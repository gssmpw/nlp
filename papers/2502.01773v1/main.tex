%%%%%%%%%%%%%%%%%%%%%%%%%%%%%%%%%%%%%%%%%%%%%%%%%%%%%%%%%%%%%%%%%%%%%%%%%%%%%%%%
%2345678901234567890123456789012345678901234567890123456789012345678901234567890
%        1         2         3         4         5         6         7         8

\documentclass[letterpaper, 10 pt, conference]{ieeeconf}  % Comment this line out if you need a4paper

%\documentclass[a4paper, 10pt, conference]{ieeeconf}      % Use this line for a4 paper

\IEEEoverridecommandlockouts                              % This command is only needed if 
                                                          % you want to use the \thanks command

\overrideIEEEmargins                                      % Needed to meet printer requirements.

%In case you encounter the following error:
%Error 1010 The PDF file may be corrupt (unable to open PDF file) OR
%Error 1000 An error occurred while parsing a contents stream. Unable to analyze the PDF file.
%This is a known problem with pdfLaTeX conversion filter. The file cannot be opened with acrobat reader
%Please use one of the alternatives below to circumvent this error by uncommenting one or the other
%\pdfobjcompresslevel=0
%\pdfminorversion=4

% See the \addtolength command later in the file to balance the column lengths
% on the last page of the document

% The following packages can be found on http:\\www.ctan.org
%\usepackage{graphics} % for pdf, bitmapped graphics files
%\usepackage{epsfig} % for postscript graphics files
%\usepackage{mathptmx} % assumes new font selection scheme installed
%\usepackage{times} % assumes new font selection scheme installed
%\usepackage{amsmath} % assumes amsmath package installed
%\usepackage{amssymb}  % assumes amsmath package installed

\usepackage{times}
% <<< from RVT

\usepackage{times}

% numbers option provides compact numerical references in the text. 
% \usepackage[numbers]{natbib}
% \usepackage{multicol}
% \usepackage[bookmarks=true]{hyperref}
\usepackage{amssymb}% http://ctan.org/pkg/amssymb
\usepackage{pifont}% http://ctan.org/pkg/pifont
% \usepackage{color}
\usepackage{soul}
\usepackage{bbm}
% additional packages
\usepackage{booktabs}
% \usepackage[table]{xcolor}
\usepackage{vcell}
\usepackage{subcaption}
% \usepackage{subfigure}
% \usepackage[font=scriptsize,labelfont=sl,labelsep=period]{caption}
\usepackage{graphicx}
\usepackage{amsmath}
\usepackage{float}
\usepackage{amssymb}
\usepackage{tikz}
\usepackage{xspace}
\usepackage{bm}
% \usepackage{wrapfig}
\usepackage{multirow}
% \usepackage{enumitem}
% numbers option provides compact numerical references in the text.
% \usepackage[bookmarks=true]{hyperref}
\usepackage{algorithm} 
\usepackage{algpseudocode} 
\usepackage{subfiles}
\usepackage{gensymb}
\newtheorem{proposition}{Proposition}

% manual commands
\newcommand{\smallsec}[1]{\noindent {\bf #1.}}
\newcommand{\method}{RVT\xspace}
\newcommand{\highlight}[1]{\textcolor{black}{#1}}
\newcommand{\tb}[1]{\textbf{#1}}
\newcommand{\yes}{\color{blue}{\ding{51}}}
\newcommand{\no}{\color{red}{\ding{55}}}
\newcommand{\blank}{\underline \underline \ }
\newcommand{\rpm}{\tiny \raisebox{.2ex}{$\scriptstyle\pm~$}}
\newcommand{\rpmh}{\huge \raisebox{.2ex}{$\scriptstyle\pm~$}}
\newcommand{\rpmx}{\small \raisebox{.2ex}{$\scriptstyle\pm~$}}
% \newcommand{\todo}[1]{\textcolor{red}{[{\bf todo: #1}]}}
\newcommand{\todo}[1]{\textcolor{black}{{ }}}

% Baseline Names
\newcommand{\bcz}{Image-BC\xspace}
\newcommand{\bczcnn}{Image-BC (CNN)\xspace}
\newcommand{\bczvit}{Image-BC (ViT)\xspace}
\newcommand{\unet}{C2F-ARM-BC\xspace}
\newcommand{\peract}{PerAct\xspace}
% from RVT >>>

\newcommand{\fix}{\marginpar{FIX}}
\newcommand{\new}{\marginpar{NEW}}
\newcommand{\SE}{\mathrm{SE}}
\newcommand{\SO}{\mathrm{SO}}
\newcommand{\pick}{\mathrm{pick}}
\newcommand{\place}{\mathrm{place}}
\newcommand{\out}{\mathrm{out}}
\newcommand{\ing}{\mathrm{in}}
\newcommand{\ours}{\textsc{???}}

\DeclareMathOperator{\key}{key}
\DeclareMathOperator{\query}{query}

\newcommand{\quot}{\mathrm{quot}}
\newcommand{\irrep}{\mathrm{irrep}}
\newcommand{\Rot}{\mathrm{Rot}}
\newcommand{\regu}{\mathrm{reg}}

\newcommand{\Ind}{\mathrm{Ind}}

\newcommand{\eref}{Equation~\ref}
\newcommand{\fgref}{Figure~\ref}


\title{\LARGE \bf
% Coarse-to-Fine $\SE(3)$ Transporter
Coarse-to-Fine 3D Keyframe Transporter
}


\author{Xupeng Zhu$^{1}$, David Klee*$^{1}$, Dian Wang*$^{1}$, Boce Hu$^{1}$, Haojie Huang$^{1}$, Arsh Tangri$^{1}$, \\ Robin Walters$^{1, 2}$, Robert Platt$^{1, 2}$ \\
\small{$^{1}$Northeastern University $^{2}$Boston Dynamics AI Institute}% <-this % stops a space
\thanks{* denotes equal contribution.}% <-this % stops a space
}


\begin{document}





\maketitle
\thispagestyle{empty}
\pagestyle{empty}


%%%%%%%%%%%%%%%%%%%%%%%%%%%%%%%%%%%%%%%%%%%%%%%%%%%%%%%%%%%%%%%%%%%%%%%%%%%%%%%%
\begin{abstract}

% Symmetry is ubiquitous in robotic manipulation policy learning. Exploiting the symmetries effectively improves the policy, especially learning from limited demonstrations using imitation learning (IL). 
Recent advances in Keyframe Imitation Learning (IL) have enabled learning-based agents to solve a diverse range of manipulation tasks.  However, most approaches ignore the rich symmetries in the problem setting and, as a consequence, are sample-inefficient. This work identifies and utilizes the bi-equivariant symmetry within Keyframe IL to design a policy that generalizes to transformations of both the workspace and the objects grasped by the gripper. We make two main contributions: First, we analyze the bi-equivariance properties of the keyframe action scheme and propose a Keyframe Transporter derived from the Transporter Networks, which evaluates actions using cross-correlation between the features of the grasped object and the features of the scene. Second, we propose a computationally efficient coarse-to-fine $\SE(3)$ action evaluation scheme for reasoning the intertwined translation and rotation action. The resulting method outperforms strong Keyframe IL baselines by an average of $>10\%$ on a wide range of simulation tasks, and by an average of $55\%$ in 4 physical experiments.

% Although recent advances in Keyframe Imitation Learning (IL) allow learning-based agents to solve a diverse range of manipulation tasks, they frequently neglect the rich symmetries in the problem setting, making those methods sample-inefficient. This work identifies and utilizes the bi-equivariant symmetry within Keyframe IL, allowing the policy to adapt simultaneously to the transformation changes on both the workspace and the objects grasped by the gripper. We make two main contributions: First, we analyze the bi-equivariance in the keyframe action scheme and propose a Keyframe Transporter derived from the Transporter Networks, which evaluates actions using cross-correlation between the features of the grasped object and the features of the scene. Second, we propose a computationally efficient coarse-to-fine $\SE(3)$ action evaluation scheme for reasoning the intertwined translation and rotation action. The resulting method outperforms strong Keyframe IL baselines by an average of $>10\%$ on a wide range of simulation tasks, and by an average of $55\%$ in 4 physical experiments.

% Keyframe Imitation Learning (IL) learns an open-loop policy from closed-loop demonstrations that can solve a diverse range of manipulation tasks \cite{james2021arm, shridhar2023perceiver}. While current methods utilize the expressiveness of Transformer \cite{vaswani2017attention} and MLP \cite{popescu2009multilayer}, they frequently neglect the rich symmetries in the Keyframe IL. This work identifies and utilizes the bi-equivariant symmetry within Keyframe IL, allowing the policy to adapt simultaneously to the transformation changes on both the workspace and the objects grasped by the gripper. We make two main contributions: First, we analyze the bi-equivariance in the keyframe action scheme \cite{james2021arm} and propose a Keyframe Transporter derived from the Transporter Networks \cite{zeng2021transporter}, which evaluates actions using cross-correlation between the features of the grasped object and the features of the scene. Second, we propose a computationally efficient coarse-to-fine $\SE(3)$ action evaluation scheme for reasoning the intertwined translation and rotation action. The resulting method outperforms strong Keyframe IL baselines by an average of $>10\%$ on 18 RLBench simulation tasks \cite{james2020rlbench} and by an average of $55\%$ in 4 physical experiments.

\end{abstract}


\section{Introduction}

% Intro:
% -- Keyframe IL is hard but important problem
% -- standard approaches do not leverage geometric structure to the extent that they could; e.g. RVT and peract are both based on transformers that destroy structure in their self attention layers
% -- we propose an approach that combines the cross attention of transporter with the c2f of c2farm to better structure the model. cross attention gives us bi-equivariance; c2f gives as additional inductive bias.
% -- the result is a significant improvement in inductive bias that improves sample efficiency
% -- this combination of cross attention and c2f is novel. rvt is completely different model architecture b/c it is based on a transformer. c2f is different b/c it does not use self attention.

Imitation Learning (IL) has emerged as an important approach for manipulation tasks. 
% IL learns policies from demonstration data, mapping sensory inputs to manipulator actions. Learning policies that act in both translation and rotation actions, i.e., $\SE(3)$ gripper pose space, is particularly important and has been an area of rapid progress~\cite{gervet2023act3d,rvt,shridhar2023perceiver, james2022coarse, chi2023diffusionpolicy}. 
IL trains a neural network on human demonstrations to map sensory inputs into robot actions. Such demonstrations are usually limited, 
% Imitation Learning usually learns from limited human demonstrations, 
since collecting demonstrations by hand is expensive. To improve sample efficiency, Keyframe IL mimics a few keyframe gripper poses in the demonstration trajectory instead of mimicking the entire trajectory. Despite the improvement, Keyframe IL ignores the geometric structures in the manipulation policy.

Current research efforts ~\cite{james2022coarse, shridhar2023perceiver,rvt,gervet2023act3d} on Keyframe IL utilize the expressiveness of Transformer~\cite{vaswani2017attention} to infer translational actions and employed Multilayer Perceptrons (MLPs)~\cite{popescu2009multilayer} to evaluate discretized Euler angle for rotation actions. However, these design choices destroy the symmetries in the policy. Transformers are not translationally equivariant due to positional embeddings and fail to enforce locality due to the global attention mechanism. On the other hand, the Euler angle representation suffers from the discontinuity or the gimbal lock issue~\cite{5D_SO3}.

% Meanwhile, symmetry is inherent in keyframe action schemes. However, current works ~\cite{james2022coarse, shridhar2023perceiver,rvt,gervet2023act3d} partially utilize the symmetry in the keyframe action, thus exhibiting poor performance when learning from limited data.

% A key challenge in BC for robotic manipulation is data efficiency, i.e. the ability to learn a policy with from few demonstrations. Since demonstrations are often collected by hand, a requirement for large numbers of demonstrations is expensive and limits the flexibility of the method. Unfortunately, many current methods are not very sample efficient~\cite{padalkar2023open,bharadhwaj2023roboagent}. 

% Transporter leverages rich geometry structure in 2d pick-place, but cannot be trivially extended to 3d and non-pick-place tasks; 
% coarse-to-fine alleviates the compute burden in 3d but ignores the intertwined translation-rotation action.

%{{{{{{{{{{{{{{{{{{{{{{{{{{{{{{

% Bi-equivariance of the robot pick-place policy has been explored in Transporter Network~\cite{zeng2021transporter} and Equivariant Transporter Network~\cite{Huang-RSS-22} wherein the desired policy generalizes to independent changes in both the pick and the place pose. However, these works use an independent pick policy and a place policy and thus are limited to pick-place tasks. Moreover, extending these $\SE(2)$ Transporter Networks to $\SE(3)$ action space is not trivial, because $\SE(3)$ action evaluation leads to significant compute increase than $\SE(2)$. For example, discretizing $\SO(2)$ action using $7.5^\circ$ bin leads to $48$ bins, but discretizing $\SO(3)$ using the similar resolution leads to $36864$ bins\cite{gorski2005healpix}, thus efficient compute is critical.

This work exploits the geometric structure of Keyframe IL to design a more sample-efficient method. We first identify a generalized form of bi-equivariant symmetry in the Keyframe IL, which extends beyond the place bi-equivariance discussed in prior works~\cite{zeng2021transporter,Huang-RSS-22,ryu2023equivariant} wherein the desired policy generalizes to independent changes in both the pick and place poses. To incorporate this property, we propose a Keyframe IL method via cross-correlation. Essentially, we derived from the place module of the Transporter Networks~\cite{zeng2021transporter, Huang-RSS-22}, where the pose actions are inferred by performing cross-correlation between a voxel representation of the scene and a dynamic kernel that represents the geometry of the grasped object. Nevertheless, extending the 2D cross-correlation in Transporter Networksto 3D suffers from curse-of-dimensionality. In 2D, the cross-correlation is performed on 3 dimensions (X, Y axes and planner rotation). In contrast, in 3D, it expands to 6 dimensions (X, Y, Z axes and roll, pitch, yaw angles), making direct computation infeasible.

%}}}}}}}}}}}}}}}}}}}}}}}}}}}}}}

%{{{{{{{{{{{{{{{{{{{{{{{{{{{{
% The coarse-to-fine technique~\cite{Gualtieri2019hierarchical,Gualtieri2020hierarchical,james2022coarse,gervet2023act3d} effectively reduces compute in translational action evaluation by evaluating the translational actions at a coarse level of resolution, identifying areas of interest, and then zooming into the translational actions in those areas to select fine-grained actions. However, these coarse-to-fine works ignore the interaction between rotation and translation and do not reduce the compute in rotation action evaluation.

% This paper is inspired by two techniques, both of which are based on leveraging geometric structures present in the robotics domain to bias the neural network model toward good policies. 
% The first technique leverages the bi-equivariance between the pick and the place pose wherein the desired policy generalizes to independent changes in both the pick and the place pose. This is perhaps best exemplified by pick/place models like Transporter Network~\cite{zeng2021transporter} and Equivariant Transporter Net~\cite{huang2023leveraging} that explicitly encode the symmetries involved in the relative rotations of pick and place poses.
% The second technique is to leverages the locality assumption about information relevant to decision making. This is embedded in a coarse-to-fine model that evaluates $\SE(3)$ actions at a coarse level of resolution, identifies areas of interest, and then zooms into the $\SE(3)$ actions in those areas to select fine-grained actions. This approach works best in the keyframe setting~\cite{rvt} where the agent selects full 6-DOF poses toward which to move the robot hand. The model first selects an approximate region and then selects a precise action. Gualtieri et al.~\cite{Gualtieri2019hierarchical,Gualtieri2020hierarchical} demonstrated that this approach significantly improved performance in the open-loop control tasks and James et al.~\cite{james2022coarse} made a similar finding in the Reinforcement Learning (RL) setting.

% Extending these two ideas is not trivial. The Transporter Networks\cite{zeng2021transporter, huang2023leveraging, cliport} are limited to pick-place and $\SE(2)$ action space. To adopt Transporter to $\SE(3)$ keyframe setup, one needs to overcome the pick-place constraint, and efficiently compute the cross-correlation in $\SE(3)=\mathbb{R}^3\rtimes\SO(3)$. Discretizing $\SO(2)$ action using $7.5^\circ$ bin leads to $48$ bins, but discretizing $\SO(3)$ using the similar resolution leads to $36864$ bins\cite{gorski2005healpix}, thus efficient compute is critical. On the other hand, the current line of coarse-to-fine works\cite{Gualtieri2020hierarchical, james2022coarse, gervet2023act3d} in robot learning ignores the intertwining between rotation and translation, only reducing compute in the translation.

% The proposed approach unleashes the power of Transporter Networks and coarse-to-fine technique in the $\SE(3)$ keyframe action setting, where the model outputs a full 6-DoF gripper pose that describes where to move the robot hand next. We recognize that the bi-equivariance property commonly exists in the keyframe action scheme (as shown in Figure \ref{fig:bi_equ}) and propose a keyframe action via cross-correlation to embed this property. Essentially, we derived from the place module of the Transporter Networks~\cite{zeng2021transporter, Huang-RSS-22} architecture to infer the pose action by performing a cross-correlation between a voxel representation of the scene and a dynamic kernel that represents the geometry of the grasped object. Moreover, this paper extend coarse-to-fine idea \cite{Gualtieri2019hierarchical,Gualtieri2020hierarchical,james2022coarse} from $\mathbb{R}^3$ to $\SE(3)=\mathbb{R}^3\rtimes\SO(3)$ to realistically evaluate the $\SE(3)$ actions. The proposed $\SE(3)$ coarse-to-fine action inference effectively alleviates the curse-of-dimensionality of $\SE(3)$ action evaluation. Our resulting model can learn a variety of manipulation behaviors including pushing, turning, using tools, etc. Our method outperforms baselines by $>10\%$ on 18 RLBench simulation tasks and by $55\%$ on 4 real-world tasks.  

To overcome the efficiency issue in 3D cross-correlation evaluation, this paper further proposes an $\SE(3)$ coarse-to-fine (C2F) calculation, extending previous translational-only C2F methods \cite{Gualtieri2020hierarchical,james2022coarse} to translation and rotation actions simultaneously. The proposed $\SE(3)$ C2F method begins by coarsely evaluating the translation and rotation actions, identifying the best $\SE(3)$ action. Then the method refines the evaluation by zooming into the best coarse $\SE(3)$ action and evaluating its neighboring translation and rotation actions. This hierarchical method drastically improved efficiency in 3D cross-correlation.

Our resulting model can learn a wide range of manipulation behaviors, including pushing, turning, using tools, etc. with a single unified architecture. This distinguishes it from prior bi-equivariant approaches~\cite{zeng2021transporter,Huang-RSS-22,ryu2023equivariant,yenchen2022mira,ryu2023diffusionedfs,huang2024fourier}, which are limited to pick-place tasks and require separated modules for picking and placing. Moreover, in contrast to existing Keyframe IL methods\cite{shridhar2023perceiver, rvt, gervet2023act3d} that are based on Transformers and Euler angles, our method evaluates action by cross-correlation on translation and rotation, effectively embeds bi-equivariance, and mitigates the gimbal lock issue. Our method outperforms Keyframe IL baselines by $>10\%$ on 18 RLBench simulation tasks with $100$ demonstrations and achieves a $55\%$ improvement on 4 real-world tasks when trained with only $10$ demonstrations.

%}}}}}}}}}}}}}}}}}}}}}}}}}}}}

\section{Related Work}

% \textbf{$\SE(3)$ robotic policy.}
% $\SE(3)$ robotic policy is desirable because that can solve various manipulation tasks that cannot be solved by $\SE(2)$ policy. 
% Previous works \cite{james2022coarse, shridhar2023perceiver, rvt, gervet2023act3d} formulate the $\SE(3)$ robotic policy as a classification problem, which estimates the probability of the optimal action from translation action bins and rotation action bins. While the translation $\mathbb{R}^3$ discretization enables efficient learning, the independent rotation $\SO(3)$ discretization in Euler angles suffers from the gimbal lock issue and fails to model the multi-modality in rotation. Another class of works formulates the policy as a regression problem \cite{end2end_deep_visuomotor}. Following this, \cite{LSTM_GMM, zhu2023viola} leverages the Gaussian mixture model and \cite{zhao2023learning} utilizes conditional variation autoencoder to address the multi-modality issue in the policy, though ignores gimbal lock issue and subject to mode collapse \cite{florence2022implicit}. Lastly, recent works propose to represent policy as diffusion process \cite{chi2023diffusionpolicy, ryu2023equivariant, xian2023chaineddiffuser} that iteratively refine the action, which efficiently addresses the multi-modality issue. Especially, the $6$D $\SO(3)$ representation \cite{5D_SO3,chi2023diffusionpolicy} avoids gimbal lock issue. However, diffusion-based policies require multiple denoise steps to infer the action. In contrast, we address the gimbal lock issue in \cite{james2022coarse, shridhar2023perceiver, rvt, gervet2023act3d} by operating in the original $\SO(3)$ representation, rather than the reduced dimension representation \cite{5D_SO3} and address the multi-modality issue in $\SO(3)$ by classifying $\SO(3)$ action candidate. Lastly, in contrast to hundreds of denoising steps during inference in diffusion methods \cite{chi2023diffusionpolicy, ryu2023diffusionedfs, xian2023chaineddiffuser}, we only need one forward pass.

\textbf{Keyframe action scheme.}
ARM \cite{james2021arm} simplifies the trajectory of a closed-loop policy by using multiple keyframes. In this way, every state in the trajectory has the action that leads to the next keyframe pose and gripper aperture, and this is called demo augmentation in \cite{james2021arm}. The keyframe action scheme can be viewed as a policy between closed-loop control and open-loop control, allowing for diverse task learning while maintaining high sample efficiency in $Q$-learning \cite{james2021arm}. Following this idea, C2F-ARM\cite{james2022coarse} extends \cite{james2021arm} from 2D CNN to 3D CNN using a hierarchical evaluation style.
% , which further improves sample efficiency by leveraging the 3D translational equivariance of 3D CNN in the $Q$-learning context. 
Later PerAct \cite{shridhar2023perceiver} adopts keyframe action in the context of imitation learning, and an extra binary collision avoidance action to let the agent seamlessly control the motion planner for complex tasks. 
% This work follows \cite{shridhar2023perceiver} to apply the keyframe action scheme in imitation learning. 
% Critically, we enforce bi-equivariance in the keyframe action scheme, thus gaining superior performance than \cite{shridhar2023perceiver, james2022coarse, rvt, gervet2023act3d} that do not have bi-equivariance.
More recent works ~\cite{shridhar2023perceiver, rvt, gervet2023act3d} employ Transformers to infer the translational actions and use discretized Euler angles for the gripper rotation. However standard Transformers\cite{vaswani2017attention} are not translationaly equivariant due to position embeddings assigning unique values to each position. The Euler angles representation suffers from the gimble lock issue~\cite{5D_SO3}. This work addresses these issues by using translational-rotational cross-correlation, which enforces translational equivariance and avoids the gimbal lock issue, thus gaining superior performance.

\textbf{Equivariance in robotic policy learning.}
%translational equivariance --> rotation 
The generalization ability of CNN is partly due to its nature of translational equivariance. \cite{zeng2018robotic,Morrison-RSS-18} showed that the translational equivariance of FCN can improve the learning efficiency of manipulation tasks.
% Rotational equivariant neural networks are studied in many recent works \cite{cohen2016group,cohensteerable,weiler2019general,thomas2018tensor,fuchs2020se,deng2021vector,cesa2022a}. 
Later, 2D rotataional equivariance are explored in \cite{wang2021equivariant,zhu2022grasp,Huang-RSS-22,zhu2023grasp,jia2023seil,wang2022so2equivariant, wang2022onrobot, zhao2022integrating,wangequivariant} and dramatically improved the sample efficiency. Several recent works~\cite{simeonov2022neural, huang2022edge,ryu2023equivariant,huang2024fourier,ryu2023diffusionedfs,huorbitgrasp,huang2024imagination} attempted encoding the 3D rotation symmetries into manipulation tasks. 
% \cite{simeonov2022neural} learned an $\SE(3)$-invariant field to manipulate the object in the same category and \cite{huang2022edge} took advantage of the $\SE(3)$-invariant property of grasp evaluation function. 
\cite{ryu2023equivariant,ryu2023diffusionedfs,huang2024fourier} achieve 3D pick-place bi-equivariance by using separate models for the pick and the place actions but are unable to perform keyframe actions. Furthermore, diffusion-based method \cite{ryu2023diffusionedfs} requires $600$ iterations for action inference, while Fourier-based method \cite{huang2024fourier} necessitates rotating a voxel grid $\sim 400$ times to perform 3D Fourier transform.
% \cite{ryu2023equivariant,ryu2023diffusionedfs} achieves 3D pick-place Bi-equivariance by encoding the $\SO(3)$-equivariant point feature with graph-based equivariant models. \cite{huang2024fourier} used the voxel grids and achieved the 3D bi-equivariance with dynamic steerable kernel in Fourier space. However, these works \cite{zeng2021transporter, Huang-RSS-22, huang2024fourier, ryu2023equivariant, ryu2023diffusionedfs} are all limited to pick-place tasks.
To the best of our knowledge, we are the first to recognize and leverage the Bi-equivariance in the keyframe action setting, allowing us to tackle a much broader range of manipulation problems beyond pick-place. Additionally, the proposed 3 levels of coarse-to-fine action evaluation enable one-shot action inference, making the approach computationally efficient during training and inference.
% Moreover, these works relied on task-specific in-hand crop size. In contrast, we propose the in-hand segmentation that enables a fixed crop size.

\textbf{Coarse-to-fine action evaluation.}
Evaluating all discretized $\SE(3)$ action candidates is expensive due to dimensionality. An effective evaluation is to follow a coarse-to-fine scheme~\cite{Gualtieri2020hierarchical, james2022coarse, gervet2023act3d} that gradually refines the translational action iteratively. Specifically, C2F-ARM \cite{james2022coarse} iteratively evaluates translation $q$ value in a finer voxel grid. RVT \cite{rvt} first evaluates the left view, the front view, and the top view $q$ value maps, then projects these maps to reconstruct the 3D translation $q$ value map. Act3D \cite{gervet2023act3d} iteratively evaluates sampled translation action candidates in the point cloud observation and then reduces the range of translation sampling range to refine action. Another stream of work \cite{wang2021policy, wang2021equivariant, zhu2022grasp} proposes to first evaluate the discretized translation action, then evaluate the discretized rotational action. While all of these works ignore the intertwining between translation and rotation action, we perform the coarse-to-fine action evaluation in translation while considering rotation.

\section{Background}

\textbf{Equivariance.} An equivariant function possesses the property that when the input is transformed, the output transforms accordingly. For instance, consider a planner equivariant grasping function $Q$ ~\cite{zhu2022grasp}, which takes the scene $s$ as input and outputs the gripper grasping location $a$. If the scene is transformed by a planar translation and rotation $g\in \SE(2)$, the gripper pose transforms accordingly:
\begin{align}
    Q(s) = a,\quad Q(gs) = ga.
\end{align}

\textbf{Keyframe Imitation Learning.} A keyframe action policy \cite{james2021arm, james2022coarse} specifies how the robot should move by defining a sequence of desired translations and rotations, i.e., $\SE(3)$ end-effector poses. When executing a keyframe action, the robot queries a collision-free motion planner to compute a trajectory that reaches the desired pose. The keyframe policy solves a task by executing a sequence of keyframe actions.
% In this paper, we learn a policy $\pi$ that maps from the current observation $o$ onto a keyframe action. 
%Similar to ~\cite{shridhar2023perceiver,rvt,gervet2023act3d}, we formulate keyframe imitation learning as a classification task
Keyframe Imitation Learning \cite{shridhar2023perceiver,rvt,gervet2023act3d} is a classification task that learns the action-value function $Q$ over a discretized action space, aiming to maximize the value for the discretized expert keyframe action $a$, given the observation $o$. The keyframe actions are extracted from expert demonstration trajectories by identifying moments when the gripper velocity is zero or its aperture changes \cite{james2021arm}. The observation, $o = \{s, p\}$, contains a scene representation $s$ (e.g., voxel or point cloud), and proprioceptive information, $p=\{\text{T}_{ee}, s_\text{open}, t\}$, where $\text{T}_{ee}$ is the gripper pose, $s_\text{open}$ is the gripper aperture, and $t$ is the time step. The action $a = \{a_\text{T}, a_\text{open}, a_\text{collide}\}$ specifies the desired $\SE(3)$ gripper pose $a_\text{T}$ with translation and rotation, the gripper open-close action  $a_\text{open}$, and a binary flag $a_\text{collide}$ to indicate whether to ignore collisions in the motion planning. Compared to higher frequency closed-loop control policies that output arm displacements, the keyframe framework significantly reduces the time horizon over which the policy must reason and thereby simplifies the policy learning problem.

\begin{figure}
    \centering
    % \vspace{-0.5cm}
    \includegraphics[width=0.4\textwidth]{figure/transporter.png}
    \caption{The place module of Transporter Networks \cite{zeng2021transporter}, along with follow-up works \cite{Huang-RSS-22, ryu2023equivariant, huang2024fourier, ryu2023equivariant} achieves bi-equivariance in place policy (e.g., picking an ``L'' shape and placing it in an ``L''-shaped receptacle) by performing cross-correlation between the scene features $f_s$ and the in-hand features $f_{ih}$. In the computed action value map $Q_\text{place}$, the height and width represent the X and Y translations of the gripper, while the channels correspond to different gripper rotations. Therefore $Q_\text{place}$ densely evaluates each trans-rotational action.}
    \label{fig:transporter}
    \vspace{-0.2cm}
\end{figure}

\textbf{The place module of Transporter Network.} The Transporter Network~\cite{zeng2021transporter, Huang-RSS-22} includes a planar pick and a planar place module that encodes rich geometric structure. The pick module is omitted for simplicity. The place module, illustrated in Figure \ref{fig:transporter}, takes the observation $s$ and the pick location $a^*_\text{pick}$ as inputs. The place module $Q_\text{place}(s, a^*_\text{pick})$ crops the observation at the pick location as the in-hand observation: $s_{ih}=crop(s, a^*_\text{pick})$. Then both the scene observation and the in-hand observation are embedded into deep latent features, maintaining the same spatial resolution, through a $\key$ and a $\query$ Unet network $f_s = \key(s), f_{ih}^i = \query(g_i s_{ih}), i \in \{1,2,...,n\}$. $f_s$ is the scene feature containing the receptacle, and $[f_{ih}^i]$ is a stack of in-hand features corresponding to each possible rotation action $g_i = \frac{2\pi i}{c}$, produced by passing rotated versions of the in-hand object observation to the $\query$ network. The place action value $Q_\text{place}$ is the result of $2$D cross-correlation in $\SE(2)$ action space between the scene features $f_s$ and each rotated in-hand feature $f_{ih}^i$, $Q_\text{place}^i = f_s \star f_{ih}^i$. The place action $a_\text{place}\in \SE(2)$ is the argmax over the place action value, $a_\text{place} = \pi(s,s_{ih}) = \arg\max Q_\text{place}(s,s_{ih})$. As \cite{Huang-RSS-22} states, if $\key$ and $\query$ networks are equivariant, then $Q_\text{place}$ is bi-equivariant due to the bi-equivariant properties of cross-correlation. However, the original Transporter Network is incompatible with keyframe action because it uses separate, specialized networks for inferring the pick action and the place action, which are coupled in a hard-coded inference sequence.
% specialized networks for inferring the pick action (equivariant actions) and the place action (bi-equivariant actions) which are coupled in a hard-coded inference sequence.

\section{Method}

% Method:

% 3.1: Keyframe IL via cross correlation
%     -- we propose an approach based on the "place" pathway in Transporter Net
%     -- briefly outline the original 2d version of transporter modified to accommodate a gripper, as we did in equi-transporter
%     -- describe a 3d version modified to use 3d convolutions. describe how this would work with a discrete set of rotations
%     -- use equation 1 in this description. It's fine to leave the \forall g \in G in eq 1; just make it clear that this means that we are evaluating the translational convolution, not rotational.
%     -- make it clear what kind of encoders (not equivariant) we are using here.
%     -- provide a picture of transporter model that has "encoders" that can be treated as either 2d or 3d.
%     -- describe how this model encodes bi-equivariance into the model. give the equations for what bi-equivariance is and describe how the model accomplishes it.
%     -- describe how this version is intractible b/c of the large number of rotations that need to be processed.
    
% 3.2: Coarse-to-fine cross correlation model
%     -- describe how c2f can fix the problem of intractibility in principle
%     -- describe c2f in detail using equation 2.
%     -- refer to fig 2. go into detail about exactly how the c2f works in the model, i.e. that features are computed once and that multiple cross convolutions occur. Note that this is different from what was done in c2f arm or gualtieri's work where features were recomputed at each level.
%     -- in fig 2, it's a little hard to understand the orange hand orientations. you should explain that somewhere.
%     -- describe how c2f provides an additional inductive bias based on an assumption that precise position and orientation only depends on local geometry around the place location.
    
% 3.3 Implementation
%     -- similar to what you already have except to improve the notation as we discussed.


% Xupeng's plan for the method section:
% Problem statement
%     -- introduce what is Keyframe IL
%     -- states bi-equ in Keyframe IL

% 3.1: Keyframe IL via cross correlation
%     -- briefly outline the original 2d version of transporter modified to accommodate a gripper, as we did in equi-transporter
%            provide a picture of transporter model that has "encoders" that can be treated as either 2d or 3d.
%            use equation 1 in this description. It's fine to leave the \forall g \in G in eq 1; just make it clear that this means that we are evaluating the translational convolution, not rotational.
%            describe how this model encodes bi-equivariance into the model.
%     -- we propose an approach based on the "place" pathway in Transporter Net
%            b/c we can make place module compatible with keyframe action
%            b/c it encodes bi-equivariance into the model
%     -- describe a keyframe version modified to use 3d convolutions.
%            describe how this would work with a discrete set of rotations
%            make it clear what kind of encoders (not equivariant) we are using here.
%            we crop and canonicalize the in-hand observation, this enables the algorithm to switch between bi-equ and equ automatically.
%     -- describe how this version is intractible b/c of the large number of rotations that need to be processed.

% 3.2: Coarse-to-fine cross correlation model
%     -- describe how c2f can fix the problem of intractibility in principle
%     -- describe c2f in detail using equation 2.
%     -- refer to fig 2. go into detail about exactly how the c2f works in the model, i.e. that features are computed once and that multiple cross convolutions occur. Note that this is different from what was done in c2f arm or gualtieri's work where features were recomputed at each level.
%     -- in fig 2, it's a little hard to understand the orange hand orientations. you should explain that somewhere.
%     -- X(describe how c2f provides an additional inductive bias)X -> we didn't have locality bias, b/c we used Unet that has global reception field
    
% 3.3 Implementation
%     -- similar to what you already have except to improve the notation as we discussed.

\begin{figure}\centering
        \includegraphics[height=3.5cm]{figure/bi_equivairance1.png}
        % \includegraphics[height=4cm]{figure/bi_equivairance2.png}
        \caption{Bi-equivariance in keyframe policies. Second column: given a scene, the policy $\pi$ prescribes an optimal action $a$. First column: if the scene is rotated by $g_1$, the optimal action should also be rotated: $g_1a$. Third column: if the in-hand object is rotated by $g_2$, the optimal action should pre-rotate to compensate: $ag_2^{-1}$.
        % The last column: our method swings the golf club properly, while RVT~\cite{rvt} cannot distinguish the head and the grip of the golf club.
        }
        \vspace{-0.2cm}
\label{fig:bi_equ}
\end{figure}

\subsection{Bi-equivariance of keyframe policy.}

We find that the keyframe action policy exhibits bi-equivariance with respect to the pose action on both the scene and the in-hand objects (see Figure \ref{fig:bi_equ}). Consider the keyframe policy, denoted by the simplified notation $\pi(s,s_{ih})=a^*_\text{T}$ which takes as input the scene observation $s$ in the world frame and the in-hand observation $s_{ih}$ (the object held by the gripper) in the gripper frame, and outputs the keyframe pose action $a^*_\text{T}$. The first equivariance is when the scene is transformed by $g_1\in\SE(3)$, the pose action should be transformed by,
\begin{align}
    g_1 a^*_\text{T} = \pi( g_1 \cdot s, s_{ih})
    \label{equ:bi_equ1}
\end{align}
The second equivariance is when the grasped object is transformed by $g_2\in\SE(3)$, the pose action should compensate for this transformation by inversely transforming by $g_2^{-1}$,
\begin{align}
    a^*_\text{T}g_2^{-1} = \pi(s, g_2 \cdot s_{ih})
    \label{equ:bi_equ2}
\end{align}

Moreover, when there is grasped object(s), both Equations \ref{equ:bi_equ1} and \ref{equ:bi_equ2} are satisfied, defining what term as bi-equivariant actions, e.g., placing and using tools. When the action depends solely on the gripper, the bi-equivariance of the policy degenerates to equivariance due to the fixed gripper pose ($g_2$ becomes identity). We refer to this type of action as equivariant actions, e.g., grasping and pushing. This framework unifies the previously separate concepts of pick equivariance \cite{zhu2022grasp,zhu2023grasp,huang2022edge} and place bi-equivariance~\cite{Huang-RSS-22, huang2024fourier, ryu2023equivariant, ryu2023diffusionedfs} under a cohesive keyframe action scheme. 

% \textbf{Bi-equivariance of keyframe policy.} We find that the keyframe policy $\pi(o) = a^*$ exhibits bi-equivariance property (with respect to $a^*_\text{T}$), as demonstrated in Figure \ref{fig:bi_equ}. Simplified the keyframe policy $\pi(s,s_{ih})=a^*_\text{T}$ that takes as input the scene observation $s$, the in-hand observation (object) $s_{ih}$, and outputs the keyframe pose action $a^*_\text{T}$. The first equivariance is when the scene is transformed by $g_1\in\SE(3)$, the pose action should be transformed by $g_1a$ accordingly,
% \begin{align}
%     g_1 a^*_\text{T} = \pi( g_1 \cdot s, s_{ih})
%     \label{equ:bi_equ1}
% \end{align}
% The second equivariance is when the grasped object is transformed by $g_2\in\SE(3)$, the pose action should compensate for this transformation by inversely transforming by $g_2^{-1}$,
% \begin{align}
%     a^*_\text{T}g_2^{-1} = \pi(s, g_2 \cdot s_{ih})
%     \label{equ:bi_equ2}
% \end{align}
% 
% This extends the pick-place bi-equivariance~\cite{Huang-RSS-22, huang2023leveraging, huang2024fourier, ryu2023equivariant, ryu2023diffusionedfs} to keyframe action scheme. Moreover, when there is grasped object(s), Equation \ref{equ:bi_equ1}, \ref{equ:bi_equ2} hold. We term this type of action as bi-equivariant actions, e.g., placing, and using tools. When the action solely depends on the gripper, then the bi-equivariance of the policy degenerates to equivariance due to the fixed gripper pose. We term this type of action as equivariant actions, e.g., grasping and pushing.


% 3.1: Keyframe IL via cross correlation
%     -- briefly outline the original 2d version of transporter modified to accommodate a gripper, as we did in equi-transporter
%            provide a picture of transporter model that has "encoders" that can be treated as either 2d or 3d.
%            use equation 1 in this description. It's fine to leave the \forall g \in G in eq 1; just make it clear that this means that we are evaluating the translational convolution, not rotational.
%            describe how this model encodes bi-equivariance into the model.
%     -- we propose an approach based on the "place" pathway in Transporter Net
%            b/c we can make place module compatible with keyframe action
%            b/c it encodes bi-equivariance into the model
%     -- describe a keyframe version modified to use 3d convolutions.
%            describe how this would work with a discrete set of rotations
%            make it clear what kind of encoders (not equivariant) we are using here.
%            we crop and canonicalize the in-hand observation, this enables the algorithm to switch between bi-equ and equ automatically.
%     -- describe how this version is intractible b/c of the large number of rotations that need to be processed.


\begin{figure*}\centering
    % \hspace*{-1.5cm}
    % \vspace{-0.5cm}
    \includegraphics[width=0.95\textwidth]{figure/ours_method.png}
    \caption{\textbf{Coarse-to-Fine $3$D Keyframe Transporter} inferences in two steps. Left: in step 1, the in-hand features $s_{ih}$ are obtained by cropping and transforming the scene features $s$ into the gripper frame. Then the $\key$ and $\query$ U-net networks map observations $s$ and $s_{ih}$ into pyramids of latent features $f_s^l$ and $f_{ih}^l$ respectively. Middle: in step 2, the action values $Q_\text{T}^l: \hat{G}_l \rightarrow \mathrm{R}$ are computed through a coarse-to-fine cross-correlation between the latent scene features $f_s^l$ and in-hand features $f_{ih}$. At the coarse level, the evaluated actions cover a wide translational-rotational range in a coarse grid. In the end, the fine level narrows the trans-roto range but provides fine resolution for precise action evaluation.
    % At the initial level, $\hat{G}_1$ covers the entire $\SE(3)$ action space with coarse resolution. In the following levels, $\hat{G}_l$ reduces translation and rotation action range but increases resolution. 
    Lastly, gripper open-close and planner collision actions are evaluated by MLP with the features from the $key$ U-net.}
    \label{fig:c2f_bi_equ}
    \vspace{-0.2cm}
\end{figure*}

\subsection{Keyframe IL via $3$D Cross-Correlation}
\label{sec:place_only}

We propose Keyframe IL via $3$D cross-correlation, inspired by the $2$D Transporter Net~\cite{zeng2021transporter}, to capture rich geometric structures of bi-equivariance. Several modifications are made to incorporate Transporter into the keyframe action scheme, as detailed below.

One modification is the use of the place module while discarding the pick module. Keyframe actions can be viewed as a special case of placing where anything held by the gripper is considered the in-hand object being placed, and the target pose is treated as the receptacle. This framework allows the in-hand object to be the gripper itself when no object is being grasped. Consequently, actions that rely solely on the gripper (e.g., grasping, pushing), can be interpreted as placing the gripper onto the target object. We represent the in-hand object $s_{ih}$ by canonicalizing (aligning) the scene voxel map $s$ to the gripper (end-effector) frame: $s_{ih} = crop(\text{T}_\text{ee}^{-1}\cdot s)$. 
% In-hand segmentation is applied to retrieve the feature of the grasped object while removing detractors. 
If the in-hand observation consists only of the canonicalized gripper, as in the case of equivariant actions (e.g., picking), the proposed bi-equivariant module naturally simplifies to a single equivariance. Conversely, when the in-hand observation includes an object, the bi-equivariance property remains intact. This adaptability ensures compatibility with the keyframe bi-equivariance.

Another modification is the adaptation of the $2$D place module to a $3$D setting. To do so, the $\key$ and $\query$ Unets are replaced with $3$D Unet\cite{Unet, 3DUnet}, and the action value becomes the result of $3$D cross-correlation between scene features and in-hand features for each discretized pose action $a_\text{T}\in\SE(3)$.


% Despite these two modifications, extending the place module of the Transporter Network to $3$D is computationally infeasible. 
However, extending the place module of the Transporter Network to $3$D poses significant computational challenges. While $2$D cross-correlation operates across 3 dimensions (X, Y axes, and planar rotation), our method performs $3$D cross-correlation across 6 dimensions (X, Y, Z axes and roll, pitch, yaw angles). This results in an exponentially increased computational cost.

% Despite these two modifications, trivially extending the place module of the Transporter Network to $3$D is infeasible because calculating the cross-correlation in $\SE(3)$ is computationally intensive. Transporter Networks perform $2$D cross-correlation that has 3 dimensions but our method performs $3$D cross-correlation on 6 dimensions (X, Y, Z axes and roll, pitch, yaw angles).

% $2$D cross-correlation of Equation \ref{equ:place} has complexity $O(n^3m^2)$ but that of $3$D cross-correlation version is $O(n^6m^3)$, where $n$ is resolution of each dimension of action space, $m$ is resolution of the features $f, f_{ih}$. We address this issue in the next Section 

\subsection{$\SE(3)$ Coarse-to-Fine Action Evaluation}
\label{sec: cross_correlation}

% 3.2: Coarse-to-fine cross correlation model
%     -- describe how c2f can fix the problem of intractibility in principle
%     -- describe c2f in detail using equation 2.
%     -- refer to fig 2. go into detail about exactly how the c2f works in the model, i.e. that features are computed once and that multiple cross convolutions occur. X(Note that this is different from what was done in c2f arm or gualtieri's work where features were recomputed at each level.)X [Removed this statement b/c Act3D did it similarly to ours.]
%     -- in fig 2, it's a little hard to understand the orange hand orientations. you should explain that somewhere.
%     -- X(describe how c2f provides an additional inductive bias)X -> we didn't have locality bias, b/c we used Unet that has global reception field

% Even though extending Transporter Network\cite{zeng2021transporter} to $\SE(3)$ could be trivially done by implementing the $\key$ and $\query$ networks with two 3D U-net \cite{Unet, 3DUnet}, calculating the cross-correlation in Transporter Network (see Eqn.~\ref{equ:place2} in Appendix for details) in $\SE(3)$ over a voxel grid is computationally intensive.
% The complexity is $\mathrm{O}\big((mnk)^3\big)$, when $m^3$ is the size of the scene voxel map $s$, $n^3$ is the size of the in-hand voxel map $s_{ih}$, and $k$ is the number of rotations used to discretize each of row, pitch, and yaw.
We present an $\SE(3)$ coarse-to-fine cross-correlation approach, extending the translational coarse-to-fine methods \cite{james2022coarse,Gualtieri2020hierarchical,gervet2023act3d} to encompass both translation and rotation. This method drastically reduces the computational complexity while maintaining high resolution in action evaluation. Additionally, we address the gimbal lock issue in \cite{shridhar2023perceiver, rvt, gervet2023act3d} by directly rotating the in-hand features multiple times to represent rotation actions. Specifically, the method first coarsely evaluates the $\SE(3)$ pose action space to identify the best coarse action. It then refines this action by zooming into the neighborhood of the coarse action and performing a finer evaluation. By repeating this process up to $l$ levels, the final level achieves high-resolution action evaluation with a significantly reduced compute.
% reduce the complexity to $\mathrm{O}\big((mnk)^\frac{3}{l}\big)$, where $l$ is number of coarse-to-fine levels. 
% We first perform cross-correlation at the coarse discretization resolution, then focus on the neighborhood of the best action, discretize with finer resolution, and repeat. By refining multiple times, we can evaluate $\SE(3)$ actions in a very fine resolution efficiently. 
% Please note our method applied coarse-to-fine to the translation and the rotation action inference simultaneously, which differs from previous works \cite{Gualtieri2019hierarchical,james2022coarse,gervet2023act3d} that only apply to translation and disregard rotation.

Defining the lift cross-correlation between an input function $b$ and a dynamic filter $k$ under a group $G$ by,
\begin{align}
    (b \star k)[g] = \int_{x\in X}b(x)(g\cdot k)(x)dx, \forall g \in G,
\end{align}
where $X$ is the domain of $b$ and $k$, i.e., X, Y, and Z dimensions in our case. In practice, $X$ is represented as a voxel grid, and the group $G$ is approximated by a discrete group $\hat{G}$, which includes a translation grid along the XYZ axes times a rotation grid over the row, pitch, and yaw axes. The term $(g\cdot k)(x)$ translates and rotates the dynamic filter $k$ by $g$, for each $g$ in the grid $\hat{G}$. Notably, while the inputs reside in $X\in\mathbb{R}^3$ (a voxel grid), the output resides in $g\in\hat{G}\subset\SE(3)$ (a voxel grid times a rotation grid). Thus this cross-correlation lifts the input signal.

As shown on the left side of Figure \ref{fig:c2f_bi_equ}, we first use a $\key$ 3D Unet\cite{Unet, 3DUnet}, based on a convolutional neural network (CNN), to embed the scene observation $s$ into a pyramid of latent features $f_s^l$ at different voxel resolution levels $l$. Then a $\query$ 3D Unet embeds the in-hand observation $s_{ih}$ into features $f^{'l}_{ih}$ and predicts a mask $Q_\text{mask}$. The mask is then applied to the in-hand features to remove noise, resulting in the final masked features: $f^{l}_{ih} = Q_\text{mask} \cdot f^{'l}_{ih}$.
% Then at the initial resolution level $l=1$, the cross-correlations are calculated over the set of group elements $\hat{G}_1 \subset \SE(3)$, consisting of group elements $\hat{G}_1 = \mathbb{Z}^3 \times H$ where $\mathbb{Z}^3$ is a translation grid and $H$ is a rotation grid (Healpix grid~\cite{gorski2005healpix}), to a coarsely discretize the entire translation-rotation action space. 
To infer the action $a^{l*}_\text{T}$ at level $l$, the lift cross-correlation is computed over the set of group elements $\forall g \in \hat{G}_l$,
\begin{align}
    Q^l_\text{T}[g] &= (f_s^l \star_l f_{ih}^l)[g] = \sum_{x\in X}f_s^l(x)(g\cdot f_{ih}^l)(x)
\end{align}
where $Q^l_\text{T}$ represents the pose action value at level $l$, and the action is greedily selected by $a^{l*}_\text{T} = \arg\max(Q^l_\text{T})$. At the coarsest level ($l=1$), the group $\hat{G}_1$ coarsely discretizes the action space into a low-resolution voxel grid times a low-resolution rotation grid. For finer levels ($l>1$), $\hat{G}_l$ refines the neighborhood around the optimal action from the previous level $a^{l-1*}_\text{T}$, by dividing the voxel-rotation grid into multiple finer voxel-rotation grid. This process is illustrated in the middle of Figure \ref{fig:c2f_bi_equ}.
% Notice that at the initial level $\hat{G}_1$ discretizes the entire translation-rotation action space. 
% Intuitively, the first level finds the best coarse translation-rotation action, then the following level evaluates a finer neighbor of the best action. In the end, the best action in the finest translation-rotation resolution is selected. 

In practice, $Q^l_\text{T}$ is a multi-channel voxel signal, where the value at each voxel corresponds to the translational action value, and each channel represents the rotational action value. To discretize the $\SE(3)$ action space into $\hat{G}_l$, we use hierarchical voxel grids for translation and Healpix grids\cite{gorski2005healpix} for rotation. The initial voxel grid has a size of $24^3$, while the rotation grid consists of $24$ discrete rotations. At each subsequent level, a voxel-rotation grid from the previous level is divided into a finer grid of size $2^3\times8$. Using a 3-level C2F process, we evaluate $\SE(3)$ action with a final resolution equivalent to a $96^3\times36864$ grid, or $1$cm in translation and $7.5^\circ$ in rotation.

\begin{figure}
    \centering
    \vspace{-0.5cm}
    % \hspace{-1cm}
    % \begin{subfigure}[b]{0.15\textwidth}\centering\includegraphics[width=\columnwidth]{figure/in_hand_seg/stack_blocks_s.png}\caption{stack\_blocks scene $s$}\end{subfigure}
    % \begin{subfigure}[b]{0.1\textwidth}\centering\includegraphics[width=\columnwidth]{figure/in_hand_seg/stack_blocks.png}\caption{in-hand crop $s_{ih}$}\end{subfigure}
    % \begin{subfigure}[b]{0.12\textwidth}\centering\includegraphics[width=\columnwidth]{figure/in_hand_seg/stack_blocks_seg.png}\caption{in-hand segmentation $Q_\text{mask}\cdot s_{ih}$}\end{subfigure}
    
    % \hspace{-1cm}
    \begin{subfigure}[b]{0.18\textwidth}\centering\includegraphics[width=\columnwidth]{figure/in_hand_seg/place_cups_s.png}\caption{scene $s$}\end{subfigure}
    \begin{subfigure}[b]{0.14\textwidth}\centering\includegraphics[width=\columnwidth]{figure/in_hand_seg/place_cups.png}\caption{in-hand $s_{ih}$}\end{subfigure}
    \begin{subfigure}[b]{0.14\textwidth}\centering\includegraphics[width=\columnwidth]{figure/in_hand_seg/place_cups_seg.png}\caption{$Q_\text{mask}\cdot s_{ih}$}\end{subfigure}
    
    
    \caption{Visualization of learned in-hand segmentation.}
    
    \vspace{-0.2cm}
    \label{fig:in_hand_segmentation}
\end{figure}

\begin{table*}[!ht]
\centering
\scriptsize
% \vspace{-1.0cm}
% \hspace*{-2cm}
\begin{tabular}{ccccccccccccccccccccc}
\toprule
       & \multicolumn{2}{c}{avg.}& \multicolumn{2}{c}{open}  & \multicolumn{2}{c}{slide}  & \multicolumn{2}{c}{sweep to} & \multicolumn{2}{c}{meat off} & \multicolumn{2}{c}{turn}  & \multicolumn{2}{c}{put in}  & \multicolumn{2}{c}{close}  & \multicolumn{2}{c}{drag}  & \multicolumn{2}{c}{stack}  \\
       & \multicolumn{2}{c}{SR$\uparrow$} & \multicolumn{2}{c}{drawer} & \multicolumn{2}{c}{block}  & \multicolumn{2}{c}{dustpan}  & \multicolumn{2}{c}{grill}    & \multicolumn{2}{c}{tap}   & \multicolumn{2}{c}{drawer}  & \multicolumn{2}{c}{jar}    & \multicolumn{2}{c}{stick} & \multicolumn{2}{c}{blocks} \\ \rule{0pt}{3ex} 
Method & 10 & 100     & 10          & 100          & 10           & 100         & 10           & 100           & 10           & 100           & 10          & 100         & 10           & 100          & 10           & 100         & 10          & 100         & 10           & 100         \\ \midrule
Ours-s  & 65 & 73 & 60        & 67           & 100          & 100         & 54           & 84            & 95           & 97            & 71          & 89          & 51           & 56           & 90           & 80          & 92          & 93          & 68           & 89            \\
RVT-s   & 62 & 61 & 98         & 94           & 89           & 93          & 60           & 40            & 77           & 96            & 89          & 96          & 37           & 42           & 81           & 84          & 99          & 95          & 25           & 31          \\
PerAct-s & 43 & 44 &89       & 93           & 100          & 100         & 1            & 0             & 98           & 98            & 83          & 77          & 19           & 28           & 56           & 73          & 21          & 30          & 85           & 59          \\
C2FARM-s & 35 & 44 &68     & 84           & 100          & 97          & 1            & 1             & 95           & 99            & 69          & 78          & 13           &  11          & 33           & 84          & 1           & 7           & 27           &  84         \\ 
% \midrule
% Act3D-m  & 49 &  65 & 92         & 93           & 66           & 93          & 82           & 92            & 58           & 94            & 64          & 94          & 82           & 90           & 90           & 92          & 52          & 92          & 6            & 12          \\ 
% RVT-m    & N/A & 63    &            & 71           &              & 82          &              & 72            &              & 88            &             & 94          &              & 88           &              & 52          &             & 99          &              & 29          \\
% PerAct-m & 30 & 43 & 68         & 80           & 32           & 72          & 72           & 56            & 68           & 84            & 72          & 80          & 16           & 68           & 32           & 60          & 36          & 68          & 12           & 36     \\
% C2FARM-m & 22 & 16 & 28       & 20           & 12           & 16          & 4            & 0             & 40           & 20            & 60          & 68           & 12          & 4            & 28           & 24          & 72          & 24          & 4             & 0          \\
\bottomrule \rule{0pt}{4ex} 
       & GPU & SGD & \multicolumn{2}{c}{screw} & \multicolumn{2}{c}{put in} & \multicolumn{2}{c}{place}    & \multicolumn{2}{c}{put in}   & \multicolumn{2}{c}{sort}  & \multicolumn{2}{c}{push}    & \multicolumn{2}{c}{insert} & \multicolumn{2}{c}{stack} & \multicolumn{2}{c}{place}  \\
       & mem $\downarrow$ & time $\downarrow$ & \multicolumn{2}{c}{bulb}   & \multicolumn{2}{c}{safe}   & \multicolumn{2}{c}{wine}     & \multicolumn{2}{c}{cupboard} & \multicolumn{2}{c}{shape} & \multicolumn{2}{c}{buttons} & \multicolumn{2}{c}{peg}    & \multicolumn{2}{c}{cups}  & \multicolumn{2}{c}{cups}   \\ \rule{0pt}{3ex} 
Method  &  &      & 10           & 100          & 10           & 100         & 10           & 100           & 10           & 100           & 10          & 100         & 10           & 100          & 10           & 100         & 10          & 100         & 10           & 100         \\ \midrule
Ours-s  & 11GB & 0.7s  & 55        & 63           & 64           & 98          & 95           & 99            & 75           & 88            & 4           & 5           & 100          & 100          & 4            & 5           & 68          & 87          & 15           & 8           \\
RVT-s  & 13GB & 0.5s & 46         & 42           & 77           & 70          & 95           & 94            & 77           & 81            & 2           & 4           & 100          & 100          & 24           & 14          & 43          & 37          & 0            & 0           \\
PerAct-s  & 13GB & 0.8s &32         & 35           & 65           & 29          & 1            & 9             & 2            & 25            & 3           & 4           & 100          & 100          & 12           & 16          & 12          & 7           & 0            & 0           \\
C2FARM-s  & 2GB & 0.1s &33       & 43           & 35           & 34          & 15           & 47            & 13           & 9             & 4           &  3          & 100          &  99          & 23           & 2           & 0           & 1           &  0           & 0           \\ 
% \midrule
% Act3D-m  &  &  & 26        & 47           & 75           & 95          & 32           & 80            & 27           & 51            & 7           & 8           & 98           & 99           & 7            & 27          & 8           & 9           & 1            & 3          \\ 
% RVT-m &  &   &                & 48           &              & 91          &              & 91            &              & 50            &             & 36          &              & 100          &              & 11          &             & 26          &              & 4           \\
% PerAct-m  &  &  & 28       & 24           & 16           & 44          & 20           & 12            & 0            & 16            & 16          & 20          & 56           & 48           & 4            & 0           & 0           & 0           & 0            & 0           \\
% C2FARM-m  &  &  & 12     & 8            & 0            & 12          & 36           & 8             & 4            & 0             & 8           & 8           & 88           & 72           & 0            &  4          & 0           & 0           & 0            & 0            \\
\bottomrule
\end{tabular}
\caption{\textbf{Success rate (\%) on RLBench.} We perform a comparison between our method and various baselines on 18 RLBench tasks. The ``-s'' suffix means the agent is trained on single-task, single-variation setups. The ``-m'' suffix means the agent is trained on multi-task, multi-variation setups. Training with $10$ demos, ours achieves a similar performance to the best baseline that is trained with $\times 10$ more demos. Training with $100$ demos, ours outperforms the best baseline by $>10\%$.}
\label{table:full_rlbench18}
\vspace{-0.5cm}
\end{table*}

\subsection{In-hand segmentation}
\label{sec:in_hand_segmentation}

Bi-equivariance assumes that the in-hand object is rigidly attached to the gripper, meaning that any gripper action will transform the in-hand object identically. However, the in-hand observation could contain distracting objects that are not grasped by the gripper, which can happen, for example, when the in-hand crop size is large, and the gripper is about to grasp or release an object.

To avoid this, we propose in-hand segmentation by adding an output channel to the $\query$ network, which predicts a mask $Q_\text{mask}$ to exclude distracting objects. The method is trained in a self-supervised manner, requiring no additional labels. We compute the ground truth in-hand segmentation mask $m$ based on the observation $(s, s_{ih}, \text{T}_\text{ee})$ at time $t$ and the observation $(s', s_{ih}', \text{T}_\text{ee}')$ at time $t+1$, as well as the gripper displacement $v = \text{T}_\text{ee}^{-1}\text{T}_\text{ee}'$,
\begin{align}
    m[x] &=
    \begin{cases}
        1 & \text{if }  x \in s_{ih} > s_{ih}' + v^{-1}(s_{ih} < s_{ih}') \\
        0 & \text{if } x \in s > s' + v(s < s') \\
        -1 & \text{elsewhere}
    \end{cases}
     \nonumber 
\end{align}
where $x$ is the $XYZ-$ coordinate of the voxel grid. We use the computed segmentation mask to train the segmentation network to predict an in-hand mask. This predicted mask is then applied to filter out the features of the distracting object by performing an element-wise dot product: $f_{ih} = Q_\text{mask} f'_{ih}$, where $f'_{ih}$ represents the embedded in-hand features from the outputs of the $\query$ Unet, see Figure.\ref{fig:in_hand_segmentation} for visualizations. This approach allows us to consistently use one large in-hand crop size across all experiments.

% The in-hand segmentation learns to mask out the features in $f'_{ih}$ that are not bi-equivariant, see Figure.\ref{fig:in_hand_segmentation} for visualizations. Notice that our method effectively segments the features that are not attached to the gripper. This improves the robustness of the bi-equivariant policy to noisy in-hand observations and allows us to use large in-hand crop sizes. In fact, we use the same in-hand crop size for all experiments.


\subsection{Bi-equivariant data augmentation}
\label{sec:data_aug}

Our method achieves discretized translational bi-equivariance through the 3D CNN backbones and approximates continuous translational and rotational bi-equivariance through bi-equivariant data augmentation. 
% The desired bi-equivariance of the method, based on Equation ~\ref{equ:bi_equ1}, \ref{equ:bi_equ2} is,
% \begin{align}
%      g_1\cdot Q_\text{T}(g_1\cdot s, g_2\cdot s_{ih})\cdot g_2^{-1} &= Q_\text{T}(s, s_{ih}), \nonumber\\
%      g_1 \in H_1,& g_2 \in H_2,
%      \label{equ:desired_bi_equ}
% \end{align}
% where the set of group elements $H_1 \subset \SE(3)$ is $[-0.15, 0.15]$m contains translations along the 3 axes and rotations of size $[-45^\circ, 45^\circ]$ along $Z-$ axis, the set of group elements $H_2 \subset \SE(3)$ has $[-0.02, 0.02]$m translations along each axis and rotations of size $[5^\circ, 5^\circ, 45^\circ]$ along the $X, Y$, or $Z$ axis. 
We augment each data point $(s, s_{ih}, a)$ in the mini-batch by applying random transformations, as described by Eqn.~\ref{equ:aug_bi_equ}, where $g_1, g_2 \in \SE(3)$ are randomly sampled,
\begin{align}
     (g_1&\cdot s, g_2\cdot s_{ih}, g_1ag_2^{-1}).
     \label{equ:aug_bi_equ}
\end{align}

\subsection{Implementation} \label{sec: implementation}
% The agent evaluates the pose action $a_\text{T}$ based on the scene voxel map $s$ and the in-hand voxel map $s_{ih}$. The in-hand voxel map $s_{ih}$ is a crop of $s$, as described in Section \ref{sec:place_only}. Besides, we include gripper pose information by adding constants to the in-hand voxel map. 
The action values $Q_\text{T}^l$ are calculated by coarse-to-fine action evaluation described in Section \ref{sec: cross_correlation}. Afterward, the agent evaluates both the gripper open action $a_\text{open}$ and the planner ignores collision action $a_\text{collide}$ using a multi-layer perceptron (MLP) based on latent features from the key network: $Q_\text{open, collide} = \text{mlp}\big(\text{maxpool}(\key(s)), \key(s)[a^*_\text{T}]\big)$, where $\text{maxpool}(\key(s))$ extracts features by maxpooling over the spatial dimension and $\key(s)[a^*_\text{T}]$ extracts features at the selected action location $a^*_\text{T}$. This MLP is similar to RVT \cite{rvt} except we do not use softmax over the feature.

The agent is trained on the expert demonstrations $\{o, a\}$ by minimizing the following loss,
\begin{align}
    \mathcal{L} = &D(Q_\text{open}, a_\text{open}) + D(Q_\text{collide}, a_\text{collide})  \nonumber\\
     & +\Sigma_{l=1}^3 D(Q_\text{T}^l, a_\text{T}^l) + \Sigma_{x}\mathbbm{1}_{m_x\ge 0}\big|\big|Q_{\text{mask}, x} - m_x\big|\big|^2_2 \nonumber
\end{align}
where $a_\text{T}^l$ is discretized expert pose action at level $l$, corresponding to the coarse-to-fine resolution $\hat{G}_l$. The indicator function $\mathbbm{1}$ equals $1$ when the mask $m_x\ge 0$, and $0$ otherwise. $Q_\text{mask}$ is the predicted in-hand segmentation mask, and $x$ refers to the position in the voxel map. We use cross-entropy loss $D$ to train action values $Q_{\{\text{T, open, collide}\}}$ and $l2$ loss $||\cdot||^2_2$ to train the in-hand segmentation mask $Q_\text{mask}$. Bi-equivariant data augmentation is applied to each sampled data point during training.
% and during evaluation the action is greedily selected: $a^* = \pi(o) = \arg\max{Q}$. For more details about training and evaluation, please see Appendix.\ref{app:data_aug}, \ref{app:train_eval}.

% We calculate the ground truth in-hand segmentation mask $m$ based on the observation $(s, s_{ih}, \text{T}_\text{ee})$ at time $t$ and the observation $(s', s_{ih}', \text{T}_\text{ee}')$ at time $t+1$ in the demonstration data, and gripper displacement $v = \text{T}_\text{ee}^{-1}\text{T}_\text{ee}'$,
% \begin{align}
%     m_x &=
%     \begin{cases}
%         1 & \text{if }  x \in s_{ih} > s_{ih}' + v^{-1}(s_{ih} < s_{ih}') \\
%         0 & \text{if } x \in s > s' + v(s < s') \\
%         -1 & \text{elsewhere}
%     \end{cases}
%      \nonumber 
% \end{align}
% where $x$ is the $XYZ-$ coordinate of the voxel grid.


\begin{figure}\centering
    % \includegraphics[width=0.48\textwidth]{figure/ours_and_rvt.png}
    
    \begin{subfigure}[b]{0.48\textwidth}\centering
    \includegraphics[width=\textwidth]{figure/simulation_tasks.png}
    \caption{}
    \end{subfigure}
    \begin{subfigure}[b]{0.22\textwidth}\centering
    \includegraphics[width=\textwidth]{figure/ours_vs_rvt_bi_equ_equ_tasks.png}
    \caption{}
    \end{subfigure}
    % \begin{subfigure}[b]{0.22\textwidth}\centering
    % \includegraphics[width=\textwidth]{figure/ours_precision_tasks.png}
    % \caption{}
    % \end{subfigure}
    \begin{subfigure}[b]{0.22\textwidth}\centering
    \includegraphics[width=\textwidth]{figure/ours_vs_rvt_bi_equ_equ_tasks_detail.png}
    \caption{}
    \end{subfigure}
    % \begin{subfigure}[b]{0.22\textwidth}\centering
    % \includegraphics[width=\textwidth]{figure/ours_precision_tasks_detail.png}
    % \caption{}
    % \end{subfigure}
    \caption{(a) shows 4 out of 18 RLBench tasks~\cite{james2020rlbench}. (b) when classifying 18 tasks by the equivariance, ours has advantages on bi-equivariant and mixed equivariance tasks but underperforms RVT on equivariant tasks. (c) ``Bi-equ.'': the top 5 tasks. ``Mix Bi-equ./Equ.'': the middle 9 tasks. ``Equ.'': the button 3 tasks.
    % (a) shows 4 out of 18 RLBench tasks~\cite{james2020rlbench}. (b) when classifying 18 tasks by the equivariance, ours has advantages on bi-equivariant and mixed equivariance tasks but underperforms RVT on equivariant tasks. (c) when sorting 18 tasks based on precision, ours learn proper policies for median- and low-precision tasks but perform badly in high-precision tasks, while constantly outperforming RVT. (d) ``Bi-equ.'': the top 5 tasks. ``Mix Bi-equ./Equ.'': the middle 9 tasks. ``Equ.'': the button 3 tasks. (e) Task performance, sorted by the Success Rate of a Noise-perturbed Expert planner (NESR). ``Low-precision'': top 6 tasks. ``Median-precision'': middle 6 tasks. ``High-precision'': bottom 6 tasks.
    }
    \vspace{-0.2cm}
    \label{fig:ours_vs_rvt}
\end{figure}

\section{Experiment}
% We evaluate our approach across diverse 3D manipulation tasks in both simulated environments and real-world settings, showcasing its ability to learn a state-of-the-art policy with a small number of robot data.


\textbf{Baselines.} 
We compare our method with strong Keyframe IL baselines. Notice that we do not compare with pick-place methods\cite{zeng2021transporter, Huang-RSS-22, ryu2023equivariant, huang2024fourier} because they can not solve all 18 RLBench tasks. E.g., ``screw\_bulb" does not belong to pick-place tasks. \textbf{C2FARMBC} Coarse-to-Fine Attention-driven Robotic Manipulation Behaviour Cloning \cite{james2022coarse, shridhar2023perceiver} is an imitation learning algorithm. The method maps voxel grid input into discretized translational actions in a coarse-to-fine scheme.
% While the translational action uses direct action mapping from the voxel using Unets\cite{Unet,3DUnet}, the other actions are the outputs of an MLP based on the features from the last level of Unets. 
% The coarse-to-fine inference effectively reduced computation. 
\textbf{PerAct} Perceiver Actor \cite{shridhar2023perceiver} and \textbf{RVT} Robot View Transformer \cite{rvt} utilize Transformer backbones to map observation into the values of translational actions, though destroying translation equivariance. Unlike PerAct which uses expensive voxel input, RVT utilizes multi-view projected images. 
% \textbf{Act3D:} Actor 3D \cite{gervet2023act3d} uses a Transformer backbone to map point-cloud observation to point-cloud of continuous translational action. 
C2FARMBC employs the coarse-to-fine method but is limited to translational actions. Moreover, all of these methods represent rotation action as discretized Euler angles, which suffer from discontinuity\cite{5D_SO3}. This rotation formulation only depends on the scene observation and does not incorporate the in-hand observation.
% The translational action is refined using coarse-to-fine point cloud levels. The other actions, orientation, gripper, and collision are inferred similar to C2FARMBC. The use of point clouds provides computation advantages and translation action accuracy advantages.
% \textbf{Act3D} Actor 3D \cite{gervet2023act3d} uses transformer backbone to map point-cloud observation to point-cloud of continuous translational action. The translational action is refined using coarse-to-fine point cloud levels. The other actions, orientation, gripper, and collision are inferred similar to C2FARMBC. The use of point clouds provides computation advantages and translation action accuracy advantages.
We train all the baselines using the same parameters from the open-sourced code, except that we train on single task setup and the iteration is reduced to 15k SGD steps.

\subsection{Simulation Tasks}


\textbf{Environment:} 
% We benchmark on RLBench \cite{james2020rlbench} based on Pyrep \cite{pyrep}, Vrep \cite{vrep}, and CoppelaSim \cite{coppeliaSim}. 
We focus on the 18 tasks on the RLBench\cite{james2020rlbench}, as shown in Figure \ref{fig:ours_vs_rvt}. We generate 10 or 100 episodes of training and 100 episodes of testing demonstrations for each task. All the baselines are trained with the same training data and are tested with the same testing scenes, restored from the 100 testing demonstrations. The demonstrations include $128\times128$ RBGD camera observations from the left shoulder, right shoulder, front, and wrist camera, the proprioceptive information $p$, and the expert action $a$.

\textbf{Tasks and success metrics:} The 18 tasks are the same as \cite{shridhar2023perceiver, rvt, gervet2023act3d}, except we uses single variation. These tasks cover a wide range of manipulation policies, that includes not only pick-place (stack\_cups, stack\_blocks), but also pushing/pulling (slide\_blocks, open\_drawer), and turning (screw\_bulb, turn\_tap), etc that the pick-place methods can not solve \cite{zeng2021transporter, Huang-RSS-22, ryu2023equivariant, huang2024fourier}. The total keyframe actions in one episode for these tasks range from 2 to 14 \cite{shridhar2023perceiver}. The metric for success is binary in $\{1, 0\}$ for success or failure. The task success depends on whether the goal state is reached within 25 steps in the RLBench simulator \cite{james2020rlbench}.


\textbf{Results:} Table \ref{table:full_rlbench18} compares ours with various baselines in the 18 RLBench tasks. To make a fair comparison, we include their performance of multi-task settings as reported in the paper~\cite{shridhar2023perceiver,rvt,gervet2023act3d}. Ours outperforms all the baselines in the average success rate when trained with 10 or 100 expert demonstrations. We further analyze the performance of our method in different task groups. We first classify the 18 tasks into three categories: ``Bi-equ. tasks'': when the task mainly contains bi-equivariant actions, e.g., stack\_cups. ``Equ. tasks'': when the task only contains equivariant actions, e.g., open\_drawer. ``Mix Bi-equ./Equ. tasks'': when the task has multiple actions that contain both, e.g., put\_in\_drawer. As is shown in Figure \ref{fig:ours_vs_rvt} (b) Our method outperforms RVT in ``Bi-equ.'' and ``Mix Bi-equ./Equ.'' tasks while underperforms RVT in ``Equ.'' tasks. This indicates that our method can effectively leverage the bi-equivariant property in the task. 
% We also sort the precision requirement of each task into ``Low-'', ``Median-'', ``High-''. As is shown in Figure \ref{fig:ours_vs_rvt} (b) Our method prevails in median-low precision tasks but suffers from high precision tasks. We hypothesize this is due to the discretization error of the voxel grid. 
% For details on task type clustering standards and analysis, please see the Appendix.\ref{app:rlbench18_analysis}. 

\begin{figure}[!t]
    \centering
    \includegraphics[width=0.49\textwidth]{figure/realworld_tasks.png}
    \caption{\textbf{Real world tasks.} \textbf{Left:} The first row shows a snapshot of 4 tasks, and the second row shows the distributions of the initial state. golf\_swing requires picking the club and aligning its head with the golf then pushing the ball to touch the goal. flip\_steak requires grasping and flipping the steak. slide\_plate requires picking the plate and then reorientating it to slide into the rack. insert\_toothpaste requires grasping the toothpaste and inserting it into the mug. \textbf{Right:} While our method is aware of the club head, RVT is not.}
    \vspace{-0.2cm}
    \label{fig:real_world_tasks}
\end{figure}

\begin{figure*}
    \centering
    \begin{minipage}{0.48\textwidth}
        \centering
        \includegraphics[width=\linewidth, trim=0 20 0 20, clip]{figures/splits_real/real_9060000_20250127-164140.pdf}
    \end{minipage}%
    \begin{minipage}{0.48\textwidth}
        \centering
        \includegraphics[width=\linewidth, trim=0 20 0 20, clip]{figures/fig8_real/real_8460000_20250127-145150.pdf}
    \end{minipage}%
    \caption{Real world flight trajectories with S5WM: \textit{State-based Split-S} task (left) and \textit{Vision-based Figure-8} task (right).}
    \label{fig:real_world}
\end{figure*}



\begin{table*}[t]
\centering
\caption{Full results of ablation test in forecasting tasks. Experiments are conducted for varying prediction lengths, includes $T\in \{96, 192, 336, 720\}$ time points.}
\label{table: full ablation forecasting}
\small  % 调整字体大小 small/scriptsize/footnotesize
\setlength{\tabcolsep}{4pt} % 调整列间距
\renewcommand{\arraystretch}{0.8} % 调整行间距
\begin{tabular}{c|c|cc|cc|cc}
\toprule
Models &  & \multicolumn{2}{c|}{GTM} & \multicolumn{2}{c|}{GTM w/o time\_gran.} & \multicolumn{2}{c|}{GTM w/o Freq.} \\ \midrule
dataset & pred\_len & MSE & MAE & MSE & MAE & MSE & MAE  \\ \midrule

 & 96 & \textbf{0.360} & \textbf{0.398} & {\color[HTML]{000000} {\ul 0.372}} & {\color[HTML]{000000} {\ul 0.406}} & 0.384 & 0.416 \\
 & 192 & \textbf{0.397} & \textbf{0.422} & {\color[HTML]{000000} {\ul 0.405}} & {\color[HTML]{000000} {\ul 0.427}} & 0.408 & 0.429 \\
ETTh1 & 336 & \textbf{0.420} & \textbf{0.437} & {\color[HTML]{000000} {\ul 0.428}} & {\color[HTML]{000000} {\ul 0.437}} & 0.433 & 0.443 \\
 & 720 & \textbf{0.438} & \textbf{0.457} & {\color[HTML]{000000} {\ul 0.450}} & {\color[HTML]{000000} {\ul 0.463}} & 0.449 & 0.466 \\
 & AVG & \textbf{0.404(3.57\%, 2.42\%)} & \textbf{0.429(2.28\%,0.92\%)} & {\color[HTML]{000000} {\ul 0.414(1.19\%)}} & {\color[HTML]{000000} {\ul 0.433(1.37\%)}} & 0.419 & 0.439 \\\midrule
 
 & 96 & \textbf{0.282} & \textbf{0.341} & {\color[HTML]{000000} {\ul 0.299}} & {\color[HTML]{000000} {\ul 0.353}} & 0.301 & 0.354 \\
 & 192 & \textbf{0.325} & \textbf{0.366} & {\color[HTML]{000000} {\ul 0.334}} & {\color[HTML]{000000} {\ul 0.372}} & 0.335 & 0.375 \\
ETTm1 & 336 & \textbf{0.353} & \textbf{0.385} & {\color[HTML]{000000} {\ul 0.360}} & {\color[HTML]{000000} {\ul 0.391}} & 0.363 & 0.393 \\
 & 720 & \textbf{0.396} & \textbf{0.410} & {\color[HTML]{000000} {\ul 0.398}} & {\color[HTML]{000000} {\ul 0.411}} & 0.398 & 0.412 \\
 & AVG & \textbf{0.339(2.87\%,2.59\%)} & \textbf{0.376(2.08\%,1.57\%)} & {\color[HTML]{000000} {\ul 0.348(0.29\%)}} & {\color[HTML]{000000} {\ul 0.382(0.52\%)}} & 0.349 & 0.384 \\\midrule
 
 & 96 & \textbf{0.147} & \textbf{0.197} & {\ul 0.153} & {\ul 0.217} & 0.158 & 0.212 \\
 & 192 & \textbf{0.192} & \textbf{0.241} & {\ul 0.206} & {\ul 0.254} & 0.208 & 0.258 \\
weather & 336 & \textbf{0.250} & \textbf{0.291} & {\ul 0.252} & {\ul 0.293} & 0.256 & 0.297 \\
 & 720 & \textbf{0.310} & \textbf{0.334} & {\ul 0.311} & {\ul 0.335} & 0.313 & 0.337 \\
 & AVG & \textbf{0.225(3.43\%,2.60\%)} & \textbf{0.266(3.62\%,3.27\%)} & {\color[HTML]{000000} {\ul 0.231(0.86\%)}} & {\color[HTML]{000000} {\ul 0.275(0.36\%)}} & 0.233 & 0.276 \\\midrule
 
 & 96 & \textbf{0.351} & \textbf{0.250} & {\ul 0.355} & {\ul 0.253} & 0.359 & 0.256 \\
 & 192 & \textbf{0.373} & \textbf{0.260} & {\ul 0.374} & {\ul 0.262} & 0.379 & 0.264 \\
traffic & 336 & \textbf{0.388} & \textbf{0.267} & {\ul 0.389} & {\ul 0.270} & 0.393 & 0.271 \\
 & 720 & \textbf{0.428} & \textbf{0.288} & {\ul 0.431} & {\ul 0.291} & 0.435 & 0.293 \\
 & AVG & \textbf{0.385(1.79\%,0.52\%)} & \textbf{0.266(1.85\%,1.12\%)} & {\color[HTML]{000000} {\ul 0.387(1.28\%)}} & {\color[HTML]{000000} {\ul 0.269(0.74\%)}} & 0.392 & 0.271 \\\midrule
 
 & 96 & \textbf{0.131} & \textbf{0.225} & {\ul 0.132} & {\ul 0.226} & 0.134 & 0.227 \\
 & 192 & \textbf{0.149} & \textbf{0.243} & {\ul 0.150} & {\ul 0.246} & 0.152 & 0.248 \\
Electricity & 336 & \textbf{0.166} & \textbf{0.259} & {\ul 0.168} & {\ul 0.262} & 0.169 & 0.264 \\
 & 720 & \textbf{0.201} & \textbf{0.292} & {\ul 0.202} & {\ul 0.295} & 0.205 & 0.296 \\
 & AVG & \textbf{0.161(2.42\%,1.23\%)} & \textbf{0.254(1.93\%,1.17\%)} & {\color[HTML]{000000} {\ul \textbf{0.163(1.21\%)}}} & {\color[HTML]{000000} {\ul \textbf{0.257(0.77\%)}}} & 0.165 & 0.259\\
         \bottomrule
\end{tabular}

\end{table*}

\subsection{Real-World Tasks}

In this section, we compare our method with RVT~\cite{rvt} in 4 complex real-world tasks (shown in Figure \ref{fig:real_world_tasks}). RVT is the best baseline in 18 RLBench tasks in Table \ref{table:full_rlbench18} when trained in the low data regimen (10 training demonstrations). The real-world tasks differ from simulation in 1) multimodal demonstrations \cite{LSTM_GMM}, 2) noisy observations \cite{zhou2024dynpoint}, and 3) limited demonstrations.


\textbf{Robot platform:} we set up the robot platform with a 6 DoFs UR5 manipulator, a Robotiq 85 gripper. The observation $\text{T}_\text{ee}, s_\text{open}$ comes from the manipulator and the gripper sensors, while the scene observation $s$ is reconstructed voxel grid from the front, the left, and the right RealSense D455 cameras. The pose action $a_\text{T}$ specifies the target pose for an off-the-shelf planner, i.e., MoveIt~\cite{moveit} motion planner with RRT-connect algorithm~\cite{rrtconnect}. We do not use the collision action $a_\text{collide}$. The Robot Operating System (ROS) is used for communication, and the workstation is equipped with a 12GB memory 2080Ti GPU. The demo is collected using an HTC VIVE controller \cite{shridhar2023perceiver}.


\textbf{Training and evaluation metrics:} For training, we first collect 10 demos for each task using the robot platform, then train our method and RVT with 15k SGD steps. The same hyper-parameters as simulations are used, except the size of the workspace is adjusted according to the robot platform. We do not cherry picking and test the last model checkpoint. We evaluate each baseline with 20 episodes. Each episode is initialized with randomized object orientation and location within the workspace, then the initialization is recorded by the cameras. We minimize the initial state between different baselines by restoring the scene to the recorded images. A task is considered a success when the success metric is achieved within 10 steps.

\textbf{Results:} Table~\ref{table:real-world} shows the evaluation results. We report two success rates, w/ means the overall success rate, and w/o means the success rate when removing the episodes that have planner failure. Our method significantly outperforms the baselines in all evaluation metrics and all 4 tasks. Training with as few as 10 demos, our method exhibits the ability to compensate for the changes of the in-hand object, e.g., correctly using the club head to hit the golf when the gripper could grasp the club with two orientations with $180^\circ$ angle (last column of Figure \ref{fig:real_world_tasks}). In contrast, RVT infers actions ignoring the in-hand state, e.g., occasionally hits the golf by the grip. We also find that the motion planner failure accounts for $10\%$ task failure. We believe this is orthogonal to our method and a better motion planner\cite{curobo_report23, zhang2019two} could effectively address this issue.


\subsection{Ablation Study}

In this section, we ablate each piece of the method to demonstrate its importance. We compare the performance on 9 RLBench tasks. All the baselines are trained with 100 demonstrations and tested with 100 episodes.

\footnotetext{The rotation grid and the in-hand size are reduced to Healpix1 and $16^3$ to match the computation overhead with ours that uses Healpix3 and $32^3$.\label{foot:changes}}

\textbf{Baselines:} \underline{no coarse-to-fine} ablates the coarse-to-fine action evaluation in Section \ref{sec: cross_correlation} by using only one level of cross-correlation instead of 3
% . In-hand voxel size and the rotation grid are reduced to match the GPU overhead with ours
\textsuperscript{\ref{foot:changes}}. \underline{no cross, C2F, seg} ablates the bi-equivariance of ours by removing the coarse-to-fine evaluation (Section \ref{sec: cross_correlation}), the cross-correlation (Section \ref{sec:place_only}), and the in-hand segmentation (Section \ref{sec:in_hand_segmentation}). This baseline only uses the $\key$ Unet with the same translation resolution as ours, and discretized Euler angles, which is identical to 1 level C2FARMBC. \underline{no segmentation} ablates the in-hand segmentation (Section \ref{sec:in_hand_segmentation}) by using the output of $\query$ Unet $f_{ih}'$ without the mask $Q_{mask}$. \underline{no augmentation} ablates the bi-equivariance data augmentation (Section \ref{sec:data_aug}) through training with the raw data.

\textbf{Results:} Table \ref{table:full_ablation} shows the results on the 9 tasks. When the coarse-to-fine evaluation is removed in \underline{no coarse-to-fine}, the performance is dropped by $57\%$ and the training time is tripled. This indicates that perhaps the most important piece of our method is the coarse-to-fine action inference. \underline{no cross, C2F, seg} ablates the bi-equivariance structure leading to $>40\%$ performance drop. \underline{no augmentation} shows the bi-equivariant data augmentation contributes to $6\%$ performance increment. This indicates both the bi-equivariance neural network architecture and data augmentation improve performance, while the proposed neural network architecture plays a more important role. \underline{no segmentation} shows that removing in-hand segmentation causes performance drops and a large variance, which indicates the necessity to mask out the distractors.

\section{Conclusion and Limitations}

In this paper, we propose the Coarse-to-fine 3D Keyframe Transporter that leverages the rich geometric structure in the $\SE(3)$ policy and achieves high success rates. We begin by analyzing bi-equivariance in the Keyframe IL, then introducing a $3$D cross-correlation architecture that embeds this geometric structure. Additionally, we proposed a novel coarse-to-fine evaluation to 
% decouple the large $\SE(3)$ action space thus 
significantly reduce computing. Simulation experiments show that our model outperforms multiple strong baselines on 18 RLBench tasks and the physical experiments demonstrate the method can effectively learn from a few demonstrations and generalize to random initial scenes.

One limitation of our framework is the aliasing effect \cite{cesa2022a,e2cnn} of using discretized voxel features, which impacts the stability of our dynamic filter and the performance on high-precision tasks.
This issue could be mitigated by using irreducibal representations \cite{e2cnn,cesa2022a,huang2024fourier} or using point-cloud-based features \cite{qi2017pointnet++,ryu2023diffusionedfs,gervet2023act3d}. 
Another limitation is the keyframe action does not provide fine-grained control of the trajectory.
This could be addressed by using an engineered trajectory controller \cite{curobo_report23, zhang2019two}, or by combining keyframe action with closed-loop controllers \cite{xian2023chaineddiffuser, ma2024hierarchical}.


%%%%%%%%%%%%%%%%%%%%%%%%%%%%%%%%%%%%%%%%%%%%%%%%%%%%%%%%%%%%%%%%%%%%%%%%%%%%%%%%


% \addtolength{\textheight}{-12cm}   % This command serves to balance the column lengths
                                  % on the last page of the document manually. It shortens
                                  % the textheight of the last page by a suitable amount.
                                  % This command does not take effect until the next page
                                  % so it should come on the page before the last. Make
                                  % sure that you do not shorten the textheight too much.

%%%%%%%%%%%%%%%%%%%%%%%%%%%%%%%%%%%%%%%%%%%%%%%%%%%%%%%%%%%%%%%%%%%%%%%%%%%%%%%%



%%%%%%%%%%%%%%%%%%%%%%%%%%%%%%%%%%%%%%%%%%%%%%%%%%%%%%%%%%%%%%%%%%%%%%%%%%%%%%%%



%%%%%%%%%%%%%%%%%%%%%%%%%%%%%%%%%%%%%%%%%%%%%%%%%%%%%%%%%%%%%%%%%%%%%%%%%%%%%%%%
% \section*{APPENDIX}


% \section*{ACKNOWLEDGMENT}


%%%%%%%%%%%%%%%%%%%%%%%%%%%%%%%%%%%%%%%%%%%%%%%%%%%%%%%%%%%%%%%%%%%%%%%%%%%%%%%%
\newpage
\bibliographystyle{IEEEtran}
\bibliography{main}
% \documentclass{MITstyle}

%\usepackage[table]{xcolor}
\usepackage{chngcntr}
\usepackage{hyperref}
\usepackage{microtype}

\title{A Lightweight and Extensible Cell Segmentation and Classification Model for Whole Slide Images}

\author{Nikita Shvetsov~$^{1, }$\footnote{Correspondence e-mail: nikita.shvetsov@uit.no}, Thomas K. Kilvaer~$^{2, 3}$, Masoud Tafavvoghi~$^{4}$, Anders Sildnes~$^{1}$, \\ Kajsa Møllersen~$^{4}$, Lill-Tove Rasmussen Busund~$^{5, 6}$, Lars Ailo Bongo~$^{1}$ \\
%
\vspace{1em} % Space between authors and afilliations
%
\normalfont{\small $^{1}$Department of Computer Science, UiT The Arctic University of Norway}\\
\normalfont{\small $^{2}$Department of Oncology, University Hospital of North Norway}\\
\normalfont{\small $^{3}$Department of Clinical Medicine, UiT The Arctic University of Norway}\\
\normalfont{\small $^{4}$Department of Community Medicine, UiT The Arctic University of Norway}\\
\normalfont{\small $^{5}$Department of Medical Biology, UiT The Arctic University of Norway} \\
\normalfont{\small $^{6}$Department of Clinical Pathology, University Hospital of North Norway} %\vspace{2em}
}

\begin{document}
\maketitle

\section*{Abstract}

% \begin{abstract}
% Developing clinically useful cell-level analysis tools in digital pathology remains challenging due to limitations in dataset granularity, inconsistent annotations, computational demands of advanced models, and difficulties in integrating new technologies into clinical workflows. To address these challenges, we propose a multi-faceted solution that enhances data quality, model performance, and usability to create a lightweight and extensible cell segmentation and classification model.

% First, we update data labels by employing a cross-relabeling process that refines the labels of two existing datasets, PanNuke and MoNuSAC, to create a new unified dataset with enhanced granularity, encompassing seven distinct cell types. Second, we leverage the H-Optimus foundation model as a fixed encoder to improve feature representation for simultaneous cell segmentation and classification tasks. Third, to address the computational demands of foundation models, we employ knowledge distillation to reduce model size and complexity while maintaining comparable performance. Finally, to facilitate integration into clinical workflows, we integrate the distilled model into the QuPath software, a widely used open-source platform in digital pathology.

% Our results demonstrate improvements in cell segmentation and classification performance using the H‑Optimus-based model compared to a CNN-based model. Specifically, the average $R^2$ improved from 0.575 to 0.871, and the average $PQ$ score improved from 0.450 to 0.492, indicating better alignment with actual cell counts and enhanced segmentation and classification quality. Furthermore, the distilled student model maintains performance comparable to the larger foundation model while reducing the parameter count by a factor of 48.
% Overall, by reducing computational complexity and integrating it into existing workflows, the proposed approach may significantly impact diagnostic processes, reduce the workload of pathologists, and contribute to improved patient outcomes. Though our approach shows potential enhancements in efficiency and usability of cell segmentation and classification models in digital pathology, extensive validation is needed to deploy these models in clinical practice.
% \end{abstract}

%%% shortened abstract
\begin{abstract}
Developing clinically useful cell-level analysis tools in digital pathology remains challenging due to limitations in dataset granularity, inconsistent annotations, high computational demands, and difficulties integrating new technologies into workflows. To address these issues, we propose a solution that enhances data quality, model performance, and usability by creating a lightweight, extensible cell segmentation and classification model. 

First, we update data labels through cross-relabeling to refine annotations of PanNuke and MoNuSAC, producing a unified dataset with seven distinct cell types. Second, we leverage the H-Optimus foundation model as a fixed encoder to improve feature representation for simultaneous segmentation and classification tasks. Third, to address foundation models' computational demands, we distill knowledge to reduce model size and complexity while maintaining comparable performance. Finally, we integrate the distilled model into QuPath, a widely used open-source digital pathology platform. 

Results demonstrate improved segmentation and classification performance using the H-Optimus-based model compared to a CNN-based model. Specifically, average $R^2$ improved from 0.575 to 0.871, and average $PQ$ score improved from 0.450 to 0.492, indicating better alignment with actual cell counts and enhanced segmentation quality. The distilled model maintains comparable performance while reducing parameter count by a factor of 48. By reducing computational complexity and integrating into workflows, this approach may significantly impact diagnostics, reduce pathologist workload, and improve outcomes. Although the method shows promise, extensive validation is necessary prior to clinical deployment.
\end{abstract}
\clearpage

\section{Introduction}
In digital pathology, accurate segmentation and classification of cells are crucial for many diagnostic, prognostic, and predictive analyses \cite{Jaber_Beziaeva_etal._2019,Lin_Pan_etal._2022,Park_Ock_etal._2022,Shen_Choi_etal._2024}. Nowadays, developments in computational pathology offer multiple solutions \cite{H._Qu_P._Wu_etal._2020,Javed_Mahmood_etal._2020} to utilize cell-level datasets to train machine learning models that solve these problems. The quality and specificity of training datasets are critical for robust and accurate models. Adhering to the principle of "garbage in, garbage out", it is essential to ensure that these datasets are extensively and accurately labeled with distinct classes that reflect the diverse biological characteristics of different cell types. Unfortunately, the number of open-source datasets comprising such high-quality annotations is limited. Existing cell segmentation datasets \cite{Gamper_Koohbanani_etal._2019,Graham_Vu_etal._2019,Verma_Kumar_etal._2021} may offer extensive annotations for certain cell types while providing more general labels for others. For example, in PanNuke, which is one of the largest open-source datasets comprising labeled cells, various types of morphologically and functionally different inflammatory cells like macrophages and lymphocytes are clustered in a broad "inflammatory" class. Consequently, these classes are frequently omitted from analyses or aggregated into broader meta-classes \cite{Gamper_Koohbanani_etal._2020} and likely interfere with other cell classes included in the dataset. This and similar inconsistencies in annotation granularity limit the ability of machine learning models to learn the comprehensive and nuanced features necessary for accurate cell segmentation and classification. To address these challenges, methods for refining and standardizing dataset annotations are essential to enhance the quality of training data.

A complementary approach to mitigate the absence of high-quality training data is the use of foundation models. Foundation models as encoders are defined as large-scale, versatile networks pre-trained on vast, diverse datasets using self-supervised learning, contrasting with convolutional neural network (CNN) pre-trained encoders that rely on supervised learning with labeled data. In practice, foundation models leverage enormous amounts of weakly or unlabeled data from millions of whole slide images (WSIs) and employ self-attention mechanisms to capture long-range dependencies and global context \cite{Chen_Ding_etal._2024,Saillard_Jenatton_etal._2024,Vorontsov_Bozkurt_etal._2024,Xu_Usuyama_etal._2024}. As a consequence, foundation models are able to produce transferable feature representations across different cell types and tissue environments. The feature representations can be leveraged by decoder networks to produce segmentation masks and pixel-level classifications. Because foundation models have comprehensive feature representations, they can be effectively fine-tuned using much smaller amounts of cell-level data compared to the large datasets needed to train models from scratch. Furthermore, foundation models incorporate adversarial training elements or contrastive learning \cite{Chen_Ding_etal._2024,Xu_Usuyama_etal._2024}, enhancing their resilience and adaptability by exposing them to challenging and varied scenarios during training. This may result in more generalizable models, often making them well-suited for diverse and complex tasks in digital pathology.

Despite the inherent advantages of foundation models, their deployment for practical use faces its own obstacles. In particular, they require substantial computational power, financial investments and rigorous testing to ensure reliability and efficacy for a given task \cite{Akkus_Dangott_etal._2022,Dragomir_Cocuz_etal._2022,Go_2022,Jafri_Farooqui_etal._2024}. Moreover, while foundation models enhance feature representation and performance, they depend on the quality of available annotations for decoder fine-tuning and, like any other model, cannot resolve existing inconsistencies or ambiguities in data labels. Therefore, there remains a critical need for solutions that address both data quality and practical deployment considerations.
Further, integrating new technologies into existing clinical workflows often encounters resistance, as it necessitates adjustments to established diagnostic processes. So, there is a need to develop solutions that could be integrated into current practices, minimizing the burden on medical professionals to adopt new tools \cite{King_Williams_etal._2023}.

Existing solutions \cite{Goldsborough_Philps_etal._2024,Hörst_Rempe_etal._2024}, while addressing some aspects of these challenges, fall short in providing a comprehensive approach. To address the data quality and clinical deployment issues, we propose a multi-faceted solution that encompasses data refinement, model optimization, and integration with existing pathology tools (\hyperref[fig:fig1]{Figure 1}). The outcome is a lightweight cell segmentation and classification model that can be integrated into digital pathology workflows for practical clinical use.

\begin{figure}[h!]
    \centering
    \includegraphics[width=\textwidth, height=0.82\textheight, keepaspectratio]{images/Figure_1.pdf}
    \caption{Overview of the proposed solution, including 1) Data refinement using cross-relabeling, 2) Teacher model development and fine tuning, 3) Student model optimization with knowledge distillation and 4) Student model and QuPath integration}
    \label{fig:fig1}
\end{figure}
\clearpage

Our approach begins with preparing the data for the fine-tuning and training of the machine learning models. We create a refined dataset, acquired via cross-relabeling two cell-level datasets, enhancing annotation specificity and consistency of the labeled data. Subsequently, we create a cell segmentation and classification model based on the foundation model. We leverage the foundation model as a fixed encoder and fine-tune a decoder using the refined dataset to improve generalization across diverse tissue- and cell types.
To ensure that the model remains lightweight and deployable in a possibly resource-constrained environment, we employ knowledge distillation to approximate the functionality of the foundation model. Finally, to facilitate the practical application of our model in digital pathology workflows, we integrate it with the QuPath \cite{Bankhead_Loughrey_etal._2017} application. Each methodological component contributes to the overarching goal of enhancing model performance, generalizability, and usability in clinical settings.

The primary contributions of this paper are:
\begin{enumerate}
    \item \textit{Data labels refinement through cross-relabeling:}
    
    We propose a new method for refining labels of cell-level datasets through cross-relabeling. This method employs classification models to re-label broad and ambiguous instances, resulting in a more diverse dataset. Our evaluation demonstrates that these classification models achieve high accuracy on test subsets, indicating the reliability of the method for label refinement.

    \item \textit{Enhanced model performance via foundation models:}
    
    We employ a foundation model as a feature extractor for the cell segmentation and classification task. In comparison with training a CNN model from scratch, the foundation model backbone only needs fine-tuning, which significantly reduces training time, computational resources and data requirements. We show that using a foundation model encoder leads to better performance in cell segmentation and classification networks than using a CNN-based encoder. This improvement may enable the model to generalize more effectively across various tissue types and imaging methods.
    
    \item \textit{Model optimization through knowledge distillation:}
    
    We show that a smaller student model trained using knowledge distillation on the refined dataset obtained via our cross-relabeling approach from a foundation model achieves comparable performance in cell segmentation and quantification tasks. As a result, this model is more suitable for deployment in environments without high-performance computing resources.
    
    \item \textit{Integration with QuPath:}
    
    We integrate the distilled cell segmentation and classification model into QuPath, a widely used open-source digital pathology platform, to accelerate clinical adaptation by enabling pathologists to more easily incorporate advanced computational tools into their existing workflows.
\end{enumerate}

Through these methodological steps, we aim to bridge the gap between advanced machine learning techniques and practical clinical applications, making accurate and efficient digital pathology accessible in a broader range of healthcare settings.

\section{Refining Existing Datasets Using Cross-Relabeling}
To address the limitations of sparse and ambiguous labeling of cell-level datasets, we propose a generalizable cross-relabeling strategy that can be applied to any dataset containing broadly categorized or imprecisely labeled cell types. This approach involves training and subsequently leveraging classification models to refine broad categories into more specific or biologically relevant classes.
When applied to cell-level data, the methodology includes extracting individual cell images from the dataset patches, preprocessing these images to standardize the size and accommodate partial cells, and then training deep learning classifiers capable of distinguishing between the finer cell subtypes within the coarser categories. 
To illustrate our approach, we focus on the PanNuke \cite{Gamper_Koohbanani_etal._2020, Gamper_Koohbanani_etal._2019} and MoNuSAC \cite{Verma_Kumar_etal._2021} datasets that we have used to train models for cell quantification in our previous works \cite{Shvetsov_Grønnesby_etal._2022,Shvetsov_Sildnes_etal._2024}. We find that for better cell differentiation we have to introduce more granular labels. PanNuke includes a broad classification of "inflammatory" cells, encompassing lymphocytes, macrophages, and neutrophils. Each cell type differs significantly in structure, function, and clinical relevance. Conversely, MoNuSAC uses the label "epithelial" for a class that comprises both benign epithelial cells and malignant neoplastic cells. This practice makes it challenging to differentiate between benign and malignant epithelial cells in the dataset, which is a critical distinction when identifying tumor areas within tissue samples. To address these issues, we implement a cross-relabeling strategy as shown in \hyperref[fig:fig2]{Figure 2}. The key components are two classification models: one is trained on singular cell images from PanNuke data to classify the epithelial meta-class into epithelial and neoplastic classes. The other is trained on MoNuSAC to refine the inflammatory class into lymphocytes, neutrophils, and macrophages.

\begin{figure}[h!]
    \centering
    \includegraphics[width=\textwidth]{images/Figure_2.pdf}
    \caption{Refined dataset generation via cross relabeling}
    \label{fig:fig2}
\end{figure}

The refining approach consists of three consecutive steps. The first is the preprocessing step, in which we extract individual cells from both datasets (\hyperref[fig:fig3]{Figure 3}). The specifics of PanNuke and MoNuSAC patch preparation before cell preprocessing are provided in \hyperref[chap:S1]{Appendix S1}.

\begin{figure}[h!]
    \centering
    \includegraphics[width=\textwidth]{images/Figure_3.pdf}
    \caption{Cell instances preprocessing including (1) cell map extraction, (2) bounding box delineation, (3) adjusting cell boxes and (4) cropping and resizing of cell images}
    \label{fig:fig3}
\end{figure}

During preprocessing, we extract cell type maps from the ground truth label mask and calculate bounding boxes around each cell instance. To accommodate partial cells at patch borders, a common issue in cropped patch images, we employ mirror padding and extend the field of view of the cell label by 15 pixels to capture adjacent cells. We then crop and resize the identified regions to $64 \times 64$ pixels using bicubic interpolation.

The preprocessed PanNuke dataset comprises 68,031 neoplastic and 23,207 epithelial cell images, while MoNuSAC comprises  33,104 lymphocytes, 1,252 neutrophils, and 1,695 macrophages, which we subsequently use in training cell classification models and classifying the cell image data \hyperref[fig:S2]{Appendix Figure S2 (1)}. 

The next step is to train two distinct ResNet50-based classifiers tailored to address the specific labeling challenges inherent in each dataset. We use ResNet50 for classification models due to its proven effectiveness for image classification tasks in histopathology \cite{pan2022reviewmachinelearningapproaches}, and its compatibility with small images. For the PanNuke dataset, we design the classifier, trained on MoNuSAC data, to disaggregate the heterogeneous "inflammatory" cell category into distinct subtypes: lymphocytes, macrophages, and neutrophils. Similarly, for the MoNuSAC dataset, the classifier is trained on PanNuke data and distinguishes between benign and malignant epithelial cells within the overarching "epithelial" label. By applying these targeted classifiers to their respective datasets, we assign more specific labels to individual cell instances, thus enabling us to create a unified dataset.
To ensure a balanced representation of classes, we train both models on datasets that had been equalized to match the size of the least represented class. Thus, we obtain datasets comprising 23,207 samples per class for PanNuke and 1,252 samples per class for MoNuSAC data. Next, we partition both of them into training (70\%), validation (20\%), and testing (10\%) subsets. To mitigate the risk of overfitting, we use a single dropout layer with a rate of p=0.5 in both models and data augmentation using randomized color perturbations, rotation, and horizontal and vertical flipping. We employ AdamW optimizer and the cross-entropy loss function for the training criterion.

To evaluate the two trained models, we measure the classification accuracy on the respective test subsets. The accuracies on the test subset for both classifiers are presented in \hyperref[tab:1]{Table 1}. The PanNuke model achieves an average accuracy of 93.57\%, with higher accuracy for neoplastic cells (96.06\%) compared to epithelial cells (86.26\%). The confusion matrix in Figure A3.1 shows that the model predominantly distinguishes accurately between epithelial and neoplastic tissues, with a substantial number of correct classifications and relatively few misclassifications. The MoNuSAC model demonstrates an average accuracy of 98.92\%, excelling in classifying lymphocytes (99.67\%) and macrophages (94.12\%), with lower performance for neutrophils (85.71\%). The confusion matrix in Figure A3.2 shows that the model identifies lymphocytes and performs reasonably well with macrophages and neutrophils.

\begin{table}[h!]
\renewcommand{\arraystretch}{1.5}
  \centering
  \caption{Cell classification results for PanNuke and MoNuSAC trained models (CI 95\%).}
  \label{tab:1}
  \begin{tabular}{|l|c|c|}
   \hline
   %\rowcolor{gray!30}
    Accuracy               & PanNuke model              & MoNuSAC model              \\
    \hline
    Average      & 0.936 (0.931--0.941)         & 0.989 (0.986--0.993)        \\
    \hline
    Neoplastic   & 0.961 (0.956--0.965)         & -                          \\
    \hline
    Epithelial   & 0.863 (0.849--0.877)         & -                          \\
    \hline
    Lymphocytes  & -                          & 0.997 (0.995--0.999)        \\
    \hline
    Neutrophils  & -                          & 0.857 (0.796--0.918)        \\
    \hline
    Macrophages  & -                          & 0.941 (0.906--0.976)        \\
    \hline
  \end{tabular}
\end{table}

Finally, during the last step, we use the model trained on PanNuke data for epithelial cells in MoNuSAC and the model trained on MoNuSAC for the inflammatory cells class in PanNuke. Specifically, we use classifier models to relabel epithelial cells in MoNuSAC and inflammatory cells in PanNuke data. Then we combine cells with refined labels and the rest of the cells in both datasets to create a refined dataset (\hyperref[fig:S2]{Appendix Figure S2 (2)}). The process of relabeling cells and visualizing them on a patch is shown in \hyperref[fig:fig4]{Figure 4}. The cell counts in the refined dataset are provided in \hyperref[tab:S4]{Appendix Table S4}.

\begin{figure}[h!]
    \centering
    \includegraphics[width=\textwidth, height=0.42\textheight, keepaspectratio]{images/Figure_4.pdf}
    \caption{Cell relabeling procedure for epithelial and inflammatory cell classes}
    \label{fig:fig4}
\end{figure}

%\hfill

Relabeling and combining datasets have been explored in a prior study \cite{Parulekar_Kanwat_etal._2023}, where consecutive fine-tuning on multiple datasets was employed to account for hierarchical class label structures. While the method presented in \cite{Parulekar_Kanwat_etal._2023} is intuitive, it often lacks consistency and requires multiple fine-tuning runs, which can be cumbersome and time-consuming. 
In contrast, cross-relabeling simplifies this process by using specialized classification models tailored to each dataset's specific labeling challenges. This approach provides better transparency and produces a unified dataset encompassing seven distinct cell types across multiple tissue samples, enhancing data diversity for further model training or fine-tuning.

Despite these improvements, cross-relabeling does not entirely resolve issues related to poor labeling quality or the amount of labeled data. Specifically, our results show lower accuracies persist for underrepresented classes, such as macrophages, which may stem from a limited sample availability and intrinsic challenges in distinguishing these cells based solely on H\&E staining. Furthermore, while our method enhances label specificity, it relies on the initial quality of the broad labels; thus, any fundamental inaccuracies in the original annotations can propagate through the relabeling process. Addressing the overall problem of limited data labels may require integrating additional data sources or utilizing complementary immunohistochemical staining methods.
Although the reported performance metrics are obtained from evaluations on the native test sets of each dataset, it is important to note that the primary application of these classifiers is to perform cross-relabeling, where a model trained on one dataset (e.g., PanNuke) is applied to another (e.g., MoNuSAC) and vice versa. We acknowledge that a more systematic evaluation of cross-dataset generalization is needed and could be performed in future work.

Overall, the refined dataset produced by our approach can enhance the supervised training or fine-tuning of cell segmentation and classification models, especially those that utilize pre-trained foundation models to improve feature extraction robustness. In addition, these models can detect nuanced classes that enable researchers to conduct more detailed analyses of biological processes in computational pathology.

\section{Foundation models for robust cell segmentation and classification}

Accurate cell segmentation and classification in digital pathology are hindered by limited labeled data and the fact that conventional CNNs are unable to capture global contextual information due to their local receptive field constraints \cite{Gheflati_Rivaz_2022,Yang_Marcus_etal.}. Traditional approaches in cell quantification have predominantly relied on CNN encoders, such as ResNet50, given their proven effectiveness in semantic segmentation tasks \cite{Deshmane_2023,Graham_Vu_etal._2019,Mukasheva_Koishiyeva_etal._2024,Stringer_Wang_etal._2021}. However, approaches that include fine-tuning of pretrained CNNs, data augmentation, and stain normalization to partially increase data variability and address staining differences often fail to achieve the necessary generalization and robustness across diverse tissue types and staining conditions \cite{G._Wang_W._Li_etal._2018,Gao_Bagci_etal._2018,Karim_El_Khoury_Martin_Fockedey_etal._2021}.

To overcome these challenges, we leverage an encoder-decoder network that uses a foundation model as the encoder and a CNN upsampling decoder (\hyperref[fig:fig5]{Figure 5}) for simultaneous cell segmentation and classification in 2D patches extracted from WSIs. Foundation models with transformer-based architectures are viable alternatives to CNN-based encoders \cite{Shamshad_Khan_etal._2023,Sourget_2023}. They enable the creation of more advanced architectures that can decode or transform learned features more effectively \cite{Chen_Duan_etal._2023,Cheng_Misra_etal._2022,Xie_Wang_etal._2021}.

\begin{figure}[h!]
    \centering
    \includegraphics[width=\textwidth]{images/Figure_5.pdf}
    \caption{UNETR-like model with foundational model as backbone}
    \label{fig:fig5}
\end{figure}

By utilizing a transformer-based encoder, we incorporate global contextual information into the feature extraction process, which is a key advantage of such architectures \cite{Chen_Lu_etal._2021}. This foundation model integration facilitates accurate pixel-wise segmentation and classification without the need for extensive encoder training, thereby potentially improving generalization across varied cellular structures and tissue types.
In our implementation, we employ a modified UNETR \cite{Hatamizadeh_Tang_etal._2021} architecture that combines a vision transformer (ViT) \cite{Dosovitskiy_Beyer_etal._2021} encoder with a CNN-based decoder. The encoder utilizes the pretrained H-Optimus foundation model, which contains 1.1 billion parameters and is trained on over 500,000 H\&E stained WSIs \cite{Saillard_Jenatton_etal._2024}. We extract outputs from four evenly spaced transformer blocks $Z_i$, where $i \in [1, 14, 26, 38]$, to serve as residual connections for the CNN decoder. We select these blocks based on our observation that features from non-adjacent levels of the encoder lead to better overall performance on the test subset.

The CNN decoder upsamples the feature representations, acquired from the transformer blocks, to generate an intermediate vector that is handled by two task-specific layers that generate cell segmentation and classification masks. The first task-specific layer is the ‘Cellpose head’,  which is used to delineate cell instances. The layer generates horizontal and vertical gradient maps to form vector fields that are refined through gradient tracking in a post-processing step using the Cellpose algorithm \cite{Stringer_Wang_etal._2021}, known for its efficacy in cell segmentation tasks and generalizability across multiple domains \cite{Pachitariu_Stringer_2022,Stringer_Pachitariu_2024}. The second task-specific layer is the "Cell type head", which assigns labels to individual pixels. In the post-processing step, we determine the output classification label of each segmented cell instance by majority voting over the labeled pixels that comprise the cell in the segmentation map.

To evaluate model performance and measure the impact of adding a foundation model as backbone, we compare it to a ResNet50-based model. ResNet50 is a widely used solution for encoders in segmentation architectures in the medical domain \cite{Deshmane_2023,Graham_Vu_etal._2019,Mukasheva_Koishiyeva_etal._2024,Stringer_Wang_etal._2021}. For the H-Optimus-based model, we utilize frozen weights for the encoder and only fine-tune the decoder to take advantage of the extensive pre-training of the foundation model. For the ResNet50-based model we start with ImageNet \cite{Deng_Dong_etal.} weights and train both encoder and decoder parts. Hyperparameters for the training step are set to be identical, where possible, for comparable evaluation. 
For this evaluation, we deliberately use the PanNuke dataset to provide a standardized and controlled comparison between the H‑Optimus and ResNet50-based models (\hyperref[fig:S2]{Appendix Figure S2 (3)}). Specifically, we use two of the default PanNuke dataset splits (66\%) for training and validation, and reserve the third split (33\%) for testing.

To address the challenge of cell class imbalance in the PanNuke dataset, which is a common characteristic in most cell-level H\&E patch datasets, both models’ training processes employ a weighted loss function comprising cross-entropy and focal loss \cite{Lin_Goyal_etal._2018}. The focal loss component is adjusted with coefficients derived from each cell class' instance frequency, emphasizing learning from underrepresented classes and enhancing the model's sensitivity to rare but significant cellular patterns. The cross-entropy loss is augmented with spectral decoupling regularization \cite{Pezeshki_Kaba_etal._2021,Pohjonen_Stürenberg_etal._2022} and spatially varying label smoothing \cite{Islam_Glocker_2021}, which potentially stabilizes training and improves generalization in case of complex tissue morphologies. For optimization, we employ the \textit{AdamW} \cite{Loshchilov_Hutter_2019} to counter unbalanced class scenarios, with cosine annealing learning rate scheduler.

We utilize the scikit-learn library \cite{Van_der_Walt_Schönberger_etal._2014} and HoVer-Net \cite{Graham_Vu_etal._2019} implementations of $R^2$ (the coefficient of determination) and $PQ$ (panoptic quality) to evaluate our experiments. Complete mathematical formulations and detailed explanations of these metrics are provided in \hyperref[chap:S5]{Appendix S5}. To compute confidence intervals, we use nonparametric bootstrapping, where after calculating the metric on the full sample, we generated 1000 bootstrap replicates by resampling with replacement and then determined the 95\% confidence intervals as the 2.5th and 97.5th percentiles of the resulting empirical distribution.

%\hfill

The model comparisons are summarized in \hyperref[tab:2]{Table 2}. The H‑Optimus-based model achieves higher $R^2$ across all cell classes compared to the ResNet50-based model, which means that its predictions are more closely aligned with the PanNuke cell counts, indicating a stronger correlation with the observed data. Notably, the improvement of $R^2_{dead}$ may be an indicator of better global contextual representations provided by the foundation model backbone. In terms of segmentation and classification quality combined, measured by the PQ score, the H‑Optimus-based model demonstrates notable improvements across most cell classes. Overall, the average $R^2$ improved from 0.575 to 0.871, while the average $PQ$ score improved from 0.450 to 0.492, demonstrating better performance of the H-Optimus-based model.

\begin{table}[h!]
\renewcommand{\arraystretch}{1.5}
  \centering
  \caption{Cell quantification metrics for baseline and proposed models (CI 95\%).}
  \label{tab:2}
  \begin{tabular}{|l|c|c|}
    \hline
    %\rowcolor{gray!30}
    Metric             & Resnet50-based            & H-optimus-based              \\
    \hline
    $R^2_{neoplastic}$    & 0.681 (0.576--0.769)       & \textbf{0.941 (0.917--0.960)} \\
    \hline
    $R^2_{inflammatory}$  & 0.863 (0.778--0.903)       & \textbf{0.949 (0.918--0.966)} \\
    \hline
    $R^2_{connective}$    & 0.600 (0.488--0.698)       & 0.609 (0.436--0.772)          \\
    \hline
    $R^2_{dead}$          & 0.097 (-11.389--0.669)     & 0.925 (0.404--0.982)          \\
    \hline
    $R^2_{epithelial}$    & 0.635 (0.490--0.747)       & \textbf{0.930 (0.886--0.964)} \\
    \hline
    $PQ_{neoplastic}$       & 0.517 (0.499--0.535)       & \textbf{0.589 (0.575--0.604)} \\
    \hline
    $PQ_{inflammatory}$     & 0.455 (0.429--0.482)       & \textbf{0.528 (0.507--0.549)} \\
    \hline
    $PQ_{connective}$       & 0.416 (0.400--0.431)       & \textbf{0.451 (0.436--0.465)} \\
    \hline
    $PQ_{dead}$             & 0.374 (0.342--0.408)       & 0.292 (0.209--0.365)          \\
    \hline
    $PQ_{epithelial}$       & 0.488 (0.460--0.519)       & \textbf{0.599 (0.579--0.618)} \\
    \hline
  \end{tabular}
\end{table}

Our results  show that integrating the H‑Optimus foundation model within the UNETR architecture enhances the model's ability to segment and classify cells across diverse tissues from PanNuke data. The pretrained transformer encoder provides robust feature representations, resulting in higher average $R^2$ and $PQ$ scores compared to the CNN-based model. This leads to more reliable cell quantification and more accurate downstream analysis. Additionally, the streamlined fine-tuning process reduces computational overhead and training time, making the model more adaptable for new data.

Despite these advancements, the foundation model-based approach does not fully resolve all challenges related to cell segmentation and classification. We observe lower metric scores for underrepresented classes in the training data. Furthermore, foundation models typically encompass billions of parameters, resulting in substantial computational and memory requirements. It therefore poses challenges for deployment in resource-constrained environments, limiting their practical applicability in certain clinical settings.

\section{Model optimization via Knowledge Distillation}

To address the limitations posed by the extensive size of foundation models, we implement knowledge distillation — a model compression technique that leverages the teacher-student paradigm \cite{Hinton_Vinyals_etal._2015}. By training a smaller, more efficient student model to replicate the output of a larger, pre-trained teacher model, we retain performance while significantly reducing the model's complexity and resource requirements (\hyperref[fig:fig6]{Figure 6}).

\begin{figure}[h!]
    \centering
    \includegraphics[width=\textwidth, height=0.45\textheight, keepaspectratio]{images/Figure_6.pdf}
    \caption{Knowledge distillation framework for training a student model using a pre-trained teacher}
    \label{fig:fig6}
\end{figure}

We employ knowledge distillation to compress the H‑Optimus-based teacher model into a more efficient student model. The teacher model is the modified UNETR architecture with the H‑Optimus foundation model described in the previous chapter. The student model is based on a UNet architecture augmented with residual connections and incorporates a smaller ViT encoder with 9 million parameters \cite{Steiner_Kolesnikov_etal._2022,Wightman_2019}. 

First, we fine-tune the teacher model using the refined dataset from the cross-relabeling procedure (Section 2). Initially we train the decoder of the teacher model while keeping the encoder weights frozen. We split the refined dataset into train (70\%), validation (20\%) and test (10\%) subsets (\hyperref[fig:S2]{Appendix Figure S2 (4)}). During fine-tuning, we use the train and validation subsets, while leaving the test subset for model evaluation. We set the training procedure and model hyperparameters to be identical to those that were used to demonstrate the utility of foundation models for the simultaneous cell segmentation and classification task.

Next, we perform knowledge distillation from teacher to student using the refined dataset used to fine-tune the teacher model. The student model is trained to replicate the teacher model's outputs. We utilize a specialized loss function that aligns the student's predicted probability distribution with the teacher's, incorporating the teacher's class probability distribution derived from the output. Following the methodology of Hinton et al. \cite{Hinton_Vinyals_etal._2015}, we experiment with various hyperparameter settings for the temperature ($T$) and the balancing coefficients ($\alpha$ and $\beta$) in the loss function. We vary $T$ from 1 to 20 and adjust $\alpha$ and $\beta$ to balance the distillation and student losses. Through iterative tuning and evaluation, we identify that setting $T=14$, $\alpha=0.3$, and $\beta=0.7$ yields a configuration that converges and closely approximates the teacher model's performance during training.

Finally, we assess the performance of both models using the $R^2$ and $PQ$ (defined in \hyperref[chap:S5]{Appendix S5}) on the test set of the refined dataset (\hyperref[tab:3]{Table 3}). We observe that the 95\% confidence intervals overlap for most cell types, so we cannot claim statistically significant performance differences between the teacher and student models. One exception appears in the neoplastic class. The teacher model produces an $R^2$ of 0.919, while the student model shows an $R^2$ of 0.852. In addition, the student model achieves higher $PQ$ values for the neoplastic and connective classes, though the confidence intervals show overlap.

\begin{table}[h!]
\renewcommand{\arraystretch}{1.5}
  \centering
  \caption{Cell quantification metrics for teacher and distilled student models (CI 95\%).}
  \label{tab:3}
  \begin{tabular}{|l|c|c|}
    \hline
    %\rowcolor{gray!30}
    Metric & Teacher & Student \\
    \hline
    $R^2_{neoplastic}$    & \textbf{0.919} (0.898--0.939) & 0.852 (0.800--0.891) \\
    \hline
    $R^2_{lymphocyte}$    & 0.969 (0.956--0.977)         & 0.969 (0.956--0.978) \\
    \hline
    $R^2_{connective}$    & 0.694 (0.548--0.809)         & 0.618 (0.469--0.741) \\
    \hline
    $R^2_{dead}$          & 0.755 (0.400--0.908)         & 0.424 (0.100--0.731) \\
    \hline
    $R^2_{epithelial}$    & 0.922 (0.870--0.958)         & 0.843 (0.738--0.917) \\
    \hline
    $R^2_{macrophage}$    & 0.384 (-0.369--0.724)        & 0.704 (0.352--0.859) \\
    \hline
    $R^2_{neutrofil}$     & 0.854 (0.578--0.929)         & 0.833 (0.502--0.925) \\
    \hline
    $PQ_{neoplastic}$       & 0.581 (0.569--0.593)         & 0.601 (0.588--0.613) \\
    \hline
    $PQ_{lymphocyte}$       & 0.536 (0.520--0.553)         & 0.563 (0.544--0.579) \\
    \hline
    $PQ_{connective}$       & 0.436 (0.421--0.451)         & 0.457 (0.441--0.474) \\
    \hline
    $PQ_{dead}$             & 0.272 (0.235--0.315)         & 0.279 (0.201--0.369) \\
    \hline
    $PQ_{epithelial}$       & 0.522 (0.500--0.545)         & 0.530 (0.506--0.555) \\
    \hline
    $PQ_{macrophage}$       & 0.524 (0.459--0.588)         & 0.474 (0.405--0.543) \\
    \hline
    $PQ_{neutrofil}$        & 0.541 (0.490--0.592)         & 0.565 (0.522--0.607) \\
    \hline
  \end{tabular}
\end{table}


We further decompose the $PQ$ metric into its $SQ$ and $DQ$ components (\hyperref[tab:S6]{Appendix Table S6}). Both models produce nearly identical $SQ$ values, which indicates that they predict instance boundaries with similar precision. Although the student model shows some improvement in $DQ$ scores for certain classes, the confidence intervals overlap and do not confirm a statistically significant difference.

We observe that the student and teacher models yield comparable detection performance despite the student model using a much smaller and simpler architecture. A model with fewer parameters reduces the risk of overfitting when training data are scarce relative to the model’s complexity \cite{Farias_Ludermir_etal._2022}. The knowledge distillation process also encourages the student model to focus on the most generalizable detection features learned from the teacher. These factors enable the student model to achieve similar detection performance across different cell types.

Additionally, considering the model sizes reported in \hyperref[tab:4]{Table 4}, the distilled model achieves a significant reduction compared to the teacher model, with a 48-fold decrease in parameter count and a 5.5-fold reduction in on-disk size. In inference mode, the teacher model requires 16 GB of VRAM for a batch size of 32, while the distilled model only needs 3 GB of VRAM for the same batch size. These reductions make the distilled model significantly more practical for fine-tuning and deployment in resource-constrained environments.

\begin{table}[h!]
\renewcommand{\arraystretch}{1.5}
  \centering
  \caption{Parameter counts and size of teacher and distilled model}
  \label{tab:4}
  \adjustbox{max width=\textwidth}{%
  \begin{tabular}{|l|c|c|c|}
    \hline
    %\rowcolor{gray!30}
    Metric & H-optimus-based (Teacher) & mobileViT-based (Student) & Magnitude of difference \\
    \hline
    Parameters count       & 1,158,917,906   & \textbf{24,093,393}   & \textbf{48x}  \\
    \hline
    Estimated Total Size (MB) & 87,912       & \textbf{15,935}    & \textbf{5.5x} \\
    \hline
  \end{tabular}%
}
\end{table}

%\hfill

With recent advancements in complex network architectures and the use of pretrained encoders to achieve state-of-the-art performance \cite{Baumann_Dislich_etal._2024,Hörst_Rempe_etal._2024} in cell segmentation and classification tasks, model size, computational complexity, and processing times have increased. This limits the scalability and accessibility of these models. As we demonstrate, this may be mitigated using knowledge distillation. Studies in the field of natural language processing have demonstrated the efficacy of knowledge distillation in retaining the capabilities of the teacher model while achieving significant reductions in size and complexity \cite{Huangpu_Gao_2024,Sun_Yu_etal.}. 

We demonstrate the feasibility of knowledge distillation in digital pathology, specifically for cell segmentation and classification tasks. Moreover, we achieve this performance while also significantly reducing the parameter count. In addressing the challenge of knowledge transfer, we found that distillation from a transformer-based model to a smaller transformer is more straightforward than attempting to map transformer features to CNN blocks. In our experiments, using a CNN-based network as a student results in worse cell quantification performance due to the structural constraints of CNN feature space dimensions. 

Although our primary approach relies on a transformer-based student model that performs well, it can be further optimized to incorporate advantages from CNN architectures. For example, employing alternative techniques such as using ViT adapters \cite{Chen_Duan_etal._2023} or $1 \times 1$ convolutions to adjust feature map sizes may be beneficial for harnessing CNN advantages like enhanced local feature extraction. Moreover, if additional performance improvements are desired, the process can be further enhanced by applying supplementary knowledge distillation techniques, such as self-distillation \cite{Zhang_Song_etal._2019} or online distillation \cite{Houyon_Cioppa_etal._2023}.

Despite these promising results, further validation on independent datasets is necessary to fully understand the model's limitations. Underrepresented classes may pose challenges when addressing complex cases. Pathologists need to validate these models to adopt them in clinical settings. While the distilled models are smaller and more deployable, a technological gap persists because pathologists traditionally rely on established methods for inspecting WSIs and diagnosing diseases. Addressing the complexities involved in deploying models for inference and supporting pathologists in adopting new tools is essential for integrating these models into clinical workflows.

\section{Model integration with QuPath}
Digital pathology tools with graphical user interfaces are essential for visualizing and analyzing WSIs. To make our student model useful in clinical pathology workflows, it needs to be integrated into a tool that enables inspecting regions, creating annotations, and providing quantitative analyses of biomarkers. Therefore, we integrate the trained student model from the previous chapter into the QuPath open‑source platform \cite{Bankhead_Loughrey_etal._2017}. QuPath provides the required annotation, visualization, and analysis tools to interpret complex histological data, including workflows for cell segmentation, classification, and quantification (\hyperref[fig:fig7]{Figure 7}). 

\begin{figure}[h!]
    \centering
    \includegraphics[width=\textwidth]{images/Figure_7.pdf}
    \caption{Visualization of model-generated cell quantification annotations (left) and the corresponding unannotated slide (right) in QuPath}
    \label{fig:fig7}
\end{figure}

To identify the regions in a WSI critical for prognosticating tumor development, such as specific tumor areas or border regions without overlapping healthy tissue, the pathologist uses QuPath to outline these regions. Then, the pathologist initiates a cell segmentation and classification script through the QuPath interface for the selected regions. The resulting annotations and quantified cell information are then directly overlaid onto the WSI in the QuPath interface. Additional design and implementation details are in \hyperref[chap:S7]{Appendix S7}. 

Two common approaches for integrating deep learning models into QuPath are Java‑based native QuPath extensions \cite{Goldsborough_Philps_etal._2024} and the execution of RESTful API requests to a model server coupled with handling the response via an extension, as demonstrated in the application of cell segmentation models applied to immunofluorescence images \cite{Sugawara_2023}. While the community is actively working on these integration strategies, there is currently no universal solution that fully addresses all integration and performance requirements.

Extensions may offer better integration with QuPath, allowing slightly improved performance and more widespread usage of the built-in QuPath models, but they lack the flexibility to customize models and modify their behavior. For example, the newest version of QuPath includes models such as StarDist \cite{Weigert_Schmidt} and InstanSeg \cite{Goldsborough_Philps_etal._2024} that can perform cell segmentation. Both models pose limitations when applied to simultaneous cell segmentation and classification. StarDist performs well only on convex, round shapes by design, whereas some neoplastic, inflammatory, and connective cells exhibit complex and non-convex shapes. InstanSeg provides only semantic segmentation without assigning classes to the segmented cells.

%\hfill

In contrast, our approach offers an alternative integration strategy. It utilizes the paquo library to directly interact with QuPath’s internal application programming interface from within Python. This enables data exchange and processing without the need for intermediate conversion steps and provides greater control over model customization, retraining, and the incorporation of custom processing steps.

The integration of our custom model with QuPath underscores its potential to significantly enhance the diagnostic process by reducing the time burden on pathologists and enabling them to focus on more complex interpretative tasks using familiar software. Leveraging a tool that is already well-established among pathologists increases the likelihood of its adoption into daily clinical workflows. The quantitative data generated through the automated workflow is critical for both clinical decision-making and research, facilitating more accurate biomarker analysis, enabling robust statistical evaluations, and supporting hypothesis generation and testing. Additionally, by streamlining cell segmentation and classification, the tool enhances the scalability and reproducibility of pathological assessments, ultimately contributing to improved diagnostic accuracy and patient outcomes.

\section{Conclusion and future work}

In this study, we address critical challenges in digital pathology and tackle the usability and deployment issues of the developed models in standard computing environments without the need for high-performance computing systems. Our multi-faceted approach encompasses data refinement through cross-relabeling, leveraging foundation models for robust cell segmentation and classification, optimizing model performance via knowledge distillation, and integrating the optimized model into the QuPath software for practical application. This approach is used to construct a capable, versatile, and adjustable model for cell segmentation and classification, with enhanced performance and usability.

\begin{sloppypar}
While our approach shows potential in the field of computational pathology, certain limitations persist. 
For example, our implementation currently exhibits lower performance in detecting macrophages. 
This serves as an instance of the broader challenge of accurately identifying complex cell types. In order to address this issue, extending our approach to incorporate additional data sources, exploring alternative modeling approaches, and integrating other imaging modalities such as immunohistochemical staining may help improve detection accuracy. Moreover, although the distilled model reduces computational demands, integrating advanced deep learning models into clinical practice requires addressing technological gaps and potential resistance to adopting new tools within established diagnostic processes.
\end{sloppypar}

Future work could focus on several key areas to refine the proposed approach and facilitate its adoption in clinical environments. Enhancing the cell-relabeling process with additional datasets \cite{Graham_Jahanifar_etal._2021} could improve the representation of underrepresented cell types and enhance overall model performance. Also, incorporating additional data sources, such as multi-modal imaging or complementary staining methods, may address limitations related to cell type differentiation and class imbalance. Exploring other foundation models \cite{Vorontsov_Bozkurt_etal._2024,Zimmermann_Vorontsov_etal._2024} or introducing additional modalities \cite{Ding_Wagner_etal._2024,Vaidya_Zhang_etal._2025} may provide alternative architectures better suited to specific tasks or offer improved efficiency. Implementing more complex knowledge distillation techniques \cite{Houyon_Cioppa_etal._2023,Zhang_Song_etal._2019} could further optimize the model's performance and adaptability. Additionally, deeper integration with QuPath or other digital pathology software could provide pathologists more control over cell quantification analysis directly within the QuPath interface, thereby increasing accessibility and usability. Such enhancements would not only refine model performance but also ensure greater adaptability and scalability within various clinical environments. Finally, extensive validation of the model by pathologists and benchmarking against independent datasets are essential steps toward establishing the model's reliability and fostering confidence in its clinical utility.

\section*{Acknowledgments} 
This work was funded in part by the Research Council of Norway grant no. 309439 SFI Visual Intelligence, and the North Norwegian Health Authority grant no. HNF1521-20.

\bibliographystyle{IEEEtran}
\begin{sloppypar}
\begin{thebibliography}{99}

\bibitem{chaplot2020neural} Chaplot, Devendra Singh, et al. "Neural topological slam for visual navigation." Proceedings of the IEEE/CVF conference on computer vision and pattern recognition. 2020.

\bibitem{maksymets2021thda} Maksymets, Oleksandr, et al. "Thda: Treasure hunt data augmentation for semantic navigation." Proceedings of the IEEE/CVF International Conference on Computer Vision. 2021.

\bibitem{mezghan2022memory} Mezghan, Lina, et al. "Memory-augmented reinforcement learning for image-goal navigation." 2022 IEEE/RSJ International Conference on Intelligent Robots and Systems (IROS). IEEE, 2022.

\bibitem{al2022zero} Al-Halah, Ziad, Santhosh Kumar Ramakrishnan, and Kristen Grauman. "Zero experience required: Plug \& play modular transfer learning for semantic visual navigation." Proceedings of the IEEE/CVF Conference on Computer Vision and Pattern Recognition. 2022.

\bibitem{ye2021auxiliary} Ye, Joel, et al. "Auxiliary tasks and exploration enable objectgoal navigation." Proceedings of the IEEE/CVF international conference on computer vision. 2021.

\bibitem{chaplot2020object} Chaplot, Devendra Singh, et al. "Object goal navigation using goal-oriented semantic exploration." Advances in Neural Information Processing Systems 33 (2020)

\bibitem{ramakrishnan2022poni} Ramakrishnan, Santhosh Kumar, et al. "Poni: Potential functions for objectgoal navigation with interaction-free learning." Proceedings of the IEEE/CVF Conference on Computer Vision and Pattern Recognition. 2022.

\bibitem{ramrakhya2022habitat} Ramrakhya, Ram, et al. "Habitat-web: Learning embodied object-search strategies from human demonstrations at scale." Proceedings of the IEEE/CVF Conference on Computer Vision and Pattern Recognition. 2022.

\bibitem{mousavian2019visual} Mousavian, Arsalan, et al. "Visual representations for semantic target driven navigation." 2019 International Conference on Robotics and Automation (ICRA). IEEE, 2019.

\bibitem{dhariwal2021diffusion} Dhariwal, Prafulla, and Alexander Nichol. "Diffusion models beat gans on image synthesis." Advances in neural information processing systems 34 (2021)

\bibitem{ho2022classifier} Ho, Jonathan, and Tim Salimans. "Classifier-free diffusion guidance." arXiv preprint arXiv:2207.12598 (2022).

\bibitem{nichol2021glide} Nichol, Alex, et al. "Glide: Towards photorealistic image generation and editing with text-guided diffusion models." arXiv preprint arXiv:2112.10741 (2021)

\bibitem{brooks2023instructpix2pix} Brooks, Tim, Aleksander Holynski, and Alexei A. Efros. "Instructpix2pix: Learning to follow image editing instructions." Proceedings of the IEEE/CVF Conference on Computer Vision and Pattern Recognition. 2023.

\bibitem{fu2023guiding} Fu, Tsu-Jui, et al. "Guiding instruction-based image editing via multimodal large language models." arXiv preprint arXiv:2309.17102 (2023).

\bibitem{geng2024instructdiffusion} Geng, Zigang, et al. "Instructdiffusion: A generalist modeling interface for vision tasks." Proceedings of the IEEE/CVF Conference on Computer Vision and Pattern Recognition. 2024.

\bibitem{zhou2024minedreamer} Zhou, Enshen, et al. "Minedreamer: Learning to follow instructions via chain-of-imagination for simulated-world control." arXiv preprint arXiv:2403.12037 (2024).

\bibitem{zhou2023esc} Zhou, Kaiwen, et al. "Esc: Exploration with soft commonsense constraints for zero-shot object navigation." International Conference on Machine Learning. PMLR, 2023.

\bibitem{yu2023l3mvn} Yu, Bangguo, Hamidreza Kasaei, and Ming Cao. "L3mvn: Leveraging large language models for visual target navigation." 2023 IEEE/RSJ International Conference on Intelligent Robots and Systems (IROS). IEEE, 2023.

\bibitem{gadre2023cows} Gadre, Samir Yitzhak, et al. "Cows on pasture: Baselines and benchmarks for language-driven zero-shot object navigation." Proceedings of the IEEE/CVF Conference on Computer Vision and Pattern Recognition. 2023.

\bibitem{shah2023navigation} Shah, Dhruv, et al. "Navigation with large language models: Semantic guesswork as a heuristic for planning." Conference on Robot Learning. PMLR, 2023.

\bibitem{cai2024bridging} Cai, Wenzhe, et al. "Bridging zero-shot object navigation and foundation models through pixel-guided navigation skill." 2024 IEEE International Conference on Robotics and Automation (ICRA). IEEE, 2024.

\bibitem{yu2023co} Yu, Bangguo, Hamidreza Kasaei, and Ming Cao. "Co-NavGPT: Multi-robot cooperative visual semantic navigation using large language models." arXiv preprint arXiv:2310.07937 (2023).

\bibitem{wu2024voronav} Wu, Pengying, et al. "Voronav: Voronoi-based zero-shot object navigation with large language model." arXiv preprint arXiv:2401.02695 (2024).

\bibitem{qin2023mp5} Qin, Yiran, et al. "Mp5: A multi-modal open-ended embodied system in minecraft via active perception." arXiv preprint arXiv:2312.07472 (2023).

\bibitem{du2024learning} Du, Yilun, et al. "Learning universal policies via text-guided video generation." Advances in Neural Information Processing Systems 36 (2024).

\bibitem{ajay2024compositional} Ajay, Anurag, et al. "Compositional foundation models for hierarchical planning." Advances in Neural Information Processing Systems 36 (2024).

\bibitem{liang2024skilldiffuser} Liang, Zhixuan, et al. "Skilldiffuser: Interpretable hierarchical planning via skill abstractions in diffusion-based task execution." Proceedings of the IEEE/CVF Conference on Computer Vision and Pattern Recognition. 2024.

\bibitem{heusel2017gans} Heusel, Martin, et al. "Gans trained by a two time-scale update rule converge to a local nash equilibrium." Advances in neural information processing systems 30 (2017).

\bibitem{zhang2018unreasonable} Zhang, Richard, et al. "The unreasonable effectiveness of deep features as a perceptual metric." Proceedings of the IEEE conference on computer vision and pattern recognition. 2018.

\bibitem{brown2020language} Brown, Tom B. "Language models are few-shot learners." arXiv preprint arXiv:2005.14165 (2020).

\bibitem{podell2023sdxl} Podell, Dustin, et al. "Sdxl: Improving latent diffusion models for high-resolution image synthesis." arXiv preprint arXiv:2307.01952 (2023).

\bibitem{brohan2022rt} Brohan, Anthony, et al. "Rt-1: Robotics transformer for real-world control at scale." arXiv preprint arXiv:2212.06817 (2022).

\bibitem{brohan2023rt} Brohan, Anthony, et al. "Rt-2: Vision-language-action models transfer web knowledge to robotic control." arXiv preprint arXiv:2307.15818 (2023).

\bibitem{li2024manipllm} Li, Xiaoqi, et al. "Manipllm: Embodied multimodal large language model for object-centric robotic manipulation." Proceedings of the IEEE/CVF Conference on Computer Vision and Pattern Recognition. 2024.

\bibitem{shah2023vint} Shah, Dhruv, et al. "ViNT: A foundation model for visual navigation." arXiv preprint arXiv:2306.14846 (2023).

\bibitem{liu2024visual} Liu, Haotian, et al. "Visual instruction tuning." Advances in neural information processing systems 36 (2024).

\bibitem{hu2021lora} Hu, Edward J., et al. "Lora: Low-rank adaptation of large language models." arXiv preprint arXiv:2106.09685 (2021).

\bibitem{qin2023supfusion} Qin, Yiran, et al. "SupFusion: Supervised LiDAR-camera fusion for 3D object detection." Proceedings of the IEEE/CVF International Conference on Computer Vision. 2023.

\bibitem{qin2024worldsimbench} Qin, Yiran, et al. "Worldsimbench: Towards video generation models as world simulators." arXiv preprint arXiv:2410.18072 (2024).

\bibitem{yu2025gamefactory} Yu, Jiwen, et al. "GameFactory: Creating New Games with Generative Interactive Videos." arXiv preprint arXiv:2501.08325 (2025).

\bibitem{zhou2024code} Zhou, Enshen, et al. "Code-as-Monitor: Constraint-aware Visual Programming for Reactive and Proactive Robotic Failure Detection." arXiv preprint arXiv:2412.04455 (2024).

\bibitem{zhang2024ad} Zhang, Zaibin, et al. "AD-H: Autonomous Driving with Hierarchical Agents." arXiv preprint arXiv:2406.03474 (2024).

\bibitem{wang2024toward} Wang, Chaoqun, et al. "Toward Accurate Camera-based 3D Object Detection via Cascade Depth Estimation and Calibration." arXiv preprint arXiv:2402.04883 (2024).

\bibitem{huang2024story3d} Huang, Yuzhou, et al. "Story3d-agent: Exploring 3d storytelling visualization with large language models." arXiv preprint arXiv:2408.11801 (2024).

\bibitem{savinov2018semi} Savinov, Nikolay, Alexey Dosovitskiy, and Vladlen Koltun. "Semi-parametric topological memory for navigation." arXiv preprint arXiv:1803.00653 (2018).

\bibitem{majumdar2022zson} Majumdar, Arjun, et al. "Zson: Zero-shot object-goal navigation using multimodal goal embeddings." Advances in Neural Information Processing Systems 35 (2022): 32340-32352.

\bibitem{yadav2023offline} Yadav, Karmesh, et al. "Offline visual representation learning for embodied navigation." Workshop on Reincarnating Reinforcement Learning at ICLR 2023. 2023.

\bibitem{yadav2023ovrl} Yadav, Karmesh, et al. "Ovrl-v2: A simple state-of-art baseline for imagenav and objectnav." arXiv preprint arXiv:2303.07798 (2023).

\bibitem{sun2024fgprompt} Sun, Xinyu, et al. "FGPrompt: fine-grained goal prompting for image-goal navigation." Advances in Neural Information Processing Systems 36 (2024).

\bibitem{zhu2017target} Zhu, Yuke, et al. "Target-driven visual navigation in indoor scenes using deep reinforcement learning." 2017 IEEE international conference on robotics and automation (ICRA). IEEE, 2017.

\bibitem{koh2024generating} Koh, Jing Yu, Daniel Fried, and Russ R. Salakhutdinov. "Generating images with multimodal language models." Advances in Neural Information Processing Systems 36 (2024).

\bibitem{krantz2022instance} Krantz, Jacob, et al. "Instance-specific image goal navigation: Training embodied agents to find object instances." arXiv preprint arXiv:2211.15876 (2022).

\bibitem{schulman2017proximal} Schulman, John, et al. "Proximal policy optimization algorithms." arXiv preprint arXiv:1707.06347 (2017).

\bibitem{anderson2018evaluation} Anderson, Peter, et al. "On evaluation of embodied navigation agents." arXiv preprint arXiv:1807.06757 (2018).

\bibitem{lin2024navcot} Lin, Bingqian, et al. "NavCoT: Boosting LLM-Based Vision-and-Language Navigation via Learning Disentangled Reasoning." arXiv preprint arXiv:2403.07376 (2024).

\bibitem{NavGPT} Zhou, Gengze, Yicong Hong, and Qi Wu. "Navgpt: Explicit reasoning in vision-and-language navigation with large language models." Proceedings of the AAAI Conference on Artificial Intelligence.

\bibitem{hahn2021no} Hahn, Meera, et al. "No rl, no simulation: Learning to navigate without navigating." Advances in Neural Information Processing Systems 34 (2021): 26661-26673.

\bibitem{li2025t2isafety} Li, Lijun, et al. "T2ISafety: Benchmark for Assessing Fairness, Toxicity, and Privacy in Image Generation." arXiv preprint arXiv:2501.12612 (2025).

\bibitem{an2024agfsync} An, Jingkun, et al. "AGFSync: Leveraging AI-Generated Feedback for Preference Optimization in Text-to-Image Generation." arXiv preprint arXiv:2403.13352 (2024).


\end{thebibliography}
\end{sloppypar}

\clearpage
\beginsupplement
\section*{Appendix}
\renewcommand{\thesubsection}{S\arabic{subsection}}

\subsection{\label{chap:S1}PanNuke and MoNuSAC preprocessing}
The PanNuke dataset comprises a set of 7,901 RGB patches, each with dimensions of $256 \times 256$ pixels, which we set as the standard patch size for our analysis. In contrast, the MoNuSAC dataset encompasses 294 images of heterogeneous dimensions. To standardize the MoNuSAC images with our experiments, we implement a standardization protocol. Specifically, for images exceeding the dimensions of $256 \times 256$ pixels, we segment them into equal-sized patches and apply mirror padding to the remaining portions to avoid information loss at the peripherals. Patches with dimensions less than $128 \times 128$ pixels are excluded from the dataset due to the insufficient resolution to capture relevant cellular details. For patches where either dimension falls between 128 and 256 pixels, we employ upsampling to achieve the standard patch size. As a result, we obtain a total of 2,823 RGB patches derived from the MoNuSAC dataset for subsequent analysis. For additional details on the MoNuSAC data preparation process, refer to the source code \cite{Shvetsov_2025a}.
\clearpage

\subsection{\label{chap:S2}Data usage for the methodology}

\counterwithin{figure}{subsection}
\renewcommand{\thefigure}{S\arabic{subsection}}

\begin{figure}[h!]
    \centering
    \includegraphics[width=\textwidth, height=0.85\textheight, keepaspectratio]{images/A2.pdf}
    \caption{Overview of the methodology for cross-labeling, dataset refinement, and model comparison. (1) Cross-relabeling - training and testing cell classification models, (2) Cross-relabeling - using cell classification models to create refined dataset, (3) Fine-tuning and training models for comparison, (4) Student knowledge distillation with refined dataset}
    \label{fig:S2}
\end{figure}
\clearpage

\subsection{\label{chap:S3}Confusion matrices for classification models}
\counterwithin{figure}{subsection}
\renewcommand{\thefigure}{S\arabic{subsection}.\arabic{figure}}

\begin{figure}[h!]
    \centering
    \includegraphics[width=\textwidth, height=0.4\textheight, keepaspectratio]{images/A3_1.pdf}
    \caption{Confusion matrix for PanNuke trained model}
    \label{fig:S3.1}
\end{figure}

\begin{figure}[h!]
    \centering
    \includegraphics[width=\textwidth, height=0.4\textheight, keepaspectratio]{images/A3_2.pdf}
    \caption{Confusion matrix for MoNuSAC trained model}
    \label{fig:S3.2}
\end{figure}

\clearpage

\subsection{\label{chap:S4}Datasets cell counts}

\counterwithin{table}{subsection}
\renewcommand{\thetable}{S\arabic{subsection}}

\begin{table}[h!]
\renewcommand{\arraystretch}{2.0}
\centering
\caption{\label{tab:S4}Cell counts for PanNuke, MoNuSAC and refined datasets. Numbers in parentheses indicate preprocessed cell counts for cell classifier models training and testing.}
%\adjustbox{max width=\textwidth}{%
\begin{tabular}{|l|c|c|c|}
\hline
%\rowcolor{gray!30}
Cell type & PanNuke & MoNuSAC & Refined \\
\hline
Neoplastic & 77,403 (68,031) & - & 105,451 \\
\hline
Epithelial & 26,572 (23,207) & - & 29,926 \\
\hline
Epithelial (benign and malignant) & - & 31,402 & - \\
\hline
Inflammatory & 32,276 & - & - \\
\hline
Lymphocytes & - & 37,045 (33,104) & 65,275 \\
\hline
Neutrophils & - & 1,355 (1,252) & 3,833 \\
\hline
Macrophage & - & 1,842 (1,695) & 3,410 \\
\hline
Dead & 2,908 & - & 2,908 \\
\hline
Connective & 50,585 & - & 50,585 \\
\hline
\end{tabular}
%
%}
\end{table}



\clearpage

\subsection{\label{chap:S5}Definition of validation metrics}
\counterwithin{equation}{subsection}
\renewcommand{\theequation}{\arabic{equation}}

\subsubsection{\label{chap:S5.1}R\textsuperscript{2}}
The coefficient of determination, denoted as $R^2$, is a statistical measure that represents the proportion of variance in the dependent variable that is predictable from the independent variables. In the context of cell quantification in pathology, $R^2$ is used to assess how well the predicted quantities of different cell types in a patch align with the actual quantities observed in the ground truth data, with higher values representing more accurate quantification. $R^2$ is defined as
\begin{equation*}
R^2 = 1 - \frac{\sum_{i=1}^n (y_i - \hat{y}_i)^2}{\sum_{i=1}^n (y_i - \bar{y})^2},
\end{equation*}
where $y_i$ represents the actual number of cells of a specific type in the $i$-th image, $\hat{y}_i$ represents the predicted number of cells of that type in the $i$-th image, $\bar{y}$ is the mean of the actual numbers across all images, and $n$ is the total number of images in the dataset.

The $R^2$ metric has a range of $(-\infty, 1]$. An $R^2$ of 1 indicates perfect prediction, where all predicted values exactly match the actual values. An $R^2$ of 0 suggests that the model explains none of the variability of the response data around its mean. If $R^2$ is negative, it indicates that the model performs worse than a model that simply predicts the mean of the actual values for all observations.

\subsubsection{\label{chap:S5.2}PQ}
Panoptic Quality ($PQ$) is a comprehensive metric used to evaluate the performance of segmentation models in tasks that require both instance segmentation and classification. $PQ$ provides a single score that encapsulates both the detection accuracy (i.e., how many objects were correctly identified) and the segmentation quality (i.e., how accurately the objects' boundaries were delineated). This metric is particularly useful in multiclass scenarios where each pixel is classified into distinct categories, such as different cell types in pathology images.

$PQ$ is calculated as the product of two terms: Detection Quality ($DQ$) and Segmentation Quality ($SQ$). It can be expressed as
\begin{equation*}
PQ = DQ \cdot SQ,
\end{equation*}
where
\begin{equation*}
DQ = \frac{TP}{TP + 0.5\, FP + 0.5\, FN},
\end{equation*}
\begin{equation*}
SQ = \frac{\sum_{(p, g) \in \mathcal{M}} IoU(p, g)}{TP}.
\end{equation*}
In these formulas, $TP$ denotes the number of correctly matched instances between ground truth and prediction, $FP$ denotes the predicted instances that have no corresponding ground truth, $FN$ denotes the ground truth instances that were not detected, $IoU(p, g)$ is the Intersection over Union for a pair of matched instances $p$ (prediction) and $g$ (ground truth), and $\mathcal{M}$ is the set of matched pairs.

The $PQ$ metric is calculated for each class and is averaged across classes to provide a global performance measure.

The $PQ$ score has a range of $[0, 1.0]$, where a higher score indicates better performance in both detecting and segmenting the instances correctly. A $PQ$ of 1 signifies perfect identification and segmentation of all instances, whereas a $PQ$ of 0 indicates that no instances were correctly identified and segmented.

\clearpage

\subsection{\label{chap:S6}Segmentation and Detection quality metrics for teacher and student models}

\begin{table}[h!]
\renewcommand{\arraystretch}{2.0}
\centering
\caption{Segmentation and detection quality for student and teacher models (CI 95\%)}
\label{tab:S6}
%\adjustbox{max width=\textwidth}{%
\begin{tabular}{|l|c|c|}
\hline
%\rowcolor{gray!30}
Metric & Teacher & Student \\
\hline
$SQ_{neoplastic}$ & 0.819 (0.815--0.823) & 0.824 (0.819--0.828) \\
\hline
$SQ_{lymphocyte}$ & 0.795 (0.788--0.802) & 0.790 (0.783--0.796) \\
\hline
$SQ_{connective}$ & 0.770 (0.762--0.776) & 0.780 (0.772--0.786) \\
\hline
$SQ_{dead}$ & 0.659 (0.623--0.688) & 0.657 (0.624--0.695) \\
\hline
$SQ_{epithelial}$ & 0.780 (0.770--0.790) & 0.788 (0.779--0.797) \\
\hline
$SQ_{macrophage}$ & 0.788 (0.760--0.810) & 0.757 (0.730--0.783) \\
\hline
$SQ_{neutrofil}$ & 0.782 (0.761--0.801) & 0.775 (0.759--0.792) \\
\hline
$DQ_{neoplastic}$ & 0.706 (0.692--0.719) & 0.727 (0.712--0.741) \\
\hline
$DQ_{lymphocyte}$ & 0.675 (0.656--0.698) & 0.713 (0.691--0.734) \\
\hline
$DQ_{connective}$ & 0.566 (0.546--0.584) & 0.583 (0.565--0.602) \\
\hline
$DQ_{dead}$ & 0.410 (0.361--0.465) & 0.435 (0.306--0.561) \\
\hline
$DQ_{epithelial}$ & 0.668 (0.639--0.694) & 0.673 (0.644--0.702) \\
\hline
$DQ_{macrophage}$ & 0.657 (0.583--0.727) & 0.615 (0.531--0.703) \\
\hline
$DQ_{neutrofil}$ & 0.691 (0.625--0.753) & 0.729 (0.679--0.778) \\
\hline
\end{tabular}
%
%}
\end{table}

\clearpage

\subsection{\label{chap:S7}QuPath integration method}
We adopt an integration strategy leveraging the paquo \cite{Bayer_AG} library, a Python package that enables direct interaction with QuPath’s internal API, thereby facilitating seamless data exchange without intermediate conversion steps. The data processing pipeline (\hyperref[fig:S7]{Appendix Figure S7}) begins with the acquisition of WSIs and their associated annotations from QuPath, which are represented as Shapely \cite{Gillies_Wel_etal._2024} polygons. Utilizing paquo, we directly read, create, and modify these annotations and detections within a QuPath project in the Python environment. Images are then cropped using these polygons and processed by cell segmentation and classification models employing standard vision processing toolkits such as OpenCV, pyvips, and PyTorch. Additionally, QuPath employs Groovy scripts to initiate a Python process that starts the entire pipeline from QuPath graphical interface: fetching polygons, extracting images from them, and running deep learning model inference on the cropped images. 
The results are returned to QuPath, leveraging paquo's Python bindings to manipulate QuPath data while minimizing the computational overhead typically associated with cross-environment communication.

\counterwithin{figure}{subsection}
\renewcommand{\thefigure}{S\arabic{subsection}}

\begin{figure}[h!]
    \centering
    \includegraphics[width=\textwidth]{images/A7.pdf}
    \caption{QuPath integration workflow using Python environment}
    \label{fig:S7}
\end{figure}

Compared to traditional workflows that involve exporting annotations as GeoJSON, classifying them in Python, and reimporting them into QuPath, our approach offers several advantages. We eliminate the need to switch between programming languages, providing a cohesive and streamlined development process entirely within QuPath software and removing the necessity to use other tools. Meanwhile, we avoid storing annotations as intermediate JSON files unless required for external use or archiving. By conducting the entire inference and post-processing workflow within the Python environment, we leverage the power and flexibility of Python libraries for image processing and machine learning. This approach also enables adjustments to any set of labels and models, thereby improving its applicability.

%\hfill

The distilled model and QuPath integration code are packaged into a Docker container, enabling streamlined execution with the Docker engine. Detailed integration code and deployment instructions can be found in the GitHub repository \cite{Shvetsov_2025b}.

Despite these benefits, we acknowledge that the paquo library is a proof‑of‑concept project in its early development stage and has not been tested across all versions of QuPath.

\clearpage

\subsection{\label{chap:S8}Data and code availability statement}
All datasets, models, and code used in this study are publicly available and can be obtained from the repositories listed below. 
The PanNuke \cite{Gamper_Koohbanani_etal._2019} and MoNuSAC \cite{Verma_Kumar_etal._2021} datasets are publicly accessible, and download information along with detailed descriptions can be found in their respective articles. Preprocessing scripts for PanNuke and MoNuSAC data, as well as individual cell extraction scripts, are available on GitHub \cite{Shvetsov_2025a}. The H-Optimus foundation model used in our experiments can be downloaded from the HuggingFace repository \cite{hoptimus2024}, and model information is available on GitHub \cite{Saillard_Jenatton_etal._2024}. In addition, the integration code for QuPath and the distilled model packaged in a Docker container are provided in the repository \cite{Shvetsov_2025b}, and paquo Python library is available from the authors GitHub repository \cite{Bayer_AG}.
\clearpage

\end{document}



\end{document}
