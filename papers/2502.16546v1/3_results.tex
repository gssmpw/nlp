


\begin{table}[]
\caption{Issue Resolution Patterns}
\label{tab:patterns}
\resizebox{\columnwidth}{!}{%
\begin{tabular}{l|l|c|c}
\hline
\multicolumn{1}{c|}{\textbf{\begin{tabular}[c]{@{}c@{}}Pattern\end{tabular}}} & \multicolumn{1}{c|}{\textbf{Description}}                                                                                                                                                                  & \textbf{\begin{tabular}[c]{@{}c@{}}Com-\\ plexity\end{tabular}} & \textbf{\begin{tabular}[c]{@{}c@{}}\# of\\ Issues\end{tabular}} \\ \hline
\texttt{\textbf{I,CR,I?}}                                                                                            & \begin{tabular}[c]{@{}l@{}}Implement the solution and review the code;\\ followed by another optional implementation.\end{tabular}                                                                         & Simple                                                          & 64                                                           \\ \hline
\texttt{\textbf{A,I,(I$\mid$CR$\mid$V)+}}                                                                                     & \begin{tabular}[c]{@{}l@{}}Analyze the problem and implement the solution;\\ followed by another optional I or CR or V or\\ any combination.\end{tabular}                                                  & Simple                                                          & 32                                                           \\ \hline
\texttt{\textbf{(I,(CR$\mid$V))+}}                                                                                        & \begin{tabular}[c]{@{}l@{}}Implement the solution; review the code and/or\\ verify the implementation; I, CR and/or V\\ repeat more than once.\end{tabular}                                                & Complex                                                         & 28                                                           \\ \hline
\texttt{\textbf{SD,I,CR,(I$\mid$V)?}}                                                                                      & \begin{tabular}[c]{@{}l@{}}Design and implement the solution and review the\\ code; followed by another optional I or V or both.\end{tabular}                                                              & Simple                                                          & 24                                                           \\ \hline
\texttt{\textbf{A,SD,I,(I$\mid$CR$\mid$V)+}}                                                                                  & \begin{tabular}[c]{@{}l@{}}Analyze the problem, design, and implement the\\ solution; followed by another optional I or CR\\ or V or any combination.\end{tabular}                                         & Simple                                                          & 22                                                           \\ \hline
\texttt{\textbf{I}}                                                                                                 & Implement the solution.                                                                                                                                                                                    & Simple                                                          & 21                                                           \\ \hline
\texttt{\textbf{I,CR,V,I?}}                                                                                           & \begin{tabular}[c]{@{}l@{}}Implement the solution, review the code, and verify\\ the implementation; followed by another optional I.\end{tabular}                                                          & Simple                                                          & 16                                                           \\ \hline
\texttt{\textbf{SD,(I,(CR$\mid$V))+}}                                                                                     & \begin{tabular}[c]{@{}l@{}}Design the solution; implement the solution,\\ review code and/or verify the implementation;\\ I, CR and/or V repeat more than once.\end{tabular}                               & Complex                                                         & 13                                                           \\ \hline
\texttt{\textbf{(SD,I,(CR$\mid$V))+}}                                                                                      & \begin{tabular}[c]{@{}l@{}}Design and implement the solution; review the\\ code, and/or verify the implementation;\\ SD,I, CR and/or V repeat more than once.\end{tabular}                                 & Complex                                                         & 12                                                           \\ \hline
\texttt{\textbf{A,(I,(CR$\mid$V))+}}                                                                                     & \begin{tabular}[c]{@{}l@{}}Analyze the problem; implement the solution,\\ review code, and/or verify the implementation;\\ I, CR and/or V repeat more than once.\end{tabular}                              & Complex                                                         & 7                                                            \\ \hline
\texttt{\textbf{(A,SD,I,(CR$\mid$V))+}}                                                                                  & \begin{tabular}[c]{@{}l@{}}Analyze the problem, design, and implement the\\ solution; review the code and/or verify the\\ implementation; A, SD, I, CR and/or V repeat\\ more than once.\end{tabular}     & Complex                                                         & 7                                                            \\ \hline
\texttt{\textbf{SD,I}}                                                                                               & Design and implement the solution.                                                                                                                                                                         & Simple                                                          & 7                                                            \\ \hline
\texttt{\textbf{R,A,SD,I,(I$\mid$CR$\mid$V)?}}                                                                               & \begin{tabular}[c]{@{}l@{}}Reproduce and analyze the problem; design and\\ implement the solution followed by another\\ optional I or CR or V or any combination.\end{tabular}                             & Simple                                                          & 6                                                            \\ \hline
\texttt{\textbf{A,SD,(I,(CR$\mid$V))+}}                                                                                  & \begin{tabular}[c]{@{}l@{}}Analyze the problem and design the solution;\\ implement the solution, review the code and/or\\ verify the implementation; I, CR and/or V\\ repeat more than once.\end{tabular} & Complex                                                         & 6                                                            \\ \hline
\texttt{\textbf{A,(SD$\mid$V)}}                                                                                          & \begin{tabular}[c]{@{}l@{}}Analyze the problem; design the solution or\\ verify the implementation.\end{tabular}                                                                                           & Simple                                                          & 6                                                            \\ \hline
\multicolumn{4}{c}{
     \scriptsize{
      \texttt{\textbf{R}}=\ir, \texttt{\textbf{A}}=\ia, \texttt{\textbf{SD}}=\sd,}} \\
  
    \multicolumn{4}{c}{
      \scriptsize{\texttt{\textbf{I}}=\impl, \texttt{\textbf{CR}}=\crv, \texttt{\textbf{V}}=\ver}}


\end{tabular}%
}
\end{table}




\section{OLD Results}
\label{sec:results}







\subsection{\ref{rq:stages}: Issue Resolution Stages}
\label{sub:results_stages}

\os{here, we are doing an analysis of  individual stages to see how  (frequently) they are found, co-occur, and "interact" in the issues, so RQ1 should be phrased to capture this}

\Cref{tab:stages} presents the identified six stages of the issue resolution process with the description, annotation codes, and \# of issues where the stage is found. For stage analysis, we used the stage sequence of each issue obtained in \Cref{sub:resolution_stages}. 

The most common stage is \impl which appears in 328 issues (92.1\%) and least common one is \ir, appearing in 47 issues (13.2\%) \os{kind of unexpected, discuss}. \crv is also common as it is documented in 264 issues (74.2\%). \os{discuss the rest}

\os{why is this important to say?}Stage sequences typically starts with \impl (44.7\% of the issues), \ia (28.7\%), and \sd (19.1\%) and ends with a \crv (34\%), \impl (29.8\%), or \ver (27.8\%). 


\textbf{Stage correlations.}
\os{this and the next paragraph sound potentially incomplete, why do we focus on these stages only?}
An additional 
\impl is observed after \crv in 137 issues (51.9\% of the issues with \crv) and after \ver in 68 issues (46.6\% of the issues with \ver). This illustrates the significance of \crv and \ver stages in the issue resolution process as they required implementation modifications to ensure solution quality -- this is inline with the Firefox's QA policies/goals~\cite{firefox-qa}.

 Among 150 issues with \sd, in 80.7\% of the issues (121), there is \impl immediately before or after \sd. Interestingly, in 114 issues \impl has an immediate preceding \sd (76\% of issues with \sd) \os{so most of them is Design --> impl}, and in 254 issues \impl is immediately followed by \crv (96.2\% of the issues with \crv). On the other hand\os{??}, in 32 issues (68.1\%), \ir is enclosed by \ia among the 47 issues with \ir. 

To understand the relative presence of the stages in the issue resolution process, we further investigated the bi-grams and tri-grams of stages in the stage sequences. Our identified most common bi-grams are: \texttt{\textbf{I,CR}} (found in 96.2\% of the issues where both appear); \texttt{\textbf{SD,I}} (found in 80.3\% of the issues where both appear); and \texttt{\textbf{IA,SD}} (found in 79.7\% of the issues where both appear). Interestingly, among all possible 30 bi-grams, we found 29 of them appear in 1 to 254 issues (36.4 issues on average with a median of 17). 
The most common tri-grams are: \texttt{\textbf{SD,I,CR}} (found in 66.1\% of the issues where all three appear); \texttt{\textbf{I,CR,V}} (found in 62.7\% of the issues where all three appear); and \texttt{\textbf{IA,SD,I}} (found in 61.5\% of the issues where all three appear). Among the 120 possible tri-grams we can form, we found 95 tri-grams, which appear in 1 to 76 issues (9.5 issues on average with a median of 4).



\cng{
To further investigate the relationship among stages, we conducted association rule mining with the stage sequences using MLExtend library \re. Top association rules based on support, confidence, and lift reveal that presence of \ia increases the probability of the presence of \sd which translates into if \ia is reported in an issue, there is a high probability of reporting \sd. Moreover, presence of \ia and/or \sd increases the probability of the presence of \ver which means if \ia and/or \sd is reported then \ver will be also reported for that issue. 
}

\finding{\impl is typically preceded by \sd and followed by \crv whereas \ir and \ia are typically performed together.\os{kind of expected?}}

\cng{\textbf{Expectation vs. reality.} \os{mention reproduction} We expect that all the issues should have an \impl \& \crv and \crv \& \ver should be found after an \impl. However, this is not always true. Although all the issues with \crv have a prior \impl stage, we observed \ver without a prior \impl in 12 issues \os{something to say about these 12? why there was ver without a prior  impl? 9 were solved in other issues. The other 3?}. Interestingly, 28 issues (7.87\% of the issues) were resolved without an \impl, yet they contain other stages: \ir (9/28), \ia (17/28), \sd (8/28), and \ver (9/28), \os{cr?}. 
The main reason for not having an \impl in these issues is that they were solved in/by other issues (25/28). \os{waste of effort?} \os{this sentence reads disconnected, what
s the problem? Situation: the related issues contain different information/code/stages and devs may not be aware of them despite there is a explicit or implicit link between issues: they may waste effort. A tool should be process both issues and find a link between, and should synch the info between issues.}Future research can investigate to extract the important information from this type of issue and merge that with the issue where it is solved. Issue tracking systems can also provide features to link these issues so that developers can identify important information from these issues easily.
}

\finding{Issues resolved without an \impl are mainly because they are solved by or in other issues. \impl is not always reviewed, and \ver can be done without any \impl. \os{last one sounds weak}}

\textbf{Issue Type.}
Among the 47 issues with \ir, 46 are defects while the other one is enhancement --- \ir is not seen for tasks. Moreover, \ia is found in only 1 issue among 26 tasks. \os{what are tasks about? why they don't have analysis?}Surprisingly \os{why?}, among 134 issues with \ia stage, 121 are defects type issues (90.3\% of the issues with \ia). \sd and \impl stages are common in all types of issues. More than 92\% of the enhancement and task issues contain a \crv while 68.2\% of issues of defects contain \crv. These results indicate the code review focuses mostly on feature implementations. 
On the other hand, \ver is more common for defects (42.2\%) and enhancements (41.7\%) than  for tasks (26.9\%). The results imply that not all the stages are equally performed and documented for all types of issues, which may indicate potential problems in implementing QA tasks, for some issue types -- not that code review and verification seem \os{seem or are?} required by Firefox's policies

\finding{Not all stages are equally performed\os{not equally performed?} across issue types. \crv is more common in enhancements and tasks, while \ver is more common in defects and enhancements.\os{reproduction?}}

\textbf{Problem Class \& Category.} 
Among 47 issues containing \ir stage, 37 issues belong to the implementation class and 9 issues belong to the testing class, while only 1 issue is about refactoring\os{expected}. Crashes (11/23), UI issues (8/33), and defective functionality issues (8/43) contain the highest percentage \ir. Of the 134 issues with \ia, 110 are of the implementation class and 20 are of the testing class, only 4 are of the refactoring class \os{so what?}. \sd is more common in issues of implementation class (47.5\%) than in refactoring (23.5\%) and testing (31.8) classes. \os{why solution and analysis are more common in impl, than in ref and testing? What's the break down by issue type?}
While both \crv and \ver stages are common in the issues of the implementation class, \crv appears higher for the issues of refactoring than testing (86.3\% vs. 56.8\% of the issues of the respective classes). %
However, \ver is more common in testing than refactoring (31.8\% vs. 19.6\% of the issues of the respective classes). This is interesting as we would expect \ver to be found for solving refactoring class issues. \os{testing is essential in verifying we don't break things with refactoring.}

\finding{Certain stages are more common for certain problem classes and categories. For example, \crv is more common in refactoring-  than testing-related issues.\os{ti strengthen this one}}




\subsection{\ref{rq:patterns}: Issue Resolution Patterns}
\label{sub:results_patterns}

\os{we probably need an analysis how some of the patterns are similar, yet it makes sense to separate them}

\cng{
For 356 issues, we derived 47 distinct patterns of issue resolution, however, 18 of them were used to resolve 5 or more issues covering 287 issues (80.6\% of all issues). Hence, we call these \textbf{18 patterns} as the \textbf{recurrent issue resolution patterns}. \Cref{tab:patterns} shows the top 10 recurrent patterns -- all the patterns are provided as the pattern catalog in our replication package~\cite{repl_pack}. The patterns represent the issue resolution process where each stage in the patterns can be performed multiple times in a row by default. For the remaining analysis of the paper, we considered all 356 issues (\ie 47 patterns) so that we can investigate and derive conclusions for all the annotated issues.

\os{this needs improvement}47 unique patterns for 356 issues (only 7.6 issues per pattern) indicates that issues are resolved in a variety of ways in practice. This implies that as-is process deviates from the to-be process and as-is process is not that straightforward as the to-be process described in the literature~\cite{Zimmermann2010,firefox-bug-handling}, discussed in \Cref{sub:background_issue_res}. In as-is process the stages of issue resolution are applied in more complicated way than it is described in the to-be process. For example, among 47 patterns only 5 contain all 6 stages, 21 contain iterative stages, 11 did not follow the ordering of the stages, and 27 are simple while the rest 20 are complex (discussed later). However, top recurrent patterns are widely used to solve issues (the top 10 patterns were used to solve 7-64 issues, 23.9 on average with a median of 21.5\os{how much of the issues  these 10 cover?}). It can be argued that we could create more generalized patterns to reduce the number of derived patterns. However, in that way, we would deviate from the actual issue resolution process for many issues. As we study the practical issue resolution process, we did not forcefully merge different patterns to reduce the number of unique patterns. We ensured that the derived patterns captured the actual issue resolution process as accurately as possible.
}





\finding{\cng{As-is process deviates from the to-be process and it is not as straightforward as described in the literature. There are 18 recurrent patterns to solve 287 issues (80.6\% of all issues) \os{talk about the 47 patterns as well}.}}


\textbf{Pattern examples.} We describe three patterns of different kinds to understand how they realize the issue resolution process. The most common pattern {\texttt{\textbf{I,CR,I?}}} means at the beginning of the issue resolution process, the assigned developer implements the solution (\impl) without performing any other stages (\eg ~{\texttt{\textbf{IR}} or \texttt{\textbf{IA}} or \texttt{\textbf{SD}}}). After the implementation is submitted, another developer reviews the code (\crv). Based on the code review, the assignee optionally \os{not sure I like this term: "optionally"} updates the implementation (\impl). \os{rephrase? all the stages are needed, but some are documented or some others do not}That means for some issues (14 out of 64), the last \texttt{\textbf{I}} is needed and others do not undergo this stage. As this pattern only involves two unique stages having one stage optionally repeated once, we classified this pattern as \textit{simple}.

The pattern \texttt{\textbf{IA,SD,I,(I$\mid$CR$\mid$V)?}} represents the process in which an analysis  of the problem is first conducted (\ia). The developers then design the solution (proposing a potential solution, reviewing a proposed solution, or both). The assignee then implements the solution, and then undergoes three optional stages: \impl, \crv, or \ver. That means after the first implementation is submitted, another developer may perform \crv and/or \ver. Based on these stages, there can be more \impl. This pattern is  \textit{simple} because the first three stages are mandatory while the last three are optional.

The pattern {\texttt{\textbf{(SD,I,(CR$\mid$V))+}}} suggests a process in which a series of \sd, \impl, \crv, and/or \ver is performed repetitively to resolve the issue. In the repetitive series {\texttt{\textbf{(CR$\mid$V)}}} means either one or both can appear after a \sd and \impl. To resolve issues with this pattern, developers need to perform 4 distinct stages where all stages are repetitive which makes this pattern \textit{complex}.



\textbf{Pattern complexity.} Among all 47 patterns, 27 are \textit{simple} and found in 252 issues (70.8\% of the issues), and 20 patterns are complex and found in 104 issues (29.2\% of the issues). Issues resolved with a simple pattern included less than 3 stages while the issues resolved with a complex pattern included about 10 stages on average.
\cng{Among the top three recurrent patterns, two are simple: {\texttt{\textbf{I,CR,I?}}} found in 64 issues and {\texttt{\textbf{IA,I,(I$\mid$CR$\mid$V)?}}} found in 32 issues, and the third pattern is complex: \texttt{\textbf{(I,(CR$\mid$V))+}} found in 28 issues. }\os{and the rest?}

\cng{
To further investigate the amount of effort required to resolve issues with simple vs. complex patterns, we performed statistical analysis on average stage sequence length (2.9 vs. 9.9), resolution time in days (58 vs. 119.8), and \# of people involved in the issue report discussion (4.4 vs. 7.4). Mann-Whitney U test \re on these criteria reveals that the differences are statistically significant and we conclude that issues resolved with complex patterns required substantially higher developer effort than the issues resolved with simple patterns. 
}
\looseness=-1


\finding{\cng{More than 70\% of the issues were solved with 27 simple patterns (3 stages on avg.) which require significantly lower developer effort than the issues resolved with complex patterns (10 stages on avg.) \os{quantify effort?}.}}

\textbf{Issue Type.}
\Cref{tab:patterns_issue_types} shows the distribution of issues having different complexity levels across issue types:  %
76.9\% of the tasks are solved using a simple pattern whereas 70.7\% of the defects and 68.3\% of the enhancements are resolved with a simple pattern. This result illustrates that tasks potentially require lower effort to be solved. 
\cng{
\Cref{fig:boxplot_issue_type_length} validates this observation as it shows significantly fewer stages required to address tasks compared to defects and enhancements, as measured by the number of stages in the sequences (3.6 vs. 5/5 stages on avg.). Other descriptive attributes, \ie avg. resolution time in days (9 vs. 82.3/76.9 on avg.) and \# of involved people (4 vs. 5.4/5.3 on avg.) also validates this observation. 
Mann-Whitney U test \re confirms that these differences are statistically significant.\os{how about the 19 recurrent ones?}
}

Among the 79 defect issues that require a complex pattern, five issues include an excessive number of stages (20 or more), as seen in \Cref{fig:boxplot_issue_type_length} as outliers. Of these five issues, two are in the feature development category, two are crashes, and the remaining one is in the code improvement category. Interestingly, no issues in both enhancement and task types require more than 15 stages.

\finding{Tasks require lower effort to resolve compared to defects and enhancements\os{strengthen}.}

The most recurrent pattern, \ie \texttt{\textbf{I,CR,I?}} is also the most recurrent pattern for all issue types: defect (39/270), enhancement (13/60), and task (12/26). 
The second most recurrent pattern for defects, \ie \texttt{\textbf{IA,I,(I$\mid$CR$\mid$V)?}} is not seen for tasks. The third and fourth most recurrent patterns for defects, \ie \texttt{\textbf{IA,SD,I,(I$\mid$CR$\mid$V)?}} and \texttt{\textbf{I}} are not seen for enhancements and tasks respectively. These two patterns were used to resolve only one issue for task and enhancement respectively.
In general, the top 9 recurrent patterns for defects and enhancements are similar, however, only 3 were not observed for tasks. This illustrates that tasks are resolved using different resolution processes than defects and enhancements which emphasizes adopting different techniques for solving different types of issues. Moreover, more unique patterns are observed for enhancement (18 patterns for 60 issues) and task (7 patterns for 26 issues) than defects (46 patterns for 270 issues).
\looseness=-1

\finding{Issue resolution patterns vary across issue types: tasks are solved using different patterns than defects and enhancements\os{spell out the differences}.}

\begin{table}[]
\caption{Number of issues across issue types}
\label{tab:patterns_issue_types}
\centering
\begin{tabular}{c|cc|c}
\hline
\multirow{2}{*}{\textbf{Issue Type}} & \multicolumn{2}{c|}{\textbf{Pattern Complexity}}        & \multirow{2}{*}{\textbf{Total}} \\ \cline{2-3}
                                     & \multicolumn{1}{c|}{\textbf{Complex}} & \textbf{Simple} &                                 \\ \hline
\textbf{Defect}                      & \multicolumn{1}{c|}{79}               & 191             & \textbf{270}                    \\ \hline
\textbf{Enhancement}                 & \multicolumn{1}{c|}{19}               & 41              & \textbf{60}                     \\ \hline
\textbf{Task}                        & \multicolumn{1}{c|}{6}                & 20              & \textbf{26}                     \\ \hline
\textbf{Total}                       & \multicolumn{1}{c|}{\textbf{104}}     & \textbf{252}    & \textbf{356}                    \\ \hline
\end{tabular}
\end{table}

\begin{figure}[t]
 \centering
 \includegraphics[scale=.35]{figures/boxplot_issue_type_length.png}
 \caption{Length of the category sequences across issue types}
 \label{fig:boxplot_issue_type_length}
 \vspace{-0.2cm}
\end{figure}


\textbf{Problem Class and Category.}
Our replication package contains the distribution of the \# of issues resolved with different pattern complexity levels across problem classes and categories~\cite{repl_pack}.  The majority of the issues in each of the three classes are solved using simple patterns. 
Among the 104 issues solved with a complex pattern, 87 issues are of implementation class where defective functionality (19/43), code design (22/75), and UI issue (14/33) categories contain the highest number of issues. 
\cng{Among 51 refactoring issues, only 9 were resolved using a complex pattern which indicates refactoring issues require less effort than implementation and testing issues. Descriptive attributes, \eg avg. stage sequence length (3.8 vs. 5.4/3.9), avg. resolution time in days (69.8 vs. 73.7/97.6), and \# of involved people (3.8 vs. 5.6/5.1) validate this observation and statistical significance was found by Mann-Whitney U test \re.
}

Interestingly, the six most frequent patterns were used to resolve issues of more than half of the categories (17 total categories). \os{and the 18?}This illustrates that the same issue resolution pattern can solve issues of different problem categories. Moreover, issues of some categories require fewer unique patterns to be resolved\os{not sure I understand this phrasing}. For example, the code improvement category only requires 12 distinct patterns to solve 32 issues and code design requires 22 unique patterns to resolve 75 issues. However, the opposite scenario is also observed. The crash category requires 18 distinct patterns to resolve only 23 issues and the UI issue category requires 23 patterns to resolve 33 issues. 

\cng{By analyzing the usage of complex patterns, we identified 5 problem categories that potentially require more effort to resolve (\ie code design, defective functionality, feature development, UI Issue, and crash). Descriptive attributes of the issues with these 5 categories vs. other categories validate the hypothesis, \ie avg. stage sequence length (5.2 vs. 4) and \# of involved people (5.8 vs. 4.4). Surprisingly, average resolution time of the issues of the 5 categories (72.4) is lower than the issues of other categories (81.6). The reason behind this is the outliers which can be understood by the median (9 vs. 6)\os{I don't get this}. Hence, we conclude that issues of these 5 categories require more effort to resolve and Mann-Whitney U test \re justifies the statistical significance. 
}


\finding{\cng{\os{rephrase this}Resolution of refactoring issues require less effort than implementation \& testing issue, on the other hand code design, defective functionality, feature development, UI Issue, and crash related problems require more effort.}}












\looseness=-1


\textbf{Pattern analysis over time.}
We conducted a statistical analysis on the \# of issues resolved with the patterns every year from 2010 to 2023. During the 14-year time span, the five most frequent patterns (used to resolve 170 issues) were observed in 11 to 14 different years. The 10 most frequent patterns were used in 7 to 14 different years. These patterns were used to resolve more than 67\% of all the issues. Moreover, all the 39 patterns used to resolve two or more issues were utilized in two or more years. \os{and the 18 patterns?} 

\finding{Mozilla Firefox developers have been using similar issue resolution patterns for a long time\os{improve?}.}








































































