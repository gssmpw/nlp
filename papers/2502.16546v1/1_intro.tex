
\section{Introduction}
\label{sec:intro}




Issue management is a fundamental process that aims to track and manage the code changes needed to address issues during the maintenance and evolution of a software project. Issue trackers are essential tools that provide the infrastructure to implement issue management~\cite{Zimmermann2009}. Such systems provide a platform for documenting software issues, facilitating discussions among stakeholders, and tracking the work and progress of solving the issues~\cite{Zimmermann2009,Zimmermann2010}. The issue management process assisted by issue trackers, typically involves steps such as issue understanding, triage, replication, and analysis, as well as issue fixing (\aka issue resolution)~\cite{Zimmermann2010,firefox-bug-handling}. 
Issue resolution is a sub-process of issue management that aims to diagnose and resolve the reported problems. 

According to existing literature~\cite{zhang2016literature,saha2015understanding,zeller2009programs,rajlich2011software,eren2023analyzing}, the typical issue resolution process includes steps such as issue reproduction, problem investigation, solution design, solution implementation, and validation/verification, which are sequentially applied to solve issues. However, while this process is meant to be generally applied to any software issue, it is unclear how developers implement it in practice for different problems and contexts and how developers discuss it in issue reports.


\looseness=-1
Understanding the issue resolution process implemented in practice is important for improving software maintenance and evolution processes. By gaining insights into how developers address software problems, we can identify bottlenecks and anomalous process implementations,  align prescribed processes with actual practices, and provide developers with better guidance for issue resolution.  Additionally, studying issue resolution can help identify common patterns and strategies that can be applied to similar problems in the future, and confirm the extent to which the implemented process deviates from the typical, linear resolution process from the literature.


This paper aims to enhance our understanding of the issue resolution process implemented in practice by identifying and analyzing the sequence of steps (\ie stages) that developers perform and discuss in issue reports when solving issues. To that end, we conducted a case study on Mozilla Firefox, a mature and widely-used open-source project. Combining qualitative and quantitative methods, we analyzed the discussions present in a sample of 356 Firefox issue reports to identify the stages of issue resolution that Firefox developers engage in, the sequences of stages that issue discussions form, recurrent transitions between stages present in the sequences, the overall issue resolution process implemented at Firefox, and the recurrent instances of this process to solve a variety of problem types reported in different issue reports. 

Using a multi-coder iterative open-coding methodology, we identified six issue resolution stages (\eg issue reproduction, solution design, implementation, and code review). The stages appear in issue reports with varying frequencies across different issue types (defects, enhancements, and tasks) and problem categories (\eg Crashes, UI Issues, and Code Improvements), and form sequences that represent particular instances of the resolution process at Firefox. The stage sequences reveal frequent relationships among stages, particularly between issue reproduction and analysis; among solution design, implementation, and code review; and among implementation, code review, and solution verification. Additionally, based on analysis of consecutive stages appearing in the sequences (\ie bi-grams), we identified the most common transitions between stages and derived the overall issue resolution process at Firefox from them. Such a process is primarily iterative, deviating from the theoretical linear process found in the literature and Firefox's documentation. In this process, developers go back and forth from one stage to another as needed to solve the issues. Finally, utilizing qualitative analysis of the sequences, we identified 47 issue resolution patterns that represent recurrent instances of the overall process of solving different types of problems. \rev{Two Mozilla developers assessed the usefulness of the patterns, identifying potential use cases to enhance Firefox's resolution process.}
\looseness=-1
	
Our study provides evidence of the iterative and diverse nature of the issue resolution process, which widely deviates from the theoretical linear process from the literature. Our methodology, stage sequences, and patterns serve to identify potential anomalies in the way Firefox developers implement the resolution process.
Our pattern catalog and results can help educate future developers and train newcomers at Firefox in the intricate process of issue resolution. Finally, we advocate for developing advanced tooling to assist developers in recording issue resolution activities more easily, as this can have great benefits for traceability, process assessment, code change rationale management, and more.






In summary, this paper makes the following contributions:
\begin{itemize}
    \item A model of Firefox's issue resolution process implemented in practice, derived from qualitative and quantitative analysis of issue report discussions. 
    \item A novel catalog of 47 patterns of issue resolution, derived from qualitative analysis of issue discussions. The patterns represent instances of Firefox's resolution process, employed by Firefox developers to address different types of software problems.
    \item A comprehensive analysis of the derived process and patterns, across different types of issues and problem categories, showing that Firefox's issue resolution is a diverse and iterative process, which deviates from the prescribed linear process from the literature.
    \item A novel dataset with annotated issues and related artifacts that enables further research in this area. We publicly release this data and the derived catalog, scripts, and other artifacts useful to validate and replicate our study~\cite{repl_pack}.
\end{itemize}








