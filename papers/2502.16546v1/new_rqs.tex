\section{New RQs}


RQ: How is the issue resolution process implemented in practice at Mozilla Firefox?

\begin{itemize}
	\item What stages  of issue resolution are discussed in issue reports and how frequently they are discussed?
	\begin{itemize}
		\item Motivation: we want to know the steps of resolution and how devs discuss them in the issues. Some of these steps may be discussed more frequently than others or not discussed at all, for certain types of problems and reports.
		\item Results: 
		
		* Not all issues include all stages, which means developers either do not perform these stages, do not need to discuss them in the issues, or they discuss them somewhere else.
		
		* IMPL and CR are frequently performed and documented, and this is expected according to the Mozilla documentation.  Also, there is integration between git/CR tool (Phabricator since 2018) and bugzilla. 
		
		* Implementation is not done in 8\% of the issues because they are solved in/by other issues.
		
		* CR is not done in 26\% of the issues. Why not? Analyzing by type of issue, defects are the least reviewed issues (only 68\% vs 92-93\%). Why? Since CR is mandated to be included in the issue report, according the Moz doc: they don't do CR (defects imply shorter code changes and more rapid issue resolution, but still need to be reviewed)
		* Testing issues have a lower CR than Refactoring and Implementation (56.8\% vs 86.3\% and 74.7\%) -- refactoring seems to have the most CR
		* Problem classes with 10+ defects that have lowest \% of CR (59\% - 69\%): Test Failure, Crash, Performance Optimization, Defective Functionality, UI Issue 
		* Problem classes with 4-9 defects that have lowest \% of CR (25\% - 57\%): Compatibility Issue, Flaky Tests, Error Handling Improvement, Incorrect page rendering, Test Update
		* 54 defects were reported in 2018+ and among them 47 (87\%) defects have a CR.  Why in 7 defects there is no CR?  Reason: None of them have IMPL. 6 of them were solved by other issue. 1 is solved by updating the technology
		* 216 defects were reported before 2018,  137 have a CR (63\%)
		* Phabricator has helped with code review (this is an example of a tool that pushes info of issue resolution in the issues). We observed that throughout the years before 2018, the CR was XXX-XXX, while after 2018, if was XXX-XXX.
		* Splinter (decomissioned in 2019), MozReview (decom in 2018 Q3) --> Phab (2018 Q3)
		
		* However, VER is not done frequently (only in 41\%). VER mainly includes: manual testing such as acceptance tests (solution\_ver) and automatic testing via unit and integration tests. Testing is a fundamental step devs must perform, according to the Mozilla documentation. Code changes are not tested at least by a QA person and there is not indication by status or code changes. Reasons for that: (1) devs or QA members don't need to discuss testing in issue reports and if not why not (problem and testing is easy)?, (2) issue comments with Try results is not always posted, (3) they use some other system to document the tests. 
		* Tasks tend to have lower VER than the other two types (27\% vs 41-42\%). Tasks include these problem categories: Code Design, Code Improvement, 	Error Handling Improvement, Feature Development, Technology Update, Unnecessary Code Removal
		* Refactoring and testing issues have lower VER than implementation issues (19.6\% and 31.8\% vs 46.7\%)
		* Problem classes with 10+ defects that have lowest \% of VER (14\% - 46\%): Code Improvement ( 3 of 22), Test Failure (10 of 29), Code Design (19 of 55), Performance Optimization (5 of 11)
		* Problem classes with 6+ enhancements that have lowest \% of VER (25-33\%): Code Design (3 of 8), 
		Unnecessary Code Removal (0 of 3),  Code Improvement (1 of 4)
		
		* Reproduction is the least frequent stage (only done in 13\%), which means devs do not discuss repro test cases and repro attempts in the issues. First of all, not all issues need reproduction, only defects need reproduction.  But, even though, only 17\% of the defect reports have a repro attempt.  Why they are not documented?  No need to discuss: in most of the cases, they reproduce the issues in their IDE or somewhere else and do it with no problem. Why is repro then discussed? (1) the issues are hard to reproduce and they reach out for help (more resolution time, more sequence length, more commentators), (2) for designing the solution or analyzing the issue, reproduction is important to find the root cause of the problem or to compose a solution.
		- solution\_Design and reproduction: 53\% of issues with repro have design (overall support: 7\%)
		- analysis and reproduction: 73\% of issues with repro have analysis (overall support: 9\%)
		- ver and repro: 60\% of issues with repro have ver (overall support: 8\%)
		* Problem classes with 10+ defects that have lowest \% of REPR (0-8.33\%): Code Improvement (0 of 22), Code Design (3 of 52), Feature Development (1 of 11)
		* Problem classes with 10+ defects that have highest \% of REPR (24-48\%): Test Failure (7 of 29), UI Issue (8 of 32), Crash (11 of 23)
		
		* Solution design (42.1\%) and analysis (37.6\%) are not that often. Why? People do not need to discuss this because it is straightforward or they do it somewhere else (chats, etc.).
		* Defects are analyzed more than enhancements and tasks (44.8\% vs 20\% and 3.9\%). These defects are the most analyzed: Crash, Flaky Tests,  Incorrect page rendering, Test Update
		* Enhancements and defects have more design discussions than tasks (42.2\% and 43.3\% vs 38.5\%)
		* Refactoring issues are almost not analyzed compared to impl and testing issues (7.8\% vs 42.2\% and 45.5\%)  and are less designed (23.5\% vs 47.5\% and 31.8\%) -- testing issues are most analyzed and impl issues are most designed
	\end{itemize}
	
	\item How do the stages interact with other stages and how frequent are these interactions?
	\begin{itemize}
		\item Motivation: the to-be process includes the stages going one after the other, mostly linearly. We want to understand if this linear process is observed in the issues and see how stages co-occur and interact in a sequence. This would inform us on which stages more commonly occur or follow after one another.
		\item Results: 
		
		* All possible bigrams/trasitions. Transitions that don't happen: CR --> RP
		* Transitions that are extremely rare (with 1 or two occurrences): IMPL->REP, VER->REP, REP -> CR, REP->VER
		* First stages: IMPL, ANLYS, DES, REP (less freq)
		* Last stages: All, but CR, IMPL, VER are most frequent
		* Stages that tend to appear together (top 5 most frequent by \# of occurrences):
				*bigrams (1430): 
							IMPL -> CR (403), CR -> IMP (187), DES->IMPL (133), CR -> VER (94), IMPL ->VER (86)  (these five cover 903 / 1430 = 63\% of the occurrences )
							      +  VER -> IMPL (73), ANL -> IMPL (70), ANL -> DES (65), CR -> DES (41)  (the 9 cover 1152 / 1430 = 80.6\% of the occurrences )
							        --> Interplay between IMPL, CR and VER, and between DES and IMPL they are quite common, as expected. IMplement leads to the other and these can request more changes. Design leads to implementation.
							        --> ANL can lead to IMPL right away or to futher design.
							        --> Interestingly, CR leads to DES
				*trigrams (1109 occurrences):  95 of 120 possible trigrams (79.2\%)
					- 25 cover 80.6\% of the occurrences
					- 7 have the most frequent with a \% of occurrence between 3.1\% and 15.2\%: 
					IMPL,CR,IMPL (168 occur, 97 issues)
					CR,IMPL,CR (114 occur, 61 issues)
					DES,IMPL,CR (85 occur, 76 issues)
					IMPL,CR,VER (83 occur, 74 issues)
					ANLYS,IMPL,CR (51 occur, 50 issues)
					ANLYS,DES,IMPL (42 occur, 40 issues)
					CR,VER,IMPL  (34 occur,34 issues)
					- Interplay between IMPL, CR and VER, but also between ANL, DES, IMPL, and CR
		* REP,ANLYS	seem to appear together quite a bit as they are frequently found in issues containing both stages (in any part of the sequence):
				ANLYS -> REP: 73.53
				REP  -> ANLYS: 70.59
	
	\end{itemize}
	
	\item What is the process of issue resolution at Mozilla?
	
	- Present stats about sequences
	- Describe how we came up with the process
	- Describe what the figure is showing
	- Do some analysis of the process based on the figure and discuss
	
	
	\item What are the patterns of issue resolution?
	
	- Describe the pattern catalog
	- Describe the examples: what they mean and how they are instances of the overall process
	- Analyze the patterns and add examples that illustrate how the stages, sequences, process, and patterns can be useful
		- have an analysis of simple vs complex across issue types, categories, etc to show that our patterns can help detect anomalous problems and  predict the effort
				- Analysis of which problem categories  and issue types have more or fewer unique patterns and more or less simple/complex patterns (the more unique implies there are more diverse ways of solving them, the more complex ones are potentially more difficult to solve --> use stats to back this up)
		- discuss the two/three examples to discuss repetitive sequences or loops, which can indicate anomalies/things to improve in the process and how certain stages are done in sequence (goal: show the patterns can help analysts identify potentially anomalous issues: the hint at problems before going deep into the issues to understand what's going on)
		
		- Diversity of patterns across problem categories):
			* We observed 6 of 17 categories with more unique patterns (14 to 23) than the other 11 problem categories (1 to 11), covering 275 / 356 = 77.2\% issues)
			Defective Functionality (23 patterns in 43  issues)
			Code Design (21 patterns in 75  issues)
			UI Issue (22 patterns in  33 issues)
			Test Failure (17 patterns in 17  issues)
			Crash (17 patterns in  17 issues)
			Feature Development (14 patterns in 39  issues)
			
			This means that solving the issues for some categories is more diverse than the issues for other categories, despite having similar number of issues. For example, UI issues are solved with 22 patterns and code improvement issues are solved with 11 patterns, despite both categorizes having 32-33 issues.
			
			
			Code Improvement (11 patterns in  32 issues)
			
		- Diversity of problems solved by the patterns.
		  * The six most frequent patterns shown in table X were used to resolve issues of more than half of the categories  (9 to 11 of 17 categories). This illustrates that the same issue resolution pattern can solve issues of different problem categories. 
		
		- Problem categories that tend to be solved with simple/complex patterns:
		While all the categories are solved with simple patterns in most of the cases (55.8\% -  85.1.7\% of the issues), 5 of 17 categories tend to have more issues solved with complex patterns (104 of 356 = 29.2\%) than other 12 categories: 
		Code Design (22 of 75 issues)
		Defective Functionality (19 of 43)
		Feature Development (13 of 39)
		UI Issue (14 of 33)
		Crash (8 of 23)
		
		This indicate that some of problems in these categories require more effort to be solved.
		
		- 
		
	\item Discussion and Implications
	
	- linear vs iterative: 
			We found evidence that the issue resolution at Firefox is iterative, diverse, and more complicated than the typical issue resolution process. Different types of issues, which show different problems, are solved differently in a variety of ways. This process, while not prescribed, reflects the nature of software development, maintenance, and evolution, which is iterative and incremental. This alights with the spirit of iterative and agile methodologies of software development~\re, in constraint to rigid methodologies such as waterfall. At the same time, our results allows us to reflect and theorize on the nature of different kind of problems, in an attempt to make issue resolution in general and problem solving in particular more standard in software engineering. While the results only apply to Firefox, we expect to find diverse and iterative processes implemented in other OSS projects, at least in those resembling the nature and development dynamics used at Firefox (i.e., worldwide community-based process).
			
			- Our patterns show that issue resolution  in practice is iterative, which aligns with agile or iterative methodologies or models of developments (Rajlich's work shows it is kind of linear, but it is not)
			
	- Detecting anomalies in the process (to be vs as is process)
		  Our methodology, including the derivation and analysis of the stage sequences, process, and patterns of issue resolution at Firefox, allow us identify potential anomalies in the process. For example, we identified missing QA activities, especially VER. CR, at least before 2019, seems to have that problem. Given the importance of these activities and Mozilla should monitor these activities closely to validate these findings.
		  
		  We also here illustrate how the stage sequences and patterns can help identify problems in the process that require further investigation. We illustrate this with two examples. 
		  
		  			- The pattern (SOL_DES,IMPL,(CR|VER))+ is a complex pattern that hints for repeated stages and high resolution effort. Using this pattern and the stage sequence, we identified issue #557586 (a bug, Defective Funct), in which devs go through 15 stages to solve the issue: SOL_DES,IMPL,SOL_DES,IMPL,CR,SOL_DES,IMPL,CR,SOL_DES,IMPL,CR,IMPL,CR,IMPL,CR
		  			
		  			This represents a potential anomalous case because there are numerous sd impl and cr. 	We investigated deeper and found that six people were involved in the discussion, particularly in the SD discussion. They did not agree on the solution. The reporter was attaching the patch and trying to convince others about his solution. Hence, he had to submit multiple patches and they had to discuss the solution for a long time. The problem here was that there was lack of communication about the best solution for the problem. 
		  			
		  			If they focus on SD first and come up with a better solution, then they should be able to avoid unnecessary back and forth IMPL and CR.
		
		  - Documentation is needed because.... if we have tools integrated it can people do things better, but this requires tooling
		  
		  - Some problems and issues types require more effort, by knowing this they prioritize their work better.  Tool support to help people do etter resultion.
	  
	  
	  
	  
	
	
	
\end{itemize}
