
\section{Threats to Validity}
\label{sec:threats}



\textbf{Construct and Internal Validity}. 
\rev{Relying solely on issue reports poses a validity threat. Issue discussions may not capture all of Firefox's resolution activities, either because certain actions do not require documentation or were discussed/recorded in other artifacts or channels. This limitation may explain why some stages (\eg issue reproduction) are absent in certain issues. Consequently, the derived patterns should be interpreted with caution, as they reflect the resolution process \textit{documented in issue reports}, which may differ from the practical process. However, according to Firefox's documentation~\cite{firefox-patching}, issue reports are one of the primary artifacts for tracking Firefox changes, and developers are encouraged to document relevant problem information within them. Moreover, we are confident in the accuracy of traces for implementation, code review, and verification stages, due to the tool integration with the issue tracker, as well as the requirement for verification to mark an issue as ``VERIFIED." This provides confidence that issue report discussions capture the implemented resolution process.}





Researcher subjectivity and potential confirmation bias introduced during issue coding, resolution pattern inference, and results interpretation represent key validity threats.
To address these,  we implemented a rigorous open-coding methodology involving multiple coding phases.
Each issue report was reviewed multiple times, accompanied by discussion sessions between annotators. Both annotators critically annotated and verified the data at each phase, resolving disagreements through consensus. The results interpretation was thoroughly discussed and supported by data-driven evidence.

\textbf{External Validity}.  Our pattern catalog and results may not generalize to all issues from Firefox and to other systems, as is typical in case studies. 
This stems from the relatively small set of issues we coded to derive the patterns. 
To strengthen generalization, our study analyzed a statistically significant sample, in which the distribution of coded issues resembles that of all Firefox issues. \rev{We also 
annotated 20 issue reports of Chromium and GnuCash, and found that some of the most frequent Firefox patterns cover the resolution workflows found in the 20 issues, which implies that at least some of the derived patterns can be generalized to these projects. While the results are indicative, in-depth studies are needed to confirm these results and establish generalizability.}







