
\section{Background, Problem, and Motivation}
\label{sec:background}


\subsection{Issue Management during Software Evolution}


\textbf{Issue management} aims to track and manage all the change requests of a software system, including new feature developments, non-functional implementations, defect corrections, and enhancements to existing functionality~\cite{Zimmermann2010,zeller2009programs,rajlich2011software}. 
Issue trackers, such as Bugzilla~\cite{bugzillaBugzilla}, Jira~\cite{jira}, and GitHub Issues~\cite{github}, are communication and social platforms that allow coordination among different stakeholders around the process of issue management~\cite{bertram2009social,bertram2010communication,xia2013accurate}.
\textbf{Issue reports} are the main artifacts created and used during issue management~\cite{Zimmermann2010}, and  
typically include an issue title, a detailed issue description, metadata  (\eg issue severity and priority, operating platform, and product versions affected by the issue), and attachments (\eg system logs and screenshots). 
\looseness=-1

During the process of managing and solving the issues, stakeholders change the status of the issues (\eg from \textit{Assigned} to \textit{Resolved} and from \textit{Verified} to \textit{Closed}~\cite{eren2023analyzing,bugzila-bug-lifecycle}), and engage in discussions when needed. These discussions are recorded in \textbf{issue comments}, which document relevant information about the reported problems~\cite{zeller2009programs}, including possible circumstances in which the problem occurs, potential causes, how the code should be changed to address the issue, and more.
\looseness=-1




\subsection{Issue Resolution within the Issue Management Process}
\label{sub:background_issue_res}




The literature has defined the different phases that compose the issue management process.  Zhang \etal~\cite{zhang2016literature} report three major phases: issue understanding, triaging, and fixing (\aka \textbf{issue resolution}). In the first step, triagers read and understand the report to determine the issue type, priority, and severity. In the second step, the triagers assign the issue to an expert developer. Finally, in the issue resolution phase, the assigned developer locates the code that needs to be changed in response to the issue and implements the desired code change.

K. Saha \etal~\cite{saha2015understanding} define similar phases, including a fourth phase called issue verification, in which a developer verifies that the code change indeed addresses the issue and conforms with the quality standards.
Zeller~\cite{zeller2009programs} includes issue reporting, duplicate report identification, and fix delivery as part of the issue management process. Zeller also decomposes the issue resolution phase for bug reports into multiple steps: bug reproduction, isolation and localization, and fix implementation.
\looseness=-1 




Rajlich~\cite{rajlich2011software} defines \textit{software change} as a general process to modify and evolve software systems, based on change requests.
Rajlich defines seven phases of software change: initiation, concept location, impact analysis,  actualization, refactoring, verification, and conclusion. Initiation includes prioritization of change requests. All these phases except for initiation and conclusion are part of issue resolution, which aims to analyze and implement the solution to an issue.


In this paper, we investigate how developers perform issue resolution, a critical sub-process of issue management that aims to diagnose and solve the reported problems~\cite{eren2023analyzing}. 
This \textbf{issue resolution process} includes reproducing the reported bugs (for bug reports), understanding and analyzing the issues, designing a solution to the issue, implementing the solution, and validating/verifying the quality of the implementation.
\looseness=-1






\subsection{The Issue Resolution Process Implemented in Practice}



Although the issue resolution process includes different activities as discussed above, it is unclear how this process is implemented in practice to address \rev{different kinds of issues}, under various operating circumstances. As we discuss in the related work section (\Cref{sec:related_work}), prior work has focused on studying different aspects of issue reports and their management, including the overall issue management process based on issue status transitions, but no prior work has studied in detail the process that developers implement in practice to analyze and solve issues. We fill in this knowledge gap by analyzing issue discussions and studying  \textbf{patterns of issue resolution}. 
\looseness=-1

We motivate our work by discussing two examples of issue resolution at Mozilla Firefox~\cite{mozilla-firefox}. 
Issue report \#1029919~\cite{firefox-bug-1029919} describes a buggy behavior in the way Firefox renders a web page: when the user hovers over HTML buttons on a page, Firefox draws a border around the button.
The issue report contains rich information in the issue comments that help us understand the process followed to solve this issue. 
At first, developers reported multiple bug reproduction attempts, asking for additional information from the reporter. After it was successfully reproduced, the developer assigned to solve the issue posted the result of his investigation, describing the potential problem location and cause. The developer then attached two fixing patches describing the root cause,  the solution, and the potential impact of the solution on the system. The patches were reviewed by another developer and after the original developer corrected a few problems, the code reviewer inspected and approved the implementation. The code change was then integrated into Mozilla's code base, the reporter verified the fix, and a triager closed the issue. 



Another example is issue \#1718748~\cite{firefox-bug-1718748}, which describes a failure in Firefox's cross-platform component %
that handles UI rendering. 
The issue states that some buttons in Firefox's toolbar customization UI become invisible when switching to a dark theme. 
The reporter is a QA member who identified a prior commit and issue that could have introduced the bug (via Mozilla's mozregression tool~\cite{mozregression}). The developer responsible for that prior commit and issue, assigned to solve the issue, \rev{provided a patch with the description of the code change to correct the defect}. Another developer reviewed the code change (via the Phabricator code review tool~\cite{phabricator}) and a QA member then successfully verified the solution, marking the issue as \textit{Verified} and \textit{Fixed}.
\looseness=-1


Both examples illustrate different ways to resolve bugs related to Firefox's web page and UI rendering. 
In the first example, we observe all the expected major steps of issue resolution, however, in the second example, the process did not include any issue reproduction and analysis.
In both cases, these issue resolution steps were performed by different stakeholders, 
recording the activities and the relevant information obtained during the issue resolution. While the nature of the problems might have been different, it is clear that the issue resolution process that we would expect from theory can be implemented and recorded in issue reports in different ways. 
Our goal is to investigate these different approaches and determine if there are recurrent patterns in the process of solving different kinds of issues. We do so by qualitatively analyzing the discussions that developers document in issue reports.  
\looseness=-1
















\subsection{The Issue Resolution Process at Mozilla Firefox}
\label{sub:firefox_process}



We selected Mozilla Firefox~\cite{mozilla-firefox} as the subject of our study because (1) it is a mature and widely-used project with 19+ years of evolution, and (2) it has well-documented 
practices for issue management~\cite{firefox-bug-handling} and software development (\eg patching~\cite{firefox-patching}, code quality~\cite{firefox-code-quality,firefox-reviewer-checklist}, testing~\cite{firefox-security-bugs}, and  debugging~\cite{Working-Firefox}), which allow us to understand Firefox's issue resolution process in detail.  Mozilla Firefox is a multi-language, multi-platform open-source project that uses BMO, an adapted version of the Bugzilla issue tracker~\cite{bugzillaBugzilla}, to manage all the changes made to Firefox's source code~\cite{firefox-patching}.
\looseness=-1

\rev{All of Firefox's code changes} are documented in issue reports by end-users, community members, QA members, and developers during system usage, testing, and analysis~\cite{firefox-bug-pipeline}. Firefox has three \textbf{issue report types}~\cite{mozilla-bug-types}: defects, enhancements (\ie user-facing improvements), and tasks (\ie back-end improvements). These issues are triaged differently by a rotating group of engineering managers who are owners of a Firefox component and by QA members~\cite{firefox-triage,firefox-feature-triage,firefox-security-approval}. These members assess the issues and assign a correct issue type, severity, priority, target release, and other metadata (\eg security flags~\cite{firefox-security-approval}) to better prioritize and manage the problems. QA members, component owners, and developers are in charge of determining the resolution state of the issues (\eg \textit{Resolved - Won't fix} or \textit{Verified - Fixed}~\cite{firefox-feature-triage}). 

The open nature of the project makes Firefox's software development, and in particular issue resolution, a worldwide and distributed process. Developers are assigned to issues and work on one or more patches to address the problems. For diagnosing and solving defects, Firefox provides guidelines for using various debugging tools across different platforms~\cite{Working-Firefox}. 
Once the patches are completed, they are attached to the issue reports, requiring a code review through the Phabricator tool~\cite{phabricator}. The tool posts comments on issues whenever a code review is submitted. The code reviewer is mainly a component owner or peer, a newcomer mentor, and/or any other developer familiar with the modified code or module~\cite{firefox-patching}. The patch is tested in Try~\cite{firefox-try}, a system for running automated tests without integrating patches into Firefox's code base. Once the patches are approved, they are integrated (\aka landed), by the code reviewer, into the `autoland' repository, where regression tests are executed~\cite{testing-firefox-ml}. 
Once the tests pass and the code changes are further validated/verified by the QA team, they are merged by `code sheriffs' into `mozilla-central',  Firefox's main development repository~\cite{testing-firefox-ml}. Merging into `mozilla-central' occurs periodically or on demand (\eg when critical security fixes are validated)~\cite{shipping-firefox,testing-firefox-ml}. 

During the resolution process, the status of the reports is updated accordingly (\eg from \textit{Assigned} to \textit{Verified and  Fixed}). 
Information relevant to the issue (\eg failing regression test results), obtained at any moment during the process, may be posted as an issue comment. For example, failing regression test results are posted in the issues. 
Code changes in `mozilla-central' are integrated into `mozilla-beta' for additional quality assurance during a four-week beta cycle. After this, a release candidate build is generated, tested thoroughly, and made available as the next version of Firefox~\cite{shipping-firefox}.












