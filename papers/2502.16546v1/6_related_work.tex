
\section{Related Work}
\label{sec:related_work}







Researchers have proposed a variety of techniques to address issue management challenges and automate several tasks in the process~\cite{zou2018practitioners,Adnan:msr25}. For example, researchers have proposed automated techniques to better report issues~\cite{song_toward_2022,Fazzini:TSE22,song2023burt}, assess issue quality~\cite{mahmud:icpc2025,chaparro2019assessing,chaparro2017detecting,song2020bee}, 
predict the priority and severity of the issues~\cite{umer2019cnn,tian2015automated}, categorize issue types~\cite{somasundaram2012automatic,catolino2019not}, assign developers to issues~\cite{xia2013accurate,chaitra2022bug}, suggest potential duplicate issues~\cite{Zhou2012a,he2020duplicate,zhang2023duplicate}, reproduce buggy behavior~\cite{Zhao2019,feng2022gifdroid,zhang2023automatically,Zhao2019,zhao2022recdroid+}, localize buggy code files~\cite{akbar2020large,lee2018bench4bl,Ye2016b,ciborowska2022fast,Wong2014,Kochhar2014,florez2021combining,chaparro2019using,chaparro2017using,chaparro2019reformulating,saha2024toward,mahmud2024using,chaparro2016reduction},  and predict re-opened issues~\cite{zimmermann2012characterizing,shihab2010predicting}. 

Researchers have studied issues for a variety of purposes: to understand  decision-making~\cite{hesse2016documented} and 
the discourse used to describe issues~\cite{chaparro2017detecting,chaparro2016vocabulary}; extract decision information~\cite{mahadi2020cross}; understand stakeholders' information needs~\cite{Breu2010}; characterize/predict different kinds of issues such as \textit{won't fix} issues~\cite{panichella2021won}, fixed/resolved issues~\cite{Guo2010}, non-reproducible bugs~\cite{rahman2022works}, and bug/issue types~\cite{Tan2014,catolino2019not,limsettho2016unsupervised}; predict issue severity~\cite{Sureka2010}; understand workarounds~\cite{yan2023programmers} and visual content in issues~\cite{agrawal2022understanding};  
questions~\cite{huang2019empirical}, and information types in issues~\cite{arya2019analysis}.










Researchers have used automated mining of issue data (\eg status changes) and version control/code review data to identify development processes and assess delays and inconsistencies~\cite{marques2018assessing,krismayer2019using}. 
They have utilized process mining techniques to integrate data from different sources (\eg VCS, issue trackers, and mail archives)~\cite{poncin2011process, gupta2014process,mittal2014process} and proposed process mining techniques~\cite{rubin2007process,gupta2014nirikshan,saini2020control} to gain insight into development processes. Other work has studied the life cycle of issues by mining and analyzing issue state transitions~\cite{eren2023analyzing,dobrzynski2016tracing,wang2012predicting,coremans2023process}.
These works focused more broadly on issue management and identified transitions of issue states (\eg from ``Assigned" to ``In progress" to ``Closed"). 
However, issue states are often too broad to provide detailed insights into how stakeholders resolve issues in practice.




Unlike prior work, our research qualitatively analyzed issue reports to identify resolution stages, develop a process model, and uncover detailed patterns of issue resolution at Firefox. This in-depth analysis led to new insights into the issue resolution process. To our knowledge, we are the first to examine how the issue resolution process is actually implemented and discussed in practice, and how it differs from the theoretical models found in the literature.

