\section{Results}
\label{sec:results}

\subsection{\ref{rq:stages}: Issue Resolution Stages}
\label{sub:results_stages}

\rev{\Cref{tab:stages} lists the six identified issue resolution stages and reveals that not all issue reports include all stages, indicating that Firefox developers do not go through these stages, do not need to discuss them in the reports or discuss them in other systems or artifacts (\eg\ instant messaging tools). We discuss the stages starting with the most frequent ones.}
\looseness=-1



\looseness=-1

\subsubsection{\textbf{\impl}} 
This stage is frequently performed and discussed (in 92.1\% of the issues), which is expected as Firefox's issue tracker is integrated with the version control system (Mercurial).
\rev{Among the 28 issues not including any \impl, 25 issues were resolved in other issues, 
two issues were resolved by updating libraries in the host operating system, and the remaining issue was closed after more than four years of being open because the issue was no longer valid.}
\looseness=-1


\subsubsection{\textbf{\crv}} 
This stage is also frequently performed and discussed (in 74.2\% of the issues), which is expected as Firefox's issue tracker is integrated with Firefox's code review tools (\eg Phabricator \cite{phabricator}). 
While \crv is frequently discussed, it is not found in 25.8\% of the reports, especially in defect reports. We found that defects are the least discussed with \crv (31.9\%), compared to enhancement and task reports (6.7\% and 7.7\%). 
\rev{Phabricator~\cite{phabricator}, adopted in 2019, replaced MozReview~\cite{mozreview} and Splinter~\cite{splinter} and became Firefox's only code review tool. Analysis shows that 35\% of defects resolved in or before 2019 lacked \crv, compared to only 17\% after 2019. We found that post-2018 issues without \crv do not include code changes, as the issues were resolved in other issues. %
}


\looseness=-1

\subsubsection{\textbf{\ver}} 

\rev{This stage, covering manual and automatic testing, appears in only 41\% of issues. It is less common in task reports (26.9\%) than in defect and enhancement reports (42.2\% and 41.7\%). Refactoring and testing issues include \ver less often (19.6\% and 31.8\%) compared to implementation-related issues (46.7\%). Categories like Code Improvement, Test Failure, Code Design, and Performance Optimization have few \ver discussions, while Crashes (65\%), Feature Dev. (61.5\%), and UI Issues (63.6\%) show higher inclusion. This indicates still low \ver discussions in the issues and this is consistently found every year.}
\looseness=-1








\subsubsection{\textbf{\ia and \sd}} 

\rev{The \ia and \sd stages are infrequently discussed, appearing in only 37.6\% and 42.1\% of the issues, respectively. Defects are analyzed more frequently (\ie\ \ia in 44.8\% of defects) compared to enhancements (20\%) and tasks (3.9\%), with Crashes, Flaky Tests, Incorrect Page Renderings, and Test Updates being the most analyzed defects. \sd is more common in enhancements (42.2\%) and defects (43.3\%) than in tasks (38.5\%). Refactoring issues are the least analyzed (7.8\%) compared to implementation (42.2\%) and testing-related issues (45.5\%).}
\looseness=-1




\subsubsection{\textbf{\ir}} 
\rev{\ir is the least frequent stage, appearing in only 13.2\% of issues, with just 17\% of defects including it. This suggests that Firefox stakeholders rarely discuss bug reproduction in issue reports.
We observed that \ir is included when bugs are difficult to reproduce or when reproduction is necessary to identify the root cause or localize the bug. The first scenario typically leads to more effort in solving the issues: compared to defects without \ir, defects with \ir take longer to resolve (avg/med: 82.9/5.5 vs. 79.3/18.5 days) and involve more commentators (avg/med: 4.9/4 vs. 7.7/7). 
This is validated by a Mann-Whitney U test~\cite{mcknight2010mann} with $\alpha = 0.05$, with p-values nearly 0.
 The second scenario is supported by the data: 73\% of the issues with \ir include \ia.}



\rqanswer{\textbf{\ref{rq:stages} Findings}: The six stages of issue resolution found in Firefox issue reports appear with varying frequency across different issue reports and problem categories. \impl and \crv are the most frequent stages while \ir is the least frequent.

\looseness=-1}



\subsection{\ref{rq:interactions}: Interactions between Issue Resolution Stages}
\label{sub:results_interactions}
We examined the 356 stage sequences obtained in \Cref{sub:process} by analyzing the frequency of stage bi-grams and tri-grams in the sequences.
Bi-grams are pairs of consecutive stages (\texttt{\textbf{S}},\texttt{\textbf{T}}) in a sequence, which represent possible stage transitions (\texttt{\textbf{S}}\trs\texttt{\textbf{T}}) in the resolution process. In our data, we found all possible bi-grams between stages, except for  \crvs \trs \irs, and five extremely rare transitions (appearing only once or twice): \impls \trs \irs, \vers \trs \irs, \irs \trs \crvs, and \irs \trs \vers. All these transitions include \ir (\irs) as the source or target stage.
\looseness=-1

Of the 1,430 bi-grams found in the sequences, nine are the most recurrent, covering 80.6\% of the bi-gram occurrences: \impls \trs \crvs (403 cases), \crvs \trs \impls (187), \sds \trs \impls (133), \crvs \trs \vers (94), \impls \trs \vers (86), \vers \trs \impls (73), \ias \trs \impls (70), \ias \trs \sds (65), and \crvs \trs \sds~(41). From these transitions, we observe the interplay among \impl, \crv, and \ver, in which \impl undergoes quality assurance activities and these also lead to additional code changes (or to \sd~-- see \crvs \trs \sds above). \sd (\sds) can lead to code changes, and \ia (\ias) can result in code changes or solution design activities. Notably, while \ias \trs \irs and \irs \trs \ias are not among the most frequent bi-grams, they appear in 73.5\% and 70.6\% of the issues containing both stages (34). This further supports our finding that \ia and \ir typically occur together.

The analysis of tri-grams, sets of three consecutive stages in a sequence (\texttt{\textbf{S}}\trs\texttt{\textbf{T}}\trs\texttt{\textbf{U}}), not only provides extra evidence of the interplay among \impl, \crv,  and \ver, but also the relationship among \ia, \sd, \impl, and \crv. Of 1,109 tri-grams found in the sequences, 25 are the most frequent, covering 80.6\% of the tri-grams, with the following three being the most frequent: \impls \trs \crvs \trs \impls (168 occurrences in 97 issues), \crvs \trs \impls \trs \crvs (114 occurrences in 61 issues), and \sds \trs \impls \trs \crvs (85 occurrences in 76 issues). 

\rqanswer{\textbf{\ref{rq:interactions} Findings}: Firefox's developers switch among different resolution stages to solve issues. 
	Stage bi-gram and tri-gram analysis reveal that developers frequently engage in three scenarios: 1) reproducing the issues (\irs) along with issue analysis (\ias)  to confirm the issues and reason about them; 2) analyzing the issues (\ias) along with solution design (\sds), and then engaging in implementation (\impls) and code reviews (\crvs); and 3) inspecting (\crvs) and verifying the implemented solution (\vers), adapting the implementation when needed (\impls).
\looseness=-1}



\subsection{\ref{rq:process}: Issue Resolution Process}
\label{sub:results_process}

\Cref{fig:overall_process} shows the overall issue resolution process at Firefox, derived from the bi-gram analysis we performed on the 356 stage sequences. The process is a directed graph in which the nodes represent the six resolution stages (\eg~\impl or \impls) and the edges represent transitions between stages (\eg~\impl \trs \crv or \impls \trs \crvs). The nodes with  \colorbox{StateGreen}{green} and \colorbox{StateRed}{red} borders are initial and end nodes, selected from the most frequent initial and end stages in the sequences. All nodes imply a loop to itself, indicating that the stage can be performed multiple times in a row.

The process includes only the most frequent bi-grams \texttt{\textbf{S}}~\trs~\texttt{\textbf{T}}, selected based on the \textit{proportion of bi-grams} starting with \texttt{\textbf{S}} (\texttt{\textbf{S}}\trs \texttt{\textbf{*}}) that contain \texttt{\textbf{S}}\trs \texttt{\textbf{T}}. This proportion is shown in \colorbox{BoxesBlue}{blue boxes} of the edges in \Cref{fig:overall_process}. For example, \impls  \trs \crvs has a frequency of 403/530 =  \colorbox{BoxesBlue}{76\%} since  from all 530 bigrams that start with \impls (\impls \trs \texttt{\textbf{*}}) there are 403 occurrences of \impls  \trs \crvs. The frequencies of all transitions starting from a given node add up to at least 90\%. For example, all transitions coming out of \impls add up to 92.3\%. The \colorbox{BoxesYellow}{yellow boxes} of the edges represent the \textit{proportion of issues} containing \texttt{\textbf{S}} and \texttt{\textbf{T}} that contain the bigram \texttt{\textbf{S}}\trs\texttt{\textbf{T}}. For example, 254 of 264 (\colorbox{BoxesYellow}{96.2\%}) issues with both \impls and \crvs contain \impls \trs \crvs.

We make two observations about the process in \Cref{fig:overall_process}:
\begin{itemize}[itemindent=0cm,leftmargin=0.3cm]
	\item The process deviates from the (theoretical) linear process outlined by the existing literature and Firefox's documentation (see \Cref{sub:background_issue_res,sub:firefox_process}). Instead of a linear sequence of stages (starting with \irs and going through every stage from left to right until \ver and/or \crv are completed --- see the path with green transitions in \Cref{fig:overall_process}), the process is more complicated than expected as it includes iterative interactions between stages. This means developers go back and forth from one stage to another, forming different workflows of issue resolution.
	\item Some nodes have a high number of incoming and outgoing transitions, indicating the level of importance of such stages in the process. Specifically, \sd has four incoming and four outgoing transitions and \impl has five incoming and two outgoing transitions. These stages are pivotal because they allow for stage switches from and to many of the other stages. \crv and \ver are the second most important stages, both having three incoming and three outgoing transitions, while \ia and \ir are less important with fewer transitions.
 \looseness=-1
\end{itemize}


\rqanswer{\textbf{\ref{rq:process} Findings}: Firefox's issue resolution follows an iterative process that deviates from the theoretical linear process. In this process, developers go back and forth from one state to another as needed to solve the issues. \sd and \impl, followed by \crv and \ver, play a key role as they are the source and target for most of the other stages.
}



\begin{figure}[t]
	\centering
	\includegraphics[width=1\linewidth]{figures/overall_process.pdf}
	\caption{Overall Issue Resolution Process of Firefox}
	\label{fig:overall_process}
\end{figure}

\subsection{\ref{rq:patterns}: Issue Resolution Patterns}
\label{sub:results_patterns}




\Cref{fig:overall_process} shows that Firefox's issue resolution process differs from the expected linear process from prior work. However, the figure does not show how much the process differs and the different instances of the process that developers follow.

Employing the qualitative approach described in \Cref{sub:resolution_patterns}, we identified 47 distinct instances of the process, which we call \textit{issue resolution patterns}. These patterns appear in the 356 issue reports with varying frequency and complexity: a pattern appears in 1 to 64 reports (7.6 on avg, 4 median), 20 patterns are categorized as complex (they imply high issue resolution effort) and 27 as simple (they imply low resolution effort). Of all patterns, 18 patterns are the most recurrent: they are found in 5 to 64 reports (16 on avg., 9.5 median), covering 287 reports (80.6\%); 12 are simple, and 6 are complex.  \Cref{tab:patterns} shows the 10 most recurrent patterns -- the entire pattern catalog can be found in our replication package~\cite{repl_pack}. 

\subsubsection{\textbf{Pattern Examples}} We describe two patterns of different kinds to illustrate different workflows of issue resolution.

The pattern \texttt{\textbf{A,SD,I,(I$\mid$CR$\mid$V)?}} represents the process in which developers first analyze the reported problem (\ias). They then design the solution (\sds) (\eg propose a potential solution or review a proposed solution), and then implement the solution~(\impls). Developers may then review the code (\crvs) and/or test the code changes to verify if they solved the issue (\vers). Based on QA feedback, more code changes may occur (\impls). This pattern is  \textit{simple} because it includes only three ``mandatory'' stages (\texttt{\textbf{A,SD,I}}) followed by three ``optional'' stages (\texttt{\textbf{(I$\mid$CR$\mid$V)?}}).
\looseness=-1

The pattern {\texttt{\textbf{(SD,I,(CR$\mid$V))+}}} suggests a process in which \sd, \impl, \crv, and/or \ver are performed repetitively to resolve the issue. In the repetitive series, {\texttt{\textbf{(CR$\mid$V)}}} means that either one or both can appear after an \sds and an \impls. To resolve issues, developers perform four distinct stages where all stages are repetitive, making this pattern \textit{complex}.
\looseness=-1



\begin{table}[t]
	\caption{Top 10 Frequent Issue Resolution Patterns}
	\label{tab:patterns}
	\resizebox{\columnwidth}{!}{%
		\begin{tabular}{l|l|c|c}
			\hline
			\multicolumn{1}{c|}{\textbf{\begin{tabular}[c]{@{}c@{}}Pattern\end{tabular}}} & \multicolumn{1}{c|}{\textbf{Description}}                                                                                                                                                                  & \textbf{\begin{tabular}[c]{@{}c@{}}Com-\\ plexity\end{tabular}} & \textbf{\begin{tabular}[c]{@{}c@{}}\# of\\ Issues\end{tabular}} \\ \hline
			\texttt{\textbf{I,CR,I?}}                                                                                            & \begin{tabular}[c]{@{}l@{}}Implement the solution and review the code;\\ followed by another optional implementation.\end{tabular}                                                                         & Simple                                                          & 64                                                           \\ \hline
			\texttt{\textbf{A,I,(I$\mid$CR$\mid$V)?}}                                                                                     & \begin{tabular}[c]{@{}l@{}}Analyze the problem and implement the solution;\\ followed by another optional I or CR or V or\\ any combination.\end{tabular}                                                  & Simple                                                          & 32                                                           \\ \hline
			\texttt{\textbf{(I,(CR$\mid$V))+}}                                                                                        & \begin{tabular}[c]{@{}l@{}}Implement the solution; review the code and/or\\ verify the implementation; I, CR and/or V\\ repeat more than once.\end{tabular}                                                & Complex                                                         & 28                                                           \\ \hline
			\texttt{\textbf{SD,I,CR,(I$\mid$V)?}}                                                                                      & \begin{tabular}[c]{@{}l@{}}Design and implement the solution and review the\\ code; followed by another optional I or V or both.\end{tabular}                                                              & Simple                                                          & 24                                                           \\ \hline
			\texttt{\textbf{A,SD,I,(I$\mid$CR$\mid$V)+}}                                                                                  & \begin{tabular}[c]{@{}l@{}}Analyze the problem, design, and implement the\\ solution; followed by another optional I or CR\\ or V or any combination.\end{tabular}                                         & Simple                                                          & 22                                                           \\ \hline
			\texttt{\textbf{I}}                                                                                                 & Implement the solution.                                                                                                                                                                                    & Simple                                                          & 21                                                           \\ \hline
			\texttt{\textbf{I,CR,V,I?}}                                                                                           & \begin{tabular}[c]{@{}l@{}}Implement the solution, review the code, and verify\\ the implementation; followed by another optional I.\end{tabular}                                                          & Simple                                                          & 16                                                           \\ \hline
			\texttt{\textbf{SD,(I,(CR$\mid$V))+}}                                                                                     & \begin{tabular}[c]{@{}l@{}}Design the solution; implement the solution,\\ review code and/or verify the implementation;\\ I, CR and/or V repeat more than once.\end{tabular}                               & Complex                                                         & 13                                                           \\ \hline
			\texttt{\textbf{(SD,I,(CR$\mid$V))+}}                                                                                      & \begin{tabular}[c]{@{}l@{}}Design and implement the solution; review the\\ code, and/or verify the implementation;\\ SD,I, CR and/or V repeat more than once.\end{tabular}                                 & Complex                                                         & 12                                                           \\ \hline
			\texttt{\textbf{A,(I,(CR$\mid$V))+}}                                                                                     
			& \begin{tabular}[c]{@{}l@{}}Analyze the problem; implement the solution,\\ review code, and/or verify the implementation;\\ I, CR and/or V repeat more than once.\end{tabular}                              & Complex                                                         & 7                                                            
			\\ \hline
			\multicolumn{4}{c}{
				\scriptsize{
					\texttt{\textbf{R}}=\ir, \texttt{\textbf{A}}=\ia, \texttt{\textbf{SD}}=\sd,}} \\
			
			\multicolumn{4}{c}{
				\scriptsize{\texttt{\textbf{I}}=\impl, \texttt{\textbf{CR}}=\crv, \texttt{\textbf{V}}=\ver}}
			
			
		\end{tabular}%
	}
\end{table}

\subsubsection{\textbf{Process and Pattern Diversity}} All the identified patterns represent instances of the issue resolution process. The 47 instances indicate a wide variety of ways to solve Firefox issues. While more generalized patterns can be formed from the 47 patterns, our qualitative approach carefully identified the patterns to accurately reflect the observed process from the issues. 
We did not forcefully merge patterns into more general ones but did validate the patterns against the process from \Cref{fig:overall_process} (which was derived quantitatively).
\looseness=-1

The diversity of the patterns/process is observed across problem categories. Six of the 17 problem categories have more unique patterns (14 to 23) than the remaining 11 categories (1 to 11 patterns). These six categories are: Defective Functionality (23 unique patterns found in 43  issues), Code Design (21 patterns in 75  issues), 
UI Issue (22 patterns in  33 issues), Test Failure (17 patterns in 17  issues), Crash (17 patterns in  17 issues), and
Feature Development (14 patterns in 39  issues). 
\looseness=-1

Issue resolution for some categories is more diverse than for other categories, despite having a similar number of issues. For example, UI Issues are solved with 22 patterns, and Code Improvement issues are solved with 11 patterns, despite both categories covering 32-33 issues. We also found that the six most frequent patterns shown in \Cref{tab:patterns} were used to resolve issues of more than half of the categories  (9-11 of 17 categories). This illustrates that the same issue resolution pattern can solve problems of different kinds.
\looseness=-1

The diversity of the patterns/process is also observed throughout Firefox's lifespan, from 2010 to 2023. During these 14 years, the five most frequent patterns (found in 48\% of the issues) were observed in 11 to 14 different years. The 10 most frequent patterns (found in 67\% of the issues) are found in 7 to 14 different years. The 39 patterns appearing in two or more issues were used in 2 to 14 years.

\subsubsection{\textbf{Pattern Complexity and Resolution Effort}} \Cref{tab:patterns_issue_types} shows that \rev{70.8\% of the issues (252 of 356)} are solved with the 27 simple patterns. \rev{These, compared to issues solved with a complex pattern, are solved faster (avg/med: 58/5 vs. 119.8/19.5 days), require fewer stages in the process (avg/med: 2.9/3 vs. 9.9/9), and include fewer commentators (avg/med: 4.4/4 vs. 7.4/7). 
	A Mann–Whitney U test~\cite{mcknight2010mann} (at $\alpha = 0.05$) confirmed these differences across all these factors with \textit{p-value} = 0.0.}
	\looseness=-1

While all the problem kinds are solved with simple patterns in most of the cases (55.8\% -  85.1.7\% of the issues), 5 of 17 categories tend to have more issues solved with complex patterns (104 of 356 = 29.2\%), compared to the other 12 categories. These five categories are:  Code Design (22 of 75 issues are solved with a complex pattern),  Defective Functionality (19 of 43), Feature Development (13 of 39), UI Issue (14 of 33), and Crash (8 of 23). This indicates that these categories contain issues that require more effort to be solved. 
\rev{Despite the issues in these categories being solved with similar resolution time (avg/med: 72.4/9 vs 81.6/6 days), they include more stages in their process (avg/med: 5.2/4 vs. 4/3) and more commentators (avg/med: 5.8/5 vs. 4.4/4) with statistical significance (Mann–Whitney U test, \textit{p-value} = 0.0), suggesting potentially higher resolution effort.}

\looseness=-1

\Cref{tab:patterns_issue_types} also reveals that 76.9\% of the tasks are solved using a simple pattern whereas 70.7\% of the defects and 68.3\% of the enhancements are solved with a simple pattern. \rev{As pattern complexity suggests, compared to defects and enhancements, tasks require less effort: they are solved significantly faster (avg/med: 9/4 vs. 82.3/7 and 76.9/11.5), require fewer process stages  (avg/med: 3.6/2 vs. 5.2/4 and 5/4), and include fewer commentators (avg/med: 4/3 vs. 5.4/4 and 5.3/5). These results are statistically significant, according to the Mann–Whitney U test (p-values = 0.02, 0.03, and 0.002, respectively.)}
\looseness=-1


\rqanswer{\textbf{\ref{rq:patterns} Findings}: The 47 identified issue resolution patterns indicate that solving issues at Firefox is done in a wide variety of ways. Of these, 18 patterns are recurrently found in 80.6\% of Firefox issue reports. 
The process of issue resolution in Firefox is diverse and far from linear. The diverse and iterative nature of the process is consistently observed throughout Firefox's 14 years of evolution.}


\begin{table}[t]
\caption{Number of Issues Across Issue Types}
\label{tab:patterns_issue_types}
\centering
\begin{tabular}{c|cc|c}
	\hline
	\multirow{2}{*}{\textbf{Issue Type}} & \multicolumn{2}{c|}{\textbf{Pattern Complexity}}        & \multirow{2}{*}{\textbf{Total}} \\ \cline{2-3}
	& \multicolumn{1}{c|}{\textbf{Complex}} & \textbf{Simple} &                                 \\ \hline
	\textbf{Defect}                      & \multicolumn{1}{c|}{79}               & 191             & \textbf{270}                    \\ \hline
	\textbf{Enhancement}                 & \multicolumn{1}{c|}{19}               & 41              & \textbf{60}                     \\ \hline
	\textbf{Task}                        & \multicolumn{1}{c|}{6}                & 20              & \textbf{26}                     \\ \hline
	\textbf{Total}                       & \multicolumn{1}{c|}{\textbf{104}}     & \textbf{252}    & \textbf{356}                    \\ \hline
\end{tabular}
\end{table}



\rev{
\subsection{\ref{rq:pattern_usefulness}: Use Cases for the Issue Resolution Patterns}
\label{sub:pattern_usefulness}

The two interviewed Mozilla developers (\ie D1 and D2) identified the following use cases for the derived patterns:

\subsubsection{\textbf{Identifying Issues with a Complex Resolution}} 
D1 and D2 suggested that the patterns could help detect issues with complex resolutions, especially those involving repetitive stages, which may signal excessive time and effort spent by developers. D1 noted that such patterns could indicate when a bug takes an unexpectedly complex path, explaining, ``If you went through three different implementations and three different verifications and it still didn't work, something went wrong here." D2 also emphasized the value of a tool that identifies these complex issues, saying, ``It might be interesting to have some sort of tool that watches the bugs and when it sees this snowball effect... it could alert a product person that this bug is chewing up a lot of time." Both developers agreed on the importance of early detection of complex resolutions. Both emphasized that detecting these issues would help understand why a process is taking longer than expected, allowing for timely corrections. 
	
\subsubsection{\textbf{Identifying Issues Not Following the Expected Process}} 
D1 suggested that a tool could be useful for detecting issues that deviate from expected workflows, especially those requiring human verification. He explained, ``Some bugs require human verification... it requires installing third-party software on a machine. [...] And it's up to developers to highlight when this is the case." D1 emphasized that a tool could help by alerting developers when an issue seems to need third-party support for verification, stating, ``If you had a tool that said, hey, this bug that you open looks like it might need some third-party support for verification, that might actually be a helpful thing."

	
\subsubsection{\textbf{Identifying Potentially Complex Code Components}} 
D2 suggested that tracking the complexity of issue resolution in specific Mozilla Firefox code components or modules could reveal underlying quality issues, such as technical debt or accumulated code complexity. As D2 put it, ``You could track the complexity of the issue resolution process in a given module... and get insights like, hey, it looks like most of the time when you touch this area of code it ends up being a slog." D2 emphasized that such insights could signal the need for refactoring, stating, ``Maybe it's time to refactor this? Maybe it's time to clean this up... this part of the code base is a tar pit and we probably want to spend some resources making it less ornery." 
\looseness=-1
	
	
\subsubsection{\textbf{Improving Bots to Detect Unsolvable Issues}} 
D2 suggested that the resolution patterns could help improve the heuristics of existing bots (\eg bugbug~\cite{bugbug}) used to process Mozilla issues, enabling them to better identify issues that are unsolvable or particularly challenging to solve. As D2 explained, ``If you could identify signs that this bug is not going to be solved or is about to fall through the cracks, that would be pretty cool." He noted that current bots already attempt to detect when issues have ``fallen through the cracks" and need attention, saying, ``We have some bots that do that kind of work." 
\looseness=-1
	
\subsubsection{\textbf{Suggesting and Decomposing Meta-Issues}} 
Both D1 and D2 suggested that issues with complex resolutions, as indicated by the complex patterns, might represent meta-issues: large issues that could be broken down into smaller, more manageable issues. They proposed that a tool capable of flagging these cases and suggesting possible decompositions would be highly beneficial. As D1 explained, ``We have this idea of a meta bug, which is a bug which hosts a whole bunch of related bugs," and suggested the tool could flag such cases and offer ideas like, "could this be split? Here are some topics that it sounds like you could split this down into."
	
\subsubsection{\textbf{Training Junior Developers}} 
D1 and D2 both indicated that the patterns could serve as valuable training tools for junior Mozilla developers, offering insights into the practical aspects of issue resolution. D1 noted that junior developers often have high expectations and approach issues linearly, seeking a perfect solution. However, D1 emphasized that the patterns show the issue-resolution process is typically incremental and iterative, involving multiple cycles of code review and verification. As D1 explained, ``You will probably have to iterate... you will probably have to go through this solution more than once. And that's okay. That's expected. It's part of the job." 

Both developers suggested that automation is needed to realize such use cases, particularly tools that identify patterns in issue discussions and classify them as simple or complex. Our future work will develop such tools to automate the identification of textual content in issue comments (\eg issue resolution activities) and use algorithms to derive sequences of stages and patterns. This process will likely combine machine learning with heuristic-based approaches.

}

\rqanswer{\textbf{\ref{rq:pattern_usefulness} Findings}: \rev{The interviewed Mozilla developers suggested that the resolution patterns could help identify complex issues, workflow deviations, and low-quality code components, improve bots for detecting unsolvable issues, decompose large issues, and train junior developers. 
}
}
