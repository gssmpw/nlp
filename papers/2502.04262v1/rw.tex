Our work draws heavily from the literature on semiparametric inference and double machine learning~\citep{robins1994estimation,robins1995semiparametric,tsiatis2006semiparametric,chernozhukov2018double}. In particular, our estimator is an optimal combination of several Augmented Inverse Probability Weighting~(\aipw) estimators, whose outcome regressions are replaced with foundation models. Importantly, the standard $\aipw$ estimator, which relies on an outcome regression estimated using experimental data alone, is also included in the combination. This approach allows \ours~to significantly reduce finite sample (and potentially asymptotic) variance while attaining the semiparametric \emph{efficiency bound}---the smallest asymptotic variance among all consistent and asymptotically normal estimators of the average treatment effect---even when the foundation models are arbitrarily biased.


\paragraph{Integrating foundation models}
Prediction-powered inference~(\ppi)~\citep{angelopoulos2023prediction} is a statistical framework that constructs valid confidence intervals using a small labeled dataset and a large unlabeled dataset imputed by a foundation model. $\ppi$ has been applied in various domains, including generalization of causal inferences~\citep{demirel24prediction}, large language model evaluation~\citep{fisch2024stratified,dorner2024limitsscalableevaluationfrontier}, and improving the efficiency of social science experiments~\citep{broskamixed,egami2024using}. However, unlike our approach, $\ppi$ requires access to an additional unlabeled dataset from the same distribution as the experimental sample, which may be as costly as labeled data. Recent work by \citet{poulet2025prediction} introduces 
Prediction-powered inference for clinical trials ($\ppct$), an adaptation of $\ppi$ to estimate  average treatment effects in randomized experiments without any additional  external data. $\ppct$ combines the difference in means estimator with an 
$\aipw$ estimator that integrates the same foundation model as the outcome regression for both treatment and control groups. However, our work differs in two key aspects:
(i) $\ppct$ integrates a single foundation model, and (ii) $\ppct$ does not include the standard $\aipw$ estimator with the outcome regression estimated from experimental data. As a result, $\ppct$ cannot achieve the efficiency bound unless the foundation model is almost surely equal to the underlying outcome regression. 


 



\paragraph{Integrating observational data} There is growing interest in augmenting randomized experiments with data from observational studies to improve statistical precision. One approach involves first testing whether the observational data is compatible with the experimental data~\citep{dahabreh2024using}---for instance, using a statistical test to assess if the mean of the outcome conditional on the covariates is invariant across studies \cite{luedtke2019omnibus,hussain2023falsification,de2024detecting}—and then combining the datasets to improve precision, if the test does not reject. These tests, however, have low statistical power, especially when the experimental sample size is small, which is precisely when leveraging observational data would be most beneficial. To overcome this, a recent line of work integrates a prognostic score estimated from observational data as a covariate when estimating the outcome regression~\citep{schuler2022increasing,liao2023prognostic}. However, increasing the dimensionality of the problem---by adding an additional covariate---can increase estimation error and inflate the finite sample variance. Finally, the work most closely related to ours is \citet{karlsson2024robust}, that integrates an outcome regression estimated from observational data into the \aipw~estimator. In contrast, our approach is not constrained by the availability of well-structured observational data, since it leverages black-box foundation models trained on external data sources.