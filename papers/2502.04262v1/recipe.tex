In this section, we provide a step-by-step guide for practitioners to implement $\ours$ using Large Language Models (LLMs). Our guide focuses on LLMs as they are both widely accessible and have demonstrated strong accuracy in predicting human behavior \citep{grossmann2023ai}. As a concrete example, we present a political science experiment that evaluates the effect of free speech framings on opposition to cancel culture among Americans~\citep{fahey2023principled}. We provide simplified prompts here and refer readers to~\Cref{apx:prompts} for the full LLM prompts. 




\begin{enumerate}
\item \textbf{Extract participant information.} Extract the tuples $Z_i = (X_i, Y_i, A_i)$ for each participant $i$ in the study. 
 In~\citet{fahey2023principled}, covariates include age, gender, ideology, income, and religion. The treatment represents a scenario where an Antifa protest is banned:
for safety reasons only ($A = 0$), or
 for safety reasons and cancel culture ($A = 1$).
The outcome is measured on a scale from 1 to 5, as the level of agreement with the statement: \emph{``Cancel culture is a big problem in today’s society."}



\item \textbf{Construct system prompts.} For each participant $i$, create a \emph{persona} that matches $X_i$ and guides the LLM in simulating  responses. In this study, personas summarize the participant's demographics. The constructed persona is then used as the \emph{system} prompt for the LLM; see \Cref{fig:example_system} for an example.








\item \textbf{Construct user prompts.} The \emph{user} prompt includes the experimental treatment, the outcome question, and instructions to guide the LLM (see \Cref{fig:example_user}). We prompt the LLM to generate a synthetic outcome for both conditions (treatment and control). The final instruction is sampled from a predefined pool to introduce variability in the LLM's responses; we provide examples in \Cref{apx:multiprompt}.




\item \textbf{Simulate 
 outcome responses.} Query the LLM using the user and system prompts. Validate that the responses are numeric and conform to the specified outcome scale. For experiments where multiple instructions are sampled, compute the average response.


\item \textbf{Estimate treatment effects.} Compute the confidence interval $\CC_{\ours}^\alpha$ following~\cref{algo:haipw}. We find that using cross-fitting for the standard $\aipw$ estimator is key for coverage in small-sample settings.


    
\end{enumerate}


\begin{figure*}[t!]
    \centering
    \begin{subfigure}{0.48\textwidth}
        \centering
        \begin{tcolorbox}[
            colframe=pierCite,
            colback=white,
            coltitle=white,
            title=Example System Prompt,
            fonttitle=\bfseries,
            boxrule=0.5mm,
            width=\linewidth
        ]\vspace{2.4mm}
        You are a 35-year-old female, politically Democrat, holding liberal views. Additionally, your religion is Christianity, and you once or twice a month attend religious services. You reside in a building with two or more apartments, and your household has a yearly income of \$85,000 to \$99,999. 
        
        You are responding to a scenario reflecting a debate involving college campus events and broader social issues.
       \vspace{2.4mm} \end{tcolorbox}
        \caption{}
        \label{fig:example_system}
    \end{subfigure}
    \hfill
    \begin{subfigure}{0.48\textwidth}
        \centering
        \begin{tcolorbox}[
            colframe=pierCite,
            colback=white,
            coltitle=white,
            title=Example User Prompt,
            fonttitle=\bfseries,
            boxrule=0.5mm,
            width=\linewidth
        ]
        \textbf{Treatment:}  \textit{A student organization denied Antifa's request for a rally, citing safety concerns due to altercations at similar events. Antifa plans to appeal the decision.} \\
        \textbf{Outcome question:} Do you agree or disagree with the statement: \\
        \emph{``Cancel culture is a big problem in today’s society.''} Choose an integer between 1 (definitely agree) and 5 (definitely disagree).\\
        \textbf{Instruction}: Reflect on the scenario and use your reasoning to assign a value. 
        \end{tcolorbox}
        \caption{}
        \label{fig:example_user}
    \end{subfigure}
\caption{\small{Examples of a system prompt and a user prompt used to generate synthetic responses for \citet{fahey2023principled}.}}



\end{figure*}