\documentclass{article}

\usepackage[margin = 0.9in]{geometry}
\usepackage{fancyhdr}
\pagestyle{fancy}
\fancyhead{}
\fancyhead[C]{Lightweight Dataset Pruning without Full Training via Example Difficulty and Prediction Uncertainty}
\usepackage[parfill]{parskip}
\usepackage[labelfont=bf]{caption}
\usepackage[compress, sort, round]{natbib}
\usepackage[x11names,table,xcdraw]{xcolor}
\usepackage{booktabs} % for professional tables
\usepackage{multirow}
\usepackage{color, colortbl, xcolor}
\usepackage[T1]{fontenc}

\RequirePackage{algorithm}
\RequirePackage{algorithmic}
\usepackage{graphicx}
\usepackage{siunitx}
\usepackage{hyperref}
% \usepackage[colorlinks=true, linkcolor=blue, citecolor=blue, urlcolor=blue, pdfborder={0 0 0}]{hyperref}

\usepackage{subcaption}
\usepackage{caption}
\usepackage{wrapfig}
% Use this instead of \paragraph{}
\def\tightparagraph#1{\noindent\textbf{#1}~~}

\title{\bfseries Lightweight Dataset Pruning without Full Training via Example Difficulty and Prediction Uncertainty}


\author{
    %1
    Yeseul Cho \thanks{Authors contributed equally to this paper.} \\
    Graduate School of AI, KAIST \\
    \texttt{cyseul@kaist.ac.kr} \\ \\
    %3
    Changmin Kang \\
    Graduate School of AI, KAIST \\
    \texttt{cmkang8128@kaist.ac.kr} \\ \\
    \and
    %2
    Baekrok Shin \footnotemark[1] \\
    Graduate School of AI, KAIST \\
    \texttt{br.shin@kaist.ac.kr} \\ \\
    %4
    Chulhee Yun \\
    Graduate School of AI, KAIST \\
    \texttt{chulhee.yun@kaist.ac.kr} \\
}

\date{}

% For theorems and such
\usepackage{amsmath}
\usepackage{amssymb}
\usepackage{mathtools}
\usepackage{amsthm}

% if you use cleveref..
\usepackage[capitalize,noabbrev]{cleveref}

%%%%%%%%%%%%%%%%%%%%%%%%%%%%%%%%
% THEOREMS
%%%%%%%%%%%%%%%%%%%%%%%%%%%%%%%%
\theoremstyle{plain}
\newtheorem{theorem}{Theorem}[section]
\newtheorem{proposition}[theorem]{Proposition}
\newtheorem{lemma}[theorem]{Lemma}
\newtheorem{corollary}[theorem]{Corollary}

\theoremstyle{definition}
\newtheorem{definition}[theorem]{Definition}
\newtheorem{assumption}[theorem]{Assumption}

\theoremstyle{remark}
\newtheorem*{remark}{\textbf{Remark}} % No Number\usepackage[capitalize,noabbrev]{cleveref}
%%%%% NEW MATH DEFINITIONS %%%%%

% \usepackage{amsmath,amsfonts,bm}
\usepackage{amsmath,amsfonts}

\usepackage{pifont}


\newcommand{\R}{\mathbb{R}}


\def\va{{\mathbf{a}}}
\def\vg{{\mathbf{g}}}

% Sets
\def\sR{\mathbb{R}}
\def\sC{\mathbb{C}}
\def\sZ{\mathbb{Z}}
\def\sN{\mathbb{N}}
\def\sQ{\mathbb{Q}}

\def\sS{\mathcal{S}}



% Vectors
\def\vzero{{\mathbf{0}}}
\def\vone{{\mathbf{1}}}
\def\vmu{{\mathbf{\mu}}}
\def\vtheta{{\mathbf{\theta}}}
\def\va{{\mathbf{a}}}
\def\vb{{\mathbf{b}}}
\def\vc{{\mathbf{c}}}
\def\vd{{\mathbf{d}}}
\def\ve{{\mathbf{e}}}
\def\vf{{\mathbf{f}}}
\def\vg{{\mathbf{g}}}
\def\vh{{\mathbf{h}}}
\def\vi{{\mathbf{i}}}
\def\vj{{\mathbf{j}}}
\def\vk{{\mathbf{k}}}
\def\vl{{\mathbf{l}}}
\def\vm{{\mathbf{m}}}
\def\vn{{\mathbf{n}}}
\def\vo{{\mathbf{o}}}
\def\vp{{\mathbf{p}}}
\def\vq{{\mathbf{q}}}
\def\vr{{\mathbf{r}}}
\def\vs{{\mathbf{s}}}
\def\vt{{\mathbf{t}}}
\def\vu{{\mathbf{u}}}
\def\vv{{\mathbf{v}}}
\def\vw{{\mathbf{w}}}
\def\vx{{\mathbf{x}}}
\def\vy{{\mathbf{y}}}
\def\vz{{\mathbf{z}}}
\def\vzeta{{\mathbf{\zeta}}}

% Matrix
\def\mA{{\mathbf{A}}}
\def\mB{{\mathbf{B}}}
\def\mC{{\mathbf{C}}}
\def\mD{{\mathbf{D}}}
\def\mE{{\mathbf{E}}}
\def\mF{{\mathbf{F}}}
\def\mG{{\mathbf{G}}}
\def\mH{{\mathbf{H}}}
\def\mI{{\mathbf{I}}}
\def\mJ{{\mathbf{J}}}
\def\mK{{\mathbf{K}}}
\def\mL{{\mathbf{L}}}
\def\mM{{\mathbf{M}}}
\def\mN{{\mathbf{N}}}
\def\mO{{\mathbf{O}}}
\def\mP{{\mathbf{P}}}
\def\mQ{{\mathbf{Q}}}
\def\mR{{\mathbf{R}}}
\def\mS{{\mathbf{S}}}
\def\mT{{\mathbf{T}}}
\def\mU{{\mathbf{U}}}
\def\mV{{\mathbf{V}}}
\def\mW{{\mathbf{W}}}
\def\mX{{\mathbf{X}}}
\def\mY{{\mathbf{Y}}}
\def\mZ{{\mathbf{Z}}}
\def\mBeta{{\mathbf{\beta}}}
\def\mPhi{{\mathbf{\Phi}}}
\def\mLambda{{\mathbf{\Lambda}}}
\def\mSigma{{\mathbf{\Sigma}}}


% Expectation
% \def\eE{\mathop{\mathbb{E}}\limits}
\def\eE{\mathbb{E}}

% Probability
\def\pP{\mathbb{P}}

% Tilde
\def\tf{\tilde{f}}
\def\tS{\tilde{S}}
\def\wtF{\widetilde{\mathcal{F}}}
\def\whR{\widehat{R}}
\def\tvx{\tilde{\mathbf{x}}}
\def\ty{\tilde{y}}


\def\defeq{\overset{\textup{def}}{=}}
% \def\defeq{\overset{.}{=}}
\def\defone{\overset{\text{\ding{172}}}{=}}
\def\deftwo{\overset{\text{\ding{173}}}{=}}
\def\leqone{\overset{\text{\ding{172}}}{\leq}}
\def\leqtwo{\overset{\text{\ding{173}}}{\leq}}
\def\leqthree{\overset{\text{\ding{174}}}{\leq}}
\def\leqfour{\overset{\text{\ding{175}}}{\leq}}
\def\eqone{\overset{\text{\ding{172}}}{=}}
\def\eqtwo{\overset{\text{\ding{173}}}{=}}
\def\eqthree{\overset{\text{\ding{174}}}{=}}
\def\eqfour{\overset{\text{\ding{175}}}{=}}
\def\geqfive{\overset{\text{\ding{176}}}{\geq}}
\usepackage[textsize=tiny]{todonotes}

\begin{document}
\pagenumbering{arabic}

\maketitle

\begin{abstract}
Recent advances in deep learning rely heavily on massive datasets, leading to substantial storage and training costs.
Dataset pruning aims to alleviate this demand by discarding redundant examples.
However, many existing methods require training a model with a full dataset over a large number of epochs before being able to prune the dataset, which ironically makes the pruning process more expensive than just training the model on the entire dataset.
To overcome this limitation, we introduce a \textbf{Difficulty and Uncertainty-Aware Lightweight (DUAL)} score, which aims to identify important samples from the early training stage by considering both example difficulty and prediction uncertainty. To address a catastrophic accuracy drop at an extreme pruning, we further propose a ratio-adaptive sampling using Beta distribution.
Experiments on various datasets and learning scenarios such as image classification with label noise and image corruption, and model architecture generalization demonstrate the superiority of our method over previous state-of-the-art (SOTA) approaches. Specifically, on ImageNet-1k, our method reduces the time cost for pruning to 66\% compared to previous methods while achieving a SOTA, specifically 60\% test accuracy at a 90\% pruning ratio. On CIFAR datasets, the time cost is reduced to just 15\% while maintaining SOTA performance.
\footnote{Our codebase is available at \href{https://github.com/behaapyy/dual-pruning.git}{\texttt{github.com/behaapyy/dual-pruning
}}.}
\end{abstract}


\section{Introduction}

\begin{figure*}
    \centering
    \includegraphics[width=\textwidth]{figures/Introduction.pdf}
    \caption{Showing the novel problem statement applied to traffic prediction use case. Multiple unstructured observations from the past are used to reconstruct a hidden traffic state from which a full traffic state is forecast with a set of query locations. }
    \label{fig:intro}
\end{figure*}

% Was sagen denn die anderen warum Traffic Prediction gut ist? 
Forecasting the traffic in the near future is an important task for city management.
Data from the near past is used to predict future traffic states with spatio-temporal Graph Neural Networks \cite{bui22}.
Accurate prediction provides the opportunity to optimize traffic flow, reduce traffic jams and increase air quality \cite{Po19}.

% Wieso ist Sparsity in allen Dimensionen wichtig.
While traffic prediction relies on the availability of data from traffic sensors, there exists a plethora of reasons why sensors may stop working temporarily, such as simple errors, energy saving, or overloaded communication systems.
Considering small- or medium-sized cities, the coverage of sensors may be low because the sensors are too expensive or not available.
Also, the sensors are typically static and do not adapt to changes in the traffic flow (e.g. caused by a construction site), which motivates moving sensors that for example could be mounted on cars. 
However, both missing and moving sensors introduce sparsity, since measurements may not be available for all locations at all times.
This sparsity must be explicitly addressed in traffic prediction for a realistic application scenario, which is illustrated in figure \ref{fig:intro}.
From one hour of data on Sunday morning, only few observations of the traffic state are available at each timestep.
The number of observations may differ throughout the observed time and the observation itself can be distributed arbitrarily in the city. 
We assume a relatively low number of sensors to account for resource saving and sensor failure in our proposed framework SUSTeR.
The task is to predict the dense traffic state one timestep after the observations at all possible sensor locations.
We study this problem on the traffic dataset Metr-LA and PEMS-BAY to test our assumption that only a fraction of the sensor values would be enough for good predictions.
By modifying an existing traffic dataset, we are able to compare our results from very sparse observations to the bottom line with all information available.
A successful study will provide insights in how sensors in new cities can be reduced before installing them and further mobile sensors would save more resources and are able to adapt to new traffic situations.
We argue that in order to be adaptable to other cities and changes in traffic flows, prior information like the road network should be neglected and just the sparse observations considered.
This comes with the added benefit of making our solution applicable in regions where no openly available road network is maintained or pathways change frequently (e.g. flood areas, animal observations). 


The aforementioned problem is novel and more challenging than the commonly considered traffic prediction problem, since there exist very few observations in each input sample.
Current works for the traffic prediction problem do not consider any missing values. \cite{Li2021, Shao22}
A common method among state of the art approaches is the usage of Graph Neural Networks on graphs that model the sensor network \cite{bui22}.
The values of a sensor are applied to the same graph node for each timestep which prohibits any non-stationary sensors . 
With fixed sensor locations, the resulting sensor network is highly correlated with the road network.
Streets connecting two intersections with sensors should be also an interesting point for correlations in the sensor network.
However, variable observations and high temporal sparsity rates can not be modeled adequately in a static network.
We show in our experiments that the road network has only a small influence on the traffic predictions.

Besides the traffic prediction for future timesteps, some works explore the field of traffic speed imputation \cite{Cini22, Cuza22} where missing sensor values are predicted.
But the amount of missing values is assumed to be at most 80\%, which on average are still over 40 given sensors in each timestep in the Metr-LA dataset with a total of 207 sensors.
We consider up to 99.9\% missing values which are on average 2.4 observations in each timestep that are used as input.
Such high sparsity rates drastically decrease the chance that multiple values are present in one input sample from the same sensor location, which makes it challenging to recognize and learn temporal correlations for each location on its own.

High sparsity rates (>95\%) result in few sensor values, but if a reconstruction of the traffic state would be possible, we question if spatio-temporal graphs require nodes for each sensor.
In SUSTeR we utilize only a small amount of graph nodes for the encoding of information and do not relate such nodes to the sensor network.
We call this the hidden graph (see figure \ref{fig:intro}), which is still able to reconstruct the complete traffic state.
Due to the reduced number of nodes SUSTeR achieves faster runtimes, as shown in the experiments.
This hidden graph is not embedded directly in the spatial domain, which is why the assignment of observations, as well as the querying of the future traffic, is done with an encoder and a decoder, implemented as neural networks.
The decoding from the hidden graph to future values depends on a set of query locations.
Figure \ref{fig:intro} shows the query locations as given from outside and in combination with the reconstructed traffic state the future values are predicted.

To construct the hidden graph we encode observations from each timestep into from multiple graphs, one for each timestep. 
The graphs are created in a residual style and information is added to the node embeddings from the previous timesteps.
We choose this method to incorporate all timesteps equally into the hidden state because the redundant information along the past is non-existing for high sparsity rates.
From the sequence of graphs where our framework inserted the observations step by step we apply STGCN \cite{Yu18}, an algorithm for traffic prediction to find and learn the spatio-temporal correlations on our small number of graph nodes.
The first future timestep of the STGCN is our hidden graph in which the traffic state is reconstructed. 

% Recent work has an implicit embedding of the graph nodes into the spatial domain as the assignment from the sensor to graph node is fixed one by one.
% Because the graph has the same structure as the road network spatio-temporal correlations can be learned between those sensors.
% We reduce the number of nodes and use a non-linear assignment learned data-driven from the observations.

We find in the experiments that SUSTeR outperforms the plain STGCN and modern traffic prediction frameworks like D2STGNN for high sparsity rates $(\geq 99\%)$.
This is equivalent to only $0.2$ to $2.4$ observation for each timestep on average.
SUSTeR uses fewer parameters than the baselines and can train faster and with less training data.
Our main contributions can be summarized as follows:
\begin{itemize}
    \item We introduce a sparse and unstructured variant of the traffic prediction problem with sparsity in all dimensions. The sensors report only a fraction of their values and are arbitrarily distributed in the spatial domain.
    \item We propose SUSTeR, a framework around the STGCN architecture, which maps sparse observations onto a dense hidden graph to reconstruct the complete traffic state.
    Our code is available at github.\footnote{https://github.com/ywoelker/SUSTeR}
    \item We conducts experiments that show that SUSTeR outperforms the baselines in very sparse situations ($\geq 95\%$) and has a competitive performance in low sparsity rates.
    % \item SUSTeR trains a third faster than the next competitor.
\end{itemize}

\section{Related Work}

\subsection{Teleoperation-Based Data Collection on Real Robots}

Collecting data using tele-operation on real robots has been explored by many previous works. Aloha \cite{zhao2023learning} introduced a low-cost teleoperation system that collects real-world demonstrations for imitation learning. A bimanual workspace is set up, where leader robots are used to control the follower robots. Followup work \cite{aldaco2024aloha} improved the performance, ergonomics, and robustness compared to the original design. In addition, a mobile version of Aloha \cite{fu2024mobile} improved data collection outside of lab settings. GELLO \cite{wu2023gello} supports a variety robot arms through a 3D-printed low-cost leader robots with off-the-shelf motors. In order to tele-operate dexterous end effectors prior work has retrieved hand motion data through visual hand tracking \cite{Qin2023AnyTeleopAG} or customized gloves \cite{wang2024dexcap}. In contrast to IRIS, none of these approaches leverages the immersive advantages of XR.

\subsection{XR-Based Data Collection in Real World}

% A major disadvantage of tele-operation using controllers or leader robots is that, for each different kinematics of the target robot, a specialized physical control device has to be built.
% To tackle this issue, some XR-based teleoperation methods have been proposed, exploiting immersive approaches for real-world interactions with robots.
Common approaches that combine XR-based tele-operation with real-world interactions typically visualize a virtual robot to show the user how human actions are mapped to robot actuation \cite{Qin2023AnyTeleopAG}. For instance, recent work developed mobile apps to allow data collection in augmented reality without the need for XR headsets \cite{ar2-d2-pmlr-v229-duan23a, eve}. However, leveraging XR headsets, allows for more intuitive robot manipulation~\cite{arcade,armada,jiang2024comprehensive}.
% , e.g., by aligning a virtual robot arm with the user's arm.
Instead of displaying the virtual robot in a third-person view, \citet{opentelevision, openteach} directly provide the first-person camera feed of the real robot to the user.
% \citet{opentelevision} and \citet{openteach} provide the camera view from real robots directly instead of displaying a virtual robot.
Many other systems~\cite{vicarios,augmentedvisualcues,wang2024robotic,immertwin,sharedctlframework,digitaltwinmr} visualize the real-world scene in the headset and control the robot arm either with controllers~\cite{sharedctlframework} or hand tracking~\cite{wang2024robotic}.
XR-based data collection for dexterous hands has also been explored. For example, \citet{arunachalam2023holo} tracks hand motion using camera and retargets it on the real robot hand. \citet{chen2024arcap} controls robot hand and robot arm at the same time. While these approaches do use XR, the robot data collection and interaction is limited to the real world, as no simulators used in the process.

% \textcolor{red}{More work about MR/VR/AR}

% \subsection{Robot Learning by Interaction}

% \textcolor{red}{what is the combination of demonstration and correction??? Interaction is not a good word}

% \textcolor{red}{a big table of comparison to other AR robot data collection}

\subsection{XR-Based Data Collection in Simulation}

Collecting data using real robots has many limitations, such as limited scene and object diversity. For gathering demonstrations more efficiently and opening access to numerous virtual 3D assets, some works have explored fully virtual data collection.
For instance, DART \cite{dexhub-park} runs a cloud-based simulation, and users can collect demonstrations in any virtualized environment from any location. \citet{mosbach2022accelerating} collects dexterous hand manipulation data with a special glove device in physics simulations.
Although \citet{meng2023virtual} also leverages simulators, their virtual scene is a replica of the real scene, thus the flexibility of simulation is not fully exploited.

This discussion reveals that state-of-the-art, XR-based data collection in simulation has barely been explored. Hence, IRIS presents a significant contribution towards immersive robot data collection and interaction. IRIS not only supports almost all popular simulation environments, but also connects seamlessly with robots in the real world, opening up enormous possibilities in human-involved robot learning in both virtual and real-world settings.
\section{Proposed Methods}
\subsection{Preliminaries}
Let $\gD \coloneq \left\{\left(\bm{x}_1, y_1\right), \cdots, \left(\bm{x}_n, y_n\right) \right\}$ be a labeled dataset of $n$ training samples, where $\bm{x}\in\gX \subset\sR^d$ and $y\in\gY \coloneq\{1, \cdots, C\}$ are the data point and the label, respectively. $C$ is a positive integer and indicates the number of classes. For each labeled data point $(\vx, y) \in \gD$, denote $\sP_k(y \mid \vx)$ as the prediction probability of $y$ given $\vx$, for the model trained with $k$ epochs. Let $\gS\subset\gD$ be the subset retained after pruning. Pruning ratio $r$ is the ratio of the size of $\gD\setminus\gS$ to $\gD$, or $r = 1-\frac{\lvert\gS\rvert}{\lvert\gD\rvert}$.

The Dynamic Uncertainty (Dyn-Unc) score~\citep{he2024large} prefers the most uncertain samples rather than easy-to-learn or hard-to-learn samples during model training. The uncertainty score is defined as the average of prediction variance throughout training. They first define the uncertainty in a sliding window of length $J$:
\begin{align}
\label{eq:U_score_window}
    \mathrm{U}_k(\bm{x}, y) \coloneq & {\sqrt{\frac{\sum_{j=0}^{J-1}\left[ \sP_{k+j}(y \mid \vx) - \Bar{\sP}_k \right]^2}{J-1}}}
\end{align}
where $\Bar{\sP}_k := \frac{\sum_{j=0}^{J-1} \sP_{k+j}  (y \mid \vx)}{J}$ is the average prediction of the model over the window $[k, k+J-1]$. Then taking the average of the uncertainty throughout the whole training process leads to the Dyn-Unc score:
\begin{equation}
\label{eq:Dyn_score}
    \mathrm{U}(\bm{x}, y) = \frac{\sum_{k=1}^{T-J+1} \mathrm{U}_k(\bm{x}, y)}{T-J+1}.
\end{equation}

\subsection{Difficulty \& Uncertainty-Aware Lightweight Score}
\label{sec:DUAL_score_compute}
Following the approach of \citet{swayamdipta2020dataset} and \citet{he2024large}, we analyze data points from ImageNet-1k based on the mean and standard deviation of predictions during training, as shown in \cref{fig:Moon_plot}. We observe data points typically ``flow'' along the ``moon'' from bottom to top direction. Data points starting from the bottom-left region with a low prediction mean and low standard deviation move to the middle region with increased mean and standard deviation, and those starting in the middle region drift toward the upper-left region with a high prediction mean and smaller standard deviation. 
This phenomenon is closely aligned with existing observations that neural networks typically learn easy samples first, then treat harder samples later ~\citep{bengio2009curriculum, arpit2017closer, jiang2020characterizing, shen2022data}. In other words, we see that the uncertainty of easy samples rises first, and then more difficult samples start to move and show an increased uncertainty score.

\cref{fig:Moon_plot} further gives a justification for this intuition. 
In \cref{fig:moon_evolution_60}, samples with the highest Dyn-Unc scores calculated at epoch 60 move upward by the end of training at epoch 90.  
It means that if we measure the Dyn-Unc score at the early stage of training, it gives the highest scores to relatively easy samples rather than the most informative samples. It seems undesirable that it results in poor test accuracy on its coreset as shown in \cref{fig:computation_30_cifar100} of Appendix~\ref{Appendix_Experiments}.

To capture the most useful samples that are likely to contribute significantly to dynamic uncertainty during the whole training process (of 90 epochs) at the earlier training stage ($e.g.$ epoch of 60), we need to target the samples located near the bottom-right region of the moon-shaped distribution, as \cref{fig:moon_evolution_90} illustrates.
Inspired by this observation, we propose a scoring metric that identifies such samples by taking the \textit{uncertainty of the predictions} and the \textit{prediction probability} into consideration.

% \begin{minipage}[t]{0.5\textwidth}
\begin{figure}[t]
    \captionsetup{type=figure}
    \centering
    % First subfigure
    \begin{subfigure}{0.48\linewidth}
        \centering
        \includegraphics[width=\linewidth]{Figures/moon_evolution_6090.png}
        \caption{Dyn-Unc score calculated at epoch $T=60$.}
        \label{fig:moon_evolution_60}
    \end{subfigure}
    % Second subfigure
    \begin{subfigure}{0.48\linewidth}
        \centering
        \includegraphics[width=\linewidth]{Figures/moon_evolution_9060.png}
        \caption{Dyn-Unc score calculated at epoch $T=90$.}
        \label{fig:moon_evolution_90}
    \end{subfigure}
    % Overall figure caption
    \captionof{figure}{The left column visualizes the prediction mean and standard deviation for each data point collected up to epoch 60, while the right column stands for epoch 90. Samples are colored by normalized Dyn-Unc score for each row. The expected cases of the original Dyn-Unc, where the score computation time step is the same as the final epoch of training, are marked with bold outlines.}
    \label{fig:Moon_plot}
\end{figure}

Here, we propose the \textbf{Difficulty and Uncertainty-Aware Lightweight (DUAL) score}, a measure that unites example difficulty and prediction uncertainty. We define the DUAL score of a data point $(\vx, y)$ at $k \in [T-J+1]$ as
\begin{equation}
\label{eq:DUAL_score_window}
    \mathrm{DUAL}_k(\bm{x}, y) \coloneq \\ \underbrace{\left( 1-\Bar{\sP}_k \right)}_{(a)} \underbrace{\sqrt{\frac{\sum_{j=0}^{J-1}\left[ \sP_{k+j}(y \mid \vx) - \Bar{\sP}_k \right]^2}{J-1}}}_{(b)}
\end{equation}
where $\Bar{\sP}_k := \frac{\sum_{j=0}^{J-1} \sP_{k+j}  (y \mid \vx)}{J}$ is the average prediction of the model over the window $[k, k+J-1]$. Note that $\mathrm{DUAL}_k$ is the product of two terms: $(a)$ $1 - \Bar{\sP}_k $ quantifies the example difficulty averaged over the window; $(b)$ is the standard deviation of the prediction probability over the same window, estimating the prediction uncertainty.

Finally, the $\rm DUAL$ score of $(\vx, y)$ is defined as the mean of $\mathrm{DUAL}_k$ scores over all windows:
\begin{equation}
\label{eq:DUAL_score}
    \mathrm{DUAL}(\bm{x}, y) = \frac{\sum_{k=1}^{T-J+1} \mathrm{DUAL}_k(\bm{x}, y)}{T-J+1}.
\end{equation}
The DUAL score reflects training dynamics by leveraging prediction probability across several epochs It provides a reliable estimation to identify the most uncertain examples. 

A theoretical analysis of a toy example further verifies the intuition above. Consider a linearly separable binary classification task $\left\{(\vx_i\in\sR^n, y_i\in\{\pm1\})\right\}_{i=1}^N$, where $N=2$ with $\lVert \vx_1\rVert$ $\ll$ $\langle \vx_1, \vx_2 \rangle < \lVert \vx_2 \rVert$. Without loss of generality, we set $y_1 = y_2 = +1$.
A linear classifier, $f(\vx; \vw) = \vw^\top \vx$, is employed as the model in our analysis. The parameter $\vw$ is initialized at zero and updated by gradient descent. 
\citet{soudry2018implicit} prove that the parameter of linear classifiers diverges to infinity, but directionally converges to the $L_2$ maximum margin separator. This separator is determined by the support vectors closest to the decision boundary. If a valid pruning method encounters this task, then it should retain the point closer to the decision boundary, which is $\vx_1$ in our case, and prune $\vx_2$.

Due to its large norm, $\vx_2$ exhibits higher score values in the early training stage for both uncertainty and DUAL scores.
It takes some time for the model to predict $\vx_1$ with high confidence, which increases its uncertainty level and prediction mean, as well as for scores of $\vx_1$ to become larger than $\vx_2$ as training proceeds.
In \cref{thm:main_shorter_time}, we show through a rigorous analysis that the moment of such a flip in order happens strictly earlier for DUAL than uncertainty.

\begin{theorem}[Informal]
\label{thm:main_shorter_time}
    Define $\sigma(z) \coloneq (1+e^{-z})^{-1}$. Let $V_{t;J}^{(i)}$ be the variance and $\mu_{t;J}^{(i)}$ be the mean of $\sigma(f(\vx_i; \vw_t))$ within a window from time $t$ to $t+J$. Denote $T_v$ and $T_{vm}$ as the first time step when $V_{t;J}^{(1)} > V_{t;J}^{(2)}$ and $V_{t;J}^{(1)}(1-\mu_{t;J}^{(i)}) > V_{t;J}^{(2)}(1-\mu_{t;J}^{(2)})$ occur, respectively. If the learning rate is small enough, then $T_{vm} < T_v$.
\end{theorem}


Technical details about \cref{thm:main_shorter_time} are provided in \Cref{sec:DUAL_theorem}, together with an empirical verification of the time-efficiency of DUAL pruning over Dyn-Unc.

\begin{wrapfigure}[13]{r}{0.53\textwidth}
\vspace{-5mm}
    \begin{center}
        \includegraphics[width=0.95\linewidth]{Figures/moon_evolution_dual_6090.png}
    \end{center}
    \caption{DUAL score also targets uncertain samples during the early epoch. In the end, selected samples are finally located in the most uncertain region.}
    \label{fig:moon_plot_dual}
\end{wrapfigure}

Empirically, as shown in Figure~\ref{fig:moon_plot_dual}, the DUAL score targets data points in the bottom-right region during the early training phase, which eventually evolve to the middle-rightmost part by the end of training. This verifies that DUAL pruning identifies the most uncertain region faster than Dyn-Unc \emph{both in theory and practice}. The distinction arises from the additional consideration of an example difficulty in our method. We believe that this adjustment leads to improved generalization performance compared to Dyn-Unc, as verified through various experiments in later sections.

However, score-based methods, including our method, 
suffer from limitations due to biased representations, leading to poor coreset test accuracy at high pruning ratios. 
ratios. To address this, we propose an additional strategy to adaptively select samples according to a ratio for pruning.

\subsection{Pruning Ratio-Adaptive Sampling}
\label{sec:betapruning}
Since the distribution of difficulty scores is dense in high-score samples, selecting only the highest-score samples may result in a biased model \citep{zhou2023probabilisticbilevelcoresetselection, maharana2023d2, choi2024bws}.
To address this, we design a sampling method to determine the subset $\gS\subset\gD$, rather than simply pruning the samples with the lowest score. We introduce a Beta distribution that varies with the pruning ratio. The primary objective of this method is to ensure that the selected subsets gradually include more easy samples into the coreset as the pruning ratio increases.

However, the concepts of ``easy'' and ``hard'' cannot be distinguished solely based on uncertainty, or DUAL score. To address this, we use the \emph{prediction mean} again for sampling. 
We utilize the Beta probability density function (PDF) to define the selection probability of each sample.
First, we assign each data point a corresponding PDF value based on its prediction mean and weight this probability using the DUAL score. The weighted probability with the DUAL score is then normalized so that the sum equals 1, and then used as the sampling probability. To be clear, sampling probability is for selecting samples, \emph{not for pruning}. Therefore, for each pruning ratio 
$r$, we randomly select $(1-r)\cdot n$ samples without replacement, where sampling probabilities are given according to the prediction mean and DUAL score as described. The detailed algorithm for our proposed pruning method with Beta sampling is provided in Algorithm~\ref{alg:DUAL}, Appendix~\ref{Appendix_explanation_of_dual_pruning}.

We design the Beta PDF to assign a sampling probability concerning a prediction mean as follows:
\begin{align}
\label{eq:alpha_beta}
\begin{split}
    \beta_r &= C \cdot (1-\mu_\gD)  \left(1-r^{c_\gD}\right)\\
    \alpha_r &= C-\beta_r, \\
\end{split}
\end{align}
where $C > 0$ is a fixed constant, and the $\mu_D$ stands for the prediction mean of the highest score sample.
Recalling that the mean of Beta distribution is $\frac{\alpha_r}{\alpha_r + \beta_r}$, the above choice makes the mean of Beta distribution move progressively with $r$, starting from $\mu_\gD$ ($r \simeq 0$, small pruning ratio) to one. In other words, with growing $r$, this Beta distribution becomes skewed towards the easier region ($r \rightarrow 1$, large pruning ratio), which in turn gives more weight to easy samples. 
The tendency of evolving should be different with datasets, thus a hyperparameter $c_\gD \geq 1$ is used to control the rate of evolution of the Beta distribution. Specifically, the choice of \( c_\gD \) depends on the complexity of the initial dataset. For smaller and more complex datasets, setting \( c_\gD \) to a smaller value retains more easy samples. For larger and simpler datasets, setting \( c_\gD \) to a larger value allows more uncertain samples to be selected.
(For your intuitive understanding, please refer to  \cref{fig:beta_pdf} and \cref{fig:cifar_coreset_visualization_beta} in the \cref{Appendix_explanation_of_dual_pruning}.) This is also aligned with the previous findings from \citet{sorscher2022beyond}; if the initial dataset is small, the coreset is more effective when it contains easier samples, while for a relatively large initial dataset, including harder samples can improve generalization performance.


\begin{remark}
BOSS~\citep{acharyabalancing} also uses the Beta distribution to sample easier data points during pruning, similar to our approach. However, a key distinction lies in how we define the Beta distribution's parameters, \(\alpha_r\) and \(\beta_r\). While BOSS adjusts these parameters to make the mode of the Beta distribution's PDF scale linearly with the pruning ratio $r$, we employ a non-linear combination. This non-linear approach has the crucial advantage of maintaining an almost stationary PDF at low pruning ratios. This stability is especially beneficial when the dataset becomes easier where there is no need to focus on easy examples. Furthermore, unlike previous methods, we define PDF values based on the prediction mean, rather than any difficulty score, which is another significant difference.
\end{remark}

\section{Experiments}
\label{sec:experiment}
\begin{table}[t]
\sisetup{table-format = 2.2, round-mode = places, round-precision=2, round-pad = false}
\caption{Comparison of test accuracy between the DUAL score method and existing techniques using ResNet-18 on CIFAR-10 and CIFAR-100 datasets. Training the model on the full dataset achieves an average test accuracy of 95.30\% on CIFAR-10 and 78.91\% on CIFAR-100. The best result in each pruning ratio is highlighted in bold.}
\setlength{\tabcolsep}{3.1pt}
\centering
\resizebox{\linewidth}{!}{
\begin{tabular}{l c@{}cc@{}cc@{}cc@{}cc@{}c | c@{}cc@{}cc@{}cc@{}cc@{}c}
    \toprule
    \textbf{Dataset ($\rightarrow$)} & \multicolumn{10}{c}{\textbf{CIFAR10}} & \multicolumn{10}{c}{\textbf{CIFAR100}}\\
    \cmidrule(lr){2-21}
    
    \textbf{Pruning Rate ($\rightarrow$)} & \multicolumn{2}{c}{\textbf{30\%}} & \multicolumn{2}{c}{\textbf{50\%}} & \multicolumn{2}{c}{\textbf{70\%}} & \multicolumn{2}{c}{\textbf{80\%}} & \multicolumn{2}{c}{\textbf{90\%}} & \multicolumn{2}{c}{\textbf{30\%}} & \multicolumn{2}{c}{\textbf{50\%}} & \multicolumn{2}{c}{\textbf{70\%}} & \multicolumn{2}{c}{\textbf{80\%}} & \multicolumn{2}{c}{\textbf{90\%}} \\
    \midrule
    
    \textbf{Random} & 94.39 & {\scriptsize \num{ +-0.2275}} & 93.20 & {\scriptsize \num{ +-0.1188}} & 90.47 & {\scriptsize \num{ +-0.1678}} & 88.28 & {\scriptsize \num{ +-0.1731}} & 83.74 & {\scriptsize \num{ +-0.2051}} & 75.15 & {\scriptsize \num{ +-0.2825}} & 71.68 & {\scriptsize \num{ +-0.3065}} & 64.86 & {\scriptsize \num{ +-0.3939}} & 59.23 & {\scriptsize \num{ +-0.6189}} & 45.09 & {\scriptsize \num{ +-1.2610}} \\
    
    \textbf{Entropy} & 93.48  & {\scriptsize \num{ +-0.0566 }} & 92.47 & {\scriptsize \num{ +-0.1707}} & 89.54 & {\scriptsize \num{ +-0.1753}} & 88.53 & {\scriptsize \num{ +-0.1864}} & 82.57 & {\scriptsize \num{ +-0.3645}} & 75.20 & {\scriptsize \num{ +-0.2486 }} & 70.90 & {\scriptsize \num{ +-0.3464}} & 61.70 & {\scriptsize \num{ +-0.4669}} & 56.24 & {\scriptsize \num{ +-0.5082}} & 42.25 & {\scriptsize \num{ +-0.3915}} \\
    
    \textbf{Forgetting} & 95.48  & {\scriptsize \num{ +-0.1412 }} & 94.94 & {\scriptsize \num{ +-0.2116}} & 89.55 & {\scriptsize \num{ +-0.6456}} & 75.47 & {\scriptsize \num{ +-1.2726}} & 46.64 & {\scriptsize \num{ +-1.9039}} & 77.52 & {\scriptsize \num{ +-0.2585 }} & 70.93 & {\scriptsize \num{ +-0.3700}} & 49.66 & {\scriptsize \num{ +-0.1962}} & 39.09 & {\scriptsize \num{ +-0.4065}} & 26.87 & {\scriptsize \num{ +-0.7318}} \\
    
    \textbf{EL2N} & 95.44  & {\scriptsize \num{ +-0.0628 }} & 95.19 & {\scriptsize \num{ +-0.1134}} & 91.62 & {\scriptsize \num{ +-0.1397}} & 74.70 & {\scriptsize \num{ +-0.4523}} & 38.74 & {\scriptsize \num{ +-0.7506}} & 77.13 & {\scriptsize \num{ +-0.2348}} & 68.98 & {\scriptsize \num{ +-0.3539}} & 34.59 & {\scriptsize \num{ +-0.4824}} & 19.52 & {\scriptsize \num{ +-0.7925}} & 8.89 & {\scriptsize \num{ +-0.2774}} \\
    
    \textbf{AUM} & 90.62  & {\scriptsize \num{ +-0.0921 }} & 87.26 & {\scriptsize \num{ +-0.1128}} & 81.28 & {\scriptsize \num{ +-0.2564}} & 76.58 & {\scriptsize \num{ +-0.3458}} & 67.88 & {\scriptsize \num{ +-0.5275}} & 74.34 & {\scriptsize \num{ +-0.1424 }} & 69.57 & {\scriptsize \num{ +-0.2100}} & 61.12 & {\scriptsize \num{ +-0.2004}} & 55.80 & {\scriptsize \num{ +-0.3256}} & 45.00 & {\scriptsize \num{ +-0.3694}} \\
    
    \textbf{Moderate} & 94.26  & {\scriptsize \num{ +-0.0904 }} & 92.79 & {\scriptsize \num{ +-0.0856}} & 90.45 & {\scriptsize \num{ +-0.2110}} & 88.90 & {\scriptsize \num{ +-0.1684}} & 85.52 & {\scriptsize \num{ +-0.2906}} & 75.20  & {\scriptsize \num{ +-0.2486 }} & 70.90 & {\scriptsize \num{ +-0.3464}} & 61.70 & {\scriptsize \num{ +-0.4699}} & 56.24 & {\scriptsize \num{ +-0.5082}} & 42.25 & {\scriptsize \num{ +-0.3915}} \\
    
    \textbf{Dyn-Unc} & 95.49  & {\scriptsize \num{ +-0.2061 }} & \textbf{95.35} & {\scriptsize \num{ +-0.1205}} & 91.78 & {\scriptsize \num{ +-0.6516}} & 83.32 & {\scriptsize \num{ +-0.9391}} & 59.67 & {\scriptsize \num{ +-1.7929}} & 77.67  & {\scriptsize \num{ +-0.1381}} & 74.23 & {\scriptsize \num{ +-0.2214}} & 64.30 & {\scriptsize \num{ +-0.1333}} & 55.01 & {\scriptsize \num{ +-0.5465}} & 34.57 & {\scriptsize \num{ +-0.6920}} \\
    
    \textbf{TDDS} & 94.42  & {\scriptsize \num{ +-0.1252 }} & 93.11 & {\scriptsize \num{ +-0.1377}} & 91.02 & {\scriptsize \num{ +-0.1908}} & 88.25 & {\scriptsize \num{ +-0.2385}} & 82.49 & {\scriptsize \num{ +-0.2799}} & 75.02  & {\scriptsize \num{ +-0.3682}} & 71.80 & {\scriptsize \num{ +-0.3323}} & 64.61 & {\scriptsize \num{ +-0.2431}} & 59.88 & {\scriptsize \num{ +-0.2110}} & 47.93 & {\scriptsize \num{ +-0.2147}} \\
    
    \textbf{CCS} & 95.31  & {\scriptsize \num{ +-0.2238}} & 95.06 & {\scriptsize \num{ +-0.1547}} & 92.68 & {\scriptsize \num{ +-0.1704}} & 91.25 & {\scriptsize \num{ +-0.2073}} & 85.92 & {\scriptsize \num{ +-0.3901}} & 77.15 & {\scriptsize \num{ +-0.2816}} & 73.83 & {\scriptsize \num{ +-0.2073}} & 68.65 & {\scriptsize \num{ +-0.3130}} & 64.06 & {\scriptsize \num{ +-0.2084}} & 54.23 & {\scriptsize \num{ +-0.4813}} \\
    
    \textbf{D2} & 94.13  & {\scriptsize \num{ +-0.2033}} & 93.26 & {\scriptsize \num{ +-0.1623}} & 92.34 & {\scriptsize \num{ +-0.1786}} & 90.38 & {\scriptsize \num{ +-0.3376}} & 86.11 & {\scriptsize \num{ +-0.2072}} & 76.47  & {\scriptsize \num{ +-0.2934}} & 73.88 & {\scriptsize \num{ +-0.2780}} & 62.99 & {\scriptsize \num{ +-0.2775}} & 61.48 & {\scriptsize \num{ +-0.3361}} & 50.14 & {\scriptsize \num{ +-0.8951}} \\
    
    \midrule
    
    \textbf{DUAL} & 95.25  & {\scriptsize \num{ +-0.17 }} & 94.95 & {\scriptsize \num{ +-0.22 }} & 91.75 & {\scriptsize \num{ +-0.98 }} & 82.02 & {\scriptsize \num{ +-1.85 }} & 54.95 & {\scriptsize \num{ +-0.42 }} & 77.43 & {\scriptsize \num{ +-0.18 }} & 74.62 & {\scriptsize \num{ +-0.47 }} & 66.41 & {\scriptsize \num{ +-0.52 }} & 56.57 & {\scriptsize \num{ +-0.57 }} & 34.38 & {\scriptsize \num{ +-1.39 }} \\
    
    \textbf{DUAL+$\beta$ sampling} & \textbf{95.51}  & {\scriptsize \num{ +-0.0634}} & 95.23 & {\scriptsize \num{ +-0.0796}} & \textbf{93.04} & {\scriptsize \num{ +-0.4282}} & \textbf{91.42} & {\scriptsize \num{ +-0.352}} & \textbf{87.09} & {\scriptsize \num{ +-0.3599}} & \textbf{77.86}  & {\scriptsize \num{ +-0.1186}} & \textbf{74.66} & {\scriptsize \num{ +-0.1173}} & \textbf{69.25} & {\scriptsize \num{ +-0.2156}} & \textbf{64.76} & {\scriptsize \num{ +-0.2272}} & \textbf{54.54} & {\scriptsize \num{ +-0.0884}} \\
    
    \bottomrule
\end{tabular}
}
\label{tbl:main_cifar}
\end{table}

\subsection{Experimental Settings}
We assessed the performance of our proposed method in three key scenarios: image classification, image classification with noisy labels and corrupted images. In addition, we validate cross-architecture generalization on three-layer CNN, VGG-16~\citep{simonyan2015deepconvolutionalnetworkslargescale}, ResNet-18 and ResNet-50~\citep{he2015deepresiduallearningimage}.

\textbf{Hyperparameters.}
For training CIFAR-10 and CIFAR-100, we train ResNet-18 for 200 epochs with a batch size of 128. SGD optimizer with momentum of 0.9 and weight decay of 0.0005 is used. The learning rate is initialized as 0.1 and decays with the cosine annealing scheduler. As \citet{zhang2024spanning} show that smaller batch size boosts performance at high pruning rates, we also halved the batch size for 80\% pruning, and for 90\% we reduced it to one-fourth. For ImageNet-1k, ResNet-34 is trained for 90 epochs with a batch size of 256 across all pruning ratios. An SGD optimizer with a momentum of 0.9, a weight decay of 0.0001, and an initial learning rate of 0.1 is used, combined with a cosine annealing scheduler.

\textbf{Baselines.} The baselines considered in this study are listed as follows\footnote{Infomax~\citep{tan2025data} was excluded as it employs different base hyperparameters in the original paper compared to other baselines and does not provide publicly available code. See Appendix~\ref{Appendix_Technical_Details_of_Baselines} for more discussion.}: (1) Random; (2) Entropy~\citep{coleman2020selectionproxyefficientdata}; (3) Forgetting~\citep{toneva2018empirical}; (4) EL2N~\citep{gordon2021data}; (5) AUM~\citep{pleiss2020identifyingmislabeleddatausing}; (6) Moderate~\citep{xia2022moderate}; (7) Dyn-Unc~\citep{he2024large}; (8) TDDS~\citep{zhang2024spanning}; (9) CCS\mbox{~\citep{zheng2022coverage}}; and (10) $\mathbb{D}^2$~\citep{maharana2023d2}. To ensure a fair comparison, all methods were trained with the same base hyperparameters for training, and the best hyperparameters reported in their respective original works for scoring. Technical details are provided in the \cref{Appendix_Technical_Details_of_Baselines}. 


\begin{wrapfigure}[16]{R}{0.5\textwidth}
\vspace{-15pt}
    \begin{center}
        \includegraphics[width=0.9\linewidth]{Figures/total_computation_time.pdf}
    \end{center}
    \vspace{-8pt}
    \caption{Comparison in total time spent on CIFAR datasets.}
    \label{fig:total_time_consumed}
    % \vspace{-10pt}
\end{wrapfigure}
\vspace{3pt}
\subsection{Image Classification Benchmarks}
\cref{tbl:main_cifar} presents the test accuracy for image classification results on CIFAR-10 and CIFAR-100. Our pruning method consistently outperforms other baselines, particularly when combined with Beta sampling. While the DUAL score exhibits competitive performance in lower pruning ratios, its accuracy degrades with more aggressive pruning. Our Beta sampling effectively mitigates this performance drop.

Notably, the DUAL score only requires training a single model for \emph{only 30 epochs}, significantly reducing the computational cost. In contrast, the second-best methods, Dyn-Unc and CCS, rely on scores computed over a full 200-epoch training cycle, making them considerably less efficient. Even considering subset selection, score computation, and subset training, the total time remains less than a single full training run, as shown in Figure~\ref{fig:total_time_consumed}. Specifically, on CIFAR-10, our method achieves lossless pruning up to a 50\% pruning ratio while saving 35.5\% of total training time. 

\begin{wrapfigure}[20]{L}{0.5\textwidth}
\vspace{2mm}
    \centering
    \captionof{table}{\label{tab:imagenet_results} Comparison of test accuracy on ImageNet-1k. The model trained with the full dataset achieves 73.1\% test accuracy. The best result in each pruning ratio is highlighted in bold.}
    \small
    \setlength{\tabcolsep}{3pt}
    \begin{tabular}{lccccc}
        \toprule
        \textbf{Pruning Rate} & \textbf{30\%} & \textbf{50\%} & \textbf{70\%} & \textbf{80\%} & \textbf{90\%} \\
        \midrule
        \textbf{Random} & 72.2 & 70.3 & 66.7 & 62.5 & 52.3 \\
        \textbf{Entropy} & 72.3 & 70.8 & 64.0 & 55.8 & 39.0 \\
        \textbf{Forgetting} & 72.6 & 70.9 & 66.5 & 62.9 & 52.3 \\
        \textbf{EL2N} & 72.2 & 67.2 & 48.8 & 31.2 & 12.9 \\
        \textbf{AUM} & 72.5 & 66.6 & 40.4 & 21.1 & 9.9 \\
        \textbf{Moderate} & 72.0 & 70.3 & 65.9 & 61.3 & 52.1 \\
        \textbf{Dyn-Unc} & 70.9 & 68.3 & 63.5 & 59.1 & 49.0 \\
        \textbf{TDDS} & 70.5 & 66.8 & 59.4 & 54.4 & 46.0 \\
        \textbf{CCS} & 72.3 & 70.5 & 67.8 & 64.5 & 57.3 \\
        \textbf{D2} & 72.9 & 71.8 & 68.1 & 65.9 & 55.6 \\
        \midrule
        \textbf{DUAL} & 72.8 & 71.5 & 68.6 & 64.7 & 53.1 \\
        \textbf{DUAL+$\beta$ sampling} & \textbf{73.3} & \textbf{72.3} & \textbf{69.4} & \textbf{66.5} & \textbf{60.0} \\
        \bottomrule
    \end{tabular}
\end{wrapfigure}
\vspace{2pt}
We also evaluate our pruning method on the large-scale dataset, ImageNet-1k. The DUAL score is computed during training, specifically at epoch 60, which is 33\% earlier than the original train epoch used to compute scores for other baseline methods. As shown in \cref{tab:imagenet_results}, Dyn-Unc performs worse than random pruning across all pruning ratios, and we attribute this undesirable performance to its limited total training epochs (only 90), which is insufficient for Dyn-Unc to fully capture the training dynamics of each sample. In contrast, our DUAL score, combined with Beta sampling, outperforms all competitors while requiring the least computational cost. The DUAL score's ability to consider both training dynamics and the difficulty of examples enables it to effectively identify uncertain samples early in the training process, even when training dynamics are limited. Remarkably, for 90\% pruned Imagenet-1K, it maintains a test accuracy of 60.0\%, surpassing the previous state-of-the-art (SOTA) by a large margin.
\vspace{15pt}
\subsection{Experiments under More Realistic Scenarios}
\subsubsection{Label Noise and Image Corruption}
Data affected by label noise or image corruption are difficult and unnecessary samples that hinder model learning and degrade generalization performance. Therefore, filtering out these samples through data pruning is crucial. Most data pruning methods, however, either focus solely on selecting difficult samples based on example difficulty~\citep{gordon2021data, pleiss2020identifyingmislabeleddatausing, coleman2020selectionproxyefficientdata} or prioritize dataset diversity~\citep{zheng2022coverage, xia2022moderate}, making them unsuitable for effectively pruning such noisy and corrupted samples.

In contrast, methods that select uncertain samples while considering training dynamics, such as Forgetting \citep{toneva2018empirical} and Dyn-Unc~\citep{he2024large}, demonstrate robustness by pruning both the hardest and easiest samples, ultimately improving generalization performance, as illustrated in Figure~\ref{fig:label_noise_20_ratio}. However, since noisy samples tend to be memorized after useful samples are learned~\citep{arpit2017closer, jiang2020characterizing}, there is a possibility that those noisy samples may still be treated as uncertain in the later stages of training and thus be included in the selected subset.

The DUAL score aims to identify high-uncertainty samples early in training by considering both training dynamics and example difficulty. Noisy data, typically under-learned compared to other challenging samples during this phase, exhibit lower uncertainty (Figure~\ref{fig:label_noise_visualization}, Appendix~\ref{Appendix_labelnoise_experiments}). Consequently, our method effectively prunes these noisy samples.

To verify this, we evaluate our method by introducing a specific proportion of symmetric label noise~\citep{patrini2017making, xia2020robust, li2022selective} and applying five different types of image corruptions~\citep{wang2018iterative, hendrycks2019benchmarking, xia2021instance}. We use CIFAR-100 with ResNet-18 and Tiny-ImageNet with ResNet-34 for these experiments. On CIFAR-100, we test label noise and image corruption ratios of 20\%, 30\%, and 40\%. For Tiny-ImageNet, we use a 20\% ratio of label noise and image corruption. We prune the label noise-added dataset using a model trained for 50 epochs and the image-corrupted dataset with a model trained for 30 epochs using DUAL pruning—both significantly lower than the 200 epochs used by other methods. For detailed experimental settings, please refer to Appendix~\ref{Appendix_Technical_Details_of_Ours}.
\begin{figure*}[t]
    \centering
    \begin{subfigure}{0.325\textwidth}
        \centering
        \includegraphics[width=\textwidth]{Figures/label_noise20_ratio.pdf}
        \caption{\label{fig:label_noise_20_ratio}Pruned mislabeled data ratio}
    \end{subfigure}
    \hfill
    \begin{subfigure}{0.325\textwidth}
        \centering
        \includegraphics[width=\textwidth]{Figures/label_noise20_testacc.pdf}
        \caption{\label{fig:label_noise_20_testacc}Test accuracy under label noise}
    \end{subfigure}
    \hfill
    \begin{subfigure}{0.325\textwidth}
        \centering
        \includegraphics[width=\textwidth]{Figures/imagecorruption_20_testacc.pdf}
    \caption{\label{fig:imagecorruption_20_testacc}Test accuracy under image noise}
    \end{subfigure}
    \caption{\label{fig:labelnoise_main}The left figure shows the ratio of pruned mislabeled data under 20\% label noise on CIFAR-100 trained with ResNet-18. When label noise is 20\%, the optimal value (black dashed line) corresponds to pruning 100\% of mislabeled data at a 20\% pruning ratio. The middle and right figures depict test accuracy under 20\% label noise and 20\% image corruption, respectively. Our method effectively prunes mislabeled data near the optimal value while maintaining strong generalization performance. Results are averaged over five random seeds.}
    \vspace{-10pt}
\end{figure*}
As shown in Figure~\ref{fig:labelnoise_main}, the left plot demonstrates that DUAL pruning effectively removes mislabeled data at a ratio close to the optimal. Notably, when the pruning ratio is 10\%, nearly \emph{all pruned samples are mislabeled data}.
Consequently, as observed in Figure~\ref{fig:label_noise_20_testacc}, DUAL pruning leads to improved test accuracy compared to training on the full dataset, even up to a pruning ratio of 70\%. At lower pruning ratios, performance improves as mislabeled data are effectively removed, highlighting the advantage of our approach in handling label noise.
Similarly, for image corruption, our method prunes more corrupted data across all corruption rates compared to other methods, as shown in Figure~\ref{fig:imagecorruption_203040_ratio},~\ref{fig:imagecorruption_all} in Appendix~\ref{Appendix_imagecorruption_experiments}. As a result, this leads to higher test accuracy, as demonstrated in Figure~\ref{fig:imagecorruption_20_testacc}. 

Detailed results, including exact numerical values for different corruption rates and Tiny-ImageNet experiments, can be found in Appendix~\ref{Appendix_labelnoise_experiments} and \ref{Appendix_imagecorruption_experiments}.  Across all experiments, DUAL pruning consistently shows \emph{strong noise robustness} and outperforms other methods by a substantial margin.

\subsubsection{Cross-Architecture Generalization}
We also evaluate the ability to transfer scores across various model architectures. To be specific, if we can get high-quality example scores for pruning by using a simpler architecture than one for the training, our DUAL pruning would become even more efficient in time and computational cost. Therefore, we focus on the cross-architecture generalization from relatively small networks to larger ones with three-layer CNN, VGG-16, ResNet-18, and ResNet-50. Competitors are selected from each categorized group of the pruning approach: EL2N from difficulty-based, Dyn-Unc from uncertainty-based, and CCS from the geometry-based group. 

For instance, we get training dynamics from the ResNet-18 and then calculate the example scores. Then, we prune samples using scores calculated from ResNet-18, and train selected subsets on ResNet-50. The result with ResNet-18 and ResNet-50 is described in Table~\ref{tab:04-cross-arch-r18-r50}. Surprisingly, the coreset shows competitive performance to the baseline, where the baseline refers to the test accuracy after training a coreset constructed based on the score calculated from ResNet-50.
For all pruning cases, we observe that our methods reveal the highest performances. Specifically, when we prune 70\% and 90\% of the original dataset, we find that all other methods fail, showing worse test accuracies than random pruning. 


\begin{table}[t]
\caption{Cross-architecture generalization performance on CIFAR-100 from ResNet-18 to ResNet-50. We report an average of five runs. `R50 $\rightarrow$ R50' stands for score computation on ResNet-50, as a baseline.}
\label{tab:04-cross-arch-r18-r50}
\setlength{\tabcolsep}{3.1pt}
\centering
\begin{tabular}{lcccc}
    \toprule
    \multicolumn{1}{c}{} & \multicolumn{4}{c}{ResNet-18 $\rightarrow$ ResNet-50} \\
    \midrule
    Pruning Rate ($\rightarrow$) & 30\% & 50\% & 70\% & 90\% \\
    \hline
    \midrule
    Random & 74.47 \scriptsize{$\pm 0.67$} & 70.09 \scriptsize{$\pm 0.42$} & 60.06 \scriptsize{$\pm 0.99$} & 41.91 \scriptsize{$\pm 4.32 $} \\
    EL2N  & 76.42 \scriptsize{$\pm 1.00$} & 69.14 \scriptsize{$\pm 1.00$} & 45.16 \scriptsize{$\pm 3.21$} & 19.63 \scriptsize{$\pm 1.15 $} \\
    Dyn-Unc  & 77.31 \scriptsize{$\pm 0.34$} & 72.12 \scriptsize{$\pm 0.68$} & 59.38 \scriptsize{$\pm 2.35$} & 31.74 \scriptsize{$\pm 2.31 $} \\
    CCS  & 74.78 \scriptsize{$\pm 0.66$} & 69.98 \scriptsize{$\pm 1.18$} & 59.75 \scriptsize{$\pm 1.41$} & 41.54 \scriptsize{$\pm 3.94 $} \\
    \midrule
    DUAL   & \textbf{78.03} \scriptsize{$\pm 0.83$} & 72.82 \scriptsize{$\pm 1.46$} & 63.08 \scriptsize{$\pm 2.45$} & 33.65 \scriptsize{$\pm 2.92 $} \\
    DUAL +$\beta$  & 77.82 \scriptsize{$\pm 0.65$} & \textbf{73.98} \scriptsize{$\pm 0.62$} & \textbf{66.36} \scriptsize{$\pm 1.66$} & \textbf{49.90} \scriptsize{$\pm 2.56 $} \\
    \hline
    \midrule
    DUAL (R50$\rightarrow$R50)  & 77.82 \scriptsize{$\pm 0.64$} & 73.66 \scriptsize{$\pm 0.85$} & 52.12 \scriptsize{$\pm 2.73 $} & 26.13 \scriptsize{$\pm 1.96 $} \\
    DUAL (R50$\rightarrow$R50)+$\beta$  & 77.57 \scriptsize{$\pm 0.23$} & 73.44 \scriptsize{$\pm 0.87$} & 65.17 \scriptsize{$\pm 0.96$} & 47.63 \scriptsize{$\pm 2.47 $} \\
    \bottomrule
    \vspace{-10pt}
\end{tabular}
\end{table}

We also test the cross-architecture generalization performance with three-layer CNN, VGG-16, and ResNet-18 in Appendix~\ref{Appendix_cross_architecture}. 
Even for a simple model like three-layer CNN, we see our methods show consistent performance, as can be seen in Table~\ref{tab:cnn-to-resnet18} in Appendix~\ref{Appendix_cross_architecture}. This observation gives rise to an opportunity to develop some small proxy networks to get example difficulty with less computational cost. 
Transfer across models with similar capacities, $e.g.$ from VGG-16 to ResNet-18 and vice versa, also supports the verification of cross-architecture compatibility.   

\subsection{Ablation Studies}
\paragraph{Hyperparameter Analysis.}
\begin{wrapfigure}[10]{r}{0.5\textwidth}
\vspace{-5mm}
    \begin{center}
        \includegraphics[width=0.92\linewidth]{Figures/TestAcc_TC_varying.pdf}
    \end{center}
    \vspace{-5mm}
    \caption{\textbf{Left}: Varying T with $J=10$ and $c_\gD=4$. \textbf{Right}: Varying $c_\gD$ with $T=30$ and $J=10$.}
    \label{fig:TC_varying}
\end{wrapfigure}
Here, we investigate the robustness of our hyperparameters, $T$, $J$, and $c_\gD$. We fix $J$ across all experiments, as it has minimal impact on selection, indicating its robustness (Fig~\ref{fig:J_varying}, Appendix~\ref{Appendix_Experiments}). In Figure~\ref{fig:TC_varying}, we assess the robustness of $T$ by varying it from 20 to 200 on CIFAR-100. We find that while $T$ remains highly robust in earlier epochs, increasing $T$ degrades generalization performance. This is expected, as larger $T$ overemphasizes difficult samples due to our difficulty-aware selection.  Thus, pruning in earlier epochs (from 30 to 50) proves to be more effective and robust. For the $c_\gD$, we vary it from 3 to 7 and find robustness, especially in the aggressive pruning regime. All results are averaged across three runs.

\paragraph{Beta Sampling Analysis.}
Next, we study the impact of our proposed pruning-ratio-adaptive Beta sampling on existing score metrics. We apply our Beta sampling strategy to prior score-based methods, including Forgetting, EL2N, and Dyn-Unc, using the CIFAR10 and CIFAR100 datasets. By comparing our sampling approach with vanilla threshold pruning, which selects only the highest-scoring samples, we observe that prior score-based methods become remarkably comparable to random pruning after Beta sampling is adjusted (see \cref{tab:abl_beta_cifar10_100_90_main}).

\begin{table*}[t]
\caption{Comparison on CIFAR-10 and CIFAR-100 for $90\%$ pruning rate. 
We report average accuracy with five runs. The best performance is in bold in each column.}
\label{tab:abl_beta_cifar10_100_90_main}
\setlength{\tabcolsep}{4.5pt}
\centering
\begin{tabular}{lcc|cc}
    \toprule
    \multicolumn{1}{c}{} & \multicolumn{2}{c}{CIFAR-10} & \multicolumn{2}{c}{CIFAR-100} \\
    \midrule
    Method & Thresholding & $\beta$-Sampling & Thresholding & $\beta$-Sampling \\
    \midrule
    Random &  \textbf{83.74}\scriptsize{$\pm$0.21} & 83.31 (-0.43) \scriptsize{$\pm$0.14} & \textbf{45.09} \scriptsize{$\pm$1.26} & 51.76 (+6.67) \scriptsize{$\pm$0.25} \\
    EL2N &  38.74 \scriptsize{$\pm$0.75} & 87.00 (+48.26) \scriptsize{$\pm$0.45} & 8.89 \scriptsize{$\pm$0.28} & 53.97 (+45.08)  \scriptsize{$\pm$0.63}  \\
    Forgetting &  46.64 \scriptsize{$\pm$1.90} & 85.67 (+39.03) \scriptsize{$\pm$0.13} & 26.87 \scriptsize{$\pm$0.73} & 52.40 (+25.53) \scriptsize{$\pm$0.43} \\
    Dyn-Unc &  59.67 \scriptsize{$\pm$1.79} & 85.33 (+32.14) \scriptsize{$\pm$0.20} & 34.57 \scriptsize{$\pm$0.69} & 51.85 (+17.28) \scriptsize{$\pm$0.35}   \\
    \midrule
    Ours & 54.95 \scriptsize{$\pm$0.42} & \textbf{87.09} (+31.51) \scriptsize{$\pm$0.36} & 34.28 \scriptsize{$\pm$1.39} & \textbf{54.54} (+20.26) \scriptsize{$\pm$0.09}  \\
    \bottomrule
\end{tabular}
\end{table*}


Even adapted for random pruning, our Beta sampling proves to perform well. Notably, EL2N, which performs poorly on its own, becomes significantly more effective when combined with our sampling method. Similar improvements are also seen with Forgetting and Dyn-Unc scores. This is because our proposed Beta sampling enhances the diversity of selected samples in turn, especially when used with example difficulty-based methods. More results conducted for 80\% pruning cases are included in the \cref{Appendix_beta_samapling}.
    

\paragraph{Additional Analysis.}
In addition to the main results presented in this paper, we also conducted various experiments to further explore the effectiveness of our method. These additional results include an analysis of coreset performance under a time budget ($e.g.$ other score metrics are also computed by using training dynamics up to epoch 30) and Spearman rank correlation was calculated between individual scores and the averaged score across five runs to assess the consistency of scores for each sample. Furthermore, additional results in extreme cases, 30\% and 40\% of label noise and image corruption can be found in \cref{Appendix_Experiments}.
\section{Conclusion}
 In this brief, we presented a 16nm reliable, time-predictable heterogeneous RISC-V SoC for \gls{ai}-enhanced mixed-criticality applications. To the best of our knowledge, this is the first SoC that combines safety features for \glspl{mcs} with hardware IPs for time-predictable execution of \glspl{mct} and leading-edge domain-specialized programmable accelerators within the same heterogeneous SoC. With a peak performance of 304.9 GOPS at 1.6TOPS/W and 260.7 GOPS/mm$^2$, and 121.8 GFLOPS at 1.1TFLOPS/W and 107GFLOPS/mm$^2$, the proposed SoC offers a comprehensive solution for reliable and deterministic execution of \gls{ai}/\gls{dsp}-enhanced \gls{mc} edge applications, achieving \gls{soa} energy efficiency under \looseness=-1  1.2~W power envelope. %The hardware description of the design is released open-source to foster future research~\footnote{\texttt{\url{https://github.com/pulp-platform/carfield}}}.
%
%To the best of our knowledge, this is the first SoC to combine safety-critical features for MCS with HW IPs for time-predictable execution of MCTs and leading edge domain-specialized programmable processor clusters into the same heterogeneous SoC, providing a comprehensive solution for reliable and deterministic execution of ML-enhanced mixed-criticality edge applications, at state-of-the-art energy efficiency and less than 1.5W power envelope. 

%\textcolor{red}{The platform and all the non-technological IPs integrated in Carfield are released open source under a liberal licence to foster future software and hardware research on reliable, time-predictable, accelerated heterogeneous computing devices for mixed-criticality applications.}


\bibliography{references}
\bibliographystyle{plainnat}

%%%%%%%%%%%%%%%%%%%%%%%%%%%%%%%%%%%%%%%%%%%%%%%%%%%%%%%%%%%%
\newpage
\appendix
\onecolumn
\section{Technical Details}

\subsection{Details on Baseline Implementation}
\label{Appendix_Technical_Details_of_Baselines}
\textbf{EL2N}~\citep{paul2021deep} is defined as the error $L2$ norm between the true labels and predictions of the model. The examples with low scores are pruned out. We calculated error norms at epoch 20 from five independent runs, then the average was used for Ethe L2N score.

\textbf{Forgetting}~\citep{toneva2018empirical} is defined as the number of forgetting events, where the model prediction goes wrong after the correct prediction, up until the end of training. Rarely unforgotten samples are pruned out.

\textbf{AUM}~\citep{pleiss2020identifyingmislabeleddatausing} accumulates the margin, which means the gap between the target probabilities and the second largest prediction of a model. They calculate the margin at every epoch and then transform it into an AUM score at the end of the training. Here samples with small margins are considered as mislabeled samples, thus data points with small AUM scores are eliminated.

\textbf{Entropy}~\citep{coleman2020selectionproxyefficientdata} is calculated as the entropy of prediction probabilities at the end of training, and then the samples that have high entropy are selected into coreset. 

\textbf{Dyn-Unc}~\citep{he2024large} is also calculated at the end of training, with the window length $J$ set as 10. Samples with high uncertainties are selected into the subset after pruning.

\textbf{TDDS}~\citep{zhang2024spanning} adapts different hyperparameter for each pruning ratio. As they do not provide full information for implementation, we have no choice but to set parameters for the rest cases arbitrarily.
The provided setting for (pruning ratio, computation epoch $T$, the length of sliding window $K$) is (0.3, 70, 10), (0.5, 90, 10), (0.7, 80, 10), (0.8, 30, 10), and (0.9, 10, 5) for CIFAR-100, and for ImageNet-(0.3, 20, 10), (0.5, 20, 10), (0.7, 30, 20). Therefore, we set the parameter for CIFAR-10 as the same as CIFAR-100, and for 80\%, 90\% pruning on ImageNet-1k, we set them as (30, 20), following the choice for 70\% pruning.

\textbf{CCS}~\citep{zheng2022coverage} for the stratified sampling method, we adapt the AUM score as the original CCS paper does. They assign different hard cutoff rates for each pruning ratio. For CIFAR-10, the cutoff rates are (30\%, 0), (50\%, 0), (70\%, 10\%), (80\%, 10\%), (90\%, 30\%). For CIFAR-100 and ImageNet-1k, we set them as the same as in the original paper.

\textbf{D2}~\citep{maharana2023d2} for $\mathbb{D}^2$ pruning, we set the initial node using forgetting scores for CIFAR-10 and CIFAR-100, we set the number of neighbors $k$, and message passing weight $\gamma$ the same as in the original paper.

Note that, Infomax~\citep{tan2025data} was excluded as it employs different base hyperparameters in the original paper compared to other baselines and does not provide publicly available code.  Additionally, implementation details, such as the base score metric used to implement Infomax, are not provided. As we intend to compare other baseline methods with the same training hyperparameters, we do not include the accuracies of Infomax in our tables. %apply any skills to boost the generalization performance specific to our method. 
To see if we can match the performance of Infomax, we tested our method with different training details. For example, if we train the subset using the same number of iterations (not epoch) as the full dataset and use a different learning rate tuned for our method, then an improved accuracy of 59\% is achievable for 90\% pruning on CIFAR-100, which surpasses the reported performance of Infomax. For the ImageNet-1k dataset, our method outperforms Infomax without any base hyperparameter tuning, while also being cost-effective.

\subsection{Detailed Experimental Settings}
\label{Appendix_Technical_Details_of_Ours}
Here we clarify the technical details of our work.
For training the model on the full dataset and the selected subset, all parameters are used identically except for batch sizes. For CIFAR-10/100, we train ResNet-18 for 200 epochs with a batch size of 128, for each pruning ratio \{30\%, 50\%, 70\%, 80\%, 90\%\} we use different batch sizes with \{128, 128, 128, 64, 32\}. We set the initial learning rate as 0.1, the optimizer as SGD with momentum 0.9, and the scheduler as cosine annealing scheduler with weight decay 0.0005.
For training ImageNet, we use ResNet-34 as the network architecture. For all coresets with different pruning rates, we train models for 300,000 iterations with a 256 batch size. We use the SGD optimizer with 0.9 momentum and 0.0001 weight decay, using a 0.1 initial learning rate. The cosine annealing learning rate scheduler was used for training. For a fair comparison, we used the same parameters across all pruning methods, including ours. All experiments were conducted using an NVIDIA A6000 GPU. We also attach the implementation in the supplementary material.

For calculating the DUAL score, we need three parameters $T$, $J$, and $c_\gD$, each means score computation epoch, the length of the sliding window, and hyperparameter regarding the training dataset. We fix $J$ as 10 for all experiments.
We use ($T$, $J$, $c_\gD$) for each dataset as follows. For CIFAR-10, we use (30, 10, 5.5), for CIFAR-100, (30, 10, 4), and for ImageNet-1k, (60, 10, 11).
We first roughly assign the term $c_\gD$ based on the size of the initial dataset and by considering the relative difficulty of each, we set $c_\gD$ for CIFAR-100 smaller than that of CIFAR-10. For the ImageNet-1k dataset, which contains 1,281,167 images, the size of the initial dataset is large enough that we do not need to set $c_\gD$ to a small value to intentionally sample easier samples. Also, note that we fix the value of $C$ of Beta distribution at 15 across all experiments. A more detailed distribution, along with visualization, can be found in Appendix~\ref{Appendix_explanation_of_dual_pruning}.

Experiments with label noise and image corruption on CIFAR-100 are conducted under the same settings as described above, except for the hyperparameters for DUAL pruning. For label noise experiments, we set $T$ to 50 and $J$ to 10 across all label noise ratios. For $c_\gD$, we set it to 6 for 20\% and 30\% noise, 8 for 40\% noise. For image corruption experiments, we set $T$ to 30, $J$ to 10, and $c_\gD$ to 6 across all image corruption ratios. 

For the Tiny-ImageNet case, we train ResNet-34 for 90 epochs with a batch size of 256 across all pruning ratios, using a weight decay of 0.0001. The initial learning rate is set to 0.1 with the SGD optimizer, where the momentum is set to 0.9, combined with a cosine annealing learning rate scheduler. For the hyperparameters used in DUAL pruning,  we set $T$ to 60, $J$ to 10, and $c_\gD$ to 6 for the label noise experiments. For the image corruption experiments, we set $T$ to 60, $J$ to 10, and $c_\gD$ to 2. We follow the ImageNet-1k hyperparameters to implement the baselines.
\clearpage
\section{More Results on Experiments}
We evaluate our proposed DUAL score through a wide range of analyses in this section. In Appendix~\ref{Appendix_labelnoise_experiments} and~\ref{Appendix_imagecorruption_experiments}, we demonstrate the robustness of the DUAL score across a variety of experiments. In Appendix~\ref{Appendix_cross_architecture}, we investigate the cross-architecture performance of DUAL pruning. 
In Appendix~\ref{Appendix_beta_samapling}, we show that beta sampling performs well even when combined with other score metrics, such as EL2N and Dyn-Unc, and also shows strong performance when compared to other sampling methods, especially CCS.


We first investigate the stability of our DUAL score compared to other baselines. We calculate the Spearman rank correlation of each score and the average score across five runs, following \citet{paul2021deep}. As shown in Figure~\ref{fig:rank_corr}, snapshot-based methods such as EL2N and Entropy exhibit relatively low correlation compared to methods that consider training dynamics. In particular, the DUAL score shows minimal score variation across runs, resulting in a high Spearman rank correlation, indicating strong stability across random seeds. Notably, even when the scores are calculated at the 30th epoch, the Spearman correlation between the individual scores and the overall average score remains approximately 0.95.

\label{Appendix_Experiments}
\begin{figure}[ht]
    \centering
    \includegraphics[width=0.6\linewidth]{Figures/Rank_Correlation.pdf}
    \caption{\label{fig:rank_corr}Average of Spearman rank correlation among independent runs and an overall average of five runs.}
\end{figure}

Next, we compute the Dyn-Unc, TDDS, and AUM scores at the 30th epoch, as we do for our method, and then compare the test accuracy on the coreset. Our pruning method, using the DUAL score and ratio-adaptive beta sampling, outperforms the others by a significant margin, as illustrated in Figure~\ref{fig:computation_30_cifar100}.
We see using epoch 30 results in insufficient training dynamics for the others, thus it negatively impacts their performance.

\begin{figure}[ht]
    \centering
    \includegraphics[width=0.55\linewidth]{Figures/Computation_30_CIFAR100.pdf}
    \caption{\label{fig:computation_30_cifar100}Test accuracy comparison under limited computation budget (epoch 30)}
\end{figure}


\begin{figure}[ht]
    \centering
    \includegraphics[width=0.5\linewidth]{Figures/TestAcc_J_varying.pdf}
    \caption{\label{fig:J_varying} J varies from 5 to 15, showing minimal differences, which demonstrates its robustness. We fix $T=30$, $C_\gD=4$. Runs are averaged over three runs.}
\end{figure}






\newpage
\subsection{Image Classification with Label Noise}
\label{Appendix_labelnoise_experiments}
We evaluated the robustness of our DUAL pruning method against label noise. We introduced symmetric label noise by replacing the original labels with labels from other classes randomly. For example, if we apply 20\% label noise to a dataset with 100 classes, 20\% of the data points are randomly selected, and each label is randomly reassigned to another label with a probability of $1/99$ for the selected data points.

Even under 30\% and 40\% random label noise, our method achieves the best performance and accurately identifies the noisy labels, as can be seen in Figure~\ref{fig:label_noise_3040_ratio}. By examining the proportion of noise removed, we can see that our method operates close to optimal. 
\begin{figure}[htbp] 
    \centering
    \begin{subfigure}{0.45\textwidth}
        \centering
        \includegraphics[width=\textwidth]{Figures/label_noise30_ratio.pdf}
        \caption{\label{fig:label_noise_30_ratio}30\% label noise}
    \end{subfigure}
    \hfill
    \begin{subfigure}{0.45\textwidth}
        \centering
        \includegraphics[width=\textwidth]{Figures/label_noise40_ratio.pdf} 
        \caption{\label{fig:label_noise_40_ratio}40\% label noise}
    \end{subfigure}
    \caption{\label{fig:label_noise_3040_ratio}Ratio of pruned mislabeled data under 30\% and 40\% label noise on CIFAR-100}
\end{figure}

Figure~\ref{fig:label_noise_visualization} shows a scatter plot of the CIFAR-100 dataset under 20\% label noise. The model is trained for 30 epochs, and we compute the prediction mean (y-axis) and standard deviation (x-axis) for each data point. Red dots represent the 20\% mislabeled data. These points remain close to the origin (0,0) during the early training phase. Therefore, pruning at this stage allows us to remove mislabeled samples nearly optimally while selecting the most uncertain ones.

\begin{figure}[ht]
    \centering
    \includegraphics[width=\linewidth]{Figures/LabelNoise_Visualization.png}
    \caption{Pruning ratio is set to 50\%. Only 116 data points over 10,000 mislabeled data are selected as a subset where red dots indicate mislabeled data.}
    \label{fig:label_noise_visualization}
\end{figure}


We evaluated the performance of our proposed method across a wide range of pruning levels, from 10\% to 90\%, and compared the final accuracy with that of baseline methods. As shown in the Table~\ref{tab:label_noise_20_cifar}-\ref{tab:label_noise_20_tinyimagenet}, our method consistently outperforms the competition with a substantial margin in most cases. For a comprehensive analysis of performance under noisy conditions, please refer to Tables~\ref{tab:label_noise_20_cifar} to \ref{tab:label_noise_40_cifar} for CIFAR-100, which show results for 20\%, 30\%, and 40\% noise, respectively. Additionally, the results for 20\% label noise in Tiny-ImageNet are shown in Table~\ref{tab:label_noise_20_tinyimagenet}.
%%%%%%%%%%%%%%%%%%%%%%%%%%%%%%%%% CIFAR-100 %%%%%%%%%%%%%%%%%%%%%%%%%%%%%%%%%%%%%
\begin{table}[ht]
\caption{\label{tab:label_noise_20_cifar}Comparison of test accuracy of DUAL pruning with existing coreset selection methods under 20\% label noise using ResNet-18 for CIFAR-100. The model trained with the full dataset achieves \textbf{65.28\%} test accuracy on average. Results are averaged over five runs.}
\setlength{\tabcolsep}{3pt}
\centering
\begin{tabular}{lccccccc}
    \toprule
    \textbf{Pruning Rate} & \textbf{10\%} & \textbf{20\%} & \textbf{30\%} & \textbf{50\%} & \textbf{70\%} & \textbf{80\%} & \textbf{90\%} \\
    \midrule
    \textbf{Random} & 64.22 \scriptsize{$\pm 0.37 $} & 63.12 \scriptsize{$\pm 0.26 $} & 61.75 \scriptsize{$\pm 0.24 $} & 58.13 \scriptsize{$\pm 0.22 $} & 50.11 \scriptsize{$\pm 0.75 $} & 44.29 \scriptsize{$\pm 1.2 $} & 32.04 \scriptsize{$\pm 0.93 $} \\
    
    \textbf{Entropy} & 63.51 \scriptsize{$\pm 0.25 $} & 60.59 \scriptsize{$\pm 0.23 $} & 56.75 \scriptsize{$\pm 0.37 $} & 44.90 \scriptsize{$\pm 0.74 $} & 24.43 \scriptsize{$\pm 0.12 $} & 16.60 \scriptsize{$\pm 0.29$} & 10.35 \scriptsize{$\pm 0.49$}\\
    
    \textbf{Forgetting} & 64.29 \scriptsize{$\pm 0.26 $} & 63.40 \scriptsize{$\pm 0.14 $} & 64.00 \scriptsize{$\pm 0.27 $} & 67.51 \scriptsize{$\pm 0.52 $} & 59.29 \scriptsize{$\pm 0.66 $} & 50.11 \scriptsize{$\pm 0.91 $} & 32.08 \scriptsize{$\pm 1.15 $} \\
    
    \textbf{EL2N} & 64.51 \scriptsize{$\pm 0.35 $} & 62.67 \scriptsize{$\pm 0.28 $} & 59.85 \scriptsize{$\pm 0.31 $} & 46.94 \scriptsize{$\pm 0.75 $} & 19.32 \scriptsize{$\pm 0.87 $} & 11.02 \scriptsize{$\pm 0.45 $} & 6.83 \scriptsize{$\pm 0.21 $} \\
    
    \textbf{AUM} & 64.54 \scriptsize{$\pm 0.23$} & 60.72 \scriptsize{$\pm 0.22$} & 50.38 \scriptsize{$\pm 0.66$} & 22.03 \scriptsize{$\pm 0.92$}& 5.55 \scriptsize{$\pm 0.26 $} & 3.00 \scriptsize{$\pm 0.18 $} & 1.68 \scriptsize{$\pm 0.10$}\\
    
    \textbf{Moderate} & 64.45 \scriptsize{$\pm 0.29 $} & 62.90 \scriptsize{$\pm 0.33 $} & 61.46 \scriptsize{$\pm 0.50 $} & 57.53 \scriptsize{$\pm 0.61 $} & 49.50 \scriptsize{$\pm 1.06 $} & 43.81 \scriptsize{$\pm 0.80 $} & 29.15 \scriptsize{$\pm 0.79 $}  \\
    
    \textbf{Dyn-Unc} & 68.17 \scriptsize{$\pm 0.26 $} & 71.56 \scriptsize{$\pm 0.27 $} & 74.12 \scriptsize{$\pm 0.15 $} & \textbf{73.43} \scriptsize{$\pm 0.12 $} & 67.21 \scriptsize{$\pm 0.27 $} & 61.38 \scriptsize{$\pm 0.27 $} & \underline{48.00} \scriptsize{$\pm 0.79 $} \\
    
    \textbf{TDDS} & 62.86 \scriptsize{$\pm 0.36 $} & 61.96 \scriptsize{$\pm 1.03 $} & 61.38 \scriptsize{$\pm 0.53 $} & 59.16 \scriptsize{$\pm 0.94 $} & 48.93 \scriptsize{$\pm 1.68 $} & 43.83  \scriptsize{$\pm 1.13 $} & 34.05 \scriptsize{$\pm 0.49 $} \\
    
    \textbf{CCS} & 64.30 \scriptsize{$\pm 0.21 $} & 63.24 \scriptsize{$\pm 0.24 $} & 61.91 \scriptsize{$\pm 0.45 $} & 58.24 \scriptsize{$\pm 0.29 $} & 50.24 \scriptsize{$\pm 0.39 $} & 43.76 \scriptsize{$\pm 1.07 $} & 30.67  \scriptsize{$\pm 0.96 $} \\
    
    \midrule
    
    \textbf{DUAL} & \underline{69.78} \scriptsize{$\pm 0.28 $} & \textbf{74.79} \scriptsize{$\pm 0.07 $} & \textbf{75.40} \scriptsize{$\pm 0.11 $} & \textbf{73.43} \scriptsize{$\pm 0.16 $} & \underline{67.57} \scriptsize{$\pm 0.18 $} & \underline{61.46} \scriptsize{$\pm 0.45 $} & 43.30 \scriptsize{$\pm 1.59 $} \\
    
    \textbf{DUAL+$\beta$} & \textbf{69.95} \scriptsize{$\pm 0.60 $} & \underline{74.68} \scriptsize{$\pm 1.22 $} & \underline{75.37} \scriptsize{$\pm 1.33 $} & \underline{73.29} \scriptsize{$\pm 0.84 $} & \textbf{68.43} \scriptsize{$\pm 0.77 $} & \textbf{63.74} \scriptsize{$\pm 0.35 $} & \textbf{54.04} \scriptsize{$\pm 0.92 $} \\
    
    \bottomrule
\end{tabular}
\end{table}





\begin{table}[ht]
\caption{\label{tab:label_noise_30_cifar}Comparison of test accuracy of DUAL pruning with existing coreset selection methods under 30\% label noise using ResNet-18 for CIFAR-100. The model trained with the full dataset achieves \textbf{58.25\%} test accuracy on average. Results are averaged over five runs.}
\setlength{\tabcolsep}{3.1pt}
\centering
\begin{tabular}{lccccccc}
    \toprule
    \textbf{Pruning Rate} & \textbf{10\%} & \textbf{20\%} & \textbf{30\%} & \textbf{50\%} & \textbf{70\%} & \textbf{80\%} & \textbf{90\%} \\
    \midrule
    \textbf{Random} & 57.67 \scriptsize{$\pm 0.52 $} & 56.29 \scriptsize{$\pm 0.55 $} & 54.70 \scriptsize{$\pm 0.60 $} & 51.41 \scriptsize{$\pm 0.38 $} & 42.67 \scriptsize{$\pm 0.80 $} & 36.86 \scriptsize{$\pm 1.01 $} & 25.64 \scriptsize{$\pm 0.82 $} \\
    
    \textbf{Entropy} & 55.51 \scriptsize{$\pm 0.42 $} & 51.87 \scriptsize{$\pm 0.36 $} & 47.16 \scriptsize{$\pm 0.58 $} & 35.35 \scriptsize{$\pm 0.49 $} & 18.69 \scriptsize{$\pm 0.76 $} & 13.61 \scriptsize{$\pm 0.42$} & 8.58 \scriptsize{$\pm 0.49$}\\
    
    \textbf{Forgetting} & 56.76 \scriptsize{$\pm 0.62 $} & 56.43 \scriptsize{$\pm 0.28 $} & 58.84 \scriptsize{$\pm 0.26 $} & 64.51 \scriptsize{$\pm 0.37 $} & 61.26 \scriptsize{$\pm 0.69 $} & 52.94 \scriptsize{$\pm 0.68 $} & 34.99 \scriptsize{$\pm 1.16 $} \\
    
    \textbf{EL2N} & 56.39 \scriptsize{$\pm 0.53 $} & 54.41 \scriptsize{$\pm 0.68 $} & 50.29 \scriptsize{$\pm 0.40 $} & 35.65 \scriptsize{$\pm 0.79 $} & 13.05 \scriptsize{$\pm 0.51 $} & 8.52 \scriptsize{$\pm 0.40 $} & 6.16 \scriptsize{$\pm 0.40 $} \\
    
    \textbf{AUM} & 56.51 \scriptsize{$\pm 0.56$} & 49.10 \scriptsize{$\pm 0.72$} & 37.57 \scriptsize{$\pm 0.66$} & 11.56 \scriptsize{$\pm 0.46$}& 2.79 \scriptsize{$\pm 0.23 $} & 1.87 \scriptsize{$\pm 0.24 $} & 1.43 \scriptsize{$\pm 0.12$}\\
    
    \textbf{Moderate} & 57.31 \scriptsize{$\pm 0.75 $} & 56.11 \scriptsize{$\pm 0.45 $} & 54.52 \scriptsize{$\pm 0.48 $} & 50.71 \scriptsize{$\pm 0.42 $} & 42.47 \scriptsize{$\pm 0.29 $} & 36.21 \scriptsize{$\pm 1.09 $} & 24.85 \scriptsize{$\pm 1.72$}  \\
    
    \textbf{Dyn-Unc} & 62.20 \scriptsize{$\pm 0.44$} & \underline{66.48} \scriptsize{$\pm 0.40 $} & 70.45 \scriptsize{$\pm 0.50 $} & \textbf{71.91} \scriptsize{$\pm 0.34 $} & \underline{66.53} \scriptsize{$\pm 0.19$} & \underline{61.95} \scriptsize{$\pm 0.46 $} & \underline{49.51} \scriptsize{$\pm 0.52$} \\
    
    \textbf{TDDS} & 57.24 \scriptsize{$\pm 0.44 $} & 55.64 \scriptsize{$\pm 0.46 $} & 53.97 \scriptsize{$\pm 0.46 $} & 49.04 \scriptsize{$\pm 1.05 $} & 39.90 \scriptsize{$\pm 1.21$} & 35.02 \scriptsize{$\pm 1.34 $} & 26.99 \scriptsize{$\pm 1.03$} \\
    
    \textbf{CCS} & 57.26 \scriptsize{$\pm 0.48 $} & 56.52 \scriptsize{$\pm 0.23 $} & 54.76 \scriptsize{$\pm 0.52 $} & 51.29 \scriptsize{$\pm 0.32 $} & 42.33 \scriptsize{$\pm 0.78 $} & 36.61 \scriptsize{$\pm 1.31 $} & 25.64 \scriptsize{$\pm 1.65 $} \\
    
    \midrule
    
    \textbf{DUAL} & \underline{62.42} \scriptsize{$\pm 0.48 $} & \textbf{67.52} \scriptsize{$\pm 0.40 $} & \textbf{72.65} \scriptsize{$\pm 0.17 $} & 71.55 \scriptsize{$\pm 0.23 $} & 66.35 \scriptsize{$\pm 0.14 $} & 61.57 \scriptsize{$\pm 0.44 $} & 48.70 \scriptsize{$\pm 0.19 $} \\
    
    \textbf{DUAL+$\beta$} & \textbf{63.02} \scriptsize{$\pm 0.41 $} & \textbf{67.52} \scriptsize{$\pm 0.24 $} & \underline{72.57} \scriptsize{$\pm 0.15 $} & \underline{71.68} \scriptsize{$\pm 0.27 $} & \textbf{66.75} \scriptsize{$\pm 0.45 $} & \textbf{62.28} \scriptsize{$\pm 0.43 $} & \textbf{52.60} \scriptsize{$\pm 0.87 $} \\
    
    \bottomrule
\end{tabular}
\end{table}




\begin{table}[ht]
\caption{\label{tab:label_noise_40_cifar}Comparison of test accuracy of DUAL pruning with existing coreset selection methods under 40\% label noise using ResNet-18 for CIFAR-100. The model trained with the full dataset achieves \textbf{52.74\%} test accuracy on average. Results are averaged over five runs.}
\setlength{\tabcolsep}{3.1pt}
\centering
\begin{tabular}{lccccccc}
    \toprule
    \textbf{Pruning Rate} & \textbf{10\%} & \textbf{20\%} & \textbf{30\%} & \textbf{50\%} & \textbf{70\%} & \textbf{80\%} & \textbf{90\%} \\
    \midrule
    \textbf{Random} & 51.13 \scriptsize{$\pm 0.71 $} & 48.42 \scriptsize{$\pm 0.46 $} & 46.99 \scriptsize{$\pm 0.29 $} & 43.24 \scriptsize{$\pm 0.46 $} & 33.60 \scriptsize{$\pm 0.50 $} & 28.28 \scriptsize{$\pm 0.81 $} & 19.52 \scriptsize{$\pm 0.79 $} \\
    
    \textbf{Entropy} & 49.14 \scriptsize{$\pm 0.32 $} & 46.06 \scriptsize{$\pm 0.58 $} & 41.83 \scriptsize{$\pm 0.73 $} & 28.26 \scriptsize{$\pm 0.37 $} & 15.64  \scriptsize{$\pm 0.19 $} & 12.21 \scriptsize{$\pm 0.68 $} & 8.23 \scriptsize{$\pm 0.40 $}\\
    
    \textbf{Forgetting} & 50.98 \scriptsize{$\pm 0.72 $} & 50.36 \scriptsize{$\pm 0.48 $} & 52.86 \scriptsize{$\pm 0.47 $} & 60.48 \scriptsize{$\pm 0.68 $} & 61.55 \scriptsize{$\pm 0.58 $} & 54.57 \scriptsize{$\pm 0.86 $} & 37.68 \scriptsize{$\pm 1.63 $} \\
    
    \textbf{EL2N} & 50.09 \scriptsize{$\pm 0.86 $} & 46.35 \scriptsize{$\pm 0.48 $} & 41.57 \scriptsize{$\pm 0.26 $} & 23.42 \scriptsize{$\pm 0.80 $} & 9.00 \scriptsize{$\pm 0.25 $} & 6.80 \scriptsize{$\pm 0.44 $} & 5.58 \scriptsize{$\pm 0.40$} \\
    
    \textbf{AUM} & 50.60 \scriptsize{$\pm 0.54 $} & 41.84 \scriptsize{$\pm 0.76 $} & 26.29 \scriptsize{$\pm 0.72 $} & 5.49 \scriptsize{$\pm 0.19 $} & 1.95 \scriptsize{$\pm 0.21 $} & 1.44 \scriptsize{$\pm 0.14 $ } & 1.43 \scriptsize{$\pm 0.24 $}\\
    
    \textbf{Moderate} & 50.62 \scriptsize{$\pm 0.27 $} & 48.70 \scriptsize{$\pm 0.79 $} & 47.01  \scriptsize{$\pm 0.21 $} & 42.73 \scriptsize{$\pm 0.39 $} & 32.35 \scriptsize{$\pm 1.29 $} & 27.72 \scriptsize{$\pm 1.69 $} & 19.85 \scriptsize{$\pm 1.11 $}  \\
    
    \textbf{Dyn-Unc} & \underline{54.46} \scriptsize{$\pm 0.27$} & \underline{59.02} \scriptsize{$\pm 0.23 $} & 63.86 \scriptsize{$\pm 0.47 $} & \underline{69.76} \scriptsize{$\pm 0.16 $} & \textbf{65.36} \scriptsize{$\pm 0.14$} & \underline{61.37} \scriptsize{$\pm 0.32 $} & \underline{50.49} \scriptsize{$\pm 0.71$} \\
    
    \textbf{TDDS} & 50.65 \scriptsize{$\pm 0.23 $} & 48.83 \scriptsize{$\pm 0.38 $} & 46.93 \scriptsize{$\pm 0.66 $} &  41.85 \scriptsize{$\pm 0.37 $} & 33.31 \scriptsize{$\pm 0.79 $} & 29.39 \scriptsize{$\pm 0.35 $} & 21.09 \scriptsize{$\pm 0.89 $} \\
    
    \textbf{CCS} & 64.30 \scriptsize{$\pm 0.29 $} & 48.54 \scriptsize{$\pm 0.35 $} & 46.81 \scriptsize{$\pm 0.45 $} & 42.57 \scriptsize{$\pm 0.32 $} & 33.19 \scriptsize{$\pm 0.88 $} & 28.32 \scriptsize{$\pm 0.59 $} & 19.61 \scriptsize{$\pm 0.75 $} \\
    
    \midrule
    
    \textbf{DUAL} & \underline{54.46} \scriptsize{$\pm 0.33 $} & 58.99 \scriptsize{$\pm 0.34 $} & \textbf{64.71} \scriptsize{$\pm 0.44 $} & \underline{69.87} \scriptsize{$\pm 0.28 $} & 64.21 \scriptsize{$\pm 0.21 $} & 59.90 \scriptsize{$\pm 0.44 $} & 49.61 \scriptsize{$\pm 0.27 $} \\
    
    \textbf{DUAL+$\beta$} & \textbf{54.53} \scriptsize{$\pm 0.06 $} & \textbf{59.65} \scriptsize{$\pm 0.41 $} & \underline{64.67} \scriptsize{$\pm 0.34 $} & \textbf{70.09} \scriptsize{$\pm 0.33 $} & \underline{65.12} \scriptsize{$\pm 0.46$} & \underline{60.62} \scriptsize{$\pm 0.30 $} & \textbf{51.51} \scriptsize{$\pm 0.41 $} \\
    
    \bottomrule
\end{tabular}
\end{table}

%%%%%%%%%%%%%%%%%%%%%%%%%%%%%%%%%%%%%%%%%%%%%%%%%%%%%%%%%%%%%%%%%%%%%%%%


%%%%%%%%%%%%%%%%%%%%%%%%%%%% Tiny ImageNet %%%%%%%%%%%%%%%%%%%%%%%%%%%%
\begin{table}[ht]
\caption{\label{tab:label_noise_20_tinyimagenet}Comparison of test accuracy of DUAL pruning with existing coreset selection methods under 20\% label noise using ResNet-34 for Tiny-ImageNet. The model trained with the full dataset achieves \textbf{42.24\%} test accuracy on average. Results are averaged over three runs.}
\setlength{\tabcolsep}{3.1pt}
\centering
\begin{tabular}{lccccccc}
    \toprule
    \textbf{Pruning Rate} & \textbf{10\%} & \textbf{20\%} & \textbf{30\%} & \textbf{50\%} & \textbf{70\%} & \textbf{80\%} & \textbf{90\%} \\
    \midrule
    \textbf{Random} & 41.09 \scriptsize{$\pm 0.29 $} & 39.24 \scriptsize{$\pm 0.39 $} & 37.17 \scriptsize{$\pm 0.23 $} & 32.93 \scriptsize{$\pm 0.45 $} & 26.12 \scriptsize{$\pm 0.63 $} & 22.11 \scriptsize{$\pm 0.42 $} & 13.88 \scriptsize{$\pm 0.60 $} \\
    
    \textbf{Entropy} & 40.69 \scriptsize{$\pm 0.06 $} & 38.14 \scriptsize{$\pm 0.92 $} & 35.93 \scriptsize{$\pm 1.56 $} & 31.24 \scriptsize{$\pm 1.76 $} & 23.65 \scriptsize{$\pm 2.05 $} & 18.53 \scriptsize{$\pm 2.10 $} & 10.52 \scriptsize{$\pm 1.64 $} \\
    
    \textbf{Forgetting} & 43.60 \scriptsize{$\pm 0.65 $} & 44.82 \scriptsize{$\pm 0.20 $} & 45.65 \scriptsize{$\pm 0.48 $} & 46.05 \scriptsize{$\pm 0.07 $} & 41.08 \scriptsize{$\pm 0.53 $} & 34.89 \scriptsize{$\pm 0.12 $} & 24.58 \scriptsize{$\pm 0.06 $} \\
    
    \textbf{EL2N} & 41.05 \scriptsize{$\pm 0.35 $} & 38.88 \scriptsize{$\pm 0.63 $} & 32.91 \scriptsize{$\pm 0.39 $} & 20.89 \scriptsize{$\pm 0.80 $} & 8.08 \scriptsize{$\pm 0.24 $} & 4.92  \scriptsize{$\pm 0.32 $} & 3.12 \scriptsize{$\pm 0.07 $} \\
    
    \textbf{AUM} & 40.20 \scriptsize{$\pm 0.27$} & 34.68 \scriptsize{$\pm 0.35 $} & 29.01 \scriptsize{$\pm 0.12$} & 10.45 \scriptsize{$\pm 0.85 $} & 2.52 \scriptsize{$\pm 0.75 $} & 1.30 \scriptsize{$\pm 0.23 $} & 0.79 \scriptsize{$\pm 0.40 $} \\
    
    \textbf{Moderate} & 41.23 \scriptsize{$\pm 0.38 $} & 38.58 \scriptsize{$\pm 0.60 $} & 37.60 \scriptsize{$\pm 0.66 $} & 32.65 \scriptsize{$\pm 1.18 $} & 25.68 \scriptsize{$\pm 0.40 $} & 21.74 \scriptsize{$\pm 0.63 $} & 14.15 \scriptsize{$\pm 0.73 $}  \\
    
    \textbf{Dyn-Unc} & \underline{45.67} \scriptsize{$\pm 0.78  $} & 47.49 \scriptsize{$\pm 0.46 $} & \underline{49.38} \scriptsize{$\pm 0.17 $} & \underline{47.47} \scriptsize{$\pm 0.32 $} & 42.49 \scriptsize{$\pm 0.39$} & 37.44 \scriptsize{$\pm 0.73 $} & \underline{28.48} \scriptsize{$\pm 0.73 $} \\
    
    \textbf{TDDS} & 36.56 \scriptsize{$\pm 0.54$} & 36.90 \scriptsize{$\pm 0.48 $} & 47.62 \scriptsize{$\pm 1.36$} & 42.44 \scriptsize{$\pm 0.63$} & 34.32 \scriptsize{$\pm 0.26$}& 24.32 \scriptsize{$\pm 0.26 $} & 17.43 \scriptsize{$\pm 0.17$}  \\
    
    \textbf{CCS} & 40.49 \scriptsize{$\pm 0.67 $} & 39.06 \scriptsize{$\pm 0.24 $} & 37.67 \scriptsize{$\pm 0.46$} & 30.83 \scriptsize{$\pm 1.02$} & 22.38 \scriptsize{$\pm 0.70 $} & 19.66 \scriptsize{$\pm 0.58$} & 12.23 \scriptsize{$\pm 0.64$} \\
    
    \midrule
    
    \textbf{DUAL} & \textbf{45.76} \scriptsize{$\pm 0.67 $} & \textbf{48.20} \scriptsize{$\pm 0.20 $} & \textbf{49.94} \scriptsize{$\pm 0.17 $} & \textbf{48.19} \scriptsize{$\pm 0.27$} & \underline{42.80} \scriptsize{$\pm 0.74 $} & \textbf{37.90} \scriptsize{$\pm 0.59$} & 27.80 \scriptsize{$\pm 0.49 $} \\
    
    \textbf{DUAL+$\beta$} & 45.21  \scriptsize{$\pm 0.08$} & \underline{47.76} \scriptsize{$\pm 0.33$} & 48.99 \scriptsize{$\pm 0.32 $} & 46.95 \scriptsize{$\pm 0.23 $} & \textbf{43.01} \scriptsize{$\pm 0.43$} & \textbf{37.91} \scriptsize{$\pm 0.28 $} & \textbf{28.78} \scriptsize{$\pm 0.57$} \\
    
    \bottomrule
\end{tabular}

\end{table}

%%%%%%%%%%%%%%%%%%%%%%%%%%%%%%%%%%%%%%%%%%%%%%%%%%%%%%%%%%%%%%%%%%%%%%%%%%%%%%%%%%%%%%%%%%%%



\clearpage

\subsection{Image Classification with Image Corruption}
\label{Appendix_imagecorruption_experiments}
We also evaluated the robustness of our proposed method against five different types of realistic image corruption: motion blur, fog, reduced resolution, rectangular occlusion, and Gaussian noise across the corruption rate from 20\% to 40\%. The ratio of each type of corruption is 4\% for 20\% corruption, 6\% for 30\% corruption, and 8\% for 40\% corruption. Example images for each type of corruption can be found in Figure~\ref{fig:example_imagecorruption}. Motion blur, reduced resolution, and rectangular occlusion are somewhat distinguishable, whereas fog and Gaussian noise are difficult for the human eye to differentiate. Somewhat surprisingly, our DUAL pruning prioritizes removing the most challenging examples, such as fog and Gaussian corrupted images, as shown in Figure~\ref{fig:imagecorruption_all}.

\begin{figure}[ht]
    \centering
    \includegraphics[width=1\linewidth]{Figures/example_imagecorruption.pdf}
    \caption{Examples of the different types of noise used for image corruption. Here we consider motion blur, fog, resolution, rectangle, and Gaussian noise.}
    \label{fig:example_imagecorruption}
\end{figure}


\begin{figure}[htbp] 
    \centering
    \begin{subfigure}{0.325\textwidth}
        \centering
        \includegraphics[width=\textwidth]{Figures/imagecorruption_20_ratio.pdf}
        \caption{20\% image corruption}
        \label{fig:imagecorruption20_ratio}
    \end{subfigure}
    \hfill
    \begin{subfigure}{0.325\textwidth}
        \centering
        \includegraphics[width=\textwidth]{Figures/imagecorruption_30_ratio.pdf} 
        \caption{30\% image corruption}
        \label{fig:imagecorruption30_ratio}
    \end{subfigure}
    \hfill
    \begin{subfigure}{0.325\textwidth}
        \centering
        \includegraphics[width=\textwidth]{Figures/imagecorruption_40_ratio.pdf} 
        \caption{40\% image corruption}
        \label{fig:imagecorruption40_ratio}
    \end{subfigure}
    \caption{Ratio of pruned corrupted samples with corruption rate of 20\%, 30\% and 40\% on CIFAR-100.}
    \label{fig:imagecorruption_203040_ratio}
\end{figure}


\begin{figure}[ht]
    \centering
    \includegraphics[width=1\linewidth]{Figures/imagecorruption_all.pdf}
    \caption{Illustration of the different types of noise used for image corruption. DUAL pruning prioritizes removing the most challenging corrupted images, such as fog and Gaussian noise.}  
    \label{fig:imagecorruption_all}
\end{figure}




We evaluated the performance of our proposed method across a wide range of pruning levels, from 10\% to 90\%, and compared the final accuracy with that of baseline methods. As shown in the table, our method consistently outperforms the competitors in most cases. For a comprehensive analysis of performance under noisy conditions, please refer to Tables~\ref{tab:image_corruption_20_cifar} to \ref{tab:image_corruption_40_cifar} for CIFAR-100, which show results for 20\%, 30\%, and 40\% corrupted images, respectively. Additionally, the results for 20\% image corruption in Tiny-ImageNet are shown in Table~\ref{tab:image_corruption_20_tinyimagenet}.


\begin{table}[ht]
\caption{\label{tab:image_corruption_20_cifar}Comparison of test accuracy of DUAL pruning with existing coreset selection methods under 20\% image corrupted data using ResNet-18 for CIFAR-100. The model trained with the full dataset achieves \textbf{75.45\%} test accuracy on average. Results are averaged over five runs.}
\setlength{\tabcolsep}{3.1pt}
\centering
\begin{tabular}{lccccccc}
    \toprule
    \textbf{Pruning Rate} & \textbf{10\%} & \textbf{20\%} & \textbf{30\%} & \textbf{50\%} & \textbf{70\%} & \textbf{80\%} & \textbf{90\%} \\
    \midrule
    \textbf{Random} & 74.54 \scriptsize{$\pm 0.14$} & 73.08 \scriptsize{$\pm 0.27 $} & 71.61 \scriptsize{$\pm 0.14 $} & 67.52 \scriptsize{$\pm 0.32 $} & 59.57 \scriptsize{$\pm 0.52 $} & 52.79 \scriptsize{$\pm 0.68 $} & 38.26 \scriptsize{$\pm 1.32 $} \\
    
    \textbf{Entropy} & 74.74 \scriptsize{$\pm 0.25 $} & 73.15 \scriptsize{$\pm 0.26$} & 71.15 \scriptsize{$\pm 0.13 $} & 64.97 \scriptsize{$\pm 0.52 $} & 49.49 \scriptsize{$ \pm 1.40$} & 35.92 \scriptsize{$\pm 0.64 $} & 17.91 \scriptsize{$\pm 0.45 $} \\
    
    \textbf{Forgetting} & 74.33 \scriptsize{$\pm 0.25 $} & 73.25 \scriptsize{$\pm 0.29$} & 71.68 \scriptsize{$\pm 0.37 $} & 67.31 \scriptsize{$\pm 0.23$} & 58.93 \scriptsize{$\pm 0.35$} & 52.01 \scriptsize{$\pm 0.62$} & 38.95 \scriptsize{$\pm 1.24$} \\
    
    \textbf{EL2N} & 75.22 \scriptsize{$\pm 0.09 $} & 74.23 \scriptsize{$\pm 0.11 $} & 72.01 \scriptsize{$\pm 0.18 $} &48.19  \scriptsize{$\pm 0.47 $} & 14.81 \scriptsize{$\pm 0.14$} & 8.68 \scriptsize{$\pm 0.06 $} & 7.60 \scriptsize{$\pm 0.18$} \\
    
    \textbf{AUM} & 75.26 \scriptsize{$\pm 0.25$} & 74.47 \scriptsize{$\pm 0.31 $} & 71.96 \scriptsize{$\pm 0.22$} & 47.50 \scriptsize{$\pm 1.39 $} & 15.35 \scriptsize{$\pm 1.79$} & 8.98 \scriptsize{$\pm 1.37 $} & 5.47 \scriptsize{$\pm 0.85$} \\
    
    \textbf{Moderate} & 75.25 \scriptsize{$\pm 0.23 $} & 74.34 \scriptsize{$\pm 0.31 $} & 72.80 \scriptsize{$\pm 0.25 $} & 68.75 \scriptsize{$\pm 0.40 $} & 60.98 \scriptsize{$\pm 0.39 $} & 54.21 \scriptsize{$\pm 0.93 $} & 38.72 \scriptsize{$\pm 0.30 $} \\
    
    \textbf{Dyn-Unc} & 75.22 \scriptsize{$\pm 0.25$} & 75.51 \scriptsize{$\pm 0.22$} & \underline{75.09} \scriptsize{$\pm 0.23$} & 72.02 \scriptsize{$\pm 0.07$} & 62.17 \scriptsize{$\pm 0.55$} & 53.49 \scriptsize{$\pm 0.47$} & 35.44 \scriptsize{$\pm 0.49$} \\
    
    \textbf{TDDS} & 73.29 \scriptsize{$\pm 0.40 $} & 72.90 \scriptsize{$\pm 0.31 $} & 71.83 \scriptsize{$\pm 0.78 $} & 67.24 \scriptsize{$\pm 0.92 $} & 57.30 \scriptsize{$\pm 3.11 $} & 55.14 \scriptsize{$\pm 1.21 $} & \underline{41.58} \scriptsize{$\pm 2.10 $} \\
    
    \textbf{CCS} & 74.31 \scriptsize{$\pm 0.14 $} & 73.04 \scriptsize{$\pm 0.23 $} & 71.83 \scriptsize{$\pm 0.25 $} & 67.61 \scriptsize{$\pm 0.48$} & 59.61 \scriptsize{$\pm 0.64$} & 53.35 \scriptsize{$\pm 0.71 $} & 39.04 \scriptsize{$\pm 1.14$} \\
    
    \midrule
    
    \textbf{DUAL} & \textbf{75.95} \scriptsize{$\pm 0.19$} & \underline{75.66} \scriptsize{$\pm 0.23$} & \textbf{75.10} \scriptsize{$\pm 0.23$} & \textbf{72.64} \scriptsize{$\pm 0.27$} & \underline{65.29} \scriptsize{$\pm 0.64$} & \underline{57.55} \scriptsize{$\pm 0.55$} & 37.34 \scriptsize{$\pm 1.70$} \\
    
    \textbf{DUAL+$\beta$} & \underline{75.50} \scriptsize{$\pm 0.21$} & \textbf{75.78} \scriptsize{$\pm 0.15 $} & \textbf{75.10} \scriptsize{$\pm 0.13 $} & \underline{72.08} \scriptsize{$\pm 0.22 $} & \textbf{65.84} \scriptsize{$\pm 0.37 $} & \textbf{62.20} \scriptsize{$\pm 0.72 $} & \textbf{53.96} \scriptsize{$\pm 0.35 $} \\
    
    \bottomrule
\end{tabular}
\end{table}



\begin{table}[ht]
\caption{\label{tab:image_corruption_30_cifar}Comparison of test accuracy of DUAL pruning with existing coreset selection methods under 30\% image corrupted data using ResNet-18 for CIFAR-100. The model trained with the full dataset achieves \textbf{73.77\%} test accuracy on average. Results are averaged over five runs.}
\setlength{\tabcolsep}{3.1pt}
\centering
\begin{tabular}{lccccccc}
    \toprule
    \textbf{Pruning Rate} & \textbf{10\%} & \textbf{20\%} & \textbf{30\%} & \textbf{50\%} & \textbf{70\%} & \textbf{80\%} & \textbf{90\%} \\
    \midrule
    \textbf{Random} & 72.71 \scriptsize{$ \pm 0.34 $} & 71.28 \scriptsize{$ \pm 0.31 $}  & 69.84 \scriptsize{$ \pm 0.24 $} & 65.42 \scriptsize{$ \pm 0.33 $} & 56.72 \scriptsize{$ \pm 0.56 $} & 49.71 \scriptsize{$ \pm 0.65 $} & 35.75 \scriptsize{$ \pm 1.41 $} \\
    
    \textbf{Entropy} & 72.94 \scriptsize{$ \pm 0.09 $} & 71.14 \scriptsize{$ \pm 0.14 $} & 68.74 \scriptsize{$ \pm 0.20 $} & 61.34 \scriptsize{$ \pm 0.59 $} & 42.70 \scriptsize{$ \pm 1.02 $} & 29.46 \scriptsize{$ \pm 1.68 $} & 12.55 \scriptsize{$ \pm 0.66 $} \\
    
    \textbf{Forgetting} & 72.67 \scriptsize{$ \pm 0.21 $} & 71.22 \scriptsize{$ \pm 0.08 $} & 69.65 \scriptsize{$ \pm 0.45 $} & 65.25 \scriptsize{$ \pm 0.33 $} & 56.47 \scriptsize{$ \pm 0.31 $} &49.07  \scriptsize{$ \pm 0.32 $} & 34.62 \scriptsize{$ \pm 1.15 $} \\
    
    \textbf{EL2N} & 73.33 \scriptsize{$ \pm 0.08 $} & 71.99 \scriptsize{$ \pm 0.11 $} & 67.72 \scriptsize{$ \pm 0.50 $} & 37.57 \scriptsize{$ \pm 0.70 $} & 10.75 \scriptsize{$ \pm 0.28 $} & 9.08 \scriptsize{$ \pm 0.30 $} & 7.75 \scriptsize{$ \pm 0.08 $} \\
    
    \textbf{AUM} & 73.73 \scriptsize{$ \pm 0.19 $} & 72.99 \scriptsize{$ \pm 0.22 $} & 70.93 \scriptsize{$ \pm 0.33 $} & 57.13 \scriptsize{$ \pm 0.42 $} & 28.98 \scriptsize{$ \pm 0.50 $} & 19.73 \scriptsize{$ \pm 0.28 $} & 12.18 \scriptsize{$ \pm 0.46 $} \\
    
    \textbf{Moderate} & \textbf{74.02} \scriptsize{$ \pm 0.28 $} & 72.70 \scriptsize{$ \pm 0.30 $} & 71.51 \scriptsize{$ \pm 0.26 $} & 67.35 \scriptsize{$ \pm 0.16 $} & 59.47 \scriptsize{$ \pm 0.34 $} & 52.95 \scriptsize{$ \pm 0.60 $} & 37.45 \scriptsize{$ \pm 1.21 $} \\
    
    \textbf{Dyn-Unc} & 73.86 \scriptsize{$\pm 0.21 $} & 73.78 \scriptsize{$ \pm 0.20 $} & \textbf{73.78} \scriptsize{$ \pm 0.12 $} & 71.01 \scriptsize{$ \pm 0.23 $} & 61.56 \scriptsize{$ \pm 0.46 $} & 52.51 \scriptsize{$ \pm 1.08 $} & 35.47 \scriptsize{$ \pm 1.34 $} \\
    
    \textbf{TDDS} & 71.58 \scriptsize{$ \pm 0.50 $} & 71.45 \scriptsize{$ \pm 0.68 $} & 69.92 \scriptsize{$ \pm 0.25 $} & 65.12 \scriptsize{$ \pm 2.08 $} & 55.79 \scriptsize{$ \pm 2.16 $} & 53.85 \scriptsize{$ \pm 0.94 $} & \underline{40.51} \scriptsize{$ \pm 1.34 $} \\
    
    \textbf{CCS} & 72.58 \scriptsize{$ \pm 0.12 $} & 71.38 \scriptsize{$ \pm 0.35 $} & 69.83 \scriptsize{$ \pm 0.26 $} & 65.45 \scriptsize{$ \pm 0.23 $} & 56.65 \scriptsize{$ \pm 0.45 $} & 49.75 \scriptsize{$ \pm 0.90 $} & 34.63 \scriptsize{$ \pm 1.79 $} \\
    
    \midrule
    
    \textbf{DUAL} & \underline{73.96} \scriptsize{$ \pm 0.20 $} & \textbf{74.07} \scriptsize{$ \pm 0.43 $} & \underline{73.74} \scriptsize{$ \pm 0.18 $} & \textbf{71.23} \scriptsize{$ \pm 0.08$} & \underline{64.76} \scriptsize{$ \pm 0.32 $} & \underline{57.47} \scriptsize{$ \pm 0.51 $} & 37.93 \scriptsize{$ \pm 2.38 $} \\
    
    \textbf{DUAL+$\beta$} & 73.91 \scriptsize{$ \pm 0.17 $} & 73.80 \scriptsize{$ \pm 0.48 $} & 73.59 \scriptsize{$ \pm 0.19 $} & \underline{71.12} \scriptsize{$ \pm 0.29 $} & \textbf{65.18} \scriptsize{$ \pm 0.44 $} & \textbf{61.07} \scriptsize{$ \pm 0.47 $} & \textbf{52.61} \scriptsize{$ \pm 0.47 $} \\
    
    \bottomrule
\end{tabular}
\end{table}




\begin{table}[ht]
\caption{\label{tab:image_corruption_40_cifar}Comparison of test accuracy of DUAL pruning with existing coreset selection methods under 40\% image corrupted data using ResNet-18 for CIFAR-100. The model trained with the full dataset achieves \textbf{72.16\%} test accuracy on average. Results are averaged over five runs.}
\setlength{\tabcolsep}{3.1pt}
\centering
\begin{tabular}{lccccccc}
    \toprule
    \textbf{Pruning Rate} & \textbf{10\%} & \textbf{20\%} & \textbf{30\%} & \textbf{50\%} & \textbf{70\%} & \textbf{80\%} & \textbf{90\%} \\
    \midrule
    \textbf{Random} & 70.78 \scriptsize{$ \pm 0.25 $} & 69.30 \scriptsize{$ \pm 0.29 $} & 67.98 \scriptsize{$ \pm 0.26 $} & 63.23 \scriptsize{$ \pm 0.26 $} & 53.29 \scriptsize{$ \pm 0.64 $} & 45.76 \scriptsize{$ \pm 0.85 $} & 32.63 \scriptsize{$ \pm 0.61 $} \\
    
    \textbf{Entropy} & 70.74 \scriptsize{$ \pm 0.18 $} & 68.90 \scriptsize{$ \pm 0.37 $} & 66.19 \scriptsize{$ \pm 0.46 $} & 57.03 \scriptsize{$ \pm 0.60 $} & 35.62 \scriptsize{$ \pm 1.58 $} & 22.50 \scriptsize{$ \pm 1.03 $} & 7.46 \scriptsize{$ \pm 0.52 $} \\
    
    \textbf{Forgetting} & 70.54 \scriptsize{$ \pm 0.10 $} & 69.17 \scriptsize{$ \pm 0.30 $} & 67.41 \scriptsize{$ \pm 0.28 $} & 62.77 \scriptsize{$ \pm 0.15 $} & 52.89 \scriptsize{$ \pm 0.36 $} & 44.94 \scriptsize{$ \pm 0.66 $} & 30.48 \scriptsize{$ \pm 0.49 $} \\
    
    \textbf{EL2N} & 71.57 \scriptsize{$ \pm 0.28 $} & 69.24 \scriptsize{$ \pm 0.16 $} & 62.95 \scriptsize{$ \pm 0.52 $} & 28.33 \scriptsize{$ \pm 0.47 $} & 9.48 \scriptsize{$ \pm 0.21 $} & 8.86 \scriptsize{$ \pm 0.21 $} & 7.58 \scriptsize{$ \pm 0.16 $} \\
    
    \textbf{AUM} & 71.66 \scriptsize{$ \pm 0.23 $} & 69.75 \scriptsize{$ \pm 0.30 $} & 62.10 \scriptsize{$ \pm 0.46 $} & 26.56 \scriptsize{$ \pm 0.62 $} & 8.93 \scriptsize{$ \pm 0.19 $} & 5.82 \scriptsize{$ \pm 0.09 $} & 4.15 \scriptsize{$ \pm 0.11 $} \\
    
    \textbf{Moderate} & \textbf{72.10} \scriptsize{$ \pm 0.14 $} & 71.55 \scriptsize{$ \pm 0.25 $} & 69.84 \scriptsize{$ \pm 0.39 $} & 65.74 \scriptsize{$ \pm 0.21 $} & 56.96 \scriptsize{$ \pm 0.52 $} & 49.04 \scriptsize{$ \pm 0.74 $} & 34.87 \scriptsize{$ \pm 0.57 $} \\
    
    \textbf{Dyn-Unc} & 71.86 \scriptsize{$ \pm 0.12 $} & 71.65 \scriptsize{$ \pm 0.18 $} & \textbf{71.79} \scriptsize{$ \pm 0.27 $} & 69.17 \scriptsize{$ \pm 0.44 $} & 59.69 \scriptsize{$ \pm 0.30 $} & 51.36 \scriptsize{$ \pm 0.70 $} & 34.02 \scriptsize{$ \pm 0.45 $} \\
    
    \textbf{TDDS} & 70.02 \scriptsize{$ \pm 0.43 $} & 69.27 \scriptsize{$ \pm 0.74 $} & 68.03 \scriptsize{$ \pm 0.55 $} & 63.42 \scriptsize{$ \pm 0.77 $} & 55.28 \scriptsize{$ \pm 1.93 $} & 51.44 \scriptsize{$ \pm 1.36 $} & \underline{38.42} \scriptsize{$ \pm 0.80 $} \\
    
    \textbf{CCS} & 70.84 \scriptsize{$ \pm 0.41 $} & 69.08 \scriptsize{$ \pm 0.41 $} & 68.11 \scriptsize{$ \pm 0.09 $} & 63.36 \scriptsize{$ \pm 0.16 $} & 53.21 \scriptsize{$ \pm 0.54 $} & 46.27 \scriptsize{$ \pm 0.52 $} & 32.72 \scriptsize{$ \pm 0.52 $} \\
    
    \midrule
    
    \textbf{DUAL} & 71.90 \scriptsize{$ \pm 0.27 $} & \textbf{72.38} \scriptsize{$ \pm 0.27 $} & \textbf{71.79} \scriptsize{$ \pm 0.11 $} & \textbf{69.69} \scriptsize{$ \pm 0.18 $} & \underline{63.35} \scriptsize{$ \pm 0.29 $} & \underline{56.57} \scriptsize{$ \pm 1.07 $} & 37.78 \scriptsize{$ \pm 0.73 $} \\
    
    \textbf{DUAL+$\beta$} & \underline{71.96} \scriptsize{$ \pm 0.13 $} & \underline{71.92} \scriptsize{$ \pm 0.22 $} & \underline{71.69} \scriptsize{$ \pm 0.18 $} & \underline{69.23} \scriptsize{$ \pm 0.15 $} & \textbf{63.73} \scriptsize{$ \pm 0.43 $} & \textbf{59.75} \scriptsize{$ \pm 0.32 $} & \textbf{51.51} \scriptsize{$ \pm 0.68 $} \\
    
    \bottomrule
\end{tabular}
\end{table}






\begin{table}[ht]
\caption{\label{tab:image_corruption_20_tinyimagenet}Comparison of test accuracy of DUAL pruning with existing coreset selection methods under 20\% image corrupted data using ResNet-34 for Tiny-ImageNet. The model trained with the full dataset achieves \textbf{57.12\%} test accuracy on average. Results are averaged over three runs.}
\setlength{\tabcolsep}{3.1pt}
\centering
\begin{tabular}{lccccccc}
    \toprule
    \textbf{Pruning Rate} & \textbf{10\%} & \textbf{20\%} & \textbf{30\%} & \textbf{50\%} & \textbf{70\%} & \textbf{80\%} & \textbf{90\%} \\
    \midrule
    \textbf{Random} & 49.59 \scriptsize{$ \pm 0.93 $} & 48.64 \scriptsize{$ \pm 0.94 $} & 45.64 \scriptsize{$ \pm 0.53 $} & 41.58 \scriptsize{$ \pm 0.66 $} & 33.98 \scriptsize{$ \pm 0.55 $} & 28.88 \scriptsize{$ \pm 0.67 $} & 18.59 \scriptsize{$ \pm 0.25 $} \\
    
    \textbf{Entropy} & 50.34 \scriptsize{$ \pm 0.19 $} & 48.02 \scriptsize{$ \pm 0.49 $} & 44.80 \scriptsize{$ \pm 0.30 $} & 36.58 \scriptsize{$ \pm 0.19 $} & 25.20 \scriptsize{$ \pm 0.53 $} & 16.55 \scriptsize{$ \pm 0.40 $} & 3.32 \scriptsize{$ \pm 0.26 $} \\
    
    \textbf{Forgetting} & 46.81 \scriptsize{$ \pm 0.26 $} & 41.16 \scriptsize{$ \pm 0.28 $} & 35.58 \scriptsize{$ \pm 0.17 $} & 26.80 \scriptsize{$ \pm 0.18 $} & 17.66 \scriptsize{$ \pm 0.23 $} & 12.61 \scriptsize{$ \pm 0.04 $} & 6.01 \scriptsize{$ \pm 0.19 $} \\
    
    \textbf{EL2N} & 50.66 \scriptsize{$ \pm 0.27 $} & 47.76 \scriptsize{$ \pm 0.25 $} & 42.15 \scriptsize{$ \pm 1.02 $} & 23.42 \scriptsize{$ \pm 0.26 $} & 8.07 \scriptsize{$ \pm 0.09 $} & 6.57 \scriptsize{$ \pm 0.36 $} & 3.75 \scriptsize{$ \pm 0.13 $} \\
    
    \textbf{AUM} &  51.11 \scriptsize{$ \pm 0.73 $} & 47.70 \scriptsize{$ \pm 0.51 $} & 42.04 \scriptsize{$ \pm 0.81 $} & 20.85 \scriptsize{$ \pm 0.79 $} & 6.87 \scriptsize{$ \pm 0.24 $} & 3.75 \scriptsize{$ \pm 0.21 $} & 2.27 \scriptsize{$ \pm 0.11 $} \\
    
    \textbf{Moderate} & 51.43 \scriptsize{$ \pm 0.76 $} & 49.85 \scriptsize{$ \pm 0.23 $} & 47.85 \scriptsize{$ \pm 0.31 $} & 42.31 \scriptsize{$ \pm 0.40 $} & 35.00 \scriptsize{$ \pm 0.49 $} & 29.63 \scriptsize{$ \pm 0.67 $} & 19.51 \scriptsize{$ \pm 0.72 $} \\
    
    \textbf{Dyn-Unc} & 51.61 \scriptsize{$ \pm 0.19 $} & 51.47 \scriptsize{$ \pm 0.34 $} & \textbf{51.18} \scriptsize{$ \pm 0.58 $} & \textbf{48.88} \scriptsize{$ \pm 0.85 $} & \underline{42.52} \scriptsize{$ \pm 0.34 $} & \underline{37.85} \scriptsize{$ \pm 0.47 $} & \underline{26.26} \scriptsize{$ \pm 0.70 $} \\
    
    \textbf{TDDS} & \underline{51.53} \scriptsize{$ \pm 0.40 $} & 49.81 \scriptsize{$ \pm 0.21 $} & 48.98 \scriptsize{$ \pm 0.27 $} & 45.81 \scriptsize{$ \pm 0.16 $} & 38.05 \scriptsize{$ \pm 0.70 $} & 33.04 \scriptsize{$ \pm 0.39 $} & 22.66 \scriptsize{$ \pm 1.28 $} \\
    
    \textbf{CCS} &  50.26 \scriptsize{$ \pm 0.78 $} & 48.00 \scriptsize{$ \pm 0.41 $} & 45.38 \scriptsize{$ \pm 0.63 $} & 40.98 \scriptsize{$ \pm 0.23 $} & 33.49 \scriptsize{$ \pm 0.04 $} & 27.18 \scriptsize{$ \pm 0.66 $} & 15.37 \scriptsize{$ \pm 0.54 $} \\
    
    \midrule
    
    \textbf{DUAL} &  51.22 \scriptsize{$ \pm 0.40 $} & \textbf{52.06} \scriptsize{$ \pm 0.55 $} & \underline{50.88} \scriptsize{$ \pm 0.64 $} & \underline{47.03} \scriptsize{$\pm 0.56 $} & 40.03 \scriptsize{$ \pm 0.09$} & 34.92 \scriptsize{$ \pm 0.15$} &20.41 \scriptsize{$ \pm 1.07$} \\
    
    \textbf{DUAL+$\beta$} & \textbf{52.15} \scriptsize{$ \pm 0.25$} & \underline{51.11} \scriptsize{$ \pm 0.34$} & 50.21 \scriptsize{$ \pm 0.36$} & 46.85 \scriptsize{$ \pm 0.27$} & \textbf{42.97} \scriptsize{$ \pm 0.28$} & \textbf{38.30} \scriptsize{$ \pm $0.06 } & \textbf{27.45} \scriptsize{$ \pm 0.50$} \\
    
    \bottomrule
\end{tabular}
\end{table}


\clearpage

\subsection{Cross-architecture generalization}
\label{Appendix_cross_architecture}
In this section, we investigate the cross-architecture generalization ability of our proposed method. Specifically, we calculate the example score on one architecture and test its coreset performance on a different architecture. This evaluation aims to assess the ability of example scores to be transferred across diverse architectural designs. 

\begin{table}[ht]
\caption{Cross-architecture generalization performance on CIFAR-100 from three layer CNN to ResNet-18. We report an average of five runs. `R18 $\rightarrow$ R18' stands for score computation on ResNet-18, as a baseline.}
\label{tab:cross-arch-cnn}
    \centering
    \begin{tabular}{lcccc}
    \toprule
    \multicolumn{4}{c}{}{3-layer CNN $\rightarrow$ ResNet-18} & \\
    \hline 
    Pruning Rate ($\rightarrow$) & 30\% & 50\% & 70\% & 90\%  \\
    \hline
    Random & 75.15 \scriptsize{$\pm 0.28$} & 71.68 \scriptsize{$\pm 0.31 $} & 64.86 \scriptsize{$\pm 0.39$} & 45.09 \scriptsize{$\pm 1.26 $} \\
    EL2N & 76.56 \scriptsize{$\pm 0.65$} & 71.78 \scriptsize{$\pm 0.32$} & 56.57 \scriptsize{$\pm 1.32$} & 22.84 \scriptsize{$\pm 3.54 $} \\
    Dyn-Unc & \textbf{76.61} \scriptsize{$\pm 0.75$} & 72.92 \scriptsize{$\pm 0.57 $} &65.97 \scriptsize{$\pm 0.53 $} & 44.25 \scriptsize{$\pm 2.47$} \\
    CCS & 75.29 \scriptsize{$\pm 0.20 $} & 72.06 \scriptsize{$\pm 0.19$} & \textbf{66.11} \scriptsize{$\pm 0.15$} & 36.98 \scriptsize{$\pm 1.47$} \\
    \hline
    DUAL & \textbf{76.61} \scriptsize{$\pm 0.08$} & \textbf{73.55} \scriptsize{$\pm 0.12$} & 65.97 \scriptsize{$\pm 0.18$} & 39.00 \scriptsize{$\pm 2.51 $} \\
    DUAL+$\beta$ sampling & 76.36 \scriptsize{$\pm 0.18 $} & 72.46 \scriptsize{$\pm 0.41 $} & 65.50 \scriptsize{$\pm 0.53$} & \textbf{48.91} \scriptsize{$\pm 0.60 $} \\
    \hline
    \hline
    DUAL (R18$\rightarrow$R18) & 77.43 \scriptsize{$\pm 0.18$} &  74.62 \scriptsize{$\pm 0.47 $} & 66.41 \scriptsize{$\pm 0.52 $} & 34.38 \scriptsize{$\pm 1.39 $} \\
    DUAL (R18$\rightarrow$R18) +$\beta$ sampling & 77.86 \scriptsize{$\pm 0.12$} & 74.66  \scriptsize{$\pm 0.12 $} & 69.25 \scriptsize{$\pm 0.22$} & 54.54 \scriptsize{$\pm 0.09$} \\
    \bottomrule
    \end{tabular}
    \label{tab:cnn-to-resnet18}
\end{table}
    
    
\begin{table}[ht]
\caption{Cross-architecture generalization performance on CIFAR-100 from three layer CNN to VGG-16. We report an average of five runs. `V16 $\rightarrow$ V16' stands for score computation on VGG-16, as a baseline.}
    \centering
    \begin{tabular}{lcccc}
    \toprule
    \multicolumn{4}{c}{}{3-layer CNN $\rightarrow$ VGG-16} & \\
    \hline 
    Pruning Rate ($\rightarrow$) & 30\% & 50\% & 70\% & 90\%  \\
    \hline
    Random & 69.47 \scriptsize{$\pm 0.27$} & 65.52 \scriptsize{$\pm 0.54 $} & 57.18 \scriptsize{$\pm 0.68 $} & 34.69 \scriptsize{$\pm 1.97 $} \\
    EL2N & 70.35 \scriptsize{$\pm 0.64$} & 63.66 \scriptsize{$\pm 1.49 $} & 46.12 \scriptsize{$\pm 6.87 $} & 20.85 \scriptsize{$\pm 9.03 $} \\
    Dyn-Unc & 71.18 \scriptsize{$\pm 0.96$} & 67.06 \scriptsize{$\pm 0.94$} & 58.87 \scriptsize{$\pm 0.83$} & 31.57 \scriptsize{$\pm 3.29 $} \\
    CCS & 69.56 \scriptsize{$\pm 0.33$} & 65.26 \scriptsize{$\pm 0.50 $} & 57.60 \scriptsize{$\pm 0.80 $} & 23.92 \scriptsize{$\pm 1.85 $} \\
    \hline
    DUAL & \textbf{71.75 }\scriptsize{$\pm 0.16$} & \textbf{67.91} \scriptsize{$\pm 0.27$} & 59.08 \scriptsize{$\pm 0.64$} & 29.16 \scriptsize{$\pm 2.28 $} \\
    DUAL+$\beta$ sampling &  70.78 \scriptsize{$\pm 0.41 $} & 67.47  \scriptsize{$\pm 0.44 $} & \textbf{60.33} \scriptsize{$\pm 0.32 $} & \textbf{43.92} \scriptsize{$\pm 1.15 $} \\
    \hline
    \hline
    DUAL (V16$\rightarrow$V16) & 73.63 \scriptsize{$\pm 0.62$} & 69.66 \scriptsize{$\pm 0.45$} & 58.49 \scriptsize{$\pm 0.77$} & 32.96 \scriptsize{$\pm 1.12 $} \\
    DUAL (V16$\rightarrow$V16) +$\beta$ sampling & 72.77 \scriptsize{$\pm 0.41$} & 68.93 \scriptsize{$\pm 0.23$} & 61.48 \scriptsize{$\pm 0.36$} & 42.99\scriptsize{$\pm 0.62 $} \\
    \bottomrule
    \end{tabular}
    \label{tab:cnn-to-vgg16}
\end{table}


\begin{table}[ht]
\caption{Cross-architecture generalization performance on CIFAR-100 from VGG-16 to ResNet-18. We report an average of five runs. `R18 $\rightarrow$ R18' stands for score computation on ResNet-18, as a baseline.}
\label{tab:cross-arch-v19-r18}
\setlength{\tabcolsep}{3.1pt}
\centering
\begin{tabular}{lcccc}
    \toprule
    \multicolumn{1}{c}{} & \multicolumn{4}{c}{VGG-16 $\rightarrow$ ResNet-18} \\
    \hline
    Pruning Rate ($\rightarrow$) & 30\% & 50\% & 70\% & 90\% \\
    \hline
    Random & 75.15 \scriptsize{$\pm 0.28$} & 71.68 \scriptsize{$\pm 0.31 $} & 64.86 \scriptsize{$\pm 0.39$} & 45.09 \scriptsize{$\pm 1.26 $} \\
    EL2N & 76.42 \scriptsize{$\pm 0.27$} & 70.44 \scriptsize{$\pm 0.48 $} & 51.87 \scriptsize{$\pm 1.27 $} & 25.74 \scriptsize{$\pm 1.53 $} \\
    Dyn-Unc & \textbf{77.59}  \scriptsize{$\pm 0.19$} & 74.20 \scriptsize{$\pm 0.22 $} & 65.24 \scriptsize{$\pm 0.36 $} & 42.95 \scriptsize{$\pm 1.14$} \\
    CCS & 75.19 \scriptsize{$\pm 0.19$} & 71.56 \scriptsize{$\pm 0.28$} & 64.83 \scriptsize{$\pm 0.25$} & \textbf{46.08} \scriptsize{$\pm 1.23$} \\
    \hline 
    DUAL & 77.40 \scriptsize{$\pm 0.36$} & \textbf{74.29} \scriptsize{$\pm 0.12$} & 63.74 \scriptsize{$\pm 0.30$} & 36.87 \scriptsize{$\pm 2.27$} \\
    DUAL+ $\beta$ sampling & 76.67 \scriptsize{$\pm 0.15 $} & 73.14 \scriptsize{$\pm 0.29 $} & \textbf{65.69} \scriptsize{$\pm 0.57 $} & 45.95 \scriptsize{$\pm 0.52 $}  \\
    \hline
    \hline
    DUAL (R18$\rightarrow$R18) & 77.43 \scriptsize{$\pm 0.18$} & 74.62 \scriptsize{$\pm 0.47$} & 66.41 \scriptsize{$\pm 0.52 $} & 34.38 \scriptsize{$\pm 1.39 $} \\
    DUAL (R18$\rightarrow$R18) +$\beta$ sampling & 77.86 \scriptsize{$\pm 0.12$} & 74.66 \scriptsize{$\pm 0.12$} & 69.25 \scriptsize{$\pm 0.22$} & 54.54 \scriptsize{$\pm 0.09$} \\
    \bottomrule
\end{tabular}
\end{table}

\begin{table}[ht]
\caption{Cross-architecture generalization performance on CIFAR-100 from ResNet-18 to VGG-16. We report an average of five runs. `V16 $\rightarrow$ V16' stands for score computation on VGG-16, as a baseline.}
\centering
\begin{tabular}{lcccc}
    \toprule
    \multicolumn{1}{c}{} & \multicolumn{4}{c}{ResNet-18 $\rightarrow$ VGG-16} \\
    \hline
    Pruning Rate ($\rightarrow$) & 30\% & 50\% & 70\% & 90\% \\
    \hline
    Random & 70.99 \scriptsize{$\pm 0.33$} & 67.34 \scriptsize{$\pm 0.21$} & 60.18 \scriptsize{$\pm 0.52$} & 41.69 \scriptsize{$\pm 0.72 $} \\
    EL2N  & 72.43 \scriptsize{$\pm 0.54$} & 65.36 \scriptsize{$\pm 0.68$} & 43.35 \scriptsize{$\pm 0.81$} & 19.92 \scriptsize{$\pm 0.89 $} \\
    Dyn-Unc  & 73.34 \scriptsize{$\pm 0.29$} & 69.24 \scriptsize{$\pm 0.39$} & 57.67 \scriptsize{$\pm 0.52$} & 31.74 \scriptsize{$\pm 0.80 $} \\
    CCS  & 71.18 \scriptsize{$\pm 0.16$} & 67.35 \scriptsize{$\pm 0.38$} & 59.77 \scriptsize{$\pm 0.43$} & 41.06 \scriptsize{$\pm 1.03 $} \\
    \hline 
    DUAL & 73.44 \scriptsize{$\pm 0.29$} & 69.87 \scriptsize{$\pm 0.35 $} & 60.07 \scriptsize{$\pm 0.47 $} & 29.74 \scriptsize{$\pm 1.70 $} \\
    DUAL +$\beta$ sampling  &\textbf{73.50} \scriptsize{$\pm 0.27$} & \textbf{70.43} \scriptsize{$\pm 0.26$} & \textbf{64.48} \scriptsize{$\pm 0.47$} & \textbf{49.61} \scriptsize{$\pm 0.49 $} \\
    \hline
    \hline
    DUAL (V16$\rightarrow$V16)  & 73.63 \scriptsize{$\pm 0.61$} & 69.66 \scriptsize{$\pm 0.45$} & 58.49 \scriptsize{$\pm 0.77$} & 32.96 \scriptsize{$\pm 1.12 $} \\
    DUAL (V16$\rightarrow$V16)+$\beta$ sampling & 72.66 \scriptsize{$\pm 0.17 $} & 68.80 \scriptsize{$\pm 0.34 $} & 60.40 \scriptsize{$\pm 0.68 $} & 41.51 \scriptsize{$\pm 0.47 $} \\
    \bottomrule
\end{tabular}
\end{table}


\begin{table}[ht]
\caption{Cross-architecture generalization performance on CIFAR-100 from ResNet-18 to ResNet-50. We report an average of five runs. `R50 $\rightarrow$ R50' stands for score computation on ResNet-50, as a baseline.}
\label{tab:cross-arch-r18-r50}
\setlength{\tabcolsep}{3.1pt}
\centering
\begin{tabular}{lcccc}
    \toprule
    \multicolumn{1}{c}{} & \multicolumn{4}{c}{ResNet-18 $\rightarrow$ ResNet-50} \\
    \hline
    Pruning Rate ($\rightarrow$) & 30\% & 50\% & 70\% & 90\% \\
    \hline
    \hline
    Random & 74.47 \scriptsize{$\pm 0.67$} & 70.09 \scriptsize{$\pm 0.42$} & 60.06 \scriptsize{$\pm 0.99$} & 41.91 \scriptsize{$\pm 4.32 $} \\
    EL2N  & 76.42 \scriptsize{$\pm 1.00$} & 69.14 \scriptsize{$\pm 1.00$} & 45.16 \scriptsize{$\pm 3.21$} & 19.63 \scriptsize{$\pm 1.15 $} \\
    Dyn-Unc  & 77.31 \scriptsize{$\pm 0.34$} & 72.12 \scriptsize{$\pm 0.68$} & 59.38 \scriptsize{$\pm 2.35$} & 31.74 \scriptsize{$\pm 2.31 $} \\
    CCS  & 74.78 \scriptsize{$\pm 0.66$} & 69.98 \scriptsize{$\pm 1.18$} & 59.75 \scriptsize{$\pm 1.41$} & 41.54 \scriptsize{$\pm 3.94 $} \\
    \hline 
    DUAL   & \textbf{78.03} \scriptsize{$\pm 0.83$} & 72.82 \scriptsize{$\pm 1.46$} & 63.08 \scriptsize{$\pm 2.45$} & 33.65 \scriptsize{$\pm 2.92 $} \\
    DUAL +$\beta$ sampling  & 77.82 \scriptsize{$\pm 0.65$} & \textbf{73.98} \scriptsize{$\pm 0.62$} & \textbf{66.36} \scriptsize{$\pm 1.66$} & \textbf{49.90} \scriptsize{$\pm 2.56 $} \\
    \hline
    \hline
    DUAL (R50$\rightarrow$R50)  & 77.82 \scriptsize{$\pm 0.64$} & 73.66 \scriptsize{$\pm 0.85$} & 52.12 \scriptsize{$\pm 2.73 $} & 26.13 \scriptsize{$\pm 1.96 $} \\
    DUAL (R50$\rightarrow$R50)+$\beta$ sampling  & 77.57 \scriptsize{$\pm 0.23$} & 73.44 \scriptsize{$\pm 0.87$} & 65.17 \scriptsize{$\pm 0.96$} & 47.63 \scriptsize{$\pm 2.47 $} \\
    \bottomrule
\end{tabular}
\end{table}

\begin{table}[ht]
\centering
\caption{Cross-architecture generalization performance on CIFAR-100 from VGG-16 to ResNet-50. We report an average of five runs. `R50 $\rightarrow$ R50' stands for score computation on ResNet-50, as a baseline}
\begin{tabular}{lcccc}
    \toprule
    \multicolumn{1}{c}{} & \multicolumn{4}{c}{VGG-16 $\rightarrow$ ResNet-50} \\
    \hline
    Pruning Rate ($\rightarrow$) & 30\% & 50\% & 70\% & 90\% \\
    \hline
    \hline
    Random & 71.13 \scriptsize{$\pm 6.52$} & 70.31 \scriptsize{$\pm 1.20$} & 61.02 \scriptsize{$\pm 1.68$} & 41.03 \scriptsize{$\pm 3.74 $} \\
    EL2N  & 76.30 \scriptsize{$\pm 0.69$} & 67.11 \scriptsize{$\pm 3.09$} & 44.88 \scriptsize{$\pm 3.65$} & 25.05 \scriptsize{$\pm 1.76 $} \\
    Dyn-Unc  & \textbf{77.91} \scriptsize{$\pm 0.54$} & \textbf{73.52} \scriptsize{$\pm 0.41$} & 62.37 \scriptsize{$\pm 0.62$} & 39.10 \scriptsize{$\pm 4.04 $} \\
    CCS  & 75.40 \scriptsize{$\pm 0.64$} & 70.44 \scriptsize{$\pm 0.49$} & 60.10 \scriptsize{$\pm 1.24$} & 41.94 \scriptsize{$\pm 3.01 $} \\
    \hline 
    DUAL   & 77.50 \scriptsize{$\pm 0.53 $} & 71.81 \scriptsize{$\pm 0.48$} & 60.68 \scriptsize{$\pm 1.67$} & 34.88 \scriptsize{$\pm 3.47 $} \\
    DUAL +$\beta$ sampling  & 76.67 \scriptsize{$\pm 0.15 $} & 73.14 \scriptsize{$\pm 0.29 $} & \textbf{65.69} \scriptsize{$\pm 0.57 $} &\textbf{45.95} \scriptsize{$\pm 0.52 $} \\
    \hline
    \hline
    DUAL (R50$\rightarrow$R50)  & 77.82 \scriptsize{$\pm 0.64$} & 73.66 \scriptsize{$\pm 0.85$} & 52.12 \scriptsize{$\pm 2.73 $} & 26.13 \scriptsize{$\pm 1.96 $} \\
    DUAL (R50$\rightarrow$R50)+$\beta$ sampling  & 77.57 \scriptsize{$\pm 0.23$} & 73.44 \scriptsize{$\pm 0.87$} & 65.17 \scriptsize{$\pm 0.96$} & 47.63 \scriptsize{$\pm 2.47 $} \\
    \bottomrule
\end{tabular}
\end{table}

\clearpage

\subsection{Effectiveness of Beta Sampling}
\label{Appendix_beta_samapling}
We study the impact of our Beta sampling on existing score metrics. We apply our Beta sampling strategy to forgetting, EL2N, and Dyn-Unc scores of CIFAR10 and 100. By comparing Beta sampling with the vanilla threshold pruning using scores, we observe that prior score-based methods become competitive, outperforming random pruning when Beta sampling is adjusted.

\begin{table}[ht]
\caption{Comparison on CIFAR-10 and CIFAR-100 for $90\%$ pruning rate. 
We report average accuracy with five runs. The best performance is in bold in each column.}
\label{tab:abl_beta_cifar10_100_90}
\setlength{\tabcolsep}{3.1pt}
\centering
\begin{tabular}{lcc|cc}
    \toprule
    \multicolumn{1}{c}{} & \multicolumn{2}{c|}{CIFAR-10} & \multicolumn{2}{c}{CIFAR-100}\\
    \midrule
    Method & Thresholding & $\beta$-Sampling & Thresholding & $\beta$-Sampling \\
    \midrule
    Random &  \textbf{83.74} \scriptsize{$\pm$ 0.21} & 83.31 (-0.43) \scriptsize{$\pm$ 0.14} & 45.09 \scriptsize{$\pm$ 1.26} & 51.76 (+6.67) \scriptsize{$\pm$ 0.25} \\
    EL2N &  38.74 \scriptsize{$\pm$ 0.75} & 87.00 (+48.26) \scriptsize{$\pm$ 0.45} & 8.89 \scriptsize{$\pm$ 0.28} & 53.97 (+45.08)  \scriptsize{$\pm$ 0.63}  \\
    Forgetting &  46.64 \scriptsize{$\pm$ 1.90} & 85.67 (+39.03) \scriptsize{$\pm$0.13} & 26.87 \scriptsize{$\pm$ 0.73} & 52.40 (+25.53) \scriptsize{$\pm$ 0.43} \\
    Dyn-Unc &  59.67 \scriptsize{$\pm$ 1.79} & 85.33 (+32.14) \scriptsize{$\pm$ 0.20} & 34.57 \scriptsize{$\pm$ 0.69} & 51.85 (+17.28) \scriptsize{$\pm$ 0.35}   \\
    \hline
    DUAL & 54.95 \scriptsize{$\pm$ 0.42} & \textbf{87.09} (+31.51) \scriptsize{$\pm$ 0.36} & 34.28 \scriptsize{$\pm$ 1.39}    & \textbf{54.54} (+20.26) \scriptsize{$\pm$ 0.09}  \\
    \bottomrule
\end{tabular}
\end{table}


\begin{table}[ht]
\caption{Comparison on CIFAR-10 and CIFAR-100 for $80\%$ pruning rate. 
We report average accuracy with five runs. The best performance is in bold in each column.}
\label{tab:abl_beta_cifar10_100_80}
\setlength{\tabcolsep}{3.1pt}
\centering
\begin{tabular}{lcc|cc}
    \toprule
    \multicolumn{1}{c}{} & \multicolumn{2}{c|}{CIFAR-10} & \multicolumn{2}{c}{CIFAR-100}\\
    \midrule
    Method & Thresholding & $\beta$-Sampling & Thresholding & $\beta$-Sampling \\
    \midrule
    Random &  \textbf{88.28} \scriptsize{$\pm$ 0.17} & 88.83 (+0.55) \scriptsize{$\pm$ 0.18} & \textbf{59.23} \scriptsize{$\pm$ 0.62} & 61.74 (+2.51) \scriptsize{$\pm$ 0.15} \\
    EL2N &  74.70 \scriptsize{$\pm$ 0.45} & 87.69 (+12.99) \scriptsize{$\pm$ 0.98} &19.52 \scriptsize{$\pm$ 0.79} & 63.98 (+44.46) \scriptsize{$\pm$ 0.73}  \\
    Forgetting &  75.47 \scriptsize{$\pm$ 1.27} & 90.86 (+15.39) \scriptsize{$\pm$ 0.07} & 39.09 \scriptsize{$\pm$ 0.41} & 63.29 (+24.20) \scriptsize{$\pm$ 0.13} \\
    Dyn-Unc &  83.32 \scriptsize{$\pm$ 0.94} & 90.80 (+7.48) \scriptsize{$\pm$ 0.30} & 55.01 \scriptsize{$\pm$ 0.55} & 62.31 (+7.30) \scriptsize{$\pm$ 0.23}  \\
    \hline
    DUAL & 82.02 \scriptsize{$\pm$ 1.85} & \textbf{91.42} (+9.68) \scriptsize{$\pm$ 0.35} & 56.57 \scriptsize{$\pm$ 0.57}    & \textbf{64.76} (+8.46) \scriptsize{$\pm$ 0.23} \\
    \bottomrule
\end{tabular}
\end{table}

We also study the impact of our pruning strategy with DUAL score combined with Beta sampling. We compare different sampling strategies $i.e.$ vanilla thresholding, stratified sampling \citep{zheng2022coverage}, and our proposed Beta sampling on CIFAR10 and 100, at 80\% and 90\% pruning rates. We observe that our proposed method mostly performs the best, especially with the high pruning ratio. 

\begin{table}[ht]
\centering
\caption{Comparison on Sampling Strategy}
\begin{tabular}{lccccc}
\toprule
\multicolumn{6}{c}{CIFAR10} \\ 
\cmidrule(lr){1-6}
Pruning Rate & $30\%$ & $50\%$ & $70\%$ & $80\%$ & $90\%$ \\
\midrule
DUAL & 95.35 & 95.08 & 91.95 & 81.74 & 55.58 \\
DUAL + CCS & \textbf{95.54} & 95.00 & 92.83 & 90.49 & 81.67 \\
DUAL + $\beta$ & 95.51 & \textbf{95.23} & \textbf{93.04} & \textbf{91.42} & \textbf{87.09} \\

\midrule 
\multicolumn{6}{c}{CIFAR100} \\ 
\cmidrule(lr){1-6}
Pruning Rate & $30\%$ & $50\%$ & $70\%$ & $80\%$ & $90\%$ \\
\midrule
DUAL & 77.61 & \textbf{74.86} & 66.39 & 56.50 & 34.28 \\
DUAL + CCS & 75.21 & 71.53 & 64.30 & 59.09 & 45.21 \\
DUAL + $\beta$ & \textbf{77.86} & 74.66 & \textbf{69.25} & \textbf{64.76} & \textbf{54.54} \\
\bottomrule
\end{tabular}
\label{tab:abl_ours_ccs}
\end{table}


\clearpage
\section{Detailed Explanation about Our Method}
\label{Appendix_explanation_of_dual_pruning}

In this section, we provide details on the implementation used across all experiments for reproducibility. Appendix~\ref{Appendix_algorithm} presents the full algorithm for our pruning method, DUAL, along with the Beta sampling strategy. Additionally, in a later subsection, we visualize the selected data using Beta sampling.

Recall that we define our sampling distribution $\mathrm{Beta}(\alpha_r, \beta_r)$ as follows:
\begin{align}
\label{eq:alpha_beta_detail}
\begin{split}
    \beta_r &= C\left(1-\mu_\gD\right)\left(1-r^{c_\gD}\right)\\
    \alpha_r &= C-\beta_r,
\end{split}
\end{align}
where $\mu_\gD\in[0, 1]$ is the probability mean of the highest DUAL score training sample. To ensure stability, we compute this as the average probability mean of the 10 highest DUAL score training samples. Additionally, as mentioned earlier, we set the value of $C$ to 15 across all experiments. For technical details, we add 1 to $\alpha_r$ to further ensure that the PDF remains stationary at low pruning ratios.

We illustrate the Beta PDF, as defined above, in Figure~\ref{fig:beta_pdf} for different values of $c_\gD$. In both subplots, we set $\mu_\gD$ as 0.25. The left subplot shows the PDF with $c_\gD=5.5$, which corresponds to the value used in CIFAR-10 experiments, while the right subplot visualizes the PDF where $c_\gD=4$, corresponding to CIFAR-100.
\begin{figure}[ht]
    \centering
    \includegraphics[width=0.7\linewidth]{Figures/beta_pdf.pdf}
    \caption{Visualization of Beta distribution for varying $c_\gD$. The left subplot corresponds to the value used in CIFAR-10, and the right subplot corresponds to the value used in CIFAR-100.}
    \label{fig:beta_pdf}
\end{figure}

\clearpage
\subsection{Visualization of Selected Data with Beta Sampling}
\label{Appendix_coreset_visualization}
Here we illustrate the sampling probability of being selected into coreset, selected samples, and pruned samples in each figure when using the DUAL score combined with Beta sampling. As the pruning ratio increases, we focus on including easier samples.

\begin{figure}[htbp] 
    \centering
    \begin{subfigure}{0.6\textwidth}
        \centering
        \includegraphics[width=\textwidth]{Figures/cifar100_beta_sample_pr0.3.png}
        \caption{CIFAR-100 at pruning ratio 30\%}
        \label{fig:cifar_beta_pr30}
    \end{subfigure}

    \begin{subfigure}{0.6\textwidth}
        \centering
        \includegraphics[width=\textwidth]{Figures/cifar100_beta_sample_pr0.5.png} 
        \caption{CIFAR-100 at pruning ratio 50\%}
        \label{fig:cifar_beta_pr50}
    \end{subfigure}
    
    \begin{subfigure}{0.6\textwidth}
        \centering
        \includegraphics[width=\textwidth]{Figures/cifar100_beta_sample_pr0.7.png} 
        \caption{CIFAR-100 at pruning ratio 70\%}
        \label{fig:cifar_beta_pr70}
    \end{subfigure}

    \begin{subfigure}{0.6\textwidth}
        \centering
        \includegraphics[width=\textwidth]{Figures/cifar100_beta_sample_pr0.9.png} 
        \caption{CIFAR-100 at pruning ratio 90\%}
        \label{fig:cifar_beta_pr90}
    \end{subfigure}

    \caption{Pruning visualization on CIFAR-100.}
    \label{fig:cifar_coreset_visualization_beta}
\end{figure}
\clearpage

\subsection{Algorithm of Proposed Pruning Method}
\label{Appendix_algorithm}
The detailed algorithms for DUAL pruning and Beta sampling are as follows:
\begin{algorithm}[htb]
\begin{algorithmic}
    \caption{DUAL pruning + $\beta$-sampling}
    \label{alg:DUAL}

    \INPUT Training dataset $\gD$, pruning ratio $r$, dataset simplicity $c_\gD$, training epoch $T$, window length $J$.
    
    \OUTPUT Subset $\gS\subset\gD$ such that $\lvert\gS\rvert = (1-r)\lvert\gD\rvert$
    
    \FOR{$(\vx_i, y_i) \in \gD$}
        \FOR{$k = 1, \cdots, T-J+1$}
            \STATE $\Bar{\mathbb{P}}_k(\vx_i, y_i) \leftarrow \frac{1}{J}\sum_{j=0}^{J-1} \mathbb{P}_{k+j}(y_i\mid \vx_i)$
            
            \STATE $\mathbb{U}_k(\vx_i, y_i) \leftarrow \sqrt{ \frac{1}{J-1} \sum_{j=0}^{J-1} \left[ \mathbb{P}_{k+j}(y_i\mid \vx_i) - \Bar{\mathbb{P}}_k(\vx_i, y_i) \right] ^2}$
            
            \STATE $\mathrm{DUAL}_k(\vx_i, y_i) \leftarrow (1-\Bar{\mathbb{P}}_k(\vx_i, y_i)) \times \mathbb{U}_k(\vx_i, y_i)$
        
        \ENDFOR
        
        \STATE $\mathrm{DUAL}(\vx_i, y_i) \leftarrow \frac{1}{T-J+1}\sum_{k=1}^{T-J+1} \mathrm{DUAL}_k(\vx_i, y_i)$
        
    \ENDFOR
    
    \IF{$\beta$-sampling}
    \FOR{$(\vx_i, y_i) \in \gD$}
        \STATE $\bar{\mathbb{P}}(\vx_i, y_i) \leftarrow \frac{1}{T}\sum_{k=1}^T \mathbb{P}_k(y_i \mid \vx_i)$
        
        \STATE $\varphi \left(\bar{\mathbb{P}}(\vx_i, y_i)\right) \leftarrow$ PDF value of $\mathrm{Beta}(\alpha_r, \beta_r)$ from \cref{eq:alpha_beta}
        
        \STATE $\Tilde{\varphi} (\vx_i) \leftarrow $ $\varphi\left(\Bar{\mathbb{P}} (\vx_i, y_i)\right) \times \mathrm{DUAL}(\vx_i, y_i)$
    
    \ENDFOR
    
    \STATE $\Tilde{\varphi}(\vx_i) \leftarrow \frac{\Tilde{\varphi}(\vx_i)}{\sum_{j \in \gD} \Tilde{\varphi}(\vx_j)}$
    
    \STATE $\gS \leftarrow$ Sample $(1-r)\lvert \gD \rvert$ data points according to $\Tilde{\varphi}(\vx_i)$
    
    \ELSE
    
    \STATE $\gS \leftarrow$ Sample $(1-r)\lvert \gD \rvert$ data points with the largest $\mathrm{DUAL}(\vx_i, y_i)$ score
    
    \ENDIF 
\end{algorithmic}
\end{algorithm}
\clearpage
\section{Theoretical Results}
\label{sec:DUAL_theorem}
Throughout this section, we will rigorously prove Theorem~\ref{thm:main_shorter_time}, providing the intuition that Dyn-Unc takes longer than our method to select informative samples.

\subsection{Proof of \texorpdfstring{\cref{thm:main_shorter_time}}{the theorem}}

Assume that the input and output (or label) space are $\gX = \sR^n$ and $\gY = \{\pm1\}$, respectively. Let the model $f: \gX \to \sR$ be of the form $f(\vx; \vw) = \vw^\top \vx$ parameterized by $\vw\in\sR^n$ with zero-initialization. Let the loss be the exponential loss, $\ell(z) = e^{-z}$. Exponential loss is reported to induce implicit bias similar to logistic loss in binary classification tasks using linearly separable datasets \citep{soudry2018implicit, gunasekar2018implicit}.


The task of the model is to learn a binary classification. The dataset $\gD$ consists only two points, i.e. $\gD = \left\{ \left(\vx_1, y_1^*\right) , \left(\vx_2, y_2^*\right) \right\}$, where without loss of generality $y_i^* = 1$ for $i=1, 2$.
% It is trivial that $\gD$ is linearly separable.
The model learns from $\gD$ with the gradient descent. The update rule, equipped with a learning rate $\eta > 0$, is:
\begin{align*}
\begin{split}
    \vw_0 &= 0\\
    \vw_{t+1} & = \vw_t - \eta\nabla_\vw\left[ \sum_{i=1}^2\ell\left( f\left(\vx_i;\vw_t\right)\right) \right]\\
    & = \vw_t + \eta\left( e^{-\vw_t^\top \vx_1}\vx_1 + e^{-\vw_t^\top \vx_2}\vx_2 \right).
\end{split}
\end{align*}

For brevity, denote the model output of the $i$-th data point at the $t$-th epoch as $y_t^{(i)} \coloneq f(\vx_i; \vw_t)$. The update rule for the parameter is simplified as:

\begin{equation}
\label{eq:synth_param_update_rule}
    \vw_{t+1} = \vw_t + \eta\left( e^{-y_t^{(1)}}\vx_1 + e^{-y_t^{(2)}}\vx_2 \right).
\end{equation}

We also derive the update rule of model output for each instance:
\begin{equation}
\label{eq:synth_output_update_rule}
\begin{cases}
    \begin{aligned}
        y_{t+1}^{(1)} &= \vw_{t+1}^\top \vx_1 = \left( \vw_t + \eta\left( e^{-y_t^{(1)}}\vx_1 + e^{-y_t^{(2)}}\vx_2 \right) \right)^\top \vx_1\\
        &= y_{t}^{(1)} + \eta e^{-y_t^{(1)}} \lVert \vx_1 \rVert^2 + \eta e^{-y_t^{(2)}}\langle \vx_1, \vx_2 \rangle,\\
        y_{t+1}^{(2)} &= y_{t}^{(2)} + \eta e^{-y_t^{(2)}} \lVert \vx_2 \rVert^2 + \eta e^{-y_t^{(1)}}\langle \vx_1, \vx_2 \rangle.
    \end{aligned}
\end{cases}
\end{equation}

Assume that $\vx_2$ is farther from the origin in terms of distance than $\vx_1$ is, but not too different in terms of angle. Formally,
\begin{assumption}
\label{eq:synth_assump}
    $\lVert \vx_2 \rVert > 1$, $4\lVert \vx_1 \rVert^2 < 2\langle \vx_1, \vx_2 \rangle < \lVert \vx_2 \rVert^2$. Moreover, $\langle \vx_1, \vx_2 \rangle < \lVert \vx_1 \rVert\lVert \vx_2 \rVert$.
\end{assumption}
Under these assumptions, as $\langle \vx_1, \vx_2 \rangle$ > 0, $\gD$ is linearly separable. Also, notice that $\vx_1$ and $\vx_2$ are not parallel. Our definition of a linearly separable dataset is in accordance with \citet{soudry2018implicit}. A dataset $\gD$ is linearly separable if there exists $\vw^*$ such that $\langle \vx_i, \vw^* \rangle > 0, \forall i$.

\begin{theorem}
\label{prop:smaller_time}
    Let $V_{t;J}^{(i)}$ be the variance and $\mu_{t;J}^{(i)}$ be the mean of $\sigma(y_t^{(i)})$ within a window from time $t$ to $t+J-1$. Denote $T_v$ and $T_{vm}$ as the first time when $V_{t;J}^{(1)} > V_{t;J}^{(2)}$ and $V_{t;J}^{(1)}(1-\mu_{t;J}^{(i)}) > V_{t;J}^{(2)}(1-\mu_{t;J}^{(2)})$ occurs, respectively. Under \Cref{eq:synth_assump},
    % if $\eta < \frac{1}{\langle \vx_1, \vx_2 \rangle + \lVert \vx_2 \rVert^2} \log \frac{\langle \vx_1, \vx_2 \rangle}{\lVert \vx_1 \rVert^2}$
    if $\eta$ is sufficiently small then $T_{vm} < T_v$.
\end{theorem}

By \citet{soudry2018implicit}, the learning is progressed as: $\vw_t$, $y_t^{(1)}$, and $y_t^{(2)}$ diverges to positive infinity (Lemma 1) but $\vw_t$ directionally converges towards $L_2$ max margin vector, $\hat{\vw} = \vx_1 / \|\vx_1\|^2$, or $\lim_{t\to\infty} \frac{\vw_t}{\lVert \vw_t \rVert} = \frac{\hat{\vw}}{\lVert \hat{\vw} \rVert}$ (Theorem 3). Moreover, the growth of $\vw$ is logarithmic, i.e. $\vw_t \approx \hat{\vw}\log t$. We hereby note that Theorem 3 of ~\citet{soudry2018implicit} holds for learning rate $\eta$ smaller than a global constant. Since our condition requires $\eta$ to be sufficiently small, we will make use of the findings of Theorem 3.


\begin{lemma}
\label{lem:dyt_diverge}
    $\Delta y_t \coloneq y_t^{(2)} - y_t^{(1)}$ is a non-negative, strictly increasing sequence. Also, $\lim_{t\to\infty}\Delta y_t = \infty$.
\end{lemma}
\begin{proof}
    \leavevmode

    1) Since $\vw_0 = 0$, $y_0^{(1)} = 0 = y_0^{(2)}$ so $\Delta y_0 = 0$. By \Cref{eq:synth_output_update_rule} and \Cref{eq:synth_assump}, $\Delta y_1 = y_1^{(2)} - y_1^{(1)} = \eta\left( \lVert \vx_2 \rVert^2 - \lVert \vx_1 \rVert^2 \right) > 0$.
    
    2)
    \begin{align*}
        \Delta y_{t+1} - \Delta y_{t} &= \eta \left[ e^{-y_{t}^{(2)}}\left( \lVert \vx_2 \rVert^2 - \langle \vx_1, \vx_2 \rangle \right) + e^{-y_{t}^{(1)}}\left( \langle \vx_1, \vx_2 \rangle - \lVert \vx_1 \rVert^2 \right) \right]\\
        &\eqcolon K_1 e^{-y_{t}^{(1)}} + K_2 e^{-y_{t}^{(2)}} > 0,
    \end{align*}

    for some positive constant $K_1, K_2$.
    As $y_t^{(i)} = \vw_t^\top \vx_i$ would logarithmically grow in terms of $t$, $e^{-y_{t}^{(i)}}$ is decreasing in $t$. Moreover, as $y_t^{(1)} = \vw_t^\top \vx_1 \approx \hat{\vw}^\top \vx_1 \log t = \log t$, $e^{-y_{t}^{(1)}}$ is (asymptotically) in scale of $t^{-1}$ and so is $\Delta y_{t+1} - \Delta y_{t}$. Hence, $\left\{ \Delta y_t \right\}$ is non-negative and increases to infinity.
\end{proof}

The notation $\Delta y_t \coloneq y_t^{(2)} - y_t^{(1)}$ will be used throughout this section. Next, we show that, under \Cref{eq:synth_assump}, $y_{t+1}^{(1)} < y_t^{(2)}$ for all $t > 0$.
\begin{lemma}
\label{lem:yt2_too_large}
    For all $t > 0$, $y_{t+1}^{(1)} < y_t^{(2)}$.
\end{lemma}
\begin{proof}
    \leavevmode
    Notice that:
    \begin{equation*}
    \begin{cases}
        y_{1}^{(1)} = \eta \lVert \vx_1 \rVert^2 + \eta \langle \vx_1, \vx_2 \rangle\\
        y_{1}^{(2)} = \eta \lVert \vx_2 \rVert^2 + \eta \langle \vx_1, \vx_2 \rangle.
    \end{cases}
    \end{equation*}
    
    1) $y_2^{(1)} < y_1^{(2)}$:
    \begin{align*}
        y_2^{(1)} &= y_{1}^{(1)} + \eta e^{-y_1^{(1)}} \lVert \vx_1 \rVert^2 + \eta e^{-y_1^{(2)}}\langle \vx_1, \vx_2 \rangle\\
        &= \eta \left(e^{-y_1^{(1)}} + 1\right) \lVert \vx_1 \rVert^2 + \eta \left(e^{-y_1^{(2)}} + 1 \right)\langle \vx_1, \vx_2 \rangle\\
        &< \eta \times 2\lVert \vx_1 \rVert^2 + \eta \times 2\langle \vx_1, \vx_2 \rangle\\
        &< \eta \langle \vx_1, \vx_2 \rangle + \eta \lVert \vx_2 \rVert^2 = y_1^{(2)}.
    \end{align*}
    
    2) Assume, for $t>0$, $y_{t+1}^{(1)} < y_t^{(2)}$.
    \begin{align*}
        y_{t+2}^{(1)} &= y_{t+1}^{(1)} + \eta e^{-y_{t+1}^{(1)}} \lVert \vx_1 \rVert^2 + \eta e^{-y_{t+1}^{(2)}}\langle \vx_1, \vx_2 \rangle\\
        &< y_{t}^{(2)} + \eta e^{-y_{t}^{(1)}} \lVert \vx_1 \rVert^2 + \eta e^{-y_{t}^{(2)}}\langle \vx_1, \vx_2 \rangle\\
        &< y_{t}^{(2)} + \eta e^{-y_{t}^{(1)}} \langle \vx_1, \vx_2 \rangle + \eta e^{-y_{t}^{(2)}} \lVert \vx_2 \rVert^2 = y_{t+1}^{(2)}.
    \end{align*}
\end{proof}
By \Cref{lem:yt2_too_large}, for all $t>0$, $\left( y_{t}^{(2)}, y_{t+1}^{(2)} \right)$ lies entirely on right-hand side of $\left( y_{t}^{(1)}, y_{t+1}^{(1)} \right)$, without any overlap.

We first analyze the following term: $\frac{y_{t+1}^{(1)} - y_t^{(1)}}{y_{t+1}^{(2)} - y_t^{(2)}}$. Observe that:
\begin{align}
\begin{split}
\label{eq:ratio_y}
    \frac{y_{t+1}^{(1)} - y_t^{(1)}}{y_{t+1}^{(2)} - y_t^{(2)}} &= \frac{\eta e^{-y_t^{(1)}} \lVert \vx_1 \rVert^2 + \eta e^{-y_t^{(2)}}\langle \vx_1, \vx_2 \rangle}{\eta e^{-y_t^{(2)}} \lVert \vx_2 \rVert^2 + \eta e^{-y_t^{(1)}}\langle \vx_1, \vx_2 \rangle}\\
    &= \frac{\lVert \vx_1 \rVert^2 + e^{-\Delta y_t}\langle \vx_1, \vx_2 \rangle}{\langle \vx_1, \vx_2 \rangle + e^{-\Delta y_t} \lVert \vx_2 \rVert^2}.
\end{split}
\end{align}

It is derived that the fraction is an increasing sequence in terms of $t$. For values $a, b, c, c', d, d' > 0, \frac{a+c}{b+d} < \frac{a+c'}{b+d'} \Leftrightarrow ad'+cb+cd' < ad+c'b+c'd$. Taking:
\begin{align*}
\begin{cases}
a = \lVert \vx_1 \rVert^2 \\ b = \langle \vx_1, \vx_2 \rangle
\end{cases}
\begin{cases}
c = e^{-\Delta y_t}\langle \vx_1, \vx_2 \rangle \\ d = e^{-\Delta y_t} \lVert \vx_2 \rVert^2
\end{cases}
\begin{cases}
c' = e^{-\Delta y_{t+1}}\langle \vx_1, \vx_2 \rangle \\ d' = e^{-\Delta y_{t+1}} \lVert \vx_2 \rVert^2
\end{cases}
,
\end{align*}

we have
\begin{align*}
    &ad'+cb+cd'\\
    =\; &e^{-\Delta y_{t+1}}\lVert \vx_1 \rVert^2 \lVert \vx_2 \rVert^2 + e^{-\Delta y_t}\langle \vx_1, \vx_2 \rangle^2 + e^{-\Delta y_t}e^{-\Delta y_{t+1}} \langle \vx_1, \vx_2 \rangle \lVert \vx_2 \rVert^2\\
    <\; &e^{-\Delta y_{t}}\lVert \vx_1 \rVert^2 \lVert \vx_2 \rVert^2 + e^{-\Delta y_{t+1}}\langle \vx_1, \vx_2 \rangle^2 + e^{-\Delta y_t}e^{-\Delta y_{t+1}} \langle \vx_1, \vx_2 \rangle \lVert \vx_2 \rVert^2\\
    =\; &ad+c'b+c'd.
\end{align*}
The inequality holds by \Cref{lem:dyt_diverge} and the Cauchy-Schwarz inequality. Taking the limit of \Cref{eq:ratio_y} as $t\to\infty$, the ratio converges to:
\begin{equation}
\label{eq:limit_ratio_y}
    R\coloneq \frac{\lVert \vx_1 \rVert^2}{\langle \vx_1, \vx_2 \rangle}.
\end{equation}
For the later uses, we also define the initial ratio, which is smaller than 1:
\begin{equation}
    R_0\coloneq \frac{y_{1}^{(1)} - y_0^{(1)}}{y_{1}^{(2)} - y_0^{(2)}} = \frac{\lVert \vx_1 \rVert^2 + \langle \vx_1, \vx_2 \rangle}{\langle \vx_1, \vx_2 \rangle + \lVert \vx_2 \rVert^2} \;(\leq R).
\end{equation}

Now we analyze a similar ratio of the one-step difference, but in terms of $\sigma\left(y_t^{(i)}\right)$ instead of $y_t^{(i)}$. There, $\sigma$ stands for the logistic function, $\sigma(z) = \left( 1+e^{-z} \right)^{-1}$. Notice that $\sigma'(z) = \sigma(z)\left(1-\sigma(z)\right)$.

\begin{lemma}
\label{lem:ratio_sig_y}
    $\gamma_V(t) \coloneq \frac{\sigma\left(y_{t+1}^{(1)}\right) - \sigma\left(y_t^{(1)}\right)}{\sigma\left(y_{t+1}^{(2)}\right) - \sigma\left(y_t^{(2)}\right)}$ monotonically increases to $+\infty$.
\end{lemma}
\begin{proof}
\begin{align*}
    \gamma_V(t) &= \frac{y_{t+1}^{(1)} - y_t^{(1)}}{y_{t+1}^{(2)} - y_t^{(2)}}\frac{\sigma'\left(\zeta_t^{(1)} \right)}{\sigma'\left(\zeta_t^{(2)}\right)} \hspace{1em} (\text{for some } \begin{cases}
        \zeta_t^{(1)}\in\left(y_t^{(1)}, y_{t+1}^{(1)}\right)\\ \zeta_t^{(2)} \in \left(y_t^{(2)}, y_{t+1}^{(2)}\right)
    \end{cases} \text{by the mean value theorem.})\\
    % \intertext{for some $\zeta_t^{(1)}\in\left(y_t^{(1)}, y_{t+1}^{(1)}\right), \zeta_t^{(2)} \in \left(y_t^{(2)}, y_{t+1}^{(2)}\right)$ by the mean value theorem.}
    &\geq \frac{y_{t+1}^{(1)} - y_t^{(1)}}{y_{t+1}^{(2)} - y_t^{(2)}}\frac{\sigma'\left(y_{t+1}^{(1)} \right)}{\sigma'\left(y_{t}^{(2)} \right)} \hspace{1em}(\because \sigma'\text{: decreasing on } \sR^+)\\
    &= \frac{y_{t+1}^{(1)} - y_t^{(1)}}{y_{t+1}^{(2)} - y_t^{(2)}} \frac{e^{-y_{t+1}^{(1)}}  \left( 1 + e^{-y_{t+1}^{(1)}} \right)^{-2}}{e^{-y_{t}^{(2)}} \left( 1 + e^{-y_{t}^{(2)}} \right)^{-2}}\\
    &\geq \frac{y_{t+1}^{(1)} - y_t^{(1)}}{y_{t+1}^{(2)} - y_t^{(2)}} \frac{1}{4} e^{y_{t}^{(2)} - y_{t+1}^{(1)}} \hspace{1em}(\because \left( 1+e^{-z} \right)^{-2}\in[1/4, 1] \text{ on }\sR^+)\\
    &\geq \frac{R_0}{4} e^{y_{t}^{(2)} - y_{t+1}^{(1)}}.
\end{align*}
As $y_{t}^{(2)} - y_{t+1}^{(1)} = y_{t}^{(2)} - y_{t}^{(1)}-\eta \left( e^{-y_{t}^{(1)}}\lVert \vx_1 \rVert^2 + e^{-y_{t}^{(2)}}\langle \vx_1, \vx_2 \rangle \right) \to \infty$, $\gamma_V(t)\to \infty$. For the part that proves $\gamma_V(t)$ is increasing, see \Cref{sec:gamma_v_inc}.
\end{proof}

Notice that $\gamma_V(0) < 1$. \Cref{lem:ratio_sig_y} implies that there exists (unique) $T_v>0$ such that for all $t \geq T_v$, $\gamma_V(t) > 1$ holds, or $\sigma\left(y_{t+1}^{(1)}\right) - \sigma\left(y_t^{(1)}\right) > \sigma\left(y_{t+1}^{(2)}\right) - \sigma\left(y_t^{(2)}\right)$. Recall that the (sample) variance of a finite dataset $\mathcal{T} = \left\{\vx_1, \cdots, \vx_n\right\}$ can be computed as:
\begin{equation*}
    \text{Var}[\mathcal{T}] = \frac{1}{n(n-1)}\sum_{i=1}^{n-1}\sum_{j=i+1}^n\left(\vx_i-\vx_j\right)^2.
\end{equation*}

Hence, for given $J$, (which corresponds to the window size,) for all $t\geq T_v$,
\begin{align*}
    V_{t;J}^{(1)}\coloneq\text{Var}\left[\left\{ \sigma\left( y_\tau^{(1)} \right) \right\}_{\tau=t}^{t+J-1}\right] &= \frac{1}{J(J-1)}\sum_{k=0}^{J-2} \sum_{l=k+1}^{J-1} \left[ \sigma\left( y_{t+l}^{(1)}\right) - \sigma\left( y_{t+k}^{(1)} \right) \right]^2\\
    &> \frac{1}{J(J-1)}\sum_{k=0}^{J-2} \sum_{l=k+1}^{J-1} \left[ \sigma\left( y_{t+l}^{(2)}\right) - \sigma\left( y_{t+k}^{(2)} \right) \right]^2\\
    &= \text{Var}\left[\left\{ \sigma\left( y_\tau^{(2)} \right) \right\}_{\tau=t}^{t+J-1}\right] \eqcolon V_{t;J}^{(2)}.
\end{align*}

It is easily derived that the converse is true: If $\gamma_V(t)$ is increasing and $V_{t;J}^{(1)} > V_{t;J}^{(2)}$ then $\gamma_V(t) > 1$.

We have two metrics: the first is only the variance (which corresponds to the Dyn-Unc score) and the second is the variance multiplied by the mean subtracted from 1 (which corresponds to the DUAL pruning score). Both the variance and the mean are calculated within a window of fixed length. At the early epoch, as the model learns $\vx_2$ first, both metrics show a smaller value for $\vx_1$ than that for $\vx_2$. At the late epoch, now the model learns $\vx_1$, so the order of the metric values reverses for both metrics.


Our goal is to show that the elapsed time of the second metric for the order to be reversed is shorter than that of the first metric. Let $T_{vm}$ be that time for our metric. We represent the mean of the logistic output within a window of length $J$ and from epoch $t$, computed for $i$-th instance by $\mu_{t;J}^{(i)}$:
\begin{equation}
    \mu_{t;J}^{(i)} \coloneq \frac{1}{J}\sum_{\tau=t}^{t+J-1}\sigma\left(y_\tau^{(i)}\right).
\end{equation}
For $t\geq T_v$, we see that the inequality still holds:
\begin{align*}
    &V_{t;J}^{(1)}\left(1 - \mu_{t;J}^{(1)}\right)\\
    &= \left[\frac{1}{J(J-1)}\sum_{k=0}^{J-2} \sum_{l=k+1}^{J-1} \left[ \sigma\left( y_{t+l}^{(1)}\right) - \sigma\left( y_{t+k}^{(1)} \right) \right]^2\right] \left[1-\frac{1}{J}\sum_{\tau=t}^{t+J-1} \sigma\left(y_\tau^{(1)}\right) \right]\\
    &> \left[\frac{1}{J(J-1)}\sum_{k=0}^{J-2} \sum_{l=k+1}^{J-1} \left[ \sigma\left( y_{t+l}^{(2)}\right) - \sigma\left( y_{t+k}^{(2)} \right) \right]^2\right] \left[1-\frac{1}{J}\sum_{\tau=t}^{t+J-1} \sigma\left(y_\tau^{(2)}\right) \right]\\
    &= V_{t;J}^{(2)}\left(1 - \mu_{t;J}^{(2)}\right).
\end{align*}
as for all $t$, $\sigma\left(y_t^{(2)}\right) > \sigma\left(y_t^{(1)}\right)$. Indeed, $T_{vm}\leq T_v$ holds, but is $T_{vm} < T_v$ true? To verify the question, we reshape the terms for a similar analysis upon $\mu$:
\begin{align}
\label{eq:var_comp_ineq}
\begin{split}
    &V_{t;J}^{(1)}\left(1 - \mu_{t;J}^{(1)}\right)\\
    &= \left[\frac{1}{J(J-1)}\sum_{k=0}^{J-2} \sum_{l=k+1}^{J-1} \left[ \sigma\left( y_{t+l}^{(1)}\right) - \sigma\left( y_{t+k}^{(1)} \right) \right]^2\right] \left[\frac{1}{J}\sum_{\tau=t}^{t+J-1} 1-\sigma\left(y_\tau^{(1)}\right) \right]\\
    &> \left[\frac{1}{J(J-1)}\sum_{k=0}^{J-2} \sum_{l=k+1}^{J-1} \left[ \sigma\left( y_{t+l}^{(2)}\right) - \sigma\left( y_{t+k}^{(2)} \right) \right]^2\right] \left[\frac{1}{J}\sum_{\tau=t}^{t+J-1} 1-\sigma\left(y_\tau^{(2)}\right) \right]\\
    &= V_{t;J}^{(2)}\left(1 - \mu_{t;J}^{(2)}\right).
\end{split}
\end{align}

The intuition is now clear: for any time before $T_v$, we know that the variance of $\vx_1$ is smaller than that of $\vx_2$, if the ratio corresponding to $1-\sigma(y)$ is large, the factors could be canceled out and the inequality still holds. If this case is possible, definitely $T_{vm}<T_v$.


Now let us analyze the ratio of $1-\sigma\left( y_t^{(i)} \right)$.
\begin{lemma}
\label{lem:ratio_comp_y}
    $\gamma_M(t) \coloneq \frac{1-\sigma\left(y_t^{(1)}\right)}{1 - \sigma\left(y_t^{(2)}\right)}$ increases to $+\infty$.
\end{lemma}
\begin{proof}
\begin{align*}
    \gamma_M(t) &= \frac{1+e^{y_t^{(2)}}}{1+e^{y_t^{(1)}}}\\
    &= e^{\Delta y_t} - \frac{e^{\Delta y_t}-1}{1+e^{y_t^{(1)}}}\\
    &\geq e^{\Delta y_t} - \frac{e^{\Delta y_t}}{1+e^{y_t^{(1)}}}\\
    &= e^{\Delta y_t} \sigma\left( y_t^{(1)} \right).
\end{align*}
The quantity in the last line indeed diverges to infinity. We now show that $\gamma_M(t)$ is increasing.
\begin{align*}
    \gamma_M(t) &= e^{\Delta y_t} - \frac{e^{\Delta y_t}}{1+e^{y_t^{(1)}}} + \frac{1}{1+e^{y_t^{(1)}}}\\
    &= e^{\Delta y_t} \sigma\left( y_t^{(1)} \right) + 1-\sigma\left( y_t^{(1)}\right)\\
    &= \left( e^{\Delta y_t} -1 \right) \sigma\left( y_t^{(1)} \right) + 1\\
    &< \left( e^{\Delta y_{t+1}} -1 \right) \sigma\left( y_{t+1}^{(1)} \right) + 1 = \gamma_M(t+1).
\end{align*}
\end{proof}

    Notice that, for $a>c>0, b>d>0, \frac{a-c}{b-d}< \frac{a}{b} \Leftrightarrow \frac{a}{b} < \frac{c}{d}$. Recall from \Cref{lem:ratio_sig_y} that $\gamma_V(t) = \frac{1-\sigma\left(y_t^{(1)}\right) - \left[ 1-\sigma\left(y_{t+1}^{(1)}\right)\right]}{1-\sigma\left(y_t^{(2)}\right) - \left[ 1-\sigma\left(y_{t+1}^{(2)}\right)\right]}$, hence $\gamma_V(t) < \gamma_M(t)$. Moreover,
\begin{align*}
    \gamma_V(t) &\leq \frac{y_{t+1}^{(1)} - y_t^{(1)}}{y_{t+1}^{(2)} - y_t^{(2)}} \frac{\sigma'\left(y_{t}^{(1)} \right)}{\sigma'\left(y_{t+1}^{(2)} \right)}\\
    &= \frac{y_{t+1}^{(1)} - y_t^{(1)}}{y_{t+1}^{(2)} - y_t^{(2)}} e^{y_{t+1}^{(2)} - y_t^{(1)}} \left( \frac{1+e^{-y_{t+1}^{(2)}}}{1+e^{-y_{t}^{(1)}}} \right)^2\\
    &\leq \frac{y_{t+1}^{(1)} - y_t^{(1)}}{y_{t+1}^{(2)} - y_t^{(2)}} e^{y_{t+1}^{(2)} - y_t^{(1)}} \left( \frac{1+e^{-y_{t+1}^{(2)}}}{1+e^{-y_{t}^{(1)}}} \right)
    \hspace{1em} \because \left( \frac{1+e^{-y_{t+1}^{(2)}}}{1+e^{-y_{t}^{(1)}}} \right) \in (0, 1].\\
    &= \frac{y_{t+1}^{(1)} - y_t^{(1)}}{y_{t+1}^{(2)} - y_t^{(2)}} e^{y_{t+1}^{(2)} - y_t^{(2)}} e^{\Delta y_t} \left( \frac{1+e^{-y_{t+1}^{(2)}}}{1+e^{-y_{t}^{(1)}}} \right)\\
    &\leq \frac{y_{t+1}^{(1)} - y_t^{(1)}}{y_{t+1}^{(2)} - y_t^{(2)}} e^{y_{t+1}^{(2)} - y_t^{(2)}} e^{\Delta y_t} \left( \frac{1+e^{-y_{t}^{(2)}}}{1+e^{-y_{t}^{(1)}}} \right)\\
    &\leq Re^{y_1^{(2)} - y_0^{(2)}}\gamma_M(t).
\end{align*}

Now we revisit \Cref{eq:var_comp_ineq}.
\begin{equation}
\begin{aligned}
\label{eq:var_mean_comp_ineq}
    \left[\frac{1}{J(J-1)}\sum_{k=0}^{J-2} \sum_{l=k+1}^{J-1} \left[ \sigma\left( y_{t+l}^{(1)}\right) - \sigma\left( y_{t+k}^{(1)} \right) \right]^2\right] \left[\frac{1}{J}\sum_{\tau=t}^{t+J-1} 1-\sigma\left(y_\tau^{(1)}\right) \right]\\
    > \left[\frac{1}{J(J-1)}\sum_{k=0}^{J-2} \sum_{l=k+1}^{J-1} \left[ \sigma\left( y_{t+l}^{(2)}\right) - \sigma\left( y_{t+k}^{(2)} \right) \right]^2\right] \left[\frac{1}{J}\sum_{\tau=t}^{t+J-1} 1-\sigma\left(y_\tau^{(2)}\right) \right].
\end{aligned}
\end{equation}
Assume, for the moment, that for some constant $C>1$, $\sigma\left(y_{t+1}^{(1)}\right) - \sigma\left(y_t^{(1)}\right) > C^{-1}\left[ \sigma\left(y_{t+1}^{(2)}\right) - \sigma\left(y_t^{(2)}\right) \right]$ but $1 - \sigma\left(y_t^{(1)}\right) > C^{2}\left[ 1 - \sigma\left(y_t^{(2)}\right) \right]$ for all large $t$. Then the ratio of the first term of the left-hand side of \cref{eq:var_mean_comp_ineq} to the first term of the right-hand side is greater than $C^{-2}$. Also, the ratio of the second term of the left-hand side of \cref{eq:var_mean_comp_ineq} to the second term of the right-hand side is greater than $C^2$. If so, we observe that 1) the inequality in \cref{eq:var_mean_comp_ineq} holds, 2) as the condition $\gamma_V(t) \geq 1$ for $T_v$ now changed to $\gamma_V(t) \geq C^{-1}$ for $T_{vm}$, hence $T_{vm} < T_v$ is guaranteed. It remains to find the constant $C$. Recall that, for all $t$,
\begin{equation*}
    \gamma_V(t) \leq Re^{y_1^{(2)} - y_0^{(2)}}\gamma_M(t).
\end{equation*}
If we set $Re^{y_1^{(2)} - y_0^{(2)}} = C^{-3}$, when $\gamma_V(t)$ becomes at least $C^{-1}$, we have $\gamma_M(t) \geq C^2$, satisfying the condition for $T_{vm}$.
If the learning rate is sufficiently small, then $\gamma_V(t)$ cannot significantly increase in one step, allowing $\gamma_V(t)$ to fall between $C^{-1}$ and $1$. 
Refer to \cref{fig:gamma_v} to observe that the graph of $\gamma_V(t)$ resembles that of a continuously increasing function.

\subsubsection{Monotonicity of \texorpdfstring{$\gamma_V(t)$}{Lemma 1.5}}
\label{sec:gamma_v_inc}

Recall that:
\begin{align*}
    \gamma_V(t) &\coloneq \frac{\sigma\left(y_{t+1}^{(1)}\right) - \sigma\left(y_t^{(1)}\right)}{\sigma\left(y_{t+1}^{(2)}\right) - \sigma\left(y_t^{(2)}\right)}\\
    &= \frac{y_{t+1}^{(1)} - y_t^{(1)}}{y_{t+1}^{(2)} - y_t^{(2)}}\frac{\sigma'\left(\zeta_t^{(1)} \right)}{\sigma'\left(\zeta_t^{(2)}\right)}
\end{align*}
for some $\zeta_t^{(1)}\in\left(y_t^{(1)}, y_{t+1}^{(1)}\right), \zeta_t^{(2)} \in \left(y_t^{(2)}, y_{t+1}^{(2)}\right)$ by the mean value theorem. The first term is shown to be increasing (to $R$). $\gamma_V(t)$ is increasing if the second term is also increasing in $t$.


Let $\Delta \zeta_t \coloneq \zeta_t^{(2)} - \zeta_t^{(1)}$. By \Cref{lem:yt2_too_large}, $\Delta \zeta_t > 0$.
\begin{align*}
    \frac{\sigma'\left(\zeta_t^{(1)} \right)}{\sigma'\left(\zeta_t^{(2)}\right)} &= \frac{e^{-\zeta_t^{(1)}}}{e^{-\zeta_t^{(2)}}}\left( \frac{1+e^{-\zeta_t^{(2)}}}{1+e^{-\zeta_t^{(1)}}} \right)^2\\
    &= e^{\Delta \zeta_t}\left( \frac{1+e^{-\zeta_t^{(1)}-\Delta\zeta_t}}{1+e^{-\zeta_t^{(1)}}} \right)^2.
\end{align*}
Define $g(x, y) \coloneq e^x\left(\frac{1+e^{-y-x}}{1+e^{-y}}\right)^2$. The partial derivatives satisfy:
\begin{equation*}
\begin{cases}
    \nabla_x g = \frac{\left( e^y-e^{-x} \right) \left( e^{x+y}+1 \right)}{\left( 1+e^{y} \right)^2}>0 \text{ for } x>0 \text{ if } y>0\\
    \nabla_y g = \frac{2e^{y-x}\left( e^x-1 \right) \left( e^{x+y}+1 \right)}{\left( 1+e^{y} \right)^3}>0, \forall y \text{ if } x>0.
\end{cases}
\end{equation*}
Notice that $\frac{\sigma'\left(\zeta_t^{(1)} \right)}{\sigma'\left(\zeta_t^{(2)}\right)} = g(\Delta \zeta_t, \zeta_t^{(1)})$. Since $\zeta_t^{(1)}\in\left(y_t^{(1)}, y_{t+1}^{(1)}\right)$ is (strictly) increasing and positive, if we show that $\Delta \zeta_t$ is increasing in $t$, we are done. Our result is that, if $y_{t+1}^{(i)} - y_{t}^{(i)}$ is small for $i=1,2$, $\zeta_t^{(i)} \approx \left( y_t^{(i)} + y_{t+1}^{(i)} \right)/2$ so $\Delta \zeta_t \approx \left( \Delta y_t + \Delta y_{t+1} \right)/2$, which is indeed increasing.


In particular, if (, assume for now) for all $t$,
\begin{align}
\label{eq:zeta_close_middle}
    \zeta_t^{(i)} &\in \left( \frac{2y_t^{(i)}+y_{t+1}^{(i)}}{3}, \frac{y_t^{(i)}+2y_{t+1}^{(i)}}{3} \right)\\ \nonumber
    \Rightarrow \; \Delta \zeta_t &\in \left( \frac{\Delta y_t + \Delta y_{t+1}}{3} + \frac{y_t^{(2)}-y_{t+1}^{(1)}}{3}, \frac{\Delta y_t + \Delta y_{t+1}}{3} +\frac{y_{t+1}^{(2)}-y_{t}^{(1)}}{3} \right)\\ \nonumber
    \Rightarrow \; \Delta \zeta_t &< \frac{\Delta y_t + \Delta y_{t+1}}{3} +\frac{y_{t+1}^{(2)}-y_{t}^{(1)}}{3} \\\nonumber
    &<\frac{\Delta y_{t+1} + \Delta y_{t+2}}{3} + \frac{y_{t+1}^{(2)}-y_{t+2}^{(1)}}{3} &(\dag)\\\nonumber
    &< \Delta \zeta_{t+1}
\end{align}
$(\dag)$ holds by \Cref{eq:synth_assump}:
\begin{align*}
    (\dag) \Leftrightarrow \;&\Delta y_{t+2} - \Delta y_{t} > y_{t+2}^{(1)} - y_{t}^{(1)}, \forall t\\
    \Leftarrow \;&\Delta y_{t+1} - \Delta y_{t} > y_{t+1}^{(1)} - y_{t}^{(1)}, \forall t\\
    \Leftrightarrow \;&\eta \left[ e^{-y_{t}^{(1)}}\left( \langle \vx_1, \vx_2 \rangle - \lVert \vx_1 \rVert^2 \right) + e^{-y_{t}^{(2)}}\left( \lVert \vx_2 \rVert^2 - \langle \vx_1, \vx_2 \rangle \right) \right] > \\ &\eta \left[ e^{-y_t^{(1)}} \lVert \vx_1 \rVert^2 + e^{-y_t^{(2)}}\langle \vx_1, \vx_2 \rangle \right], \forall t.
\end{align*}
It remains to show \Cref{eq:zeta_close_middle}. To this end, we use \Cref{lem:small_step_MVT}.
\begin{lemma}
\label{lem:small_step_MVT}
    Let $z_2>z_1(\geq0)$ be reals and $\zeta \in \left(z_1, z_2\right)$ be a number that satisfies the following: $\sigma\left(z_2\right)-\sigma\left(z_1\right) = \left(z_2-z_1\right)\sigma'(\zeta)$. Denote the midpoint of $\left(z_1, z_2\right)$ as $m \coloneq \left(z_1 + z_2\right)/2$. For $(1 \gg)\epsilon > 0$, if $z_2-z_1 < \mathcal{O} \left( \sqrt{\epsilon} \right)$ then $\lvert \zeta - m \rvert < \epsilon$.
\end{lemma}
\begin{proof}
    Expand the Taylor series of $\sigma$ at $m$ for $z_i$:
    \begin{equation*}
        \sigma\left(z_i\right) = \sigma(m) + \sigma'(m)\left(z_i-m\right) + \frac{1}{2!}\sigma''(m)\left(z_i-m\right)^2 + \frac{1}{3!}\sigma'''(m)\left(z_i-m\right)^3 + \mathcal{O}\left( \lvert z_i-m \rvert^4 \right)
    \end{equation*}
    We have:
    \begin{equation*}
        \sigma\left(z_2\right) - \sigma\left(z_1\right) = \sigma'(m)\left(z_2-z_1\right) + \frac{1}{24}\sigma'''(m)\left(z_2-z_1\right)^3 + \mathcal{O}\left( \left(z_2-z_1\right)^5 \right)
    \end{equation*}
    \begin{equation*}
        \sigma'\left(\zeta\right) = \sigma'(m) + \frac{1}{24}\sigma'''(m)\left(z_2-z_1\right)^2 + \mathcal{O}\left( \left(z_2-z_1\right)^4 \right)
    \end{equation*}

    Now, expand the Taylor series of $\sigma'$ at $m$ for $\zeta$:
    \begin{equation*}
        \sigma'\left(\zeta\right) = \sigma'(m) + \sigma''(m)\left(\zeta -m\right) + \frac{1}{2!}\sigma'''(m)\left(\zeta -m\right)^2 + \mathcal{O}\left( \lvert \zeta-m \rvert^3 \right)
    \end{equation*}

    Comparing the above two lines gives
    \begin{equation*}
        24 \sigma^{\prime \prime}(m)(\zeta-m)+12 \sigma^{\prime \prime \prime}(m)(\zeta-m)^2=\sigma^{\prime \prime \prime}(m)\left(z_2-z_1\right)^2+\mathcal{O}\left(\left(z_2-z_1\right)^3\right)
    \end{equation*}

    If $\sigma'''(m) = 0$ then $|\zeta - m| = \mathcal{O}\left(\left(z_2-z_1\right)^3\right)$, so $z_2-z_1 = \mathcal{O}\left( \sqrt{\epsilon} \right)$ is sufficient.
    
    Otherwise, we can solve the above for $\zeta-m$ from the fact that $\sigma''(z) < 0$ for $z > 0$:
    \begin{align*}
        12\sigma'''(m)(\zeta-m)&=-12 \sigma^{\prime \prime}(m)-\sqrt{\left(12 \sigma^{\prime \prime}(m)\right)^2+12 \sigma^{\prime \prime \prime}(m)\left[\sigma^{\prime \prime \prime}(m)\left(z_2-z_1\right)^2+\mathcal{O}\left(\left(z_2-z_1\right)^3\right)\right]}\\
        &=\frac{12\sigma'''(m)^2(z_2-z_1)^2}{24\sigma''(m)} + \mathcal{O}\left(\left(z_2-z_1\right)^3\right)
    \end{align*}
    The last equality is from the Taylor series $\sqrt{1+\frac{a}{x^2}}-1 = \frac{a}{2x^2} + \mathcal{O}\left( a^2x^{-4} \right)$, or $\sqrt{x^2+a}-x = \frac{a}{2x} + \mathcal{O}\left( a^2x^{-3} \right)$.
    We have $\lvert \zeta-m \rvert = \Theta\left( (z_2-z_1)^2 \right)$.
\end{proof}

For $\lvert \zeta_t^{(i)} - \left( y_t^{(i)} + y_{t+1}^{(i)} \right)/2 \rvert < \left( y_{t+1}^{(i)} - y_{t}^{(i)} \right)/6$, it suffices to have $y_{t+1}^{(i)} - y_{t}^{(i)} < \mathcal{O}\left( \sqrt{\left( y_{t+1}^{(i)} - y_{t}^{(i)} \right)/6} \right)$. 
This generally holds for sufficiently small $\eta$.

\vspace{1em}
\subsection{Experimental Results under Synthetic Setting}
\label{sec:syn_expt}
This section displays the figures plotted from the experiments on the synthetic dataset. We choose $\gX = \sR^2$ and $\gD = \left\{\ ((0.1, 0.1), 1), ((10, 5), 1) \right\}$. We fix $J=10$ and $\eta=0.01$ (unless specified). The total time of training $T$ is specified for each figure for neat visualization. In this setting, the upper bound for the learning rate is $\log(75)/126.5 \approx 0.034$.

\begin{figure}[ht]
    \centering
    \label{fig:syn_weight_evolve}
    \includegraphics[width=0.65\linewidth]{Figures/figure_synthetic/weight_evolution.pdf}
    \caption{Illustration of the evolution of the weight as the model learns from the two-point dataset. Observe that the weight learns $\vx_2$ first (closer to the orange dashed line), but gradually moves towards $\vx_1$ (closer to the brown dashed line). Here $T=10,000$.}
\end{figure}

We also empirically validate our statements of \cref{sec:gamma_v_inc}. \cref{fig:empirical_valid_inc} shows that $\gamma_V(t)$ and $\Delta\zeta_t$ are indeed increasing functions. \cref{fig:empirical_valid_midpoint} shows that $\zeta_t^{(i)}$ is sufficiently close to the midpoint of the interval it lies in, $\left(y_t^{(i)}, y_{t+1}^{(i)}\right)$.

\begin{figure}[htbp] 
    \centering
    \begin{subfigure}{0.4\textwidth}
        \centering
        \includegraphics[width=\textwidth]{Figures/figure_synthetic/gamma_V.pdf}
        \caption{$\gamma_V(t)$ in log scale.}
        \label{fig:gamma_v}
    \end{subfigure}
    \hfill
    \begin{subfigure}{0.4\textwidth}
        \centering
        \includegraphics[width=\textwidth]{Figures/figure_synthetic/Dzeta.pdf} 
        \caption{$\Delta\zeta_t$}
        \label{fig:delta_zetat}
    \end{subfigure}
    \caption{Empirical validations of the critical statements in \cref{sec:gamma_v_inc}. We ran experiments and plot the results that both $\gamma_V(t)$ (left---in log scale) and $\Delta \zeta_t$ (right) are an increasing sequence in terms of $t$. Here, we set $\eta=0.0005$. The reason is that if the learning rate is larger, $\sigma(y_t^{(2)})$ quickly saturates to 1, leading to a possibility of division by zero in $\gamma_V(t)$ and degradation in numerical stability of $\Delta \zeta_t$. Moreover, notice that the graph of $\gamma_V(t)$ in the log scale closely resembles that of $\Delta\zeta_t$ in the original scale.}
    \label{fig:empirical_valid_inc}
\end{figure}

\begin{figure}[htbp] 
    \centering
    \begin{subfigure}{0.4\textwidth}
        \centering
        \includegraphics[width=\textwidth]{Figures/figure_synthetic/zeta1_almost_midpoint.pdf}
        \caption{$|\zeta_t^{(1)} - (y_t^{(1)} + y_{t+1}^{(1)})/2|$ in log scale.}
        \label{fig:zeta1}
    \end{subfigure}
    \hfill
    \begin{subfigure}{0.4\textwidth}
        \centering
        \includegraphics[width=\textwidth]{Figures/figure_synthetic/zeta2_almost_midpoint.pdf} 
        \caption{$|\zeta_t^{(2)} - (y_t^{(2)} + y_{t+1}^{(2)})/2|$ in log scale.}
        \label{fig:zeta2}
    \end{subfigure}
    \caption{Empirical validations of the critical statements in \cref{sec:gamma_v_inc}. We ran experiments and plot the results that both $\zeta_t^{(1)}$ (left) and $\zeta_t^{(2)}$ (right) are extremely close to the midpoint $(y_t^{(1)} + y_{t+1}^{(1)})/2$ and $(y_t^{(2)} + y_{t+1}^{(2)})/2$, compared to the interval length, respectively. In both plots, the blue line is the true distance while the orange line is the interval length. Here, we set $\eta=0.0005$ for the same reasoning of \cref{fig:empirical_valid_inc}. Empirically, the noise introduced by MVT is too small to deny that $\Delta \zeta_t$ is an increasing sequence.}
    \label{fig:empirical_valid_midpoint}
\end{figure}

\pagebreak
We also show that we can observe the ``flow'' of the moon plot as in \cref{fig:Moon_plot} for the synthetic dataset.

\begin{figure}[ht]
    \centering
    
    \includegraphics[width=0.98\linewidth]{Figures/figure_synthetic/synthetic_moonplot.pdf}
    \caption{Evolution of $\vx_1, \vx_2$ by their mean and standard deviation in prediction probabilities at different epochs. The marker `o' and `x' stands for $\vx_1$ and $\vx_2$, respectively. The red color indicates the sample to be selected, and the blue color indicates the sample to be pruned. Observe that the path that each data point draws resembles is of moon-shape. Here $T=30,000$.}
    \label{fig:syn_moon}
\end{figure}
\end{document}
