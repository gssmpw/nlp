% arXiv: use pdflatex
\documentclass[sigconf, nonacm]{acmart}

\setcopyright{none}
\settopmatter{printacmref=false}
\renewcommand\footnotetextcopyrightpermission[1]{}
\usepackage{hyperref}   % Load this first
\usepackage{hyperxmp}   
%\usepackage[table]{xcolor}
\usepackage{xcolor}
\usepackage{colortbl}
\usepackage{tikz}
\usepackage{balance}
\usepackage{amsmath,amsfonts}
\usepackage{graphicx}
\usepackage{capt-of}% or \usepackage{caption}
\usepackage{booktabs}
\usepackage{varwidth}
\newsavebox \tmpbox
\usepackage[acronym]{glossaries}
\usepackage{enumitem}
\usepackage{multirow}
\usepackage{subcaption}
\usepackage{xspace}
\usepackage{makecell}
\usepackage{lscape}
\usepackage{algorithm}
\usepackage{algpseudocode}
\usepackage{amsmath}
\usepackage{multicol}
\usepackage{pifont}
\newcommand{\cmark}{\ding{51}}%
\newcommand{\xmark}{\ding{55}}%
\usepackage{multicol}
\newcommand{\fraudzen}{\textit{FraudZen}\xspace}
\newcommand{\simbox}{\textit{SIMBox}\xspace}
\newcommand{\sign}{\textit{SigN}\xspace}
\usepackage{pifont}


\newcommand{\josc}[1]{[\textcolor{purple}{Josiane: #1}]}
\newcommand{\jos}[1]{\textcolor{purple}{#1}}
\newcommand{\ali}[1]{\textcolor{blue}{#1}}
\newcommand{\alic}[1]{\textcolor{red}{[Aline:] #1}}
\newcommand{\alac}[1]{[\textcolor{green}{Alain: #1}]}
\newcommand{\ala}[1]{\textcolor{green}{#1}}
\usepackage{color}
\definecolor{lightred}{rgb}{1,0.6,0.6}
\definecolor{lightgray}{gray}{0.8}
\definecolor{beaublue}{rgb}{0.74, 0.83, 0.9}


\newcommand\greybox[1]{%
  %\vskip\baselineskip%
  \par\noindent\colorbox{lightgray}{%
    \begin{minipage}{0.47\textwidth}#1\end{minipage}%
  }%
  \vskip\baselineskip%
}

\newcommand\bluebox[1]{%
  %\vskip\baselineskip%
  \par\noindent\colorbox{beaublue}{%
    \begin{minipage}{0.47\textwidth}#1\end{minipage}%
  }%
  \vskip\baselineskip%
}



\AtBeginDocument{%
  \providecommand\BibTeX{{%
    \normalfont B\kern-0.5em{\scshape i\kern-0.25em b}\kern-0.8em\TeX}}}

%% Rights management information.  This information is sent to you
%% when you complete the rights form.  These commands have SAMPLE
%% values in them; it is your responsibility as an author to replace
%% the commands and values with those provided to you when you
%% complete the rights form.
\setcopyright{acmcopyright}
\copyrightyear{2018}
\acmYear{2018}
\acmDOI{XXXXXXX.XXXXXXX}


\acmJournal{POMACS}
\acmVolume{37}
\acmNumber{4}
\acmArticle{111}
\acmMonth{8}

\makeglossaries
\begin{document}


\title{\textit{SigN}: SIMBox Activity Detection Through Latency Anomalies at the Cellular Edge}

% \author{Anne Josiane Kouam}
% \affiliation{\institution{TU Berlin} \country{Germany}}
% \email{kouam.djuigne@tu-berlin.de}
\author{Anne Josiane Kouam}
\affiliation{\institution{TU Berlin, Germany} \country{}}
%\email{TU Berlin, Germany \quad kouam.djuigne@tu-berlin.de}

\author{Aline Carneiro Viana}
\affiliation{\institution{INRIA, France} \country{}}
%\email{aline.viana@inria.fr}

\author{Philippe Martins}
\affiliation{
\institution{Telecom Paris, France} \country{}}
%\email{martins@telecom-paris.fr}

\author{Cedric Adjih}
\affiliation{\institution{INRIA, France} \country{}}
%\email{cedric.adjih@inria.fr}

\author{Alain Tchana}
\affiliation{
\institution{Grenoble INP, France} \country{}
}
%\email{alain.tchana@grenoble-inp.fr}


\renewcommand{\shortauthors}{Kouam, et al.}


\begin{abstract}

Despite their widespread adoption, cellular networks face growing vulnerabilities due to their inherent complexity and the integration of advanced technologies. One of the major threats in this landscape is Voice over IP (VoIP) to GSM gateways, known as \simbox devices. These devices use multiple SIM cards to route VoIP traffic through cellular networks, enabling international bypass fraud with losses of up to \$3.11 billion annually. Beyond financial impact, \simbox activity degrades network performance, threatens national security, and facilitates eavesdropping on communications.
Existing detection methods for \simbox activity are hindered by evolving fraud techniques and implementation complexities, limiting their practical adoption in operator networks.
 
This paper addresses the limitations of current detection methods by introducing \sign, a novel approach to identifying \simbox activity at the cellular edge. The proposed method focuses on detecting \textit{remote SIM card association}, a technique used by \simbox appliances to mimic human mobility patterns. 
The method detects latency anomalies between \simbox and standard devices by analyzing cellular signaling during network attachment.
Extensive indoor and outdoor experiments demonstrate that \simbox devices generate significantly higher attachment latencies, particularly during the authentication phase, where latency is up to 23 times greater than that of standard devices. We attribute part of this overhead to immutable factors such as LTE authentication standards and Internet-based communication protocols. Therefore, our approach offers a robust, scalable, and practical solution to mitigate \simbox activity risks at the network edge.
\end{abstract}


% \begin{CCSXML}
% 	<ccs2012>
% 	<concept>
% 	<concept_id>10003033.10003039.10003045.10003046</concept_id>
% 	<concept_desc>Networks~Routing protocols</concept_desc>
% 	<concept_significance>500</concept_significance>
% 	</concept>
% 	</ccs2012>
% \end{CCSXML}

% \begin{CCSXML}
% 	<ccs2012>
% 	<concept>
% 	<concept_id>10002978.10003014.10003015</concept_id>
% 	<concept_desc>Security and privacy~Security protocols</concept_desc>
% 	<concept_significance>500</concept_significance>
% 	</concept>
% 	</ccs2012>
% \end{CCSXML}

% \ccsdesc[500]{Networks~Routing protocols}
% \ccsdesc[500]{Security and privacy~Security protocols}

\keywords{Cellular signaling, Network attachment, Cellular authentication}

% \received{20 February 2007}
% \received[revised]{12 March 2009}
% \received[accepted]{5 June 2009}


\maketitle

\section{Introduction}
\label{sec:intro}




%\alic{josiane, all my ocmments are in comment in overleaf.} 
%Cellular networks provide digital communications for more than five billion people around the globe. Yet, their openness to the general public and complexity have exposed cellular networks to increased attacks over the previous decades. In this context, 
%International bypass termination, also known as \simbox fraud, is one of the most prevalent scams affecting cellular networks. 

Cellular networks provide digital communications for more than five billion people around the globe. However, their accessibility to the general public, inherent complexity, and integration of multiple advanced technologies have exposed these networks to numerous attacks, which have significantly increased over the past decades.

In this context, Voice over IP (VoIP) to GSM gateways, also known as \simbox, are a significant source of security challenges within cellular networks. \simbox appliances bridge two telecommunication technologies by converting VoIP traffic to traditional GSM cellular networks. This allows them to route calls initiated over the internet through cellular networks by re-originating them from one of their multiple SIM cards. %they contain.

\textit{Although \simbox appliances may have legitimate uses}, such as reducing telecommunication costs or automating calls in dedicated companies, \textit{this paper highlights that their potential for misuse poses significant security risks}. Indeed, \simbox appliances are at the basis of international bypass frauds in cellular networks, recognized as one of the top four phone system frauds causing substantial losses to mobile network operators~\cite{CFCA:2021}. As depicted in Fig. \ref{fig:simbox_scheme}, International bypass fraud, or simply \simbox fraud, involves intercepting international mobile calls routing and diverting them through an internet flow (VoIP) to a \simbox in the destination country. The \simbox then re-originates the received VoIP traffic as a local mobile call from one of its SIM cards to the receiving party. 
%Fraudsters thus bypass the regular interconnect operator, circumventing the international termination charges to pay the local call termination charges that are much lower and benefit from the difference. 
Fraudsters bypass the regular interconnect operator, avoiding international termination charges by paying the lower local call termination charges, thus profiting from the difference.


Therefore, beyond the \textit{growing revenue loss for operators}, estimated at $2.7$ billion in 2019~\cite{CFCA:2019} and $3.11$ billion in 2021~\cite{CFCA:2021}\footnote{The CFCA’s 2023 survey summary~\cite{cfca_telecom_fraud_2023} indicates a 12\% increase in global telecom fraud losses compared to 2021, highlighting the continuing rise in the economic impact of fraud. However, the full report is not publicly available for verification.}, \simbox usage \textit{negatively impacts network quality} for legitimate consumers and \textit{compromises national security}. Specifically, \simbox fraud degrades the quality of experience for consumers due to call initiation delays and network unavailability, which in turn increases churn. Moreover, \simbox's re-originated calls introduce bias into operators' network usage records, distorting call origins and locations and affecting various analyses and research\cite{Naboulsi:2016}.
More critically, \simbox usage enables international attackers to masquerade as national subscribers, a vulnerability that could be exploited for covert operations, including by terrorists. Furthermore, \simbox appliances provide attackers with the ability to eavesdrop on international phone conversations~\cite{goantifraud:call_recording}, endangering user privacy and facilitating international espionage.





%Beyond a growing revenue loss for operators estimated to \$2.7 Billion in 2019~\cite{CFCA:2019} and \$3.11 Billion in 2021~\cite{CFCA:2021}, \simbox fraud negatively impacts network quality for legitimate consumers and national security.  In particular, \simbox fraudulent activities create network hotspots through the injection of huge volumes of tunneled calls into the local network cells, degrading network availability and reliability. More critically, \simbox fraud allows international attackers to act as national subscribers, which terrorists could use for covert operations. \simbox appliances also allow attackers eavesdropping on international phone conversations~\cite{goantifraud:call_recording}, impeding user privacy and giving the possibility of international espionage. 


%As international voice traffic transits through multiple worldwide-located parties (i.e., transit carriers), there is no cohesion and interoperability of the technologies used along the route, leading to call routing opaqueness~\cite{Merve:2017}. This impedes any states’ cooperation for fraud mitigation, exacerbated by the fact that the notion of legal can vary significantly from one state to another. Therefore, 


As a result, investigations into detecting \simbox activity in cellular networks have gained the attention of researchers. The objective is to provide mobile operators with means to detect and regulate \simbox usage on their networks by implementing legal registration for legitimate \simbox operations (as exemplified in \cite{t-mobile}) while blocking undeclared usage. Such investigations are typically conducted at the destination operator level, where fraud occurs. The most common approach involves analyzing network users' cellular activity traces to differentiate between \simbox patterns and legitimate ones.

%Thus, investigations on the detection of \simbox activity in cellular networks have attracted the attention of researchers in the field. The objective is to provide mobile operators with means to detect and regulate \simbox usage on their networks by implementing legal registration for legitimate \simbox operations (as exemplified in \cite{t-mobile}) while blocking undeclared \simbox usage. Such investigations are typically conducted at the destination operator level, where fraud occurs. The most common approach involves analyzing network users' cellular activity traces to differentiate between \jos{\simbox patterns and legitimate ones.}
%fraudulent SIM cards (used inside fraudsters' \simbox appliances) and legitimate ones.
%Such analysis can be performed \jos{(i) at \textit{the network core} through the use of historical data for the extraction of patterns that identify a \simbox or (ii) at \textit{the network edge} through the monitoring of users' network activities in real-time.}
%\textit{(i) offline} through the use of historical data for the extraction of patterns to identify fraud a posteriori, i.e., after it has occurred or \textit{(ii) online } through the  monitoring of users' network activities in real-time. 

%\josc{Aller directement avec quelque chose comme ils extraient des CDRs des comportements propres à l'utilisation d'une carte SIM dans une SIMBox: low mobility, high traffic, par opposition au comportement humain. Ils sont efficients et implémentés dans le core network.. pratiques mais pas adapté à l'évolution de la simbox.}

Most detection methods from the literature~\cite{Sallehuddin:2013, Sallehuddin:2015, Murynets:2014, Hagos:2018, Fitsum:2020, Veloso:2020, Marah:2015} extract the spatiotemporal communication behavior of each SIM card by relying on \textit{\acrfull{CDR} traces}. %\alic{check comment. remove:}communication and mobility behavior of 
CDRs are time-stamped and geo-referenced recordings of mobile device-generated events (i.e., call, text, data) collected by network operators. SIM cards used within \simbox appliances tend to exhibit automated behavior, distinct from human or natural patterns, characterized by low mobility, repetitive calls at odd hours, or many contacts, as noted in \cite{Murynets:2014}. %These 
Such literature contributions have demonstrated excellent detection performance (i.e., an average accuracy of 94.5\%) and are implemented offline, leveraging historical data collected at the network core without impacting network performance.\\
However, \simbox appliances currently available on the market offer functionalities that enable fraudsters to automatically mimic more advanced and human-like behavior, thereby evading CDR-based \simbox activity detection~\cite{Kouam:2024}. 

%Offline detection benefits from operators' vast data sources, providing a comprehensive view of fraudulent activities over time with no impact on the network performance.
%Hence, most detection methods from the literature are offline and rely on \textit{\acrfull{CDR} traces}~\cite{Sallehuddin:2013, Sallehuddin:2015, Murynets:2014, Hagos:2018, Fitsum:2020, Veloso:2020, Marah:2015}. %, which 
% CDRs are time-stamped and geo-referenced recordings of mobile devices' generated events (i.e., call/text/data), done by network operators.
% \acrshort{CDR}-based detection extracts users' communication and mobility behavior from such CDRs to uncover SIM cards with abnormal patterns by applying Machine Learning classification models. These contributions show excellent detection performances (i.e., on average 94.5\% of accuracy) for frauds with naive behavior, i.e., low mobility, repetitive calls at odd hours, or high number of contacts, as in \cite{Murynets:2014}. However, \simbox appliances currently in the market offer functionalities such as \textit{remote SIM card allocation}, which enable fraudsters to reproduce more advanced and human-like behavior~\cite{Kouam:2021}. This makes it harder to detect such frauds using the literature CDR-based detection techniques, justifying fraudsters' trend towards advanced human behavior simulation techniques~\cite{goantifraud_mobility_simulation}.


%\josc{Aller directement: ils ont un problème de practical relevancy car ils sont proposés pour être appliqués au réseau d'acces. pas de gros focus sur la méthode de Beomseok stp.}

% a few contributions are applied at the cellular edge and propose a real-time monitoring of users' network activity to detect \simbox patterns. The related analysed data consists of \textit{call audio} quality~\cite{Reaves:2015} or speakers' voices~\cite{Elrajubi:2017} to uncover possible degradation due to \simbox routing. Besides, more recently, \textit{cellular signaling data} has been leveraged in \cite{Beomseok:2023} to build device model fingerprints and propose an access-control-list prevention methodology.
% Unfortunately, such contributions overlook the computational challenge related to cellular-edge-based deployment, affecting their practical relevancy. Indeed, as they should run smoothly at each of the~-- hundreds to  thousands of~-- cell towers that compose the cellular network edge, such solutions should provide sufficient indicators to reliably detect \simbox patterns while limiting the computational resources incurred by the processing of these indicators across the entire network. This unsolved scalability challenge explains their low adoption in practice: call-audio-based solution requires investigate all local calls across the network, while signaling-fingerprint requires maintaining at each base station an exhaustive list of device fingerprints to consult regularly.
Conversely, a few contributions focus on the cellular edge, proposing real-time monitoring of users' network activity to detect \simbox patterns. These analyses include monitoring \textit{call audio} quality~\cite{Reaves:2015} and speakers' voices~\cite{Elrajubi:2017} to identify potential degradation due to \simbox routing. More recently, \textit{cellular signaling data} has been leveraged~\cite{Beomseok:2023} to create device model fingerprints and suggest an access-control-list prevention methodology.\\
Unfortunately, these approaches often overlook the computational challenges of cellular-edge-based deployment, which affects their practical relevance. Since these solutions must operate efficiently across the hundreds to thousands of cell towers comprising the cellular network edge, they must provide reliable indicators for detecting \simbox patterns while minimizing the computational resources needed to process them network-wide. This scalability challenge remains unresolved and explains their limited practical adoption (cf. \S \ref{sec:motivation}): e.g., \textit{call-audio}-based solutions require monitoring all local calls across the network. Similarly,
signaling-based fingerprinting necessitates maintaining an exhaustive list of device fingerprints at each base station for regular consultation.


% online fraud detection methods from the literature are efficient regardless of the fraudulent SIM cards' behavior, giving fraudsters fewer evasion possibilities. However, for its practical relevancy, online detection should efficiently address a challenge that has been overlooked in related literature:  \textit{providing sufficient indicators to reliably detect fraudulent network users while limiting the computational resources incurred by the real-time processing of these indicators across the entire network.} Indeed, online detection solutions should run smoothly at each of the~-- hundreds to  thousands of~-- cell towers that compose the cellular network edge. In particular:%We discuss this challenge by distinguishing online detection of the literature as follows:
% \begin{itemize}[leftmargin=*]
%     \item %Audio-based detection:
%     A few contributions proposed the real-time analysis of \textit{call audio} quality~\cite{Reaves:2015} or speakers' voices~\cite{Elrajubi:2017} to uncover possible degradation due to \simbox fraud. However, this implies the investigation of \textit{all local calls} across the network, which is hardly feasible owing to regulations limiting operators’ access to their subscribers’ call audio, and the related scalability issue.
%     \item %Signaling-based detection: 
%     More recently, \textit{cellular signaling data} has been leveraged for online \simbox fraud detection. Signaling data results from logging user devices' control-plane procedures (e.g., network attachment, call setup, mobility management). Beomseok Oh et al.~\cite{Beomseok:2023} demonstrated the potential to create device model fingerprints using network-layer signaling information, from which they proposed an access-control-list prevention methodology. Therefore, if an attaching device model's fingerprint does not correspond to the one in a network's database, it is considered fraudulent. Despite the originality and preciseness of the analyses conducted in \cite{Beomseok:2023}, its effective capability to mitigate \simbox fraud is limited by \textit{the need to have at each network's base station an exhaustive list of all (existing) device model fingerprints}, without which it is incapacitated. This \textit{complexity} restrains the solution widespread deployment while the exhaustivity can hardly be met.
% \end{itemize}

This paper aims to bridge the gap posed by the limitations of the current solutions: \textit{It addresses the detection of \simbox patterns remaining undetected through CDR-based detection or overlooking computational costs, and proposes a novel and practical approach to unmask \simbox activities at the cellular edge.} 
% This paper introduces \sign, a novel online approach for preventing \simbox fraud at the network edge.
% \sign focuses on the mitigation of advanced \simbox frauds, i.e., with a realistic human behavior, bypassing traditional CDR-based methodologies. Indeed, such advanced frauds are the most prevalent in the fraud ecosystem~\cite{goantifraud_fraud_success, goantifraud:simbank_importance} while naive frauds have been efficiently addressed in the offline literature.  Hence, \sign effectively identifies and prevents the occurrence of advanced \simbox frauds at the network edge while addressing unresolved challenges in online literature to provide a largely deployable solution with no impact on the network performance.
The proposed approach, i.e., named \sign, identifies and leverages an indicator of \simbox activity: the \textit{remote SIM card association}.
\textit{Remote SIM card association} is a ground technology used in the \simbox to mimic human mobility pattern. It allows fraudsters to avoid the resource-intensive and easily-detected movements of \simbox appliances by enabling the binding of a SIM card to a distant gateway (with cellular antenna), as depicted in Fig. \ref{fig:simbox_architecture}. 
%As \simbox appliances can be bulky and require a fixed wired internet connection, they are inherently static. 
%\textit{Remote SIM card association} frees fraudsters from resource-demanding and easily-detected movements of \simbox appliances by allowing them to bind a SIM card to a distant \simbox gateway (with cellular antenna), as depicted in Fig. \ref{fig:simbox_architecture}. 
%Fraudsters thus automate the realistic simulation of human behavior, as shown in Fig. \ref{fig:simbox_mobility}, by binding a SIM card to gateways at different locations, causing an erroneous (yet human-like) network recording of the SIM card locations. 
To the best of our knowledge, the \simbox is the only system capable of physically separating the SIM card from the cellular antenna. \textit{Remote SIM card association} is thus a distinct signature of \simbox activity resulting in decoupled network devices, unlike traditional coupled devices (e.g., phones).
%This decoupling is a distinct signature of \simbox activity, unlike traditional coupled devices (e.g., mobile phones).}
%\alic{i reviewed this paragraph, verify. original version is commented. }
%To the best of our knowledge, the \simbox is the only system with such ability to physically separate the SIM card from the cellular antenna. %Mobile Equipment (with antenna). 
%\textit{Remote SIM card association} is thus a signature for \simbox activity yielding decoupled network devices, as opposed to traditional devices (e.g., mobile phones) that are coupled. 


\sign precisely detects such \simbox-decoupled network devices at their attachment to the network by analyzing their generated cellular signaling at the network edge. By especially characterizing the \textit{latency of devices' signaling}, we provide empirical evidence of \textit{a significant dissimilarity between \simbox-decoupled devices and coupled ones during the network attachment}. 
%\alic{very good, but \simbox-originated decoupled network devices is too long. why not \simbox-decoupled network devices or \simbox-decoupled network devices?}
%\textit{\sign analyzes the cellular signaling at the network edge to detect such fraudulent \simbox devices before their attachment to the network, therefore, preventing fraudsters' implementation of advanced \simbox frauds.}
%\textit{Accordingly, \sign's goal is detecting and blocking such \simbox \textit{virtual} UEs before their attachment to the network, guaranteeing a severe limitation of \simbox fraud flexibility and its accurate detection through literature approaches. }
%\sign focuses on the \textit{cellular signaling of user devices during the network attachment}, aiming to block \simbox appliances with fraudulent intents before they can connect to the network.
%Precisely, \sign methodology builds up on \textit{the characterization of signaling latency} to provide empirical evidence of \textit{an important dissimilarity of the fraudulent \simbox devices compared to legitimate devices during the network attachment}.

\noindent This includes the following contributions, outlined in Fig. \ref{fig:methodology}:
%\josc{WORK ON THE CONTRIBS BELOW}
\begin{itemize}[leftmargin=*]
%\vspace{-0.5em}
    \item We set up inside a Faraday shield, realistic urban settings of an operator 4G/LTE radio access and core networks with specialized equipment. Indeed, 4G is the most widespread cellular technology, particularly in the developing countries 
    %\alic{verifie, mais je pense que le bon nom est:  lower-to-upper-income developing countries. en tout cas, pas world}
    where \simbox activity is the most striking, with 5G still several years away. Our testbed provides real-time access to the cellular edge network attachment signaling from 12 phones and 7 LTE \simbox appliances from two major manufacturers, collected at the base station (cf. \textbf{\S \ref{subsec:experimental_setup}}).  
    %4G is growingly adopted in most parts of the globe for its better quality of experience than 3G and 2G. Indeed, although the deployment of 5G is getting concrete, its democratization is still several years away in regions where fraud is the most striking (i.e., developing countries). 
   % \vspace{-0.5em}
   % \item %Thereon,
    \item We make the first literature empirical characterization of network attachment latency, to the best of our knowledge (cf. \textbf{\S \ref{subsec:measurement}}). We report \simbox decoupled devices generate at least $5$ times %$5\times$ 
    more latency than standard phones, particularly during the \textit{authentication phase} where their minimum latency is $23$ times higher. 
    %\vspace{-0.5em}
    \item We explain the latency overhead by analyzing the interactions between the SIM card and Mobile Equipment during the \textit{authentication phase} for a SIMBox decoupled device compared to standard devices. 
    %    We provide explainability of such latency overhead by analyzing SIM card to Mobile Equipment interactions during the \textit{authentication phase} for a \simbox decoupled device compared to the standards. %baseline. 
    Our investigation shows that the authentication latency in \simbox decoupled devices is influenced by unavoidable factors, such as LTE authentication standards and Internet-based communication protocols and vagaries. 
    Despite optimizations, this latency cannot match that of streamlined, legitimate devices, maintaining a clear distinction between \simbox decoupled devices and their coupled counterparts (cf. \textbf{\S \ref{subsec:explainability}}).
    %Even with optimization efforts, this latency cannot match that of stripped-down, legitimate devices, maintaining a clear distinction between \simbox decoupled devices and their legitimate counterparts (cf. \textbf{\S \ref{subsec:explainability}}).
    %\textit{Our study attests the generality of fraudulent \simbox devices' latency, revealing it is caused by factors beyond fraudsters reach and shared by all \simbox models} such as LTE standardized exchanges and TCP/UDP protocols' related procedures.
   % \vspace{-0.5em}
    \item On the other hand, we show through data collection in an actual operator network that a standard phone cannot reach such high authentication latency peaks regardless of the network signal conditions (cf. \textbf{\S \ref{subsec:tresholding}}). Our empirical findings confirm that \textit{authentication latency} is a reliable, practical, and robust metric for distinguishing \simbox decoupled devices from coupled ones.
    %\jos{On the other hand, we show through data collection in an actual operator network that a standard phone cannot reach such high authentication latency peaks regardless of the wireless network channel conditions (cf. \textbf{\S \ref{subsec:tresholding}}). This result confirms that the distinction of fraudulent \simbox devices is significant enough for identifying and stopping their connection to the network at the cellular edge.}
    %\vspace{-0.5em}
    % \item Based on this empirical evidence, we elaborate in \textbf{\S \ref{sec:real_world_deployment}} on the practicality of the widespread implementation of \sign \textit{without overhead} through a novel monitoring of the authentication latency collected at the cellular edge. 
    % Through statistical hypothesis testing, we show \sign has an almost 1-probability of correctly detecting the network attachment of a fraudulent \simbox device if it occurs, given the latency distributions.      
    \item Based on this empirical evidence, we demonstrate in \textbf{\S \ref{sec:real_world_deployment}} the practicality of \sign by introducing a novel method to monitor authentication latency at the cellular edge \textit{without added overhead}. 
    Through latency distribution analysis, \sign identifies devices with \textit{consistently unusual latency values, enabling operators to promptly investigate potential threats}. Therefore, our statistical analysis shows that \sign achieves near-perfect accuracy in detecting \simbox-decoupled device attachments.
    %Our statistical analysis shows that \sign achieves near-perfect accuracy in detecting \simbox decoupled device attachments through latency distribution analysis. Typically \sign highliht \textit{devices with consistently unusually high values and prompting further investigation by the operator.}
    %Based on this empirical evidence, we elaborate in \textbf{\S \ref{sec:real_world_deployment}} on the practicality of implementing \sign through a novel monitoring of authentication latency at the cellular edge \textit{without overhead}. 
    %Our statistical analysis demonstrates that \sign achieves near-certainty in detecting \simbox decoupled devices' attachments based on latency distributions.
    \item To ensure the reproducibility of our results, we have released the \sign datasets and code \href{https://anonymous.4open.science/r/SigN-E485/Readme.md}{\textit{at this anonymous link}.} 
\end{itemize}
%\vspace{-0.5em}
Additionally, \textbf{\S \ref{sec:background}} provides the background for our work, \textbf{\S \ref{sec:motivation}} discusses the motivation, and \textbf{\S \ref{sec:threat-model}} outlines the threat model and our defense objectives. Finally, we conclude in \textbf{\S \ref{sec:conclude}}. 
Readers can refer to the appendix for a list of acronyms used throughout the paper.

\section{Setting the stage}
\label{sec:background}
This section outlines the background for our research, covering the cellular network ecosystem and \simbox architecture.

\subsection{Cellular networks}
\vspace{0.13cm}
We overview several aspects of the 4G cellular networks, the most widespread cellular technology, particularly in the developing world where \simbox activity is the most prevalent. %, with 5G still several years away

\noindent\textbf{Architecture.}  The cellular network infrastructure consists of end devices, also known as User Equipment or UE (e.g., phone, tablet), base stations, and the core network.
A User Equipment (UE) is a mobile device %with a SIM card provided by the network operator, 
registering to the network to receive access to communication services.
It comprises two distinctive elements, the Mobile Equipment (ME) and the SIM card provided by the network operator. %(cf. Fig. \ref{fig:simbox_scheme}).
% The ME is the hardware device containing a processor, memory, transceiver, battery, etc. It is uniquely identified by a 15-digit IMEI (International Mobile Equipment Identity) code.
%A physical SIM card is inserted in the ME, or an eSIM is activated with a subscription. 
Base stations, called eNodeB in 4G networks, are intermediate connectors responsible for the radio transmission %and reception 
with the devices. At last, the core network handles administrative tasks such as the devices' authentication, security, and mobility management, intending to provide permanent service access.

\vspace{0.13cm}
\noindent\textbf{The network attachment} signaling procedure establishes a connection between end devices and the network. It occurs in four circumstances: when a device is powered on, when it moves into a new tracking area, when it loses connection with the network, or at a network trigger. In 4G, the network attachment (cf. Fig. \ref{fig:methodology}, step 2) consists of several steps aiming (i) the acquisition of the device identity, i.e., \acrfull{IMSI}, (ii) the mutual device and network authentication, (iii) the \acrfull{NAS} security setup, (iv) the device location update, and (v) the \acrfull{EPS} session establishment~\cite{3gpp_lte_attach}. 
%The cellular networks' fourth generation (4G), called Long Term Evolution (LTE), is growingly adopted in most parts of the globe for its better quality of experience than 3G and 2G. Moreover, although the deployment of 5G is getting concrete, its democratization is still several years away, notably in regions where fraud is the most striking (i.e., developing countries). 
%We therefore focus on LTE networks for our experimentation but note this paper's methodologies can easily be ported to other cellular standards.

\vspace{0.13cm}
\noindent\textbf{The SIM card} binds the mobile subscription to the network device.
It securely stores the \acrshort{IMSI} subscriber identifier and a secret symmetric key called the subscriber key (or $K_i$, in short) used in steps (ii) and (iii) of the network attachment. It also represents an environment protected from attackers where the network authentication and security algorithms are run following the \acrshort{AKA} protocol~\cite{3gpp_aka}.
 %Its most common form is a small, tamper-resistant integrated circuit embedded in a plastic card and removable from the ME. Such properties allow the SIM card to play a central role in the UE attachment procedure. Indeed, e

\subsection{\simbox architecture and fraud}

\begin{figure}
\centering
\includegraphics[width=0.9\linewidth]{Figures/simbox_fraud_scheme_1.pdf}
\caption{International call routing: (Flow 1) Legitimate scheme, (Flow 2) Fraudulent scheme.}
\label{fig:simbox_scheme}
\end{figure}

\vspace{0.13cm}
\noindent\textbf{\simbox fraud scheme.} As depicted in Fig. \ref{fig:simbox_scheme}, \simbox fraud interferes with the regular international voice call routing (\textit{flow 1}). 
In a regular routing, the call traffic leaves the caller's mobile operator (Operator A) and is routed to the destination country through a set of transit operators. The traffic is received directly by the called party's operator (Operator B), who terminates it.
Nevertheless, a transit carrier can be fraudulent. Indeed, transit carriers perform traffic interconnection between countries by buying and reselling international termination routes. A fraudulent carrier instead diverts the traffic it receives through a low cost VoIP trunk, as in the \textit{flow 2} on Fig. \ref{fig:simbox_scheme}.
The diverted traffic is sent to a \simbox (VoIP to GSM gateway) located in the destination country and re-originated as a national mobile call to its recipient. 
Once in the destination country, there are two possible fraudulent termination scenarios: (i) 2-1 
is an on-net termination when the re-originated call is made using a SIM card of Operator B, the same operator of the called party, and (ii) 2-2 is an off-net termination when the fraudster uses a SIM card from a different local operator in the destination country.


\vspace{0.13cm}
\noindent\textbf{The \simbox} operates 
%is a system operating 
as a VoIP GSM gateway. It receives a diverted call traffic as a VoIP client and terminates it by re-originating a  cellular mobile call using one of its numerous SIM cards. 
%(e.g., the GoIP324 \simbox model~\cite{GoIP324} has 128 SIM slots).
The \simbox continuously creates network devices by associating SIM cards and GSM modules (providing wireless link to the network). 
%In this association, the SIM card binds the mobile subscription to the formed device, while the GSM module provides the formed device's wireless link to the network.
The \simbox includes three kinds of hardware components:
%This includes three kind of components:
%\vspace{-0.6em}
\begin{itemize}[leftmargin=*]
    \item The \textit{gateway} is a rack with a set of GSM modules maintaining the wireless communication inside a given cellular frequency range 
    %of cellular technologies 
    (i.e.,2G/3G/4G). It receives incoming VoIP traffic and distributes it to the GSM modules 
    for termination as mobile calls. The gateway plays the role of Mobile Equipment for the formed \simbox devices. 
    Hence, the recorded network location of a \simbox device is the location of its belonging gateway.
    Most gateways in the market include SIM slots for operation. % (cf. Table \ref{tab:simbox_review}).
    %\vspace{-0.6em}
    %providing SIM cards for operation (
    \item The \textit{SIMBank} is an appliance with numerous SIM slots that remotely holds a bundle of SIM cards (e.g., 128 in the SMB128 model~\cite{SMB128}). It manages \simbox SIM cards, including their addition, removal, and data transfer. %to other components.
   % \vspace{-0.6em}
    \item The \textit{control server} is a web server providing the \simbox control functions, i.e., binding of SIM cards to GSM modules and architecture configuration. It can be hosted online to ease remote access from a web client.
   % \vspace{-0.6em}
\end{itemize}
%Hence, we distinguish two \simbox architectures: \textit{standalone} and \textit{distributed} ones. In the standalone architecture, the \simbox consists of a unique gateway equipped with SIM slots, which can thus handle all components' functions at once. 

%We distinguish two types of SIMBox architectures: \textit{standalone} \simbox, which consists of a unique gateway with SIM slots, and \textit{distributed} \simbox, which involves the control server for managing the different appliances. 
Distributed \simbox architecture involves the interaction of such appliances over an IP network using TCP or UDP protocols. Hence, 
%\textit{standalone} \simbox that consists of a unique gateway with SIM slots and \textit{distributed} \simbox that involves the control server for managing the different appliances. %includes at least one gateway, at least one SIM card holder (i.e., a SIMBank or a gateway), and the control server. 
%as shown in Fig. \ref{fig:simbox_architecture}, we distributed architectures allow the formation of \textit{actual UEs} through \textit{local SIM card association} if the SIM card is in the same appliance as its associated GSM module, and \textit{virtual UEs} through \textit{remote SIM card association} if the SIM card is from another appliance, i.e., the SIMBank.
as shown in Fig. \ref{fig:simbox_architecture}, \simbox devices formation can be done through \textit{local SIM card association} if the SIM card is in the same appliance as its associated GSM module, or \textit{remote SIM card association} if the SIM card is from another appliance, i.e., the SIMBank. \textit{Local SIM card association} results in \textit{coupled UEs}, while \textit{remote SIM card association} yields \textit{decoupled UEs}.


% \vspace{0.13cm}
% \noindent\textbf{Usage.}
% Legitimate companies may use \simbox architectures with an IP-PBX to extend the voice communication coverage from an internal IP network to external multi-operator cellular networks. This is a typical setup in call centers, enabling cost-effective operations, balancing incoming and outgoing calls across multiple SIM cards, and allowing the seamless expansion of SIM cards or gateways to handle increasing call volumes.
% Unfortunately, such \simbox functionalities have been misused for various fraudulent purposes. This paper focuses on \simbox international bypass fraud.




\begin{figure}
\centering
\includegraphics[width=\linewidth]{Figures/simbox_architecture.pdf}
\caption{Example of a \simbox distributed architecture.}
\label{fig:simbox_architecture}
\end{figure}

\section{Unraveling literature gaps}
\label{sec:motivation}
Despite the significant impact of \simbox activity, it has received limited attention in the literature, with only 15 detection methods proposed since 2011. We categorize these contributions based on the type of cellular network data they use and how it is processed.
%Despite the striking impact of \simbox activity, this problem has remained little tackled in the literature, with only 15 detection contributions since 2011. 
%\alic{is it still the same? no other new one?}
%We consider such literature contributions based on the cellular network data type they rely on and how such data is handled.
Specifically, we distinguish between methodologies that rely on core network data (e.g., Call Detail Records/CDRs, \S \ref{subsec:offline}) and those that use edge network data (e.g., call audio and cellular signaling, \S \ref{subsec:online}). 
This section highlights the strengths and weaknesses of current \simbox detection approaches and positions \sign to address the remaining gaps. 
For a complete survey of \simbox fraud solutions before 2021, refer to \cite{Kouam:2021}.
%Therefore, we distinguish methodologies based on network-core data (Call Detail Records/CDRs, i.e., \S \ref{subsec:offline}) and those based on network-edge data (call audio and cellular signaling, i.e., \S \ref{subsec:online}).
%This section sheds light on the nuanced strengths and weaknesses of the current \simbox detection literature and strategically positions \sign to address the remaining gaps in the literature.
%In particular, network-core-based contributions show a \textit{resiliency deficiency}, making them inefficient \jos{against more recent \simbox patterns,} 
%against advanced frauds, 
%while \textit{impracticality} is a notable weakness in network-edge-based contributions. Drawing from these insights, we strategically position \sign to address the remaining gaps in the literature.
%identifies and discusses the strengths and weaknesses of such contributions, thereby positioning \sign drive in addressing the remaining gaps. 
%This section discusses the limitations inherent in \simbox fraud detection literature, thereby positioning \sign in addressing these critical gaps.
%The reader can refer to \cite{Kouam:2021} for a complete survey on \simbox fraud solutions prior to 2021. 
%\alic{and... add FraudZen for a deep investigation on types of frauds  even if it is CDR based, it shows the background on the subject.}



\subsection{Network-core-based detection}%: strength and resilience deficiency}
\label{subsec:offline}

% \jos{
% As aforementioned, the vast majority of works~\cite{Sallehuddin:2013, Sallehuddin:2015, Murynets:2014, Hagos:2018, Fitsum:2020, Veloso:2020, Marah:2015} leverage \acrshort{CDR}s to distinguish between \simbox SIM cards and those used by legitimate users, by inferring per-user communication and mobility behavior. 
% Therefore, CDR-based approaches detect with high accuracy \simbox fraudsters' use of naive behavior as it induces SIM cards with no or little mobility or cluster movements of SIM cards~\cite{Murynets:2014}. Yet, the evolution of fraudster techniques in the realistic simulation of human traffic and mobility through \textit{remote SIM card association} has significantly reduced such fraud detectability~\cite{Kouam:2021}.


Literature approaches operating at the network core~\cite{Sallehuddin:2013, Sallehuddin:2015, Murynets:2014, Hagos:2018, Fitsum:2020, Veloso:2020, Marah:2015} aim to identify \simbox activity %fraudulent SIM cards 
from SIM cards' communication and mobility behavior extracted from CDRs datasets. These methods rely on various features (e.g., \#calls at night, \#contacts, \#incoming calls, \#locations) to distinguish SIM cards used for \simbox termination from SIM cards used by genuine consumers. \textit{Such contributions have demonstrated high accuracy, averaging 94.5\%, in detecting \simbox activity characterized by unusual patterns in communication or mobility, as in \cite{Marah:2015, Murynets:2014, Kehelwala:2015}: numerous outgoing calls, few stay points, SIM card clusters, or no incoming calls.}

Unfortunately, \simbox devices have evolved with functionalities that automatically mimic human behavior in CDR datasets, known as \acrfull{HBS}~\cite{Kouam:2021}. \acrshort{HBS} techniques enable \simbox devices to operate while maintaining human-like behavior in terms of communication and mobility. In communication, this is achieved by thresholding the number and duration of initiated calls and controlling their contacts and timing. For mobility, fraudsters use \textit{remote SIM card association}, binding a SIM card to a remote gateway (i.e., Mobile Equipment), resulting in an erroneous network recording of the SIM card location. 
%As depicted in Fig. \ref{fig:simbox_mobility}, 
Notably, the automatic binding of a \simbox SIM card to gateways in different locations at various times creates human-like movements between network cells in CDRs at no cost to fraudsters.

Recent research~\cite{Kouam:2024} empirically examines the performance of \acrshort{HBS}-generated \simbox patterns compared to CDR-based \simbox activity detection. The results indicate that the current \acrshort{HBS} functionalities produce \simbox patterns that closely mimic human behavior, enabling them to evade detection by CDR-based methods with a high degree of success. 
%\alic{i know it is correct, but before was "allowing them to evade CDR-based existing detection with 100\% accuracy. " but for a ready it is just striking 100\% of accuracy. so, i changed. feel free to come back to the other form.}
\textit{This finding underscores the limitations of CDR-based approaches, which are insufficient to unmask all existing \simbox patterns.}
%A recent contribution on the subject~\cite{Kouam:2024} empirically studies the performance of such \acrshort{HBS}-generated \simbox patterns against CDR-based \simbox activity detection. 
%The results show that currently implemented \acrshort{HBS} functionalities produce \simbox patterns that closely mimic human behavior, allowing them to evade CDR-based existing detection with 100\% accuracy. 
%\textit{This finding attests CDR-based approaches are not enough to unmask all existing \simbox patterns.}

\vspace{0.13cm}
\greybox{
\noindent \textbf{Insight:} \textit{The wide adoption of \acrshort{HBS} techniques in the \simbox ecosystem limits the effectiveness of existing network-core-based \simbox activity detection, justifying the need for detection techniques tailored to the fraud evolution.}}


%The mimic of human communication behavior consists of thresholding the number and duration of initiated calls and controlling the corresponding timing. In contrast, reproducing human mobility for \simbox gateways is less straightforward, as such appliances can be bulky and require a fixed wired internet connection, making them inherently stationary. Fraudsters leverage \textit{remote SIM card association} that enables the binding of a SIM card to a distant gateway (i.e., with cellular antennas), resulting in an erroneous network recording of the SIM card location. Therefore, as depicted in Fig. \ref{fig:simbox_mobility}, binding a SIM card to gateways in different locations during different daytime periods induces human-like movements between network cells in CDRs at no cost to fraudsters.
%Moreover, even  if possible, manual movements of \simbox gateways (e.g., in a car) would be detected with high accuracy by CDR-based approaches as cluster movements of SIM cards~\cite{Murynets:2014}.

%Accordingly, \textit{the use of HBS techniques is a default practice for fraudsters, explaining why \simbox fraud continues to plague}. For instance, GoAntiFraud~\cite{goantifraud} is a cloud-based service assisting \simbox termination businesses. Since 2013, they have helped over 2000 fraudsters in more than 31 countries~\cite{goantifraud:blog}. Their top-2 most viewed article~\cite{goantifraud_fraud_success} indicates "\textit{5 Efficient Ways to Bypass AntiFraud systems}," highlighting the necessity for realistic human mobility and communication behavior simulation they propose as a service.

% \vspace{0.13cm}
% \greybox{
% \noindent \textbf{Insight:} \textit{The wide adoption of HBS techniques in the fraudulent ecosystem limits the accuracy of CDR-based \simbox fraud detection, justifying the need for detection techniques tailored to the fraud evolution.}}



%It gives fraudsters more flexibility for the realistic simulation of human behavior, making it more challenging to detect by literature detection methods. 
%For instance, GoAntiFraud~\cite{goantifraud} is a cloud-based service assisting \simbox termination businesses. Since 2013, they have helped over 2000 fraudsters in more than 31 countries.
%Their top-2 most popular article~\cite{goantifraud_fraud_success} indicates "\textit{5 Efficient Ways to Bypass AntiFraud systems}", highlighting the necessity to use a SIMBank and a SIM server to simulate human mobility. They further describe how to use \textit{remote SIM card association} to reproduce human mobility routine~\cite{goantifraud_mobility_simulation}.

%using a SIMBank %and a SIM server to mimic human mobility. They further describe how to use \textit{remote SIM card association} to reproduce human mobility routine efficiently~\cite{goantifraud_mobility_simulation}.
%To circumvent operator detection, fraudsters have developed various strategies and features known in the literature as Human Behavior Simulation (HBS)~\cite{Kouam:2021}. HBS techniques allow fraudulent SIM cards to be indistinguishable from legitimate ones in traffic and mobility. For instance, regarding traffic, fraudsters limit the number and duration of their initiated calls and control the timing of these calls. Regarding mobility, as illustrated in Fig. \ref{fig:simbox_mobility}, they use \textit{remote SIM card association} to allocate a SIM card with gateways in different locations, inducing movements between network cells in operators' logs.

\subsection{Network-edge-based detection}%strength and impracticality}
\label{subsec:online}


Network-edge-based detection methods operate at each base station within the cellular network, monitoring activity in real-time to detect \simbox patterns. Unlike network-core-based solutions, these methods often face scalability challenges that affect their practical efficiency. Specifically, existing solutions analyze either \textit{call audio} or \textit{cellular signaling data}.
% Network-edge-based detection contributions operate at each base station within the cellular topology, monitoring network activity in real-time to identify patterns indicative of a \simbox. Unlike network-core-based solutions, these methods can impact network performance if scalability is not adequately addressed in their design. Unfortunately, existing solutions in this category often fail to meet this challenge, affecting their practical efficiency. As discussed below, these solutions analyze \textit{call audio} and \textit{cellular signaling data}.}


% First, \textit{call audio-based} solutions use techniques that examine speakers' voices~\cite{Elrajubi:2017} or call quality~\cite{Reaves:2015} to detect local calls terminated through a \simbox. Despite their proven efficiency in laboratory settings~\cite{Reaves:2015}, the real-world deployment of these solutions requires investigating all local calls across the network. This not only raises concerns regarding operators' access to their subscribers' phone conversations but also poses notable scalability issues.
First, \textit{call audio-based} solutions, which examine speakers' voices~\cite{Elrajubi:2017} or call quality~\cite{Reaves:2015}, have demonstrated efficiency in lab settings but face real-world deployment challenges. Investigating all local calls across the network raises privacy concerns and significant scalability issues.



In contrast, Oh et al. \cite{Beomseok:2023} leverages \textit{cellular signaling data} for \simbox detection, proposing a fingerprinting-based approach, referred to as \textit{ACLPrint}. This method compares the device's fingerprint and factory identifier code (i.e., \acrshort{TAC}) to a pre-established database, rejecting devices with mismatched fingerprints.
However, \textit{ACLPrint} faces significant scalability issues. Frequent updates to the relying 3GPP LTE specifications (cf. Fig. \ref{fig:spef_updates}), which occur roughly every three months, require constant manual monitoring of extensive documents and their numerous references, along with adjustment of the fingerprinting process. Additionally, \textit{ACLPrint} relies on a pre-established database, making it vulnerable to new, unrecorded \acrshort{ME} models and brute-force attacks, where fraudsters modify their \simbox identifiers until they find a bypass. This also implies each network base station maintains and regularly consults such a vast database, complicating deployment.

These findings highlight the scalability challenges of \textit{ACLPrint} and similar methods, limiting their real-world efficiency.
% Second, to the best of our knowledge, the study by Oh et al. \cite{Beomseok:2023} is the only one that leverages \textit{cellular signaling data} for \simbox activity detection, similar to \sign. The authors propose a fingerprinting-based methodology to prevent \simbox gateways from accessing the network.
% In this approach, the fingerprint of each \acrshort{ME} model is empirically determined through a manual analysis of the network-requested \acrshort{ME} features during the \acrshort{UE}'s attach procedure, as specified in LTE standards. Consequently, at each attach procedure, the methodology extracts and compares the attaching \acrshort{ME} model's fingerprint and factory identifier code (i.e., \acrshort{TAC}) to those recorded in a pre-established database. If there is a mismatch, the attaching device is rejected. For simplicity, we use the term "\textit{ACLPrint}" to refer to this approach in the following discussion.
% \jos{
% Similar to call-audio-based solutions, \textit{ACLPrint} fails to provide the scalability necessary for practical real-world efficiency. Indeed, its setup is complex and requires constant manual monitoring with cellular network expertise, as discussed in the following:}
% \begin{itemize}[leftmargin=*]
%     \vspace{-0.75em}
%     \item \jos{The \textit{ACLPrint} fingerprint is based on 3GPP LTE specifications, specifically, \cite{NAS_3gpp} for NAS and \cite{RRC_3gppp} for RRC. In Fig. \ref{fig:spef_updates}, we report that these specifications undergo frequent updates, averaging every three months over a five-year period. Unfortunately, such a constant evolution directly impacts \textit{ACLPrint}'s scalability.
%     Indeed, \textit{ACLPrint}'s first phase involves a manual review of these specifications to extract the required information from \acrshort{ME}s. This process would entail, at each specification update, thorough searches for newly introduced modifications across extensive documents and their numerous references. Furthermore, the existence of multiple concurrent releases of specifications allows devices the flexibility to select which release to adhere to, thereby adding complexity to the implementation of \textit{ACLPrint} (cf. Fig. \ref{fig:rrc_updates}).}
%     \vspace{-0.75em}
%     \item \jos{Moreover, \textit{ACLPrint} relies on a pre-established database containing fingerprints of \acrshort{ME} models along with their corresponding identifiers (i.e., \acrshort{TAC}). Consequently, it is susceptible to unknown \acrshort{ME} models: if the identifier and fingerprint of an attaching \acrshort{ME} model are not yet in the database, \textit{ACLPrint} cannot make a detection decision. This vulnerability allows fraudsters using new, unexplored \simbox models to evade detection by \textit{ACLPrint}. Similarly, fraudsters can execute brute-force attacks by attempting multiple modifications of their \simbox identifier (i.e., \acrshort{TAC}) until they find a value that is not stored in the database.}
%     \vspace{-0.75em}
% \end{itemize}

%Such findings underscore the practical challenges associated with deploying \textit{ACLPrint} in real-world scenarios
\vspace{0.13cm}
\greybox{
\noindent \textbf{Insight:}  \textit{The effectiveness of network-edge-based detection methods hinges on their scalability — their capacity to function across the entire network without compromising network performance. However, this challenge remains unmet in current literature contributions.}}


\begin{figure}
\begin{subfigure}{0.49\textwidth}
\centering
\includegraphics[scale=0.39]{Figures/nas_updates.pdf}
\caption{NAS spef. updates from 27-03-2019 to 03-01-2023.} 
\label{fig:nas_updates}
\end{subfigure}
\hfill
\begin{subfigure}{0.5\textwidth}
\centering
\includegraphics[scale=0.39]{Figures/rrc_updates.pdf}
\caption{RRC spef. updates from 19-02-2019 to 13-01-2023.} 
\label{fig:rrc_updates}
\end{subfigure}
\caption{NAS\cite{NAS_3gpp} and RRC\cite{RRC_3gppp} protocol specifications size. Number of pages (\#Page) in solid line. Number of references (\#Ref) in dashed line.}
\label{fig:spef_updates}
\vspace{-0.5cm}
\end{figure}

%\section{Threat model and defense goals}
\section{Establishing \sign grounds}
\label{sec:threat-model}
 

% \vspace{0.13cm}
% \bluebox{
% \noindent \textbf{\sign positioning.}
% \textit{This paper falls into the category of online detection and deals with the unresolved challenges from the literature. Introducing \sign, we effectively tackle advanced \simbox frauds (bypassing traditional CDR-based countermeasures) while ensuring users' privacy, robustness against fraudster evasion attempts, and enabling effortless adoption on a broad scale.}}

In this section, we overview our investigation standpoint. In \S \ref{subsec:threat_model}, we first outline the \sign threat model, defense objectives, and key insights. Then, in \S \ref{subsec:preliminaries}, we explain why signaling latency is central to the \sign methodology. Further discussions in \S \ref{subsec:network_attachment_focus} provide an overview of signaling procedures in LTE, justifying \sign focus on the network attachment procedure signaling.

%we explain the rationale for using signaling latency as the central element of the \sign methodology.}

%In this section, we give a concise overview of our investigation standpoint, first outlining in \S \ref{subsec:threat_model} \sign threat model, defense objectives, and key insights, and second, establishing in \S \ref{subsec:preliminaries} the rationale of using signaling latency as the central element of \sign methodology.

\subsection{Threat models and defense goals}
\label{subsec:threat_model}

\vspace{0.13cm}
\noindent\textbf{Threat model:}
\sign is designed in a complementary viewpoint to literature methodologies. It addresses "advanced \simbox activity" that is undetectable with CDR-based approaches due to the use of \acrfull{HBS} (cf. \S \ref{subsec:offline}), and with network-edge-based methods due to privacy/scalability limitations (cf. \S \ref{subsec:online}).

Therefore, we focus on detecting advanced \simbox patterns derived from HBS techniques implementation. As described in \S \ref{subsec:offline}, these patterns involve \textit{remote SIM card associations} to mimic human-like mobility behavior. We, therefore, assume that adversaries implement \textit{remote SIM card association} using a distributed \simbox architecture (cf. Fig \ref{fig:simbox_architecture}) with lawfully-issued local SIM cards held within a SIMBank.

%\jos{We do not consider \simbox activity not implementing \textit{remote SIM card association} since, as established in previous work~\cite{Kouam:2024}, such activity results in distinctive communication behaviors that are efficiently detected through existing literature contributions.}

\vspace{0.13cm}
\noindent\textbf{Defense objective:} 
%Our goal is the efficient and online prevention of advanced \simbox activity in a privacy-preserving and real-world applicable manner.  We introduce \sign, \textit{a network-edge-based \simbox activity detection methodology based on cellular signaling data.} 
% \sign aims to prevent fraudulent \simbox activity on the mobile network surface. This implies providing sufficient information to reliably identify network devices originating from a \simbox activity at the cellular edge. Importantly, this identification occurs before any fraudulent call, effectively preventing \simbox owners from gaining any financial advantage. \sign thus effectively  provide mobile operators with means to detect and regulate \simbox usage on their networks by implementing legal registration for legitimate \simbox operations (as exemplified in \cite{t-mobile}) while blocking undeclared usage.
%mobile operators to enforce effective network access control, requiring \simbox devices to be preregistered, a critical measure in thwarting unauthorized use while preserving lawful \simbox applications like in call centers.
Our goal is the efficient, online prevention of such advanced \simbox activity in a privacy-preserving and practical manner. We introduce \sign, \textit{a network-edge-based \simbox activity detection methodology based on cellular signaling data}. \sign aims to prevent fraudulent \simbox activity on the mobile network surface. This is done through the reliable identification of network devices with advanced \simbox patterns at the cellular edge \textit{before any fraudulent calls} are made. Such an early detection effectively \textit{prevents \simbox owners from gaining financial advantage}. \textit{\sign thus provides mobile operators with the means to detect and regulate \simbox usage, enforcing legal registration for legitimate operations while blocking undeclared usage.
}

We design \sign keeping in mind the open challenges of the network-edge-based \simbox fraud detection literature (cf. \S \ref{subsec:online}). To ensure high real-world relevancy, we establish the following requirements: (i) \textit{Privacy}: \sign should rely on network device features that operators can access without impeding privacy. (ii) \textit{Practicality}: \sign implementation should be scalable and require minimal, non-constant effort from operators for wide deployment on the network surface.


% \sign aims to prevent fraudsters' use of \textit{remote SIM card association}, making it impossible for them to reproduce human-like (mobility) behavior. We stress that apart from the \simbox, no other user end device (smartphone, tablets, laptops, IoT devices, modems, etc.) separates the SIM card from the \acrfull{ME} during network operations, attesting \textit{remote SIM card association is a proxy for \simbox activity}.  It detects attaching \acrshort{UE}s' type at the cellular edge, i.e., \textit{integrated UE/IUE} (from local SIM association) or \textit{decoupled UE/DUE} (from remote SIM association), and intercepts \textit{DUEs}, thus enforcing \simbox devices to be preregistered at the network operator to get network access. 


\vspace{0.13cm}
\noindent\textbf{Key insights:} 
As advanced \simbox patterns result from carefully crafted communications and movements to mimic human behavior, providing precise online detection indicators for mitigation at the cellular edge is a genuine challenge.

Our approach to addressing this challenge builds upon the indicator of advanced \simbox activity: \textit{remote SIM card association}. Specifically:
\begin{enumerate}[leftmargin=*]
%\vspace{-0.7em}
\item \textit{Indispensability of \textit{remote SIM card association}}: \textit{Remote SIM card association} is essential for \simbox to mimic human behavior (cf. \S \ref{subsec:offline}). Previous work~\cite{Kouam:2024} establishes that \simbox activity without \textit{remote SIM card association} results in distinctive communication behaviors efficiently detected through existing methods. %\alic{can we add this last insight from asiaCCS in the introduction? it is essential.}
%\vspace{-0.5em}
\item \textit{Uniqueness of \textit{remote SIM card association} to \simbox}: No other user end device (smartphones, tablets, laptops, IoT devices, modems, etc.) separates the SIM card from the \acrfull{ME} during network operations, making \textit{remote SIM card association} an explicit proxy for advanced \simbox activity.
%\vspace{-0.5em}
\end{enumerate}

Henceforth, by detecting the use of \textit{remote SIM card association}, \sign effectively controls advanced \simbox activity and prevents any malicious usage.
\sign analyzes cellular signaling to determine if an attaching device is either \textit{coupled} or \textit{decoupled} via a \simbox-operated \textit{remote SIM card association}.
%\alic{attention, by saying legitimate, automatically you say the other is not-legitimate, again. i think it is the legitimate word that causes problem. in sec 2.2, I think we have to come back to the discussion of the introduction that "SIMBOX appliances may have legitimate uses, even the decoupled ones. non?? but that are the decoupled that are used to fraud.  }
This approach follows the intuition that signaling messages from \simbox-decoupled devices exhibit higher latency compared to coupled devices, as below.

%Our extensive empirical studies validate the significance of signaling latency for preventing advanced \simbox activity, enabling privacy-preserving and readily applicable mitigation in real-world operator networks.

\subsection{Preliminaries}
\label{subsec:preliminaries}
% This section establishes the rationale of using signaling latency as the central element of \sign methodology. It describes how it meets identified goals, i.e., %privacy-friendly, 
% \jos{privacy and practicality.}
% Then, it provides an overview of signaling procedures that sets \sign focus on network attachment signaling.

\begin{figure}
    \centering
    \includegraphics[width=0.6\linewidth]{Figures/sign_latency_new.pdf}
    \caption{Signaling latency of coupled and \simbox-decoupled devices.}
    \label{fig:sig_latency}
    \vspace{-0.7cm}
\end{figure}

Standard devices in cellular networks are a combination of a \acrfull{ME} and a SIM card integrated within the \acrshort{ME}, as depicted in Fig. \ref{fig:sig_latency}: \textit{it is a coupled ME-to-SIM combination}. 
They thus present a coupled signaling latency $L_{co}$, corresponding to the interaction time between the base station and the \acrshort{ME}. \
%alic{here you completly excluded the simbox-coupled devices as legitimate?}

On the other hand, \simbox-decoupled devices make a \textit{logical IP-based} binding of a GSM module (inside the gateway operating as the \acrshort{ME}) to a SIM card (inside the SIMBank) done at the level of the \textit{control server}: \textit{it is a decoupled ME-to-SIM combination.} 
%For the \simbox \acrshort{UE}s, the combination of the \acrshort{ME} (gateway) and SIM cards is either \textit{manual} in the standalone \simbox architecture or \textit{logical} at the level of the SIM server in the distributed architecture. Yet, most \simbox deployments are distributed due to the limitations induced by \simbox standalone architecture described below. 
Accordingly, the signaling latency of a \simbox-decoupled device, i.e., $L_{simbox}$, includes:% has two components:
\begin{itemize}
    %\vspace{-0.5em}
    \item $L_{simbox}^{(1)}$ corresponding to the interaction time between the base station and the gateway (i.e., \acrshort{ME} of the \simbox-decoupled device), and
    %\vspace{-0.5em}
    \item $L_{simbox}^{(2)}$ corresponding to the interaction time between the gateway (i.e., ME) and the \simbox-decoupled device's SIM card inside the SIMBank
    %\vspace{-0.5em}
\end{itemize}
\noindent such that $L_{simbox} = L_{simbox}^{(1)} + L_{simbox}^{(2)}$.
%Hence, as depicted in Fig. \ref{fig:sig_latency}, the latency of a legitime device's signaling operation, i.e., $L_{leg}$, is approximately equal to $L_{simbox-1}$. \alic{not clear why!! say the sim card of simbox1 is inside the gateway} 

\vspace{0.2cm}
Therefore, compared to coupled devices' signaling latencies $L_{leg}$, \simbox-decoupled devices' signaling latency $L_{simbox}$,  will tend to be larger due to component $L_{simbox}^{(2)}$ involving one or more exchanges of \simbox components over the Internet during the signaling operation.

%However, the total signaling latency for fraudulent UEs $L_{simbox} = L_{simbox-1} + L_{simbox-2}$ is increased by the latency $L_{simbox-2}$ due to the \textit{interaction of \simbox components} over the Internet, during the signaling operation.

%In contrast, for \simbox \acrshort{UE}s
% Therefore, as shown in Fig. \ref{fig:sig_latency}, the signaling latency of \simbox \acrshort{UE}s  $L_{simbox} = L_{simbox-1} + L_{simbox-2}$ where $L_{simbox-1} \approx L_r^{(0)} \approx L_r^{(1)}$ is the regular latency of the signaling operation within the network cell. On the other hand, $L_{simbox-2}$ is the latency overhead experienced by the \simbox due to the \textit{interaction of its components} over the Internet, during the signaling operation.

%\sign identifies such latency overhead, $L_{simbox-2}$, during the network attach procedure at the base station.
\sign methodology aims to identify such latency overhead, at the base station, to distinguish between coupled devices and \simbox-decoupled ones. It, therefore, fulfills our mitigation requirements as follows: 
\begin{enumerate}[leftmargin=*]
%\vspace{-0.5em}
\item Privacy: Mobile operators have a natural access to signaling latency measurements as these do not relate to any specific individual or device model's identifier or data content and, thus, are not privacy-impeding. %Accordingly, operators can have a natural access to this data type.
%\item Beyond fraudsters' reach: Fraudsters can hardly prevent their latency overhead in signaling. This is caused by parameters out of their reach such as the standardized size of signaling messages, the imposed number of signaling exchanges between the \acrshort{ME} and the SIM card, or Internet network and transport layers' protocols. Also, uncontrollable vagaries of the Internet link quality between \simbox appliances may cause latency.
%factors difficult to control such as vagaries related to Internet link quality between \simbox appliances may cause latency.
%\vspace{-0.5em}
\item Practicality: Inspecting signaling latency is straightforward and is already implemented in LTE using \textit{timers}. This practical approach ensures ease of deployment at no additional cost across the entire network edge.
%\vspace{-0.5em}
\end{enumerate}

\begin{table}
\centering
\begin{minipage}{\linewidth}
\caption{Summary of signalling procedure analysis}
\label{tab:signaling-proc}
\resizebox{\columnwidth}{!}{%
\begin{tabular}{l|l|l|l}
\hline
\textbf{\begin{tabular}[c]{@{}l@{}}Signaling \\ procedure\end{tabular}}    & \textbf{Description}                                                                                                 & \textbf{\begin{tabular}[c]{@{}l@{}}\#device\\ processing\end{tabular}} & \textbf{\begin{tabular}[c]{@{}l@{}}Moment of occurrence\end{tabular}}                                                                                                        \\ \hline
\begin{tabular}[c]{@{}l@{}}Network \\ attachment\end{tabular}              & \begin{tabular}[c]{@{}l@{}}Device connection and \\ authentication to the \\ network\end{tabular}                    & 4                                                                      & \begin{tabular}[c]{@{}l@{}}- At the device power on\\ - Device mobility dependent\\ - Network initiated\end{tabular} \\ \hline
Handover (X2)                                                              & \begin{tabular}[c]{@{}l@{}}Direct device connection's\\ transfer between network\\ base stations\end{tabular}        & 1                                                                      & Device mobility dependent                                                                                            \\ \hline
Handover (S1)                                                              & \begin{tabular}[c]{@{}l@{}}Device connection's transfer\\ between base stations via \\ the core network\end{tabular} & 1                                                                      & Device mobility dependent                                                                                            \\ \hline
CQI update                                                                 & \begin{tabular}[c]{@{}l@{}}Device's information to the\\  network of channel quality\end{tabular}                    & 0                                                                      & \begin{tabular}[c]{@{}l@{}}Network dependent \\ (periodic or aperiodic)\end{tabular}                                 \\ \hline
\begin{tabular}[c]{@{}l@{}}Data bearer \\ establishment\end{tabular}       & \begin{tabular}[c]{@{}l@{}}Setting up data\\  transmission channel\end{tabular}                                      & 2                                                                      & \begin{tabular}[c]{@{}l@{}}Device communication \\ dependent\end{tabular}                                            \\ \hline
\begin{tabular}[c]{@{}l@{}}Mobile originated \\ SMS signaling\end{tabular} & \begin{tabular}[c]{@{}l@{}}Texts transmission from\\ the device to the network\end{tabular}                          & 1                                                                      & \begin{tabular}[c]{@{}l@{}}Device communication\\ dependent\end{tabular}                                             \\ \hline
\begin{tabular}[c]{@{}l@{}}CS Fallback\\ call setup\end{tabular}      & \begin{tabular}[c]{@{}l@{}}Establishment of a\\ traditional voice call circuit\end{tabular}                          & 7                                                                      & \begin{tabular}[c]{@{}l@{}}Device communication \\ dependent\end{tabular}                                            \\ \hline
\end{tabular}%
}
\end{minipage}
\end{table}

\subsection{Focusing on the network attachment}
\label{subsec:network_attachment_focus}


LTE standards provide several signaling procedures to deliver communication services to network devices.
Aiming to distinguish \simbox-decoupled devices by their latency overhead, we analyze the most common of such signaling procedures (cf. Table \ref{tab:signaling-proc}) based on two criteria:
First, their \textit{ability to involve device's processing}, i.e., the number of device processing necessarily occurring during the signaling procedure (\textit{\#device processing}), that maximizes the latency checking possibilities to detect latency anomalies of \simbox-decoupled devices; 
Second, their \textit{moment of occurrence} indicating when and how often a latency checking can done and whether such checking depends on the network or on the device behavior. 

Our investigation relies upon the related 3GPP specifications and 
%Unfortunately, LTE specifications are often limited concerning the first criterion (i.e., ME-to-SIM interactions), as they do not always distinguish between the ME and the SIM card but consider the network device as a whole. To bridge this gap, we make the simplistic assumption that any network request processing by the network device involves an ME-to-SIM interaction. 
reports in Table \ref{tab:signaling-proc} the uncovered \textit{\#device processing} and \textit{moment of occurrence} per signaling procedure. We make the following observations:
\vspace{-0.1cm}
\begin{itemize}[leftmargin=*]
    %\vspace{-0.5em}
    \item The number of device processing varies from one procedure to another, indicating that procedures with greater values are more suitable for our \simbox activity detection goal. For instance, CQI updates, though fully network-controlled, do not involve any device processing, therefore, not allowing to uncover \simbox latency overhead.
    %\vspace{-0.5em}
    \item Concerning the moment of occurrence, signaling procedures are triggered either by the device's communication or mobility behavior or by the network itself. Network-initiated or mandatory procedures are more relevant to guarantee minimal interference by fraudsters. For instance, 
    %handovers are only executed when a device moves from one network cell to another, and 
    data bearer establishment are only executed when a device starts a mobile data session, that can be avoided by fraudsters. 
    %\vspace{-0.5em}
\end{itemize}

Based on these insights, \textit{the network attachment procedure is the optimal choice for building \sign}, as it incurs a sufficient number of device processing compared to other signaling procedures. This procedure is mandatorily carried out by all network devices (coupled or \simbox-decoupled) when they connect to an operator network upon being powered on.
It, therefore, enables the implementation of a  network access control that prevents any \simbox activity-induced damage. Furthermore, it can be triggered by the operator (as a Tracking Area Update), independently of the device's behavior, to increase attempts to detect \simbox activity.

In the following steps, we carry out an in-depth empirical study of the network attachment signaling latency to assess if this metric is satisfactory in distinguishing between coupled and \simbox-decoupled devices. %and provides precise information for \sign implementation in a real-world operator network.


\section{Latency empirical study}
\label{sec:latency_detection}


% \begin{figure*}
%     \centering
%     \includegraphics[width=0.75\linewidth]{Figures/methodology-1.pdf}
%     \caption{\sign attachment latency analysis methodology.\josc{To enlarge/verbreiten and increase text size}}
%     \label{fig:methodology}
% \end{figure*}

\begin{figure*}
  \centering
  \includegraphics[width=\linewidth]{Figures/methodology_long.png}
  \caption{\sign attachment latency analysis methodology.}
  \label{fig:methodology}
\end{figure*}

In this section, we want to evaluate how well the network attachment signaling latency (referred to as \textit{attachment latency}, for simplicity) can be used to differentiate  between coupled and \simbox-decoupled devices through experimental studies. To this end, we answer the following questions: 
%\begin{enumerate}[label=\[\textbf{Q\arabic*}\],leftmargin=*]
\begin{enumerate}
  %\vspace{-0.5em}
  \item[$\textbf{[Q1]}$] \textit{How different is the attachment latency between coupled and \simbox-decoupled devices?} %And where is this difference observed? } 
  %\vspace{-0.5em}
  %\item[$\textbf{[Q2]}$] \textit{How well do fraudsters have a reach on the observed \simbox-decoupled devices' latency, allowing them to reduce it and thereby preclude distinction? }
  \item[$\textbf{[Q2]}$] \textit{What factors explain the attachment latency of \simbox-decoupled devices, and is this latency reliable?}
  %\vspace{-0.5em}
  %\item[$\textbf{[Q3]}$] \textit{How well can the observed coupled devices' latency approximate the one of \simbox-decoupled devices under certain conditions and thereby preclude distinction?}
  \item[$\textbf{[Q3]}$] \textit{What factors influence the attachment latency of coupled devices, and how might these variations compare to the latency observed in \simbox-decoupled devices?}
  %\vspace{-0.5em}
\end{enumerate}

Through extensive indoor and outdoor experiments, represented in Fig. \ref{fig:methodology}, we make important observations relatively to the previous questions, which are summarized as follows: 
%we thoroughly analyze network attachment signaling latency , following the methodology hereafter, also represented in Fig. \ref{fig:methodology}: % , t

\begin{enumerate}
  %\vspace{-0.5em}
  \item[$\textbf{[O1]}$] \textit{\simbox-decoupled devices generate at least $5\times$ more attachment latency compared to coupled ones %(i.e., standard phones),
  in particular during the \textit{authentication phase} where their minimum latency is  $23\times$ higher (cf. \S \ref{subsec:measurement}). }
  %\vspace{-0.5em}
  \item[$\textbf{[O2]}$] \textit{\simbox-decoupled devices' latency is induced by both (i) \simbox implementation and (ii) network protocols-imposed procedures (i.e., LTE standards and TCP/UDP correction and retransmission mechanisms). While the former allows for fraudsters improvement, the latter is beyond fraudsters reach and guarantee a minimum latency still $2\times$ higher than coupled devices, in the authentication phase} (cf. \S \ref{subsec:explainability}).
  %\vspace{-0.5em}
  \item[$\textbf{[O3]}$] \textit{Regardless of the wireless network channel conditions, coupled devices' latency, in the authentication phase, cannot reach high values comparable to the one of \simbox-decoupled devices (cf. \S \ref{subsec:tresholding}). Therefore, the authentication latency \textit{unambiguously separates} coupled and \simbox-decoupled devices.}
  %\vspace{-0.5em}
\end{enumerate}


%In an effort to validate the proposed detection approach, we thoroughly analyze network attachment signaling latency , following the methodology hereafter, also : % , to build the \sign detection approach. Precisely, \sign focus on the \acrshort{UE}s' latency during the network attach \alic{1st time you mention network attachment from \acrshort{UE}, you mention signaling, authentication, etc. the differences and which step are you focusing, have to be clear before this section. attach or network attachment?} procedure, that fraudsters should mandatorily carry out to connect to the network.



\subsection{Experimental Setup}
\label{subsec:experimental_setup}

% To ensure our study remains in compliance with regulations and avoids any interference with live operator networks, we have designed a unique, high-performance testbed that accurately simulates a real-world 4G cellular network. This setup is powered by Amarisoft's professional suite, a trusted service provider for Mobile Network Operators (MNOs), and is housed within a $30m^2$ Faraday shield. This configuration guarantees that we generate results that reflect what would be expected in a commercial network environment. 
To ensure our study complies with regulations and avoids interference with live operator networks, we have designed a high-performance testbed that accurately simulates a real-world 4G cellular network. This setup utilizes Amarisoft's professional suite, a trusted solution in the wireless industry, and is housed within a  $30m^2$ Faraday shield. Amarisoft's software-based technology is fully compliant with 3GPP standards and compatible with off-the-shelf hardware, including the physical layer~\cite{amarisoftWebsite}. With over 1,000 customers in more than 60 countries, including numerous public and private network operators, Amarisoft's solutions are widely adopted for both laboratory and field applications. This widespread adoption underscores the reliability and accuracy of Amarisoft's technology in replicating authentic network environments. Consequently, the signaling we capture in our testbed mirrors the exact procedures employed in operational networks, ensuring that our results are both precise and reflective of real-world conditions.

Our testbed employs a single PC to run both the base station and core network nodes, including the \acrshort{MME}, \acrshort{IMS}, and \acrshort{SGW}. This PC handles baseband processing, while radio processing is managed by a software-defined \acrshort{USRP} B210 connected to the PC, enabling seamless integration of the baseband and SDR systems. Detailed specifications of all testbed components, including the featured 4G cell and its radio parameters, are provided in Table \ref{tab:testbed} in the appendix. Signal quality, specifically \acrshort{RSRP}, has been rigorously validated using a radio spectrum analyzer, showing excellent performance (around -71 dBm) consistent with real-world urban network conditions as documented in recent studies~\cite{Krawczeniuk:2019} (cf. Fig. \ref{fig:spectrum0}).

Our setup includes 12 mobile phones from five different vendors and 7 \simbox devices from two manufacturers—Hybertone, the leader in the \simbox market~\cite{goantifraud_top5}, and Portech. These devices are equipped with programmable SIM cards~\cite{sysmocom}, ensuring they connect seamlessly to the LTE network inside the shield. Notably, Hybertone \simbox appliances support both TCP and UDP protocols, while Portech devices only support UDP. %(cf. Table \ref{tab:simbox_review} in the appendix).
% To avoid any law-forbidden interference with the deployed operator networks, 
% we setup a unique testbed for this study that relies on an emulated 4G cellular network, i.e., the Amarisoft suite, deployed inside an $30m^2$ Faraday shield. The specifications of used components are detailed in Table \ref{tab:testbed} in appendix. We use a single PC to host the base station and core network nodes (\acrshort{MME}, \acrshort{IMS}, and \acrshort{SGW}). This PC is in charge of the baseband processing, while the (software-defined) radio processing is handled by a \acrshort{USRP} B210 connected to the PC. %through an USB3 interface. The baseband processing communicates with \acrshort{SDR} via a libuhd 4.3.0 TRX API. 

% The testbed deploys a single 4G cell with radio parameters described in Table \ref{tab:testbed}. Using a radio spectrum analyzer, we validate the signal quality (i.e., \acrshort{RSRP}) inside the Faraday shield is of excellent quality (around -71dBm) as in real urban scenarios measurements~\cite{Krawczeniuk:2019} (cf. Fig. \ref{fig:spectrum0} in appendix). 
% %shows the \acrfull{RSRP} at the level of \acrshort{UE}s to be around -71dBm (cf. Fig. \ref{fig:spectrum0}) reflecting an excellent signal quality. 

% Our testbed includes 12 phones from 5 distinct vendors and 7 \simbox appliances from two manufacturers, i.e., Hybertone being the Top-1 of the \simbox market~\cite{goantifraud_top5} and Portech. % (see Table \ref{tab:testbed}). 
% Such devices are equipped with programmable SIM cards~\cite{sysmocom}, which are set to connect to the LTE network inside the Faraday shield.  
% Note that Hybertone \simbox appliances support TCP and UDP, while Portech ones only supports UDP. %(cf. Table \ref{tab:simbox_review}, in appendix).

% \simbox control server software is hosted in LAN-connected Ubuntu-Server and Windows 7 PCs according to Hybertone and Portech requirements. Such PC hardware specifications are described in Table \ref{tab:testbed}.
As illustrated in Fig. \ref{fig:methodology}, step 1, we use the \simbox appliances of both manufacturers to deploy (i) \textit{remote SIM card association} and (ii) \textit{local SIM card association}. 
The \textit{remote SIM card associations} a \simbox control server hosted on LAN-connected PCs (cf. Table \ref{tab:testbed}).
% Therefore, the terms \textit{SMBHyb\_rem} and \textit{SMBPor\_rem} refer to \textit{remote SIM card association}-originated devices of Hybertone and Portech manufacturers, respectively. They are \simbox-decoupled devices.
% Similarly, the terms \textit{SMBHyb\_loc} and \textit{SMBPor\_loc} refer to \textit{local SIM card association}-originated devices of Hybertone and Portech manufacturers, respectively. Such coupled devices may be fraudulent but only give place to an explicit \simbox activity, out of the scope of this research (cf. \S \ref{subsec:threat_model}). 
Thus, \textit{SMBHyb\_rem} and \textit{SMBPor\_rem} refer to devices using \textit{remote SIM card association} from Hybertone and Portech, respectively, which are \simbox-decoupled. Likewise, \textit{SMBHyb\_loc} and \textit{SMBPor\_loc} refer to devices using \textit{local SIM card association} from the same manufacturers. These coupled devices, though potentially fraudulent, fall outside the scope of this research, as they involve already-addressed \simbox activity (cf. \S\ref{subsec:threat_model})




\subsection{Measuring attachment latency}
\label{subsec:measurement}

Here we collect and analyze, for all the coupled and \simbox-decoupled devices of our testbed, the latency at each step of network attachment procedure (cf. Fig. \ref{fig:methodology}, step 2). 
%the   presents a statistical analysis of the attachment latency computed for each of our testbed's device. 
We first detail the methodology for latency collection and computation and then present and discuss the obtained results.


\vspace{0.13cm}
\noindent\textbf{Methodology.}
For each network device (i.e., phone model, \simbox coupled and decoupled devices), we carry out 50 executions of the network attachment procedure. The resulting cellular signaling logs are recorded at the level of the base station.
We consider only NAS-layer logs, as they provide information on the signaling between the device and the core network during the network attachment. These logs consist of 11 messages, as represented in Fig. \ref{fig:methodology}, step 2. %in a format following LTE specifications (e.g., ASN.1 for RRC~\cite{RRC_3gppp})
For each message we use the following associated fields for latency computation:  
\verb|time, layer, direction, device_id, message|. %where the fields \verb|rnti| (Radio Network Temporary Identifier), \verb|sfn| (System Frame Number), and \verb|channel| are optional.
The communication direction, i.e., uplink or downlink, indicates the message originator as the device or the network, respectively. 
Therefore, we compute the latency of each message as $L_i = T_i - T_{i-1}$, i.e, the delta time between the arrival of a message $i$ and its previous one $i-1$. Depending on the message direction (uplink/downlink), $L_i$ refers to the network's or the device’s processing time along with the message transmission time to the base station. The total network attachment latency of a network device is thus $\sum_{i=1}^{n} T_i - T_{i-1}$ with $n=10$ steps (cf. Fig. \ref{fig:methodology}, step 2).


\vspace{0.13cm}
\noindent\textbf{Results.}
Table \ref{tab:detailed_latency} reports the obtained latency's mean and standard deviation values for each step of the network attachment. A particular interest is on the lines with \textit{uplink} direction, %(column 2 text in red),
enabling us to determine and compare the processing time per network device. We make the following observations:
%Note that some devices (e.g., \textit{SMBPOR}) may not execute the identity request if their attach request already includes the device's identifier (IMSI). Similarly, the network determines the \acrshort{ESM} information request inclusion in case of the need for additional information about the device, such as its supported \acrshort{EPS} bearers or QoS parameters. 
\begin{itemize}[leftmargin=*]
  \item  Regardless of the model, all phones have comparable latencies per step, similar to the \simbox devices resulting from \textit{local SIM association}. However, \textit{distinguishing from coupled devices, the attachment latency for \simbox-decoupled devices %of the two \simbox constructors
  is significantly higher i.e., $\approx 9\times$ for Hybertone and $\approx 5\times$ times for Portech.}
  \item \textit{Such latency distinction of \simbox-decoupled devices emerges at step 4 (authentication response)}, which consists of mutual authentication of the network and device, following the \acrshort{AKA} procedure~\cite{3gpp_aka}. 
  The authentication phase involves a computation internal to the SIM card (in the remote SIMBank) and, therefore, necessarily imputes IP-based interactions between \simbox components, explaining the overhead. Particularly in the authentication phase, \simbox-decoupled devices show a latency approximately $29\times$ (for Hybertone) and $23\times$ (for Portech) higher than coupled devices' latency.
\end{itemize}

\greybox{\noindent\textbf{Insight.} \textit{The previous results spotlight the authentication phase as the primary source of latency distinction for \simbox-decoupled devices during the network attachment. Henceforth, we narrow the following investigations to 
understand such authentication latency.}}
\vspace{-0.5em}



\begin{table*}[]
\centering
\caption{Latency (in ms) per device model reported per network attachment step}
\label{tab:detailed_latency}
\resizebox{\textwidth}{!}{%
\begin{tabular}{|ll|l|l|l|l|l|l|l|l|l|l|>{\columncolor{lightred}}l|l|>{\columncolor{lightred}}l|}
\hline
\multicolumn{1}{|l|}{\textbf{Step}} &
\textbf{Direction} &
\textbf{\begin{tabular}[c]{@{}l@{}}\rotatebox[origin=c]{90}{\makecell{Fair\\Phone5G}}\end{tabular}} &
\textbf{\rotatebox[origin=c]{90}{\makecell{Galaxy\\A90}}} &
\textbf{\rotatebox[origin=c]{90}{\makecell{Galaxy\\Note4}}} &
\textbf{\rotatebox[origin=c]{90}{\makecell{Galaxy\\S3}}} &
\textbf{\begin{tabular}[c]{@{}l@{}}\rotatebox[origin=c]{90}{\makecell{Galaxy\\ZFold25G}}\end{tabular}} &
\textbf{\begin{tabular}[c]{@{}l@{}}\rotatebox[origin=c]{90}{\makecell{OnePlus\\Nord}}\end{tabular}} &
\textbf{\begin{tabular}[c]{@{}l@{}}\rotatebox[origin=c]{90}{\makecell{Sony\\XPERIA}}\end{tabular}} &
\textbf{\begin{tabular}[c]{@{}l@{}}\rotatebox[origin=c]{90}{\makecell{Xiaomi10\\Lite5G}}\end{tabular}} &
\textbf{\begin{tabular}[c]{@{}l@{}}\rotatebox[origin=c]{90}{\makecell{Xiaomi9\\Pro5G}}\end{tabular}} &
\textbf{\begin{tabular}[c]{@{}l@{}}\rotatebox[origin=c]{90}{\makecell{SMBHyb\\\_loc}}\end{tabular}} &
\textbf{\begin{tabular}[c]{@{}l@{}}\rotatebox[origin=c]{90}{\makecell{SMBHyb\\\_rem}}\end{tabular}} &
\textbf{\begin{tabular}[c]{@{}l@{}}\rotatebox[origin=c]{90}{\makecell{SMBPor\\\_loc}}\end{tabular}} &
\textbf{\begin{tabular}[c]{@{}l@{}}\rotatebox[origin=c]{90}{\makecell{SMBPor\\\_rem}}\end{tabular}} \\ \hline
\multicolumn{1}{|l|}{\textbf{0. Attach request}} &
{\color[HTML]{C00000} Uplink} &
0 &
0 &
0 &
0 &
0 &
0 &
0 &
0 &
0 &
0 &
0 &
0 &
0 \\ \hline
\multicolumn{1}{|l|}{\textbf{1. Identity request}} &
{\color[HTML]{000000} Downlink} &
1$\pm$ 0 &
1$\pm$ 0 &
1 $\pm$ 0 &
\begin{tabular}[c]{@{}l@{}}0.9$\pm$\\ 0.3\end{tabular} &
1$\pm$ 0 &
1$\pm$ 0 &
1$\pm$ 0 &
1$\pm$ 0 &
1$\pm$ 0 &
1 $\pm$ 0 &
\begin{tabular}[c]{@{}l@{}}0.9 $\pm$ \\ 0.2\end{tabular} &
/ &
/ \\ \hline
\multicolumn{1}{|l|}{\textbf{2. Identity response}} &
{\color[HTML]{C00000} Uplink} &
31$\pm$0 &
27$\pm$6 &
\begin{tabular}[c]{@{}l@{}}38.3 $\pm$ \\ 2.3\end{tabular} &
\begin{tabular}[c]{@{}l@{}}31.0$\pm$\\ 10.4\end{tabular} &
31$\pm$0 &
\begin{tabular}[c]{@{}l@{}}31.8$\pm$\\ 2.4\end{tabular} &
\begin{tabular}[c]{@{}l@{}}25.0$\pm$\\ 6.4\end{tabular} &
31$\pm$ 0 &
31$\pm$ 0 &
\begin{tabular}[c]{@{}l@{}}31.8 $\pm$ \\ 3.5\end{tabular} &
\begin{tabular}[c]{@{}l@{}}31.0  $\pm$ \\ 4.3\end{tabular} &
/ &
/ \\ \hline
\multicolumn{1}{|l|}{\textbf{\begin{tabular}[c]{@{}l@{}}3. Authentication\\  request\end{tabular}}} &
Downlink &
1$\pm$0 &
1 $\pm$ 0 &
\begin{tabular}[c]{@{}l@{}}1.0 $\pm$ \\ 0.3\end{tabular} &
1$\pm$0 &
1$\pm$0 &
1$\pm$ 0 &
1$\pm$ 0 &
1$\pm$ 0 &
1 $\pm$ 0 &
1 $\pm$ 0 &
\begin{tabular}[c]{@{}l@{}}0.9 $\pm$\\  0.3\end{tabular} &
\begin{tabular}[c]{@{}l@{}}0.9$\pm$\\  0.1\end{tabular} &
1  $\pm$ 0 \\ \hline
\multicolumn{1}{|l|}{\textbf{\begin{tabular}[c]{@{}l@{}}4. Authentication\\  response\end{tabular}}} &
{\color[HTML]{C00000} Uplink} &
\begin{tabular}[c]{@{}l@{}}57.6 $\pm$\\ 11.4\end{tabular} &
\begin{tabular}[c]{@{}l@{}}74.1 $\pm$\\ 22.1\end{tabular} &
\begin{tabular}[c]{@{}l@{}}84.5 $\pm$ \\ 36.5\end{tabular} &
\begin{tabular}[c]{@{}l@{}}67.9$\pm$\\ 12.2\end{tabular} &
\begin{tabular}[c]{@{}l@{}}70.2$\pm$\\ 18.2\end{tabular} &
\begin{tabular}[c]{@{}l@{}}69.8 $\pm$\\ 10.0\end{tabular} &
\begin{tabular}[c]{@{}l@{}}69.1$\pm$\\ 5.9\end{tabular} &
\begin{tabular}[c]{@{}l@{}}69.9$\pm$\\ 8.2\end{tabular} &
\begin{tabular}[c]{@{}l@{}}67.9$\pm$\\ 16.2\end{tabular} &
\begin{tabular}[c]{@{}l@{}}71.7 $\pm$ \\ 10.8\end{tabular} &
{\color[HTML]{C00000} \textbf{\begin{tabular}[c]{@{}l@{}}2122.7$\pm$ \\ 309.9\end{tabular}}} &
\begin{tabular}[c]{@{}l@{}}71.2$\pm$\\ 10.7\end{tabular} &
{\color[HTML]{C00000} \textbf{\begin{tabular}[c]{@{}l@{}}1640.2$\pm$\\ 286.7\end{tabular}}}\\ \hline
\multicolumn{1}{|l|}{\textbf{\begin{tabular}[c]{@{}l@{}}5. Security mode\\  command\end{tabular}}} &
Downlink &
1$\pm$0 &
1$\pm$ 0 &
1$\pm$0 &
1$\pm$0.1 &
1$\pm$0.1 &
1$\pm$ 0 &
1$\pm$ 0 &
1$\pm$ 0 &
1 $\pm$ 0 &
1 $\pm$ 0 &
\begin{tabular}[c]{@{}l@{}}0.9 $\pm$ \\ 0.3\end{tabular} &
1  $\pm$ 0 &
1 $\pm$ 0 \\ \hline
\multicolumn{1}{|l|}{\textbf{\begin{tabular}[c]{@{}l@{}}6. Security mode\\  complete\end{tabular}}} &
{\color[HTML]{C00000} Uplink} &
\begin{tabular}[c]{@{}l@{}}20.5$\pm$\\ 3.2\end{tabular} &
\begin{tabular}[c]{@{}l@{}}19.3$\pm$ \\ 1.6\end{tabular} &
\begin{tabular}[c]{@{}l@{}}37.0$\pm$ \\ 6.3\end{tabular} &
\begin{tabular}[c]{@{}l@{}}33.0$\pm$\\ 9.5\end{tabular} &
\begin{tabular}[c]{@{}l@{}}21.8$\pm$\\ 14.3\end{tabular} &
\begin{tabular}[c]{@{}l@{}}31.3$\pm$\\ 12.5\end{tabular} &
\begin{tabular}[c]{@{}l@{}}19.6$\pm$\\ 2.6\end{tabular} &
\begin{tabular}[c]{@{}l@{}}21.9$\pm$\\ 4.6\end{tabular} &
\begin{tabular}[c]{@{}l@{}}21.8$\pm$\\ 10.5\end{tabular} &
\begin{tabular}[c]{@{}l@{}}22.4 $\pm$ \\ 5.9\end{tabular} &
\begin{tabular}[c]{@{}l@{}}20.1$\pm$ \\ 3.7\end{tabular} &
\begin{tabular}[c]{@{}l@{}}19 .7 $\pm$\\  2.7\end{tabular} &
\begin{tabular}[c]{@{}l@{}}21.1$\pm$\\ 5.8\end{tabular} \\ \hline
\multicolumn{1}{|l|}{\textbf{\begin{tabular}[c]{@{}l@{}}7. ESM information\\  request\end{tabular}}} &
Downlink &
1$\pm$0 &
1 $\pm$ 0 &
\begin{tabular}[c]{@{}l@{}}0.9 $\pm$\\ 0.2\end{tabular} &
/ &
1 $\pm$0 &
1$\pm$ 0 &
1$\pm$ 0 &
1$\pm$0 &
\begin{tabular}[c]{@{}l@{}}0.9$\pm$\\ 0.1\end{tabular} &
1 $\pm$ 0 &
1.0$\pm$ 0 &
/ &
/ \\ \hline
\multicolumn{1}{|l|}{\textbf{\begin{tabular}[c]{@{}l@{}}8. ESM information\\  response\end{tabular}}} &
{\color[HTML]{C00000} Uplink} &
19$\pm$0 &
\begin{tabular}[c]{@{}l@{}}19.7 $\pm$\\ 2.6\end{tabular} &
\begin{tabular}[c]{@{}l@{}}37.3 $\pm$ \\ 5.5\end{tabular} &
/ &
\begin{tabular}[c]{@{}l@{}}22.8$\pm$\\ 21.4\end{tabular} &
\begin{tabular}[c]{@{}l@{}}26.2$\pm$ \\ 9.3\end{tabular} &
\begin{tabular}[c]{@{}l@{}}19.6$\pm$\\ 2.3\end{tabular} &
\begin{tabular}[c]{@{}l@{}}22.6$\pm$\\ 5.6\end{tabular} &
\begin{tabular}[c]{@{}l@{}}20.7$\pm$ \\ 4.2\end{tabular} &
\begin{tabular}[c]{@{}l@{}}22.9 $\pm$ \\ 5.8\end{tabular} &
\begin{tabular}[c]{@{}l@{}}20.6 $\pm$\\ 3.9\end{tabular} &
/ &
/ \\ \hline
\multicolumn{1}{|l|}{\textbf{9. Attach accept}} &
Downlink &
\begin{tabular}[c]{@{}l@{}}50.4 $\pm$\\ 4.8\end{tabular} &
\begin{tabular}[c]{@{}l@{}}48.7$\pm$\\ 2.5\end{tabular} &
\begin{tabular}[c]{@{}l@{}}66.2 $\pm$\\ 6.8\end{tabular} &
\begin{tabular}[c]{@{}l@{}}56.9 $\pm$\\ 8.7\end{tabular} &
\begin{tabular}[c]{@{}l@{}}50.0$\pm$\\ 4.4\end{tabular} &
\begin{tabular}[c]{@{}l@{}}66.5 $\pm$\\ 14.3\end{tabular} &
\begin{tabular}[c]{@{}l@{}}50.9$\pm$\\ 5.9\end{tabular} &
\begin{tabular}[c]{@{}l@{}}48.8$\pm$\\ 3.9\end{tabular} &
\begin{tabular}[c]{@{}l@{}}49.3$\pm$ \\ 4.3\end{tabular} &
\begin{tabular}[c]{@{}l@{}}46.9 $\pm$ \\ 10.3\end{tabular} &
\begin{tabular}[c]{@{}l@{}}43.7$\pm$\\ 9.3\end{tabular} &
\begin{tabular}[c]{@{}l@{}}50.7$\pm$ \\ 6.8\end{tabular} &
\begin{tabular}[c]{@{}l@{}}57.9 $\pm$ \\ 26.1\end{tabular} \\ \hline
\multicolumn{1}{|l|}{\textbf{10. Attach complete}} &
{\color[HTML]{C00000} Uplink} &
\begin{tabular}[c]{@{}l@{}}32.4$\pm$\\ 1.9\end{tabular} &
\begin{tabular}[c]{@{}l@{}}32.8$\pm$ \\ 3.4\end{tabular} &
\begin{tabular}[c]{@{}l@{}}49.7 $\pm$ \\ 7.1\end{tabular} &
\begin{tabular}[c]{@{}l@{}}60.1$\pm$\\ 1.1\end{tabular} &
\begin{tabular}[c]{@{}l@{}}34.3$\pm$\\ 6.0\end{tabular} &
\begin{tabular}[c]{@{}l@{}}35.5 $\pm$\\ 8.2\end{tabular} &
\begin{tabular}[c]{@{}l@{}}54.5$\pm$\\ 6.4\end{tabular} &
\begin{tabular}[c]{@{}l@{}}38.8$\pm$ \\ 3.9\end{tabular} &
\begin{tabular}[c]{@{}l@{}}33.5$\pm$\\ 4.1\end{tabular} &
\begin{tabular}[c]{@{}l@{}}57.3 $\pm$ \\ 10.1\end{tabular} &
53.2 $\pm$ 9.5 &
\begin{tabular}[c]{@{}l@{}}78.5$\pm$ \\ 6.8\end{tabular} &
\begin{tabular}[c]{@{}l@{}}52.2$\pm$ \\ 4.7\end{tabular} \\ \hline
\multicolumn{2}{|l|}{\textbf{Total}} &
\begin{tabular}[c]{@{}l@{}}215.0$\pm$\\ 21.3\end{tabular} &
\begin{tabular}[c]{@{}l@{}}225.6$\pm$\\ 38.2\end{tabular} &
\begin{tabular}[c]{@{}l@{}}316.8$\pm$\\ 65.5\end{tabular} &
\begin{tabular}[c]{@{}l@{}}251.9$\pm$\\ 42.4\end{tabular} &
\begin{tabular}[c]{@{}l@{}}234.2$\pm$\\ 64.6\end{tabular} &
\begin{tabular}[c]{@{}l@{}}265.1$\pm$\\ 56.9\end{tabular} &
\begin{tabular}[c]{@{}l@{}}242.8$\pm$\\ 29.5\end{tabular} &
\begin{tabular}[c]{@{}l@{}}237.1$\pm$\\ 33.5\end{tabular} &
\begin{tabular}[c]{@{}l@{}}228.2$\pm$\\ 38.5\end{tabular} &
\begin{tabular}[c]{@{}l@{}}257.1$\pm$\\ 42.7\end{tabular} &
{\color[HTML]{C00000} \textbf{\begin{tabular}[c]{@{}l@{}}2295.5$\pm$\\ 341.7\end{tabular}}} &
\begin{tabular}[c]{@{}l@{}}253.9$\pm$\\ 31.3\end{tabular} &
{\color[HTML]{C00000} \textbf{\begin{tabular}[c]{@{}l@{}}1773.5$\pm$\\ 323.3\end{tabular}}} \\ \hline
\end{tabular}
\vspace{-1cm}
}
\end{table*}


%\subsection{\simbox-decoupled devices' authentication latency: Internal interactions vs. standards }
\subsection{Decoupled devices' authentication latency}
\label{subsec:explainability}


%We pursue experimental investigations of \simbox-decoupled devices' latency during the authentication (i.e., Table \ref{tab:detailed_latency}, step 4) to uncover the cause of the latency overhead induced by \textit{remote SIM card association} and \jos{how steady it is for a distinction}. 
This section investigates the latency introduced by \simbox-decoupled devices during authentication (i.e., Table \ref{tab:detailed_latency}, step 4) to uncover the cause and consistency of this overhead due to \textit{remote SIM card association}. 
%To this end, we first scrutinize, as a baseline, ME-to-SIM interactions during the authentication as imposed by 3GPP standards. Then, we perform a similar authentication latency breakdown for a \simbox-decoupled device. 
To this end, we capture ME-to-SIM interactions during authentication for both a coupled device, serving as a baseline reflecting 3GPP standards, and a \simbox-decoupled device. Then by comparing these interactions, we explain decoupled devices' authentication latency, classifying its sources as either (i) specific to \simbox implementation  or (ii) imposed by standards and protocols.
%Finally, by comparing both outcomes, we explain fraudulent authentication latency, classifying its sources as either (i) specific to \simbox implementation  or (ii) imposed by standards and protocols.

\subsubsection{Methodology}

First, we describe the experimental process for capturing ME-to-SIM interactions during authentication for both a coupled device and a \simbox-decoupled device.

\vspace{0.13cm}
\noindent\textbf{Coupled devices.} Aiming to capture standardized \acrshort{ME}-to-SIM interactions during the authentication phase, we separate a coupled device's SIM card and radio processing, similarly to \textit{remote SIM card association}: As depicted in Fig. \ref{fig:methodology}, step 3 we set up a coupled network device combining
(i) a srsUE softphone, i.e., a 4G phone implemented entirely in software, running on a Linux-system PC and connecting to the shielded LTE network, (ii) a  physical SIM card within a SIM card reader connected to the softphone through an USB interface, and (iii) physical cellular antennas handled by a connected software-defined radio system (i.e., \acrshort{USRP} B210). We then perform the network attachment of the formed device and align two sets of resulting timestamped logs: (i) signaling logs at the base station and (ii) SIM card logs at the softphone.
\textit{The use of the srsUE softphone, developed in the widely adopted srsRAN 4G framework~\cite{srs4gdoc}, in our experiments attests to its generality and fidelity to 4G/LTE standards. Hence, our experiments thus provide insights into the actual implementation of 3GPP standards for the authentication phase (i.e., the \acrshort{AKA} procedure~\cite{3gpp_aka}), publicly available in \cite{srsUE_code}.}

\vspace{0.13cm}
\noindent\textbf{\simbox-decoupled devices.} 
Such devices disconnect the \acrshort{ME} (i.e., the \simbox gateway) from the SIM card (within the SIMBank), causing ME-to-SIM interactions to occur as packet exchanges over an IP network (cf. Fig \ref{fig:sig_latency}).
In order to capture these packet exchanges, we perform the network attachment of a \simbox-decoupled device and monitor packets at the control server-hosting PC using Wireshark. This setup enables us to gain insights into the interactions between the SIMBank and the gateway, which are proprietary \simbox appliances and thus typically concealed. Transport protocols (i.e., TCP or UDP) ensure reliability, order, and flow control in these IP-based interactions. We noted variations in the number of packets exchanged depending on the transport protocol used. By correlating the timing of these packets with network attachment signaling logs collected at the base station, we make specific observations for each transport protocol.




\subsubsection{Observations}
\label{subsubsec:fraud_device_obs}
% Aiming to capture standardized \acrshort{ME}-to-SIM interactions during the authentication phase, we separate a coupled device's SIM card and radio processing, similarly to \textit{remote SIM card association}: As depicted in Fig. \ref{fig:methodology}, step 3 we set up a coupled network device combining
% (i) a srsUE softphone, i.e., a 4G phone implemented entirely in software, running on a Linux-system PC and connecting to the shielded LTE network, (ii) a  physical SIM card within a SIM card reader connected to the softphone through an USB interface, and (iii) physical cellular antennas handled by a connected software-defined radio system (i.e., \acrshort{USRP} B210). We then perform the network attachment of the formed device and align two sets of resulting timestamped logs: (i) signaling logs at the base station and (ii) SIM card logs at the softphone.
% \textit{The use of the srsUE softphone, developed in the widely adopted srsRAN 4G framework~\cite{srs4gdoc}, in our experiments attests to its generality and fidelity to 4G/LTE standards. Hence, our experiments thus provide insights into the actual implementation of 3GPP standards for the authentication phase (i.e., the \acrshort{AKA} procedure~\cite{3gpp_aka}), publicly available in \cite{srsUE_code}.}
From the previous experiments we make the following observations, summarized in Table \ref{tab:sim_me_interactions} in the appendix.
\begin{itemize}[leftmargin=*]
    \item First, confirming Table \ref{tab:detailed_latency} insights, authentication is the primary phase involving ME-to-SIM interactions, making it the best context for identifying any latency overhead. For coupled devices, these interactions occur \textit{only during authentication}, while \simbox-decoupled devices also show \textit{minor exchanges during the attach complete} phase. Specifically, TCP interactions involve 63 packets during authentication and 4 packets during the attach complete phase, while UDP interactions consist of 36 packets for authentication and 2 packets for attach complete (cf. Figs \ref{fig:hyb_tcp_interactions}, \ref{fig:hyb_udp_interactions}). 
    
    Shared by coupled and \simbox-decoupled devices, interactions during the authentication split into ME-to-SIM \textit{transfer} and \textit{processing} phases. Transfers consist of \textit{information transmission} from/to the ME/SIM card, while the processing phases are \textit{internal computations} within the ME/SIM card following these transfers.
    
    \item \textit{Transfers}: Logs from coupled devices reveal two physical layer round-trip transfers (four transfers in total) between the \acrshort{ME} (i.e., softphone) and the SIM card, following the ISO/IEC 7816-4 protocol~\cite{3GPP_SIM_Card} illustrated on Fig. \ref{fig:UE_internals}. These transfers are rapid, averaging 0.12 ms due to physical layer communication via \acrshort{UART} serial interfaces.\\
    In contrast, \simbox-decoupled devices show a significantly higher number of transfers. %While TCP transfers are more observable, UDP transfers are less noticeable due to their unordered nature. 
    They are observable with TCP while UDP's unordered nature makes them less distinguishable. Fig. \ref{fig:HYP_TCP_time_interactions} illustrates 15 transfer sessions during authentication, each involving four packets (in total 60 packets) exchanged between the \acrshort{ME} (i.e., gateway) and the SIM card in the SIMBank through the control server, which acknowledges and re-transmits the packets. On a local network, these transfers take on average 4.7 ms, totaling 70.6 ms, which 
    %accounts for only 2.1\% of the authentication latency and 
    is a lower bound compared to real-world \simbox deployments that are Internet-based.
    
    \textit{This comparison highlights that transfer latency is a consistent indicator for distinguishing \simbox-decoupled devices.} Specifically, the 4 transfers mandated by standards result in a latency at least 39$\times$ higher than that of coupled devices, and \textit{hardly controllable due to its dependence on (i) the transport protocol and (ii) Internet vagaries. Regarding the transport protocol, while TCP increases the number of packet exchanges, UDP poorly handles network congestion, leading to retransmissions and delays.} For instance, a comparison of the latency distribution over 50 authentications of \textit{SMBHyb\_rem} configured with TCP and UDP, as shown in Fig. \ref{fig:tcp_udp}, indicates that latency with UDP is significantly higher than with TCP. Additionally, \textit{internet vagaries further contribute to transfer latency overhead.} We estimate this overhead by performing network attachment with the control server online, showing an average additional latency of 460 ms, and by measuring the median RTT of internet communication within the same country based on an empirical distribution of 1000 RTTs (cf. Fig. \ref{fig:rtt_latency}). The median value of 57.4 ms suggests that\textit{ the 2 RTTs imposed by the standard guarantee a transfer latency overhead of 114.8 ms for \simbox-decoupled devices, which is almost twice the avg. auth. latency of coupled devices (cf. Table \ref{tab:detailed_latency}).}
    %\vspace{-0.5em}
    \begin{figure*}
    \centering
    \includegraphics[width=\linewidth]{Figures/HYB_simbox_ip_latency_legend.png}
    \caption{Hybertone \simbox components TCP interactions during the authentication phase.}
    \label{fig:HYP_TCP_time_interactions}
    \vspace{-0.3cm}
    \end{figure*}

     %\vspace{-0.7cm}
    \item \textit{Processing}-induced latency is much higher than transfer latency.
    Analysis of coupled devices' logs reveals two standard-imposed processing phases on the SIM card and three on the \acrshort{ME}, averaging 15.6 ms and 9.4 ms respectively (cf. Table \ref{tab:sim_me_interactions}).
    In contrast, \simbox-decoupled devices exhibit as many as 14 processing phases (8 for the SIM card and 6 for the \acrshort{ME}), averaging 218 ms and 211 ms respectively with TCP, and 12 (6 each for the SIM card and the \acrshort{ME}), averaging 236.1 ms and 139.3 ms respectively with UDP.
     
    \textit{This comparison highlights that the elevated processing latency in \simbox-decoupled devices is likely due to their implementation involving a higher number of processing phases than necessary and significantly longer average times. Although this could be optimized by \simbox manufacturers, some overhead may be inevitable due to the simultaneous control of multiple SIM cards and GSM modules, as well as the encapsulation/decapsulation of information exchanged during authentication into IP packets. This overhead is challenging to quantify for proprietary devices.}
\end{itemize}



\greybox{\noindent\textbf{Insight.} \textit{In essence, our investigations show that \simbox-decoupled devices' authentication latency is influenced by factors beyond \simbox owners' control. Even with optimization efforts, i.e., reducing the transfer count and processing time, this latency cannot match that of stripped-down, coupled devices, maintaining a consistent distinction between \simbox-decoupled devices and their coupled counterparts.} }

%\subsection{coupled devices' authentication latency:  Investigating network and external Factors}
\subsection{Coupled devices' authentication latency}
\label{subsec:tresholding}



Here, we assess the feasibility of instances where a coupled device's latency could be high enough to be mistaken for a \simbox-decoupled one. To this end, we scrutinize the latency of coupled devices during the authentication phase by breaking down a coupled device's auth. latency into two components:
%Here we inspect coupled devices' latency during the authentication phase, to uncover cases where a coupled device is identified as a \simbox-decoupled one if such cases are possible.
%To this end, we breaking down a coupled device's authentication latency into two parts:
(i) \textit{A transmission latency} that includes the wireless propagation time along with any delay related to the wireless network channel condition (\S \ref{subsubsec:transmission}) and 
(ii) \textit{A processing latency} in which the device locally runs the authentication algorithm until a response is generated (\S \ref{subsubsec:processing}). Note that these latencies differ from the ones detailed in \S \ref{subsubsec:fraud_device_obs}, which focused on intra-device interactions. The current context instead aims to analyze the latency in the device's communication with the network through the base station.

%\alic{if you do not agree with re-organizing the sections (merging baseline of 6.3.1 with this section and moving it to 6.3), here it would be important to make the difference with what was discussed before (baseline), and explain here it concerns the communication between UE-to-BS or SIM-to-BS or still ME-to-SIM-to-BS) and not a local (internal to the device) communication, ME-to-SIM discussed in 6.3.1. }. 

\subsubsection{Transmission latency of the authentication}
\label{subsubsec:transmission}

Transmission latency refers to the back-and-forth communication time between the device and the base station during an authentication request and response. 
Predicting this latency is challenging due to various factors that affect the quality of each device's experience. In particular, although the \textit{wireless signal propagation} time is negligible as the signal moves at the speed of light, and LTE employs an admission control mechanism~\cite{AdmissionControl} to prevent delays during the attachment caused by \textit{network congestion}, the impact of \textit{signal quality} on authentication latency remains to be determined.

Indeed, poor signal quality, indicated by an LTE \acrshort{RSRP} of less than -110 dBm, often results from significant distance to the cell tower, interference, or device sensitivity issues. This can cause re-transmissions and signaling delays. In the following, we assess this impact on authentication latency by measuring it in an outdoor network and generalizing the results in a controlled indoor testbed (cf. Fig. \ref{fig:methodology}, step 4).
% According to 3GPP requirements, such one-way transmission latency is a maximum of 100ms for LTE and 50ms for LTE-Advanced~\cite{3gpp_requirements_lte_a}. However, transmission latency remains challenging to predict due to several factors impacting the quality of each device's experience, notably \textit{wireless signal propagation}, \textit{network congestion} and \textit{signal quality}: 
% \begin{itemize}[leftmargin=*]
%     \item \textit{Wireless signal propagation}: The signal propagation component of this latency is \textit{negligible as the speed of electromagnetic signals in the air is essentially the speed of light, approximately 300,000 km/s.}
%     \item \textit{Network congestion} can cause signaling messages delay. %in case of high traffic demands.
%     However, LTE implements various mechanisms to prevent congestion while maintaining an acceptable quality of service for connected users, such as dynamic scheduling, fast resource allocation, load control, or admission control. Specifically, the admission control mechanism~\cite{AdmissionControl} limits the number of users connecting to a base station based on available resources and network capacity. 
%     %It constantly monitors and makes real-time updates of various parameters to determine the availability of resources, such as the number of users already accessing the network, the amount of bandwidth available, the quality of the radio channels, and the network capacity. 
%     Hence, a new user attach request will fail in the case of radio resource scarcity due to congestion 
%     %generating the "cause \#22" of attach reject, i.e., "Congestion," according to 3GPP specifications
%     ~\cite{3gpp_attach_reject}. \textit{Therefore, although generic signaling messages may suffer delay due to congestion, the network attachment required for a new user connection to the base station will not be executed in case of congestion, preventing high latency values.}
%     \item \textit{Signal quality} or strength measured in LTE networks by the \acrfull{RSRP} can also contribute to latency in LTE signaling. Indeed, a poor \acrshort{RSRP} (less than -110 dBm) indicates that the device experiences network signal attenuation due to a considerable distance to the cell tower, obstacles, interference, or any receiver sensitivity limitation of the device. This may lead to %physical-layer 
%     re-transmissions according to the implemented correction mechanism to improve wireless communication's reliability while inducing some signaling delay. \textit{Therefore, signal quality can influence the transmission latency during authentication.} 
%     \end{itemize}
\vspace{-0.13cm}
\paragraph{\textbf{Methodology.}} 
Regarding \textit{outdoor signal attenuation}, we measure the outdoor signal quality over three days with varying weather conditions (sunny, rainy, and windy) using a Samsung Galaxy Note4 (referred to as \textit{GalaxyNote4}) in a vehicle covering over 80km of city roads in a central urban area in the Paris region. As open phone signaling data is not public, we use the QCSuper open-source diagnostic logging tool~\cite{QCSuper}, which is compatible only with Qualcomm-based phones, to capture the network attachment signaling messages. %QCSuper is a real-time data collection and diagnostic logging tool communicating with Qualcomm-based phones to capture raw 4G radio frames. 
Data collection is done with a QCSuper-installed laptop connected to a \textit{GalaxyNote4} phone, the only compatible device in our testbed. 
The collected outdoor dataset comprises 2287 network attachments of the \textit{GalaxyNote4} phone. Each network attachment marks the phone's entrance into a new network cell %which could be a previously visited location, 
and induces an authentication process. From the collected logs, we extract signal quality measurements (i.e., \acrshort{RSRP}) within the respective cell for each network attachment.

To extend these findings across various phone models, we replicate the network \textit{indoor signal attenuation} in a controlled environment (cf. \S \ref{subsec:experimental_setup}) using static attenuators.
Specifically, we apply two attenuation levels to the initial network signal quality of -71 dBm (excellent quality), resulting in measurements of (i) -90 dBm (medium quality) and (ii) -100 dBm (poor quality) (cf. Fig. \ref{fig:spectrum}, in the appendix). 
For each signal quality condition, we conduct 50 network attachments for every phone model (described in Table \ref{tab:testbed}), including the \textit{GalaxyNote4} phone used in the outdoor measurement scenario, and record the corresponding authentication latency.

\vspace{-1em}
\paragraph{\textbf{Observations.}}  
%First, Fig. \ref{fig:galaxynote4} shows the distribution of the total attachment latency of the \textit{GalaxyNote4} phone in the described outdoor scenario, ranging from 400 to 800 ms. 
In Fig. \ref{fig:outdoor_latency_steps}, we break down the attachment latency distribution for the \textit{GalaxyNote4} in the outdoor scenario, detailing each step of the network attachment process. Since latency data is collected at the device level, our focus is on the downlink messages (UE←BS), which include the transmission latency of interest. Specifically, we examine the impact of signal quality (i.e., \acrshort{RSRP}) on the latency of the \textit{"security mode command"} message, which inherently captures the transmission latency of the \textit{"authentication response."}  To streamline interpretation, we approximate the \textit{"security mode command"} latency as the transmission latency of the authentication process in Fig. \ref{fig:outdoor_rsrp_security}. With signal quality measurements ranging from -65 to -119 dBm, our results cover all radio frequency conditions, from "Cell Edge" to "Excellent"~\cite{3GPP_RSRP_range}, ensuring the representativeness and depth of our findings.

Notably, Fig. \ref{fig:outdoor_rsrp_security} shows that the upper limit for the auth. response's transmission latency is negligible. Although some outliers around 200 ms, the majority of the values indicate low latency, irrespective of meteorological conditions (denoted by the days) or signal quality (denoted by the \acrshort{RSRP}). \textit{This result convincingly shows that signal quality has a minimal impact on authentication transmission latency. Furthermore, linear regression of latency and signal quality confirms this trend across a broader signal quality spectrum.} %, affirming the robustness of our results.}

Figure \ref{fig:attenuation_cage}, shows the authentication latency under indoor signal attenuation, extending our findings to other phone models. The results indicate that signal quality has a negligible impact on the authentication latency for the \textit{GalaxyNote4, GalaxyS3,} and \textit{GalaxyZFold5G}, confirming our earlier interpretations from outdoor scenarios. However, the six remaining phone models did not initiate network attachment under medium to high signal attenuation, likely due to the lower sensitivity of their receivers. In a carrier network, these phones would have connected to nearby cells with stronger signals, thereby avoiding any latency issues related to signal quality.
% Figure \ref{fig:attenuation_cage} reports the obtained network authentication latency values for each phone model, varying from 45 ms to a maximum of 120 ms.  
% It shows a negligible impact of the signal quality on the authentication latency of the \textit{GalaxyNote4, GalaxyS3}, and \textit{GalaxyZFold5G} phones, confirming our previous interpretations in outdoor scenarios. However, the six remaining phone models do not initiate any network attachment in case of medium to high signal attenuation due to the lower sensitivity of their receivers. In a carrier network, such phone models would have attached to surrounding cells with a stronger signal, thus preventing any latency related to signal quality.


\vspace{0.13cm}
\greybox{\noindent \textbf{Insight:} \textit{
Our findings demonstrate that the round-trip transmission latency during the authentication phase is negligible and robust, showing little sensitivity to fluctuations in network signal quality. Even in the worst-case scenario, poor signal quality will cause mobile devices to switch network cells rather than increase transmission latency.}}

\begin{table*}
\begin{minipage}{0.24\linewidth}
\centering
\includegraphics[width=\linewidth]{Figures/SIMBoxHYB_tcp_udp.pdf}
\captionof{figure}{\textit{SMBHYB\_rem} TCP vs UDP auth. latency.}
\label{fig:tcp_udp}
\end{minipage}
\hfill
\begin{minipage}{0.25\linewidth}
\centering
\includegraphics[width=0.95\linewidth]{Figures/RTT_distrib.pdf}
\captionof{figure}{RTT latency distribution over Internet}
\label{fig:rtt_latency}
\end{minipage}
\hfill
\begin{minipage}{0.48\linewidth}
\centering
\includegraphics[scale=0.45]{Figures/galaxyNote4OutdoorLatency.pdf}
\captionof{figure}{Latency (in log-scale) in different attachment steps for \textit{GalaxyNote4}'s outdoor scenario}
\label{fig:outdoor_latency_steps}
\end{minipage}
\end{table*}
%\vspace{-0.8cm}
\begin{table*}
% \begin{minipage}{0.4\linewidth}
% \centering
% \includegraphics[width=0.7\linewidth]{Figures/uplink_downlink.pdf}
% \captionof{figure}{Uplink vs Downlink messages' captured latency using QCSuper}
% \label{fig:up_vs_down}
% \end{minipage}
\begin{minipage}{0.32\linewidth}
\includegraphics[width=0.98\linewidth]{Figures/outdoor_rsrp_security.pdf}
\captionof{figure}{Outdoor transmission latency (in log-scale) of the authentication response w.r.t. the signal quality.}
\label{fig:outdoor_rsrp_security}
\end{minipage}
\hfill
\begin{minipage}{0.4\linewidth}
\centering
\includegraphics[width=0.94\linewidth]{Figures/latency_attenuation_legend.pdf}
\captionof{figure}{Indoor network authentication latency per phone model w.r.t. the signal quality.}
%\alic{why caption is so different from fig 14? why not ''Indoor transmission latency of the attachment w.r.t. to signal quality''?}}
\label{fig:attenuation_cage}
\end{minipage}
\hfill
\begin{minipage}{0.22\linewidth}
\centering
\includegraphics[width=\linewidth]{Figures/galaxyNote4AuthAlgo.pdf}
\captionof{figure}{\textit{GalaxyNote4}'s authentication latency variation w.r.t. the SIM authentication algorithm.}
\label{fig:auth_algo}
\end{minipage}
\end{table*}


\subsubsection{Processing latency of the authentication}
\label{subsubsec:processing}

%The processing latency refers to the authentication request's computation time inside the coupled device. With processing done in the operator-provided SIM card, processing latency is not substantial and should not vary much depending on the phone model. As the measurements in Table \ref{tab:detailed_latency} step 4 shows, this time maximum value is less than 84.5ms for the whole phone set, which is negligible compared to that of fraudulent devices. 
The processing latency denotes the time required for authentication computations within coupled devices, specifically within the SIM cards provided by the operator. As a result, it barely varies across different phone models or depends on phone features (i.e., processor, battery, or RAM), since modern phones are designed to handle much more resource-intensive applications.
Thus, the authentication processing latency is primarily determined by the \textit{SIM card  dauthentication algorithm}, which runs inside the SIM card to compute the expected network Authentication Response (RES). Chosen by each operator and kept secret to prevent account impersonation, these algorithms are typically variants of standardized algorithms like XOR~\cite{3GPP_Xor}, Milenage~\cite{ETSI_milenage}, or Tuak~\cite{3GPP_Tuak}.

We evaluate the impact of the standardized SIM card authentication algorithms on the authentication latency. Specifically, we configure inside our indoor testbed (cf. \S \ref{subsec:experimental_setup}) such different authentication algorithms on the \textit{GalaxyNote4} phone and conduct 50 network attachments for each algorithm, recording the resulting authentication latency values. The findings in Fig. \ref{fig:auth_algo} confirm that the authentication algorithm impacts both the processing latency and its variability. \textit{Thus, mobile operators can achieve lower processing latency by carefully selecting their SIM authentication algorithm.}

\section{SigN implementation}
\label{sec:real_world_deployment}
This section delves into utilizing the authentication latency metric for the practical implementation of \sign for \simbox activity detection at the mobile edge. 
%Our objective is to ensure \sign's lightweight design and its seamless operation at a large scale in real-time scenarios. To achieve this, we introduce an implementation approach centered around the monitoring of mobile devices' authentication latency. This monitoring process involves utilizing a LTE-standardized timer (T3460~\cite{3GPP_Auth_timer}), which \sign adapts to enhance its functionality.
%In the following discussion, we elaborate on how we leverage LTE timing mechanisms to develop a \sign prevention method tailored for advanced SIMBox fraud mitigation. We then discuss the efficiency of this approach and its effectiveness in countering fraud.

\vspace{0.13cm}
\noindent\textbf{Supporting insights.}
The experiments conducted in \S \ref{sec:latency_detection} underscored a \textit{significant distinction} in attachment latency between coupled and \simbox-decoupled devices, specifically \textit{during the authentication phase} (cf. Table \ref{tab:detailed_latency}). Acknowledging its variability influenced by internal/external factors, we further investigated the authentication latency of both coupled and \simbox-decoupled devices. First, \textit{coupled devices authentication latency in outdoor networks remains within a consistent range and is markedly lower than observed for \simbox-decoupled devices in indoor settings} (cf. Fig. \ref{fig:outdoor_latency_steps}).
Second, our investigations reveal that \textit{a non-negligible portion of the observed authentication latency in \simbox-decoupled devices is imposed by factors beyond fraudsters' control}, such as LTE standards and Internet-based communication protocols and vagaries (cf. \S \ref{subsec:explainability}). Moreover, like coupled devices, \textit{\simbox-decoupled devices experience additional transmission latency} in the real-life conditions of operator networks. Given these findings, \textit{\textbf{monitoring authentication latency at the network edge proves to be a reliable and practical method for distinguishing \simbox activity from regular one.}} %\alic{review: ...for distinguishing \simbox activity likely related to malicious uses.}
%\sign leverages this ascertainment as a strategy to prevent advanced \simbox frauds at the cellular edge, in \S \ref{sec:real_world_deployment}
%\subsection{\sign deployment strategy}

\vspace{0.13cm}
\noindent\textbf{Monitoring the authentication latency.} 
%In LTE, the authentication procedure starts when the network sends an "\textit{authentication request}" to the device, initiating a timer (T3460) with a default value of 6 seconds~\cite{3GPP_Auth_timer}. The device must respond with an "\textit{authentication response}" within this time frame. If the timer expires, the network retransmits the "\textit{authentication request}" and restarts the timer, repeating this process up to four times. On the fifth expiry, the network aborts the authentication procedure and terminates the network attachment.
% In particular, the device shall respond to the network's "\textit{authentication request}" message with an "\textit{authentication response}" message within T3460's time span. On the first expiry of the timer T3460, the network retransmits the "\textit{authentication request}" message as well as resets and starts the timer T3460. This retransmission is repeated four times. On the fifth expiry of timer T3460, the network aborts the authentication procedure and stops the network attachment. %releases the NAS signaling connection. 
%\noindent\textbf{\sign originality.} 
%To the best of our knowledge, LTE standards do not justify the default T3460 timer value of 6 seconds and allow operators to adjust it. Our experiments show that this default value is excessively high. Evidence includes the fact that phone authentication latency is typically in the millisecond range, even with poor signal quality (cf. \S \ref{subsubsec:transmission}), while \simbox-decoupled devices exhibit latencies from hundreds of milliseconds to 2.5 seconds, still below the 6-second threshold (cf. Table \ref{tab:detailed_latency}).
%These findings underscore the need to re-evaluate the T3460 timer configuration, which has primarily addressed network congestion but should also account for vulnerabilities related to \simbox activity.
% To the best of our knowledge, LTE standards give no grounds for the default T3460 value of 6s and allow any operator to modify it easily. 
% Our experiments' results strongly argue against the adequacy of this default timer value, which is significantly high. The evidence supporting this claim is twofold. First, our tests reveal that the authentication latency for phones is typically in the millisecond range, even under conditions of low signal quality (cf. \S \ref{subsubsec:transmission}). 
% Moreover, for some phone models, an excessively low signal quality triggers a new attachment to a neighboring cell tower, preventing authentication latency from increasing (cf. Fig. \ref{fig:attenuation_cage}).
% Second, \simbox-decoupled devices exhibit authentication latencies at least four times higher than phones, ranging from hundreds of milliseconds to 2.5 seconds, but still far under the 6s default value (cf. Table \ref{tab:detailed_latency}).
% This work alerts the critical importance of strategically configuring the timer T3460, until now primarily aimed at dealing with network congestion. Our findings underscore the need to consider additional factors when configuring this timer, especially in vulnerabilities against \simbox patterns detection.
% Precisely, our research sheds light on the vulnerability of the \textit{remote SIM card association} operation, an indispensable technique for advanced \simbox fraud generation. The extended authentication latency associated with this operation proves to be a robust weakness for fraudsters. Our conclusive results demonstrate its effectiveness as a distinctive factor for fraud detection.
%According to our experimental evidence, such a 6s default value is significantly high, facilitating the seamless attachment to the network of \simbox \simbox-decoupled devices.
In LTE, the authentication procedure initiates the monitoring of the induced latency through a logging mechanism. Consequently, the latency of each authentication is automatically logged at each base station, and is useful for network performance monitoring, optimization, and Quality of Service (QoS) management.

However, \textit{experiments in this paper highlight that authentication latency is not only an indicator of network QoS but also a robust metric for identifying vulnerabilities related to \simbox activity}. Our findings demonstrate that, while the latency is generally acceptable (i.e., within the 6s timer range~\cite{3GPP_Auth_timer}) for both coupled and \simbox-decoupled devices, it creates a clear distinction between \simbox activity and regular one. 
%\alic{no!! simbox activity of coupled is like regular one too. so: ...distiction between the activities of \simbox devices more likely to be used for fraudulent purposes. }

% However experiments of this paper alert the importance of the authentication latency not only as a sign of the network quality of service but also as a robust indice improve the network vulnerability related to \simbox activity. Indeed, our evidence show that though acceptable (i.e., below the 6s timer range) for both coupled and \simbox-decoupled devices, the authentication cree une distinction claire entre l'activités simbox et celle qui ne l'est pas. 
% show that this default value is excessively high. Evidence includes the fact that phone authentication latency is typically in the millisecond range, even with poor signal quality (cf. \S \ref{subsubsec:transmission}), while \simbox-decoupled devices exhibit latencies from hundreds of milliseconds to 2.5 seconds, still below the 6-second threshold (cf. Table \ref{tab:detailed_latency}).
% These findings underscore the need to re-evaluate the T3460 timer configuration, which has primarily addressed network congestion but should also account for vulnerabilities related to \simbox activity.
Therefore, \textit{\sign approach suggests a new monitoring usage of authentication latency in cellular networks, allowing the detection of \simbox activity through its distribution.} According to 3GPP standards, \textbf{network operators have the flexibility to initiate at any time an authentication procedure when a signaling connection with a device exists}~\cite{3GPP_Auth_timer}. This flexibility enables operators to capture the distribution of mobile devices' authentication latency by initiating multiple authentications at different times throughout the day. Randomly timed authentications are essential to accurately reflect a device’s behavior, as passive collection could be manipulated by \simbox operators who might switch between remote and local SIM card associations to skew the latency distribution.

Relying on the distribution rather than a single measurement is crucial for ensuring robustness against high-value outliers that may occur for coupled devices (cf. Fig. \ref{fig:outdoor_rsrp_security}). Hence, analyzing a full day’s data yields key metrics such as the mean, median, and standard deviation of authentication latencies, highlighting \textit{devices with consistently unusually high values and prompting further investigation by the operator.} %\alic{ discussion of last sentence has to appear at introduction and conclusion}

Such monitoring is lightweight, leveraging existing automatic functions in cellular networks, i.e., logs collected by each base station. As a result, it allows operators to make informed decisions without the network edge's overhead.

\vspace{0.13cm}
\noindent\textbf{Statistical support.} 
% We statistically support the efficiency of implementing \sign approach from our network measurements.
% Fig. \ref{fig:latency_pdf} shows the distribution of the authentication latency of (i) coupled devices with the outdoor transmission latency (ii) current measurements of \simbox-decoupled devices with the outdoor transmission latency and (iii) the most optimized fraudulent devices given by their local deployment plus the RTT distributions with the outdoor  transmission latency. 
% We perform a \textit{t-test}, a fundamental statistical tool, that assesses the true difference between the means of two groups: in our case, coupled devices and fraudulent ones, both current and optimized. 
% Details about the t-test operation are described in  in the appendix.
% Applying such statistical test to our data samples, we obtain a \textit{t-ratio} $=15.29$ from the values ($t=25.27$ and \textit{critical value} $=1.65$) which denotes a high statistical dissimilarity between the the coupled and the fraudulent groups. The \textit{t-test} also provides a \textit{p-value} of $1.3\times 10^{-102}$, indicating a near-0 likelihood of similarity between the two groups in the hypothesis test, drawn from a t-distribution table. 
% Therefore, there is a 1-probability of correctly detecting a \simbox-decoupled device attachment if it occurs, given the latency distribution. 
We statistically validate the effectiveness of the \sign approach from our network measurements. 
Fig. \ref{fig:latency_pdf} plots realistic distributions of authentication latency for: (i) coupled devices with outdoor transmission latency, (ii) Current measurements of \simbox-decoupled devices with outdoor transmission latency, and (iii) Optimized \simbox-decoupled devices (cf. \ref{subsubsec:fraud_device_obs}) consisting of \simbox coupled device with outdoor transmission latency and simulated reduced ME-to-SIM transfers (2 RTTs as on Fig. \ref{fig:rtt_latency}).
We employed a \textit{t-test}, a key statistical tool, to compare the means of coupled and \simbox-decoupled devices (both current and optimized). Details are provided in the appendix (cf. \S \ref{sec:ttest}). %\ali{We employed a \textit{t-test}, a key statistical tool, to compare the latency means of such devices.}

The \textit{t-ratio} of 15.29 (with $t=25.27$ and \textit{critical value} $=1.65$) reveals a significant statistical difference between the coupled and \simbox-decoupled device groups. The \textit{p-value} of $1.3\times 10^{-102}$ indicates an almost zero chance of overlap between the two groups and  
%This confirms a high probability of correctly identifying the most optimized \simbox-decoupled device attachments based on latency distribution. \alic{
a high probability of correctly identifying the attachment activity even of the most optimized \simbox-decoupled device. 

%, most likely for malicious usage.

\begin{figure}
    \centering
    \includegraphics[width=0.9\linewidth]{Figures/authentication_latency_PDF_unique.pdf}
    \caption{Distribution of authentication latency of (i) coupled devices (\textit{dashed}), (ii) current (\textit{bold label}) and (iii) optimized (\textit{italic label}) \simbox-decoupled devices }
    \label{fig:latency_pdf}
    \vspace{-0.6cm}
\end{figure}

\section{Limitations}
\label{sec:limitations}
Acknowledging the limitations of this study is crucial to contextualizing its findings and guiding future research efforts. While we believe our approach and results provide valuable insights into monitoring \simbox activity in mobile networks, there are areas where constraints, assumptions, and specific conditions may have influenced the outcomes.

First, due to the proprietary nature of \simbox devices, we were unable to perform an in-depth reverse engineering of their hardware and software. This limitation constrained our ability to fully uncover the specific implementation techniques used by \simbox manufacturers. Instead, we adopted a standards-based approach to evaluate potential baseline measures that fraudsters must adhere to. 
%For example, we considered whether a faster implementation of remote association authentication might be feasible using SIM card secrets directly on mobile equipment. 
While our approach sheds light on baseline security assumptions, future research could explore advanced logic analysis techniques to extract and study SIM card secrets from these devices, providing deeper insights into their operation.

Second, while our experimentation was conducted on a simulated 4G network powered by the Amarisoft suite, we acknowledge that this does not fully replicate the complexities of a live operator network. However, the testbed design and the use of Amarisoft's professional-grade software, widely trusted by Mobile Network Operators (MNOs), ensure results that closely approximate those observed in real-world conditions. Additionally, testing in a controlled environment allowed for precise measurements and analysis without the risk of interfering with live operator networks. Despite this, future validations on live networks could provide additional insights into practical deployment scenarios.

Lastly, the TCP and UDP latency measurements presented in Figure~\ref{fig:rtt_latency} are derived from a single network configuration, which may not fully capture the variability across different operator networks. While these measurements are representative and align with general expectations, they may overestimate latency in certain cases. We believe this limitation can be mitigated by adapting our detection method to each operator's specific network conditions through a straightforward calibration process. Future work should consider extending these measurements across diverse network environments to enhance the generalizability of our findings.

\vspace{-0.3cm}


\section{Conclusion}
\label{sec:conclude}
This paper introduced \sign, an online prevention solution to uncover \simbox activity at the cellular edge. Based on an empirical study of network attachment latency in coupled and \simbox-decoupled devices, we found that \simbox-decoupled devices exhibit higher authentication latency. \sign optimizes existing cellular monitoring, improving \simbox activity detection efficiency.

\sign's significance lies in its effectiveness and practicality, enabling easy integration into operator networks. This offers substantial economic benefits and resolves challenges faced by current network-edge solutions, making \sign a key advancement in securing networks against \simbox activity.

Note that while this paper measurements focus on smartphones, the findings apply to other devices with a physical or embedded SIM card, including tablets, laptops, and IoT devices. The key distinction is the separation of the SIM card from the Mobile Equipment in \simbox-decoupled devices.


\bibliographystyle{ACM-Reference-Format}
\bibliography{references}

\appendix
%\section{List of acronyms}
\newacronym{UART}{UART}{Universal Asynchronous Receiver/Transceiver} 
\newacronym{CDR}{CDR}{Call Detail Records}
\newacronym{GSM}{GSM}{Global System for Mobile Communications}
\newacronym{ESM}{ESM}{EPS Session Management}
\newacronym{EPS}{EPS}{Evolved Packet System}
\newacronym{AKA}{AKA}{Authentication and Key Agreement}
\newacronym{ME}{ME}{Mobile Equipment}
\newacronym{NAS}{NAS}{Non-Access Stratum}
\newacronym{UE}{UE}{User Equipment}
\newacronym{TAC}{TAC}{Type Allocation Code}
\newacronym{IMEI}{IMEI}{International Mobile Equipment Identity}
\newacronym{HBS}{HBS}{Human Behavior Simulation}
\newacronym{CA}{CA}{Carrier Aggregation}
\newacronym{SDR}{SDR}{Software-Defined Radio}
\newacronym{MME}{MME}{Mobility Management Entity}
\newacronym{IMS}{IMS}{IP Multimedia Subsystem}
\newacronym{SGW}{SGW}{Serving Gateway}
\newacronym{USRP}{USRP}{Universal Software Radio Peripheral}
\newacronym{RSRP}{RSRP}{Reference Signal Received Power}
\newacronym{PLMN}{PLMN}{Public Land Mobile Network}
\newacronym{AUTN}{AUTN}{Authentication Token}
\newacronym{XRES}{XRES}{Expected Response}
\newacronym{KASME}{KASME}{Key Agreement Key Stored in the \acrlong{ME}}
\newacronym{IMSI}{IMSI}{International Mobile Subscriber Identity}

\printglossary[type=\acronymtype]

% \section{Focusing on the network attachment}
% \label{sec:network_attachment_focus}

% LTE standards provide several signaling procedures to deliver communication services to network devices.
% Aiming to distinguish \simbox-decoupled devices by their latency overhead, we analyze the most common of such signaling procedures (cf. Table \ref{tab:signaling-proc}) based on two criteria:
% First, their \textit{ability to involve device's processing}, i.e., the number of device processing necessarily occurring during the signaling procedure (\textit{\#device processing}), that maximizes the latency checking possibilities to detect latency anomalies of \simbox-decoupled devices; 
% Second, their \textit{moment of occurrence} indicating when and how often a latency checking can done and whether such checking depends on the network or on the device behavior. 

% Our investigation relies upon the related 3GPP specifications and 
% %Unfortunately, LTE specifications are often limited concerning the first criterion (i.e., ME-to-SIM interactions), as they do not always distinguish between the ME and the SIM card but consider the network device as a whole. To bridge this gap, we make the simplistic assumption that any network request processing by the network device involves an ME-to-SIM interaction. 
% reports in Table \ref{tab:signaling-proc} the uncovered \textit{\#device processing} and \textit{moment of occurrence} per signaling procedure. We make the following observations:
% \begin{itemize}[leftmargin=*]
%     %\vspace{-0.5em}
%     \item The number of device processing varies from one procedure to another, indicating that procedures with greater values are more suitable for our \simbox activity detection goal. For instance, CQI updates, though fully network-controlled, do not involve any device processing, therefore, not allowing to uncover \simbox latency overhead.
%     %\vspace{-0.5em}
%     \item Concerning the moment of occurrence, signaling procedures are triggered either by the device's communication or mobility behavior or by the network itself. Network-initiated or mandatory procedures are more relevant to guarantee minimal interference by fraudsters. For instance, 
%     %handovers are only executed when a device moves from one network cell to another, and 
%     data bearer establishment are only executed when a device starts a mobile data session, that can be avoided by fraudsters. 
%     %\vspace{-0.5em}
% \end{itemize}

% Based on these insights, \textit{the network attachment procedure is the optimal choice for building \sign}, as it incurs a sufficient number of device processing compared to other signaling procedures. This procedure is mandatorily carried out by all network devices (coupled or \simbox-decoupled) when they connect to an operator network upon being powered on.
% It, therefore, enables the implementation of a  network access control that prevents any \simbox activity-induced damage. Furthermore, it can be triggered by the operator (as a Tracking Area Update), independently of the device's behavior, to increase attempts to detect \simbox activity.

% In the following steps, we carry out an in-depth empirical study of the network attachment signaling latency to assess if this metric is satisfactory in distinguishing between coupled and \simbox-decoupled devices. %and provides precise information for \sign implementation in a real-world operator network.


\section{T-Test Procedure and Statistical Analysis}
\label{sec:ttest}
% After computing groups' means ($\mu_l$ and $\mu_f$), groups' sizes ($n_l$ and $n_f$), and the pooled standard error of both groups ($SE$, cf. equation \ref{eq:se}), the t-statistic ($t$) computes the ratio of the difference of each group's mean over the pooled standard error of both groups as in equation \ref{eq:t}.  

% \begin{multicols}{2}
%   \begin{equation}
%   \label{eq:se}
%     SE = \sqrt{\left(\frac{\sigma_{l}^2}{n_{l}}\right) + \left(\frac{\sigma_{f}^2}{n_{f}}\right)}
%   \end{equation}\break
%   \begin{equation}
%   \label{eq:t}
%      t = \lvert \frac{\mu_{f} - \mu_{l}}{\sqrt{SE^2(\frac{1}{n_f}+\frac{1}{n_l})}} \rvert
%   \end{equation}
% \end{multicols}

% \noindent Differently from the classical standard deviation ($\sigma_l$ or $\sigma_f$), measuring the data dispersion within each group, the pooled standard error of the two groups ($SE$) takes into account the variability within each group to provide a more accurate estimate of the whole population standard deviation.
% A higher t-statistic value indicates a more significant difference between groups. To ascertain the statistical significance of this observed difference, the t-statistic is compared to a \textit{critical value}, typically determined based on a predetermined level of confidence, such as 95\%. 
% Suppose the computed t-statistic ($t$) exceeds the \textit{critical value} (i.e., the ratio of both metrics, or \textit{t-ratio} $\gg 1$). In that case, the disparity between the groups is significant and not attributable to random chance, thereby supporting the scientific validity of the findings.

To perform a t-test, we first calculate the means of the two groups ($\mu_l$ and $\mu_f$), their respective sizes ($n_l$ and $n_f$), and the pooled standard error ($SE$) of the two groups, as shown in equation \ref{eq:se}. The t-statistic ($t$) is then computed as the ratio of the difference between the group means to the pooled standard error, as detailed in equation \ref{eq:t}.

\begin{multicols}{2}
  \begin{equation}
  \label{eq:se}
    SE = \sqrt{\left(\frac{\sigma_{l}^2}{n_{l}}\right) + \left(\frac{\sigma_{f}^2}{n_{f}}\right)}
  \end{equation}\break
  \begin{equation}
  \label{eq:t}
     t = \lvert \frac{\mu_{f} - \mu_{l}}{\sqrt{SE^2(\frac{1}{n_f}+\frac{1}{n_l})}} \rvert
  \end{equation}
\end{multicols}

\noindent Unlike the classical standard deviation ($\sigma_l$ or $\sigma_f$), which measures dispersion within each group, the pooled standard error ($SE$) incorporates variability from both groups to provide a more accurate estimate of the population standard deviation.

A higher t-statistic value suggests a more significant difference between the groups. To determine the statistical significance of this difference, we compare the computed t-statistic to a \textit{critical value}, which is typically based on a predetermined confidence level (e.g., 95\%). If the computed t-statistic exceeds this critical value, it indicates that the observed difference is statistically significant and unlikely due to random chance, thereby supporting the validity of the findings.

\begin{table}
\begin{minipage}{\linewidth}
\centering
\includegraphics[width=0.8\linewidth]{Figures/UE_internals.pdf}
\captionof{figure}{coupled \acrshort{ME} to SIM card interactions during the authentication}
\label{fig:UE_internals}
\end{minipage}
\end{table}

\begin{figure}
\centering
\begin{subfigure}{0.5\textwidth}
\centering
\includegraphics[scale=0.5]{Figures/spectrum_basis.png}
\caption{Reference state: -71 dBm} 
\label{fig:spectrum0}
\end{subfigure}
\hfill
\begin{subfigure}{0.5\textwidth}
\centering
\includegraphics[scale=0.5]{Figures/spectrum_30db.png}
\caption{-90dBm RSRP attenuation} 
\label{fig:spectrum1}
\end{subfigure}
\hfill
\begin{subfigure}{0.5\textwidth}
\centering
\includegraphics[scale=0.5]{Figures/spectrum_60db.png}
\caption{-100dBm RSRP attenuation} 
\label{fig:spectrum2}
\end{subfigure}
\caption{\acrfull{RSRP} measurement  inside the testbed: x-axis: RSRP (dBm), y-axis: Frequency (Hz)}
\label{fig:spectrum}
\end{figure}




\begin{table*}
\centering
\caption{Testbed component specifications}
\label{tab:testbed}
\resizebox{\textwidth}{!}{%
\begin{tabular}{|ll|l|}
\hline
\multicolumn{2}{|l|}{\textbf{Parameters}}             & \textbf{Values}                                  \\ \hline
\multicolumn{2}{|l|}{\begin{tabular}[c]{@{}l@{}}Host PC (BS, MME, SGW)\end{tabular}} &
  \begin{tabular}[c]{@{}l@{}}Intel(R) Core(TM) i9-10900K CPU@3.70GHz, 16GB RAM, GB Ethernet controller\end{tabular} \\ \hline
\multicolumn{1}{|l|}{}         & Bandwidth            & 5MHz FDD                                         \\ \cline{2-3} 
\multicolumn{1}{|l|}{}         & Configuration        & SISO (Single Input Single Output)                                          \\ \cline{2-3} 
\multicolumn{1}{|l|}{\multirow{-3}{*}{Cell}} &
  Frequency &
  \begin{tabular}[c]{@{}l@{}}Downlink center frequency: 1845 MHz, Band 3\end{tabular} \\ \hline
\multicolumn{2}{|l|}{\begin{tabular}[c]{@{}l@{}}Programmable SIM cards\end{tabular}} &
  Sysmocom SysmoSIM-SJS1 \\ \hline
\multicolumn{2}{|l|}{}                                & Samsung Galaxy Note 4 (x3)                       \\ \cline{3-3} 
\multicolumn{2}{|l|}{}                                & Samsung Galaxy S3                                \\ \cline{3-3} 
\multicolumn{2}{|l|}{}                                & Xiaomi Redmi Note 9                              \\ \cline{3-3} 
\multicolumn{2}{|l|}{}                                & Xiaomi 10 Lite 5G (x2)                           \\ \cline{3-3} 
\multicolumn{2}{|l|}{}                                & FairPhone 4 5G                                   \\ \cline{3-3} 
\multicolumn{2}{|l|}{}                                & OnePlus Nord Model 5G                            \\ \cline{3-3} 
\multicolumn{2}{|l|}{}                                & Sony XPERIA                                      \\ \cline{3-3} 
\multicolumn{2}{|l|}{}                                & {\color[HTML]{333333} Samsung galaxy Z Fold2 5G} \\ \cline{3-3} 
\multicolumn{2}{|l|}{\multirow{-9}{*}{Mobile Phones}} & Samsung Galaxy A90 5G                            \\ \hline
\multicolumn{2}{|l|}{} &
  \begin{tabular}[c]{@{}l@{}}Hybertone\\ - SIMBank: SMB32 - Gateway: GoIP8 (x2)\\- Control server v. 2022-5-11 (Host PC:
  Intel(R) Core(TM) i5-4590 CPU @ 3.30GHz, 8GB RAM, GB Ethernet controller)\end{tabular} \\ \cline{3-3} 
\multicolumn{2}{|l|}{\multirow{-2}{*}{\simbox appliances}} &
  \begin{tabular}[c]{@{}l@{}}Portech\\- SIMBank: SBK-32 - Gateway: MV-374\\- Control server SS-128 (Host PC:
    Intel(R) Core(TM) i7-4610M CPU @ 3.00GHz, 16GB RAM, GB Ethernet controller)  \end{tabular} \\ \hline
    \end{tabular}%
}
\end{table*}


\begin{table*}[]
\centering
\caption{Finegrained analysis of authentication latency (in ms)}
\label{tab:sim_me_interactions}
\resizebox{\textwidth}{!}{%
\begin{tabular}{ll|cllcll|cll|cll|}
\cline{3-14}
\multicolumn{2}{l|}{} &
  \multicolumn{6}{c|}{\textit{SMBHyb\_rem}} &
  \multicolumn{3}{c|}{\textit{SMBPor\_rem}} &
  \multicolumn{3}{c|}{} \\ \cline{3-11}
\multicolumn{2}{l|}{\multirow{-2}{*}{}} &
  \multicolumn{3}{c|}{\textit{TCP}} &
  \multicolumn{3}{c|}{\textit{UDP}} &
  \multicolumn{3}{c|}{\textit{UDP}} &
  \multicolumn{3}{c|}{\multirow{-2}{*}{\textit{srsUE softphone}}} \\ \hline
\multicolumn{1}{|l|}{\textbf{Step}} &
  \textbf{Dir.} &
  \multicolumn{1}{c|}{\textbf{latency}} &
  \multicolumn{1}{c|}{\textbf{Transfer}} &
  \multicolumn{1}{c|}{\textbf{Processing}} &
  \multicolumn{1}{c|}{\textbf{latency}} &
  \multicolumn{1}{c|}{\textbf{Transfer}} &
  \multicolumn{1}{c|}{\textbf{Processing}} &
  \multicolumn{1}{c|}{\textbf{latency}} &
  \multicolumn{1}{c|}{\textbf{Transfer}} &
  \multicolumn{1}{c|}{\textbf{Processing}} &
  \multicolumn{1}{c|}{\textbf{latency}} &
  \multicolumn{1}{c|}{\textbf{Transfer}} &
  \multicolumn{1}{c|}{\textbf{Processing}} \\ \hline
\multicolumn{1}{|l|}{\textbf{\begin{tabular}[c]{@{}l@{}}4. Authen-\\tication \\ response\end{tabular}}} &
  {\color[HTML]{C00000} Uplink} &
  \multicolumn{1}{l|}{3259} &
  \multicolumn{1}{l|}{\begin{tabular}[c]{@{}l@{}}15 sessions\\ 4.7 $\pm$ 9.2\\ total: 70.6\end{tabular}} &
  \multicolumn{1}{l|}{\begin{tabular}[c]{@{}l@{}}14 occurences:\\ \\ - SIMBank (8)\\ 218$\pm$8\\ total: 1744.3\\ \\ - Gateway (6)\\ 211$\pm$6\\ total: 1265.8\end{tabular}} &
  \multicolumn{1}{l|}{2379} &
  \multicolumn{1}{l|}{\begin{tabular}[c]{@{}l@{}}Not clearly\\ identified\end{tabular}} &
  \begin{tabular}[c]{@{}l@{}}12 occurences:\\ \\ - SIMBank (6)\\ 236.1$\pm$116.3\\ total:1416.4\\ \\ - Gateway (6)\\ 139.3$\pm$73.6\\ total: 835.9\end{tabular} &
  \multicolumn{1}{l|}{1199} &
  \multicolumn{1}{l|}{\begin{tabular}[c]{@{}l@{}}1 session\\ total: 4.2\end{tabular}} &
  \begin{tabular}[c]{@{}l@{}}Not clearly\\ identified\\ - Before\\transfer\\ 774.4\\ - After\\transfer\\ 420.4\end{tabular} &
  \multicolumn{1}{l|}{59} &
  \multicolumn{1}{l|}{\begin{tabular}[c]{@{}l@{}}4 sessions\\ 0.12$\pm$0.15\\ total: 0.58\end{tabular}} &
  \begin{tabular}[c]{@{}l@{}}5 occurences\\ \\ - SIM card (2)\\15.6$\pm$14.5\\ total: 31.1\\ \\ - \acrshort{ME} (3)\\ 9.4$\pm$10.8\\ total: 28.1\end{tabular} \\ \hline
\multicolumn{1}{|l|}{\textbf{\begin{tabular}[c]{@{}l@{}}10. Attach \\ complete\end{tabular}}} &
  {\color[HTML]{C00000} Uplink} &
  \multicolumn{1}{l|}{40} &
  \multicolumn{1}{l|}{\begin{tabular}[c]{@{}l@{}}1 session\\ total: 2.8\end{tabular}} &
  \multicolumn{1}{l|}{/} &
  \multicolumn{1}{l|}{60} &
  \multicolumn{1}{l|}{\begin{tabular}[c]{@{}l@{}}1 session\\ total: 0.2\end{tabular}} &
  / &
  \multicolumn{1}{l|}{52} &
  \multicolumn{1}{l|}{/} &
  / &
  \multicolumn{1}{l|}{48} &
  \multicolumn{1}{l|}{/} &
  / \\ \hline
\multicolumn{2}{|l|}{\textbf{Total latency}} &
  \multicolumn{3}{c|}{3411 ms} &
  \multicolumn{3}{c|}{2521 ms} &
  \multicolumn{3}{c|}{1320 ms} &
  \multicolumn{3}{c|}{259 ms} \\ \hline
\end{tabular}%
}
\end{table*}
% \begin{figure*}
% \centering
% \includegraphics[width=0.3\linewidth]{Figures/release_year_hist.pdf}
% \caption{Distribution of release year of \textit{ACLPrint}~\cite{Beomseok:2023}’s tested phones.}
% \label{fig:release_year}
% \end{figure*}

% \begin{figure*}[t]
% \begin{subfigure}{\textwidth}
% \centering
% \includegraphics[scale=0.5]{Figures/HYB_simbox_interactions.png}
% \caption{TCP} 
% \label{fig:hyb_tcp_interactions}
% \end{subfigure}
% \begin{subfigure}{\textwidth}
% \centering
% \includegraphics[scale=0.5]{Figures/HYB_UDP_simbox_interactions.png}
% \caption{UDP} 
% \label{fig:hyb_udp_interactions}
% \end{subfigure}
% \caption{Hybertone \simbox components TCP interactions during authentication}
% \label{fig:simbox_interactions}
% \end{figure*}

\begin{figure*}
\centering
\includegraphics[width=0.6\linewidth]{Figures/HYB_simbox_interactions.png}
\caption{Hybertone \simbox components TCP interactions during authentication}
\label{fig:hyb_tcp_interactions}
\end{figure*}


\begin{figure*}
\centering
\includegraphics[width=0.6\linewidth]{Figures/HYB_UDP_simbox_interactions.png}
\caption{Hybertone \simbox components UDP interactions during authentication}
\label{fig:hyb_udp_interactions}
\end{figure*}

% \begin{figure*}
% \begin{subfigure}{0.45\textwidth}
% \centering
% \includegraphics[scale=0.45]{Figures/simbank_architecture.png}
% \caption{SIMBank.} 
% \label{fig:simbank_arch}
% \end{subfigure}
% \hfill
% \begin{subfigure}{0.45\textwidth}
% \centering
% \includegraphics[scale=0.45]{Figures/gateway_architecture.png}
% \caption{\simbox Gateway.} 
% \label{fig:gateway_arch}
% \end{subfigure}
% \caption{SIMBox internal architecture~\cite{patent2Main}.}
% \label{fig:simbox_arch}
% \end{figure*}

% \section{Reproducibility Instructions}
% \label{app:reproducibility}

% To ensure the reproducibility of the results presented in this paper, we have released the \textit{SigN} datasets and code at the following anonymous link: \href{https://anonymous.4open.science/r/SigN-E485/Readme.md}{\textit{access here}}. This section provides a detailed guide to replicating the results, organized by the figures and tables in the paper.

% \subsection{Installation and Setup}

% \textbf{Install Dependencies}
% \begin{itemize}
%     \item Ensure you have Python (version 3.7 or later) and Jupyter Notebook installed on your system.
%     \item The following Python libraries are required to run the code:
%         \begin{itemize}
%             \item \texttt{numpy}
%             \item \texttt{matplotlib}
%             \item \texttt{scikit-learn}
%             \item \texttt{pandas}
%         \end{itemize}
%     \item Install these libraries using pip as needed. For example:
% \begin{verbatim}
% pip install numpy matplotlib scikit-learn pandas
% \end{verbatim}
% \end{itemize}

% \textbf{Downloading the Repository}
% \begin{itemize}
%     \item Visit the repository at the \href{https://anonymous.4open.science/r/SigN-E485/Readme.md}{\textit{given anonymous link}}.
%     \item Download the repository by clicking the "Download" button on the page.
%     \item Extract the contents of the downloaded ZIP file to a directory of your choice.
% \end{itemize}

% \textbf{Setting Up the Environment}
% \begin{itemize}
%     \item Navigate to the extracted directory in your terminal or command prompt.
%     \item To run the Jupyter Notebooks, start Jupyter Notebook in this directory:
% \begin{verbatim}
% jupyter notebook
% \end{verbatim}
%     \item Ensure the \texttt{helper/} folder is included in your Python path to access utility functions from \texttt{helper.py}. You can add the following lines at the start of each notebook:
% \begin{verbatim}
% import sys
% sys.path.append('helper/')
% \end{verbatim}
% \end{itemize}
% \textbf{Repository Overview}

% The repository includes datasets, code, and logs corresponding to various experimental setups. Below is an overview of the main directories:
% \begin{itemize}
%     \item \textbf{Indoor measures:} Data and analysis for indoor experiments (Table \ref{tab:detailed_latency}, Fig. \ref{fig:rtt_latency}, Fig. \ref{fig:latency_pdf}).
%     \item \textbf{GalaxyNote4 auth. algo:} Authentication algorithm experiments (Fig. \ref{fig:auth_algo}).
%     \item \textbf{Indoor static attenuation:} Data for signal attenuation studies (Fig. \ref{fig:attenuation_cage}).
%     \item \textbf{Outdoor measures:} Analysis of outdoor experiments (Fig. \ref{fig:outdoor_latency_steps}, Fig. \ref{fig:outdoor_rsrp_security}).
%     \item \textbf{SIMBox IP interactions:} Logs of SIMBox IP-level interactions (Fig. \ref{fig:HYP_TCP_time_interactions}, Fig. \ref{fig:tcp_udp}).
%     \item \textbf{srsUE:} Data from LTE user equipment experiments (Table \ref{tab:sim_me_interactions}).
%     \item \textbf{Plots:} Pre-generated plots and PDFs.
% \end{itemize}

% \subsection{Reproducing Results}

% This section provides a step-by-step guide to reproduce the results, organized by the figures and tables.

% \subsubsection*{Table 1: Indoor Latency Measures}
% \begin{itemize}
%     \item \textbf{Directory:} \texttt{Indoor measures/}
%     \item \textbf{Code:} \texttt{code.ipynb}
%     \item \textbf{Data:} Logs organized by phone (e.g., \texttt{Phone1}, \texttt{Phone2}) and SIMBox manufacturer (\texttt{\_rem}, \texttt{\_loc}).
%     \item \textbf{Instructions:} Run the \texttt{code.ipynb} notebook to analyze latency logs and reproduce the table.
% \end{itemize}

% \subsubsection*{Figure 6: SIMBox IP Interactions}
% \begin{itemize}
%     \item \textbf{Directory:} \texttt{SIMBox IP interactions/}
%     \item \textbf{Code:} \texttt{code.ipynb}
%     \item \textbf{Data:}
%         \begin{itemize}
%             \item \texttt{SMBHyb\_rem/}: TCP and UDP logs for Hybertone SIMBox interactions.
%             \item \texttt{SMBPor\_rem/}: Logs for Portech SIMBox interactions.
%         \end{itemize}
%     \item \textbf{Instructions:} Run the \texttt{code.ipynb} notebook to analyze IP-level interactions and generate the figure.
% \end{itemize}

% \subsubsection*{Figure 7: TCP vs UDP in Hybertone Experiments}
% \begin{itemize}
%     \item \textbf{Directory:} \texttt{SMBHyb\_rem TCP vs UDP/}
%     \item \textbf{Code:} \texttt{code.ipynb}
%     \item \textbf{Data:}
%         \begin{itemize}
%             \item \texttt{TCP/}: Logs for TCP traffic.
%             \item \texttt{UDP/}: Logs for UDP traffic.
%         \end{itemize}
%     \item \textbf{Instructions:} Execute the \texttt{code.ipynb} notebook to produce comparison plots for TCP and UDP traffic.
% \end{itemize}

% \subsubsection*{Figure 8, Figure 13: Indoor Latency Analysis}
% \begin{itemize}
%     \item \textbf{Directory:} \texttt{Indoor measures/}
%     \item \textbf{Code:} \texttt{code.ipynb}
%     \item \textbf{Data:} Same as for Table 1.
%     \item \textbf{Instructions:} Use the same analysis in \texttt{Indoor measures/code.ipynb} to render these figures.
% \end{itemize}

% \subsubsection*{Figure 9, Figure 10: Outdoor Measures}
% \begin{itemize}
%     \item \textbf{Directory:} \texttt{Outdoor measures/}
%     \item \textbf{Code:} \texttt{code.ipynb}
%     \item \textbf{Data:}
%         \begin{itemize}
%             \item Logs from different days (\texttt{Day1}, \texttt{Day2-1}, etc.).
%             \item Supplementary CSV files: \texttt{rsrp\_correspondance.csv}, \texttt{rsrq\_correspondance.csv}.
%         \end{itemize}
%     \item \textbf{Instructions:} Run \texttt{code.ipynb} to analyze outdoor experiment logs and generate the figures.
% \end{itemize}

% \subsubsection*{Figure 11: Indoor Static Attenuation}
% \begin{itemize}
%     \item \textbf{Directory:} \texttt{Indoor static attenuation/}
%     \item \textbf{Code:} \texttt{code.ipynb}
%     \item \textbf{Data:} Logs organized by signal strength (\texttt{-100dBm}, \texttt{-90dBm}, etc.).
%     \item \textbf{Instructions:} Execute \texttt{code.ipynb} to analyze attenuation data and reproduce the figure.
% \end{itemize}

% \subsubsection*{Figure 12: Galaxy Note 4 Authentication Algorithms}
% \begin{itemize}
%     \item \textbf{Directory:} \texttt{GalaxyNote4 auth. algo/}
%     \item \textbf{Code:} \texttt{code.ipynb}
%     \item \textbf{Data:} Subdirectories \texttt{milenage/}, \texttt{tuak/}, \texttt{xor/} containing logs for respective algorithms.
%     \item \textbf{Instructions:} Run the \texttt{code.ipynb} notebook to analyze authentication algorithm data and generate the figure.
% \end{itemize}

% \subsubsection*{Table 4: srsUE Experiments}
% \begin{itemize}
%     \item \textbf{Directory:} \texttt{srsUE/}
%     \item \textbf{Data:} Log files for LTE user equipment experiments.
%     \item \textbf{Instructions:} Review the provided log files for insights relevant to Table 4.
% \end{itemize}

% \textbf{Plot Files}
% \begin{itemize}
%     \item \textbf{Directory:} \texttt{Plots/}
%     \item \textbf{Description:} Contains all figures generated from the analysis.
%     \item \textbf{Instructions:} Running the notebooks in each directory will regenerate and overwrite these plots.
% \end{itemize}

\end{document}

