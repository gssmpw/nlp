\section{Proxy Columns}
\label{sec:pivot}
As mentioned in Section~\ref{sec:intro}, it is beneficial to model c2c  joinabilities. However, it is time-consuming to perform online computations to assess the relationship of the query column with each column in the repository. To tackle this, we introduce the concepts of proxy column and proxy column matrix.
% Proxy columns are representative columns in the column space $\mathbb{C}$, based on which, each column in the repository can pre-compute the relationships with proxy columns. Analogous to the column matrix, we define the proxy column matrix as follows. 

\begin{myDef}
\textnormal{\textbf{(Proxy Column)}}. Given a column space $\mathbb{C}$ which is a collection of textual columns, a proxy column $P = \{p_1, p_2, \dots, p_m \} \in \mathbb{C}$ is a representative one, based on which, each column in the repository $\mathcal{R}$ can pre-compute the relationships with proxy columns. 
\end{myDef}



\begin{myDef}
\textnormal{\textbf{(Proxy Column Matrix)}}. Given a proxy column $P = \{p_1, p_2, \dots, p_m \}$ and a cell embedding function $h(\cdot)$, the proxy column matrix $\mathbf{P} = h (P) = \{\mathbf{p}_1, \mathbf{p}_2, \dots, \mathbf{p}_m \} \in \mathbb{R}^{m \times d}$, where $\mathbf{p}_i \in \mathbb{R}^d$. 
\end{myDef}

Note that the specific column repository $\mathcal{R}$ is just a subset of the column space $\mathbb{C}$.  Since \textsf{Snoopy} requires the column matrix and proxy column matrix as inputs (detailed later), we define the proxy-guided column projection given a proxy column matrix as follows.
% Since \textsf{Snoopy} requires the column matrix and proxy column matrix as inputs (detailed later),
% For simplicity, we use the terms ``column" and ``proxy column" to refer to ``column matrix" and ``proxy column matrix", respectively, when the context is clear. We now define the proxy-guided column projection.

% \begin{myDef}
% \textnormal{\textbf{(Column Projection)}}.
% Given a set of proxy columns $\mathcal{P} = \{\mathbf{P}_1, \mathbf{P}_2, \dots, \mathbf{P}_l \} \in \mathbb{R}^{l\times m \times d}$, a column $\mathbf{C}$ is mapped to a point in the column embedding space:
% \begin{equation}
% \phi(\mathbf{C})= \left[\pi_{\mathbf{P}_1} (\mathbf{C}), \pi_{\mathbf{P}_2}(\mathbf{C}), \ldots \pi_{\mathbf{P}_l} (\mathbf{C})\right] \in \mathbb{R}^l
% \end{equation}
% \noindent where $\pi_{\mathbf{P}_i} (\cdot)$ is a  projection operator that projects the column $\mathbf{C}$ to a specific dimension of the embedding space via proxy column $\mathbf{P}_i$.
% \end{myDef}

\begin{myDef}
\textnormal{\textbf{(Column Projection)}}.
Given a proxy column matrix $\mathbf{P}$,  the proxy-guided column projection is a function $\pi_{\mathbf{P}} (\cdot)$, which projects a column matrix $\mathbf{C}$ to $\pi_{\mathbf{P}} (\mathbf{C}) \in \mathbb{R}$ indicating the value of a specific dimension in the column embedding space $\mathbb{R}^l$.
\end{myDef}

Based on the column projection, we can obtain the column embeddings. Specifically,   given a set of proxy column matrices  $\mathcal{P} = \{\mathbf{P}_1, \mathbf{P}_2, \dots, \mathbf{P}_l \} \in \mathbb{R}^{l\times m \times d}$, the column embedding of $\mathbf{C}$ is denoted as $
\phi(\mathbf{C})= \left[\pi_{\mathbf{P}_1} (\mathbf{C}), \pi_{\mathbf{P}_2}(\mathbf{C}), \ldots \pi_{\mathbf{P}_l} (\mathbf{C})\right] \in \mathbb{R}^l$.
 
 