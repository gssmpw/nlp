\documentclass[lettersize,journal]{IEEEtran}
\usepackage{amsmath,amsfonts}
\usepackage{amsthm}
\usepackage{algorithmic}
% \newtheorem{myDef}{Definition}
\usepackage[ruled,vlined]{algorithm2e}
\usepackage{array}

\usepackage[caption=false,font=footnotesize,labelfont=rm,textfont=rm]{subfig}
% \usepackage{subfigure}
\usepackage{textcomp}
\usepackage{multirow}
\usepackage{threeparttable}
\usepackage{setspace}
\usepackage{bbding}
\usepackage{pifont}
\usepackage{stfloats}
\usepackage{subfloat}
\usepackage{subfig}
\usepackage{url}
\usepackage{verbatim}
\usepackage{makecell}
\usepackage{graphicx}
\usepackage{booktabs}
\usepackage{cite}
\usepackage{tikz}
\usepackage[hang,flushmargin]{footmisc}
\newtheorem{myThm}{Theorem}
 


\newcommand*\circled[1]{\tikz[baseline=(char.base)]{
            \node[shape=circle, draw, fill=black, text=white, inner sep=1pt, font=\bfseries\scriptsize] (char) {#1};}}

\newcommand{\up}{\vspace*{-0.1in}}
\newcommand{\down}{\vspace*{0.05in}}
 
\newcommand{\mathds}[1]{\text{\usefont{U}{dsrom}{m}{n}#1}}

 
\hyphenation{op-tical net-works semi-conduc-tor IEEE-Xplore}

\usepackage{xcolor} % 如果需要自定义颜色
\usepackage{amsthm} % 定理环境
\newtheoremstyle{example}
  {3pt} % Space above
  {3pt} % Space below
  {} % Body font
  {} % Indent amount
  {\bfseries} % Theorem head font
  {.} % Punctuation after theorem head
  {.5em} % Space after theorem head
  {} % Theorem head spec

\theoremstyle{example}
\newtheorem{example}{Example}

\usepackage{enumitem}
\setlist[itemize]{leftmargin=*}

\usepackage{xcolor} % 如果需要自定义颜色
\usepackage{amsthm} % 定理环境
\newtheoremstyle{Definition}
  {3pt} % Space above
  {3pt} % Space below
  {} % Body font
  {} % Indent amount
  {\bfseries} % Theorem head font
  {.} % Punctuation after theorem head
  {.5em} % Space after theorem head
  {} % Theorem head spec

\theoremstyle{Definition}
\newtheorem{myDef}{Definition}

\usepackage{xcolor} % 如果需要自定义颜色
\usepackage{amsthm} % 定理环境
\newtheoremstyle{Property}
  {3pt} % Space above
  {3pt} % Space below
  {} % Body font
  {} % Indent amount
  {\bfseries} % Theorem head font
  {.} % Punctuation after theorem head
  {.5em} % Space after theorem head
  {} % Theorem head spec

\theoremstyle{Property}
\newtheorem{myProp}{Property}

\usepackage{xcolor} % 如果需要自定义颜色
\usepackage{amsthm} % 定理环境
\newtheoremstyle{Proposition}
  {3pt} % Space above
  {3pt} % Space below
  {} % Body font
  {} % Indent amount
  {\bfseries} % Theorem head font
  {.} % Punctuation after theorem head
  {.5em} % Space after theorem head
  {} % Theorem head spec

\theoremstyle{Proposition}
\newtheorem{prop}{Proposition}


\usepackage{xcolor} % 如果需要自定义颜色
\usepackage{amsthm} % 定理环境
\newtheoremstyle{Problem}
  {3pt} % Space above
  {3pt} % Space below
  {} % Body font
  {} % Indent amount
  {\bfseries} % Theorem head font
  {.} % Punctuation after theorem head
  {.5em} % Space after theorem head
  {} % Theorem head spec

\theoremstyle{Problem}
\newtheorem{Prob}{Problem}


% updated with editorial comments 8/9/2021

\begin{document}

% \title{A Sample Article Using IEEEtran.cls\\ for IEEE Journals and Transactions}
\title{Snoopy: Effective and Efficient Semantic Join Discovery via Proxy Columns}


\author{Yuxiang Guo,
        Yuren Mao,
        Zhonghao Hu,
        Lu~Chen,
        Yunjun~Gao, ~\IEEEmembership{Senior Member,~IEEE}
       
        % <-this % stops a space
% \thanks{This paper was produced by the IEEE Publication Technology Group. They are in Piscataway, NJ.}% <-this % stops a space
\thanks{Y. Guo,  L. Chen and Y. Gao are with the College of Computer Science, Zhejiang University, Hangzhou 310027, China (e-mail: guoyx@zju.edu.cn; luchen@zju.edu.cn; gaoyj@zju.edu.cn).}
\thanks{Y. Mao, Z. Hu are with the School of Software Technology, Zhejiang University, Ningbo 315048, China (e-mail: yuren.mao@zju.edu.cn; zhonghao.hu@zju.edu.cn).}
\thanks{*Corresponding author: Yunjun Gao (e-mail: gaoyj@zju.edu.cn)}
}



% % The paper headers
% \markboth{IEEE Transactions on Knowledge and Data Engineering,~Vol.~XX, No.~XX, XXX~XXXX}{Li \MakeLowercase{\textit{et al.}}:Snoopy: Effective and Efficient Semantic Join
% Discovery via Proxy Columns}

% \IEEEpubid{0000--0000/00\$00.00~\copyright~2021 IEEE}
% Remember, if you use this you must call \IEEEpubidadjcol in the second
% column for its text to clear the IEEEpubid mark.

\maketitle

\begin{abstract}
Semantic join discovery, which aims to find columns in a table repository with high  semantic joinabilities to a given query column, plays an essential role in dataset discovery.
Existing methods can be divided into two categories: cell-level methods and column-level methods. However, both of them cannot simultaneously ensure proper effectiveness and efficiency. 
Cell-level methods, which compute the joinability by counting cell matches between columns, enjoy ideal effectiveness but suffer poor efficiency. In contrast, column-level methods, which determine joinability only by computing the similarity of column embeddings, enjoy proper efficiency but suffer poor effectiveness due to the issues occurring in their column embeddings: (i) semantics-joinability-gap, (ii) size limit, and (iii) permutation sensitivity. To address these issues, this paper proposes to compute column embeddings via proxy columns; furthermore, a novel column-level 
% effective and efficient
semantic join  discovery framework, \textsf{Snoopy}, is presented, leveraging proxy-column-based embeddings to bridge effectiveness and efficiency. Specifically, the proposed column embeddings are derived from the implicit column-to-proxy-column relationships, which are captured by the lightweight approximate-graph-matching-based column projection. To acquire good proxy columns for guiding the column projection, we introduce a rank-aware contrastive learning paradigm.
Extensive experiments on four real-world datasets demonstrate that \textsf{Snoopy} outperforms SOTA column-level methods by 16\% in Recall@25 and 10\% in NDCG@25, and achieves superior efficiency—being at least 5 orders of magnitude faster than cell-level solutions, and 3.5x faster than existing column-level methods.
\end{abstract}


\begin{IEEEkeywords}
 Semantic Join Discovery, Similarity Search, Proxy Columns, Representation Learning
\end{IEEEkeywords}

\section{Introduction}

In recent years, with advancements in generative models and the expansion of training datasets, text-to-speech (TTS) models \cite{valle, voicebox, ns3} have made breakthrough progress in naturalness and quality, gradually approaching the level of real recordings. However, low-latency and efficient dual-stream TTS, which involves processing streaming text inputs while simultaneously generating speech in real time, remains a challenging problem \cite{livespeech2}. These models are ideal for integration with upstream tasks, such as large language models (LLMs) \cite{gpt4} and streaming translation models \cite{seamless}, which can generate text in a streaming manner. Addressing these challenges can improve live human-computer interaction, paving the way for various applications, such as speech-to-speech translation and personal voice assistants.

Recently, inspired by advances in image generation, denoising diffusion \cite{diffusion, score}, flow matching \cite{fm}, and masked generative models \cite{maskgit} have been introduced into non-autoregressive (NAR) TTS \cite{seedtts, F5tts, pflow, maskgct}, demonstrating impressive performance in offline inference.  During this process, these offline TTS models first add noise or apply masking guided by the predicted duration. Subsequently, context from the entire sentence is leveraged to perform temporally-unordered denoising or mask prediction for speech generation. However, this temporally-unordered process hinders their application to streaming speech generation\footnote{
Here, “temporally” refers to the physical time of audio samples, not the iteration step $t \in [0, 1]$ of the above NAR TTS models.}.


When it comes to streaming speech generation, autoregressive (AR) TTS models \cite{valle, ellav} hold a distinct advantage because of their ability to deliver outputs in a temporally-ordered manner. However, compared to recently proposed NAR TTS models,  AR TTS models have a distinct disadvantage in terms of generation efficiency \cite{MEDUSA}. Specifically, the autoregressive steps are tied to the frame rate of speech tokens, resulting in slower inference speeds.  
While advancements like VALL-E 2 \cite{valle2} have boosted generation efficiency through group code modeling, the challenge remains that the manually set group size is typically small, suggesting room for further improvements. In addition,  most current AR TTS models \cite{dualsteam1} cannot handle stream text input and they only begin streaming speech generation after receiving the complete text,  ignoring the latency caused by the streaming text input. The most closely related works to SyncSpeech are CosyVoice2 \cite{cosyvoice2.0} and IST-LM \cite{yang2024interleaved}, both of which employ interleaved speech-text modeling to accommodate dual-stream scenarios. However, their autoregressive process generates only one speech token per step, leading to low efficiency.



To seamlessly integrate with  upstream LLMs and facilitate dual-stream speech synthesis, this paper introduces \textbf{SyncSpeech}, designed to keep the generation of streaming speech in synchronization with the incoming streaming text. SyncSpeech has the following advantages: 1) \textbf{low latency}, which means it begins generating speech in a streaming manner as soon as the second text token is received,
and
2) \textbf{high efficiency}, 
which means for each arriving text token, only one decoding step is required to generate all the corresponding speech tokens.

SyncSpeech is based on the proposed \textbf{T}emporal \textbf{M}asked generative \textbf{T}ransformer (TMT).
During inference, SyncSpeech adopts the Byte Pair Encoding (BPE) token-level duration prediction, which can access the previously generated speech tokens and performs top-k sampling. 
Subsequently, mask padding and greedy sampling are carried out based on  the duration prediction from the previous step. 

Moreover, sequence input is meticulously constructed to incorporate duration prediction and mask prediction into a single decoding step.
During the training process, we adopt a two-stage training strategy to improve training efficiency and model performance. First, high-efficiency masked pretraining is employed to establish a rough alignment between text and speech tokens within the sequence, followed by fine-tuning the pre-trained model to align with the inference process.

Our experimental results demonstrate that, in terms of generation efficiency, SyncSpeech operates at 6.4 times the speed of the current dual-stream TTS model for English and at 8.5 times the speed for Mandarin. When integrated with LLMs, SyncSpeech achieves latency reductions of 3.2 and 3.8 times, respectively, compared to the current dual-stream TTS model for both languages.
Moreover, with the same scale of training data, SyncSpeech performs comparably to traditional AR models in terms of the quality of generated English speech. For Mandarin, SyncSpeech demonstrates superior quality and robustness compared to current dual-stream TTS models. This showcases the potential of  SyncSpeech as a foundational model to integrate with upstream LLMs.


\section{Preliminaries}
\label{sec:preliminaries}
In this section, we provide background on the Circuit Analysis~\citep{zoom, circuitgrokking, conmy2023towards}, which represents the model's computation through structured subgraph of its components. 

\subsection{Circuit Analysis}
\label{subsec:circuit-analysis}
Circuit analysis represents a transformer's computation as a directed acyclic graph (DAG) $G = (N, E)$, where each node in $N$ corresponds to a distinct component in the model: attention heads $A_{l,j}$ (at layer $l$ and head $j$), MLP modules $M_l$ for each layer, the input node $I$ (embeddings), and the output node $O$ (logits). 
Thus, we formally define the set of nodes as:
\begin{equation}
    N = \{ I, A_{l,j}, M_l, O \}.
\end{equation}
The edges in $E$ represent residual connections that propagate activations between these nodes:
\begin{equation}
    E = \{ (n_x, n_y) \mid n_x, n_y \in N \}.
\end{equation}

A \emph{circuit} is defined as a subgraph $C \subseteq (N, E)$ selected to explain a specific behavior of interest--for instance, how certain tokens influence the model's output or how factual knowledge is stored and elicited.  
By examining which nodes and edges are crucial for producing a particular prediction, we can identify the subgraph (the circuit) that governs each behavior.

\subsection{Knowledge Circuit}
\label{subsec:knowledge_circuit}
A \emph{knowledge circuit}~\citep{KC} focuses on how a model treats the subject $s$, and relation $r$ to generate the object $o$ using a knowledge triplet $(s,r,o)$. 
By systematically \emph{ablating} (i.e.\ zeroing) parts of the model, it identifies the crucial nodes responsible for this generation and constructs a subgraph $KC \subseteq (N,E)$ whose removal \emph{breaks} the model’s ability to produce the correct object.
Concretely, it define a performance metric as:
\begin{equation}
\begin{split}
S(e_i) = &\; \log\bigl(p_G(o \mid s,r)\bigr) \\
         &- \log\bigl(p_{G/e_i}(o \mid s,r)\bigr).
\end{split}
\label{eq:knw-circuit-score}
\end{equation}
where $p_{G/e_i}$ denotes the model’s probability of next-token prediction after \emph{ablating} (i.e.\ zeroing) the activation of a node or edge $e_i$. 
If $S(e_i)$ exceeds a threshold $\tau$, $e_i$ is deemed \emph{critical} and retained in $KC$; otherwise, $e_i$ is pruned. 
This yields a \emph{minimal} set of heads/MLPs whose connections critically shape the binding of $(s,r)$ to the correct answer $o$.

Unlike a generic circuit for any functionality, a knowledge circuit specifically captures the local subgraph dedicated to storing and relaying factual content for the knowledge triplet at hand.
We specifically utilize effective attribution pruning-integrated gradients (EAP-IG), which ablating (zeroing) candidate edges and measuring drops in correct prediction~\citep{eapig}. 
For more details, see Appendix~\ref{sec:EAP-IG}.

\section{Proxy Columns}
\label{sec:pivot}
As mentioned in Section~\ref{sec:intro}, it is beneficial to model c2c  joinabilities. However, it is time-consuming to perform online computations to assess the relationship of the query column with each column in the repository. To tackle this, we introduce the concepts of proxy column and proxy column matrix.
% Proxy columns are representative columns in the column space $\mathbb{C}$, based on which, each column in the repository can pre-compute the relationships with proxy columns. Analogous to the column matrix, we define the proxy column matrix as follows. 

\begin{myDef}
\textnormal{\textbf{(Proxy Column)}}. Given a column space $\mathbb{C}$ which is a collection of textual columns, a proxy column $P = \{p_1, p_2, \dots, p_m \} \in \mathbb{C}$ is a representative one, based on which, each column in the repository $\mathcal{R}$ can pre-compute the relationships with proxy columns. 
\end{myDef}



\begin{myDef}
\textnormal{\textbf{(Proxy Column Matrix)}}. Given a proxy column $P = \{p_1, p_2, \dots, p_m \}$ and a cell embedding function $h(\cdot)$, the proxy column matrix $\mathbf{P} = h (P) = \{\mathbf{p}_1, \mathbf{p}_2, \dots, \mathbf{p}_m \} \in \mathbb{R}^{m \times d}$, where $\mathbf{p}_i \in \mathbb{R}^d$. 
\end{myDef}

Note that the specific column repository $\mathcal{R}$ is just a subset of the column space $\mathbb{C}$.  Since \textsf{Snoopy} requires the column matrix and proxy column matrix as inputs (detailed later), we define the proxy-guided column projection given a proxy column matrix as follows.
% Since \textsf{Snoopy} requires the column matrix and proxy column matrix as inputs (detailed later),
% For simplicity, we use the terms ``column" and ``proxy column" to refer to ``column matrix" and ``proxy column matrix", respectively, when the context is clear. We now define the proxy-guided column projection.

% \begin{myDef}
% \textnormal{\textbf{(Column Projection)}}.
% Given a set of proxy columns $\mathcal{P} = \{\mathbf{P}_1, \mathbf{P}_2, \dots, \mathbf{P}_l \} \in \mathbb{R}^{l\times m \times d}$, a column $\mathbf{C}$ is mapped to a point in the column embedding space:
% \begin{equation}
% \phi(\mathbf{C})= \left[\pi_{\mathbf{P}_1} (\mathbf{C}), \pi_{\mathbf{P}_2}(\mathbf{C}), \ldots \pi_{\mathbf{P}_l} (\mathbf{C})\right] \in \mathbb{R}^l
% \end{equation}
% \noindent where $\pi_{\mathbf{P}_i} (\cdot)$ is a  projection operator that projects the column $\mathbf{C}$ to a specific dimension of the embedding space via proxy column $\mathbf{P}_i$.
% \end{myDef}

\begin{myDef}
\textnormal{\textbf{(Column Projection)}}.
Given a proxy column matrix $\mathbf{P}$,  the proxy-guided column projection is a function $\pi_{\mathbf{P}} (\cdot)$, which projects a column matrix $\mathbf{C}$ to $\pi_{\mathbf{P}} (\mathbf{C}) \in \mathbb{R}$ indicating the value of a specific dimension in the column embedding space $\mathbb{R}^l$.
\end{myDef}

Based on the column projection, we can obtain the column embeddings. Specifically,   given a set of proxy column matrices  $\mathcal{P} = \{\mathbf{P}_1, \mathbf{P}_2, \dots, \mathbf{P}_l \} \in \mathbb{R}^{l\times m \times d}$, the column embedding of $\mathbf{C}$ is denoted as $
\phi(\mathbf{C})= \left[\pi_{\mathbf{P}_1} (\mathbf{C}), \pi_{\mathbf{P}_2}(\mathbf{C}), \ldots \pi_{\mathbf{P}_l} (\mathbf{C})\right] \in \mathbb{R}^l$.
 
 
\section{The \textsf{Snoopy} Framework}
\label{sec:Snoopy}
In this section, we first  
overview the \textsf{Snoopy} framework. Then, we design the column representation via proxy column matrices, and present a rank-aware contrastive learning paradigm to obtain good proxy column matrices. After that, we illustrate the index and online search process. Finally, we devise two training data generation strategies for self-supervised training.


% \begin{myDef}
% \textnormal{\textbf{(Column Mapping)}}.
% Given an input column $\mathbf{C}$, a column mapping operation guided by the proxy column $\mathbf{P}$ maps the column $\mathbf{C}$ to a real number $\phi(\mathbf{C}, \mathbf{P}) \in \mathbb{R}$.
% \end{myDef}



\subsection{Overview}
\label{subsec:overview}

\textsf{Snoopy} framework is composed of two stages: offline and online, as illustrated in Fig.~\ref{fig:framework}.

\noindent\textbf{Offline stage.} Given a table repository $\mathcal{T}$, the textual columns are extracted to form the column repository $\mathcal{R}$. In the \ding{172} training phase, the column representation process transforms each column in $\mathcal{R}$ into a column embedding by concatenating the proxy-guided column projection values. Proxy column matrices are treated as learnable parameters and are updated via rank-aware contrastive learning.
% , which aims to identify good proxy columns that can
% map the joinable columns closely while pushing non-joinable ones far apart in the column embedding space.
% pull joinable columns closer, while pushing non-joinable
% columns far away in the embedding space.
\ding{173} After training, \textsf{Snoopy} uses the learned proxy column matrices to pre-compute all the column embeddings, and stores them in a Vector Storage. The indexes (e.g. HNSW~\cite{HNSW}) are constructed to accelerate the subsequent online search.

\noindent\textbf{Online stage.} Given a  query column $C_Q$ from table $T_Q$, \textsf{Snoopy} first computes the embedding of $C_Q$ using the previously learned proxy column  matrices. 
Then, it uses the embedding of $C_Q$ to search the top-$k$ similar embeddings from the Vector Storage. Finally, \textsf{Snoopy} returns the joinable columns according to the retrieved embeddings. 

% \subsection{Properties of Column Representation}
% Recall that, the effectiveness of column-level methods highly depends on column representations.
% % Given a column $C\in \mathcal{R}$, we aim to design an embedding function $f(\cdot)$ to transform the column $C$ to an embedding $f(C)\in \mathbb{R}^l$.
% Thus, we formalize and analyze several desirable properties of column representations for column-level semantically joinable table search. 


% \begin{figure}
%   \centering
%   \includegraphics[width=1\linewidth]{Framework - 副本 - 副本.pdf} \vspace{-5mm}
%   \caption{{Overview of \textsf{Snoopy}.}
%   \label{fig:framework}}
%   \vspace{-4mm}
% \end{figure}

% % \begin{myProp}
% % \textnormal{\textbf{(\textcolor{red}{Size Unlimited})}.} Given a column $C =\{ c_1, c_2, \dots,\\ c_{|C|} \}$, and another column $C' =\{ c'_1, c'_2, \dots, c'_{|C'|} \}$, the extension of $C$ by $C'$ is denoted as $C||C' = \{c_1, c_2, \dots, c_{|C|}, c'_1, c'_2, \dots, c'_{|C'|} \}$. The column representation is expected to satisfy:
% % \begin{equation}
% % \nexists C, \text{s.t.} \; f(C) \equiv f(C||C'), \forall C^{\prime} \in \mathcal{R}
% % \end{equation}
% % \end{myProp}

% % If a column embedding function has an input size limit $\epsilon$, any column $C$ with size $\epsilon$ satisfies $f(C) \equiv f(C||C')$, as any extension of $C$ exceeds the size $\epsilon$ and would be ignored.



% \begin{myProp}
% \label{prop:1}
%     \textnormal{\textbf{(Size Unlimited)}}. Given a column $C =\{ c_1, c_2, \dots,\\ c_n \} \in \mathcal{R}$, and the embedding function $f: \mathcal{R} \rightarrow \mathbb{R}^l$, the size $n$ of the input column $C$ is arbitrary.
% \end{myProp}


 
% % \subsubsection{Permutation Invariance }
% \begin{myProp}
% \label{prop:2}
% \textnormal{\textbf{(Permutation Invariance)}.} Given a column $C = \{c_1, c_2, \dots, c_n \}$, and an arbitrary bijective function $\delta: \{1, 2, \dots, n\} \rightarrow \{1, 2, \dots, n\}$, a permutation of column $C$ is denoted as $\Tilde{C} = \{c_{\delta(1)}, c_{\delta(2)},\\ \dots, c_{\delta(n)} \}$. The column representation is expected to satisfy:
% \begin{equation}
%     f(C) \equiv f(\Tilde{C}) 
% \end{equation}
% \end{myProp}
% % If a column embedding function treats the input column as an ordered sequence of cells, the output varies with permutations, thereby not satisfying the permutation invariance property.

% % \subsubsection{Rank Preserving}
% \begin{myProp}
% \label{prop:3}
% \textnormal{\textbf{(Order Preserving)}.} 
% % Recall that the embedding-based methods aim to find the top-$k$ results through the embedding of columns.
% % Thus, the ideal column embedding function $f(\cdot)$ is expected 
% % to preserve the ranks of columns obtained by computing the joinability defined in Equation~\ref{eq:js}. We formalize this expected property as follows.
% Given a query column $Q$, two candidate columns $C_1$ and $C_2$, the joinability function $J(\cdot)$, and the column-level similarity measurement $\operatorname{sim}(\cdot)$, the IDEAL column representation is expected to satisfy:
% \begin{equation}
% \small
%     \operatorname{sim} \bigl(f(Q), f(C_1)\bigr) \geq \operatorname{sim} \bigl(f(Q), f(C_2) \bigr)   \Leftrightarrow  J(Q, C_1) \geq J(Q, C_2)
% \end{equation}
% \end{myProp}
% % \textcolor{blue}{We design a column representation via proxy columns (Section~\ref{subsec:ColRep}) which satisfies the \textsc{Property}~\ref{prop:1} and \textsc{Property}~\ref{prop:2}. 
% Note that, \textsc{Property}~\ref{prop:3} is the ultimate goal ensuring that column-level methods achieve the same effectiveness as cell-level ones.
% However, it is challenging due to inevitable information loss resulting from coarse computations at the column level.  
% Thus, we design a training paradigm to learn the optimal proxy columns capable of deriving column representations that approximate this goal (Section~\ref{subsubsec:PivotLearn}).



\begin{figure}
  \centering
  \includegraphics[width=1\linewidth]{Framework.pdf} \vspace{-4mm}
  \caption{Overview of \textsf{Snoopy}. ``Column" is abbreviated as ``Col".}
  \label{fig:framework}
  \vspace{-4mm}
\end{figure}

\subsection{Column Representation}
The effectiveness of column-level methods highly depends on column representations. However, existing methods just adopt suboptimal PTMs as column encoders, and the desirable properties of column representations for column-level methods remain under-explored. Thus, we first formalize several desirable properties of column representations, and then propose the AGM-based column projection function to deduce the column embeddings.

\subsubsection{Desirable Properties}
% \vspace{1mm}
% \noindent\textbf{Desirable Properties.}
We provide some key observations based on the definition of joinability and formalize the desirable properties of column representations for column-level semantic join discovery.




The first observation is that each cell in the column may contribute to the joinability score. Consider the example in Fig.~\ref{fig:exm1}, the joinability between  $C_Q$ and $C_1$ is $\frac{3}{3}$=1. If we neglect any cell in the  $C_1$, the join score declines. 
This observation implies that the column embedding function should consider all the cells within the column and not be
% consider all cells within the column, without being
constrained by the column size, which is formalized as follows.

 
\begin{myProp}
\label{prop:1}
 \textnormal{\textbf{(Size-unlimited)}.}
     Given a column $C =\{ c_1, c_2,\\ \dots, c_n \} \in \mathcal{R}$, which is the input of the embedding function $f: \mathcal{R} \rightarrow \mathbb{R}^l$, the size $n$ of the input column $C$ is arbitrary.
\end{myProp}


% \noindent  \underline{\textit{Size-unlimited.}} The first observation is that each cell in the column may contribute to the joinability score. Consider the example in Fig.~\ref{fig:exm1}, the joinability between  $C_Q$ and $C_1$ is $\frac{3}{3}$=1. If we neglect any cell in the  $C_1$, the join score declines. 
% This observation implies that the column embedding function should consider all cells within the column, without being constrained by the   column size. 

 
% \begin{myProp}
% \label{prop:1}
%      Given a column $C =\{ c_1, c_2, \dots, c_n \} \in \mathcal{R}$, which is the input of the embedding function $f: \mathcal{R} \rightarrow \mathbb{R}^l$, the size $n$ of the input column $C$ is arbitrary.
% \end{myProp}



% \noindent \underline{\textit{Permutation-invariant.}} The second observation is that the joinability between two columns is agnostic to the permutations of cells within each column. For instance, if we permute the column  $C_Q$  in Fig.~\ref{fig:exm1} from \{``Los Angeles", ``New York", ``Washington"\} to \{``Washington", ``Los Angeles", ``New York"\}, the joinability between $C_Q$  and $C_1$ is still $\frac{3}{3}$=1. This suggests that the column embedding should be permutation-invariant, which is formalized as follows.
% \begin{myProp}
% \label{prop:2}
% Given a column $C = \{c_1, c_2, \dots, c_n \}$, and an arbitrary bijective function $\delta: \{1, 2, \dots, n\} \rightarrow \{1, 2, \dots, n\}$, a permutation of column $C$ is denoted as $\Tilde{C} = \{c_{\delta(1)}, c_{\delta(2)}, \dots, c_{\delta(n)} \}$. The column representation is expected to satisfy:
% \begin{equation}
%     f(C) \equiv f(\Tilde{C}) 
% \end{equation}
% \end{myProp}

The second observation is that the joinability between two columns is agnostic to the permutations of cells within each column. For instance, if we permute the column  $C_Q$  in Fig.~\ref{fig:exm1} from \{``Los Angeles", ``New York", ``Washington"\} to \{``Washington", ``Los Angeles", ``New York"\}, the joinability between $C_Q$  and $C_1$ is still $\frac{3}{3}$=1. This suggests that the column embedding should be permutation-invariant, which is formalized as follows.
\begin{myProp}
\label{prop:2}
\textnormal{\textbf{(Permutation-invariant)}.} 
Given a column $C = \{c_1, c_2, \dots, c_n \}$, and an arbitrary bijective function $\delta: \{1, 2, \dots, n\} \rightarrow \{1, 2, \dots, n\}$, a permutation of column $C$ is denoted as $\Tilde{C} = \{c_{\delta(1)}, c_{\delta(2)}, \dots, c_{\delta(n)} \}$. The column representation is expected to satisfy:
\begin{equation}
    f(C) \equiv f(\Tilde{C}) 
\end{equation}
\end{myProp}


% \noindent \underline{\textit{Order-preserving.}} Recall that the column-level methods return the top-$k$ results through the similarity comparison of column embeddings. 
% Hence, an ideal column embedding function $f(\cdot)$ needs to preserve the ranking order of columns as determined by their joinabilities in Definition~\ref{def:js}.


% \begin{myProp}
% \label{prop:3}
% % \textnormal{\textbf{(Order Preserving)}.} 
% Given a query column $C_Q$, two candidate columns $C_1$ and $C_2$, the joinability $J(\cdot)$ as defined in Definition~\ref{def:js}, and the column-level similarity measurement $\operatorname{sim}(\cdot)$, the ideal column representation satisfies:
% \begin{equation}
% \begin{gathered}
% \operatorname{sim}\left(f\left(C_Q\right), f\left(C_1\right)\right) \geq \operatorname{sim}\left(f\left(C_Q\right), f\left(C_2\right)\right) \\
% \Leftrightarrow J\left(C_Q, C_1\right) \geq J\left(C_Q, C_2\right)
% \end{gathered}
% \end{equation}
% \end{myProp}

Recall that the column-level methods return the top-$k$ results through the similarity comparison of column embeddings. 
Hence, an ideal column embedding function $f(\cdot)$ needs to preserve the ranking order of columns as determined by their joinabilities in Definition~\ref{def:js}.


\begin{myProp}
\label{prop:3}
\textnormal{\textbf{(Order-preserving)}.} 
Given a query column $C_Q$, two candidate columns $C_1$ and $C_2$, the joinability $J(\cdot)$ as defined in Definition~\ref{def:js}, and the column-level similarity measurement $\operatorname{sim}(\cdot)$, the ideal column representation satisfies:
\begin{equation}
\begin{gathered}
\operatorname{sim}\left(f\left(C_Q\right), f\left(C_1\right)\right) \geq \operatorname{sim}\left(f\left(C_Q\right), f\left(C_2\right)\right) \\
\Leftrightarrow J\left(C_Q, C_1\right) \geq J\left(C_Q, C_2\right)
\end{gathered}
\end{equation}

\end{myProp}


The property~\ref{prop:3} is the ultimate goal ensuring that column-level methods achieve the same effectiveness as exact solutions.
However, it is challenging to achieve this due to the inevitable information loss resulting from coarse computations at the column level.  
Thus, we design a training paradigm to learn good proxy column matrices capable of deriving column representations that approximate this goal (Section~\ref{subsubsec:PivotLearn}). Note that the defined joinability in Definition~\ref{def:js} relies on a cell embedding function $h(\cdot)$. If $h(\cdot)$ is inferior, then the ground truth in evaluation and the learned  column embedding function $f(\cdot)$ may be affected.
% Several alternatives are available for the cell embedding function, including pre-trained language models~\cite{fasttext, sentencebert} and models tailored for entity matching~\cite{camper, ditto}. Designing a good cell embedding function is orthogonal to this work.
In this paper, we follow~\cite{Pexeso} to employ fastText~\cite{fasttext} as a cell embedding function, and show how the ground truth shifts when adopting different cell embedding functions (see Section~\ref{subsec:further_exp}). 




\subsubsection{AGM-based Column Projection}
\label{subsec:ColRep}
% \vspace{1mm}
% \noindent \textbf{AGM-based Column Projection.}
In order to consider all the cells within the column $C$, 
\textsf{Snoopy} first transforms it into the column matrix $\mathbf{C} = h(C)$ by the cell embedding function $h(\cdot)$. 
Then, it computes the column
representation using $\mathbf{C}$ as the input. 
The cell embedding function $h(\cdot)$ is utilized to capture the cell semantics, so that the cells that are not exactly the same but semantically equivalence can be matched and contribute to the joinability.  
Several alternatives are available for the cell embedding function, including pre-trained language models~\cite{fasttext, sentencebert} and models tailored for entity matching~\cite{camper, ditto}.
Note that, designing a good cell embedding function is orthogonal to this work.
% In this paper, we follow~\cite{Pexeso} to employ fastText~\cite{fasttext} as a non-parametric cell embedding function.  


% Given a column $C$, \textsf{Snoopy} first transforms it into the column matrix $\mathbf{C} = h(C)$ by the cell embedding function $h(\cdot)$. 
% Then, it computes the column
% representation using $\mathbf{C}$ as the input. 
% Note that, designing a good cell embedding function is orthogonal to this work. In this paper, we follow~\cite{Pexeso} to employ fastText~\cite{fasttext} as a non-parametric cell embedding function. 
% We now define the concept of pivot column, and column mapping which is the core of pivot-column-based column representation. 



As mentioned in Section~\ref{sec:intro}, the core of proxy-column-based column representation lies in the design of column projection function  $\pi_{\mathbf{P}}(\cdot)$  that aims to well capture the c2pc relationships. Recall that, the semantic joinability in Equation (\ref{eq:js}) represents the c2c relationship captured by the cell-level methods, and it is naturally \textit{size-unlimited} and \textit{permutation-invariant} by definition. 
Thus, a straightforward way is to define $\pi_{\mathbf{P}}(\mathbf{C}) = J(\mathbf{C}, \mathbf{P})$. However, it results in significant information loss.  We begin our analysis by rewriting the joinability in Equation~(\ref{eq:js}) into the following equivalent  mathematical form:
\begin{equation}
\label{eq:js2}
    J(\mathbf{C}, \mathbf{P})= \sum_{i=1}^n \mathds{1}\left(\min _{\mathbf{p}_j \in \mathbf{P}}  d\left(\mathbf{c}_i, \mathbf{p}_j\right) \leq \tau\right) /|\mathbf{C}|
\end{equation}
\noindent
where $\mathds{1}(\mathbf{x})$ is a binary indicator function that returns 1 if the  predicate $\mathbf{x}$ is true  otherwise returns 0. Since the threshold $\tau$ is typically small~\cite{Pexeso} (otherwise, it would introduce many falsely matched cells), the predicate  $\left(\min _{\mathbf{p}_j \in \mathbf{P}}  d\left(\mathbf{c}_i, \mathbf{p}_j\right)\leq\tau\right)$  is typically NOT true.  Consequently, $\mathds{1}(\cdot)$ frequently returns a value of 0, resulting in $J(\mathbf{C}, \mathbf{P})$ being close to 0 for numerous different $\mathbf{C}$. Hence, the projection values tend to be small and similar to each other, resulting in a significant information loss.
\begin{example}
\label{example-4}
Consider the example in Fig.~\ref{fig:exm1}. According to the cells within the columns,  the column matrix $\mathbf{C}_Q$ is supposed to be similar to $\mathbf{C}_1$ while dissimilar to $\mathbf{C}_3$. Thus, we assume that  $\mathbf{C}_Q = [[-0.1, 0.2], [0.2, 0.3], [-0.2,$ $ 0.3]], \mathbf{C}_1 = [[-0.1, 0.25], [0.15, 0.35], $ $ [-0.15, 0.3]]$, and $\mathbf{C}_3 = [[-0.8, 0.2],$ $[0.3, 0.8], [0.5, 0.3]]$. We also assume that a proxy column matrix $\mathbf{P} = [[1, -1], [-1, -1]]$ is used, the Euclidean distance is adopted as $d(\cdot)$ and the threshold $\tau = 0.1$. If we set $\pi_{\mathbf{P}}(\mathbf{C}) = J(\mathbf{C}, \mathbf{P})$, we would get $\pi_{\mathbf{P}}(\mathbf{C}_Q) = \pi_{\mathbf{P}}(\mathbf{C}_1) = \pi_{\mathbf{P}}(\mathbf{C}_3) = 0$, despite the differences of the three columns.
\end{example}

This observation motivates us to explore a column projection mechanism to better capture the c2pc joinabilities. 
Since both the column $\mathbf{C}$ and proxy column $\mathbf{P}$ can be treated as order-insensitive cell sets, we resort to the maximum bipartite matching which has been extensively employed in set similarity measurement~\cite{SilkMoth,TokenJoin}. Given an input column $\mathbf{C} = \{\mathbf{c}_1, \mathbf{c}_2, \dots, \mathbf{c}_n\}$, a proxy column $\mathbf{P} = \{\mathbf{p}_1, \mathbf{p}_2, \dots, \mathbf{p}_m\}$, and a similarity measurement $\operatorname{sim}(\cdot)$, we construct a bipartite graph $\mathcal{G} = (\mathbf{C}, \mathbf{P}, E)$, where $\mathbf{C}$ and $\mathbf{P}$ are two disjoint vertex sets, $E = \{e_{ij}\}$ is an edge set where $e_{ij}$ connects vertices $\mathbf{c}_i$ and $\mathbf{p}_j$ and associated with a weight $\operatorname{sim}(\mathbf{c}_i, \mathbf{p}_j)$.
\begin{myDef}
    \textnormal{\textbf{(Maximum Weighted Bipartite Matching)}}. Given a bipartite graph $\mathcal{G} = (\mathbf{C}, \mathbf{P}, E)$, the maximum weighted bipartite matching aims to
    find a subset of edges   $M \subset E$ that maximizes the sum of edge weights while ensuring no two edges in $M$ share a common vertex. The matching value $\mathcal{M}(\mathbf{C}, \mathbf{P})$ is formulated as follows:
\begin{equation}
\label{eq:BM}
\begin{aligned}
&  {\mathcal{M}}(\mathbf{C}, \mathbf{P})=\max \sum_{i=1}^n \sum_{j=1}^m u_{i j} \operatorname{sim}\left(\mathbf{c}_{i},\mathbf{p}_{j}\right),  \text{subject to:} \\
% & \text{subject to:}\\
& \textnormal{(i)} \; \forall  i \in\{1,2, \ldots, n\}, \sum_{j = 1}^m  u_{i j} \leq 1,  u_{i j} \in\{0,1\}\\
& \textnormal{(ii)} \; \forall j \in\{1,2, \ldots, m\}, \sum_{i = 1}^n  u_{i j} \leq 1,  u_{i j} \in\{0,1\}
\end{aligned}
\end{equation}
\end{myDef}
\noindent
where $u_{ij} = 1$ (or $0$) indicates the edge $e_{i j}$ is (or not) in $M$.

However, since the time complexity of weighted bipartite matching is 
% $\mathcal{O}\left((n+m)^3 \log (n+m)\right)$,
$\mathcal{O}(\{\max(m,n)\}^3)$ using the classical Hungarian algorithm, it is rather inefficient to perform the computation.
% Moreover, deriving gradients with respect to the proxy columns in the subsequent learning phase is not straightforward.
To tackle this,
% instead of using $\mathcal{M}$ as the column mapping operation,
we remove the second constraint (ii) in Equation~(\ref{eq:BM}) to formulate an approximate graph matching (AGM) problem. In this way, two edges are allowed to have common vertices in set $\mathbf{P}$. Thus, the AGM can be solved by a lightweight greedy mechanism and we define the projection $\pi_{\mathbf{P}}(\mathbf{C})$ as the result  $\mathcal{M}'(\mathbf{C}, \mathbf{P})$ of AGM as follows:
% for each vertex $\mathbf{c}_i$, select the edge with the maximum weight $sim(\mathbf{c}_i, \mathbf{p}_j)$:
\begin{equation}
\label{eq:ABM}
     \pi_{\mathbf{P}}(\mathbf{C}) = \mathcal{M}'(\mathbf{C}, \mathbf{P})=\sum_{i=1}^n  \max _{\mathbf{p}_j\in \mathbf{P}} \operatorname{sim}\left(\mathbf{c}_{i},\mathbf{p}_{j}\right)
\end{equation}
\noindent This approximation reduces the time complexity to $\mathcal{O}(mn)$, and the operations can be executed on a GPU, further enhancing efficiency. We use dot product as the measurement $\operatorname{sim}(\cdot)$.



Finally, given a proxy column set $\mathcal{P} = \{\mathbf{P}_1, \mathbf{P}_2, \dots, \mathbf{P}_l \}$, we can obtain the column embedding of column $C$:
% To incorporate more information from different proxy columns, we apply a set of proxy columns $\mathcal{P} = \{\mathbf{P}_1, \mathbf{P}_2, \dots, \mathbf{P}_l \}$ and concatenate the column mapping values to obtain the column representation as follows:
\begin{equation}
\label{eq:pi}
   % f(C) = \phi(\mathbf{C}) = [\mathcal{M}'(\mathbf{C}, \mathbf{P}_1), \mathcal{M}'(\mathbf{C}, \mathbf{P}_2), \cdots, \mathcal{M}'(\mathbf{C}, \mathbf{P}_l)]
    f(C)  =  \phi(\mathbf{C}) = \left[\pi_{\mathbf{P}_1} (\mathbf{C}), \pi_{\mathbf{P}_2}(\mathbf{C}), \ldots \pi_{\mathbf{P}_l} (\mathbf{C})\right] 
\end{equation}

\noindent where $\pi_{\mathbf{P}_i}(\cdot)$ is desinged as Equation (\ref{eq:ABM}).



% \begin{prop}
%     The projection value $\pi_{\mathbf{P}}(\mathbf{C})$ in Equation\textnormal{~(\ref{eq:ABM})} is an upper bound of $\mathcal{M}(\mathbf{C}, \mathbf{P})$ defined in Equation\textnormal{~(\ref{eq:BM})}.
% \end{prop}

\begin{myThm}
    The proposed column representation $f(C)$ is size-unlimited  and permutation-invariant.
\end{myThm}

\begin{proof}
   The  transformation $h(\cdot$) from the input column $C$ to column matrix $\mathbf{C}$ is size-unlimited, as all cells are processed independently. The function $\pi_{\mathbf{P}_i} (\cdot)$ in Equation (\ref{eq:ABM}) is also size-unlimited, as it will go through each cell embedding $\mathbf{c}_i$. Thus, $f(C)$ is size-unlimited.
    Permuting the order of cells in $C$ changes the order of $\{\mathbf{c}_i\}$ in column matrix $\mathbf{C}$. However, $\pi_{\mathbf{P}_i} (\cdot)$ is independent of this order,  making $\phi(\mathbf{C})$ permutation-invariant. Thus, $f(C)$ is permutation-invariant.
\end{proof}

% \subsubsection{Design Analysis}
\noindent \textbf{Discussion.}
We now analyze the correlation between the column projection in Equation (\ref{eq:ABM}) and the joinability definition in Equation (\ref{eq:js2}).
First, we substitute the distance function $d(\cdot)$ in Equation (\ref{eq:js2}) with a similarity function $\operatorname{sim}(\cdot)$ and correspondingly replace the distance threshold $\tau$ with a similarity threshold $\alpha$. Since the smaller distance means the higher similarity, we can rewrite the Equation (\ref{eq:js2}) as follows:
\begin{equation}
\label{eq:js3}
    J(\mathbf{C}, \mathbf{P})= \sum_{i=1}^n \mathds{1}\left(\max _{\mathbf{p}_j \in \mathbf{P}}  \operatorname{sim}\left(\mathbf{c}_i, \mathbf{p}_j\right)>\alpha\right) /|\mathbf{C}|
\end{equation}

It is observed that we can simplify 
% our designed column projection function is a smoothy version of
the term $\sum_{i=1}^n \mathds{1}\left(\max _{\mathbf{p}_j \in \mathbf{P}}  \operatorname{sim}\left(\mathbf{c}_i, \mathbf{p}_j\right)>\alpha\right)$ in Equation (\ref{eq:js3}) by eliminating the inequality comparison condition $> \alpha$ and the step-like indicator function $\mathds{1}(\cdot)$, and obtain the Equation (\ref{eq:ABM}).
This indicates that our designed column projection is a
smoothed version of the primary component of Equation (\ref{eq:js3}).
It mitigates information loss from the step-like indicator function, preserving more c2pc relationship details.
 

\begin{example}
\label{example-6}
    Continuing with Example~\ref{example-4}, applying the AGM-based column projection yields  $\pi_{\mathbf{P}}(\mathbf{C}_1) = -0.3$, $\pi_{\mathbf{P}}(\mathbf{C}_2) = -0.55$, and $\pi_{\mathbf{P}}(\mathbf{C}_3) = 0.3$. Thus, the phenomenon of information loss is alleviated.
\end{example}


\subsection{Rank-Aware Contrastive Learning}
\label{subsubsec:PivotLearn}
To achieve Property~\ref{prop:3}, it is crucial to identify good proxy column matrices $\mathcal{P} \in \mathbb{R}^{l \times m \times d }$ that can map the joinable columns closer and push the non-joinable columns far apart in the column embedding space. Traditional pivot selection methods in metric spaces~\cite{ZhuCGJ22} can be extended to select proxy columns from the given column repository $\mathcal{R}$, and then derive the corresponding proxy column matrices. However, these methods are constrained in the subspace $\mathcal{R}$ of the column space  $\mathbb{C}$, and are not designed for order-preserving property, resulting in inferior accuracy (see ablation study in Section~\ref{sec:exp_ablation}). Also, it is non-trivial to directly select good proxy columns that can achieve order-preserving, as columns consist of discrete cell values. To this end, we present a novel perspective that regards the continuous proxy column matrices $\mathcal{P} \in \mathbb{R}^{l \times m \times d }$ as learnable parameters, and introduce the rank-aware contrastive learning paradigm 
to learn $\mathcal{P}$
guided by the order-preserving objective.


% \subsubsection{Contrastive Learning Paradigm}

% \noindent \textbf{Contrastive Learning Paradigm.}
Order-preserving entails high similarity for joinable column embeddings and low similarity for non-joinable ones. We introduce the contrastive learning paradigm to achieve this objective. Specifically, given an anchor column $C$, a positive column $ C^{+}$ with high joinability $J(C, C^+)$,  a negative column set $\{C_i^-\}_{i=1}^u$ with low joinabilities $\{J(C, C_i^-)\}_{i=1}^u$, and the column embedding function $f(\cdot)$, we have InfoNCE loss~\cite{Moco} as follows:
\begin{equation}
\label{eq:cl}
\mathcal{L} =-\log \frac{e^{  f(C)^{\top} f(C^+)/ t}}{ e^{ f(C)^{\top} f({C}^+)/ t} + {\sum_i e^{  f({C})^{\top} f({C_i}^-)/ t}}}
\end{equation}
\noindent
where $t$ is a  temperature scaling parameter and we fix it to be 0.08 empirically. Note $ f(C)^{\top} f(C_i)$ denotes the  cosine similarity, as we normalize the column embedding to ensure that $\| f(\cdot) \| = 1$.  

\noindent \textbf{Rank-Aware Optimization.}
% \subsubsection{Rank-Aware Optimization}
In traditional InfoNCE loss, each anchor column has just one positive column, and thus, it is difficult to distinguish the ranks of different positive columns~\cite{rank}. To incorporate the rank awareness of positive
columns, we introduce a positive ranking list $L_C = [C_1^+, C_2^+, \dots, C_s^+]$ for each anchor $C$. The positive columns within $L_C$ are sorted in descending order by joinabilities, i.e., $J(C, C_1^+) > J(C, C_2^+) > \dots > J(C, C_s^+)$. During training, each column $C_j^+ \in \mathcal{R}$ is regarded as a positive column or pseudo-negative column. Specifically, we recursively regard $C_j^+$ as a positive column and $\{C_{j+1}^+, C_{j+2}^+, \dots, C_s^+\}$ as pseudo-negative columns.
The pseudo-negative columns, along with the true negative columns $\{C_i^-\}_{i=1}^u$, are combined to form the negative columns for training, as shown in Fig.~\ref{fig:rank}.
Note that, the negative column set $\{C_i^-\}_{i=1}^u$ is generated from the negative queue $\mathcal{Q}$, which will be detailed  in Algorithm~\ref{algo:algo1} later.
% It is a common generation method~\cite{Deepjoin, Moco}, since most column pairs in the huge repository have low joinabilities.
As for the positive ranking list, we design two kinds of data generation strategies to automatically construct it (see Section~\ref{subsec:data_gen}).
We adopt the rank-aware contrastive loss~\cite{rank} $\mathcal{L}_{rank} = \sum_{j=1} ^ s \ell_j$, where $\ell_j$ is defined as follows:
% \begin{equation}
% \label{eq:rcl}
% \ell_j =- \log \frac{e^{  f(C)^{\top}f(C_j^+)/ \tau}}{{\sum \limits_{t \geq j} e^{  f({C})^{\top} f({C_t}^+)/ \tau}} + {\sum_i e^{  f({C})^{\top} f({C_i}^-)/ \tau}}}
% \end{equation} 
\begin{equation}
\label{eq:rcl}
\ell_j =- \log \frac{e^{  f(C)^{\top}f(C_j^+)/ t}}{{ Z+\sum \limits_{r \geq (j+1)} e^{  f({C})^{\top} f({C_r}^+)/ t}}  }
\end{equation} 


\noindent where $Z = { e^{ f(C)^{\top} f({C}{_j}^+)/ t} + {\sum_i e^{  f({C})^{\top} f({C_i}^-)/ t}}}$ in Equation~(\ref{eq:cl}). Equation~(\ref{eq:rcl}) takes the ranks into consideration. Specifically, minimizing $\ell_j$ requires
% increasing $f(C)^{\top} f(C_j^+)$ and
decreasing $f(C)^{\top} f(C_{j+1}^+)$; while minimizing $\ell_{j+1}$ requires increasing $f(C)^{\top} f(C_{j+1}^+)$. Thus, $\ell_j$ and $\ell_{j+1}$ will compete for the  $f(C)^{\top} f(C_{j+1}^+)$. Since $C_{j+1}^+$ would be regarded as negative columns $j$ times while $C_j^+$ would be regarded as negative columns only $(j-1)$ times,  it would guide a ranking of positive columns with $f(C)^{\top} f(C_j^+) > f(C)^{\top} f(C_{j+1}^+)$.




\begin{figure}
  \centering
  \includegraphics[width=1\linewidth]{Rank.pdf} \vspace{-7mm}
  \caption{An illustration of rank-aware contrastive learning.}
  \label{fig:rank}
  \vspace{-1mm}
\end{figure}




\begin{algorithm}[!tb]

	\small
	\LinesNumbered
	\caption{\protect\mbox {\textsf{Rank-aware Contrastive Learning (RCL)}}}
 \label{algo:algo1}
        \KwIn{a column repository $\mathcal{R}$, the training epoch $\mathcal{E}$, the list $L$ of positive ranking lists for each column,  the length $s$ of each ranking list,
        % the target/momentum pivot columns $\mathcal{P}^\text{t}$/ $\mathcal{P}^\text{m}$,
        the length $\beta$ of the queue $\mathcal{Q}$}
	\KwOut{the learned proxy column matrices  $\mathcal{P}^\text{t}$}
	Initialize target proxy column matrices  $\mathcal{P}^\text{t}$ and momentum proxy column matrices $\mathcal{P}^\text{m}$ in the same way\\
	\ForEach{ e $\in \left \{0, 1, ..., \mathcal{E}\right \}$}
    {     Get a $batch$ of columns from $\mathcal{R}$\\
	   $\mathcal{Q} \leftarrow$ Enqueue($\mathcal{Q}$, $batch$) \\
           \If{$len(\mathcal{Q}) >= \beta+1$}
           {
              CurrentBatch $B \leftarrow$ Dequeue($\mathcal{Q}$)\\
              \ForEach{ column $C$ in $B$  }
              {     Get the positive ranking list $L_C$ of $C$ from $L$\\
                   \ForEach{ j $\in \left \{0, 1, ..., s\right \}$  }
                   { 
                     $C_j^+ \leftarrow L_C\left[j\right]$\\
                     $\{C_r^+\} \leftarrow L_C\left[j+1:\right]$\\
                     $\{C_i^-\} \leftarrow$ All the columns in the current $\mathcal{Q}$\\
                     Compute $\ell_j$ using $C$, $C_j^+$,$\{C_r^+\}$, and $\{C_i^-\}$ \text{// Equation (\ref{eq:rcl}) }\\
                   }
                   Compute $\mathcal{L}_C$ using $\{\ell_j\}$
              }
              Compute $\mathcal{L}_B$ using $\{\mathcal{L}_C\}$\\
              Compute the gradients $\nabla_{{\mathcal{P}}^\text{t}}\mathcal{L}_B$\\
              Update $\mathcal{P}^\text{t}$ using stochastic gradient descent\\
              Update $\mathcal{P}^\text{m}$ with momentum\text{// Equation (~\ref{eq:mom}) }\\
              
           }
	
	}
	\Return{$\mathcal{P}^\text{t}$}
  
\end{algorithm}


In practice, considering the training efficiency, we freeze the cell embedding function $h(\cdot)$, and thus, the only learnable parameters are the proxy column  matrices $\mathcal{P} = \{\mathbf{P}_1, \mathbf{P}_2, \dots, \mathbf{P}_l \}$.
% \textcolor{blue}{Note that, since the computation involves a non-differentiable operation -- $\operatorname{max(\cdot)}$, we apply straight-through gradient estimation~\cite{sun2024learning} to back-propagate the gradient from rank-aware contrastive loss.}
Since enlarging the number of negative samples typically brings performance improvement in contrastive learning~\cite{camper}, we maintain a  negative column queue $\mathcal{Q}$ with the pre-defined length $\beta$ to consider more negative columns.
We present the rank-aware contrastive learning process (\textsf{RCL}) in Algorithm~\ref{algo:algo1}.  At the beginning, \textsf{RCL} would not implement the gradient update until the  queue reaches the pre-defined length $\beta$ + 1 (line 5). Next, the oldest batch is dequeued and becomes the current batch (line 6).
For each column $C$ in the current batch, \textsf{RCL} gets the positive ranking list $L_C$ (line 8), and there are $s$ iterations to be performed to compute the loss $\mathcal{L}_C$. 
In each iteration, \textsf{RCL} first gets the positive column $C_j^+$, the pseudo-negative columns $\{C_r^+\}$, and  the true negative columns $\{C_i^-\}$ (line 10--12).
% Note that the true negative columns are from the current negative queue, which is a common strategy~\cite{starmine,selfKG}, as most column pairs in the huge repository have low joinabilities.
Then, \textsf{RCL} computes the loss $\ell_j$ using Equation (\ref{eq:rcl}). After $s$ iterations,  the loss $\mathcal{L}_C$ can be obtained (line 14). 
After getting losses of each column in the current batch, \textsf{RCL} computes $\mathcal{L}_B$ and the gradients of proxy column  matrices (line 15--16).
We adapt the momentum technique~\cite{Moco} to mitigate the obsolete column embeddings. Specifically, two sets of proxy column  matrices (i.e., the target proxy column  matrices ${\mathcal{P}}^\text{t}$ and the momentum proxy column matrices ${\mathcal{P}^\text{m}}$) are maintained. While ${\mathcal{P}}^\text{t}$ is instantly updated with the backpropagation, ${\mathcal{P}}^\text{m}$ is updated with momentum as follows:
\begin{equation}
\label{eq:mom}
    \mathbf{\mathcal{P}}^\text{m} \leftarrow \alpha \cdot \mathbf{\mathcal{P}}^\text{m}+(1-\alpha) \cdot \mathbf{\mathcal{P}}^\text{t}, \alpha \in[0,1)
\end{equation}

\noindent where  $\alpha$ is the momentum coefficient. Note that ${\mathcal{P}}^\text{t}$ and ${\mathcal{P}}^\text{m}$ are initialized in the same way before training (line 1).
% The learned ${\mathcal{P}}^\text{t}$ are utilized to compute the embeddings for both the columns in the repository and the query column during online process.
The learned $\mathcal{P}^\text{t}$ is used to compute the column embeddings during offline and online processes.

% In practice, considering the training efficiency, we freeze the cell embedding function $h(\cdot)$, and thus, the only learnable parameters are the proxy columns $\mathcal{P} = \{\mathbf{P}_1, \mathbf{P}_2, \dots, \mathbf{P}_l \}$.
% Since enlarging the number of negative samples typically brings performance improvement in contrastive learning~\cite{selfKG,ContrastiveBox}, we maintain a  negative column queue $\mathcal{Q}$ with the pre-defined length $\beta$ to consider more negative columns.
% % during the mini-batch training, a negative column queue $\mathcal{Q}$ is maintained to store the columns in previous batches. At the beginning, we would not implement the gradient update until the  queue reaches the pre-defined length $\beta$ + 1. Then, the oldest batch is dequeued to become the current batch, and all the other columns in $\mathcal{Q}$ form the negative column set.
% We adapt the momentum technique~\cite{Moco} to avoid the obsolete column representations. Specifically, two sets of proxy columns (i.e., the target proxy columns ${\mathcal{P}}^\text{t}$ and the momentum proxy columns ${\mathcal{P}^\text{m}}$) are maintained. While ${\mathcal{P}}^\text{t}$ is instantly updated with the backpropagation, ${\mathcal{P}}^\text{m}$ is updated with momentum as follows:
% \begin{equation}
% \label{eq:mom}
%     \mathbf{\mathcal{P}}^\text{m} \leftarrow \alpha \cdot \mathbf{\mathcal{P}}^\text{m}+(1-\alpha) \cdot \mathbf{\mathcal{P}}^\text{t}, \alpha \in[0,1)
% \end{equation}

% Note ${\mathcal{P}}^\text{t}$ and ${\mathcal{P}}^\text{m}$ are initialized in the same way before training.
% The learned ${\mathcal{P}}^\text{t}$ are utilized to compute the embeddings for both the columns in the repository and the query column.






\subsection{Index and Search}
\label{subsec:OnlineSearch}
After contrastive learning, \textsf{Snoopy} uses the learned $\mathcal{P}^\text{t}$ to pre-compute the embeddings of all the columns in the repository $\mathcal{R}$.
% To further accelerate the search process, \textsf{Snoopy} adopts the approximate nearest neighbor (ANN) search methods. Thus, during the offline stage, 
Then, \textsf{Snoopy} can construct the indexes for column embeddings using any prevalent indexing techniques, to boost the online approximate nearest neighbor (ANN) search. Since graph-based methods are a proven superior trade-off in terms of accuracy versus efficiency~\cite{WangXY021}, \textsf{Snoopy} adopts HNSW~\cite{HNSW} to construct indexes.

For online processing, when a query column $C_Q$ comes, \textsf{Snoopy} uses the learned $\mathcal{P}^\text{t}$ to get $f(C_Q)$. Compared with the existing PTM-based methods, \textsf{Snoopy}'s online encoding process is more efficient due to the proposed lightweight AGM-based column projection. Then, \textsf{Snoopy}  performs the ANN search using the indexes constructed offline and finds the top-$k$ similar column embeddings to $f(C_Q)$ from the Vector Storage. Finally, the corresponding top-$k$ joinable columns in the table repository are returned. We analyze the time complexity of  the online and offline stages in Appendix A.

% \vspace{1mm}
% \noindent \textbf{Time Complexity}. 
% We denote the cardinality of proxy column set  as $l$ and the cardinality per proxy column as $m$. During the offline stage, the complexity of pre-computing all the column embeddings is $\mathcal{O}(ml|\overline{C}||\mathcal{R}|) = \mathcal{O}(|\overline{C}||\mathcal{R}|)$, where $|\mathcal{R}|$ denotes the number of columns in the repository, and $|\overline{C}|$ denotes the average size of columns in the repository; and the time complexity of index construction using HNSW is $\mathcal{O}(|\mathcal{R}| \operatorname{log}|\mathcal{R}|)$. During the online stage, the complexity of computing the query column embedding is $\mathcal{O}(ml|C_Q|) = \mathcal{O}(|C_Q|)$, where $|C_Q|$ is the size of the query column $C_Q$; and the time complexity of ANN search is  $\mathcal{O}(\operatorname{log}|\mathcal{R}|)$. 
% Consequently, the overall time complexity of online processing is $\mathcal{O}(|C_Q| + \operatorname{log}|\mathcal{R}|)$.


\subsection{Training Data Generation}
\label{subsec:data_gen}
During contrastive learning, the negative column set is generated from the negative queue $\mathcal{Q}$, as most column pairs in the huge repository have low joinabilities.
As for the positive columns,
% previous work~\cite{Deepjoin}   
% utilizes the exact algorithms to label some column pairs in the table repository with a joinability score exceeding a specified high threshold.
% However, positive column pairs with high joinability scores are rare in the real table repository. Moreover,
% in our rank-aware paradigm, we require each anchor to have a list of positives, incurring numerous anchor columns struggling to find enough positive columns. Additionally, relying on the data in the repository, it tends to find positive columns lacking diversity.
% To address the aforementioned limitations,
we design two kinds of data generation strategies.
Our objective is to synthesize positive columns with expected joinability scores. In this way, we can generate a ranked 
positive list $L_C$ according to given ranked  scores.

% \begin{algorithm}[!tb]
 
% 	\small
% 	\LinesNumbered
%  \caption{\textsf{Text-level Column Synthesis (TCS)}}
% 	\label{alg:tcs}
%         \KwIn{a column $C$, the sample rate $x\%$, the distance function $d$ and threshold $\tau$}
% 	\KwOut{the anchor column $C_a$ with positive column $C_a^+$}
% 	Divide $C$ into two subsequences $C_a$ and $C_b$\\
%     $S_a \leftarrow$Randomly sample $x\%$ cells from $C_a$\\
%     $S_a' \leftarrow \emptyset$\\
  
% 	\ForEach{ c in $S_a$}
%     {     Randomly sample an augmentation operator $\operatorname{aug}(\cdot)$\\
% 	   $c' \leftarrow \operatorname{aug}(c)$ \\
%           $Flag \leftarrow d(h(c), h(c')) \geq \tau$\\ 
%            \If{$Flag == 1$}
%            {
%               $c' = c$
%            }
%            $S'_a.append(c')$
	
% 	}
%     $C_a^+ \leftarrow \text{shuffle}(S'_a\|C_b)$\\
% 	\Return{$C_a$, $C_a^+$}
 
% \end{algorithm}




% \begin{figure}
%   \centering
%   \includegraphics[width=1\linewidth]{DataGen.pdf} \vspace{-7mm}
%   \caption{{Illustration of text-level positive column synthesis.}
%   \label{fig:text-level}}
%   \vspace{-5mm}
% \end{figure}

% \subsubsection{Text-level Synthesis}
\vspace{1mm}
\noindent \textbf{Text-level Synthesis}.
Given a column $C$, we randomly divide it into two sub-columns $C_a$ and $C_b$. We use $C_a$ as the anchor column, and $C_b$ as the residual column. 
The reason why we maintain the residual column $C_b$ will be detailed later.
Now we illustrate how to synthesize a positive column $C_a^+$ for $C_a$ with a joinability approximated to a specified score $x\%$.


We randomly sample $x\%$ cells from the anchor $C_a$ to form a sampling column $S_a$. However, directly using $S_a$ as the positive column has two shortcomings: (i) the matched cells between $C_a$ and $S_a$ are exactly the same, without covering the semantically-equivalent cases, and (ii) each cell in $S_a$ can find matched cell in the anchor $C_a$, which is not realistic.
To tackle (i), we follow~\cite{Watchog,starmine} to apply straightforward augmentation operations to each cell $c \in S_a$ in text-level.  Note we should guarantee that the augmentation operator would not be too aggressive that $c$ and the augmented $c'$ are no longer matched under the distance threshold $\tau$ in Definition~\ref{def:cellmatch}. If that happens, we give up this augmentation and let $c'= c$. After that, we obtain an augmentation version $S'_a$ of $S_a$.
To tackle (ii), we concatenate $S'_a$ with the residual sub-column $C_b$ to obtain $S'_a || C_b$. Since we can assume that duplicates are few in the original column~\cite{autofuzzyjoin},  most cells in the sub-column $C_b$ do not have matches in the sub-column $C_a$, effectively simulating non-matched cells between the positive column and the anchor column.
Finally, we apply shuffling to $S'_a || C_b$ to obtain the required positive column $C_a^+$.
% Note that, $J(C_a, C_a^+)$ may slightly exceed $x\%$, but it would not affect the efficacy of our model training. In this way, we can easily construct $L_C$ according to a given sorted list of joinability scores.



% \subsubsection{Embedding-level Synthesis}
\vspace{1mm}
\noindent\textbf{Embedding-level Synthesis.}
An obstacle to text-level synthesis is that it is hard to determine the granularity of the applied augmentation operator, especially in an extremely small threshold $\tau$. To this end, we propose another embedding-level column synthesis strategy.

Given a column matrix $\mathbf{C}$, analogous to the text-level method, we first horizontally divide $\mathbf{C}$ into two sub-matrices $\mathbf{C}_a$ and $\mathbf{C}_b$. Then we randomly samples $x\%$ rows from $\mathbf{C}_a$, and denote it as $\mathbf{S}_a$. Note that each row of $\mathbf{S}_a$ is a cell embedding $\mathbf{c}$. Next, we  augment each cell embedding vector $\mathbf{c}$ by random perturbation. Specifically, we generate a random vector $\mathbf{r} $ following normal distribution $\mathcal{N}(0, \sigma)$, and gets the augmented $\mathbf{c}' = \mathbf{c} + \mathbf{r}$. Here, we also need a validation step to ensure that $d(\mathbf{c}, \mathbf{c}') \leq \tau$. But the advantage is that we can reduce $\sigma$ by multiplying a coefficient $\gamma \in (0,1)$ until the augmentation is proper to make $\mathbf{c}'$ matches $\mathbf{c}$ under the threshold $\tau$. Since the augmentation is operated in the continual embedding space, it is more flexible than the text-level method.



% \begin{algorithm}[!tb]
 
% 	\small
% 	\LinesNumbered
%  \caption{\protect\mbox{\textsf{Embedding-level Column Synthesis (ECS)}}}
% 	\label{alg:mcsa}
%          \KwIn{a column matrix $\mathbf{C}$, the sample rate $x\%$, the distance function $d$ and threshold $\tau$}
% 	\KwOut{\protect\mbox{the anchor matrix $\mathbf{C}_a$ with positive matrix $\mathbf{C}_a^+$}}
% 	Divide $\mathbf{C}$ into two matrices $\mathbf{C}_a$ and $\mathbf{C}_b$\\
%     $\mathbf{S}_a \leftarrow$Randomly sample $x\%$ rows from $\mathbf{C}_a$\\
%     $\mathbf{S}_a' \leftarrow \emptyset$ \\
  
% 	\ForEach{ \text{row} $\mathbf{c}$  in $\mathbf{S}_a$}
%     {     $Flag \leftarrow 1$ \\
%           \While{Flag == \textnormal{1}} 
%           {Generate a random vector $\mathbf{r}\sim \mathcal{N}(0, std)$ \\
% 	   $\mathbf{c}' \leftarrow \mathbf{c}+\mathbf{r}$ \\
%           $Flag \leftarrow d(\mathbf{c}, \mathbf{c}') \geq \tau$\\ 
%           std=std⋅γstd = std \cdot \gamma
%           } 
%           S′a.append(c′)\mathbf{S}'_a.append(\mathbf{c}')
% 	}
%     C+a←shuffle(S′a‖C_a^+ \leftarrow \text{shuffle}(\mathbf{S}'_a\|\mathbf{C}_b)\\
% 	\Return{\mathbf{C}_a\mathbf{C}_a, \mathbf{C}_a^+\mathbf{C}_a^+}
 
% \end{algorithm}



% Given a column matrix \mathbf{C}\mathbf{C}, analogous to the text-level method, \textsf{ECS} first horizontally divides \mathbf{C}\mathbf{C} into two matrices \mathbf{C}_a\mathbf{C}_a and \mathbf{C}_b\mathbf{C}_b (line 1). Then it randomly samples x\%x\% rows from \mathbf{C}_a\mathbf{C}_a and denote it as \mathbf{S}_a\mathbf{S}_a (line 2). Note each row of \mathbf{S}_a\mathbf{S}_a is a cell embedding \mathbf{c}\mathbf{c}. Next, \textsf{ECS}  augments each cell embedding vector \mathbf{c}\mathbf{c} by random perturbation. Specifically, it generates a random vector \mathbf{r} \mathbf{r}  following normal distribution \mathcal{N}(0, \sigma)\mathcal{N}(0, \sigma) (line 7), and gets the augmented \mathbf{c}' = \mathbf{c} + \mathbf{r}\mathbf{c}' = \mathbf{c} + \mathbf{r} (line 8). Here it also needs a validation step to ensure that d(\mathbf{c}, \mathbf{c}') \leq \taud(\mathbf{c}, \mathbf{c}') \leq \tau (line 9). But the advantage is that it can reduce \sigma\sigma by multiplying a coefficient \gamma < 1\gamma < 1 (line 10) until the augmentation is proper to make \mathbf{c}'\mathbf{c}' matches \mathbf{c}\mathbf{c} under the threshold \tau\tau and the Flag is equal to 0. Since the augmentation is operated in the continual embedding space, it is more flexible than the text-level method.




\begin{table*}[ht]
\centering
\caption{\wqruan{The online communication size profiling results of four secure CNN model inference processes outputted by \texttt{CryptFlow2}~\cite{rathee2020cryptflow2} and \hawkeye.} We report the proportion of each operator's online communication size to the total online communication size and the online communication size of each operator. Following Ganesan et al.~\cite{ganesan2022efficient}, we list the online communication size across linear and non-linear operators. }
\scalebox{0.7}{
\begin{tabular}{c|c|cccc}
\hline
Model                          & Framework                                                                              & \multicolumn{3}{c}{\% (GB) of linear operators}                                                                                                                                                                                                                                                      & \% (GB) of non-linear operators                                                       \\ \hline
\multicolumn{1}{l|}{}         &                                                                                        & \multicolumn{1}{c}{\% (GB) of Conv2d}                                                               & \multicolumn{1}{c}{\% (GB) of other linear operators}                                                       & \% (GB) of all                                                                     & \multicolumn{1}{l}{}                                                        \\ \hline
\multirow{3}{*}{DenseNet-121} & \texttt{CrypTFlow2}~\cite{rathee2020cryptflow2} & \multicolumn{1}{c}{\wqruan{94.14\% (646.05GB)}}                                                              & \multicolumn{1}{c}{\wqruan{3.60\% (24.72GB)}}                                                              & \wqruan{97.74\% (670.78GB)}                                                              & \wqruan{2.26\% (15.52GB)}                    \\ \cline{2-6} 
                 & \hawkeye                                                                & \multicolumn{1}{c}{\begin{tabular}[c]{@{}c@{}}\wqruan{93.92\% (657.06GB)}\\ \wqruan{-0.22\% (+11.01GB)}\end{tabular}} & \multicolumn{1}{c}{\begin{tabular}[c]{@{}c@{}}\wqruan{3.77\% (26.33GB)}\\ \wqruan{+0.17\% (+1.61GB)}\end{tabular}} & \begin{tabular}[c]{@{}c@{}} \wqruan{97.69\% (683.39GB)}\\ \wqruan{-0.05\% (+12.61GB)}\end{tabular} & \begin{tabular}[c]{@{}c@{}}\wqruan{2.31\% (16.17GB)}\\ \wqruan{+0.05\% (+0.65GB)}\end{tabular}                       \\ \hline
\multirow{3}{*}{ResNet-50}     & \texttt{CrypTFlow2}~\cite{rathee2020cryptflow2} & \multicolumn{1}{c}{\wqruan{96.17\% (851.76GB)}}                                                              & \multicolumn{1}{c}{\wqruan{2.04\% (18.05GB)}}                                                              & \wqruan{98.21\% (869.81GB)}                                                              & \wqruan{1.79\% (15.83GB)}                        \\ \cline{2-6} 
    & \hawkeye                                                                & \multicolumn{1}{c}{\begin{tabular}[c]{@{}c@{}}\wqruan{96.14\% (863.11GB)}\\ \wqruan{-0.03\% (+11.35GB)}\end{tabular}} & \multicolumn{1}{c}{\begin{tabular}[c]{@{}c@{}}\wqruan{2.05\% (18.43GB)}\\ \wqruan{+0.01\% (+0.38GB)}\end{tabular}}  & \begin{tabular}[c]{@{}c@{}}\wqruan{98.19\% (881.54GB)}\\ \wqruan{-0.02\% (+11.73GB)}\end{tabular} & \begin{tabular}[c]{@{}c@{}}\wqruan{1.81\% (16.25GB)}\\ \wqruan{+0.02\% (+0.42GB)}\end{tabular}                                                                 \\ \hline
\multirow{3}{*}{MobileNet-V3}  & \texttt{CrypTFlow2}~\cite{rathee2020cryptflow2} & \multicolumn{1}{c}{\wqruan{75.14\% (73.30GB)}}                                                               & \multicolumn{1}{c}{\wqruan{18.62\% (18.16GB)}}                                                             & \wqruan{93.76\% (91.46GB)}                                                               & \wqruan{6.24\% (6.09GB)}                          \\ \cline{2-6} 
  & \hawkeye                                                                & \multicolumn{1}{c}{\begin{tabular}[c]{@{}c@{}}\wqruan{74.98\% (73.33GB)}\\ \wqruan{-0.16\% (+0.03GB)}\end{tabular}}   & \multicolumn{1}{c}{\begin{tabular}[c]{@{}c@{}}\wqruan{18.57\% (18.16GB)}\\ \wqruan{-0.05\% (+0.00GB)}\end{tabular}} & \begin{tabular}[c]{@{}c@{}}\wqruan{93.55\% (91.49GB)}\\ \wqruan{-0.21\% (+0.03GB)}\end{tabular}   & \begin{tabular}[c]{@{}c@{}}\wqruan{6.45\% \wqruan{(6.31GB)}}\\ \wqruan{+0.21\% (+0.22GB)}\end{tabular}                                                                   \\ \hline
\multirow{3}{*}{ShuffleNet-V2}   & \texttt{CrypTFlow2}~\cite{rathee2020cryptflow2} & \multicolumn{1}{c}{\wqruan{86.18\% (38.42GB)}}                                                               & \multicolumn{1}{c}{\wqruan{9.20\% (4.10GB)}}                                                              & \wqruan{95.38\% (42.52GB)}                                                               & \wqruan{4.62\% (2.06GB)} \\ \cline{2-6} 
& \hawkeye                                                                & \multicolumn{1}{c}{\begin{tabular}[c]{@{}c@{}}\wqruan{86.14\% (38.87GB)}\\ \wqruan{-0.04\% (+0.45GB)}\end{tabular}}   & \multicolumn{1}{c}{\begin{tabular}[c]{@{}c@{}}\wqruan{9.20\% (4.15GB)}\\ \wqruan{+0.00\% (+0.05GB)}\end{tabular}}  & \begin{tabular}[c]{@{}c@{}}\wqruan{95.34\% (43.02GB)}\\ \wqruan{-0.04\% (+0.50GB)}\end{tabular}   & \begin{tabular}[c]{@{}c@{}}\wqruan{4.66\% (2.10GB)}\\ \wqruan{+0.04\% (+0.04GB)}\end{tabular}                                                                                            \\ \hline
\end{tabular}
}
\label{tab:cryptflow2}
\end{table*}


\section{Performance Evaluation}~\label{sec:exp}
\vspace{-6mm}
\subsection{\wqruan{Implementation}} 
We implement \hawkeye by supplementing about 12k lines of code of Python (including test codes) to the \mpspdz compiler. At first, we modify the instruction generation process of the \mpspdz compiler to implement our proposed block labeling method. Then, we modify the aggregate functions of ReqNode and ReqChild in the \mpspdz compiler to implement our proposed block analysis method. 

\wqruan{For the Autograd library of \hawkeye, we implement all of its operators and functions according to the description of \texttt{PyTorch} API document~\footnote{\url{https://pytorch.org/docs/stable/torch.html}}. Concretely, we implement functions of five \texttt{PyTorch}-related modules (Tensor module, Functional module, NN module, Optimizers, and Dataloader) described in Section~\ref{subsec:autograd_architecture} based on the secure fixed-point numbers computation functions provided by the \mpspdz compiler. The above Autograd library modules are integrated into the \mpspdz compiler as one of its standard libraries. Thus, in \hawkeye, when model designers construct models they want to profile, they can directly import the above modules to construct models in \texttt{PyTorch}.}

% \begin{table*}[ht]
% \centering
% \caption{The online communication round profiling results of four secure CNN model inference processes outputted by \texttt{CryptFlow2}~\cite{rathee2020cryptflow2} and \hawkeye. }
% \scalebox{0.7}{
% \begin{tabular}{c|c|ccc|c}
% \hline
% Model                          & Framework                                                                              & \multicolumn{3}{c|}{\% (Rounds) of linear operators}                                                                                                                                                                                                                       & \% (Rounds) of non-linear operators                                      \\ \hline
% \multicolumn{1}{l|}{}          &                                                                                        & \multicolumn{1}{c|}{\% (Rounds) of Conv2d}                                                     & \multicolumn{1}{c|}{\% (Rounds) of other linear operators}                                    & \% (Rounds) of all                                                        & \multicolumn{1}{l}{}                                                 \\ \hline
% \multirow{3}{*}{DenseNet-121}  & \texttt{CrypTFlow2}~\cite{rathee2020cryptflow2} & \multicolumn{1}{c|}{\wqruan{83.35\% (10,317)}}                                                           & \multicolumn{1}{c|}{\wqruan{10.83\% (1,340)}}                                                           & \wqruan{94.18\% (11,657)}                                                           & \wqruan{5.82\% (720)}                                                         \\ \cline{2-6} 
%                                & \hawkeye                                                                & \multicolumn{1}{c|}{\begin{tabular}[c]{@{}c@{}}\wqruan{83.49\% (10,278)}\\ \wqruan{+0.14\% (-39)}\end{tabular}}   & \multicolumn{1}{c|}{\begin{tabular}[c]{@{}c@{}}\wqruan{10.67\% (1,313)}\\ \wqruan{-0.17\% (-27)}\end{tabular}}   & \begin{tabular}[c]{@{}c@{}}\wqruan{94.15\% (11,591)}\\ \wqruan{-0.03\% (-66)}\end{tabular}   & \begin{tabular}[c]{@{}c@{}}\wqruan{5.85\% (720)}\\ \wqruan{+0.03\% (+0)}\end{tabular}  \\ \hline
% \multirow{3}{*}{ResNet-50}     & \texttt{CrypTFlow2}~\cite{rathee2020cryptflow2} & \multicolumn{1}{c|}{\wqruan{87.52\% (7,155)}}                                                            & \multicolumn{1}{c|}{\wqruan{7.12\% (582)}}                                                             & \wqruan{94.64\% (7,737)}                                                           & \wqruan{5.36\% (438)}                                                         \\ \cline{2-6} 
%                                & \hawkeye                                                                & \multicolumn{1}{c|}{\begin{tabular}[c]{@{}c@{}}\wqruan{87.84\% (7,148)}\\ \wqruan{+0.32\% (-7)}\end{tabular}}     & \multicolumn{1}{c|}{\begin{tabular}[c]{@{}c@{}}\wqruan{7.07\% (575)}\\ \wqruan{-0.05\% (-7)}\end{tabular}}      & \begin{tabular}[c]{@{}c@{}}\wqruan{94.91\% (7,723)}\\ \wqruan{+0.27\% (-14)}\end{tabular}    & \begin{tabular}[c]{@{}c@{}}\wqruan{5.09\% (414)}\\ \wqruan{-0.27\% (-24)}\end{tabular} \\ \hline
% \multirow{3}{*}{MobileNet-V3}  & \texttt{CrypTFlow2}~\cite{rathee2020cryptflow2} & \multicolumn{1}{c|}{\wqruan{80.90\% (12,131)} }                                                          & \multicolumn{1}{c|}{\wqruan{15.02\% (2,252)}}                                                           & \wqruan{95.92\% (14,383)}                                                           & \wqruan{4.08\% (612)}                                                         \\ \cline{2-6} 
%                                & \hawkeye                                                                & \multicolumn{1}{c|}{\begin{tabular}[c]{@{}c@{}}\wqruan{44.94\% (2,234)}\\ \wqruan{-35.96\% (-9,897)}\end{tabular}} & \multicolumn{1}{c|}{\begin{tabular}[c]{@{}c@{}}\wqruan{42.75\% (2,125)}\\ \wqruan{+27.73\% (-127)}\end{tabular}} & \begin{tabular}[c]{@{}c@{}}\wqruan{87.69\% (4,359)}\\ \wqruan{-8.23\% (-10,024)}\end{tabular} & \begin{tabular}[c]{@{}c@{}}\wqruan{12.31\% (612)}\\ \wqruan{+8.23\% (+0)}\end{tabular} \\ \hline
% \multirow{3}{*}{ShuffleNet-V2} & \texttt{CrypTFlow2}~\cite{rathee2020cryptflow2} & \multicolumn{1}{c|}{\wqruan{88.61\% (6,827)} }                                                           & \multicolumn{1}{c|}{\wqruan{7.89\% (608)}}                                                             & \wqruan{96.50\% (7,435)}                                                            & \wqruan{3.50\% (270)}                                                         \\ \cline{2-6} 
%                                & \hawkeye                                                                & \multicolumn{1}{c|}{\begin{tabular}[c]{@{}c@{}}\wqruan{67.37\% (1,745)}\\ \wqruan{-21.24\% (-5,082)}\end{tabular}}  & \multicolumn{1}{c|}{\begin{tabular}[c]{@{}c@{}}\wqruan{23.13\% (599)}\\ \wqruan{+15.24\% (-9)}\end{tabular}}    & \begin{tabular}[c]{@{}c@{}}\wqruan{90.50\% (2,344)}\\ \wqruan{-6.00\% (-5091)}\end{tabular}   & \begin{tabular}[c]{@{}c@{}}\wqruan{9.50\% (246)}\\ \wqruan{+6.00\% (-24)}\end{tabular} \\ \hline
% \end{tabular}
% }
% \label{tab:cryptflow2_round}
% \end{table*}

\begin{figure*}[ht]
    \centering
    \includegraphics[width=0.8\textwidth]{figures/round_crypTFlow.pdf}
    \caption{\wqruan{The online communication round profiling results of four secure CNN model inference processes outputted by \texttt{CryptFlow2}~\cite{rathee2020cryptflow2} and \hawkeye.}}
    \label{fig:round_crypTFlow}
\end{figure*}

\subsection{Accuracy of \hawkeye}~\label{subsec:accuracy_garnet}
\begin{table}[ht]
\centering
\caption{\wqruan{The online/offline communication size profiling results of secure BERT$_{\textsc{base}}$ model inference process outputted by \texttt{CrypTen}~\cite{crypten2020} and \hawkeye.} We report the proportion of each operator's communication size to the total communication size and the communication size of each operator. Following Li et al.~\cite{li2023mpcformer}, we list the communication sizes of MatMul, Gelu, and Softmax operators. }
\scalebox{0.7}{
\begin{tabular}{c|c|c|c}
\hline
Operator                 & Framework                                                                  & \% (GB) of online phase           & \% (GB) of offline phase                                                                  \\ \hline
\multirow{3}{*}{MatMul}  & \texttt{CrypTen}~\cite{crypten2020} & \wqruan{4.75\% (3.22GB)}                                         & \wqruan{7.05\% (1.37GB)}                                                                                                                                     \\ \cline{2-4} 
                         & \hawkeye                                                    & \begin{tabular}[c]{@{}c@{}}\wqruan{4.74\% (3.22GB)} \\ \wqruan{-0.01\% (+0.00GB)}\end{tabular} 
                         & \begin{tabular}[c]{@{}c@{}} \wqruan{7.09\% (1.37GB)}\\ \wqruan{+0.04\% (+0.00GB)}\end{tabular}  
                           \\ \hline
\multirow{3}{*}{Gelu}    & \texttt{CrypTen}~\cite{crypten2020} & \wqruan{26.15\% (17.72GB)}                                       & \wqruan{23.87\% (4.64GB)}                                                                                                                                    \\ \cline{2-4} 
                         & \hawkeye                                                    & \begin{tabular}[c]{@{}c@{}}\wqruan{26.49\% (18.00GB)} \\ \wqruan{+0.34\% (+0.28GB)}\end{tabular} 
                         & \begin{tabular}[c]{@{}c@{}} \wqruan{24.12\% (4.64GB)}\\ \wqruan{+0.25\%(+0.00GB)}\end{tabular}        \\ \hline
\multirow{3}{*}{Softmax} & \texttt{CrypTen}~\cite{crypten2020} & \wqruan{69.10\% (46.83GB)}                                            & \wqruan{69.08\% (13.43GB)}                                                                                                                       \\ \cline{2-4} 
                         & \hawkeye                                                    & \begin{tabular}[c]{@{}c@{}}\wqruan{68.77\% (46.73GB)} \\ \wqruan{-0.33\% (-0.10GB)}\end{tabular} 
                         & \begin{tabular}[c]{@{}c@{}} \wqruan{68.79\% (13.23GB)}\\ \wqruan{-0.29\% (-0.20GB)}\end{tabular} \\
                         \hline
\end{tabular}
}
\label{tab:mpcformer}
\end{table}

\begin{figure}[ht]
    \centering
    \includegraphics[width=0.4\textwidth]{figures/round_crypTen.pdf}
    \caption{\wqruan{The online communication round profiling results of secure BERT$_{\textsc{base}}$ model inference process outputted by \texttt{CrypTen}~\cite{crypten2020} and \hawkeye.}}
    \label{fig:round_crypTen}
\end{figure}
\noindent\textbf{Experiment Setup.} We verify the accuracy of \hawkeye by comparing the static profiling results outputted by \hawkeye with dynamic profiling results obtained from two MPL frameworks, \wqruanother{i.e.,} \texttt{CypTFlow2}~\cite{rathee2020cryptflow2} and \texttt{CrypTen}~\cite{crypten2020}. We first describe two dynamic profiling processes we compare with.  Firstly, Ganesan et al.~\cite{ganesan2022efficient} dynamically profile secure inference processes of four popular CNN models (\wqruanother{i.e.,} DenseNet-121~\cite{Huang_2017_CVPR}, ResNet-50~\cite{7780459}, MobileNetV3~\cite{howard2019searching}, ShuffleNetV2~\cite{Ma_2018_ECCV}) on \texttt{CrypTFlow2}~\cite{rathee2020cryptflow2}. Secondly, Li et al.~\cite{li2023mpcformer} dynamically profile the secure BERT$_{\textsc{base}}$ model inference process on the two-party backend of \texttt{CrypTen}~\cite{crypten2020}.  We run the open-source codes of the two MPL frameworks~\footnote{The code repository address of \texttt{CrypTFlow2} is \url{https://github.com/mpc-msri/EzPC}. We use the codes at commit \textit{4bae530}.}~\footnote{The code repository address of \texttt{CrypTen} is \url{https://github.com/facebookresearch/CrypTen}. We use \texttt{CrypTen0.4.1}.} to obtain the dynamic profiling results. For dynamic profiling on \texttt{CrypTFlow2}, following Ganesan et al.~\cite{ganesan2022efficient}, we set the bit length as $60$ and the bit length of fixed-point numbers' fractional part as $23$. For dynamic profiling on \texttt{CrypTen}, we set the bit length as $64$ and the bit length of fixed-point numbers' fractional part as $16$. We set the statistical security parameter and computation security parameter of these two MPL frameworks as $40$ and $128$. For the input data of CNN models, we set their sizes as $1 \times 3 \times 224 \times 224$. For the input data of BERT$_{\textsc{base}}$, following Li et al.~\cite{li2023mpcformer}, we set their sizes as $1 \times 512 \times 28996$. Because \texttt{CrypTen} does not support the tanh function, we replace the tanh function with the hardtanh function following Li et al.~\cite{li2023mpcformer}. \wqruan{Note that because \texttt{CrypTFlow2} does not have the offline phase, we only show the offline communication cost profiling results of \texttt{CrypTen}. }
% Besides, because the offline preparation of \texttt{CrypTen} does not depend on circuit structures, it can be finished in the constant round, and we do not consider its offline communication round here.

For static profiling on \hawkeye, we first configure the communication cost of \texttt{CryptoFlow2}~\cite{rathee2020cryptflow2} and \texttt{CrypTen}~\cite{crypten2020}. After that, we implement the models profiled by Ganesan et al.~\cite{ganesan2022efficient} and Li et al.~\cite{li2023mpcformer} in \hawkeye. Finally, we run \hawkeye under the same parameter setting with \texttt{CypTFlow2} and \texttt{CrypTen} to obtain the static profiling results. 

\smallskip
\noindent\textbf{Experimental results.} \wqruan{As is shown in Table~\ref{tab:cryptflow2}, Figure~\ref{fig:round_crypTFlow}, Table~\ref{tab:mpcformer}, and Figure~\ref{fig:round_crypTen}, \hawkeye can accurately profile model communication cost on different MPL frameworks. For communication size, the proportions of operators' online/offline communication size to the total online/offline communication size outputted by \hawkeye, \texttt{CrypTFlow2}, and \texttt{CrypTen} are almost the same, i.e., the proportion differences between baselines and \hawkeye are all smaller than 0.87\%.  The slight differences between the communication size profiling results outputted by \hawkeye and dynamic profiling results could be caused by the following two reasons: (1) The communication complexity of basic operations is asymptotic rather than actual. For example, \texttt{CrypTFlow2} analyzes the communication complexity upper bound of truncation and comparison operations. Because we configure the communication cost for truncation and comparison operations of \texttt{CrypTFlow2} with the upper bound rather than the actual complexity, the communication sizes outputted by \hawkeye are slightly larger than those outputted by \texttt{CrypTFlow2}  (2) The implementations of high-level operations are different. For example, \texttt{CrypTen} implements the max operation by mixing the tree-based and pairwise comparison methods, while \hawkeye purely uses the tree-based method. Thus, \texttt{CrypTen} has a larger communication size on the softmax function than \hawkeye.}

\wqruan{For communication rounds, as is shown in Figure~\ref{fig:round_crypTFlow} and Figure~\ref{fig:round_crypTen}, \hawkeye can accurately profile the communication rounds of most CNN operators on \texttt{CrypTFlow2} and all transformer operators on \texttt{CrypTen}. The main difference between the results outputted by \texttt{CrypTFlow2} and \hawkeye lies in the communication rounds of the Conv2d operators in MobileNet-V3 and ShuffleNet-V2 models. The main reason is that the implementation of the group convolution operator, which is used by MobileNet-V3 and ShuffleNet-V2, in \texttt{CrypTFlow2} is not parallel. Therefore, the group convolution operator in \texttt{CrypTFlow2} would require the group number times of communication rounds than the parallel implementation in \hawkeye. The above results show that besides helping model designers analyze model communication costs, \hawkeye could also help MPL framework developers find the performance issues in their implementation. }


% Concretely, for secure CNN model inference processes, although the online communication sizes of operators outputted by \texttt{CrypTFlow2} are slightly lower than those outputted by \hawkeye, the proportions of operators' online communication size to the total online communication size outputted by \texttt{CrypTFlow2} and \hawkeye are almost the same, \wqruanother{i.e.,} the proportion differences between \texttt{CrypTFlow2} and \hawkeye are all smaller than 1.25\%. For the secure BERT$_{\textsc{base}}$ model inference process, the proportions and online communication sizes of operators outputted by \texttt{CrypTen} and \hawkeye are both almost the same. The slight differences between the profiling results outputted by \hawkeye and dynamic profiling results could be caused by the following two reasons: (1) The communication complexity of basic operations is asymptotic rather than actual. For example, \texttt{CrypTFlow2} analyzes the communication complexity upper bound of truncation and comparison operations. Because we configure the communication cost for truncation and comparison operations of \texttt{CrypTFlow2} with the upper bound rather than the actual complexity, the online communication sizes outputted by \hawkeye are slightly larger than those outputted by \texttt{CrypTFlow2}  (2) The implementations of high-level operations are different. For example, \texttt{CrypTen} implements the max operation by mixing the tree-based and pairwise comparison methods, while \hawkeye purely uses the tree-based method. Thus, \texttt{CrypTen} has a larger communication size on the softmax function than \hawkeye. In summary, even though the asymptotic communication complexity of basic operations and different implementations of high-level operations cause slight bias, the bias has negligible impacts on finding the communication bottleneck of models. 

\wqruan{Besides \texttt{CrypTFlow2} and \texttt{CrypTen}, we compare \hawkeye with two mixed-protocol MPL frameworks (\texttt{Delphi}~\cite{mishra2020delphi} and \texttt{Cheetah}~\cite{Cheetah}) that rely on garbled circuit (GC) and HE. Due to the space limitation, we show the experimental results in Appendix~\ref{appendix:mixed-protocol}. Furthermore, to show the accuracy of \hawkeye in secure model training scenarios, we compare \hawkeye with the two-party backend of \texttt{SecretFlow-SPU}, i.e., \texttt{SecretFlow-SEMI2K}, in Appendix~\ref{appdendix:secure_training}.}


\subsection{Efficiency of \hawkeye}
To demonstrate the efficiency of \hawkeye, \wqruan{we compare the runtimes of \hawkeye and the runtimes of \texttt{CrypTFlow2} and \texttt{CrypTen} in the experiments of Section~\ref{subsec:accuracy_garnet}.} All experiments are run on a Linux server equipped with two 32-core 2.30 GHz Intel Xeon CPUs and 512GB of RAM.

\begin{table}[ht]
\centering
\caption{\wqruan{Runtimes of static and dynamic profiling for five secure model inference processes in Section~\ref{subsec:accuracy_garnet}}.  We report the average results of five runs and show the standard deviations in brackets.}
\scalebox{0.7}{
\begin{tabular}{c|cc}
\hline
 Network                        & \multicolumn{1}{c|}{ \wqruan{Static profiling (s)}}   & \wqruan{Dynamic profiling (s)} \\ \hline
Densenet-121           & \multicolumn{1}{c|}{\wqruan{34.38 ($\pm$ 0.62)}}   & \wqruan{4,429.01 ($\pm$ 65.74)}  \\ 
Resnet-50              & \multicolumn{1}{c|}{\wqruan{39.24 ($\pm$ 0.13)}}   & \wqruan{5,698.77 ($\pm$ 80.06)}  \\ 
\wqruanother{Mobilenet-V3}           & \multicolumn{1}{c|}{\wqruan{31.47 ($\pm$ 0.31}}   & \wqruan{811.92 ($\pm$ 1.81)}  \\ 
Shufflenet-V2          & \multicolumn{1}{c|}{ \wqruan{15.50 ($\pm$ 0.12)}}   & \wqruan{319.34 ($\pm$ 3.44)}   \\ 
BERT$_{\textsc{base}}$ & \multicolumn{1}{c|}{\wqruan{471.54 ($\pm$ 4.98)}} & \wqruan{618.90 ($\pm$ 2.11)}  \\ \hline
\end{tabular}}
\label{tab:compilation_time}
\end{table}
\wqruan{As is shown in Table~\ref{tab:compilation_time}, the efficiency of \hawkeye is promising for model communication cost profiling. Concretely, \hawkeye can profile all CNN models within one minute. The profiling time is about eight minutes even for large BERT$_{\textsc{base}}$ model. In contrast, dynamic profiling the models using \texttt{CrypTFlow2} or \texttt{CrypTen} requires much more time than \hawkeye ($1.31\times \sim 145.23\times$). As a result, \hawkeye enables model designers to efficiently profile the model communication cost and design MPC-friendly models agilely.}

\subsection{Ease of Use}~\label{subsec:ease_use}
We then report on the ease of using \hawkeye to profile the communication cost of models in \texttt{PyTorch} by showing an example of modifying a \texttt{PyTorch}-based logistics regression model training codes to be \hawkeye-based.

\begin{lstlisting}[escapechar=!,mathescape,xleftmargin=2em,framexleftmargin=2em, language=python, columns=fullflexible, caption= {An example of modifying a \texttt{PyTorch}-based logistics regression model training codes to be \hawkeye-based. The removed \texttt{PyTorch} codes are highlighted on a red background and labeled with a minus sign at the beginning of the line. The newly added \hawkeye codes are highlighted on a green background and labeled with a plus sign at the beginning of the line.}, label={listing:code_example}]
class LogisticRegression(nn.Module):
    def __init__(self, n_inputs, n_outputs):
        super(LogisticRegression, self).__init__()
        self.linear = nn.Linear(n_inputs, n_outputs)
    def forward(self,x):
        out = F.sigmoid(self.linear(x))
        return out
!\colorbox[RGB]{255,235,233}{$-$ mnist = datasets.MNIST(root='./data', train=True)}!
!\colorbox[RGB]{230,255,236}{$+$ x = Tensor(60000, 784).get\_input\_from(0)}!
!\colorbox[RGB]{230,255,236}{$+$ y = Tensor(60000, 10).get\_input\_from(0)}!
!\colorbox[RGB]{255,235,233}{$-$ dataloader = DataLoader(mnist, batch\_size=128)}!
!\colorbox[RGB]{230,255,236}{$+$ dataloader = DataLoader(x, y, batch\_size = 128)}!
model = LogisticRegression(784, 10)
optimizer = optim.SGD(model.parameters(), lr = 0.01)
criterion = nn.CrossEntropyLoss()
model.train()
!\colorbox[RGB]{255,235,233}{$-$ for i in range(10):}!
!\colorbox[RGB]{230,255,236}{$+$ @for\_range(10)}!
!\colorbox[RGB]{230,255,236}{$+$ def \_(i):}!
    x, labels = dataloader[i]
    optimizer.zero_grad()
    output = model(x)
    loss = criterion(output, labels)
    loss.backward()
    optimizer.step()
\end{lstlisting}


As is shown in Listing~\ref{listing:code_example}, the model construction process, model forward process, model backward process, and model optimization process of \hawkeye-based codes are fully consistent with those of \texttt{PyTorch}-based codes. The main differences fall on the data loading and the definition of the loop: (1) Rather than loading local data, in \hawkeye, model designers need to specify the data source and then use the input data to initialize the dataloader (Line 8-12). (2) The loop interface of \hawkeye slightly differs from that of \texttt{PyTorch} (Line 17-19). The differences should be subtle. Meanwhile, because the Autograd library of \hawkeye integrates the communication cost profiling method, model designers can profile the communication cost of the secure logistics regression model training process by directly running \hawkeye to analyze the program shown in Listing~\ref{listing:code_example} without manually inserting test instruments. As a result, model designers can use \hawkeye to profile the communication cost of complex models (\wqruanother{e.g.,} transformers) in \texttt{PyTorch} by modifying less than ten lines of code. In contrast, Li et al.~\cite{li2023mpcformer} have to manually insert about 100 lines of codes to dynamically profile the communication cost of secure transformer inference processes on \texttt{CrypTen}. More examples of implementing complex models in \hawkeye can be found in our source codes.





\subsection{Case Studies}~\label{subsec:case_studies}
In this section, we conduct three case studies to show the practical applications of \hawkeye. Firstly, to show that \hawkeye can help model designers find communication bottlenecks of models on MPL frameworks with different security models, we apply \hawkeye to profile model communication cost on four MPL frameworks whose security models are different. Secondly, to show that \hawkeye can help model designers choose a proper optimizer for secure model training, we apply \hawkeye to profile the communication cost of secure model training processes with two different optimizers. Finally, we apply \hawkeye in model computational graph optimization to improve the efficiency of secure model inference.

\subsubsection{Communication bottlenecks of models on MPL frameworks with different security models}
\noindent\textbf{Security models of MPL frameworks.} We first introduce security models of MPL frameworks. The security model of an MPL framework refers to assumptions the MPL framework makes about parties. In scenarios with different security requirements, model designers usually need to employ MPL frameworks with the corresponding security models. One security model generally compromises two dimensions: the behavior of the parties and the number of colluded parties. Depending on whether parties follow the protocols, the security models of MPL frameworks can be classified as semi-honest and malicious. Depending on whether the number of colluded parties is strictly below half of the total number of parties or not, the security models of MPL frameworks can be classified as honest-majority and dishonest-majority.

We apply \hawkeye to profile the secure inference processes of Resnet-50 and BERT$_{\textsc{base}}$ models on four MPL frameworks, \wqruanother{i.e.,}  \texttt{ABY}~\cite{aby}, \texttt{SPDZ-2k}~\cite{spdz2k}, \texttt{ABY3}~\cite{mohassel2018aby3}, and \texttt{Falcon}~\cite{wagh2020falcon}, whose security models are (semi-honest, dishonest-majority), (malicious, dishonest-majority), (semi-honest, honest-majority),  (malicious, honest-majority) respectively.
For parameter settings, we set the bit length as $64$, the statistical security parameter as $40$, the computational security parameter as $128$, the bit length of fixed-point numbers' fractional part as $16$, and the number of parties as two for \texttt{ABY} and \texttt{SPDZ-2k}, three for \texttt{ABY3} and \texttt{Falcon}. The input data sizes are consistent with those used in Section~\ref{subsec:accuracy_garnet}.
% We show communication cost configurations of the four MPL frameworks in Appendix~\ref{appendix:protocol_config}.

As is shown in Figure~\ref{fig:protocol_profiling}, two dimensions of the security model have significantly different impacts on communication bottlenecks of models: (1) Under the same assumption on the number of colluded parties, switching the assumption on the behavior of parties from semi-honest to malicious slightly changes the profiling results. Therefore, MPC-friendly models optimized for semi-honest MPL frameworks could remain effective for malicious MPL frameworks.  (2) When underlying MPL frameworks are designed for a dishonest majority, the communication bottlenecks are linear operators, \wqruanother{i.e.,} Matmul or Conv2d. In contrast, the communication bottlenecks become non-linear operators (\wqruanother{i.e.,} Softmax, Relu, Gelu, MaxPool) when underlying MPL frameworks are designed for an honest majority. Therefore, model designers would need to tailor MPC-friendly models for different scenarios where the assumptions on the number of colluded parties are different. 
The above results show that \hawkeye can help model designers efficiently find communication bottlenecks of models on MPL frameworks with different security models. 

\begin{figure}[ht]
    \centering
    \includegraphics[width=0.4\textwidth]{figures/protocol_profiling.pdf}
    \caption{The proportion of each operator's online communication size to the total online communication size on four MPL frameworks with different security models.}
    \label{fig:protocol_profiling}
\end{figure}

\subsubsection{Choice of optimizers in secure model training}  We apply \hawkeye to profile the communication cost of secure model training processes with two optimizers. A few studies~\cite{10.1145/3411501.3419427,DBLP:journals/popets/AttrapadungHIKM22} show that in secure model training, Adam~\cite{adam} could be a better optimizer than SGD because Adam can significantly improve the convergence speed. However, Adam incurs much more communication overhead than SGD. To quantitatively analyze the communication cost of optimizers, we apply \hawkeye to profile the communication cost of three secure CNN model training processes (LeNet, AlexNet, VGG-16) on two MPL frameworks (\texttt{ABY}, \texttt{ABY3}). Meanwhile, following previous studies~\cite{watson22piranha,cryptGPU}, we replace MaxPool in the CNN models with AvgPool. Finally, we set the input data size as $128 \times 3 \times 28 \times 28$ for LeNet, $128 \times 3 \times 224 \times 224$ for AlexNet and VGG-16, where $128$ is the batch size. Other parameter settings are the same as those of the first case study. Note that we run the secure model training processes on one batch of data.  Because the model training process is the same for each batch of data, the proportion of each part's online communication size to the total online communication size remains unchanged as the batch number changes.

\begin{table}[ht]
\centering
\caption{The online communication size of gradient computation and optimization of three secure CNN model training processes. We report the proportion of each part's online communication size to the total online communication size and the online communication size of each part.}
\scalebox{0.7}{
\begin{tabular}{c|c|cc}
\hline
Framework              & Model        & Grad Computation & Optimization \\ \hline
\multirow{6}{*}{\texttt{ABY}~\cite{aby}}  & Lenet-SGD    & 100.00\% (10.47GB)               & 0.00\% (0GB)                       \\ 
                      & Lenet-Adam   & 92.17\% (10.47GB)                      & 7.83\% (0.89GB)                  \\ 
                      & AlexNet-SGD  & 100.00\% (12,526.23GB)                   & 0.00\% (0GB)                       \\ 
                      & AlexNet-Adam & 97.06\% (12,526.23GB)                   & 2.94\% (379.67GB)                \\ 
                      & VGG-16-SGD   &  100.00\% (189,079.51GB)                  & 0.00\% (0GB)                       \\  
                      & VGG-16-Adam  & 99.55\% (189,079.51GB)                  & 0.45\% (859.72GB)                \\ \hline
\multirow{6}{*}{\texttt{ABY3}~\cite{mohassel2018aby3}} & Lenet-SGD    & 98.94\% (0.10GB)                   & 1.06\% ($<0.01$GB)         \\ 
                      & Lenet-Adam   & 32.26\% (0.10GB)                      & 67.74\% (0.21GB)                  \\ 
                      & AlexNet-SGD  & 96.34\% (12.12GB)                     & 3.66\% (0.46GB)                  \\ 
                      & AlexNet-Adam & 3.99\% (12.12GB)                     & 96.01\% (291.35GB)                \\ 
                      & VGG-16-SGD   & 99.80\% (521.60GB)                    & 0.20\% (1.03GB)                 \\ 
                      & VGG-16-Adam  & 44.15\% (521.60GB)                    & 55.85\% (659.74GB)                \\ \hline
\end{tabular}
}
\label{tab:choice_optimizer}
\end{table}

As is shown in Table~\ref{tab:choice_optimizer}, the extra communication cost incurred by Adam significantly differs among different models and MPL frameworks. When training models on \texttt{ABY}, the online communication size of Adam only accounts for $0.45\% \sim 7.83\%$ of the total online communication size. In this case, replacing SGD with Adam would significantly improve the training efficiency because the extra communication cost incurred by Adam is minor compared with the communication cost of gradient computation. In contrast, when training models on \texttt{ABY3}, replacing SGD with Adam would increase the total online communication size by $2.26 \sim 24.12$ times. Especially when securely training the AlexNet model on \texttt{ABY3}, replacing SGD with Adam would cause the total online communication size to increase by $24.12$ times. In this case, SGD should be a better optimizer than Adam because the convergence speed improvement brought by Adam cannot cover its extra communication cost. 

\subsubsection{\wqruan{Computational graph optimization}}
\wqruan{To further show the practical application of \hawkeye, we combine \hawkeye with TASO~\cite{10.1145/3341301.3359630}, a classical computational graph optimization method for deep learning models, to effectively reduce the communication overhead of secure model inference. TASO improves the efficiency of secure model inference by changing the structure of the computational graphs that represent the model inference process. Meanwhile, TASO ensures that the optimized computational graph is equivalent to the original computational graph. In this experiment, we use the online communication size outputted by \hawkeye as the cost model of TASO. For the underlying MPL framework, we use the three-party protocol of \texttt{Deep-MPC} proposed by Keller and Sun~\cite{pmlr-v162-keller22a}, whose source codes are included in \texttt{MP-SPDZ}, as our target MPL framework. We show the communication cost configuration of \texttt{Deep-MPC} in Appendix~\ref{appendix:protocol_config}. Meanwhile, following Jia et al.~\cite{10.1145/3341301.3359630}, we choose the backbone network (i.e., model components excluding the stem component and the final classifier) of ResNet-18 and ResNet-50 models as target models.}

\begin{table}[ht]
\centering
\caption{\wqruan{The communication size and communication time of two CNN models that are optimized by original TASO and \hawkeye-enhanced TASO (TASO-\hawkeye). The communication time is obtained under the WAN setting where round-trip time is 72ms, and bandwidth is 9 MBps. We report the average results of five runs and show the standard deviations in brackets.}}
\scalebox{0.7}{
\begin{tabular}{c|c|c|c}
\hline
Model                      & Method       & Comm Size (MB) & Comm Time (s)    \\ \hline
\multirow{3}{*}{ResNet-18} & Original     & \wqruan{286.10MB}       & \wqruan{41.30s ($pm$ 0.68s)}   \\ \cline{2-4} 
                           & TASO   &  \wqruan{247.49MB }      &  \wqruan{37.00s ($pm$ 0.74s)}   \\ \cline{2-4} 
                           & TASO-\hawkeye &  \wqruan{210.92MB}       &  \wqruan{30.54s ($pm$ 1.26s) }  \\ \hline
\multirow{3}{*}{ResNet-50} & Original     &  \wqruan{ 1758.85MB }     &  \wqruan{207.71s ($pm$ 8.17s)} \\ \cline{2-4} 
                           & TASO   &  \wqruan{1647.08MB }     &  \wqruan{193.54s ($pm$ 8.72s)}  \\ \cline{2-4} 
                           & TASO-\hawkeye &  \wqruan{1492.64MB  }    &  \wqruan{174.27s ($pm$ 5.48s)}  \\ \hline
\end{tabular}}
\label{tab:taso}
\end{table}
\wqruan{As is shown in Table~\ref{tab:taso}, \hawkeye-enhanced TASO significantly outperforms the original TASO. Concretely, \hawkeye-enhanced TASO can reduce the communication overhead by 15.14\% $\sim$ 26.28\% and the communication time by 16.10\% $\sim$ 26.05\%, which is 1.95 $\sim$ 2.38 times and 2.36 $\sim$ 2.50 times that of the original TASO, respectively. The above results show that \hawkeye can effectively help the efficiency optimization of secure model inference.}
% \wqruan{We apply HawkEye to redesign the cost model in the computation graph optimization for MPL. We just simply extend TASO, which is a graph optimization framework for DNN models. TASO uses three plain machine learning metrics (i.e. FLOPs, memory usage, and number of kernel launches) as the cost model. Its relaxed graph substitution algorithm uses a priority queue to select the lowest-cost graph in the queue as the source and use those subustution rules matching the graph to produce new graphs from the source. The new graphs whose cost are exceed $\alpha$ higer than the best graph are dropped, and the left are added to the searching queue for next iteration. $\alpha$ is a searching parameter designed manually. }

% \wqruan{We compare the optimizing graph's online commucation between the HawkEye-based cost model and the TASO-based cost model, using the above graph substitution algorithm searching for 600 graphs for two DNN models (ResNet18 and ResNet50). For parameter setting, we set the bit length as 64 and the bottom MPC protocol as ABY3, the searching parameter $\alpha$ as 1.05. The results are shown in the following table. the input data size is 1 × 3 × 224 × 224, the batch size is 1 for secure inference scenario.}


% \wqruan{As is shown in Table, Hawkeye help improving the graph optimizing process for MPL. For resnet18, the hawkeye-optimized online communication is 42.29\% of the origin model while the TASO-optimized online communication is 68.64\% of the origin model. For resnet50, the hawkeye-optimized online communication is 85.05\% of the origin model while the TASO-optimized online communication is 69.28\% of the origin model. At the same searching graphs amount, Haweye get in the less online communication cost and the faster end-to-end model inference lantency. Compared with the base TASO cost model, the Hawkeye cost model can reduce the searching time to find the nearly best graph for MPL and achieve less commuication result owing to the accurate profiling model for MPL.}
\section{Related Work}
\label{sec:relatedwork}
\subsection{Dataset Discovery} 
Dataset discovery has been widely studied in the data management community~\cite{paton2023dataset, TabelDiscovery}, with table search  as the primary application. 
% A prevalent line  is query-driven discovery that aims to search for the tables from data lakes or large open data repositories, according to a user's query.  
% Earlier studies~\cite{AdelfioS13,BrickleyBN19} aim to find web tables related to the given keywords with the help of metadata. 
The main sub-tasks of table search are joinable table search and table union search.
% When given a query table for search, the main sub-tasks are joinable table search and table union search.

\textbf{Joinable table search.}
To support joinable table search, most studies~\cite{Aurum, DatasetDiscovery,LSH,JOSIE,CrossDataDis,Correlation,MATE} focus on equi-join and utilize syntactic similarity measures to determine joinanility between columns. 
% Aurum~\cite{Aurum} and D3L~\cite{DatasetDiscovery} use Jaccard distance, while LSH Ensemble~\cite{LSH} and Josie~\cite{JOSIE} adopt Jaccard set containment to alleviate the bias to shorter columns. Some recent studies extend the equi-join to correlated table discovery~\cite{Correlation} or n-ary joins~\cite{MATE}.
% These methods do not consider semantics of columns.
To take semantics into consideration, PEXESO~\cite{Pexeso} proposes a semantic joinability measure, and designs a cell-level exact algorithm under this measure using word embeddings. To enhance efficiency, the following column-level solutions, such as DeepJoin~\cite{Deepjoin} and WarpGate~\cite{WarpGate}, perform coarser computation at column-level to approximate the results of cell-level solutions. However, the effectiveness is poor due to the suboptimal column embeddings. In contrast, our $\textsf{Snoopy}$ is an effective column-level framework powered by the proxy-column-based column embeddings.
Recent works~\cite{koios,SilkMoth} on semantic overlap set search are related to join discovery, but adopt a different semantic overlap (join) measure from the previous studies~\cite{Deepjoin,Pexeso} and ours. 
OmniMatch~\cite{omnimatch}, a concurrent work with ours, detects both equi-joins and fuzzy-joins by combining multiple similarity measures. However, it views join discovery as an offline procedure~\cite{omnimatch}, unlike online procedures where high efficiency is a crucial demand.


\textbf{Table union search.} The goal of table union search is to find tables that can be unioned with the query table to extend it with tuples. TUS~\cite{TUS} defines table union search based on attribute unionability, and formalizes three probabilistic models to describe how unionable attributes are
generated from different domains.
D3L~\cite{DatasetDiscovery} further adds in measures that include formatting similarity and attribute names. SANTOS~\cite{santos}
% and Starmie~\cite{starmine} consider the relationships between columns and achieve better accuracy.
considers the relationships between columns and uses a knowledge base to identify  unionable tables. Starmie~\cite{starmine} extends the notion of capturing binary relationships to use the context of the table to determine union-ability. 
 
 





% \noindent\textbf{Joinable table search.} 
% % \subsection{Joinable Table Search}
% To find joinable tables from large table repositories or data lakes, most studies~\cite{Aurum, DatasetDiscovery,LSH,JOSIE,CrossDataDis,Correlation,MATE,CrossDataDis} focus on equi-join and utilize syntactic similarity metrics to measure joinability between columns. Aurum~\cite{Aurum} and D3L~\cite{DatasetDiscovery} use Jaccard distance as the similarity measure. Since Jaccard similarity has a bias to shorter columns, solutions based on Jaccard set containment~\cite{LSH,JOSIE,CrossDataDis} have been proposed.
% % , which are more robust under different cardinalities.
% Some recent studies extend the requirement of column joinability, such as the Join-Correlation Search~\cite{Correlation} which aims to find joinable and correlated tables, and MATE~\cite{MATE} which aims to find n-ary joinable tables.
% To take semantics into consideration, PEXESO~\cite{Pexeso}, the first semantically joinable table search solution, uses word embeddings to capture the semantics of cell values as multi-dimensional vectors, and finds tables that can be fuzzy joined using similarity predicates.




% All the studies above, however, exclude the semantic-joinable tables, and would miss substantial join opportunities. To take semantics into consideration, PEXESO~\cite{Pexeso}, the first semantically joinable table search solution, uses word embeddings to capture the semantics of cell values as multi-dimensional vectors, and finds tables that can be fuzzy joined using similarity predicates. Although the pivot-based index and filtering techniques are adopted to accelerate the search procedure, the efficiency is still limited due to the complex cell-level similarity computations. By contrast, the column-level solutions perform coarser computation at the column level, improving efficiency by transforming the entire column to a fixed-dimensional embedding.
% To the best of our knowledge, DeepJoin~\cite{Deepjoin} and WarpGate~\cite{WarpGate} are the only existing column-level methods.
% DeepJoin~\cite{Deepjoin} encodes each column via the pre-trained language models. WarpGate~\cite{WarpGate} suggests using the pre-trained table embedding models~\cite{tableembed} to encode columns. 
% The main shortcoming of existing column-level methods is the sub-optimal column representation derived by transformer-based column encoders. 
% Our work falls into the column-level category, but differs
% from existing methods mainly in the column representation, i.e., we design a novel column representation for effective and efficient joinable table search via the the proposed pivot columns.

% \vspace{1mm}
\subsection{Table Representation}
% \subsection{Table Representation}
Many researchers are exploring how to represent tabular data (i.e., structured data) with neural models~\cite{tableembed, badaro2023transformers}. Due to the huge success in natural language processing (NLP), pre-trained language models (e.g., BERT~\cite{bert}, SBERT~\cite{sentencebert}, E5-base~\cite{E5-Base-4k}, etc.) have been widely applied to represent different levels of tabular data, including entity matching (row-level)~\cite{camper,ditto}, column type annotation (column-level)~\cite{doduo,Watchog}, etc. To model the row-and-column structure as well as integrate the heterogeneous information from different components of tables, transformer-based table embedding models have been proposed, such as TURL~\cite{turl}, TaBERT~\cite{tabert}, TAPAS~\cite{tapas}, etc. These models are based on Transformer architecture, and thus, enforce a length limit to token sequences (e.g. 512) due to the high computational complexity of the self-attention mechanism~\cite{attention}. In contrast, our proposed column representation is size unlimited, which can well handle the long columns in the real table repositories.

% \textbf{Pivot-based techniques.} The pivot (proxy) concept has been widely used in diverse fields, such as representation learning~\cite{KimKCK20}, data management in metric spaces~\cite{metricsurvey}, etc. In the field  of representation learning, proxies represent the classes of data points and are used to better represent the objects in continuous spaces ~\cite{LiangZWA21}, improve the model generalization~\cite{yao2022pcl}, etc. In the field  of metric spaces, to reduce the number of distance computations and accelerate the similarity search, pivot-based filtering and indexing techniques~\cite{ChavezNBM01,metricsurvey} have been proposed to pre-compute and store distances from every object in the metric space to a set of so-called pivots, and then utilize these distances to prune unqualified objects during search. 
% To improve the effectiveness of pivot-based techiniques in metric space,
% various pivot selection methods~\cite{fft,pca,ZhuCGJ22} have been proposed.
% % For example, Farthest First Traversal (FFT)~\cite{fft} iteratively identifies a new pivot, which is the farthest from the current pivot set, and utilizes it to expand the existing pivot set. PCA for Pivot Selection (PCA)~\cite{pca} performs dimensionality reduction to select high-quality pivots, based on the FFT.
% For more information on pivot selection, please refer to the survey~\cite{ZhuCGJ22}. Although these pivot selection methods can be easily extended to select pivot columns in our problem, we provide a novel perspective to regard pivot columns as learnable parameters. We show that our learning-based method can achieve better performance than the traditional pivot selection strategies via experimental validation in Section~\ref{sec:exp_ablation}. 

\subsection{Contrastive Learning}
Contrastive learning~\cite{Moco} (CL) is a discriminative approach that aims to pull similar samples closer and push apart dissimilar ones in the embedding space, and has achieved huge success in diverse domains.
In data discovery and preparation, CL is an effective method for learning high-quality data representations. Pylon~\cite{Pylon} and Starmie~\cite{starmine} leverage CL to learn column representations for table discovery. Ember~\cite{Ember}  enables a general keyless join operator by constructing an index populated with task-specific embeddings via CL. Sudowoodo~\cite{Sudowoodo} applies CL to learn entity, column, and cell representations to address multiple tasks in data preparation. Instead of directly learning the data items, in this paper, we leverage CL to learn proxy column matrices, which are then used to derive column representations efficiently. In contrast to the standard CL that requires a strict binary separation of the training pairs into similar and dissimilar samples, RINCE~\cite{2022ranking} first proposes a new mechanism that preserves a ranked ordering of positive samples. Inspired by that, we incorporate rank awareness into the pivot column matrix learning process, focusing on designing a data generation method to synthesize ranked joinable columns to enable rank-aware CL.

% \vspace{-2ex}
\section{Conclusion}

This paper presented a comprehensive study of the core viewers of VTubers on Bilibili, the primary platform for VTuber livestreaming in China. Our findings offer valuable insights into the behaviors and characteristics of core VTuber viewers, which we use to develop a tool that can help VTubers identify potential high-quality viewers and effectively grow their fan communities. Additionally, our results underscore the challenges of retaining core viewers, building a unique fan community culture, and moderating toxic behaviors during livestreams. In the future, we aim to extend our analysis to other platforms, such as YouTube for Japanese VTubers and Twitch for non-Asian VTubers.

% \vspace{-1.5ex}
\subsection*{Acknowledgment}
This work was supported in part by the Guangzhou Science and Technology Bureau (2024A03J0684), the Guangzhou Municipal Key Laboratory on Future Networked Systems (024A03J0623), the Guangdong Provincial Key Lab of Integrated Communication, Sensing and Computation for Ubiquitous Internet of Things \\(No.2023B1212010007), and the Guangzhou Municipal Science and Technology Project (2023A03J0011).
\subsection{Lloyd-Max Algorithm}
\label{subsec:Lloyd-Max}
For a given quantization bitwidth $B$ and an operand $\bm{X}$, the Lloyd-Max algorithm finds $2^B$ quantization levels $\{\hat{x}_i\}_{i=1}^{2^B}$ such that quantizing $\bm{X}$ by rounding each scalar in $\bm{X}$ to the nearest quantization level minimizes the quantization MSE. 

The algorithm starts with an initial guess of quantization levels and then iteratively computes quantization thresholds $\{\tau_i\}_{i=1}^{2^B-1}$ and updates quantization levels $\{\hat{x}_i\}_{i=1}^{2^B}$. Specifically, at iteration $n$, thresholds are set to the midpoints of the previous iteration's levels:
\begin{align*}
    \tau_i^{(n)}=\frac{\hat{x}_i^{(n-1)}+\hat{x}_{i+1}^{(n-1)}}2 \text{ for } i=1\ldots 2^B-1
\end{align*}
Subsequently, the quantization levels are re-computed as conditional means of the data regions defined by the new thresholds:
\begin{align*}
    \hat{x}_i^{(n)}=\mathbb{E}\left[ \bm{X} \big| \bm{X}\in [\tau_{i-1}^{(n)},\tau_i^{(n)}] \right] \text{ for } i=1\ldots 2^B
\end{align*}
where to satisfy boundary conditions we have $\tau_0=-\infty$ and $\tau_{2^B}=\infty$. The algorithm iterates the above steps until convergence.

Figure \ref{fig:lm_quant} compares the quantization levels of a $7$-bit floating point (E3M3) quantizer (left) to a $7$-bit Lloyd-Max quantizer (right) when quantizing a layer of weights from the GPT3-126M model at a per-tensor granularity. As shown, the Lloyd-Max quantizer achieves substantially lower quantization MSE. Further, Table \ref{tab:FP7_vs_LM7} shows the superior perplexity achieved by Lloyd-Max quantizers for bitwidths of $7$, $6$ and $5$. The difference between the quantizers is clear at 5 bits, where per-tensor FP quantization incurs a drastic and unacceptable increase in perplexity, while Lloyd-Max quantization incurs a much smaller increase. Nevertheless, we note that even the optimal Lloyd-Max quantizer incurs a notable ($\sim 1.5$) increase in perplexity due to the coarse granularity of quantization. 

\begin{figure}[h]
  \centering
  \includegraphics[width=0.7\linewidth]{sections/figures/LM7_FP7.pdf}
  \caption{\small Quantization levels and the corresponding quantization MSE of Floating Point (left) vs Lloyd-Max (right) Quantizers for a layer of weights in the GPT3-126M model.}
  \label{fig:lm_quant}
\end{figure}

\begin{table}[h]\scriptsize
\begin{center}
\caption{\label{tab:FP7_vs_LM7} \small Comparing perplexity (lower is better) achieved by floating point quantizers and Lloyd-Max quantizers on a GPT3-126M model for the Wikitext-103 dataset.}
\begin{tabular}{c|cc|c}
\hline
 \multirow{2}{*}{\textbf{Bitwidth}} & \multicolumn{2}{|c|}{\textbf{Floating-Point Quantizer}} & \textbf{Lloyd-Max Quantizer} \\
 & Best Format & Wikitext-103 Perplexity & Wikitext-103 Perplexity \\
\hline
7 & E3M3 & 18.32 & 18.27 \\
6 & E3M2 & 19.07 & 18.51 \\
5 & E4M0 & 43.89 & 19.71 \\
\hline
\end{tabular}
\end{center}
\end{table}

\subsection{Proof of Local Optimality of LO-BCQ}
\label{subsec:lobcq_opt_proof}
For a given block $\bm{b}_j$, the quantization MSE during LO-BCQ can be empirically evaluated as $\frac{1}{L_b}\lVert \bm{b}_j- \bm{\hat{b}}_j\rVert^2_2$ where $\bm{\hat{b}}_j$ is computed from equation (\ref{eq:clustered_quantization_definition}) as $C_{f(\bm{b}_j)}(\bm{b}_j)$. Further, for a given block cluster $\mathcal{B}_i$, we compute the quantization MSE as $\frac{1}{|\mathcal{B}_{i}|}\sum_{\bm{b} \in \mathcal{B}_{i}} \frac{1}{L_b}\lVert \bm{b}- C_i^{(n)}(\bm{b})\rVert^2_2$. Therefore, at the end of iteration $n$, we evaluate the overall quantization MSE $J^{(n)}$ for a given operand $\bm{X}$ composed of $N_c$ block clusters as:
\begin{align*}
    \label{eq:mse_iter_n}
    J^{(n)} = \frac{1}{N_c} \sum_{i=1}^{N_c} \frac{1}{|\mathcal{B}_{i}^{(n)}|}\sum_{\bm{v} \in \mathcal{B}_{i}^{(n)}} \frac{1}{L_b}\lVert \bm{b}- B_i^{(n)}(\bm{b})\rVert^2_2
\end{align*}

At the end of iteration $n$, the codebooks are updated from $\mathcal{C}^{(n-1)}$ to $\mathcal{C}^{(n)}$. However, the mapping of a given vector $\bm{b}_j$ to quantizers $\mathcal{C}^{(n)}$ remains as  $f^{(n)}(\bm{b}_j)$. At the next iteration, during the vector clustering step, $f^{(n+1)}(\bm{b}_j)$ finds new mapping of $\bm{b}_j$ to updated codebooks $\mathcal{C}^{(n)}$ such that the quantization MSE over the candidate codebooks is minimized. Therefore, we obtain the following result for $\bm{b}_j$:
\begin{align*}
\frac{1}{L_b}\lVert \bm{b}_j - C_{f^{(n+1)}(\bm{b}_j)}^{(n)}(\bm{b}_j)\rVert^2_2 \le \frac{1}{L_b}\lVert \bm{b}_j - C_{f^{(n)}(\bm{b}_j)}^{(n)}(\bm{b}_j)\rVert^2_2
\end{align*}

That is, quantizing $\bm{b}_j$ at the end of the block clustering step of iteration $n+1$ results in lower quantization MSE compared to quantizing at the end of iteration $n$. Since this is true for all $\bm{b} \in \bm{X}$, we assert the following:
\begin{equation}
\begin{split}
\label{eq:mse_ineq_1}
    \tilde{J}^{(n+1)} &= \frac{1}{N_c} \sum_{i=1}^{N_c} \frac{1}{|\mathcal{B}_{i}^{(n+1)}|}\sum_{\bm{b} \in \mathcal{B}_{i}^{(n+1)}} \frac{1}{L_b}\lVert \bm{b} - C_i^{(n)}(b)\rVert^2_2 \le J^{(n)}
\end{split}
\end{equation}
where $\tilde{J}^{(n+1)}$ is the the quantization MSE after the vector clustering step at iteration $n+1$.

Next, during the codebook update step (\ref{eq:quantizers_update}) at iteration $n+1$, the per-cluster codebooks $\mathcal{C}^{(n)}$ are updated to $\mathcal{C}^{(n+1)}$ by invoking the Lloyd-Max algorithm \citep{Lloyd}. We know that for any given value distribution, the Lloyd-Max algorithm minimizes the quantization MSE. Therefore, for a given vector cluster $\mathcal{B}_i$ we obtain the following result:

\begin{equation}
    \frac{1}{|\mathcal{B}_{i}^{(n+1)}|}\sum_{\bm{b} \in \mathcal{B}_{i}^{(n+1)}} \frac{1}{L_b}\lVert \bm{b}- C_i^{(n+1)}(\bm{b})\rVert^2_2 \le \frac{1}{|\mathcal{B}_{i}^{(n+1)}|}\sum_{\bm{b} \in \mathcal{B}_{i}^{(n+1)}} \frac{1}{L_b}\lVert \bm{b}- C_i^{(n)}(\bm{b})\rVert^2_2
\end{equation}

The above equation states that quantizing the given block cluster $\mathcal{B}_i$ after updating the associated codebook from $C_i^{(n)}$ to $C_i^{(n+1)}$ results in lower quantization MSE. Since this is true for all the block clusters, we derive the following result: 
\begin{equation}
\begin{split}
\label{eq:mse_ineq_2}
     J^{(n+1)} &= \frac{1}{N_c} \sum_{i=1}^{N_c} \frac{1}{|\mathcal{B}_{i}^{(n+1)}|}\sum_{\bm{b} \in \mathcal{B}_{i}^{(n+1)}} \frac{1}{L_b}\lVert \bm{b}- C_i^{(n+1)}(\bm{b})\rVert^2_2  \le \tilde{J}^{(n+1)}   
\end{split}
\end{equation}

Following (\ref{eq:mse_ineq_1}) and (\ref{eq:mse_ineq_2}), we find that the quantization MSE is non-increasing for each iteration, that is, $J^{(1)} \ge J^{(2)} \ge J^{(3)} \ge \ldots \ge J^{(M)}$ where $M$ is the maximum number of iterations. 
%Therefore, we can say that if the algorithm converges, then it must be that it has converged to a local minimum. 
\hfill $\blacksquare$


\begin{figure}
    \begin{center}
    \includegraphics[width=0.5\textwidth]{sections//figures/mse_vs_iter.pdf}
    \end{center}
    \caption{\small NMSE vs iterations during LO-BCQ compared to other block quantization proposals}
    \label{fig:nmse_vs_iter}
\end{figure}

Figure \ref{fig:nmse_vs_iter} shows the empirical convergence of LO-BCQ across several block lengths and number of codebooks. Also, the MSE achieved by LO-BCQ is compared to baselines such as MXFP and VSQ. As shown, LO-BCQ converges to a lower MSE than the baselines. Further, we achieve better convergence for larger number of codebooks ($N_c$) and for a smaller block length ($L_b$), both of which increase the bitwidth of BCQ (see Eq \ref{eq:bitwidth_bcq}).


\subsection{Additional Accuracy Results}
%Table \ref{tab:lobcq_config} lists the various LOBCQ configurations and their corresponding bitwidths.
\begin{table}
\setlength{\tabcolsep}{4.75pt}
\begin{center}
\caption{\label{tab:lobcq_config} Various LO-BCQ configurations and their bitwidths.}
\begin{tabular}{|c||c|c|c|c||c|c||c|} 
\hline
 & \multicolumn{4}{|c||}{$L_b=8$} & \multicolumn{2}{|c||}{$L_b=4$} & $L_b=2$ \\
 \hline
 \backslashbox{$L_A$\kern-1em}{\kern-1em$N_c$} & 2 & 4 & 8 & 16 & 2 & 4 & 2 \\
 \hline
 64 & 4.25 & 4.375 & 4.5 & 4.625 & 4.375 & 4.625 & 4.625\\
 \hline
 32 & 4.375 & 4.5 & 4.625& 4.75 & 4.5 & 4.75 & 4.75 \\
 \hline
 16 & 4.625 & 4.75& 4.875 & 5 & 4.75 & 5 & 5 \\
 \hline
\end{tabular}
\end{center}
\end{table}

%\subsection{Perplexity achieved by various LO-BCQ configurations on Wikitext-103 dataset}

\begin{table} \centering
\begin{tabular}{|c||c|c|c|c||c|c||c|} 
\hline
 $L_b \rightarrow$& \multicolumn{4}{c||}{8} & \multicolumn{2}{c||}{4} & 2\\
 \hline
 \backslashbox{$L_A$\kern-1em}{\kern-1em$N_c$} & 2 & 4 & 8 & 16 & 2 & 4 & 2  \\
 %$N_c \rightarrow$ & 2 & 4 & 8 & 16 & 2 & 4 & 2 \\
 \hline
 \hline
 \multicolumn{8}{c}{GPT3-1.3B (FP32 PPL = 9.98)} \\ 
 \hline
 \hline
 64 & 10.40 & 10.23 & 10.17 & 10.15 &  10.28 & 10.18 & 10.19 \\
 \hline
 32 & 10.25 & 10.20 & 10.15 & 10.12 &  10.23 & 10.17 & 10.17 \\
 \hline
 16 & 10.22 & 10.16 & 10.10 & 10.09 &  10.21 & 10.14 & 10.16 \\
 \hline
  \hline
 \multicolumn{8}{c}{GPT3-8B (FP32 PPL = 7.38)} \\ 
 \hline
 \hline
 64 & 7.61 & 7.52 & 7.48 &  7.47 &  7.55 &  7.49 & 7.50 \\
 \hline
 32 & 7.52 & 7.50 & 7.46 &  7.45 &  7.52 &  7.48 & 7.48  \\
 \hline
 16 & 7.51 & 7.48 & 7.44 &  7.44 &  7.51 &  7.49 & 7.47  \\
 \hline
\end{tabular}
\caption{\label{tab:ppl_gpt3_abalation} Wikitext-103 perplexity across GPT3-1.3B and 8B models.}
\end{table}

\begin{table} \centering
\begin{tabular}{|c||c|c|c|c||} 
\hline
 $L_b \rightarrow$& \multicolumn{4}{c||}{8}\\
 \hline
 \backslashbox{$L_A$\kern-1em}{\kern-1em$N_c$} & 2 & 4 & 8 & 16 \\
 %$N_c \rightarrow$ & 2 & 4 & 8 & 16 & 2 & 4 & 2 \\
 \hline
 \hline
 \multicolumn{5}{|c|}{Llama2-7B (FP32 PPL = 5.06)} \\ 
 \hline
 \hline
 64 & 5.31 & 5.26 & 5.19 & 5.18  \\
 \hline
 32 & 5.23 & 5.25 & 5.18 & 5.15  \\
 \hline
 16 & 5.23 & 5.19 & 5.16 & 5.14  \\
 \hline
 \multicolumn{5}{|c|}{Nemotron4-15B (FP32 PPL = 5.87)} \\ 
 \hline
 \hline
 64  & 6.3 & 6.20 & 6.13 & 6.08  \\
 \hline
 32  & 6.24 & 6.12 & 6.07 & 6.03  \\
 \hline
 16  & 6.12 & 6.14 & 6.04 & 6.02  \\
 \hline
 \multicolumn{5}{|c|}{Nemotron4-340B (FP32 PPL = 3.48)} \\ 
 \hline
 \hline
 64 & 3.67 & 3.62 & 3.60 & 3.59 \\
 \hline
 32 & 3.63 & 3.61 & 3.59 & 3.56 \\
 \hline
 16 & 3.61 & 3.58 & 3.57 & 3.55 \\
 \hline
\end{tabular}
\caption{\label{tab:ppl_llama7B_nemo15B} Wikitext-103 perplexity compared to FP32 baseline in Llama2-7B and Nemotron4-15B, 340B models}
\end{table}

%\subsection{Perplexity achieved by various LO-BCQ configurations on MMLU dataset}


\begin{table} \centering
\begin{tabular}{|c||c|c|c|c||c|c|c|c|} 
\hline
 $L_b \rightarrow$& \multicolumn{4}{c||}{8} & \multicolumn{4}{c||}{8}\\
 \hline
 \backslashbox{$L_A$\kern-1em}{\kern-1em$N_c$} & 2 & 4 & 8 & 16 & 2 & 4 & 8 & 16  \\
 %$N_c \rightarrow$ & 2 & 4 & 8 & 16 & 2 & 4 & 2 \\
 \hline
 \hline
 \multicolumn{5}{|c|}{Llama2-7B (FP32 Accuracy = 45.8\%)} & \multicolumn{4}{|c|}{Llama2-70B (FP32 Accuracy = 69.12\%)} \\ 
 \hline
 \hline
 64 & 43.9 & 43.4 & 43.9 & 44.9 & 68.07 & 68.27 & 68.17 & 68.75 \\
 \hline
 32 & 44.5 & 43.8 & 44.9 & 44.5 & 68.37 & 68.51 & 68.35 & 68.27  \\
 \hline
 16 & 43.9 & 42.7 & 44.9 & 45 & 68.12 & 68.77 & 68.31 & 68.59  \\
 \hline
 \hline
 \multicolumn{5}{|c|}{GPT3-22B (FP32 Accuracy = 38.75\%)} & \multicolumn{4}{|c|}{Nemotron4-15B (FP32 Accuracy = 64.3\%)} \\ 
 \hline
 \hline
 64 & 36.71 & 38.85 & 38.13 & 38.92 & 63.17 & 62.36 & 63.72 & 64.09 \\
 \hline
 32 & 37.95 & 38.69 & 39.45 & 38.34 & 64.05 & 62.30 & 63.8 & 64.33  \\
 \hline
 16 & 38.88 & 38.80 & 38.31 & 38.92 & 63.22 & 63.51 & 63.93 & 64.43  \\
 \hline
\end{tabular}
\caption{\label{tab:mmlu_abalation} Accuracy on MMLU dataset across GPT3-22B, Llama2-7B, 70B and Nemotron4-15B models.}
\end{table}


%\subsection{Perplexity achieved by various LO-BCQ configurations on LM evaluation harness}

\begin{table} \centering
\begin{tabular}{|c||c|c|c|c||c|c|c|c|} 
\hline
 $L_b \rightarrow$& \multicolumn{4}{c||}{8} & \multicolumn{4}{c||}{8}\\
 \hline
 \backslashbox{$L_A$\kern-1em}{\kern-1em$N_c$} & 2 & 4 & 8 & 16 & 2 & 4 & 8 & 16  \\
 %$N_c \rightarrow$ & 2 & 4 & 8 & 16 & 2 & 4 & 2 \\
 \hline
 \hline
 \multicolumn{5}{|c|}{Race (FP32 Accuracy = 37.51\%)} & \multicolumn{4}{|c|}{Boolq (FP32 Accuracy = 64.62\%)} \\ 
 \hline
 \hline
 64 & 36.94 & 37.13 & 36.27 & 37.13 & 63.73 & 62.26 & 63.49 & 63.36 \\
 \hline
 32 & 37.03 & 36.36 & 36.08 & 37.03 & 62.54 & 63.51 & 63.49 & 63.55  \\
 \hline
 16 & 37.03 & 37.03 & 36.46 & 37.03 & 61.1 & 63.79 & 63.58 & 63.33  \\
 \hline
 \hline
 \multicolumn{5}{|c|}{Winogrande (FP32 Accuracy = 58.01\%)} & \multicolumn{4}{|c|}{Piqa (FP32 Accuracy = 74.21\%)} \\ 
 \hline
 \hline
 64 & 58.17 & 57.22 & 57.85 & 58.33 & 73.01 & 73.07 & 73.07 & 72.80 \\
 \hline
 32 & 59.12 & 58.09 & 57.85 & 58.41 & 73.01 & 73.94 & 72.74 & 73.18  \\
 \hline
 16 & 57.93 & 58.88 & 57.93 & 58.56 & 73.94 & 72.80 & 73.01 & 73.94  \\
 \hline
\end{tabular}
\caption{\label{tab:mmlu_abalation} Accuracy on LM evaluation harness tasks on GPT3-1.3B model.}
\end{table}

\begin{table} \centering
\begin{tabular}{|c||c|c|c|c||c|c|c|c|} 
\hline
 $L_b \rightarrow$& \multicolumn{4}{c||}{8} & \multicolumn{4}{c||}{8}\\
 \hline
 \backslashbox{$L_A$\kern-1em}{\kern-1em$N_c$} & 2 & 4 & 8 & 16 & 2 & 4 & 8 & 16  \\
 %$N_c \rightarrow$ & 2 & 4 & 8 & 16 & 2 & 4 & 2 \\
 \hline
 \hline
 \multicolumn{5}{|c|}{Race (FP32 Accuracy = 41.34\%)} & \multicolumn{4}{|c|}{Boolq (FP32 Accuracy = 68.32\%)} \\ 
 \hline
 \hline
 64 & 40.48 & 40.10 & 39.43 & 39.90 & 69.20 & 68.41 & 69.45 & 68.56 \\
 \hline
 32 & 39.52 & 39.52 & 40.77 & 39.62 & 68.32 & 67.43 & 68.17 & 69.30  \\
 \hline
 16 & 39.81 & 39.71 & 39.90 & 40.38 & 68.10 & 66.33 & 69.51 & 69.42  \\
 \hline
 \hline
 \multicolumn{5}{|c|}{Winogrande (FP32 Accuracy = 67.88\%)} & \multicolumn{4}{|c|}{Piqa (FP32 Accuracy = 78.78\%)} \\ 
 \hline
 \hline
 64 & 66.85 & 66.61 & 67.72 & 67.88 & 77.31 & 77.42 & 77.75 & 77.64 \\
 \hline
 32 & 67.25 & 67.72 & 67.72 & 67.00 & 77.31 & 77.04 & 77.80 & 77.37  \\
 \hline
 16 & 68.11 & 68.90 & 67.88 & 67.48 & 77.37 & 78.13 & 78.13 & 77.69  \\
 \hline
\end{tabular}
\caption{\label{tab:mmlu_abalation} Accuracy on LM evaluation harness tasks on GPT3-8B model.}
\end{table}

\begin{table} \centering
\begin{tabular}{|c||c|c|c|c||c|c|c|c|} 
\hline
 $L_b \rightarrow$& \multicolumn{4}{c||}{8} & \multicolumn{4}{c||}{8}\\
 \hline
 \backslashbox{$L_A$\kern-1em}{\kern-1em$N_c$} & 2 & 4 & 8 & 16 & 2 & 4 & 8 & 16  \\
 %$N_c \rightarrow$ & 2 & 4 & 8 & 16 & 2 & 4 & 2 \\
 \hline
 \hline
 \multicolumn{5}{|c|}{Race (FP32 Accuracy = 40.67\%)} & \multicolumn{4}{|c|}{Boolq (FP32 Accuracy = 76.54\%)} \\ 
 \hline
 \hline
 64 & 40.48 & 40.10 & 39.43 & 39.90 & 75.41 & 75.11 & 77.09 & 75.66 \\
 \hline
 32 & 39.52 & 39.52 & 40.77 & 39.62 & 76.02 & 76.02 & 75.96 & 75.35  \\
 \hline
 16 & 39.81 & 39.71 & 39.90 & 40.38 & 75.05 & 73.82 & 75.72 & 76.09  \\
 \hline
 \hline
 \multicolumn{5}{|c|}{Winogrande (FP32 Accuracy = 70.64\%)} & \multicolumn{4}{|c|}{Piqa (FP32 Accuracy = 79.16\%)} \\ 
 \hline
 \hline
 64 & 69.14 & 70.17 & 70.17 & 70.56 & 78.24 & 79.00 & 78.62 & 78.73 \\
 \hline
 32 & 70.96 & 69.69 & 71.27 & 69.30 & 78.56 & 79.49 & 79.16 & 78.89  \\
 \hline
 16 & 71.03 & 69.53 & 69.69 & 70.40 & 78.13 & 79.16 & 79.00 & 79.00  \\
 \hline
\end{tabular}
\caption{\label{tab:mmlu_abalation} Accuracy on LM evaluation harness tasks on GPT3-22B model.}
\end{table}

\begin{table} \centering
\begin{tabular}{|c||c|c|c|c||c|c|c|c|} 
\hline
 $L_b \rightarrow$& \multicolumn{4}{c||}{8} & \multicolumn{4}{c||}{8}\\
 \hline
 \backslashbox{$L_A$\kern-1em}{\kern-1em$N_c$} & 2 & 4 & 8 & 16 & 2 & 4 & 8 & 16  \\
 %$N_c \rightarrow$ & 2 & 4 & 8 & 16 & 2 & 4 & 2 \\
 \hline
 \hline
 \multicolumn{5}{|c|}{Race (FP32 Accuracy = 44.4\%)} & \multicolumn{4}{|c|}{Boolq (FP32 Accuracy = 79.29\%)} \\ 
 \hline
 \hline
 64 & 42.49 & 42.51 & 42.58 & 43.45 & 77.58 & 77.37 & 77.43 & 78.1 \\
 \hline
 32 & 43.35 & 42.49 & 43.64 & 43.73 & 77.86 & 75.32 & 77.28 & 77.86  \\
 \hline
 16 & 44.21 & 44.21 & 43.64 & 42.97 & 78.65 & 77 & 76.94 & 77.98  \\
 \hline
 \hline
 \multicolumn{5}{|c|}{Winogrande (FP32 Accuracy = 69.38\%)} & \multicolumn{4}{|c|}{Piqa (FP32 Accuracy = 78.07\%)} \\ 
 \hline
 \hline
 64 & 68.9 & 68.43 & 69.77 & 68.19 & 77.09 & 76.82 & 77.09 & 77.86 \\
 \hline
 32 & 69.38 & 68.51 & 68.82 & 68.90 & 78.07 & 76.71 & 78.07 & 77.86  \\
 \hline
 16 & 69.53 & 67.09 & 69.38 & 68.90 & 77.37 & 77.8 & 77.91 & 77.69  \\
 \hline
\end{tabular}
\caption{\label{tab:mmlu_abalation} Accuracy on LM evaluation harness tasks on Llama2-7B model.}
\end{table}

\begin{table} \centering
\begin{tabular}{|c||c|c|c|c||c|c|c|c|} 
\hline
 $L_b \rightarrow$& \multicolumn{4}{c||}{8} & \multicolumn{4}{c||}{8}\\
 \hline
 \backslashbox{$L_A$\kern-1em}{\kern-1em$N_c$} & 2 & 4 & 8 & 16 & 2 & 4 & 8 & 16  \\
 %$N_c \rightarrow$ & 2 & 4 & 8 & 16 & 2 & 4 & 2 \\
 \hline
 \hline
 \multicolumn{5}{|c|}{Race (FP32 Accuracy = 48.8\%)} & \multicolumn{4}{|c|}{Boolq (FP32 Accuracy = 85.23\%)} \\ 
 \hline
 \hline
 64 & 49.00 & 49.00 & 49.28 & 48.71 & 82.82 & 84.28 & 84.03 & 84.25 \\
 \hline
 32 & 49.57 & 48.52 & 48.33 & 49.28 & 83.85 & 84.46 & 84.31 & 84.93  \\
 \hline
 16 & 49.85 & 49.09 & 49.28 & 48.99 & 85.11 & 84.46 & 84.61 & 83.94  \\
 \hline
 \hline
 \multicolumn{5}{|c|}{Winogrande (FP32 Accuracy = 79.95\%)} & \multicolumn{4}{|c|}{Piqa (FP32 Accuracy = 81.56\%)} \\ 
 \hline
 \hline
 64 & 78.77 & 78.45 & 78.37 & 79.16 & 81.45 & 80.69 & 81.45 & 81.5 \\
 \hline
 32 & 78.45 & 79.01 & 78.69 & 80.66 & 81.56 & 80.58 & 81.18 & 81.34  \\
 \hline
 16 & 79.95 & 79.56 & 79.79 & 79.72 & 81.28 & 81.66 & 81.28 & 80.96  \\
 \hline
\end{tabular}
\caption{\label{tab:mmlu_abalation} Accuracy on LM evaluation harness tasks on Llama2-70B model.}
\end{table}

%\section{MSE Studies}
%\textcolor{red}{TODO}


\subsection{Number Formats and Quantization Method}
\label{subsec:numFormats_quantMethod}
\subsubsection{Integer Format}
An $n$-bit signed integer (INT) is typically represented with a 2s-complement format \citep{yao2022zeroquant,xiao2023smoothquant,dai2021vsq}, where the most significant bit denotes the sign.

\subsubsection{Floating Point Format}
An $n$-bit signed floating point (FP) number $x$ comprises of a 1-bit sign ($x_{\mathrm{sign}}$), $B_m$-bit mantissa ($x_{\mathrm{mant}}$) and $B_e$-bit exponent ($x_{\mathrm{exp}}$) such that $B_m+B_e=n-1$. The associated constant exponent bias ($E_{\mathrm{bias}}$) is computed as $(2^{{B_e}-1}-1)$. We denote this format as $E_{B_e}M_{B_m}$.  

\subsubsection{Quantization Scheme}
\label{subsec:quant_method}
A quantization scheme dictates how a given unquantized tensor is converted to its quantized representation. We consider FP formats for the purpose of illustration. Given an unquantized tensor $\bm{X}$ and an FP format $E_{B_e}M_{B_m}$, we first, we compute the quantization scale factor $s_X$ that maps the maximum absolute value of $\bm{X}$ to the maximum quantization level of the $E_{B_e}M_{B_m}$ format as follows:
\begin{align}
\label{eq:sf}
    s_X = \frac{\mathrm{max}(|\bm{X}|)}{\mathrm{max}(E_{B_e}M_{B_m})}
\end{align}
In the above equation, $|\cdot|$ denotes the absolute value function.

Next, we scale $\bm{X}$ by $s_X$ and quantize it to $\hat{\bm{X}}$ by rounding it to the nearest quantization level of $E_{B_e}M_{B_m}$ as:

\begin{align}
\label{eq:tensor_quant}
    \hat{\bm{X}} = \text{round-to-nearest}\left(\frac{\bm{X}}{s_X}, E_{B_e}M_{B_m}\right)
\end{align}

We perform dynamic max-scaled quantization \citep{wu2020integer}, where the scale factor $s$ for activations is dynamically computed during runtime.

\subsection{Vector Scaled Quantization}
\begin{wrapfigure}{r}{0.35\linewidth}
  \centering
  \includegraphics[width=\linewidth]{sections/figures/vsquant.jpg}
  \caption{\small Vectorwise decomposition for per-vector scaled quantization (VSQ \citep{dai2021vsq}).}
  \label{fig:vsquant}
\end{wrapfigure}
During VSQ \citep{dai2021vsq}, the operand tensors are decomposed into 1D vectors in a hardware friendly manner as shown in Figure \ref{fig:vsquant}. Since the decomposed tensors are used as operands in matrix multiplications during inference, it is beneficial to perform this decomposition along the reduction dimension of the multiplication. The vectorwise quantization is performed similar to tensorwise quantization described in Equations \ref{eq:sf} and \ref{eq:tensor_quant}, where a scale factor $s_v$ is required for each vector $\bm{v}$ that maps the maximum absolute value of that vector to the maximum quantization level. While smaller vector lengths can lead to larger accuracy gains, the associated memory and computational overheads due to the per-vector scale factors increases. To alleviate these overheads, VSQ \citep{dai2021vsq} proposed a second level quantization of the per-vector scale factors to unsigned integers, while MX \citep{rouhani2023shared} quantizes them to integer powers of 2 (denoted as $2^{INT}$).

\subsubsection{MX Format}
The MX format proposed in \citep{rouhani2023microscaling} introduces the concept of sub-block shifting. For every two scalar elements of $b$-bits each, there is a shared exponent bit. The value of this exponent bit is determined through an empirical analysis that targets minimizing quantization MSE. We note that the FP format $E_{1}M_{b}$ is strictly better than MX from an accuracy perspective since it allocates a dedicated exponent bit to each scalar as opposed to sharing it across two scalars. Therefore, we conservatively bound the accuracy of a $b+2$-bit signed MX format with that of a $E_{1}M_{b}$ format in our comparisons. For instance, we use E1M2 format as a proxy for MX4.

\begin{figure}
    \centering
    \includegraphics[width=1\linewidth]{sections//figures/BlockFormats.pdf}
    \caption{\small Comparing LO-BCQ to MX format.}
    \label{fig:block_formats}
\end{figure}

Figure \ref{fig:block_formats} compares our $4$-bit LO-BCQ block format to MX \citep{rouhani2023microscaling}. As shown, both LO-BCQ and MX decompose a given operand tensor into block arrays and each block array into blocks. Similar to MX, we find that per-block quantization ($L_b < L_A$) leads to better accuracy due to increased flexibility. While MX achieves this through per-block $1$-bit micro-scales, we associate a dedicated codebook to each block through a per-block codebook selector. Further, MX quantizes the per-block array scale-factor to E8M0 format without per-tensor scaling. In contrast during LO-BCQ, we find that per-tensor scaling combined with quantization of per-block array scale-factor to E4M3 format results in superior inference accuracy across models. 


% \section*{Acknowledgments}
% This should be a simple paragraph before the References to thank those individuals and institutions who have supported your work on this article.



%{\appendices
%\section*{Proof of the First Zonklar Equation}
%Appendix one text goes here.
% You can choose not to have a title for an appendix if you want by leaving the argument blank
%\section*{Proof of the Second Zonklar Equation}
%Appendix two text goes here.}



% \section{References Section}
% You can use a bibliography generated by BibTeX as a .bbl file.
%  BibTeX documentation can be easily obtained at:
%  http://mirror.ctan.org/biblio/bibtex/contrib/doc/
%  The IEEEtran BibTeX style support page is:
%  http://www.michaelshell.org/tex/ieeetran/bibtex/
 
 % argument is your BibTeX string definitions and bibliography database(s)
%\bibliography{IEEEabrv,../bib/paper}
%
% \section{Simple References}
% You can manually copy in the resultant .bbl file and set second argument of $\backslash${\tt{begin}} to the number of references
%  (used to reserve space for the reference number labels box).

 

% \begin{thebibliography}{1}
\bibliographystyle{IEEEtran}
% \input{refer.bbl}
\bibliography{IEEEabrv,sample-base}
 
% \end{thebibliography}


% \vspace*{-7ex}
% \begin{IEEEbiography}
% [\vspace*{-10ex}{\includegraphics[width=0.64in,height=0.8in,clip,keepaspectratio]{author_fig/gyx.jpg}}]{Yuxiang Guo}
% received the B.S. degree in computer science from Beijing Institute of Technology, China, in 2021.
% He is currently working toward the PhD degree in  Zhejiang University, China. His research interests include data discovery and data integration.
% \end{IEEEbiography}


% \vspace*{-15ex}
% \begin{IEEEbiography}
% [\vspace*{-6ex}{\includegraphics[width=0.64in,height=0.9in,clip,keepaspectratio]{author_fig/myr.jpg}}]{Yuren Mao}
% received the PhD degree under the supervision of Prof. Xuemin Lin in computer science from University of New South Wales, Australia in 2022. He is currently an assistant professor with the School of Software Technology, Zhejiang University, China. His current research interests include Machine Learning and its applications.
% \end{IEEEbiography}


 

% \vspace*{-10ex}
% \begin{IEEEbiography}
% [\vspace*{-11ex}{\includegraphics[width=0.64in,height=0.9in,clip,keepaspectratio]{author_fig/hzh.jpg}}]{Zhonghao Hu} 
% received the B.S. degree in software engineering from Henan University, China, in 2023. He is currently working toward his M.S. degree in Zhejiang University, China. His research interests include data discovery.
% \end{IEEEbiography}




% \vspace*{-11ex}
% \begin{IEEEbiography}[\vspace*{-6ex}{\includegraphics[width=0.64in,height=0.9in,clip,keepaspectratio]{author_fig/luchen.eps}}]{Lu Chen}
% received the PhD degree in computer science from Zhejiang University, China, in 2016. She is currently a professor in the College of Computer Science, Zhejiang University, China. Her research interests include indexing and querying metric spaces.
% \end{IEEEbiography}

% \vspace*{-11ex}
% \begin{IEEEbiography}
% [\vspace*{-6ex}{\includegraphics[width=0.64in,height=0.9in,clip,keepaspectratio]{author_fig/yjgao.eps}}]{Yunjun Gao}
% (Senior Member, IEEE) received the
% PhD degree in computer science from Zhejiang
% University, China, in 2008. He is currently a professor
% in the College of Computer Science, Zhejiang
% University, China. His research interests include
% database, Big Data management and analytics, and
% AI interaction with DB technology.
% \end{IEEEbiography}



% \subsection{Lloyd-Max Algorithm}
\label{subsec:Lloyd-Max}
For a given quantization bitwidth $B$ and an operand $\bm{X}$, the Lloyd-Max algorithm finds $2^B$ quantization levels $\{\hat{x}_i\}_{i=1}^{2^B}$ such that quantizing $\bm{X}$ by rounding each scalar in $\bm{X}$ to the nearest quantization level minimizes the quantization MSE. 

The algorithm starts with an initial guess of quantization levels and then iteratively computes quantization thresholds $\{\tau_i\}_{i=1}^{2^B-1}$ and updates quantization levels $\{\hat{x}_i\}_{i=1}^{2^B}$. Specifically, at iteration $n$, thresholds are set to the midpoints of the previous iteration's levels:
\begin{align*}
    \tau_i^{(n)}=\frac{\hat{x}_i^{(n-1)}+\hat{x}_{i+1}^{(n-1)}}2 \text{ for } i=1\ldots 2^B-1
\end{align*}
Subsequently, the quantization levels are re-computed as conditional means of the data regions defined by the new thresholds:
\begin{align*}
    \hat{x}_i^{(n)}=\mathbb{E}\left[ \bm{X} \big| \bm{X}\in [\tau_{i-1}^{(n)},\tau_i^{(n)}] \right] \text{ for } i=1\ldots 2^B
\end{align*}
where to satisfy boundary conditions we have $\tau_0=-\infty$ and $\tau_{2^B}=\infty$. The algorithm iterates the above steps until convergence.

Figure \ref{fig:lm_quant} compares the quantization levels of a $7$-bit floating point (E3M3) quantizer (left) to a $7$-bit Lloyd-Max quantizer (right) when quantizing a layer of weights from the GPT3-126M model at a per-tensor granularity. As shown, the Lloyd-Max quantizer achieves substantially lower quantization MSE. Further, Table \ref{tab:FP7_vs_LM7} shows the superior perplexity achieved by Lloyd-Max quantizers for bitwidths of $7$, $6$ and $5$. The difference between the quantizers is clear at 5 bits, where per-tensor FP quantization incurs a drastic and unacceptable increase in perplexity, while Lloyd-Max quantization incurs a much smaller increase. Nevertheless, we note that even the optimal Lloyd-Max quantizer incurs a notable ($\sim 1.5$) increase in perplexity due to the coarse granularity of quantization. 

\begin{figure}[h]
  \centering
  \includegraphics[width=0.7\linewidth]{sections/figures/LM7_FP7.pdf}
  \caption{\small Quantization levels and the corresponding quantization MSE of Floating Point (left) vs Lloyd-Max (right) Quantizers for a layer of weights in the GPT3-126M model.}
  \label{fig:lm_quant}
\end{figure}

\begin{table}[h]\scriptsize
\begin{center}
\caption{\label{tab:FP7_vs_LM7} \small Comparing perplexity (lower is better) achieved by floating point quantizers and Lloyd-Max quantizers on a GPT3-126M model for the Wikitext-103 dataset.}
\begin{tabular}{c|cc|c}
\hline
 \multirow{2}{*}{\textbf{Bitwidth}} & \multicolumn{2}{|c|}{\textbf{Floating-Point Quantizer}} & \textbf{Lloyd-Max Quantizer} \\
 & Best Format & Wikitext-103 Perplexity & Wikitext-103 Perplexity \\
\hline
7 & E3M3 & 18.32 & 18.27 \\
6 & E3M2 & 19.07 & 18.51 \\
5 & E4M0 & 43.89 & 19.71 \\
\hline
\end{tabular}
\end{center}
\end{table}

\subsection{Proof of Local Optimality of LO-BCQ}
\label{subsec:lobcq_opt_proof}
For a given block $\bm{b}_j$, the quantization MSE during LO-BCQ can be empirically evaluated as $\frac{1}{L_b}\lVert \bm{b}_j- \bm{\hat{b}}_j\rVert^2_2$ where $\bm{\hat{b}}_j$ is computed from equation (\ref{eq:clustered_quantization_definition}) as $C_{f(\bm{b}_j)}(\bm{b}_j)$. Further, for a given block cluster $\mathcal{B}_i$, we compute the quantization MSE as $\frac{1}{|\mathcal{B}_{i}|}\sum_{\bm{b} \in \mathcal{B}_{i}} \frac{1}{L_b}\lVert \bm{b}- C_i^{(n)}(\bm{b})\rVert^2_2$. Therefore, at the end of iteration $n$, we evaluate the overall quantization MSE $J^{(n)}$ for a given operand $\bm{X}$ composed of $N_c$ block clusters as:
\begin{align*}
    \label{eq:mse_iter_n}
    J^{(n)} = \frac{1}{N_c} \sum_{i=1}^{N_c} \frac{1}{|\mathcal{B}_{i}^{(n)}|}\sum_{\bm{v} \in \mathcal{B}_{i}^{(n)}} \frac{1}{L_b}\lVert \bm{b}- B_i^{(n)}(\bm{b})\rVert^2_2
\end{align*}

At the end of iteration $n$, the codebooks are updated from $\mathcal{C}^{(n-1)}$ to $\mathcal{C}^{(n)}$. However, the mapping of a given vector $\bm{b}_j$ to quantizers $\mathcal{C}^{(n)}$ remains as  $f^{(n)}(\bm{b}_j)$. At the next iteration, during the vector clustering step, $f^{(n+1)}(\bm{b}_j)$ finds new mapping of $\bm{b}_j$ to updated codebooks $\mathcal{C}^{(n)}$ such that the quantization MSE over the candidate codebooks is minimized. Therefore, we obtain the following result for $\bm{b}_j$:
\begin{align*}
\frac{1}{L_b}\lVert \bm{b}_j - C_{f^{(n+1)}(\bm{b}_j)}^{(n)}(\bm{b}_j)\rVert^2_2 \le \frac{1}{L_b}\lVert \bm{b}_j - C_{f^{(n)}(\bm{b}_j)}^{(n)}(\bm{b}_j)\rVert^2_2
\end{align*}

That is, quantizing $\bm{b}_j$ at the end of the block clustering step of iteration $n+1$ results in lower quantization MSE compared to quantizing at the end of iteration $n$. Since this is true for all $\bm{b} \in \bm{X}$, we assert the following:
\begin{equation}
\begin{split}
\label{eq:mse_ineq_1}
    \tilde{J}^{(n+1)} &= \frac{1}{N_c} \sum_{i=1}^{N_c} \frac{1}{|\mathcal{B}_{i}^{(n+1)}|}\sum_{\bm{b} \in \mathcal{B}_{i}^{(n+1)}} \frac{1}{L_b}\lVert \bm{b} - C_i^{(n)}(b)\rVert^2_2 \le J^{(n)}
\end{split}
\end{equation}
where $\tilde{J}^{(n+1)}$ is the the quantization MSE after the vector clustering step at iteration $n+1$.

Next, during the codebook update step (\ref{eq:quantizers_update}) at iteration $n+1$, the per-cluster codebooks $\mathcal{C}^{(n)}$ are updated to $\mathcal{C}^{(n+1)}$ by invoking the Lloyd-Max algorithm \citep{Lloyd}. We know that for any given value distribution, the Lloyd-Max algorithm minimizes the quantization MSE. Therefore, for a given vector cluster $\mathcal{B}_i$ we obtain the following result:

\begin{equation}
    \frac{1}{|\mathcal{B}_{i}^{(n+1)}|}\sum_{\bm{b} \in \mathcal{B}_{i}^{(n+1)}} \frac{1}{L_b}\lVert \bm{b}- C_i^{(n+1)}(\bm{b})\rVert^2_2 \le \frac{1}{|\mathcal{B}_{i}^{(n+1)}|}\sum_{\bm{b} \in \mathcal{B}_{i}^{(n+1)}} \frac{1}{L_b}\lVert \bm{b}- C_i^{(n)}(\bm{b})\rVert^2_2
\end{equation}

The above equation states that quantizing the given block cluster $\mathcal{B}_i$ after updating the associated codebook from $C_i^{(n)}$ to $C_i^{(n+1)}$ results in lower quantization MSE. Since this is true for all the block clusters, we derive the following result: 
\begin{equation}
\begin{split}
\label{eq:mse_ineq_2}
     J^{(n+1)} &= \frac{1}{N_c} \sum_{i=1}^{N_c} \frac{1}{|\mathcal{B}_{i}^{(n+1)}|}\sum_{\bm{b} \in \mathcal{B}_{i}^{(n+1)}} \frac{1}{L_b}\lVert \bm{b}- C_i^{(n+1)}(\bm{b})\rVert^2_2  \le \tilde{J}^{(n+1)}   
\end{split}
\end{equation}

Following (\ref{eq:mse_ineq_1}) and (\ref{eq:mse_ineq_2}), we find that the quantization MSE is non-increasing for each iteration, that is, $J^{(1)} \ge J^{(2)} \ge J^{(3)} \ge \ldots \ge J^{(M)}$ where $M$ is the maximum number of iterations. 
%Therefore, we can say that if the algorithm converges, then it must be that it has converged to a local minimum. 
\hfill $\blacksquare$


\begin{figure}
    \begin{center}
    \includegraphics[width=0.5\textwidth]{sections//figures/mse_vs_iter.pdf}
    \end{center}
    \caption{\small NMSE vs iterations during LO-BCQ compared to other block quantization proposals}
    \label{fig:nmse_vs_iter}
\end{figure}

Figure \ref{fig:nmse_vs_iter} shows the empirical convergence of LO-BCQ across several block lengths and number of codebooks. Also, the MSE achieved by LO-BCQ is compared to baselines such as MXFP and VSQ. As shown, LO-BCQ converges to a lower MSE than the baselines. Further, we achieve better convergence for larger number of codebooks ($N_c$) and for a smaller block length ($L_b$), both of which increase the bitwidth of BCQ (see Eq \ref{eq:bitwidth_bcq}).


\subsection{Additional Accuracy Results}
%Table \ref{tab:lobcq_config} lists the various LOBCQ configurations and their corresponding bitwidths.
\begin{table}
\setlength{\tabcolsep}{4.75pt}
\begin{center}
\caption{\label{tab:lobcq_config} Various LO-BCQ configurations and their bitwidths.}
\begin{tabular}{|c||c|c|c|c||c|c||c|} 
\hline
 & \multicolumn{4}{|c||}{$L_b=8$} & \multicolumn{2}{|c||}{$L_b=4$} & $L_b=2$ \\
 \hline
 \backslashbox{$L_A$\kern-1em}{\kern-1em$N_c$} & 2 & 4 & 8 & 16 & 2 & 4 & 2 \\
 \hline
 64 & 4.25 & 4.375 & 4.5 & 4.625 & 4.375 & 4.625 & 4.625\\
 \hline
 32 & 4.375 & 4.5 & 4.625& 4.75 & 4.5 & 4.75 & 4.75 \\
 \hline
 16 & 4.625 & 4.75& 4.875 & 5 & 4.75 & 5 & 5 \\
 \hline
\end{tabular}
\end{center}
\end{table}

%\subsection{Perplexity achieved by various LO-BCQ configurations on Wikitext-103 dataset}

\begin{table} \centering
\begin{tabular}{|c||c|c|c|c||c|c||c|} 
\hline
 $L_b \rightarrow$& \multicolumn{4}{c||}{8} & \multicolumn{2}{c||}{4} & 2\\
 \hline
 \backslashbox{$L_A$\kern-1em}{\kern-1em$N_c$} & 2 & 4 & 8 & 16 & 2 & 4 & 2  \\
 %$N_c \rightarrow$ & 2 & 4 & 8 & 16 & 2 & 4 & 2 \\
 \hline
 \hline
 \multicolumn{8}{c}{GPT3-1.3B (FP32 PPL = 9.98)} \\ 
 \hline
 \hline
 64 & 10.40 & 10.23 & 10.17 & 10.15 &  10.28 & 10.18 & 10.19 \\
 \hline
 32 & 10.25 & 10.20 & 10.15 & 10.12 &  10.23 & 10.17 & 10.17 \\
 \hline
 16 & 10.22 & 10.16 & 10.10 & 10.09 &  10.21 & 10.14 & 10.16 \\
 \hline
  \hline
 \multicolumn{8}{c}{GPT3-8B (FP32 PPL = 7.38)} \\ 
 \hline
 \hline
 64 & 7.61 & 7.52 & 7.48 &  7.47 &  7.55 &  7.49 & 7.50 \\
 \hline
 32 & 7.52 & 7.50 & 7.46 &  7.45 &  7.52 &  7.48 & 7.48  \\
 \hline
 16 & 7.51 & 7.48 & 7.44 &  7.44 &  7.51 &  7.49 & 7.47  \\
 \hline
\end{tabular}
\caption{\label{tab:ppl_gpt3_abalation} Wikitext-103 perplexity across GPT3-1.3B and 8B models.}
\end{table}

\begin{table} \centering
\begin{tabular}{|c||c|c|c|c||} 
\hline
 $L_b \rightarrow$& \multicolumn{4}{c||}{8}\\
 \hline
 \backslashbox{$L_A$\kern-1em}{\kern-1em$N_c$} & 2 & 4 & 8 & 16 \\
 %$N_c \rightarrow$ & 2 & 4 & 8 & 16 & 2 & 4 & 2 \\
 \hline
 \hline
 \multicolumn{5}{|c|}{Llama2-7B (FP32 PPL = 5.06)} \\ 
 \hline
 \hline
 64 & 5.31 & 5.26 & 5.19 & 5.18  \\
 \hline
 32 & 5.23 & 5.25 & 5.18 & 5.15  \\
 \hline
 16 & 5.23 & 5.19 & 5.16 & 5.14  \\
 \hline
 \multicolumn{5}{|c|}{Nemotron4-15B (FP32 PPL = 5.87)} \\ 
 \hline
 \hline
 64  & 6.3 & 6.20 & 6.13 & 6.08  \\
 \hline
 32  & 6.24 & 6.12 & 6.07 & 6.03  \\
 \hline
 16  & 6.12 & 6.14 & 6.04 & 6.02  \\
 \hline
 \multicolumn{5}{|c|}{Nemotron4-340B (FP32 PPL = 3.48)} \\ 
 \hline
 \hline
 64 & 3.67 & 3.62 & 3.60 & 3.59 \\
 \hline
 32 & 3.63 & 3.61 & 3.59 & 3.56 \\
 \hline
 16 & 3.61 & 3.58 & 3.57 & 3.55 \\
 \hline
\end{tabular}
\caption{\label{tab:ppl_llama7B_nemo15B} Wikitext-103 perplexity compared to FP32 baseline in Llama2-7B and Nemotron4-15B, 340B models}
\end{table}

%\subsection{Perplexity achieved by various LO-BCQ configurations on MMLU dataset}


\begin{table} \centering
\begin{tabular}{|c||c|c|c|c||c|c|c|c|} 
\hline
 $L_b \rightarrow$& \multicolumn{4}{c||}{8} & \multicolumn{4}{c||}{8}\\
 \hline
 \backslashbox{$L_A$\kern-1em}{\kern-1em$N_c$} & 2 & 4 & 8 & 16 & 2 & 4 & 8 & 16  \\
 %$N_c \rightarrow$ & 2 & 4 & 8 & 16 & 2 & 4 & 2 \\
 \hline
 \hline
 \multicolumn{5}{|c|}{Llama2-7B (FP32 Accuracy = 45.8\%)} & \multicolumn{4}{|c|}{Llama2-70B (FP32 Accuracy = 69.12\%)} \\ 
 \hline
 \hline
 64 & 43.9 & 43.4 & 43.9 & 44.9 & 68.07 & 68.27 & 68.17 & 68.75 \\
 \hline
 32 & 44.5 & 43.8 & 44.9 & 44.5 & 68.37 & 68.51 & 68.35 & 68.27  \\
 \hline
 16 & 43.9 & 42.7 & 44.9 & 45 & 68.12 & 68.77 & 68.31 & 68.59  \\
 \hline
 \hline
 \multicolumn{5}{|c|}{GPT3-22B (FP32 Accuracy = 38.75\%)} & \multicolumn{4}{|c|}{Nemotron4-15B (FP32 Accuracy = 64.3\%)} \\ 
 \hline
 \hline
 64 & 36.71 & 38.85 & 38.13 & 38.92 & 63.17 & 62.36 & 63.72 & 64.09 \\
 \hline
 32 & 37.95 & 38.69 & 39.45 & 38.34 & 64.05 & 62.30 & 63.8 & 64.33  \\
 \hline
 16 & 38.88 & 38.80 & 38.31 & 38.92 & 63.22 & 63.51 & 63.93 & 64.43  \\
 \hline
\end{tabular}
\caption{\label{tab:mmlu_abalation} Accuracy on MMLU dataset across GPT3-22B, Llama2-7B, 70B and Nemotron4-15B models.}
\end{table}


%\subsection{Perplexity achieved by various LO-BCQ configurations on LM evaluation harness}

\begin{table} \centering
\begin{tabular}{|c||c|c|c|c||c|c|c|c|} 
\hline
 $L_b \rightarrow$& \multicolumn{4}{c||}{8} & \multicolumn{4}{c||}{8}\\
 \hline
 \backslashbox{$L_A$\kern-1em}{\kern-1em$N_c$} & 2 & 4 & 8 & 16 & 2 & 4 & 8 & 16  \\
 %$N_c \rightarrow$ & 2 & 4 & 8 & 16 & 2 & 4 & 2 \\
 \hline
 \hline
 \multicolumn{5}{|c|}{Race (FP32 Accuracy = 37.51\%)} & \multicolumn{4}{|c|}{Boolq (FP32 Accuracy = 64.62\%)} \\ 
 \hline
 \hline
 64 & 36.94 & 37.13 & 36.27 & 37.13 & 63.73 & 62.26 & 63.49 & 63.36 \\
 \hline
 32 & 37.03 & 36.36 & 36.08 & 37.03 & 62.54 & 63.51 & 63.49 & 63.55  \\
 \hline
 16 & 37.03 & 37.03 & 36.46 & 37.03 & 61.1 & 63.79 & 63.58 & 63.33  \\
 \hline
 \hline
 \multicolumn{5}{|c|}{Winogrande (FP32 Accuracy = 58.01\%)} & \multicolumn{4}{|c|}{Piqa (FP32 Accuracy = 74.21\%)} \\ 
 \hline
 \hline
 64 & 58.17 & 57.22 & 57.85 & 58.33 & 73.01 & 73.07 & 73.07 & 72.80 \\
 \hline
 32 & 59.12 & 58.09 & 57.85 & 58.41 & 73.01 & 73.94 & 72.74 & 73.18  \\
 \hline
 16 & 57.93 & 58.88 & 57.93 & 58.56 & 73.94 & 72.80 & 73.01 & 73.94  \\
 \hline
\end{tabular}
\caption{\label{tab:mmlu_abalation} Accuracy on LM evaluation harness tasks on GPT3-1.3B model.}
\end{table}

\begin{table} \centering
\begin{tabular}{|c||c|c|c|c||c|c|c|c|} 
\hline
 $L_b \rightarrow$& \multicolumn{4}{c||}{8} & \multicolumn{4}{c||}{8}\\
 \hline
 \backslashbox{$L_A$\kern-1em}{\kern-1em$N_c$} & 2 & 4 & 8 & 16 & 2 & 4 & 8 & 16  \\
 %$N_c \rightarrow$ & 2 & 4 & 8 & 16 & 2 & 4 & 2 \\
 \hline
 \hline
 \multicolumn{5}{|c|}{Race (FP32 Accuracy = 41.34\%)} & \multicolumn{4}{|c|}{Boolq (FP32 Accuracy = 68.32\%)} \\ 
 \hline
 \hline
 64 & 40.48 & 40.10 & 39.43 & 39.90 & 69.20 & 68.41 & 69.45 & 68.56 \\
 \hline
 32 & 39.52 & 39.52 & 40.77 & 39.62 & 68.32 & 67.43 & 68.17 & 69.30  \\
 \hline
 16 & 39.81 & 39.71 & 39.90 & 40.38 & 68.10 & 66.33 & 69.51 & 69.42  \\
 \hline
 \hline
 \multicolumn{5}{|c|}{Winogrande (FP32 Accuracy = 67.88\%)} & \multicolumn{4}{|c|}{Piqa (FP32 Accuracy = 78.78\%)} \\ 
 \hline
 \hline
 64 & 66.85 & 66.61 & 67.72 & 67.88 & 77.31 & 77.42 & 77.75 & 77.64 \\
 \hline
 32 & 67.25 & 67.72 & 67.72 & 67.00 & 77.31 & 77.04 & 77.80 & 77.37  \\
 \hline
 16 & 68.11 & 68.90 & 67.88 & 67.48 & 77.37 & 78.13 & 78.13 & 77.69  \\
 \hline
\end{tabular}
\caption{\label{tab:mmlu_abalation} Accuracy on LM evaluation harness tasks on GPT3-8B model.}
\end{table}

\begin{table} \centering
\begin{tabular}{|c||c|c|c|c||c|c|c|c|} 
\hline
 $L_b \rightarrow$& \multicolumn{4}{c||}{8} & \multicolumn{4}{c||}{8}\\
 \hline
 \backslashbox{$L_A$\kern-1em}{\kern-1em$N_c$} & 2 & 4 & 8 & 16 & 2 & 4 & 8 & 16  \\
 %$N_c \rightarrow$ & 2 & 4 & 8 & 16 & 2 & 4 & 2 \\
 \hline
 \hline
 \multicolumn{5}{|c|}{Race (FP32 Accuracy = 40.67\%)} & \multicolumn{4}{|c|}{Boolq (FP32 Accuracy = 76.54\%)} \\ 
 \hline
 \hline
 64 & 40.48 & 40.10 & 39.43 & 39.90 & 75.41 & 75.11 & 77.09 & 75.66 \\
 \hline
 32 & 39.52 & 39.52 & 40.77 & 39.62 & 76.02 & 76.02 & 75.96 & 75.35  \\
 \hline
 16 & 39.81 & 39.71 & 39.90 & 40.38 & 75.05 & 73.82 & 75.72 & 76.09  \\
 \hline
 \hline
 \multicolumn{5}{|c|}{Winogrande (FP32 Accuracy = 70.64\%)} & \multicolumn{4}{|c|}{Piqa (FP32 Accuracy = 79.16\%)} \\ 
 \hline
 \hline
 64 & 69.14 & 70.17 & 70.17 & 70.56 & 78.24 & 79.00 & 78.62 & 78.73 \\
 \hline
 32 & 70.96 & 69.69 & 71.27 & 69.30 & 78.56 & 79.49 & 79.16 & 78.89  \\
 \hline
 16 & 71.03 & 69.53 & 69.69 & 70.40 & 78.13 & 79.16 & 79.00 & 79.00  \\
 \hline
\end{tabular}
\caption{\label{tab:mmlu_abalation} Accuracy on LM evaluation harness tasks on GPT3-22B model.}
\end{table}

\begin{table} \centering
\begin{tabular}{|c||c|c|c|c||c|c|c|c|} 
\hline
 $L_b \rightarrow$& \multicolumn{4}{c||}{8} & \multicolumn{4}{c||}{8}\\
 \hline
 \backslashbox{$L_A$\kern-1em}{\kern-1em$N_c$} & 2 & 4 & 8 & 16 & 2 & 4 & 8 & 16  \\
 %$N_c \rightarrow$ & 2 & 4 & 8 & 16 & 2 & 4 & 2 \\
 \hline
 \hline
 \multicolumn{5}{|c|}{Race (FP32 Accuracy = 44.4\%)} & \multicolumn{4}{|c|}{Boolq (FP32 Accuracy = 79.29\%)} \\ 
 \hline
 \hline
 64 & 42.49 & 42.51 & 42.58 & 43.45 & 77.58 & 77.37 & 77.43 & 78.1 \\
 \hline
 32 & 43.35 & 42.49 & 43.64 & 43.73 & 77.86 & 75.32 & 77.28 & 77.86  \\
 \hline
 16 & 44.21 & 44.21 & 43.64 & 42.97 & 78.65 & 77 & 76.94 & 77.98  \\
 \hline
 \hline
 \multicolumn{5}{|c|}{Winogrande (FP32 Accuracy = 69.38\%)} & \multicolumn{4}{|c|}{Piqa (FP32 Accuracy = 78.07\%)} \\ 
 \hline
 \hline
 64 & 68.9 & 68.43 & 69.77 & 68.19 & 77.09 & 76.82 & 77.09 & 77.86 \\
 \hline
 32 & 69.38 & 68.51 & 68.82 & 68.90 & 78.07 & 76.71 & 78.07 & 77.86  \\
 \hline
 16 & 69.53 & 67.09 & 69.38 & 68.90 & 77.37 & 77.8 & 77.91 & 77.69  \\
 \hline
\end{tabular}
\caption{\label{tab:mmlu_abalation} Accuracy on LM evaluation harness tasks on Llama2-7B model.}
\end{table}

\begin{table} \centering
\begin{tabular}{|c||c|c|c|c||c|c|c|c|} 
\hline
 $L_b \rightarrow$& \multicolumn{4}{c||}{8} & \multicolumn{4}{c||}{8}\\
 \hline
 \backslashbox{$L_A$\kern-1em}{\kern-1em$N_c$} & 2 & 4 & 8 & 16 & 2 & 4 & 8 & 16  \\
 %$N_c \rightarrow$ & 2 & 4 & 8 & 16 & 2 & 4 & 2 \\
 \hline
 \hline
 \multicolumn{5}{|c|}{Race (FP32 Accuracy = 48.8\%)} & \multicolumn{4}{|c|}{Boolq (FP32 Accuracy = 85.23\%)} \\ 
 \hline
 \hline
 64 & 49.00 & 49.00 & 49.28 & 48.71 & 82.82 & 84.28 & 84.03 & 84.25 \\
 \hline
 32 & 49.57 & 48.52 & 48.33 & 49.28 & 83.85 & 84.46 & 84.31 & 84.93  \\
 \hline
 16 & 49.85 & 49.09 & 49.28 & 48.99 & 85.11 & 84.46 & 84.61 & 83.94  \\
 \hline
 \hline
 \multicolumn{5}{|c|}{Winogrande (FP32 Accuracy = 79.95\%)} & \multicolumn{4}{|c|}{Piqa (FP32 Accuracy = 81.56\%)} \\ 
 \hline
 \hline
 64 & 78.77 & 78.45 & 78.37 & 79.16 & 81.45 & 80.69 & 81.45 & 81.5 \\
 \hline
 32 & 78.45 & 79.01 & 78.69 & 80.66 & 81.56 & 80.58 & 81.18 & 81.34  \\
 \hline
 16 & 79.95 & 79.56 & 79.79 & 79.72 & 81.28 & 81.66 & 81.28 & 80.96  \\
 \hline
\end{tabular}
\caption{\label{tab:mmlu_abalation} Accuracy on LM evaluation harness tasks on Llama2-70B model.}
\end{table}

%\section{MSE Studies}
%\textcolor{red}{TODO}


\subsection{Number Formats and Quantization Method}
\label{subsec:numFormats_quantMethod}
\subsubsection{Integer Format}
An $n$-bit signed integer (INT) is typically represented with a 2s-complement format \citep{yao2022zeroquant,xiao2023smoothquant,dai2021vsq}, where the most significant bit denotes the sign.

\subsubsection{Floating Point Format}
An $n$-bit signed floating point (FP) number $x$ comprises of a 1-bit sign ($x_{\mathrm{sign}}$), $B_m$-bit mantissa ($x_{\mathrm{mant}}$) and $B_e$-bit exponent ($x_{\mathrm{exp}}$) such that $B_m+B_e=n-1$. The associated constant exponent bias ($E_{\mathrm{bias}}$) is computed as $(2^{{B_e}-1}-1)$. We denote this format as $E_{B_e}M_{B_m}$.  

\subsubsection{Quantization Scheme}
\label{subsec:quant_method}
A quantization scheme dictates how a given unquantized tensor is converted to its quantized representation. We consider FP formats for the purpose of illustration. Given an unquantized tensor $\bm{X}$ and an FP format $E_{B_e}M_{B_m}$, we first, we compute the quantization scale factor $s_X$ that maps the maximum absolute value of $\bm{X}$ to the maximum quantization level of the $E_{B_e}M_{B_m}$ format as follows:
\begin{align}
\label{eq:sf}
    s_X = \frac{\mathrm{max}(|\bm{X}|)}{\mathrm{max}(E_{B_e}M_{B_m})}
\end{align}
In the above equation, $|\cdot|$ denotes the absolute value function.

Next, we scale $\bm{X}$ by $s_X$ and quantize it to $\hat{\bm{X}}$ by rounding it to the nearest quantization level of $E_{B_e}M_{B_m}$ as:

\begin{align}
\label{eq:tensor_quant}
    \hat{\bm{X}} = \text{round-to-nearest}\left(\frac{\bm{X}}{s_X}, E_{B_e}M_{B_m}\right)
\end{align}

We perform dynamic max-scaled quantization \citep{wu2020integer}, where the scale factor $s$ for activations is dynamically computed during runtime.

\subsection{Vector Scaled Quantization}
\begin{wrapfigure}{r}{0.35\linewidth}
  \centering
  \includegraphics[width=\linewidth]{sections/figures/vsquant.jpg}
  \caption{\small Vectorwise decomposition for per-vector scaled quantization (VSQ \citep{dai2021vsq}).}
  \label{fig:vsquant}
\end{wrapfigure}
During VSQ \citep{dai2021vsq}, the operand tensors are decomposed into 1D vectors in a hardware friendly manner as shown in Figure \ref{fig:vsquant}. Since the decomposed tensors are used as operands in matrix multiplications during inference, it is beneficial to perform this decomposition along the reduction dimension of the multiplication. The vectorwise quantization is performed similar to tensorwise quantization described in Equations \ref{eq:sf} and \ref{eq:tensor_quant}, where a scale factor $s_v$ is required for each vector $\bm{v}$ that maps the maximum absolute value of that vector to the maximum quantization level. While smaller vector lengths can lead to larger accuracy gains, the associated memory and computational overheads due to the per-vector scale factors increases. To alleviate these overheads, VSQ \citep{dai2021vsq} proposed a second level quantization of the per-vector scale factors to unsigned integers, while MX \citep{rouhani2023shared} quantizes them to integer powers of 2 (denoted as $2^{INT}$).

\subsubsection{MX Format}
The MX format proposed in \citep{rouhani2023microscaling} introduces the concept of sub-block shifting. For every two scalar elements of $b$-bits each, there is a shared exponent bit. The value of this exponent bit is determined through an empirical analysis that targets minimizing quantization MSE. We note that the FP format $E_{1}M_{b}$ is strictly better than MX from an accuracy perspective since it allocates a dedicated exponent bit to each scalar as opposed to sharing it across two scalars. Therefore, we conservatively bound the accuracy of a $b+2$-bit signed MX format with that of a $E_{1}M_{b}$ format in our comparisons. For instance, we use E1M2 format as a proxy for MX4.

\begin{figure}
    \centering
    \includegraphics[width=1\linewidth]{sections//figures/BlockFormats.pdf}
    \caption{\small Comparing LO-BCQ to MX format.}
    \label{fig:block_formats}
\end{figure}

Figure \ref{fig:block_formats} compares our $4$-bit LO-BCQ block format to MX \citep{rouhani2023microscaling}. As shown, both LO-BCQ and MX decompose a given operand tensor into block arrays and each block array into blocks. Similar to MX, we find that per-block quantization ($L_b < L_A$) leads to better accuracy due to increased flexibility. While MX achieves this through per-block $1$-bit micro-scales, we associate a dedicated codebook to each block through a per-block codebook selector. Further, MX quantizes the per-block array scale-factor to E8M0 format without per-tensor scaling. In contrast during LO-BCQ, we find that per-tensor scaling combined with quantization of per-block array scale-factor to E4M3 format results in superior inference accuracy across models. 


\vfill

\end{document}


