\section{Related Works}
Recent years have witnessed several measurement campaigns which investigate the Starlink's performance characteristics \cite{Dominic_2024,Sami,Beckman,Melisa,Bin,Zhao,Michel,Kassem}. Table \ref{tab:TabIII} lists the selected related works, publication year, and information about the experimental setup and the geographical region. In \cite{Sami}, the authors conducted measurements from urban and remote areas of Canada using Starlink Generation-1 (Gen-1, round-shaped) and Generation-2 \footnote{It appears from the figures (see Fig. 13 (a) in \cite{Sami}) that Gen-2 rectangular dish was Standard Actuated terminal.} (Gen-2, rectangular) ground terminals. The results confirm that Starlink has the potential to provide ubiquitous Internet connectivity. Both Gen-1 and Gen-2 terminals offered comparable performance. Notably, it was observed that Starlink's throughput and latency fluctuate over time compared to terrestrial networks, and the connection experiences frequent outages. Additionally, in-motion tests were conducted for 30 minute period, with vehicle speeds ranging from 40 km/h to 70 km/h, revealed very unstable performance and frequent outages. It is worth emphasizing that these mobility tests were conducted using a Gen-1 terminal, which was primarily designed for stationary use. To the best of our knowledge, the Gen-1 terminal is now obsolete, and Starlink no longer advertises or sells this round-shaped model. 

The work in \cite{Michel} presents measurements conducted in Western Europe, evaluating Starlink's  latency, packet loss rate and throughput using Transmission Control Protocol 
 (TCP) and Quick UDP Internet Connection (QUIC). The authors in~\cite{Kassem} examine website browser performance using data from eighteen Starlink user terminals across the UK, USA, EU, and Australia. Additionally, the work in \cite{Zhao} discusses measurements conducted in urban and rural areas of the USA. Collectively, the studies \cite{Michel, Kassem, Zhao,Sami} reveal that Starlink can support demanding applications including live streaming, video conferencing, and cloud gaming. In \cite{WetLinks}, stationary measurements were conducted to collect throughput data using the UDP and to examine the impact of weather conditions on performance in Germany and the Netherlands.

As of today, only a few studies examine the Starlink performance when in-motion. In \cite{Bin,Sami,Melisa}, residential terminals are used for in-motion measurements. However, these terminals are not designed for mobility. Starlink advises against in-motion use of these terminals due to the risk of equipment falling onto the road and serious road accidents\footnote{https://support.starlink.com/topic?category=26}.


The works in \cite{Bin} and \cite{Beckman} compare the Starlink FHP terminal and cellular networks in motion measurements conducted in the USA and Sweden, respectively. The comparison indicates that, on average, Starlink outperforms cellular networks. However, Starlink experiences more frequent connection outages. Starlink measurements in Sweden within the Arctic Circle solely focus on the downlink analysis \cite{Beckman}. More recently in \cite{Dominic_2024}, the FHP terminal download and upload UDP throughput results are reported for stationary and in-motion scenarios in Germany that locates in Central Europe. It was observed that the Starlink in-motion performance is significantly worse than stationary performance. Specifically, the uplink and downlink performance decreases each by nearly 10 \%. Additionally, a lower throughput was observed in urban areas due to LOS obstruction by buildings. However, the number of samples for stationary and in-motion measurements were not equal.

Based on the literature review, there is a clear need to conduct a more thorough bi-directional measurement campaign to study the performance of the new FHP terminal in Northern Europe for both stationary and in-motion use. Our paper addresses this shortcoming with measurements to better understand its throughput and usefulness for applications.

\if{0}