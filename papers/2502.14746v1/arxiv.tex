% \documentclass[11pt,reqno]{amsart}
\documentclass[conference]{IEEEtran}
% \pdfoutput=1
\def\method{\text MixMin~}
\def\methodnospace{\text MixMin}
\def\genmethod{$\mathbb{R}$\text Min~}
\def\genmethodnospace{ $\mathbb{R}$\text Min}

\usepackage{enumitem,diagbox}
\usepackage{stfloats}
\newcommand{\nnote}[1]{{\highlightname{Nolan}{#1}{neworange}}}
\newcommand{\snote}[1]{{\highlightname{AB}{#1}{newred}}}
\renewcommand\thesection{\arabic{section}} 
\renewcommand\thesubsectiondis{\thesection.\arabic{subsection}}
\renewcommand\thesubsubsectiondis{\thesubsectiondis.\alph{subsubsection}}
\renewcommand\theparagraphdis{\arabic{paragraph}.}
\setlength{\abovedisplayskip}{3.5pt}
\setlength{\belowdisplayskip}{3.5pt}
\usepackage{cite}
%\setlength{\topsep}{0pt plus3pt minus0.5pt}
%\newcommand{\nnote}[1]{}

\newcommand{\bfsl}{\bfseries\slshape}
\newcommand{\bfit}{\bfseries\itshape}
\newcommand{\sfsl}{\sffamily\slshape}
\newcommand{\dfn}{\sffamily\slshape\small}

\newcommand\nnfootnote[1]{%
   \begin{NoHyper}
    \renewcommand\thefootnote{}\footnote{#1}%
    \addtocounter{footnote}{-1}%
   \end{NoHyper}
}
\makeatletter
\newcommand\footnoteref[1]{\protected@xdef\@thefnmark{\ref{#1}}\@footnotemark}
\makeatother
%\addtolength{\skip\footins}{-.05in}
\pagestyle{plain}

\usepackage{balance}

\title{Classical and quantum Coxeter codes:\\ Extending the Reed--Muller family}
\author{%
   \IEEEauthorblockN{{\sc Nolan J. Coble}  \qquad    {\sc Alexander Barg}}
    \IEEEauthorblockA{University of Maryland, College Park, USA}
 }
\date{}

\setstretch{1.}

\begin{document}
\maketitle

\begin{abstract}
We introduce a class of binary linear codes that generalizes the Reed--Muller family by replacing the group $\ZZ_2^m$ with an arbitrary finite Coxeter group. Similar to the Reed--Muller codes, this class is closed under duality and has rate determined by a Gaussian distribution. We also construct quantum CSS codes arising from the Coxeter codes, which admit transversal logical operators outside of the Clifford group.
\vspace*{-.17in}
\end{abstract}

\nnfootnote{N.C. was partially supported by NSF grant DMS-2231533. A.B. was supported in part by NSF grant CCF-2330909.}



% \newpage
% \tableofcontents % uncomment to include table of contents page


% \printlen[10][cm]{\linewidth}


%%%%%%%%%%%%%%%%%%%%%%%%%%%%%%%%%%%%%%%%%%%%%%%%
%%%%%%%%%%%%%%%%%%%%%%%%%%%%%%%%%%%%%%%%%%%%%%%%
%%%%%%%%%%%%%%%%%%%%%%%%%%%%%%%%%%%%%%%%%%%%%%%%

\section{Introduction}
Reed--Muller (RM) codes form a classic family studied for its interesting algebraic and combinatorial properties \cite{MS77,Assmus98} as well as from the perspective of information transmission \cite{YeAbbe2020,abbe2023reed}. 
They achieve Shannon capacity of the basic binary channel models such as channels with independent erasures or flip errors
\cite{Kudekar2015ReedMullerCA,AbbeSandon2023}. They also give rise to a large family of quantum codes \cite{Steane1999} with well-understood logical operators \cite{kubica2015universal,campbell_magic-state_2012,rengaswamy2020optimality,barg2024geometric}. This motivated us to look into possible extensions of the RM
code family, viewing them as codes in the Coxeter complex of the group $\ZZ_2^m$. The starting point of this
research is a realization than an RM code $RM(r,m)$ is spanned by (the indicator vectors of) the $(m-r)$-dimensional faces
of the $m$-dimensional Boolean cube, i.e., the Cayley graph of the group $\ZZ_2^m$. Once we adopt this description, the next step is to replace $\ZZ_2^m$ with an arbitrary (finite) Coxeter group, $W$. 
Coxeter groups naturally give rise to Cayley graphs, which are $m$-dimensional polytopes whose faces are
themselves defined through Coxeter subgroups. These polytopes and their suitable generalizations are often studied in combinatorial group theory \cite{BB05,AB08}. We define a Coxeter code of order $r$ as a
binary linear code obtained as an $\FF_2$-linear span of the set of $(m-r)$-dimensional faces. We show that the duality relation $RM(r,m)^\bot=RM(m-r-1,m)$ extends to all Coxeter codes. We also find the dimension of the codes
in terms of the $W$-polynomial of the group, whose components are given by Eulerian numbers associated to $W$ \cite{petersen2015eulerian,BB05}. Codes arising from Coxeter systems, such as the one from the permutation group, exhibit dependence of the rate on the parameters $m,r$ similar to that of RM codes; in particular, the asymptotic behavior of the rate parallels that of RM codes.

One of the motivations to study Coxeter codes is derived from our earlier work \cite{barg2024geometric}, which 
explored the structure of quantum RM codes and their transversal logical gates in terms of the faces of the cubical complex (cosets of $\ZZ_2^m$). In \cref{sec:quantum} we extend some of the results of \cite{barg2024geometric} to Coxeter codes.


\vspace*{.05in}
\noindent{\bfit 1.1. Reed-Muller codes.} Let $\FF\coloneqq\FF_2$ be the binary field and let $S_m=\br{e_1,\dots,e_m}$ be the standard basis of the $m$-dimensional cube $\ZZ_2^m$.
A \emph{standard $\ell$-cube} is a subgroup $\langle J \rangle\leq\ZZ_2^m$ spanned by a subset $J\subseteq S_m, |J|=\ell$. An \emph{$\ell$-cube} is a shift of a standard $\ell$-cube, i.e., a set $x+\langle J \rangle,$ where $x\in\ZZ_2^m$. 
%
\begin{definition}\label{def: RM}
    For $m\ge 2, r\in \{-1,0,\dots,m\}$ let 
    $$
    H_i:=\{x+\langle J \rangle\mid  x\in\ZZ_2^m, J\subseteq S, |J|=i\}.
    $$
The \emph{order-$r$ Reed-Muller code} $RM(r,m)$ is the $\FF$-linear subspace of $\FF^{2^m}$ spanned by the indicators of the $(m-r)$-subcubes, $RM(r,m)=\standard{\1_A, A\in H_{m-r}}$ \cite{barg2024geometric}.\footnote{\cite{barg2024geometric} may be not the first place to define RM codes in this way, although we are not aware of earlier references.} Note that $H_{m+1}=\emptyset$ and $RM(-1,m)=0^{2^m}$.
\end{definition}
%(\!\!\cite{barg2024geometric} may be not the first place to define RM codes in this way, although we are not aware of earlier references).
Other definitions of Reed-Muller codes rely on evaluations of polynomials of $m$ variables \cite{MS77} or the group algebra formalism \cite{willems2021codes}. We mention the second of these because the generalization of RM codes we consider is based on the perspective of combinatorial group theory, for which \cref{def: RM} is particularly
well suited.

% \subsection{Group algebra codes} Let $G$ be a finite group and let $\FF$ be a finite field. The group algebra $\FF G$ is an $\FF$-vector space $\{a=\sum_{g\in G} a_g g\mid a_g\in \FF\}$ and a {\em group code} is a (right) ideal in $\FF G$. Group codes were introduced by Berman \cite{berman1967theory} and MacWilliams \cite{macwilliams1970binary}; see \cite{willems2021codes} for a recent overview of the literature on them. Standard examples of group codes include cyclic codes and binary RM codes, obtained from cyclic groups and $\ZZ_2^m$, respectively. Another example often studied in the literature is codes obtained from the dihedral group $D_{2n}$ \cite{VD21,sales2024codes}. 
% While we consider code vectors as elements of the group algebra, our approach departs from these references by adopting a combinatorial geometry perspective rather than focusing on ideals in group algebras.



\section{Coxeter systems and codes}


\noindent
{\bfit 2.1 Coxeter systems.} Before we introduce the Coxeter code family (\cref{def: Coxeter codes}), we will prepare the combinatorial background, listing several facts about Coxeter systems in
the form and level of generality suitable for our needs. A more general presentation of finite Coxeter systems appears in  comprehensive references \cite{AB08,BB05}.

\begin{definition}
    Let $S\coloneqq\br{s_1,\dots,s_m}$ be a set of $m<\infty$ letters and consider the group, $W$, given by the presentation
    $$
        W\coloneqq \left\langle S\Bigmid (s_i s_j)^{M(i,j)}=1 \right\rangle,
    $$
    where %$M(i,j)=\text{ord}(s_is_j),$ 
    $M(i,i)=1$ and $M(i,j)=M(j,i)\in \ZZ_{\geq 2}$. %By convention, $M(i,j)=\infty$ means that there is no relation between between $s_i$ and $s_j$.
    We say that $W$ is a \emph{Coxeter group} and that the pair $(W,S)$ is a \emph{Coxeter system}. The cardinality $\abs{S}=m$  is called the \emph{rank} of $(W,S)$. Throughout, we will assume that $W$ is a finite group (finite Coxeter groups, a.k.a. finite reflection groups, are completely classified \cite[App.A.1]{BB05}).
\end{definition}

\begin{definition}[\sc Standard subgroups and cosets]
    For a fixed Coxeter system, $(W,S)$, and a subset $J\subseteq S$ of generators, the subgroup $\langle J\rangle\leq W$ is called a \emph{standard subgroup of 
    $W$}, and the \emph{type} of $\langle J\rangle$ is $J$. In particular, $(\standard{J},J)$ is a Coxeter system in its own right. % We will denote standard subgroups of type $J$ by $\standard{J}\coloneqq\langle J\rangle$. 
    A \emph{standard (left) coset} of $W$ is any coset of the form $R\coloneqq \sigma\standard{J}$ for $\sigma\in W$, $J\subseteq S$, with $J$ referred to as the \emph{type} of the coset. The \emph{rank} of $R=\sigma\standard{J}$ is $\rank (R)\coloneqq \abs{J}$.
    The collection of all standard cosets is denoted by $\Sigma\coloneqq \br{\sigma\standard{J}\mid \sigma\in W,\; J\subseteq S}$.
\end{definition}


\begin{definition}[\sc Cayley graphs]
    The \emph{(right) Cayely graph} of a finite Coxeter system, $(W,S)$, is a graph $\mcG(W,S)$ whose vertices are indexed by elements of $W$, and for $r,t\in W$, there is an edge between them whenever $t=rs_i$ for some $s_i\in S$. Since each $s_i$ is an involution, $\mcG(W,S)$ is an undirected graph and each vertex of $\mcG(W,S)$ is incident to precisely $m$ edges, one for each generator $s_i\in S$.
\end{definition}

\begin{remark}
    The Cayley graph $\mcG(W,S)$ of any Coxeter system of rank-$m$ is a polytope in the $m$-dimensional space, with the $i$-dimensional faces corresponding to the rank-$i$ standard cosets of $(W,S)$. For example, $\mcG(\ZZ_2^m,S_m)$ is simply an $m$-dimensional hypercube. For the symmetric group on $m+1$ elements, $A_m\coloneqq(\mathrm{Sym}(m+1),S)$, generated by adjacent transpositions $S\coloneqq\br{(i\;\;\, i+1)\mid i\in[m]}$, $\mcG(A_m)$ is an $m$-dimensional \emph{permutohedron}. See \cref{fig: permutohedron}.\hfill$\lhd$
\end{remark}



\begin{figure}[t]
    \centering
    \includegraphics[width=.8\linewidth]{images/A3_descents.pdf}
    \caption{The Cayley graph $\mcG(A_3)$ for the symmetric group on 4 letters is a 3-dimensional polytope called a permutohedron. The dark gray vertex is the identity element of the group, and the three colored edges indicate right multiplication by a pairwise swap, $(i\;\;i+1)$. The vertices are labeled with the descent number of the corresponding group element (\cref{def:descents}).\vspace{-0.5em}}
    \label{fig: permutohedron}
\end{figure}


\noindent{\bfit 2.2. Coxeter codes.}
We will now use the structure of Coxeter systems and standard cosets to build a family of linear codes that generalizes the RM family. Throughout, we assume that $(W,S)$ is a finite Coxeter system of rank $m$ and we denote the binary field by $\FF\coloneqq\FF_2$ and $n\coloneqq\abs{W}$. Consider the group algebra $\FF W$, which is an $n$-dimensional vector space over $\FF$ whose elements are of the form $v=\sum_{w\in W} c_w w$, for $c_w\in\FF$. We can view $\FF W$ as a vector space whose basis vectors are indexed by vertices of the Cayley graph $\mcG(W,S)$. We will not make explicit use of $\mcG(W,S)$, though it is a useful picture to keep in mind.
By abuse of notation we will consider each standard coset $R\coloneqq\sigma\standard{J}$ as an element of $\FF W$ by setting
$
    R\coloneqq \sum_{w\in R} 1\cdot w
$
and conflating subsets and their indicators.

\begin{figure}[t]
    \centering
    \includegraphics[width=.8\linewidth]{images/A3_codeword.pdf}
    \caption{The codes $\coxeter{A_3}{S}{1}\subset \coxeter{A_3}{S}{2}$ are generated by the faces and edges 
    of the permutohedron $\mcG(A_3)$, respectively. The  bit assignment shown in the figure represents the codeword in $\coxeter{A_3}{S}{1}$ generated by the colored hexagonal and square faces. The same codeword within
    the code $\coxeter{A_3}{S}{2}$ is equivalently generated by the three solid red edges.\vspace{-0.15em}}
    \label{fig: codes}
\end{figure}

% \vspace{1em}
\begin{definition}\label{def: Coxeter codes}
    Given $r\in\br{-1,\dots,m}$, the \emph{order-$r$ Coxeter code of type $(W,S)$}, $\Coxeter{r}$, is defined to be the $\FF$-linear span
    of all rank-$(m-r)$ standard cosets in $\Sigma$,
    $$
        \Coxeter{r} \coloneqq \Bigg\{\sum_{\substack{R\in \Sigma,\\ \rank(R)=m-r}} c_R R\Biggmid c_R\in\FF\Bigg\}.
    $$
    See \cref{fig: codes} for an illustration.
\end{definition}

\begin{remark}
    \hspace{0em}
    \begin{itemize}[leftmargin=*]
        \item The elementary Abelian 2-group $\ZZ_2^m$ with its standard generating set $S\coloneqq \br{e_i}_{i\in[m]}$ is a finite Coxeter system of rank $m$. As remarked above, the order-$r$ Coxeter code of type $(\ZZ_2^m,S)$ is, in fact, the code
        $RM(r,m)$.
        % Coxeter codes are thus a broad class of error-correcting codes which generalize the classic Reed--Muller family.
        \item For every Coxeter system the code $\Coxeter{-1}=0^{\abs{W}}$ is the trivial $\abs{W}$-bit code (given by an empty generating set), the code $\Coxeter{0}$ is the $\abs{W}$-bit repetition code, the code $\Coxeter{m-1}$ is the $\abs{W}$-bit single parity-check code and the code $\Coxeter{m}=\FF W$ is the entire vector space $\FF^{|W|}$. 
        \item The collection $\Sigma$ is invariant under the left action of $W$, so Coxeter codes are ideals in the group algebra $\FF W$. \hfill$\lhd$
    \end{itemize}
\end{remark}

% \vspace{1em}

We prove in Section~\hyperlink{sec: basis}{3.1} that some well-known structural results about the RM family extend to \emph{any} Coxeter code. First, Coxeter codes are a nested family of codes:
\begin{theorem}\label{thm: nested}
    For integers $q\leq r\leq m$, the order-$q$ Coxeter code of type $(W,S)$ is contained in the order-$r$ code:
    $$
        \Coxeter{q}\subseteq\Coxeter{r}.
    $$
\end{theorem}
The intuition for \cref{thm: nested} is that any coset $\sigma\standard{J}$ with $\abs{J}>m-r$ can be partitioned into $\abs{S}/\abs{J}$ cosets $\sigma_i\standard{J'}$ where $J'\subseteq J$ is any choice of $\abs{J'}=m-r$ elements in $J$. 

Like RM codes, Coxeter codes are also closed under duality:
\vspace{-1em}
\begin{theorem}\label{thm: Coxeter duality}
    The dual of the order-$r$ Coxeter code of type $(W,S)$ is the corresponding order-$(m-r-1)$ Coxeter code:
    $$
        \Coxeter{r}^\perp = \Coxeter{m-r-1}.
    $$
\end{theorem}

\begin{figure}[t!]
    \centering
    \includegraphics[width=.8\linewidth]{images/A3_extensions.pdf}
    \caption{The solid blue hexagon and the dashed red edge adjacent to the vertex $w_1$ represent the extension $R_{w_1}$ and reverse extension $\overline{R}_{w_1}$, respectively, of $w_1$. The solid blue edge and the dashed red square adjacent to the vertex $w_2$ represent the extension $R_{w_2}$ and reverse extension $\overline{R}_{w_2}$, respectively, of $w_2$. For $i\in\br{1,2}$, $w_i$ is the unique element of $R_{w_i}$ closest to the identity (dark vertex) and the unique element of $\overline R_{w_i}$ farthest from the identity (\cref{lem: unique shortest longest}).\vspace{-0.6em}
    }    \label{fig: extensions1}
\end{figure}

\section{Code parameters}

% \noindent{\bfit 3.1 Combinatorics of Coxeter systems.}
Coxeter systems carry a natural \emph{length function}, $\ell\coloneqq W\rightarrow \NN$, where the length of an element, $w$, is the smallest number of elements from $S$ needed to generate $w$. That is, $\ell(w)$ is the smallest $\ell'$ for which there is a decomposition $w=s_{i_1} s_{i_2}\dots s_{i_{\ell'}}$, where each $i_j\in [m]$, and \emph{any} such decomposition of $w$ must contain at least $\ell'$ elements. Even though $l(w)$ is well defined, there usually are multiple ways of writing $w$ as a word of length $l(w)$. The length function satisfies the following natural properties:
\begin{fact}[\cite{BB05}, Prop.1.4.2]
    The length function satisfies:
    \begin{enumerate}
        \item $\ell(e)=0$,
        \item $\ell(w^{-1})=\ell(w)$ for all $w\in W$,
        \item $\ell(w_1 w_2) \leq \ell(w_1)+\ell(w_2)$ for all $w_1,w_2\in W$, and
        \item $\ell(ws)=\ell(w)\pm 1$ for all $w\in W$, $s\in S$.
    \end{enumerate}
    In particular, property (4) implies that multiplication of an element in $W$ by a generator necessarily changes the length of the element.
\end{fact}


%\vspace*{.05in}

\begin{figure}[t!]
    \centering
    \includegraphics[width=.8\linewidth]{images/torus_extensions.pdf}
    \caption{Consider the Coxeter system $A_2\times A_2$ given by the direct product of two copies of the symmetric group on $3$ letters. The solid blue strip and the dashed red edge adjacent to the vertex $w_1$ represent the extension $R_{w_1}$ and reverse extension $\overline{R}_{w_1}$, respectively, of $w_1$. The solid blue square and the dashed red square adjacent to the vertex $w_2$ represent the extension $R_{w_2}$ and reverse extension $\overline{R}_{w_2}$, respectively, of $w_2$.}
    \label{fig: extensions2}
\end{figure}
\begin{definition}\label{def:descents}
    For $w\in W$, the subset of generators $D(w)\subseteq S$, defined as
    \begin{align*}
        D(w)\coloneqq \br{s\in S\bigmid \ell(ws)<\ell(w)},
    \end{align*}
    is called the (right) \emph{descent set} of $w$. The value $d(w)\coloneqq\abs{D(w)}$ is called the (right) \emph{descent number} of $w$. The $W$-Eulerian numbers \cite{petersen2015eulerian}, \cite[Sec.7.2]{BB05}, denoted $\euler{W}{i}$, correspond to numbers of elements in $W$ with particular descent numbers,
    \begin{equation*}
        \euler{W}{i}:=|\{{w\in W\Bigmid d(w)=i}\}|,
    \end{equation*}
    and satisfy the so-called {\em Dehn--Sommerville equations}
    \begin{equation*}
        \euler{W}{i} = \euler{W}{m-i},\tag{\hypertarget{eq:DS}{DS}}
    \end{equation*}
    for all $i\in\br{0,\dots,m}$. We note that the above definitions are dependent on the choice of the generating set $S$, but we suppress this dependence in the notations for simplicity, as is standard. 
\end{definition}
\begin{remark}
    If $W=\ZZ_2^m$ then $\euler Wi=\binom mi$. If $(W,S)=A_m$ then $\euler W i$ is the classic Eulerian number, i.e., the count of permutations in $W$ with $i$ descents \cite[p.6]{petersen2015eulerian}. See \cref{sec: computing Eulerian numbers} for expressions computing $W$-Eulerian numbers for reducible and irreducible Coxeter systems. \hfill$\lhd$
\end{remark}
\vspace{-0.2em}

The dimension of a Coxeter code is given by the sum of $W$-Eulerian numbers:
\vspace{-0.3em}
\begin{theorem}\label{thm: dimension of order r}
    The dimension of the order-$r$ Coxeter code of type $(W,S)$ is given by
    \begin{equation}\label{eq:dimension}
        \dim \Coxeter{r} = \sum_{i=0}^r \euler{W}{i}.
    \end{equation}
\end{theorem}
    For the RM case when $W=\ZZ_2^m$, this recovers the standard formula $\dim RM(r,m)=\sum_{i=0}^r\binom mi$. 
 We prove \cref{thm: dimension of order r} by constructing a basis of $\Coxeter{r}$.

\begin{definition}
    For $w\in W$, the coset $R_w\coloneqq w\standard{S\setminus D(w)}$ is called the \emph{extension} of $w$ in $W$. The coset $\overline R_w\coloneqq w\standard{D(w)}$ is called the \emph{reverse extension} of $w$ in $W$. Note that $\rank(R_w) = m-d(w)$ and $\rank (\overline R_w)=d(w)$. See \cref{fig: extensions1,fig: extensions2}.
\end{definition}



\begin{definition}
The set of all extensions (reverse extensions) in $W$ is denoted by $\mcB$ ($\overline\mcB$). For $i\in\br{0,\dots,m}$, let $\overline \mcB_i$ and $\overline{\mcB_i}$ denote the subset of extensions and reverse extensions of rank equal to $i$, which are, in turn, given by
    \begin{align*}
        \mcB_i=\br{R_w \bigmid d(w)=m-i},\;\;
        \overline{\mcB}_i =\br{\overline{R}_w \bigmid d(w)=i}.
    \end{align*}
    By the Dehn--Sommerville equations, we have
    \begin{align*}
        \abs{\mcB_{m-i}} = \abs{\overline\mcB_{m-i}}
        = \abs{\mcB_{i}} = \abs{\overline\mcB_{i}} =\euler{W}{i}.
    \end{align*}
\end{definition}



\vspace*{.03in}
\begin{lemma}\label{lem: unique shortest longest}
    For $w\in W$, $w$ is the unique shortest (resp. longest) element of its extension (resp. reverse extension). 
\end{lemma}
\begin{proof}
   Let $w'\in \standard{D(w)}, w'\ne e$. Proposition 2.17 of \cite{AB08} states that $\ell (w'w)=\ell(w)-\ell(w')>0$,
proving the claim for the reverse extension. This proposition also 
    implies that $\ell(w)>\ell(ww')$ for all $w'\in\standard{D(w)}\setminus\br{e}$. Further, \cite[Prop.2.20]{AB08} states that the minimal element $w_1\in w\standard{S\setminus D(w)}$ is uniquely characterized by the property $\ell(w_1s)=\ell(w_1)+1$ for all $s\in S\setminus D(w)$, which is satisfied by $w$ by construction of $D(w)$. \end{proof}

\begin{lemma}\label{lem: lemma for independence}
    Let $w\in W$. If $U\subseteq W\setminus\br{w}$ is a subset satisfying $\ell(u)\geq\ell(w)$ for all $u\in U$, then $w\notin R_u$ for any $u\in U$.
\end{lemma}
\begin{proof}
    Suppose $w\in R_u$ for some $u\in U$. As $w\neq u$, \cref{lem: unique shortest longest} implies that $\ell(w)>\ell(u)$, contradicting the assumption on $U$.
\end{proof}



Lastly, the following two simple results will be crucial in proving duality.

\begin{fact}\label{fact: even order}
    A non-trivial, finite Coxeter group has even order.
\end{fact}
\begin{proof}
   As the order of any $s\in S\neq\emptyset$ is 2, the result holds by Lagrange's theorem.
\end{proof}
As the intersection of two cosets is either empty, or a coset of the intersection of the component subgroups, for two standard cosets we have the following:
\begin{lemma}\label{lem: even overlap}
    Let $\sigma_1\standard{J_1}$ and $\sigma_2\standard{J_2}$ be two standard cosets. If $\abs{J_1}+\abs{J_2}>m$ then $\abs{\sigma_1\standard{J_1}\cap \sigma_2\standard{J_2}}$ is even.
\end{lemma}
\begin{proof}
    The result is true if the cosets have trivial overlap. Otherwise, $\sigma_1\standard{J_1}\cap \sigma_2\standard{J_2} = \sigma\standard{J_1\cap J_2}$ for some $\sigma \in W$. As $\abs{J_1}+\abs{J_2}>m$ but $\abs{J_1},\abs{J_2}\leq m$, the intersection $J_1\cap J_2$ is non-empty. Thus $\abs{\sigma_1\standard{J_1}\cap \sigma_2\standard{J_2}}=\abs{\standard{J_1\cap J_2}}$, which is even by \cref{fact: even order}.
\end{proof}

\vspace*{.03in}
\noindent{\bfit \hypertarget{sec: basis}{3.1.} A basis for the code. }%\label{sec: structural}
For $r\in\br{-1,\dots,m}$ consider the collection of extensions with rank at least $m-r$, $\mcB_{\geq m-r}\coloneqq\bigcup_{i\geq m-r}\mcB_i$. For the RM case when $W=\ZZ_2^m$, this collection is precisely the evaluations of monomials in $m$ variables with degree at most $r$, i.e., the standard RM basis. We will prove that $\mcB_{\geq m-r}$ is a basis for the order-$r$ Coxeter code of type $(W,S)$, from which \cref{thm: nested,thm: dimension of order r,thm: Coxeter duality} will all follow. 

\begin{lemma}\label{lem: independence}
    The collection $\mcB$ is linearly independent.
\end{lemma}
\begin{proof}
    Suppose for contradiction that $\sum c_u R_u = 0$ is a non-trivial relation on $\mcB$. As $W$ is finite, there must exist a $w\in W$, $c_w\neq 0$, whose length is minimal among the elements with non-zero coefficients. Denoting the set $U\coloneqq\br{u\in W\mid c_u\neq 0,\ u\neq w}$, this means that $\ell(w)\leq\ell(u)$ for all $u\in U$. By the linear relation we further have that \begin{equation}\label{eq: relation}
        R_w=\sum_{u\in U}R_u.
    \end{equation}
    We can apply \cref{lem: lemma for independence} to the set $U$, which implies that the element $w$ does not appear on the RHS of \cref{eq: relation}. However, $w$ clearly appears on the LHS of \cref{eq: relation}, 
    making this equality impossible.
\end{proof}

This, of course, implies that the $\mcB_{\geq m-r}$ are linearly independent, as well. It also implies that $\abs{\mcB} = \abs{W} = \sum_{i=0}^m\euler{W}{i}$.

We now show that the span of $\mcB_{\geq m-r}$ satisfies the desired duality structure. 
As we are treating standard cosets $R_1,R_2$ as elements of $\FF W$, we can consider their dot product $R_1\cdot R_2 = \abs{R_1\cap R_2}\pmod{2}$.

\begin{lemma}\label{lem: basis duality}
    For each $r\in\br{-1,\dots,m}$ we have
    \begin{equation*}
        \Span \mcB_{\geq m-r} \subseteq \left(\Span \mcB_{\geq r+1}\right)^\perp.
    \end{equation*}
\end{lemma}
\begin{proof}
    We must show that for each $R_1\in \mcB_{\geq m-r}$ and each $R_2\in \mcB_{\geq r+1}$ we have $R_1\cdot R_2=0$. By construction, the rank of $R_1$ is at least $r_1\geq m-r$ and similarly the rank of $R_2$ is at least $r_2\geq r+1$. As $r_1+r_2 > m$, the result holds by \cref{lem: even overlap}.
\end{proof}




Using the symmetry of $W$-Eulerian numbers given by the DS equations, these two spaces are, in fact, equal:
\begin{lemma}\label{lem: dual spans}
    For each $r\in\br{-1,\dots,m}$ we have
    \begin{equation*}
        \Span \mcB_{\geq m-r} =\left(\Span \mcB_{\geq r+1}\right)^\perp.
    \end{equation*}
\end{lemma}
\begin{proof}
    By \cref{lem: basis duality} and the fact that $\dim C+\dim C^\perp = n$ for all length-$n$ linear codes, we simply must show that $\dim (\Span \mcB_{\geq m-r}) +\dim(\Span \mcB_{\geq r+1} )= \abs{W}$. Using (\DS) and the linear independence of $\mcB_{\geq m-r}$, we have
    \begin{align*}
        \dim (\Span \mcB_{\geq m-r}) &= \sum_{i=m-r}^m \euler{W}{i} %\since{\DS}
        \stackrel{\rm (\DS)}=\sum_{i=0}^{r} \euler{W}{i} %\sum_{i=m-r}^m \euler{W}{m-i},\\
 %       &= \sum_{i=0}^{r} \euler{W}{i},
    \end{align*}
    where we have reindexed the summation to start from 0. Thus,
    \begin{align*}
        \dim (\Span \mcB_{\geq m-r}) +\dim(\Span \mcB_{\geq r+1})%\hspace{4em}\\
        %= \sum_{i=0}^{r} \euler{W}{i} +\sum_{i=r+1}^{m} \euler{W}{i} 
        = \abs{W},
    \end{align*}
    as desired.
\end{proof}

Finally, we have the following:
\begin{lemma}\label{lem: extensions form a basis}
    For $r\in\br{-1,\dots,m}$, $\mcB_{\geq m-r}$ is a basis for the order-$r$ Coxeter code of type $(W,S)$:
    \begin{equation}\label{eq: basis for Coxeter code}
        \Coxeter{r} = \Span \mcB_{\geq m-r}.
    \end{equation}
\end{lemma}
\begin{proof}
    ($\supseteq$) Recall that $\Coxeter{r}$ is the span of \emph{all} standard cosets with rank exactly equal to $m-r$. Consider an $R_w\in \mcB_{\geq m-r}$, which by definition is equal to $R_w=w\standard{S\setminus D(w)}$. Let $J\subseteq S\setminus D(w)$ be any subset of $\abs{J}=m-r$ elements of $S\setminus D(w)$, which must exist since $\rank (R_w)\geq m-r$. As the cosets of $\standard{J}$ in $\standard{S\setminus D(w)}$, denoted by $\standard{S\setminus D(w)}/\standard{J}$, form a partition of $\standard{S\setminus D(w)}$, we have that
    $$
        R_w = \sum_{R\in \standard{S\setminus D(w)}/\standard{J}} R,
    $$
    where each $R$ has rank $(m-r)$  by construction.

    ($\subseteq$) Let $R$ be a standard coset with rank $(m-r)$. Consider an arbitrary standard coset $R'$ with rank at least $r+1$. Since $\rank(R)+\rank (R')>m$, by \cref{lem: even overlap} we have that $R\cdot R'=0$. In particular, this applies to all $R'\in\mcB_{r+1}$, so we necessarily have that $R\in (\Span \mcB_{r+1})^\perp$ which equals $\Span\mcB_{m-r}$ by \cref{lem: dual spans}.
\end{proof}
\begin{proof}[Proofs of \cref{thm: nested,thm: dimension of order r,thm: Coxeter duality}]
    \cref{thm: nested} is a trivial consequence of \cref{lem: extensions form a basis}. \cref{thm: Coxeter duality} holds by \cref{lem: extensions form a basis,lem: dual spans}. \cref{thm: dimension of order r} holds by \cref{lem: extensions form a basis} and the definition of $W$-Eulerian numbers. See also the proof of \cref{lem: dual spans}.
\end{proof}

\begin{proposition}
    The results in this section hold if $\mcB$ is replaced with $\overline{\mcB}$. In particular, $\Coxeter{r} = \Span \overline\mcB_{\geq m-r}$.
\end{proposition}

% \begin{lemma}\label{lem: reverse extensions form a basis}
%     For $r\in\br{-1,\dots,m}$, $\overline\mcB_{\geq m-r}$ is a basis for the order-$r$ Coxeter code of type $(W,S)$:
%     \begin{equation}
        
%     \end{equation}
% \end{lemma}


\vspace{0.5em}
\noindent {\bfit 3.2. Rate of the codes.}
The rate of the Reed-Muller code $RM(r,m)$ equals $2^{-m}\sum_{k=0}^r\binom mi$. By the standard asymptotic arguments,
it changes from near zero to near one when $r$ crosses $m/2$, and is about $1/2$ if $r=\lfloor m/2\rfloor$, with more
precise information derived from the standard Gaussian distribution. Here we argue that largely the same behavior extends
to many Coxeter codes. We address the three infinite series of groups in the Coxeter-Dynkin classification, namely
$A_m$ (the symmetric group on $m+1$ elements), $B_m$ (the hyperoctahedral group of order $2^mm!$), and $D_m$ (the generalized dihedral group of order $2^{m-1}m!$). 
The dimension of the code $\C_{W}(r)$ is given in \eqref{eq:dimension}, from which the rate is found to be
     $$
  R(\C_W(r))=\frac 1{|W|}\sum_{i=0}^r \euler W i.
   $$
There are no closed-form expressions for any of the three cases (for that matter, there is no such expression even for RM codes),
but asymptotic analysis of Eulerian numbers of types $A,B,D$ has been addressed in many places in the literature, with \cite{HCD19}
being the most comprehensive source. We combine several results from \cite{HCD19} into the following theorem:
\begin{theorem}\label{thm: Gaussian} Suppose that $(W,S)$ is one of the irreducible Coxeter families $A_m,B_m$, or $D_m$. Then the code rate 
$R(\Coxeter{r})$ is asymptotically normal, namely, 
$$\frac{R(\Coxeter{\lfloor x\rfloor})-m/2}{m/12}\longrightarrow \frac 1{\sqrt{2\pi}}\int_{-\infty}^x {e^{-t^2/2}}dt$$
as $m\to\infty$.
\end{theorem}
    

This implies that for $R(\C_{W}(r))$ not to tend to 0 or 1 as $m\to\infty$, the quantity
$r/m$ should be separated from 0 and 1. Moreover, the variance $\text{Var}(X_r)=\frac m{12}$ implies that concentration around the mean is sharper for these Coxeter codes than for RM codes where it is controlled by the binomial distribution with variance $\frac m4$. Lastly, we note that the product structure of the $W$-polynomials of Coxeter groups implies that the rate of any infinite family
of Coxeter codes, including the ones constructed from reducible systems (\cref{sec: computing Eulerian numbers}), exhibits a behavior similar to \cref{thm: Gaussian}.
\begin{table}[t!]
    \centering
    \begin{tabular}{|l|c|c|c|c|c|}
    \hline
 %       Group\textbackslash $r$   & 1 & 2  \\ \hline\hline
 \diagbox[width=\dimexpr .6\textwidth/8+2\tabcolsep\relax, height=.55cm]{ $W$ }{$r$}  & 1 & 2  \\ \hline\hline
        $A_3$  & $[24, 12, 4]$ & $[24, 23, 2]$  \\ \hline
        $A_4$  & $[120, 27, 12]$ & $[120, 93, 4]$   \\ \hline
        % $I_2(n)$     & $[2n , 2n-1, 2]$   & $[2n , 2n, 1]$        \\ \hline
        % $I_2(n)\times I_2(n)$ & $[4n^2 , 1, 4n^2 ]$  & $[4n^2  , 4n-3, 4n]$   & $[4n^2  , 4n^2-4n+3, 4]$   & $[4n^2 , 4n^2-1, 2]$  & $[4n^2 , 4n^2, 1]$  \\ \hline
        $I_2(3)^2$   & $[36  , 9, 12]$   & $[36  , 27, 4]$       \\ \hline
        $I_2(4)^2$   & $[64  , 13, 16]$   & $[64  , 51, 4]$       \\ \hline
        $B_3$  & $[48, 24, 4]$ & $[48, 47, 2]$   \\ \hline
        $A_3\times A_1$  & $[48, 13, 8]$ & $[48, 35, 4]$  \\ \hline
        $B_3\times A_1$  & $[96, 25, 8]$ & $[96, 71, 4]$  \\ \hline
    \end{tabular}
    \vspace{0.5em}
    \caption{Parameters for various Coxeter codes.}
    \label{tab: parameters}
\end{table}

\vspace{.03in}\noindent {\bfit 3.3. Distance of the codes.} Given that $\Coxeter{r}$ is generated by standard cosets of rank $m-r$, there is a trivial upper bound on the code distance given by the \emph{smallest} such coset. We conjecture that this bound is, in fact, tight:
\begin{conjecture}\label{conj: distance}
    The distance of $\Coxeter{r}$ is given by 
    \begin{equation}
        \mathrm{dist}(\Coxeter{r})=\min_{J\subseteq S, \abs{J}=m-r} \abs{\standard{J}}.
    \end{equation}
\end{conjecture}
This conjecture is known to be true for RM codes and the family of Coxeter codes given by the dihedral groups, $I_2(n)$, for all $n\geq 2$. We have further verified it by computer for all nontrivial Coxeter codes of length at most $120$ (some of them are listed in \cref{tab: parameters}).


\section{Quantum codes from Coxeter groups}\label{sec:quantum}
We denote by $[[n,k]]$ the parameters of a qubit stabilizer code that encodes $k$ logical qubits into $n$ physical qubits. Given binary $[n,k_i]$ codes $C_i$, $i\in\br{1,2}$, such that $C_1^\perp\subseteq C_2$ there is an $[[n,k_1+k_2-n]]$ stabilizer code, known as the CSS code associated to $C_1$, $C_2$, denoted by $\CSS(C_1,C_2)$. The codes $C_1^\perp$ and $C_2^\perp$ represent the $X$ and $Z$ stabilizers of $\CSS(C_1,C_2)$, respectively. That is, denoting $X^x\coloneqq \bigotimes_{i\in[n]}X^{x_i}$ and $Z^z\coloneqq \bigotimes_{i\in[n]}Z^{z_i}$ where $X$ and $Z$ are the Pauli matrices, the operators
  \begin{equation}\label{eq: XZ}
    \br{X^x, Z^z\Bigmid x\in C_1^\perp, z\in C_2^\perp},
 \end{equation}
commute and have a joint $+1$ eigenspace in $\CC^{2^n}$ of dimension $2^{k_1+k_2-n}$. The codes $C_1$ and $C_2$ likewise represent the space of logical $Z$ and $X$ Pauli operators, respectively.

Let $(W,S)$ be a finite Coxeter system of rank $m\geq 1$.
For $-1\leq q\leq r\leq m$, \cref{thm: nested} implies that $\Coxeter{q}\subseteq\Coxeter{r}$, and so we immediately construct a quantum code using Coxeter codes:
\begin{definition}[Quantum Coxeter code]\label{def: Quantum Coxeter}
     The \emph{order-$(q,r)$ quantum Coxeter code of type $(W,S)$}, $\QCoxeter{q,r}$, is defined to be the CSS code $$\QCoxeter{q,r}\coloneqq\CSS(\Coxeter{m-q-1},\Coxeter{r})$$ with parameters $[[n=\abs{W}, \kappa=\sum_{i=q+1}^r\euler{W}{i}]]$.
\end{definition}
Consider $n=\abs{W}$ physical qubits indexed by the elements of $W$. For a subset $A\subseteq W$ let $X_{A}$ denote the $n$-qubit Pauli operator acting as $X$ on the qubits in $A$ and $\eye$ elsewhere, and analogously for $Z_{w\standard{J}}$.
\begin{lemma}\label{lem: Quantum Coxeter}
    Given $q,r\in\br{-1,\dots,m}$, $q\leq r$, consider the collections of rank-$(m-q)$ and rank-$(r+1)$ standard cosets in $(W,S)$: 
    \begin{align*}
        \Sigma_{m-q}&\coloneqq\br{{w\standard{J}}\bigmid w\in W,J\subseteq S, \abs{J} = m-q},\\
        \Sigma_{r+1}&\coloneqq\br{{w\standard{J}}\bigmid \sigma\in W,J\subseteq S, \abs{J} = r+1}.
    \end{align*}
    The operators $\br{X_{R_1},Z_{R_2}\mid R_1\in\Sigma_{m-q}, R_2\in\Sigma_{r+1}}$ form a (redundant) generating set for the stabilizers of $\QCoxeter{q,r}$.\footnote{As a simple example, consider the dihedral group $I_2(n)$ whose Cayley graph is a $2n$-cycle. Then $\QCoxeter{0,1}$ is the Iceberg code generated by global $X^{\otimes 2n}$ and $Z^{\otimes 2n}$ stabilizers.}
\end{lemma}
\cref{lem: Quantum Coxeter} is a simple consequence of the definition of classical Coxeter codes and their duality structure given in \cref{thm: Coxeter duality}. 


In prior work \cite{barg2024geometric}, we utilized the geometric and combinatorial structure of the group $\ZZ_2^m$ with its standard generating set to study transversal logical operators in higher levels of the Clifford hierarchy of the quantum RM family, $QRM_m(q,r) = \Qcoxeter{\ZZ_2^m}{}{q,r}$. For instance, the exact nature of the logic implemented by certain transversal operators acting on a standard coset depends only on the rank of the coset. This result holds in the case of arbitrary quantum Coxeter codes.
\begin{claim}\label{lem: validity} Let $\QCoxeter{q,r}$ be the quantum Coxeter code and let $R$ be a standard coset.
    For the single-qubit operator 
    $$Z(k)\coloneqq\ketbra{0}+e^{i\frac{\pi}{2^k}}\ketbra{1},$$
    \begin{enumerate}[leftmargin=*]
        \item If $\rank(R) \leq q+kr$ then applying $Z(k)$ to the qubits in $R$ does not preserve the code space.
        \item If $q+kr+1\leq \rank(R)\leq (k+1)r$ then applying $Z(k)$ to the qubits in $R$ implements a non-trivial logical operation the code space.
        \item If $\rank(R)\geq (k+1)r+1$ then applying $Z(k)$ to the qubits in $R$  implements a logical identity on the code space.
    \end{enumerate}
\end{claim}
The proof of \cref{lem: validity} is identical to the proof of Theorem 6.2 in \cite{barg2024geometric}, which relied only on the Coxeter group structure of $\ZZ_2^m$. A natural future direction, following the main results of \cite{barg2024geometric}, is to give a combinatorial description of the logical circuit implemented by a $Z(k)_R$ operator when $q+kr+1\leq \rank(R)\leq (k+1)r$. A necessary first step would be to construct a so-called ``symplectic basis'' for $\QCoxeter{q,r}$. In a few cases--- including the QRM family--- the collections of forward and reverse extensions satisfy the symplectic condition. At the same time, in many cases this fails to be true, including some small quantum Coxeter codes. Examples of groups for which the symplectic condition fails, include $A_3$, the symmetric group on 4 letters generated by pairwise swaps, and $I_2(4)$, the dihedral group of order 8 generated by two reflections meeting at a 45{\textdegree}  angle.

The codes $\QCoxeter{0,1}$ for the Coxeter systems $A_3$, $B_3$, and $H_3$ appear in \cite{Vasmer2022} as examples of 3D ball codes. The authors of \cite{Vasmer2022} note that a global transversal $T$ operator is a non-trivial logical operator for these codes; this is also a consequence of our \cref{lem: validity}.\footnote{\cite{Vasmer2022} technically considers a \emph{signed} version of transversal $T$ which acts as $T$ on half of the qubits and $T^\dagger$ on the remaining qubits.}

\section{Conclusion and outlook}
We have introduced a broad family of binary codes which generalizes and shares several features with the classic Reed--Muller family. It is natural to wonder what other properties of RM codes are shared with the Coxeter code family beyond our conjectured value of the distance. For instance, the codewords of minimum weight in RM codes are given by flats in the affine geometry; is there a characterization of the minimum weight codewords for arbitrary Coxeter codes, as well?
One can also ask what the equivalent notion of a \emph{projective} RM code is in the case of Coxeter codes. Switching to a probabilistic view of the Coxeter codes, one could also study their capacity-achieving \cite{Kudekar2015ReedMullerCA,AbbeSandon2023} and local-testability properties \cite{blum1990self,alon2005testing}.

Instead of considering the Cayley graphs of Coxeter groups we could alternatively consider their \emph{dual} polytopes, which are guaranteed to be simplicial, i.e., their facets are $(m-1)$-simplices. In this view, the order-$r$ Coxeter code would have bits indexed by the facets and parity checks given by the incidence vectors of $(m-r-2)$-simplices in the polytope. In a certain sense, Coxeter codes are related to codes on simplicial complexes where local codes are placed on simplices of a given dimension, e.g., some locally-testable codes based on high-dimensional expanders \cite{dinur2023new}.

In this simplicial view, finite Coxeter systems form a subclass of objects known as {\em spherical buildings} \cite{AB08}. Every
such object is assembled of multiple pieces, called {\em apartments}, each of which is isomorphic to a fixed
Coxeter system. There are two suitable generalizations of Coxeter codes to buildings: one where the code
is generated by subbuildings within the apartments, and another where the code is generated by simplices of a given dimension. In the example of the building associated to the Fano plane, these families happen to be dual to each other. Perhaps this is a general feature of codes on arbitrary buildings; we leave this direction to future work.





\section{Computing \texorpdfstring{$W$}{W}-Eulerian numbers}\label{sec: computing Eulerian numbers}
To find the code dimension \eqref{eq:dimension}, it is useful to have explict expressions for the $W$-Eulerian numbers. For the irreducible families of Coxeter groups, they appear in many references, e.g., \cite{petersen2015eulerian,hyatt2016recurrences,BRENTI1994417}. We give these expressions in our notation, along with an expression to compute the $W$-Eulerian numbers for direct products of Coxeter groups.

For every finite Coxeter system $(W,S)$ of rank $m$, the 0-th and $m$-th $W$-Eulerian numbers equal 1, $\euler{W}{0}=\euler{W}{m}=1$.

\vspace{0.5em}
\noindent {\bfit Type A.} \cite[A008292]{oeis} The $A_n$-\emph{Eulerian numbers} can be computed by the recurrence relation
\begin{equation*}
    \euler{A_n}{k} = (n-k+1)\euler{A_{n-1}}{k-1} + (k+1)\euler{A_{n-1}}{k}.
\end{equation*}


\vspace{0.5em}
\noindent {\bfit Type B.} \cite[A060187]{oeis} The $B_n$-Eulerian numbers can be computed by the recurrence relation
\begin{equation*}
    \euler{B_n}{k} = (2n-2k+1)\euler{B_{n-1}}{k-1} + (2k+1)\euler{B_{n-1}}{k}.
\end{equation*}


\vspace{0.5em}
\noindent {\bfit Type D.} \cite[A066094]{oeis} The $D_n$-Eulerian numbers can be computed from the $A_n$- and $B_n$-Eulerian numbers via
\begin{equation*}
    \euler{D_n}{k} = \euler{B_n}{k}-n 2^{n-1}\euler{A_{n-2}}{k-1}.
\end{equation*}


\vspace{0.5em}
\noindent {\bfit Dihedral group.}  The non-trivial $I_2(n)$-Eulerian number is given by $$\euler{I_2(n)}{1} = 2n-2.$$

\vspace{0.5em}
\noindent {\bfit Exceptional types.} See \cref{tab: exceptional}.

\begin{table*}[t!]
    \centering
    \begin{tabular}{|l|c|c|c|c|c|c|c|c|c|}
    \hline
 %       Group\textbackslash $r$   & 1 & 2  \\ \hline\hline
 \diagbox[width=\dimexpr .6\textwidth/8+2\tabcolsep\relax, height=.55cm]{ $W$ }{$k$} &0 & 1& 2& 3& 4& 5& 6& 7& 8  \\ \hline\hline
        $E_6$  &1& 1272 &12183 & 24928&12183 &1272 &1 & & \\ \hline
        $E_7$  & 1& 17635 & 309969 &1123915  & 1123915 & 309969 & 17635&1  &   \\ \hline
        $E_8$  & 1& 881752& 28336348& 169022824&300247750  & 169022824& 28336348& 881752&    1  \\ \hline
        $F_4$  &1& 236& 678 & 236&1 & & & &       \\ \hline
        $H_3$  &1& 59& 59& 1& & & & &    \\ \hline
        $H_4$  &1& 2636& 9126& 2636& 1& & & &   \\ \hline
    \end{tabular}
    \vspace{0.5em}
    \caption{$W$-Eulerian numbers for the groups of exceptional type \cite{petersen2015eulerian}.}
    \label{tab: exceptional}
\end{table*}

\vspace{0.5em}
\noindent {\bfit Reducible systems.}
Let $(W_1,S_1)$ and $(W_2,S_2)$ be finite Coxeter systems of ranks $m_1$ and $m_2$, respectively. Their direct product $(W,S)\coloneqq (W_1,S_1)\times (W_2,S_2)$ is a finite Coxeter system of rank $m_1+m_2$ where $$S\coloneqq \br{ (s_1,e_{W_2})\mid s_1\in S_1}\cup \br{ (e_{W_1},s_2)\mid s_2\in S_2}.$$

The $W$-Eulerian numbers can be computed from the component $W_1$- and $W_2$-Eulerian numbers by the equation
\begin{equation*}
    \euler{W}{k} = \sum_{i+j=k}\euler{W_1}{i}\euler{W_2}{j},
\end{equation*}
which can be proven by noting that the $W$-polynomial for a direct product of Coxeter groups is the product of the component $W$-polynomials \cite[p.202]{BB05}.
% For a more general direct product, $(W,S)\coloneqq\prod_{\ell=1}^n(W_\ell,S_\ell)$, this yields
% \begin{equation*}
%     \euler{W}{k} = \sum_{\sum_{p=1}^n i_p =k} \left(\prod_{\ell=1}^n \euler{W_\ell}{i_p}\right).
% \end{equation*}

% \newpage
\vspace{0.5em}
{
\balance
\bibliographystyle{IEEEtranS}
\bibliography{ref}
}




\end{document}

