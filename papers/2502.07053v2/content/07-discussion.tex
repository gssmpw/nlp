\section{Discussion \label{disc}}
\setlength\cellspacetoplimit{3pt}
\setlength\cellspacebottomlimit{3pt}

\begin{table*}[hbt!]
        \footnotesize
        % \scriptsize
        % \small
        \vspace{-2em}
        \centering\captionsetup{justification = centering}        
        %{\centering}        
        {\begin{tabular}{|Sc|Sc|Sc|Sc|Sc|Sc|Sc|}
            \hline
            \rowcolor{gray!20}
            {\bf $\boldsymbol{\ra}$ Method} & {\bf Type} & {\bf  Passive/Active} & $\boldsymbol{\toctoura}$ & {\bf Network TCB} & {\bf Attestation Time} & {\bf Platform} \\            
            \thickhline
            RealSWATT \cite{surminski2021realswatt} & SW &
            Passive & \greencheck  & \redcross & $O(n)$ & ESP32 \\
            \hline
            PISTIS \cite{grisafi2022pistis} & SW &
            Passive & \redcross & \redcross & $O(n)$ & openMSP430 \\
            \hline
            SANCUS \cite{noorman2013sancus} & HW &
            Passive & \redcross & \redcross & $O(n)$ & openMSP430 \\
            \hline
            TrustVisor \cite{mccune2010trustvisor} & HW &
            Passive & \redcross & \redcross & $O(n)$ & AMD \\
            \hline
            VRASED \cite{vrased} & Hybrid &
            Passive & \redcross & \redcross & $O(n)$ & openMSP430 \\
            \hline
            IDA \cite{arkannezhadida} & Hybrid &
            Passive & \greencheck & \redcross & $O(n)$ & openMSP430 \\
            \hline
            RATA \cite{rata} & Hybrid &
            Passive & \greencheck & \redcross & $O(1)$ & openMSP430 \\
            \hline
            GAROTA \cite {garota} & Hybrid &
            Active & \redcross & \greencheck  & $O(n)$ & openMSP430 \\
            \hline
            CASU \cite {casu} & Hybrid &
            Active & \greencheck & \redcross & $O(1)$ & openMSP430 \\            
            \thickhline
            \rowcolor{yellow!20}
            {\bf TRAIN} & {\bf Hybrid} &
            {\bf Passive/Active} & \greencheckt & \greencheckt  & {\bf $O(1)$} & {\bf openMSP430} \\
            \hline
        \end{tabular}}
        \vspace{.1cm}
        \caption{Comparison with Other Individual Attestation Schemes ($n$: attested area size)}
        \label{table:comp_ind_att}
        \vspace{-1.5em}
\end{table*}
%\vspace{0.2cm}
\begin{table*}[hbt!]
    % \footnotesize
    % \small
    \scriptsize
    \centering\captionsetup{justification = centering}
    {\centering}    \resizebox{\textwidth}{!}
    {\begin{tabular}{|Sc|Sc|Sc|Sc|Sc|}
        \hline
        \rowcolor{gray!20}
        {\bf $\boldsymbol{\sa}$ Method} & {\bf $\boldsymbol{\toctousa}$} & {\bf Simulator} & {\bf \makecell{Underlying \\ Platform}} & {\bf Remark} \\            
        \thickhline
        SEDA \cite{asokan2015seda} & \redcross &
        OMNeT++ & SMART/TrustLite & Provides pioneering scheme using secure hop-by-hop aggregation \\
        \hline
        SANA \cite{ambrosin2016sana} & \redcross &
        OMNeT++ & TyTan & Extends SEDA with aggregate signatures and sub-networks \\ 
        \hline
        LISA \cite{carpent2017lightweight} & \redcross &
        CORE & Unspecified & Introduces neighbor-based communication and quality metric \\ 
        \hline
        SeED \cite{ibrahim2017seed} & \redcross &
        OMNeT++ & SMART/TrustLite & Extends SEDA with self-initiated \ra\\ 
        \thickhline
        DARPA \cite{ibrahim2016darpa} & \redcross &
        OMNeT++ & SMART & Exchanges heartbeat messages to detect physically compromised devices \\
        \hline
        SCAPI \cite{kohnhauser2017scapi} & \redcross &
        OMNeT++ & ARM Cortex-M4 & Extends DARPA with regular session key generation and distribution on \prv-s \\
        \hline
        SAP \cite{nunes2019towards} & \redcross &
        OMNeT++ & TrustLite & Constructs formal model with security notions for \sa \\ 
        \hline
        SALAD \cite {kohnhauser2018salad} & \redcross &
        OMNeT++ & ARM Cortex-M4 & Offers lightweight message aggregation in dynamic topology \\ 
        \hline
        SCRAPS \cite {petzi2022scraps} & \redcross &
        Python-based & ARM Cortex-M33 & Constructs Pub/Sub protocol using blockchain-hosted smart contracts\\
        \hline
        ESDRA \cite {kuang2019esdra} & \redcross &
        OMNeT++ & Unspecified & Presents many-to-one \sa scheme to eliminate fixed \vrf \\
        \hline
        DIAT \cite {abera2019diat} & \redcross &
        OMNeT++ & PX4 & Introduces control-flow attestation for autonomous collaborative systems \\
        \hline
        \rowcolor{yellow!20}
        {\bf TRAIN} & \greencheckt &
        {\bf OMNeT++} & {\bf CASU/RATA} & Minimizes \toctou window, \ra overhead, and isolates \ra functionality\\
        \hline
    \end{tabular}}      
    \vspace{.1cm}
    \caption{Comparison with Other Network Attestation Schemes}
    \vspace{-1em}
    \label{table:comp_swa_att}
\end{table*}



\subsection{\system Compatibility}
%
The rationale behind our choice of \prv\ \ra platforms (i.e., \casu and \rata) is due to their minimal
\ra\ overhead (HMAC over minimal fixed size input), \toctoura\ mitigation, and extensibility,
which facilitates the construction of TimerTCB and NetTCB with a few hardware modifications.
However, \system is also compatible with other \ra platforms to minimize the \toctousa window,
while losing the benefits of \casu and \rata.
% 
Some examples of compatible devices are:
%
\begin{compactitem}
    \item Devices with custom hardware \rot, e.g., Sancus~\cite{noorman2013sancus} or TrustVisor~\cite{mccune2010trustvisor}.
    \item Devices with off-the-shelf TEE, such as TrustZone-A or TrustZone-M \cite{trustzone}.
    \item Devices with hybrid (HW/SW) \rot, such as SMART~\cite{smart}, VRASED~\cite{vrased}, TyTAN~\cite{tytan}, TrustLite~\cite{trustlite}.
    \item Devices without any hardware \rot. In this case, the device OS must be trusted.
\end{compactitem}

\subsection{\rata vs. \casu} \label{subsec:rata-casu-comparison}
%
Given that \rata operates as a passive \rot and \casu functions as an active \rot, 
it is natural to question the necessity of \rata and why \casu is not utilized exclusively.
The justification for employing \rata over \casu stems from three primary reasons:
(1) Memory Constraints: In \casu, only half of the program memory (PMEM) can store authorized software, while the other half is reserved for the secure update process. This significant (50\%) PMEM reservation can be prohibitive for low-end devices with limited memory. 
(2) Access to Non-Volatile Memory: \casu prevents normal software from modifying PMEM. However, some software may require access to non-volatile memory (e.g., flash) for benign purposes, such as storing text or image files. \rata allows such access and is preferred in these circumstances.
(3) Hardware Overheads: \rata has slightly lower hardware overheads compared to \casu. 

\subsection{\toctousa Minimization in \trapsnortc }\label{dis:precise_sync}
%
Even though \trapsnortc cannot achieve perfect synchronization without RTCs, it significantly 
reduces the \toctousa window compared to na\"ive approaches where the window scales with both 
spanning tree depth and network congestion. Recall that Section~\ref{subsec:security-analysis} illustrates the reduction in \toctousa window through a concrete example. By computing attestation timing based on the 
network topology, \trapsnortc effectively eliminates the spanning tree traversal component 
of the \toctousa window, leaving only network delay as a factor influencing the
imperfection of the synchronization. 
