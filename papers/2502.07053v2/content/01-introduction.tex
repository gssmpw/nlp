\section{Introduction \label{intro}}
% 
Rapid expansion and popularity of the Internet of Things (IoT) devices 
and Cyber-Physical Systems (CPS) have resulted in the deployment of vast numbers of 
Internet-connected and inter-connected devices. Such networks, 
composed of numerous devices, collaboratively execute sensing and/or actuation 
tasks in diverse settings, such as smart factories, warehouses, agriculture,
and environmental monitoring. However, 
the resource-constrained nature of IoT devices makes them vulnerable to 
remote attacks. This poses significant risks: malicious actors 
can compromise data integrity or even jeopardize safety within critical 
control loops. Given the safety-critical functions they perform 
and the sensitive data they collect, protecting IoT devices 
against such attacks is essential.
Remote attestation (\ra), a well-established security service, detects
malware on remote devices by verifying the integrity of their 
software state \cite{nunes2019towards, rata, li2011viper}. However, applying 
single-device \ra techniques to large IoT networks incurs high overhead.
Many techniques, including 
\cite{asokan2015seda,ambrosin2016sana,carpent2017lightweight, kohnhauser2017scapi,petzi2022scraps}, 
made progress towards efficient network  (aka swarm) attestation (\sa).
Nonetheless, they have substantial limitations, which form the motivation for this work.

\noindent{\bf Time-of-Check to Time-of-Use ({\toctou}):}
Prior techniques do not guarantee simultaneous (synchronized) attestation 
across all networked devices. Network structure, potential mobility, intermittent
connectivity, and congestion can lead to staggered reception of \ra requests,  
thus widening the time window for discrepancies in \ra timing.  
Also, even if networked devices are all of the same type, varying memory sizes 
and application task scheduling can result in different execution times of \ra.
These factors lead to a potentially long \toctou window, 
where the state of network devices is captured over an 
interval of time rather than at the same time. This increases the risk of 
undetected transient malware presence.  
\begin{figure}
    \includegraphics [width=\columnwidth]
    % [height=2.6in,width=\columnwidth]
    {images/TOCTOU.pdf}
   \captionsetup{justification=centering}
   \vspace{-1.8em}
   \caption{TOCTOU Window Minimized by \system}
    \label{fig: toctou window}
    \vspace{-1.5em}
\end{figure}
The \toctou problem arises in two cases: 

\noindent $\boldsymbol{\toctoura}$ -- 
the window of vulnerability between two successive \ra instances performed by a device, 
during which the state of the software is unknown and potentially compromised without detection; 
colored orange in Figure \ref{fig: toctou window}(a). \toctoura can be exploited by 
transient malware that: (1) infects a device, (2) remains active for a 
while, and (3) erases itself and restores the device software to its ``good'' state, as 
long as (1)-(3) occur between two successive \ra instances.

\noindent $\boldsymbol{\toctousa}$ -- the inter-device \toctou window,
i.e., the time variance between the earliest and the latest \ra performed across networked
devices. colored red in Figure \ref{fig: toctou window}(a).
Consider a situation where, the verifier performs network attestation. Device-A receives an 
attestation request, performs its attestation at time $t_0$ and replies to the verifier.
At time $t_1>t_0$, the verifier receives device-A's attestation report, checks it, and decides that 
device-A is benign. However, device-A is compromised at time $t_2>t_1$. Meanwhile, due to network delay, device-B performs its attestation at $t_3>t_2$ and replies.
The verifier receives device-B's attestation report at $t_4>t_3$ and 
(erroneously) concludes that both devices are now benign. 


\noindent \textbf{Synchronized Attestation} -- An important requirement for \sa is 
that all attestation reports should accurately reflect current system state.
If devices are attested at different times, the verifier cannot determine 
if the network as a whole is (or was) in a secure state, even if all individual 
\ra reports reflect the benign state. This undermines trust in current attestation methods, 
motivating the need for more synchronized network-wide attestation.

\noindent {\bf Performance Overhead:}
Attesting the entire software state of a 
device is computationally expensive. For safety-critical IoT devices, 
minimizing time spent on non-safety-critical tasks (e.g., \ra) is crucial to 
maintain responsiveness and real-time performance. Even a lightweight \ra, 
which is typically based on a device computing a relatively fast Message Authentication Code (MAC) 
(usually implemented as a keyed hash) requires doing so over the entire application program memory.
This introduces a non-negligible delay which is a function of memory size. For example, a TI MSP430 
microcontroller unit (MCU) running at $8$MHz takes $\approx450$ms to compute an
SHA2-256 HMAC over $4$KB of program memory \cite{vrased}. 
This delay is significant for real-time or safety-critical systems with 
tight timing constraints. 

\noindent {\bf Energy Overhead:}
Execution of \ra consumes
energy on battery-powered or energy-harvesting IoT devices. This is particularly 
problematic for devices deployed in remote or inaccessible locations where 
battery replacement is difficult or infeasible. Reducing power consumption 
is therefore both beneficial and important. 

\noindent {\bf Unreliable Communication:}
Malware-infected devices can 
subvert the attestation process by dropping or modifying attestation 
requests and replies. Prior techniques do not adequately address this 
problem.

\noindent To this end, we construct \system: \systemtext. It 
offers two protocol variants: \trapsrtc\ -- for devices equipped with  
real-time clocks (RTCs), and \trapsnortc\ -- for devices without such clocks. 
\system is designed to work with low-end IoT devices that have a small set 
of security features, based on \rata \cite{rata} or \casu \cite{casu} \ra techniques
which were originally developed for a single-device \ra setting. \system pairs
these \ra techniques with \garota\cite{garota} -- another recent technique that constructs
a minimal {\bf active} Root-of-Trust (\rot) for low-end devices and guarantees
operation even if a device is fully malware-compromised. Specifically, \system\
uses NetTCB and TimerTCB of \garota which ensure, respectively: (1) timely
sending and receiving messages, and (2) starting attestation on time, 
with no interference from any other software.

\noindent Contributions of this work are:
%
\begin{compactenum}[(1)]
%
\item \textbf{Reduced \toctou Window}: \system employs time synchronization 
(using RTCs or a depth-based mechanism) to ensure nearly simultaneous 
attestation across all devices in the network, substantially reducing the \toctousa window. 
The \toctoura window is mitigated by the use of \casu or \rata security features.
Figure \ref{fig: toctou window}(b) shows decreased \toctou windows by \system.
%
\item \textbf{Efficient and resilient \ra}: \system combines a few \rot constructions 
to minimize \ra-induced performance overhead and power consumption for individual 
devices, while guaranteeing timely \ra execution by isolating it from any potential 
malware interference.
%
\item \textbf{Open-Source Implementation}: \system's practicality and 
cost-effectiveness is confirmed via a fully functional prototype 
on a popular low-end IoT device platform -- TI MSP430 micro-controller,
using FPGA \cite{TRAPSAnonOpenSource}.
%
\end{compactenum}
