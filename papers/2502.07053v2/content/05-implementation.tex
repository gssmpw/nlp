\section{Implementation \label{impl}}
%
\system is prototyped atop openMSP430\cite{openMSP430}, an open-source implementation of TI MSP430 MCU, 
written in Verilog HDL. OpenMSP430 can execute software generated by any MSP430 toolchain 
\cite{msp430-gcc} with near-cycle accuracy. We extended both \rata and \casu architectures to 
support \system. In this implementation, \prv and \vrf are connected via UART.

\subsection{\system Software}
%
Using the native msp430-gcc toolchain, \system software on \prv is compiled to generate software images 
compatible with the memory layout of the modified openMSP430. \system software, responsible for processing 
\system protocol messages and generating attestation responses, is housed in ROM.
NetTCB is triggered whenever a \system protocol message is received; this is determined by the cleartext 
message type in the header.

Also, TimerTCB is triggered to start attestation whenever the timer expires in the {\bf Attest-Wait} state.
For cryptographic operations we use a formally verified cryptographic library, HACL* \cite{hacl}.
It provides high-assurance implementations of essential cryptographic primitives, such as hash 
functions and MAC-s. SHA2-256 and HMAC are used for hash and MAC, respectively.
Both \rata and \casu implement their respective cryptographic operations using HACL*.


To emulate \vrf, we developed a Python application with $\approx~200$ lines of code, 
as described in Sections \ref{sec:one} and \ref{sec:two}. The application runs on an
Ubuntu 20.04 LTS laptop with an Intel i5-11400 processor @$2.6$GHZ with $16$GB of RAM.

%\vspace{0.1cm}
\subsection{\system Hardware}
%\vspace{0.2cm}

As mentioned earlier, \prv-s in \system can adopt either \casu or \rata architecture,
possibly equipped with different system resources (e.g., CPU clock, 
memory, peripherals). We refer to \casu-based \prv-s as $\boldsymbol{\trapscasu}$ and 
\rata-based \prv-s as $\boldsymbol{\trapsrata}$. We implemented and evaluated both 
as part of the proof-of-concept.

The design is synthesized using Xilinx Vivado 2023.1, a popular logic synthesis tool. 
It generates the hardware implementation for the FPGA platform. The synthesized design 
is then deployed on a Basys3 Artix-7 FPGA board for prototyping and evaluating hardware design.

\begin{figure}
    \captionsetup{justification = centering}
    \centering
    \includegraphics[height=1.3in,width=0.8\columnwidth]{images/PoC.pdf}
    \caption{\system Proof-Of-Concept with Three \prv-s}  
    \vspace{-1.5em}
    \label{fig: PoC}
\end{figure}

Figure \ref{fig: PoC} shows a proof-of-concept implementation of \system. In it,
three \prv-s (implemented on Basys3 FPGA boards) are connected to \vrf. For the sake of simplicity, 
\prv-s are deployed using a star topology for signal routing. All three \prv-s in Figure \ref{fig: PoC} are \trapscasu 
devices. However, we also implemented \system with \trapsrata devices for performance evaluation.

%\vspace{0.2cm}