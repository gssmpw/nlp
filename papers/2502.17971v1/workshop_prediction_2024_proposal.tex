\documentclass[letterpaper, 10 pt, conference]{ieeeconf}
%\usepackage[top=2cm, bottom=2cm, left=2cm, right=2cm]{geometry}
\usepackage{amsmath}
\usepackage{amsfonts}
\usepackage{amssymb}
\usepackage{graphicx}
\usepackage{pgfplots}
\usepackage{hyperref}
\usepackage{pdfpages}
\usepackage{multirow}
\usepackage{todonotes}
\renewcommand\footnotemark{}
\renewcommand\footnoterule{}
\renewcommand{\baselinestretch}{0.99}
\IEEEoverridecommandlockouts

\title{Multimodal Interaction and Intention Communication for Industrial Robots}
%	Towards Superhuman Motion Prediction}

\author{Tim Schreiter$^{1*}$, Andrey Rudenko$^{2*}$, Jens V. Rüppel$^{1}$, Martin Magnusson$^{3}$, Achim J. Lilienthal$^{1,3}$%<-this % stops a space
    \thanks{$^{1}$ \raggedright {Technical University of Munich, MIRMI,
    Chair of Perception for Intelligent Systems, Germany, \url{tim.schreiter@tum.de}}}
    \thanks{$^{2}$ Bosch Corporate Research, Germany,\url{andrey.rudenko@de.bosch.com}
    (${^*}$Shared first author)}     
    \thanks{$^{3}$ \"Orebro University, Sweden}
	%\thanks{${^*}$ Shared first author}
}

\date{}
\begin{document}

\maketitle

\begin{abstract}
%
%
Successful adoption of industrial robots will strongly depend on their ability to safely and efficiently operate in human environments, engage in natural communication, understand their users, and express intentions intuitively while avoiding unnecessary distractions. To achieve this advanced level of Human-Robot Interaction (HRI), robots need to acquire and incorporate knowledge of their users' tasks and environment and adopt multimodal communication approaches with expressive cues that combine speech, movement, gazes, and other modalities. This paper presents several methods to design, enhance, and evaluate expressive HRI systems for non-humanoid industrial robots. We present the concept of a small anthropomorphic robot communicating as a proxy for its non-humanoid host, such as a forklift. We developed a multimodal and LLM-enhanced communication framework for this robot and evaluated it in several lab experiments, using gaze tracking and motion capture to quantify how users perceive the robot and measure the task progress. %Finally, we quantify the effect of LLM-enhanced responses on a representative interaction with an industrial robot involving approach, instruction, and object manipulation.
\end{abstract}

\section{Introduction}
\label{sec:introduction}
%The increasing presence of robots in human-occupied spaces necessitates the development of effective communication strategies to facilitate successful human-robot interaction. The success of robot communication is dependent on multiple interconnected elements, including the delivery of clear and timely messages, the complementation of verbal communication with appropriate non-verbal signals, the guidance of attention to task-relevant environmental features, and the interpretation of human non-verbal cues such as spatial positioning, gestures, and gaze patterns. These abilities are crucial in industrial contexts, where effective collaboration between humans and robots relies on mutual understanding and predictable behavior. However, integrating these communication elements into systems that can seamlessly adapt to the dynamic environment of human-robot interactions poses significant challenges. Robots are often constrained by their native design, which limits their ability to produce legible social cues that humans can intuitively understand and respond to.

Robots are increasingly used in shared environments with humans, making effective communication necessary for successful human-robot interaction. Many aspects of robot behavior define successful communication: generating clear, concise, and timely messages, supporting these messages with appropriate signals (verbal and non-verbal), directing the attention towards the relevant parts of the task and the environment while avoiding unnecessary distractions, and reading user feedback and task engagement from non-verbal cues such as position in space, gestures, and gaze direction. Combining these elements in a system that naturally fits dynamic human environments is challenging. Robots are often limited by their native design, making it difficult for them to produce legible social cues. Nevertheless, strong communication abilities are crucial in industrial contexts, where effective collaboration between humans and robots relies on mutual understanding and predictable behavior.

As part of the EU project DARKO\footnote{\url{https://darko-project.eu/}}, we develop methods for the next generation of agile production robots that are aware of humans and their intentions to smoothly and intuitively interact with them. Key to our research are \textit{Transferability} and \textit{Quantification} aspects of our methods. Aiming to address the inherent need to design transferable solutions to HRI that can be applied and verified on different robotic platforms \cite{cha2018survey}, we develop the concept of an ``Anthropomorphic Robotic Mock Driver'' (ARMoD) to communicate on behalf of the non-humanoid host platform. Here, we investigate the application of the ARMoD supporting the communication of an industrial robot in a representative interaction, involving approach, spoken instruction, and object manipulation (see Fig.~\ref{fig1}). To support the interaction in these settings, we utilize the developed expressive multimodal communication architecture, which includes robot speech, gaze, and gestures, directed to the task-relevant parts of the environment. To quantify the effect of the various communication styles, we adopt human gaze tracking as a measure of attention, intention, and task progress. Finally, we compare the traditional, partially scripted interaction to an LLM-enhanced one, investigating the potential to adapt the robot responses to the inherently dynamic and unpredictable human behavior.

\begin{figure}
    \centering
    \includegraphics[width=\linewidth]{sHRI.png}
    \caption{\textbf{Focus points and methods in our HRI Studies:}
(1) Anthropomorphic Communication Proxy for non-humanoid platforms
(2) Multimodal and LLM-enhanced communication
(3) Gaze tracking and motion capture
(4) Controlled user studies}
    \label{fig1}
    %\vspace{-1cm}
\end{figure}

%The increasing deployment of robots in shared environments with humans highlights the need for effective communication to ensure successful interaction. Building on our previous research \cite{schreiter2023advantages,schreiter2024human,schreiter2025evaluating}, we emphasize the importance of gaze tracking as a critical measure of non-verbal communication. Gaze serves as an objective indicator of user attention, intention, and task engagement by capturing where and how long users focus on specific elements of the environment. In our research, analyzing gaze dynamics, such as fixation points and head rotations, has enabled us to quantify the impact of robot behaviors on directing human attention and facilitating natural interaction. This approach validates our methods in controlled industrial settings and informs the development of adaptive and transferable communication strategies for next-generation robotic systems.

\section{Methods}

\subsection{Lab studies}
\label{sec:lab}

% To study natural interactions between robots and humans while being able to isolate the factors and get accurate, objective measurements, we opt for lab studies.

% We propose a scripted interaction model, which includes many elements found in industrial environments and a more spontaneous one, taking place in a more extensive human motion study.

We opted to investigate human-robot interaction scenarios in controlled laboratory settings. Although online studies offer scalability \cite{toris2013bringing} and ``in-the-wild'' experiments allow validation in complex social settings \cite{jung2018robots}, lab studies strike a balance for precise measurements of real human behavior and allow to isolate and condition the factors that may influence the interaction. Controlled environments also facilitate experimental repeatability and high-quality data collection. Specifically, in our recent studies \cite{schreiter2022effect,schreiter2023advantages,schreiter2024human,schreiter2025evaluating}, we investigated human-robot interaction in a scripted setup, which includes many elements found in industrial environments, and also more spontaneous interactions, which are recorded as part of a large-scope study of indoor human motion \cite{schreiter2024thor}.

The scripted interaction features several steps, relevant for industrial robots. Participants are instructed to deliver a tin can to the table, where they are approached by the robot asking for their assistance. The robot asks to pick up a large box and place it on the forks of the forklift. The interaction concludes afterwards with a disengagement.
%The scripted model structures robot-human interactions in a controlled manner, integrating predefined robot communication behaviors such as looking, pointing, and running preprogrammed behaviors that allow for repeatability in experimentation while considering natural responses from human participants.
%We implement this interaction with a forklift and a mobile manipulator platform and use it as a basis to investigate how robot communication styles affect attention fixation and reaction times and validate the anthropomorphic robotic driver concept by analyzing human perception and trust in industrial human-robot collaboration.

In contrast, the spontaneous interactions \cite{schreiter2024thor} involve the robot being approached by a person in different positions and settings. The robot communicates its next goal point and asks the person to accompany it. In these interactions, the robot moves either differentially or omnidirectionally.
%Unlike scripted interactions, which follow predefined sequences, the spontaneous interaction model captures how humans interact with a mobile robot in real-time scenarios. To make communication more intuitive and responsive to real-world unpredictability, we deploy the ``Wizard of Oz'' technique \cite{salber1993applying} alongside the scripted behavior. %LLMs are well known for their potential to enhance interaction along these lines \cite{kim2024understanding}. We study the performance of an LLM-enhanced communication against the scripted interaction model (see Section \ref{sec:llms})

\subsection{Multimodal and LLM-enhanced communication}\label{sec:llms}

%Using robot communication channels is crucial for successfully deploying robots in shared environments \cite{bonarini2020communication}. During active communication, expressive robots can achieve multimodality to convey robotic intent to the human user \cite{pascher2023communicate} by combining their available communication channels, including gestures, gazes, speech, movement, light, and color. We follow the hypothesis of Salem et al.$\,$\cite{salem2011friendly} that human interaction partners will evaluate robots capable of multimodal communication more positively. This multimodality can also provide better support by referring to objects in the environment, giving non-verbal feedback, and enhancing the naturalness and efficiency of interactions. Research shows that multimodal approaches can improve interaction speed, accuracy, and naturalness by better mimicking human communication patterns, leading to fixations being concentrated on the robot face \cite{salem2011friendly, kompatsiari2019measuring}.

When actively interacting with the user, robots can benefit from a wide variety of available modalities \cite{pascher2023communicate} to support and enrich their messages with non-verbal cues, acknowledge the reception of user's commands and refer to the objects in the environment. Research shows that multimodal approaches can improve interaction speed, accuracy, and naturalness by better mimicking human communication patterns \cite{salem2011friendly, kompatsiari2019measuring}. Furthermore, users will likely evaluate robots capable of multimodal communication more positively \cite{salem2011friendly}.
Following the interaction designs presented in Sec. ~\ref{sec:lab}, we implemented a multimodal communication design to support users' tasks and compared it to verbal-only conditions.

In addition to expressive multimodal communication, robots need to flexibly adapt their messages and actions to the environment's context of the environment and the status of the interaction. We use Large Language Models (LLMs), owing to their advanced reasoning capabilities, to extend our multimodal communication framework with real-time context interpretation and natural language response generation capabilities. The potential to improve the interaction flow, in comparison to more traditional pre-scripted behavior, is yet to be qualified in practice. Specifically, we compared the scripted interaction from Sec.~\ref{sec:lab} with an equivalent one, which benefits from LLM-enhanced responses \cite{schreiter2025evaluating}.

%Integrating Large Language Models (LLMs) into a multimodal communication framework increases the capabilities of Human-Robot Interaction (HRI) systems \cite{schreiter2025evaluating}. During traditional multimodal approaches (such as using ARMoD) that leverage speech, gaze, and gestural cues to foster natural communication, LLMs introduce a dynamic, context-sensitive layer that enhances the robot's ability to interpret and generate human-like language in real-time. This fusion allows for adaptive responses that extend beyond pre-scripted interactions, resulting in more fluid and engaging exchanges. Our studies indicate that LLM-enhanced systems increase user engagement and improve communication and relevance in industrial tasks \cite{schreiter2025evaluating}. By balancing the robustness of multimodal cues with the flexibility of LLMs, we create transferable communication strategies across diverse robots that are adaptable to varying task environments.

\subsection{Anthropomorphic communication proxies}
\label{sec:armod}

To be capable of expressive multimodal communication, robots need specialized modalities that are intuitively interpretable by people. However, the function-driven design of non-humanoid service and industrial robots limits their ability to express human-readable cues. To address these conflicting requirements, we introduced the Anthropomorphic Robotic Mock Driver (ARMoD) concept of a small robotic entity that extends the host system (e.g., a non-humanoid robot) and can communicate with natural, human-readable signals. ARMoD is designed to standardize communication patterns across diverse robotic platforms \cite{schreiter2023advantages}. We designed a multimodal communication protocol for ARMoD that combines speech, gaze, and referential gestures. The ARMoD was deployed in all interactions, presented in Sec.~\ref{sec:lab}

\begin{figure}
\vspace{0.2cm}
    \centering
    \includegraphics[width=\linewidth]{RobotHeatmaps.png}
    \caption{Heatmaps showing participant gaze distribution on two robot platforms (including the ARMoD) for two interaction styles (verbal-only and multimodal).
    In the multimodal style, eye fixations are more concentrated on the ARMoD humanoid robot. 
    }
    %\caption{Heatmaps showing participant gaze distribution on two robot platforms for two interaction styles. One deployed \textbf{verbal-only communication (left)}. Fixations are spread widely across the robot and its sensory equipment, with multiple areas of high attention (red) on the ARMoD's body and one on the RGBD camera. Another one deployed \textbf{multimodal communication (right)}. Participants focus more strongly on the ARMoD, as indicated by only one single area of high attention on the robot's face.}
    \label{fig2}
    %\vspace{-1cm}
\end{figure}

%This consistency allows for a thorough assessment of the effectiveness of interactions. It ensures that solutions created in one experimental setting can easily be adapted to new robots and different task environments (see Fig. \ref{fig1} A and B). %In doing so, ARMoD advances the overall goal of designing intuitive and adaptable human-robot interfaces, essential for both controlled laboratory experiments and future in-the-wild applications.



%A structured environment accommodates studies of anthropomorphic communication \cite{schreiter2023advantages} and natural motion analysis (\cite{schreiter2024thor}. We implement two distinct interaction paradigms: a structured protocol for studying specific communication modalities and trust development with anthropomorphic proxies and a more naturalistic setting allowing spontaneous human motion and gaze behavior analysis during navigation and object manipulation tasks. Our dual approach enables us to examine scripted interactions to analyze and compare different communication strategies (verbal-only vs. multimodal) and more dynamic scenarios that reveal natural behavioral patterns in human-robot shared spaces. The laboratory environment, equipped with high-precision motion capture and gaze tracking systems, provides controlled conditions for studying subtle behavioral cues while maintaining sufficient ecological validity to generate insights applicable to industrial settings.

\subsection{Gaze tracking and motion capture}
\label{sec:thor}

One of the key challenges of HRI research is assessing the benefit of novel robot behaviors \cite{cha2018survey}. In our experiments, we use gaze tracking to objectively measure user eye movements and head rotations \cite{schreiter2024human} as participants navigate, explore, and manipulate objects in shared environments with robots. By integrating these measures with motion capture data, we directly correlate gaze behavior with task performance and attention distribution during motion, providing critical insights into how users perceive and adjust their behavior in the presence of our robot communication strategies \cite{schreiter2025evaluating}. This approach validates the effectiveness of our novel behaviors and informs the improvement of HRI systems for more natural and intuitive interactions.

\section{Results}

Combining the insights from 3D position and head orientation motion capture and gaze tracking, we could derive several notable conclusions about human behavior in scripted and spontaneous interactions with robots.

We found that users react faster in collaborative tasks with the robot equipped with an ARMoD and multimodal interaction style. When the robot gives instructions supported by gaze, users are quicker to localize goal points and objects of interest \cite{schreiter2023advantages}. We also notice the concentration of fixations on the ARMoD using a multimodal communication style, as opposed to a verbal-only interaction (see Fig.~\ref{fig2}). In contrast, we do not find significant differences between the perception of the robot moving differentially vs. omnidirectionally in the THÖR-MAGNI dataset \cite{schreiter2024human}. Similarly, participants did not achieve higher task efficiency when the robot guided the interaction with LLM-enhanced responses, as compared to the fully scripted scenario \cite{schreiter2025evaluating}.

These examples illustrate our attempts to achieve more efficient and natural human-robot interaction, that can be transferred to different robots and quantified beyond subjective questionnaire ratings. In our future work, we aim to transfer these communication and evaluation methods to other domains outside of industry, such as elderly care.

%In summary, our research aims to transfer the methods described herein to other areas of robotics application outside of industry, such as elderly support. The integrated approach used in this study combines anthropomorphic communication proxies, multimodal and LLM-enhanced interactions, precise gaze tracking, and motion capture. This approach has proven effective in enhancing the naturalness and efficiency of human-robot interactions. We have validated these techniques through rigorous user studies, establishing a robust framework that optimizes task performance in industrial settings. This framework promises to improve user engagement and safety in contexts such as elderly care. The transferability of these methods opens opportunities to tailor adaptive, intuitive robotic systems that meet the diverse needs of different user populations beyond the factory floor.

%\pagebreak
\bibliographystyle{IEEEtran}
\bibliography{prediction_state_of_the_art}

\end{document}