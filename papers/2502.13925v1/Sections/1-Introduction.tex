\section{Introduction}

\begin{quote}
    \textit{In the space between the panels, human imagination takes separate images and transforms them into a single idea.}\par\raggedleft--- Scott McCloud (1993)
\end{quote}
% "Here in the limbo of the gutter, human imagination takes two separate images and transforms them into a single idea." — Scott McCloud, Understanding Comics (1993), Chapter 3
% In the space between the panels, human imagination takes separate images and transforms them into a single idea.

\begin{figure}[t]
\centering
\includegraphics[width=0.98\columnwidth]{fig/crop_image1.pdf}
\caption{An example from the \dataset dataset includes the correct answers for our three sub-tasks: comprehension, frame prediction, and reordering. All these tasks are presented as multiple-choice questions, with distractors excluded due to limited context. Star means correct answer.}
\label{fig:intro}
\end{figure}

Recent advancements in Large Multimodal Models (LMMs), particularly GPT-4o~\citep{hurst2024gpt40}, have yielded significant breakthroughs in a various visual-language tasks, including image captioning~\cite{liu2023llava, Ghandi:2024imagecaption}, visual question answering~\cite{Lu:2023vqa, Zhu:2024minigpt-4}, video understanding~\cite{Zhang:2023video-llama, maaz:2024video-chatgpt}, and so on. LMMs have shown promising efficacy in processing and interpreting visual content~\cite{Yin:2024lmmsurvey}, greatly enhancing their capacity to interact with the real world.

\section{Backup: compare with previous works}

\paragraph{Comparison with Theorem 1 of \cite{srikant2024rates}.} While the framework of our proof of Theorem \ref{thm:Srikant-generalize} is mainly inspired by the proof of Theorem 1 of \cite{srikant2024rates}, there are some noteworthy differences. Most importantly, we observe that in the equation beginning from the bottom of Page 7 and continuing to the start of Page 8, the right-most side contains a term
\begin{align}\label{eq:Srikant-error}
-\frac{1}{n-k+1} \mathsf{Tr}\left(\bm{\Sigma}_{\infty}^{-\frac{1}{2}}(\bm{\Sigma}_k - \bm{\Sigma}_{\infty})\bm{\Sigma}_{\infty}^{-\frac{1}{2}}\mathbb{E}[\nabla^2 f(\tilde{\bm{Z}}_k)]\right);
\end{align}
the author argued that ``by taking an expectation to remove conditioning, and defining $\bm{A}_k$ to be $\mathbb{E}[\nabla^2 f(\tilde{\bm{Z}}_k)]$'', this term can be transformed to the term
\begin{align}\label{eq:Srikant-wrong}
-\frac{1}{n-k+1} \mathsf{Tr}\left(\bm{A}_k \left(\bm{\Sigma}_{\infty}^{-\frac{1}{2}} \mathbb{E}[\bm{\Sigma}_k]\bm{\Sigma}_{\infty}^{-\frac{1}{2}}-\bm{I}\right)\right)
\end{align}
in the expression of Theorem 1. However, we note that the function $f(\cdot)$, as defined on Page 6 as the solution to the Stein's equation with respect to $\tilde{h}(\cdot)$, is \emph{dependent on} $\mathcal{F}_{k-1}$; in fact, $f$ corresponds to the function $f_k$ in our proof. Consequently, the terms $\bm{A}_k = \mathbb{E}[\nabla^2 f(\tilde{\bm{Z}}_k)]$ (which is actually a conditional expectation with respect to $\mathcal{F}_{k-1}$), and $\bm{\Sigma}_k$ (which corresponds to $\bm{V}_k$ in our proof), are confounded by $\mathcal{F}_{k-1}$ and hence \emph{not independent}. Therefore, taking expectation, with respect to $\mathcal{F}_0$, on \eqref{eq:Srikant-error} should yield
\begin{align}\label{eq:Srikant-right}
-\frac{1}{n-k+1} \mathbb{E}\left\{\mathsf{Tr}\left(\bm{A}_k \left(\bm{\Sigma}_{\infty}^{-\frac{1}{2}} \bm{\Sigma}_k\bm{\Sigma}_{\infty}^{-\frac{1}{2}}-\bm{I}\right)\right)\right\}
\end{align}
Notice that the expectation is taken over the trace as a whole, instead of only $\bm{\Sigma}_k$. However, also due to the confounding bewteen $\bm{A}_k$ and $\bm{\Sigma}_k$, there is no guarantee that the sum of \eqref{eq:Srikant-right} is bounded as shown in the proof of Theorem 2 in \cite{srikant2024rates} on page 10. In other words, the framework of the proof needs a substantial correction to obtain a meaningful Berry-Esseen bound. 

Our solution in the proof of Theorem \ref{thm:Srikant-generalize} is to replace the matrix $\bm{Q}=\sqrt{n-k+1}\bm{\Sigma}_{\infty}$, as defined on Page 6 of \cite{srikant2024rates}, with the matrix $\bm{P}_k$, following the precedent of \cite{JMLR2019CLT}. This essentially eliminates the term \eqref{eq:Srikant-right}, but would require $\bm{P}_k$ to be measurable with respect to $\mathcal{F}_{k-1}$. For this purpose, we impose the assumption that $\bm{P}_1 = n\bm{\Sigma}_n$ almost surely, also following the precedent of \cite{JMLR2019CLT}. The relaxation of this assumption would be addressed in Theorem \ref{thm:Berry-Esseen-mtg}. 

Another important improvement we made in Theroem \ref{thm:Srikant-generalize} is to tighten the upper bound through a closer scrutiny of the smoothness of the solution to the Stein's equation, as is indicated in Proposition \ref{prop:Stein-smooth}. This paves the way for Corollary \ref{cor:Wu}, the proof of which we present in the next subsection. 


Despite recent advances, the capability of LMMs to process and reason over sequential images remains underexplored, even though sequential visual inputs are prevalent in real-world applications~\cite{yang2024:deepsemantic, liu2024:iibench}. While existing benchmarks primarily evaluate LMMs on single images, often emphasizing surface-level understanding, they fail to address the complexities of sequential dependencies. In multi-image contexts, the ability to discern implicit meanings and contextual relationships is essential for comprehensive interpretation. In particular, understanding nuanced concepts such as sarcasm~\cite{cai-etal-2019-multi, tang:2024sarcasm}, humor~\cite{PatroL:2021humor, hessel2023:androids}, and other multi-faceted deep meanings~\cite{Zhang:2024CDeep} requires reasoning beyond isolated frames. This gap highlights the need for a deeper investigation into the reasoning capabilities of LMMs over dynamic image sequences.


To address this gap, we introduce \dataset, a novel benchmark designed to assess the reasoning ability of LMMs on temporal image sequences, including contextual structure, temporal relationships among images, and underlying semantics. \dataset consists of three challenging subtasks: visual narrative comprehension, contextual frame prediction, and temporal frames reordering, as shown in Figure~\ref{fig:intro}. 

With \dataset, we aim to advance the development of LMMs in temporal-visual comprehension while identifying their current limitations.

Our comprehensive evaluation of $16$ state-of-the-art LMMs on \dataset reveals a substantial performance gap between AI and human capabilities in sequential image comprehension, especially in the reordering task. Most notably, GPT-4o achieves only $23.93\%$ accuracy in the reordering subtask and trails human performance by $30\%$ in visual narrative comprehension. Further quantitative analysis identifies several key factors affecting the sequential understanding performance of LMMs, highlighting the fundamental challenges that remain in the development of LMMs.

% Our key contributions are as follows:
% We introduce STRIPCIPHER, the first comprehensive benchmark that evaluates LMMs on implicit visual narrative reasoning.
% We propose three novel subtasks (narrative comprehension, frame prediction, and reordering) that challenge existing LMMs beyond single-image understanding.
% We systematically benchmark 9 state-of-the-art LMMs, revealing substantial gaps between AI and human capabilities.
% Through extensive analysis, we uncover key factors affecting LMMs' sequential reasoning, paving the way for future model improvements."