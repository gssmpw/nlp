% This must be in the first 5 lines to tell arXiv to use pdfLaTeX, which is strongly recommended.
\pdfoutput=1
% In particular, the hyperref package requires pdfLaTeX in order to break URLs across lines.

\documentclass[11pt]{article}

% Change "review" to "final" to generate the final (sometimes called camera-ready) version.
% Change to "preprint" to generate a non-anonymous version with page numbers.
% \usepackage[review]{acl}
\usepackage[final]{acl}
% Standard package includes
\usepackage{times}
\usepackage{latexsym}

% For proper rendering and hyphenation of words containing Latin characters (including in bib files)
\usepackage[T1]{fontenc}
% For Vietnamese characters
% \usepackage[T5]{fontenc}
% See https://www.latex-project.org/help/documentation/encguide.pdf for other character sets

% This assumes your files are encoded as UTF8
\usepackage[utf8]{inputenc}

% This is not strictly necessary, and may be commented out,
% but it will improve the layout of the manuscript,
% and will typically save some space.
\usepackage{microtype}

% This is also not strictly necessary, and may be commented out.
% However, it will improve the aesthetics of text in
% the typewriter font.
\usepackage{inconsolata}

\usepackage{hyperref}
\usepackage{url}
\newcommand{\explain}[2]{\underbrace{#1}_{{\footnotesize\raggedright #2}}}

\usepackage[most]{tcolorbox} % 载入 tcolorbox 宏包
% for figures
\usepackage{graphicx}
\usepackage{subfigure}
\usepackage{wrapfig}

% for equation
\usepackage{amsmath}
\usepackage{amstext}
\usepackage{amsfonts}
\usepackage{bm}

\usepackage{bbm}
% for table
\usepackage{multirow}
\usepackage{booktabs}
\usepackage{array}
\usepackage{caption}
\usepackage{multirow}
\usepackage{booktabs}
\usepackage{array}
\usepackage{caption}
\usepackage{color}
\usepackage{colortbl}
\usepackage{tablefootnote}
\usepackage{adjustbox}

% for algorithm
\usepackage{caption}
\usepackage{algorithm}
\usepackage{algpseudocode}
\newcommand{\mycolor}[1]{\textcolor[RGB]{64,101,149}{#1}}
\newcommand{\mydarkcolor}[1]{\textcolor[RGB]{64,101,149}{#1}}
\algnewcommand{\LineComment}[1]{\Statex ~~~~~~\textsc{//}~\textit{#1}}

%for itemize
\usepackage{enumitem}
\setenumerate[1]{itemsep=0pt,partopsep=0pt,parsep=\parskip,topsep=5pt}
\setitemize[1]{itemsep=0pt,partopsep=0pt,parsep=\parskip,topsep=5pt}
\setdescription{itemsep=0pt,partopsep=0pt,parsep=\parskip,topsep=5pt}

% for highlight
\usepackage{soul}

% for taxonomy
\usepackage{tikz}
\usepackage[edges]{forest}
\definecolor{hidden-draw}{RGB}{64,101,149}
% \definecolor{hidden-pink}{RGB}{255,245,247}
\definecolor{hidden-pink}{RGB}{231,239,250}

% for notation
\usepackage[mathscr]{euscript}
\newcommand{\dataset}{\textsc{StripCipher}\xspace}

%for itemize
\usepackage{amssymb}  
\usepackage{pifont}
\newcommand{\cmark}{\ding{51}}
\newcommand{\xmark}{\ding{55}}
\newcommand{\greenyes}{\textcolor{green}{\ding{51}}}
\newcommand{\redno}{\textcolor{red}{\ding{55}}}
\newcommand{\cpm}[1]{\textcolor{gray}{$_{\pm #1}$}}

% for logo
\usepackage{scalerel}
\usepackage{graphicx}  % 使得表格可以自动调整
\usepackage{array}     % 允许使用 p{} 设定列宽
\usepackage{booktabs}  % 提供更好的表格格式
\usepackage{ragged2e}  % 使得文本可以左对齐且换行


% for title
\usepackage{xspace}

% If the title and author information does not fit in the area allocated, uncomment the following
%
%\setlength\titlebox{<dim>}
%
% and set <dim> to something 5cm or larger.

% From Still to Motion: Evaluating Large Multimodal Models' Understanding of Comic Strips
% Cracking the Code of Visual Narratives: Can LMMs Uncover Implicit Meanings Behind Comic Strips?
% Cracking the Code of Visual Narratives: Can LMMs Uncover Implicit Meanings Behind Image Sequences?

% \scalerel*{\includegraphics{fig/comic.png}}{{\rule{3ex}{3ex}}}
% \texorpdfstring{\includegraphics[width=18pt]{fig/comic.png}}{}
\title{
\texorpdfstring{\includegraphics[width=18pt]{fig/comic.png}}{}
\textit{Beyond Single Frames:} Can LMMs Comprehend  Temporal and Contextual Narratives in Image Sequences?
}

% \author{\textbf{Xiaochen Wang}\textsuperscript{1,$*$} \textbf{Heming Xia}\textsuperscript{2,\thanks{equal contribution}}, \textbf{Jialin Song}\textsuperscript{1,$\diamond$}, \textbf{Longyu Guan}\textsuperscript{1,$\diamond$}, \textbf{Yixin Yang}\textsuperscript{1}, \textbf{Qingxiu Dong}\textsuperscript{1}, \\
% \textbf{Weiyao Luo}\textsuperscript{1}, \textbf{Yiru Wang}\textsuperscript{4}, \textbf{Yifan Pu}\textsuperscript{3}, \textbf{Xiangdi Meng}\textsuperscript{1}, \textbf{Wenjie Li}\textsuperscript{2}, \textbf{Zhifang Sui}\textsuperscript{ 1,\thanks{Corresponding Author}}\\
% \textsuperscript{1} State Key Laboratory of Multimedia Information Processing, Peking University\\
%   \textsuperscript{2} Department of Computing, The Hong Kong Polytechnic University \\ 
%    \textsuperscript{3} Tsinghua University      \textsuperscript{4} ModelTC \\ 
%    }

\author{%
\textbf{Xiaochen Wang}\textsuperscript{1 $*$},
\textbf{Heming Xia}\textsuperscript{2 $*$},
\textbf{Jialin Song}\textsuperscript{1 $\dagger$},
\textbf{Longyu Guan}\textsuperscript{1 $\dagger$},
\textbf{Yixin Yang}\textsuperscript{1},
\textbf{Qingxiu Dong}\textsuperscript{1}, \\
\textbf{Weiyao Luo}\textsuperscript{1}, 
\textbf{Yiru Wang}\textsuperscript{4}, 
\textbf{Yifan Pu}\textsuperscript{3}, 
\textbf{Xiangdi Meng}\textsuperscript{1}, 
\textbf{Wenjie Li}\textsuperscript{2}, 
\textbf{Zhifang Sui}\textsuperscript{1 $\ddagger$} \\
\textsuperscript{1} State Key Laboratory of Multimedia Information Processing, Peking University\\
\textsuperscript{2} Department of Computing, The Hong Kong Polytechnic University\\
\textsuperscript{3} Tsinghua University \quad
\textsuperscript{4} ModelTC \\
  \texttt{ wangxiaochen@stu.pku.edu.cn}
}



\begin{document}
\maketitle

\renewcommand{\thefootnote}{\fnsymbol{footnote}}
% \footnotetext{\textsuperscript{$*$} Equal first contribution. \textsuperscript{$\dagger$} Equal second contribution.} 
% \footnotetext{\textsuperscript{$\ddagger$} Corresponding Author.} 
\footnotetext{\textsuperscript{$*$} Equal first contribution. } 

\footnotetext{\textsuperscript{$\dagger$} Equal second contribution.}
\footnotetext{\textsuperscript{$\ddagger$} Corresponding author.} 

% \footnotetext[1]{Equal first contribution.} \footnotetext[2]{Equal second contribution.}
% \footnotetext[3]{Corresponding Author.}
% \renewcommand{\thefootnote}{\arabic{footnote}}

\begin{abstract}
Large Multimodal Models (LMMs) have achieved remarkable success across various visual-language tasks. However, existing benchmarks predominantly focus on single-image understanding, leaving the analysis of image sequences largely unexplored. To address this limitation, we introduce \dataset, a comprehensive benchmark designed to evaluate capabilities of LMMs to comprehend and reason over sequential images. \dataset comprises a human-annotated dataset and three challenging subtasks: visual narrative comprehension, contextual frame prediction, and temporal narrative reordering. Our evaluation of $16$ state-of-the-art LMMs, including GPT-4o and Qwen2.5VL, reveals a significant performance gap compared to human capabilities, particularly in tasks that require reordering shuffled sequential images. For instance, GPT-4o achieves only $23.93\%$ accuracy in the reordering subtask, which is $56.07\%$ lower than human performance. Further quantitative analysis discuss several factors, such as input format of images, affecting LMMs’ performance in sequential understanding, underscoring the fundamental challenges that remain in the development of LMMs.   
\end{abstract}

\section{Introduction}

Deep Reinforcement Learning (DRL) has emerged as a transformative paradigm for solving complex sequential decision-making problems. By enabling autonomous agents to interact with an environment, receive feedback in the form of rewards, and iteratively refine their policies, DRL has demonstrated remarkable success across a diverse range of domains including games (\eg Atari~\citep{mnih2013playing,kaiser2020model}, Go~\citep{silver2018general,silver2017mastering}, and StarCraft II~\citep{vinyals2019grandmaster,vinyals2017starcraft}), robotics~\citep{kalashnikov2018scalable}, communication networks~\citep{feriani2021single}, and finance~\citep{liu2024dynamic}. These successes underscore DRL's capability to surpass traditional rule-based systems, particularly in high-dimensional and dynamically evolving environments.

Despite these advances, a fundamental challenge remains: DRL agents typically rely on deep neural networks, which operate as black-box models, obscuring the rationale behind their decision-making processes. This opacity poses significant barriers to adoption in safety-critical and high-stakes applications, where interpretability is crucial for trust, compliance, and debugging. The lack of transparency in DRL can lead to unreliable decision-making, rendering it unsuitable for domains where explainability is a prerequisite, such as healthcare, autonomous driving, and financial risk assessment.

To address these concerns, the field of Explainable Deep Reinforcement Learning (XRL) has emerged, aiming to develop techniques that enhance the interpretability of DRL policies. XRL seeks to provide insights into an agent’s decision-making process, enabling researchers, practitioners, and end-users to understand, validate, and refine learned policies. By facilitating greater transparency, XRL contributes to the development of safer, more robust, and ethically aligned AI systems.

Furthermore, the increasing integration of Reinforcement Learning (RL) with Large Language Models (LLMs) has placed RL at the forefront of natural language processing (NLP) advancements. Methods such as Reinforcement Learning from Human Feedback (RLHF)~\citep{bai2022training,ouyang2022training} have become essential for aligning LLM outputs with human preferences and ethical guidelines. By treating language generation as a sequential decision-making process, RL-based fine-tuning enables LLMs to optimize for attributes such as factual accuracy, coherence, and user satisfaction, surpassing conventional supervised learning techniques. However, the application of RL in LLM alignment further amplifies the explainability challenge, as the complex interactions between RL updates and neural representations remain poorly understood.

This survey provides a systematic review of explainability methods in DRL, with a particular focus on their integration with LLMs and human-in-the-loop systems. We first introduce fundamental RL concepts and highlight key advances in DRL. We then categorize and analyze existing explanation techniques, encompassing feature-level, state-level, dataset-level, and model-level approaches. Additionally, we discuss methods for evaluating XRL techniques, considering both qualitative and quantitative assessment criteria. Finally, we explore real-world applications of XRL, including policy refinement, adversarial attack mitigation, and emerging challenges in ensuring interpretability in modern AI systems. Through this survey, we aim to provide a comprehensive perspective on the current state of XRL and outline future research directions to advance the development of interpretable and trustworthy DRL models.
\section{Related Work}

\subsection{Large Language Models in Biosciences}
Large language models (LLMs) have emerged as powerful tools for natural language comprehension and generation~\cite{llms-survey}. Beyond their application in traditional natural language tasks, there is a growing interest in leveraging LLMs to accelerate scientific research. Early studies revealed that general-purpose LLMs, owing to their rich pre-training data, exhibit promise across various research domains~\cite{ai4science}. Subsequent efforts have focused on directly training LLMs using domain-specific data, aiming to extend the transfer learning paradigm from natural language processing (NLP) to biosciences. This body of work primarily falls into three categories: molecular LLMs, protein LLMs, and genomic LLMs.

For molecular modeling, extensive work has been conducted on training with various molecular string representations, such as SMILES~\cite{Smiles-bert,space-of-chemical,large-scale-chemical}, SELFIES~\cite{SELFIES,chemberta,chemberta2}, and InChI~\cite{inchi}. Additionally, several studies address the modeling of molecular 2D~\cite{mol-2d} and 3D structures~\cite{uni-mol} to capture more detailed molecular characteristics. In the realm of protein LLMs, related work~\cite{msa-transformer,esm2,Prottrans} mainly concentrates on modeling the primary structure of proteins (amino acid sequences), providing a solid foundation for protein structure prediction~\cite{AlphaFold2,AlphaFold3}. For genomic sequences, numerous studies have attempted to leverage the power of LLMs for improved genomic analysis and understanding. These efforts predominantly involve training models on DNA~\cite{BPNet,DNABERT,enformer,nucleotide-transformer,DNABERT-2,GROVER,gena-lm,Caduceus,dnagpt,megaDNA,HyenaDNA,Evo} and RNA~\cite{RNAErnie,uni-rna,Rinalmo} sequences. In the following section, we delve deeper into genomic LLMs specifically designed for DNA sequence modeling.

\subsection{DNA Language Models}
In the early stages, \citeauthor{BPNet} introduced the BPNet convolutional architecture to learn transcription factor binding patterns and their syntax in a supervised manner. Prior to the emergence of large-scale pre-training, BPNet was widely used in genomics for supervised learning on relatively small datasets. With the advent of BERT~\cite{BERT}, DNABERT~\cite{DNABERT} pioneered the application of pre-training on the human genome using K-mer tokenizers. To effectively capture long-range interactions, Enformer~\cite{enformer} advanced human genome modeling by incorporating convolutional downsampling into transformer architectures.

Following these foundational works, numerous models based on the transformer encoder architecture have emerged. A notable example is the Nucleotide Transformer (NT)~\cite{nucleotide-transformer}, which scales model parameters from 100 million to 2.5 billion and includes a diverse set of multispecies genomes. Recent studies, DNABERT-2~\cite{DNABERT-2} and GROVER~\cite{GROVER}, have investigated optimal tokenizer settings for masked language modeling, concluding that Byte Pair Encoding (BPE) is better suited for masked DNA LLMs. The majority of these models face the limitation of insufficient context length, primarily due to the high computational cost associated with extending the context length in the transformer architecture. To address this limitation, GENA-LM~\cite{gena-lm} employs sparse attention, and Caduceus~\cite{Caduceus} uses the more lightweight BiMamba architecture~\cite{Mamba}, both trained on the human genome.

Although these masked DNA LLMs effectively understand and predict DNA sequences, they lack generative capabilities, and generative DNA LLMs remain in the early stages of development. An early preprint~\cite{dnagpt} introduced DNAGPT, which learns mammalian genomic structures through three pre-training tasks, including next token prediction. Recent works, such as HyenaDNA~\cite{HyenaDNA} and megaDNA~\cite{megaDNA}, achieve longer context lengths by employing the Hyena~\cite{Hyena} and multiscale transformer architectures respectively, though they are significantly limited by their data and model scales. A more recent influential study, Evo~\cite{Evo}, trained on an extensive dataset of prokaryotic and viral genomes, has garnered widespread attention for its success in designing CRISPR-Cas molecular complexes, thus demonstrating the practical utility of generative DNA LLMs in the genomic field.

\section{Dataset and Task Overview}
To investigate the capabilities of LMMs to comprehend sequential images, we introduce \dataset, a novel benchmark consisting of three subtasks:

\begin{itemize}
     \item \textbf{Visual Narrative Comprehension:} Examines whether models accurately interpret the narrative content of image sequences.
      \item \textbf{Contextual Frame Prediction:} Assesses the model reasoning ability to predict missing frames in image sequences based contextual.
      \item \textbf{Temporal narrative Reordering:} Evaluates whether models correctly infer and restore the chronological order of image sequences based causal temporal relationship.
\end{itemize}

The instructions for three subtasks are presented in Table~\ref{prompt}. These subtasks provide a rigorous and multifaceted assessment of LMMs, offering insights into their strengths and limitations in sequential image understanding. In the following, we may use their full names or refer to them as \textit{comprehension}, \textit{prediction}, and \textit{reordering} for simplicity.


\definecolor{titlecolor}{rgb}{0.9, 0.5, 0.1}
\definecolor{anscolor}{rgb}{0.2, 0.5, 0.8}
\definecolor{labelcolor}{HTML}{48a07e}
\begin{table*}[h]
	\centering
	
 % \vspace{-0.2cm}
	
	\begin{center}
		\begin{tikzpicture}[
				chatbox_inner/.style={rectangle, rounded corners, opacity=0, text opacity=1, font=\sffamily\scriptsize, text width=5in, text height=9pt, inner xsep=6pt, inner ysep=6pt},
				chatbox_prompt_inner/.style={chatbox_inner, align=flush left, xshift=0pt, text height=11pt},
				chatbox_user_inner/.style={chatbox_inner, align=flush left, xshift=0pt},
				chatbox_gpt_inner/.style={chatbox_inner, align=flush left, xshift=0pt},
				chatbox/.style={chatbox_inner, draw=black!25, fill=gray!7, opacity=1, text opacity=0},
				chatbox_prompt/.style={chatbox, align=flush left, fill=gray!1.5, draw=black!30, text height=10pt},
				chatbox_user/.style={chatbox, align=flush left},
				chatbox_gpt/.style={chatbox, align=flush left},
				chatbox2/.style={chatbox_gpt, fill=green!25},
				chatbox3/.style={chatbox_gpt, fill=red!20, draw=black!20},
				chatbox4/.style={chatbox_gpt, fill=yellow!30},
				labelbox/.style={rectangle, rounded corners, draw=black!50, font=\sffamily\scriptsize\bfseries, fill=gray!5, inner sep=3pt},
			]
											
			\node[chatbox_user] (q1) {
				\textbf{System prompt}
				\newline
				\newline
				You are a helpful and precise assistant for segmenting and labeling sentences. We would like to request your help on curating a dataset for entity-level hallucination detection.
				\newline \newline
                We will give you a machine generated biography and a list of checked facts about the biography. Each fact consists of a sentence and a label (True/False). Please do the following process. First, breaking down the biography into words. Second, by referring to the provided list of facts, merging some broken down words in the previous step to form meaningful entities. For example, ``strategic thinking'' should be one entity instead of two. Third, according to the labels in the list of facts, labeling each entity as True or False. Specifically, for facts that share a similar sentence structure (\eg, \textit{``He was born on Mach 9, 1941.''} (\texttt{True}) and \textit{``He was born in Ramos Mejia.''} (\texttt{False})), please first assign labels to entities that differ across atomic facts. For example, first labeling ``Mach 9, 1941'' (\texttt{True}) and ``Ramos Mejia'' (\texttt{False}) in the above case. For those entities that are the same across atomic facts (\eg, ``was born'') or are neutral (\eg, ``he,'' ``in,'' and ``on''), please label them as \texttt{True}. For the cases that there is no atomic fact that shares the same sentence structure, please identify the most informative entities in the sentence and label them with the same label as the atomic fact while treating the rest of the entities as \texttt{True}. In the end, output the entities and labels in the following format:
                \begin{itemize}[nosep]
                    \item Entity 1 (Label 1)
                    \item Entity 2 (Label 2)
                    \item ...
                    \item Entity N (Label N)
                \end{itemize}
                % \newline \newline
                Here are two examples:
                \newline\newline
                \textbf{[Example 1]}
                \newline
                [The start of the biography]
                \newline
                \textcolor{titlecolor}{Marianne McAndrew is an American actress and singer, born on November 21, 1942, in Cleveland, Ohio. She began her acting career in the late 1960s, appearing in various television shows and films.}
                \newline
                [The end of the biography]
                \newline \newline
                [The start of the list of checked facts]
                \newline
                \textcolor{anscolor}{[Marianne McAndrew is an American. (False); Marianne McAndrew is an actress. (True); Marianne McAndrew is a singer. (False); Marianne McAndrew was born on November 21, 1942. (False); Marianne McAndrew was born in Cleveland, Ohio. (False); She began her acting career in the late 1960s. (True); She has appeared in various television shows. (True); She has appeared in various films. (True)]}
                \newline
                [The end of the list of checked facts]
                \newline \newline
                [The start of the ideal output]
                \newline
                \textcolor{labelcolor}{[Marianne McAndrew (True); is (True); an (True); American (False); actress (True); and (True); singer (False); , (True); born (True); on (True); November 21, 1942 (False); , (True); in (True); Cleveland, Ohio (False); . (True); She (True); began (True); her (True); acting career (True); in (True); the late 1960s (True); , (True); appearing (True); in (True); various (True); television shows (True); and (True); films (True); . (True)]}
                \newline
                [The end of the ideal output]
				\newline \newline
                \textbf{[Example 2]}
                \newline
                [The start of the biography]
                \newline
                \textcolor{titlecolor}{Doug Sheehan is an American actor who was born on April 27, 1949, in Santa Monica, California. He is best known for his roles in soap operas, including his portrayal of Joe Kelly on ``General Hospital'' and Ben Gibson on ``Knots Landing.''}
                \newline
                [The end of the biography]
                \newline \newline
                [The start of the list of checked facts]
                \newline
                \textcolor{anscolor}{[Doug Sheehan is an American. (True); Doug Sheehan is an actor. (True); Doug Sheehan was born on April 27, 1949. (True); Doug Sheehan was born in Santa Monica, California. (False); He is best known for his roles in soap operas. (True); He portrayed Joe Kelly. (True); Joe Kelly was in General Hospital. (True); General Hospital is a soap opera. (True); He portrayed Ben Gibson. (True); Ben Gibson was in Knots Landing. (True); Knots Landing is a soap opera. (True)]}
                \newline
                [The end of the list of checked facts]
                \newline \newline
                [The start of the ideal output]
                \newline
                \textcolor{labelcolor}{[Doug Sheehan (True); is (True); an (True); American (True); actor (True); who (True); was born (True); on (True); April 27, 1949 (True); in (True); Santa Monica, California (False); . (True); He (True); is (True); best known (True); for (True); his roles in soap operas (True); , (True); including (True); in (True); his portrayal (True); of (True); Joe Kelly (True); on (True); ``General Hospital'' (True); and (True); Ben Gibson (True); on (True); ``Knots Landing.'' (True)]}
                \newline
                [The end of the ideal output]
				\newline \newline
				\textbf{User prompt}
				\newline
				\newline
				[The start of the biography]
				\newline
				\textcolor{magenta}{\texttt{\{BIOGRAPHY\}}}
				\newline
				[The ebd of the biography]
				\newline \newline
				[The start of the list of checked facts]
				\newline
				\textcolor{magenta}{\texttt{\{LIST OF CHECKED FACTS\}}}
				\newline
				[The end of the list of checked facts]
			};
			\node[chatbox_user_inner] (q1_text) at (q1) {
				\textbf{System prompt}
				\newline
				\newline
				You are a helpful and precise assistant for segmenting and labeling sentences. We would like to request your help on curating a dataset for entity-level hallucination detection.
				\newline \newline
                We will give you a machine generated biography and a list of checked facts about the biography. Each fact consists of a sentence and a label (True/False). Please do the following process. First, breaking down the biography into words. Second, by referring to the provided list of facts, merging some broken down words in the previous step to form meaningful entities. For example, ``strategic thinking'' should be one entity instead of two. Third, according to the labels in the list of facts, labeling each entity as True or False. Specifically, for facts that share a similar sentence structure (\eg, \textit{``He was born on Mach 9, 1941.''} (\texttt{True}) and \textit{``He was born in Ramos Mejia.''} (\texttt{False})), please first assign labels to entities that differ across atomic facts. For example, first labeling ``Mach 9, 1941'' (\texttt{True}) and ``Ramos Mejia'' (\texttt{False}) in the above case. For those entities that are the same across atomic facts (\eg, ``was born'') or are neutral (\eg, ``he,'' ``in,'' and ``on''), please label them as \texttt{True}. For the cases that there is no atomic fact that shares the same sentence structure, please identify the most informative entities in the sentence and label them with the same label as the atomic fact while treating the rest of the entities as \texttt{True}. In the end, output the entities and labels in the following format:
                \begin{itemize}[nosep]
                    \item Entity 1 (Label 1)
                    \item Entity 2 (Label 2)
                    \item ...
                    \item Entity N (Label N)
                \end{itemize}
                % \newline \newline
                Here are two examples:
                \newline\newline
                \textbf{[Example 1]}
                \newline
                [The start of the biography]
                \newline
                \textcolor{titlecolor}{Marianne McAndrew is an American actress and singer, born on November 21, 1942, in Cleveland, Ohio. She began her acting career in the late 1960s, appearing in various television shows and films.}
                \newline
                [The end of the biography]
                \newline \newline
                [The start of the list of checked facts]
                \newline
                \textcolor{anscolor}{[Marianne McAndrew is an American. (False); Marianne McAndrew is an actress. (True); Marianne McAndrew is a singer. (False); Marianne McAndrew was born on November 21, 1942. (False); Marianne McAndrew was born in Cleveland, Ohio. (False); She began her acting career in the late 1960s. (True); She has appeared in various television shows. (True); She has appeared in various films. (True)]}
                \newline
                [The end of the list of checked facts]
                \newline \newline
                [The start of the ideal output]
                \newline
                \textcolor{labelcolor}{[Marianne McAndrew (True); is (True); an (True); American (False); actress (True); and (True); singer (False); , (True); born (True); on (True); November 21, 1942 (False); , (True); in (True); Cleveland, Ohio (False); . (True); She (True); began (True); her (True); acting career (True); in (True); the late 1960s (True); , (True); appearing (True); in (True); various (True); television shows (True); and (True); films (True); . (True)]}
                \newline
                [The end of the ideal output]
				\newline \newline
                \textbf{[Example 2]}
                \newline
                [The start of the biography]
                \newline
                \textcolor{titlecolor}{Doug Sheehan is an American actor who was born on April 27, 1949, in Santa Monica, California. He is best known for his roles in soap operas, including his portrayal of Joe Kelly on ``General Hospital'' and Ben Gibson on ``Knots Landing.''}
                \newline
                [The end of the biography]
                \newline \newline
                [The start of the list of checked facts]
                \newline
                \textcolor{anscolor}{[Doug Sheehan is an American. (True); Doug Sheehan is an actor. (True); Doug Sheehan was born on April 27, 1949. (True); Doug Sheehan was born in Santa Monica, California. (False); He is best known for his roles in soap operas. (True); He portrayed Joe Kelly. (True); Joe Kelly was in General Hospital. (True); General Hospital is a soap opera. (True); He portrayed Ben Gibson. (True); Ben Gibson was in Knots Landing. (True); Knots Landing is a soap opera. (True)]}
                \newline
                [The end of the list of checked facts]
                \newline \newline
                [The start of the ideal output]
                \newline
                \textcolor{labelcolor}{[Doug Sheehan (True); is (True); an (True); American (True); actor (True); who (True); was born (True); on (True); April 27, 1949 (True); in (True); Santa Monica, California (False); . (True); He (True); is (True); best known (True); for (True); his roles in soap operas (True); , (True); including (True); in (True); his portrayal (True); of (True); Joe Kelly (True); on (True); ``General Hospital'' (True); and (True); Ben Gibson (True); on (True); ``Knots Landing.'' (True)]}
                \newline
                [The end of the ideal output]
				\newline \newline
				\textbf{User prompt}
				\newline
				\newline
				[The start of the biography]
				\newline
				\textcolor{magenta}{\texttt{\{BIOGRAPHY\}}}
				\newline
				[The ebd of the biography]
				\newline \newline
				[The start of the list of checked facts]
				\newline
				\textcolor{magenta}{\texttt{\{LIST OF CHECKED FACTS\}}}
				\newline
				[The end of the list of checked facts]
			};
		\end{tikzpicture}
        \caption{GPT-4o prompt for labeling hallucinated entities.}\label{tb:gpt-4-prompt}
	\end{center}
\vspace{-0cm}
\end{table*}
% \vspace{-5mm}
% \begin{figure*}[htbp]
%     % 左侧图片
%     \begin{minipage}{0.77\linewidth}  % 调整宽度
%         \centering
%         \includegraphics[width=\linewidth]{images/benchmark_construction.pdf}
%     \end{minipage}%
%     % 间隔
%     \hfill
%     % 右侧表格
%     \begin{minipage}{0.23\linewidth}  % 调整宽度
%         \centering
%         \resizebox{\linewidth}{!}{  % 调整表格至合适的宽度
%             \begin{tabular}{lcc}
%                 \toprule
%                 \textbf{Statistic} & \textbf{Number} \\
%                 \midrule
%                 \rowcolor[HTML]{F2F2F2} 
%                 \textit{Domain Count} &  \\
%                 \midrule
%                 Domain & 103 \\
%                 Requirement & 8 \\
%                 \midrule
%                 \rowcolor[HTML]{F2F2F2} 
%                 \textit{Token Count} &  \\
%                 \midrule
%                 Description & 851.6 $\pm$ 515.2 \\
%                 - Min/Max & [159, 2814] \\
%                 Domain & 1187.2 $\pm$ 1212.1 \\
%                 - Min/Max & [85, 7514] \\
%                 \midrule
%                 \rowcolor[HTML]{F2F2F2} 
%                 \textit{Line Count} &  \\
%                 \midrule
%                 Domain & 75.4 $\pm$ 62.9 \\
%                 - Min/Max & [9, 394] \\
%                 \midrule
%                 \rowcolor[HTML]{F2F2F2} 
%                 \textit{Component Count} &  \\
%                 \midrule
%                 Actions & 4.5 $\pm$ 2.8 \\
%                 - Min/Max & [1, 16] \\
%                 Predicates & 8.1 $\pm$ 4.8 \\
%                 - Min/Max & [1, 25] \\
%                 Types & 1.1 $\pm$ 1.3 \\
%                 - Min/Max & [1, 8] \\
%                 \bottomrule
%             \end{tabular}
%         }
%     \end{minipage}
%     % 公共标题
%     \caption{Dataset construction process (left) and key statistics (right) of the \texttt{\benchmark} dataset.     Dataset construction process including: (a) \textit{Data Acquisition} (\S\ref{sec:data_acquisition}); (b) \textit{Data Filtering and Manual Selection} (\S\ref{sec:data_filtering}); (c) \textit{Data Annotation and Quality Assurance}(\S\ref{sec:data_annotation} and \S\ref{sec:quality_assurance}). Tokens are counted by GPT-2~\cite{openai2019gpt2} tokenizer.}
%     \label{fig:combined}
% \end{figure*}


Detailed statistics is displayed on Table~\ref{tab:statistics}.
Overall, our proposed \dataset includes $896$ image sequences, with an average frame length of $4.09$ of each sequence. The number of frames ranges from 3 to 8. Each task is designed in the form of multiple-choice questions, except for the reorder task, which also includes a question-answering format. Since its options are simple and well-defined, accuracy can be directly computed. In our tasks, the input format uniformly consists of images paired with textual prompts. Specifically, the comprehension task utilizes the whole images without split, the reordering task takes shuffled image sequence as input,  and the frame prediction task involves masking second-to-last frame within the image sequence.
For the frame prediction task, we select the second-to-last frame, as it typically serves as a bridge between the preceding frames and the final frame. The start and end frames are generally more challenging to predict.


% 上面写具体的statistics,比如task A每个sample几个选项,task B & task C,平均的选项长度是多少。每个task的数据格式是怎么样的,比如reorder input就是打乱的图片,comprehension是顺序图片,frmae prediction我们会mask某一帧的图片。可以在这里讲我们是mask倒数第二帧以及为什么。

% 最后一帧一般是脑洞大开的反转,很难预测,所以我们选择了他的前一帧进行预测,并且只筛选了frmae数大于等于4的连环画。但是有些倒数第二帧的内容是有多种可能的,我们在人类标注员check的时候删除了那些倒数第二帧不唯一的图。

% Reference:

%The \method{} dataset includes 1,001 samples, each with an image and three manually annotated components: a description, a title, and deep semantics. The statistical information about the text is displayed in Table \ref{tab:statistics}. To enable quantitative evaluation, we additionally craft multiple-choice questions to test the understanding of descriptions, titles, and deep semantics. Each segment is represented by 1,001 questions, where each question presents an image, a question text, and four potential answers. Only one answer is correct, while the others serve as distractors. Figure~\ref{fig:fig_sim} illustrates examples of the manually annotated components and the multiple-choice questions.

% We describe the proposed PMR task with an example in Figure 2 I&II. Given a source image and a textual premise, the inference model should perceive and understand the image in combination with the premise so as to choose the exclusive correct action among the four hypothetical candidates. The premise would serve as the background knowledge or domain-specific commonsense for the given image. In the running example, the model should be able to recognize what [person2] wears from the image and infer whether [person4] would give his seat to [person2] under the premise “[person4] is very friendly”. The corrected answer is ‘C’ according to the visual and textual clues. In total, we collect about 15k instances for PMR. We list the statistics for PMR in Table 2.
\section{Dataset Construction}



% Image Source
% Prerequisite: Annotator Training
% (1) AI-assisted Data Annotation for Answers & Distractor Generation
% (2) Cross-Check Examination
% (3) Subtask Composition
% (4) Dataset Evaluation

We construct our \dataset dataset in a multi-step crowd-sourcing pipeline, including 1) annotator training, 2) data annotation, and 3) cross-check examination. An overall demonstration of our dataset construction pipeline is illustrated in Figure ~\ref{fig:construct}. 

\subsection{Image Source}

We use silent comic strips, comic with panels and no dialogue, as our primary data source. As a distinct art form, comic strips often encapsulate complex narratives within concise visual sequences, addressing deeper themes such as social satire, humor, and inspiration. These characteristics make comic strips a particularly challenging medium for evaluating ability of LMMs to understand visual sequences. 
The dataset comprises samples from well-known comics, such as \textit{Father and Son} and \textit{Peanuts}, along with web-scraped images from \texttt{GoComics} \footnote{https://www.gocomics.com/}, \texttt{Google}, and \texttt{Facebook} \footnote{https://www.facebook.com/}. 
Initially, we collected $1,260$ images and then refined the dataset through a filtering process. We conducted a thorough manual inspection to eliminate unclear, toxic, overly simplistic images, along withmulti-panel comics lacking a clear temporal sequence. Moreover, comics with dialogue will also be removed to prevent the model from using OCR to understand the meaning of the comics through text rather than through images. As a result, the final dataset was reduced to 896 images.

% \subsection{Prerequisite: Annotator Training}
% We posted job descriptions on online forums and received over 50 applications from candidates with at least a Bachelor's degree. To ensure dataset quality, we provided training sessions that included online pre-annotation instructions and a qualification test to assess candidates' performance. Only those scoring above 95\% were selected. candidates were assigned to one of two groups: annotators or inspectors. Ultimately, we hired 13 annotators and 7 inspectors for our data annotation process.

% During the data collection process, we primarily rely on the powerful GPT-4o model to abtain annotation, combined with multiple rounds of human review.

\subsection{Phase 1: Data Annotation.}

\paragraph{Annotator Training}
We posted job descriptions on online forums and received over 50 applications from candidates with at least a Bachelor's degree. To ensure dataset quality, we provided training sessions that included online pre-annotation instructions and a qualification test to assess candidates' performance. Only those scoring above $95\%$ were selected. candidates were assigned to one of two groups: annotators or inspectors. Ultimately, we hired $13$ annotators and $7$ inspectors for our data annotation process.
To optimize efficiency and reduce costs, we implement a semi-automated pipeline for \dataset annotation, leveraging \texttt{GPT-4o}\footnote{We use the \texttt{gpt-4o-2024-11-20} version for the data annotation process and subsequent evaluations in this work.}. Specifically, our data annotation process consists of two substeps: \textit{answer creation} and \textit{distractor generation}.

\paragraph{Answer Creation.}
The bottom panel of Figure~\ref{fig:construct} illustrates the process of our answer creation phase. Notably, only the comprehension task and frame prediction task need option annotation. The reordering task only necessitates using a program to randomly shuffle the frame order as the answer order, without the need for manual selection of the correct answer. We adopt an AI-assisted annotation approach in which human annotators refine pre-generated answers instead of creating them from scratch. Initially, we leverage \texttt{GPT-4o, GPT-4o-Mini} \cite{hurst2024gpt40} and Gemini \cite{reid2024gemini} to generate diverse candidate answers. Human annotators then evaluate the image sequence and candidate answers, selecting the most appropriate ground truth. If none of the candidate answers is suitable, annotators are instructed to either refine a specific answer or create a new ground truth from scratch, which is about $28\%$. 

\paragraph{Distractor Generation.}
We use candidates with plausible hallucinations from the previous sub-step as strong distractors.  To ensure diversity in the multiple-choice options, we also prompt \texttt{GPT-4o, GPT-4o-mini} \cite{hurst2024gpt40} to generate intentionally incorrect responses as weak distractors. 
Typically, the responses of \texttt{GPT-4o-mini}  are not very accurate and are mostly used as distractors. A detailed example of model annotation can be found in the appendix~\ref{appendix:annotation}
Annotators are instructed to evaluate the quality of these distractors and select top-3 options, refining them if necessary. Finally, each ground truth is paired with three high-quality distractors for evaluation.

%We carefully curated a new set of challenging distractors that:;Maintain semantic relevance to the comic content;Share similar visual elements or themes with the correct answer;Avoid obvious logical contradictions;Represent plausible but incorrect interpretations

% To ensure annotation quality, annotators are provided with detailed examples and rejection guidelines. 

% The annotation interface and full specifications are provided in Appendix{ \color{red}X[TODO].}

% 对于理解任务,我们在采样4o的回答作为标注的时候发现模型经常出现过度解读升华漫画的情况。所以一些非常简单无深意的图片会被删除,此外人类标注员在check的时候会删除过度解读。

% The annotation process concluded with two rounds of cross-verification to ensure answer accuracy, maintain stylistic consistency, and minimize potential biases.


\subsection{Phase 2: Cross-Check Examination}
We implement a cross-check examination mechanism to ensure rigorous screening of high-quality annotations. During the data annotation process, hired inspectors review the annotated data and corresponding image sequences. If they encounter low-quality annotations, they have the option to reject them. Each annotation is reviewed by two inspectors. If both inspectors reject the annotation, it is discarded, and the image is returned to the dataset for re-annotation. If an image sequence is rejected in two rounds of annotation, it suggests that this sample is not suitable for the current subtask (e.g., the meaning of the sample is unclear), and the image is subsequently removed from the subtask. 

% During this phase, we also use Cohen's kappa to quantify inter-annotator agreement, obtaining an average score of { \color{red}0.701} across all tasks, indicating substantial agreement.

After annotation, both advanced annotators and inspectors, acting as final examiners, review the annotations to ensure they meet the required standards. Each annotation undergoes review by three examiners, who vote on whether to accept the annotated sample. Only the samples that receive a majority vote are approved. To ensure the quality of the examiners' work, we randomly sample 10\% of the annotations for verification. 

% This review is conducted by the authors of this paper, who provide timely feedback to the examiners. Further details on the pricing strategy and the hierarchical supervision process are provided in Appendix B.



\subsection{Data Composition}
It is important to note that the reordering subtask does not require human annotation, as described in the previous process. For this subtask, we select suitable image sequences based on the criterion that the correct ordering must be unique. To ensure this, we conduct a manual review to verify that each sequence follows a logically unambiguous order. A script is then run to perform the initial splitting of panels within specific comics, followed by a random shuffling of these panels. Human annotators are tasked with verifying the format and quality of the frames to ensure they meet the required standards. These processed image sequences serve as the evaluation data for the reordering subtask.

The final version of our 32-day annotated \dataset contains 896 items (see Table 2), encompassing three subtasks: \textit{visual narrative comprehension}, \textit{contextual frame prediction}, and \textit{temporal narrative reordering}. In each of these subtasks, each sample consists of an image sequence paired with a multiple-choice question offering four options. The evaluated LMMs are required to select the option they deem most appropriate from the four. More information and examples of \dataset can be found in Appendix~\ref{example}. 

% Please add the following required packages to your document preamble:

% Beamer presentation requires \usepackage{colortbl} instead of \usepackage[table,xcdraw]{xcolor}
\begin{table*}[t]
\centering
\caption{Main Results. Eurus-2-7B-PRIME demonstrates the best reasoning ability.}
\label{tab:main_results}
\resizebox{\textwidth}{!}{
\begin{tabular}{lcccccc}
\toprule
\textbf{Model}                     & \textbf{AIME 2024}                           & \textbf{MATH-500} & \textbf{AMC}          & \textbf{Minerva Math} & \textbf{OlympiadBench} & \textbf{Avg.}          \\ \midrule
\textbf{GPT-4o}                    & 9.3                                          & 76.4              & 45.8                  & 36.8                  & \textbf{43.3}          & 43.3                   \\
\textbf{Llama-3.1-70B-Instruct}    & 16.7                                         & 64.6              & 30.1                  & 35.3                  & 31.9                   & 35.7                   \\
\textbf{Qwen-2.5-Math-7B-Instruct} & 13.3                                         & \textbf{79.8}     & 50.6                  & 34.6                  & 40.7                   & 43.8                   \\
\textbf{Eurus-2-7B-SFT}            & 3.3                                          & 65.1              & 30.1                  & 32.7                  & 29.8                   & 32.2                   \\
\textbf{Eurus-2-7B-PRIME}          & \textbf{26.7 {\color[HTML]{009901} (+23.3)}} & 79.2 {\color[HTML]{009901}(+14.1)}      & \textbf{57.8 {\color[HTML]{009901}(+27.7)}} & \textbf{38.6 {\color[HTML]{009901}(+5.9)}}  & 42.1 {\color[HTML]{009901}(+12.3) }          & \textbf{48.9 {\color[HTML]{009901}(+ 16.7)}} \\ \bottomrule
\end{tabular}
}
\end{table*}


\section{Experiments}
\label{sec: experiments}

\begin{table*}
        % \centering
        \caption{A comparison between our proposed method with other advanced methods on the nuScenes test set.
        % This leaderboard is available at \href{https://www.nuscenes.org/tracking?externalData=all&mapData=all&modalities=Any}{nuScenes official benchmark}.
        $\ddagger$ means the GPU device.
        \textcolor{black}{The reported runtimes of all methods exclude the detection time.}
        Poly-MOT~\cite{li2023poly}, Fast-Poly~\cite{li2024fast} and Easy-Poly rely entirely on the detector input, as they do not utilize any visual or deep features \textcolor{black}{during tracking}.}
        \label{table:nu_test}
        % \renewcommand{\arraystretch}{0.7}
        \setlength{\tabcolsep}{1.6mm}
        {
        \begin{tabular}{cccc|ccc|ccc}
        \toprule
        \multicolumn{1}{c}{\textbf{Method}} & \textbf{Device} & \textbf{Detector} & \textbf{Input} & \textbf{AMOTA}$\uparrow$ & \textbf{MOTA}$\uparrow$ & \textbf{FPS}$\uparrow$ & \textbf{IDS}$\downarrow$ & \textbf{FN}$\downarrow$ & \textbf{FP}$\downarrow$ \\ \midrule
                                     EagerMOT~\cite{kim2021eagermot}               & \textbf{\text{--}}       & CenterPoint~\cite{yin2021center}\&Cascade R-CNN~\cite{cai2018cascade}         & 2D+3D       & 67.7      & 56.8     & 4    & 1156    & 24925   & 17705   \\
                                   CBMOT~\cite{benbarka2021score}     & I7-9700          & CenterPoint~\cite{yin2021center}\&CenterTrack~\cite{zhou2020tracking}          & 2D+3D         & 67.6      & 53.9      & \textcolor{red}{80.5}    & 709    & 22828   & 21604   \\
                                   % ShaSTA~\cite{sadjadpour2023shasta} & A100$\ddagger$       & CenterPoint~\cite{yin2021center}         & 3D      & 69.6      & 57.8     & 10     & 473    & 21293   & 16746   \\
                                   Minkowski~\cite{gwak2022minkowski} & TITAN$\ddagger$  & Minkowski~\cite{gwak2022minkowski}         & 3D      & 69.8      & 57.8     & 3.5    & 325    & 21200   & 19340   \\
                                   ByteTrackv2~\cite{zhang2023bytetrackv2} & \textbf{\text{--}}       & TransFusion-L~\cite{bai2022transfusion}         & 3D      & 70.1      & 58     & \textbf{\text{--}}    & 488    & 21836   & 18682   \\
                                   3DMOTFormer~\cite{ding20233dmotformer}& 2080Ti$\ddagger$       & BEVFuison~\cite{liu2023bevfusion}       & 2D+3D      & 72.5      & 60.9     & \textcolor{blue}{54.7}    & 593    & 20996   & \textcolor{blue}{17530}   \\  
                                   %CAMO-MOT~\cite{li2023camo}& 3090Ti$\ddagger$   & BEVFuison~\cite{liu2023bevfusion}\&FocalsConv~\cite{chen2022focal}   & 2D+3D      & 75.3      & \textbf{63.5}     & \textbf{\text{--}}    & 324    & 18192   & 17269   \\
                                   Poly-MOT~\cite{li2023poly}               & 9940X       & LargeKernel3D~\cite{chen2022scaling}         &2D+3D       & \textcolor{blue}{75.4}      & \textcolor{blue}{62.1}     & 3    & \textcolor{blue}{292}    & \textcolor{blue}{17956}   & 19673   \\ 
                                  Fast-Poly~\cite{li2024fast}               & 7945HX       & LargeKernel3D~\cite{chen2022scaling}         &2D+3D       & \textcolor{blue}{75.8}      & \textcolor{blue}{62.8}     & 34.2    & \textcolor{blue}{326}    & \textcolor{blue}{18415}   &  \textcolor{blue}{17098}  \\ \midrule

                                  \textbf{Easy-Poly (Ours)}               & 4090Ti$\ddagger$       & LargeKernel3D~\cite{chen2022scaling}         &2D+3D       &  \textcolor{red}{75.9}      & \textcolor{red}{63.0}     & \textcolor{blue}{34.9}    & \textcolor{red}{287}    & \textcolor{red}{17620}   &   \textcolor{red}{16718}  \\
        \bottomrule
        \end{tabular}}
        % \vspace{-1.5em}
 \end{table*}


\begin{table*}
\vspace{0.5em}
\begin{center}
\caption{
{A comparison of existing methods applied to the nuScenes val set.}}
\label{table:nu_val}
% \renewcommand{\arraystretch}{0.7}
\setlength{\tabcolsep}{2.4mm}
{
\begin{tabular}{cccccccc}
\toprule
\bf{Method} & \bf{Detector} & \bf{Input Data} & \bf{MOTA$\uparrow$} & \bf{AMOTA$\uparrow$} & \bf{AMOTP$\downarrow$} & \bf{FPS$\uparrow$} & \bf{IDS$\downarrow$}  \\ \hline
CBMOT~\cite{benbarka2021score}   & CenterPoint~\cite{yin2021center} \& CenterTrack~\cite{zhou2020tracking} & 2D + 3D & -- & 72.0 & \textbf{\textcolor{red}{48.7}}  & -- & 479   \\
EagerMOT~\cite{kim2021eagermot}  & CenterPoint~\cite{yin2021center} \& Cascade R-CNN~\cite{cai2018cascade} & 2D + 3D  & -- & 71.2   & 56.9 & 13  & 899    \\
SimpleTrack~\cite{pang2022simpletrack}  & CenterPoint~\cite{yin2021center} & 3D & 60.2 & 69.6  & 54.7 & 0.5  & 405  \\
CenterPoint~\cite{yin2021center}   & CenterPoint~\cite{yin2021center} & 3D & -- & 66.5  & 56.7 & -- & 562 \\ 
OGR3MOT~\cite{zaech2022learnable}  & CenterPoint~\cite{yin2021center} &3D & 60.2 & 69.3  & 62.7 & 12.3 & \textbf{\textcolor{blue}{262}}  \\ \hline

\textbf{Poly-MOT}~\cite{li2023poly}      & CenterPoint~\cite{yin2021center} & 3D & 61.9 & 73.1   & \textbf{\textcolor{blue}{52.1}} & 5.6 & 281   \\ 
% \textbf{Poly-MOT}~\cite{li2023poly}      & LargeKernel3D-L~\cite{chen2022scaling}  & 3D   & \textbf{\textcolor{red}{75.2}}    & 54.1   & \textbf{\textcolor{red}{252}} \\ 

\textbf{Poly-MOT}~\cite{li2023poly}      & LargeKernel3D-L~\cite{chen2022scaling}  & 3D & 54.1 & \textbf{\textcolor{red}{75.2}}    & 54.1  & 8.6 & 252 \\

\textbf{Fast-Poly}~\cite{li2024fast}      & CenterPoint~\cite{yin2021center} & 3D & \textbf{\textcolor{blue}{63.2}}  & 73.7   &  -- & \textbf{\textcolor{blue}{28.9}} & 414   \\ \hline
 
\textbf{Easy-Poly (Ours)}       & CenterPoint~\cite{yin2021center} & 2D + 3D & \textbf{\textcolor{blue}{64.4}} & \textbf{\textcolor{blue}{74.5}}   & 54.9  & \textbf{\textcolor{blue}{34.6}} & \textbf{\textcolor{blue}{272}}   \\
\textbf{Easy-Poly (Ours)}       & LargeKernel3D~\cite{chen2022scaling}  & 2D + 3D  & \textbf{\textcolor{red}{64.8}} & \textbf{\textcolor{blue}{75.0}}    & \textbf{\textcolor{blue}{53.6}}  & \textbf{\textcolor{red}{34.9}}  & \textbf{\textcolor{red}{242}} \\ \hline
% \vspace{-4.5em}
% \setlength{\abovecaptionskip}{3pt}
% \setlength{\belowcaptionskip}{3pt}
\end{tabular}}
\end{center}
\end{table*}

\subsection{Datasets}

% The nuScenes dataset~\cite{caesar2020nuscenes} consists of 850 training and 150 test sequences, capturing a wide range of driving scenarios, including challenging weather conditions and nighttime environments. Each sequence contains approximately 40 frames, with keyframes sampled at 2Hz and fully annotated. In addition to this, it provides annotations for object-level attributes such as visibility, activity, pose, and more. It includes a large volume of RGB and point-cloud data (in PCD format). The official evaluation protocol utilizes \textbf{AMOTA}, \textbf{MOTA}, and \textbf{sAMOTA}~\cite{weng20203d} as primary metrics, evaluating performance across seven object categories: Car (\textit{Car}), Bicycle (\textit{Bic}), Motorcycle (\textit{Moto}), Pedestrian (\textit{Ped}), Bus (\textit{Bus}), Trailer (\textit{Tra}), and Truck (\textit{Tru}). Given the substantial size of the complete nuScenes dataset, users often prefer the nuScenes-mini dataset. Notably, Poly-MOT, Fast-Poly, and our proposed Easy-Poly methods exclusively utilize keyframes for tracking tasks.


The nuScenes dataset~\cite{caesar2020nuscenes} consists of 850 training and 150 test sequences, capturing a wide range of driving scenarios, including challenging weather conditions and nighttime environments. Each sequence contains approximately 40 frames, with keyframes sampled at 2Hz and fully annotated. The official evaluation protocol utilizes \textbf{AMOTA}, \textbf{MOTA}, and \textbf{sAMOTA}~\cite{weng20203d} as primary metrics, evaluating performance across seven object categories: Car (\textit{Car}), Bicycle (\textit{Bic}), Motorcycle (\textit{Moto}), Pedestrian (\textit{Ped}), Bus (\textit{Bus}), Trailer (\textit{Tra}), and Truck (\textit{Tru}). Notably, Poly-MOT, Fast-Poly, and our proposed Easy-Poly methods exclusively utilize keyframes for tracking tasks.

\subsection{Implementation Details}

% Our tracking method is implemented in Python under the Nvidia 4090X GPU.  Hyperparameters are chosen based on the best AMOTA identified in the validation set. SF thresholds are category-specific and detector-specific, which are (\textit{Bic}: 0.15; are (\textit{Car}: 0.16; are (\textit{Moto}: 0.16; \textit{Bus}: 0.12; \textit{Tra}: 0.13; \textit{Tru}: 0; \textit{Ped}: 0.13) on nuScenes on Waymo. The NMS thresholds are 0.08 on all categories and datasets. We also employ Scale-NMS~\cite{huang2021bevdet} on (\textit{Bic}, \textit{Ped}) on nuScenes. With default $IoU_{bev}$ in NMS, we additionally utilize our proposed $A\text{-}gIoU_{bev}$ to describe similarity for (\textit{Bic}, \textit{Ped}, \textit{Bus}, \textit{Tru}) on nuScenes. The motion models and filters are consistent with~\cite{li2023poly}. The lightweight filter is implemented by the median filter with $l_{lw}=5$ on all datasets. The association metrics are all implemented by $A\text{-}gIoU$ on all datasets. The first association thresholds $\theta_{fm}$ are category-specific, which are (\textit{Bic, Moto, Bus}: 1.6; \textit{Car, Tru}: 1.2; \textit{Tra}: 1.16;\textit{Ped}: 1.78) on nuScenes. Voxel mask size $\theta_{vm}$ is 5\textit{m} on nuScenes.  The count-based and output file strategies are consistent with~\cite{li2023poly}. In the confidence-based part, decay rates $\sigma$ are category-specific, which are (\textit{Ped}: 0.18; \textit{Car}: 0.26; \textit{Tru, Moto}: 0.28; \textit{Tra}: 0.22; \textit{Bic,Bus}: 0.24) on nuScenes. The delete threshold $\theta_{dl}$ are (\textit{Bus}: 0.08,  \textit{Ped}: 0.1 and 0.04 for other categories) on nuScenes.

Our tracking framework is implemented in Python and executed on an Nvidia 4090X GPU. Hyperparameters are optimized based on the highest AMOTA achieved on the validation set. The following category-specific and SF thresholds are employed for nuScenes: (\textit{Bic}: 0.15; \textit{Car}: 0.16; \textit{Moto}: 0.16; \textit{Bus}: 0.12; \textit{Tra}: 0.13; \textit{Tru}: 0; \textit{Ped}: 0.13). NMS thresholds are uniformly set to 0.08 across all categories and datasets. Additionally, we implement Scale-NMS~\cite{huang2021bevdet} for (\textit{Bic}, \textit{Ped}) categories on nuScenes.
In conjunction with the default in NMS, we introduce our novel  metric to enhance similarity assessment for (\textit{Bic}, \textit{Ped}, \textit{Bus}, \textit{Tru}) categories on nuScenes. Motion models and filters are consistent with those described in~\cite{li2023poly}. A lightweight filter, implemented as a median filter with , is applied across all datasets. Association metrics universally employ  across all datasets.
Category-specific first association thresholds  for nuScenes are as follows: (\textit{Bic, Moto, Bus}: 1.6; \textit{Car, Tru}: 1.2; \textit{Tra}: 1.16; \textit{Ped}: 1.78). The voxel mask size  is set to 5\textit{m} on nuScenes. Count-based and output file strategies align with those presented in~\cite{li2023poly}.
In the confidence-based component, category-specific decay rates for nuScenes are: (\textit{Ped}: 0.18; \textit{Car}: 0.26; \textit{Tru, Moto}: 0.28; \textit{Tra}: 0.22; \textit{Bic, Bus}: 0.24). The delete thresholds  are set as follows: (\textit{Bus}: 0.08, \textit{Ped}: 0.1, and 0.04 for all other categories) on nuScenes.

In the object tracking phase of 3D MOT, Easy-Poly exhibits exceptional performance following a series of optimizations. These enhancements include pre-processing, Kalman filtering, motion modeling, and tracking cycle refinements. The integration of these techniques significantly improves the algorithm's effectiveness in complex 3D environments.

Easy-Poly exhibits exceptional performance on the test set, achieving a \textbf{75.9\%} AMOTA score, surpassing the majority of existing 3D MOT methods. As shown in Table \ref{table:nu_test}, Easy-Poly attains a remarkably low IDS count of \textbf{287} while maintaining the highest AMOTA (\textbf{75.9\%}) among all modal methods. This underscores Easy-Poly's ability to maintain stable tracking without compromising recall. Notably, Easy-Poly achieves state-of-the-art performance without relying on additional image data input. Easy-Poly significantly outperforms competing algorithms in the critical 'Car' category. With minimal computational overhead, it delivers impressive results, highlighting its potential for integration into real-world autonomous driving systems. The False Negative and False Positive metrics in Table \ref{table:nu_test} further demonstrate Easy-Poly's robust continuous tracking capability while maintaining high recall.

For validation set experiments in Table \ref{table:nu_val}, we utilize CenterPoint~\cite{yin2021center} as the detector to ensure fair comparisons. As illustrated in Table \ref{table:nu_val}, Easy-Poly significantly outperforms most deep learning-based methods in both tracking accuracy (\textbf{75.0\%} AMOTA, \textbf{64.8\%} MOTA) and computational efficiency (\textbf{34.9} FPS). Compared to the baseline FastPoly~\cite{li2024fast}, Easy-Poly achieves substantial improvements of \textbf{+1.3\%} in MOTA and \textbf{+1.6\%} in AMOTA, while operating \textbf{1.5x} faster under identical conditions. When integrated with the high-performance LargeKernel3D~\cite{chen2022scaling} detector, Easy-Poly demonstrates even more impressive detection and tracking capabilities. Furthermore, when employing the multi-camera detector \textcolor{black}{DETR3D}~\cite{DETR3D} with constrained performance, Easy-Poly exhibits robust real-time performance without compromising accuracy. Furthermore, the lower AMOTP and IDS metrics demonstrate Easy-Poly's exceptional capability in tracking small objects and maintaining performance in complex scenarios and adverse weather conditions. These results underscore the algorithm's robustness across diverse and challenging environments.
% It is noteworthy that achieving optimal AMOTA necessitates a stringent score filter threshold, which marginally reduces the latency advantage of our method. Nevertheless, Easy-Poly maintains a favorable balance between accuracy and computational efficiency.

\begin{table}
% \vspace{0.5em}
        \caption{Comparing different data association algorithms using CenterPoint and (lines 1-7) and LargeKernel3D (lines 8-11) methods on nuScenes val set. Among them, the algorithms lines 1-3 are the Fast-Poly framework and in lines 4-11 are the latest our Easy-Poly framework.}

        \label{table:nus_assoc}
        % \renewcommand{\arraystretch}{0.7}
        \setlength{\tabcolsep}{0.1mm}
        \begin{tabular}{cccccc}
        \toprule
        \multicolumn{1}{c}{\textbf{Algorithms}} & \textbf{MOTA}$\uparrow$ & \textbf{AMOTA}$\uparrow$ & \textbf{AMOTP}$\downarrow$ & \textbf{IDS}$\downarrow$ & \textbf{FN}$\downarrow$\\ 
        
        \midrule
         %  Mutual Nearest Neighbor (MNN)
         MNN  & 62.2  & 72.5 & 52.4  & 433 & 16644  \\  
         Greedy    & 62.3  & 72.7 &  53.4  & 428  & 17647 \\
         Hungarian & \textbf{\textcolor{blue}{63.2}}  & \textbf{\textcolor{blue}{73.7}} & \textbf{\textcolor{blue}{52.1}} & \textbf{\textcolor{blue}{414}}  & \textbf{\textcolor{blue}{15996}}  \\
        
        \midrule  
        % 多模态+数据增强
         %  Mutual Nearest Neighbor (MNN)
        \textbf{MNN (Ours)}  & 63.7  & 73.6 & 54.8  & 406 &  \textbf{\textcolor{red}{15873 }}\\  
        \textbf{Greedy (Ours)}    &  64.0  & 73.7  &  54.6  & 368  & 16736 \\
        \textbf{Hungarian (Ours)}  & 64.3  & 74.3 & \textbf{\textcolor{red}{54.3}} &  335 & 16892  \\

        \textbf{DTO (Ours)}   & \textbf{\textcolor{red}{64.4}}  &   \textbf{\textcolor{red}{74.5}} & 54.9 &  \textbf{\textcolor{red}{272}} & 16982  \\

        
         \midrule
         \textbf{MNN (Ours)}  & 64.1  & 73.9 & 54.0  & 370 & 15865  \\  
         \textbf{Greedy (Ours)}    & 64.5  & 74.3 & 53.7  & 307  & 16014 \\
        \textbf{Hungarian (Ours)}  & 64.7  & 74.8 & 53.9  & 291  & 15923 \\

        \textbf{DTO (Ours)}  & \textbf{\textcolor{red}{64.8}}  & \textbf{\textcolor{red}{75.0}} & \textbf{\textcolor{red}{53.6}}  & \textbf{\textcolor{red}{242}}  & \textbf{\textcolor{red}{15488}} \\

    \bottomrule
\end{tabular}
\end{table}



% \begin{table}
% \caption{The ablation study of whether or not to use Score Filter and Non-Maximum Suppression, including the Run-Time, which represents the execution time of the Pre-processing Module. We compared Poly-MOT~\cite{li2023poly} (rows 1-3) with our proposed Easy-Poly method (rows 4-6).}
% \label{table:nu_NMSsf}
% % \renewcommand{\arraystretch}{0.7}
% \setlength{\tabcolsep}{2.7mm}
% {
% \begin{tabular}{cccc}
% \toprule

% \textbf{Variable} & \textbf{AMOTA$\uparrow$} & \textbf{IDS$\downarrow$} &
% \textbf{Run-Time (s) $\downarrow$}
% \\ 
% \midrule
% NMS + SF & \textbf{\textcolor{blue}{73.1}}   & \textbf{\textcolor{blue}{281}}  & 0.055    \\
% NMS     & 71.8  & 320  & 0.093     \\
% SF       & 68.6  & 354  & \textbf{\textcolor{blue}{0.008}}     \\ 
% \midrule
% \textbf{NMS + SF (Ours)} & \textbf{\textcolor{red}{75.0}}   & \textbf{\textcolor{red}{242}}  & 0.037    \\
% \textbf{NMS (Ours)}      & 73.6   & 273  & 0.068    \\
% \textbf{SF (Ours)}       & 71.2   & 308  & \textbf{\textcolor{red}{0.008}}     \\\hline
% % \vspace{-3.2em}
% \end{tabular}}
% \end{table}



\subsection{Comparative Evaluations} 
\label{sec: Comparative}

% Our proposed method, Easy-Poly, achieves state-of-the-art performance on the nuScenes validation set, demonstrating \textbf{75.0\%} AMOTA at \textbf{34.9} FPS, surpassing existing approaches. Utilizing an identical detector, Easy-Poly outperforms Fast-Poly \cite{li2024fast} across nearly all key metrics, with notable improvements in accuracy (\textbf{+1.3\%} AMOTA, \textbf{+1.6\%} MOTA) and speed (\textbf{+6.0 FPS}). While marginally slower than CBMOT \cite{benbarka2021score} and 3DMOTFormer \cite{ding20233dmotformer}, Easy-Poly significantly exceeds their accuracy while maintaining robust real-time performance. Our open-source implementation establishes a strong baseline for 3D MOT, providing a solid foundation for future advancements in the field.

In this study, we conduct a comprehensive evaluation of the association stage, focusing on four algorithms: Hungarian, Greedy, MNN, and the novel DTO. Our extensive experiments, summarized in Table~\ref{table:nus_assoc}, reveal that the Easy-Poly consistently outperforms Fast-Poly across both CenterPoint and LargeKernel3D frameworks. Notably, LargeKernel3D demonstrates superior performance over CenterPoint, particularly in complex tracking scenarios. Among the association algorithms, Hungarian and DTO consistently yield superior results, underscoring their robustness and efficacy in diverse multi-object tracking contexts. Compared to the Hungarian algorithm, DTO not only achieves similarly excellent AMOTA values but also provides more robust and accurate tracking performance, especially in challenging scenarios involving occlusions, missed detections, or false positives. These findings highlight the critical role of algorithm selection and model optimization in advancing the state-of-the-art in 3D object tracking.

\begin{figure*}[t]
    \centering
    \includegraphics[width=0.95\linewidth]{Images/Ablation_study_line_chart_for_NMS.pdf}
    \vspace{-8pt}
    \caption
    {
      The ablation study of whether or not to use Score Filter and Non-Maximum Suppression, including the Run-Time, which represents the execution time of the Pre-processing Module. We compared Poly-MOT with our proposed Easy-Poly method.
     }
     \Description{}
    \label{fig: NMSFS}
\end{figure*}


\textcolor{black}{Table \ref{table:nu_life} demonstrates the efficacy of our proposed methods. The Fast-Poly tracklet termination strategy (line 3) significantly outperforms baseline score refinement~\cite{benbarka2021score} (line 2), yielding a \textbf{2.7\%} improvement in AMOTA and reducing FN by \textbf{2366}. This enhancement mitigates tracker vulnerabilities in mismatch scenarios, including occlusions. Further performance gains are achieved through smoother score prediction (line 4), resulting in additional improvements of \textbf{0.4\%} AMOTA, \textbf{0.1\%} MOTA, and a reduction of \textbf{926 FN}.}
Our Easy-Poly model exhibits even more substantial performance enhancements. The tracklet termination strategy (line 7) surpasses the baseline score refinement (line 6) by \textbf{2.8\%} in AMOTA while decreasing FN by \textbf{1844}. Furthermore, the integration of smoother score prediction (line 8) further boosts the tracking performance, resulting in improvements of \textbf{0.4\%} in AMOTA and \textbf{0.6\%} in MOTA, along with a reduction of \textbf{575} FN.


\begin{table}
% \vspace{0.5em}
        \caption{A comparison on distinct life-cycle modules on nuScenes val set.
        \textbf{Average} means using the online average score to delete.
        \textbf{Latest} means using the latest score to delete.
        \textbf{Max-age} means using the continuous mismatch time to delete.
        Other settings are under the best performance. Methods in lines 1-4 and lines 5-8 use Fast-Poly~\cite{li2024fast} and Our Easy-Poly.
        }
        \label{table:nu_life}
        % \renewcommand{\arraystretch}{0.7}
        \setlength{\tabcolsep}{0.1mm}
        \begin{tabular}{ccccc}
        \toprule
        \multicolumn{1}{c}{\textbf{Strategy}} & \textbf{AMOTA}$\uparrow$ & \textbf{MOTA}$\uparrow$ & \textbf{FPS}$\uparrow$ & \textbf{FN}$\downarrow$\\ \midrule
 
         Count                \& Max-age            & 73.3      & 62.9     & 23.0  & 16523    \\
         %predict:normal update:normal+delete = -100
         Confidence~\cite{benbarka2021score} \& Latest       & 70.6      & 63.2     & \textbf{\textcolor{blue}{45.8}}  & 19192   \\%predict:minus update:multi+latest
         Confidence~\cite{benbarka2021score} \& Average      & 73.3      & 63.1     & 28.3 & 16826   \\%predict:minus update:multi+average
         Confidence  \& Average      & \textbf{\textcolor{blue}{73.7}}      & \textbf{\textcolor{blue}{63.2}}     & 28.9  & \textbf{\textcolor{blue}{15900}}   \\%predict:normal update:multi+average 27.7or28.9
         \midrule
         \textbf{Count \& Max-age (Ours)}          & 74.3      & 63.8     & 34.3  & 16024    \\
         %predict:normal update:normal+delete = -100
         \textbf{Confidence~\cite{benbarka2021score} \& Latest (Ours)}        & 71.8      & 63.7     & \textbf{\textcolor{red}{50.2}}  & 17907   \\%predict:minus update:multi+latest
         \textbf{Confidence~\cite{benbarka2021score} \& Average (Ours)}       & 74.6      & 64.2     & 35.6 & 16063   \\%predict:minus update:multi+average
         \textbf{Confidence \& Average (Ours)}      & \textbf{\textcolor{red}{75.0}}      & \textbf{\textcolor{red}{64.8}}     & 36.9  & \textbf{\textcolor{red}{15488}}   \\%predict:normal update:multi+average 36.9
        \bottomrule
        \end{tabular}
    \end{table}


\subsection{Ablation Studies}

% The Sf can filter out low-score bounding boxes while NMS can remove duplicate bounding boxes with high confidence, which makes the remaining bounding boxes have superior quality. For the Poly-MOT, using SF before NMS brings inference 40\% reduction in pre-processing inference time while boosting AMOTA by 1.3\% compared with only using NMS. The Fast-Poly has better performance, using SF before NMS brings inference 50\% reduction in pre-processing inference time while boosting AMOTA by 1.4\% compared with only using NMS, as demonstrate in Table~\ref{table:nu_NMSsf}.  It is clear from the table that when only SF is used without NMS, although the running time is fast, both AMOTA and IDS values become lower and the values drop very significantly.

We optimize two-stage filtering approach that synergistically combines SF and NMS to significantly enhance bounding box quality in multi-object tracking scenarios. This novel method effectively integrates SF to eliminate low-score detections and NMS to remove high-confidence duplicates, resulting in a set of superior quality bounding boxes. Our comprehensive experimental results, presented in Figure~\ref{fig: NMSFS}, demonstrate substantial improvements in both computational efficiency and tracking accuracy across multiple state-of-the-art models. For the Poly-MOT model, our approach of applying SF before NMS yields a remarkable \textbf{40\%} reduction in pre-processing inference time while simultaneously improving AMOTA by \textbf{1.3\%} compared to using NMS alone. These results highlight the significant potential of our method in improving real-time tracking capabilities without compromising accuracy. The Easy-Poly model exhibits even more impressive performance gains, further validating the scalability and effectiveness of our approach. By applying SF before NMS, we achieve a substantial \textbf{50\%} reduction in pre-processing time coupled with a \textbf{1.4\%} improvement in AMOTA. This notable improve in both speed and accuracy underscores the robustness of our method across different model architectures. Importantly, our analysis reveals critical insight into the interplay between SF and NMS. Although using SF without NMS accelerates processing, it leads to significant degradation in both AMOTA and Identity Switches IDS metrics. This observation underscores the crucial importance of our combined SF-NMS approach in maintaining an optimal balance between processing speed and tracking accuracy. The synergistic effect of SF and NMS not only enhances the quality of bounding boxes but also optimizes the trade-off between computational efficiency and tracking performance. This balance is particularly vital in real-world applications where both speed and accuracy are paramount, such as autonomous driving and surveillance systems. 
% The consistent improvements observed across different models suggest broad applicability and potential for integration into various tracking frameworks, paving the way for more efficient and accurate tracking systems in complex, real-world environments.


% \subsection{Visualization}

\section{Analysis}
Our analysis addresses the following questions:
\begin{figure*}[t]
\centering
\includegraphics[width=0.99\textwidth]{fig/crop_case.pdf}
\caption{Sample outputs of our three tasks generated by different vision language models, along with gold truth. We highlight errors in distractors. }
\label{fig:case}
\vspace{-4mm}
\end{figure*}

\paragraph{Does fine-tuning with reorder task help?}
Yes, it does. We fine-tune Qwen2-VL using 3,160 samples for one epoch. This not only significantly improves performance on the reordering VQA task but also enhances comprehension tasks.
To construct the training dataset, we applied data augmentation to 790 images using the reorder task. Specifically, we randomly shuffled the sequence of images four times, generating a total of approximately 3,160 distinct samples. For evaluation on the reorder task, we used only the remaining 100 samples. For the comprehension task, we conducted a full test set evaluation, as the training data provided only images without any analytical content. Meanwhile, we excluded the frame prediction task from testing due to potential data leakage. The experimental results are presented in the table~\ref{tab:finetune}. 
Overall, our reordering data is useful for fine-tuning, as it can enhance the LMMs to reason on sequential images. However, the ultimate performance still depends on the base capability of the model. From the table, it can be seen that Qwen2.5-VL benefits greatly from fine-tuning, with improvements not only in the reordering task but also in comprehension tasks. The improvement in generation tasks in the reordering task is much larger than that in choice tasks. We believe this is because the generation task's instructions are very challenging, and the model has not been trained on them before, leading to difficulty in solving them. On the other hand, for choice tasks, the model can make educated guesses based on the options provided. Due to the relatively small amount of training samples, the model's improvement in choice tasks is limited. In contrast, LLaVA shows improvement only in the reorder-choice task after training.


\begin{table}[htbp]
    \centering
     \resizebox{0.45\textwidth}{!}{% 缩小整体表格
    \begin{tabular}{c|c|c|c}
    \toprule
        Tasks  & Comprehension & Reordering-G & Reordering-C \\
        \midrule
       Qwen2-VL & 58.53 & 6.00 & 31.00 \\
       +finetune & 62.94 & 31.00 & 38.00 \\
       \midrule
       LLaVA-1.6 & 34.41 &3.00& 26.00 \\
       +finetune & 33.82 &2.00 & 32.00 \\
       \bottomrule
    \end{tabular}}
    \caption{Performance on Qwen2-VL-7B and LLaVA-1.6-7B finetuned with reordering task data. Reordering-C refers to the Reordering-choice. Reordering-G refers to the Reordering-generation}
    \label{tab:finetune}
\end{table}

\paragraph{Does GPT-4o understand sequence images as well as humans?}
While GPT-4o achieves the highest performance among all tested models, there remains a substantial gap between its capabilities and human performance, particularly in novel tasks like frame reordering. Through our preliminary data annotation experiments, we observed that while GPT-4o can comprehend basic comic content and provide interpretations, it frequently generates hallucinated content and struggles with comics that depict unconventional or imaginative scenarios rarely encountered in real life.

In the visual narrative comprehension and next-frame prediction tasks, the multiple-choice format allows models to leverage similarity matching between options. Our investigation revealed that model performance is heavily influenced by the quality of distractor options. In initial experiments with weak distractors (generated using GPT-4o with instructions to provide distractors with hallucinations, the prompt is followed HalluEval), the model achieved accuracy rates up to 90\%. Upon analysis, we found these initial distractors were too obviously incorrect or irrelevant to the comic content, making the selection task trivial. To address this limitation, we carefully curated a new set of challenging distractors. With these enhanced distractors, performance of GPT-4o decreased significantly to more realistic levels ($61.60\%$ for understanding and $69.95\%$ for prediction), better reflecting the true challenges in comic comprehension. The scores obtained from multiple-choice questions with semantically transparent options tend to be inflated. In subsequent reordering tasks, where options lack explicit semantic meanings, coupled with open-ended questions, the scores provided a more authentic assessment of the LMMs.
% 这种选择题的形式(且选项内容有直接的含义)测出来的分数是偏高的。后来在reorder中,选项内容没有直接的实际含义,他的分数并且还测了问答题。这个任务的分数比较真实的反应模型的能力。

% 总的来说,模型的表现还是不如人类。尤其是在新颖的任务reorder中,表现差距非常大。在前面测试4o是否可以拿来标数据的时候,我们发现模型可以理解漫画的部分内容并做出一些解读,但是也会经常有幻觉内容。对于一些脑洞大开在现实生活中很少见的漫画,模型就很难理解。在visual narrative comprehensive 和 next-frame prediction 任务中,因为是选择题的形式,所以可以根据选项内容进行相似度匹配。模型的选择受干扰项的影响很大。在最初的任务标注中,干扰项使用4o模型直接告诉他要生成带幻觉的干扰项,这些弱干扰项和答案一起输给LMM时,4o的准确率可以达到90%。经过分析例子,我们发现是那些弱干扰项错误的非常明显,所以四选一可以轻松的选对答案。后面我们又生成了新的强干扰项,然后模型的准确率才下降。
% 怎么讲清楚,understanding,prediction的选项内容跟reorder的是不一样的?前者有直接的含义,而后者需要转义无实际含义。
\paragraph{Does input format of images influence performance?}


% 每个帧的单独输入可以帮助模型更清楚地提取每一格的信息,而不会因为它们挤在一张图上导致信息混杂。
% 帧间逻辑更明确:通过人工提供帧的顺序,模型能更好地理解帧之间的逻辑关系(比如时间顺序、因果关系等),避免因整图布局复杂而误解帧的排列。
% 但是也考察模型对多图片的整合处理方式。
%我们做了更细粒度的实验,把整图按照frame拆分,然后一次性按序输入。这种方式对于人类来说会更简单,但是由于模型处理多图走的是video方式,这可能对模型来说也是挑战。结果表明差别不大,所以我们进一步把帧的顺序打乱继续测模型对隐含意的理解。
Considering the distinct computational pathways that LMMs employ in processing individual versus multiple images, we designed following experiments to measure the differential impact of varied input formats using Qwen2.5VL as our test case. We compared three input formats: (1) whole image - the entire comic strip as a single image, (2) sequential frames - individually separated frames input in order, and (3) shuffled sequence - separated frames input in random order. Table~\ref{tab:understanding} shows surprisingly consistent performance across all three formats ($56.03\%$, $54.56\%$, and $57.65\%$ respectively).

This consistency suggests that while separated frames might theoretically help models extract clearer information from each panel and avoid visual confusion from complex layouts, the current video-like processing mechanism used by LMMs for multiple images might not fully capitalize on these advantages. The similar performance with shuffled sequences further indicates that models might rely more on individual frame content rather than sequential relationships for understanding tasks.


\begin{table}[htbp]
    \centering
     \resizebox{0.45\textwidth}{!}{% 缩小整体表格
    \begin{tabular}{c|c|c}
     \toprule
       Input  & GPT-4o-mini & Qwen2.5-VL \\
       \midrule
       Whole Image  & 53.23 & 56.03 \\
      Image Sequence & 51.03 &  54.56 \\
       Shuffled Sequence & 49.56 & 57.65 \\
       \bottomrule
    \end{tabular}}
    \caption{Performance with different input format for understanding task. }
    \label{tab:understanding}
\end{table}
\paragraph{Does implicit meaning help reordering task?}
% 新的prompt = f"The sequence of the comic strips provided below is incorrect, Your task is to find out the correctorder of the comic strips based on the storyline, temporal relationships, and common-sense logic. Here is thedepiction of the comic strips: \"fga.get('understanding’)}\" \n Number each comic strip in the order they shouldappear, starting from 1.\n\n" 
To investigate whether poor reordering performance stems from inadequate semantic understanding, we enhanced the reordering task by providing explicit semantic annotations along with shuffled images. As shown in Table~\ref{tab:reorder_enhanced}, this additional semantic information only marginally improved performance (from 30.01\% to 32.54\%). This modest improvement suggests that the bottleneck in reordering tasks lies not in semantic understanding but in the fundamental capability to reason about temporal and logical sequences in visual narratives.

\begin{table}[htbp]
    \centering
     \resizebox{0.45\textwidth}{!}{% 缩小整体表格
    \begin{tabular}{c|c|c}
    \toprule
        Input & GPT-4o-mini & Qwen2.5-VL \\
        \midrule
       Shuffled image & 26.07  & 30.01 \\
       +Meaning & 28.40 & 32.54 \\
       \bottomrule
    \end{tabular}}
    \caption{Performance with enhanced data (correct answer for comprehension task) for reordering task.}
    \label{tab:reorder_enhanced}
\end{table}
% 从表中可以看到,给模型输入额外的漫画含义后,排序的分数并没有很大的提升。所以LMMs在reorder任务上的瓶颈并不是理解图片含义。以后的模型应该增强对此任务的训练。

% \paragraph{Can video-VLM outperform VLLM?}
% Video-LLM is trained with video data containing richer temporal information. Comic strip is like the key frame for video, so it omitted the step of extracting keyframes from the video. 


% \paragraph{Does CoT help with three tasks?}
% To achieve better performance, we conducted experiments using the Chain of Thought (CoT) approach by adding "Let's think step by step" to the input. However, as shown in table~\ref{tab:cot}, we found that this modification did not improve performance on the first two tasks and only provided a boost in the reorder task. We speculate that this is because the instructions for the reorder task are relatively more complex compared to the other tasks.
% \begin{table}[htbp]
%     \centering
%      \resizebox{0.45\textwidth}{!}{% 缩小整体表格
%     \begin{tabular}{c|c|c|c}
%     \toprule
%         Tasks  & Comprehension & Frame Prediction & Reordering-MCQ \\
%         \midrule
%        origin input & 56.03  & 64.00 & 29.21 \\
%         +CoT        & 52.94 & 57.50  & 31.69 \\
%        \bottomrule
%     \end{tabular}}
%     \caption{Compare the performance with and without CoT (Chain of Thought).}
%     \label{tab:cot}
% \end{table}

\paragraph{How does model size affect?}
In this section, we will discuss the relationship between model parameters size and reasoning performance for tasks due to the scaling law. We examine on LLaVA and Qwen2.5VL, from 3B scale to 34B scale.
There is a clear scaling effect across model sizes, as demonstrated by LLaVA1.5's performance improving from $34.41\%$ (7B) to $46.03\%$ (13B) to $52.94\%$ (34B). This suggests that model scale plays a crucial role in comprehending implicit meanings in visual narratives.
Figure~\ref{fig:scale} provide a visual representation of the performance trend for five models.

\begin{figure}[t]
\centering
\includegraphics[width=0.96\columnwidth]{fig/model.pdf}
\caption{ Comparison of the accuracy results between Qwen2.5-3B vs Qwen2.5-7B and LLaVA-1.6-7B vs LLaVA-1.6-13B vs LLaVA-1.6-34B}
\label{fig:scale}
\vspace{-4mm}
\end{figure}



\paragraph{Where do LMMs fail?}
We present sample outputs of three tasks generated by vision language models (VLMs) in Figure~\ref{fig:case}. These images are easy for human but hard for VLMs. VLMs can understand one comic strip but they can still make mistakes with reordering task. 

% \paragraph{Does repeating images help with three tasks?}
% % 类型,模型处理inspiring的会更好。对于讽刺类的表现不好。我们推测这是由于模型在训练的时候被灌输大量的积极的内容,而消极内容训练的比较少,并被要求减少负面输出。
% \paragraph{Error analysis}
%  % visual misinterpretation  
%   % hallucination 
%   % 常识  打破镜子而不是窗户

% Our experiments reveal that while models achieve reasonable scores on semantic understanding tasks, their abilities remain superficial. Notably, we observe that model predictions remain  unchanged even when presented with scrambled frame sequences, indicating that current LMMs rely more on matching similar visual elements and semantic concepts across frames rather than truly comprehending the sequential narrative structure of comics. This observation is further supported by their poor performance on frame reordering tasks, highlighting their limitations in temporal reasoning and sequential understanding.
% \section{Discussion and Future Work}

% %Onscad is insample data cause these LLMS have seen openscad

% The decision space of language design is enormous, so we had to make some decisions about what to explore in the language design of AIDL. In particular, we did not build a new constraint system from scratch and instead developed ours based on an open-source constraint solver. This limited the types of primitives we allow, e.g. ellipses are not currently supported. \jz{Additionally, rectangles in AIDL are constrained to be axis-aligned by default because we found that in most use cases, a rectangle being rotated by the solver was unintuitive, and we included a parameter in the language allows rectangles to be marked as rotatable. While this feature was included in the prompts to the LLM, it was never used by the model. We hope to explore prompt-engineering techniques to rectify this issue in the future. Similarly, we hope to reduce the frequency of solver errors by providing better prompts for explaining the available constraints.} \adriana{Add two other limitations to this paragraph: that we typically noticed that things are axis alignment, say why we use this as default and in the future could try to get the gpt to not use default more often. Mention that we still have Solver failures that could be addressed by better engineering in future. }

% In testing our front-end, we observed that repeated instances of feedback tends to reduce the complexity of models as the LLM would frequently address the errors by removing the offending entity. This leads to unnecessarily removed details. More extensive prompt-engineering could be employed in future work to encourage the LLM to more frequently modify, rather than remove, to fix these errors. \adriana{no idea what this paragraph is trying to say}


% \adriana{This seems  like a future work paragraph so maybe start by saying that in the future you could do other front end or fine tune a model with aidl, we just tested the few shot.  } \jz{In the future, we hope to improve our front-end generation pipeline by finetuning a pretrained LLM on example AIDL programs.} In addition, multi-modal vision-langauge model development has exploded in recent months. Visual modalities are an obvious fit for CAD modeling -- in fact, most procedural CAD models are produced in visual editors -- but we decided not to explore visual inputs yet based on reports ([PH] cite OPENAIs own GPT4V paper) that current vision-language models suffter from the same spatial reasoning issues as purely textual models do (identifying relative positions like above, left of, etc.). This also informed our decision to omit spline curves which are difficult to describe in natural language. This deficit is being addressed by the development of new spatial reasoning datasets ([PH] cite visual math reasoning paper), so allowing visual user input as well as visual feedback in future work with the next generation of models seems promising.

% The decision space of language design is enormous, so it was impossible to explore it all here. We had to make some decisions about what to explore, guided by experience, conjecture, technical limitations, and anecdotal experience. Since we primarily explore the interaction between language design and language models in order to overcome the shortcomings in the latter, we did not wish to focus effort on building new constraint systems. This led us to use an open-source constraint solver to build our solver off of. This limited the types of primitives we allow; in particular, most commercial geometric solvers also support ellipses.

% In testing our generation frontend, we observed that repeated instances of feedback tended to reduce the complexity of models as the LLM would frequently address the errors by removing the offending entity. This is a fine strategy for over-constrained systems, but can unnecessarily remove detail when done in response to a syntax or validation error. More extensive prompt-engineering could be employed to encourage the LLM to more frequently modify, rather than remove, to fix these errors


% In recent months, multi-modal vision-language model development has exploded. Visual modalities are an obvious fit for CAD modeling -- in fact, most procedural CAD models are produced in visual editors -- but we decided not to explore visual input yet based on reports (cite OpenAIs own GPT4V paper) that current vision-language models suffer from the same spatial reasoning issues as purely textual models do (identifying relative positions like above, left of, etc.). This also informed our decision to omit spline curves; they are not easily described in natural language. This deficit is being addressed by the development of new spatial reasoning datasets (cite visual math reasoning paper), so allowing visual user input as well as visual feedback in future work with the next generation of models seems promising.



\section{Conclusion}

AIDL is an experiment in a new way of building graphics systems for language models; what if, instead of tuning a model for a graphics system, we build a graphics system tailored for language models? By taking this approach, we are able to draw on the rich literature of programming languages, crafting a language that supports language-based dependency reasoning through semantically meaningful references, separation of concerns with a modular, hierarchical structure, and that compliments the shortcomings of LLMs with a solver assistance. Taking this neurosymbolic, procedural approach allows our system to tap into the general knowledge of LLMs as well as being more applicable to CAD by promoting precision, accuracy, and editability. Framing AI CAD generation as a language design problem is a complementary approach to model training and prompt engineering, and we are excited to see how advance in these fields will synergize with AIDL and its successors, especially as the capabilities of multi-modal vision-language models improve. AI-driven, procedural design coming to CAD, and AIDL provides a template for that future.

% Using procedural generation instead of direct geometric generation enables greater editability, accuracy, and precision
% Using a general language model allows for generalizability beyond existing CAD datasets and control via common language.
% Approaches code gen in LLMs through language design rather than training the model or constructing complexing querying algorithms. This could be a complimentary approach
% Embedding as a DSL in a popular language allows us to leverage the LLMs syntactic knowledge while exploiting our domain knowledge in the language design
% LLM-CAD languages should hierarchical, semantic, support constraints and dependencies




%In this paper, we proposed AIDL, a language designed specifically for LLM-driven CAD design. The AIDL language simultaneously supports 1) references to constructed geometry (\dgone{}), 2) geometric constraints between components (\dgtwo{}), 3) naturally named operators (\dgthree{}), and 4) first-class hierarchical design (\dgfour{}), while none of the existing languages supports all the above. These novel designs in AIDL allow users to tap into LLMs' knowledge about objects and their compositionalities and generate complex geometry in a hierarchical and constrained fashion. Specifically, the solver for AIDL supports iterative editing by the LLM by providing intermediate feedback, and remedies the LLM's weakness of providing explicit positions for geometries.

%\adriana{This seems  like a future work paragraph so maybe start by saying that in the future you could do other front end or fine tune a model with aidl, we just tested the few shot.  }
%\paragraph{Future work} In recent months, multi-modal vision-language model development has exploded. Visual modalities are an obvious fit for CAD modeling -- in fact, most procedural CAD models are produced in visual editors -- but we decided not to explore visual input yet based on reports (cite OpenAIs own GPT4V paper) that current vision-language models suffer from the same spatial reasoning issues as purely textual models do (identifying relative positions like above, left of, etc.). This also informed our decision to omit spline curves; they are not easily described in natural language. This deficit is being addressed by the development of new spatial reasoning datasets (cite visual math reasoning paper), so allowing visual user input as well as visual feedback in future work with the next generation of models seems promising. 

\section*{Limitations}

While Drift contributes promising advances in implicit personal preferences, several limitations remain that should be addressed in future research.

\paragraph{Needs of Online Human Evaluation Benchmarks.}  
A major challenge in personal preference research is the absence of reproducible human evaluations. Even if future benchmarks collect more user-specific annotations beyond PRISM, evaluating personalized generation outputs requires \textit{re-engaging with the same users for feedback}. Although we designed the Perspective dataset to align the label construction and test set evaluation pipelines, it still relies on virtual personas. Therefore, to advance this field, there is a need for online evaluation benchmarks that can reproducibly assess personalized generation using real user feedback.

\paragraph{Limited Analysis Between Drift Attributes and Actual Users.}  
Due to practical and ethical issues, we do not have full access to the backgrounds of actual users. While the PRISM dataset provides basic information (e.g., the intended use of LLMs and brief self-introductions), our analysis (as seen in Figure~\ref{fig:activated-attributes-prism}) is limited in explaining why certain attributes are activated and how these relate to user characteristics. A more in-depth investigation into the correlation between Drift attributes and real user profiles should be studied with future benchmarks.

\paragraph{Biases in Differential Prompting.}  
Our study does not thoroughly analyze the limitations of the zero-shot rewarding mechanism used for each attribute. It is possible that differential prompting may fail to capture certain attributes accurately, and methods like those employed in Helpsteer2—where data is explicitly constructed—could offer more precise evaluations. Nevertheless, given the vast diversity of personal preferences, a zero-shot approach remains essential. As shown in Figure~\ref{fig:rm-results}, this approach yields significantly higher performance, and Figure~\ref{fig:attributes_num} demonstrates that even when the number of attributes is reduced to levels comparable to those used in Helpsteer2, performance remains robust. In essence, unreliable attributes are unlikely to be used during decoding, which mitigates this limitation. Moreover, as future research develops to enable LLM to follow system prompts more precisely, these advances will directly enhance Drift.

\paragraph{Tokenizer Dependency.}  
Drift Decoding adjusts the next-token distribution at each step, which requires that the LLM and the sLM share the same support—that is, they must use the same tokenizer. 

\paragraph{Limited Baselines.}  
Due to the scarcity of datasets for implicit personal preferences, this domain is far less mature compared to explicit preferences. As highlighted in Table~\ref{tab:method_comparison}, the limited availability of extensive baselines forced us to concentrate primarily on analyzing the unique characteristics of Drift.

\section*{Ethical Statement}

While Drift effectively integrates users’ implicit preferences into generated outputs, it also introduces several ethical risks that must be carefully managed. Notably, the Drift Approximation stage allows us to directly assess the activation levels of each attribute, which provides an opportunity to identify and preemptively block system prompts that could lead to harmful or undesirable content before they are incorporated into the decoding process. This capability underscores the importance of further research into combining Drift with diverse system prompts, ensuring that the generation of undesirable content is minimized while still delivering personalized services.

Additionally, considering that existing research~\citep{kim2024guaranteed} indicates it is impossible to obtain filtered autoregressive distributions under certain conditions, it is necessary to combine rejection sampling on final outputs using safeguards~\citep{lifetox} such as LlamaGuard~\citep{llamaguard} and ShieldGemma~\citep{shieldgemma}. This approach can further enhance the safety of the final generated content.



% Bibliography entries for the entire Anthology, followed by custom entries
%\bibliography{anthology,custom}
% Custom bibliography entries only
\bibliography{custom}

\clearpage

\appendix

\section*{Appendix}
\section{Additional Experimental Details}
\label{appen:experimental details}
\subsection{Datasets}
\label{appen:datasets}
In this work, we categorize the datasets into \textit{in-context} and \textit{out-of-context} datasets. The videos from \textit{in-context} datasets consist of actions with frequent static context, \eg swimming in the swimming pool, while the videos from \textit{out-of-context} datasets contain actions occurring with an unusual static context, \eg dancing in the mall~\cite{choi2019can}.
We conduct the experiments on five \textit{in-context} benchmarks: Kinectics-400~\cite{k400} (K400), Kinectis-600~\cite{k600} (K600), UCF101~\cite{UCF101} (UCF), HMDB51~\cite{HMDB51} (HMDB), and Something-Something V2~\cite{ssv2} (SSv2). Additionally, we evaluate our approach on two \textit{out-of-context} benchmarks: SCUBA~\cite{li2023mitigating} and HAT~\cite{chung2022enabling}.

\noindent {\bf K400 and K600} are both comprehensive video datasets for human action recognition. K400 contains 400 action categories of approximately 240k training and 20k validation videos collected from YouTube, which covers a wide range of human actions, including sports activities, daily life actions, and various interactions, serving as a widely-used action recognition dataset for pre-training. The duration of video clips in K400 varies, with most clips being around 10 seconds long. This diversity in video duration helps models learn temporal dynamics and context for action recognition. K600 extends K400 by incorporating 220 additional new categories, thus enabling the evaluation of zero-shot learning capabilities on these novel categories. 

\noindent {\bf UCF} is a human action recognition dataset collected from YouTube, and consists of 13,320 video clips, which are classified into 101 categories. These 101 categories encompass a wide range of realistic actions including body motion, human-human interactions, human-object interactions, playing musical instruments and sports. Officially, there are three splits allocating 9,537 videos for training and 3,783 videos for testing.

\noindent {\bf HMDB} is a relatively small video dataset comprising a diverse range of sources, including movies, public databases, and YouTube videos, and is composed of 6,766 videos across 51 action categories (such as ``jump'', ``kiss'' and ``laugh''), ensuring at least 101 clips within each category. The original evaluation scheme employs three distinct training/testing splits, allocating 70 clips for training and 30 clips for testing of each category in each split.

\noindent {\bf SSv2} is a temporally focused video dataset across 174 fine-grained action categories, consisting of 168,913 training videos and 24,777 testing videos showing the objects and the actions performed on them. These action categories are presented using object-agnostic templates, such as ``Dropping [something] into [something]'' containing slots (``[something]'') that serve as placeholders for objects. This dataset focuses on basic, physical concepts rather than higher-level human activities, which challenges the temporal modeling capabilities.

\noindent {\bf SCUBA} is an out-of-distribution (OOD) video benchmark designed to quantitatively evaluate static bias in the background. It comprises synthetic out-of-context videos derived from the first test split of HMDB and UCF, as well as the validation set of K400. These videos are created by superimposing action regions from one video onto diverse scenes, including those from Place365~\cite{zhou2017places} and VQGAN-CLIP~\cite{crowson2022vqgan} generated scenes. 
Due to the differences in test sets and background sources, the domain gaps of SCUBA benchmarks vary. A domain gap is defined as the ratio of accuracies between the original test sets and synthetic datasets obtained by a 2D reference network, where a higher ratio indicates a greater domain gap with respect to static features.
The UCF-SCUBA and K400-SCUBA used in our experiments consist of 4,550 and 10,190 videos with domain gaps of $20.49$ and $6.09$, respectively, whose backgrounds are replaced by the test set of Place365.

\noindent {\bf HAT} is a more ``realistic-looking'' mixed-up benchmark for quantitative evaluation of the background bias by automatically generating synthetic counterfactual validation videos with different visual cues. It provides four Action-Swap sets with distinct characteristics: \textit{Random} and \textit{Same} refer to the swap of actions and backgrounds from different and same classes, respectively, while \textit{Close} and \textit{Far} denote the swap of videos from a class with similar and very different backgrounds, respectively. The UCF-HAT benchmark used in our experiments consists of Action-Swap videos in \textit{Close} and \textit{Far} sets from 5 closest and 30 farthest action categories, respectively, following the literature~\cite{chung2022enabling}. Note that we only consider videos from the first test split of UCF where all frames have human masks taking up 5\% to 50\% of the pixels to ensure that sufficient human and background cues are present in each generated Action-Swap video.

\subsection{Evaluation Protocols}
\label{appen:EP}
For the experimental settings, we follow the previous works~\cite{weng2023open,rasheed2023fine,ni2022expanding} for in-context generalization evaluations and perform the newly proposed out-of-context generalization evaluations described below.

\noindent {\bf In-context base-to-novel generalization.}
Under this setting, we divide the entire set of action categories into two equal halves: base and novel, with the most frequently occurring classes designated as the base classes. We conduct generalization evaluations on four in-context datasets, \ie K400, HMDB, UCF and SSv2, where the models are initially trained on the base classes within the training splits of the dataset, and evaluated on both base and novel classes within the validation splits. Every training split consists of 16 video clips of each base class. During inference within HMDB and UCF datasets, only the novel class samples in the first validation splits are used for evaluation. For K400 and SSv2 datasets, the full validation split of each is used for evaluation here. We report the results of the average top-1 accuracies for both base and novel classes as well as the harmonic mean.

\noindent {\bf In-context cross-dataset generalization.}
Under this setting, the models are fine-tuned on the training set of K400, and evaluated on three in-context cross-datasets, \ie UCF, HMDB and K600. We report top-1 average accuracies with performance variances on the three validation splits in case of UCF and HMDB. For K600, the models are evaluated on non-overlapping 220 categories with K400, and we report top-1 average accuracies over three randomly sampled splits of 160 categories.

\noindent {\bf Out-of-context cross-dataset generalization.}
Under the more challenging out-of-context cross-dataset setting, the models are also trained on K400, and then evaluated on two out-of-context datasets based on UCF, \ie UCF-SCUBA and UCF-HAT. We report the top-1 and top-5 average accuracies over the synthetic counterfactual validation splits from UCF's first validation split. We further conduct the closed-set out-of-context evaluation based on the K400-SCUBA benchmark and report the harmonic mean of the accuracies under in-context and out-of-context settings to comprehensively analyze the generalization of the models.

\subsection{Implementation Details}
\label{appen:implementation}
Each training video clip is sampled with $8$ frames uniformly, and each sampled frame is spatially scaled in the shorter side to $256$ pixels and is processed with basic augmentations like color jittering, random flipping and random cropping of $224\times 224$.
We leverage multi-view inference with $3$ temporal and $1$ spatial views per video and linearly aggregate the recognition results. 
For our Gaussian Weight Average scheme, we use $\mu=7$ and $\sigma^2=10$ for in-context base-to-novel generalization and $\mu=15$ and $\sigma^2=10$ for in-context and out-of-context cross-dataset generalization.
We also adopt decision aggregation with pre-trained CLIP with the video learner for in-context evaluations.
The experiments are conducted on two computing clusters with four NVIDIA RTX 24G 4090 GPUs.


\end{document}
