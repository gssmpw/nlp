% This must be in the first 5 lines to tell arXiv to use pdfLaTeX, which is strongly recommended.
\pdfoutput=1
% In particular, the hyperref package requires pdfLaTeX in order to break URLs across lines.

\documentclass[11pt]{article}

% Change "review" to "final" to generate the final (sometimes called camera-ready) version.
% Change to "preprint" to generate a non-anonymous version with page numbers.
% \usepackage[review]{acl}
\usepackage[final]{acl}
% Standard package includes
\usepackage{times}
\usepackage{latexsym}

% For proper rendering and hyphenation of words containing Latin characters (including in bib files)
\usepackage[T1]{fontenc}
% For Vietnamese characters
% \usepackage[T5]{fontenc}
% See https://www.latex-project.org/help/documentation/encguide.pdf for other character sets

% This assumes your files are encoded as UTF8
\usepackage[utf8]{inputenc}

% This is not strictly necessary, and may be commented out,
% but it will improve the layout of the manuscript,
% and will typically save some space.
\usepackage{microtype}

% This is also not strictly necessary, and may be commented out.
% However, it will improve the aesthetics of text in
% the typewriter font.
\usepackage{inconsolata}

\usepackage{hyperref}
\usepackage{url}
\newcommand{\explain}[2]{\underbrace{#1}_{{\footnotesize\raggedright #2}}}

\usepackage[most]{tcolorbox} % 载入 tcolorbox 宏包
% for figures
\usepackage{graphicx}
\usepackage{subfigure}
\usepackage{wrapfig}

% for equation
\usepackage{amsmath}
\usepackage{amstext}
\usepackage{amsfonts}
\usepackage{bm}

\usepackage{bbm}
% for table
\usepackage{multirow}
\usepackage{booktabs}
\usepackage{array}
\usepackage{caption}
\usepackage{multirow}
\usepackage{booktabs}
\usepackage{array}
\usepackage{caption}
\usepackage{color}
\usepackage{colortbl}
\usepackage{tablefootnote}
\usepackage{adjustbox}

% for algorithm
\usepackage{caption}
\usepackage{algorithm}
\usepackage{algpseudocode}
\newcommand{\mycolor}[1]{\textcolor[RGB]{64,101,149}{#1}}
\newcommand{\mydarkcolor}[1]{\textcolor[RGB]{64,101,149}{#1}}
\algnewcommand{\LineComment}[1]{\Statex ~~~~~~\textsc{//}~\textit{#1}}

%for itemize
\usepackage{enumitem}
\setenumerate[1]{itemsep=0pt,partopsep=0pt,parsep=\parskip,topsep=5pt}
\setitemize[1]{itemsep=0pt,partopsep=0pt,parsep=\parskip,topsep=5pt}
\setdescription{itemsep=0pt,partopsep=0pt,parsep=\parskip,topsep=5pt}

% for highlight
\usepackage{soul}

% for taxonomy
\usepackage{tikz}
\usepackage[edges]{forest}
\definecolor{hidden-draw}{RGB}{64,101,149}
% \definecolor{hidden-pink}{RGB}{255,245,247}
\definecolor{hidden-pink}{RGB}{231,239,250}

% for notation
\usepackage[mathscr]{euscript}
\newcommand{\dataset}{\textsc{StripCipher}\xspace}

%for itemize
\usepackage{amssymb}  
\usepackage{pifont}
\newcommand{\cmark}{\ding{51}}
\newcommand{\xmark}{\ding{55}}
\newcommand{\greenyes}{\textcolor{green}{\ding{51}}}
\newcommand{\redno}{\textcolor{red}{\ding{55}}}
\newcommand{\cpm}[1]{\textcolor{gray}{$_{\pm #1}$}}

% for logo
\usepackage{scalerel}
\usepackage{graphicx}  % 使得表格可以自动调整
\usepackage{array}     % 允许使用 p{} 设定列宽
\usepackage{booktabs}  % 提供更好的表格格式
\usepackage{ragged2e}  % 使得文本可以左对齐且换行


% for title
\usepackage{xspace}

% If the title and author information does not fit in the area allocated, uncomment the following
%
%\setlength\titlebox{<dim>}
%
% and set <dim> to something 5cm or larger.

% From Still to Motion: Evaluating Large Multimodal Models' Understanding of Comic Strips
% Cracking the Code of Visual Narratives: Can LMMs Uncover Implicit Meanings Behind Comic Strips?
% Cracking the Code of Visual Narratives: Can LMMs Uncover Implicit Meanings Behind Image Sequences?

% \scalerel*{\includegraphics{fig/comic.png}}{{\rule{3ex}{3ex}}}
% \texorpdfstring{\includegraphics[width=18pt]{fig/comic.png}}{}
\title{
\texorpdfstring{\includegraphics[width=18pt]{fig/comic.png}}{}
\textit{Beyond Single Frames:} Can LMMs Comprehend  Temporal and Contextual Narratives in Image Sequences?
}

% \author{\textbf{Xiaochen Wang}\textsuperscript{1,$*$} \textbf{Heming Xia}\textsuperscript{2,\thanks{equal contribution}}, \textbf{Jialin Song}\textsuperscript{1,$\diamond$}, \textbf{Longyu Guan}\textsuperscript{1,$\diamond$}, \textbf{Yixin Yang}\textsuperscript{1}, \textbf{Qingxiu Dong}\textsuperscript{1}, \\
% \textbf{Weiyao Luo}\textsuperscript{1}, \textbf{Yiru Wang}\textsuperscript{4}, \textbf{Yifan Pu}\textsuperscript{3}, \textbf{Xiangdi Meng}\textsuperscript{1}, \textbf{Wenjie Li}\textsuperscript{2}, \textbf{Zhifang Sui}\textsuperscript{ 1,\thanks{Corresponding Author}}\\
% \textsuperscript{1} State Key Laboratory of Multimedia Information Processing, Peking University\\
%   \textsuperscript{2} Department of Computing, The Hong Kong Polytechnic University \\ 
%    \textsuperscript{3} Tsinghua University      \textsuperscript{4} ModelTC \\ 
%    }

\author{%
\textbf{Xiaochen Wang}\textsuperscript{1 $*$},
\textbf{Heming Xia}\textsuperscript{2 $*$},
\textbf{Jialin Song}\textsuperscript{1 $\dagger$},
\textbf{Longyu Guan}\textsuperscript{1 $\dagger$},
\textbf{Yixin Yang}\textsuperscript{1},
\textbf{Qingxiu Dong}\textsuperscript{1}, \\
\textbf{Weiyao Luo}\textsuperscript{1}, 
\textbf{Yiru Wang}\textsuperscript{4}, 
\textbf{Yifan Pu}\textsuperscript{3}, 
\textbf{Xiangdi Meng}\textsuperscript{1}, 
\textbf{Wenjie Li}\textsuperscript{2}, 
\textbf{Zhifang Sui}\textsuperscript{1 $\ddagger$} \\
\textsuperscript{1} State Key Laboratory of Multimedia Information Processing, Peking University\\
\textsuperscript{2} Department of Computing, The Hong Kong Polytechnic University\\
\textsuperscript{3} Tsinghua University \quad
\textsuperscript{4} ModelTC \\
  \texttt{ wangxiaochen@stu.pku.edu.cn}
}



\begin{document}
\maketitle

\renewcommand{\thefootnote}{\fnsymbol{footnote}}
% \footnotetext{\textsuperscript{$*$} Equal first contribution. \textsuperscript{$\dagger$} Equal second contribution.} 
% \footnotetext{\textsuperscript{$\ddagger$} Corresponding Author.} 
\footnotetext{\textsuperscript{$*$} Equal first contribution. } 

\footnotetext{\textsuperscript{$\dagger$} Equal second contribution.}
\footnotetext{\textsuperscript{$\ddagger$} Corresponding author.} 

% \footnotetext[1]{Equal first contribution.} \footnotetext[2]{Equal second contribution.}
% \footnotetext[3]{Corresponding Author.}
% \renewcommand{\thefootnote}{\arabic{footnote}}

\begin{abstract}
Large Multimodal Models (LMMs) have achieved remarkable success across various visual-language tasks. However, existing benchmarks predominantly focus on single-image understanding, leaving the analysis of image sequences largely unexplored. To address this limitation, we introduce \dataset, a comprehensive benchmark designed to evaluate capabilities of LMMs to comprehend and reason over sequential images. \dataset comprises a human-annotated dataset and three challenging subtasks: visual narrative comprehension, contextual frame prediction, and temporal narrative reordering. Our evaluation of $16$ state-of-the-art LMMs, including GPT-4o and Qwen2.5VL, reveals a significant performance gap compared to human capabilities, particularly in tasks that require reordering shuffled sequential images. For instance, GPT-4o achieves only $23.93\%$ accuracy in the reordering subtask, which is $56.07\%$ lower than human performance. Further quantitative analysis discuss several factors, such as input format of images, affecting LMMs’ performance in sequential understanding, underscoring the fundamental challenges that remain in the development of LMMs.   
\end{abstract}

\section{Introduction}

Chain-of-Thought (CoT) prompting~\cite{Nye:2021, cot, Kojima:2022cotzero} has emerged as a cornerstone strategy for enhancing Large Language Models (LLMs) in complex reasoning tasks. By eliciting step-by-step inference, CoT enables LLMs to decompose intricate problems into manageable subtasks, thereby improving their problem-solving performance~\cite{Yao:2023tot, Wang:2023self-consistency, Zhou:2023least, Shinn:2023Reflexion}. Recent advancements, such as OpenAI's o1~\cite{o1} and DeepSeek-R1~\cite{deepseekr1}, further demonstrate that scaling up CoT lengths from hundreds to thousands of reasoning steps could continuously improve LLM reasoning. These breakthroughs have underscored CoT’s potential to advance LLM capabilities, expanding the boundaries of AI-driven problem-solving.

\begin{figure}[t]
\centering
    \includegraphics[width=0.95\columnwidth]{fig/intro.pdf}
    \caption{In contrast to vanilla CoT that generates all reasoning tokens sequentially, \method enables LLMs to \textit{skip} tokens with less semantic importance (\textit{e.g.,} \includegraphics[width=7pt]{fig/token.pdf}~) and learn shortcuts between critical reasoning tokens, facilitating controllable CoT compression.}
    \label{fig:intro}
\end{figure}

Despite its effectiveness, the increased length of CoT sequences introduces substantial computational overhead. Due to the autoregressive nature of LLM decoding, longer CoT outputs lead to proportional increases in both inference latency and memory footprints of key-value cache. Additionally, the quadratic computational cost of attention layers further exacerbates this burden. These issues become particularly pronounced when CoT sequences extend into thousands of reasoning steps, resulting in significant computational costs and prolonged response times. While prior research has explored methods for selectively skipping reasoning steps~\cite{Ding:2024cotshortcut, liu2024skipstep}, recent findings~\cite{jin:2024cotlength, Merrill:2024cotlength} suggest that such reductions may conflict with test-time scaling~\cite{o1-blog, snell2025scaling}, ultimately impairing LLM reasoning performance. Therefore, striking an optimal balance between CoT efficiency and reasoning accuracy remains a critical open challenge.

In this work, we delve into CoT efficiency and seek the answer to an important question: \textit{``Does every token in the CoT output contribute equally to deriving the answer?''} We empirically analyze the semantic importance of tokens within CoT outputs and reveal that their contributions to the reasoning performance vary, as depicted in Figure 2. Building on this insight, we introduce \method, a simple yet effective approach that enables LLMs to \textit{skip} less important tokens within CoT sequences and learn shortcuts between critical reasoning tokens, thereby allowing for controllable CoT compression with adjustable ratios. Specifically, as shown in Figure~\ref{fig:intro}, \method constructs compressed CoT training data with various compression ratios, by pruning unimportance tokens from original LLM CoT trajectories. Then, it conducts a general supervised fine-tuning process on target LLMs with this training data, facilitating LLMs to automatically trim redundant tokens during reasoning.

We conduct extensive experiments across various models, including LLaMA-3.1-8B-Instruct and the Qwen2.5-Instruct series, using two widely recognized math reasoning benchmarks: GSM8K and MATH-500. The results validate the effectiveness of \method in compressing CoT outputs while maintaining robust reasoning performance. Notably, Qwen2.5-14B-Instruct exhibits almost \textbf{NO} performance drop (less than $0.4\%$) with a $\bm{40\%}$ reduction in token usage on GSM8K. On the challenging MATH-500 dataset, LLaMA-3.1-8B-Instruct effectively reduces CoT token usage by $\bm{30}\%$ with a performance decline of less than $4\%$, resulting in a $\bm{1.4}\times$ inference speedup. Further analysis underscores the coherence of \method in specified compression ratios and its potential scalability with stronger compression techniques.

\method is distinguished by its low training cost. For Qwen2.5-14B-Instruct, \method fine-tunes only 0.2\% of the model's parameters using LoRA. The size of the compressed CoT training data is no larger than that of the original training set, with 7,473 examples in GSM8K and 7,500 in MATH. The training is completed in approximately 2 hours for the 7B model and 2.5 hours for the 14B model on two 3090 GPUs. These characteristics make \method an efficient and reproducible approach, suitable for use in efficient and cost-effective LLM deployment.

To sum up, our key contributions are:
\begin{enumerate}
    \item To the best of our knowledge, this work is the \textit{first} to investigate the potential of enhancing CoT efficiency through \textit{token skipping}, inspired by the varying semantic importance of tokens in CoT trajectories of LLMs.
    \item We introduce \method, a simple yet effective approach that enables LLMs to skip redundant tokens within CoTs and learn shortcuts between critical tokens, facilitating CoT compression with adjustable ratios.
    \item Our experiments validate the effectiveness of \method. When applied to Qwen2.5-14B-Instruct, \method reduces reasoning tokens by $40\%$ (from 313 to 181) on GSM8K, with less than a $0.4\%$ performance drop.
\end{enumerate}

\section{Related Work}
Researchers have been leveraging eye tracking methodologies from human perception research to model how people perceive images~\cite{shanmuga2015eye, bonhage2015combined, conklin2016using}.
These models help assess the appearance and salience of visual representations, enabling eye movement tracking to understand the perceptual and cognitive mechanisms of scene perception~\cite{itti1998model} and object detection~\cite{borji2015salient}.
The existing saliency models perform well in naturalistic scenes
%and real-world object detection
; however, there are unique perception rules and cognitive biases in the artificial world of data visualization 
%does not always follow the rules of perception in the natural world
~\cite{franconeri2021science, correll2012comparing, polatsek2018exploring, knittel2024gridlines}, and, thus, these models do not accurately predict where people would look in visualizations. 
Visualization researchers have been building visual saliency models geared to visualizations~\cite{DVSaliencyModel2017Matzen, bylinskii2016should}. %and adopting them for predicting eye gaze on visualizations. %enabling the prediction of visual saliency across design styles~\cite{fosco2020predicting}.
However, these models rely on handcrafted features, making it difficult to generalize to complex visualizations. Additionally, these models cannot incorporate textual information to generate task-specific saliency maps since the prediction is solely based on visual inputs.

With the advent of deep learning, gaze data were used as the ground truth of saliency models~\cite{fosco2020predicting, scannerDeeply, scanpath}, leading to higher performance in saliency prediction while enabling task-specific saliency~\cite{salchartQA}. 
These models usually need large-scale datasets to learn complex patterns. However, gathering precise gaze data is 
%challenging and requires specialized eye-tracking devices. While these devices provide accurate results, they tend to be 
costly and cumbersome, which limits large-scale data collection efforts. 
Many researchers, therefore, proposed several proxies for eye gaze. WebGaze~\cite{webgaze} uses a webcam for cheap and easy deployment in online studies yet suffers from data quality issues due to low-resolution cameras and uncontrolled calibration.
Therefore, mouse-(cursor-)based annotation tools~\cite{jiang2015salicon,bubbleView,importAnnot} were proposed to improve data quality. Among these methods, BubbleView~\cite{bubbleView} was the most used tool for capturing visual saliency and importance~\cite{graphicDesignImportance, salchartQA}.
However, BubbleView is primarily designed for exploring images and gathering information, which differs slightly from the goal of capturing perceived importance. As a result, while BubbleView is well-suited for measuring visual saliency, it may not be the best tool for capturing %instruction-tuned \yao{I would keep it consistent saying task-specific}
task-specific importance~\cite{turkeyes}. Built upon these prior approaches' limitations, our Grid Labeling aims to collect responses that cover all essential areas of the visualization with minimum noise, leading to more efficient data collection.



% One key motivation to our grid-based approach is to help people 
% We also demonstrate that the grid-based approaches can minimize biases in annotation to disproportionally emphasize text elements~\cite{DVSaliencyModel2017Matzen}
% % blurring the visualization can disproportionately emphasize text elements~\cite{DVSaliencyModel2017Matzen}, potentially misrepresenting a user's true areas of interest.
% More recently, 
% % \ms{Changed a bit using Yao's work (task-dependent saliency), but not sure whether it looks ok}
% Yao et al.~\cite{salchartQA} collect task-dependent saliency using the BubbleView method, % but their approach had some limitations. 
% and made a significant improvement on existing saliency models.
% First, the blurred visualization allowed users to perceive the overall structure of the chart, which prevented the system from capturing the specific action of identifying the maximum value. However, increasing the blur to address this issue introduced another challenge. As the structure became less visible, users had to explore the entire image, leading to the consideration of irrelevant regions as salient.
\section{Dataset and Task Overview}
To investigate the capabilities of LMMs to comprehend sequential images, we introduce \dataset, a novel benchmark consisting of three subtasks:

\begin{itemize}
     \item \textbf{Visual Narrative Comprehension:} Examines whether models accurately interpret the narrative content of image sequences.
      \item \textbf{Contextual Frame Prediction:} Assesses the model reasoning ability to predict missing frames in image sequences based contextual.
      \item \textbf{Temporal narrative Reordering:} Evaluates whether models correctly infer and restore the chronological order of image sequences based causal temporal relationship.
\end{itemize}

The instructions for three subtasks are presented in Table~\ref{prompt}. These subtasks provide a rigorous and multifaceted assessment of LMMs, offering insights into their strengths and limitations in sequential image understanding. In the following, we may use their full names or refer to them as \textit{comprehension}, \textit{prediction}, and \textit{reordering} for simplicity.

\begin{tcolorbox}[title={The Prompt used for Translation}]
You are a highly skilled translator tasked with translating various types of content from English into \{\{ language \}\}. Follow these instructions carefully to complete the translation task.

You will receive a user-bot conversation in XML format. Please follow a three-step translation process:

\begin{enumerate}
  \item \textbf{Initial Translation:} Translate the input content into \{\{ language \}\}, preserving the original intent and keeping the original paragraph and text format unchanged. Do not delete or omit any content, and ensure that all original Markdown elements (e.g., images, code blocks) are preserved.
  \item \textbf{Reflection and Feedback:} Carefully review both the source text and your translation. Provide constructive criticism and specific suggestions to improve the translation in terms of:
    \begin{enumerate}[label=(\roman*)]
      \item \textbf{Accuracy:} Correct errors of addition, mistranslation, omission, or untranslated text.
      \item \textbf{Fluency:} Apply \{\{ language \}\} grammar, spelling, and punctuation rules while avoiding unnecessary repetitions.
      \item \textbf{Style:} Ensure that the translation reflects the style of the source text and considers any relevant cultural context.
    \end{enumerate}
  \item \textbf{Refinement:} Based on your reflections, refine and polish your translation.
  \item \textbf{Fallback:} If you are not confident in translating the conversation, please return ``\texttt{<stop></stop>}''.
\end{enumerate}

\bigskip
\textbf{Output:}

For each step of the translation process, output your results within the appropriate XML tags as follows:
\begin{verbatim}
<step1_initial_translation>
[Insert your initial translation here]
</step1_initial_translation>

<step2_reflection>
[Insert your reflection on the translation, including a list 
of specific, helpful, and constructive suggestions for 
improvement. Each suggestion should address a specific 
part of the translation.]
</step2_reflection>

<step3_refined_translation>
[Insert your refined and polished translation here]
</step3_refined_translation>
\end{verbatim}

Ensure that your final translation in step 3 accurately reflects the original meaning while sounding natural in \{\{ language \}\}.

Here is the original conversation:
\label{box:trans_prompt}
\end{tcolorbox}

% \vspace{-5mm}
% \begin{figure*}[htbp]
%     % 左侧图片
%     \begin{minipage}{0.77\linewidth}  % 调整宽度
%         \centering
%         \includegraphics[width=\linewidth]{images/benchmark_construction.pdf}
%     \end{minipage}%
%     % 间隔
%     \hfill
%     % 右侧表格
%     \begin{minipage}{0.23\linewidth}  % 调整宽度
%         \centering
%         \resizebox{\linewidth}{!}{  % 调整表格至合适的宽度
%             \begin{tabular}{lcc}
%                 \toprule
%                 \textbf{Statistic} & \textbf{Number} \\
%                 \midrule
%                 \rowcolor[HTML]{F2F2F2} 
%                 \textit{Domain Count} &  \\
%                 \midrule
%                 Domain & 103 \\
%                 Requirement & 8 \\
%                 \midrule
%                 \rowcolor[HTML]{F2F2F2} 
%                 \textit{Token Count} &  \\
%                 \midrule
%                 Description & 851.6 $\pm$ 515.2 \\
%                 - Min/Max & [159, 2814] \\
%                 Domain & 1187.2 $\pm$ 1212.1 \\
%                 - Min/Max & [85, 7514] \\
%                 \midrule
%                 \rowcolor[HTML]{F2F2F2} 
%                 \textit{Line Count} &  \\
%                 \midrule
%                 Domain & 75.4 $\pm$ 62.9 \\
%                 - Min/Max & [9, 394] \\
%                 \midrule
%                 \rowcolor[HTML]{F2F2F2} 
%                 \textit{Component Count} &  \\
%                 \midrule
%                 Actions & 4.5 $\pm$ 2.8 \\
%                 - Min/Max & [1, 16] \\
%                 Predicates & 8.1 $\pm$ 4.8 \\
%                 - Min/Max & [1, 25] \\
%                 Types & 1.1 $\pm$ 1.3 \\
%                 - Min/Max & [1, 8] \\
%                 \bottomrule
%             \end{tabular}
%         }
%     \end{minipage}
%     % 公共标题
%     \caption{Dataset construction process (left) and key statistics (right) of the \texttt{\benchmark} dataset.     Dataset construction process including: (a) \textit{Data Acquisition} (\S\ref{sec:data_acquisition}); (b) \textit{Data Filtering and Manual Selection} (\S\ref{sec:data_filtering}); (c) \textit{Data Annotation and Quality Assurance}(\S\ref{sec:data_annotation} and \S\ref{sec:quality_assurance}). Tokens are counted by GPT-2~\cite{openai2019gpt2} tokenizer.}
%     \label{fig:combined}
% \end{figure*}


Detailed statistics is displayed on Table~\ref{tab:statistics}.
Overall, our proposed \dataset includes $896$ image sequences, with an average frame length of $4.09$ of each sequence. The number of frames ranges from 3 to 8. Each task is designed in the form of multiple-choice questions, except for the reorder task, which also includes a question-answering format. Since its options are simple and well-defined, accuracy can be directly computed. In our tasks, the input format uniformly consists of images paired with textual prompts. Specifically, the comprehension task utilizes the whole images without split, the reordering task takes shuffled image sequence as input,  and the frame prediction task involves masking second-to-last frame within the image sequence.
For the frame prediction task, we select the second-to-last frame, as it typically serves as a bridge between the preceding frames and the final frame. The start and end frames are generally more challenging to predict.


% 上面写具体的statistics,比如task A每个sample几个选项,task B & task C,平均的选项长度是多少。每个task的数据格式是怎么样的,比如reorder input就是打乱的图片,comprehension是顺序图片,frmae prediction我们会mask某一帧的图片。可以在这里讲我们是mask倒数第二帧以及为什么。

% 最后一帧一般是脑洞大开的反转,很难预测,所以我们选择了他的前一帧进行预测,并且只筛选了frmae数大于等于4的连环画。但是有些倒数第二帧的内容是有多种可能的,我们在人类标注员check的时候删除了那些倒数第二帧不唯一的图。

% Reference:

%The \method{} dataset includes 1,001 samples, each with an image and three manually annotated components: a description, a title, and deep semantics. The statistical information about the text is displayed in Table \ref{tab:statistics}. To enable quantitative evaluation, we additionally craft multiple-choice questions to test the understanding of descriptions, titles, and deep semantics. Each segment is represented by 1,001 questions, where each question presents an image, a question text, and four potential answers. Only one answer is correct, while the others serve as distractors. Figure~\ref{fig:fig_sim} illustrates examples of the manually annotated components and the multiple-choice questions.

% We describe the proposed PMR task with an example in Figure 2 I&II. Given a source image and a textual premise, the inference model should perceive and understand the image in combination with the premise so as to choose the exclusive correct action among the four hypothetical candidates. The premise would serve as the background knowledge or domain-specific commonsense for the given image. In the running example, the model should be able to recognize what [person2] wears from the image and infer whether [person4] would give his seat to [person2] under the premise “[person4] is very friendly”. The corrected answer is ‘C’ according to the visual and textual clues. In total, we collect about 15k instances for PMR. We list the statistics for PMR in Table 2.
\section{Dataset Construction}



% Image Source
% Prerequisite: Annotator Training
% (1) AI-assisted Data Annotation for Answers & Distractor Generation
% (2) Cross-Check Examination
% (3) Subtask Composition
% (4) Dataset Evaluation

We construct our \dataset dataset in a multi-step crowd-sourcing pipeline, including 1) annotator training, 2) data annotation, and 3) cross-check examination. An overall demonstration of our dataset construction pipeline is illustrated in Figure ~\ref{fig:construct}. 

\subsection{Image Source}

We use silent comic strips, comic with panels and no dialogue, as our primary data source. As a distinct art form, comic strips often encapsulate complex narratives within concise visual sequences, addressing deeper themes such as social satire, humor, and inspiration. These characteristics make comic strips a particularly challenging medium for evaluating ability of LMMs to understand visual sequences. 
The dataset comprises samples from well-known comics, such as \textit{Father and Son} and \textit{Peanuts}, along with web-scraped images from \texttt{GoComics} \footnote{https://www.gocomics.com/}, \texttt{Google}, and \texttt{Facebook} \footnote{https://www.facebook.com/}. 
Initially, we collected $1,260$ images and then refined the dataset through a filtering process. We conducted a thorough manual inspection to eliminate unclear, toxic, overly simplistic images, along withmulti-panel comics lacking a clear temporal sequence. Moreover, comics with dialogue will also be removed to prevent the model from using OCR to understand the meaning of the comics through text rather than through images. As a result, the final dataset was reduced to 896 images.

% \subsection{Prerequisite: Annotator Training}
% We posted job descriptions on online forums and received over 50 applications from candidates with at least a Bachelor's degree. To ensure dataset quality, we provided training sessions that included online pre-annotation instructions and a qualification test to assess candidates' performance. Only those scoring above 95\% were selected. candidates were assigned to one of two groups: annotators or inspectors. Ultimately, we hired 13 annotators and 7 inspectors for our data annotation process.

% During the data collection process, we primarily rely on the powerful GPT-4o model to abtain annotation, combined with multiple rounds of human review.

\subsection{Phase 1: Data Annotation.}

\paragraph{Annotator Training}
We posted job descriptions on online forums and received over 50 applications from candidates with at least a Bachelor's degree. To ensure dataset quality, we provided training sessions that included online pre-annotation instructions and a qualification test to assess candidates' performance. Only those scoring above $95\%$ were selected. candidates were assigned to one of two groups: annotators or inspectors. Ultimately, we hired $13$ annotators and $7$ inspectors for our data annotation process.
To optimize efficiency and reduce costs, we implement a semi-automated pipeline for \dataset annotation, leveraging \texttt{GPT-4o}\footnote{We use the \texttt{gpt-4o-2024-11-20} version for the data annotation process and subsequent evaluations in this work.}. Specifically, our data annotation process consists of two substeps: \textit{answer creation} and \textit{distractor generation}.

\paragraph{Answer Creation.}
The bottom panel of Figure~\ref{fig:construct} illustrates the process of our answer creation phase. Notably, only the comprehension task and frame prediction task need option annotation. The reordering task only necessitates using a program to randomly shuffle the frame order as the answer order, without the need for manual selection of the correct answer. We adopt an AI-assisted annotation approach in which human annotators refine pre-generated answers instead of creating them from scratch. Initially, we leverage \texttt{GPT-4o, GPT-4o-Mini} \cite{hurst2024gpt40} and Gemini \cite{reid2024gemini} to generate diverse candidate answers. Human annotators then evaluate the image sequence and candidate answers, selecting the most appropriate ground truth. If none of the candidate answers is suitable, annotators are instructed to either refine a specific answer or create a new ground truth from scratch, which is about $28\%$. 

\paragraph{Distractor Generation.}
We use candidates with plausible hallucinations from the previous sub-step as strong distractors.  To ensure diversity in the multiple-choice options, we also prompt \texttt{GPT-4o, GPT-4o-mini} \cite{hurst2024gpt40} to generate intentionally incorrect responses as weak distractors. 
Typically, the responses of \texttt{GPT-4o-mini}  are not very accurate and are mostly used as distractors. A detailed example of model annotation can be found in the appendix~\ref{appendix:annotation}
Annotators are instructed to evaluate the quality of these distractors and select top-3 options, refining them if necessary. Finally, each ground truth is paired with three high-quality distractors for evaluation.

%We carefully curated a new set of challenging distractors that:;Maintain semantic relevance to the comic content;Share similar visual elements or themes with the correct answer;Avoid obvious logical contradictions;Represent plausible but incorrect interpretations

% To ensure annotation quality, annotators are provided with detailed examples and rejection guidelines. 

% The annotation interface and full specifications are provided in Appendix{ \color{red}X[TODO].}

% 对于理解任务,我们在采样4o的回答作为标注的时候发现模型经常出现过度解读升华漫画的情况。所以一些非常简单无深意的图片会被删除,此外人类标注员在check的时候会删除过度解读。

% The annotation process concluded with two rounds of cross-verification to ensure answer accuracy, maintain stylistic consistency, and minimize potential biases.


\subsection{Phase 2: Cross-Check Examination}
We implement a cross-check examination mechanism to ensure rigorous screening of high-quality annotations. During the data annotation process, hired inspectors review the annotated data and corresponding image sequences. If they encounter low-quality annotations, they have the option to reject them. Each annotation is reviewed by two inspectors. If both inspectors reject the annotation, it is discarded, and the image is returned to the dataset for re-annotation. If an image sequence is rejected in two rounds of annotation, it suggests that this sample is not suitable for the current subtask (e.g., the meaning of the sample is unclear), and the image is subsequently removed from the subtask. 

% During this phase, we also use Cohen's kappa to quantify inter-annotator agreement, obtaining an average score of { \color{red}0.701} across all tasks, indicating substantial agreement.

After annotation, both advanced annotators and inspectors, acting as final examiners, review the annotations to ensure they meet the required standards. Each annotation undergoes review by three examiners, who vote on whether to accept the annotated sample. Only the samples that receive a majority vote are approved. To ensure the quality of the examiners' work, we randomly sample 10\% of the annotations for verification. 

% This review is conducted by the authors of this paper, who provide timely feedback to the examiners. Further details on the pricing strategy and the hierarchical supervision process are provided in Appendix B.



\subsection{Data Composition}
It is important to note that the reordering subtask does not require human annotation, as described in the previous process. For this subtask, we select suitable image sequences based on the criterion that the correct ordering must be unique. To ensure this, we conduct a manual review to verify that each sequence follows a logically unambiguous order. A script is then run to perform the initial splitting of panels within specific comics, followed by a random shuffling of these panels. Human annotators are tasked with verifying the format and quality of the frames to ensure they meet the required standards. These processed image sequences serve as the evaluation data for the reordering subtask.

The final version of our 32-day annotated \dataset contains 896 items (see Table 2), encompassing three subtasks: \textit{visual narrative comprehension}, \textit{contextual frame prediction}, and \textit{temporal narrative reordering}. In each of these subtasks, each sample consists of an image sequence paired with a multiple-choice question offering four options. The evaluated LMMs are required to select the option they deem most appropriate from the four. More information and examples of \dataset can be found in Appendix~\ref{example}. 

    
    
\begin{table*}[t!]
    \centering
    \small
    
    \scalebox{0.90}{
    \setlength{\tabcolsep}{1.0pt}
    \begin{tabular}{l c c c r | c c c c c c |c  c c }
    \toprule
    \multirow{1}{*}{Method} & \multirow{1}{*}{Recipe} & \multirow{1}{*}{Complexity} & \multirow{1}{*}{\# P.} & \multirow{1}{*}{\# T.P.}& MME & MMB &POPE & \multicolumn{1}{c} {SEED} & MMMU & MM-Vet& TQA & SQA-I  & \multicolumn{1}{c}{GQA} \\
    \midrule
    \rowcolor{gray!14}
    \multicolumn{14}{l}{\textbf{\textit{Encoder-based VLMs}}} \\ 
    OpenFlamingo~\cite{openflamingo} & \underline{PT, SFT}& Quadratic & 9B& 96.6\%  & - & 4.6 & - & - & - & - & 33.6 & - & - \\
    MiniGPT-4~\cite{minigpt} & \underline{PT, SFT}& Quadratic & 13B& 94.8\%  & 581.7 & 23.0 & - & - & -& 22.1 & - & - & 32.2  \\
    Qwen-VL~\cite{qwenvl} & \underline{PT, SFT}& Quadratic & 7B& 100.0\%  & - & 38.2 & - & 56.3 & - & - & 63.8 & 67.1 & 59.3\\ 
    LLaVA-Phi~\cite{llavaphi}  & \underline{PT, SFT}& Quadratic & 3B& 90.0\%  & 1335.1 & 59.8 & 85.0 & - & - & 28.9& 48.6 & 68.4 & - \\
    MobileVLM-3B~\cite{mobilevlm} & \underline{PT, SFT}& Quadratic & 3B& 90.0\%  & 1288.9 & 59.6 & 84.9 & - & - & - & 47.5 & 61.0 & 59.0  \\
    VisualRWKV~\cite{visualrwkv} & \underline{PT, SFT}&  \textbf{Linear} & 3B& 90.0\%  & 1369.2 & 59.5 & 83.1 & - & - & - & 48.7 & 65.3 & 59.6 \\
    VL-Mamba~\cite{vlmamba} & \underline{PT, SFT}&  \textbf{Linear} & 3B& 90.0\%  & 1369.6 & 57.0 & 84.4 & - & -& 32.6 & 48.9 & 65.4 & 56.2 \\
    Cobra~\cite{cobra} & \underline{PT, SFT}&  \textbf{Linear} & 3.5B& 82.6\%  & - & - & \textbf{88.4} & - & - & - & 58.2 & - & \textbf{62.3}\\
    \midrule
    \rowcolor{gray!14}
    \multicolumn{14}{l}{\textbf{\textit{Decoder-only VLMs}}} \\
    Fuyu-8B (HD)~\cite{fuyu} & \underline{PT, SFT}& Quadratic & 8B& 100.0\%  & 728.6 & 10.7 & 74.1 & - & - & 21.4 & - & - & -\\
    SOLO~\cite{solo} & \underline{PT, SFT}& Quadratic &  7B& 100.0\%   & 1001.3 & - & - & 64.4 & - & - & - & 73.3 & -   \\    
    Chameleon-7B~\cite{chameleon}  & \underline{PT, SFT}& Quadratic &  7B& 100.0\%   & 170 & 31.1 & - & 30.6 & 25.4 & 8.3 & 4.8 & 47.2 & -\\  
    EVE-7B~\cite{eve}  & \underline{PT, SFT}& Quadratic &  7B& 100.0\%  & 1217.3 & 49.5 & 83.6 & 61.3 & \underline{32.3} & 25.6& 51.9 & 63.0 & 60.8 \\
    Emu3~\cite{emu3} & \underline{PT, SFT}& Quadratic & 8B& 100.0\%  & - & 58.5 & 85.2 & \underline{68.2} & 31.6 & \underline{37.2} & \underline{64.7} & \underline{89.2} & 60.3\\
    HoVLE~\cite{hovle} & DT, PT, SFT & Quadratic & \textbf{2.6B}& 100.0\%  & \textbf{1433.5} & \textbf{71.9} & \underline{87.6} & \textbf{70.7} & \textbf{33.7} & \textbf{44.3} & \textbf{66.0} & \textbf{94.8} & \underline{60.9} \\
    \rowcolor{green!15}
    \name{} & \textbf{DT} & \textbf{Linear} & \underline{2.7B}& \underline{14.7\%}  &1303.5 & 57.2 & 85.2 & 62.9& 30.7  & 31.1 &47.7 & 79.2 & 57.4 \\
    \rowcolor{yellow!15}
    \name{} & \textbf{DT} & \underline{Hybrid} & \underline{2.7B}& \textbf{11.2\%}  & \underline{1371.1} & \underline{63.7} & 86.7 & 66.3 & \underline{32.3} & 36.9 & 55.1 & 86.9 & 59.3  \\
    
    \bottomrule
    \end{tabular}
    }
    \vspace{-1em}
    \caption{\textbf{Comparison with existing VLMs on general VLM benchmarks.} ``Recipe'' denotes the adopted training recipe. ``PT'', ``SFT'', and ``DT'' denote the pre-training, supervised fine-tuning, and distillation training, respectively. ``Complexity'' denotes the model computation complexity with respect to the number of tokens. ``\# P.'' denotes the number of total parameters. ``\# T.P.'' denotes the percentage of trainable parameters ($\frac{\text{trainable paramters}}{\text{total parameters}}$). The best performance is highlighted in \textbf{bold} and the second-best result is \underline{underlined}.}
    \label{tab:results_general}
    \end{table*}



\section{Experiments}

\subsection{Models}
To comprehensively evaluate on LMMs, we conducted zero-shot inference across both commercial and open-source models. Our evaluation suite includes leading commercial models GPT-4o~\cite{hurst2024gpt40} and Gemini1.5-Pro~\cite{Gemini} alongside state-of-the-art open-source alternatives of varying scales: Qwen2.5-VL~\cite{qwen2.5-VL}, Qwen2-VL~\cite{wang2024qwen2}, LLaVA-v1.6~\cite{liu2023llava}, CogVLM~\cite{wang2023cogvlm}, MiniCPM-o-2.6~\cite{yao2024minicpm}, mPlug-Owl2~\cite{ye2023mplugowl2}, InternVL2v5~\cite{chen2024internvl},LLaVA-NEXT-Video~\cite{zhang2024llavanextvideo} and Cambrian~\cite{tong2024cambrian1}. Besides, Janus-Pro~\cite{chen2025januspro}, which unifies multimodal understanding and generation, is included to test the abilities between Unified Model and Vision Language Model.  This diverse selection enables us to analyze how model scale, architecture, and training approaches influence comic comprehensive capabilities. 


% Detailed specifications and inference configurations for each model are provided in Appendix~\ref{appendix:model}.

\subsection{Experimental Details}
% 隐含含义理解和预测帧内容这两个任务都是选择题,因为他们的标准答案是不唯一的很难衡量。如果模型选择了正确的答案选项,则认为是正确的,也就是说accuracy是主要的metric。 而对于排序任务,这一个任务形式非常新颖,对于一个comic strip,其输入顺序可以随便打乱,且答案是确切的。所以这个任务我们既采用了选择题的题型又使用了问答题。
The task prompts is displayed in Table ~\ref{prompt}. For visual narrative comprehension task, model is provided with the whole image. But for next-frame prediction and multi-frame sequence reordering task, LMMs infer with image sequences.
The hyper-parameters for each LMMs in the experiments including possible settings are detailed in Appendix~\ref{appendix:hyper-param}. Furthermore, to assess human capabilities in these tasks, we randomly select 100 questions from the dataset for each task and instruct human evaluators to answer. This allows us to benchmark the performance of human participants against our models, offering a thorough comparison of both human and LMMs proficiency in these specific tasks. 


\subsection{Main Results}
Our comprehensive evaluation reveals that while LMMs show promising capabilities in comprehension and prediction tasks, they significantly underperformed in sequence reordering tasks. Moreover, there remains a substantial performance gap between current models and human performance across all tasks. Unified Model underperformed than Vision Language Model.

\paragraph{Contextual Frame Prediction}
The frame prediction task appears to be the most tractable among the three tasks. GPT-4o achieves the highest score of $69.95\%$, followed closely by Qwen2-VL at $64.00\%$.This demonstrates that the performance gap between closed and open-source models is relatively small for this task. However, Janus-Pro perform notably below expectations ($27.50\%$), possibly due to its unified model architectural.

\paragraph{Visual Narrative Comprehension}
For visual narrative comprehension, we observe a similar pattern but with generally lower scores. GPT-4o leads with $61.60\%$, while other models show varying degrees of capability. 

\paragraph{Temporal Narrative Reordering}
The frame reordering task proves to be the most challenging, with all models performing significantly below human capability. Even the best-performing models struggle to exceed $30\%$ accuracy, with many achieving scores around $25-26\%$, which is slightly higher than random selection. 
Notably, several models (marked with *) are unable to perform this task due to their architectural limitations in processing multiple images simultaneously. For these models, we attempted to accommodate their single-image constraint by concatenating multiple frames horizontally into a single image, with white margins serving as frame boundaries. However, this workaround appears to be suboptimal, as these models likely struggle to properly distinguish individual frame boundaries and maintain the semantic independence of each frame, ultimately leading to their poor performance on the reordering task. 

The poor performance on reordering task suggests that current LMMs, regardless of their scale or architecture, have not yet developed robust capabilities for understanding temporal relationships and sequential logic in visual narratives.




\section{Analysis}
Our analysis addresses the following questions:
\begin{figure*}[t]
\centering
\includegraphics[width=0.99\textwidth]{fig/crop_case.pdf}
\caption{Sample outputs of our three tasks generated by different vision language models, along with gold truth. We highlight errors in distractors. }
\label{fig:case}
\vspace{-4mm}
\end{figure*}

\paragraph{Does fine-tuning with reorder task help?}
Yes, it does. We fine-tune Qwen2-VL using 3,160 samples for one epoch. This not only significantly improves performance on the reordering VQA task but also enhances comprehension tasks.
To construct the training dataset, we applied data augmentation to 790 images using the reorder task. Specifically, we randomly shuffled the sequence of images four times, generating a total of approximately 3,160 distinct samples. For evaluation on the reorder task, we used only the remaining 100 samples. For the comprehension task, we conducted a full test set evaluation, as the training data provided only images without any analytical content. Meanwhile, we excluded the frame prediction task from testing due to potential data leakage. The experimental results are presented in the table~\ref{tab:finetune}. 
Overall, our reordering data is useful for fine-tuning, as it can enhance the LMMs to reason on sequential images. However, the ultimate performance still depends on the base capability of the model. From the table, it can be seen that Qwen2.5-VL benefits greatly from fine-tuning, with improvements not only in the reordering task but also in comprehension tasks. The improvement in generation tasks in the reordering task is much larger than that in choice tasks. We believe this is because the generation task's instructions are very challenging, and the model has not been trained on them before, leading to difficulty in solving them. On the other hand, for choice tasks, the model can make educated guesses based on the options provided. Due to the relatively small amount of training samples, the model's improvement in choice tasks is limited. In contrast, LLaVA shows improvement only in the reorder-choice task after training.


\begin{table}[htbp]
    \centering
     \resizebox{0.45\textwidth}{!}{% 缩小整体表格
    \begin{tabular}{c|c|c|c}
    \toprule
        Tasks  & Comprehension & Reordering-G & Reordering-C \\
        \midrule
       Qwen2-VL & 58.53 & 6.00 & 31.00 \\
       +finetune & 62.94 & 31.00 & 38.00 \\
       \midrule
       LLaVA-1.6 & 34.41 &3.00& 26.00 \\
       +finetune & 33.82 &2.00 & 32.00 \\
       \bottomrule
    \end{tabular}}
    \caption{Performance on Qwen2-VL-7B and LLaVA-1.6-7B finetuned with reordering task data. Reordering-C refers to the Reordering-choice. Reordering-G refers to the Reordering-generation}
    \label{tab:finetune}
\end{table}

\paragraph{Does GPT-4o understand sequence images as well as humans?}
While GPT-4o achieves the highest performance among all tested models, there remains a substantial gap between its capabilities and human performance, particularly in novel tasks like frame reordering. Through our preliminary data annotation experiments, we observed that while GPT-4o can comprehend basic comic content and provide interpretations, it frequently generates hallucinated content and struggles with comics that depict unconventional or imaginative scenarios rarely encountered in real life.

In the visual narrative comprehension and next-frame prediction tasks, the multiple-choice format allows models to leverage similarity matching between options. Our investigation revealed that model performance is heavily influenced by the quality of distractor options. In initial experiments with weak distractors (generated using GPT-4o with instructions to provide distractors with hallucinations, the prompt is followed HalluEval), the model achieved accuracy rates up to 90\%. Upon analysis, we found these initial distractors were too obviously incorrect or irrelevant to the comic content, making the selection task trivial. To address this limitation, we carefully curated a new set of challenging distractors. With these enhanced distractors, performance of GPT-4o decreased significantly to more realistic levels ($61.60\%$ for understanding and $69.95\%$ for prediction), better reflecting the true challenges in comic comprehension. The scores obtained from multiple-choice questions with semantically transparent options tend to be inflated. In subsequent reordering tasks, where options lack explicit semantic meanings, coupled with open-ended questions, the scores provided a more authentic assessment of the LMMs.
% 这种选择题的形式(且选项内容有直接的含义)测出来的分数是偏高的。后来在reorder中,选项内容没有直接的实际含义,他的分数并且还测了问答题。这个任务的分数比较真实的反应模型的能力。

% 总的来说,模型的表现还是不如人类。尤其是在新颖的任务reorder中,表现差距非常大。在前面测试4o是否可以拿来标数据的时候,我们发现模型可以理解漫画的部分内容并做出一些解读,但是也会经常有幻觉内容。对于一些脑洞大开在现实生活中很少见的漫画,模型就很难理解。在visual narrative comprehensive 和 next-frame prediction 任务中,因为是选择题的形式,所以可以根据选项内容进行相似度匹配。模型的选择受干扰项的影响很大。在最初的任务标注中,干扰项使用4o模型直接告诉他要生成带幻觉的干扰项,这些弱干扰项和答案一起输给LMM时,4o的准确率可以达到90%。经过分析例子,我们发现是那些弱干扰项错误的非常明显,所以四选一可以轻松的选对答案。后面我们又生成了新的强干扰项,然后模型的准确率才下降。
% 怎么讲清楚,understanding,prediction的选项内容跟reorder的是不一样的?前者有直接的含义,而后者需要转义无实际含义。
\paragraph{Does input format of images influence performance?}


% 每个帧的单独输入可以帮助模型更清楚地提取每一格的信息,而不会因为它们挤在一张图上导致信息混杂。
% 帧间逻辑更明确:通过人工提供帧的顺序,模型能更好地理解帧之间的逻辑关系(比如时间顺序、因果关系等),避免因整图布局复杂而误解帧的排列。
% 但是也考察模型对多图片的整合处理方式。
%我们做了更细粒度的实验,把整图按照frame拆分,然后一次性按序输入。这种方式对于人类来说会更简单,但是由于模型处理多图走的是video方式,这可能对模型来说也是挑战。结果表明差别不大,所以我们进一步把帧的顺序打乱继续测模型对隐含意的理解。
Considering the distinct computational pathways that LMMs employ in processing individual versus multiple images, we designed following experiments to measure the differential impact of varied input formats using Qwen2.5VL as our test case. We compared three input formats: (1) whole image - the entire comic strip as a single image, (2) sequential frames - individually separated frames input in order, and (3) shuffled sequence - separated frames input in random order. Table~\ref{tab:understanding} shows surprisingly consistent performance across all three formats ($56.03\%$, $54.56\%$, and $57.65\%$ respectively).

This consistency suggests that while separated frames might theoretically help models extract clearer information from each panel and avoid visual confusion from complex layouts, the current video-like processing mechanism used by LMMs for multiple images might not fully capitalize on these advantages. The similar performance with shuffled sequences further indicates that models might rely more on individual frame content rather than sequential relationships for understanding tasks.


\begin{table}[htbp]
    \centering
     \resizebox{0.45\textwidth}{!}{% 缩小整体表格
    \begin{tabular}{c|c|c}
     \toprule
       Input  & GPT-4o-mini & Qwen2.5-VL \\
       \midrule
       Whole Image  & 53.23 & 56.03 \\
      Image Sequence & 51.03 &  54.56 \\
       Shuffled Sequence & 49.56 & 57.65 \\
       \bottomrule
    \end{tabular}}
    \caption{Performance with different input format for understanding task. }
    \label{tab:understanding}
\end{table}
\paragraph{Does implicit meaning help reordering task?}
% 新的prompt = f"The sequence of the comic strips provided below is incorrect, Your task is to find out the correctorder of the comic strips based on the storyline, temporal relationships, and common-sense logic. Here is thedepiction of the comic strips: \"fga.get('understanding’)}\" \n Number each comic strip in the order they shouldappear, starting from 1.\n\n" 
To investigate whether poor reordering performance stems from inadequate semantic understanding, we enhanced the reordering task by providing explicit semantic annotations along with shuffled images. As shown in Table~\ref{tab:reorder_enhanced}, this additional semantic information only marginally improved performance (from 30.01\% to 32.54\%). This modest improvement suggests that the bottleneck in reordering tasks lies not in semantic understanding but in the fundamental capability to reason about temporal and logical sequences in visual narratives.

\begin{table}[htbp]
    \centering
     \resizebox{0.45\textwidth}{!}{% 缩小整体表格
    \begin{tabular}{c|c|c}
    \toprule
        Input & GPT-4o-mini & Qwen2.5-VL \\
        \midrule
       Shuffled image & 26.07  & 30.01 \\
       +Meaning & 28.40 & 32.54 \\
       \bottomrule
    \end{tabular}}
    \caption{Performance with enhanced data (correct answer for comprehension task) for reordering task.}
    \label{tab:reorder_enhanced}
\end{table}
% 从表中可以看到,给模型输入额外的漫画含义后,排序的分数并没有很大的提升。所以LMMs在reorder任务上的瓶颈并不是理解图片含义。以后的模型应该增强对此任务的训练。

% \paragraph{Can video-VLM outperform VLLM?}
% Video-LLM is trained with video data containing richer temporal information. Comic strip is like the key frame for video, so it omitted the step of extracting keyframes from the video. 


% \paragraph{Does CoT help with three tasks?}
% To achieve better performance, we conducted experiments using the Chain of Thought (CoT) approach by adding "Let's think step by step" to the input. However, as shown in table~\ref{tab:cot}, we found that this modification did not improve performance on the first two tasks and only provided a boost in the reorder task. We speculate that this is because the instructions for the reorder task are relatively more complex compared to the other tasks.
% \begin{table}[htbp]
%     \centering
%      \resizebox{0.45\textwidth}{!}{% 缩小整体表格
%     \begin{tabular}{c|c|c|c}
%     \toprule
%         Tasks  & Comprehension & Frame Prediction & Reordering-MCQ \\
%         \midrule
%        origin input & 56.03  & 64.00 & 29.21 \\
%         +CoT        & 52.94 & 57.50  & 31.69 \\
%        \bottomrule
%     \end{tabular}}
%     \caption{Compare the performance with and without CoT (Chain of Thought).}
%     \label{tab:cot}
% \end{table}

\paragraph{How does model size affect?}
In this section, we will discuss the relationship between model parameters size and reasoning performance for tasks due to the scaling law. We examine on LLaVA and Qwen2.5VL, from 3B scale to 34B scale.
There is a clear scaling effect across model sizes, as demonstrated by LLaVA1.5's performance improving from $34.41\%$ (7B) to $46.03\%$ (13B) to $52.94\%$ (34B). This suggests that model scale plays a crucial role in comprehending implicit meanings in visual narratives.
Figure~\ref{fig:scale} provide a visual representation of the performance trend for five models.

\begin{figure}[t]
\centering
\includegraphics[width=0.96\columnwidth]{fig/model.pdf}
\caption{ Comparison of the accuracy results between Qwen2.5-3B vs Qwen2.5-7B and LLaVA-1.6-7B vs LLaVA-1.6-13B vs LLaVA-1.6-34B}
\label{fig:scale}
\vspace{-4mm}
\end{figure}



\paragraph{Where do LMMs fail?}
We present sample outputs of three tasks generated by vision language models (VLMs) in Figure~\ref{fig:case}. These images are easy for human but hard for VLMs. VLMs can understand one comic strip but they can still make mistakes with reordering task. 

% \paragraph{Does repeating images help with three tasks?}
% % 类型,模型处理inspiring的会更好。对于讽刺类的表现不好。我们推测这是由于模型在训练的时候被灌输大量的积极的内容,而消极内容训练的比较少,并被要求减少负面输出。
% \paragraph{Error analysis}
%  % visual misinterpretation  
%   % hallucination 
%   % 常识  打破镜子而不是窗户

% Our experiments reveal that while models achieve reasonable scores on semantic understanding tasks, their abilities remain superficial. Notably, we observe that model predictions remain  unchanged even when presented with scrambled frame sequences, indicating that current LMMs rely more on matching similar visual elements and semantic concepts across frames rather than truly comprehending the sequential narrative structure of comics. This observation is further supported by their poor performance on frame reordering tasks, highlighting their limitations in temporal reasoning and sequential understanding.
\section{Conclusion}
We operationalized the theory of instrumental interaction for generative AI, with an in-depth unpacking of the principles of reification of user intent, reflection, and grounding. We argue that leveraging this re-appropriated and refined theory can drive the creation of a \textit{new generation of expressive AI-Instruments} that afford better expression of intent, make it easier to discover what is possible, and provide powerful degrees of freedom for steering the generation towards the best possible results. Those new tools and instruments can truly leverage the polymorphic and non-deterministic behavior of generative AI models, unleashing new and empowering forms of expressive HCI+AI experiences. 

Beyond our focus on AI-Instruments, theories play an important role in the advancement of our wider research field~\cite{rogers_hci_2012, halverson_activity_2002}. Rogers argues that there is a need for theories as lenses bringing critical design characteristics into focus, and which can function as a generative source: providing "\textit{design dimensions and constructs to inform the design and selection of interactive representations}"~\cite{rogers_new_2004}. We hope that our work on operationalizing the theory of instrumental interaction for AI can inspire other new -- and re-appropriated -- theories to advance HCI+AI. 









\section*{Limitations}
\label{subsec:limitation} 

Due to computational constraints, experiments with larger LLMs, such as Qwen2.5-32B-Instruct and Qwen2.5-72B-Instruct, were not conducted. We believe that \method could achieve a more favorable trade-off between reasoning performance and CoT token usage on these models. Additionally, the token importance measurement used in our study, derived from the LLMLingua-2 compressor~\cite{pan:2024llmlingua2}, was not specifically trained on mathematical data. This limitation may affect the compression effectiveness, as the model is not optimized for handling numerical tokens and mathematical expressions. Furthermore, experiments with long-CoT LLMs, such as QwQ-32B-Preview, were also excluded due to computational constraints. We plan to explore these aspects in future work, as we anticipate that \method’s potential can be further realized in these contexts.

\section*{Ethics Statement}
\label{subsec:ethics} 
The datasets used in our experiment are publicly released and labeled through interaction with humans in English. In this process, user privacy is protected, and no personal information is contained in the dataset. The scientific artifacts that we used are available for research with permissive licenses. And the use of these artifacts in this paper is consistent with their intended use. Therefore, we believe that our research work meets the ethics of ACL. 



% Bibliography entries for the entire Anthology, followed by custom entries
%\bibliography{anthology,custom}
% Custom bibliography entries only
\bibliography{custom}

\clearpage

\appendix

\section*{Appendix}
\section{Additional Experimental Details}
\label{appen:experimental details}
\subsection{Datasets}
\label{appen:datasets}
In this work, we categorize the datasets into \textit{in-context} and \textit{out-of-context} datasets. The videos from \textit{in-context} datasets consist of actions with frequent static context, \eg swimming in the swimming pool, while the videos from \textit{out-of-context} datasets contain actions occurring with an unusual static context, \eg dancing in the mall~\cite{choi2019can}.
We conduct the experiments on five \textit{in-context} benchmarks: Kinectics-400~\cite{k400} (K400), Kinectis-600~\cite{k600} (K600), UCF101~\cite{UCF101} (UCF), HMDB51~\cite{HMDB51} (HMDB), and Something-Something V2~\cite{ssv2} (SSv2). Additionally, we evaluate our approach on two \textit{out-of-context} benchmarks: SCUBA~\cite{li2023mitigating} and HAT~\cite{chung2022enabling}.

\noindent {\bf K400 and K600} are both comprehensive video datasets for human action recognition. K400 contains 400 action categories of approximately 240k training and 20k validation videos collected from YouTube, which covers a wide range of human actions, including sports activities, daily life actions, and various interactions, serving as a widely-used action recognition dataset for pre-training. The duration of video clips in K400 varies, with most clips being around 10 seconds long. This diversity in video duration helps models learn temporal dynamics and context for action recognition. K600 extends K400 by incorporating 220 additional new categories, thus enabling the evaluation of zero-shot learning capabilities on these novel categories. 

\noindent {\bf UCF} is a human action recognition dataset collected from YouTube, and consists of 13,320 video clips, which are classified into 101 categories. These 101 categories encompass a wide range of realistic actions including body motion, human-human interactions, human-object interactions, playing musical instruments and sports. Officially, there are three splits allocating 9,537 videos for training and 3,783 videos for testing.

\noindent {\bf HMDB} is a relatively small video dataset comprising a diverse range of sources, including movies, public databases, and YouTube videos, and is composed of 6,766 videos across 51 action categories (such as ``jump'', ``kiss'' and ``laugh''), ensuring at least 101 clips within each category. The original evaluation scheme employs three distinct training/testing splits, allocating 70 clips for training and 30 clips for testing of each category in each split.

\noindent {\bf SSv2} is a temporally focused video dataset across 174 fine-grained action categories, consisting of 168,913 training videos and 24,777 testing videos showing the objects and the actions performed on them. These action categories are presented using object-agnostic templates, such as ``Dropping [something] into [something]'' containing slots (``[something]'') that serve as placeholders for objects. This dataset focuses on basic, physical concepts rather than higher-level human activities, which challenges the temporal modeling capabilities.

\noindent {\bf SCUBA} is an out-of-distribution (OOD) video benchmark designed to quantitatively evaluate static bias in the background. It comprises synthetic out-of-context videos derived from the first test split of HMDB and UCF, as well as the validation set of K400. These videos are created by superimposing action regions from one video onto diverse scenes, including those from Place365~\cite{zhou2017places} and VQGAN-CLIP~\cite{crowson2022vqgan} generated scenes. 
Due to the differences in test sets and background sources, the domain gaps of SCUBA benchmarks vary. A domain gap is defined as the ratio of accuracies between the original test sets and synthetic datasets obtained by a 2D reference network, where a higher ratio indicates a greater domain gap with respect to static features.
The UCF-SCUBA and K400-SCUBA used in our experiments consist of 4,550 and 10,190 videos with domain gaps of $20.49$ and $6.09$, respectively, whose backgrounds are replaced by the test set of Place365.

\noindent {\bf HAT} is a more ``realistic-looking'' mixed-up benchmark for quantitative evaluation of the background bias by automatically generating synthetic counterfactual validation videos with different visual cues. It provides four Action-Swap sets with distinct characteristics: \textit{Random} and \textit{Same} refer to the swap of actions and backgrounds from different and same classes, respectively, while \textit{Close} and \textit{Far} denote the swap of videos from a class with similar and very different backgrounds, respectively. The UCF-HAT benchmark used in our experiments consists of Action-Swap videos in \textit{Close} and \textit{Far} sets from 5 closest and 30 farthest action categories, respectively, following the literature~\cite{chung2022enabling}. Note that we only consider videos from the first test split of UCF where all frames have human masks taking up 5\% to 50\% of the pixels to ensure that sufficient human and background cues are present in each generated Action-Swap video.

\subsection{Evaluation Protocols}
\label{appen:EP}
For the experimental settings, we follow the previous works~\cite{weng2023open,rasheed2023fine,ni2022expanding} for in-context generalization evaluations and perform the newly proposed out-of-context generalization evaluations described below.

\noindent {\bf In-context base-to-novel generalization.}
Under this setting, we divide the entire set of action categories into two equal halves: base and novel, with the most frequently occurring classes designated as the base classes. We conduct generalization evaluations on four in-context datasets, \ie K400, HMDB, UCF and SSv2, where the models are initially trained on the base classes within the training splits of the dataset, and evaluated on both base and novel classes within the validation splits. Every training split consists of 16 video clips of each base class. During inference within HMDB and UCF datasets, only the novel class samples in the first validation splits are used for evaluation. For K400 and SSv2 datasets, the full validation split of each is used for evaluation here. We report the results of the average top-1 accuracies for both base and novel classes as well as the harmonic mean.

\noindent {\bf In-context cross-dataset generalization.}
Under this setting, the models are fine-tuned on the training set of K400, and evaluated on three in-context cross-datasets, \ie UCF, HMDB and K600. We report top-1 average accuracies with performance variances on the three validation splits in case of UCF and HMDB. For K600, the models are evaluated on non-overlapping 220 categories with K400, and we report top-1 average accuracies over three randomly sampled splits of 160 categories.

\noindent {\bf Out-of-context cross-dataset generalization.}
Under the more challenging out-of-context cross-dataset setting, the models are also trained on K400, and then evaluated on two out-of-context datasets based on UCF, \ie UCF-SCUBA and UCF-HAT. We report the top-1 and top-5 average accuracies over the synthetic counterfactual validation splits from UCF's first validation split. We further conduct the closed-set out-of-context evaluation based on the K400-SCUBA benchmark and report the harmonic mean of the accuracies under in-context and out-of-context settings to comprehensively analyze the generalization of the models.

\subsection{Implementation Details}
\label{appen:implementation}
Each training video clip is sampled with $8$ frames uniformly, and each sampled frame is spatially scaled in the shorter side to $256$ pixels and is processed with basic augmentations like color jittering, random flipping and random cropping of $224\times 224$.
We leverage multi-view inference with $3$ temporal and $1$ spatial views per video and linearly aggregate the recognition results. 
For our Gaussian Weight Average scheme, we use $\mu=7$ and $\sigma^2=10$ for in-context base-to-novel generalization and $\mu=15$ and $\sigma^2=10$ for in-context and out-of-context cross-dataset generalization.
We also adopt decision aggregation with pre-trained CLIP with the video learner for in-context evaluations.
The experiments are conducted on two computing clusters with four NVIDIA RTX 24G 4090 GPUs.


\end{document}
