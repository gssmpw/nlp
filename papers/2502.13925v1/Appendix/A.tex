\section{Model}
\label{appendix:model}
Our evaluation suite includes leading commercial models GPT-4o~\cite{hurst2024gpt40} and Gemini1.5-Pro~\cite{Gemini} alongside state-of-the-art open-source alternatives of varying scales: Qwen2.5-VL~\cite{qwen2.5-VL}, Qwen2-VL~\cite{wang2024qwen2}, LLaVA-v1.6~\cite{liu2023llava}, CogVLM~\cite{wang2023cogvlm}, MiniCPM-o 2.6~\cite{yao2024minicpm}, mPlug-Owl2~\cite{ye2023mplugowl2}, InternVL2v5~\cite{chen2024internvl},LLaVA-NEXT-Video~\cite{zhang2024llavanextvideo} and Cambrian~\cite{tong2024cambrian1}. Besides, Janus-Pro~\cite{chen2025januspro}, which unifies multimodal understanding and generation, is included to test the abilities between Unified Model and Vision Language Model.


% \begin{itemize}

% \item \textbf{LLaVA-1.5}~\cite{liu2023improvedllava} is an end-to-end LMM extended from Vicuna\cite{vicuna2023}, augmented with vision encoder.

% \item \textbf{MiniGPT-4}~\cite{zhu2023minigpt} is an extension of Vicuna, incorporating ViT \cite{vit2021image} and Q-former \cite{li2023blip} as the vision encoder, while also featuring a single linear projection layer sandwiched between them.

% \item \textbf{mPLUG-Owl2}~\cite{ye2023mplugowl2} is an extension of LLaMA-2-7B\cite{touvron2023llama2}, using ViT-L/14\cite{radford2021learning} as the vision encoder, and introducing a visual abstractor between them.

% \item \textbf{CogVLM}~\cite{wang2023cogvlm} is also an extension of Vicuna, incorporating ViT\cite{vit2021image} as the vision encoder,  a two-layer MLP\cite{shazeer2020glu} as adapter, and introducing Visual expert module.

% \item \textbf{Qwen-VL}~\cite{Qwen-VL} is an extension of Qwen-7B\cite{qwen}, incorporating ViT\cite{vit2021image} as the vision encoder, and introducing a vision-language adapter that compresses the image features.

% \item \textbf{InstructBlip2}~\cite{Dai2023InstructBLIPTG} employs ViT-g/14 \cite{fang2022eva} as image encoder, and four different LLMs as language decoders. In our following tests, we utilize vicuna-13B and vicuna-7B \cite{vicuna2023} versions.

% \item \textbf{Fuyu}~\cite{fuyu-8b} employs a decoder-only architecture, devoid of a dedicated image encoder for image processing. This design choice enables the model to support arbitrary image resolutions.

% \item \textbf{GPT-4V}~\cite{yang2023dawn} is OpenAI's cutting-edge language model redefining natural language processing with advanced contextual understanding and versatile linguistic abilities.
% \end{itemize}

\section{Model Hyper-parameter Details}
\label{appendix:hyper-param}

% 有待修改
We use the default hyper-parameter values of the models. In the LLaVa-1.5-7B and LLaVa-1.5-13B, the temperature is set to 0.2. For MiniGPT-4, the temperature is set to 1.0, and num\_beams is also set to 1.0. The temperature for mPlug-Owl-2 is set to 0.7. For CogVLM, the temperature is set to 0.4, top\_p is set to 0.8, and top\_k is set to 1.0.

%本次测试具体例子
In the LLaVa-1.6-7B, LLaVa-1.6-13B and LLaVa-1.6-34B, the temperature is set to 0.2. In the Qwen2-VL-7B, Qwen2.5-VL-3B and Qwen2.5-VL-7B, the temperature is set to 0.01, top\_p is set to 0.001, and top\_k is set to 1. For CogVLM-17B, the temperature is set to 0.4, top\_p is set to 0.8, and top\_k is set to 1.0. For InternVL2-26B, do\_sample is set to False. For Cambrian-13B, the temperature is set to 0.2.

\section{Annotation}
\label{appendix:annotation}
The following listed prompts are used to construct data. By instructing to different LMMs, we can obtain option candidate pool.

        \textbf{Prompt1:} \textit{"""What happened in comic strip? Conclude the whole story, then carefully analyze the implicit meaning of comic. Ouput in 35 words."""}         
        
        \textbf{Prompt2:} \textit{""" You are now a mature hallucination generator. Please generate one strong distractor option for the following question. You can use any method you have learned that is misleading for the given question."""} 
        
                  
\textbf{Prompt3: } \textit{""" Task Overview:
Strive to understand this story and analyze its implicit meaning, then complete multi-choice question for test. You should act in two roles to complete tasks. 
Image Context:
Pic1: The first picture shows the complete comic strip. Read it from left to right, top to bottom, to understand the full narrative arc.
Pic2: The second picture is the second-to-last frame from Pic1, which is the target frame in Task 1.}

\textit{\#\#\#Role 1 - Excellent Comic Analysis Expert:}

\textit{Task 1: Contextual Scene Description
Question 1: Based on the overall story, what is happening in the second-to-last frame (Pic2) of the comic strip?
Requirements:
1. Provide a clear and detailed description of the key visual elements, characters, relationships and actions in Pic2. 2. Ensure narrative continuity with the events of the entire comic. 3. Output this as the right option for Task 2 with 30-40 words.}

\textit{Task 2: Implicit Meaning Analysis
Question 2: What happened in comic strip (Pic1)? Describe the whole story in detail, then analyze its implicit meaning.
Requirements:
1. Describe the whole story in detail and analyze its implicit meaning and sentiment. 2. Provide three sentences with 40-50 words as right option for Task 4.
}

\textit{\#\#\#Role 2 - Strong Distractor Options Generator:
}
\textit{Task 3: Frame Scene Options Generation
Question 1: Based on the overall story, what is happening in the second-to-last frame (Pic2) of the comic strip?
Requirements:
1. Generate three plausible but incorrect options for \#Question 1\#. 2.The length of each option should be similar with the correct answer from Task 1 (around 30-40 words)! 3. Ensure that the incorrect options are consistent with the overall story but misinterpret the events of Pic2. }

\textit{Task 4: Comic Strip Analysis Options Generation
\#Question 2:\# What happened in comic strip (Pic1)? Describe the whole story in detail, then analyze its implicit meaning.  
Requirements:
1. Generate three plausible but incorrect options for \#Question 2\#. 2. Each incorrect option should be composed of 3 sentences and share the same length with the correct answer from Task 2! 3. Avoid obviously wrong or nonsensical answers.
}

\textit{\#Please strictly adhere to the word count requirement! The final output should be in the following format:\#}
\textit{\#Question 1:\# Based on the overall story, what is happening in the second-to-last frame (Pic2) of the comic strip?}
\textit{\#Reasoning Chain 1:\# [Describe the story in sequence.]}

\textit{\#Options in Q1:\# A. [Hallucination option with 30-40 words from Task 3] \#\#\#
B. [Hallucination option with 30-40 words from Task 3] \#\#\#
C. [Hallucination option with 30-40 words from Task 3] \#\#\#
D. [Right option with 30-40 words from Task 1] \#\#\#}


\textit{\#Right Answer 1:\# D}


\textit{\#Question 2:\# What happened in comic strip (Pic1)? Describe the whole story in detail. Carefully analyze the implicit meaning of comic.}

\textit{\#Reasoning Chain 2:\# [Reason step by step here.]}

\textit{\#Options in Q2:\# A. [Hallucination option with 40-50 words from Task 4] \#\#\#
B. [Hallucination option with 40-50 words from Task 4] \#\#\#
C. [Hallucination option with 40-50 words from Task 4] \#\#\#
D. [Right option with 40-50 words from Task 2] \#\#\#
}

\textit{\#Right Answer 2:\# D """}



\textbf{prompt4:} \textit{"""Predict what happened in the blank panel? Output in 35 words."""}

\textbf{prompt5:} \textit{"""You are now a mature hallucination generator. Please generate one strong distractor option for the following question. You can use any method you have learned that is misleading for the given question. 
Question: Predict what happened in the blank panel. Please output with 35 words without any additional text or explanation."""}


\section{Examples}
Here is an example of data construction in Figure~\ref{fig:examples}.
\label{example}





\section{Category}
\label{categories}
We evaluated the performance of LLMs on comprehension tasks across 6 categories of comic strip.

\begin{figure}[h]
    \centering
    \includegraphics[width=\linewidth]{fig/crop_piechart.pdf}
    \caption{The distribution of six categories of \dataset.}
    \label{fig:pie}
\end{figure}

\begin{figure}[h]
    \centering
    \includegraphics[width=\linewidth]{fig/radar_compare_optimized_v2.pdf}
    \caption{Comprehension task performance comparision of different LLMs on different categories.}
    \label{fig:radar}
\end{figure}




\begin{table*}[]
    \centering
    % \small
    \begin{tabular}{lp{13cm}}
    \toprule
       \textbf{Category}  &  \textbf{Definition} \\ 
       \midrule
         Satirical &  \RaggedRight The comic uses irony, exaggeration, or ridicule to criticize social, political, or cultural issues. It often highlights contradictions, hypocrisy, or absurdity in a way that provokes thought or debate. The humor may be sharp or biting but serves a critical purpose. \\ 
         \midrule
Inspiring& \RaggedRight The comic presents a positive or uplifting message, often encouraging personal growth, motivation, or perseverance. It may depict acts of kindness, success against adversity, or wisdom that encourages the reader to strive for betterment.\\ \midrule
Touching& \RaggedRight The comic evokes emotions such as empathy, nostalgia, or affection. It may explore themes of love, friendship, loss, or family bonds, aiming to create a sentimental or heartfelt response from the audience.\\ 
\midrule
Philosophical& \RaggedRight The comic explores deep, abstract, or existential ideas about life, morality, meaning, or human nature. It prompts the reader to reflect on profound questions, often using metaphors or thought-provoking dialogue rather than direct humor or emotion.\\
\midrule
Critical& \RaggedRight The comic highlights flaws or problems in society, institutions, or human behavior with a serious or analytical tone. Unlike satire, which uses humor as a tool for critique, a critical comic may adopt a more straightforward, serious, or thought-provoking approach to expose issues and encourage awareness.\\ 
\midrule
Humorous& \RaggedRight The comic's primary goal is to entertain and amuse the audience. It relies on lighthearted jokes, wordplay, or visual gags without necessarily conveying a deeper message or critique. The tone is playful, aiming for laughter rather than serious reflection. \\
\bottomrule
    \end{tabular}
    \caption{The types and definition of the categories in \dataset.}
    \label{tab:categories}
\end{table*}


\begin{figure*}
    \centering
    \includegraphics[width=0.95\linewidth]{fig/crop_appendix.pdf}
    \caption{An detailed example of data construction.}
    \label{fig:examples}
\end{figure*}