\section{Conclusion}
\label{sec:conclusion}
We revisited the debiasing effects of normalization on embedding vectors of ReID models and explored the risk of camera bias inherent in unsupervised learning for ReID models.
We found that the existing ReID models are biased towards camera labels on unseen domain, and the unsupervised models even have a large camera bias to their training data.
We analyzed why the camera-specific feature normalization has debiasing effects and explored its potential and applicability for ReID tasks in comprehensive empirical studies.
It was observed that, for a camera variation, the sensitivity of each feature dimension is quite different and features tend to move consistently in sensitive dimensions.
Then, it was shown that the feature normalization is a simple but effective bias elimination method for ReID models in general, including biases towards low-level properties and body angle.
Also, we empirically showed the detrimental effects of biased pseudo labels using toy examples and achieved significant performance improvements with simple modifications to the existing unsupervised algorithms.
We hope that the insights from this work will serve as an insightful foundation for researching biases of ReID models and developing debiasing techniques for ReID models.