\section{RELATED WORK}
LeCun et al. \cite{c3} presented early work on computational graphs in neural networks, particularly in the architecture of Convolutional Neural Networks (CNNs), focusing on the flow of information through structured layers. This study laid the foundation for the graph-based analysis of neural networks by emphasizing the role of structured information flow in model performance.  

Kipf and Welling \cite{c4} introduced Graph Convolutional Networks (GCNs), expanding neural networks to process non-Euclidean data represented as graphs. GCNs are a class of neural networks that extend CNNs to graph-based data, enabling more efficient processing of non-Euclidean data. Their work demonstrated that graph structures could significantly improve the ability of models to capture relationships within complex datasets, such as social networks and molecules.  

The seed papers, You et al. \cite{c5}, \cite{c6}, \cite{c7}, \cite{c8} systematically investigates the role of graph structures in neural networks. Their analysis of graph metrics, such as clustering coefficient and average path length, identified optimal configurations that balance scalability and accuracy in models such as MLPs, CNNs, and ResNets. This study highlighted the importance of graph structure in optimizing neural architectures for predictive tasks. These papers directly address separability and geometry of object manifolds in deep neural networks and explore classification and geometry of perceptual manifolds.

Watts and Strogatz \cite{c9} introduced the small-world network model, providing key insights into the clustering coefficient and path length metrics, which are now crucial in evaluating the performance and scalability of neural networks when treated as graphs. This work forms the theoretical basis for many modern graph generation techniques applied in neural networks.  

In \cite{c10}, Chen et al. explored graph pruning and rewiring techniques as methods to enhance the efficiency of neural network architectures. By leveraging centrality measures like between-ness centrality, the study demonstrated that pruning unnecessary connections could improve both performance and computational efficiency.  

In the context of neural architecture search, Ying et al. \cite{c11} developed NAS-Bench-101, which focuses on optimizing connectivity patterns by searching for efficient Directed Acyclic Graphs (DAGs). A DAG is a graph where the edges have a direction, and no cycles are present, meaning no node can be revisited once traversed. This is relevant to the discussion on efficient graph-based network design and its implications for computational cost.  

In \cite{c12}, Barabasi and Psfai contributed foundational work in network science, which applies directly to understanding the structural properties of neural networks when modeled as graphs. Their research on scale-free networks influences how we think about network connectivity and the distribution of connections in complex architectures.  

Bassett and Bullmore \cite{c13} drew parallels between biological neural networks and artificial models, highlighting the similarities in small-world properties. This work provides insights into how the brain's efficient organization can inform the optimization of artificial neural network architectures.  

Recent work from Szegedy et al. \cite{c14} on deep convolutional networks further explores the impact of depth and complexity on scalability. Their findings are relevant to graph-based neural networks, particularly in how layer depth and graph complexity interact to influence network performance.  

Lastly, in \cite{c15}, Elsen et al. investigated sparse convolutional networks, focusing on how reducing graph complexity could enhance the computational efficiency of large-scale networks. This research aligns with the efforts to optimize graph representations for larger neural networks and complex architectures.