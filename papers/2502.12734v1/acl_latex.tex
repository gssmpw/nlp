% This must be in the first 5 lines to tell arXiv to use pdfLaTeX, which is strongly recommended.
\pdfoutput=1
% In particular, the hyperref package requires pdfLaTeX in order to break URLs across lines.

\documentclass[11pt]{article}

% Change "review" to "final" to generate the final (sometimes called camera-ready) version.
% Change to "preprint" to generate a non-anonymous version with page numbers.
\usepackage[review]{acl}

% Standard package includes
\usepackage{times}
\usepackage{latexsym}

% For proper rendering and hyphenation of words containing Latin characters (including in bib files)
\usepackage[T1]{fontenc}
% For Vietnamese characters
% \usepackage[T5]{fontenc}
% See https://www.latex-project.org/help/documentation/encguide.pdf for other character sets

% This assumes your files are encoded as UTF8
\usepackage[utf8]{inputenc}

% This is not strictly necessary, and may be commented out,
% but it will improve the layout of the manuscript,
% and will typically save some space.
\usepackage{microtype}

% This is also not strictly necessary, and may be commented out.
% However, it will improve the aesthetics of text in
% the typewriter font.
\usepackage{inconsolata}

%Including images in your LaTeX document requires adding
%additional package(s)
\usepackage{graphicx}

% If the title and author information does not fit in the area allocated, uncomment the following
%
%\setlength\titlebox{<dim>}
%
% and set <dim> to something 5cm or larger.

\usepackage{microtype}
\microtypecontext{spacing=nonfrench}

\usepackage{hyperref}
\usepackage{longtable}
\usepackage{arydshln}
\usepackage{bm}     
\usepackage{booktabs}
\usepackage{array}
\usepackage{makecell}
\usepackage{amsmath, amssymb, amsthm}
\usepackage{cleveref}  
\usepackage{amsmath, etoolbox}

\makeatletter
\patchcmd{\gathered@d@t@}
  {\tagform@{\thetag@form@}\fi}
  {%
    \setbox\z@\hbox{\tagform@{\thetag@form@}}%
    \dimen@=\ht\z@ \advance\dimen@ by \dp\z@
    \rlap{\raisebox{0.5\dimen@}{\tagform@{\thetag@form@}}}%
  }{}{}
\makeatother
\newtheorem{definition}{Definition}
\newtheorem{theorem}{Theorem}
\newtheorem{lemma}{Lemma}
\usepackage{algorithm}
\usepackage{algpseudocode}
\usepackage{ragged2e} 
\usepackage{graphicx}
\usepackage{xspace}
\usepackage{multirow}
\usepackage{pifont} 
\usepackage{colortbl}
\PassOptionsToPackage{table}{xcolor} 
\usepackage{xcolor}
\usepackage{comment}
\newcommand{\modelname}{\textsc{Greater}\xspace}
\newcommand{\attackname}{\textsc{Greater-A}\xspace}
\newcommand{\defensename}{\textsc{Greater-D}\xspace}
\newcommand{\greencheckmark}{{\color{green}{\checkmark}}}
\newcommand{\redxmark}{{\color{red}{\ding{55}}}}
\newcommand{\doubleunderline}[1]{%
    \underline{\underline{#1}}}
\newcommand{\etal}{\emph{et~al.}\xspace}
\newcommand{\eg}{\emph{e.g.},\xspace}
\newcommand{\ie}{\emph{i.e.},\xspace}
\newcommand{\etc}{etc.\xspace}
\newcommand\figref[1]{Figure~\ref{#1}}
\newcommand\figsubref[1]{Figure~\subref{#1}}
\newcommand\tabref[1]{Table~\ref{#1}}
\newcommand\tabsubref[1]{Table~\subref{#1}}
\newcommand\secref[1]{\S\ref{#1}}
\newcommand\equref[1]{Eqn.~(\ref{#1})}
\newcommand\appref[1]{Appendix~\ref{#1}}
\newcommand{\fakeparagraph}[1]
{\vspace{1mm}\noindent\textbf{#1}}

\title{\Large Iron Sharpens Iron: Defending Against Attacks in Machine-Generated Text Detection with Adversarial Training}

% Author information can be set in various styles:
% For several authors from the same institution:
% \author{Author 1 \and ... \and Author n \\
%         Address line \\ ... \\ Address line}
% if the names do not fit well on one line use
%         Author 1 \\ {\bf Author 2} \\ ... \\ {\bf Author n} \\
% For authors from different institutions:
% \author{Author 1 \\ Address line \\  ... \\ Address line
%         \And  ... \And
%         Author n \\ Address line \\ ... \\ Address line}
% To start a separate ``row'' of authors use \AND, as in
% \author{Author 1 \\ Address line \\  ... \\ Address line
%         \AND
%         Author 2 \\ Address line \\ ... \\ Address line \And
%         Author 3 \\ Address line \\ ... \\ Address line}

\author{Yuanfan Li\textsuperscript{$1,\dagger$}, Zhaohan Zhang\textsuperscript{$2,\dagger$}, Chengzhengxu Li\textsuperscript{$1,\dagger$},  Chao Shen\textsuperscript{$1$}, Xiaoming Liu\textsuperscript{$1,\ast$} \\
        \textsuperscript{1}Faculty of Electronic and Information Engineering, Xi'an Jiaotong University\\ 
        %\\ No.28, Xianning West Road, Xi'an, China\\
        \textsuperscript{2}Queen Mary University of London \\
        \textsuperscript{$\dagger$} Equal contribution, \textsuperscript{$\ast$} Corresponding author\\
        \texttt{
        %\textsuperscript{$\ast$} Corresponding author, 
        liyuan7716@gmail.com, czx.li@stu.xjtu.edu.cn
        }\\
        \texttt{
        zhaohan.zhang@qmul.ac.uk, \{chaoshen, xm.liu\}@xjtu.edu.cn 
        }\\
        %\textsuperscript{$\ast$} Corresponding author
        }

%\author{
%  \textbf{First Author\textsuperscript{1}},
%  \textbf{Second Author\textsuperscript{1,2}},
%  \textbf{Third T. Author\textsuperscript{1}},
%  \textbf{Fourth Author\textsuperscript{1}},
%\\
%  \textbf{Fifth Author\textsuperscript{1,2}},
%  \textbf{Sixth Author\textsuperscript{1}},
%  \textbf{Seventh Author\textsuperscript{1}},
%  \textbf{Eighth Author \textsuperscript{1,2,3,4}},
%\\
%  \textbf{Ninth Author\textsuperscript{1}},
%  \textbf{Tenth Author\textsuperscript{1}},
%  \textbf{Eleventh E. Author\textsuperscript{1,2,3,4,5}},
%  \textbf{Twelfth Author\textsuperscript{1}},
%\\
%  \textbf{Thirteenth Author\textsuperscript{3}},
%  \textbf{Fourteenth F. Author\textsuperscript{2,4}},
%  \textbf{Fifteenth Author\textsuperscript{1}},
%  \textbf{Sixteenth Author\textsuperscript{1}},
%\\
%  \textbf{Seventeenth S. Author\textsuperscript{4,5}},
%  \textbf{Eighteenth Author\textsuperscript{3,4}},
%  \textbf{Nineteenth N. Author\textsuperscript{2,5}},
%  \textbf{Twentieth Author\textsuperscript{1}}
%\\
%\\
%  \textsuperscript{1}Affiliation 1,
%  \textsuperscript{2}Affiliation 2,
%  \textsuperscript{3}Affiliation 3,
%  \textsuperscript{4}Affiliation 4,
%  \textsuperscript{5}Affiliation 5
%\\
%  \small{
%    \textbf{Correspondence:} \href{mailto:email@domain}{email@domain}
%  }
%}

\begin{document}
\maketitle
\begin{abstract}
Machine-generated Text (MGT) detection is crucial for regulating and attributing online texts.
While the existing MGT detectors achieve strong performance, they remain vulnerable to simple perturbations and adversarial attacks.
To build an effective defense against malicious perturbations, we view MGT detection from a threat modeling perspective, that is, analyzing the model's vulnerability from an adversary's point of view and exploring effective mitigations. 
To this end, we introduce an adversarial framework for training a robust MGT detector, named \textbf{GRE}edy \textbf{A}dversary
Promo\textbf{T}ed
Defend\textbf{ER}
(\modelname).
The \modelname consists of two key components: an adversary \attackname and a detector \defensename.
The \defensename learns to defense against the adversarial attack from \attackname and generalizes the defense to other attacks.
\attackname identifies and perturbs the critical tokens in embedding space, along with greedy search and pruning to generate stealthy and disruptive adversarial examples. 
Besides, we update the \attackname and \defensename synchronously, encouraging the \defensename to generalize its defense to different attacks and varying attack intensities. 
Our experimental results across 9 text perturbation strategies and 5 adversarial attacks show that our \defensename reduces the Attack Success Rate (ASR) by \textbf{10.61\%} compared with SOTA defense methods while our \attackname is demonstrated to be more effective and efficient than SOTA attack approaches.
\end{abstract}

%插入章节
\section{Introduction}
\label{sec:introduction}
The business processes of organizations are experiencing ever-increasing complexity due to the large amount of data, high number of users, and high-tech devices involved \cite{martin2021pmopportunitieschallenges, beerepoot2023biggestbpmproblems}. This complexity may cause business processes to deviate from normal control flow due to unforeseen and disruptive anomalies \cite{adams2023proceddsriftdetection}. These control-flow anomalies manifest as unknown, skipped, and wrongly-ordered activities in the traces of event logs monitored from the execution of business processes \cite{ko2023adsystematicreview}. For the sake of clarity, let us consider an illustrative example of such anomalies. Figure \ref{FP_ANOMALIES} shows a so-called event log footprint, which captures the control flow relations of four activities of a hypothetical event log. In particular, this footprint captures the control-flow relations between activities \texttt{a}, \texttt{b}, \texttt{c} and \texttt{d}. These are the causal ($\rightarrow$) relation, concurrent ($\parallel$) relation, and other ($\#$) relations such as exclusivity or non-local dependency \cite{aalst2022pmhandbook}. In addition, on the right are six traces, of which five exhibit skipped, wrongly-ordered and unknown control-flow anomalies. For example, $\langle$\texttt{a b d}$\rangle$ has a skipped activity, which is \texttt{c}. Because of this skipped activity, the control-flow relation \texttt{b}$\,\#\,$\texttt{d} is violated, since \texttt{d} directly follows \texttt{b} in the anomalous trace.
\begin{figure}[!t]
\centering
\includegraphics[width=0.9\columnwidth]{images/FP_ANOMALIES.png}
\caption{An example event log footprint with six traces, of which five exhibit control-flow anomalies.}
\label{FP_ANOMALIES}
\end{figure}

\subsection{Control-flow anomaly detection}
Control-flow anomaly detection techniques aim to characterize the normal control flow from event logs and verify whether these deviations occur in new event logs \cite{ko2023adsystematicreview}. To develop control-flow anomaly detection techniques, \revision{process mining} has seen widespread adoption owing to process discovery and \revision{conformance checking}. On the one hand, process discovery is a set of algorithms that encode control-flow relations as a set of model elements and constraints according to a given modeling formalism \cite{aalst2022pmhandbook}; hereafter, we refer to the Petri net, a widespread modeling formalism. On the other hand, \revision{conformance checking} is an explainable set of algorithms that allows linking any deviations with the reference Petri net and providing the fitness measure, namely a measure of how much the Petri net fits the new event log \cite{aalst2022pmhandbook}. Many control-flow anomaly detection techniques based on \revision{conformance checking} (hereafter, \revision{conformance checking}-based techniques) use the fitness measure to determine whether an event log is anomalous \cite{bezerra2009pmad, bezerra2013adlogspais, myers2018icsadpm, pecchia2020applicationfailuresanalysispm}. 

The scientific literature also includes many \revision{conformance checking}-independent techniques for control-flow anomaly detection that combine specific types of trace encodings with machine/deep learning \cite{ko2023adsystematicreview, tavares2023pmtraceencoding}. Whereas these techniques are very effective, their explainability is challenging due to both the type of trace encoding employed and the machine/deep learning model used \cite{rawal2022trustworthyaiadvances,li2023explainablead}. Hence, in the following, we focus on the shortcomings of \revision{conformance checking}-based techniques to investigate whether it is possible to support the development of competitive control-flow anomaly detection techniques while maintaining the explainable nature of \revision{conformance checking}.
\begin{figure}[!t]
\centering
\includegraphics[width=\columnwidth]{images/HIGH_LEVEL_VIEW.png}
\caption{A high-level view of the proposed framework for combining \revision{process mining}-based feature extraction with dimensionality reduction for control-flow anomaly detection.}
\label{HIGH_LEVEL_VIEW}
\end{figure}

\subsection{Shortcomings of \revision{conformance checking}-based techniques}
Unfortunately, the detection effectiveness of \revision{conformance checking}-based techniques is affected by noisy data and low-quality Petri nets, which may be due to human errors in the modeling process or representational bias of process discovery algorithms \cite{bezerra2013adlogspais, pecchia2020applicationfailuresanalysispm, aalst2016pm}. Specifically, on the one hand, noisy data may introduce infrequent and deceptive control-flow relations that may result in inconsistent fitness measures, whereas, on the other hand, checking event logs against a low-quality Petri net could lead to an unreliable distribution of fitness measures. Nonetheless, such Petri nets can still be used as references to obtain insightful information for \revision{process mining}-based feature extraction, supporting the development of competitive and explainable \revision{conformance checking}-based techniques for control-flow anomaly detection despite the problems above. For example, a few works outline that token-based \revision{conformance checking} can be used for \revision{process mining}-based feature extraction to build tabular data and develop effective \revision{conformance checking}-based techniques for control-flow anomaly detection \cite{singh2022lapmsh, debenedictis2023dtadiiot}. However, to the best of our knowledge, the scientific literature lacks a structured proposal for \revision{process mining}-based feature extraction using the state-of-the-art \revision{conformance checking} variant, namely alignment-based \revision{conformance checking}.

\subsection{Contributions}
We propose a novel \revision{process mining}-based feature extraction approach with alignment-based \revision{conformance checking}. This variant aligns the deviating control flow with a reference Petri net; the resulting alignment can be inspected to extract additional statistics such as the number of times a given activity caused mismatches \cite{aalst2022pmhandbook}. We integrate this approach into a flexible and explainable framework for developing techniques for control-flow anomaly detection. The framework combines \revision{process mining}-based feature extraction and dimensionality reduction to handle high-dimensional feature sets, achieve detection effectiveness, and support explainability. Notably, in addition to our proposed \revision{process mining}-based feature extraction approach, the framework allows employing other approaches, enabling a fair comparison of multiple \revision{conformance checking}-based and \revision{conformance checking}-independent techniques for control-flow anomaly detection. Figure \ref{HIGH_LEVEL_VIEW} shows a high-level view of the framework. Business processes are monitored, and event logs obtained from the database of information systems. Subsequently, \revision{process mining}-based feature extraction is applied to these event logs and tabular data input to dimensionality reduction to identify control-flow anomalies. We apply several \revision{conformance checking}-based and \revision{conformance checking}-independent framework techniques to publicly available datasets, simulated data of a case study from railways, and real-world data of a case study from healthcare. We show that the framework techniques implementing our approach outperform the baseline \revision{conformance checking}-based techniques while maintaining the explainable nature of \revision{conformance checking}.

In summary, the contributions of this paper are as follows.
\begin{itemize}
    \item{
        A novel \revision{process mining}-based feature extraction approach to support the development of competitive and explainable \revision{conformance checking}-based techniques for control-flow anomaly detection.
    }
    \item{
        A flexible and explainable framework for developing techniques for control-flow anomaly detection using \revision{process mining}-based feature extraction and dimensionality reduction.
    }
    \item{
        Application to synthetic and real-world datasets of several \revision{conformance checking}-based and \revision{conformance checking}-independent framework techniques, evaluating their detection effectiveness and explainability.
    }
\end{itemize}

The rest of the paper is organized as follows.
\begin{itemize}
    \item Section \ref{sec:related_work} reviews the existing techniques for control-flow anomaly detection, categorizing them into \revision{conformance checking}-based and \revision{conformance checking}-independent techniques.
    \item Section \ref{sec:abccfe} provides the preliminaries of \revision{process mining} to establish the notation used throughout the paper, and delves into the details of the proposed \revision{process mining}-based feature extraction approach with alignment-based \revision{conformance checking}.
    \item Section \ref{sec:framework} describes the framework for developing \revision{conformance checking}-based and \revision{conformance checking}-independent techniques for control-flow anomaly detection that combine \revision{process mining}-based feature extraction and dimensionality reduction.
    \item Section \ref{sec:evaluation} presents the experiments conducted with multiple framework and baseline techniques using data from publicly available datasets and case studies.
    \item Section \ref{sec:conclusions} draws the conclusions and presents future work.
\end{itemize}
% Consider a lasso optimization procedure with potentially distinct regularization penalties:
% \begin{align}
%     \hat{\beta} = \arg\min_{\beta}\{\|y-X\beta\|^2_2+\sum_{i=1}^{N}\lambda_i|\beta_i|\}.
% \end{align}
\subsection{Supervised Data-Driven Learning}\label{subsec:supervised}
We consider a generic data-driven supervised learning procedure. Given a dataset \( \mathcal{D} \) consisting of \( n \) data points \( (x_i, y_i) \in \mathcal{X} \times \mathcal{Y} \) drawn from an underlying distribution \( p(\cdot|\theta) \), our goal is to estimate parameters \( \theta \in \Theta \) through a learning procedure, defined as \( f: (\mathcal{X} \times \mathcal{Y})^n \rightarrow \Theta \) 
that minimizes the predictive error on observed data. 
Specifically, the learning objective is defined as follows:
\begin{align}
\hat{\theta}_f := f(\mathcal{D}) = \arg\min_{\theta} \mathcal{L}(\theta, \mathcal{D}),
\end{align}
where \( \mathcal{L}(\cdot,\mathcal{D}) := \sum_{i=1}^{n} \mathcal{L}(\cdot, (x_i, y_i))\), and $\mathcal{L}$ is a loss function quantifying the error between predictions and true outcomes. 
Here, $\hat{\theta}_f$ is the parameter that best explains the observed data pairs \( (x_i, y_i) \) according to the chosen loss function \( \mathcal{L} (\cdot) \).

\paragraph{Feature Selection.}
Feature selection aims to improve model \( f \)'s predictive performance while minimizing redundancy. 
%Formally, given data \( X \), response \( y \), feature set \( \mathcal{F} \), loss function \( \mathcal{L}(\cdot) \), and a feature limit \( k \), the objective is:
% \begin{align}
% \mathcal{S}^* = \arg \min_{\mathcal{S} \subseteq \mathcal{F}, |\mathcal{S}| \leq k} \mathcal{L}(y, f(X_\mathcal{S})) + \lambda R(\mathcal{S}),
% \end{align}
% where \( X_\mathcal{S} \) is the submatrix of \( X \) for selected features \( \mathcal{S} \), \( \lambda \) is a regularization parameter, and \( R(\mathcal{S}) \) penalizes feature redundancy.
 State-of-the-art techniques fall into four categories: (i) filter methods, which rank features based on statistical properties like Fisher score \citep{duda2001pattern,song2012feature}; (ii) wrapper methods, which evaluate model performance on different feature subsets \citep{kohavi1997wrappers}; (iii) embedded methods, which integrate feature selection into the learning process using techniques like regularization \citep{tibshirani1996LASSO,lemhadri2021lassonet}; and (iv) hybrid methods, which combine elements of (i)-(iii) \citep{SINGH2021104396,li2022micq}. This paper focuses on embedded methods via Lasso, benchmarking against approaches from (i)-(iii).

\subsection{Language Modeling}
% The objective of language modeling is to learn a probability distribution \( p_{LM}(x) \) over sequences of text \( x = (X_1, \ldots, X_{|x|}) \), such that \( p_{LM}(x) \approx p_{text}(x) \), where \( p_{text}(x) \) represents the true distribution of natural language. This process involves estimating the likelihood of token sequences across variable lengths and diverse linguistic structures.
% Modern large language models (LLMs) are trained on vast datasets spanning encyclopedias, news, social media, books, and scientific papers \cite{gao2020pile}. This broad training enables them to generalize across domains, learn contextual knowledge, and perform zero-shot learning—tackling new tasks using only task descriptions without fine-tuning \cite{brown2020gpt3}.
Language modeling aims to approximate the true distribution of natural language \( p_{\text{text}}(x) \) by learning \( p_{\text{LM}}(x) \), a probability distribution over text sequences \( x = (X_1, \ldots, X_{|x|}) \). Modern large language models, trained on diverse datasets \citep{gao2020pile}, exhibit strong generalization across domains, acquire contextual knowledge, and perform zero-shot learning—solving new tasks using only task descriptions—or few-shot learning by leveraging a small number of demonstrations \citep{brown2020gpt3}.
\paragraph{Retrieval-Augmented Generation (RAG).} Retrieval-Augmented Generation (RAG) enhances the performance of generative language models by  integrating a domain-specific information retrieval process  \citep{lewis2020retrieval}. The RAG framework comprises two main components: \textit{retrieval}, which extracts relevant information from external knowledge sources, and \textit{generation}, where an LLM generates context-aware responses using the prompt combined with the retrieved context. Documents are indexed through various databases, such as relational, graph, or vector databases \citep{khattab2020colbert, douze2024faiss, peng2024graphretrievalaugmentedgenerationsurvey}, enabling efficient organization and retrieval via algorithms like semantic similarity search to match the prompt with relevant documents in the knowledge base. RAG has gained much traction recently due to its demonstrated ability to reduce incidence of hallucinations and boost LLMs' reliability as well as performance \citep{huang2023hallucination, zhang2023merging}. 
 
% image source: https://medium.com/@bindurani_22/retrieval-augmented-generation-815c1ae438d8
\begin{figure}
    \centering
\includegraphics[width=1.03\linewidth]{fig/fig1.pdf}
\vspace{-0.6cm}
\scriptsize 
    \caption{Retrieval Augmented Generation (RAG) based $\ell_1$-norm weights (penalty factors) for Lasso. Only feature names---no training data--- are included in LLM prompt.} 
    \label{fig:rag}
\end{figure}
% However, for the RAG model to be effective given the input token constraints of the LLM model used, we need to effectively process the retrieval documents through a procedure known as \textit{chunking}.

\subsection{Task-Specific Data-Driven Learning}
LLM-Lasso aims to bridge the gap between data-driven supervised learning and the predictive capabilities of LLMs trained on rich metadata. This fusion not only enhances traditional data-driven methods by incorporating key task-relevant contextual information often overlooked by such models, but can also be especially valuable in low-data regimes, where the learning algorithm $f:\mathcal{D}\rightarrow\Theta$ (seen as a map from datasets $\mathcal{D}$ to the space of decisions $\Theta$) is susceptible to overfitting.

The task-specific data-driven learning model $\tilde{f}:\mathcal{D}\times\mathcal{D}_\text{meta}\rightarrow\Theta$ can be described as a metadata-augmented version of $f$, where a link function $h(\cdot)$ integrates metadata (i.e. $\mathcal{D}_\text{meta}$) to refine the original learning process. This can be expressed as:
\[
\tilde{f}(\mathcal{D}, \mathcal{D}_\text{meta}) := \mathcal{T}(f(\mathcal{D}),  h(\mathcal{D}_{\text{meta}})),
\]
where the functional $\mathcal{T}$ takes the original learning algorithm $f(\mathcal{D})$ and transforms it into a task-specific learning algorithm $\tilde{f}(\mathcal{D}, \mathcal{D}_\text{meta})$ by incorporating the metadata $\mathcal{D}_\text{meta}$. 
% In particular, the link function $h(\mathcal{D}_{\text{meta}})$ provides a structured mechanism summarizing the contextual knowledge.

There are multiple approaches to formulate $\mathcal{T}$ and $h$.
%to ``inform" the data-driven model $f$ of %meta knowledge. 
For instance, LMPriors \citep{choi2022lmpriorspretrainedlanguagemodels} designed $h$ and $\mathcal{T}$ such that $h(\mathcal{D}_{\text{meta}})$ first specifies which features to retain (based on a probabilistic prior framework), and then $\mathcal{T}$ keeps the selected features and removes all the others from the original learning objective of $f$. 
Note that this approach inherently is restricted as it selects important features solely based on $\mathcal{D}_\text{meta}$ without seeing $\mathcal{D}$.

In contrast, we directly embed task-specific knowledge into the optimization landscape through regularization by introducing a structured inductive bias. This bias guides the learning process toward solutions that are consistent with metadata-informed insights, without relying on explicit probabilistic modeling. Abstractly, this can be expressed as:
\begin{align}
    \!\!\!\!\!\hat{\theta}_{\tilde{f}} := \tilde{f}(\mathcal{D},\mathcal{D}
    _\text{meta})= \arg\min_{\theta} \mathcal{L}(\theta, \mathcal{D}) + \lambda R(\theta, \mathcal{D}_{\text{meta}}),
\end{align}
where \( \lambda \) is a regularization parameter, \( R(\cdot) \) is a regularizer, and $\theta$ is the prediction parameter.
%We explain our framework with more details in the following section.


% Our research diverges from both aforementioned approaches by positioning the LLM not as a standalone feature selector but as an enhancement to data-driven models through an embedded feature selection method, L-LASSO. L-LASSO incorporates domain expertise—auxiliary natural language metadata about the task—via the LLM-informed LASSO penalty, which is then used in statistical models to enhance predictive performance. This method integrates the rich, context-sensitive insights of LLMs with the rigor and transparency of statistical modeling, bridging the gap between data-driven and knowledge-driven feature selection approaches. To approach this task, we need to tackle two key components: (i). train an LLM that is expert in the task-specific knowledge; (ii). inform data-driven feature selector LASSO with LLM knowledge.

% In practice, this involves combining techniques like prompt engineering and data engineering to develop an effective framework for integrating metadata into existing data-driven models. We will go through this in detail in Section \ref{mthd} and \ref{experiment}.


\section{Research Methodology}~\label{sec:Methodology}

In this section, we discuss the process of conducting our systematic review, e.g., our search strategy for data extraction of relevant studies, based on the guidelines of Kitchenham et al.~\cite{kitchenham2022segress} to conduct SLRs and Petersen et al.~\cite{PETERSEN20151} to conduct systematic mapping studies (SMSs) in Software Engineering. In this systematic review, we divide our work into a four-stage procedure, including planning, conducting, building a taxonomy, and reporting the review, illustrated in Fig.~\ref{fig:search}. The four stages are as follows: (1) the \emph{planning} stage involved identifying research questions (RQs) and specifying the detailed research plan for the study; (2) the \emph{conducting} stage involved analyzing and synthesizing the existing primary studies to answer the research questions; (3) the \emph{taxonomy} stage was introduced to optimize the data extraction results and consolidate a taxonomy schema for REDAST methodology; (4) the \emph{reporting} stage involved the reviewing, concluding and reporting the final result of our study.

\begin{figure}[!t]
    \centering
    \includegraphics[width=1\linewidth]{fig/methodology/searching-process.drawio.pdf}
    \caption{Systematic Literature Review Process}
    \label{fig:search}
\end{figure}

\subsection{Research Questions}
In this study, we developed five research questions (RQs) to identify the input and output, analyze technologies, evaluate metrics, identify challenges, and identify potential opportunities. 

\textbf{RQ1. What are the input configurations, formats, and notations used in the requirements in requirements-driven
automated software testing?} In requirements-driven testing, the input is some form of requirements specification -- which can vary significantly. RQ1 maps the input for REDAST and reports on the comparison among different formats for requirements specification.

\textbf{RQ2. What are the frameworks, tools, processing methods, and transformation techniques used in requirements-driven automated software testing studies?} RQ2 explores the technical solutions from requirements to generated artifacts, e.g., rule-based transformation applying natural language processing (NLP) pipelines and deep learning (DL) techniques, where we additionally discuss the potential intermediate representation and additional input for the transformation process.

\textbf{RQ3. What are the test formats and coverage criteria used in the requirements-driven automated software
testing process?} RQ3 focuses on identifying the formulation of generated artifacts (i.e., the final output). We map the adopted test formats and analyze their characteristics in the REDAST process.

\textbf{RQ4. How do existing studies evaluate the generated test artifacts in the requirements-driven automated software testing process?} RQ4 identifies the evaluation datasets, metrics, and case study methodologies in the selected papers. This aims to understand how researchers assess the effectiveness, accuracy, and practical applicability of the generated test artifacts.

\textbf{RQ5. What are the limitations and challenges of existing requirements-driven automated software testing methods in the current era?} RQ5 addresses the limitations and challenges of existing studies while exploring future directions in the current era of technology development. %It particularly highlights the potential benefits of advanced LLMs and examines their capacity to meet the high expectations placed on these cutting-edge language modeling technologies. %\textcolor{blue}{CA: Do we really need to focus on LLMs? TBD.} \textcolor{orange}{FW: About LLMs, I removed the direct emphase in RQ5 but kept the discussion in RQ5 and the solution section. I think that would be more appropriate.}

\subsection{Searching Strategy}

The overview of the search process is exhibited in Fig. \ref{fig:papers}, which includes all the details of our search steps.
\begin{table}[!ht]
\caption{List of Search Terms}
\label{table:search_term}
\begin{tabularx}{\textwidth}{lX}
\hline
\textbf{Terms Group} & \textbf{Terms} \\ \hline
Test Group & test* \\
Requirement Group & requirement* OR use case* OR user stor* OR specification* \\
Software Group & software* OR system* \\
Method Group & generat* OR deriv* OR map* OR creat* OR extract* OR design* OR priorit* OR construct* OR transform* \\ \hline
\end{tabularx}
\end{table}

\begin{figure}
    \centering
    \includegraphics[width=1\linewidth]{fig/methodology/search-papers.drawio.pdf}
    \caption{Study Search Process}
    \label{fig:papers}
\end{figure}

\subsubsection{Search String Formulation}
Our research questions (RQs) guided the identification of the main search terms. We designed our search string with generic keywords to avoid missing out on any related papers, where four groups of search terms are included, namely ``test group'', ``requirement group'', ``software group'', and ``method group''. In order to capture all the expressions of the search terms, we use wildcards to match the appendix of the word, e.g., ``test*'' can capture ``testing'', ``tests'' and so on. The search terms are listed in Table~\ref{table:search_term}, decided after iterative discussion and refinement among all the authors. As a result, we finally formed the search string as follows:


\hangindent=1.5em
 \textbf{ON ABSTRACT} ((``test*'') \textbf{AND} (``requirement*'' \textbf{OR} ``use case*'' \textbf{OR} ``user stor*'' \textbf{OR} ``specifications'') \textbf{AND} (``software*'' \textbf{OR} ``system*'') \textbf{AND} (``generat*'' \textbf{OR} ``deriv*'' \textbf{OR} ``map*'' \textbf{OR} ``creat*'' \textbf{OR} ``extract*'' \textbf{OR} ``design*'' \textbf{OR} ``priorit*'' \textbf{OR} ``construct*'' \textbf{OR} ``transform*''))

The search process was conducted in September 2024, and therefore, the search results reflect studies available up to that date. We conducted the search process on six online databases: IEEE Xplore, ACM Digital Library, Wiley, Scopus, Web of Science, and Science Direct. However, some databases were incompatible with our default search string in the following situations: (1) unsupported for searching within abstract, such as Scopus, and (2) limited search terms, such as ScienceDirect. Here, for (1) situation, we searched within the title, keyword, and abstract, and for (2) situation, we separately executed the search and removed the duplicate papers in the merging process. 

\subsubsection{Automated Searching and Duplicate Removal}
We used advanced search to execute our search string within our selected databases, following our designed selection criteria in Table \ref{table:selection}. The first search returned 27,333 papers. Specifically for the duplicate removal, we used a Python script to remove (1) overlapped search results among multiple databases and (2) conference or workshop papers, also found with the same title and authors in the other journals. After duplicate removal, we obtained 21,652 papers for further filtering.

\begin{table*}[]
\caption{Selection Criteria}
\label{table:selection}
\begin{tabularx}{\textwidth}{lX}
\hline
\textbf{Criterion ID} & \textbf{Criterion Description} \\ \hline
S01          & Papers written in English. \\
S02-1        & Papers in the subjects of "Computer Science" or "Software Engineering". \\
S02-2        & Papers published on software testing-related issues. \\
S03          & Papers published from 1991 to the present. \\ 
S04          & Papers with accessible full text. \\ \hline
\end{tabularx}
\end{table*}

\begin{table*}[]
\small
\caption{Inclusion and Exclusion Criteria}
\label{table:criteria}
\begin{tabularx}{\textwidth}{lX}
\hline
\textbf{ID}  & \textbf{Description} \\ \hline
\multicolumn{2}{l}{\textbf{Inclusion Criteria}} \\ \hline
I01 & Papers about requirements-driven automated system testing or acceptance testing generation, or studies that generate system-testing-related artifacts. \\
I02 & Peer-reviewed studies that have been used in academia with references from literature. \\ \hline
\multicolumn{2}{l}{\textbf{Exclusion Criteria}} \\ \hline
E01 & Studies that only support automated code generation, but not test-artifact generation. \\
E02 & Studies that do not use requirements-related information as an input. \\
E03 & Papers with fewer than 5 pages (1-4 pages). \\
E04 & Non-primary studies (secondary or tertiary studies). \\
E05 & Vision papers and grey literature (unpublished work), books (chapters), posters, discussions, opinions, keynotes, magazine articles, experience, and comparison papers. \\ \hline
\end{tabularx}
\end{table*}

\subsubsection{Filtering Process}

In this step, we filtered a total of 21,652 papers using the inclusion and exclusion criteria outlined in Table \ref{table:criteria}. This process was primarily carried out by the first and second authors. Our criteria are structured at different levels, facilitating a multi-step filtering process. This approach involves applying various criteria in three distinct phases. We employed a cross-verification method involving (1) the first and second authors and (2) the other authors. Initially, the filtering was conducted separately by the first and second authors. After cross-verifying their results, the results were then reviewed and discussed further by the other authors for final decision-making. We widely adopted this verification strategy within the filtering stages. During the filtering process, we managed our paper list using a BibTeX file and categorized the papers with color-coding through BibTeX management software\footnote{\url{https://bibdesk.sourceforge.io/}}, i.e., “red” for irrelevant papers, “yellow” for potentially relevant papers, and “blue” for relevant papers. This color-coding system facilitated the organization and review of papers according to their relevance.

The screening process is shown below,
\begin{itemize}
    \item \textbf{1st-round Filtering} was based on the title and abstract, using the criteria I01 and E01. At this stage, the number of papers was reduced from 21,652 to 9,071.
    \item \textbf{2nd-round Filtering}. We attempted to include requirements-related papers based on E02 on the title and abstract level, which resulted from 9,071 to 4,071 papers. We excluded all the papers that did not focus on requirements-related information as an input or only mentioned the term ``requirements'' but did not refer to the requirements specification.
    \item \textbf{3rd-round Filtering}. We selectively reviewed the content of papers identified as potentially relevant to requirements-driven automated test generation. This process resulted in 162 papers for further analysis.
\end{itemize}
Note that, especially for third-round filtering, we aimed to include as many relevant papers as possible, even borderline cases, according to our criteria. The results were then discussed iteratively among all the authors to reach a consensus.

\subsubsection{Snowballing}

Snowballing is necessary for identifying papers that may have been missed during the automated search. Following the guidelines by Wohlin~\cite{wohlin2014guidelines}, we conducted both forward and backward snowballing. As a result, we identified 24 additional papers through this process.

\subsubsection{Data Extraction}

Based on the formulated research questions (RQs), we designed 38 data extraction questions\footnote{\url{https://drive.google.com/file/d/1yjy-59Juu9L3WHaOPu-XQo-j-HHGTbx_/view?usp=sharing}} and created a Google Form to collect the required information from the relevant papers. The questions included 30 short-answer questions, six checkbox questions, and two selection questions. The data extraction was organized into five sections: (1) basic information: fundamental details such as title, author, venue, etc.; (2) open information: insights on motivation, limitations, challenges, etc.; (3) requirements: requirements format, notation, and related aspects; (4) methodology: details, including immediate representation and technique support; (5) test-related information: test format(s), coverage, and related elements. Similar to the filtering process, the first and second authors conducted the data extraction and then forwarded the results to the other authors to initiate the review meeting.

\subsubsection{Quality Assessment}

During the data extraction process, we encountered papers with insufficient information. To address this, we conducted a quality assessment in parallel to ensure the relevance of the papers to our objectives. This approach, also adopted in previous secondary studies~\cite{shamsujjoha2021developing, naveed2024model}, involved designing a set of assessment questions based on guidelines by Kitchenham et al.~\cite{kitchenham2022segress}. The quality assessment questions in our study are shown below:
\begin{itemize}
    \item \textbf{QA1}. Does this study clearly state \emph{how} requirements drive automated test generation?
    \item \textbf{QA2}. Does this study clearly state the \emph{aim} of REDAST?
    \item \textbf{QA3}. Does this study enable \emph{automation} in test generation?
    \item \textbf{QA4}. Does this study demonstrate the usability of the method from the perspective of methodology explanation, discussion, case examples, and experiments?
\end{itemize}
QA4 originates from an open perspective in the review process, where we focused on evaluation, discussion, and explanation. Our review also examined the study’s overall structure, including the methodology description, case studies, experiments, and analyses. The detailed results of the quality assessment are provided in the Appendix. Following this assessment, the final data extraction was based on 156 papers.

% \begin{table}[]
% \begin{tabular}{ll}
% \hline
% QA ID & QA Questions                                             \\ \hline
% Q01   & Does this study clearly state its aims?                  \\
% Q02   & Does this study clearly describe its methodology?        \\
% Q03   & Does this study involve automated test generation?       \\
% Q04   & Does this study include a promising evaluation?          \\
% Q05   & Does this study demonstrate the usability of the method? \\ \hline
% \end{tabular}%
% \caption{Questions for Quality Assessment}
% \label{table:qa}
% \end{table}

% automated quality assessment

% \textcolor{blue}{CA: Our search strategy focused on identifying requirements types first. We covered several sources, e.g., ~\cite{Pohl:11,wagner2019status} to identify different formats and notations of specifying requirements. However, this came out to be a long list, e.g., free-form NL requirements, semi-formal UML models, free-from textual use case models, UML class diagrams, UML activity diagrams, and so on. In this paper, we attempted to primarily focus on requirements-related aspects and not design-level information. Hence, we generalised our search string to include generic keywords, e.g., requirement*, use case*, and user stor*. We did so to avoid missing out on any papers, bringing too restrictive in our search strategy, and not creating a too-generic search string with all the aforementioned formats to avoid getting results beyond our review's scope.}


%% Use \subsection commands to start a subsection.



%\subsection{Study Selection}

% In this step, we further looked into the content of searched papers using our search strategy and applied our inclusion and exclusion criteria. Our filtering strategy aimed to pinpoint studies focused on requirements-driven system-level testing. Recognizing the presence of irrelevant papers in our search results, we established detailed selection criteria for preliminary inclusion and exclusion, as shown in Table \ref{table: criteria}. Specifically, we further developed the taxonomy schema to exclude two types of studies that did not meet the requirements for system-level testing: (1) studies supporting specification-driven test generation, such as UML-driven test generation, rather than requirements-driven testing, and (2) studies focusing on code-based test generation, such as requirement-driven code generation for unit testing.




\section{Experiments}
\label{sec:exp}
Following the settings in Section \ref{sec:existing}, we evaluate \textit{NovelSum}'s correlation with the fine-tuned model performance across 53 IT datasets and compare it with previous diversity metrics. Additionally, we conduct a correlation analysis using Qwen-2.5-7B \cite{yang2024qwen2} as the backbone model, alongside previous LLaMA-3-8B experiments, to further demonstrate the metric's effectiveness across different scenarios. Qwen is used for both instruction tuning and deriving semantic embeddings. Due to resource constraints, we run each strategy on Qwen for two rounds, resulting in 25 datasets. 

\subsection{Main Results}

\begin{table*}[!t]
    \centering
    \resizebox{\linewidth}{!}{
    \begin{tabular}{lcccccccccc}
    \toprule
    \multirow{3}*{\textbf{Diversity Metrics}} & \multicolumn{10}{c}{\textbf{Data Selection Strategies}} \\
    \cmidrule(lr){2-11}
    & \multirow{2}*{\textbf{K-means}} & \multirow{2}*{\vtop{\hbox{\textbf{K-Center}}\vspace{1mm}\hbox{\textbf{-Greedy}}}}  & \multirow{2}*{\textbf{QDIT}} & \multirow{2}*{\vtop{\hbox{\textbf{Repr}}\vspace{1mm}\hbox{\textbf{Filter}}}} & \multicolumn{5}{c}{\textbf{Random}} & \multirow{2}{*}{\textbf{Duplicate}} \\ 
    \cmidrule(lr){6-10}
    & & & & & \textbf{$\mathcal{X}^{all}$} & ShareGPT & WizardLM & Alpaca & Dolly &  \\
    \midrule
    \rowcolor{gray!15} \multicolumn{11}{c}{\textit{LLaMA-3-8B}} \\
    Facility Loc. $_{\times10^5}$ & \cellcolor{BLUE!40} 2.99 & \cellcolor{ORANGE!10} 2.73 & \cellcolor{BLUE!40} 2.99 & \cellcolor{BLUE!20} 2.86 & \cellcolor{BLUE!40} 2.99 & \cellcolor{BLUE!0} 2.83 & \cellcolor{BLUE!30} 2.88 & \cellcolor{BLUE!0} 2.83 & \cellcolor{ORANGE!20} 2.59 & \cellcolor{ORANGE!30} 2.52 \\    
    DistSum$_{cosine}$  & \cellcolor{BLUE!30} 0.648 & \cellcolor{BLUE!60} 0.746 & \cellcolor{BLUE!0} 0.629 & \cellcolor{BLUE!50} 0.703 & \cellcolor{BLUE!10} 0.634 & \cellcolor{BLUE!40} 0.656 & \cellcolor{ORANGE!30} 0.578 & \cellcolor{ORANGE!10} 0.605 & \cellcolor{ORANGE!20} 0.603 & \cellcolor{BLUE!10} 0.634 \\
    Vendi Score $_{\times10^7}$ & \cellcolor{BLUE!30} 1.70 & \cellcolor{BLUE!60} 2.53 & \cellcolor{BLUE!10} 1.59 & \cellcolor{BLUE!50} 2.23 & \cellcolor{BLUE!20} 1.61 & \cellcolor{BLUE!30} 1.70 & \cellcolor{ORANGE!10} 1.44 & \cellcolor{ORANGE!20} 1.32 & \cellcolor{ORANGE!10} 1.44 & \cellcolor{ORANGE!30} 0.05 \\
    \textbf{NovelSum (Ours)} & \cellcolor{BLUE!60} 0.693 & \cellcolor{BLUE!50} 0.687 & \cellcolor{BLUE!30} 0.673 & \cellcolor{BLUE!20} 0.671 & \cellcolor{BLUE!40} 0.675 & \cellcolor{BLUE!10} 0.628 & \cellcolor{BLUE!0} 0.591 & \cellcolor{ORANGE!10} 0.572 & \cellcolor{ORANGE!20} 0.50 & \cellcolor{ORANGE!30} 0.461 \\
    \midrule    
    \textbf{Model Performance} & \cellcolor{BLUE!60}1.32 & \cellcolor{BLUE!50}1.31 & \cellcolor{BLUE!40}1.25 & \cellcolor{BLUE!30}1.05 & \cellcolor{BLUE!20}1.20 & \cellcolor{BLUE!10}0.83 & \cellcolor{BLUE!0}0.72 & \cellcolor{ORANGE!10}0.07 & \cellcolor{ORANGE!20}-0.14 & \cellcolor{ORANGE!30}-1.35 \\
    \midrule
    \midrule
    \rowcolor{gray!15} \multicolumn{11}{c}{\textit{Qwen-2.5-7B}} \\
    Facility Loc. $_{\times10^5}$ & \cellcolor{BLUE!40} 3.54 & \cellcolor{ORANGE!30} 3.42 & \cellcolor{BLUE!40} 3.54 & \cellcolor{ORANGE!20} 3.46 & \cellcolor{BLUE!40} 3.54 & \cellcolor{BLUE!30} 3.51 & \cellcolor{BLUE!10} 3.50 & \cellcolor{BLUE!10} 3.50 & \cellcolor{ORANGE!20} 3.46 & \cellcolor{BLUE!0} 3.48 \\ 
    DistSum$_{cosine}$ & \cellcolor{BLUE!30} 0.260 & \cellcolor{BLUE!60} 0.440 & \cellcolor{BLUE!0} 0.223 & \cellcolor{BLUE!50} 0.421 & \cellcolor{BLUE!10} 0.230 & \cellcolor{BLUE!40} 0.285 & \cellcolor{ORANGE!20} 0.211 & \cellcolor{ORANGE!30} 0.189 & \cellcolor{ORANGE!10} 0.221 & \cellcolor{BLUE!20} 0.243 \\
    Vendi Score $_{\times10^6}$ & \cellcolor{ORANGE!10} 1.60 & \cellcolor{BLUE!40} 3.09 & \cellcolor{BLUE!10} 2.60 & \cellcolor{BLUE!60} 7.15 & \cellcolor{ORANGE!20} 1.41 & \cellcolor{BLUE!50} 3.36 & \cellcolor{BLUE!20} 2.65 & \cellcolor{BLUE!0} 1.89 & \cellcolor{BLUE!30} 3.04 & \cellcolor{ORANGE!30} 0.20 \\
    \textbf{NovelSum (Ours)}  & \cellcolor{BLUE!40} 0.440 & \cellcolor{BLUE!60} 0.505 & \cellcolor{BLUE!20} 0.403 & \cellcolor{BLUE!50} 0.495 & \cellcolor{BLUE!30} 0.408 & \cellcolor{BLUE!10} 0.392 & \cellcolor{BLUE!0} 0.349 & \cellcolor{ORANGE!10} 0.336 & \cellcolor{ORANGE!20} 0.320 & \cellcolor{ORANGE!30} 0.309 \\
    \midrule
    \textbf{Model Performance} & \cellcolor{BLUE!30} 1.06 & \cellcolor{BLUE!60} 1.45 & \cellcolor{BLUE!40} 1.23 & \cellcolor{BLUE!50} 1.35 & \cellcolor{BLUE!20} 0.87 & \cellcolor{BLUE!10} 0.07 & \cellcolor{BLUE!0} -0.08 & \cellcolor{ORANGE!10} -0.38 & \cellcolor{ORANGE!30} -0.49 & \cellcolor{ORANGE!20} -0.43 \\
    \bottomrule
    \end{tabular}
    }
    \caption{Measuring the diversity of datasets selected by different strategies using \textit{NovelSum} and baseline metrics. Fine-tuned model performances (Eq. \ref{eq:perf}), based on MT-bench and AlpacaEval, are also included for cross reference. Darker \colorbox{BLUE!60}{blue} shades indicate higher values for each metric, while darker \colorbox{ORANGE!30}{orange} shades indicate lower values. While data selection strategies vary in performance on LLaMA-3-8B and Qwen-2.5-7B, \textit{NovelSum} consistently shows a stronger correlation with model performance than other metrics. More results are provided in Appendix \ref{app:results}.}
    \label{tbl:main}
    \vspace{-4mm}
\end{table*}


\begin{table}[t!]
\centering
\resizebox{\linewidth}{!}{
\begin{tabular}{lcccc}
\toprule
\multirow{2}*{\textbf{Diversity Metrics}} & \multicolumn{3}{c}{\textbf{LLaMA}} & \textbf{Qwen}\\
\cmidrule(lr){2-4} \cmidrule(lr){5-5} 
& \textbf{Pearson} & \textbf{Spearman} & \textbf{Avg.} & \textbf{Avg.} \\
\midrule
TTR & -0.38 & -0.16 & -0.27 & -0.30 \\
vocd-D & -0.43 & -0.17 & -0.30 & -0.31 \\
\midrule
Facility Loc. & 0.86 & 0.69 & 0.77 & 0.08 \\
Entropy & 0.93 & 0.80 & 0.86 & 0.63 \\
\midrule
LDD & 0.61 & 0.75 & 0.68 & 0.60 \\
KNN Distance & 0.59 & 0.80 & 0.70 & 0.67 \\
DistSum$_{cosine}$ & 0.85 & 0.67 & 0.76 & 0.51 \\
Vendi Score & 0.70 & 0.85 & 0.78 & 0.60 \\
DistSum$_{L2}$ & 0.86 & 0.76 & 0.81 & 0.51 \\
Cluster Inertia & 0.81 & 0.85 & 0.83 & 0.76 \\
Radius & 0.87 & 0.81 & 0.84 & 0.48 \\
\midrule
NovelSum & \textbf{0.98} & \textbf{0.95} & \textbf{0.97} & \textbf{0.90} \\
\bottomrule
\end{tabular}
}
\caption{Correlations between different metrics and model performance on LLaMA-3-8B and Qwen-2.5-7B.  “Avg.” denotes the average correlation (Eq. \ref{eq:cor}).}
\label{tbl:correlations}
\vspace{-2mm}
\end{table}

\paragraph{\textit{NovelSum} consistently achieves state-of-the-art correlation with model performance across various data selection strategies, backbone LLMs, and correlation measures.}
Table \ref{tbl:main} presents diversity measurement results on datasets constructed by mainstream data selection methods (based on $\mathcal{X}^{all}$), random selection from various sources, and duplicated samples (with only $m=100$ unique samples). 
Results from multiple runs are averaged for each strategy.
Although these strategies yield varying performance rankings across base models, \textit{NovelSum} consistently tracks changes in IT performance by accurately measuring dataset diversity. For instance, K-means achieves the best performance on LLaMA with the highest NovelSum score, while K-Center-Greedy excels on Qwen, also correlating with the highest NovelSum. Table \ref{tbl:correlations} shows the correlation coefficients between various metrics and model performance for both LLaMA and Qwen experiments, where \textit{NovelSum} achieves state-of-the-art correlation across different models and measures.

\paragraph{\textit{NovelSum} can provide valuable guidance for data engineering practices.}
As a reliable indicator of data diversity, \textit{NovelSum} can assess diversity at both the dataset and sample levels, directly guiding data selection and construction decisions. For example, Table \ref{tbl:main} shows that the combined data source $\mathcal{X}^{all}$ is a better choice for sampling diverse IT data than other sources. Moreover, \textit{NovelSum} can offer insights through comparative analyses, such as: (1) ShareGPT, which collects data from real internet users, exhibits greater diversity than Dolly, which relies on company employees, suggesting that IT samples from diverse sources enhance dataset diversity \cite{wang2024diversity-logD}; (2) In LLaMA experiments, random selection can outperform some mainstream strategies, aligning with prior work \cite{xia2024rethinking,diddee2024chasing}, highlighting gaps in current data selection methods for optimizing diversity.



\subsection{Ablation Study}


\textit{NovelSum} involves several flexible hyperparameters and variations. In our main experiments, \textit{NovelSum} uses cosine distance to compute $d(x_i, x_j)$ in Eq. \ref{eq:dad}. We set $\alpha = 1$, $\beta = 0.5$, and $K = 10$ nearest neighbors in Eq. \ref{eq:pws} and \ref{eq:dad}. Here, we conduct an ablation study to investigate the impact of these settings based on LLaMA-3-8B.

\begin{table}[ht!]
\centering
\resizebox{\linewidth}{!}{
\begin{tabular}{lccc}
\toprule
\textbf{Variants} & \textbf{Pearson} & \textbf{Spearman} & \textbf{Avg.} \\
\midrule
NovelSum & 0.98 & 0.96 & 0.97 \\
\midrule
\hspace{0.10cm} - Use $L2$ distance & 0.97 & 0.83 & 0.90\textsubscript{↓ 0.08} \\
\hspace{0.10cm} - $K=20$ & 0.98 & 0.96 & 0.97\textsubscript{↓ 0.00} \\
\hspace{0.10cm} - $\alpha=0$ (w/o proximity) & 0.79 & 0.31 & 0.55\textsubscript{↓ 0.42} \\
\hspace{0.10cm} - $\alpha=2$ & 0.73 & 0.88 & 0.81\textsubscript{↓ 0.16} \\
\hspace{0.10cm} - $\beta=0$ (w/o density) & 0.92 & 0.89 & 0.91\textsubscript{↓ 0.07} \\
\hspace{0.10cm} - $\beta=1$ & 0.90 & 0.62 & 0.76\textsubscript{↓ 0.21} \\
\bottomrule
\end{tabular}
}
\caption{Ablation Study for \textit{NovelSum}.}
\label{tbl:ablation}
\vspace{-2mm}
\end{table}

In Table \ref{tbl:ablation}, $\alpha=0$ removes the proximity weights, and $\beta=0$ eliminates the density multiplier. We observe that both $\alpha=0$ and $\beta=0$ significantly weaken the correlation, validating the benefits of the proximity-weighted sum and density-aware distance. Additionally, improper values for $\alpha$ and $\beta$ greatly reduce the metric's reliability, highlighting that \textit{NovelSum} strikes a delicate balance between distances and distribution. Replacing cosine distance with Euclidean distance and using more neighbors for density approximation have minimal impact, particularly on Pearson's correlation, demonstrating \textit{NovelSum}'s robustness to different distance measures.






\vspace{-0.2cm}
\section{Discussion of Assumptions}\label{sec:discussion}
In this paper, we have made several assumptions for the sake of clarity and simplicity. In this section, we discuss the rationale behind these assumptions, the extent to which these assumptions hold in practice, and the consequences for our protocol when these assumptions hold.

\subsection{Assumptions on the Demand}

There are two simplifying assumptions we make about the demand. First, we assume the demand at any time is relatively small compared to the channel capacities. Second, we take the demand to be constant over time. We elaborate upon both these points below.

\paragraph{Small demands} The assumption that demands are small relative to channel capacities is made precise in \eqref{eq:large_capacity_assumption}. This assumption simplifies two major aspects of our protocol. First, it largely removes congestion from consideration. In \eqref{eq:primal_problem}, there is no constraint ensuring that total flow in both directions stays below capacity--this is always met. Consequently, there is no Lagrange multiplier for congestion and no congestion pricing; only imbalance penalties apply. In contrast, protocols in \cite{sivaraman2020high, varma2021throughput, wang2024fence} include congestion fees due to explicit congestion constraints. Second, the bound \eqref{eq:large_capacity_assumption} ensures that as long as channels remain balanced, the network can always meet demand, no matter how the demand is routed. Since channels can rebalance when necessary, they never drop transactions. This allows prices and flows to adjust as per the equations in \eqref{eq:algorithm}, which makes it easier to prove the protocol's convergence guarantees. This also preserves the key property that a channel's price remains proportional to net money flow through it.

In practice, payment channel networks are used most often for micro-payments, for which on-chain transactions are prohibitively expensive; large transactions typically take place directly on the blockchain. For example, according to \cite{river2023lightning}, the average channel capacity is roughly $0.1$ BTC ($5,000$ BTC distributed over $50,000$ channels), while the average transaction amount is less than $0.0004$ BTC ($44.7k$ satoshis). Thus, the small demand assumption is not too unrealistic. Additionally, the occasional large transaction can be treated as a sequence of smaller transactions by breaking it into packets and executing each packet serially (as done by \cite{sivaraman2020high}).
Lastly, a good path discovery process that favors large capacity channels over small capacity ones can help ensure that the bound in \eqref{eq:large_capacity_assumption} holds.

\paragraph{Constant demands} 
In this work, we assume that any transacting pair of nodes have a steady transaction demand between them (see Section \ref{sec:transaction_requests}). Making this assumption is necessary to obtain the kind of guarantees that we have presented in this paper. Unless the demand is steady, it is unreasonable to expect that the flows converge to a steady value. Weaker assumptions on the demand lead to weaker guarantees. For example, with the more general setting of stochastic, but i.i.d. demand between any two nodes, \cite{varma2021throughput} shows that the channel queue lengths are bounded in expectation. If the demand can be arbitrary, then it is very hard to get any meaningful performance guarantees; \cite{wang2024fence} shows that even for a single bidirectional channel, the competitive ratio is infinite. Indeed, because a PCN is a decentralized system and decisions must be made based on local information alone, it is difficult for the network to find the optimal detailed balance flow at every time step with a time-varying demand.  With a steady demand, the network can discover the optimal flows in a reasonably short time, as our work shows.

We view the constant demand assumption as an approximation for a more general demand process that could be piece-wise constant, stochastic, or both (see simulations in Figure \ref{fig:five_nodes_variable_demand}).
We believe it should be possible to merge ideas from our work and \cite{varma2021throughput} to provide guarantees in a setting with random demands with arbitrary means. We leave this for future work. In addition, our work suggests that a reasonable method of handling stochastic demands is to queue the transaction requests \textit{at the source node} itself. This queuing action should be viewed in conjunction with flow-control. Indeed, a temporarily high unidirectional demand would raise prices for the sender, incentivizing the sender to stop sending the transactions. If the sender queues the transactions, they can send them later when prices drop. This form of queuing does not require any overhaul of the basic PCN infrastructure and is therefore simpler to implement than per-channel queues as suggested by \cite{sivaraman2020high} and \cite{varma2021throughput}.

\subsection{The Incentive of Channels}
The actions of the channels as prescribed by the DEBT control protocol can be summarized as follows. Channels adjust their prices in proportion to the net flow through them. They rebalance themselves whenever necessary and execute any transaction request that has been made of them. We discuss both these aspects below.

\paragraph{On Prices}
In this work, the exclusive role of channel prices is to ensure that the flows through each channel remains balanced. In practice, it would be important to include other components in a channel's price/fee as well: a congestion price  and an incentive price. The congestion price, as suggested by \cite{varma2021throughput}, would depend on the total flow of transactions through the channel, and would incentivize nodes to balance the load over different paths. The incentive price, which is commonly used in practice \cite{river2023lightning}, is necessary to provide channels with an incentive to serve as an intermediary for different channels. In practice, we expect both these components to be smaller than the imbalance price. Consequently, we expect the behavior of our protocol to be similar to our theoretical results even with these additional prices.

A key aspect of our protocol is that channel fees are allowed to be negative. Although the original Lightning network whitepaper \cite{poon2016bitcoin} suggests that negative channel prices may be a good solution to promote rebalancing, the idea of negative prices in not very popular in the literature. To our knowledge, the only prior work with this feature is \cite{varma2021throughput}. Indeed, in papers such as \cite{van2021merchant} and \cite{wang2024fence}, the price function is explicitly modified such that the channel price is never negative. The results of our paper show the benefits of negative prices. For one, in steady state, equal flows in both directions ensure that a channel doesn't loose any money (the other price components mentioned above ensure that the channel will only gain money). More importantly, negative prices are important to ensure that the protocol selectively stifles acyclic flows while allowing circulations to flow. Indeed, in the example of Section \ref{sec:flow_control_example}, the flows between nodes $A$ and $C$ are left on only because the large positive price over one channel is canceled by the corresponding negative price over the other channel, leading to a net zero price.

Lastly, observe that in the DEBT control protocol, the price charged by a channel does not depend on its capacity. This is a natural consequence of the price being the Lagrange multiplier for the net-zero flow constraint, which also does not depend on the channel capacity. In contrast, in many other works, the imbalance price is normalized by the channel capacity \cite{ren2018optimal, lin2020funds, wang2024fence}; this is shown to work well in practice. The rationale for such a price structure is explained well in \cite{wang2024fence}, where this fee is derived with the aim of always maintaining some balance (liquidity) at each end of every channel. This is a reasonable aim if a channel is to never rebalance itself; the experiments of the aforementioned papers are conducted in such a regime. In this work, however, we allow the channels to rebalance themselves a few times in order to settle on a detailed balance flow. This is because our focus is on the long-term steady state performance of the protocol. This difference in perspective also shows up in how the price depends on the channel imbalance. \cite{lin2020funds} and \cite{wang2024fence} advocate for strictly convex prices whereas this work and \cite{varma2021throughput} propose linear prices.

\paragraph{On Rebalancing} 
Recall that the DEBT control protocol ensures that the flows in the network converge to a detailed balance flow, which can be sustained perpetually without any rebalancing. However, during the transient phase (before convergence), channels may have to perform on-chain rebalancing a few times. Since rebalancing is an expensive operation, it is worthwhile discussing methods by which channels can reduce the extent of rebalancing. One option for the channels to reduce the extent of rebalancing is to increase their capacity; however, this comes at the cost of locking in more capital. Each channel can decide for itself the optimum amount of capital to lock in. Another option, which we discuss in Section \ref{sec:five_node}, is for channels to increase the rate $\gamma$ at which they adjust prices. 

Ultimately, whether or not it is beneficial for a channel to rebalance depends on the time-horizon under consideration. Our protocol is based on the assumption that the demand remains steady for a long period of time. If this is indeed the case, it would be worthwhile for a channel to rebalance itself as it can make up this cost through the incentive fees gained from the flow of transactions through it in steady state. If a channel chooses not to rebalance itself, however, there is a risk of being trapped in a deadlock, which is suboptimal for not only the nodes but also the channel.

\section{Conclusion}
This work presents DEBT control: a protocol for payment channel networks that uses source routing and flow control based on channel prices. The protocol is derived by posing a network utility maximization problem and analyzing its dual minimization. It is shown that under steady demands, the protocol guides the network to an optimal, sustainable point. Simulations show its robustness to demand variations. The work demonstrates that simple protocols with strong theoretical guarantees are possible for PCNs and we hope it inspires further theoretical research in this direction.
\vspace{-0.1cm}
\section{Conclusion}
In this work, we propose a simple yet effective approach, called SMILE, for graph few-shot learning with fewer tasks. Specifically, we introduce a novel dual-level mixup strategy, including within-task and across-task mixup, for enriching the diversity of nodes within each task and the diversity of tasks. Also, we incorporate the degree-based prior information to learn expressive node embeddings. Theoretically, we prove that SMILE effectively enhances the model's generalization performance. Empirically, we conduct extensive experiments on multiple benchmarks and the results suggest that SMILE significantly outperforms other baselines, including both in-domain and cross-domain few-shot settings.
\section*{Limitations and Ethical Considerations}

\noindent\textbf{Limitations.} The primary limitation of our work is that it extends only the dataset provided by MUSE and employs DeepSeek-v3 for question generation. 
To mitigate this generalization risk, we have released our code and the generated audit suite, allowing researchers to utilize our framework to create additional audit datasets and evaluate their quality. Meanwhile, this is also our future work to extend our framework to other benchmarks.

\noindent\textbf{Ethical Considerations.} Machine unlearning can be employed to mitigate risks associated with LLMs in terms of privacy, security, bias, and copyright. Our work is dedicated to providing a comprehensive evaluation framework to help researchers better understand the unlearning effectiveness of LLMs, which we believe will have a positive impact on society.
Our benchmark comprises contexts and instructions related to multiple real-world safety topics, aiming to provide comprehensive evaluation and support further improvement on LLM safety. However, some of the data may contain toxic contents that could pose potential risks if misused. To mitigate these risks, we plan to conduct careful inspections before open-sourcing the benchmark, and restrict data access to individuals who adhere to stringent ethical guidelines.

During the data collection procedure, we inform the crowd workers from China in advance of the possibility of encountering harmful contents and the future use of the annotated data. Participation of the workers is entirely voluntary, and they are free to withdraw from the study at any time without burden. We pay the workers about 8.5 USD per hour, which is above the average wage of local residents.

\bibliography{GENSHIN}     

\clearpage
\newpage
\appendix
\section*{Appendix A}
\label{AppendixA}

We used a series of data sets in our case studies and as examples in our paper. 

\subsection*{Bike Sharing}
The bike sharing data set is used to predict the number of bike rentals per hour. 

We trained a MLPC Regression model.

We used an 80:20 train:test split resulting in 13903 instances being used for training and in \textsc{Finch}.

In our example case, we used the following features:
\begin{itemize}
    \item count (target): the number of bike rentals that hour
    \item hour=3: the hour for which the bike rentals where recorded. Here: 3am.
    \item workingday=0: if the instance was recorded on a workingday or not. Categoric feature. 0=no, 1=yes.
    \item season=0: in which season the instance was recorded. Categoric feature. 0=winter, 1=spring, 2=summer, 3=autumn.
\end{itemize}



\begin{figure}[h!]
    \centering
    \begin{subfigure}[b]{0.5\textwidth}
        \centering
        \includegraphics[width=1\linewidth]{california1.png}
        \caption{house value per median income}
        \label{fig:california1}
    \end{subfigure}
    \hspace{0.001\textwidth} % Adds some space between the two images
    \begin{subfigure}[b]{0.5\textwidth}
        \centering
        \includegraphics[width=1\linewidth]{california2.png}
        \caption{house value per median income for areas with a population of 2000}
        \label{fig:california2}
    \end{subfigure}
    \hspace{0.001\textwidth} % Adds some space between the two images
    \begin{subfigure}[b]{0.5\textwidth}
        \centering
        \includegraphics[width=1\linewidth]{california3.png}
        \caption{house value per median income for areas with a population of 2000 and 1000 total rooms}
        \label{fig:california3}
    \end{subfigure}
    \hspace{0.001\textwidth} % Adds some space between the two images
    \begin{subfigure}[b]{0.5\textwidth}
        \centering
        \includegraphics[width=1\linewidth]{california4.png}
        \caption{The ground truth is even lower.}
        \label{fig:california4}
    \end{subfigure}
    \caption{The incremental visualization of the interaction of median income, population and total rooms in the california housing data set. Colored areas visualize the change in each step.}
\end{figure}

\subsection*{Titanic}
The titanic data set is used to predict the survival of people on board the titanic. 

We trained an MLPC classifier. The resulting accuracy was 70.23\%.

We used an 80:20 train:test split resulting in 1047 instances being used for training and in \textsc{Finch}.

In the described interaction, we used the following features:
\begin{itemize}
    \item survival(target): If the current person survived.
    \item pclass=1: Which passenger class the current person belonged to. Categoric feature. 1=first class, 2=second class, 3=third class. 
    \item sex=1: The sex of the person. Categoric feature. 0=male, 1=female.
    \item age:30: The age of the person. Here: 30yo.
\end{itemize}




\subsection*{California housing}
The california housing data set is used to predict housing values for block groups in California and was derived from the 190 US census. It contains only continuous variables. The mean predicted housing value is 200.000. 

We trained a GradientBoostingRegressor model on the data. 
It was trained with 100 boosting stages, a learning rate of 0.1 and squared error as the loss function.
The resulting R2 score was 0.77.

We used an 80:20 train:test split resulting in 16,512 instances being used for training and in \textsc{Finch}.

In our observed interaction, we used the following features:
\begin{itemize}
    \item housing value (target): Median house value in US Dollars.
    \item median income: The median income of that block group in 100,000 US Dollars.
    \item population: The number of people residing in the block group.
    \item total rooms: The total number of rooms in that block group.
\end{itemize}

\subsection*{Diabetes}
The diabetes risk factor data set. It is based on the BRFSS telephone study that is performed yearly in the united states.

We used a subset of 10,000 instances for model training. Using a 80:20 train/test split, this resulted in 8000 instnaces being used in \textsc{Finch}.

For better model training, half of the instances are diabetes positive, and half negative. Therefore, the probabilities generated by the model and \textsc{finch} cannot be directly used on a general public.

In our interaction, we considered the following features:
\begin{itemize}
    \item diabetes risk (target): The diabetes risk for the person, that the model predicted.
    \item sex=0: The sex of the person. 0=male, 1=female.
    \item exercise=1: If the person exercises. 1=yes, 0=no.
    \item high blood pressure=1: If the person has high blood pressure. Categoric feature. 1=yes, 0=no.
    \item weight category=3: The weight category of the person. Categoric feature. 0=underweight, 1= normal weight, 2=overweight, 3=obese.  
\end{itemize}

\begin{figure}[h]
    \centering
    \begin{subfigure}[b]{0.5\textwidth}

        \includegraphics[width=0.7\linewidth]{diabetes1.png}
        \caption{diabetes risk per sex}
        \label{fig:diabetes1}
    \end{subfigure}
    \hspace{0.001\textwidth} % Adds some space between the two images
    \begin{subfigure}[b]{0.5\textwidth}
        \centering
        \includegraphics[width=1\linewidth]{diabetes2.png}
        \caption{diabetes risk per sex for people who exercise}
        \label{fig:diabetes2}
    \end{subfigure}
    \hspace{0.001\textwidth} % Adds some space between the two images
    \begin{subfigure}[b]{0.5\textwidth}
        \centering
        \includegraphics[width=1\linewidth]{diabetes3.png}
        \caption{diabetes risk per sex for people who exercise but have high blood pressure}
        \label{fig:diabetes3}
    \end{subfigure}
    \hspace{0.001\textwidth} % Adds some space between the two images
    \begin{subfigure}[b]{0.5\textwidth}
        \centering
        \includegraphics[width=1\linewidth]{diabetes4.png}
        \caption{diabetes risk per sex for people who exercise, have high blood pressure and are obese}
        \label{fig:diabetes4}
    \end{subfigure}
    \caption{Diabetes risk per sex. Incremental interaction of sex, exercise, high blood pressure and weight. Based on the BRFSS data set.}
\end{figure}


\chapter{\textcolor{black}{Edge Network optimization}}\label{app: EN_ib}

In this section the mathematical solution of the optimization problem \eref{eq: EN_ib initial opt problem} in \sref{sec: EN_ib} reported below:

\begin{mini}|s|[0]
    {\mathbf{\Psi}(t)}{\lim_{T \to +\infty}\; \frac{1}{T} \sum_{t=1}^T  \mathbb{E}[P^{tot}(t)] }
    {}{}
    \addConstraint{\lim_{T \to +\infty}\; \frac{1}{T} \sum_{t=1}^T  \mathbb{E}[D_k^{tot}(t)] \leq D_k^{avg}\qquad \forall k }{}
    \addConstraint{ \lim_{T \to +\infty}\; \frac{1}{T} \sum_{t=1}^T  \mathbb{E}[G_k(t)] \leq G_k^{avg}\qquad \forall k }{}
    \addConstraint{0 \leq f_k(t) \leq f_k^{max} \qquad \forall k,t }{}
    \addConstraint{0 \leq R_k(t) \leq R_k^{max}(t) \qquad \forall k,t }{}
    \addConstraint{\beta_k(t) \in \mathcal{B}_k  \qquad \forall k,t}{}
    \addConstraint{0 \leq f^{es}(t) \leq f_{es}^{max} \qquad \forall t}{}
    \addConstraint{f_k^{es}(t) \geq 0 \quad \forall k,t}, \qquad {\sum_{k=1}^K f_k^{es}(t) \leq f_c(t)  \quad \forall t,}{}
\end{mini}

These virtual queues associated to the long-term delay and evaluation metric constraints, $T_k(t)$ and $U_k(t)$ respectively are introduced as follows \cite{Neely2010Lyapunov}:
\begin{align}
    T_k(t+1) &= \max [0,T_k(t) + \varepsilon_k(D_k^{tot}(t) - D_k^{avg})] \\
    U_k(t+1) &= \max [0,U_k(t) + \nu_k(G_k(t) - G_k^{avg})],  
\end{align}
where $\epsilon_k$ and $ \nu_k $ are the learning rate for the update of the virtual queues. 

Based on these virtual queues is possible to define the \textit{Lyapunov function} $L(\mathbf{\Theta}(t))$ as:
\begin{equation}
    L(\mathbf{\Theta}(t)) = \frac{1}{2} \sum_{k=1}^K T_k^2(t) + U_k^2(t),
    \tag{\ref{eq: EN_ib Lyapunov function}}
    \label{app: EN_ib Lyapunov function}
\end{equation}
where $\mathbf{\Theta}(t) = [\{T_k(t)\}_k, \{U_k(t)\}_k]$ is the vector composed by all the virtual queues at time $t$. The idea is to use this Lyapunov function to satisfy the constraints on $D_k^{avg}$ and $G_k^{avg}$ by enforcing the stability of $L(\mathbf{\Theta}(t))$. 

To this scope it is introduced the so called \textit{drift-plus-penalty function}:
\begin{align}
    \Delta(\Theta(t)) &= \mathbb{E}\left[L({\Theta}(t+1))-L({\Theta}(t))+V\cdot P^{tot}(t)  \;\Big|\; \Theta(t)\right] \\
    &=\mathbb{E}\left[\;\sum_{k=1}^K \frac{T_k^2(t+1)-T_k^2(t)}{2} +  \frac{U_k^2(t+1)-U_k^2(t)}{2} +V\cdot P^{tot}(t)\;\; \Big|\;\; \Theta(t)\right]\\
    &= \mathbb{E}\left[\;\sum_{k=1}^K \Delta_{T_k} +  \Delta_{U_k} +V\cdot P^{tot}(t) \;\; \Big|\;\; \Theta(t)\right],
    \label{app: EN_ib drift plus penalty}
\end{align}
where, starting from a generic virtual queue evolving as 
$H(t+1) = \max [0,H(t) +h(t) - \Bar{h}]$ the quantity $\Delta_H$ is defined as follows:
\begin{align*}
    \Delta_H &= \frac{H^2(t+1)-H^2(t)}{2} = \frac{\max [0,(H(t) +h(t) - \Bar{h})^2]-H^2(t)}{2} \\
   &\leq   \frac{(h(t) - \Bar{h})^2}{2} + H(t)[h(t)-\Bar{h}].
\end{align*} 

By applying the same upper bound to $\Delta_{T_k}$ it is possible to obtain:
\begin{align}
    \Delta_{T_k} &= \frac{T_k^2(t+1)-T_k^2(t)}{2} = \frac{\max [0,(T_k(t) + \nu_k(D_k^{tot}(t) - D_k^{avg}))^2]-T_k^2(t)}{2} \\
    &\leq   \nu_k^2\frac{(D_k^{tot}(t) - D_k^{avg})^2}{2} + \nu_k T_k(t)[D_k^{tot}(t) - D_k^{avg}] \\
    &\leq \nu_k^2\frac{(D_k^{max} - D_k^{avg})^2}{2}  + \nu_k T_k(t)[D_k(t) - D_k^{avg}],
    \label{app: EN_ib delta U_k}
\end{align}
where $D_k^{max}(t)$ is the maximum delay allowed for the $k$-th \gls{ed}.

By applying the same reasoning to $\Delta_{U_k}$ it is possible to obtain:
\begin{equation}
    \Delta_{U_k} \leq \nu_k^2\frac{(G_k^{max} - G_k^{avg})^2}{2}  + \nu_k U_k(t)[G_k(t) - G_k^{avg}],
    \label{app: EN_ib delta U_k}
\end{equation}
where $G_k^{max}(t)$ is the maximum value allowed for the evaluation metric for the $k$-th \gls{ed}.

Substituting now \eref{app: EN_ib delta U_k} and \eqref{app: EN_ib delta U_k} inside \eref{app: EN_ib drift plus penalty} and rearranging the terms it is possible to obtain:

\begin{align}
    \Delta_p(\Theta(t)) &\leq
    \sum_{k=1}^K \Bigg{[} \nu_k^2\frac{(D_k^{max} - D_k^{avg})^2}{2} + \nu_k^2\frac{(G_k^{max}(t) - G_k^{avg})^2}{2}  \Bigg{]}  \\ &\;\;\;
    + \mathbb{E} \Bigg{[}\;\sum_{k=1}^K \Big{[} - \varepsilon_k Z_k(t)Q_k^{avg} - \nu_k S_k(t)G_k^{avg}   + \Big|\;\; \Theta(t) \Bigg{]} \\ &\;\;\; + \mathbb{E} \Bigg{[}\;\sum_{k=1}^K \Big{[} \varepsilon_k Z_k(t)Q_k^{tot}(t)  +  \nu_k S_k(t)G_k(t) \Big{]} + V\cdot P^{tot}\;\; \Big|\;\; \Theta(t) \Bigg{]}, 
\end{align}
where some constants that have been taken out of the expected value (first line), while others even if within the expected value do not depend on the optimization parameters (second line).

Pivoting therefore on the Lyapunov optimization it is possible to neglect all these terms. Moreover, it is possible to remove the expected value to obtain the following per-slot optimization:

\begin{mini}|s|[0]
    {\mathbf{\Psi}(t)}{\sum_{k=1}^K \bigg[ \frac{\epsilon_kT_k(t)N_k(t)}{R_k(t)} + \frac{\epsilon_kT_k(t)W_k(t)}{f_k(t)\rho_k } + \frac{\epsilon_kT_k(t)W_{max}^{es}}{f_k^{es}(t) \rho_k^{es}}+}{}{} \breakObjective{\qquad +  \frac{B_k N_0}{h_k(t)} {\rm exp} \left(\frac{R_k(t) ln(2)}{B_k} \right) + V \Gamma_k \eta_k (f_k(t))^3 +}{}{} \breakObjective{\;+  V \eta (f_c(t))^3 + \nu_k U_k(t)G_k(t)\bigg]}{}{}
    \addConstraint{\mathbf{\Psi}(t) \in \mathcal{T}(t),}{}
    \label{eq: EN_ib per-slot opt problem structure}
\end{mini}
where $\mathcal{T}(t)$ indicates the space of possible solutions given by the constraints on the optimization variables. 

at this point it is possible to split the problem for the resource allocation at the \gls{ed} and at the \gls{es}.

\section{Edge Device optimization}\label{app: EN_ib ed opt}
The sub-problem for the \gls{ed} as defined in \eref{eq: EN_ib per-slot opt ed} can be split in two further sub-problems for the transmission rate $R_k(t)$ and the clock frequency $f_k(t)$.

\subsection*{Transmission rate optimal solution}
The sub-problem associated to the transmission rate $R_k(t)$ can be defined as follows:
\begin{mini}|s|[0]
    {R_k(t)}{\frac{\epsilon_kT_k(t)N_k(t)}{R_k(t)} +  V \frac{B_k N_0}{h_k(t)} {\rm exp} \left(\frac{R_k(t) ln(2)}{B_k} \right) }{}{}
    \addConstraint{0 \leq R_k(t) \leq R_k^{max}(t)}{} 
\end{mini}

To simplify the notation, define:
\[
A = \epsilon_k T_k(t) N_k(t), \quad B = V \dfrac{B_k N_0}{h_k(t)}, \quad C = \dfrac{\ln(2)}{B_k}.
\]

Computing the derivative of the objective function $J(R_k(t))$ with respect to $R_k(t)$and set it to zero it is possible to obtain:
\[
\frac{dJ}{dR_k(t)} = -\dfrac{A}{[R_k(t)]^2} + B C \exp\left( C R_k(t) \right) = 0.
\]

By defining Let $x = C R_k(t)$ and $d = \dfrac{A C}{B}$ the derivative can be rearranged as:
\[
x e^{\frac{x}{2}} = \sqrt{d}.
\]

Fortunately, there is an exact solution to this problem and it is based on the \textit{Lambert W function}. By applying the definition and substituting back all the terms it is possible to obtain the final solution:
\begin{equation}
    R_k^*(t) = \frac{2 B_k}{ln(2)}\; W\! \!\left(\sqrt{\frac{\epsilon_k T_k(t)\; ln(2)\; h_k(t)N_k(t)\; }{4 B_k^2\;V \;N_0}}\right)\; \Biggr|_0^{R_k^{max}(t)}
\end{equation}

\subsection*{Clock frequency optimal solution}
The sub-problem associated to the transmission rate $R_k(t)$ can be defined as follows:
\begin{mini}|s|[0]
    {f_k(t)}{\frac{\epsilon_k T_k(t)W_k(t)}{f_k(t)\rho_k } +  V \Gamma_k \eta_k (f_k(t))^3 }{}{}
    \addConstraint{0 \leq f_k(t) \leq f_k^{max}}{} 
\end{mini}

To simplify the notation define:
\[
A = \dfrac{\epsilon_k T_k(t) W_k(t)}{\rho_k}, \quad B = V \Gamma_k \eta_k
\]

Computing the derivative of the objective function $J(f_k(t))$ with respect to $f_k(t)$ and set it to zero it is possible to obtain:
\[
\frac{dJ}{df_k(t)} = -\dfrac{A}{[f_k(t)]^2} + 3B [f_k(t)]^2 = 0
\]

After multiply both sides by $[f_k(t)]^2$, rearranging the terms and substituting back  $A$ and $B$ the final solution is:
\[
f_k(t) = \left( \dfrac{A}{3B} \right)^{1/4} \; \Biggr|_0^{f_k^{max}} \implies f_k^* (t) = \sqrt[4]{\frac{\epsilon_k T_k(t) W_k(t)}{3 V \Gamma_k \eta_k \rho_k} }\; \Biggr|_0^{f_k^{max}},
\]


\section{Edge Server optimization}\label{app: EN_ib es opt}


\begin{mini}|s|[0]
    {\{f_f^{es}(t)\}_k, f_c(t)}{\sum_{k=1}^K \frac{\epsilon_k T_k(t)W_{max}^{es}}{f_k^{es}(t)\rho_k^{es}} + V \eta (f_c(t))^3 }{}{}
    \addConstraint{0 \leq f_c(t) \leq f_c^{max} }{}
    \addConstraint{f_k^{es}(t) \geq 0 \quad \forall k}, \qquad {\sum_{k=1}^K f_k^{es}(t) \leq f_c(t)}{}
\end{mini}

Define:
\[
A_k = \dfrac{ \epsilon_k T_k(t) W_{\text{max}}^{es} }{ \rho_k^{es} }, \quad B = V \eta, \quad S = \sum_{k=1}^K \sqrt{ A_k }
\]


The objective function becomes:
\[
J(\{f_k^{es}(t)\}_k,\ f_c(t)) = \sum_{k=1}^K \dfrac{A_k}{f_k^{es}(t)} + B [f_c(t)]^3
\]

As a first step it is possible to define the associated Lagrangian $L$ of the sub-problem with respect to  $f_k^{es}(t)$ given $f_c(t)$ as:
\[
L = \sum_{k=1}^K \dfrac{A_k}{f_k^{es}(t)} + \lambda \left( \sum_{k=1}^K f_k^{es}(t) - f_c(t) \right)
\]

By deriving it and isolating with respect to $f_k^{es}(t)$ it is possible to obtain:

Solve for $f_k^{es}(t)$:
\[
    \frac{\partial L}{\partial f_k^{es}(t)} = -\dfrac{A_k}{[f_k^{es}(t)]^2} + \lambda = 0  \implies [f_k^{es}(t)]^2 = \dfrac{A_k}{\lambda} \implies f_k^{es}(t) = \sqrt{ \dfrac{A_k}{\lambda} }
\]

Apply the coupling constraint on $f_c(t)$ and by solving for $\lambda$ it is possible to identify:
\[
\sum_{k=1}^K f_k^{es}(t) = \dfrac{1}{\sqrt{\lambda}} \sum_{k=1}^K \sqrt{ A_k } = f_c(t) \implies \sqrt{\lambda} = \dfrac{ S }{ f_c(t) } \implies \lambda = \left( \dfrac{ S }{ f_c(t) } \right)^2
\]

Therefore:
\[
f_k^{es}(t) = \dfrac{ \sqrt{ A_k } }{ S } f_c(t)
\]

This term can now be substituted back into the objective function that is then derived with respect to $f_c(t)$ and set to zero as:

\[
J(f_c(t)) = \dfrac{ S^2 }{ f_c(t) } + B [f_c(t)]^3 \implies \frac{dJ}{df_c(t)} = - \dfrac{ S^2 }{ [f_c(t)]^2 } + 3 B [f_c(t)]^2 = 0
\]

By solving for $f_c(t)$, substituting back the expressions of $A$, $B$ and $S$ and applying the constraints it is possible to obtain the final solution:
\[
    f_c^*(t) = \left[ \left( \dfrac{ S^2 }{ 3 B } \right)^{1/4} \right]_0^{f_c^{\text{max}}}  = \frac{\sqrt{\sum_{k=1}^K \sqrt{\frac{\epsilon_k T_k(t)W_{max}^{es}}{\rho_k^{es}}}}}{\sqrt[4]{3V\eta}} \; \Biggr|_0^{f_{c}^{max}}
\]


Therefore, for every $k$:
\[
f_k^{es}(t) = \dfrac{ \sqrt{ A_k } }{ S } f_c^*(t) = f_k^{es*}(t) = \frac{\sqrt{\frac{\epsilon_k T_k(t)W_{max}^{es}}{\rho_k^{es}}}}{\sqrt{\sum_{k=1}^K \sqrt{\frac{\epsilon_k T_k(t)W_{max}^{es}}{\rho_k^{es}}}}\sqrt[4]{3V\eta} }
\]

\section{Example: H1 Failure on 4-Operand Addition}
\label{appendix:C}
Below is an example in which the heuristic \textbf{H1} fails in 4-operand addition, visualized in Figure \ref{fig:carry_4_op_fail}: 

\noindent\textbf{186 + 261 + 198 + 256.}
\[
\begin{split}
    t_{1} =8 + 6 + 9 + 5 = 28\\
    c_{2}^{h} \in \{ \left\lfloor \frac{c_{min} + 28}{10} \right\rfloor,\\
    \left\lfloor \frac{c_{max} + 28}{10} \right\rfloor \}
\end{split}
\]

with \(c_{max} = 3\)
\[c_{2}^{h} \in \{ \left\lfloor \frac{28}{10} \right\rfloor, \left\lfloor \frac{31}{10} \right\rfloor \} = \{2, 3\}\]
therefore $c_{2}^{h}$ is chosen uniformly at random between $2$ and $3$.
The heuristic thus fails in solving \textbf{186 + 261 + 198 + 256} with a chance of 50\%. 

\begin{figure}[ht]
    \centering
    \includegraphics[width=0.5\textwidth]{Images/figure9.png} 
    \caption{4-operand addition in which \textbf{H1} fails.} 
    \label{fig:carry_4_op_fail}
\end{figure}
\section{Accuracy Prediction of Heuristic}
\label{appendix:D}

\renewcommand{\arraystretch}{1.2}
\setlength{\tabcolsep}{5pt}
\begin{table*}[ht]
\centering
\begin{tabular}{|c|c|c|c|c|}
\hline
Nr. Operands $k$ & \textbf{\(c_{max}(k)\)} & Values of \(t_i\) in which H1 fails & Expected acc. on \(s_d\)\\ \hline
2 & 1 & 1 fail:= 9 & $\frac{18\times1.0 + 1 \times 0.5}{19} = 0.974$ \\ \hline
3 & 2 & 4 fails:= 8, 9, 18, 19 & $\frac{24\times1.0 + 4 \times 0.5}{28} = 0.928$\\ \hline
4 & 3 & 9 fails:= 7, 8, 9, 17, 18, 19, 27, 28, 29 & $\frac{28\times1.0 + 9 \times 0.5}{37} = 0.878$ \\ \hline
5 & 4 & 16 fails:= 6, 7, 8, 9, 16, ..., 39 & $\frac{ 30 \times1.0 + 16 \times 0.5}{46} = 0.826$\\ \hline
6 & 5 & 25 fails:= 5, 6, 7, 8, 9, 15, ..., 49 & $\frac{ 30 \times1.0 + 25 \times 0.5}{55} = 0.773$\\ \hline
7 & 6 & 36 fails:= 4, 5, 6, ..., 59 & $\frac{ 28 \times1.0 + 36 \times 0.5}{64} = 0.719$\\ \hline
8 & 7 & 49 fails:= 3, 4, 5, ..., 69 & $\frac{ 24 \times1.0 + 49 \times 0.5}{73} = 0.664$\\ \hline
9 & 8 & 64 fails:= 2, 3, 4, ..., 79 & $\frac{ 18 \times1.0 + 64 \times 0.5}{82} = 0.610$\\ \hline
10 & 9 & 81 fails:= 1, 2, 3, ..., 89 & $\frac{ 10 \times1.0 + 81 \times 0.5}{91} = 0.555$\\ \hline
11 & 9 & 89 fails:= 1, 2, 3, ..., 99 & $\frac{ 10 \times1.0 + 90 \times 0.5}{100} = 0.540$\\ \hline
%12 & 108 & 10 & 109 fails:= 0, 1, 2, ..., 108 & /109 = 0.1 \\ \hline
\end{tabular}
\caption{Predicted accuracy on the first result digit $s_d$ in the addition of multiple numbers according to \textbf{H1}.}
\label{tab:heuristic}
\end{table*}

Table \ref{tab:heuristic} contains, for addition tasks with different numbers of operands $k$, the maximum value of the carry \(c_{max}(k)\). Based on \(c_max\) it list those values of \(t_i\) in which \textbf{H1} is insufficient to accurately predict \(s_2\). Based on the proportion of values of \(t_i\) for which \textbf{H1} is sufficient to the total number of possible values, it lists the predicted accuracy for \(s_2\).

\newpage


\end{document}
