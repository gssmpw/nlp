\section{Discussion}
\label{sec:discussion}

\subsection{Impact of Attack Strength of \attackname on \defensename}
In this section, we investigate how the strength of adversarial training affects the performance of both our \defensename and \attackname.
We define the attack strength as the max number of query the adversary is allowed to make in the training process.
Generally, if the adversary queries the target model more frequently, its final output tends to be more effective.
\begin{figure}[t]
    \centering
    \resizebox{0.48\textwidth}{!}{    \includegraphics{genshin/two_dataset_comparison.png} 
    }
	\caption{
    \textbf{Impact of attack strength in \modelname.}
    We normalize the attack strength for better visualization of the results.
	}
    \label{fig:compare} 
    \vspace{-0.5cm}
\end{figure}
We increase attack strength in \modelname and evaluate the performance of \defensename under 7 attacks and \attackname on the fine-tuned \texttt{XLM-RoBERTa-Base} detector and present the results in Figure~\ref{fig:compare}. 
The ASR significantly increases as the attack strength grows, which proves the rationale of the attack strength measure.
We observe that the \defensename becomes more robust with the increasing of attack strength, indicated by the decreasing ASR under all attacks.
Notably, the ASR under \textit{Paraphrasing} Attack decreases from \textbf{10.80} to \textbf{3.45} as the attack strength increases from \textbf{0.0} to \textbf{1.0}, which is \textbf{3.13} times lower.

The experimental results indicate that attack strength is a key factor influencing the robustness of the MGT detector. 
However, an increased number of queries comes with more cost on time and budget.
Thus, there exists a trade-off between the effectiveness and efficiency in adversarial training.