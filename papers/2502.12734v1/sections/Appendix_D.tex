\section{Adversarial Training Process}
\label{appd: pseudo}
Algorithm \ref{alg:algorithm1_short} shows the detailed optimization process of \attackname and \defensename. The two components are updated alternatively in the same training step.

\begin{algorithm}[h]
\small
\caption{Adversarial Training Procedure}
\label{alg:algorithm1_short}
\begin{algorithmic}[1]
        \State \textbf{Input:} Training set $D_{train}$, surrogate detector $\mathcal{M}_{sur}$.

    \begin{center}
            \textit{**** training phase begins ****}
        \end{center}
    \State \textbf{Initialize:} target detector $\mathcal{M}_{tar}$, importance scoring network $\mathcal{F}_{\theta}$ and $epoch \leftarrow 0$.
    \While{$epoch< epoch_{max}$}
       \For{each batch samples $\{X_i, c_i\}_{i=0}^{N}$ in $D_{train}$}
       \State $D_{adv}\leftarrow \{\}$
       \For{each MGT in $\{X_i, c_i\}_{i=0}^{N}$}
       \State Obtain the last layer hidden state by Eq.\eqref{eq:1}.
       \State Obtain the importance score by Eq.\eqref{eq:2}.
       \State Construct the Important-token Set $\mathbf{I}$ by Eq.\eqref{eq:3}.
       \State Execute the Greedy Search Algorithm\ref{alg:algorithm2_short}.
       \State Execute the Greedy Pruning Algorithm\ref{alg:algorithm3_short}.
       \State{Get the adversarial example:}
       \begin{center}
            $D_{adv} \gets D_{adv} \cup \{X_i, c_i\}_{i=0}^{N}$.
        \end{center}
       \EndFor
       \State $D_{adv} \gets D_{adv} \cup \{\tilde{X}_i, c_i\}$.
       \State Classify $D_{adv}$ via $\mathcal{M}_{tar}$ and obtain the output.
       \State Calculate loss $\mathcal{L}_{\text{A}}$ by Eq.\eqref{eq:14}.
       \State Calculate loss $\mathcal{L}_{\text{D}}$ by Eq.\eqref{eq:9}.
       \State Update $\mathcal{M}_{sar}$ and $\mathcal{F}_{\theta}$ via SGD \shortcite{robbins1951stochastic}.
       \EndFor 
    \EndWhile
    \State \textbf{Output:} Trained detector $\mathcal{M}_{tar}$ and $\mathcal{F}_{\theta}$.
\end{algorithmic}
\end{algorithm}


\section{Case Study}

Table \ref{tab:case_study_results} presents a case study of our adversary \attackname, in which our approach \attackname outperforms other SOTA methods regarding semantic preservation and reduction in perturbation rate.



\begin{table*}[!htbp]
\centering
\renewcommand\arraystretch{1.4}
\resizebox{1\textwidth}{!}{

\begin{tabular}{lm{0.75\textwidth}>{\centering\arraybackslash}m{0.2\textwidth}}
\hline
\textbf{Method} & \textbf{Text} & \textbf{Result} \\ 
\hline

\textbf{Original MGT} & Suggested by the Scottish Parliamentary Constituencies Commission in 2003, and adopted at Holyrood on 1 May 2004. 
Area of Scotland's 32nd largest council area - covering parts of East Renfrewshire Council Area & Machine-written
 \includegraphics[width=0.5cm]{genshin/now.png}\\

\hline

\textbf{PWWS} & Suggested by the Scottish Parliamentary Constituencies Commission in 2003, and \textcolor{red}{adoptions} at Holyrood on 1 \textcolor{red}{Probability} 2004. 
Area of Scotland's 32nd \textcolor{red}{highest} council area - covering \textcolor{red}{item} of East Renfrewshire Council Area & Human-written

 (Succeeded) \makebox[-1pt][l]{\includegraphics[width=0.5cm]{genshin/smiley.png}} \\

\hline

\textbf{TextFooler} & Suggested by the Scottish Parliamentary Constituencies Commission in 2003, and adopted at Holyrood on 1 May 2004. 
Area of Scotland's 32nd largest council area - covering parts of East Renfrewshire Council Area & Machine-written

 (Failed) \makebox[-1pt][l]{\includegraphics[width=0.5cm]{genshin/frownie.png}} \\

\hline

\textbf{BERTAttack} & Suggested by the Scottish \textcolor{red}{rural} Constituencies Commission in 2003, and \textcolor{red}{abolished} at Holyrood on 1 May 2004. 
Area of \textcolor{red}{ward} 32nd \textcolor{red}{most} council \textcolor{red}{constituency} - \textcolor{red}{almost} \textcolor{red}{all} of East Renfrewshire Council Area & Human-written

 (Succeeded) \makebox[-1pt][l]{\includegraphics[width=0.5cm]{genshin/smiley.png}} \\

\hline

\textbf{A2T} & Suggested by the Scottish Parliamentary Constituencies Commission in 2003, and adopted at Holyrood on 1 May 2004. 
Area of Scotland's 32nd largest council area - covering parts of East Renfrewshire Council Area &  Machine-written

 (Failed) \makebox[-1pt][l]{\includegraphics[width=0.5cm]{genshin/frownie.png}}\\

\hline

\textbf{FastTextDodger} & Suggested by the Scottish Parliamentary Constituencies Commission in 2003\textcolor{red}{:} and adopted at Holyrood on 1 May 2004. 
Area \textcolor{red}{of' 's} 32nd largest council area - covering parts of East \textcolor{red}{council Council} & Human-written

 (Succeeded) \makebox[-1pt][l]{\includegraphics[width=0.5cm]{genshin/smiley.png}} \\

\hline

\textbf{ABP} & Suggested \textcolor{red}{aside} the Scottish Parliamentary Constituencies Commission \textcolor{red}{indium} 2003, and adopted \textcolor{red}{atomic number 85} Holyrood on \textcolor{red}{ace Crataegus oxycantha} 2004. 
Area of Scotland's 32nd largest council area - covering parts of East Renfrewshire Council Area & Human-written

 (Succeeded) \makebox[-1pt][l]{\includegraphics[width=0.5cm]{genshin/smiley.png}}\\

\hline

\textbf{HQA} & Suggested by the Scottish Parliamentary Constituencies Commission in 2003, and adopted at Holyrood on 1 May 2004. 
Area of Scotland's \textcolor{red}{2004. sphere council} area \textcolor{red}{thirty-second prominent council} of East Renfrewshire Council Area & Human-written

 (Succeeded) \makebox[-1pt][l]{\includegraphics[width=0.5cm]{genshin/smiley.png}} \\

\hline

\textbf{T-PGD} & \textcolor{red}{Introduced. by.} Scottish Parliamentary Resituiances Commission in 2002, and adopted at Holyrood on 1 May 2004. 
\textcolor{red}{All of Scotland's 32 The largest councils} area - covering parts of East Renfrewshire Council Area & Human-written

 (Succeeded) \makebox[-1pt][l]{\includegraphics[width=0.5cm]{genshin/smiley.png}} \\

\hline

\textbf{\attackname} & Suggested by the \textcolor{red}{Grave} Parliamentary Constituencies Commission in 2003, and adopted at Holyrood on 1 May 2004. 
Area Scotland 32nd \textcolor{red}{biggest} council area - covering parts of East Renfrewshire Council Area & Human-written

 (Succeeded) \makebox[-1pt][l]{\includegraphics[width=0.5cm]{genshin/smiley.png}}\\

\hline

\end{tabular}
}
\caption{Case study of semantic preservation of the adversarial texts generated by various attack methods. Words modified during the attacks are highlighted in \textcolor{red}{red}. }
\label{tab:case_study_results}
\end{table*}

