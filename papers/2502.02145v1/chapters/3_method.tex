%!TeX spellcheck = en_US
% !TeX root = ../main.tex

\section{Methodology}
\label{sec:method}
In this paper, we analyze scenarios from real-world datasets re-simulated  within the CommonRoad simulation framework, which serves as a benchmark for testing motion planning algorithms, and our proposed analysis framework is illustrated in Figure~\ref{fig:framework}. Initially, we focus on the CommonRoad scenario format, where a CommonRoad XML file encapsulates a Driving simulation environment and a PlanningProblem.

Subsequently, we extend the 6-layer model introduced in \cite{scholtes20216}, which describes the six essential elements for the driving scenario by integrating the driving simulation environment and PlanningProblem with the ego layer from motion planner to comprehensively represent driving scenario within the context of motion planning. In our paper, we use one of the high performance modular motion Planner - Frenetix motion planner \cite{Frenetix} from our lab. After that we parse relevant information from road, traffic sign, obstacle, and ego layers to construct effective prompting context for downstream analysis.

Using a combination of role prompting, chain-of-thought reasoning and template-based techniques, we leverage safety-critical metrics to guide LLMs in analyzing the safety-criticality of the scenarios. These metrics ensure a robust analysis, enabling identifying and evaluating potential hazards and safety concerns within the scenarios.

\subsection{CommonRoad Scenario format}
The CommonRoad scenario format is a standardized representation that facilitates the simulation of various driving scenarios, providing a comprehensive framework for testing and benchmarking motion planning algorithms. It encapsulates the following driving element and is shown in Figure~\ref{fig:framework}.
\begin{itemize}
    \item \textit{Meta information} includes the ID of benchmark,  location with the map's name, and GPS information, also tags for scenarios with categories like time of day, highway, multi-lane, etc.
    \item \textit{Road network} consists of lanelet with boundary information, the position of each traffic sign, and the graph of the road network by using elements like predecessor, successor.
     \item \textit{Traffic sign} consists of the current traffic signs and traffic lights from Germany, like speed limit and stop. 
    \item \textit{Obstacles} are classified into staticObstacle and dynamicObstacle to represent the participants like, parked vehicles, cars, buses, and motorcycles within the scenario, including the shape size, initial state, and trajectory including their positions, orientation, velocities, and accelerations for each obstacle.
    \item \textit{PlanningProblem} defines the motion planning problem by specifying the ego vehicle's initial state, such as its position, velocity, acceleration, orientation, and one or more goal state(s).
\end{itemize}

\subsection{Layer Model}
The CommonRoad XML format is complex and challenging for LLMs to interpret directly. Therefore, we need to decode this format by importing the layer model. The 6-layer model is a conceptual framework that describes the six essential elements of a driving scenario: the road, the traffic sign, temporary change, the traffic participants, the traffic environment, and the digital map. We do not need to consider the temporary change and environmental conditions for CommonRoad simulation. Therefore, in the layer model we used to analysis CommonRoad XML file, we got information about the element lanelet representing the road layer, the traffic sign layer model from traffic sign and traffic light elements, the obstacle elements represent the traffic participants layer. Additionally, to build the new ego layer, we consider the Planning Problem, which represents the ego vehicle's initial state and goal state, to combine the simulated trajectory from the Frenetix motion planner.
Moreover, for each Scenario $\mathcal{S}$ could be describe by:
\begin{equation}
    S = \{\mathcal{L}, \mathcal{T}, \mathcal{O}, \mathcal{E}\}
\end{equation}
where $\mathcal{L}$ represents the lanelet, $\mathcal{T}$ denotes the traffic sign, $\mathcal{O}$ stands for the obstacle, and $\mathcal{E}$ represents the ego vehicle.

Furthermore, this decoding process involves extracting relevant information from the XML file and from ego trajectory CSV file by converting these into JSON files and translating them into a structured format for each layer.
After that we need to parse  layer model information into a context for the prompt message of LLMs.

\subsection{Prompt techniques}
Prompt techniques are essential for guiding LLMs in analyzing complex problems. General prompt techniques include zero-shot\cite{wei2021finetuned} and few-shot prompting \cite{touvron2023llama}. Zero-shot prompting involves providing the model with a task description without any examples, while few-shot prompting includes a few examples to help the model understand the task better. However, these general techniques are not able to analyze complicated scenarios due to their complexity and the need for detailed reasoning. 

To address these challenges, we employ more advanced techniques, such as role prompting, chain-of-thought reasoning, and template-based prompts. We use role prompting techniques by introducing detailed \textit{system messages} to provide more information about the role of our LLMs agent, the definition of the risk score, and the used metrics. This helps the LLM adopt the perspective of a domain expert and generate responses aligned with that expertise. In the user message, combining the chain-of-thought reasoning, we provide the context of the scenario and the guidance for the LLMs to analyze the safety criticality of the scenario, ensuring a more robust and transparent evaluation.

Additionally, we use template-based techniques with specific context parts and outputs format to ensure
the robustness, accuracy of each prompting, and consistency
of the response of LLMs. In the Figure~\ref{fig:prompt}, we have listed three prompt templates: \textit{Cartesian coordinates template, Frenet coordinates template, and safety-critical metrics template}. 
Specifically, each template uses a different context $\mathcal{C}$. To combine the relevant information from the context $\mathcal{C}$ by parsing the lanelet information, the status of obstacles, and the ego vehicle for each timestep which allows the LLMs to understand the motion development of each vehicle and process the information, leading to more accurate and reliable analysis.  
To convert Scenario $\mathcal{S}$ into context $\mathcal{C}$ by using the parsing function $\mathcal{F}$
\begin{equation}
    \mathcal{C} = \mathcal{F}(\mathcal{S})
\end{equation}

\begin{itemize}
    \item \textit{Cartesian coordinates context} provides a global view with x and y Cartesian coordinates of the scenario, including the positions, speeds, accelerations, and the location in the lanelet of all vehicles, which are easy to get. 
    The format of the parsing function $ \mathcal{F}_{\text{cart}}$ is shown below,
    {\small
    \[
    \mathcal{F}_{\text{cart}} =
    \begin{cases}
    \text{ID: } \text{Vehicle\_id (with ego Vehicle)} \\
    \text{Position: } (x, y)  \text{ m} \\
    \text{Orientation: } \theta_{\text{cart}} \text{ rad } \\
    \text{Velocity: } v \text{ m/s} \\
    \text{Acceleration: } a \text{ m/s}^2 \\
    \text{Lanelet: } l
    \end{cases}
    \]}
\begin{figure*}[ht]
    \centering
    \includegraphics[width=1\textwidth]{figures/templates.jpg} 
    \caption{Three proposed prompts templates submitted to LLM.}
    \label{fig:prompt} % Optional: for referencing the figure
\end{figure*}
    \item \textit{Frenet\cite{werling2012optimal} coordinates context} provides a detailed view of the scenario based on the coordinates of the ego car, including the relative directions, positions, speeds, and accelerations of the ego vehicle and the surrounding vehicles. We convert the parsing function \( \mathcal{F}_{\text{cart}} \) into \( \mathcal{F}_{\text{fren}} \):

{\small\[
    \mathcal{F}_{\text{fren}} =
    \begin{cases}
    \text{ID: } \text{Obstacle\_id}, \\
    \text{Relative direction: } \theta_{\text{fren}},\\
    \text{Position: } (s, d) \text{ m}, \\
    \text{Velocity: } (v_s, v_d) \text{ m/s}, \\
    \text{Acceleration: } (a_s, a_d) \text{ m/s}^2, \\
    \text{Motion: } (m_s, m_d).    
    \end{cases}
\]}
where \( \theta_{\text{fren}} \) is defined as:
{\scriptsize
\begin{itemize}
    \item \textbf{Front}:  If \( s > L_{\text{veh}} \land |d| \leq W_{\text{veh}} \).
    \item \textbf{Front-Left}: If \( s > L_{\text{veh}} \land d > W_{\text{veh}} \).
    \item \textbf{Front-Right}: If \( s > L_{\text{veh}} \land d < -W_{\text{veh}} \).
    \item \textbf{Behind}: If \( s < -L_{\text{veh}} \land |d| \leq W_{\text{veh}} \).
    \item \textbf{Rear-Left}: If \( s < -L_{\text{veh}} \land d > W_{\text{veh}} \).
    \item \textbf{Rear-Right}: If \( s < -L_{\text{veh}} \land d < -W_{\text{veh}} \).
    \item \textbf{Left}: If \( d > W_{\text{veh}} \land |s| \leq L_{\text{veh}} \).
    \item \textbf{Right}: If \( d < -W_{\text{veh}} \land |s| \leq L_{\text{veh}} \).
\end{itemize}
}
and \( L_{\text{veh}}, W_{\text{veh}} \) are the length and width of the ego vehicle. Based on the relative direction, motion \( m \) is determined as:
{\scriptsize
\[
    m_s =
    \begin{cases}
    M_s[0], & \text{if } v_{\text{rel}, s} > 0 \land \theta_{\text{fren}} \in \{\text{Front}, \text{Front-Left}, \text{Front-Right}\}, \\
    M_s[1], & \text{if } v_{\text{rel}, s} < 0 \land \theta_{\text{fren}} \in \{\text{Front}, \text{Front-Left}, \text{Front-Right}\}, \\
    M_s[2], & \text{if } v_{\text{rel}, s} > 0 \land \theta_{\text{fren}} \in \{\text{Behind}, \text{Rear-Left}, \text{Rear-Right}\}, \\
    M_s[3], & \text{if } v_{\text{rel}, s} < 0 \land \theta_{\text{fren}} \in \{\text{Behind}, \text{Rear-Left}, \text{Rear-Right}\}, \\
    M_s[4], & \text{if } v_{\text{rel}, s} = 0, \\
    M_s[5], & \text{otherwise}.
    \end{cases}
\]
}
{\scriptsize
\[
    m_d =
    \begin{cases}
    M_d[0], & \text{if } v_{\text{rel}, d} > 0 \land \theta_{\text{fren}} \in \{\text{Left}, \text{Front-Left}, \text{Rear-Left}\}, \\
    M_d[1], & \text{if } v_{\text{rel}, d} < 0 \land \theta_{\text{fren}} \in \{\text{Left}, \text{Front-Left}, \text{Rear-Left}\}, \\
    M_d[2], & \text{if } v_{\text{rel}, d} > 0 \land \theta_{\text{fren}} \in \{\text{Right}, \text{Front-Right}, \text{Rear-Right}\}, \\
    M_d[3], & \text{if } v_{\text{rel}, d} < 0 \land \theta_{\text{fren}} \in \{\text{Right}, \text{Front-Right}, \text{Rear-Right}\}, \\
    M_d[4], & \text{if } v_{\text{rel}, d} = 0, \\
    M_d[5], & \text{otherwise}.
    \end{cases}
\]
}
{\scriptsize
\[
    M_s =
    \begin{bmatrix}
    \makecell[l]{\text{"Obstacle is moving away longitudinally in front"}} \\
    \makecell[l]{\text{"Obstacle is driving toward the ego car longitudinally from front"}} \\
    \makecell[l]{\text{"Obstacle is moving away longitudinally in behind"}} \\
    \makecell[l]{\text{"Obstacle is driving toward the ego car longitudinally from behind"}} \\
    \makecell[l]{\text{"No longitudinal relative motion"}} \\
    \makecell[l]{\text{"Exact longitudinal alignment"}}.
    \end{bmatrix}
\]
}
{\scriptsize
\[
    M_d =
    \begin{bmatrix}
     \makecell[l]{\text{"Obstacle is moving away laterally to the left"}}, \\
     \makecell[l]{\text{"Obstacle is driving toward the ego car laterally from the left"}}, \\
     \makecell[l]{\text{"Obstacle is moving away laterally to the right"}}, \\
     \makecell[l]{\text{"Obstacle is driving toward the ego car laterally from the right"}}, \\
     \makecell[l]{\text{"No lateral relative motion"}}, \\
     \makecell[l]{\text{"Exact lateral alignment"}}.
    \end{bmatrix}
\]
}

    \item \textit{Safety critical metrics context} provides the key values, such as the relative distance and the time to collision between vehicles and ego car. Furthermore, the system message provides more information about calculating the risk score. The format of the parsing function $\mathcal{F}_{\text{metrics}}$ for the context is shown as:
{\small
    \[
    \mathcal{F}_{\text{metrics}} =
    \begin{cases}
    \text{ID: } \text{Obstacle\_id}, \\
    \text{Relative Direction: } \theta_{\text{fren}}, \\
    \text{Distance to Collision: } (dtc_{\text{long}},  dtc_{\text{lat}}) \text{ m}, \\
    \text{Time to Collision: } (ttc_{\text{long}}, ttc_{\text{lat}}) \text{ rad},\\
    \text{Motion: } (m_s, m_d).
    \end{cases}
\]
}
\end{itemize}
Meanwhile, it is important to note that use the specific constrained output format can ensure the consistency and correctness of the response from LLMs. For instance, one of the output format for safty-critical metrics template is in the following:

{\small
\begin{verbatim}
### Safety analysis for timestep <timesteps>: 
Obstacle Analysis:
- Obstacle ID: <numeric ID>
- Relative Direction: <Front......>
- Distance Risk Reason: <description>
- Longitudinal Distance Safety Score:<LongDSC>  
- Lateral Distance Safety Score: <LatDSC>
- Overall Distance Safety Score: <DSC>
- Time Risk Reason: <description>
- Longitudinal Time Safety Score: <LongTSC>
- Lateral Time Safety Score: <LatTSC>
- Overall Time Safety Score: <TSC>
- Overall Risk Score: <Risk Score>

### Summary in JSON Format:
Go through all of the Obstacle Analysis again
and write the related obstacle in the 
following JSON format. 
If they don't exist, set them as null:  
{    "CollisionObstacle": {
        "ObstacleID": "<Obstacle ID>",
        "OverallRiskScore": "<RS = 0>"},
    "ExtremeRiskObstacle": {
        "ObstacleID": "<Obstacle ID>",
        "OverallRiskScore": "<RS = 1>"},
    "HighRiskObstacle": {
        "ObstacleID": "<Obstacle ID>",
        "OverallRiskScore": "<RS = 2>"},
    "MediumRiskObstacle": {
        "ObstacleID": "<Obstacle ID>",
        "OverallRiskScore": "<RS = 3>"}}
\end{verbatim}
}
With these three proposed prompts, we can empirically evaluate the performance of LLMs for our 100 collision scenario. To identify the safety criticality of these scenarios, we need to consider the safety-critical metrics in the following subsection.

\subsection{Safety-Critical Metrics}
In our LLMs-based analysis, we consider two kinds of classical safety-critical metrics. One is time to collision (TTC)\cite{vogel2003comparison}, and the other is a distance-based metric minimal distance to collision (MDC), which is a simplified version of Proportion of Stopping Distance (PSD) \cite{astarita2012new} independent of braking dynamics since it is unknown in the CommonRoad simulation. These definitions are in the following:
\[
\text{TTC}_i(t) = \frac{\mathbf{X}_{i-1}(t) - \mathbf{X}_i(t) - \mathbf{L}_i}{\mathbf{V}_i(t) - \mathbf{V}_{i-1}(t)}
\]
\[
\text{MDC}_i(t) = |\mathbf{X}_{i-1}(t) - \mathbf{X}_i(t) - \mathbf{L}_i|,
\]

\begin{itemize}
    \item \(\mathbf{V}\): Vehicle velocity vector
    \item \(\mathbf{X}\): Vehicle position vector
    \item \(\mathbf{L}\): Subject vehicle's length 
\end{itemize}
Combining TTC and MDC provides a multi-dimensional view of safety-criticality, addressing both the temporal and spatial factors of potential collisions. This synergy enhances the ability to detect, assess the collision situation.

It is important to note that we did not give the detailed equations of TTC and MDC metrics in the \textbf{Cartesian coordinate prompt} and \textbf{Frenet coordinates prompt}  as shown in Figure~\ref{fig:prompt}. We also do not calculate the related metrics since we want to test LLMs' reasoning and understanding ability. Compared to these two prompts, in safety-critical metrics prompts, we provide the detailed related values of TTC and MDC in the context by using the parsing function $\mathcal{F}_{\text{metrics}}$ and give the logical threshold in the system message. However, LLMs still need to understand the logical information to determine the final risk score based on these frames.




 