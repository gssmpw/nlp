\section{Related Work}
In this section, we will introduce the primary challenges encountered by CSP methods (Section \ref{profor}), commonly used crystal structure characterization techniques (Section \ref{cryrep}), encoding network models (Section \ref{equnet}) and crystal generation models (Section \ref{crygenmodel}).

\subsection{Problem Formulation}
\label{profor}
A crystal structure consists of atoms interacting with each other, where each atom is characterized by its element type and coordinates. The goal of learning its representation is to train an encoder that accurately maps the crystal structure into a representation within a sampling space. To preserve the properties of the crystal structure after transformations, the structure must maintain translation, rotation, inversion, and permutation invariance\cite{keating1966effect, lin2022general}. Among these, translation, rotation, and inversion operations form E(3) symmetry, while translation and rotation operations form SE(3) symmetry. However, current models often struggle to effectively capture key features of crystal structures like symmetry and periodicity, particularly global symmetry, which can result in generated crystal structures failing to retain correct physical properties. Effectively incorporating these symmetries into material structure representation and neural networks has emerged as a significant challenge in crystal structure prediction. This issue also represents a classic problem in structural encoding.

\subsection{Crystal Representation}
\label{cryrep}

The crystal structure representation methods form the foundation for studying the relationship between crystal properties and structures. Previous representation techniques, such as those based on simple matrices\cite{oubari:hal-04464893, dan2020generative, kim2020generative, long2021constrained}, are inadequate for representing complex systems. The advent of graph representation methods has significantly improved the ability of models to learn crystal properties\cite{xie2018crystal, schutt2018schnet, chen2019graph, chen2022universal, yuan2024tripartite}. Notably, the incorporation of information regarding three-body interactions has further improved the predictive power of these models.

\subsection{Equivariant Network}
\label{equnet}
To facilitate the representation of crystal structure symmetries in generative network models, e3nn \cite{mario_geiger_2022_6459381}, based on previous work\cite{Thomas2018TensorFN, weiler20183d}, is specifically designed to handle the symmetry of E(3). E3nn simplifies the process of constructing and training neural networks with these characteristics.

Equivariant networks\cite{kondor2018clebsch,unke2021se, tholke2022equivariant, brandstetter2022geometric} utilize geometric functions constructed from spherical harmonics and irreducible features to implement 3D rotation and translation equivariance, as proposed in Tensor Field Networks (TFN)\cite{Thomas2018TensorFN}. The SE(3) Transformer\cite{fuchs2020se} employs equivariant dot product (DP) attention\cite{vaswani2017attention} with linear messages. Equiformer \cite{liao2023equiformer} integrates MLP attention with non-linear messages and various types of support vectors, enhancing the model's expressiveness.

\subsection{Crystal Generative Model}
\label{crygenmodel}
The generative model paradigm has been extensively utilized to generate material structures, including VAEs\cite{oubari:hal-04464893, noh2019inverse, hoffmann2019data, Ren2020InverseDO, court20203, xie2021crystal}, GANs\cite{nouira2018crystalgan, dan2020generative, kim2020generative, long2021constrained}, and diffusion models\cite{jiao2023crystal, yang2024scalable, zeni2025generative}. Most of these studies focus on binary compounds\cite{noh2019inverse, long2021constrained}, ternary compounds\cite{nouira2018crystalgan, kim2020generative}, or simpler materials within the cubic system\cite{hoffmann2019data, court20203}. Xie et al.\cite{xie2021crystal} incorporated the diffusion concept into the decoder component of the VAE, addressing the challenge of predicting material coordinates, while also retaining the crystal reconstruction capabilities of the VAE. The diffusion model\cite{jiao2023crystal, yang2024scalable, zeni2025generative} facilitates applications in more complex systems, such as the Materials Project dataset\cite{jain2013commentary}, and is capable of performing conditional generation tasks.