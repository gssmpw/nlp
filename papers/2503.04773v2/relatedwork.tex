\section{Related Work}
%\subsection{Urban Segregation}
\noindent \textbf{Experienced Segregation.}
The study of segregation traces its roots back to the early 20th century, when sociologists examined the division of urban spaces into ``ecological niches'', each inhabited by distinct social groups~\cite{park25city}. %such as Robert Park and Ernest Burgess
%This phenomenon gained further attention after the U.S. Civil Rights Movement. 
Researchers revealed the detrimental effects of enduring segregation even in the absence of ``legalized racial segregation'', on education, income, housing, and crime~\cite{king1973racial,massey1987effect,charles2003dynamics}.
Albeit offering valuable insights, these studies are highly constrained by data availability, overlooking the reality that individuals spend significant time and engage in numerous interactions beyond the confines of their residential neighborhoods~\cite{wang2018urban}.
With the rapid proliferation of smart mobile devices, the ability to track people's intricate movements in urban spaces has emerged~\cite{xu2016understanding}.
This has paved the way for a novel research avenue aimed at quantifying and understanding the type of segregation individuals actually \textit{experience} in their daily movements~\cite{moro2021mobility,athey2021estimating,de2024people}, including how it may change during disasters~\cite{yabe2023behavioral,chen2023getting} and scale with city sizes~\cite{nilforoshan2023human}.
Nevertheless, existing studies primarily focus on static spatial distribution of demographic groups and physical movement, overlooking the complex socioeconomic factors behind the segregation phenomenon, such as cultural resonance and local involvement.
Thus, these works can only provide retrospective measurement but have limited predictive power. 
In contrast, our work establishes an LLM-based method to automatically extract nuanced features from online reviews for experienced segregation prediction.

\noindent \textbf{Mining Web Data with LLMs.}
The digital footprints and rich textual contents people generate on Web platforms have proven to be informative for a wide array of social phenomena, including health outcome~\cite{nguyen2016building}, mental well-being~\cite{mitchell2013geography}, crime rates~\cite{fatehkia2019correlated}, and neighborhood disparities~\cite{rama2020facebook,iqbal2023lady}. However, these studies often rely on pre-calculated indices or pre-defined word lists to engineer and extract features.
This is not only labor-intensive but also lacks adaptability to evolving research questions and contexts.
Given the remarkable language understanding and reasoning capabilities of LLMs, recent research has explored the potential of leveraging LLMs to mine Web data for social good, such as revealing food-related social prejudice~\cite{luo2024othering}, evaluating public accessibility~\cite{li2024toward}, and capturing urban perception~\cite{santos2024real}.
Nevertheless, these studies often lack an examined framework for insight extraction from massive social media content, which limits their effectiveness in capturing the full complexity of Web data.
In contrast, our work presents the first step toward unlocking the reasoning power of LLMs to code free-form online reviews, extracting human-comprehensible, informative insights for segregation prediction.


\begin{figure*}[t]
    \centering
    \includegraphics[width=\linewidth]{figs/reflective_coding.png}
    \caption{Demonstration of Reflective Coding.}
    %\Description{Demonstration of Reflective Coding.}
    \label{fig:coding}
\end{figure*}