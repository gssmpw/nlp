\pdfoutput=1
\documentclass[11pt]{article}
\usepackage[preprint]{acl}
\usepackage{times}
\usepackage{latexsym}
\usepackage[T1]{fontenc}
\usepackage[utf8]{inputenc}
\usepackage{microtype}
\usepackage{inconsolata}
\usepackage{graphicx}
\usepackage{subfigure}
% \usepackage[x11names]{xcolor}
\usepackage{tcolorbox}
\usepackage{colortbl}
\usepackage{hyperref}
\usepackage{float}

\usepackage{pifont}       % \ding{xx}
\usepackage{bbding}       % \Checkmark,\XSolid,... (需要和pifont宏包共同使用)
\usepackage{fontawesome}  % \faCheck,\faTimes

\usepackage{booktabs}
\usepackage{multirow}  % 支持合并行
\usepackage{multicol}  % 可选,若需要多列功能

\newcommand{\hzz}[1]{\textcolor{blue}{\bf\small [#1 --huangzhongzhen]}}
\newcommand{\mlj}[1]{\textcolor{red}{\bf\small [#1]}}
\title{MMXU: A Multi-Modal and Multi-X-ray Understanding Dataset for Disease Progression}

% Author information can be set in various styles:
% For several authors from the same institution:
% \author{Author 1 \and ... \and Author n \\
%         Address line \\ ... \\ Address line}
% if the names do not fit well on one line use
%         Author 1 \\ {\bf Author 2} \\ ... \\ {\bf Author n} \\
% For authors from different institutions:
% \author{Author 1 \\ Address line \\  ... \\ Address line
%         \And  ... \And
%         Author n \\ Address line \\ ... \\ Address line}
% To start a separate "row" of authors use \AND, as in
% \author{Author 1 \\ Address line \\  ... \\ Address line
%         \AND
%         Author 2 \\ Address line \\ ... \\ Address line \And
%         Author 3 \\ Address line \\ ... \\ Address line}

% \author{
% Linjie Mu \\
% Shanghai Jiao Tong University\\
% {linjiemu@sjtu.edu.cn} \\\And
% Zhongzhen Huang \\
% Shanghai Jiao Tong University\\  
% {huangzhongzhen@sjtu.edu.cn} \\\And
% Shengqian Qin\\
% Shanghai Jiao Tong University\\  
% {qinshengqian@sjtu.edu.cn} 
  % Second Author \\
  % Affiliation / Address line 1 \\
  % Affiliation / Address line 2 \\
  % Affiliation / Address line 3 \\
  %  {email@domain} \\\And
  % Second Author \\
  % Affiliation / Address line 1 \\
  % Affiliation / Address line 2 \\
  % Affiliation / Address line 3 \\
  %  {email@domain} \\\And
  % Second Author \\
  % Affiliation / Address line 1 \\
  % Affiliation / Address line 2 \\
  % Affiliation / Address line 3 \\
  %  {email@domain} \\\And
  % Second Author \\
  % Affiliation / Address line 1 \\
  % Affiliation / Address line 2 \\
  % Affiliation / Address line 3 \\
  %  {email@domain}
   % }

\author{
 \textbf{Linjie Mu\textsuperscript{1}},
 \textbf{Zhongzhen Huang\textsuperscript{1}},
 \textbf{Shengqian Qin\textsuperscript{1}},
 \textbf{Yakun Zhu\textsuperscript{1}},
\\
 \textbf{Shaoting Zhang\textsuperscript{1}},
 \textbf{Xiaofan Zhang\textsuperscript{1}}
\\
\\
 \textsuperscript{1}Shanghai Jiao Tong University
\\
 \small{
   \textbf{Correspondence:} 
   \href{linjiemu@sjtu.edu.cn}{linjiemu@sjtu.edu.cn}, 
   \href{xiaofan.zhang@sjtu.edu.cn}{xiaofan.zhang@sjtu.edu.cn}
 }
}

\begin{document}
\maketitle
\begin{abstract}
% Medical Visual Question Answering (MedVQA) plays an indispensable role in enhancing the ability of large vision-language models (LVLMs) to comprehend complex medical inquiries and assimilate domain-specific knowledge.
% However, the majority of contemporary datasets in chest X-ray (CXR) research primarily concentrate on the analysis of individual images, often neglecting the critical importance of clinical history.
% To address this gap, we propose MMXU, a multi-image chest X-ray dataset designed to analyze the progression of medical conditions through visual question answering. Specifically, we first selected comparative sentences from the Chest ImaGenome dataset, filtered them based on specific rules, identified target regions for each sentence, and then used GPT-4o to construct region-level QAs, followed by post-processing. 
% Furthermore, by leveraging MMXU, we established a robust benchmark to assess LVLMs' capabilities in utilizing medical records, specifically evaluating the progression of three distinct disease trajectories: ``Improved,'' ``Worsen,'' and ``No Change.''
% Experimental findings demonstrate that leveraging medical records significantly enhances the model's accuracy across diverse disease trajectories. 
% In a zero-shot setting, the inclusion of region-specific information yields the most substantial improvement. Upon fine-tuning, the model's contextual comprehension is refined, and the integration of more comprehensive global data further elevates its performance.
% Medical Visual Question Answering (MedVQA) plays a crucial role in enhancing the ability of Large Vision-Language Models (LVLMs) to comprehend complex medical inquiries and integrate domain-specific knowledge. However, most contemporary benchmarks and datasets in Chest X-Ray (CXR) research primarily focus on individual image analysis, often overlooking the critical importance of clinical history. To address this gap, we propose MMXU, a framework for multi-modal and multi-X-ray understanding aimed at analyzing the progression of medical conditions through visual question answering. MMXU includes a training dataset, MMXU-dev, and a benchmark, MMXU-test. Specifically, we begin by selecting comparative sentences from the reports in the MIMIC-CXR dataset, followed by filtering and identifying the target comparative regions. We then employ GPT-4o to generate questions and perform post-processing.  
% Human experts and several well-known open-source and closed-source models were simultaneously evaluated on the benchmark.
% Experimental results demonstrate that even the most advanced models still fall approximately 40\% short of the accuracy achieved by human experts. To address this, we propose the MedRecord-Augmented Generation method, which enhances LVLMs' generation capabilities by incorporating global or regional medical records. Experimental results show that referencing historical medical records significantly improves performance, both with and without fine-tuning.
% Medical Visual Question Answering (MedVQA) plays a pivotal role in advancing the capability of Large Vision-Language Models (LVLMs) to interpret intricate medical queries and integrate domain-specific expertise. However, existing benchmarks and datasets in Chest X-Ray (CXR) research predominantly emphasize single-image analysis, often neglecting the critical influence of clinical history in diagnostic reasoning.
% To bridge this gap, we introduce MMXU, a novel framework for multi-modal and multi-X-ray understanding, designed to assess disease progression through visual question answering. MMXU comprises a dedicated training dataset, MMXU-dev, and a robust benchmark, MMXU-test. 
% % ⬇️⬇️⬇️ 构造数据集的pipeline(可要可不要)
% Our approach begins with the selection of comparative sentences from radiology reports in the MIMIC-CXR dataset, followed by meticulous filtering and identification of target comparative regions. We then leverage GPT-4o to generate questions, incorporating a rigorous post-processing pipeline to refine the dataset.
% To evaluate the MMXU-test, we conducted extensive experiments involving both human experts and leading open-source and proprietary models. The results reveal a striking gap—state-of-the-art models still fall short by approximately 40\% in accuracy compared to human performance. To address this challenge, we propose MedRecord-Augmented Generation, a novel method that enhances the generative capabilities of LVLMs by integrating global and regional medical historical records. Our findings demonstrate that referencing medical historical records significantly boosts performance, both in zero-shot settings and with fine-tuning on MMXU-dev.

Large vision-language models (LVLMs) have shown great promise in medical applications, particularly in visual question answering (MedVQA) and diagnosis from medical images. However, existing datasets and models often fail to consider critical aspects of medical diagnostics, such as the integration of historical records and the analysis of disease progression over time. In this paper, we introduce MMXU (Multimodal and MultiX-ray Understanding), a novel dataset for MedVQA that focuses on identifying changes in specific regions between two patient visits. Unlike previous datasets that primarily address single-image questions, MMXU enables multi-image questions, incorporating both current and historical patient data. We demonstrate the limitations of current LVLMs in identifying disease progression on MMXU-\textit{test}, even those that perform well on traditional benchmarks. To address this, we propose a MedRecord-Augmented Generation (MAG) approach, incorporating both global and regional historical records.
Our experiments show that integrating historical records significantly enhances diagnostic accuracy by at least 20\%, bridging the gap between current LVLMs and human expert performance. Additionally, we fine-tune models with MAG on MMXU-\textit{dev}, which demonstrates notable improvements. We hope this work could illuminate the avenue of advancing the use of LVLMs in medical diagnostics by emphasizing the importance of historical context in interpreting medical images.
Our dataset is released at \href{https://github.com/linjiemu/MMXU}{https://github.com/linjiemu/MMXU}.

\end{abstract}

\section{Introduction}


\begin{figure}[t]
  \includegraphics[width=\columnwidth]{images/dataset.pdf}
  \caption{The upper part of the figure presents an example from our constructed dataset, which includes two images and a question regarding their changes. The lower part illustrates the performance of human experts and several models on three types of disease progression questions, emphasizing a notable gap between the models and human experts.}
  \label{fig:dataset}
\end{figure}


% Large vision-language models (LVLMs) ~\cite{zhu2023minigpt,bai2023qwen,liu2023llava,achiam2023gpt,lu2024deepseek,chen2024internvl}, which integrate pre-trained vision encoders into large language models (LLMs) ~\cite{touvron2023llama,cai2024internlm2,yang2024qwen2,guo2025deepseek} through customized connectors, have achieved remarkable capability in general visual scene perception and understanding, as well as in generating textual descriptions. This success has also driven significant advancements in medical field~\cite{li2023chatdoctor,wu2024pmc,li2024llava,chen2024huatuogpto1medicalcomplexreasoning}.



% Medical Visual Question Answering (MedVQA), a critical downstream task in medical AI, plays a vital role in assisting healthcare professionals with diagnosis and analysis, medical education, and clinical decision-making~\cite{liu2024gemex}. Serving as invaluable sources of evidence in the clinical diagnostic process, medical history and records are frequently utilized by doctors as essential references for diagnosing current conditions~\cite{lorkowski2022medical}. However, as shown in Figure \ref{fig:dataset}, most existing datasets designed for MedVQA predominantly focus on the analysis and questioning of single images~\cite{bae2024ehrxqa,liu2024gemex}, neglecting the rich historical information and medical records that are integral to real-world clinical scenarios. Although there are a few datasets~\cite{hu2023medical} that involve referencing historical medical records for comparison, they still exhibit significant shortcomings in terms of question depth and the granularity of comparison regions.

Stemming from the ever-increasing number of parameters and large-scale training corpus, large vision-language models (LVLMs)~\cite{zhu2023minigpt,bai2023qwen,liu2023llava,achiam2023gpt,lu2024deepseek,chen2024internvl} have demonstrated remarkable capabilities in general visual scene perception and understanding, as well as in generating textual descriptions. As the development of LVLMs accelerates, this paradigm has spurred significant advancements in the medical field~\cite{li2023chatdoctor,wu2024pmc,li2024llava,chen2024huatuogpto1medicalcomplexreasoning}, particularly in the analysis and diagnosis of medical images.


Although current medical LVLMs have shown strong performance in downstream tasks such as medical visual question answering (MedVQA)~\cite{hu2024omnimedvqa,sun-etal-2024-self,saeed-2024-medifact} and medical report generation~\cite{zhou-wang-2024-divide,bu-etal-2024-dynamic,huang-etal-2025-cmeaa,yin-etal-2025-kia} on public benchmarks, they are still limited in responding to basic visual questions (i.e., those involving a single image and a brief description ~\cite{bae2024ehrxqa,liu2024gemex}). In real-world scenarios, diagnosis often requires physicians to integrate both case history and current evidence~\cite{lorkowski2022medical}. Case history serves as an invaluable source of evidence, encompassing factors such as previous medical conditions, treatments, and patient demographics~\cite{liu2024gemex}. Identifying the differences between symptoms, clinical signs, and diagnostic outcomes across the timeline is key to accurate diagnosis. Based on these considerations, we ask: Can LVLMs enhance the diagnostic process by identifying these critical factors?



In this paper, we introduce MMXU (Multimodal and MultiX-ray Understanding), leveraging the rich resources of patient electronic medical records (EMRs) from the MIMIC-CXR dataset. As shown in Figure \ref{fig:dataset}, unlike previous works~\cite{bae2024ehrxqa,liu2024gemex}, which focus solely on the current image or are restricted to simple questions, this benchmark is specifically designed to inquire about differences in specific regions between a patient’s two visits. MMXU contains two splits: \textit{test} and \textit{test}. MMXU-\textit{test} consists of 3,000 entries from 1,201 patients and 2,469 studies, while MMXU-\textit{dev} contains 121K QA pairs involving 114K images. We evaluate a range of open-source and closed-source models and conduct a user study to assess human performance. The results indicate that LVLMs struggle to identify differences between two visits, even the LVLM that demonstrates notable performance on other medical benchmarks exhibits a significant performance gap—nearly 40\%—when compared to human performance.

% Moreover, there is a bias where LVLMs tends to consider the conditions progression into one 

Recognizing the limitations of current LVLMs in such scenarios, we take steps toward bridging this gap. 
Since physicians often rely on patient history, previous treatments, and other contextual information during the diagnostic process, we propose a novel approach, MedRecord-Augmented Generation (MAG) to facilitate LVLMs. We introduce two types of records as contextual information for diagnosis: global records (e.g., patient demographics and overall medical report) and regional records (e.g., specific regional details and diagnostic results). 
% Our experiments show that incorporating global and local records leads to a significant accuracy improvement of at least 20\% for several famous models.
Our experiments show that incorporating global or regional medical records leads to a significant accuracy improvement of at least 20\% for several well-known models.
% Acknowledging the inadequacy of current training datasets in this domain and the observed performance deficiencies, we introduce MMXU-dev, where each data entry includes relevant global and local records.
Acknowledging the inadequate training on such tasks, 
we apply the MAG method for fine-tuning the MMXU-\textit{dev} dataset and substantiate the effectiveness of both the dataset and the MAG approach in enhancing overall performance.

In summary, this paper presents three key contributions. First, we introduce MMXU, the first multi-image MedVQA dataset designed to investigate differences in specific regions of CXR images between a patient’s two visits, addressing the gap between current MedVQA benchmarks and real-world clinical scenarios. Second, our evaluation results show that current large vision-language models (LVLMs), including proprietary models such as GPT-4o, face significant challenges in identifying disease progression between two visits. Third, we propose the MedRecord-Augmented Generation (MAG) method to enhance the diagnosis of X-rays over time by leveraging the patient’s global and regional historical records. Experimental results demonstrate the effectiveness of MAG, underscoring the potential of incorporating contextual information in diagnostic processes.


\begin{table*}[ht]
    \centering\small
    \resizebox{\textwidth}{!}{
        \begin{tabular}{ccccccc}
            \toprule
            \textbf{Dataset} & \textbf{\#Images} & \textbf{\# QA Pairs} & \textbf{\#Question Type} & \textbf{\#Regional} & \textbf{\#Comparative} &\textbf{\#Complex}\\
            \midrule
            \textbf{MIMIC-CXR-VQA}~\cite{bae2024ehrxqa}&142K&377K&Single&\ding{51}&\ding{55}&\ding{51}                  \\
            \textbf{GEMeX}~\cite{liu2024gemex}&151K&1.6M&Single&\ding{51}&\ding{55}&\ding{51}                           \\
            \textbf{Medical-Diff-VQA}~\cite{hu2023medical}&164K&700K&Single\&Multiple&\ding{55}&\ding{51}&\ding{55}       \\
            \textbf{MMXU-\textit{dev} (Ours)}&114K&121K&Multiple&\ding{51}&\ding{51}&\ding{51}                                        \\
            \bottomrule
        \end{tabular}
    }
    \caption{Comparison of the MedVQA dataset constructed on MIMIC-CXR. In the \#Question Type column, ``Single'' refers to questions about a single image, and ``Multiple'' refers to questions about multiple images. Most existing datasets primarily focus on observations in a single image. While Medical-Diff-VQA contains 131k QA pairs that focus on multi-image changes, it has a simple structure and only considers global-level differences. Our MMXU-\textit{dev} dataset is the first to focus on complex changes in the same regions across multiple images at the regional level.}
    \label{tab:dataset-comparison}
\end{table*}

\section{Related Work}

\noindent \textbf{Large Vision-Language Models} ~ ~ Large Vision-Language Models, which integrate vision encoders, connectors, and large language models to enhance cross-modal understanding, have emerged as powerful frameworks that combine visual and textual information for a wide range of tasks. These models can be systematically categorized based on the type of connector. The first category comprises approaches utilizing cross-attention-based connectors, such as Flamingo~\cite{alayrac2022flamingo} and CogVLM~\cite{wang2023cogvlm,hong2024cogvlm2}, which exploit attention mechanisms to facilitate the exchange of information between the vision encoder and the language model. The second category includes methods that employ query-based connectors, such as BLIP-2~\cite{li2023blip}, Instruct-BLIP~\cite{dai2023instructblip}, mPLUG-owl2~\cite{ye2024mplug}, and Qwen-VL~\cite{bai2023qwen}, wherein queries are leveraged to orchestrate the interaction between visual and textual modalities, thereby enhancing the alignment and coherence of visual and linguistic representations. Furthermore, projection-based connector methods, exemplified by LLaVA~\cite{liu2023llava}, Mini-GPT4~\cite{zhu2023minigpt}, DeepSeek-VL~\cite{lu2024deepseek}, and Mini-Gemini~\cite{li2024mini}, project visual data into a shared embedding space, thereby fostering seamless integration with textual information.
These innovations offer a range of solutions for cross-modal understanding, driving the potential applications of intelligent systems in multi-task learning.

% The innovation of various methods offers a range of solutions for cross-modal understanding, thereby driving the potential applications of intelligent systems in multi-task learning.


\noindent \textbf{MedVQA Dataset on radiology} ~ ~ Medical visual question answering (MedVQA) datasets play a pivotal role in advancing AI-driven clinical decision-making. VQA-RAD~\cite{Lau2018}, as an early pioneering work, introduces a meticulously curated dataset for radiology images, featuring clinician-generated questions and corresponding answers tailored to clinically relevant tasks. SLAKE~\cite{liu2021slake} stands out as a large, bilingual dataset, enriched with extensive semantic annotations and spanning a wide range of radiological modalities. 

MIMIC-CXR~\cite{johnson2019mimic} provides a vast collection of 371,920 chest X-rays from 65,079 patients, serving as the foundation for numerous subsequent studies. The comparison of these datasets is shown in Table \ref{tab:dataset-comparison}.  MIMIC-CXR-VQA~\cite{bae2024ehrxqa} seamlessly integrates chest X-rays with Electronic Health Records (EHRs), facilitating multi-modal question answering with an emphasis on region-specific queries. The Medical-Diff-VQA~\cite{hu2023medical} is notable for its inclusion of seven distinct question types, particularly focusing on the comparative analysis of current and reference images for diagnostic purposes. GEMeX~\cite{liu2024gemex} offers a large-scale, explainable VQA benchmark, complete with detailed visual and textual explanations, thus addressing the growing need for a diverse array of clinical questions. Most of these datasets primarily focus on observations from a single image. While Medical-Diff-VQA contains 131k QA pairs that address multi-image changes, its structure is relatively simple and only accounts for global-level differences. Our MMXU-\textit{dev} dataset is the first to understand complex changes in the same regions across multiple chest X-ray images of the same patient at the regional level, spanning several visits.



\section{Dataset Construction}
In this section, we outline the pipeline of constructing MMXU, as shown in Figure \ref{fig:framework}. The process starts with the Chest ImaGenome dataset~\cite{wu2021chest}, which includes the \textit{silver\_dataset} section containing annotations for 243,310 images from 63,945 patients. These annotations cover bounding boxes for 29 anatomical regions, along with corresponding region-level report \textit{phrases}, labeled \textit{attributes}, and \textit{relationships}. The entire method consists of four distinct phases: (1) Comparative Sentences Extraction (\S \ref{sec:stage1}), (2) Comparative Targets Selection (\S \ref{sec:stage2}), (3) QA pairs Generation (\S \ref{sec:stage3}), and (4) Post-Processing (\S \ref{sec:stage4}).

\subsection{Comparative Sentences Extraction}
\label{sec:stage1}
In the first stage, our objective is to identify and extract sentences that contain comparative information, forming the foundation for generating question-answer pairs in subsequent stages. For example, the sentence ``Previously seen ill-defined peribronchial lower lobe opacity seen on lateral view has resolved,'' along with its associated \textit{relationships} label ``comparison|yes|improved'' explicitly indicates a comparison with prior conditions, highlighting the resolution of the lower lobe opacity and thus signaling an improvement in the patient's condition. In total, we extract 232,247 comparative sentences, encompassing 22,770 patients and 102,606 reports. A more detailed example of such a sentence is provided in Appendix \ref{app:stage1}.

\begin{figure*}[t]
  \centering
  \includegraphics[width=2.05\columnwidth]{images/framework.pdf}
  \caption{The pipeline of constructing MMXU. In the first stage, we extract all comparative sentences and their associated labels from the Chest ImaGenome dataset. In the second stage, we refine our selection by applying precise filtering rules and identifying regions of interest within these sentences. The third stage involves leveraging GPT-4o with meticulously crafted prompts to generate region-level question-answer pairs. In the final stage, we further refine and filter generated QA pairs, thereby constructing the MMXU-\textit{test} and MMXU-\textit{dev}.}
  \label{fig:framework}
\end{figure*}


\subsection{Comparative Targets Selection}
\label{sec:stage2}
% In this stage, we classify comparative sentences into three categories based on the disease progression: ``Worsen'' (108,734), ``Improved'' (91,084), and ``No Change'' (264,676). A significant portion of the data is concentrated in the ``No Change'' category. Since the incidence of medical conditions often follows a long-tail distribution~\cite{wu2024medical}, it is crucial to achieve a more balanced distribution for the rigorous evaluation of LVLMs. To this end, we adopt two principles for sentence selection: (1) Retain only one comparative sentence per report, prioritizing those that reflect a change, in order to enhance data diversity and minimize overlap. (2) Randomly discard a subset of ``No Change'' sentences, ensuring that the number of ``Improved'' and ``Worsen'' sentences aligns with that of the ``No Change'' category.
At this stage, we categorize comparative sentences into three groups based on disease progression: ``Worsen'' (108,734), ``Improved'' (91,084), and ``No Change'' (264,676), with the majority concentrated in the ``No Change'' category. Given that the prevalence of medical conditions often follows a long-tail distribution~\cite{wu2024medical}, it is essential to achieve a more balanced dataset for a robust evaluation of LVLMs. To this end, we implement two key principles for sentence selection: (1) Retaining only one comparative sentence per report, prioritizing those indicative of change, to enhance data diversity and reduce redundancy. (2) Randomly discarding a portion of ``No Change'' sentences to equalize their count with that of the ``Improved'' and ``Worsen'' categories, thereby mitigating class imbalance.

% Subsequently, we perform regional selection to improve the precision of the questions. In real-world scenarios, a comparative sentence may refer to multiple anatomical regions. For example, the sentence ``In comparison with the study of \_\_\_, there has been worsening of the increased opacification at the left base with silhouetting of the hemidiaphragm and blunting of the costophrenic angle'' simultaneously describes the ``left lung'', ``left lower lung zone'', ``left costophrenic angle'', and ``left hemidiaphragm''. Therefore, it is essential to select specific regions referenced in the sentence to avoid ambiguity. Our strategy is as follows:
% (1) Discard regions in both the current and historical images that lack an explicitly defined bounding box. (2) Discard regions not described in previous reports, as it is impossible to obtain historical region-level reports. (3) If multiple regions remain, randomly select one based on the original distribution.
Next, we implement a regional selection process to improve query accuracy, where one region is chosen for each sentence to question. In real-world applications, a comparative statement may reference multiple anatomical regions simultaneously. For instance, the sentence ``In comparison with the study of \_\_\_, there has been worsening of the increased opacification at the left base with silhouetting of the hemidiaphragm and blunting of the costophrenic angle'' concurrently describes the ``left lung'', ``left lower lung zone'', ``left costophrenic angle'' and ``left hemidiaphragm''. Consequently, it is crucial to select the precise regions mentioned in the sentence to eliminate ambiguity. Our approach is as follows: (1) Exclude regions in both current and historical images that lack a clearly defined bounding box. (2) Eliminate regions that are not referenced in prior reports, as it is infeasible to retrieve historical region-level reports. (3) In cases where multiple regions remain, randomly select one based on the original distribution.

\subsection{QA pairs Generation}
\label{sec:stage3}

In the third stage, we employ GPT-4o as the generator, leveraging comparative sentences, \textit{relationship} labels, and reports from two visits as foundational data to guide the model in producing question-answer pairs. Our observations suggest that a single sentence can encapsulate multiple changes. For instance, the sentence ``Bilateral pleural effusions are again seen, and atelectasis is present'' concurrently describes alterations in both ``pleural effusions'' and ``atelectasis''. To fully harness this data, we direct GPT-4o to generate up to three distinct question-answer pairs, ensuring maximum diversity. Additionally, to facilitate the future validation of data accuracy, we require the model to provide justifications for the generated pairs. 

In total, we have established six rules to guide GPT-4o generating QA pairs that capture changes in the same area across the two reports. 
Appendix \ref{app:stage3} presents our prompt template and a detailed example during the QA generation process.

\begin{figure}[ht]
    \centering\small
\begin{tcolorbox}[colframe=black]
\ttfamily
Here are two chest X-RAY images reports of the same patient. Previous chest X-RAY is Image-1. Current chest X-RAY is Image-2. Your task is to evaluate the differences between the two images based on the provided report and question.\\
The report of Image-1 is: \%s
\\
The report of Image-2 is: \%s
\\
Question:\%s
\\
Options:\\
A: \%s\\
B: \%s\\
C: \%s\\
D: \%s\\
Answer with the option's letter from the given choices directly.

\end{tcolorbox}
\caption{Text template during the data post-processing.}
    \label{fig:post-processing}
\end{figure}

\subsection{Post-Processing}
\label{sec:stage4}

Finally, we conduct post-processing to further ensure the quality of generated QA-pairs. Since imbalanced answer options can mislead the model during training and result in unfair evaluations during testing, we first balance the distribution of four answer choices in the QA pairs generated by GPT-4o. Moreover, we employ GPT-4o to answer the generated questions with the prompt detailed in Figure  \ref{fig:post-processing}, which contains the information used during question generation. Questions that were answered incorrectly are considered either excessively difficult or erroneous and are removed. As illustrated in Table \ref{tab:post}, 95.1\% of the questions were answered correctly. The final candidate set comprises 121,800 QA pairs, including 47,000 instances of ``Worsen'', 41,000 instances of ``Improved'' and 35,000 instances of ``No Change''.

\begin{table}[ht]
    \centering\small
    \begin{tabular}{c|ccc}
        \toprule
       & \textbf{\#Examples} & \textbf{\#Correctness} & \textbf{Rate}\\
        \midrule
        \textbf{Worsen}&50,305&45,395&90.2\%\\
        \textbf{Improved}&40,647&39,956&98.3\%\\
        \textbf{No change}&37,124&33,449&90.1\%\\
        \textbf{Overall}&128,076&121,800 & 95.1\%\\
        \bottomrule
    \end{tabular}
    \caption{Post-processing validation results of QA pairs using GPT-4o. Here, \#Total and \#Correctness represent the total number of original QA pairs and the number of correctly answered pairs, respectively.}
    \label{tab:post}
\end{table}

\noindent \textbf{Data Splitting} ~ ~% 数据集划分的原则
To facilitate a rigorous evaluation for LVLMs, we carefully curate the MMXU-\textit{test} benchmark comprising 1,000 data samples for each of the three categories—``Worsen'', ``Improved'' and ``No Change''. The remaining data forms the MMXU-\textit{dev} dataset. During the selection process, we meticulously ensured that questions derived from the same patient did not appear in both the \textit{dev} and \textit{test} sets, thereby preventing data leakage. Furthermore, we maintained a balanced regional distribution across the training and test sets to preserve the integrity of the evaluation. Finally, the MMXU-\textit{test} contains 3,000 QA pairs, and MMXU-\textit{dev} contains 118,800 QA pairs.


% With these considerations in place, we then conducted a randomized partitioning of the data to uphold fairness and impartiality in the assessment process.
\section{Benchmark Results}
To ensure the professionalism and accuracy of MMXU-\textit{test} benchmark, we recruited a panel of 5 board-certified chest radiology experts to assess it. Following that, we evaluated the performance of several prominent open-source and closed-source LVLMs capable of supporting multi-image VQA on the MMXU-\textit{test} benchmark. Since all the questions from our benchmark are single-choice, we use accuracy as the metric. 

% During the testing process, we closely focus on the results of the three aforementioned question categories.
\subsection{Evaluation Models}
Prominent medical large vision-language models (LVLMs), such as LLaVA-Med-v1~\cite{li2024llava}, LLaVA-Med-v1.5~\cite{li2024llava}, MiniGPT-Med~\cite{alkhaldi2024minigpt}, and XrayGPT~\cite{thawkar2023xraygpt}, have predominantly been trained on single-image visual question answering (VQA) tasks. This limitation poses challenges in accurately evaluating our dataset's performance on these models. Therefore, all the models used in this paper belong to the general domain, as listed below:

\noindent\textbf{Open-source LVLMs}: Qwen2-VL 2B\& 7B~\cite{bai2023qwen}, DeepSeek-VL 1.3B\& 7B~\cite{lu2024deepseek}, InternVL2 1B\&2B\& 4B\&8B~\cite{chen2024internvl}, IDEFICS2 8B~\cite{laurenccon2024matters} and Llama3.2-Vision 11B~\cite{touvron2023llama}
\noindent\textbf{Closed-source LVLMs}: GPT-4o~\cite{achiam2023gpt} and  Claude-3-5-sonnet


\begin{table}[ht]
    \centering\small
    \begin{tabular}{c|ccc}
        \toprule
        \textbf{Expert ID} & \textbf{\#Examples} & \textbf{\#Correctness} & \textbf{Rate} \\
        \midrule
        1 & 500 & 487 & 97.4\% \\
        2 & 500 & 480 & 96.0\% \\
        3 & 500 & 453 & 90.6\% \\
        4 & 750 & 732 & 97.6\% \\
        5 & 750 & 723 & 96.4\% \\
        \midrule
        \textbf{Total} & 3,000 & 2,875 & 95.8\% \\
        \bottomrule
    \end{tabular}
    \caption{Evaluation results of five experienced chest radiology human experts on the MMXU-\textit{test} benchmark.}
    \label{tab:expert_evaluation}
\end{table}




\begin{table*}[ht!]
\centering
\small
\begin{tabular}{c|c|c|ccc|cccc}
\toprule
\multirow{2}{*}{\textbf{Source}}&\multirow{2}{*}{\textbf{LVLMs}}& \multirow{2}{*}{\textbf{Size}}& 
\multicolumn{3}{c|}{\textbf{VQA-RAD}} & \multicolumn{4}{c}{\textbf{MMXU-\textit{test}}} \\
\cmidrule{4-10}
& & & \rule{0pt}{8pt}\textbf{Closed} & \textbf{Open} & \textbf{Overall}  
 & \textbf{Worsen} & \textbf{Improved} & \textbf{No change} & \textbf{Overall}  \\
\midrule
\multirow{10}{*}{Open}&\multirow{2}{*}{Qwen2-VL}&
   2B& 0.594&0.380&0.499&\textbf{0.712}&0.258&0.284&0.418\\
& &7B& \textbf{0.745}&0.430&\textbf{0.605}&0.331&0.554&0.500&0.458\\
\cmidrule{2-10}
\rule{0pt}{8pt}&\multirow{2}{*}{DeepSeek-VL}
&1.3B& 0.566&0.250&0.426&0.292&0.294&0.288&0.291\\
& &7B& 0.582&0.300&0.457&0.389&0.310&\underline{0.606}&0.435\\
\cmidrule{2-10}
\rule{0pt}{8pt}&\multirow{4}{*}{InternVL2}
  &1B& 0.490&0.265&0.390& 0.553&0.214&0.113&0.293\\
& &2B& 0.641&0.350&0.512& 0.560&0.182&0.559&0.434\\
& &4B& 0.649&0.370&0.525&0.476&0.571&0.315&0.454\\
& &8B& 0.665&0.480&0.583&0.423&0.483& 0.495 & 0.467 \\
\cmidrule{2-10}
\rule{0pt}{10pt} &IDEFICS2&8B& \underline{0.673}&0.450&0.574& 0.234& 0.570 & \textbf{0.668}&0.491\\
\cmidrule{2-10}
\rule{0pt}{10pt} &Llama3.2-vision&11B& 0.649&\textbf{0.515}&\underline{0.590}&\underline{0.596}&\underline{0.608}&0.356&\underline{0.520}\\
\midrule
\multirow{2}{*}{Closed}& GPT-4o&-&0.578& 0.480&0.534&0.480&\textbf{0.675}&0.559&\textbf{0.571}\\
\cmidrule{2-10}
\rule{0pt}{10pt}& Claude-3.5&-& 0.622&\underline{0.510}&0.572&0.494&0.518&0.493&0.502\\
\bottomrule
\end{tabular}
\caption{Evaluation results of several mainstream open-source and closed-source LVLMs supporting multi-image question answering on the VQA-RAD and MMXU-\textit{test} benchmarks. For open-ended questions in VQA-RAD, we used GPT-4o to evaluate LVLMs' responses. The results in \textbf{bold} and \underline{underlined} represent the best and the second-best
results, respectively.}
\label{tab:zero-shot}
\end{table*}



\subsection{Human Expert Evaluation}
\label{sec:humanexpertevaluation}
To evaluate the quality of the MMXU-\textit{test} benchmark, we conduct the human expert evaluation with five radiologists. The data from the MMXU-test benchmark was randomly divided into five parts, containing 500, 500, 500, 750, and 750 questions, respectively. We ensured that the three question categories were distributed as evenly as possible within each subset. The evaluation results are presented in Table \ref{tab:expert_evaluation}. Except for Expert 3, all experts achieved an accuracy rate of at least 96.0\%, with an overall accuracy reaching 95.3\%. These findings demonstrate that the MMXU-\textit{test} benchmark is both highly professional and well-structured.

%After completing the evaluation, all experts acknowledged the benchmark’s quality and validity.

\subsection{Model Performance}

We evaluated the performance of several well-known open-source and closed-source models on two benchmarks: VQA-RAD, which focuses on single-image visual question answering, and MMXU-\textit{test}, which emphasizes multi-image difference analysis. The experimental results are presented in Table \ref{tab:zero-shot}. For open-ended questions in VQA-RAD~\cite{Lau2018}, we utilized GPT-4o to compare the model-generated answers with the ground truth, incorporating a certain level of tolerance to assess correctness ~\cite{he2024pefomed}.

Comparing the performance of these models on single-image and multi-image benchmarks, it is clear that \textbf{the accuracy of the same model on overall accuracy on MMXU-\textit{test} is lower than its VQA-RAD closed-ended questions}.
For example, with the Qwen2-VL 7B model, the accuracy on VQA-RAD closed questions reaches an impressive 74.5\%. However, its performance on MMXU-\textit{test} drops significantly to just 45.8\%. This discrepancy highlights the gap between current MedVQA benchmarks and the demands of real-world scenarios, suggesting that models that perform well on public benchmarks may not be effective in supporting clinical diagnosis. Furthermore, as shown in Table \ref{tab:expert_evaluation}, even the best-performing model lags behind human experts by nearly 40\%. This highlights the existing limitations of these models in multi-image MedVQA tasks.

Additionally, we can observe that almost all open-source models \textbf{exhibit significant accuracy discrepancies across the three types of questions} in the MMXU-\textit{test}, suggesting that these models have certain biases when interpreting disease progression. This bias is particularly noticeable in smaller models, such as Qwen2-VL 2B, which shows an accuracy of 71.2\% on the ``Worsen'' question, while its accuracy on the other two question types is below 30\%. Clearly, this model tends to favor results associated with disease deterioration.  Additionally, such biases are less apparent in closed-source models. \textbf{As the model size increases, the overall accuracy improves}, and the bias of progression gradually diminishes. Despite all models being general-domain models, their results still exhibit noticeable differences.
% For the InternVL2 1B and 2B models, the accuracy of one category is notably higher—several times greater than that of the other categories. However, in the 8B model, the accuracy across all three categories becomes more balanced. 
% Additionally, we observe that smaller-scale models yield relatively similar overall results, with discrepancies becoming more pronounced as the model size exceeds 7B. For example, DeepSeek-VL 1.3B and InternVL2 1B, as well as Qwen2-VL 2B and InternVL2 2B, show comparable performance. In contrast, larger models like DeepSeek-VL 7B, InternVL2 8B, and IDEFICS2 8B exhibit more significant variation in their results.


\section{MedRecord-Augmented Generation}

Through prior experiments, we pinpointed that current LVLMs face significant challenges in identifying disease progression. To address this challenge, we propose a novel approach, MedRecord-Augmented Generation (MAG). In routine diagnostic practice, physicians often rely on a patient’s historical records to inform their analysis and diagnosis of current conditions. To replicate this process, we integrate historical records directly into the prompt as contextual information. Our study investigates the effectiveness of global reports derived from historical images, alongside regional reports related to the specified questions, which we categorize as global and regional historical records, respectively. 

\subsection{Effectiveness of MAG}

\begin{table}[ht!]
\centering
\small
\resizebox{0.5\textwidth}{!}{%
\begin{tabular}{lcccc}
\toprule
\textbf{Method}& \textbf{Worsen} & \textbf{Improved} & \textbf{No change} & \textbf{Overall} \\
\midrule
\rowcolor[HTML]{EDEDED}\multicolumn{5}{c}{\textit{\textbf{InternVL2 8B}}}\\
- &0.423&0.483&0.495&0.467\\
+ G-MRec&0.440&0.644&0.838&0.641\\
+ R-MRec&\textbf{0.483}&\textbf{0.830}&\textbf{0.842}&\textbf{0.718}\\
\midrule
\rowcolor[HTML]{EDEDED}\multicolumn{5}{c}{\textit{\textbf{Qwen2-VL 7B}}}\\
- &0.331&0.544&0.500&0.458\\
+ G-MRec&0.427&0.683&0.784&0.631\\
+ R-MRec&\textbf{0.458}&\textbf{0.755}&\textbf{0.805}&\textbf{0.673}\\
\midrule
\rowcolor[HTML]{EDEDED}\multicolumn{5}{c}{\textit{\textbf{GPT-4o}}}\\
- &\textbf{0.480}&0.675&0.559&0.571\\
+ G-MRec&0.380&0.760&0.629&0.590\\
+ R-MRec&0.374&\textbf{0.802}&\textbf{0.765}&\textbf{0.647}\\
\bottomrule
\end{tabular}
}
\caption{Results of MedRecord-augmented generation on the MMXU-\textit{test} benchmark without fine-tuning. G-MRec and R-MRec denote generation augmentation using global and regional historical records, respectively.}
\label{tab:zero-shot-mag}
\end{table}

Firstly, we conducted experiments with MedRecord-augmented generation tests on both the open-source models InternVL2 8B, Qwen2-VL 7B, and the closed-source model GPT-4o. The outcomes of these experiments are summarized in Table \ref{tab:zero-shot-mag}. It is clear that \textbf{providing historical records significantly improves the models' overall accuracy}. With historical records, the open-source models achieved accuracy comparable to that of closed-source commercial models. Except for GPT-4o, using medical records led to improvements across all three types of questions, indicating that these models could understand and reason from historical records instead of merely repeating past information. Moreover, it can be observed that \textbf{regional historical records offer greater improvements than global records.} This may be attributed to the models' limited ability to analyze contextual and region-specific details, making it difficult for them to identify the most relevant information. By providing precise, region-specific information, the models are better able to extract valuable information.


\begin{table}[ht]
\centering
\small
\resizebox{0.48\textwidth}{!}{%
\begin{tabular}{lcccc}
\toprule
\textbf{Method }& \textbf{Worsen} & \textbf{Improved} & \textbf{No change} & \textbf{Overall} \\
\midrule
\rowcolor[HTML]{EDEDED}\multicolumn{5}{c}{\textit{\textbf{InternVL2 8B + 20\% MMXU-\textit{dev}}}}\\
-&0.834&0.802&0.810&0.815\\
+ G-MRec&0.856&0.841&0.843&0.847\\
+ R-MRec&0.866&0.830&0.848&0.848\\
\midrule
\rowcolor[HTML]{EDEDED}\multicolumn{5}{c}{\textit{\textbf{InternVL2 8B + 40\% MMXU-\textit{dev}}}}\\
-&0.838&0.813&0.839&0.830\\
+ G-MRec&0.867&0.857&0.867&0.864\\
+ R-MRec&0.868&0.845&0.874&0.862\\
\midrule
\rowcolor[HTML]{EDEDED}\multicolumn{5}{c}{\textit{\textbf{InternVL2 8B + 60\% MMXU-\textit{dev}}}}\\
-&0.855&0.822&0.841&0.839\\
+ G-MRec&0.884&0.855&0.873&0.871\\
+ R-MRec&0.880&0.855&0.869&0.868\\
\midrule
\rowcolor[HTML]{EDEDED}\multicolumn{5}{c}{\textit{\textbf{InternVL2 8B + 80\% MMXU-\textit{dev}}}}\\
-&0.857&0.814&0.866&0.846\\
+ G-MRec&0.881&0.861&0.884&0.876\\
+ R-MRec&0.873&0.869&0.881&0.875\\
\midrule
\rowcolor[HTML]{EDEDED}\multicolumn{5}{c}{\textit{\textbf{InternVL2 8B + 100\% MMXU-\textit{dev}}}}\\
-&0.860&0.840&0.852&0.851\\
+ G-MRec&0.887&0.878&0.882&0.883\\
+ R-MRec&0.884&0.871&0.888&0.881\\
\bottomrule
\end{tabular}
}
\caption{Fine-tuning Results of InternVL2 8B on MMXU-\textit{dev} dataset with MAG Method. G-MRec and R-MRec denote generation augmentation using global and regional historical records, respectively.}
\label{fig:finetuned}
\end{table}


The extent to which historical records contribute to performance improvement varies across different question categories. As illustrated in Table \ref{tab:zero-shot-mag}, historical records notably enhance performances in the ``Improved'' and ``No Change'' categories, while providing only minimal benefit in the ``Worsen'' category. In fact, for GPT-4o, performance in this category even deteriorated. This indicates that the use of historical records may \textbf{not effectively address biases} in a high-quality and robust manner.


\subsection{MAG Fine-tuning on MMXU-\textit{dev}}

Furthermore, we assess the efficacy of MMXU-\textit{dev} dataset and our proposed MAG method by fine-tuning the InternVL2 8B. The model was fine-tuned using 20\%, 40\%, 60\%, 80\%, and 100\% of the MMXU-\textit{dev} dataset. We evaluate the performance on MMXU-\textit{test} benchmark. The corresponding fine-tuning results are presented in Table \ref{fig:finetuned}.


\begin{figure*}[t]
  \centering
  \includegraphics[width=2.05\columnwidth]{images/case.pdf}
  \caption{Two Examples from MMXU answered by some LVLMs. The left image shows successful cases with medical records enhancing the answers. The right image shows failed cases where historical records mislead the answers.}
  \label{fig:case}
\end{figure*}


We observed that even with just 20\% of the MMXU-\textit{dev}, the model achieved substantial improvements in accuracy across all three problem categories, and \textbf{the bias was almost eliminated}. As the dataset size expanded, the model's overall performance continued to improve. When utilizing the full 100\% of MMXU-\textit{dev}, the performance gap between the model and human experts narrowed to approximately 10\%. This clearly underscores the efficacy of the MMXU-\textit{dev} dataset we developed. Additionally, we found that the MAG method continues to deliver significant improvements after model fine-tuning. 
Notably, as the volume of training data increases,\textbf{ the enhancement effect of global historical records surpasses that of regional historical records}.
Fine-tuning with larger datasets enables the model to more effectively capture relevant information, resulting in more comprehensive and refined final outputs.



\subsection{Case Study}

In Figure \ref{fig:case}, we present two answer samples provided by GPT-4o, InternVL2 8B, Qwen2-VL 7B, and fine-tuned InternVL2 8B, evaluated under various historical record strategies. Case 1 illustrates a successful scenario where all models adjusted their responses accurately when regional historical records were incorporated.  However, the zero-shot InternVL model misinterpreted critical information, primarily due to the global records containing terms such as ``unchanged'' and ``minimally,'' which led to an erroneous conclusion. Fine-tuning effectively mitigated this issue. In Case 2, historical records resulted in misleading predictions. When historical information was absent, all models delivered correct answers. However, the current right pleural effusion had diminished in size, and prior diagnoses indicated no changes. The inclusion of historical records resulted in an incorrect response, and fine-tuning did not yield an improvement in this case. For more case studies, please refer to Appendix \ref{app:more-case-study}.


\section{Conclusion}
% In this paper, we introduce MMXU-test, a multi-image chest X-ray dataset designed to analyze the progression of medical conditions through visual question answering. Furthermore, we propose a benchmark for medical records aimed at advancing Medical VQA, simulating clinical scenarios where physicians rely on medical records for diagnosis. We fine-tuned and evaluated several open-source and commercial models that support multi-image VQA. Experimental results show that incorporating medical records enhances answer accuracy, regardless of condition changes. We anticipate that MMXU-test will make a significant contribution to improving LVLMs' ability to comprehend multi-image discrepancies, analyze condition progression, and effectively utilize historical medical records.

% In this paper, we introduce MMXU, a dataset for multi-modal and multi-X-ray understanding in MedVQA. We first invited human experts to evaluate the benchmark and then conducted evaluations using several well-known open-source and closed-source LVLMs supporting multi-image VQA. The experimental results reveal that even the best-performing models still show a substantial gap compared to human experts—nearly a 40\%. Furthermore, we explored the impact of historical records on the performance of LVLMs in this benchmark and proposed the MAG method. Through zero-shot testing and fine-tuning using MMXU-\textit{dev} dataset, we further validated the effectiveness of this method.
In this paper, we introduce MMXU, a dataset designed for multi-modal and multi-X-ray understanding in MedVQA. First, we propose a benchmark, MMXU-\textit{test}, and invite five chest X-ray experts to evaluate the performance. Then, we conduct evaluations using several well-known open-source and closed-source large vision-language models (LVLMs) that support multi-image VQA. The experimental results indicate that even the best-performing models exhibit a significant performance gap—nearly 40\%—compared to human experts. To bridge this gap, we propose the MAG method, which leverages historical records to enhance the understanding of disease progression. We further evaluate its performance both without fine-tuning and with fine-tuning on MMXU-\textit{dev} and the experiment confirm the effectiveness.

\section*{Limitation}
Although we have carefully designed our MMXU-\textit{test} benchmark and established the MMXU-\textit{dev} dataset, there are still some limitations: 1) Our dataset is based on MIMIC-CXR, a chest X-ray dataset, which somewhat limits its generalization when applied to other datasets. 2) Most current medical LVLMs, which have extensive medical knowledge, only support single-image VQA, thus restricting our evaluation to general-domain LVLMs capable of handling multiple images. 3) We have proposed the MAG method, inspired by clinical scenarios, to validate the effectiveness of historical historical records in enhancing LVLMs' medical responses. Although we aim to replicate clinical scenarios as closely as possible, the scarcity of data means we can only use previous reports as historical records for research purposes.

\bibliography{custom}

\appendix

\begin{figure*}[htbp]
    \centering
    \subfigure[]{
        \includegraphics[width=0.45\textwidth,height=5cm]{images/m-3.pdf}
        \label{fig:m-3}
    }
    \subfigure[]{
        \includegraphics[width=0.45\textwidth]{images/m-1.pdf}
        \label{fig:m-1}
    }
    \hspace{0.5cm}
    
    \caption{ 
    Figure (a) shows the distribution of the number of patients corresponding to different study counts. We can see that a large number of patients have only one study.
    Figure (b) shows the \textcolor[HTML]{6571EE}{blue line} representing the changes between two studies of the same patient in MIMIC-CXR as the study order increases. The \textcolor[HTML]{EA3323}{red dashed lines} indicate the differences across patients with a number of studies no less than specific values. 
    }
    \label{fig:tm}
\end{figure*}

\newpage
\section{Pilot Studies on Medical Diff-VQA}
\label{sec:plot_study}
\subsection{Motivation}
In the clinical diagnostic process, medical experts integrate a patient's medical historical records with current medical evidence to guide their diagnostic decisions. To explore whether this approach can enhance the diagnostic capabilities of large vision-language models in medical settings, we fine-tuned and evaluated InternVL2 8B and Qwen2-VL 8B on the publicly available Medical-Diff-VQA dataset~\cite{hu2023medical}, incorporating historical medical records.

\begin{figure}[ht]
    \centering\small
\begin{tcolorbox}[colframe=black]
\ttfamily
Assume you are a professional thoracic doctor. You are now provided with a chest radiology image. There is also a certain probability that you will be given the patient’s previous treatment image and report. Based on the provided information, please answer the corresponding questions. \\
The current study image is <image>.\\
The reference image is <image>.\\
The report of reference image is: \{historical records\}\\
The question is: \{question\}

\end{tcolorbox}
\caption{Text template on Medical-Diff-VQA using historical records.}
    \label{fig:medical-diff-vqa-with}
\end{figure}

\begin{figure}[ht]
    \centering\small
\begin{tcolorbox}[colframe=black]
\ttfamily
Assume you are a professional thoracic doctor. You are now provided with a chest radiology image and a question. Based on the provided information, please answer the corresponding questions.\\
The study image is <image>.\\
The question is: \{question\} 
\end{tcolorbox}
\caption{Text template on Medical-Diff-VQA without historical records.}
    \label{fig:medical-diff-vqa-without}
\end{figure}

The Medical-Diff-VQA dataset comprises seven categories of questions: abnormality, location, type, level, view, presence, and difference. However, we excluded the view questions, as they do not require historical information. Therefore, we selected six categories of questions from the Medical-Diff-VQA dataset for our study: \textbf{abnormality}, \textbf{location}, \textbf{type}, \textbf{level}, \textbf{presence}, and \textbf{difference}. The distribution of the training and test sets for each category is detailed in Table \ref{tab:number}. For each category, two distinct templates were used as model inputs: one incorporating historical records and the other devoid of additional information, as illustrated in Figure \ref{fig:medical-diff-vqa-with} and Figure \ref{fig:medical-diff-vqa-without}. These templates were employed to fine-tune and evaluate the model performance. It is worth noting that not all questions are associated with historical information; some are constructed based on records from a patient's initial visit, seen in Figure \ref{fig:m-3}. In such instances, we directly employed the template without additional information as the model input.


\begin{table*}[ht]
\centering\small
\begin{tabular}{l|cccccc}
\toprule
 & Abnormality & Level & Location & Presence & Type & Difference \\
\midrule
Training set & 116,394 & 53,728 & 67,187 & 124,654 & 22,067 & 131,563 \\
Test set & 14,515 & 6,846 & 8,496 & 15,523 & 2,702 & 16,389 \\
\bottomrule
\end{tabular}
\caption{The number of training and test samples for the six categories of questions in Medical-Diff-VQA.}
\label{tab:number}
\end{table*}

\begin{figure}[ht]
\includegraphics[width=0.5\textwidth]{images/m-2.pdf}
\caption{Visualization of experimental results for different settings on the Medical-Diff-VQA dataset.}
\label{fig:m-2}
\end{figure}

For historical records, we utilized the report in MIMIC-CXR dataset, which contains data from 65,079 subjects, each representing an individual patient. A subject may have multiple studies, where each study corresponds to a patient's visit and includes multiple chest X-ray (CXR) images along with an associated medical report. As shown in Figure \ref{fig:m-1}, the blue line shows the average L1 distance variation between the Chexpert~\cite{ye2020weakly} labels of different studies within the same subject. The red dashed line indicates the average L1 distance between patients who meet certain conditions in terms of the number of studies. It is evident that as the study order increases, the differences between studies grow significantly. Furthermore, under the same conditions, the differences between patients are much greater than those within a single patient. Based on these observations, we \textbf{selected the most recent frontal CXR and report} as the historical records for our study.




\subsection{Experimental}
\textbf{Experimental Setting} We benchmarked our approach against the DRax method~\cite{10.1007/978-3-031-73471-7_10}, employing distinct evaluation metrics: \textbf{token recall} for open-ended questions and \textbf{accuracy} for close-ended ones. The models utilized in the experiments were InternVL2 8B and Qwen2-VL 8B. During the fine-tuning process, the LoRA rank was set to 16, the learning rate to 4e-5, and a cosine learning rate scheduling strategy was implemented. Training was conducted using the DeepSpeed Zero Stage 2~\cite{rajbhandari2020zero} distributed strategy on a 8-GPU RTX 3090 server.


\textbf{Experimental Results} Figure \ref{fig:m-2} shows the visualization of experimental results on 5 types of single-image questions (including closed-ended questions and open-ended questions). The results clearly indicate that integrating historical records as references significantly enhances VQA performance on both Qwen2-VL 8B and InternVL2 8B. 

\begin{table*}[ht!]
\centering
\resizebox{\textwidth}{!}{
\begin{tabular}{l|ccc|ccc|ccc|c}
\toprule
\multirow{2}{*}{\textbf{Method}} & \multicolumn{3}{c|}{\textbf{NLG metrics}} & \multicolumn{3}{c|}{\textbf{Missed Labels}} & \multicolumn{3}{c|}{\textbf{Added Labels}}& \multirow{2}{*}{\textbf{Average}}\\
\cmidrule{2-10}
 & \textbf{BLUE-1} & \textbf{Token Recall} & \textbf{Token F1} & \textbf{ML Acc} & \textbf{ML Recall} & \textbf{ML F1} & \textbf{AL Acc} & \textbf{AL Recall} & \textbf{AL F1} \\
\midrule
w/\ \ \  MedRecord& 0.464435 & 0.614229 & 0.59154 & 0.884127 & 0.267652 & 0.259081 & 0.885768 & 0.283076 & 0.275101 & 0.502779 \\
w/o MedRecord &\textbf{ 0.550117} & \textbf{0.684575} & \textbf{0.674029} & \textbf{0.947046} & \textbf{0.629887} & \textbf{0.615927} & \textbf{0.901559} & \textbf{0.345302} & \textbf{0.334627} & \textbf{0.631452} \\
\bottomrule
\end{tabular}
}
\caption{Evaluation results for the differences category in Medical-Diff-VQA using different methods.}
\label{tab:mdv6}
\end{table*}

\begin{figure*}[ht!]
\centering\small
\begin{tcolorbox}[colframe=black]
\ttfamily
\{\\
\hspace*{5mm}"qid": 53804,\\
\hspace*{5mm}"question": "Considering the changes observed, what can be concluded about the left lung in the recent CXR images?",\\
\hspace*{5mm}"question\_type": "single choice",\\
\hspace*{5mm}"content\_type": "worsen",\\
\hspace*{5mm}"options": \{\\
\hspace*{10mm}"A": "Worsened with increased effusion.",\\
\hspace*{10mm}"B": "Stable with no change.",\\
\hspace*{10mm}"C": "Improved or resolving condition.",\\
\hspace*{10mm}"D": "Complete resolution."\\
\hspace*{5mm}\},\\
\hspace*{5mm}"answer": "A",\\
\hspace*{5mm}"region\_name": "left lung",\\
\hspace*{5mm}"current\_image\_path": "files/p19/p19839145/s55247427/79b946ab-8d938e59-52027cad-8b4a4268-dab 951d0.jpg",\\
\hspace*{5mm}"regional\_current\_bbox": [1637, 122, 2796, 2196],\\
\hspace*{5mm}"history\_image\_path": "files/p19/p19839145/s59400960/36b4f554-95744e28-8b7baa81-fa621df8-41a d666f.jpg" ,\\
\hspace*{5mm}"regional\_history\_bbox": [1514, 463, 2442, 2278],\\
\hspace*{5mm}"regional\_history\_report": "Opacities in the left upper lobe have markedly improved. Moderate bilateral effusions larger on the left side with adjacent atelectasis are grossly unchanged. There is no pneumothorax.", \\
\hspace*{5mm}"global\_history\_report": "Mild cardiomegaly is stable. The aorta is tortuous. Cardiomediastinum is shifted to the right as before. Moderate bilateral effusions larger on the left side with adjacent atelectasis are grossly unchanged. Opacities in the left upper lobe have markedly improved. There is no pneumothorax."\\
\}
\end{tcolorbox}
\caption{A detailed data case sampled from the MMXU dataset.}
    \label{fig:detailed_data_case}
\end{figure*}

For the sixth category, the difference questions, we also investigated the effect of providing reference images along with historical records on the outcomes. These results are shown in Table \ref{tab:mdv6}. The responses to ``difference'' questions consist primarily of two components: the 14 ChexPert labels~\cite{ye2020weakly} that are either missing or added in the main image compared to the reference. The ``Missed Labels'' and ``Added Labels'' sections of the table employ accuracy, recall, and F1 scores for the quantitative assessment of the model outputs. Our findings indicate that the inclusion of historical records resulted in a substantial improvement across all evaluation metrics. (The 14 clinical efficacy labels in chexpert include: atelectasis, cardiomegaly, consolidation, edema, enlarged cardiomediastinum, fracture, lung lesion, lung opacity, no finding, pleural effusion, pleural other, pneumonia, pneumothorax, support devices.)

\begin{figure*}[ht!]
\centering\small
\begin{tcolorbox}[colframe=black]
\ttfamily
\{\\
\hspace*{5mm}"cmp\_id": 3,\\
\hspace*{5mm}"subject\_id": "14731346",\\
\hspace*{5mm}"study\_id": "54684841",\\
\hspace*{5mm}"cur\_image\_path": "files/p14/p14731346/s54684841/56ec2474-cfdbbdcd-5b133f15-6c0bc409-17435e c2.jpg",\\
\hspace*{5mm}"prior\_image\_path": "files/p14/p14731346/s51369333/33585d85-edf829eb-51303048-7bd9062a-b6ee3 a5b.jpg",\\
\hspace*{5mm}"cur\_report": "Right lower quadrant pain. In comparison with the study of \_\_\_, there has been worsening of the increased opacification at the left base with silhouetting of the hemidiaphragm and blunting of the costophrenic angle. These findings are consistent with a combination of volume loss in the left lower lobe and pleural effusion. Right lung is clear, and there is no evidence of vascular congestion.",\\
\hspace*{5mm}"prior\_report": "Hyperglycemia, intubated for airway protection, please assess NG tube and ET tube placement. AP radiograph of the chest was compared to prior study obtained the same day earlier. The ET tube tip is approximately 4.3 cm above the carina and slightly impinging the right wall of the trachea and should be repositioned. The NG tube tip is in the stomach. Right subclavian line tip is at the level of cavoatrial junction. Heart size and mediastinum are unremarkable. There is no pneumothorax. Minimal bibasilar opacities most likely reflect areas of atelectasis. No pulmonary edema is seen.",\\
\hspace*{5mm}"comp\_sent": "FINDINGS: In comparison with the study of \_\_\_, there has been worsening of the increased opacification at the left base with silhouetting of the hemidiaphragm and blunting of the costophrenic angle.",\\
\hspace*{5mm}"relationships": \{\\
\hspace*{10mm}"left lung": ["comparison|yes|worsened"],\\
\hspace*{10mm}"left lower lung zone": ["comparison|yes|worsened"],\\
\hspace*{10mm}"left costophrenic angle": ["comparison|yes|worsened"],\\
\hspace*{10mm}"left hemidiaphragm": ["comparison|yes|worsened"]\\
\hspace*{5mm}\},\\
\hspace*{5mm}"related\_region\_names": ["left lung", "left lower lung zone", "left costophrenic angle", "left hemidiaphragm"],\\
\hspace*{5mm}"cur\_image\_size": [2544, 3056],\\
\hspace*{5mm}"cur\_image\_bboxs": \{\\
\hspace*{10mm}"left lung": [1186, 777, 2087, 2210],\\
\hspace*{10mm}"left lower lung zone": [1241, 1705, 2087, 2210],\\
\hspace*{10mm}"left costophrenic angle": [1719, 1719, 1991, 1991],\\
\hspace*{10mm}"left hemidiaphragm": [1186, 1691, 2605, 2769] \\ 
\hspace*{5mm}\},\\
\hspace*{5mm}"prior\_image\_size": [[2544, 3056]],\\
\hspace*{5mm}"prior\_image\_bboxs": \{\\
\hspace*{10mm}"left lung": [[1268, 327, 2537, 2878]],\\
\hspace*{10mm}"left lower lung zone": [[1487, 1705, 2537, 2878]],\\
\hspace*{10mm}"left costophrenic angle": [[]],\\
\hspace*{10mm}"left hemidiaphragm": [[]] \\ 
\hspace*{5mm}\}\\
\}
\end{tcolorbox}
\caption{A detailed data case about \textbf{comparative sentence} during dataset construction.}
\label{fig:detailed_comp_sentence}
\end{figure*}

However, the Medical-Diff-VQA dataset does not fully capture the nuanced relationship between these questions and the evolution of historical historical records. This shortcoming complicates the determination of whether the model is truly leveraging historical information or simply reiterating prior data. Such behavior falls short of the expectations for real-world clinical applications. As a result, \textbf{there is an urgent need for a dataset that seamlessly integrates and balances historical medical records}, thereby supporting their meaningful utilization in practical healthcare settings.


\section{Detailed Data Case}
Figure \ref{fig:detailed_data_case} illustrates a detailed example from the MMXU dataset we developed, which is specifically tailored to analyze changes between two historical records of a patient related to chest X-rays. Each sample includes a single-choice question focused on changes in the condition of a particular region, along with comprehensive records from the patient's two visits. For the initial visit, the dataset provides images, bounding boxes marking the relevant regions, region-level reports, and overall-level reports. The questions are divided into three categories: ``Worsened'', ``Improved'', and ``No Change'', based on the comparative statements within the raw report. This dataset serves as a valuable resource for advancing research in multi-image comparison, historical record integration, and the application of visual grounding. 

\section{Detailed of Dataset Construction}
In this section, we will introduce some additional information during the  construction process.

\begin{figure*}[ht!]
    \centering
\begin{tcolorbox}[colframe=black]
\ttfamily
\small
You are a chest x-ray assistant and you are presented with a sentence involving a comparison in the original and current reports of a patient's two visits before and after, as well as some additional related explanatory information about the sentence. Please generate up to 3 single choice questions about the changes in the images from the two visits, based primarily on the sentence involving the comparison and the additional information, with the aid of referencing the two reports. Format the output into JSON format.
\\
\hrule
\vspace{5pt}
\textbf{Here are some rules:}
\vspace{5pt}\\
(1) Your questions are generated primarily around the KEY COMPARISON SENTENCE IN THE CURRENT REPORT and related explanatory information, with the content of the two reports as a secondary aid. \\
(2) Your question should be whether the chest CXR images have changed in some way in the same area, including abnormality, disease, location, severity, etiology, etc.\\
(3) For each question, you should clearly identify the area in which the question is asked and provide the correct answer, along with an explanation of why that answer was chosen.\\
(4) Your question is intended to answer the case where only images of the two diagnoses are known. Therefore, please do not include the words report comparison in your question; what should be included is image comparison.\\
(5) The JSON in related explanatory information consists of a "region" and a "list of explanatory information". For example, "right lung": ["comparison|yes|improved"] indicates that the right lung region has improved, and "left lung": ["comparison|yes|worsen"] indicates that the left lung region has worsened.\\
(6) Avoid asking questions that do not involve a change in two pictures. Ensuring all the questions are as diverse as possible.
\\
\hrule
\vspace{5pt}
\textbf{Here is one example:}\\
\{...\}\\
\hrule
\vspace{5pt}
\textbf{Here is some information:}\\
\{comparison sentence\},\ \{explanatory information\},\ \{current report\},\ \{prior report\}
\end{tcolorbox}
\caption{Text prompt for generating multiple image-based question-answer pairs in MMXU using GPT-4o}
    \label{fig:gpt4oprompt}
\end{figure*}

\subsection{Comparative Sentences Extraction}
\label{app:stage1}
Figure \ref{fig:detailed_comp_sentence} presents a detailed example of a comparative sentence extracted during the dataset construction process. The example includes the comparative sentence itself, the specific regions it pertains to, and the two corresponding images. Additionally, the bounding boxes of the relevant regions within the images, along with the comparative relationship, are illustrated. This detailed data structure is crucial for building a robust dataset that facilitates accurate comparisons during QA generation.


\subsection{QA Pairs Generation}
\label{app:stage3}
The version of GPT-4o we utilized is gpt-4o-2024-08-06. We employ the prompt template shown in Figure \ref{fig:gpt4oprompt}, where the "explanatory information" is derived from the details associated with the \textit{relationship} label. Figure \ref{fig:detailedgpt4oprompt} illustrates a detailed example of the input prompt and output content for the question-answer pair model using GPT-4o. We provided six specific rules and supplied GPT-4o with comparative sentences, \textit{relationship} labels, reports from two visits, and the region name of interest. Notably, the answers generated by GPT-4o are not limited to simple responses regarding improvements, deteriorations, or stability. Instead, we emphasize capturing more detailed changes in the regions.

\begin{figure*}[ht!]
\centering\small
\begin{tcolorbox}[colframe=black]
\ttfamily
Image-1: <image>\\
Image-2: <image>\\
Here are two chest X-RAY images of the same patient. Previous chest X-RAY is Image-1. And current chest X-RAY is Image-2.
Your task is to evaluate the differences between the two images based on the provided report and question.\\
\\
The report of Image-1 about is: \\
\%s\\
\\
Question:\\
\%s\\
\\
Options:\\
A: \%s\\
B: \%s\\
C: \%s\\
D: \%s\\
\\
Answer with the option's letter from the given choices directly.
\end{tcolorbox}
\caption{Text prompt template for evaluation on MMXU-\textit{test} benchmark with full report as historical records}
    \label{fig:templatewithfull}
\end{figure*}

\begin{figure*}[ht!]
\centering\small
\begin{tcolorbox}[colframe=black]
\ttfamily
Image-1: <image>\\
Image-2: <image>\\
Here are two chest X-RAY images of the same patient. Previous chest X-RAY is Image-1. And current chest X-RAY is Image-2.
Your task is to evaluate the differences between the two images based on the provided regional report and question.\\
\\
The report of Image-1 about region \%s is: \\
\%s\\
\\
Question:\\
\%s\\
\\
Options:\\
A: \%s\\
B: \%s\\
C: \%s\\
D: \%s\\
\\
Answer with the option's letter from the given choices directly.
\end{tcolorbox}
\caption{Text prompt template for evaluation on MMXU-\textit{test} benchmark with regional report as historical records}
    \label{fig:templatewithregional}
\end{figure*}



\section{Evaluation on MMXU-\textit{test} benchmark}

\noindent\textbf{Experimental Settings} We respectively utilized specialized prompt templates for three distinct scenarios: direct evaluation, enhancement through the integration of global and regional historical reports as historical records. For all generation processes, we set  {do\_sample} to  {False} and  {temperature} to  {0}. During the InternVL2 8B fine-tuning process, we set  {epochs} to  {1},  {max\_dynamic\_patch} to  {6},  {down\_sample\_ratio} to  {0.5},  {lora\_rank} to  {16}, and the  {learning\_rate} to  {4e-5}. The learning rate scheduler type is set to "cosine," with a warmup ratio of  {0.03}. The fine-tuning is carried out using the DeepSpeed Zero Stage 2~\cite{rajbhandari2020zero} distributed strategy on an 8-GPU NVIDIA RTX 3090 server. 

\begin{figure*}[ht]
\centering\small
\begin{tcolorbox}[colframe=black]
\ttfamily
Image-1: <image>\\
Image-2: <image>\\
Here are two chest X-RAY images of the same patient. Previous chest X-RAY is Image-1. And current chest X-RAY is Image-2.
Your task is to evaluate the differences between the two images based on the provided question.\\
\\
Question:\\
\%s \\
\\
Options:\\
A: \%s\\
B: \%s\\
C: \%s\\
D: \%s\\
\\
Answer with the option's letter from the given choices directly.
\end{tcolorbox}
\caption{Text prompt template for direct evaluation on MMXU-\textit{test} benchmark without historical records}
    \label{fig:templatewithout}
\end{figure*}


\noindent\textbf{Text Prompt} To assess the impact of historical  records on medical diagnosis, we performed evaluations on our MMXU-\textit{test} dataset under three distinct conditions: without historical records, with global reports as historical records, and with regional reports as historical records. Figure \ref{fig:templatewithout} illustrates the prompt template used in the absence of historical records, where only the basic instructions, questions, and options are provided. Figure \ref{fig:templatewithfull} depicts the prompt template incorporating global reports as historical records, in which we enhance the original template by including the global report. Figure \ref{fig:templatewithregional} showcases the prompt template with regional reports as historical records. In contrast to the global reports, these regional reports are tailored to the specific region relevant to the question, with the region's name explicitly indicated.



When generating region-level reports as historical record enhancements, we distinguish regions using the 29 anatomical names provided in the Chest ImaGenome dataset, including: right lung, right upper lung zone, right mid lung zone, right lower lung zone, right hilar structures, right apical zone, right costophrenic angle, right hemidiaphragm, left lung, left upper lung zone, left mid lung zone, left lower lung zone, left hilar structures, left apical zone, left costophrenic angle, left hemidiaphragm, trachea, spine, right clavicle, left clavicle, aortic arch, mediastinum, upper mediastinum, svc, cardiac silhouette, cavoatrial junction, right atrium, carina, and abdomen.


\begin{figure*}[ht!]
  \centering
  \includegraphics[width=2.05\columnwidth]{images/case-worsen.pdf}
  \caption{Two hard examples from ``Worsen'' category of MMXU answered by some LVLMs. }
  \label{fig:case-worsen}
\end{figure*}



\section{More Case Study}
\label{app:more-case-study}

% 图\ref{fig:case-worsen}展示了MMXU中两个困难样例 of “Worsen” category。在CASE 1中,所以的没有进行微调的模型均无法正确回答问题,只有使用了MAG方法的微调模型正确回答了问题。在CASE 2中,只有GPT-4o在提供historical records后可以正确回答问题。而对于其他模型,由于R-MRec阐述了之前没有变化的观点,所有均错误选择了A不变的情况,这说明这些模型没有正确理解变化,在单纯的重复。

Figure \ref{fig:case-worsen} shows two challenging examples from the "Worsen" category in MMXU. In CASE 1, none of the models without fine-tuning were able to correctly answer the question, and only the fine-tuned model using the MAG method provided the correct answer. This may be due to the difficulty of the question itself, as well as the challenge of extracting useful information to answer the question from the historical records.In CASE 2, only GPT-4o was able to answer the question correctly after providing historical records. For other models, since R-MRec reiterated previously unchanged views, all incorrectly chose option A (no change), indicating that these models failed to correctly understand the change and simply repeated the previous information.




\begin{figure*}[ht]
    \centering

\resizebox{0.98\textwidth}{!}{
\begin{tcolorbox}[colframe=black]
\ttfamily
\small
\textbf{USER:} You are a chest x-ray assistant and you are presented with a sentence involving a comparison in the original and current reports of a patient's two visits before and after, as well as some additional related explanatory information about the sentence. Please generate at most 3 single choice questions about the changes in the images from the two visits, based primarily on the sentence involving the comparison and the additional information, with the aid of referencing the two reports. Format the output into JSON format.
\\
Here are some rules:
\\
(1) Your questions are generated primarily around the KEY COMPARISON SENTENCE IN THE CURRENT REPORT and related explanatory information, with the content of the two reports as a secondary aid. \\
(2) Your question should be whether the chest CXR images have changed in some way in the same area, including abnormality, disease, location, severity, etiology, etc.\\
(3) For each question, you should clearly identify the area in which the question is asked and provide the correct answer, along with an explanation of why that answer was chosen.\\
(4) Your question is intended to answer the case where only images of the two diagnoses are known. Therefore, please do not include the words report comparison in your question; what should be included is image comparison.\\
(5) The JSON in related explanatory information consists of a "region" and a "list of explanatory information". For example, "right lung": ["comparison|yes|improved"] indicates that the right lung region has improved, and "left lung": ["comparison|yes|worsen"] indicates that the left lung region has worsened.\\
(6) Avoid asking questions that do not involve a change in two pictures. Ensuring all the questions are as diverse as possible.
\\
Here is one example:\\
\{ \\
\hspace*{5mm}"question": ..., \\
\hspace*{5mm}"region": ...,\\
\hspace*{5mm}"content\_type": ..., \# worsen, improved or no change\\
\hspace*{5mm}"options": \{ \\
\hspace*{10mm}"A": ..., \\
\hspace*{10mm}"B": ..., \\
\hspace*{10mm}"C": ..., \\
\hspace*{10mm}"D": ... \}\\
\hspace*{5mm}"answer": .., \\
\hspace*{5mm}"reason": ...\\
\}\\
Here is some information: \\
KEY COMPARISON SENTENCE IN THE CURRENT REPORT: "Widening mediastinum has improved." \\
Explanatory Information: "upper mediastinum": ["comparison|yes|improved"] \\
Current Report: Moderate to severe cardiomegaly is stable.  Widening mediastinum has improved. Bilateral effusions are small. Opacities in the lower lobes bilaterally larger on the left side are likely atelectasis, superimposed infection cannot be totally excluded. There is no evident pneumothorax. Sternal wires are aligned. Degenerative changes in the thoracic spine are mild.  There is mild vascular congestion. Bilateral healed rib fractures are again noted. Small bilateral effusions with adjacent opacities, these are likely atelectasis but superimposed infection cannot be totally excluded. Mild 
vascular congestion.\\
Prior Report: IN COMPARISON TO \_\_\_ RADIOGRAPH, CARDIOMEGALY IS ACCOMPANIED BY IMPROVING PULMONARY EDEMA.  A MORE CONFLUENT OPACITY IN THE LEFT RETROCARDIAC REGION HAS WORSENED, AND MAY REFLECT A COMBINATION OF PLEURAL effusion and adjacent atelectasis or consolidation.  No other relevant change.
\vspace{5pt}
\hrule
\vspace{5pt}
\textbf{GPT-4o:} \\
\{\\
\hspace*{5mm}"question": "What change has been observed in the upper mediastinum region of the chest CXR images?", \\
\hspace*{5mm}"region": "upper mediastinum",\\
\hspace*{5mm}"content\_type": "improved",\\ 
\hspace*{5mm}"options": \{ \\
\hspace*{10mm}"A": "The widening has worsened.", \\
\hspace*{10mm}"B": "There is new vasculature congestion.", \\
\hspace*{10mm}"C": "The widening has improved.", \\
\hspace*{10mm}"D": "There is no notable change."\\
\hspace*{5mm}\}, \\
\hspace*{5mm}"answer": "C",  \\
\hspace*{5mm}"reason": "The related explanatory information indicates that there is an improvement in the upper mediastinum, corresponding to the general improvement of the mediastinum's widening mentioned in the key comparison sentence."\\
\}

\end{tcolorbox}
}
\caption{A detailed example of the input and output content for the question-answer pair model using GPT-4o}
\label{fig:detailedgpt4oprompt}
\end{figure*}

\end{document}
