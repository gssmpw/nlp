\begin{table}[!t]
\centering
\scalebox{0.65}{
    \begin{tabular}{l cccc}
      \toprule
      & \multicolumn{4}{c}{\textbf{Recruiting Dataset}}\\
      & \multicolumn{2}{c}{Rank Resume} & \multicolumn{2}{c}{Rank Job} \\
      \cmidrule(lr){2-3} \cmidrule(lr){4-5} 
      \textbf{Method}
      &  Recall@100 & nDCG@100 & Recall@10 & nDCG@10 \\
      \midrule
      BM25(top-10)
      & 82.95 &44.62 &86.75 &64.36
      \\
      \midrule
      \RunnerUpMiningShort{}(0\%-1\%)
      & 80.80 &48.41 &86.00 &66.17 
      \\
      \RunnerUpMiningShort{}(1\%-2\%)
      & 85.13 &49.92 &85.13 &66.85
      \\
      \RunnerUpMiningShort{}(2\%-3\%)
      & 83.08 &49.09 &90.25 &70.21 
      \\
      \RunnerUpMiningShort{}(3\%-4\%)
      & 85.43 &\textbf{50.99} &\textbf{91.38} &\textbf{71.34} 
      \\
      \RunnerUpMiningShort{}(4\%-5\%)
      & \textbf{85.56}  &50.88 &87.25 &69.28
      \\
      \bottomrule

    \end{tabular}
  }
\caption{Comparing \RunnerUpMiningShort{} with BM25 methods, as well as \RunnerUpMiningShort{} under different percentiles ranges. We denote this as ``\emph{\RunnerUpMiningShort{}(L\%-H\%)}''.}
\label{tbl:RUM}
\end{table}