\section{Conclusion}  
In this paper, we address the critical and previously underexplored problem of adversarial robustness in two-stage Learning-to-Defer systems. We introduce two novel adversarial attack strategies—untargeted and targeted—that exploit inherent vulnerabilities in existing L2D frameworks. To mitigate these threats, we propose \name{}, a robust deferral algorithm that provides theoretical guarantees based on Bayes consistency and \((\mc{R}, \mc{G})\)-consistency. We evaluate our approach across classification, regression, and multi-task scenarios. Our experiments demonstrate the effectiveness of the proposed adversarial attacks in significantly degrading the performance of existing two-stage L2D baselines. In contrast, \name{} exhibits strong robustness against these attacks, consistently maintaining high performance.




% \section*{Impact Statement}
% This paper introduces methods to improve the adversarial robustness of two-stage Learning-to-Defer frameworks, which allocate decision-making tasks between AI systems and human experts. The work has the potential to advance the field of Machine Learning, particularly in high-stakes domains such as healthcare, finance, and safety-critical systems, where robustness and reliability are essential. By mitigating vulnerabilities to adversarial attacks, this research ensures more secure and trustworthy decision-making processes. 

% The societal implications of this work are largely positive, as it contributes to enhancing the reliability and fairness of AI systems. However, as with any advancement in adversarial robustness, there is a potential for misuse if adversarial strategies are exploited for harmful purposes. While this paper does not directly address these ethical concerns, we encourage further exploration of safeguards and responsible deployment practices in future research.

% No immediate or significant ethical risks have been identified in this work, and its societal impacts align with the well-established benefits of improving robustness in Machine Learning systems.