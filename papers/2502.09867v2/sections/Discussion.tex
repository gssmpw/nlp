\section{DISCUSSION}
Based on a formative study with experts, we prototyped a creativity-support tool to explore how dimensional scaffolding can bridge the gap between novice designers’ capabilities and the complex demands of product design using generative AI. At its core, this research proposes and evaluates \toolname{}, an interaction paradigm that surfaces key product design dimensions from user-curated images into a selectable palette to enhance text prompts for text-to-image generation models.

The formative study revealed insights into client-designer dynamics and novice-expert gaps in design exploration. Designers frequently rely on visual aids and multiple iterations to communicate ideas and align with clients’ preferences, emphasizing the importance of managing trade-offs and balancing creative goals with practical constraints. These findings underscored the need for an interface that emphasizes \textit{visual feedback} and \textit{structured exploration}, especially for users without domain-specific knowledge. These insights informed \toolname{} ’s "dimensional scaffolding" paradigm, which we hypothesized would allow novices to discover domain language and more fluently explore design spaces typically reserved for experts. In the study, dimensional scaffolding in \toolname{} enhanced textual descriptions and supported iterative design processes, enabling novice designers to produce semantically diverse, novel outcomes while fostering greater self-efficacy through perceived control and improvement. Below, we delve into the interaction paradigm of \toolname{}, leveraging relevant theories to explain the outcomes and implications.

\subsection{Why did \toolname{} yield more diverse image outputs?}
Participants using \toolname{} produced longer prompts with richer vocabularies, contributing to images of greater diversity. By surfacing key design dimensions and associated vocabulary, \toolname{} encouraged users to craft detailed textual descriptions that articulated nuanced design ideas. These longer and richer prompts provided generative models with the necessary context to produce varied and novel design outputs.

While the \toolname{} creations were more diverse, participants’ iterative prompting exhibited higher semantic similarity than the baseline participants. Those minor edits tended to be meaningful refinements, leveraging the dimensional scaffolding framework. This suggests that while participants refined their designs, they preserved a consistent conceptual design. At the same time, the higher Levenshtein edit distances across iterations indicate that participants made substantial textual edits to their prompts. These edits introduced nuanced semantic adjustments, which allowed participants to explore new design possibilities while staying grounded in their core design goals. This coherence aligns with cognitive theories of design iteration in which minor, deliberate modifications within a structured framework lead to richer, more meaningful creative outputs \cite{karimi2019relating, davis2024fashioning}. Such iterative exploration and diversity in outputs reflect the value of generating and sharing multiple prototypes \cite{dow2011prototyping, hartmann2008design, kim2022mixplorer}, which promotes deeper creative exploration and more well-rounded design solutions.


\subsection{To what extent did \toolname{} participants understand the design space?}

Participants using \toolname{} demonstrated a better ability to navigate the design space, as evidenced by their significantly higher ratings compared to the baseline group for exploring different design dimensions and their significantly lower reported difficulty in converting ideas into text-based prompts.

This enhanced exploration is attributed to the dual feedback loop facilitated by \toolname{} ’s interaction paradigm (see section~\ref{finding6.3:perception}). Dimensional scaffolding allowed participants to map textual inputs to visual outputs, offering immediate feedback on how specific design terms influenced generated images \cite{yen2024give}. Simultaneously, the inspect feature supported the reverse process by linking visual outputs to associated design tags, making implicit design elements explicit. This iterative interplay between exploration and interpretation enabled participants to refine their mental models of the design space and articulate their creative intentions more effectively. Such scaffolding is particularly valuable in design contexts where visual thinking dominates \cite{xu2024jamplate,heyman2024supermind, dhillon2024shaping}, but articulating visual concepts verbally can often be challenging \cite{zamfirescu2023johnny}.

\subsection{How did \toolname{} affect the co-creation process?}
Participants using \toolname{} reported significantly higher ratings for “My design improved throughout the iterative process,” reflecting their positive perception of the system’s ability to support iterative refinement. Expert evaluations further confirmed these results, with significantly higher ratings for the novelty and alignment with the client’s design brief in participants’ final designs than in the baseline group. These outcomes demonstrate \toolname{}’s effectiveness in fostering creative and contextually relevant designs.

This improvement can be partly attributed to the mechanisms discussed above, where participants were able to explore a broader range of design possibilities and simultaneously deepen their understanding of the design space. The system’s bi-directional feedback loop and dimensional scaffolding enabled participants to refine the prompt dynamically, connecting design dimensions with visual outputs \cite{green1989cognitive, davis2018scaffolding}. By engaging with these iterative processes, participants can refine their prompts effectively and articulate nuanced design ideas, balancing creativity with alignment with client goals.

However, despite these positive outcomes, participants did not express significantly higher satisfaction with the generated images. This discrepancy likely stems from heightened expectations created by the system’s structured and adaptable framework. By enabling iterative refinement and supporting creative exploration, \toolname{} may have amplified participants’ expectations for the specificity and quality of the generated outputs \cite{amershi2014power, zhou2024understanding}. While effective at capturing broad design ideas, current generative models often struggle to deliver the detailed and precise characteristics participants sought. This highlights a critical gap between user empowerment and the technical limitations of existing generative AI, suggesting a need for advancements that better align outputs with users’ elevated expectations \cite{lee2024holistic}. 

\subsection{Limitations}

While the study demonstrates the potential of dimensional scaffolding in enhancing novice designers’ creative workflows, several limitations should be acknowledged. 

Firstly, although participants using \toolname{} reported an increased sense of control with structured prompts, we also received complaints when the generated images did not meet their expectations, reflecting the widening execution and evaluation gulfs \cite{subramonyam2024bridging}. This underscores a limitation in current text-to-image models \cite{lee2024holistic}, which struggle to consistently render detailed prompts accurately. As these models evolve, they may better satisfy user needs by accurately interpreting nuanced design specifications. Nonetheless, novices' mastery of domain-specific design language remains crucial.

Secondly, the study was confined to the domain of product design, specifically focusing on chair designs. Future studies in different domains, such as fashion \cite{jeon2021fashionq}, UI design \cite{vaithilingam2024dynavis}, or architecture \cite{aseniero2024experiential, zhang2023generative}, will help strengthen insights on scaffolding novice interactions with AI models. 
% Additionally, our design task was relatively simple, and the participants were primarily novice designers. The results may differ when applied to more complex projects or experienced professionals who might interact with the system differently and have different needs regarding creative freedom versus structured guidance.

Lastly, while dimensional scaffolding provides a helpful structure for thinking about well-established aspects of a design space, it may also constrain creativity by limiting users to primary dimensions and tags. Finding the optimal balance between offering helpful guidance and allowing creative flexibility is challenging \cite{chi2014nature}, especially for users with varying levels of expertise \cite{cross2004expertise, ericsson2009development}. The system’s current design might not fully accommodate the needs of more experienced designers who may prefer less restrictive tools.

\subsection{Future Work}

Building on our findings, several future research and development directions emerge. Enhancing designers’ control over generated outputs and addressing system limitations are key to improving the creative process. Features like image- \cite{zhang2023adding} or sketch-conditioned generation \cite{chang2020workflow, sarukkai2024block}, in-painting \cite{lugmayr2022repaint}, and direct manipulation \cite{pan2023drag} could give users greater control by allowing them to retain and modify specific design geometries during exploration \cite{brade2023promptify}. Additionally, well-timed feedback \cite{yen2024give}, dynamic scaffolding \cite{xu2024jamplate, suh2024luminate}, analogical thinking \cite{lin2025inkspire, kang2024biospark}, and advanced agentic generative models \cite{park2023generative, shaikh2024rehearsal} could bridge the gap between user expectations and outputs. These enhancements would offer more accurate and consistent translations of detailed prompts into visual designs, making the creative process more intuitive, satisfying, and productive.

Second, investigating how to tailor the scaffolding to suit different levels of expertise could help find the optimal balance between providing guidance and allowing creative freedom, ensuring that both novices and experienced designers benefit from the approach. 
While predefined dimensions offer essential guidance for novices, the ideal level of customization for experienced users remains an open question. Future research could compare tools prioritizing strict predefined structures with those offering greater customization to explore how different scaffolding strategies impact user satisfaction, creativity, and outcomes. Studying tools with varying degrees of dimensional scaffolding across different expertise levels could reveal the optimal balance between structure and flexibility.

Furthermore, improving the visualization and management of design metadata is essential, especially when dealing with many generated designs. Exploring alternative visualization methods—such as hierarchical trees, interactive maps, or layered sheets—could help users better navigate and understand the design space \cite{jiang2023graphologue, wootton2024charting}. Enhanced visualization tools could enable designers to track their iterative changes more effectively, compare different design variations, and make more informed decisions throughout the creative process.

Finally, adapting the system for collaborative environments presents an intriguing opportunity. Future work could explore how dimensional scaffolding might support collaborative design workflows, enabling multiple designers or designer-client teams to co-create with AI assistance. Investigating collaborative setups could provide deeper insights into supporting diverse creative processes and enhancing team-based design projects \cite{bodker2000creativity}. Future work could explore the potential for LLMs to facilitate communication and idea sharing among collaborators \cite{he2024ai, heyman2024supermind}, to support convergence processes among various stakeholders \cite{rayan2024exploring}, or to enable consensus-building \cite{liu2018consensus}.

