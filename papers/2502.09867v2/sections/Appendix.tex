\appendix

\section{PROMPTS USED IN THE LLM PIPELINES}
\label{appA:prompts}
\subsection{Document Digestion}
\label{appA1:prompts_doc_digest}
\begin{itemize}
    \item \textbf{prompt}: "Get the three most important design dimensions from the requirement doc. For each dimension, generate 3-5 tags. Be concise."
\end{itemize}

\subsection{Prompt Generation \& Update}
\label{appA2:prompts_prompt_gen_and_update}
\begin{itemize}
    \item \textbf{system\_message}: "You are a design generalist that converts design tags and weights into descriptive prompts. Your task is to update the prompt according to the given old and new tags comparison and their corresponding weights, making sure to remove any references to tags that have been removed or neutralized (weight = 0). Preserve as much of the original prompt as possible, but reflect all tag changes accurately."
    \item \textbf{user\_message}: "Create a product rendering of a dining room chair that stands out prominently against a white background. Update the old prompt by comparing the old and new tags and weights pairs. Any tags with a weight of 0 should be removed from the prompt. Any tags with a weight of 1 should be included in the prompt.
    \newline
    New Tags: \$\{JSON.stringify(transformedNewTags)\}
    \newline
    Old Tags: \$\{JSON.stringify(transformedOldTags)\}
    \newline
    Old Prompt: "\$\{promptText\}"
    \newline
    Just return the prompt itself. Use complete sentences to describe the design."
    \item \textbf{model}: "gpt-4o"
    \item \textbf{messages}: [system\_message, user\_message]
\end{itemize}

\subsection{Image Generation}
\label{appA3:prompts_img_gen}
\begin{itemize}
    \item \textbf{model}: "dall-e-3"
    \item \textbf{quality}: hd
    \item \textbf{num}: 3
    \item \textbf{size}: "1024x1024"
    \item \textbf{quality}: "standard"
\end{itemize}

\subsection{Tag Extraction}
\label{appA4:prompts_tag_extraction}
\begin{itemize}
    \item \textbf{system\_message}: "You are a creative and helpful designer who assists in identifying and categorizing aesthetic dimensions of product designs. The response should be format like: \{newtags:[\{'name':'Dimension Name', 'tags':['tag1', 'tag2', 'tag3' ... \}]\}"
    \item \textbf{text\_prompt}: "What relevant aesthetic dimensions and design element tags are in this image? Reference the existing tags, think outside the box, and include all relevant dimensions. On top of the old tags, generate 1-5 new tags that either append to existing design dimensions or create new dimensions while avoiding creating similar dimensions to the old ones. Provide the output in a JSON format." + JSON.stringify(JSON\_prompt);
    \item user\_message: [
    \newline
    \{type: "text", text: text\_prompt\}, 
    \newline
    \{type: "image\_url", image\_url: \{url: url,detail: "low",\}
    \newline
    \}]
    \item \textbf{model}: "gpt-4o-mini"
    \item \textbf{response\_format}: \{ type: "json\_object" \}
    \item \textbf{messages}: [system\_message, user\_message]
\end{itemize}

\subsection{Tag Recommendation}
\label{appA5:prompts_tag_recommendation}

\begin{itemize}
    \item \textbf{system\_message}: "You are a helpful assistant that provides concise five distinct design recommendations based on existing design tags for dining room chair design."
    \item \textbf{user\_message}: "Based on the current design tags in the category '\$\{currentCategory.name\}', suggest five new distinct design options. Please provide a simple list of options separated by commas and nothing else. Don't add numbers or bullet points."
    \item \textbf{model}: "gpt-4o-mini"
    \item \textbf{messages}: [system\_message, user\_message]
\end{itemize}


\subsection{Dimension Recommendation}
\label{appA6:prompts_dimension_extraction}
\begin{itemize}
    \item \textbf{system\_message}: "You are a helpful assistant that provides concise design recommendations based on existing design dimensions for dining room chair design."
    \item \textbf{user\_message}: "Based on the current design dimensions: [\$\{categoriesList\}], suggest 5 new distinct dimensions. Please provide a simple list of dimensions separated by commas and nothing else. Don't add numbers or bullet points."
    \item \textbf{model}: "gpt-4o-mini"
    \item \textbf{messages}: [system\_message, user\_message]
\end{itemize}


\section{USER STUDY RELATED MATERIAL}
\subsection{Design Document}
\label{appB:dd}
\subsubsection{Design requirement}

(Email from our client David)

Dear Designer,

I hope this message finds you in great spirits. As an architect deeply passionate about modern and sustainable design, I'm reaching out with a project that's very close to my heart. I'm in the process of bringing a vision to life—a set of 10 dining room chairs that's not just a piece of furniture but a statement of my lifestyle and values.

\textbf{Here's What I Envision:}
I'm drawn to contemporary style and embrace minimalism with open arms. My home is a testament to my love for clean lines and functional aesthetics, and I seek to extend this philosophy to this new chair design. I imagine it in neutral tones, with customization options allowing for the occasional bold accent. The materials, of course, need to echo my commitment to sustainability—a combination of natural wood and high-quality, eco-friendly fabrics would be ideal.

\textbf{Design Specifications:}
\begin{itemize}
    \item \textbf{Dimensions}: A height of 36 inches, with a seat height of 18 inches, seems perfect. The width and depth should be about 18 and 16 inches, respectively, providing ample space without compromising the minimalist design.
    \item \textbf{Comfort and Ergonomics}: Given my love for hosting dinner parties, the chair must offer comfort for long conversations. A curved backrest for lumbar support and a cushioned seat are essential.
    \item \textbf{Materials}: Solid oak or a similar hardwood for the frame would complement my home's aesthetic, paired with durable and stain-resistant fabric sourced responsibly.
\end{itemize}

\textbf{Functionality Needs:}
The chair should be lightweight, making it easy to move around, yet sturdy enough to withstand the joyous chaos of my gatherings. I also need to ensure the feet are kind to my flooring—no scratches are welcome.

\textbf{Sustainability and Quality:}
I value furniture that reflects my dedication to sustainable living and modern design. It should be durable and sourced from eco-friendly suppliers, aligning with my values and lifestyle.

\textbf{Budget and Timeline:}
I've allocated \$1000 to \$1500 per unit for this project, aiming to balance impeccable quality and affordability. Ideally, the design would be finalized within 30 days, with a prototype ready for review 60 days later and production commencing 90 days after prototype approval.

\textbf{Additional Considerations:}
Ease of assembly and eco-friendly packaging that ensures safe transport without compromising our planet's health is crucial.

I'm thrilled at the prospect of working together to create a piece that serves its purpose and does so with style and conscience. Your talent in design and understanding of functional aesthetics make you the perfect partner for this venture.

Looking forward to your thoughts and ideas.

Best regards,
David Thompson

\subsubsection{Client’s Persona}
\begin{itemize}
    \item David Thompson
    \item Age: 35
    \item Occupation: Architect
    \item Location: San Francisco, California
    \item Marital Status: Married
    \item Interests:
    \begin{itemize}
        \item Passionate about modern and sustainable architecture.
        \item Enjoys reading about interior design and home renovation.
        \item Loves outdoor activities like hiking and cycling.
    \end{itemize}
    \item Lifestyle:
    \begin{itemize}
        \item Lives in a well-designed, modern home.
        \item Prefers a minimalist and functional aesthetic.
        \item Often hosts dinner parties and small gatherings.
        \item Has a few dogs of varying ages
    \end{itemize}
    \item Purchase Motivations:
    \begin{itemize}
        \item Seeking furniture that reflects his taste for modern design.
        \item Values sustainability and eco-friendly materials.
        \item Wants durable, high-quality furniture that can withstand regular use.
    \end{itemize}
    \item Functional Needs:
    \begin{itemize}
        \item Comfortable for long dinner conversations, easy to move, and matches with a diverse range of dining tables.
        \item Able to stand pets' daily activities
    \end{itemize}
    \item Aesthetic Preferences:
    \begin{itemize}
        \item Prefers neutral tones with occasional bold accents.
        \item Likes clean lines and uncluttered spaces.
        \item Appreciates furniture that makes a statement but remains timeless.
    \end{itemize}
\end{itemize}

\subsubsection{Sketches \& Mood boards}
Here we present the sketches and mood boards we showed to our participants in the design document (see \autoref{fig:design_document_combined}).

\begin{figure}[htbp]
    \centering
    % First row
    \begin{subfigure}[t]{0.45\linewidth}
        \centering
        \includegraphics[width=\linewidth]{figures/design_document/design_document_sketch.png}
        \caption{Design Document Sketches}
        \Description{An image with various sketches about chair designs}
        \label{fig:design_document_sketch}
    \end{subfigure}
    \hfill
    \begin{subfigure}[t]{0.45\linewidth}
        \centering
        \includegraphics[width=\linewidth]{figures/design_document/design_document_moodboard1.png}
        \caption{Mood Board 1}
        \Description{An image with cool design inspirations about chairs}
        \label{fig:design_document_moodboard1}
    \end{subfigure}
    
    % Second row
    \vspace{0.5cm} % Adjust space between rows if needed
    \begin{subfigure}[t]{0.45\linewidth}
        \centering
        \includegraphics[width=\linewidth]{figures/design_document/design_document_moodboard2.png}
        \caption{Mood Board 2}
        \Description{An image with cool design inspirations about chairs}
        \label{fig:design_document_moodboard2}
    \end{subfigure}
    \hfill
    \begin{subfigure}[t]{0.45\linewidth}
        \centering
        \includegraphics[width=\linewidth]{figures/design_document/design_document_moodboard3.png}
        \caption{Mood Board 3}
        \Description{An image with cool design inspirations about chairs}
        \label{fig:design_document_moodboard3}
    \end{subfigure}

    \caption{Design Document: Sketches and Mood Boards}
    \label{fig:design_document_combined}
\end{figure}

\section{Survey \& Interview Questions}
\label{appC:siq}
\subsection{Formative Study - Screening Survey}
\label{appC1:fs_ss}
\textbf{Design Interview Study}

Thank you for your interest in participating in our study! We are a group of researchers from the xxx at the xxx. We seek designers and creators with prior experience customizing products, solutions, or other client designs. Our goal is to understand authentic design workflows better.

Participants will receive \$xx for approximately 30 minutes of their time (\$yy/hour). The interview will be recorded for research purposes. If you are interested, please provide your background information and availability for the interview.

\subsubsection*{Participant Information}
\begin{enumerate}
    \item \textbf{Full Name:} \noindent\rule{3cm}{0.4pt}
    \item \textbf{Email Address:} \noindent\rule{3cm}{0.4pt}
\end{enumerate}

\subsubsection*{Design Experience}
\begin{enumerate}
    \item \textbf{What is your background in creating custom solutions in your domain?}  
    (e.g., Furniture, product design, graphics, architecture, etc.)  
    \begin{itemize}
        \item Please describe your experience, the types of products you design, and how long you have been in the field. 
    \end{itemize}
    \item \textbf{[Optional] Please share any links to your past projects or portfolio pieces that showcase your custom design work:} \noindent\rule{3cm}{0.4pt}
\end{enumerate}

\subsubsection*{Availability}
\begin{enumerate}
    \item \textbf{Preferred Interview Times} (check all that apply):
    \begin{itemize}
        \item \textbf{9 AM -- 12 PM:} Mon/ Tue/ Wed/ Thu/ Fri/ Sat/ Sun
        \item \textbf{12 -- 3 PM:} Mon/ Tue/ Wed/ Thu/ Fri/ Sat/ Sun
        \item \textbf{3 -- 6 PM:} Mon/ Tue/ Wed/ Thu/ Fri/ Sat/ Sun
        \item \textbf{Other}
    \end{itemize}

    \item \textbf{[Optional] Additional Comments or Questions} (e.g., time zone considerations)
\end{enumerate}

\subsection{Formative Study - Expert Interview Protocol}
\label{appC2:fs_eip}
\subsubsection*{Overview}
The following protocol outlines the structure and questions for conducting expert interviews as part of our research to better understand authentic design workflows.

\subsubsection*{Welcome Script}
\textbf{Month/Date @ Time - Interview with [NAME]}  

Hi [Interviewee Name],  

My name is xxx, and I am a researcher from the yyy Group at zzz Affiliation. Thank you so much for your interest and time in interviewing with us. We sincerely appreciate your help.  

To give you a brief overview, we are researching authentic design workflows to understand them better. We would love to ask you some questions to learn more about your background, work style, design process, and communication methods with clients.  

All of the information during this interview will be kept confidential, and with your consent, we might directly quote your insights. May we have your permission to record this interview for further analysis? The recording is intended to capture all the details you mention. Are you available to participate in this interview?  

\textbf{Permission to record interview:} [Yes/No]  
If yes: "Great! I will start the recording now."  

\subsubsection*{Interview Questions}
Can you tell me a little about your role and experience in the design field?

\textbf{Design Process}
\begin{enumerate}[label=\arabic*.]
    \setcounter{enumi}{1}
    \item Can you walk me through your typical design process, from ideation to completion?
        \begin{itemize}
            \item If possible, could you share details about a specific portfolio piece?
        \end{itemize}
    \item How do you handle client interactions and negotiations throughout the design process?
        \begin{itemize}
            \item E.g., trade-offs related to budget, aesthetics, and materials.
        \end{itemize}
\end{enumerate}

\textbf{Key Design Dimensions}
\begin{enumerate}[label=\arabic*.]
    \setcounter{enumi}{3}
    \item What are your key design considerations when designing a product?
        \begin{itemize}
            \item E.g., size, weight, depth, budget, materials, form, style.
        \end{itemize}
    \item Can you discuss a specific project where you faced challenges related to these design dimensions?
\end{enumerate}

\textbf{Negotiating with Clients}
\begin{enumerate}[label=\arabic*.]
    \setcounter{enumi}{5}
    \item How do you approach discussions with clients about design preferences?
        \begin{itemize}
            \item E.g., in-person, Zoom, Skype, Slack, Email.
        \end{itemize}
    \item Can you share an example of a trade-off situation and how you and the client resolved it?
        \begin{itemize}
            \item E.g., handling feedback and revisions.
        \end{itemize}
\end{enumerate}

\textbf{Client Constraints and Input Modalities}
\begin{enumerate}[label=\arabic*.]
    \setcounter{enumi}{7}
    \item What are some common constraints you encounter when working with clients?
    \item How do you manage and prioritize these constraints?
    \item How do clients typically provide input, and how do you incorporate this feedback into your designs?
        \begin{itemize}
            \item E.g., text descriptions, images, sketches.
        \end{itemize}
\end{enumerate}

\textbf{End}\\[0.5em]
Thank you so much for sharing your insights and experiences. Your input is incredibly valuable to our research. Could you also provide your preferred contact method for follow-up opportunities like user testing? Our research team will provide a [\$xx reward] for your participation as a token of appreciation. Do you have any questions or additional thoughts you would like to share before we conclude the interview?


\subsection{Summative Study - Screening Survey}
\label{appC3:ss_ss}
\subsubsection*{Participant Information}
\begin{enumerate}
    \item \textbf{Full Name:} \noindent\rule{3cm}{0.4pt}
    \item \textbf{Email Address:} \noindent\rule{3cm}{0.4pt}
    \item \textbf{Experiment Version:}
    \begin{itemize}
        \item In-person
        \item Remote
    \end{itemize}
\end{enumerate}

\subsubsection*{Demographic Information}
\begin{enumerate}
    \item \textbf{English Language Proficiency:}
    \begin{itemize}
        \item Native
        \item Advanced
        \item Intermediate
        \item Basic
        \item No proficiency
    \end{itemize}
    \item \textbf{Gender:}
    \begin{itemize}
        \item Female
        \item Male
        \item Non-binary/ Non-conforming
        \item Prefer not to respond
    \end{itemize}
    \item \textbf{Age:} \noindent\rule{3cm}{0.4pt}
\end{enumerate}

\subsubsection*{Design Experience}
\begin{enumerate}
    \item \textbf{Familiarity with Design:}
    \begin{itemize}
        \item Never
        \item Beginner (0–1 year)
        \item Intermediate (2–5 years)
        \item Advanced (>5 years)
    \end{itemize}
    \item \textbf{Design Fields (select all that apply):}
    \begin{itemize}
        \item Graphic Design
        \item Industrial Design
        \item UI/UX Design
        \item Architectural Design
        \item Fashion Design
        \item Other: \noindent\rule{3cm}{0.4pt}
    \end{itemize}
    \item \textbf{Describe Your Design Experience:} \noindent\rule{2cm}{0.4pt}
\end{enumerate}

\subsubsection*{Technical Experience}
\begin{enumerate}
    \item \textbf{Frequency of Using Large Language Models (e.g., ChatGPT, Gemini):}
    \begin{itemize}
        \item Never
        \item Tried a few times
        \item Monthly
        \item Weekly
        \item Daily
    \end{itemize}
    \item \textbf{Frequency of Using Image Generative Models (e.g., MidJourney, DALL-E):}
    \begin{itemize}
        \item Never
        \item Tried a few times
        \item Monthly
        \item Weekly
        \item Daily
    \end{itemize}
\end{enumerate}

\subsubsection*{Additional Information}
\begin{enumerate}
    \item \textbf{Do You Have Any Questions?} 
    \item \textbf{Participant ID:} Assigned ID (record this for future use).
\end{enumerate}

\subsection{Summative Study - Post-Experiment Survey}

\subsubsection*{Participant Details}
\begin{enumerate}
    \item \textbf{Participant ID:} \noindent\rule{3cm}{0.4pt}
    \item \textbf{Experiment Version:} \noindent\rule{3cm}{0.4pt}
    \begin{itemize}
        \item In-person
        \item Remote
    \end{itemize}
\end{enumerate}

\subsubsection*{Self-Evaluation}
\begin{enumerate}
    \item \textbf{What criteria did you use to select the final image for the client?}
    \item \textbf{Why does this image best fit your client's needs?} 
    \item \textbf{List the design dimensions and options you learned about the chair design space:} \noindent\rule{2cm}{0.4pt}
\end{enumerate}

\subsubsection*{Tool Feedback}
\begin{enumerate}[label=\arabic*.]
    \item \textbf{Image Satisfaction:}
    How satisfied were you with the generated images? (1=Very Unsatisfied, 7=Very Satisfied)  
    \emph{What factors influenced your rating?}
    
    \item \textbf{Image Alignment:}
    How well did the images match your expectations? (1=Very Misaligned, 7=Very Aligned)  
    \emph{Which elements aligned or deviated from your expectations?}
    
    \item \textbf{Ease of Converting Ideas to Prompts:}
    How easy was it to translate your ideas into prompts? (1=Very Difficult, 7=Very Easy)  
    \emph{What aspects made the process easier or more challenging?}
\end{enumerate}

\subsubsection*{Tool Usefulness}
\begin{enumerate}[label=\arabic*.]
    \item \textbf{Rate these statements} [1=Strongly Disagree, 7=Strongly Agree]
    \begin{itemize}
        \item The tool visualized my ideas.
        \item It helped explore design dimensions.
        \item It aided in initial prompt creation.
        \item It refined my design concept.
    \end{itemize}
    \item \textbf{Additional comments:} \noindent\rule{2cm}{0.4pt}
\end{enumerate}

\subsubsection*{Iterative Design Process}
\begin{enumerate}[label=\arabic*.]
    \item \textbf{Rate these statements} [1=Strongly Disagree, 7=Strongly Agree]
    \begin{itemize}
        \item My design improved over iterations.
        \item My prompts became more detailed.
    \end{itemize}
    \item \textbf{Additional comments:} \noindent\rule{2cm}{0.4pt}
\end{enumerate}

\subsubsection*{Future Directions}
\begin{enumerate}[label=\arabic*.]
    \item \textbf{Which aspects of the Design Assistant were most intuitive/useful and which were confusing/difficult?}
    \item \textbf{Suggestions for improvement:} \noindent\rule{2cm}{0.4pt}
\end{enumerate}


\section{Additional Analysis}
\begin{figure*}[htbp]
    \centering
    \includegraphics[width=\textwidth]{figures/findings/finding_image_side_by_side_heatmaps.png}
    \caption{Image semantic similarity heatmaps for Baseline (left),  DesignWeaver (middle), and their differences (right). Baseline participants' generated images are more semantically similar.}
    \label{fig:finding_image_side_by_side_heatmaps}
    \Description{This figure includes three heatmaps. The first (left) represents the image-to-image correlations for the Baseline, illustrating how image outputs correlate across different iterations. The second (middle) shows the same for DesignWeaver, highlighting how DesignWeaver affects the image output process. The third (right) heatmap shows the correlations between Baseline and DesignWeaver, where red indicates areas of increased correlation and blue indicates areas of decreased correlation in the DesignWeaver group.}
\end{figure*}

\begin{figure*}[htbp]
    \centering
    \includegraphics[width=\textwidth]{figures/findings/finding_prompt_side_by_side_heatmaps.png}
    \caption{Prompt similarity heatmaps for Baseline (left),  DesignWeaver (middle), and their differences (right). DesignWeaver participants' prompts are more semantically similar.}
    \label{fig:finding_prompt_side_by_side_heatmaps}
    \Description{This figure includes three heatmaps. The first (left) represents the prompt-to-prompt Similarity for Baseline, showing how similar the prompts are across different iterations. The second (middle) represents the prompt Similarity for DesignWeaver, highlighting how the DesignWeaver process influences the Similarity between prompts over time. The third (right) shows the difference in prompt Similarity between Baseline and DesignWeaver, where blue areas indicate a decrease in Similarity and red areas indicate an increase in Similarity within the DesignWeaver group compared to the Baseline.}
\end{figure*}

\begin{figure*}[htbp]
    \centering
    \includegraphics[width=\textwidth]{figures/appendix/design_dimensions_frequency.png}
    \caption{Frequency of design dimensions generated by DesignWeaver participants.}
    \label{fig:design_dimensions_frequency}
    \Description{
        The bar graph displays the count of each unique design dimension mentioned across participants in the DesignWeaver condition. The x-axis lists design dimensions (e.g., comfort, material, Durability), and the y-axis shows the count of occurrences for each dimension. Bars are labeled with the exact count of mentions. A dashed horizontal line at 27 represents the total number of participants in this condition, with an annotation reading "Total DesignWeaver Condition Users." The most frequent dimensions are comfort (22 mentions), material (12 mentions), and Durability (6 mentions), with others mentioned less frequently.
    }
    \label{fig:user_study_bar_chart}
\end{figure*}

\subsection{Prompt Comparison}
\label{appD:pc}

Here are randomly sampled example prompts from similar stages of the design process from \toolname{} and Baseline conditions (see \autoref{table:prompt_comparison}). 

\begin{table}[htbp]
\centering
\begin{tabular}{|p{0.32\linewidth}|p{0.66\linewidth}|} % Adjust column widths to fit the text width
\hline
\textbf{Baseline \newline (P18 10th Iteration)} & \textbf{\toolname{} \newline (P7 7th Iteration)} \\ \hline
A dining chair that is beige in color, made out of wood and leather, with a unique leg design, a wide sitting space, an ergonomic design, a curved and tall back, and a minimalistic aesthetic. It is described as comfortable. & 
The design features a neutral vibe, bold accents, and contemporary elements. It includes unique, eye-catching details and emphasizes eco-friendly and durable elements such as sustainable materials, renewable resources, and energy-efficient production. Ergonomics, lightweight construction, sturdy build, and scratch-resistant surfaces further enhance its functionality. Wood adds natural warmth, and the curved back with dynamic posture support improves comfort. Playful geometry adds to the modern aesthetic. \\ \hline
\end{tabular}
\caption{Prompts generated in the \toolname{} condition are more developed in fewer iterations.}
\label{table:prompt_comparison}
\end{table}

\subsection{Semantic Difference}
\label{appD:sd}
\autoref{fig:finding_image_side_by_side_heatmaps} and \autoref{fig:finding_prompt_side_by_side_heatmaps} shows more fine-grained across iteration comparison of the semantic difference using CLIP. The figures indicate that \toolname{} has more semantically diverse visual outputs but more semantically similar prompts. We believe the more semantically similar prompts in \toolname{} condition could retrieve more semantically diverse images in the latent embedding space in the T2I model we used because they are more detailed and nuanced. One interesting pattern in \autoref{fig:finding_prompt_side_by_side_heatmaps} is that prompts are becoming more semantically similar as \toolname{} participants accumulate more tags in one prompt as time progresses.



\subsection{Design Dimensions}
\label{appD:dd}

\autoref{tab:designweaver_generated_dimensions} shows all the design dimensions 27 \toolname{} participants generated, which are 38 in total. After merging some with essentially the same meaning, we obtained the following \autoref{fig:design_dimensions_frequency}, which shows the frequency of newly generated design dimensions (other than the three dimensions we pre-populated to help with the cold start problem). These dimensions are all reasonable and align with the design document requirements.
 

\begin{table}[htbp]
    \centering
    \begin{tabular}{|p{0.95\linewidth}|} 
    \hline
    artistry, color, color palette, comfort, construction, cost-effectiveness, craftsmanship, cultural influence, customizability, customization, depth, dimensions, Durability, environmental impact, ergonomic design, ergonomics, height, innovation, interaction, maintenance, maintenance ease, material, material durability, material innovation, material quality, materiality, materials, modern, setting, shape, space efficiency, space harmony, style, timelessness, user experience, versatility, visual appeal, width. \\
    \hline
    \end{tabular}
    \Description{This table lists 38 distinct design dimensions alphabetically, including factors such as artistry, ergonomics, and material qualities.}
    \caption{List of 38 DesignWeaver Participant Generated Design Dimensions in Alphabetical Order}
    \label{tab:designweaver_generated_dimensions}
\end{table}
