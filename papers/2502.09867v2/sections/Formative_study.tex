\section{FORMATIVE STUDY}
This study aims to understand how designers explore the design space to meet client goals, communicate with clients to clarify preferences, and negotiate constraints to converge on a solution for further development. We conducted semi-structured, one-on-one interviews with twelve experienced designers who specialized in diverse design fields and had professional experience ranging from over two years in furniture design to more than twenty years in speculative and architectural design (see \autoref{tab:formative_participants}). 

\textbf{\textit{Participants:}} To ensure broad representation, we recruited expert designers from academia, design studios, corporate settings, and independent shops with roles in consumer electronics, medical devices, startups, bespoke furniture, and outdoor environments. Recruitment involved targeted outreach on LinkedIn, Reddit, and Discord within design-focused communities. Interested individuals signed up via an online form detailing their background and experience in creating customized client designs. We prioritized diversity in domains (e.g., furniture, product design, architecture) and experience levels (minimum of two years in the field). Our sample size aligns with formative studies in prior HCI research \cite{jeon2021fashionq, kang2021metamap, lin2024jigsaw}, offering rich insights into design generation and client communication. Continuous data analysis revealed clear convergence in participant feedback, which is shared below.

\textbf{\textit{Method:}} Semi-structured interviews lasting 30 to 60 minutes were conducted over video calls. These interviews followed a consistent protocol while allowing flexibility for participants to share unique insights. The questions focused on their design processes, client interactions, key considerations, challenges, and strategies for managing trade-offs, constraints, and client feedback. For the full list of interview questions, please refer to Appendix~\ref{appC2:fs_eip}. A qualitative analysis of the session transcripts was performed using an open coding scheme, grouping quotes into recurring themes. This approach indicated data saturation and supported the validity and robustness of our findings.

\begin{table}[htbp]
\centering
\begin{tabular}{lll}
\toprule
\textbf{Experts} & \textbf{Years of Experience} & \textbf{Design Domain(s)} \\
\midrule
E1 & 10 & Architecture, Product \\
E2 & 21 & Speculative, Architecture \\
E3 & 3 & Furniture \\
E4 & 11 & Furniture \\
E5 & 16 & Product \\
E6 & 16 & Industrial \\
E7 & 11 & Industrial \\
E8 & 3 & Product \\
E9 & 4 & Product, Furniture \\
E10 & 3 & Furniture \\
E11 & 5 & Landscape Architecture \\
E12 & 6 & Product, Industrial \\
\bottomrule
\end{tabular}
\caption{Characteristics of formative study participants.}
\label{tab:formative_participants}
\end{table}

\subsection{Insight 1: Clients Often Struggle to Articulate Specific Preferences and Rely on Visuals to Express Ideas} 
Few clients can clearly articulate their vision with specific preferences, often turning to visual aids or physical materials to communicate their ideas more effectively. As \textit{[E3]} points out, \begin{quote}\textit{“Sometimes, on top of rough sketches and photos of inspiration, clients even provide fabric swatches, color samples, or other physical materials to guide the design process.”}\end{quote} These visuals help bridge the gap between vague ideas and concrete preferences. Designers, in turn, must carefully analyze these materials to \textit{“identify key themes, categorize preferences, and highlight main priorities” (E3).} 

Designers, likewise, share visual representations with clients to allow them to get into specific details but in a colloquial manner, as \textit{[E4]} explains, \textit{“When I send a sketch to the client, they might circle parts they don’t like and say, ‘Okay, I don’t like this.’”} However, even with the help of visual aids, \textit{[E9] points out that “it can still be difficult for clients to communicate their needs fully”}. Some clients go further by bringing external inspirations, as \textit{[E8] }highlights: \textit{“Some clients bring pictures from places like France or friends, and others share videos of the furniture they want.”} Visual tools are also crucial in preventing unwanted outcomes, as \textit{[E7]} emphasizes the importance of avoiding negative surprises, \textit{“with a product that surprised users, but not in a good way,”} where clients may react with \textit{“Oh, I wasn't expecting this, and this doesn’t feel right” (E7).}

\subsection{Insight 2: Designers Present Multiple Visual Alternatives to Support Tacit Communication}
This approach typically starts with \textit{[E3] “exploring various forms and ideas before narrowing down”} to the most feasible ones. This initial exploration often leads to the presentation of \textit{[E5] “three strong options, detailing why each one is specific and discussing one standout option.”} By presenting multiple alternatives, designers can encourage more diverse and authentic feedback from clients, as this approach helps surface key design principles through comparison and exploration, ultimately fostering better design outcomes and increased confidence in the design process \cite{tohidi2006getting, dow2010parallel}. As \textit{[E6]} notes, \textit{“we impress clients by involving them throughout the brainstorming session...we display ideas on the wall and present numerous concepts that we feel strongly about.”} This method allows designers to share creative ideas and timely reference materials while quickly ruling out what works and what doesn’t for better outcomes. \textit{[E7]} further reinforces this by engaging in this activity \textit{“before moving to a full-scale model.”} This method also helps designers and clients identify key constraints, such as budget and materials, and assess where compromises can be made. \begin{quote}\textit{[E3] “By offering different design options that vary in cost, materials, and aesthetics, this helps to identify what aspects are non-negotiable for the client and where they are willing to make compromises. For instance, if a material is too expensive, we present a more affordable alternative that aligns with their design goals.”}\end{quote}

Designers often face challenges when moving into the manufacturing phase, where decisions about form and shape can cause regret. \textit{[E6]} reflects, \textit{“In hindsight, we should have designed a different product to simplify the manufacturing process…ensuring the texture and color were correct was notably difficult.”} Collaboration with engineering and manufacturing teams is crucial, as high prototyping costs, often thousands of dollars, require efficient communication and well-refined choices before implementation. The designers we interviewed consistently showed three versions with detailed breakdowns to highlight the impact of individual changes on the overall project, aiming to exceed client expectations and minimize major revisions.

\subsection{Insight 3: Designers Convey Trade-offs Across Multiple Dimensions to Educate and Manage Expectations}
During the design process, clients often bring ambitious, idealistic visions that require adjustments to align with practical realities. As noted by \textit{[E11], “[clients] will also be more idealistic. Because sometimes I find that an idea and its practicality don't go hand in hand.”} For physical products, especially furniture, space constraints often require adjustments to form and materials to ensure suitability: \textit{[E10] “I may find the need to communicate that the size constraints they’ve mentioned are not aesthetically suitable or practical for their space.”} 

Social media can further complicate these discussions, as clients often want materials they’ve seen online without considering practical factors. As highlighted by \textit{[E11], “They want the exact same materials as seen on TikTok or Instagram, and do not consider other things like their local environment or the most suitable materials.”} Budget limitations often pose a significant challenge, as clients frequently underestimate the costs of their desired designs, leading to a misalignment between their expectations and the final deliverable. \textit{[E4]} shared, “Clients often demand specific colors without realizing their budget covers less than half the cost,” highlighting how limited material availability and custom sourcing can significantly increase costs. Designers guide clients through these decisions and explain why certain choices are necessary, often convincing them to trust their expertise. As \textit{[E7]} noted, \begin{quote}\textit{“There are always situations where clients have specific preferences about the look or colors, and I need to help them understand that my choices are based on professional expertise and are the best fit for the project.”}\end{quote} 

In many cases, clients prioritize appearance over durability, making it challenging for designers to convey the importance of longevity over aesthetics for a lasting, valuable purchase. Similarly, \textit{[E10]} shared \textit{“Sometimes, clients may ask for cheaper alternatives ‘cause their initial budget is insufficient for the desired look.”} Timelines and material availability create additional obstacles, as supported by \textit{[E10]: “When I find their request impractical for their space, and when I adjust the design, it sometimes exceeds their budget and deadlines, forcing me to redo the work.”} Despite these challenges, most designers find that clients are typically satisfied with the outcome as long as the design captures their original vision.