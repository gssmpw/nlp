
\section{Introduction}

Scientific publishing has grown tremendously in recent decades \cite{white2019publications}, but many researchers still lack access to crucial papers \cite{suber2012open}. This access gap exists even in the wealthiest academic libraries in the world and is much worse for Researchers in developing countries, small independent labs or for independent scholars and educators. In 2008, Harvard had 98,900 subscriptions, Yale 73,900 and the best-funded research library in India only 10,600 \cite{suber2012open}. This disparity, coupled with the unsustainable surge in academic journal subscription costs, has been a pivotal driving force behind the Open Access (OA) movement. Although OA has become more common, this \textit{scholarly communication crisis} still remains an issue today. In July 2018 up to 200 German institutions lost access to journals of Elsevier \cite{else2018dutch} for five years after unsuccessful negotiations. While the negotiations since succeeded this temporary cut from new publications had a measurable effect on publishing behavior \cite{NoDealFraser}. Furthermore, in 2019 major US universities canceled their subscriptions to certain journals stating unreasonable price surges as the reason \cite{gaind2019huge}. This barrier slows scientific progress and as such ``copyright laws work counter to prevailing scientific norms" \cite{stodden2008legal}. 

\textbf{Why this position paper?} Recent advances in language models (LLMs), like GPT-4 \cite{achiam2023gpt}, Llama \cite{touvron2023llama}, and recently Deepseek R1 \cite{deepseekai2025deepseekr1incentivizingreasoningcapability} allow us to democratize access to scholarly knowledge, such as answering questions based on existing scientific research. The core issue lies in scholarly texts containing both \textit{information} and \textit{artistic} elements — such as wording, style, and unique phrasing — of which the latter is protected by copyright. With LLMs, for the first time, we can extract knowledge at scale and free scientific content while respecting authors' rights to copyright.

\textbf{Contribution.} In this paper, we take the position that separating the factual information in scholarly works from the copyrighted creative expression is \textit{technically and legally feasible}. We advocate for a \textit{Project Alexandria}, 
to realize this vision by creating \textit{Knowledge Units} using LLMs, which systematically separate reusable \textit{information} from the \textit{artistic expressions} inherent in scholarly writings. \textit{Knowledge Units} are structured records capturing entities, relationships, and attributes extracted from scholarly texts in a database.
Our study tests the \textit{legal} and \textit{technical} feasibility of preserving scientific knowledge that researchers and students need to learn or apply. We posit it must satisfy two desiderata:

\textit{\textbf{1. Legal Defensibility.}} Under interpretations of German copyright law and the U.S. Fair Use doctrine, facts themselves are not subject to copyright protection, only their creative expression. In addition to these jurisdictions, concept of the idea-expression dichotomy is also adopted by many other jurisdictions, such as the UK and India \cite{jain2012principle, adhikari2021idea}. We provide a legal analysis showcasing that Knowledge Units could preserve only the factual substance — definitions, measurements, causal relationships, and methodological details. They are designed explicitly to exclude the original phrasing and stylistic elements, offering what we believe is a legally permissible way to share scientific knowledge openly.

\textit{\textbf{2. Information Preservation.}} We evaluate the fidelity of Knowledge Units through performing question-answering (QA) experiments using them, where language models answer multiple-choice questions based on abstracts and full-text articles across biology, physics, mathematics, and computer science. Models provided with Knowledge Units achieve high accuracy, nearly matching the performance of those given the original texts. This provides evidence that the vast majority of relevant information should be accurately captured in this format.

\textbf{Impact.} Freeing factual information from copyright restrictions can help more researchers, educators, and language models access and share facts from scholarly work.  As a result, researchers worldwide can discuss and build on one another’s findings without legal uncertainty or relying on expensive paywalls. We believe this is a key step toward a more open and inclusive global research community. We present a clear vision, a practical mechanism, and an open-source infrastructure aimed at fostering a more inclusive and collaborative global scientific ecosystem.

In the rest of this paper, we present how Knowledge Units work, discuss the legal principles that support this approach, show evidence of high factual retention, and consider alternate perspectives. We conclude by describing future plans to build massive open databases of factual knowledge and encourage the community to adopt this approach.