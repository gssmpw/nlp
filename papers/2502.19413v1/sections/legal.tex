\vspace{-0.2cm}
\section{Legal Considerations for Fair Knowledge Extraction}

We start our legal considerations under the broader legal framework of the Universal Declaration of Human Rights (UDHR), Article 27(a) which states ``[e]veryone has the right freely to participate in the cultural life of the community, to enjoy the arts and to share in scientific advancement and its benefits." However, it also specifies the need to support and respect the rights of authors in their “moral and material interests resulting from any scientific, literary or artistic production of which (s)he is the author” as also stated under the UDHR 27(b). These rights of authors must co-exist with everyone's rights to scientific knowledge, and indeed freeing such knowledge from copyright burdens will allow scholars to participate in the marketplace of ideas - to continue new research. From this starting point, we analyzed in detail the legal implications of extracting knowledge from copyrighted text under two legal frameworks: German and U.S. copyright law, presented below.

\subsection{German Copyright Law}

Under German law, \textbf{Urheberrecht} (copyright) serves to protect creative expressions, ensuring that authors maintain exclusive rights over their original works \cite{wandtke2010urheberrecht, geller2009german}. Therefore, only the aesthetic design of a text is protected, but not the content itself. An important exception exists for complex narratives that stem from an author's imagination, which may receive copyright protection. In contrast, mere facts and scientific discoveries remain unprotected.

For a work to qualify for copyright protection, it must satisfy the \emph{Schöpfungshöhe} (original creative threshold) \cite{wandtke2010urheberrecht, hoffmann2011copyrights}, and only human authors — excluding AI systems — can hold authorship rights \cite{gratz2021kunstliche, legner2019erzeugnisse}. The \emph{extraction of information} from copyrighted texts does not infringe upon the original rights holder’s exclusive privileges \citep{bmbf2020copyright}, provided three specific conditions are met: no protected text is copied, the extracted text is fact-centric, and data mining exemptions are adhered to.

\textbf{Accessing Original Text:} According to Sections 44b and 60d of the \emph{Urheberrechtsgesetz} (UrhG), or the German Copyright Act, there is explicit permission to temporarily store copyrighted works for the purpose of extracting insights such as patterns, trends, or correlations, particularly in the context of scientific research (\emph{Text and Data Mining} )~\citep{heidrich2023legal}. Both provisions are exceptions to the principle that works protected by copyright may only be reproduced with the permission of the author or rights holder. Organizations operating as non-profits in the scientific research sector that engage in large-scale text analysis are offered robust support under \S44b and \S60d UrhG, being authorized to perform such activities, provided they systematically delete the underlying works once the factual extraction process is complete.

\textbf{Protected Text is Not Extracted:} The process of abstracting or summarizing the main ideas from a work can be permissible under German copyright law, \emph{provided the summary does not replicate the original’s creative form} \citep{heidrich2023legal}. Knowledge Units aim to comply to this principle by avoiding the storage of even summary phrases. Instead, it focuses solely on maintaining relationships, domain-specific concepts, or numeric attributes. If Knowledge Units do not reveal the text’s distinctive arrangement or style, they cannot be deemed “unfreie Bearbeitung.” (unfree adaptation) 

\textbf{Publication and Use of the Extracted Knowledge Units:} Under \S15 UrhG, authors retain control over the reproduction and public communication of their works. However, when it comes to the publication of extracted Knowledge Units, if what is being disseminated is \emph{non-protected factual content} and the creative aspects of the original work are \emph{not} reproduced in this process, they would be exempt. We highlight that any new textual or data-based work that arises solely from \emph{factual extraction} is inherently authored by those who develop the new structure. Such works may be considered unprotected if they are exclusively machine-generated without any human creative input, aligning with the legal framework that restricts authorship rights to human creators~\cite{heidrich2023legal}. As a result, it is viable to publish these Knowledge Units without infringing upon the original author’s rights. 

Overall, the processes of accessing original texts, extracting Knowledge Units, and publishing these extracted units can be conducted in a manner that neither reproduces nor stores the original phrasing, sentence structures, or distinctive literary qualities of the source material. This allows that there is no \emph{unfreie Bearbeitung} (unfree adaptation) involved, thereby avoiding any breach of German copyright laws. By explicitly prompting, we seek to adhere to the stipulated conditions — avoiding the copying of protected text, focusing on fact-centric extraction, and following data mining exemptions — allowing the  extracted Knowledge Units to be published while complying with the legal protections afforded to original creative works under German law.

\subsection{The Idea-Expression Dichotomy and US Fair Use Doctrine in Copyright}

In the United States, the legal framework regarding copyright differs slightly from Germany but leads to similar conclusions as other jurisdictions. U.S.\ copyright law does not protect facts or ideas, only the \emph{expressions} of those facts, an axiom that courts have repeatedly affirmed. Notably, the \textbf{Fair Use} doctrine (17 U.S.C.\ \S107) also allows provides a flexible framework that accommodates new and transformative uses such as text and data mining (TDM)~\citep{USfairuseTDM2015, reichman2012copyright}. The basis for using the knowledge downstream is the idea-expression dichotomy \citep{yen1989first} codified in 17 U.S.C.\ \S102 (b):  
\vspace{-1.1em}
\begin{quote}
    In no case does copyright protection for an original work of authorship extend to any idea, procedure, process, system, method of operation, concept, principle, or discovery, regardless of the form in which it is described, explained, illustrated, or embodied in such work.
\end{quote}
\vspace{-0.5em}
Title 17, Section 107 of the U.S.\ Code also enumerates four factors to evaluate fair use. For TDM, the first factor, \textbf{Purpose/Character}, demands for output text (in our case, Knowledge Units) to be typically “highly transformative,” especially if the copying is for nonprofit research or distinct from the original text’s use. The second factor, \textbf{Nature of work}, although many TDM cases involve creative works, courts have often downplayed or treated this factor as neutral if the use remains transformative. The third factor, \textbf{Amount/Substantiality}, permits copying \emph{the entire text} to achieve meaningful analysis, with courts finding that the “all or nothing” nature of TDM demands full copying~\citep{USfairuseTDM2015}. Lastly, the fourth factor, \textbf{Effect on market}, is generally favorable for TDM because it does not serve as a substitute for reading or consuming the original work, thereby rarely damaging the market for the original.

When these factors are weighed collectively, courts usually find that TDM constitutes fair use—particularly in academic or research settings~\citep{USfairuseTDM2015, hathitrustcase}. Knowledge Units aim to operate in accordance with the best practice recommendations from “The Code of Best Practices in Fair Use for Academic and Research Libraries,” which explicitly endorses the creation of TDM databases, provided that full-text or near-verbatim distributions are not made publicly available~\citep{USfairuseTDM2015}. Our pipeline aims to adheres to these recommendations by ensuring that no substantial original expression is published; Knowledge Units contain only factual statements, short style descriptors, and minimal numeric references. Additionally, there is no end-user access to entire works, as the original copyrighted text is neither exposed nor distributed. Instead, it is either deleted after analysis or stored for ephemeral TDM tasks within the scope of allowable research usage. Furthermore, we focus on non-consumptive research, providing derived knowledge for advanced AI, searching, and research purposes, rather than for reading or substituting the original content.

\textit{Past Relevant Case Law} illustrates the application of these principles. In \emph{Authors Guild v. HathiTrust}, the court emphasized that scanning entire works to facilitate full-text search and enable computational analysis was \emph{“quintessentially transformative”} \cite{hathitrustcase}. Similarly, in \emph{Authors Guild v. Google}~\cite{campbell2016authors}, the massive digitization for Google Books was held to be fair use, partly because it “transformed the book text into data for the purpose of substantive research” \cite{googleBooksCase}.

Overall, our approach likely aligns with past cases and could gain broad acceptance under U.S.\ fair use precedents, as it fosters \textit{public-interest scholarship} with measures taken to avoid threatening authors’ legitimate markets or moral rights to their expression. In the U.S. context, such re-purposing contributes distinct value and fosters new lines of inquiry, such as large-scale pattern identification. Moreover, it does not replace the original text as reading material, minimizing risk of market harm.

