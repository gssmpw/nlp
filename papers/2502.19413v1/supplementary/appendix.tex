\onecolumn
\section{Appendix}
\subsection{Example Knowledge Unit}
\label{sec:example_ku}
The following shows the first of six KUs of "Sparsity-certifying Graph Decompositions" \cite{streinu2009sparsity}

\textbf{Title:} Sparsity-certifying Graph Decompositions\\
\textbf{Authors:} Ileana Streinu; Louis Theran\\
\textbf{Genre:} Academic Journal, Mathematics, Computer Science\\
\textbf{Style:} The writing style is formal and highly technical, employing specialized terminology from graph theory and matroid theory.  The tone is objective and expository, characteristic of a research paper presenting new algorithms and theoretical results. The text is dense with mathematical notation and\\
\subsubsection*{Knowledge Unit 1 of 6}
\textbf{Context.} The provided context is empty.  Therefore, this summary will describe the text excerpt. The paper introduces a novel algorithm, the (k, $\lambda$)-pebble game with colors, to characterize (k, $\lambda$)-sparse graphs.  This algorithm offers solutions to problems involving tree decompositions of graphs, a topic relevant to rigidity theory. The research builds upon and extends prior work in the field, providing a new proof of the Tutte-Nash-Williams arboricity characterization.

\lstset{
    basicstyle=\ttfamily\small,
    columns=fullflexible,
    keepspaces=true,
    frame=single,
    backgroundcolor=\color{gray!10},
    keywordstyle=\color{blue},
    showstringspaces=false,
    breaklines=true,        % Enable automatic line breaking
    postbreak=\mbox{\textcolor{red}{$\hookrightarrow$}\space},  % Mark continuation
}

\begin{lstlisting}[language=TeX]
'Ileana Streinu': {
  'relations': {
      'authored': 'Sparsity-certifying Graph Decompositions',
      'affiliated_with': 'Smith College',
      'email': 'streinu@cs.smith.edu'
  },
  'attributes': {
      'department': 'Computer Science'
  }
},
'Louis Theran': {
  'relations': {
      'authored': 'Sparsity-certifying Graph Decompositions',
      'affiliated_with': 'University of Massachusetts Amherst',
      'email': 'theran@cs.umass.edu'
  },
  'attributes': {
      'department': 'Computer Science'
  }
},
'Sparsity-certifying Graph Decompositions': {
  'relations': {
      'authors': ['Ileana Streinu', 'Louis Theran'],
      'introduces': '(k, $\lambda$)-pebble game with colors',
      'characterizes': '(k, $\lambda$)-sparse graphs',
      'provides_solutions_for': 'Tree decompositions of graphs',
      'extends_work_of': ['Lee and Streinu', 'Gabow', 'Gabow and Westermann', 'Hendrickson'],
      'proves': 'Tutte-Nash-Williams characterization of arboricity'
  },
  'attributes': {
      'type': 'Academic Journal, Mathematics, Computer Science',
      'topic': 'Graph decompositions',
      'focus': '(k, $\lambda$)-sparse graphs'
  }
},
'(k, $\lambda$)-pebble game with colors': {
  'relations': {
      'introduced_in': 'Sparsity-certifying Graph Decompositions',
      'generalizes': 'Previous results of Lee and Streinu'
  },
  'attributes': {
      'type': 'Algorithm'
  }
},
'(k, $\lambda$)-sparse graphs': {
  'relations': {
      'characterized_by': '(k, $\lambda$)-pebble game with colors',
      'definition': 'No subset of n vertices spans more than $k n - \lambda$ edges'
  },
  'attributes': {
      'range': '$k \geq \lambda \geq 2k-1$ (upper range), $0 \geq \lambda \geq k$ (lower range)'
  }
},
'(k, $\lambda$)-tight graphs': {
  'relations': {
      'are_a_type_of': '(k, $\lambda$)-sparse graphs'
  },
  'attributes': {
      'edge_count': '$k n - $\lambda$'
  }
},
'Tree decompositions of graphs': {
  'relations': {
      'addressed_by': 'Sparsity-certifying Graph Decompositions'
  },
  'attributes': {
      'relevance': 'Rigidity theory'
  }
},
'Tutte-Nash-Williams characterization of arboricity': {
  'relations': {
      'proven_by': 'Sparsity-certifying Graph Decompositions'
  },
  'attributes': {
      'type': 'Theorem'
  }
},
'Decomposition (certifying sparsity)': {
  'relations': {
      'based_on': '(k, $\lambda$)-pebble game with colors',
      'presented_in': 'Sparsity-certifying Graph Decompositions'
  },
  'attributes': {
      'property': 'Sparse graphs and graphs admitting the decomposition coincide'
  }
},
'Algorithms (efficient)': {
  'relations': {
      'presented_in': 'Sparsity-certifying Graph Decompositions',
      'apply_to': 'Upper range of $\lambda$'
  },
  'attributes': {
      'purpose': 'Finding decompositions that certify sparsity'
  }
},
'Previous work': {
  'relations': {
      'referenced_by': 'Sparsity-certifying Graph Decompositions',
      'authors': ['Gabow', 'Gabow and Westermann', 'Hendrickson']
  }
}
\end{lstlisting}
% <source_sentence_min_hash: [ 37210904  44237933   9024081  31149170  14019373  28097237 116077693
%   32682572  37381873  11261647  87741497  17470031  33479172   8053145
%   16595777   3978670] >
% </segment 1>


\newpage

\section{Example Multiple-Choice Questions}

\subsection{Abstract-Level Analysis}

\begin{table}[h!]
    \centering
    % \small
    \caption{Abstract \cite{virnau2006intricate} and generated multiple-choice questions using the Gemini Pro 1.5 002 model. This is a paper from the Medical domain. Correct answers indicated by "*".}
    \label{tab:abstract_medical_mcqs}
    \begin{tabular}{p{0.5\textwidth}|p{0.5\textwidth}}
    \hline
    \textbf{Abstract} & \textbf{Questions} \\
    \hline
    \multirow{3}{0.5\textwidth}{A number of recently discovered protein structures incorporate a rather unexpected structural feature: a knot in the polypeptide backbone. These knots are extremely rare, but their occurrence is likely connected to protein function in as yet unexplored fashion. Our analysis of the complete Protein Data Bank reveals several new knots which, along with previously discovered ones, can shed light on such connections. In particular, we identify the most complex knot discovered to date in human ubiquitin hydrolase, and suggest that its entangled topology protects it against unfolding and degradation by the proteasome. Knots in proteins are typically preserved across species and sometimes even across kingdoms. However, we also identify a knot which only appears in some transcarbamylases while being absent in homologous proteins of similar structure. The emergence of the knot is accompanied by a shift in the enzymatic function of the protein. We suggest that the simple insertion of a short DNA fragment into the gene may suffice to turn an unknotted into a knotted structure in this protein.
} & \begin{minipage}[t]{\linewidth}
        \textit{Which of the following best describes the relationship between protein knots and their function?}\\
        A) Knots invariably enhance protein stability and resistance to degradation.\\
        B) The presence of knots always indicates a shared evolutionary ancestry between proteins.\\
        C*) Knots can influence protein function, but the exact mechanisms are still being investigated.\\
        D) Knot formation is a random event with negligible impact on protein activity.
        
    \end{minipage} \\
    % \cline{2}
    & \vspace{0.2cm}\begin{minipage}[t]{\linewidth}
        \textit{The study of protein knots in ubiquitin hydrolase suggests that:}\\ A) The knot's complexity hinders its interaction with the proteasome.\\
        B*) The knot protects the protein from denaturation and breakdown by the proteasome.\\
        C) The knot facilitates the protein's interaction with other ubiquitin-related proteins.\\
        D) The knot is a result of post-translational modifications.
    \end{minipage} \\
    & \vspace{0.2cm}\begin{minipage}[t]{\linewidth}
        \textit{The example of transcarbamylases illustrates that:} \\
        A) Knots in proteins are always conserved across species and kingdoms. \\
        B*) The emergence of a knot can be linked to a change in the protein's enzymatic activity. \\
        C) Knot formation requires significant alterations to the protein's primary structure. \\
        D) Unknotted proteins are inherently less efficient than their knotted counterparts.
    \end{minipage} \\
    \hline
    \end{tabular}
\end{table}

\newpage

\begin{table}[h!]
    \centering
    % \small
    \caption{Abstract \cite{DBLP:journals/gc/StreinuT09} and generated multiple-choice questions using the Gemini Pro 1.5 002 model. This is a paper from the Mathematics domain. Correct answers indicated by "*".}
    \label{tab:abstract_math_mcqs}
    \begin{tabular}{p{0.5\textwidth}|p{0.5\textwidth}}
    \hline
    \textbf{Abstract} & \textbf{Questions} \\
    \midrule
    \multirow{3}{0.5\textwidth}{We describe a new algorithm, the $(k,\ell)$-pebble game with colors, and use it to obtain a characterization of the family of $(k,\ell)$-sparse graphs and algorithmic solutions to a family of problems concerning tree decompositions of graphs. Special instances of sparse graphs appear in rigidity theory and have received increased attention in recent years. In particular, our colored pebbles generalize and strengthen the previous results of Lee and Streinu and give a new proof of the Tutte-Nash-Williams characterization of arboricity. We also present a new decomposition that certifies sparsity based on the $(k,\ell)$-pebble game with colors. Our work also exposes connections between pebble game algorithms and previous sparse graph algorithms by Gabow, Gabow and Westermann and Hendrickson.
} & \begin{minipage}[t]{\linewidth}
        \textit{Which of the following best describes the relationship between the $(k,\ell)$-pebble game with colors and the Tutte-Nash-Williams characterization of arboricity, according to the text?}\\
        A) The pebble game provides a counterexample to the Tutte-Nash-Williams characterization.\\
        B*) The pebble game offers new proof and strengthens previous results related to the Tutte-Nash-Williams characterization. \\
        C) The Tutte-Nash-Williams characterization is a specific instance of the $(k,\ell)$-pebble game with colors. \\
        D) The pebble game and the Tutte-Nash-Williams characterization address unrelated graph properties.
    \end{minipage} \\
    % \cline{2}
    
    & \vspace{0.2cm}\begin{minipage}[t]{\linewidth}
        \textit{The described algorithm connects pebble game algorithms with prior sparse graph algorithms by which of the following researchers?}\\ 
        A) Dijkstra and Kruskal \\
        B) Prim and Tarjan \\
        C*) Gabow, Gabow and Westermann, and Hendrickson \\
        D) Ford and Fulkerson
    \end{minipage} \\
    & \vspace{0.2cm}\begin{minipage}[t]{\linewidth}
        \textit{The "new decomposition" mentioned in the text certifies sparsity based on which of the following?} \\
        A) The chromatic number of the graph \\
        B*) The $(k,\ell)$-pebble game with colors \\
        C) The maximum flow through the graph \\
        D) The minimum spanning tree of the graph
    \end{minipage} \\
    \midrule
    \end{tabular}
\end{table}

\newpage

\begin{table}[h!]
    \centering
    % \small
    \caption{Abstract \cite{DBLP:journals/corr/abs-0704-0671} and generated multiple-choice questions using the Gemini Pro 1.5 002 model. This is a paper from the Computer Science domain. Correct answers indicated by "*".}
    \label{tab:abstract_cs_mcqs}
    \begin{tabular}{p{0.5\textwidth}|p{0.5\textwidth}}
    \hline
    \textbf{Abstract} & \textbf{Questions} \\
    \midrule
    \multirow{3}{0.5\textwidth}{The problem of statistical learning is to construct a predictor of a random variable $Y$ as a function of a related random variable $X$ on the basis of an i.i.d. training sample from the joint distribution of $(X,Y)$. Allowable predictors are drawn from some specified class, and the goal is to approach asymptotically the performance (expected loss) of the best predictor in the class. We consider the setting in which one has perfect observation of the $X$-part of the sample, while the $Y$-part has to be communicated at some finite bit rate. The encoding of the $Y$-values is allowed to depend on the $X$-values. Under suitable regularity conditions on the admissible predictors, the underlying family of probability distributions and the loss function, we give an information-theoretic characterization of achievable predictor performance in terms of conditional distortion-rate functions. The ideas are illustrated on the example of nonparametric regression in Gaussian noise.


    
} & \begin{minipage}[t]{\linewidth}
        \textit{What is the primary challenge addressed in the described statistical learning problem when the Y-part of the sample is communicated at a finite bit rate?}\\
        A) Reconstructing the joint distribution of (X,Y) with minimal error.\\
        B) Minimizing the computational complexity of encoding the Y-values.\\
        C*) Balancing predictor performance against the constraints imposed by the limited bit rate for Y.\\
        D) Determining the optimal bit rate allocation between X and Y for achieving a desired prediction accuracy.
    \end{minipage} \\
    % \cline{2}
    
    & \vspace{0.2cm}\begin{minipage}[t]{\linewidth}
        \textit{Under what circumstances does the information-theoretic characterization of achievable predictor performance hold, in terms of conditional distortion-rate functions?}\\ A) When the loss function is convex and the admissible predictors are drawn from a parametric class. \\
        B) When the training sample is drawn from a non-i.i.d. distribution and the predictors are nonparametric. \\
        C*) When suitable regularity conditions are met on admissible predictors, the underlying probability distributions, and the loss function. \\
        D) When the X-part of the sample is partially observed and the Y-part is communicated at an infinite bit rate.
    \end{minipage} \\
    & \vspace{0.2cm}\begin{minipage}[t]{\linewidth}
        \textit{How is the concept of conditional distortion-rate functions related to predictor performance in the given scenario?} \\
        A) It quantifies the trade-off between the complexity of the predictor class and the achievable prediction accuracy. \\
        B) It establishes a lower bound on the expected loss of any predictor given the finite bit rate constraint on Y. \\
        C*) It characterizes the achievable predictor performance by quantifying the trade-off between the distortion in representing Y and the bit rate used. \\
        D) It provides a method for selecting the optimal predictor from the admissible class based on the observed X-values.
    \end{minipage} \\
    \midrule
    \end{tabular}
\end{table}

\newpage

\begin{table}[h!]
    \centering
    % \small
    \caption{Abstract \cite{pan2007evolution} and generated multiple-choice questions using the Gemini Pro 1.5 002 model. This is a paper from the Physics domain. Correct answers indicated by "*".}
    \label{tab:abstract_phy_mcqs}
    \begin{tabular}{p{0.5\textwidth}|p{0.5\textwidth}}
    \hline
    \textbf{Abstract} & \textbf{Questions} \\
    \midrule
    \multirow{3}{0.5\textwidth}{
  The evolution of the Earth-Moon system is described by the dark matter field fluid model proposed in the Meeting of Division of Particle and Field 2004, American Physical Society. The current behavior of the Earth-Moon system agrees with this model very well and the general pattern of the evolution of the Moon-Earth system described by this model agrees with geological and fossil evidence. The closest distance of the Moon to Earth was about 259000 km at 4.5 billion years ago, which is far beyond the Roche's limit. The result suggests that the tidal friction may not be the primary cause for the evolution of the Earth-Moon system. The average dark matter field fluid constant derived from Earth-Moon system data is $4.39 x 10^(-22) s^(-1)m^(-1)$. This model predicts that the Mars's rotation is also slowing with the angular acceleration rate about $-4.38 x 10^(-22) rad s^(-2)$.
    
} & \begin{minipage}[t]{\linewidth}
        \textit{What is the primary implication of the dark matter field fluid model's agreement with the current Earth-Moon system behavior and geological evidence?}\\
        A) Tidal forces are the primary driver of the Earth-Moon system's evolution.\\
        B) The Moon originated from a collision between Earth and a Mars-sized object.\\
        C) The Moon's closest approach to Earth was within the Roche limit.\\
        D*) The tidal friction may not be the primary influence on the Earth-Moon system's evolution.
    \end{minipage} \\
    % \cline{2}
    
    & \vspace{0.2cm}\begin{minipage}[t]{\linewidth}
        \textit{According to the dark matter field fluid model, what was the approximate distance between the Earth and the Moon 4.5 billion years ago?}\\ 
        A) 125,000 km\\
        B*) 259,000 km\\
        C) 384,400 km\\
        D) 450,000 km
    \end{minipage} \\
    & \vspace{0.2cm}\begin{minipage}[t]{\linewidth}
        \textit{The passage mentions a dark matter field fluid constant derived from Earth-Moon system data.  Which of the following best describes the significance of this constant in relation to Mars?} \\
        A) It predicts the rate of decrease in Mars's orbital velocity.\\
        B) It calculates the rate at which Mars's magnetic field is decaying.\\
        C*) It predicts the angular acceleration rate of Mars's rotation.\\
        D) It estimates the rate of expansion of Mars's crust due to internal heating.
    \end{minipage} \\
    \midrule
    \end{tabular}
\end{table}

\newpage

\subsection{Full-Paper Analysis}



\begin{table}[h!]
    \centering
    % \small
    \caption{Eight of ten generated multiple-choice questions by the Gemini Pro 1.5 002 model from \cite{chabot2014interrelationships}. This is a paper from the Medical domain. Correct answers indicated by "*".}
    \label{tab:fullpaper_medical_mcqs}
    \begin{tabular}{p{0.5\textwidth}|p{0.5\textwidth}}
    \hline
    \multicolumn{2}{c}{\textbf{Questions}} \\
    \midrule
    \begin{minipage}[t]{\linewidth}
    \textit{What is the primary mechanism by which ghrelin stimulates growth hormone (GH) release?}\\
    A) Direct activation of the growth hormone releasing hormone receptor (GHRHr)\\
    B) Stimulation of somatostatin release from the hypothalamus\\
    C*) Activation of the ghrelin receptor (GHSR), specifically the GHSR1a subtype\\
    D) Modulation of opioid peptide activity in the pituitary
    \end{minipage} & \begin{minipage}[t]{\linewidth}
    \textit{Which statement best describes the relationship between ghrelin and insulin sensitivity in healthy individuals after acute administration?}\\
    A) Ghrelin enhances insulin sensitivity, leading to increased glucose uptake.\\
    B) Ghrelin has no significant effect on insulin sensitivity.\\
    C*) Ghrelin impairs insulin sensitivity, potentially through direct effects on the liver.\\
    D) Ghrelin's effect on insulin sensitivity is entirely dependent on GH secretion.
    \end{minipage} \\
    \midrule
    \begin{minipage}[t]{\linewidth}
    \textit{What is the primary site of ghrelin production in the body?}\\
    A) Hypothalamus\\
    B) Pituitary gland\\
    C*) Stomach\\
    D) Small intestine
    \end{minipage} & \begin{minipage}[t]{\linewidth}
    \textit{How does unacylated ghrelin (UAG) affect the metabolic actions of acylated ghrelin (AG)?}\\
    A) UAG amplifies the hyperglycemic and hyperinsulinemic effects of AG.\\
    B) UAG has no impact on the metabolic effects of AG.\\
    C*) UAG counteracts the hyperglycemic and hyperinsulinemic effects of AG.\\
    D) UAG mimics the effects of AG on GH secretion but not on glucose metabolism.
    \end{minipage} \\
    \midrule
    \begin{minipage}[t]{\linewidth}
    \textit{What is the role of GOAT in ghrelin processing?}\\
    A) Cleavage of the ghrelin precursor into its active form\\
    B*) Octanoylation of the serine-3 residue of ghrelin\\
    C) Deacylation of acylated ghrelin to form UAG\\
    D) Binding and transport of ghrelin in the bloodstream
    \end{minipage} & \begin{minipage}[t]{\linewidth}
    \textit{What is the effect of long-term ghrelin treatment on plasma glucose and insulin levels?}\\
    A) Consistently decreases both glucose and insulin levels\\
    B*) Increases glucose levels, while insulin levels remain unchanged or increase\\
    C) Decreases glucose levels, while insulin levels increase\\
    D) Has no consistent effect on either glucose or insulin levels
    
    \end{minipage} \\
    \midrule
    \begin{minipage}[t]{\linewidth}
    \textit{What role does the vagus nerve play in ghrelin's effects on insulin secretion?}\\
    A) Ghrelin stimulates the vagus nerve to enhance insulin secretion.\\
    B*) Ghrelin inhibits the vagus nerve to suppress insulin secretion, particularly through the hepatic branch.\\
    C) Ghrelin's effects on insulin secretion are independent of vagal activity.\\
    D) Ghrelin acts synergistically with vagal stimulation to increase insulin secretion.
    \end{minipage} & \begin{minipage}[t]{\linewidth}

    \textit{Which of the following best describes the effect of ghrelin on glucose-stimulated insulin secretion (GSIS) in isolated pancreatic islets and cell lines?}
    A) Ghrelin consistently enhances GSIS.\\
    B) Ghrelin consistently inhibits GSIS.\\
    C) Ghrelin has no effect on GSIS.\\
    D*) Ghrelin's effect on GSIS is complex and may depend on factors like glucose concentration and ghrelin dose.
    \end{minipage} \\
    \midrule
    \end{tabular}
\end{table}



\begin{table}[h!]
    \centering
    % \small
    \caption{Eight of ten generated multiple-choice questions by the Gemini Pro 1.5 002 model from \cite{cipolloni2023entanglement}. This is a paper from the Physics domain. Correct answers indicated by "*".}
    \label{tab:fullpaper_phy_mcqs}
    \begin{tabular}{p{0.5\textwidth}|p{0.5\textwidth}}
    \hline
    \multicolumn{2}{c}{\textbf{Questions}} \\
    \midrule
    \begin{minipage}[t]{\linewidth}
    \textit{What is the primary focus of the paper discussed in the text?}\\
    A) Developing a new definition of entanglement entropy in gauge theories.\\
    B) Exploring the entanglement structure of strongly coupled Yang-Mills theories.\\
    C*) Utilizing recent technical advancements to understand ground state entanglement in weakly coupled Yang-Mills theories.\\
    D) Comparing different approaches to calculating entanglement entropy in gauge theories and establishing their equivalence.
    \end{minipage} & \begin{minipage}[t]{\linewidth}
    \textit{What is the main difficulty in defining entanglement entropy in gauge theories?}\\
    A) The non-Abelian nature of the gauge group makes it challenging to define subsystems.\\
    B) Gauge invariance introduces nonlocality at the UV scale, making subsystem definition difficult.\\
    C) The presence of both electric and magnetic terms in the Hamiltonian complicates the calculation.\\
    D*) The lack of a clear separation between physical and unphysical degrees of freedom makes it hard to define a reduced density operator.
    \end{minipage} \\
    \midrule
    \begin{minipage}[t]{\linewidth}
    \textit{Which approach does the paper primarily follow to define entanglement entropy?}\\
    A) Embedding the physical Hilbert space into a larger direct product space.\\
    B*) Using the replica trick and Euclidean path integral methods.\\
    C) Employing the Ryu-Takayanagi prescription in the holographic dual.\\
    D) Constructing a gauge-invariant density operator within a subalgebra of observables.
    \end{minipage} & \begin{minipage}[t]{\linewidth}
    \textit{How does the paper address the issue of the physical Hilbert space not admitting a direct product decomposition?}\\
    A) It introduces a new type of gauge-invariant operator that allows for a direct product decomposition.\\
    B) It utilizes a gauge-fixing procedure that eliminates the nonlocal effects of gauge invariance.\\
    C*) It works with an extended basis of gauge-variant states and accounts for the entropy contribution from splitting flux lines.\\
    D) It restricts the algebra of observables to a subalgebra that does admit a direct product decomposition.
    \end{minipage} \\
    \midrule
    \begin{minipage}[t]{\linewidth}
    \textit{What is the significance of the ubiquitous term identified in the entanglement entropy of Yang-Mills theories?}\\
    A) It represents the contribution of edge modes to the entanglement entropy.\\
    B*) It is a universal term that dominates the entanglement entropy in 3+1 dimensions.\\
    C) It arises from the presence of topological defects in the gauge theory.\\
    D) It is a non-universal term that depends on the specific lattice regularization.
    \end{minipage} & \begin{minipage}[t]{\linewidth}
    \textit{How is the Yang-Mills theory related to the principal chiral model in the paper's calculation?}\\
    A) The Yang-Mills theory is dual to the principal chiral model.\\
    B*) The Yang-Mills theory can be expressed as a principal chiral model after gauge fixing to axial gauge.\\
    C) The principal chiral model is used as a toy model to understand the qualitative features of the Yang-Mills theory.\\
    D) The principal chiral model provides a non-perturbative definition of the Yang-Mills theory.
    
    \end{minipage} \\
    \midrule
    \begin{minipage}[t]{\linewidth}
    \textit{What is the role of Nambu-Goldstone bosons in the entanglement entropy calculation?}\\
    A) They represent the gauge degrees of freedom that are fixed in the axial gauge.\\
    B*) Their enhanced entanglement of the softest mode contributes to the logarithmic term in the entropy.\\
    C) They mediate the interactions between the electric and magnetic degrees of freedom.\\
    D) Their zero mode fluctuations determine the Shannon entropy contribution to the entanglement entropy.
    \end{minipage} & \begin{minipage}[t]{\linewidth}

    \textit{What is the connection between the logarithmic term in the entanglement entropy and topological entanglement entropy?}
    A*) The logarithmic term is a generalization of topological entanglement entropy to continuous gauge groups.\\
    B) The logarithmic term is a correction to topological entanglement entropy at weak coupling.\\
    C) The logarithmic term is equivalent to topological entanglement entropy in the planar limit.\\
    D) The logarithmic term is unrelated to topological entanglement entropy.
    \end{minipage} \\
    \midrule
    \end{tabular}
\end{table}


\section{Similarity Overlaps}
\label{sec:appsimilarity}

In Figure \ref{fig:simtop}, we highlight the top similarity overlaps between Original Texts and Knowledge Units, while Figure \ref{fig:simtoprecon} focuses on overlaps between Original Texts and Reconstructed Texts. In both cases, the shared segments predominantly consist of scientific jargon or references to particular issues, illustrating the specialized nature of the content.

\begin{figure}[h]
    \centering
    \fbox{\includegraphics[width=\linewidth]{supplementary/2.png}}
    \vspace{1em} 
    \fbox{\includegraphics[width=\linewidth]{supplementary/3.png}}
    \vspace{1em}
    \fbox{\includegraphics[width=\linewidth]{supplementary/4.png}}
    \vspace{1em}
    \caption{\textbf{Similarity Overlaps:} Overlap between Top-3 Most Similar Original Texts and Knowledge Units using an online plagiarism checker \href{https://app.copyleaks.com/text-compare}{tool}.}
    \label{fig:simtop}
\end{figure}


\begin{figure}[h]
    \centering
    \fbox{\includegraphics[width=\linewidth]{supplementary/r1.png}}
    \vspace{1em} 
    \fbox{\includegraphics[width=\linewidth]{supplementary/r2.png}}
    \vspace{1em}
    \fbox{\includegraphics[width=\linewidth]{supplementary/r3.png}}
    \vspace{1em}
    \caption{\textbf{Similarity Overlaps:} Overlap between Top-3 Most Similar Original Texts and reconstructured text via Knowledge Units using an online plagiarism checker \href{https://app.copyleaks.com/text-compare}{tool}.}
    \label{fig:simtoprecon}
\end{figure}

\clearpage

\section{Legal Opinion}
\label{sec:legal}

We include, alongside this paper, a legal opinion from a team of lawyers which formed the source material used to derive legal insights about the German law, but not publicly available. This opinion corresponds to \citet{heidrich2023legal} and is provided here for completeness.

\includepdf[pages=-]{supplementary/heidrich2023legal.pdf}