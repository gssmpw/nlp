\begin{figure*}[ht]
    \centering
    \begin{subfigure}{.44\textwidth}
        \includegraphics[width=1.01\linewidth]{figures/sym_buildup.png} % Add figure file
        \caption{Buildup for a symmetric system with 3 servers with capacity $\mu$ and 3 queues with arrival rates $\lambda$ as a function of $\lambda/\mu$.}
        \label{fig:fig1a-buildup-vs-capacity-ratio-symmentric}
    \end{subfigure}
    \hfill
    \begin{subfigure}{.44\textwidth}
        \includegraphics[width=1.01\linewidth]{figures/rand_buffer_only.png} % Add figure file
        \caption{Buildup for an ensemble of 200 randomized systems with $5$ queues and $6$ servers as a function of $\sum_i \lambda_i/\sum_j \mu_j$.}
        \label{fig:fig1b-buildup-vs-capacity-ratio-non-symmentric}
    \end{subfigure}
    \caption{Empirical buildup of total queue sizes normalized by $n \cdot T$ as a function of the ratio $\sum_i \lambda_i/\sum_j \mu_j$ across different scenarios. The dashed vertical lines mark the ratios of $1/3$ and $1/2$ from our analysis.}
    \label{fig:fig1-buildup-vs-capacity-ratio}
\end{figure*}

In the preceding section, we derived theoretical stability conditions and worst-case bounds on the buildup of queue lengths. Specifically, our proof showed that in a system with singleton buffers, whenever a queue grows large, given that a condition of the total service capacity is satisfied, a no-regret property of the server-selection process guarantees that it will shrink at a linear rate, keeping the total lengths of queues bounded. In this section, we complement these theoretical guarantees with computational experiments to observe the typical behavior of these systems beyond the worst case, and explore the empirical impact of buffers on the dynamics.

We conduct simulations using the EXP3 algorithm \cite{auer2002nonstochastic} (see also \cite{slivkins2019introduction}), implemented in Python. For the simulation parameters, we use $\gamma$ for exploration rate and $T$ for the time horizon of the simulation. 

\subsection{Empirical Queue Buildup}

\begin{figure*}[ht!]
% \vspace{-8pt}
    \centering
    \begin{subfigure}{.44\textwidth}
        \includegraphics[width=1.01\linewidth]{figures/probability_buildup.png} % Add figure file
        \caption{Probability that a randomized system with time horizon $T$ has a queue with buildup greater than $\sqrt{T}$ as a function of the ratio $\sum_i \lambda_i/\sum_j \mu_j$. Shaded regions show the 95\% binomial-proportion confidence interval for the probability estimation.}
        \label{fig:fig2a-buildup-with-vs-without-buffers}
    \end{subfigure}
    \hfill
    \begin{subfigure}{.44\textwidth}
        \includegraphics[width=1.01\linewidth]{figures/clearing_rates.png} % Add figure file
        \caption{Empirical clearing rates (the number of packets cleared by a system normalized by time horizon $T$) across $1$,$000$ samples of a system with fixed parameters. The left histogram is without buffers and the right one is with buffers.}
        \label{fig:fig2b-secvice-rate-with-vs-without-buffers}
    \end{subfigure}
    \caption{A comparison of identical systems with and without buffers.}
    \label{fig:fig2-with-vs-without-buffers}
\end{figure*}

\begin{figure*}[h!]
\vspace{8pt}
    \centering
    \begin{subfigure}{.325\textwidth}
        \includegraphics[width=1.07\linewidth]{figures/4a.png} % Add figure file
        \caption{Dynamics in a system with $2$ queues with $\lambda_1 = 1/3$, $\lambda_2 = 1/6$ and 3 servers with $\mu_1 = 2/3$, $\mu_2 = 2/9$, $\mu_3 = 1/9$.\vspace{22pt}}
        \label{fig:fig3a-dynamics}
    \end{subfigure}
    \hfill
    \begin{subfigure}{.325\textwidth}
        \includegraphics[width=1.07\linewidth]{figures/4b.png} % Add figure file
        \caption{Probability of sending to server 1 in 
        a system with two queues with 
        % for two queues 
        % $\lambda_1 = \lambda_2 = 1/4$ and two servers with $\mu_1 = \mu_2 = 2/3$ compared to the Nash equilibria.}
        $\lambda_i = 1/4$, and two servers with $\mu_i = 2/3$, compared to the Nash equilibria. Dashed lines are the time average.}
        \label{fig:fig3b-dynamics-and-NE}
    \end{subfigure}
    \hfill
    \begin{subfigure}{.325\textwidth}
        \includegraphics[width=1.07\linewidth]{figures/4c.png} % Add figure file
        \caption{Total variation distance  
        from a pure Nash equilibrium as a function of time in multiple simulations in the system from Figure \ref{fig:fig3b-dynamics-and-NE}. 
        % The 95\% confidence interval is shaded.}
        Shaded regions are 95\% confidence intervals.}
        \label{fig:fig3c-TV-from-NE}
    \end{subfigure}
    \caption{Dynamics of probability distributions and the empirical play across different scenarios.}
    \label{fig:fig3-dynamics}
    % \hfill
\end{figure*}

We start by examining the relationship between the system's arrival rate and its service capacity. The service capacity must strictly exceed the arrival rate to prevent queue buildup. This observation is seen also in typical experimental scenarios.

Figure \ref{fig:fig1-buildup-vs-capacity-ratio} illustrates the total empirical buildup, which is the sum of all packets left in queues after $T$ iterations, normalized by $n \cdot T$ (vertical axis) as a function of the ratio $\sum_i \lambda_i/\sum_j \mu_j$ (horizontal axis). 
Figure \ref{fig:fig1a-buildup-vs-capacity-ratio-symmentric} focuses on a symmetric system with three servers such that $\mu_1 = 0.8$ and $\mu_2 = \mu_3 = 0.2$ and three queues with equal arrival rates $\lambda_i$, which range between $0.02$ and $0.4$. The simulations run for $T = 50$,$000$ steps with $\gamma = 1/\sqrt{T}$. The solid line represents the average buildup over $200$ independent simulations for each value of $\lambda_i$, and the shaded region indicates the range between minimum and maximum buildup values observed in the simulations. A clear transition emerges: when $\sum_i \lambda_i/\sum_j \mu_j > 0.9$, the buildup becomes proportional to $T$. Crucially, this transition occurs at a point above $0.5$ and below $1$ (where buildup is inevitable). 

Figure \ref{fig:fig1b-buildup-vs-capacity-ratio-non-symmentric} presents the same analysis for an ensemble of randomized systems with $5$ queues and $6$ servers. To generate such instances, we proceed as follows: generate $200$ base parameters by selecting 5 queue arrival rates, $\lambda_i$, and 6 server capacity rates, $\mu_j$, uniformly at random in $(0,1)$. For each value of the ratio $r = \sum_i \lambda_i/\sum_j \mu_j$ (with $r$ in $(0, 1]$), rescale the arrival rates such that the ratio $\sum_i \lambda_i/\sum_j\mu_j = r$ is satisfied. 
This process is repeated for each of the $200$ sets of base parameters. The simulations are run for $T = 50$,$000$ and %\jmcomment{Should we have the learning rate here too?} 
$\gamma = 1/\sqrt{T}$. In Figure \ref{fig:fig1b-buildup-vs-capacity-ratio-non-symmentric}, the solid line represents the average buildup over the $200$ simulation for each $r$ value; the shaded region indicates the range between the $2.5$ and $97.5$ percentiles of buildup values. The results confirm that, even for asymmetric and randomized systems, the transition point remains above $0.5$ and below $1$. In both Figure \ref{fig:fig1a-buildup-vs-capacity-ratio-symmentric} and Figure \ref{fig:fig1b-buildup-vs-capacity-ratio-non-symmentric} the vertical lines indicate where $r = \frac{1}{3}$ and $r = \frac{1}{2}$. The empirical buildups in these simulations show that while the worst case bound is $\sum_i\lambda_i < \frac{1}{3} \sum_j \mu_j$, systems with learning typically do better than the worst case at clearing packets.


\subsection{The Impact of Buffers}
We now compare systems with vs. without buffers. The methodology follows the one described in the previous subsection. Figure \ref{fig:fig2a-buildup-with-vs-without-buffers} illustrates the effect of adding buffers to a system in the random-instance scenario, with $n=5$ queues and $m=6$ servers as described earlier. It shows the probability that some queue in the system has at time $T$ a buildup greater than $\sqrt{T}$ as a function of the arrival-to-capacity ratio for systems with and without buffers. The solid lines show the observed frequency in $300$ independent simulations for each value of $r = \sum_i \lambda_i/\sum_j \mu_j$ in $(0, 1]$; the shaded region shows 
the $95\%$ confidence interval in estimating these probabilities. We observe that systems without buffers require a higher capacity relative to the arrival rate to prevent buildup.

Figure \ref{fig:fig2b-secvice-rate-with-vs-without-buffers} shows the 
distributions of packet clearing rates for two identical systems, with and without buffers, over $1$,$000$ simulations with $T=10$,$000$ and $\gamma = 1/\sqrt{T}$. There are $4$ queues with $\lambda_1 = 0.6, \lambda_2 = \lambda_3 = \lambda_4 = 0.3$ and $5$ servers with $\mu_1 = 0.8, \mu_2 = \mu_3 = 0.4, \mu_4 = \mu_5 = 0.2$. The clear separation between the distributions demonstrates that buffers increase efficiency. In the system without buffers, 
queues preferred the servers with higher capacity, which led to more competition and a lower clearing rate since the three high-capacity servers could not process all packets.\footnote{The total service rate of the three highest-capacity servers is exactly the same as the total arrival rate, hence not enough for stability.} In the system with buffers, queues do learn to extract good service rates also from the lower capacity servers, which increases the overall clearing rate of the system. 


\subsection{Dynamics and Convergence}

Next, we look at the dynamics of the learning agents in our systems. We observe that in all parameter configurations we have tested, the dynamics converge in last iterate (up to a small noise level due to the finite time horizon and step size in our simulations) to Nash equilibria, as illustrated in Figure \ref{fig:fig3-dynamics}.  This is surprising, as even without the carryover effects that our games have, no-regret dynamics need not converge to Nash equilibria.\footnote{In fact, no-regret dynamics may fail to converge to any single coarse correlated equilibrium even in the average-iterate sense \cite{kolumbus2022auctions,kolumbus2022and}.}
Figure \ref{fig:fig3a-dynamics} depicts the dynamics in systems with $2$ queues and $3$ servers. It can be seen that the probabilities of play of the algorithms approximately converge to a stationary distribution. Note that a stationary distribution of play and the no-regret condition imply that this distribution is a Nash equilibrium. 
Figure \ref{fig:fig3b-dynamics-and-NE} shows a dynamic in a simpler system which has two pure Nash equilibria and one mixed equilibrium. As can be seen in the instance in the figure, which is a typical one, the dynamic in this system converges to a pure equilibrium. 
Figure \ref{fig:fig3c-TV-from-NE} depicts the total variation distance from the closest Nash equilibrium as a function of the horizon $T$ in $200$ simulations of the same system for each $T$. The strategies quickly converge, and so the distance of the historical average of play shrinks as well, roughly as $1/\sqrt{T}$.

