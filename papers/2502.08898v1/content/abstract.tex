Routers in networking use simple learning algorithms to find the best way to deliver packets to their desired destination. This simple, myopic and distributed decision system makes large queuing systems simple to operate, but at the same time, the system needs more capacity than would be required if all traffic were centrally coordinated. In a recent paper, Gaitonde and Tardos (EC 2020 and JACM 2023) initiate the study of such systems, modeling them as an infinitely repeated game in which routers compete for servers and the system maintains a state (number of packets held by each queue) that results from outcomes of previous rounds. Queues get to send a packet at each step to one of the servers, and servers attempt to process only one of the arriving packets, modeling routers. However, their model assumes that servers have no buffers at all, so queues have to resend all packets that were not served successfully. They show that, even with hugely increased server capacity relative to what is needed in the centrally-coordinated case, ensuring that the system is stable requires the use of timestamps and priority for older packets.\vspace{3pt}

We consider a system with two important changes, which make the model more realistic: first we add a very small buffer to each server, allowing the server to hold on to a single packet to be served later (even if it fails to serve it); and second, we do not require timestamps or priority for older packets. Our main result is to show that when queues are learning, a small constant factor increase in server capacity, compared to what would be needed if centrally coordinating, suffices to keep the system stable, even if servers select randomly among packets arriving simultaneously. \vspace{3pt}

This work contributes to the growing literature on the impact of selfish learning in systems with carryover effects between rounds: when outcomes in the present round affect the game in the future.
