\section{Related Work}
\label{sec:related_work}

We now review five tools that use AV scan reports to label malware. \textit{AVClass} is the seminal work in this area ____. The tool has a module for generating a token taxonomy and identifying aliases, although the module is not enabled by default. AVClass's successor, \textit{AVClass2}, adds functionality to tag malware according to file properties, behaviors, and other malicious attributes ____. AVClass and AVClass2 have since been combined into the same codebase and are collectively referred to as "AVClass" in this work ____. \textit{EUPHONY} is an AV tagging tool that was developed for labeling Android malware in particular. It uses a graph-based community detection approach for labeling clusters of similar files ____. \textit{AVClass++} is a fork of the original AVClass tool (prior to the merge with AVClass2) that can perform label propagation ____. \textit{Sumav} automatically generates a token taxonomy by forming a graph of related tokens, from which it identifies families ____. \textit{TagClass} uses an incremental parsing strategy that accurately identifies family names appearing after behavior- and file-related tokens ____. In subsequent work, the authors of TagClass were the first to propose a family labeling approach based on the Dawid-Skene algorithm ____.
The code for TagClass' AV parsers is available online, but the authors have not yet published an implementation of their labeling methodology based on the Dawid-Skene algorithm.

Accurate malware labeling is a critical issue in both research ____ and industry ____ settings. We found that each of the surveyed tools makes crucial oversights in one or more of the following areas, leading to labeling errors.

% \begin{enumerate}
    (1) \textbf{AV Detection Parsing}. Special care must be taken to avoid mistakes when parsing AV detections. Improper parsing can result in family names being ignored or non-family tokens being treated as family names.
    
    (2)  \textbf{Alias Resolution}. Each AV product uses their own preferred naming conventions. Tokens with different spellings but the same meaning are called \emph{aliases} and must be identified. Existing tools make frequent aliasing errors or ignore the alias problem entirely. 
    
    (3) \textbf{Aggregation Strategy.} Individual AV products may have better or worse coverage of particular malware families, and correlations between AV products must be taken into account. Existing tools use aggretation strategies that are na\"ive or that do not scale to large dataset sizes.
% \end{enumerate}