% CVPR 2025 Paper Template; see https://github.com/cvpr-org/author-kit

\documentclass[10pt,twocolumn,letterpaper]{article}

%%%%%%%%% PAPER TYPE  - PLEASE UPDATE FOR FINAL VERSION
% \usepackage{cvpr}              % To produce the CAMERA-READY version
% \usepackage[review]{cvpr}      % To produce the REVIEW version
\usepackage[pagenumbers]{cvpr} % To force page numbers, e.g. for an arXiv version

% Import additional packages in the preamble file, before hyperref
%
% --- inline annotations
%
\newcommand{\red}[1]{{\color{red}#1}}
\newcommand{\todo}[1]{{\color{red}#1}}
\newcommand{\TODO}[1]{\textbf{\color{red}[TODO: #1]}}
% --- disable by uncommenting  
% \renewcommand{\TODO}[1]{}
% \renewcommand{\todo}[1]{#1}



\newcommand{\VLM}{LVLM\xspace} 
\newcommand{\ours}{PeKit\xspace}
\newcommand{\yollava}{Yo’LLaVA\xspace}

\newcommand{\thisismy}{This-Is-My-Img\xspace}
\newcommand{\myparagraph}[1]{\noindent\textbf{#1}}
\newcommand{\vdoro}[1]{{\color[rgb]{0.4, 0.18, 0.78} {[V] #1}}}
% --- disable by uncommenting  
% \renewcommand{\TODO}[1]{}
% \renewcommand{\todo}[1]{#1}
\usepackage{slashbox}
% Vectors
\newcommand{\bB}{\mathcal{B}}
\newcommand{\bw}{\mathbf{w}}
\newcommand{\bs}{\mathbf{s}}
\newcommand{\bo}{\mathbf{o}}
\newcommand{\bn}{\mathbf{n}}
\newcommand{\bc}{\mathbf{c}}
\newcommand{\bp}{\mathbf{p}}
\newcommand{\bS}{\mathbf{S}}
\newcommand{\bk}{\mathbf{k}}
\newcommand{\bmu}{\boldsymbol{\mu}}
\newcommand{\bx}{\mathbf{x}}
\newcommand{\bg}{\mathbf{g}}
\newcommand{\be}{\mathbf{e}}
\newcommand{\bX}{\mathbf{X}}
\newcommand{\by}{\mathbf{y}}
\newcommand{\bv}{\mathbf{v}}
\newcommand{\bz}{\mathbf{z}}
\newcommand{\bq}{\mathbf{q}}
\newcommand{\bff}{\mathbf{f}}
\newcommand{\bu}{\mathbf{u}}
\newcommand{\bh}{\mathbf{h}}
\newcommand{\bb}{\mathbf{b}}

\newcommand{\rone}{\textcolor{green}{R1}}
\newcommand{\rtwo}{\textcolor{orange}{R2}}
\newcommand{\rthree}{\textcolor{red}{R3}}
\usepackage{amsmath}
%\usepackage{arydshln}
\DeclareMathOperator{\similarity}{sim}
\DeclareMathOperator{\AvgPool}{AvgPool}

\newcommand{\argmax}{\mathop{\mathrm{argmax}}}     



% It is strongly recommended to use hyperref, especially for the review version.
% hyperref with option pagebackref eases the reviewers' job.
% Please disable hyperref *only* if you encounter grave issues, 
% e.g. with the file validation for the camera-ready version.
%
% If you comment hyperref and then uncomment it, you should delete *.aux before re-running LaTeX.
% (Or just hit 'q' on the first LaTeX run, let it finish, and you should be clear).
\definecolor{cvprblue}{rgb}{0.21,0.49,0.74}
\usepackage[pagebackref,breaklinks,colorlinks,allcolors=cvprblue]{hyperref}

% DK
\usepackage{xcolor}
\usepackage{multirow}
% JH
\usepackage{adjustbox}

%%%%%%%%% PAPER ID  - PLEASE UPDATE
\def\paperID{16035} % *** Enter the Paper ID here
\def\confName{CVPR}
\def\confYear{2025}

%%%%%%%%% TITLE - PLEASE UPDATE
% \title{\LaTeX\ Author Guidelines for \confName~Proceedings}
% \title{360 Indoor Reconstruction}
\title{IM360: Textured Mesh Reconstruction for Large-scale Indoor Mapping with 360\textdegree\ Cameras}

%%%%%%%%% AUTHORS - PLEASE UPDATE
\author{Dongki Jung$^{\thanks{These two authors contributed equally.}}$\:\: Jaehoon Choi$^{\footnotemark[1]}$\:\: Yonghan Lee\:\: Dinesh Manocha\\
University of Maryland
}



\begin{document}
\maketitle
% \begin{abstract}
% We propose a comprehensive 3D reconstruction pipeline specifically tailored for 360° cameras to address challenges in indoor environments. Traditional Structure-from-Motion (SfM) methods struggle with issues such as textureless surfaces and sparse capture settings, which are critical for efficient data collection. To overcome these limitations, our approach leverages the wide field of view provided by omnidirectional images, employing a spherical camera model with dense feature matching for SfM. To further enhance geometric reconstruction, we integrate a neural implicit surface reconstruction technique, capable of generating high-quality surfaces from sparse input data. Additionally, we utilize a mesh-based neural rendering framework to refine texture maps, accurately capturing view-dependent properties by combining diffuse and specular components. We validate our pipeline on large-scale indoor scenes from the Matterport3D and Stanford2D3D datasets, demonstrating superior performance in textured mesh reconstruction. Comparative evaluations against existing localization and texture mapping techniques highlight the robustness and accuracy of our method. Project page: \url{anonymousim360.github.io/}


 We present a novel 3D reconstruction pipeline for 360° cameras for 3D mapping and rendering of indoor environments. Traditional Structure-from-Motion (SfM) methods may not work well in large-scale indoor scenes due to the prevalence of textureless and repetitive regions. To overcome these challenges, our approach (IM360) leverages the wide field of view of omnidirectional images and integrates the spherical camera model into every core component of the SfM pipeline. In order to develop a comprehensive 3D reconstruction solution, we integrate a neural implicit surface reconstruction technique to generate high-quality surfaces from sparse input data. Additionally, we utilize a mesh-based neural rendering approach to refine texture maps and accurately capture view-dependent properties by combining diffuse and specular components. We evaluate our pipeline on large-scale indoor scenes from the Matterport3D and Stanford2D3D datasets. In practice, IM360 demonstrate superior performance in terms of textured mesh reconstruction over SOTA. We observe accuracy improvements in terms of camera localization and registration as well as rendering high frequency details. 
 %Comparative evaluations against existing localization and texture mapping techniques highlight the robustness and accuracy of our method. 
 % Project page: \url{anonymousim360.github.io/}
\end{abstract}





% - (1) No existing SfM method is suitable for sparsely captured indoor scenes, which are typical in practical data collections such as Matterport3D. Current methods only perform well on indoor datasets with extremely dense captures, failing to address inherent challenges in indoor environments, such as textureless surfaces and repetitive structures.

% - (2) We propose a "complete pipeline" that processes raw 360° images into textured meshes with diffuse and specular components, effectively "eliminating spherical distortion at each step" of the reconstruction process.
\begin{abstract}
% We propose a comprehensive 3D reconstruction pipeline specifically tailored for 360° cameras to address challenges in indoor environments. Traditional Structure-from-Motion (SfM) methods struggle with issues such as textureless surfaces and sparse capture settings, which are critical for efficient data collection. To overcome these limitations, our approach leverages the wide field of view provided by omnidirectional images, employing a spherical camera model with dense feature matching for SfM. To further enhance geometric reconstruction, we integrate a neural implicit surface reconstruction technique, capable of generating high-quality surfaces from sparse input data. Additionally, we utilize a mesh-based neural rendering framework to refine texture maps, accurately capturing view-dependent properties by combining diffuse and specular components. We validate our pipeline on large-scale indoor scenes from the Matterport3D and Stanford2D3D datasets, demonstrating superior performance in textured mesh reconstruction. Comparative evaluations against existing localization and texture mapping techniques highlight the robustness and accuracy of our method. Project page: \url{anonymousim360.github.io/}


 We present a novel 3D reconstruction pipeline for 360° cameras for 3D mapping and rendering of indoor environments. Traditional Structure-from-Motion (SfM) methods may not work well in large-scale indoor scenes due to the prevalence of textureless and repetitive regions. To overcome these challenges, our approach (IM360) leverages the wide field of view of omnidirectional images and integrates the spherical camera model into every core component of the SfM pipeline. In order to develop a comprehensive 3D reconstruction solution, we integrate a neural implicit surface reconstruction technique to generate high-quality surfaces from sparse input data. Additionally, we utilize a mesh-based neural rendering approach to refine texture maps and accurately capture view-dependent properties by combining diffuse and specular components. We evaluate our pipeline on large-scale indoor scenes from the Matterport3D and Stanford2D3D datasets. In practice, IM360 demonstrate superior performance in terms of textured mesh reconstruction over SOTA. We observe accuracy improvements in terms of camera localization and registration as well as rendering high frequency details. 
 %Comparative evaluations against existing localization and texture mapping techniques highlight the robustness and accuracy of our method. 
 Project page: \url{anonymousim360.github.io/}
\end{abstract}





% - (1) No existing SfM method is suitable for sparsely captured indoor scenes, which are typical in practical data collections such as Matterport3D. Current methods only perform well on indoor datasets with extremely dense captures, failing to address inherent challenges in indoor environments, such as textureless surfaces and repetitive structures.

% - (2) We propose a "complete pipeline" that processes raw 360° images into textured meshes with diffuse and specular components, effectively "eliminating spherical distortion at each step" of the reconstruction process.

%%%%%%%%%%%%%%%%%%%%%%%
%% Introduction
%%%%%%%%%%%%%%%%%%%%%%%
\section{Introduction}
\label{introduction}
% % sfm 설명
% sfm 예시들 colmap, pixelsfm, theiasfm, spehresfm, etc
% challenges
% feature matching (especially dense matching)
% dense feature track 연결? 이거 내 contribution 아니니까 대충 얘기하거나 넘어가도 될 듯.
The 3D geometry of a scene is essential for a wide range of computer vision applications and has been a long-standing challenge in the field.
General Structure-from-Motion (SfM) software such as COLMAP \cite{schonberger2016structure}, TheiaSfM \cite{sweeney2015theia}, and OpenMVG \cite{moulon2017openmvg} have significantly advanced the capabilities of 3D reconstruction.

Indoor 3D mapping typically relies on LiDAR sensors or dense video scans to achieve accurate reconstruction. 
However, these requirements pose significant challenges when dealing with sparsely scanned indoor scenes captured solely by visual sensors. 
The primary issue lies in the prevalence of large textureless regions, which degrade the accuracy of both feature detection and matching, thereby impairing the overall performance of the SfM pipeline. 
Additionally, frequent occlusions in indoor environments, combined with sparse input views, make the feature matching process even more challenging. 
In such scenarios, conventional perspective cameras further worsen the problem due to their limited field of view, which reduces the overlap between views. 
This lack of overlap complicates the estimation of accurate camera poses, making the reconstruction task considerably more difficult.

Recently, detector-free matchers \cite{sun2021loftr,melekhov2019dgc,truong2020glu,truong2021learning,edstedt2023dkm,edstedt2024roma} have emerged as a promising alternative to detector-based approaches \cite{detone2018superpoint,revaud2019r2d2,tyszkiewicz2020disk}, particularly in handling repetitive or indiscriminate regions where keypoint detection performance tends to degrade.
However, detector-free matchers face issues with multi-view inconsistency in subpixel correspondences due to their pairwise dependency \cite{he2024detector}.
To mitigate this, Detector-Free SfM \cite{he2024detector} proposes a strategy that quantizes feature locations to form tracks across views, enhancing its compatibility with existing SfM systems.

\begin{figure}[t]
    \centering
    % \vspace*{-5mm}
    \includegraphics[width=1.\linewidth]{figures/fig1_mp3d.pdf}
    % \vspace*{-2mm}
    \caption{blue sphere wireframes indicate camera location. sparse, hard,}
    \label{fig:introduction}
\end{figure}


\paragraph{Main Results}
In this paper, we present a novel 3D indoor reconstruction pipeline designed for omnidirectional images, addressing challenges that conventional SfM approaches fail to overcome.
We tackle the problem of extensive textureless regions by employing detector-free matching methods, which improve matching accuracy without the need for explicit feature detection. 
To mitigate the frequent occlusions encountered in sparsely scanned images, we utilize a spherical camera model with a wide field of view, increasing the overlap between images and enhancing robustness in multiview matching.
To the best of our knowledge, this is the first work to present dense matching and SfM tailored for the spherical camera model.
Our framework also extends to mesh reconstruction and texturing, providing a comprehensive solution for indoor 3D reconstruction. 
The proposed method demonstrates significant improvements over traditional approaches, particularly in challenging indoor environments with sparse and textureless scenes.
The main contributions of this paper are summarized as follows:
\begin{itemize}
    \item We present a unified framework to reconstruct the textured mesh for sparsely scanned large scale indoor scenes.
    \item the first dense matching \& spherical sfm
    \item texture refine
\end{itemize}

% sfm 설명
% sfm 예시들 colmap, pixelsfm, theiasfm, spehresfm, etc
% challenges
% feature matching (especially dense matching)
% dense feature track 연결? 이거 내 contribution 아니니까 대충 얘기하거나 넘어가도 될 듯.


Indoor 3D mapping and photorealistic rendering are core technologies for various applications in computer vision, robotics, and graphics. High-fidelity digitization of the real world enables immersive experiences in AR/VR and helps bridge the sim-to-real gap in robotic applications. However, the conventional image acquisition process \cite{dai2017scannet,baruch2021arkitscenes,barron2022mip} is often labor-intensive and time-consuming due to the limited field-of-view and sparse scene coverage, as shown in Fig \ref{fig:introduction}. 
Previous studies \cite{janiszewski2022rapid,herban2022use} have demonstrated that omnidirectional cameras can significantly reduce acquisition time, which is crucial for capturing various indoor environments.
Nevertheless, most recent research \cite{schonberger2016structure,moulon2017openmvg,mildenhall2021nerf,kerbl20233d,waechter2014TexRecon} relies on conventional cameras with limited fields of view, which results in inadequate scene coverage, susceptibility to motion blur, and substantial time requirements for capturing large-scale indoor environments. 

Unlike conventional cameras, which benefit from well-established photogrammetric software such as COLMAP \cite{schonberger2016structure} and OpenMVG \cite{moulon2017openmvg}, research on omnidirectional cameras remains fragmented across various domains and lacks a robust pipeline for Structure-from-Motion (SfM) and photorealistic rendering. Notably, there has been limited academic exploration of sparse scanning scenarios aimed at significantly reducing image acquisition time. There are two key challenges in developing 3D mapping and rendering for omnidirectional cameras. 



\begin{figure}[t]
    \centering
    % \vspace*{-5mm}
    \includegraphics[width=1.\linewidth]{figures/fig1_mp3d.pdf}
    % \vspace*{-2mm}
    \caption{This example compares two benchmark datasets: one captured with a conventional camera \cite{dai2017scannet} and the other with an omnidirectional camera \cite{chang2017matterport3d}. The blue wireframes indicate the camera locations. While the wide field of view of the omnidirectional camera significantly reduces the data acquisition time, it introduces challenges such as sparse views and occlusions. Our approach provides an effective solution to handle issues in sparsely scanned indoor environments and enables efficient reconstruction and rendering using omnidirectional cameras.}
    \label{fig:introduction}
    \vspace*{-5mm}
\end{figure}

First, the SfM pipeline in sparsely scanned indoor environments suffer from the prevalence of large textureless regions, which degrade the accuracy of both feature detection and matching. Moreover, frequent occlusions in indoor environments, combined with sparse input views as shown in Fig. \ref{fig:introduction}, make the feature matching process even more challenging. In such scenarios, conventional perspective cameras further worsen the problem due to their limited field of view, which reduces the overlap between views. This lack of overlap complicates the estimation of accurate camera poses, making the reconstruction task considerably more difficult. Recently, detector-free or dense matchers \cite{sun2021loftr,melekhov2019dgc,truong2020glu,truong2021learning,edstedt2023dkm,edstedt2024roma} have emerged as a promising alternative to detector-based approaches \cite{detone2018superpoint,revaud2019r2d2,tyszkiewicz2020disk}, particularly in handling repetitive or indiscriminate regions where keypoint detection performance tends to degrade.
% However, detector-free matchers face issues with multi-view inconsistency in subpixel correspondences due to their pairwise dependency \cite{he2024detector}.
% To mitigate this, Detector-Free SfM \cite{he2024detector} proposes a strategy that quantizes feature locations to form tracks across views, enhancing its compatibility with existing SfM systems.

Additionally, in sparsely scanned scenarios, recent neural rendering methods \cite{mildenhall2021nerf,kerbl20233d} often face challenges in producing high-quality novel-view images. It is widely recognized that both neural radiance fields (NeRF) \cite{mildenhall2021nerf} and 3D Gaussian splatting (3DGS) \cite{kerbl20233d} are optimized for densely captured scenes with substantial overlap between viewpoints. Prior studies \cite{xiong2023sparsegs,wang2023sparsenerf,martins2024feature,lee2024modegs} have demonstrated that neural rendering algorithms often exhibit overfitting to the training views and a poor rendering quality on novel-view images far from training views.

% \begin{figure}[t]
%     \centering
%     % \vspace*{-5mm}
%     \includegraphics[width=0.9\linewidth]{figures/fig1_intro.pdf}
%     % \vspace*{-2mm}
%     \caption{makk geurheo noeun geot}
%     \label{fig:introduction}
% \end{figure}

% The 3D geometry of a scene is essential for a wide range of computer vision applications and has been a long-standing challenge in the field.
% General Structure-from-Motion (SfM) software such as COLMAP \cite{schonberger2016structure}, TheiaSfM \cite{sweeney2015theia}, and OpenMVG \cite{moulon2017openmvg} have significantly advanced the capabilities of 3D reconstruction.
% Indoor 3D mapping typically relies on LiDAR sensors or dense video scans to achieve accurate reconstruction. 
% However, these requirements pose significant challenges when dealing with sparsely scanned indoor scenes captured solely by visual sensors. 
% The primary issue lies in the prevalence of large textureless regions, which degrade the accuracy of both feature detection and matching, thereby impairing the overall performance of the SfM pipeline. 
% Additionally, frequent occlusions in indoor environments, combined with sparse input views, make the feature matching process even more challenging. 
% In such scenarios, conventional perspective cameras further worsen the problem due to their limited field of view, which reduces the overlap between views. 
% This lack of overlap complicates the estimation of accurate camera poses, making the reconstruction task considerably more difficult.

% Recently, detector-free matchers \cite{sun2021loftr,melekhov2019dgc,truong2020glu,truong2021learning,edstedt2023dkm,edstedt2024roma} have emerged as a promising alternative to detector-based approaches \cite{detone2018superpoint,revaud2019r2d2,tyszkiewicz2020disk}, particularly in handling repetitive or indiscriminate regions where keypoint detection performance tends to degrade.
% However, detector-free matchers face issues with multi-view inconsistency in subpixel correspondences due to their pairwise dependency \cite{he2024detector}.
% To mitigate this, Detector-Free SfM \cite{he2024detector} proposes a strategy that quantizes feature locations to form tracks across views, enhancing its compatibility with existing SfM systems.

\paragraph{Main Results}
We present a novel indoor 3D mapping and rendering pipeline designed for omnidirectional images and address the challenges of sparsely scanned indoor environments.
We tackle the problem of extensive textureless regions by employing detector-free matching methods~\cite{EDM}.
%which improve matching accuracy without the need for explicit feature detection. 
To mitigate the frequent occlusions encountered in sparsely scanned images, we utilize an spherical camera model with a wide field of view, increasing the overlap between images and enhancing robustness in multi-view matching.
% To the best of our knowledge, this is the first work to present dense matching and SfM designed for the spherical camera model. Our method achieves improved camera poses without the need for redundant or repetitive feature matching.  
% (\textcolor{blue}{(YH) 
 While traditional methods, such as OpenMVG \cite{moulon2017openmvg} and COLMAP \cite{schonberger2016structure}, fail to register nearly half of the images in challenging datasets like Matterport3D \cite{chang2017matterport3d} and Stanford2D3D \cite{Stanford2d3d}, our method achieves unparalleled registration performance with high pose accuracy, thanks to the tight integration of a wide-view spherical model.
% })
To the best of our knowledge, ours is the first work to present a complete SfM pipeline dedicated to spherical camera models, where entire step—feature matching, two-view geometry estimation, track triangulation, and bundle adjustment—is conducted on a spherical manifold.

We extend our approach to mesh reconstruction and texturing and provide a comprehensive solution for indoor 3D mapping and rendering.  For mesh reconstruction, we leverage neural surface reconstruction \cite{xiao2024debsdf} by training a signed distance field and extracting a mesh. Classical texture mapping \cite{waechter2014TexRecon} is then combined with texture optimization through differentiable rendering. We optimize textures and train small multi-layer perceptrons (MLPs) to represent specular color. By jointly optimizing diffuse and specular textures along with the MLPs via differentiable rendering \cite{nvdiffrast} under image supervision, our approach achieves superior rendering quality. Additionally, it demonstrates robustness in sparsely scanned indoor environments, outperforming recent neural rendering methods. The proposed method demonstrates significant improvements over traditional approaches, particularly in challenging indoor environments with sparse and textureless scenes. 

% We evaluate our approach using the Matterport3D dataset \cite{chang2017matterport3d}, which spans 101.82k $m^2$ and offers a larger physical scale than typical indoor datasets like ScanNet \cite{dai2017scannet} (0.56k $m^2$) and Replica \cite{straub2019replica} (39.98 $m^2$), according to \cite{ramakrishnan2021hm3d}. 

We evaluate our approach using the Matterport3D dataset~\cite{chang2017matterport3d}, which offers a larger physical scale than typical indoor datasets like ScanNet, according to~\cite{ramakrishnan2021hm3d}. We also evaluate the performance on
%Additionally, to demonstrate the generalization capability of our method, we validate it on 
the Stanford2D3D datasets \cite{Stanford2d3d}. The novel contributions of our work include:
%The main contributions of this paper are summarized as follows:
\begin{itemize}
    \item We introduce a unified framework for reconstructing textured meshes using omnidirectional cameras in sparsely scanned, large-scale indoor scenes.
    \item We propose the first spherical Structure-from-Motion approach utilizing dense matching features specifically designed for omnidirectional cameras.
    \item We integrate classical texture mapping with neural texture fine-tuning through differentiable rendering, demonstrating improved rendering quality in sparsely scanned, in-the-wild indoor environments. 
\end{itemize}
We  compare our method with SOTA  and observe higher accuracy in terms of camera localization and rendering. In particular our approach can render higher frequency details and results in lower noise in the reconstructed models.


\section{Related Work}
% \subsection{Structure from Motion}
% feature matching - SPSG, DKM,
% structure from motion - colmap, dfsfm, 
% SphereGlue, SphereSfM 
% EDM 
% 하지만 아직 omnidirectional camera 에서  dense matching과 detector free sfm 을 결합한 연구가 없음을 강조
% Traditional, Learning detector-based feature and SfM 
% Dense Matching and Detector Free SfM

Structure from Motion (SfM) is a fundamental problem in computer vision. Traditional SfM frameworks typically begin with the detection of keypoints and descriptors, which serve as a core component, ranging from classical algorithms \cite{SIFT} to learning-based techniques \cite{detone2018superpoint,sarlin2020superglue,revaud2019r2d2}. Then, they match these keypoints by either nearest neighbor \cite{schonberger2016structure} or learning-based methods \cite{sarlin2020superglue}. The matching pair is verified by recovering the two-view geometry \cite{hartley2003multiple} with RANSAC \cite{RANSAC}. Based on these matching pairs, popular SfM methods \cite{schonberger2016structure,moulon2017openmvg,sweeney2015theia} follow incremental approach, sequentially registering new images and reconstructing their 3D structures through triangulation \cite{hartley2003multiple} and bundle adjustment \cite{Bundleadjustment}. Alternatively, a recent work \cite{GLOMAP} employs a global approach, recovering the camera geometry for all input images simultaneously. The 2D correspondence matching significantly impact the overall performance of SfM pipeline. In textureless region of indoor environment, these methods suffer from poor feature detection \cite{he2024detector}. Detector-free or dense matching methods  \cite{sun2021loftr,edstedt2023dkm,edstedt2024roma} is proposed to solve this issue by estimating dense feature matches at pixel level. DfSfM \cite{he2024detector} builds these detector-free matches \cite{sun2021loftr} and refines the tracks and geometry of
the coarse SfM model by enforcing multi-view consistency. 

% However, there are relatively few open-source software packages \cite{moulon2017openmvg,jiang20243d} that support omnidirectional cameras.  These frameworks typically adopt an incremental SfM workflow based on a spherical camera model \cite{jiang20243d}. To address the challenge of keypoint matching across multiple spherical images, as opposed to perspective images, various detector-based feature methods have been proposed \cite{gava2023sphereglue}. More recently, the first dense matching method, EDM \cite{EDM}, for spherical images has also been introduced. In this paper, we first introduce a detector-free SfM framework for spherical cameras by leveraging this dense matching approach.
Conversely, there are relatively few open-source software packages \cite{moulon2017openmvg,jiang20243d} that support omnidirectional cameras. 
These frameworks typically employ an incremental SfM workflow based on a spherical camera model \cite{jiang20243d}. 
However, existing methods lack a comprehensive pipeline specifically designed for omnidirectional images. 
Recently, the first dense matching method for omnidirectional images, EDM \cite{EDM}\footnote{EDM is our previous work.}, has been introduced. 
Building upon this feature matching and two-view geometry estimation \cite{solarte2021robust}, we propose a complete pipeline for spherical SfM to address the challenges inherent in indoor scene reconstruction.


\subsection{Textured Mesh Reconstruction}
Textured mesh is the basic component of photorealistic rendering. Traditional computer vision has developed effective techniques for mapping between texture and geometry. Classical texture mapping \cite{lempitsky2007seamless,waechter2014TexRecon,fu2018texture} employs a Markov Random Field to select the optimal color image for each mesh face and applies global color adjustment for consistency \cite{waechter2014TexRecon}. Recently, numerous neural rendering techniques have been developed for surface reconstruction \cite{wang2021neus,yu2022monosdf,choi2023tmo,xiao2024debsdf} and rendering \cite{mildenhall2021nerf,barron2022mip,barron2023zip,kerbl20233d}. Notably, some approaches leverage differentiable rendering \cite{nvdiffrast,PyTorch3D,Mitsuba3,munkberg2021nvdiffrec,Goel_2022_CVPR} to learn neural representations of texture, lighting, and geometry. Additionally, certain NeRF variants \cite{SNeRG,chen2023mobilenerf,NeuRas,DNMP,Choi2024LTM} employ differentiable volumetric rendering with mesh primitives, which is similar to rasterization-based rendering. Some neural rendering methods are tailored for omnidirectional cameras. EgoNeRF \cite{choi2023balanced} utilizes a spherical feature grid to represent a distant environment for rendering. OmniSDF \cite{kim2024omnisdf} introduces an adaptive spherical binoctree method for surface reconstruction. However, in sparsely scanned indoor environment, which contains lots of textureless regions and sparse viewpoints, most neural rendering methods struggle with severe artifacts. To address this limitation, we propose a hybrid approach that combines classical texture mapping with neural texture optimization.



% \subsection{Omnidirectional Camera Model}
% omnisfm, Sphereglue 
% omnisdf, omninerf
% omnigs
% EDM 
% 하지만 아직 omnidirectional camera 에서  dense matching과 detector free sfm 을 결합한 연구가 없음을 강조
% For spherical cameras, SPHORB, SphereGlue, PanoPoint, CoVisPose
% OpenMVG\cite{moulon2017openmvg}
% SphereSfM\cite{jiang20243d}
% 360-8pa \cite{solarte2021robust}
\subsection{Structure from Motion}
% feature matching - SPSG, DKM,
% structure from motion - colmap, dfsfm, 
% SphereGlue, SphereSfM 
% EDM 
% 하지만 아직 omnidirectional camera 에서  dense matching과 detector free sfm 을 결합한 연구가 없음을 강조
% Traditional, Learning detector-based feature and SfM 
% Dense Matching and Detector Free SfM

Structure from Motion (SfM) is a fundamental problem in computer vision. Traditional SfM frameworks typically begin with the detection of keypoints and descriptors, which serve as a core component, ranging from classical algorithms \cite{SIFT} to learning-based techniques \cite{detone2018superpoint,sarlin2020superglue,revaud2019r2d2}. Then, they match these keypoints by either nearest neighbor \cite{schonberger2016structure} or learning-based methods \cite{sarlin2020superglue}. The matching pair is verified by recovering the two-view geometry \cite{hartley2003multiple} with RANSAC \cite{RANSAC}. Based on these matching pairs, popular SfM methods \cite{schonberger2016structure,moulon2017openmvg,sweeney2015theia} follow incremental approach, sequentially registering new images and reconstructing their 3D structures through triangulation \cite{hartley2003multiple} and bundle adjustment \cite{Bundleadjustment}. Alternatively, a recent work \cite{GLOMAP} employs a global approach, recovering the camera geometry for all input images simultaneously. The 2D correspondence matching significantly impact the overall performance of SfM pipeline. In textureless region of indoor environment, these methods suffer from poor feature detection \cite{he2024detector}. Detector-free or dense matching methods  \cite{sun2021loftr,edstedt2023dkm,edstedt2024roma} is proposed to solve this issue by estimating dense feature matches at pixel level. DetectorFreeSfM \cite{he2024detector} builds these detector-free matches \cite{sun2021loftr} and refines the tracks and geometry of
the coarse SfM model by enforcing multi-view consistency. 

% However, there are relatively few open-source software packages \cite{moulon2017openmvg,jiang20243d} that support omnidirectional cameras.  These frameworks typically adopt an incremental SfM workflow based on a spherical camera model \cite{jiang20243d}. To address the challenge of keypoint matching across multiple spherical images, as opposed to perspective images, various detector-based feature methods have been proposed \cite{gava2023sphereglue}. More recently, the first dense matching method, EDM \cite{EDM}, for spherical images has also been introduced. In this paper, we first introduce a detector-free SfM framework for spherical cameras by leveraging this dense matching approach.
Conversely, there are relatively few open-source software packages \cite{moulon2017openmvg,jiang20243d} that support omnidirectional cameras. 
These frameworks typically employ an incremental SfM workflow based on a spherical camera model \cite{jiang20243d}. 
However, existing methods lack a comprehensive pipeline specifically designed for omnidirectional images. 
Recently, the first dense matching method for omnidirectional images, EDM \cite{EDM}, has been introduced. 
Building upon this feature matching and two-view geometry estimation \cite{solarte2021robust}, we propose a complete pipeline for spherical SfM to address the challenges inherent in indoor scene reconstruction.


\subsection{Textured Mesh Reconstruction}
Textured mesh is the basic component of photorealistic rendering. Traditional computer vision has developed effective techniques for mapping between texture and geometry. Classical texture mapping \cite{lempitsky2007seamless,waechter2014TexRecon,fu2018texture} employs a Markov Random Field to select the optimal color image for each mesh face and applies global color adjustment for consistency \cite{waechter2014TexRecon}. Recently, numerous neural rendering techniques have been developed for surface reconstruction \cite{wang2021neus,yu2022monosdf,choi2023tmo,xiao2024debsdf} and rendering \cite{mildenhall2021nerf,barron2022mip,barron2023zip,kerbl20233d}. Notably, some approaches leverage differentiable rendering \cite{nvdiffrast,PyTorch3D,Mitsuba3,munkberg2021nvdiffrec,Goel_2022_CVPR} to learn neural representations of texture, lighting, and geometry. Additionally, certain NeRF variants \cite{SNeRG,chen2023mobilenerf,NeuRas,DNMP,Choi2024LTM} employ differentiable volumetric rendering with mesh primitives, which is similar to rasterization-based rendering. Some neural rendering methods are tailored for omnidirectional cameras. EgoNeRF \cite{choi2023balanced} utilizes a spherical feature grid to represent a distant environment for rendering. OmniSDF \cite{kim2024omnisdf} introduces an adaptive spherical binoctree method for surface reconstruction. However, in sparsely scanned indoor environment, which contains lots of textureless regions and sparse viewpoints, most neural rendering methods struggle with severe artifacts. To address this limitation, we propose a hybrid approach that combines classical texture mapping with neural texture optimization.



% \subsection{Omnidirectional Camera Model}
% omnisfm, Sphereglue 
% omnisdf, omninerf
% omnigs
% EDM 
% 하지만 아직 omnidirectional camera 에서  dense matching과 detector free sfm 을 결합한 연구가 없음을 강조
% For spherical cameras, SPHORB, SphereGlue, PanoPoint, CoVisPose
% OpenMVG\cite{moulon2017openmvg}
% SphereSfM\cite{jiang20243d}
% 360-8pa \cite{solarte2021robust}


\section{Our Method}
% \begin{figure}[h]
    \centering
    % \vspace*{-5mm}
    \includegraphics[width=1.0\linewidth]{figures/fig2_pipeline.pdf}
    % \vspace*{-2mm}
    \caption{Overview of Our Pipeline. It is comprised of three steps: Spherical Structure from Motion (Sec. \ref{sec:3_1}), Geometry Reconstruction (Sec. \ref{sec:3_2}), and Texture Optimization (Sec. \ref{sec:3_3}). Our method }
    \label{fig:pipeline}
\end{figure}
Given a set of 360-degree equirectangular projection (ERP) images, our objective is to reconstruct the geometry and texture of real-world scenes (refer to Figure X). 
Our pipeline is composed of three primary components: Structure from Motion (SfM) using a spherical camera model, geometry reconstruction, and texture optimization.

To accurately estimate camera poses from sparsely scanned images, we perform SfM utilizing a spherical camera model. 
This approach leverages the wide field of view inherent in spherical imagery, addressing challenges associated with limited overlap and frequent occlusions in indoor environments.

For 3D geometric reconstruction, we employ a neural model that represents the zero-level set of a signed distance function (SDF). 
After training the model, we extract the mesh using the Marching Cubes algorithm. 
Initially, we attempted to define rays using the spherical camera model for training but observed a decline in reconstruction quality. 
Similarly, OmniSDF did not yield accurate results. 
Consequently, we converted ERP images into cubemaps and conducted training on the resulting perspective images to improve performance.

For texture map optimization, we leverage differentiable rasterization techniques by integrating Multi-View Stereo Texturing (MVS-Texturing) and Gaussian Splatting. 
MVS-Texturing is particularly effective in sparse scenes where details are not prominently visible, while Gaussian Splatting excels in densely scanned scenes. 
By potentially combining these two methods, we aim to capitalize on their respective strengths. 
Additionally, the use of differentiable rendering allows us to reduce noise in areas where seams occur, enhancing overall texture quality.

Given a set of equirectangular projection (ERP) images, our goal is to reconstruct the geometry and texture of real-world scenes. In Fig. \ref{fig:pipeline}, 
our pipeline is composed of three primary components: Spherical Structure from Motion (SfM), geometry reconstruction, and texture optimization. To address challenges associated with limited overlap and frequent occlusions in indoor environments, our method leverages the wide field of view inherent in spherical imagery. We perform SfM utilizing a spherical camera model and dense features to estimate camera poses from sparsely scanned images in textureless regions. For 3D geometric reconstruction accurately, we converted ERP images into cubemaps and conducted training on the resulting perspective images. Then, we employ a neural model \cite{xiao2024debsdf} that represents a signed distance function (SDF). After training the model, we extract the mesh using the Marching Cubes algorithm \cite{marchingcube} and apply classical texture mapping \cite{waechter2014TexRecon} to initialize texture. Finally, we optimize the texture for diffuse, specular and small MLP components using differentiable rendering \cite{nvdiffrast}.    
% Initially, we attempted to define rays using the spherical camera model for training but observed a decline in reconstruction quality. Similarly, OmniSDF did not yield accurate results. 
% Then, we converted ERP images into cubemaps and conducted training on the resulting perspective images.

% For texture map optimization, we leverage differentiable rasterization techniques by integrating Multi-View Stereo Texturing (MVS-Texturing) and Gaussian Splatting. 
% MVS-Texturing is particularly effective in sparse scenes where details are not prominently visible, while Gaussian Splatting excels in densely scanned scenes. 
% By potentially combining these two methods, we aim to capitalize on their respective strengths. 
% Additionally, the use of differentiable rendering allows us to reduce noise in areas where seams occur, enhancing overall texture quality.



\begin{figure}[t]
    \centering
    % \vspace*{-5mm}
    \includegraphics[width=1.0\linewidth]{figures/fig2_pipeline_JH.pdf}
    % \vspace*{-2mm}
    \caption{Overview of Our Pipeline. It is comprised of three steps: Spherical Structure from Motion (Sec. \ref{sec:3_1}), Geometry Reconstruction (Sec. \ref{sec:3_2}), and Texture Optimization (Sec. \ref{sec:3_3}). We introduce a spherical SfM framework for estimating camera poses from omnidirectional images in challenging indoor environments. This is followed by geometry reconstruction to generate a 3D mesh, combined with texture mapping and neural texture optimization to enable realistic rendering of the entire scene.}
    \vspace*{-2mm}
    \label{fig:pipeline}
\end{figure}


\begin{figure*}[t]
    \centering
    % \vspace*{-5mm}
    \includegraphics[width=1.0\linewidth]{figures/fig3_method.pdf}
    % \vspace*{-28mm}
    \caption{Our Approach: In Section \ref{sec:3_1}, 
    % \textcolor{red}{EXPLAIN!} 
    spherical SfM is utilized to estimate the camera poses for omnidirectional images.
    Then, in Section \ref{sec:3_2}, our method apply cubemap projection and estimate monocular depth and normal to train geometry $f_{sdf}$ representation using volumetric rendering. In Section \ref{sec:3_3}, we initlize texture using classical texture mapping and jointly optimize diffuse, specular texture along with small MLP for modeling appearance.}
     \vspace*{-2mm}
    \label{fig:method}
\end{figure*}

\subsection{Dense Matching and Structure from Motion}
\label{sec:3_1}
% Dense matching involves establishing dense correspondences and estimating the 3D geometry between two images.
% EDM \cite{EDM} proposes a dense matching algorithm specifically designed for omnidirectional images by leveraging a spherical coordinate system.
% However, dense features often encounter challenges related to inconsistent multi-view correspondences. To address this, detector-free SfM \cite{he2024detector} merges nearby matches to specific pixel grid positions through match quantization. Subsequently, it performs feature track formation and incremental mapping using triangulation and bundle adjustment.
%
% new version
% Large-scale indoor scenes often present significant challenges for localization due to the presence of numerous occlusions and textureless regions, making it difficult to achieve accurate localization using conventional methods such as COLMAP \cite{schonberger2016structure}. To effectively handle occlusions, we utilize a sphere camera that provides a comprehensive omnidirectional field of view. In addition, to address the prevalent issue of textureless regions, we employ a dense matching algorithm for omnidirectional images, EDM \cite{EDM}, which improves matching performance in such challenging environments. However, dense features derived from image pairs often struggle with inconsistencies in multi-view correspondences. Thus, we employ match quantization \cite{he2024detector}, which merges nearby matches into specific pixel grid positions. Subsequently, we perform feature track formation and incremental mapping using spherical triangulation and bundle adjustment \cite{jiang20243d},
 
Large-scale indoor scenes often present significant challenges for camera localization when using conventional SfM methods such as COLMAP \cite{schonberger2016structure} or OpenMVG \cite{moulon2017openmvg}, due to numerous occlusions and textureless regions.
To effectively reconstruct sparsely captured indoor scenes, we integrate a spherical camera model into every step of SfM pipeline, pipelining spherical dense matching, two-view geometry estimation, and optimization methods in a unified and complete 3D reconstruction framework.
%
Instead of using the traditional pinhole camera model and normalized coordinates, we build our pipeline based on the unit bearing vector representation $u \in \mathcal{S}^2$ and a spherical projection mapping $S$, where $S$ projects as $S: P \rightarrow u, \quad \mathbb{R}^3 \mapsto \mathbb{S}^2$. We provide a description of our SfM core components with unified spherical representations.

\noindent\textbf{Spherical Dense Matching}
In our approach, we borrow EDM \cite{EDM}, which extends Dense Kernelized Matching (DKM) \cite{edstedt2023dkm} with a spherical camera model. DKM formulates the feature matching problem as a form of Gaussian Process Regression (GPR) problem, assuming a mapping $f: \varphi \rightarrow \mathcal{X}$ as a Gaussian process to be estimated, where $\varphi$ and $\mathcal{X}$ denotes feature descriptors and pixel coordinate embeddings, respectively. 
%
Given paired images $A$ and $B$, EDM predicts the posterior Gaussian distribution of spherical coordinate embeddings $\mathcal{U}_A \, | \, \varphi_A, \mathcal{U}_B,  \varphi_B \sim \mathcal{N}(\mu, \Sigma)$ conditioned on the observations $(\mathcal{U}_B, \varphi_B)$ from the image $B$, which is formulated as,
% Extending DKM with spherical representations, EDM predicts the posterior Gaussian distribution of spherical coordinate embeddings $\mathcal{U}_A \, | \, \mathcal{U}_B, \varphi_A, \varphi_B \sim \mathcal{N}(\mu, \Sigma)$ from the observations on image $B$, $(\mathcal{U}_B, \varphi_B)$, which is formulated as,
%
\begin{align}
\mu&= K_{AB} (K_{BB} + \sigma_n^2 I)^{-1} \mathcal{U}_B, \\
\Sigma&= K_{AA} - K_{AB} (K_{BB} + \sigma_n^2 I)^{-1} K_{BA},
\end{align}
where $\mu$ is refined through CNN-based warp refiners \cite{EDM, edstedt2023dkm}.

% Extending DKM with spherical representations, EDM replaces the coordinate embedding $\mathcal{X}$ with spherical coordinate embeddings $\mathcal{U} = E(u)$, where $u$ represent bearing vectors $u \in \mathcal{S}^2$ and $E$ is the encoding from bearing vectors to embeding vectors $U \in \mathbb{R}^c$.

% Given image pairs $A$ and $B$, EDM predicts the posterior Gaussian distribution $\mathcal{U}_A \, | \, \mathcal{U}_B, \varphi_A, \varphi_B \sim \mathcal{N}(\mu, \Sigma)$, which is the conditional Gaussian of positional embeddings $\mathcal{U}_A $ from features $\varphi_A$ conditioned on observation on image $B$, $(\mathcal{U}_B, \varphi_B)$


\noindent\textbf{Spherical Two-view Geometry Estimation}
It can be easily proven that the traditional two-view epipolar constraint for normalized coordinates can be extended to unit bearing vectors on spherical images,
%
\begin{equation}
u_1^T  E  u_2 = 0 \; \text{with} \; E = R [t]_\times.
\end{equation}
%
where $E \in \mathbb{R}^{3\times3}$ is essential matrix built from camera rotation $R \in \text{SO(3)}$, and position $t \in \mathbb{R}^3$. 
% As can be seen from the matrix form, bearing vector $u$ corresponds to the left and right null space of $E$ as it aligns with the direction of the normalized coordinates.
%
%
Solarte et al. \cite{solarte2021robust} provides efficient Singular Value Decomposition (SVD) solutions for spherical two-view geometry $E$, incorporating normalization to improve numerical stability. This approach extends traditional perspective-based 8-point algorithms, which is based on the Direct Linear Transform (DLT) framework \cite{hartley2003multiple}.


\noindent\textbf{Spherical Bundle Adjustment}
By representing each feature track observation as a bearing vector, conventional Bundle Adjustment (BA) can be reformulated in a manner similar to the perspective case, as follows:
%
\begin{equation} \label{eq:textureupdate}
    L = \sum_{i}\sum_{j}\rho(||S(P_j; R_i, t_i) - u_{ij}||^2),
\end{equation}
where $\rho$ denotes the robust Huber norm, and $S$ represents the spherical projection, which maps the 3D scene point $X_j$ to a unit bearing vector, given camera poses $R_j \in \text{SO(3)}$ and $t_j \in \mathbb{R}^3$.




\subsection{Geometric Reconstruction}
\label{sec:3_2}
To address visual localization and mapping in textureless and highly occluded sparsely scanned indoor scenes, we use neural rendering techniques \cite{yu2022monosdf, xiao2024debsdf} for surface reconstruction instead of traditional 3D reconstruction methods such as multi-view stereo (MVS)~\cite{schonberger2016pixelwise}.
Many methods leverage monocular geometric priors~\cite{yu2022monosdf, xiao2024debsdf} and demonstrate higher robustness in textureless and sparsely covered areas. However, 
in terms of ERP images there are no  robust methods for zero-shot monocular depth and normal estimation.
%has yet to be developed, primarily due to the absence of a sufficiently large-scale dataset.
%Initially, we experimented with defining rays on a sphere camera model; however, we observed slower convergence rates and reduced surface quality.
We also obsered  that OmniSDF \cite{kim2024omnisdf}, which learns signed distance functions (SDF) from equirectangular projection (ERP) images, failed to converge when applied to large-scale indoor datasets. 
% We believe that \textcolor{red}{EXPLAIN!} % 이유가 그래도 있으면 좋을거 같긴한데 ... Sparsely scanned scene? 대략의 이라도 
%We believe that representing rays in spherical camera models is inappropriate due to the inherent distortions present in omnidirectional cameras.

To overcome these limitations, we transformed ERP images into cubemaps, estimated depth $D$ and normal maps $N$ from all six perspective images using a pretrained Omnidata model \cite{eftekhar2021omnidata}, and utilized these geometric cues to train neural surface reconstruction effectively.
Following the DebSDF \cite{xiao2024debsdf}, we jointly train two MLPs using the differentiable volumetric rendering, (i) $f_{sdf}$, which represents the scene geometry as a signed distance function, and (ii) $f_{color}$, a color network. 
The training process \cite{xiao2024debsdf} incorporates a combination of losses, including color reconstruction loss $L_{\text{rgb}} = \sum_{r \in R}||\hat{C}(r) - C(r)||_{1}$, Eikonal loss \cite{eikonal} $L_{\text{eikonal}} = \sum_{x \in \chi}(||\triangledown f_{sdf}(x)||_{2} - 1)^2$ , and depth and normal losses,
%
% \begin{align} \label{eq:depthnormalloss}
%     L_{\text{depth}} &= \sum_{r \in R}||(w \hat{D}(r) + q) - D(r)||^2, \\
%     L_{\text{normal}} &= \sum_{r \in R} ||\hat{N}(r) - N(r)||_1 + ||1 - \hat{N}(r)^\top N(r)||_1.
% \end{align}
%
The latter are derived from prior geometric cues by comparing the rendered depth $\hat{D}(t)$ and normals $\hat{N}(t)$ with the corresponding prior depth $D$ and normals $N$. Color image $\hat{C}$ is volumetrically rendered by ray marching $\hat{C} = \sum_{i \in I} \alpha_i \, C_i \, T_i$ along with $\hat{D}$ and $\hat{N}$. We then utilize the learned SDF ($f_{sdf}$ evaluated over a uniform grid) to extract a mesh $M$ using the Marching Cubes algorithm \cite{marchingcube}.
% \textcolor{red}{Loss functions?}

%% + Add Mraching Cube and Mesh Reconstruction Part 

% --------------------------------------------------------

% To overcome these limitations, we transformed ERP images into cubemaps, estimated depth $D$ and normal maps $N$ from all six perspective images using a pretrained Omnidata model \cite{eftekhar2021omnidata}, and utilized these geometric cues to train neural surface reconstruction effectively.
% Following the DebSDF \cite{xiao2024debsdf}, we jointly train two MLPs using the differentiable volumetric rendering, (i) $f_{sdf}$, which represents the scene geometry as a signed distance function, and (ii) $f_{color}$, a color network. 
% The training process \cite{xiao2024debsdf} incorporates a combination of losses, including color reconstruction loss , , and depth and normal losses,

% $\hat{C}$ are volumetrically rendered by ray marching $\hat{C} = \sum_{i \in I} \alpha_i \, C_i \, T_i$
% %

% \begin{equation} \label{eq:depthnormalloss}
% L = L_{\text{color}} + L_{\text{depth}} + L_{\text{normal}}
% + L{\text{eikonal}}
% \end{equation}
% We refer to read (cite) for details for each loss
% % \begin{align} \label{eq:depthnormalloss}
% % L_{\text{color}} &= \sum_{r \in R}||\hat{C}(r) - C(r)||_{1} \\
% %     L_{\text{depth}} &= \sum_{r \in R}||(w \hat{D}(r) + q) - D(r)||^2, \\
% %     L_{\text{normal}} &= \sum_{r \in R} ||\hat{N}(r) - N(r)||_1 + ||1 - \hat{N}(r)^\top N(r)||_1.
% % \end{align}

% Eikonal loss \cite{eikonal} $L_{\text{eikonal}} = \sum_{x \in \chi}(||\triangledown f_{\theta}(x)||_{2} - 1)^2$ is additionally incorporated to preserve implicit constraint of SDF
% To address visual localization and mapping in textureless and highly occluded sparsely scanned indoor scenes, we employ neural rendering techniques for surface reconstruction instead of traditional 3D reconstruction methods such as multi-view stereo (MVS).
In particular, several works leveraging monocular geometric priors \cite{yu2022monosdf, xiao2024debsdf} demonstrate higher robustness in textureless and sparsely covered areas.
Initially, we experimented with defining rays on a sphere camera model; however, we observed slower convergence rates and reduced surface quality.
Similarly, we found that OmniSDF \cite{kim2024omnisdf}, which learns signed distance functions (SDF) from equirectangular projection (ERP) images, failed to converge when applied to large-scale indoor datasets.
To overcome these limitations, we transformed ERP images into cubemaps, estimated depth and normal maps from perspective images using a pretrained Omnidata model \cite{eftekhar2021omnidata}, and utilized these geometric cues to train DebSDF \cite{xiao2024debsdf} effectively.

%% + Add Mraching Cube and Mesh Reconstruction Part 


\subsection{Texture Optimization}
\label{sec:3_3}
%%%%%%%%%%%%%%%%%%%%%%%%%%%%%
% 3D GS 및 Neural Rendering 알고리즘 언급
%%%%%%%%%%%%%%%%%%%%%%%%%%%%%
In this stage, we project textures onto the 3D mesh $M$ generated in the preceding step. In comparison to recent neural rendering methods such as NeRF \cite{mildenhall2021nerf} and 3DGS \cite{kerbl20233d}, it is important to highlight that texture mapping provides a more robust solution for novel view synthesis in sparsely scanned indoor environments.
Utilizing images obtained from cubemap projections along with their corresponding camera poses, we initially employ TexRecon \cite{waechter2014TexRecon} to construct the texture map $T$. However, this conventional texture mapping approach suffers from visible seams between texture patches and is prone to geometric inaccuracies.

Following TMO \cite{choi2023tmo}, we parameterize the texture map and leverage a differentiable rasterizer $\mathcal{R}$ \cite{nvdiffrast} to render an image from a given viewpoint $P$. Unlike TMO \cite{choi2023tmo}, which is limited to representing diffuse textures and cannot capture view-dependent effects, our approach overcomes this limitation. The diffuse texture $K_{d} \in \mathbb{R}^3$ is directly converted into an RGB image. Additionally, we initialize a specular feature $K_{s} \in \mathbb{R}^3$ and employ a small MLP $f_{s}$ as a fragment shader. This MLP takes the specular feature $K_{s}$ and viewing direction $v$ as inputs to compute the specular color as follows: 
\begin{equation} \label{eq:textureupdate}
    \hat{I_{d}} = \mathcal{R}(M, K_{d}, P), ~   \hat{I_{s}} = f_{s}(\mathcal{R}(M, K_{s}, P), v) 
\end{equation}
The final rendered color $\hat{I} \in \mathbb{R}^3$ is obtained by combining the diffuse and specular components, $\hat{I} = \hat{I_{d}} + \hat{I_{s}}$. 
To optimize the diffuse, specular features, and an MLP, we employ a combination of L1 and SSIM \cite{SSIM} as the photometric loss in image space. This loss is calculated between the final rendered image and the ground truth image from the corresponding viewpoint, as follows:
\begin{equation} \label{eq:textureupdate1}
    L_{photo} = (1-\alpha)\parallel\hat{I} - I\parallel + \alpha\:(1 - SSIM(\hat{I}, I))
\end{equation}
where $\alpha$ is set to 0.2 and the weight to balance the losses.


% By measuring the loss between this rendered image and the ground truth image, we can fine-tune the texture map.

\section{Implementation and Performance}
\subsection{Experiment Settings}
\paragraph{Matterport3D Dataset}
The Matterport3D dataset \cite{chang2017matterport3d} consists of 90 indoor scenes encompassing a total of 10,800 panoramic images. Following the EDM approach \cite{EDM}, we define the ground truth camera poses of these images. From the validation and test sets of the official benchmark split, we selected three scenes each, ensuring that these scenes are entirely indoors or composed of levels no higher than the second floor.
Since dense matching requires overlapped image pairs as input,
we assume that during 360-degree camera scanning, the information about which room each camera belongs to is known.
To utilize this information effectively, we leveraged the $house\_segmentation$ labels available in the Matterport3D dataset. Images taken within the same room were selected as matching pairs, while for transitions between different rooms, only one image pair with the highest overlap was linked.
For EDM \cite{EDM}, the ERP images were resized to $640 \times 320$, whereas images for other feature matching methods were resized to $1024 \times 512$.

% \paragraph{OmniBlender Dataset}
% OmniBlender \cite{choi2023balanced} is a photorealistic synthetic dataset comprising 11 distinct scenes characterized by circular camera motion. For our experiments, we selected four scenes that represent indoor environments. Since this dataset is captured sequentially within a single space, matching pairs were created by pairing a reference image with its two adjacent images. Similar to the Matterport3D experiments, ERP images were resized to $640 \times 320$ for EDM, while images for other feature matching methods were resized to $1024 \times 512$.
\paragraph{Stanford2D3D Datatset}
The Stanford2D3D dataset \cite{Stanford2d3d} comprises scenes from six large-scale indoor environments collected from three different buildings. 
We utilize this dataset to demonstrate that our spherical SfM method outperforms existing SfM approaches in terms of registration performance and accurate pose estimation. 
Specifically, we selected three scenes containing a moderate number of images. 
Assuming that overlapping image pairs were predefined during the scanning process, we determined covisibility and defined matching pairs using geometric meshes and ground truth poses. 
Additionally, we resized the equirectangular projection (ERP) images following the same methodology employed in the Matterport3D dataset.


\paragraph{Implementation Details}
We construct the Spherical Structure from Motion based on SphereSfM \cite{jiang20243d} and COLMAP \cite{schonberger2016structure}. To address challenges in textureless or occluded regions, we leverage the dense matching algorithm EDM \cite{EDM} for enhanced feature extraction. Following the DebSDF \cite{xiao2024debsdf}, the SDF and color networks are implemented using an 8-layer MLP and a 2-layer MLP, respectively, each with a hidden dimension of 256. The networks are optimized for geometric reconstruction using the Adam optimizer \cite{kingma2014adam}, starting with an initial learning rate of $5 \times 10^{-4}$ and applying exponential decay at each iteration. Input images are resized to $384 \times 384$, and geometric cues are provided by the pretrained Omnidata model \cite{eftekhar2021omnidata}. In Texture Optimization, the specular features are initialized to zero values, matching the resolution of the diffuse texture image. The small MLP consists of 2 layers with 32 hidden dimensions. We employ the Adam optimizer \cite{kingma2014adam} with a learning rate of $5 \times 10^{-4}$ and utilize a learning rate scheduler. The model is trained for 7,000 iterations. We render images at a resolution of $512 \times 512$.

% \textcolor{red}{TODO: Add a description for the texturing process.}
%%%% TODO: Add a description for the texturing process.

\begin{figure}[t]
    \centering    
    \includegraphics[width=1.0\linewidth]{figures/figno_intro_JH.pdf}
    \caption{
    We provide a visual comparison between sparse matching perspective SfM and our proposed dense matching spherical SfM. 
    In (a), ERP images are converted into a cubemap representation, after which feature matching is performed across all 36 possible image pairs, resulting in sparse and noisy correspondence matches. (b) demonstrates our approach, which directly finds dense and accurate correspondences on ERP images, thereby facilitating the construction of a detailed 3D structure.
    }
    \label{fig:sfm}
\end{figure}


\subsection{Experimental Results}
\paragraph{Camera Pose Estimation}
% TODO : colmap 은 rig 로 돌린거 강조해야함.
Accurate geometric 3D reconstruction depends on precise camera pose estimation. 
We compare our proposed method, which leverages dense matching \cite{EDM} and Spherical SfM, against four different methods:
1) \textbf{OpenMVG:} An open-source SfM pipeline that supports spherical camera models \cite{moulon2017openmvg}.
2) \textbf{SPSG COLMAP:} This approach employs SuperPoint \cite{detone2018superpoint} and SuperGlue \cite{sarlin2020superglue} for feature detection and matching on perspective images. ERP images are transformed into a cubemap representation, yielding six perspective views per ERP image. Feature matching is performed in a brute-force manner across all 36 possible image pairs. Subsequently, incremental triangulation and bundle adjustment are performed under the rig constraints for each 360 camera node.
3) \textbf{DKM COLMAP:} This method leverages DKM \cite{edstedt2023dkm} to establish dense correspondences without relying on explicit feature extraction. Since DKM is designed for perspective images, it also follows the cubemap projection and exhaustive pairwise matching, and rig-based optimization.
4) \textbf{SphereGlue COLMAP:} Building on COLMAP, SphereSfM \cite{jiang20243d} incorporates spherical camera models and spherical bundle adjustment to handle ERP images. For feature extraction and matching, SuperPoint \cite{detone2018superpoint} with a local planar approximation \cite{eder2020tangent} and SphereGlue \cite{gava2023sphereglue} are utilized to mitigate distortion in ERP images.
To estimate initial poses for incremental triangulation, we follow the normalization strategy and non-linear optimization proposed by Solarte et al. \cite{solarte2021robust}.

Table \ref{table:mp3d} presents the quantitative results of camera pose estimation on the Matterport3D dataset. 
While previous methods also performed well in smaller scenes with significant image overlap, our proposed method demonstrated superior robustness across diverse scenes. 
By leveraging a wide field of view (FoV) for matching, our approach proved particularly effective on datasets with sparse input views. 
Notably, it achieved the most accurate registrations in scenes with frequent occlusions. 
Figure \ref{fig:sfm_result} visualizes the estimated camera poses and triangulated 3D points, demonstrating that our method estimated both the largest number of 3D points and the most camera poses.


\begin{figure}[t]
    \centering
    \includegraphics[width=1.0\linewidth]{figures/fig4_exp.pdf}
    \caption{\textbf{Qualitative Comparison of SfM results on Matterport3D.} By leveraging dense features on equirectangular projection (ERP) images, our method effectively finds correspondences in textureless regions. }
    %which builds accurate track information and leads to a denser 3D structure.}
    \label{fig:sfm_result}
\end{figure}




%% table
% \begin{table}[t]
%     \begin{center}
%     \caption{\label{table:mp3d_stfd}QUANTITATIVE COMPARISONS WITH RECENT ALGORITHMS}
%     \resizebox{1.0\linewidth}{!}{
%     \begin{tabular}{l|c|c|c|ccc}
%         \toprule
%         \multirow{2}{*}{Method} &\multirow{2}{*}{Dataset} &\multirow{2}{*}{Image} &\multirow{2}{*}{Feature} &\multicolumn{3}{c}{AUC} \\
%         & & & & @5\textdegree & @10\textdegree & @20\textdegree \\
%         \midrule
%         SPHORB \cite{zhao2015sphorb} &Matterport3D &ERP &sparse &0.38 &1.41 &3.99 \\
%         SphereGlue \cite{gava2023sphereglue} &Matterport3D &ERP &sparse &11.29 &19.95 &31.10 \\
%         \midrule
%         DKM \cite{edstedt2023dkm} &Matterport3D &perspective &dense &18.43 &28.50 &38.44 \\
%         RoMa \cite{edstedt2023roma} &Matterport3D &perspective &dense &12.45 &22.37 &34.24 \\
%         \midrule
%         \textbf{EDM (ours)} &Matterport3D &ERP &dense &\textbf{45.15} &\textbf{60.99} &\textbf{73.60} \\
%         \midrule\midrule
%         SPHORB \cite{zhao2015sphorb} &Stanford2D3D &ERP &sparse &0.14 &1.01 &4.08 \\
%         SphereGlue \cite{gava2023sphereglue} &Stanford2D3D &ERP &sparse &11.25 &22.41 &36.57 \\
%         \midrule
%         DKM \cite{edstedt2023dkm} &Stanford2D3D &perspective &dense &12.46 &22.18 &34.13 \\
%         RoMa \cite{edstedt2023roma} &Stanford2D3D &perspective &dense &11.48 &22.52 &37.07 \\
%         \midrule
%         \textbf{EDM (ours)} &Stanford2D3D &ERP &dense &\bf{55.08} &\bf{71.65} &\bf{82.72} \\
%         \bottomrule
%     \end{tabular}
%     }
%     \end{center}
%     EDM improve AUC@5\textdegree\ by 26.72 on Matterport3D and by 42.62 on Stanford2D3D.
% \end{table}
\begin{table*}[t]
    \begin{center}
    \resizebox{0.95\linewidth}{!}{
    \begin{tabular}{l|cccc cccc cccc}
        \toprule
        \multirow{2}{*}{Method} & \multicolumn{4}{c}{2t7WUuJeko7} & \multicolumn{4}{c}{8194nk5LbLH} & \multicolumn{4}{c}{pLe4wQe7qrG} \\
        \cmidrule{2-5}\cmidrule(lr){6-9}\cmidrule(lr){10-13}
        & \# Registered & AUC @3\textdegree & AUC @5\textdegree & AUC @10\textdegree & \# Registered & AUC @3\textdegree & AUC @5\textdegree & AUC @10\textdegree & \# Registered & AUC @3\textdegree & AUC @5\textdegree & AUC @10\textdegree \\
        \midrule
        OpenMVG & 5 / 37 & 0.82 & 1.10 & 1.30 &2 / 20 & 0.27&0.37 &0.45 &25 / 31 & 45.11 &52.82 &58.67 \\
        SPSG COLMAP & 16 / 37 & 14.30 & 15.79 & 16.90 & 12 / 20 & 26.64 & 29.88 & 32.31 & \textbf{31 / 31} & 75.02 & 85.01 & 92.51 \\
        DKM COLMAP & 22 / 37& 25.14 & 27.70 & 29.62 & 9 / 20 & 14.64 & 16.36 & 17.66 & \textbf{31 / 31} & \textbf{75.94} & \textbf{85.56} & \textbf{92.78} \\
        SphereGlue COLMAP & 21 / 37 & 23.95 & 26.98 & 29.26 & 12 / 20 & 23.11  & 27.76 & 31.24 & \textbf{31 / 31} & 66.83 & 80.06 & 90.03 \\
        IM360 (Ours) & \textbf{37 / 37} & \textbf{49.16} & \textbf{69.05} & \textbf{84.53} & \textbf{20 / 20} & \textbf{34.91} & \textbf{44.16} & \textbf{66.87} & \textbf{31 / 31} & 73.95 & 84.37 & 92.18 \\
        \midrule
        \multirow{2}{*}{Method} & \multicolumn{4}{c}{RPmz2sHmrrY} & \multicolumn{4}{c}{YVUC4YcDtcY} & \multicolumn{4}{c}{zsNo4HB9uLZ} \\
        \cmidrule{2-5}\cmidrule(lr){6-9}\cmidrule(lr){10-13}
        & \# Registered & AUC @3\textdegree & AUC @5\textdegree & AUC @10\textdegree & \# Registered & AUC @3\textdegree & AUC @5\textdegree & AUC @10\textdegree & \# Registered & AUC @3\textdegree & AUC @5\textdegree & AUC @10\textdegree \\
        \midrule
        OpenMVG & 22 / 59 & 9.86 &11.31 &12.41 &8 / 46 &2.22 &2.41 &2.56 &2 / 53 &-- &-- &-- \\
        SPSG COLMAP & 34 / 59 & 22.93 & 26.87 & 29.83 & \textbf{46 / 46} & \textbf{84.21} & \textbf{90.53} & \textbf{95.26} & 19 / 53 & 7.84 & 9.22 & 10.55 \\
        DKM COLMAP & 51 / 59 & 41.56 & 53.56 & 62.58 & \textbf{46 / 46} & 56.33 & 61.00 & 64.67 & 19 / 53 & 7.87 & 8.71 & 9.41 \\
        SphereGlue COLMAP & 34 / 59 & 22.23 & 26.45 & 29.62 & \textbf{46 / 46} & 78.32 & 86.97 & 93.48 & 22 / 53 & 6.25 & 7.56 & 9.46 \\
        IM360 (Ours) & \textbf{59 / 59} & \textbf{53.44} & \textbf{71.87} & \textbf{85.93} & \textbf{46 / 46} & 72.40 & 83.40 & 91.71 & \textbf{53 / 53} & \textbf{52.35} & \textbf{71.14} & \textbf{85.49} \\
        \bottomrule
    \end{tabular}
    }
    \end{center}
    \vspace*{-4mm}
    \caption{\label{table:mp3d} \textbf{Quantitative Evaluation of Camera Localization Performance} on the Matterport3D dataset \cite{chang2017matterport3d}. Our method demonstrates superior results in terms of registration accuracy and achieves comparable performance in pose estimation across most scenes.}
\end{table*}

\begin{table*}[t]
    \begin{center}
    \resizebox{0.95\linewidth}{!}{
    \begin{tabular}{l|cccc cccc cccc}
        \toprule
        \multirow{2}{*}{Method} & \multicolumn{4}{c}{area 3} & \multicolumn{4}{c}{area 4} & \multicolumn{4}{c}{area 5a} \\
        \cmidrule{2-5}\cmidrule(lr){6-9}\cmidrule(lr){10-13}
        & \# Registered & AUC @3\textdegree & AUC @5\textdegree & AUC @10\textdegree & \# Registered & AUC @3\textdegree & AUC @5\textdegree & AUC @10\textdegree & \# Registered & AUC @3\textdegree & AUC @5\textdegree & AUC @10\textdegree \\
        \midrule
        OpenMVG & 6 / 85 & 0.35 &0.38 &0.40 & 17 / 258 &0.34 &0.37 &0.39 &  8 / 143 &0.23 &0.25 &0.26 \\
        SPSG COLMAP & 28 / 85 & 6.71 & 7.47 & 8.29 & 73 / 258 & 4.75 & 5.77 & 6.55 & 54 / 143 & 8.84 & 10.93 & 12.51 \\
        DKM COLMAP & 45 / 85& 13.10 & 14.92 & 16.32 & 79 / 258 & 3.39 & 3.96 & 4.40 & 78 / 143 & 15.27 & 18.71 & 21.38  \\
        SphereGlue COLMAP & 30 / 85 & 5.87 & 7.38 & 8.75 & 116 / 258 & 8.62  & 11.64 & 14.13 & 69 / 143 & 5.10 & 8.06 &  11.73\\
        IM360 (Ours) & \textbf{85 / 85} & \textbf{32.20} & \textbf{45.12} & \textbf{58.23} & \textbf{258 / 258} & \textbf{28.70} & \textbf{50.50} & \textbf{71.29} & \textbf{138 / 143} & \textbf{19.52} & \textbf{32.54} & \textbf{51.72}  \\
        \bottomrule
    \end{tabular}
    }
    \end{center}
    \vspace*{-4mm}
    \caption{\label{table:stfd} \textbf{Quantitative evaluation of camera Localization Performance} on the Stanford2D3D dataset \cite{Stanford2d3d}. Our method outperforms all other approaches in terms of registration accuracy and pose estimation.}
\end{table*}

% \begin{table*}[t]
%     \begin{center}
%     \resizebox{1.0\linewidth}{!}{
%     \begin{tabular}{l|cccc cccc cccc}
%         \toprule
%         % \multirow{2}{*}{Method} & \multicolumn{4}{c}{2t7WUuJeko7} & \multicolumn{4}{c}{8194nk5LbLH} & \multicolumn{4}{c}{pLe4wQe7qrG} \\
%         \multirow{2}{*}{Method} & \multicolumn{12}{c}{\# Registered \quad AUC @3\textdegree \quad AUC @5\textdegree \quad AUC @10\textdegree} \\
%         & \multicolumn{4}{c}{2t7WUuJeko7} & \multicolumn{4}{c}{8194nk5LbLH} & \multicolumn{4}{c}{pLe4wQe7qrG} \\
%         % \cmidrule{2-5}\cmidrule(lr){6-9}\cmidrule(lr){10-13}
%         % & \# Registered & AUC @3\textdegree & AUC @5\textdegree & AUC @10\textdegree & \# Registered & AUC @3\textdegree & AUC @5\textdegree & AUC @10\textdegree & \# Registered & AUC @3\textdegree & AUC @5\textdegree & AUC @10\textdegree \\
%         \midrule
%         OpenMVG & & & & & & & & & & & & \\
%         SPSG COLMAP & 16 / 37 & 14.30 & 15.79 & 16.90 & 12 / 20 & 26.64 & 29.88 & 32.31 & 31 / 31 & 75.02 & 85.01 & 92.51 \\
%         DKM COLMAP & 22 / 37& 25.14 & 27.70 & 29.62 & 9 / 20 & 14.64 & 16.36 & 17.66 & 31 / 31 & 75.94 & 85.56 & 92.78 \\
%         SphereGlue COLMAP & 21 / 37 & 23.95 & 26.98 & 29.26 & 12 / 20 & 23.11 & 27.76 & 31.24 & 31 / 31 & 66.83 & 80.06 & 90.03 \\
%         Ours & 37 / 37 & 49.16 & 69.05 & 84.53 & 20 / 20 & 34.91 & 44.16 & 66.87 & 31 / 31 & 73.95 & 84.37 & 92.18 \\
%         \midrule
%         Method & \multicolumn{4}{c}{RPmz2sHmrrY} & \multicolumn{4}{c}{YVUC4YcDtcY} & \multicolumn{4}{c}{zsNo4HB9uLZ} \\
%         % \cmidrule{2-5}\cmidrule(lr){6-9}\cmidrule(lr){10-13}
%         % & \# Registered & AUC @3\textdegree & AUC @5\textdegree & AUC @10\textdegree & \# Registered & AUC @3\textdegree & AUC @5\textdegree & AUC @10\textdegree & \# Registered & AUC @3\textdegree & AUC @5\textdegree & AUC @10\textdegree \\
%         \midrule
%         OpenMVG & & & & & & & & & & & & \\
%         SPSG COLMAP & 34 / 59 & 22.93 & 26.87 & 29.83 & 46 / 46 & 84.21 & 90.53 & 95.26 & 19 / 53 & 7.84 & 9.22 & 10.55 \\
%         DKM COLMAP & 51 / 59 & 41.56 & 53.56 & 62.58 & 46 / 46 & 56.33 & 61.00 & 64.67 & 19 / 53 & 7.87 & 8.71 & 9.41 \\
%         SphereGlue COLMAP & 34 / 59 & 22.23 & 26.45 & 29.62 & 46 / 46 & 78.32 & 86.97 & 93.48 & 22 / 53 & 6.25 & 7.56 & 9.46 \\
%         Ours & 59 / 59 & 53.44 & 71.87 & 85.93 & 46 / 46 & 72.40 & 83.40 & 91.71 & 53 / 53 & 52.35 & 71.14 & 85.49 \\
%         \bottomrule
%     \end{tabular}
%     }
%     \end{center}
%     \caption{\label{table:mp3d}Quantitative comparisons with recent algorithms}
% \end{table*}



\begin{table*}[t]
    \centering
    \begin{adjustbox}{width=0.95\linewidth,center}
    \begin{tabular}{l|c|ccccccc}
    \toprule
    \multirow{2}{*}{Method}&\multirow{2}{*}{Rendering}&\multicolumn{7}{c}{PSNR \(\uparrow\)  / SSIM \(\uparrow\)  / LPIPS \(\downarrow\) } \\
    & & 2t7WUuJeko7 & 8194nk5LbLH & pLe4wQe7qrG & RPmz2sHmrrY & YVUC4YcDtcY & zsNo4HB9uLZ & Mean \\
    \midrule
    ZipNeRF \cite{barron2023zip} & Volume &  15.1 / 0.51 / 0.69 & 11.9 / 0.44 / 0.74 & 14.1 / 0.43 / 0.70 & 13.1 / 0.50 / 0.69 & 15.3 / 0.53 / 0.68 & 14.1 / 0.65 / 0.68 & 13.9 / 0.51 / 0.68 \\
    3DGS \cite{kerbl20233d} & Splat & 14.4 / 0.47 / 0.53 & 11.9 / 0.38 / 0.60 & 13.5 / 0.37 / 0.56 & 12.9 / 0.51 / 0.55 & 14.2 / 0.49 / 0.52 & 13.7 / 0.60 / 0.50 & 13.4 / 0.47 / 0.55 \\
    SparseGS \cite{xiong2023sparsegs} & Splat & 15.4 / 0.48 / 0.50 & 12.8 / 0.36 / 0.58 & 13.8 / 0.37 / 0.52 & 13.6 / 0.46 / 0.57 & 15.5 / 0.50 / 0.49 & 14.4 / 0.58 / 0.55 & 14.3 / 0.46 / 0.53 \\
    TexRecon \cite{waechter2014TexRecon} & Mesh & 17.1 / 0.51 / 0.43 & 14.2 / 0.47 / 0.49 & 15.9 / 0.45 / 0.46 & 16.7 / 0.60 / 0.38 & 15.0 / 0.59 / 0.42 & 17.0 / 0.51 / 0.42 & 15.9 / 0.54 / 0.43 \\
    \midrule
    IM360* (Ours) & Mesh & 19.6 / 0.60 / 0.37 & 16.7 / 0.60 / 0.43 & 18.5 / 0.65 / 0.36 & 18.9 / 0.65 / 0.38 & 18.3 / 0.60 / 0.42 & 19.8 / 0.68 / 0.41 & 18.6 / 0.63 / 0.39 \\
    IM360 (Ours) & Mesh & \textbf{19.8} / \textbf{0.61} / \textbf{0.38} & \textbf{16.9} / \textbf{0.61} / \textbf{0.43} & \textbf{18.8} / \textbf{0.66} / \textbf{0.35} & \textbf{20.6} / \textbf{0.75} / \textbf{0.32} & \textbf{19.4} / \textbf{0.66} / \textbf{0.32} & \textbf{21.0} / \textbf{0.74} / \textbf{0.37} & \textbf{19.4} / \textbf{0.67} / \textbf{0.37} \\
    \bottomrule
    \end{tabular}
    \end{adjustbox}    
    \caption{\textbf{Quantitative Comparison of Rendering Performance} on the Matterport3D Dataset. Our method shows higher rendering quality of our method, IM360,across all scenes, achieving a 3.5 PSNR over the TexRecon \cite{waechter2014TexRecon}. The best-performing algorithms for each metric are shown in boldface. 
    }
    \label{table:mp3d-rendering-comparion}
\end{table*}



% \begin{table}[t]
%     \centering
%     \begin{adjustbox}{width=\linewidth,center}
%     \begin{tabular}{l|c|ccccc}
%     \toprule
%     \multirow{2}{*}{Method}&\multirow{2}{*}{Rendering}&\multicolumn{5}{c}{PSNR \(\uparrow\)  / SSIM \(\uparrow\)  / LPIPS \(\downarrow\) } \\
%     & & archiviz & barbershop & classroom & restroom & Mean \\
%     \midrule
%     ZipNeRF \cite{barron2023zip} & Volume &  11.1 / 1.11 / 1.11 & 11.1 / 1.11 / 1.11 & 11.1 / 1.11 / 1.11 & 11.1 / 1.11 / 1.11 & 11.1 / 1.11 / 1.11 \\
%     3DGS \cite{kerbl20233d} & Splat & 11.1 / 1.11 / 1.11 & 11.1 / 1.11 / 1.11 & 11.1 / 1.11 / 1.11 & 11.1 / 1.11 / 1.11  \\
%     SparseGS \cite{xiong2023sparsegs} & Splat & 11.1 / 1.11 / 1.11 & 11.1 / 1.11 / 1.11 & 11.1 / 1.11 / 1.11 & 11.1 / 1.11 / 1.11  \\
%     TexRecon \cite{waechter2014TexRecon} & Mesh & 11.1 / 1.11 / 1.11 & 11.1 / 1.11 / 1.11 & 11.1 / 1.11 / 1.11 & 11.1 / 1.11 / 1.11  \\
%     \midrule
%     Ours & Mesh & 11.1 / 1.11 / 1.11 & 11.1 / 1.11 / 1.11 & 11.1 / 1.11 / 1.11 & 11.1 / 1.11 / 1.11  \\
%     Ours & Mesh & 11.1 / 1.11 / 1.11 & 11.1 / 1.11 / 1.11 & 11.1 / 1.11 / 1.11 & 11.1 / 1.11 / 1.11  \\
%     \bottomrule
%     \end{tabular}
%     \end{adjustbox}    
%     \caption{
%     }
%     \label{table:omniblender-psnr-comparion}
% \end{table}



\paragraph{Surface Reconstruction}
Recently, neural surface reconstruction methods based on volumetric rendering techniques \cite{wang2021neus, yu2022monosdf, xiao2024debsdf} have shown strong performance in dense 3D reconstruction. 
In particular, several approaches that leverage monocular geometry priors \cite{yu2022monosdf, xiao2024debsdf} demonstrate robustness in handling large textureless regions and sparse input views. 
Therefore, we adopt DebSDF \cite{xiao2024debsdf} to estimate the signed distance function and apply the marching cubes algorithm \cite{marchingcube} to convert the neural implicit representation into triangle meshes. Figure \ref{fig:geometry} presents shading meshes extracted without texture, the most effective way to visualize geometric differences. This highlights the effectiveness of our SfM method in producing decent geometric meshes from omnidirectional data.  

\begin{figure}[t]
    \centering
    \includegraphics[width=1.0\linewidth]{figures/fig6_geometry.pdf}
    \caption{
    We visualize the shading meshes extracted without texture on Matterport3D. Leveraging the estimated poses from IM360, we achieve accurate surface reconstruction, demonstrating the effectiveness of our SfM method in generating decent geometric meshes from omnidirectional data.}
    \label{fig:geometry}
\end{figure}



\paragraph{Texture Map Optimization}
We now evaluate the results of our view synthesis method by comparing it against four different approaches: 1) \textbf{ZipNeRF} \cite{barron2023zip}, the state-of-the-art neural radiance field method; 2) \textbf{3DGS} \cite{kerbl20233d}, which employs 3D Gaussian splats rendered through rasterization; 3) \textbf{SparseGS} \cite{xiong2023sparsegs}, designed to address sparse-view challenges in 3D Gaussian splatting; and 4) \textbf{TexRecon} \cite{waechter2014TexRecon}, a classical texture mapping method for large-scale scenes. Table \ref{table:mp3d-rendering-comparion} reports the rendering quality with the PSNR, SSIM \cite{SSIM}, and LPIPS \cite{LPIPS} on Matterport3D dataset. We consistently observe that neural rendering methods, including NeRF \cite{barron2023zip} and 3DGS \cite{kerbl20233d}, perform inferior to traditional texture reconstruction techniques \cite{waechter2014TexRecon}. The significant discrepancy between training PSNR and novel view PSNR indicates that this gap arises due to the sparsity of the scanned images. To address this, we evaluated the state-of-the-art SparseGS \cite{xiong2023sparsegs} approach, specifically designed for sparse-view scenarios. While SparseGS \cite{xiong2023sparsegs} achieves a 1 PSNR improvement in rendering quality, this gain indicates that sparse scanning remains a significant challenge. Our method achieves improved rendering quality across all scenes, demonstrating a 3.5 PSNR increase over the TexRecon \cite{waechter2014TexRecon} method.

In Fig. \ref{fig:rendering-comparions-matterport}, we visually compare our method to several recent rendering approaches: TexRecon \cite{waechter2014TexRecon}, SparseGS \cite{xiong2023sparsegs}, and ZipNeRF \cite{barron2023zip}. We visualize only SparseGS instead of 3DGS, as both methods employ similar rendering techniques, but SparseGS demonstrates superior rendering quality in sparsely scanned indoor environments. 
TexRecon shows the visual seams between face texture and color misalignment.  ZipNeRF tends to generate ambiguous pixels, resulting in severe artifacts. SparseGS contains obvious floating artifacts in ground and floor regions. In contrast, our method produces much higher quality textures and consistently demonstrate improvement across various scenes. 

\subsection{Ablation Study}
We observed that when initial poses are determined through two-view geometry without utilizing the method proposed by Solarte et al.~\cite{solarte2021robust}, the registration performance significantly deteriorates, resulting in fragmented SfM models. Furthermore, as shown in Table \ref{table:mp3d} and \ref{table:stfd}, by modifying the feature matching in COLMAP-based methods and comparing the registration and pose accuracy, we confirmed that our approach demonstrates superior robustness.
% \textcolor{red}{sfm ablation study}
% robust 360 8pa 유무 -> registration 차이
% dense feature 방식인데, 기존 spherical sfm 방식들은 indoor 에서 작동 X. feature 적인 특징..
In table \ref{table:mp3d-rendering-comparion}, IM360* only finetunes diffuse texture without utilizing specular features. Without texture optimization stage outlined in Sec \ref{sec:3_3}, our rendering quality is equivalent to that of TexRecon \cite{waechter2014TexRecon}. Although we apply texture optimization soley to the diffuse texture map, our method achieves a substantial improvement in rendering quality, with a 2.7 PSNR increase. Additionally, our final approach, which combines both diffuse and specular colors, provides a further 0.8 PSNR gain by accounting for view-dependent components.      





\begin{figure*}[h]
    \centering
    \includegraphics[width=0.97\linewidth]{figures/render_comparison.pdf}
    % \vspace{-2mm}
    \caption{\textbf{Qualitative Comparisons with Existing Methods.} We visually compare our proposed approach with TexRecon \cite{waechter2014TexRecon}, SparseGS \cite{xiong2023sparsegs}, and ZipNeRF \cite{barron2023zip}. Our method can render high frequency details and results in lower noise.}
    % \vspace{-5mm}
    \label{fig:rendering-comparions-matterport}
\end{figure*}


\section{Conclusion, Limitations and Future Work}
% Limitations of Texturing 
% Mesh 의 geometric artifacts laed to rendering error and texture artifacts

We introduce a novel framework for textured mesh reconstruction for omnidirectional cameras in sparsely scanned indoor environments. To address the challenges associated with textureless regions and sparse viewpoints, we propose a spherical Structure-from-Motion (SfM) that integrates spherical camera models into all the core steps of SfM. Additionally, we reconstruct texture maps and leverage differentiable rendering to enhance its quality, improving the rendering performance in sparse-view scenarios. However, our approach has certain limitations.
% \textcolor{red}{Limitation of SfM}.
First, the performance of dense matching depends on correct matching pairs. Therefore, if we lack an appropriate image retrieval technique, we must rely on covisibility annotations during the scanning process.
Furthermore, our texture mapping method is based on a mesh rasterization approach, making the rendering quality highly sensitive to the accuracy of the underlying mesh geometry. Consequently, any inaccuracies in the mesh can result in rendering artifacts. In the future, we may explore ERP image retrieval methods to extract covisible image pairs and the joint optimization of geometry and texture to refine geometric artifacts.

{
    \small
    \bibliographystyle{ieeenat_fullname}
    \bibliography{main}
}

% WARNING: do not forget to delete the supplementary pages from your submission 
% \clearpage
\pagenumbering{gobble}
\maketitlesupplementary

\section{Additional Results on Embodied Tasks}

To evaluate the broader applicability of our EgoAgent's learned representation beyond video-conditioned 3D human motion prediction, we test its ability to improve visual policy learning for embodiments other than the human skeleton.
Following the methodology in~\cite{majumdar2023we}, we conduct experiments on the TriFinger benchmark~\cite{wuthrich2020trifinger}, which involves a three-finger robot performing two tasks: reach cube and move cube. 
We freeze the pretrained representations and use a 3-layer MLP as the policy network, training each task with 100 demonstrations.

\begin{table}[h]
\centering
\caption{Success rate (\%) on the TriFinger benchmark, where each model's pretrained representation is fixed, and additional linear layers are trained as the policy network.}
\label{tab:trifinger}
\resizebox{\linewidth}{!}{%
\begin{tabular}{llcc}
\toprule
Methods       & Training Dataset & Reach Cube & Move Cube \\
\midrule
DINO~\cite{caron2021emerging}         & WT Venice        & 78.03     & 47.42     \\
DoRA~\cite{venkataramanan2023imagenet}          & WT Venice        & 81.62     & 53.76     \\
DoRA~\cite{venkataramanan2023imagenet}          & WT All           & 82.40     & 48.13     \\
\midrule
EgoAgent-300M & WT+Ego-Exo4D      & 82.61    & 54.21      \\
EgoAgent-1B   & WT+Ego-Exo4D      & \textbf{85.72}      & \textbf{57.66}   \\
\bottomrule
\end{tabular}%
}
\end{table}

As shown in Table~\ref{tab:trifinger}, EgoAgent achieves the highest success rates on both tasks, outperforming the best models from DoRA~\cite{venkataramanan2023imagenet} with increases of +3.32\% and +3.9\% respectively.
This result shows that by incorporating human action prediction into the learning process, EgoAgent demonstrates the ability to learn more effective representations that benefit both image classification and embodied manipulation tasks.
This highlights the potential of leveraging human-centric motion data to bridge the gap between visual understanding and actionable policy learning.



\section{Additional Results on Egocentric Future State Prediction}

In this section, we provide additional qualitative results on the egocentric future state prediction task. Additionally, we describe our approach to finetune video diffusion model on the Ego-Exo4D dataset~\cite{grauman2024ego} and generate future video frames conditioned on initial frames as shown in Figure~\ref{fig:opensora_finetune}.

\begin{figure}[b]
    \centering
    \includegraphics[width=\linewidth]{figures/opensora_finetune.pdf}
    \caption{Comparison of OpenSora V1.1 first-frame-conditioned video generation results before and after finetuning on Ego-Exo4D. Fine-tuning enhances temporal consistency, but the predicted pixel-space future states still exhibit errors, such as inaccuracies in the basketball's trajectory.}
    \label{fig:opensora_finetune}
\end{figure}

\subsection{Visualizations and Comparisons}

More visualizations of our method, DoRA, and OpenSora in different scenes (as shown in Figure~\ref{fig:supp pred}). For OpenSora, when predicting the states of $t_k$, we use all the ground truth frames from $t_{0}$ to $t_{k-1}$ as conditions. As OpenSora takes only past observations as input and neglects human motion, it performs well only when the human has relatively small motions (see top cases in Figure~\ref{fig:supp pred}), but can not adjust to large movements of the human body or quick viewpoint changes (see bottom cases in Figure~\ref{fig:supp pred}).

\begin{figure*}
    \centering
    \includegraphics[width=\linewidth]{figures/supp_pred.pdf}
    \caption{Retrieval and generation results for egocentric future state prediction. Correct and wrong retrieval images are marked with green and red boundaries, respectively.}
    \label{fig:supp pred}
\end{figure*}

\begin{figure*}[t]
    \centering
    \includegraphics[width=0.9\linewidth]{figures/motion_prediction.pdf}
    \vspace{-0.5mm}
    \caption{Motion prediction results in scenes with minor changes in observation.}
    \vspace{-1.5mm}
    \label{fig:motion_prediction}
\end{figure*}

\subsection{Finetuning OpenSora on Ego-Exo4D}

OpenSora V1.1~\cite{opensora}, initially trained on internet videos and images, produces severely inconsistent results when directly applied to infer future videos on the Ego-Exo4D dataset, as illustrated in Figure~\ref{fig:opensora_finetune}.
To address the gap between general internet content and egocentric video data, we fine-tune the official checkpoint on the Ego-Exo4D training set for 50 epochs.
OpenSora V1.1 proposed a random mask strategy during training to enable video generation by image and video conditioning. We adopted the default masking rate, which applies: 75\% with no masking, 2.5\% with random masking of 1 frame to 1/4 of the total frames, 2.5\% with masking at either the beginning or the end for 1 frame to 1/4 of the total frames, and 5\% with random masking spanning 1 frame to 1/4 of the total frames at both the beginning and the end.

As shown in Fig.~\ref{fig:opensora_finetune}, despite being trained on a large dataset, OpenSora struggles to generalize to the Ego-Exo4D dataset, producing future video frames with minimal consistency relative to the conditioning frame. While fine-tuning improves temporal consistency, the moving trajectories of objects like the basketball and soccer ball still deviate from realistic physical laws. Compared with our feature space prediction results, this suggests that training world models in a reconstructive latent space is more challenging than training them in a feature space.


\section{Additional Results on 3D Human Motion Prediction}

We present additional qualitative results for the 3D human motion prediction task, highlighting a particularly challenging scenario where egocentric observations exhibit minimal variation. This scenario poses significant difficulties for video-conditioned motion prediction, as the model must effectively capture and interpret subtle changes. As demonstrated in Fig.~\ref{fig:motion_prediction}, EgoAgent successfully generates accurate predictions that closely align with the ground truth motion, showcasing its ability to handle fine-grained temporal dynamics and nuanced contextual cues.

\section{OpenSora for Image Classification}

In this section, we detail the process of extracting features from OpenSora V1.1~\cite{opensora} (without fine-tuning) for an image classification task. Following the approach of~\cite{xiang2023denoising}, we leverage the insight that diffusion models can be interpreted as multi-level denoising autoencoders. These models inherently learn linearly separable representations within their intermediate layers, without relying on auxiliary encoders. The quality of the extracted features depends on both the layer depth and the noise level applied during extraction.


\begin{table}[h]
\centering
\caption{$k$-NN evaluation results of OpenSora V1.1 features from different layer depths and noising scales on ImageNet-100. Top1 and Top5 accuracy (\%) are reported.}
\label{tab:opensora-knn}
\resizebox{0.95\linewidth}{!}{%
\begin{tabular}{lcccccc}
\toprule
\multirow{2}{*}{Timesteps} & \multicolumn{2}{c}{First Layer} & \multicolumn{2}{c}{Middle Layer} & \multicolumn{2}{c}{Last Layer} \\
\cmidrule(r){2-3}   \cmidrule(r){4-5}  \cmidrule(r){6-7}  & Top1           & Top5           & Top1            & Top5           & Top1           & Top5          \\
\midrule
32        &  6.10           & 18.20             & 34.04               & 59.50             & 30.40             & 55.74             \\
64        & 6.12              & 18.48              & 36.04               & 61.84              & 31.80         & 57.06         \\
128       & 5.84             & 18.14             & 38.08               & 64.16              & 33.44       & 58.42 \\
256       & 5.60             & 16.58              & 30.34               & 56.38              &28.14          & 52.32        \\
512       & 3.66              & 11.70            & 6.24              & 17.62              & 7.24              & 19.44  \\ 
\bottomrule
\end{tabular}%
}
\end{table}

As shown in Table~\ref{tab:opensora-knn}, we first evaluate $k$-NN classification performance on the ImageNet-100 dataset using three intermediate layers and five different noise scales. We find that a noise timestep of 128 yields the best results, with the middle and last layers performing significantly better than the first layer.
We then test this optimal configuration on ImageNet-1K and find that the last layer with 128 noising timesteps achieves the best classification accuracy.

\section{Data Preprocess}
For egocentric video sequences, we utilize videos from the Ego-Exo4D~\cite{grauman2024ego} and WT~\cite{venkataramanan2023imagenet} datasets.
The original resolution of Ego-Exo4D videos is 1408×1408, captured at 30 fps. We sample one frame every five frames and use the original resolution to crop local views (224×224) for computing the self-supervised representation loss. For computing the prediction and action loss, the videos are downsampled to 224×224 resolution.
WT primarily consists of 4K videos (3840×2160) recorded at 60 or 30 fps. Similar to Ego-Exo4D, we use the original resolution and downsample the frame rate to 6 fps for representation loss computation.
As Ego-Exo4D employs fisheye cameras, we undistort the images to a pinhole camera model using the official Project Aria Tools to align them with the WT videos.

For motion sequences, the Ego-Exo4D dataset provides synchronized 3D motion annotations and camera extrinsic parameters for various tasks and scenes. While some annotations are manually labeled, others are automatically generated using 3D motion estimation algorithms from multiple exocentric views. To maximize data utility and maintain high-quality annotations, manual labels are prioritized wherever available, and automated annotations are used only when manual labels are absent.
Each pose is converted into the egocentric camera's coordinate system using transformation matrices derived from the camera extrinsics. These matrices also enable the computation of trajectory vectors for each frame in a sequence. Beyond the x, y, z coordinates, a visibility dimension is appended to account for keypoints invisible to all exocentric views. Finally, a sliding window approach segments sequences into fixed-size windows to serve as input for the model. Note that we do not downsample the frame rate of 3D motions.

\section{Training Details}
\subsection{Architecture Configurations}
In Table~\ref{tab:arch}, we provide detailed architecture configurations for EgoAgent following the scaling-up strategy of InternLM~\cite{team2023internlm}. To ensure the generalization, we do not modify the internal modules in InternML, \emph{i.e.}, we adopt the RMSNorm and 1D RoPE. We show that, without specific modules designed for vision tasks, EgoAgent can perform well on vision and action tasks.

\begin{table}[ht]
  \centering
  \caption{Architecture configurations of EgoAgent.}
  \resizebox{0.8\linewidth}{!}{%
    \begin{tabular}{lcc}
    \toprule
          & EgoAgent-300M & EgoAgent-1B \\
          \midrule
    Depth & 22    & 22 \\
    Embedding dim & 1024  & 2048 \\
    Number of heads & 8     & 16 \\
    MLP ratio &    8/3   & 8/3 \\
    $\#$param.  & 284M & 1.13B \\
    \bottomrule
    \end{tabular}%
    }
  \label{tab:arch}%
\end{table}%

Table~\ref{tab:io_structure} presents the detailed configuration of the embedding and prediction modules in EgoAgent, including the image projector ($\text{Proj}_i$), representation head/state prediction head ($\text{MLP}_i$), action projector ($\text{Proj}_a$) and action prediction head ($\text{MLP}_a$).
Note that the representation head and the state prediction head share the same architecture but have distinct weights.

\begin{table}[t]
\centering
\caption{Architecture of the embedding ($\text{Proj}_i$, $\text{Proj}_a$) and prediction ($\text{MLP}_i$, $\text{MLP}_a$) modules in EgoAgent. For details on module connections and functions, please refer to Fig.~2 in the main paper.}
\label{tab:io_structure}
\resizebox{\linewidth}{!}{%
\begin{tabular}{lcl}
\toprule
       & \multicolumn{1}{c}{Norm \& Activation} & \multicolumn{1}{c}{Output Shape}  \\
\midrule
\multicolumn{3}{l}{$\text{Proj}_i$ (\textit{Image projector})} \\
\midrule
Input image  & -          & 3$\times$224$\times$224 \\
Conv 2D (16$\times$16) & -       & Embedding dim$\times$14$\times$14    \\
\midrule
\multicolumn{3}{l}{$\text{MLP}_i$ (\textit{State prediction head} \& \textit{Representation head)}} \\
\midrule
Input embedding  & -          & Embedding dim \\
Linear & GELU       & 2048          \\
Linear & GELU       & 2048          \\
Linear & -          & 256           \\
Linear & -          & 65536     \\
\midrule
\multicolumn{3}{l}{$\text{Proj}_a$ (\textit{Action projector})} \\
\midrule
Input pose sequence  & -          & 4$\times$5$\times$17 \\
Conv 2D (5$\times$17) & LN, GELU   & Embedding dim$\times$1$\times$1    \\
\midrule
\multicolumn{3}{l}{$\text{MLP}_a$ (\textit{Action prediction head})} \\
\midrule
Input embedding  & -          & Embedding dim$\times$1$\times$1 \\
Linear & -          & 4$\times$5$\times$17     \\
\bottomrule
\end{tabular}%
}
\end{table}


\subsection{Training Configurations}
In Table~\ref{tab:training hyper}, we provide the detailed training hyper-parameters for experiments in the main manuscripts.

\begin{table}[ht]
  \centering
  \caption{Hyper-parameters for training EgoAgent.}
  \resizebox{0.86\linewidth}{!}{%
    \begin{tabular}{lc}
    \toprule
    Training Configuration & EgoAgent-300M/1B \\
    \midrule
    Training recipe: &  \\
    optimizer & AdamW~\cite{loshchilov2017decoupled} \\
    optimizer momentum & $\beta_1=0.9, \beta_2=0.999$ \\
    \midrule
    Learning hyper-parameters: &  \\
    base learning rate & 6.0E-04 \\
    learning rate schedule & cosine \\
    base weight decay & 0.04 \\
    end weight decay & 0.4 \\
    batch size & 1920 \\
    training iters & 72,000 \\
    lr warmup iters & 1,800 \\
    warmup schedule & linear \\
    gradient clip & 1.0 \\
    data type & float16 \\
    norm epsilon & 1.0E-06 \\
    \midrule
    EMA hyper-parameters: &  \\
    momentum & 0.996 \\
    \bottomrule
    \end{tabular}%
    }
  \label{tab:training hyper}%
\end{table}%

\clearpage


\end{document}
