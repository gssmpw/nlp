% sfm 설명
% sfm 예시들 colmap, pixelsfm, theiasfm, spehresfm, etc
% challenges
% feature matching (especially dense matching)
% dense feature track 연결? 이거 내 contribution 아니니까 대충 얘기하거나 넘어가도 될 듯.
The 3D geometry of a scene is essential for a wide range of computer vision applications and has been a long-standing challenge in the field.
General Structure-from-Motion (SfM) software such as COLMAP \cite{schonberger2016structure}, TheiaSfM \cite{sweeney2015theia}, and OpenMVG \cite{moulon2017openmvg} have significantly advanced the capabilities of 3D reconstruction.

Indoor 3D mapping typically relies on LiDAR sensors or dense video scans to achieve accurate reconstruction. 
However, these requirements pose significant challenges when dealing with sparsely scanned indoor scenes captured solely by visual sensors. 
The primary issue lies in the prevalence of large textureless regions, which degrade the accuracy of both feature detection and matching, thereby impairing the overall performance of the SfM pipeline. 
Additionally, frequent occlusions in indoor environments, combined with sparse input views, make the feature matching process even more challenging. 
In such scenarios, conventional perspective cameras further worsen the problem due to their limited field of view, which reduces the overlap between views. 
This lack of overlap complicates the estimation of accurate camera poses, making the reconstruction task considerably more difficult.

Recently, detector-free matchers \cite{sun2021loftr,melekhov2019dgc,truong2020glu,truong2021learning,edstedt2023dkm,edstedt2024roma} have emerged as a promising alternative to detector-based approaches \cite{detone2018superpoint,revaud2019r2d2,tyszkiewicz2020disk}, particularly in handling repetitive or indiscriminate regions where keypoint detection performance tends to degrade.
However, detector-free matchers face issues with multi-view inconsistency in subpixel correspondences due to their pairwise dependency \cite{he2024detector}.
To mitigate this, Detector-Free SfM \cite{he2024detector} proposes a strategy that quantizes feature locations to form tracks across views, enhancing its compatibility with existing SfM systems.

\begin{figure}[t]
    \centering
    % \vspace*{-5mm}
    \includegraphics[width=1.\linewidth]{figures/fig1_mp3d.pdf}
    % \vspace*{-2mm}
    \caption{blue sphere wireframes indicate camera location. sparse, hard,}
    \label{fig:introduction}
\end{figure}


\paragraph{Main Results}
In this paper, we present a novel 3D indoor reconstruction pipeline designed for omnidirectional images, addressing challenges that conventional SfM approaches fail to overcome.
We tackle the problem of extensive textureless regions by employing detector-free matching methods, which improve matching accuracy without the need for explicit feature detection. 
To mitigate the frequent occlusions encountered in sparsely scanned images, we utilize a spherical camera model with a wide field of view, increasing the overlap between images and enhancing robustness in multiview matching.
To the best of our knowledge, this is the first work to present dense matching and SfM tailored for the spherical camera model.
Our framework also extends to mesh reconstruction and texturing, providing a comprehensive solution for indoor 3D reconstruction. 
The proposed method demonstrates significant improvements over traditional approaches, particularly in challenging indoor environments with sparse and textureless scenes.
The main contributions of this paper are summarized as follows:
\begin{itemize}
    \item We present a unified framework to reconstruct the textured mesh for sparsely scanned large scale indoor scenes.
    \item the first dense matching \& spherical sfm
    \item texture refine
\end{itemize}
