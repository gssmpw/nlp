\begin{figure}[h]
    \centering
    % \vspace*{-5mm}
    \includegraphics[width=1.0\linewidth]{figures/fig2_pipeline.pdf}
    % \vspace*{-2mm}
    \caption{Overview of Our Pipeline. It is comprised of three steps: Spherical Structure from Motion (Sec. \ref{sec:3_1}), Geometry Reconstruction (Sec. \ref{sec:3_2}), and Texture Optimization (Sec. \ref{sec:3_3}). Our method }
    \label{fig:pipeline}
\end{figure}
Given a set of 360-degree equirectangular projection (ERP) images, our objective is to reconstruct the geometry and texture of real-world scenes (refer to Figure X). 
Our pipeline is composed of three primary components: Structure from Motion (SfM) using a spherical camera model, geometry reconstruction, and texture optimization.

To accurately estimate camera poses from sparsely scanned images, we perform SfM utilizing a spherical camera model. 
This approach leverages the wide field of view inherent in spherical imagery, addressing challenges associated with limited overlap and frequent occlusions in indoor environments.

For 3D geometric reconstruction, we employ a neural model that represents the zero-level set of a signed distance function (SDF). 
After training the model, we extract the mesh using the Marching Cubes algorithm. 
Initially, we attempted to define rays using the spherical camera model for training but observed a decline in reconstruction quality. 
Similarly, OmniSDF did not yield accurate results. 
Consequently, we converted ERP images into cubemaps and conducted training on the resulting perspective images to improve performance.

For texture map optimization, we leverage differentiable rasterization techniques by integrating Multi-View Stereo Texturing (MVS-Texturing) and Gaussian Splatting. 
MVS-Texturing is particularly effective in sparse scenes where details are not prominently visible, while Gaussian Splatting excels in densely scanned scenes. 
By potentially combining these two methods, we aim to capitalize on their respective strengths. 
Additionally, the use of differentiable rendering allows us to reduce noise in areas where seams occur, enhancing overall texture quality.