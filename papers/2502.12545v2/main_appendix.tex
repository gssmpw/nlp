% CVPR 2025 Paper Template; see https://github.com/cvpr-org/author-kit

\documentclass[10pt,twocolumn,letterpaper]{article}

%%%%%%%%% PAPER TYPE  - PLEASE UPDATE FOR FINAL VERSION
% \usepackage{cvpr}              % To produce the CAMERA-READY version
% \usepackage[review]{cvpr}      % To produce the REVIEW version
\usepackage[pagenumbers]{cvpr} % To force page numbers, e.g. for an arXiv version

% Import additional packages in the preamble file, before hyperref
%
% --- inline annotations
%
\newcommand{\red}[1]{{\color{red}#1}}
\newcommand{\todo}[1]{{\color{red}#1}}
\newcommand{\TODO}[1]{\textbf{\color{red}[TODO: #1]}}
% --- disable by uncommenting  
% \renewcommand{\TODO}[1]{}
% \renewcommand{\todo}[1]{#1}



\newcommand{\VLM}{LVLM\xspace} 
\newcommand{\ours}{PeKit\xspace}
\newcommand{\yollava}{Yo’LLaVA\xspace}

\newcommand{\thisismy}{This-Is-My-Img\xspace}
\newcommand{\myparagraph}[1]{\noindent\textbf{#1}}
\newcommand{\vdoro}[1]{{\color[rgb]{0.4, 0.18, 0.78} {[V] #1}}}
% --- disable by uncommenting  
% \renewcommand{\TODO}[1]{}
% \renewcommand{\todo}[1]{#1}
\usepackage{slashbox}
% Vectors
\newcommand{\bB}{\mathcal{B}}
\newcommand{\bw}{\mathbf{w}}
\newcommand{\bs}{\mathbf{s}}
\newcommand{\bo}{\mathbf{o}}
\newcommand{\bn}{\mathbf{n}}
\newcommand{\bc}{\mathbf{c}}
\newcommand{\bp}{\mathbf{p}}
\newcommand{\bS}{\mathbf{S}}
\newcommand{\bk}{\mathbf{k}}
\newcommand{\bmu}{\boldsymbol{\mu}}
\newcommand{\bx}{\mathbf{x}}
\newcommand{\bg}{\mathbf{g}}
\newcommand{\be}{\mathbf{e}}
\newcommand{\bX}{\mathbf{X}}
\newcommand{\by}{\mathbf{y}}
\newcommand{\bv}{\mathbf{v}}
\newcommand{\bz}{\mathbf{z}}
\newcommand{\bq}{\mathbf{q}}
\newcommand{\bff}{\mathbf{f}}
\newcommand{\bu}{\mathbf{u}}
\newcommand{\bh}{\mathbf{h}}
\newcommand{\bb}{\mathbf{b}}

\newcommand{\rone}{\textcolor{green}{R1}}
\newcommand{\rtwo}{\textcolor{orange}{R2}}
\newcommand{\rthree}{\textcolor{red}{R3}}
\usepackage{amsmath}
%\usepackage{arydshln}
\DeclareMathOperator{\similarity}{sim}
\DeclareMathOperator{\AvgPool}{AvgPool}

\newcommand{\argmax}{\mathop{\mathrm{argmax}}}     



% It is strongly recommended to use hyperref, especially for the review version.
% hyperref with option pagebackref eases the reviewers' job.
% Please disable hyperref *only* if you encounter grave issues, 
% e.g. with the file validation for the camera-ready version.
%
% If you comment hyperref and then uncomment it, you should delete *.aux before re-running LaTeX.
% (Or just hit 'q' on the first LaTeX run, let it finish, and you should be clear).
\definecolor{cvprblue}{rgb}{0.21,0.49,0.74}
\usepackage[pagebackref,breaklinks,colorlinks,allcolors=cvprblue]{hyperref}

% DK
\usepackage{xcolor}
\usepackage{multirow}
% JH
\usepackage{adjustbox}

%%%%%%%%% PAPER ID  - PLEASE UPDATE
\def\paperID{16035} % *** Enter the Paper ID here
\def\confName{CVPR}
\def\confYear{2025}

%%%%%%%%% TITLE - PLEASE UPDATE
% \title{\LaTeX\ Author Guidelines for \confName~Proceedings}
% \title{360 Indoor Reconstruction}
\title{IM360: Textured Mesh Reconstruction for Large-scale Indoor Mapping with 360\textdegree\ Cameras}

%%%%%%%%% AUTHORS - PLEASE UPDATE
\author{First Author\\
Institution1\\
Institution1 address\\
{\tt\small firstauthor@i1.org}
% For a paper whose authors are all at the same institution,
% omit the following lines up until the closing ``}''.
% Additional authors and addresses can be added with ``\and'',
% just like the second author.
% To save space, use either the email address or home page, not both
\and
Second Author\\
Institution2\\
First line of institution2 address\\
{\tt\small secondauthor@i2.org}
}

\begin{document}

\clearpage
\setcounter{page}{1}
\maketitlesupplementary

In this supplementary material, we provide additional qualitative results, highlighting the superior performance of our method compared to other approaches.
Appendix \ref{sphericalsfm} highlights the advantages of the spherical camera model and the dense matching algorithm for indoor reconstruction. Appendix \ref{texture} demonstrates the effectiveness of our novel texturing method.

\section{Spherical Structure from Motion}
\label{sphericalsfm}
Due to page limitations, we present our spherical structure from motion results in the supplementary material: 1) \textbf{OpenMVG:} An open-source SfM pipeline that supports spherical camera models \cite{moulon2017openmvg}.
2) \textbf{SPSG COLMAP:} SuperPoint \cite{detone2018superpoint} and SuperGlue \cite{sarlin2020superglue} are used with cubemap and equirectangular projection.
3) \textbf{DKM COLMAP:} This method leverages DKM \cite{edstedt2023dkm} to establish dense correspondences, utilizing cubemap and equirectangular projection.
4) \textbf{SphereGlue COLMAP:} SuperPoint \cite{detone2018superpoint} with a local planar approximation \cite{eder2020tangent} and SphereGlue \cite{gava2023sphereglue} are utilized to mitigate distortion in ERP images.
The experimental results discussed in the main paper for Matterport3D \cite{chang2017matterport3d} and Stanford2D3D \cite{Stanford2d3d} are shown in Fig. \ref{fig:sfm_mp3d} and Fig. \ref{fig:sfm_stfd}, respectively.

\section{Texture Map Optimization}
\label{texture}
We compare our method with several recent rendering approaches, including \textbf{TexRecon} \cite{waechter2014TexRecon}, \textbf{SparseGS} \cite{xiong2023sparsegs}, and \textbf{ZipNeRF} \cite{barron2023zip}. 
Our method outperforms these approaches by delivering higher frequency details and producing seamless texture maps.
The results of the textured mesh and rendering are shown in Fig. \ref{fig:textured_mesh} and Fig. \ref{fig:render1} - \ref{fig:render4}.

% \section{Rationale}
% \label{sec:rationale}
% % 
% Having the supplementary compiled together with the main paper means that:
% % 
% \begin{itemize}
% \item The supplementary can back-reference sections of the main paper, for example, we can refer to \cref{sec:intro};
% \item The main paper can forward reference sub-sections within the supplementary explicitly (e.g. referring to a particular experiment); 
% \item When submitted to arXiv, the supplementary will already included at the end of the paper.
% \end{itemize}
% % 
% To split the supplementary pages from the main paper, you can use \href{https://support.apple.com/en-ca/guide/preview/prvw11793/mac#:~:text=Delete%20a%20page%20from%20a,or%20choose%20Edit%20%3E%20Delete).}{Preview (on macOS)}, \href{https://www.adobe.com/acrobat/how-to/delete-pages-from-pdf.html#:~:text=Choose%20%E2%80%9CTools%E2%80%9D%20%3E%20%E2%80%9COrganize,or%20pages%20from%20the%20file.}{Adobe Acrobat} (on all OSs), as well as \href{https://superuser.com/questions/517986/is-it-possible-to-delete-some-pages-of-a-pdf-document}{command line tools}.

\begin{figure*}[t]
    \centering
    \includegraphics[width=1.0\linewidth]{figures_sup/sup_sfm.pdf}
    \caption{Qualitative Comparison of SfM results on Matterport3D.
    While other approaches failed to achieve pose registration, our method successfully estimates poses by leveraging the spherical camera model and dense matching.
    }
    \label{fig:sfm_mp3d}
\end{figure*}

\begin{figure*}[t]
    \centering
    \includegraphics[width=1.0\linewidth]{figures_sup/sup_sfm_stfd.pdf}
    \caption{Qualitative Comparison of SfM results on Stanford2D3D.
    While other approaches failed to achieve pose registration, our method successfully estimates poses by leveraging the spherical camera model and dense matching.}
    \label{fig:sfm_stfd}
\end{figure*}

\begin{figure*}[t]
    \centering
    \includegraphics[width=1.0\linewidth]{figures_sup/texture_map.pdf}
    \caption{Qualitative Comparisons of Textured Mesh Results on Matterport3D.
    A comparison between TexRecon \cite{waechter2014TexRecon} and ours shows that our method effectively reduces noise in the texture maps, leading to improved visual quality and detail.}
    \label{fig:textured_mesh}
\end{figure*}


% % \maketitle
% % \clearpage
% \setcounter{page}{1}
% \maketitlesupplementary
\begin{figure*}[t]
    \centering
    \includegraphics[width=1.0\linewidth]{figures_sup/render_comparison_supp1.pdf}
    \caption{Qualitative Comparisons with Existing Methods. Our method can render high frequency details and results in lower noise.}
    \label{fig:render1}
\end{figure*}

\begin{figure*}[t]
    \centering
    \includegraphics[width=1.0\linewidth]{figures_sup/render_comparison_supp2.pdf}
    \caption{Qualitative Comparisons with Existing Methods. Our method can render high frequency details and results in lower noise.}
    \label{fig:render2}
\end{figure*}

\begin{figure*}[t]
    \centering
    \includegraphics[width=1.0\linewidth]{figures_sup/render_comparison_supp3.pdf}
    \caption{Qualitative Comparisons with Existing Methods. Our method can render high frequency details and results in lower noise.}
    \label{fig:render3}
\end{figure*}

\begin{figure*}[t]
    \centering
    \includegraphics[width=1.0\linewidth]{figures_sup/render_comparison_supp4.pdf}
    \caption{Qualitative Comparisons with Existing Methods. Our method can render high frequency details and results in lower noise.}
    \label{fig:render4}
\end{figure*}



{
    \small
    \bibliographystyle{ieeenat_fullname}
    \bibliography{main}
}


\end{document}