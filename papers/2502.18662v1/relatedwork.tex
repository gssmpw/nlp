\section{Related Work}
\label{relatedwork}

Among the refactoring actions indicated in the literature are team restructuring, defining and/or restructuring communication plans, and mentoring team members \cite{catolino2020, sarmento2022, tahsin2022}. However, despite the presence of these refactoring strategies, several challenges remain. For instance, Caballero-Espinosa, Carver, and Stowers~\cite{caballero2023} highlighted in a systematic literature review the existence of Community Smells without indication of refactoring strategies, strategies with unproven or partially proven effectiveness, and a lack of details about the strategies indicated in the literature.

This study sought to identify strategies defined in the retrospective meetings for refactoring Community Smells. However, due to the limitation of the Community Smells addressed, it does not present details of the execution or effectiveness of the strategies, results that would contribute to the gaps identified by Caballero-Espinosa, Carver, and Stowers\cite{caballero2023}.
Other studies reveal the effects of agile software development practices on social factors within teams, highlighting, for example, that retrospective meetings are associated with work engagement and trust among team members \cite{law2005, mchugh2012, muller2021}. Frequent and open communication within the team, knowledge sharing, and obtaining feedback are factors that impact trust among members and are related to the practice of retrospective meetings \cite{mchugh2012}.

The importance of retrospectives as a tool for continuous improvement is evident in agile development contexts. A longitudinal study examined retrospective meeting reports from teams in Large-Scale Agile Development to identify problems and improvement actions, aiming to understand how retrospectives are conducted and how improvement actions could be enhanced\cite{dingsoyr2018}. The research identified 109 points, of which 65 were areas where the team could make improvements. However, while many of these points were categorized under ``People and Relationships'', this category received limited attention in the final analysis because the authors evaluated theses as less related to how the large-scale agile development methodology influences teams.

Similarly, Martini, Stray, and Moe~\cite{martini2019} conducted a study to characterize the issues identified in retrospective meetings of large-scale agile development teams. The case study consisted of interviews, observations, and document analysis to investigate what teams discuss in intra-team and inter-team retrospective meetings concerning technical debt, social debt, and process debt. The reported results show that most of the problems in intra-team retrospectives are related to social debt, which suggests the need for further research.
In this context, retrospective meetings are highlighted as a data source for investigating different aspects of software development. Unlike such works \cite{martini2019, law2005, mchugh2012, muller2021, dingsoyr2018}, this study has a specific focus on the practice of retrospective meetings and their role in refactoring social debt, particularly in the circumstances known as Community Smells, showing how these smells are highlighted, what strategies are defined and which monitoring and prevention are carried out through the retrospective meetings.