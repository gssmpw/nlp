\documentclass{article}


\usepackage{arxiv}
% \usepackage[utf8]{inputenc} % allow utf-8 input
\usepackage[T1]{fontenc}    % use 8-bit T1 fonts
\usepackage{amsmath, amssymb, amsfonts}
\usepackage{newtxtext,newtxmath} % Times font with small caps and math support
\usepackage{hyperref}       % hyperlinks
\usepackage{url}            % simple URL typesetting
\usepackage{booktabs}       % professional-quality tables
\usepackage{nicefrac}       % compact symbols for 1/2, etc.
\usepackage{microtype}      % microtypography
\usepackage{graphicx}
\usepackage{doi}
\usepackage{bm}
\usepackage{multirow}
\usepackage{todonotes}
\usepackage{subcaption}
\usepackage[backend=biber,style=apa]{biblatex}
\addbibresource{ref.bib}

\title{OOD Detection with \textbf{immature} Models}
\author{
  Behrooz Montazeran \\
  Computer Vision and Learning Lab \\
  University of Heidelberg \\
  \texttt{behrooz.montazeran@stud.uni-heidelberg.de} \\
  \And
  Ullrich Köthe \\
  Interdisciplinary Center for Scientific Computing \\
  University of Heidelberg \\
  \texttt{ullrich.koethe@iwr.uni-heidelberg.de} \\
}

\date{} % Remove date

\begin{document}
\maketitle

\begin{abstract}
    Likelihood-based deep generative models (DGMs) have gained significant attention for their ability to approximate the distributions of high-dimensional data. However, these models lack a performance guarantee in assigning higher likelihood values to in-distribution (ID) inputs—data the models are trained on—compared to out-of-distribution (OOD) inputs. This counter-intuitive behaviour is particularly pronounced when ID inputs are more complex than OOD data points. One potential approach to address this challenge involves leveraging the gradient of a data point with respect to the parameters of the DGMs. A recent OOD detection framework proposed estimating the joint density of layer-wise gradient norms for a given data point as a model-agnostic method, demonstrating superior performance compared to the Typicality Test across likelihood-based DGMs and image dataset pairs. In particular, most existing methods presuppose access to fully converged models, the training of which is both time-intensive and computationally demanding. In this work, we demonstrate that using immature models—stopped at early stages of training—can mostly achieve equivalent or even superior results on this downstream task compared to mature models capable of generating high-quality samples that closely resemble ID data. This novel finding enhances our understanding of how DGMs learn the distribution of ID data and highlights the potential of leveraging partially trained models for downstream tasks. Furthermore, we offer a possible explanation for this unexpected behaviour through the concept of support overlap. The source code for the implementation is available in GitHub \emph{repository}\footnote{\url{https://github.com/BehroozMontazeran/ood_detection}}.
\end{abstract}

\section{Introduction}
\label{sec:Introduction}
The application of machine learning models in high-stakes scenarios, where safety and reliability are paramount, has sparked significant debate (\cite{goodman2017european,cluzeau2020concepts,zhang2021understanding}), particularly in domains such as finance (\cite{chen2024generalized}), autonomous driving (\cite{voronin2024enhancing}), and medical diagnostics (\cite{varoquaux2022machine,su2024machine}). These models, despite their strong performance on data similar to their training distribution, often exhibit overly confident or erroneous behaviour when encountering out-of-distribution (OOD) inputs (\cite{wei2022mitigating, tony2012isolation}).
In this work, we explore OOD detection using likelihood-based deep generative models (DGMs) in an unsupervised method, especially flow-based models (\cite{kingma2018glow}), which are designed to estimate the underlying probability density of the observed data. Density estimation techniques do not assume the existence of an anomaly distribution at training time. These models, trained with objectives such as maximum likelihood, aim to maximize the likelihood of the training data, inherently normalizing the probability densities across the data distribution. This normalization suggests that OOD samples should theoretically yield lower likelihoods compared to in-distribution data (\cite{bishop1994novelty}). Notable likelihood-based DGMs, including normalizing flows (NFs) (\cite{papamakarios2021normalizing}), diffusion models (DMs) (\cite{sohl-dickstein2015deep, ho2020denoising}), variational autoencoders (VAEs) (\cite{kingma2014autoencoding,rezende2014stochastic}), autoregressive models (ARMs) (\cite{oord2016wavenet,oord2016pixel}), and energy-based models (EBMs) (\cite{lecun2006tutorial,xie2016theory}), have demonstrated remarkable capabilities, particularly in generating high-quality and highly realistic images. Building on this success, a natural approach for OOD detection is to assess the likelihood of a given sample under the trained model and classify it based on a threshold (\cite{ren2019likelihood, nalisnick2019detecting, kamkari2024geometric, xiao2020likelihood}).
However, despite their theoretical appeal, recent studies reveal a surprising inconsistency in this approach (see Fig.\ref{fig:nnl}, showing performance inconsistencies). These models, which approximate data log-likelihoods or their surrogates, often assign higher likelihoods to OOD samples than to those from the training distribution (\cite{choi2019waic,nalisnick2019deep}), especially when ID inputs are more complex than the OOD sample points. This paradoxical phenomenon, observed across various probabilistic DGMs, highlights a fundamental challenge in using likelihood as a standalone metric for OOD detection. Recent studies (\cite{choi2021robust, dauncey2023on, dauncey2024approximations}) proposed using layer-wise gradient norms as a more effective indicator for OOD detection. This approach is based on the intuitive observation that OOD data typically produce larger gradient norms compared to ID data when the model undergoes a single-step back-propagation, starting with parameters optimized on the ID dataset, as illustrated in (Fig.\ref{fig:layerwise_scores}). In essence, this method evaluates how much meaningful signal can be derived solely from the gradients, offering a fresh perspective on OOD detection.


\begin{figure}[ht!]
    \centering
    \begin{subfigure}[t]{0.49\textwidth}
        \centering
        \includegraphics[width=\textwidth]{images/NNL_cifar_model_svhn.png}
    \end{subfigure}
    \hfill
    \begin{subfigure}[t]{0.49\textwidth}
        \centering
        \includegraphics[width=\textwidth]{images/NNL_celeba_model_svhn.png}
    \end{subfigure}
    \caption{\textbf{\emph{Visualization of the anomalous behaviour in density-based generative models (GLOW)}}\\[0.5em]Despite training the model on (left) CIFAR-10 and (right) CelebA as in-distribution (ID) datasets, the model assigns higher likelihoods (higher negative bits per dimension values) to OOD samples from SVHN. This surprising observation is especially pronounced when the complexity of the ID dataset is higher than that of the OOD dataset, highlighting a key limitation of likelihood-based OOD detection in deep generative models. Further comparisons of this phenomenon are illustrated in \S~\ref{sec:NLL_as_OOD}}
    \label{fig:nnl}
\end{figure}



Authors (\cite{choi2021robust,dauncey2024approximations}), along with studies such as (\cite{hendrycks2018deep, nalisnick2019detecting, xiao2020likelihood,havtorn2021hierarchical, bergamin2022model, Haroush2021statistical}) and more operate under the assumption that the parameters of a trained model are optimal, implying that fully trained models are inherently superior for downstream tasks. However, \textbf{in this contribution} we challenge this assumption by demonstrating that a layer-wise score-based OOD detection method achieves superior results when applied to partially trained models. Specifically, our findings suggest that fully optimized models may not always be the best choice for tasks like OOD detection, especially when the ID data points are less complex than the OODs samples.

To substantiate our claim, we utilized the GLOW model \cite{kingma2018glow}, evaluating its performance at different stages of training. The results, presented in Tables \ref{tab:auc_results_three_channel} and \ref{tab:auc_results_one_channel}, reveal that full convergence is unnecessary when the objective is limited to OOD detection. In fact, partially trained models not only match but mostly exceed the performance of fully trained models, especially when the OOD samples are semantically dissimilar to the in-distribution data.


We attribute this phenomenon to the evolving gap (or overlap) between the support of ID and OOD data as training progresses. Specifically, when a score-based method is used to detect OOD samples, the gap between the histograms of layer-wise gradient norms of samples under the parameters of DGMs widens progressively during training, which can impose an additional computational burden on such downstream tasks. Conversely, in some cases, this gap transitions to an overlap, or an existing overlap becomes more pronounced at the later stages of convergence. When likelihood alone, or in combination with other measures, is used as an OOD score, partial training reduces the extent of this overlap, thereby enhancing the model's ability to distinguish OOD samples. These phenomena, illustrated in \S~\ref{sec:partial_vs_full}, highlight the potential benefits of partial training. This insight offers a fresh perspective on OOD detection, emphasizing the value of leveraging partially trained models for efficient and effective performance in specialized tasks.



\begin{figure}[ht!]
    \centering
    \begin{subfigure}[t]{0.49\textwidth}
        \centering
        \includegraphics[width=\textwidth]{images/histogram_imagenet32.png}
    \end{subfigure}
    \hfill
    \begin{subfigure}[t]{0.49\textwidth}
        \centering
        \includegraphics[width=\textwidth]{images/histogram_celeba.png}
    \end{subfigure}
    \caption{\textbf{\emph{Layer-wise gradient-based OOD scoring effectively separates ID and OOD samples.}}\\[0.5em]The GLOW model was trained on two ID datasets: (left) ImageNet32 and (right) CelebA, while tested against four OOD datasets from : SVHN, GTSRB, CIFAR-10, CelebA and ImageNet32. The gradient values are demonstrating variability across layers and the distinct separation between ID and OOD data distributions. The scoring function 
    $S_{\bm{\theta}^{(l)}}(\bm{x}_{b=\{1,5\}}) = \log \big\{ \big\| \nabla_{\bm{\theta}^{(l)}} \big( \sum_b \ell(\bm{x}_b) \big) \big\|_2^2 \big\}$
    is computed using $b=5$, indicating that each score is calculated using a batch of five random samples. The near-perfect separation observed between ID and OOD samples highlights the effectiveness of this method. Additional results with varying batch sizes (e.g., $b=1$ and $b=5$) and other IDs are detailed in \S~\ref{sec:Additional_Results}.}
    \label{fig:layerwise_scores}
\end{figure}






\section{Methods to detect OOD data (Related work)}
\subsection{Density Scoring Methods}
\label{sec:density_scoring_methods}
This section provides an in-depth examination of recent advancements in OOD detection methodologies, focusing on their underlying principles, strengths, and limitations. First, we explore likelihood-based scoring methods, tracing their evolution, and analysing their primary bottlenecks, including their dependence on accurate data representation and susceptibility to misleading likelihood estimates for OOD samples. Following this, we delve into gradient-based methods, presenting them as a promising alternative. These methods leverage layer-wise gradient information to address the challenges inherent in likelihood-based approaches, offering improved robustness and interpretability in OOD detection.



A foundational strategy for OOD detection, applicable in both labelled and unlabelled settings, involves learning a density model \(\mathcal{M}\) that approximates the true distribution \(p^* (\mathcal{X}_{\text{ID}})\) of training inputs \(x_{\text{ID}} \in \mathcal{X}_{\text{ID}}\) (\cite{bishop1994novelty}). The underlying assumption is that if the approximation is sufficiently accurate, such that \(p(x_{\text{ID}} | \mathcal{M}) \approx p^* (x_{\text{ID}})\), then OOD inputs should yield low likelihood scores under \(\mathcal{M}\). While this approach was historically considered infeasible for high-dimensional data like images or audio due to the challenge of constructing high-quality density models, recent advancements in generative modelling have made significant progress in this area.



Autoregressive and invertible models, such as PixelCNN++ (\cite{salimans2017pixelcnn}) and Glow (\cite{kingma2018glow}), have shown the capability to approximate complex data distributions with remarkable accuracy. These models fit a generative distribution \(p_\theta (x_{\text{ID}})\) to input data and evaluate the likelihood of new inputs to differentiate ID and OOD samples. However, studies by (\cite{choi2019waic}) and (\cite{nalisnick2019deep}) revealed that using model likelihood as an OOD score often results in counter-intuitive behaviour when datasets of varying complexity are paired, undermining its reliability (see our replication of their results in Fig. \ref{fig:nnl} and \S~\ref{sec:NLL_as_OOD}.



To address these limitations, various techniques have been proposed. The \textbf{Typicality Test}, introduced by (\cite{nalisnick2019deep}), uses the concept of a "typical set" from information theory (\cite{shannon1948mathematical}). It evaluates whether test data likely belongs to the generative model’s distribution by computing a test statistic \(\hat{\epsilon}\), derived from negative log-likelihoods of samples. This test operates as an omnibus goodness-of-fit (GoF) measure (\cite{eubank1992asymptotic}), ensuring that ID data falls within a high-probability region.




\textbf{Likelihood Ratio} Methods proposed by (\cite{ren2019likelihood}) isolate semantic information from background statistics in data. This approach trains a background model by adding perturbations to corrupt semantic structure, ensuring it captures only population-level background characteristics. Using the ratio of likelihoods from the original model and the background model, this method distinguishes ID data from OOD data, providing a background contrastive score for detection.


\textbf{Generative Ensembles}, suggested by (\cite{choi2019waic}), combine likelihood-based models with predictive uncertainty estimation. The method uses ensemble variance and the Watanabe-Akaike Information Criterion (WAIC) to adjust likelihood estimates, penalizing high variance across independently trained models. This approach distinguishes anomalies by incorporating Bayesian posterior approximations and leveraging variance-based metrics to highlight distributional differences.



\textbf{Hierarchical Log-Likelihood Ratios}, explored by (\cite{schirrmeister2020advances, havtorn2021hierarchical}), address the dominance of low-level features in likelihood-based methods. By training generative models on both general and specific ID datasets, the ratio of their likelihoods isolates semantic information, mitigating biases from low-level features. Additionally, an outlier loss function, incorporating temperature-scaled log-likelihood ratios, enhances the robustness of OOD detection. However, finding a comprehensive general dataset in practice is not feasible.



\textbf{Density of States Estimation (DoSE)}, developed by (\cite{morningstar2021density}), introduces a statistical mechanics-inspired technique. It estimates the density of states from summary statistics of pre-trained generative models. By fitting distributions to these statistics on ID data, DoSE identifies OOD samples as those with low density support, providing a novel perspective on OOD detection.


(\cite{kamkari2024geometric}) introduced a technique leveraging \textbf{Local Intrinsic Dimension (LID)} (\cite{bengio2012unsupervised, pope2021intrinsic}) to address the paradoxical behaviour of generative models assigning high density but low probability mass to OOD regions. Using LID estimates alongside the model likelihood, the method classifies data points into three cases: (1) low likelihood, (2) high likelihood with low LID, and (3) high likelihood and high LID. In the first two cases, the assigned probability mass is negligible, indicating OOD classification, while the third case suggests the data point is ID. This approach demonstrates the significance of considering both density and LID for reliable OOD detection.


A critical challenge in all density-based methods is ensuring \emph{representation invariance} (\cite{lan2021perfect, serran2019input}), which is often not guaranteed except in methods involving density ratios. However, even ratio-based approaches face practical hurdles, such as identifying an appropriate global dataset to serve as the denominator—a task that is often unattainable. This lack of invariance highlights the susceptibility of density-based methods to arbitrary manipulations, such as using different data representations (e.g., RGB or HSV colour spaces), which can introduce subjectivity and compromise reliability.



Our method distinguishes itself from existing likelihood-based OOD score methods by utilizing the layer-wise gradient of parameters, a feature absent in traditional approaches. Unlike \cite{dauncey2024approximations} and \cite{choi2021robust}, which rely on statistics derived from fully trained models, our approach leverages partially trained models, demonstrating that these can perform better for OOD detection. Additionally, while \cite{zhang2021understanding} focuses on explaining the performance of immature models with histogram of support overlap, our method provides a novel perspective by analysing the gap between the OOD scores logarithm gradient $L^2$-norms $S_{\bm{\theta}^{(l)}}(\bm{x})$, offering a new explanation for why partially trained models are more effective.




\subsection{Gradient Scoring (Score-based) Method}
\label{sec:Gradient_scoring}



Gradient scoring methods are renowned for representation-invariant techniques such that the choice of representation does not dramatically influence the output of the OOD detection. At a conceptual level, new sample data \( \bm{x} \) at test time can be categorized based on the variations observed in the parameters of a corresponding trained model. Minimal alterations in optimal trained parameters \( \hat{\bm{\theta}} \) typically indicate that the input data originates from the ID (\( \bm{x} \sim p^* \)). Conversely, substantial changes in the parameters are indicative of input samples belonging to OOD (\( \bm{x} \not\sim p^* \)). 

To model this formally, we represent the layer-wise parameters as \( \hat{\bm{\theta}}^{(l)} \), where \( l \) denotes the layer index. For a new input sample \( \bm{x} \), the change in parameters after incorporating \( \bm{x} \), at layer \( l \), can be expressed in terms of the likelihood \( p(\hat{\bm{\theta}}_{\bm{x}}^{(l)} \mid \cdot) \). Define the change in parameters for layer \( l \) as:



\begin{align}
\Delta_{\bm{\theta}}^{(l)}(\bm{x}) &= \log p(\hat{\bm{\theta}}_{\bm{x}}^{(l)} \mid \bm{x}) - \log p(\hat{\bm{\theta}}^{(l)} \mid \bm{x}). 
\label{eq:delta}
\end{align}



Based on the magnitude of \( \Delta^{(l)} \), \( \bm{x} \) is classified  as:



\begin{align}
\begin{cases} 
\text{ID}, & \Delta \leq \varepsilon, \\
\text{OOD}, & \Delta > \varepsilon. 
\end{cases} \label{eq:classification}
\end{align}


While this approach is theoretically sound, retraining the model with new data points to measure parameter changes directly is computationally prohibitive. To address this inefficiency, (\cite{choi2021robust}) propose using a second-order analysis to approximate \( \Delta^{(l)} \):



\begin{align}
\Delta_{\bm{\theta}}^{(l)}(\bm{x}) &\approx 
\big[ \nabla_{\bm{\theta}^{(l)}} \log p(\bm{x} \mid \bm{\theta}^{(l)}) \big]^\top F(\bm{\theta})^{-1} \nabla_{\bm{\theta}^{(l)}} \log p(\bm{x} \mid \bm{\theta}^{(l)}), \label{eq:approx_delta}
\end{align}



where \( F(\bm{\theta}) \) is the Fisher information matrix (FIM) for \( p^*(\bm{x}_{\text{ID}}) \) evaluated at \( \bm{\theta} \). This approximation provides a layer-wise representation-invariant framework for deriving a more informative OOD score than the raw model likelihood. By leveraging the Fisher Information Metric (\cite{fisher1920mathematical}), which naturally quantifies the size of the gradient, this method achieves computational efficiency and robustly distinguishes ID from OOD samples (\cite{ly2017tutorial}).

The score function  $\big\| \nabla_{\bm{\theta}} \ell(\bm{x}) \big\|_{\text{FIM}}^2$ is given by:




\begin{align}
\nabla_{\bm{\theta}} \ell(\bm{x}) &= \nabla_{\bm{\theta}} \log p(\bm{x} \mid \bm{\theta}), \label{eq:gradient_score} \\
\big\| \nabla_{\bm{\theta}} \ell(\bm{x}) \big\|_{\text{FIM}}^2 &= 
\big[ \nabla_{\bm{\theta}} \ell(\bm{x}) \big]^\top F(\bm{\theta})^{-1} \nabla_{\bm{\theta}} \ell(\bm{x}), 
\label{eq:fim_score} \\
F(\bm{\theta}) &= \mathbb{E}_{\bm{x}_{\text{ID}} \sim p_{\bm{\theta}}} \big[ \nabla_{\bm{\theta}} \ell(\bm{x}_{\text{ID}}) \nabla_{\bm{\theta}} \ell(\bm{x}_{\text{ID}})^\top \big]. 
\label{eq:fim_matrix}
\end{align}




However, directly computing the inverse of \emph{Fisher Information Matrix} \( F(\bm{\theta}) \) is computationally impractical for large-scale DGMs due to the high dimensionality of the parameter space (\( |\bm{\theta}| \times |\bm{\theta}| \)) that exceeds millions. To address this, approximation of \( F(\bm{\theta})^{-1} \) is necessary. The following methods are commonly used for such approximation:




\begin{itemize}
    \item \textbf{Identity} (\cite{bergamin2022model}): \( F(\bm{\theta}_0) \approx \mathbb{I}, \)
    \item \textbf{Diagonal} (\cite{kirkpatrick2016overcoming}): \( F(\bm{\theta}_0) \approx \mathbb{E}_{\bm{x}_{\text{ID}} \sim p_{\bm{\theta}}} \big[ \text{diag} \big( \nabla_{\bm{\theta}} \ell(\bm{x}_{\text{ID}}) \nabla_{\bm{\theta}} \ell(\bm{x}_{\text{ID}})^\top \big) \big], \)
    \item \textbf{EKFAC} (\cite{martens2016second, george2018fast}): \( F(\bm{\theta}_0) \approx \mathbb{E}_{\bm{x}_{\text{ID}} \sim p_{\bm{\theta}}} \big[ \nabla_{\bm{\theta}} \ell(\bm{x}_{\text{ID}}) \nabla_{\bm{\theta}} \ell(\bm{x}_{\text{ID}})^\top \big] = U \Sigma U^\top, \)
    \item \textbf{EKFAC using Kronecker product} (\cite{choi2021robust}): \( \mathbb{E}_{\bm{x}_{\text{ID}} \sim p_{\bm{\theta}}} \big[ \nabla_{\bm{\theta}} \ell(\bm{x}_{\text{ID}}) \nabla_{\bm{\theta}} \ell(\bm{x}_{\text{ID}})^\top \big] \approx A \otimes B, \)
\end{itemize}




where \( U \) is an orthogonal matrix, \( \Sigma \) is a diagonal matrix, and \( A \) and \( B \) represent smaller matrices derived from (\( h \))  represents the layer's input features, and (\( \delta \)) represents the back-propagated gradients with respect to the pre-activations. 

The identity approximation simplifies the FIM-based norm \( \| \cdot \|_{\text{FIM}}^2 \) to the Euclidean norm \( \| \cdot \|_2^2 \), offering a computationally efficient alternative for evaluating gradient magnitudes. This approach is particularly useful in large-scale models, where direct computation of the FIM can be prohibitively expensive. Formally, this approximation can be expressed as:




\begin{align}
\big\| \nabla_{\bm{\theta}^{(l)}} \ell(\bm{x}) \big\|_{\text{FIM}}^2 &\approx \big\| \nabla_{\bm{\theta}^{(l)}} \ell(\bm{x}) \big\|_2^2. \label{eq:euclidean_norm}
\end{align}



(\cite{bartlett1946statistical}) demonstrated that $\big\| \nabla_{\bm{\theta}^{(l)}} \ell(\bm{x}) \big\|_{\text{FIM}}^2$ follows a chi-squared (\( \chi^2 \)) distribution with a large degrees of freedom \( |\bm{\theta}| \), where \( |\bm{\theta}| \) denotes the dimensionality of the parameter space, typically estimated via MLE. Consequently, the logarithm of this squared norm should approximately follow a normal distribution (\cite{dauncey2024approximations}). Leveraging this property, the OOD score based on the log-transformed gradient norm is defined as:



\begin{align}
S_{\bm{\theta}^{(l)}}(\bm{x}) &= \log \big\{ \| \nabla_{\bm{\theta}^{(l)}} \ell(\bm{x}) \|_2^2 \big\}. \label{eq:ood_score}
\end{align}

\section{Methodology}
\label{sec:Methodology}
In the following section, we revisit and redefine a method for OOD detection based on layer-wise gradients, as proposed by (\cite{dauncey2024approximations}), and formalized in (Equation \ref{eq:ood_score}). This approach exploits GLOW (\cite{kingma2018glow}) and demonstrates that partially trained models can deliver equivalent or superior performance compared to fully trained models in OOD detection tasks.

Our findings suggest that the superior performance of partially trained models is tied to the phenomenon of sufficient gap between ID and OOD data. As training progresses on complex datasets (e.g., ImageNet32), this gap increases but is not necessarily required for achieving optimal results. Conversely, on less complex datasets (e.g., SVHN), the gap diminishes and transitions into overlap, which reduces the effectiveness of OOD detection criteria in fully converged models. This observation is explored in greater detail in \S~\ref{sec:partial_vs_full}, where we provide a comprehensive analysis of how training dynamics influence OOD performance.

\subsection{Training the Model}
\label{sec:Training_the_model}
Unlike conventional OOD detection methods that rely on fully converged models and their associated parameter statistics as OOD metrics, we propose a novel approach applying the GLOW model. Specifically, we train the model on various three-channel colour image datasets, including SVHN, GTSRB, CIFAR-10, CelebA, and ImageNet32, as well as one-channel colour such as (MNIST, FashionMNIST, KMNIST and Omniglot), but only partially. This approach extracts parameter values from ID datasets without requiring full convergence, challenging the traditional reliance on fully optimized models (see the experimental setup in \ref{sec:Experimental_Setups}). By significantly reducing computational time and energy, this method achieves remarkable improvements in OOD detection performance, yielding an AUROC metric of about perfect across nearly all tested dataset pairs (refer to Tables \ref{tab:auc_results_three_channel} and \ref{tab:auc_results_one_channel}) and challenges the necessity of fully trained model parameters.

\subsection{Extracting OOD scores}
\label{sec:Extracting_ood_scores}


To compute the OOD scores, the method allows for two approaches: single-sample OOD detection (singleton designation) and detection based on a batch of five samples, utilizing a goodness-of-fit (GoF) test. After partially training the model, samples from the in-distribution and out-of-distribution test datasets are used for one step of back-propagation. This step is initialized with the parameters of the partially trained model and captures the gradient changes caused by the new samples. These gradient changes are evaluated layer by layer, producing layer-wise scores that reflect the divergence between ID and OOD data. The scores are then combined to compute a final OOD score, which effectively determines whether the given sample or batch of samples belongs to the ID or OOD data distribution.


Having approximated the Fisher Information Matrix (FIM) as identity, the score is expressed as:


\begin{align}
S_{\bm{\theta}^{(l)}}(\bm{x}_{b=\{1,5\}}) &= \log \big\{ \big\| \nabla_{\bm{\theta}^{(l)}} \big( \sum_b \ell(\bm{x}_b) \big) \big\|_2^2 \big\}, 
\label{eq:score_single_batch}
\end{align}


where \( \bm{x} \) is the new data sample from the test dataset, and \( b = \{1,5\} \) represents the batch size. A batch size of \( b = 1 \) corresponds to single-sample OOD detection, while \( b = 5 \) enhances performance through a goodness-of-fit test.

The computed layer-wise score \( S_{\bm{\theta}^{(l)}}(\bm{x}_{b=\{1,5\}}) \) follows an approximately normal distribution, then a natural method of combining the layer-wise \( L^2 \)-norms is by fitting normal distributions to each score independently:
\begin{align}
\mu^{(l)} &= \text{MEAN} \big(S_{\bm{\theta}^{(l)}}(\bm{x}_{b=\{1,5\}})^{(N)}\big), \label{eq:mean_layer} \\
\sigma^{(l)} &= \text{VARIANCE} \big(S_{\bm{\theta}^{(l)}}(\bm{x}_{b=\{1,5\}})^{(N)}\big), \label{eq:variance_layer}
\end{align}
where \( N \) represents the number of samples from the ID dataset used to calculate the mean and variance for each layer independently. These statistics are derived from a fit dataset that is not used during the model's training phase 

Using the layer-specific means and variances, we calculate the Gaussian negative log-likelihood as a comprehensive OOD score:


\begin{align}
S_{\bm{\theta}}(\bm{x}_{b=\{1,5\}}) &= \frac{1}{2} \sum_{l=1}^{L} \bigg( \frac{\big(S_{\bm{\theta}^{(l)}}(\bm{x}_{b=\{1,5\}}) - \mu^{(l)}\big)^2}{(\sigma^{(l)})^2 + \varepsilon} 
+ \log \big(2\pi((\sigma^{(l)})^2 + \varepsilon)\big) \bigg), \label{eq:ood_score_final}
\end{align}
where \( \mu^{(l)} \) and \( \sigma^{(l)} \) are the layer-specific mean and variance from fit dataset and $\varepsilon$ is a small constant added for numerical stability.


This formulation quantifies the deviation of log-transformed gradient magnitudes from their expected mean, normalized by variance, across all layers. By summing these normalized deviations, the final OOD score effectively integrates layer-wise gradient information to detect out-of-distribution samples. As depicted in \S~\ref{sec:Additional_Results}, this summation does not necessarily result in a normal distribution but instead produces a right-skewed distribution, which arises due to the high correlation between adjacent layers.





\section{Immature vs. Mature Model in OOD Detection}
\label{sec:partial_vs_full}

We investigate the relationship between the gap (or overlap) in histograms of OOD scores (using Equation \ref{eq:ood_score_final}) across different datasets. Then, we reference the context of support overlap as discussed in (\cite{zhang2021understanding}) to demonstrate that partial training can outperform full training in OOD detection when using log-likelihood models and their associated NLL statistics.

\noindent Extracting the OOD score as explained in \S~\ref{sec:Methodology}, two major phenomena can be observed: (1) training the model on a complex dataset (e.g., ImageNet32) and (2) training on a less complex dataset (e.g., SVHN). In the first case, as the model is trained to achieve lower bits per dimension (BPD) and potentially better image quality, the gap between the ID and OOD histograms of OOD scores becomes increasingly wider (see Fig. \ref{fig:layerwise_scores_complex_comparison}). However, this widening gap does not significantly affect the AUROC results. Therefore, a minimal yet sufficient gap, achievable during earlier training epochs, is enough to yield AUROC results comparable to those of a fully converged model (see Tables \ref{tab:auc_results_three_channel} and \ref{tab:auc_results_one_channel}). 

\noindent In the second case, when the model is trained on a simpler dataset, as it reduces the BPD, its ability to detect OOD samples diminishes. The gap observed in the earlier epochs transitions into overlap in the fully trained model or an exiting overlap becomes more pronounced (refer to Table \ref{tab:auc_results_one_channel}). This suggests that OOD detection primarily operates on low-level image features (e.g., textures and curves) rather than high-level features (e.g., objects in images) (see Fig. \ref{fig:layerwise_scores_simple_comparison}).

\noindent These phenomena are consistent across different combinations of ID and OOD datasets for both three-channel and one-channel colour images (additional comparisons are provided in \S~\ref{sec:Additional_Results}). Notably, for certain datasets, such as SVHN and Omniglot, effective OOD detection can be achieved using only the first epoch, leading to better AUROC results. It is important to note that if the size of the training dataset is sufficiently large relative to the dataset's complexity, allowing the model to adequately learn the ID samples, a gap will emerge between the histograms of OOD scores. This indicates a clear distinction between ID and OOD samples, as seen in datasets like Imagenet32, Celeba, SVHN, and Omniglot. However, synthetically increasing the dataset size—such as through data augmentation, adding generated samples to the training set, or utilizing mix-up/cutout techniques—does not significantly boost AUROC scores to nearly perfect levels for distinguishing singleton OOD samples, as demonstrated with CIFAR-10 and GTSRB. Therefore, their results are not reported.






\begin{figure}[ht!]
    \centering
    \begin{subfigure}[t]{0.49\textwidth}
        \centering
        \includegraphics[width=\textwidth]{images/histogram_imagenet32_b5_partial.png}
    \end{subfigure}
    \hfill
    \begin{subfigure}[t]{0.49\textwidth}
        \centering
        \includegraphics[width=\textwidth]{images/histogram_imagenet32_b5_full.png}
    \end{subfigure}
    \caption{\textbf{\emph{Progressive widening of the gap in histograms of layer-wise gradient-based OOD scores with batch size of $5$.}}\\[0.5em]
    The figure illustrates how training on a complex ID dataset, such as ImageNet32, affects the gap between histograms of OOD scores for ID and OOD samples from (GTSRB, CIFAR-10, CelebA, and SVHN). Figure (left) represents the results after 10 epochs, while figure (right) shows the results after 250 epochs. Despite the increasing gap as training progresses, AUROC scores remain unchanged compared to a partially trained model, indicating that early training may suffice for OOD detection tasks. This widening gap, while reflecting improved separation, incurs higher computational costs. Additional experiments, including results for other batch sizes (e.g., $b=1$) and ID datasets, are discussed in \S~\ref{sec:Additional_Results}.}
    \label{fig:layerwise_scores_complex_comparison}
\end{figure}


\begin{figure}[ht!]
    \centering
    \begin{subfigure}[t]{0.49\textwidth}
        \centering
        \includegraphics[width=\textwidth]{images/histogram_svhn_b5_partial.png}
    \end{subfigure}
    \hfill
    \begin{subfigure}[t]{0.49\textwidth}
        \centering
        \includegraphics[width=\textwidth]{images/histogram_svhn_b5_full.png}
    \end{subfigure}
    \caption{\textbf{\emph{Transition from gap to overlap in histograms of layer-wise gradient-based OOD scores using batch size of $5$ as training progresses.}}\\[0.5em]This phenomenon occurs when the ID dataset is less complex compared to the OOD samples, resulting in a decline in OOD detection performance with a fully converged model that generates higher-quality images. The GLOW model was trained on the ID dataset SVHN for: (left) one epoch and (right) 250 epochs, and tested against four OOD datasets: GTSRB, CIFAR-10, CelebA, and ImageNet32. As training progresses, the distinct separation between ID and OOD data distributions deteriorates. Additional results with varying batch sizes (e.g., $b=1$ and $b=5$) and other IDs are detailed in \S~\ref{sec:Additional_Results}.}
    \label{fig:layerwise_scores_simple_comparison}
\end{figure}


\noindent\\ Consider the case when NLL is used as the OOD score. Using the general concept of support overlap, it can be observed that as the model progresses in training, the support overlap between the ID and OOD distributions increases. This increasing overlap degrades the quality of OOD scores generated by statistics such as log-likelihood or its combinations (see Fig. \ref{fig:support_overlap}). Formally, when the supports of the in-distribution \(P\) and the out-distribution \(Q\) overlap (i.e., there exists \(x \in X\) such that both \(P(x) > 0\) and \(Q(x) > 0\)), a misestimated model \(P_\theta\), constructed based on the likelihood ratio, can lead to improved OOD detection performance compared to using the true distribution \(P\). Specifically, \(P_\theta(x)\) is defined as being proportional to the likelihood ratio of \(P(x)\) and \(Q(x)\), normalized by a constant \(C\) assuming integrability:

\[
P_\theta(x) = \frac{1}{C} \frac{P(x)}{Q(x)}, \quad \text{where } C = \int_X \frac{P(x)}{Q(x)} \, dx.
\]

\noindent Using this construction, the test statistic \(\phi_{P_\theta}\) becomes proportional to the likelihood ratio \(\frac{P(x)}{Q(x)}\). According to the Neyman-Pearson Lemma (\cite{neyman1933most}), the likelihood ratio test is uniformly most powerful for distinguishing \(P\) from \(Q\). As a result, OOD detection using \(P_\theta\) achieves a higher Area Under the Curve (AUC) than detection using the true distribution \(P\), expressed as:

\[
\text{Pr}(\phi_{P_\theta}(x) > \phi_{P_\theta}(y)) > \text{Pr}(\phi_P(x) > \phi_P(y)),
\]

\noindent where \(x \sim P\) and \(y \sim Q\). This improvement holds for any model \(P_\theta\) that makes values of \(\phi_{P_\theta}(x)\); \(x \sim P\) higher
relative to \(\phi_{P_\theta}(y)\); \(y \sim Q\) leading to more effective OOD detection.


\begin{figure}[ht!]
    \centering
    \begin{subfigure}[t]{0.49\textwidth}
        \centering
        \includegraphics[width=\textwidth]{images/CIFAR10_vs_GTSRB_overlap_partial.png}
    \end{subfigure}
    \hfill
    \begin{subfigure}[t]{0.49\textwidth}
        \centering
        \includegraphics[width=\textwidth]{images/CIFAR10_vs_GTSRB_overlap_full.png}
    \end{subfigure}
    \caption{\textbf{\emph{Overlap Coefficient (OVL)}}\\[0.5em] Overlap area (\cite{walker2021newmeasure, weitzman1970overlap} ) between the PDFs of negative BIDs for an ID dataset (CIFAR-10) and an OOD dataset (GTSRB), evaluated using the Glow model trained on CIFAR-10 at different epochs. Figure (a) shows the OVL value of 0.8357 after 50 epochs, and (b) value of 0.8635 for the fully trained model (lower values are better). The increase in OVL reflects a larger overlap of distributions as training progresses, which correlates with a decline in AUC values of OOD detection: 0.6310 (50 epochs), and 0.5502 (fully trained), (higher values are better).}
    \label{fig:support_overlap}
\end{figure}





\section{Experimental Results}
\label{sec:Experimental_Results}
\subsection{OOD Results}
\label{sec:OOD_Results}



In this section, the results of a partially trained versus fully trained model is demonstrated. To maintain comparability of results, GLOW (\cite{kingma2018glow}) as the generative model is employed and trained for 250 epochs on five three-channel datasets: \emph{ImageNet32, CIFAR-10, CelebA, GTSRB, and SVHN}, with all images resized to a uniform resolution of \( 32 \times 32 \) pixels
and in another experiment using four one-channel datasets \emph{MNIST, FashionMNIST, KMNIST and Omniglot} with a resolution of \( 28 \times 28 \), however, limited the selection of epochs to $(1st, 10th, 20th, 30th, 40th, 50th, 70th, 80th, 100th, 150th )$ as the nominee of partially training and the last epoch as fully training.

\noindent For each experiment, one dataset is designated as the ID dataset, while the remaining datasets in each family channel are treated as OOD. The training split of the ID dataset is utilized to train the model and the test split to fit Gaussian distributions to the logarithmic  trained parameters. From the test splits of all datasets, 1,000 samples are randomly selected to compute OOD scores and generate AUROC results, as summarized in Tables~\ref{tab:auc_results_three_channel} and ~\ref{tab:auc_results_one_channel}.

\noindent The effectiveness of OOD detection is evaluated based on the test's ability to correctly classify OOD samples while minimizing the misclassification of ID samples as OOD. This performance is calculated through a Receiver Operating Characteristic (ROC) curve, which checks the True Positive Rate (TPR)—the proportion of correctly identified OOD samples—against the False Positive Rate (FPR)—the proportion of misclassified ID samples. The ROC curve illustrates the trade-off between sensitivity (TPR) and specificity (\( 1 - \text{FPR} \)) across various threshold settings.

\noindent The Area Under the ROC Curve (AUROC) provides a single metric to quantify the overall performance of the OOD detection method. A higher AUROC value, closer to 1, reflects better performance, indicating the test's ability to effectively distinguish OOD samples from ID samples. As a summary statistic, the AUROC captures the balance between correctly detecting OOD samples and avoiding the misclassification of ID samples.


\begin{table}[ht!]
\centering
\begin{tabular}{lll|cccc}
\toprule
\textbf{(Train ↓)}\textbf{(Test →)}&&\textbf{Epochs} & \textbf{MNIST} & \textbf{FashionMNIST} & \textbf{KMNIST} & \textbf{Omniglot} \\
\midrule
\multirow{6}{*}{\textbf{MNIST}} 
    & & $80^{th}$      && \textbf{0.9666} & \textbf{0.9522} &\textbf{1.0000} \\
    &$(b=1)$ &fully trained  &-& 0.9653 & 0.9511 & \textbf{1.0000} \\
\cmidrule{2-7}
    && $80^{th}$      && \textbf{0.9995} & 0.9997 &\textbf{1.0000} \\
    & $(b=5)$ &fully trained  &-& \textbf{0.9995} & \textbf{0.9998}&\textbf{1.0000} \\

\midrule
\midrule
\multirow{6}{*}{\textbf{FashionMNIST}}   
    & &$70^{th}$       & \textbf{0.9547}   &&\textbf{0.9025}&\textbf{1.0000}  \\
    &$(b=1)$ &fully trained    & 0.9546   &-& 0.9020 & \textbf{1.0000} \\
\cmidrule{2-7}
    &  &$50^{th}$       & \textbf{0.9991}&& \textbf{0.9936}&\textbf{1.0000}  \\
    &$(b=5)$ &fully trained& 0.9985&-& 0.9935 & \textbf{1.0000}\\
\midrule
\midrule
\multirow{6}{*}{\textbf{KMNIST}}
    & &$30^{th}$      & \textbf{0.4715}   & \textbf{0.7315} && \textbf{0.9983}\\
    & $(b=1)$&fully trained   & 0.4661   & 0.6737 &-& 0.9960\\
\cmidrule{2-7}
    &&$20^{th}$      & \textbf{0.6595}   & \textbf{0.9590} &&\textbf{1.0000}\\
    & $(b=5)$ &fully trained   & 0.6290   &0.9499 &-& \textbf{1.0000}\\
\midrule
\midrule
\multirow{6}{*}{\textbf{Omniglot}}    
    & &$1^{st}$&  \textbf{1.0000}&\textbf{1.0000} &\textbf{1.0000}& \\
    &$(b=1)$ &fully trained&\textbf{1.0000}&\textbf{1.0000} &\textbf{1.0000}&-\\
\cmidrule{2-7}
    & &$1^{st}$&  \textbf{1.0000}&\textbf{1.0000}&\textbf{1.0000}&\\
    &$(b=5)$ &fully trained& \textbf{1.0000}&\textbf{1.0000}&\textbf{1.0000}&-\\
\bottomrule
\end{tabular}
\vspace{0.5cm}

\caption{\textbf{\emph{AUROC Results for One-Channel colour Datasets}}\\[0.5em]
This table presents the AUROC results for various ID and OOD dataset pairs (\emph{MNIST, FashionMNIST, KMNIST, Omniglot}) trained with GLOW and evaluated at two training stages: partial training (\emph{e.g., 1st, 20th, 30th, 50th, 70th, and 80th epochs}) and fully trained models. Results are reported for single OOD detection (\(b=1\)) and batch size \(b=5\) using the OOD scores defined in Equation~\ref{eq:ood_score_final}. For most dataset pairs, partial training—depending on the dataset—outperforms fully trained models in OOD detection, even as the fully trained models produce higher-quality images. Bolded values highlight the highest AUROC scores, indicating the most effective training stage for distinguishing OOD data. Generated samples from each model can be found in \S~\ref{sec:Generated_Samples}.}
\label{tab:auc_results_one_channel}
\end{table}



\begin{table}[ht!]
\centering
\begin{tabular}{lll|ccccc}
\toprule
\textbf{(Train ↓)}\textbf{(Test →)} & &\textbf{Epochs} & \textbf{SVHN} & \textbf{CelebA} & \textbf{CIFAR-10} & \textbf{GTSRB} & \textbf{ImageNet32} \\
\midrule
\multirow{6}{*}{\textbf{SVHN}}
    &&$10^{th}$&& \textbf{0.9852} & \textbf{0.9765} & \textbf{0.9561} & \textbf{0.9833} \\
    &$(b=1)$&fully trained&-& 0.9377 & 0.9243 & 0.9215 & 0.9091 \\
\cmidrule{2-8}
    & &$1^{st}$& & \textbf{1.0000}&\textbf{1.0000} & \textbf{0.9998}& \textbf{1.0000} \\
    &$(b=5)$&fully trained&-& \textbf{1.0000} & 0.9994 & 0.9996 & 0.9999 \\
\midrule
\midrule
\multirow{6}{*}{\textbf{CelebA}}   
    &&$50^{th}$& \textbf{0.9896}&& \textbf{0.9799} & \textbf{0.9755} & \textbf{0.9865} \\
    &$(b=1)$&fully trained& 0.9876 &-& 0.9797 & 0.9753 & 0.9790 \\
\cmidrule{2-8}
    &&$20^{th}$& \textbf{1.0000}&& \textbf{1.0000} & \textbf{1.0000}& \textbf{1.0000} \\
    &$(b=5)$&fully trained& \textbf{1.0000}&-& \textbf{1.0000} & \textbf{1.0000}& \textbf{1.0000} \\
\midrule
\midrule
\multirow{6}{*}{\textbf{CIFAR-10}}
    &&$100^{th}$& \textbf{0.8652}& \textbf{0.5410} && \textbf{0.7191}& 0.5893 \\
    &$(b=1)$&fully trained& 0.8185 & 0.4142 &-& 0.6798 & \textbf{0.7001} \\
\cmidrule{2-8}
    &&$50^{th}$& \textbf{0.9987}& \textbf{0.9377} && \textbf{0.9388} & 0.8247 \\
    &$(b=5)$&fully trained& 0.9984& 0.9140 &-& 0.9140 & \textbf{0.9480} \\
\midrule
\midrule
\multirow{6}{*}{\textbf{GTSRB}}  
    &&$150^{th}$& \textbf{0.9583}& \textbf{0.9249} & 0.9079 && \textbf{0.9509} \\
    &$(b=1)$&fully trained& 0.9579& 0.9177 & \textbf{0.9221} &-& 0.9499 \\
\cmidrule{2-8}
    &&$100^{th}$& \textbf{1.0000}& \textbf{0.9999} & 0.9990 & -& 0.9999 \\
    &$(b=5)$&fully trained& \textbf{1.0000}& \textbf{0.9999} & \textbf{0.9997} && \textbf{1.0000} \\
\midrule
\midrule
\multirow{6}{*}{\textbf{ImageNet32}} 
    &&$40^{th}$& 0.8949& \textbf{0.9999} & \textbf{0.9996} & \textbf{0.9999} & \\
    &$(b=1)$&fully trained& \textbf{0.9994}& 0.9998 & \textbf{0.9996} & 0.9998 &- \\
\cmidrule{2-8}
    &&$10^{th}$& \textbf{1.0000}& \textbf{1.0000} & \textbf{1.0000} & \textbf{1.0000} &\\
    &$(b=5)$&fully trained & \textbf{1.0000}& \textbf{1.0000} & \textbf{1.0000} & \textbf{1.0000}&-\\
\bottomrule
\end{tabular}
\vspace{0.5cm}

\caption{\textbf{\emph{Three-channel colour Dataset AUROC Results}}\\[0.5em]
The AUROC results for OOD detection across various three-channel datasets (\emph{SVHN}, \emph{CelebA}, \emph{CIFAR-10}, \emph{GTSRB}, and \emph{ImageNet32}) evaluated at different training epochs (e.g., 1st, 10th, 20th, 40th, 50th, 100th, 150th and fully trained). The OOD detection scores, calculated as $S_{\bm{\theta}^{(l)}}(\bm{x}_{b=\{1,5\}}) = \log \big\{ \big\| \nabla_{\bm{\theta}^{(l)}} \big( \sum_b \ell(\bm{x}_b) \big) \big\|_2^2 \big\}$, are provided for both batch sizes of 1 and 5. Each row corresponds to a specific in-distribution (ID) dataset (Train ↓) evaluated against various out-of-distribution (OOD) datasets (Test →). Bolded values highlight the highest AUROC scores achieved for a given dataset pair across the training epochs. The results show that in many cases, the Glow model achieves near-optimal or optimal OOD detection performance after partial training, often surpassing or equalling the performance of the fully trained model. Notably, the epochs selected for each dataset may not necessarily be the lowest but are instead sampled from every tenth epoch, demonstrating that partial training can outperform or match fully converged models. These findings emphasize the diminishing returns of extended training for OOD detection tasks in many scenarios. See generated samples \ref{sec:Generated_Samples} and visualization of results in \ref{sec:Additional_Results}.}
\label{tab:auc_results_three_channel}
\end{table}

\section{Conclusion}
\label{sec:Conclusion}
This study demonstrates that using layer-wise changes in parameter norms with respect to individual data points, partially trained models outperform fully trained models in OOD detection tasks. The results suggest that the models excel in identifying low-level features such as dominant colors, textures, and structures, which are more prominent in earlier training stages. However, this work has two key limitations. First, future experiments should explore other DGMs to validate the findings across a broader range of architectures. Second, while this study provides an extensive empirical benchmark for DGMs, OOD, and ID datasets, future research should expand beyond image datasets to include other data modalities, such as text, and evaluate performance on large language models.


\section*{Acknowledgments}
\label{sec:Acknowledgments}
The authors would like to thank Felix Draxler, Armand Rousselot and members of CVL for their guidance and insightful discussions throughout this work.


\printbibliography
% \bibliography{main}
% % This must be in the first 5 lines to tell arXiv to use pdfLaTeX, which is strongly recommended.
\pdfoutput=1
% In particular, the hyperref package requires pdfLaTeX in order to break URLs across lines.

\documentclass[11pt]{article}

% Change "review" to "final" to generate the final (sometimes called camera-ready) version.
% Change to "preprint" to generate a non-anonymous version with page numbers.
\usepackage{acl}

% Standard package includes
\usepackage{times}
\usepackage{latexsym}

% Draw tables
\usepackage{booktabs}
\usepackage{multirow}
\usepackage{xcolor}
\usepackage{colortbl}
\usepackage{array} 
\usepackage{amsmath}

\newcolumntype{C}{>{\centering\arraybackslash}p{0.07\textwidth}}
% For proper rendering and hyphenation of words containing Latin characters (including in bib files)
\usepackage[T1]{fontenc}
% For Vietnamese characters
% \usepackage[T5]{fontenc}
% See https://www.latex-project.org/help/documentation/encguide.pdf for other character sets
% This assumes your files are encoded as UTF8
\usepackage[utf8]{inputenc}

% This is not strictly necessary, and may be commented out,
% but it will improve the layout of the manuscript,
% and will typically save some space.
\usepackage{microtype}
\DeclareMathOperator*{\argmax}{arg\,max}
% This is also not strictly necessary, and may be commented out.
% However, it will improve the aesthetics of text in
% the typewriter font.
\usepackage{inconsolata}

%Including images in your LaTeX document requires adding
%additional package(s)
\usepackage{graphicx}
% If the title and author information does not fit in the area allocated, uncomment the following
%
%\setlength\titlebox{<dim>}
%
% and set <dim> to something 5cm or larger.

\title{Wi-Chat: Large Language Model Powered Wi-Fi Sensing}

% Author information can be set in various styles:
% For several authors from the same institution:
% \author{Author 1 \and ... \and Author n \\
%         Address line \\ ... \\ Address line}
% if the names do not fit well on one line use
%         Author 1 \\ {\bf Author 2} \\ ... \\ {\bf Author n} \\
% For authors from different institutions:
% \author{Author 1 \\ Address line \\  ... \\ Address line
%         \And  ... \And
%         Author n \\ Address line \\ ... \\ Address line}
% To start a separate ``row'' of authors use \AND, as in
% \author{Author 1 \\ Address line \\  ... \\ Address line
%         \AND
%         Author 2 \\ Address line \\ ... \\ Address line \And
%         Author 3 \\ Address line \\ ... \\ Address line}

% \author{First Author \\
%   Affiliation / Address line 1 \\
%   Affiliation / Address line 2 \\
%   Affiliation / Address line 3 \\
%   \texttt{email@domain} \\\And
%   Second Author \\
%   Affiliation / Address line 1 \\
%   Affiliation / Address line 2 \\
%   Affiliation / Address line 3 \\
%   \texttt{email@domain} \\}
% \author{Haohan Yuan \qquad Haopeng Zhang\thanks{corresponding author} \\ 
%   ALOHA Lab, University of Hawaii at Manoa \\
%   % Affiliation / Address line 2 \\
%   % Affiliation / Address line 3 \\
%   \texttt{\{haohany,haopengz\}@hawaii.edu}}
  
\author{
{Haopeng Zhang$\dag$\thanks{These authors contributed equally to this work.}, Yili Ren$\ddagger$\footnotemark[1], Haohan Yuan$\dag$, Jingzhe Zhang$\ddagger$, Yitong Shen$\ddagger$} \\
ALOHA Lab, University of Hawaii at Manoa$\dag$, University of South Florida$\ddagger$ \\
\{haopengz, haohany\}@hawaii.edu\\
\{yiliren, jingzhe, shen202\}@usf.edu\\}



  
%\author{
%  \textbf{First Author\textsuperscript{1}},
%  \textbf{Second Author\textsuperscript{1,2}},
%  \textbf{Third T. Author\textsuperscript{1}},
%  \textbf{Fourth Author\textsuperscript{1}},
%\\
%  \textbf{Fifth Author\textsuperscript{1,2}},
%  \textbf{Sixth Author\textsuperscript{1}},
%  \textbf{Seventh Author\textsuperscript{1}},
%  \textbf{Eighth Author \textsuperscript{1,2,3,4}},
%\\
%  \textbf{Ninth Author\textsuperscript{1}},
%  \textbf{Tenth Author\textsuperscript{1}},
%  \textbf{Eleventh E. Author\textsuperscript{1,2,3,4,5}},
%  \textbf{Twelfth Author\textsuperscript{1}},
%\\
%  \textbf{Thirteenth Author\textsuperscript{3}},
%  \textbf{Fourteenth F. Author\textsuperscript{2,4}},
%  \textbf{Fifteenth Author\textsuperscript{1}},
%  \textbf{Sixteenth Author\textsuperscript{1}},
%\\
%  \textbf{Seventeenth S. Author\textsuperscript{4,5}},
%  \textbf{Eighteenth Author\textsuperscript{3,4}},
%  \textbf{Nineteenth N. Author\textsuperscript{2,5}},
%  \textbf{Twentieth Author\textsuperscript{1}}
%\\
%\\
%  \textsuperscript{1}Affiliation 1,
%  \textsuperscript{2}Affiliation 2,
%  \textsuperscript{3}Affiliation 3,
%  \textsuperscript{4}Affiliation 4,
%  \textsuperscript{5}Affiliation 5
%\\
%  \small{
%    \textbf{Correspondence:} \href{mailto:email@domain}{email@domain}
%  }
%}

\begin{document}
\maketitle
\begin{abstract}
Recent advancements in Large Language Models (LLMs) have demonstrated remarkable capabilities across diverse tasks. However, their potential to integrate physical model knowledge for real-world signal interpretation remains largely unexplored. In this work, we introduce Wi-Chat, the first LLM-powered Wi-Fi-based human activity recognition system. We demonstrate that LLMs can process raw Wi-Fi signals and infer human activities by incorporating Wi-Fi sensing principles into prompts. Our approach leverages physical model insights to guide LLMs in interpreting Channel State Information (CSI) data without traditional signal processing techniques. Through experiments on real-world Wi-Fi datasets, we show that LLMs exhibit strong reasoning capabilities, achieving zero-shot activity recognition. These findings highlight a new paradigm for Wi-Fi sensing, expanding LLM applications beyond conventional language tasks and enhancing the accessibility of wireless sensing for real-world deployments.
\end{abstract}

\section{Introduction}

In today’s rapidly evolving digital landscape, the transformative power of web technologies has redefined not only how services are delivered but also how complex tasks are approached. Web-based systems have become increasingly prevalent in risk control across various domains. This widespread adoption is due their accessibility, scalability, and ability to remotely connect various types of users. For example, these systems are used for process safety management in industry~\cite{kannan2016web}, safety risk early warning in urban construction~\cite{ding2013development}, and safe monitoring of infrastructural systems~\cite{repetto2018web}. Within these web-based risk management systems, the source search problem presents a huge challenge. Source search refers to the task of identifying the origin of a risky event, such as a gas leak and the emission point of toxic substances. This source search capability is crucial for effective risk management and decision-making.

Traditional approaches to implementing source search capabilities into the web systems often rely on solely algorithmic solutions~\cite{ristic2016study}. These methods, while relatively straightforward to implement, often struggle to achieve acceptable performances due to algorithmic local optima and complex unknown environments~\cite{zhao2020searching}. More recently, web crowdsourcing has emerged as a promising alternative for tackling the source search problem by incorporating human efforts in these web systems on-the-fly~\cite{zhao2024user}. This approach outsources the task of addressing issues encountered during the source search process to human workers, leveraging their capabilities to enhance system performance.

These solutions often employ a human-AI collaborative way~\cite{zhao2023leveraging} where algorithms handle exploration-exploitation and report the encountered problems while human workers resolve complex decision-making bottlenecks to help the algorithms getting rid of local deadlocks~\cite{zhao2022crowd}. Although effective, this paradigm suffers from two inherent limitations: increased operational costs from continuous human intervention, and slow response times of human workers due to sequential decision-making. These challenges motivate our investigation into developing autonomous systems that preserve human-like reasoning capabilities while reducing dependency on massive crowdsourced labor.

Furthermore, recent advancements in large language models (LLMs)~\cite{chang2024survey} and multi-modal LLMs (MLLMs)~\cite{huang2023chatgpt} have unveiled promising avenues for addressing these challenges. One clear opportunity involves the seamless integration of visual understanding and linguistic reasoning for robust decision-making in search tasks. However, whether large models-assisted source search is really effective and efficient for improving the current source search algorithms~\cite{ji2022source} remains unknown. \textit{To address the research gap, we are particularly interested in answering the following two research questions in this work:}

\textbf{\textit{RQ1: }}How can source search capabilities be integrated into web-based systems to support decision-making in time-sensitive risk management scenarios? 
% \sq{I mention ``time-sensitive'' here because I feel like we shall say something about the response time -- LLM has to be faster than humans}

\textbf{\textit{RQ2: }}How can MLLMs and LLMs enhance the effectiveness and efficiency of existing source search algorithms? 

% \textit{\textbf{RQ2:}} To what extent does the performance of large models-assisted search align with or approach the effectiveness of human-AI collaborative search? 

To answer the research questions, we propose a novel framework called Auto-\
S$^2$earch (\textbf{Auto}nomous \textbf{S}ource \textbf{Search}) and implement a prototype system that leverages advanced web technologies to simulate real-world conditions for zero-shot source search. Unlike traditional methods that rely on pre-defined heuristics or extensive human intervention, AutoS$^2$earch employs a carefully designed prompt that encapsulates human rationales, thereby guiding the MLLM to generate coherent and accurate scene descriptions from visual inputs about four directional choices. Based on these language-based descriptions, the LLM is enabled to determine the optimal directional choice through chain-of-thought (CoT) reasoning. Comprehensive empirical validation demonstrates that AutoS$^2$-\ 
earch achieves a success rate of 95–98\%, closely approaching the performance of human-AI collaborative search across 20 benchmark scenarios~\cite{zhao2023leveraging}. 

Our work indicates that the role of humans in future web crowdsourcing tasks may evolve from executors to validators or supervisors. Furthermore, incorporating explanations of LLM decisions into web-based system interfaces has the potential to help humans enhance task performance in risk control.






\section{Related Work}
\label{sec:relatedworks}

% \begin{table*}[t]
% \centering 
% \renewcommand\arraystretch{0.98}
% \fontsize{8}{10}\selectfont \setlength{\tabcolsep}{0.4em}
% \begin{tabular}{@{}lc|cc|cc|cc@{}}
% \toprule
% \textbf{Methods}           & \begin{tabular}[c]{@{}c@{}}\textbf{Training}\\ \textbf{Paradigm}\end{tabular} & \begin{tabular}[c]{@{}c@{}}\textbf{$\#$ PT Data}\\ \textbf{(Tokens)}\end{tabular} & \begin{tabular}[c]{@{}c@{}}\textbf{$\#$ IFT Data}\\ \textbf{(Samples)}\end{tabular} & \textbf{Code}  & \begin{tabular}[c]{@{}c@{}}\textbf{Natural}\\ \textbf{Language}\end{tabular} & \begin{tabular}[c]{@{}c@{}}\textbf{Action}\\ \textbf{Trajectories}\end{tabular} & \begin{tabular}[c]{@{}c@{}}\textbf{API}\\ \textbf{Documentation}\end{tabular}\\ \midrule 
% NexusRaven~\citep{srinivasan2023nexusraven} & IFT & - & - & \textcolor{green}{\CheckmarkBold} & \textcolor{green}{\CheckmarkBold} &\textcolor{red}{\XSolidBrush}&\textcolor{red}{\XSolidBrush}\\
% AgentInstruct~\citep{zeng2023agenttuning} & IFT & - & 2k & \textcolor{green}{\CheckmarkBold} & \textcolor{green}{\CheckmarkBold} &\textcolor{red}{\XSolidBrush}&\textcolor{red}{\XSolidBrush} \\
% AgentEvol~\citep{xi2024agentgym} & IFT & - & 14.5k & \textcolor{green}{\CheckmarkBold} & \textcolor{green}{\CheckmarkBold} &\textcolor{green}{\CheckmarkBold}&\textcolor{red}{\XSolidBrush} \\
% Gorilla~\citep{patil2023gorilla}& IFT & - & 16k & \textcolor{green}{\CheckmarkBold} & \textcolor{green}{\CheckmarkBold} &\textcolor{red}{\XSolidBrush}&\textcolor{green}{\CheckmarkBold}\\
% OpenFunctions-v2~\citep{patil2023gorilla} & IFT & - & 65k & \textcolor{green}{\CheckmarkBold} & \textcolor{green}{\CheckmarkBold} &\textcolor{red}{\XSolidBrush}&\textcolor{green}{\CheckmarkBold}\\
% LAM~\citep{zhang2024agentohana} & IFT & - & 42.6k & \textcolor{green}{\CheckmarkBold} & \textcolor{green}{\CheckmarkBold} &\textcolor{green}{\CheckmarkBold}&\textcolor{red}{\XSolidBrush} \\
% xLAM~\citep{liu2024apigen} & IFT & - & 60k & \textcolor{green}{\CheckmarkBold} & \textcolor{green}{\CheckmarkBold} &\textcolor{green}{\CheckmarkBold}&\textcolor{red}{\XSolidBrush} \\\midrule
% LEMUR~\citep{xu2024lemur} & PT & 90B & 300k & \textcolor{green}{\CheckmarkBold} & \textcolor{green}{\CheckmarkBold} &\textcolor{green}{\CheckmarkBold}&\textcolor{red}{\XSolidBrush}\\
% \rowcolor{teal!12} \method & PT & 103B & 95k & \textcolor{green}{\CheckmarkBold} & \textcolor{green}{\CheckmarkBold} & \textcolor{green}{\CheckmarkBold} & \textcolor{green}{\CheckmarkBold} \\
% \bottomrule
% \end{tabular}
% \caption{Summary of existing tuning- and pretraining-based LLM agents with their training sample sizes. "PT" and "IFT" denote "Pre-Training" and "Instruction Fine-Tuning", respectively. }
% \label{tab:related}
% \end{table*}

\begin{table*}[ht]
\begin{threeparttable}
\centering 
\renewcommand\arraystretch{0.98}
\fontsize{7}{9}\selectfont \setlength{\tabcolsep}{0.2em}
\begin{tabular}{@{}l|c|c|ccc|cc|cc|cccc@{}}
\toprule
\textbf{Methods} & \textbf{Datasets}           & \begin{tabular}[c]{@{}c@{}}\textbf{Training}\\ \textbf{Paradigm}\end{tabular} & \begin{tabular}[c]{@{}c@{}}\textbf{\# PT Data}\\ \textbf{(Tokens)}\end{tabular} & \begin{tabular}[c]{@{}c@{}}\textbf{\# IFT Data}\\ \textbf{(Samples)}\end{tabular} & \textbf{\# APIs} & \textbf{Code}  & \begin{tabular}[c]{@{}c@{}}\textbf{Nat.}\\ \textbf{Lang.}\end{tabular} & \begin{tabular}[c]{@{}c@{}}\textbf{Action}\\ \textbf{Traj.}\end{tabular} & \begin{tabular}[c]{@{}c@{}}\textbf{API}\\ \textbf{Doc.}\end{tabular} & \begin{tabular}[c]{@{}c@{}}\textbf{Func.}\\ \textbf{Call}\end{tabular} & \begin{tabular}[c]{@{}c@{}}\textbf{Multi.}\\ \textbf{Step}\end{tabular}  & \begin{tabular}[c]{@{}c@{}}\textbf{Plan}\\ \textbf{Refine}\end{tabular}  & \begin{tabular}[c]{@{}c@{}}\textbf{Multi.}\\ \textbf{Turn}\end{tabular}\\ \midrule 
\multicolumn{13}{l}{\emph{Instruction Finetuning-based LLM Agents for Intrinsic Reasoning}}  \\ \midrule
FireAct~\cite{chen2023fireact} & FireAct & IFT & - & 2.1K & 10 & \textcolor{red}{\XSolidBrush} &\textcolor{green}{\CheckmarkBold} &\textcolor{green}{\CheckmarkBold}  & \textcolor{red}{\XSolidBrush} &\textcolor{green}{\CheckmarkBold} & \textcolor{red}{\XSolidBrush} &\textcolor{green}{\CheckmarkBold} & \textcolor{red}{\XSolidBrush} \\
ToolAlpaca~\cite{tang2023toolalpaca} & ToolAlpaca & IFT & - & 4.0K & 400 & \textcolor{red}{\XSolidBrush} &\textcolor{green}{\CheckmarkBold} &\textcolor{green}{\CheckmarkBold} & \textcolor{red}{\XSolidBrush} &\textcolor{green}{\CheckmarkBold} & \textcolor{red}{\XSolidBrush}  &\textcolor{green}{\CheckmarkBold} & \textcolor{red}{\XSolidBrush}  \\
ToolLLaMA~\cite{qin2023toolllm} & ToolBench & IFT & - & 12.7K & 16,464 & \textcolor{red}{\XSolidBrush} &\textcolor{green}{\CheckmarkBold} &\textcolor{green}{\CheckmarkBold} &\textcolor{red}{\XSolidBrush} &\textcolor{green}{\CheckmarkBold}&\textcolor{green}{\CheckmarkBold}&\textcolor{green}{\CheckmarkBold} &\textcolor{green}{\CheckmarkBold}\\
AgentEvol~\citep{xi2024agentgym} & AgentTraj-L & IFT & - & 14.5K & 24 &\textcolor{red}{\XSolidBrush} & \textcolor{green}{\CheckmarkBold} &\textcolor{green}{\CheckmarkBold}&\textcolor{red}{\XSolidBrush} &\textcolor{green}{\CheckmarkBold}&\textcolor{red}{\XSolidBrush} &\textcolor{red}{\XSolidBrush} &\textcolor{green}{\CheckmarkBold}\\
Lumos~\cite{yin2024agent} & Lumos & IFT  & - & 20.0K & 16 &\textcolor{red}{\XSolidBrush} & \textcolor{green}{\CheckmarkBold} & \textcolor{green}{\CheckmarkBold} &\textcolor{red}{\XSolidBrush} & \textcolor{green}{\CheckmarkBold} & \textcolor{green}{\CheckmarkBold} &\textcolor{red}{\XSolidBrush} & \textcolor{green}{\CheckmarkBold}\\
Agent-FLAN~\cite{chen2024agent} & Agent-FLAN & IFT & - & 24.7K & 20 &\textcolor{red}{\XSolidBrush} & \textcolor{green}{\CheckmarkBold} & \textcolor{green}{\CheckmarkBold} &\textcolor{red}{\XSolidBrush} & \textcolor{green}{\CheckmarkBold}& \textcolor{green}{\CheckmarkBold}&\textcolor{red}{\XSolidBrush} & \textcolor{green}{\CheckmarkBold}\\
AgentTuning~\citep{zeng2023agenttuning} & AgentInstruct & IFT & - & 35.0K & - &\textcolor{red}{\XSolidBrush} & \textcolor{green}{\CheckmarkBold} & \textcolor{green}{\CheckmarkBold} &\textcolor{red}{\XSolidBrush} & \textcolor{green}{\CheckmarkBold} &\textcolor{red}{\XSolidBrush} &\textcolor{red}{\XSolidBrush} & \textcolor{green}{\CheckmarkBold}\\\midrule
\multicolumn{13}{l}{\emph{Instruction Finetuning-based LLM Agents for Function Calling}} \\\midrule
NexusRaven~\citep{srinivasan2023nexusraven} & NexusRaven & IFT & - & - & 116 & \textcolor{green}{\CheckmarkBold} & \textcolor{green}{\CheckmarkBold}  & \textcolor{green}{\CheckmarkBold} &\textcolor{red}{\XSolidBrush} & \textcolor{green}{\CheckmarkBold} &\textcolor{red}{\XSolidBrush} &\textcolor{red}{\XSolidBrush}&\textcolor{red}{\XSolidBrush}\\
Gorilla~\citep{patil2023gorilla} & Gorilla & IFT & - & 16.0K & 1,645 & \textcolor{green}{\CheckmarkBold} &\textcolor{red}{\XSolidBrush} &\textcolor{red}{\XSolidBrush}&\textcolor{green}{\CheckmarkBold} &\textcolor{green}{\CheckmarkBold} &\textcolor{red}{\XSolidBrush} &\textcolor{red}{\XSolidBrush} &\textcolor{red}{\XSolidBrush}\\
OpenFunctions-v2~\citep{patil2023gorilla} & OpenFunctions-v2 & IFT & - & 65.0K & - & \textcolor{green}{\CheckmarkBold} & \textcolor{green}{\CheckmarkBold} &\textcolor{red}{\XSolidBrush} &\textcolor{green}{\CheckmarkBold} &\textcolor{green}{\CheckmarkBold} &\textcolor{red}{\XSolidBrush} &\textcolor{red}{\XSolidBrush} &\textcolor{red}{\XSolidBrush}\\
API Pack~\cite{guo2024api} & API Pack & IFT & - & 1.1M & 11,213 &\textcolor{green}{\CheckmarkBold} &\textcolor{red}{\XSolidBrush} &\textcolor{green}{\CheckmarkBold} &\textcolor{red}{\XSolidBrush} &\textcolor{green}{\CheckmarkBold} &\textcolor{red}{\XSolidBrush}&\textcolor{red}{\XSolidBrush}&\textcolor{red}{\XSolidBrush}\\ 
LAM~\citep{zhang2024agentohana} & AgentOhana & IFT & - & 42.6K & - & \textcolor{green}{\CheckmarkBold} & \textcolor{green}{\CheckmarkBold} &\textcolor{green}{\CheckmarkBold}&\textcolor{red}{\XSolidBrush} &\textcolor{green}{\CheckmarkBold}&\textcolor{red}{\XSolidBrush}&\textcolor{green}{\CheckmarkBold}&\textcolor{green}{\CheckmarkBold}\\
xLAM~\citep{liu2024apigen} & APIGen & IFT & - & 60.0K & 3,673 & \textcolor{green}{\CheckmarkBold} & \textcolor{green}{\CheckmarkBold} &\textcolor{green}{\CheckmarkBold}&\textcolor{red}{\XSolidBrush} &\textcolor{green}{\CheckmarkBold}&\textcolor{red}{\XSolidBrush}&\textcolor{green}{\CheckmarkBold}&\textcolor{green}{\CheckmarkBold}\\\midrule
\multicolumn{13}{l}{\emph{Pretraining-based LLM Agents}}  \\\midrule
% LEMUR~\citep{xu2024lemur} & PT & 90B & 300.0K & - & \textcolor{green}{\CheckmarkBold} & \textcolor{green}{\CheckmarkBold} &\textcolor{green}{\CheckmarkBold}&\textcolor{red}{\XSolidBrush} & \textcolor{red}{\XSolidBrush} &\textcolor{green}{\CheckmarkBold} &\textcolor{red}{\XSolidBrush}&\textcolor{red}{\XSolidBrush}\\
\rowcolor{teal!12} \method & \dataset & PT & 103B & 95.0K  & 76,537  & \textcolor{green}{\CheckmarkBold} & \textcolor{green}{\CheckmarkBold} & \textcolor{green}{\CheckmarkBold} & \textcolor{green}{\CheckmarkBold} & \textcolor{green}{\CheckmarkBold} & \textcolor{green}{\CheckmarkBold} & \textcolor{green}{\CheckmarkBold} & \textcolor{green}{\CheckmarkBold}\\
\bottomrule
\end{tabular}
% \begin{tablenotes}
%     \item $^*$ In addition, the StarCoder-API can offer 4.77M more APIs.
% \end{tablenotes}
\caption{Summary of existing instruction finetuning-based LLM agents for intrinsic reasoning and function calling, along with their training resources and sample sizes. "PT" and "IFT" denote "Pre-Training" and "Instruction Fine-Tuning", respectively.}
\vspace{-2ex}
\label{tab:related}
\end{threeparttable}
\end{table*}

\noindent \textbf{Prompting-based LLM Agents.} Due to the lack of agent-specific pre-training corpus, existing LLM agents rely on either prompt engineering~\cite{hsieh2023tool,lu2024chameleon,yao2022react,wang2023voyager} or instruction fine-tuning~\cite{chen2023fireact,zeng2023agenttuning} to understand human instructions, decompose high-level tasks, generate grounded plans, and execute multi-step actions. 
However, prompting-based methods mainly depend on the capabilities of backbone LLMs (usually commercial LLMs), failing to introduce new knowledge and struggling to generalize to unseen tasks~\cite{sun2024adaplanner,zhuang2023toolchain}. 

\noindent \textbf{Instruction Finetuning-based LLM Agents.} Considering the extensive diversity of APIs and the complexity of multi-tool instructions, tool learning inherently presents greater challenges than natural language tasks, such as text generation~\cite{qin2023toolllm}.
Post-training techniques focus more on instruction following and aligning output with specific formats~\cite{patil2023gorilla,hao2024toolkengpt,qin2023toolllm,schick2024toolformer}, rather than fundamentally improving model knowledge or capabilities. 
Moreover, heavy fine-tuning can hinder generalization or even degrade performance in non-agent use cases, potentially suppressing the original base model capabilities~\cite{ghosh2024a}.

\noindent \textbf{Pretraining-based LLM Agents.} While pre-training serves as an essential alternative, prior works~\cite{nijkamp2023codegen,roziere2023code,xu2024lemur,patil2023gorilla} have primarily focused on improving task-specific capabilities (\eg, code generation) instead of general-domain LLM agents, due to single-source, uni-type, small-scale, and poor-quality pre-training data. 
Existing tool documentation data for agent training either lacks diverse real-world APIs~\cite{patil2023gorilla, tang2023toolalpaca} or is constrained to single-tool or single-round tool execution. 
Furthermore, trajectory data mostly imitate expert behavior or follow function-calling rules with inferior planning and reasoning, failing to fully elicit LLMs' capabilities and handle complex instructions~\cite{qin2023toolllm}. 
Given a wide range of candidate API functions, each comprising various function names and parameters available at every planning step, identifying globally optimal solutions and generalizing across tasks remains highly challenging.



\section{Preliminaries}
\label{Preliminaries}
\begin{figure*}[t]
    \centering
    \includegraphics[width=0.95\linewidth]{fig/HealthGPT_Framework.png}
    \caption{The \ourmethod{} architecture integrates hierarchical visual perception and H-LoRA, employing a task-specific hard router to select visual features and H-LoRA plugins, ultimately generating outputs with an autoregressive manner.}
    \label{fig:architecture}
\end{figure*}
\noindent\textbf{Large Vision-Language Models.} 
The input to a LVLM typically consists of an image $x^{\text{img}}$ and a discrete text sequence $x^{\text{txt}}$. The visual encoder $\mathcal{E}^{\text{img}}$ converts the input image $x^{\text{img}}$ into a sequence of visual tokens $\mathcal{V} = [v_i]_{i=1}^{N_v}$, while the text sequence $x^{\text{txt}}$ is mapped into a sequence of text tokens $\mathcal{T} = [t_i]_{i=1}^{N_t}$ using an embedding function $\mathcal{E}^{\text{txt}}$. The LLM $\mathcal{M_\text{LLM}}(\cdot|\theta)$ models the joint probability of the token sequence $\mathcal{U} = \{\mathcal{V},\mathcal{T}\}$, which is expressed as:
\begin{equation}
    P_\theta(R | \mathcal{U}) = \prod_{i=1}^{N_r} P_\theta(r_i | \{\mathcal{U}, r_{<i}\}),
\end{equation}
where $R = [r_i]_{i=1}^{N_r}$ is the text response sequence. The LVLM iteratively generates the next token $r_i$ based on $r_{<i}$. The optimization objective is to minimize the cross-entropy loss of the response $\mathcal{R}$.
% \begin{equation}
%     \mathcal{L}_{\text{VLM}} = \mathbb{E}_{R|\mathcal{U}}\left[-\log P_\theta(R | \mathcal{U})\right]
% \end{equation}
It is worth noting that most LVLMs adopt a design paradigm based on ViT, alignment adapters, and pre-trained LLMs\cite{liu2023llava,liu2024improved}, enabling quick adaptation to downstream tasks.


\noindent\textbf{VQGAN.}
VQGAN~\cite{esser2021taming} employs latent space compression and indexing mechanisms to effectively learn a complete discrete representation of images. VQGAN first maps the input image $x^{\text{img}}$ to a latent representation $z = \mathcal{E}(x)$ through a encoder $\mathcal{E}$. Then, the latent representation is quantized using a codebook $\mathcal{Z} = \{z_k\}_{k=1}^K$, generating a discrete index sequence $\mathcal{I} = [i_m]_{m=1}^N$, where $i_m \in \mathcal{Z}$ represents the quantized code index:
\begin{equation}
    \mathcal{I} = \text{Quantize}(z|\mathcal{Z}) = \arg\min_{z_k \in \mathcal{Z}} \| z - z_k \|_2.
\end{equation}
In our approach, the discrete index sequence $\mathcal{I}$ serves as a supervisory signal for the generation task, enabling the model to predict the index sequence $\hat{\mathcal{I}}$ from input conditions such as text or other modality signals.  
Finally, the predicted index sequence $\hat{\mathcal{I}}$ is upsampled by the VQGAN decoder $G$, generating the high-quality image $\hat{x}^\text{img} = G(\hat{\mathcal{I}})$.



\noindent\textbf{Low Rank Adaptation.} 
LoRA\cite{hu2021lora} effectively captures the characteristics of downstream tasks by introducing low-rank adapters. The core idea is to decompose the bypass weight matrix $\Delta W\in\mathbb{R}^{d^{\text{in}} \times d^{\text{out}}}$ into two low-rank matrices $ \{A \in \mathbb{R}^{d^{\text{in}} \times r}, B \in \mathbb{R}^{r \times d^{\text{out}}} \}$, where $ r \ll \min\{d^{\text{in}}, d^{\text{out}}\} $, significantly reducing learnable parameters. The output with the LoRA adapter for the input $x$ is then given by:
\begin{equation}
    h = x W_0 + \alpha x \Delta W/r = x W_0 + \alpha xAB/r,
\end{equation}
where matrix $ A $ is initialized with a Gaussian distribution, while the matrix $ B $ is initialized as a zero matrix. The scaling factor $ \alpha/r $ controls the impact of $ \Delta W $ on the model.

\section{HealthGPT}
\label{Method}


\subsection{Unified Autoregressive Generation.}  
% As shown in Figure~\ref{fig:architecture}, 
\ourmethod{} (Figure~\ref{fig:architecture}) utilizes a discrete token representation that covers both text and visual outputs, unifying visual comprehension and generation as an autoregressive task. 
For comprehension, $\mathcal{M}_\text{llm}$ receives the input joint sequence $\mathcal{U}$ and outputs a series of text token $\mathcal{R} = [r_1, r_2, \dots, r_{N_r}]$, where $r_i \in \mathcal{V}_{\text{txt}}$, and $\mathcal{V}_{\text{txt}}$ represents the LLM's vocabulary:
\begin{equation}
    P_\theta(\mathcal{R} \mid \mathcal{U}) = \prod_{i=1}^{N_r} P_\theta(r_i \mid \mathcal{U}, r_{<i}).
\end{equation}
For generation, $\mathcal{M}_\text{llm}$ first receives a special start token $\langle \text{START\_IMG} \rangle$, then generates a series of tokens corresponding to the VQGAN indices $\mathcal{I} = [i_1, i_2, \dots, i_{N_i}]$, where $i_j \in \mathcal{V}_{\text{vq}}$, and $\mathcal{V}_{\text{vq}}$ represents the index range of VQGAN. Upon completion of generation, the LLM outputs an end token $\langle \text{END\_IMG} \rangle$:
\begin{equation}
    P_\theta(\mathcal{I} \mid \mathcal{U}) = \prod_{j=1}^{N_i} P_\theta(i_j \mid \mathcal{U}, i_{<j}).
\end{equation}
Finally, the generated index sequence $\mathcal{I}$ is fed into the decoder $G$, which reconstructs the target image $\hat{x}^{\text{img}} = G(\mathcal{I})$.

\subsection{Hierarchical Visual Perception}  
Given the differences in visual perception between comprehension and generation tasks—where the former focuses on abstract semantics and the latter emphasizes complete semantics—we employ ViT to compress the image into discrete visual tokens at multiple hierarchical levels.
Specifically, the image is converted into a series of features $\{f_1, f_2, \dots, f_L\}$ as it passes through $L$ ViT blocks.

To address the needs of various tasks, the hidden states are divided into two types: (i) \textit{Concrete-grained features} $\mathcal{F}^{\text{Con}} = \{f_1, f_2, \dots, f_k\}, k < L$, derived from the shallower layers of ViT, containing sufficient global features, suitable for generation tasks; 
(ii) \textit{Abstract-grained features} $\mathcal{F}^{\text{Abs}} = \{f_{k+1}, f_{k+2}, \dots, f_L\}$, derived from the deeper layers of ViT, which contain abstract semantic information closer to the text space, suitable for comprehension tasks.

The task type $T$ (comprehension or generation) determines which set of features is selected as the input for the downstream large language model:
\begin{equation}
    \mathcal{F}^{\text{img}}_T =
    \begin{cases}
        \mathcal{F}^{\text{Con}}, & \text{if } T = \text{generation task} \\
        \mathcal{F}^{\text{Abs}}, & \text{if } T = \text{comprehension task}
    \end{cases}
\end{equation}
We integrate the image features $\mathcal{F}^{\text{img}}_T$ and text features $\mathcal{T}$ into a joint sequence through simple concatenation, which is then fed into the LLM $\mathcal{M}_{\text{llm}}$ for autoregressive generation.
% :
% \begin{equation}
%     \mathcal{R} = \mathcal{M}_{\text{llm}}(\mathcal{U}|\theta), \quad \mathcal{U} = [\mathcal{F}^{\text{img}}_T; \mathcal{T}]
% \end{equation}
\subsection{Heterogeneous Knowledge Adaptation}
We devise H-LoRA, which stores heterogeneous knowledge from comprehension and generation tasks in separate modules and dynamically routes to extract task-relevant knowledge from these modules. 
At the task level, for each task type $ T $, we dynamically assign a dedicated H-LoRA submodule $ \theta^T $, which is expressed as:
\begin{equation}
    \mathcal{R} = \mathcal{M}_\text{LLM}(\mathcal{U}|\theta, \theta^T), \quad \theta^T = \{A^T, B^T, \mathcal{R}^T_\text{outer}\}.
\end{equation}
At the feature level for a single task, H-LoRA integrates the idea of Mixture of Experts (MoE)~\cite{masoudnia2014mixture} and designs an efficient matrix merging and routing weight allocation mechanism, thus avoiding the significant computational delay introduced by matrix splitting in existing MoELoRA~\cite{luo2024moelora}. Specifically, we first merge the low-rank matrices (rank = r) of $ k $ LoRA experts into a unified matrix:
\begin{equation}
    \mathbf{A}^{\text{merged}}, \mathbf{B}^{\text{merged}} = \text{Concat}(\{A_i\}_1^k), \text{Concat}(\{B_i\}_1^k),
\end{equation}
where $ \mathbf{A}^{\text{merged}} \in \mathbb{R}^{d^\text{in} \times rk} $ and $ \mathbf{B}^{\text{merged}} \in \mathbb{R}^{rk \times d^\text{out}} $. The $k$-dimension routing layer generates expert weights $ \mathcal{W} \in \mathbb{R}^{\text{token\_num} \times k} $ based on the input hidden state $ x $, and these are expanded to $ \mathbb{R}^{\text{token\_num} \times rk} $ as follows:
\begin{equation}
    \mathcal{W}^\text{expanded} = \alpha k \mathcal{W} / r \otimes \mathbf{1}_r,
\end{equation}
where $ \otimes $ denotes the replication operation.
The overall output of H-LoRA is computed as:
\begin{equation}
    \mathcal{O}^\text{H-LoRA} = (x \mathbf{A}^{\text{merged}} \odot \mathcal{W}^\text{expanded}) \mathbf{B}^{\text{merged}},
\end{equation}
where $ \odot $ represents element-wise multiplication. Finally, the output of H-LoRA is added to the frozen pre-trained weights to produce the final output:
\begin{equation}
    \mathcal{O} = x W_0 + \mathcal{O}^\text{H-LoRA}.
\end{equation}
% In summary, H-LoRA is a task-based dynamic PEFT method that achieves high efficiency in single-task fine-tuning.

\subsection{Training Pipeline}

\begin{figure}[t]
    \centering
    \hspace{-4mm}
    \includegraphics[width=0.94\linewidth]{fig/data.pdf}
    \caption{Data statistics of \texttt{VL-Health}. }
    \label{fig:data}
\end{figure}
\noindent \textbf{1st Stage: Multi-modal Alignment.} 
In the first stage, we design separate visual adapters and H-LoRA submodules for medical unified tasks. For the medical comprehension task, we train abstract-grained visual adapters using high-quality image-text pairs to align visual embeddings with textual embeddings, thereby enabling the model to accurately describe medical visual content. During this process, the pre-trained LLM and its corresponding H-LoRA submodules remain frozen. In contrast, the medical generation task requires training concrete-grained adapters and H-LoRA submodules while keeping the LLM frozen. Meanwhile, we extend the textual vocabulary to include multimodal tokens, enabling the support of additional VQGAN vector quantization indices. The model trains on image-VQ pairs, endowing the pre-trained LLM with the capability for image reconstruction. This design ensures pixel-level consistency of pre- and post-LVLM. The processes establish the initial alignment between the LLM’s outputs and the visual inputs.

\noindent \textbf{2nd Stage: Heterogeneous H-LoRA Plugin Adaptation.}  
The submodules of H-LoRA share the word embedding layer and output head but may encounter issues such as bias and scale inconsistencies during training across different tasks. To ensure that the multiple H-LoRA plugins seamlessly interface with the LLMs and form a unified base, we fine-tune the word embedding layer and output head using a small amount of mixed data to maintain consistency in the model weights. Specifically, during this stage, all H-LoRA submodules for different tasks are kept frozen, with only the word embedding layer and output head being optimized. Through this stage, the model accumulates foundational knowledge for unified tasks by adapting H-LoRA plugins.

\begin{table*}[!t]
\centering
\caption{Comparison of \ourmethod{} with other LVLMs and unified multi-modal models on medical visual comprehension tasks. \textbf{Bold} and \underline{underlined} text indicates the best performance and second-best performance, respectively.}
\resizebox{\textwidth}{!}{
\begin{tabular}{c|lcc|cccccccc|c}
\toprule
\rowcolor[HTML]{E9F3FE} &  &  &  & \multicolumn{2}{c}{\textbf{VQA-RAD \textuparrow}} & \multicolumn{2}{c}{\textbf{SLAKE \textuparrow}} & \multicolumn{2}{c}{\textbf{PathVQA \textuparrow}} &  &  &  \\ 
\cline{5-10}
\rowcolor[HTML]{E9F3FE}\multirow{-2}{*}{\textbf{Type}} & \multirow{-2}{*}{\textbf{Model}} & \multirow{-2}{*}{\textbf{\# Params}} & \multirow{-2}{*}{\makecell{\textbf{Medical} \\ \textbf{LVLM}}} & \textbf{close} & \textbf{all} & \textbf{close} & \textbf{all} & \textbf{close} & \textbf{all} & \multirow{-2}{*}{\makecell{\textbf{MMMU} \\ \textbf{-Med}}\textuparrow} & \multirow{-2}{*}{\textbf{OMVQA}\textuparrow} & \multirow{-2}{*}{\textbf{Avg. \textuparrow}} \\ 
\midrule \midrule
\multirow{9}{*}{\textbf{Comp. Only}} 
& Med-Flamingo & 8.3B & \Large \ding{51} & 58.6 & 43.0 & 47.0 & 25.5 & 61.9 & 31.3 & 28.7 & 34.9 & 41.4 \\
& LLaVA-Med & 7B & \Large \ding{51} & 60.2 & 48.1 & 58.4 & 44.8 & 62.3 & 35.7 & 30.0 & 41.3 & 47.6 \\
& HuatuoGPT-Vision & 7B & \Large \ding{51} & 66.9 & 53.0 & 59.8 & 49.1 & 52.9 & 32.0 & 42.0 & 50.0 & 50.7 \\
& BLIP-2 & 6.7B & \Large \ding{55} & 43.4 & 36.8 & 41.6 & 35.3 & 48.5 & 28.8 & 27.3 & 26.9 & 36.1 \\
& LLaVA-v1.5 & 7B & \Large \ding{55} & 51.8 & 42.8 & 37.1 & 37.7 & 53.5 & 31.4 & 32.7 & 44.7 & 41.5 \\
& InstructBLIP & 7B & \Large \ding{55} & 61.0 & 44.8 & 66.8 & 43.3 & 56.0 & 32.3 & 25.3 & 29.0 & 44.8 \\
& Yi-VL & 6B & \Large \ding{55} & 52.6 & 42.1 & 52.4 & 38.4 & 54.9 & 30.9 & 38.0 & 50.2 & 44.9 \\
& InternVL2 & 8B & \Large \ding{55} & 64.9 & 49.0 & 66.6 & 50.1 & 60.0 & 31.9 & \underline{43.3} & 54.5 & 52.5\\
& Llama-3.2 & 11B & \Large \ding{55} & 68.9 & 45.5 & 72.4 & 52.1 & 62.8 & 33.6 & 39.3 & 63.2 & 54.7 \\
\midrule
\multirow{5}{*}{\textbf{Comp. \& Gen.}} 
& Show-o & 1.3B & \Large \ding{55} & 50.6 & 33.9 & 31.5 & 17.9 & 52.9 & 28.2 & 22.7 & 45.7 & 42.6 \\
& Unified-IO 2 & 7B & \Large \ding{55} & 46.2 & 32.6 & 35.9 & 21.9 & 52.5 & 27.0 & 25.3 & 33.0 & 33.8 \\
& Janus & 1.3B & \Large \ding{55} & 70.9 & 52.8 & 34.7 & 26.9 & 51.9 & 27.9 & 30.0 & 26.8 & 33.5 \\
& \cellcolor[HTML]{DAE0FB}HealthGPT-M3 & \cellcolor[HTML]{DAE0FB}3.8B & \cellcolor[HTML]{DAE0FB}\Large \ding{51} & \cellcolor[HTML]{DAE0FB}\underline{73.7} & \cellcolor[HTML]{DAE0FB}\underline{55.9} & \cellcolor[HTML]{DAE0FB}\underline{74.6} & \cellcolor[HTML]{DAE0FB}\underline{56.4} & \cellcolor[HTML]{DAE0FB}\underline{78.7} & \cellcolor[HTML]{DAE0FB}\underline{39.7} & \cellcolor[HTML]{DAE0FB}\underline{43.3} & \cellcolor[HTML]{DAE0FB}\underline{68.5} & \cellcolor[HTML]{DAE0FB}\underline{61.3} \\
& \cellcolor[HTML]{DAE0FB}HealthGPT-L14 & \cellcolor[HTML]{DAE0FB}14B & \cellcolor[HTML]{DAE0FB}\Large \ding{51} & \cellcolor[HTML]{DAE0FB}\textbf{77.7} & \cellcolor[HTML]{DAE0FB}\textbf{58.3} & \cellcolor[HTML]{DAE0FB}\textbf{76.4} & \cellcolor[HTML]{DAE0FB}\textbf{64.5} & \cellcolor[HTML]{DAE0FB}\textbf{85.9} & \cellcolor[HTML]{DAE0FB}\textbf{44.4} & \cellcolor[HTML]{DAE0FB}\textbf{49.2} & \cellcolor[HTML]{DAE0FB}\textbf{74.4} & \cellcolor[HTML]{DAE0FB}\textbf{66.4} \\
\bottomrule
\end{tabular}
}
\label{tab:results}
\end{table*}
\begin{table*}[ht]
    \centering
    \caption{The experimental results for the four modality conversion tasks.}
    \resizebox{\textwidth}{!}{
    \begin{tabular}{l|ccc|ccc|ccc|ccc}
        \toprule
        \rowcolor[HTML]{E9F3FE} & \multicolumn{3}{c}{\textbf{CT to MRI (Brain)}} & \multicolumn{3}{c}{\textbf{CT to MRI (Pelvis)}} & \multicolumn{3}{c}{\textbf{MRI to CT (Brain)}} & \multicolumn{3}{c}{\textbf{MRI to CT (Pelvis)}} \\
        \cline{2-13}
        \rowcolor[HTML]{E9F3FE}\multirow{-2}{*}{\textbf{Model}}& \textbf{SSIM $\uparrow$} & \textbf{PSNR $\uparrow$} & \textbf{MSE $\downarrow$} & \textbf{SSIM $\uparrow$} & \textbf{PSNR $\uparrow$} & \textbf{MSE $\downarrow$} & \textbf{SSIM $\uparrow$} & \textbf{PSNR $\uparrow$} & \textbf{MSE $\downarrow$} & \textbf{SSIM $\uparrow$} & \textbf{PSNR $\uparrow$} & \textbf{MSE $\downarrow$} \\
        \midrule \midrule
        pix2pix & 71.09 & 32.65 & 36.85 & 59.17 & 31.02 & 51.91 & 78.79 & 33.85 & 28.33 & 72.31 & 32.98 & 36.19 \\
        CycleGAN & 54.76 & 32.23 & 40.56 & 54.54 & 30.77 & 55.00 & 63.75 & 31.02 & 52.78 & 50.54 & 29.89 & 67.78 \\
        BBDM & {71.69} & {32.91} & {34.44} & 57.37 & 31.37 & 48.06 & \textbf{86.40} & 34.12 & 26.61 & {79.26} & 33.15 & 33.60 \\
        Vmanba & 69.54 & 32.67 & 36.42 & {63.01} & {31.47} & {46.99} & 79.63 & 34.12 & 26.49 & 77.45 & 33.53 & 31.85 \\
        DiffMa & 71.47 & 32.74 & 35.77 & 62.56 & 31.43 & 47.38 & 79.00 & {34.13} & {26.45} & 78.53 & {33.68} & {30.51} \\
        \rowcolor[HTML]{DAE0FB}HealthGPT-M3 & \underline{79.38} & \underline{33.03} & \underline{33.48} & \underline{71.81} & \underline{31.83} & \underline{43.45} & {85.06} & \textbf{34.40} & \textbf{25.49} & \underline{84.23} & \textbf{34.29} & \textbf{27.99} \\
        \rowcolor[HTML]{DAE0FB}HealthGPT-L14 & \textbf{79.73} & \textbf{33.10} & \textbf{32.96} & \textbf{71.92} & \textbf{31.87} & \textbf{43.09} & \underline{85.31} & \underline{34.29} & \underline{26.20} & \textbf{84.96} & \underline{34.14} & \underline{28.13} \\
        \bottomrule
    \end{tabular}
    }
    \label{tab:conversion}
\end{table*}

\noindent \textbf{3rd Stage: Visual Instruction Fine-Tuning.}  
In the third stage, we introduce additional task-specific data to further optimize the model and enhance its adaptability to downstream tasks such as medical visual comprehension (e.g., medical QA, medical dialogues, and report generation) or generation tasks (e.g., super-resolution, denoising, and modality conversion). Notably, by this stage, the word embedding layer and output head have been fine-tuned, only the H-LoRA modules and adapter modules need to be trained. This strategy significantly improves the model's adaptability and flexibility across different tasks.


\section{Experiment}
\label{s:experiment}

\subsection{Data Description}
We evaluate our method on FI~\cite{you2016building}, Twitter\_LDL~\cite{yang2017learning} and Artphoto~\cite{machajdik2010affective}.
FI is a public dataset built from Flickr and Instagram, with 23,308 images and eight emotion categories, namely \textit{amusement}, \textit{anger}, \textit{awe},  \textit{contentment}, \textit{disgust}, \textit{excitement},  \textit{fear}, and \textit{sadness}. 
% Since images in FI are all copyrighted by law, some images are corrupted now, so we remove these samples and retain 21,828 images.
% T4SA contains images from Twitter, which are classified into three categories: \textit{positive}, \textit{neutral}, and \textit{negative}. In this paper, we adopt the base version of B-T4SA, which contains 470,586 images and provides text descriptions of the corresponding tweets.
Twitter\_LDL contains 10,045 images from Twitter, with the same eight categories as the FI dataset.
% 。
For these two datasets, they are randomly split into 80\%
training and 20\% testing set.
Artphoto contains 806 artistic photos from the DeviantArt website, which we use to further evaluate the zero-shot capability of our model.
% on the small-scale dataset.
% We construct and publicly release the first image sentiment analysis dataset containing metadata.
% 。

% Based on these datasets, we are the first to construct and publicly release metadata-enhanced image sentiment analysis datasets. These datasets include scenes, tags, descriptions, and corresponding confidence scores, and are available at this link for future research purposes.


% 
\begin{table}[t]
\centering
% \begin{center}
\caption{Overall performance of different models on FI and Twitter\_LDL datasets.}
\label{tab:cap1}
% \resizebox{\linewidth}{!}
{
\begin{tabular}{l|c|c|c|c}
\hline
\multirow{2}{*}{\textbf{Model}} & \multicolumn{2}{c|}{\textbf{FI}}  & \multicolumn{2}{c}{\textbf{Twitter\_LDL}} \\ \cline{2-5} 
  & \textbf{Accuracy} & \textbf{F1} & \textbf{Accuracy} & \textbf{F1}  \\ \hline
% (\rownumber)~AlexNet~\cite{krizhevsky2017imagenet}  & 58.13\% & 56.35\%  & 56.24\%& 55.02\%  \\ 
% (\rownumber)~VGG16~\cite{simonyan2014very}  & 63.75\%& 63.08\%  & 59.34\%& 59.02\%  \\ 
(\rownumber)~ResNet101~\cite{he2016deep} & 66.16\%& 65.56\%  & 62.02\% & 61.34\%  \\ 
(\rownumber)~CDA~\cite{han2023boosting} & 66.71\%& 65.37\%  & 64.14\% & 62.85\%  \\ 
(\rownumber)~CECCN~\cite{ruan2024color} & 67.96\%& 66.74\%  & 64.59\%& 64.72\% \\ 
(\rownumber)~EmoVIT~\cite{xie2024emovit} & 68.09\%& 67.45\%  & 63.12\% & 61.97\%  \\ 
(\rownumber)~ComLDL~\cite{zhang2022compound} & 68.83\%& 67.28\%  & 65.29\% & 63.12\%  \\ 
(\rownumber)~WSDEN~\cite{li2023weakly} & 69.78\%& 69.61\%  & 67.04\% & 65.49\% \\ 
(\rownumber)~ECWA~\cite{deng2021emotion} & 70.87\%& 69.08\%  & 67.81\% & 66.87\%  \\ 
(\rownumber)~EECon~\cite{yang2023exploiting} & 71.13\%& 68.34\%  & 64.27\%& 63.16\%  \\ 
(\rownumber)~MAM~\cite{zhang2024affective} & 71.44\%  & 70.83\% & 67.18\%  & 65.01\%\\ 
(\rownumber)~TGCA-PVT~\cite{chen2024tgca}   & 73.05\%  & 71.46\% & 69.87\%  & 68.32\% \\ 
(\rownumber)~OEAN~\cite{zhang2024object}   & 73.40\%  & 72.63\% & 70.52\%  & 69.47\% \\ \hline
(\rownumber)~\shortname  & \textbf{79.48\%} & \textbf{79.22\%} & \textbf{74.12\%} & \textbf{73.09\%} \\ \hline
\end{tabular}
}
\vspace{-6mm}
% \end{center}
\end{table}
% 

\subsection{Experiment Setting}
% \subsubsection{Model Setting.}
% 
\textbf{Model Setting:}
For feature representation, we set $k=10$ to select object tags, and adopt clip-vit-base-patch32 as the pre-trained model for unified feature representation.
Moreover, we empirically set $(d_e, d_h, d_k, d_s) = (512, 128, 16, 64)$, and set the classification class $L$ to 8.

% 

\textbf{Training Setting:}
To initialize the model, we set all weights such as $\boldsymbol{W}$ following the truncated normal distribution, and use AdamW optimizer with the learning rate of $1 \times 10^{-4}$.
% warmup scheduler of cosine, warmup steps of 2000.
Furthermore, we set the batch size to 32 and the epoch of the training process to 200.
During the implementation, we utilize \textit{PyTorch} to build our entire model.
% , and our project codes are publicly available at https://github.com/zzmyrep/MESN.
% Our project codes as well as data are all publicly available on GitHub\footnote{https://github.com/zzmyrep/KBCEN}.
% Code is available at \href{https://github.com/zzmyrep/KBCEN}{https://github.com/zzmyrep/KBCEN}.

\textbf{Evaluation Metrics:}
Following~\cite{zhang2024affective, chen2024tgca, zhang2024object}, we adopt \textit{accuracy} and \textit{F1} as our evaluation metrics to measure the performance of different methods for image sentiment analysis. 



\subsection{Experiment Result}
% We compare our model against the following baselines: AlexNet~\cite{krizhevsky2017imagenet}, VGG16~\cite{simonyan2014very}, ResNet101~\cite{he2016deep}, CECCN~\cite{ruan2024color}, EmoVIT~\cite{xie2024emovit}, WSCNet~\cite{yang2018weakly}, ECWA~\cite{deng2021emotion}, EECon~\cite{yang2023exploiting}, MAM~\cite{zhang2024affective} and TGCA-PVT~\cite{chen2024tgca}, and the overall results are summarized in Table~\ref{tab:cap1}.
We compare our model against several baselines, and the overall results are summarized in Table~\ref{tab:cap1}.
We observe that our model achieves the best performance in both accuracy and F1 metrics, significantly outperforming the previous models. 
This superior performance is mainly attributed to our effective utilization of metadata to enhance image sentiment analysis, as well as the exceptional capability of the unified sentiment transformer framework we developed. These results strongly demonstrate that our proposed method can bring encouraging performance for image sentiment analysis.

\setcounter{magicrownumbers}{0} 
\begin{table}[t]
\begin{center}
\caption{Ablation study of~\shortname~on FI dataset.} 
% \vspace{1mm}
\label{tab:cap2}
\resizebox{.9\linewidth}{!}
{
\begin{tabular}{lcc}
  \hline
  \textbf{Model} & \textbf{Accuracy} & \textbf{F1} \\
  \hline
  (\rownumber)~Ours (w/o vision) & 65.72\% & 64.54\% \\
  (\rownumber)~Ours (w/o text description) & 74.05\% & 72.58\% \\
  (\rownumber)~Ours (w/o object tag) & 77.45\% & 76.84\% \\
  (\rownumber)~Ours (w/o scene tag) & 78.47\% & 78.21\% \\
  \hline
  (\rownumber)~Ours (w/o unified embedding) & 76.41\% & 76.23\% \\
  (\rownumber)~Ours (w/o adaptive learning) & 76.83\% & 76.56\% \\
  (\rownumber)~Ours (w/o cross-modal fusion) & 76.85\% & 76.49\% \\
  \hline
  (\rownumber)~Ours  & \textbf{79.48\%} & \textbf{79.22\%} \\
  \hline
\end{tabular}
}
\end{center}
\vspace{-5mm}
\end{table}


\begin{figure}[t]
\centering
% \vspace{-2mm}
\includegraphics[width=0.42\textwidth]{fig/2dvisual-linux4-paper2.pdf}
\caption{Visualization of feature distribution on eight categories before (left) and after (right) model processing.}
% 
\label{fig:visualization}
\vspace{-5mm}
\end{figure}

\subsection{Ablation Performance}
In this subsection, we conduct an ablation study to examine which component is really important for performance improvement. The results are reported in Table~\ref{tab:cap2}.

For information utilization, we observe a significant decline in model performance when visual features are removed. Additionally, the performance of \shortname~decreases when different metadata are removed separately, which means that text description, object tag, and scene tag are all critical for image sentiment analysis.
Recalling the model architecture, we separately remove transformer layers of the unified representation module, the adaptive learning module, and the cross-modal fusion module, replacing them with MLPs of the same parameter scale.
In this way, we can observe varying degrees of decline in model performance, indicating that these modules are indispensable for our model to achieve better performance.

\subsection{Visualization}
% 


% % 开始使用minipage进行左右排列
% \begin{minipage}[t]{0.45\textwidth}  % 子图1宽度为45%
%     \centering
%     \includegraphics[width=\textwidth]{2dvisual.pdf}  % 插入图片
%     \captionof{figure}{Visualization of feature distribution.}  % 使用captionof添加图片标题
%     \label{fig:visualization}
% \end{minipage}


% \begin{figure}[t]
% \centering
% \vspace{-2mm}
% \includegraphics[width=0.45\textwidth]{fig/2dvisual.pdf}
% \caption{Visualization of feature distribution.}
% \label{fig:visualization}
% % \vspace{-4mm}
% \end{figure}

% \begin{figure}[t]
% \centering
% \vspace{-2mm}
% \includegraphics[width=0.45\textwidth]{fig/2dvisual-linux3-paper.pdf}
% \caption{Visualization of feature distribution.}
% \label{fig:visualization}
% % \vspace{-4mm}
% \end{figure}



\begin{figure}[tbp]   
\vspace{-4mm}
  \centering            
  \subfloat[Depth of adaptive learning layers]   
  {
    \label{fig:subfig1}\includegraphics[width=0.22\textwidth]{fig/fig_sensitivity-a5}
  }
  \subfloat[Depth of fusion layers]
  {
    % \label{fig:subfig2}\includegraphics[width=0.22\textwidth]{fig/fig_sensitivity-b2}
    \label{fig:subfig2}\includegraphics[width=0.22\textwidth]{fig/fig_sensitivity-b2-num.pdf}
  }
  \caption{Sensitivity study of \shortname~on different depth. }   
  \label{fig:fig_sensitivity}  
\vspace{-2mm}
\end{figure}

% \begin{figure}[htbp]
% \centerline{\includegraphics{2dvisual.pdf}}
% \caption{Visualization of feature distribution.}
% \label{fig:visualization}
% \end{figure}

% In Fig.~\ref{fig:visualization}, we use t-SNE~\cite{van2008visualizing} to reduce the dimension of data features for visualization, Figure in left represents the metadata features before model processing, the features are obtained by embedding through the CLIP model, and figure in right shows the features of the data after model processing, it can be observed that after the model processing, the data with different label categories fall in different regions in the space, therefore, we can conclude that the Therefore, we can conclude that the model can effectively utilize the information contained in the metadata and use it to guide the model for classification.

In Fig.~\ref{fig:visualization}, we use t-SNE~\cite{van2008visualizing} to reduce the dimension of data features for visualization.
The left figure shows metadata features before being processed by our model (\textit{i.e.}, embedded by CLIP), while the right shows the distribution of features after being processed by our model.
We can observe that after the model processing, data with the same label are closer to each other, while others are farther away.
Therefore, it shows that the model can effectively utilize the information contained in the metadata and use it to guide the classification process.

\subsection{Sensitivity Analysis}
% 
In this subsection, we conduct a sensitivity analysis to figure out the effect of different depth settings of adaptive learning layers and fusion layers. 
% In this subsection, we conduct a sensitivity analysis to figure out the effect of different depth settings on the model. 
% Fig.~\ref{fig:fig_sensitivity} presents the effect of different depth settings of adaptive learning layers and fusion layers. 
Taking Fig.~\ref{fig:fig_sensitivity} (a) as an example, the model performance improves with increasing depth, reaching the best performance at a depth of 4.
% Taking Fig.~\ref{fig:fig_sensitivity} (a) as an example, the performance of \shortname~improves with the increase of depth at first, reaching the best performance at a depth of 4.
When the depth continues to increase, the accuracy decreases to varying degrees.
Similar results can be observed in Fig.~\ref{fig:fig_sensitivity} (b).
Therefore, we set their depths to 4 and 6 respectively to achieve the best results.

% Through our experiments, we can observe that the effect of modifying these hyperparameters on the results of the experiments is very weak, and the surface model is not sensitive to the hyperparameters.


\subsection{Zero-shot Capability}
% 

% (1)~GCH~\cite{2010Analyzing} & 21.78\% & (5)~RA-DLNet~\cite{2020A} & 34.01\% \\ \hline
% (2)~WSCNet~\cite{2019WSCNet}  & 30.25\% & (6)~CECCN~\cite{ruan2024color} & 43.83\% \\ \hline
% (3)~PCNN~\cite{2015Robust} & 31.68\%  & (7)~EmoVIT~\cite{xie2024emovit} & 44.90\% \\ \hline
% (4)~AR~\cite{2018Visual} & 32.67\% & (8)~Ours (Zero-shot) & 47.83\% \\ \hline


\begin{table}[t]
\centering
\caption{Zero-shot capability of \shortname.}
\label{tab:cap3}
\resizebox{1\linewidth}{!}
{
\begin{tabular}{lc|lc}
\hline
\textbf{Model} & \textbf{Accuracy} & \textbf{Model} & \textbf{Accuracy} \\ \hline
(1)~WSCNet~\cite{2019WSCNet}  & 30.25\% & (5)~MAM~\cite{zhang2024affective} & 39.56\%  \\ \hline
(2)~AR~\cite{2018Visual} & 32.67\% & (6)~CECCN~\cite{ruan2024color} & 43.83\% \\ \hline
(3)~RA-DLNet~\cite{2020A} & 34.01\%  & (7)~EmoVIT~\cite{xie2024emovit} & 44.90\% \\ \hline
(4)~CDA~\cite{han2023boosting} & 38.64\% & (8)~Ours (Zero-shot) & 47.83\% \\ \hline
\end{tabular}
}
\vspace{-5mm}
\end{table}

% We use the model trained on the FI dataset to test on the artphoto dataset to verify the model's generalization ability as well as robustness to other distributed datasets.
% We can observe that the MESN model shows strong competitiveness in terms of accuracy when compared to other trained models, which suggests that the model has a good generalization ability in the OOD task.

To validate the model's generalization ability and robustness to other distributed datasets, we directly test the model trained on the FI dataset, without training on Artphoto. 
% As observed in Table 3, compared to other models trained on Artphoto, we achieve highly competitive zero-shot performance, indicating that the model has good generalization ability in out-of-distribution tasks.
From Table~\ref{tab:cap3}, we can observe that compared with other models trained on Artphoto, we achieve competitive zero-shot performance, which shows that the model has good generalization ability in out-of-distribution tasks.


%%%%%%%%%%%%
%  E2E     %
%%%%%%%%%%%%


\section{Conclusion}
In this paper, we introduced Wi-Chat, the first LLM-powered Wi-Fi-based human activity recognition system that integrates the reasoning capabilities of large language models with the sensing potential of wireless signals. Our experimental results on a self-collected Wi-Fi CSI dataset demonstrate the promising potential of LLMs in enabling zero-shot Wi-Fi sensing. These findings suggest a new paradigm for human activity recognition that does not rely on extensive labeled data. We hope future research will build upon this direction, further exploring the applications of LLMs in signal processing domains such as IoT, mobile sensing, and radar-based systems.

\section*{Limitations}
While our work represents the first attempt to leverage LLMs for processing Wi-Fi signals, it is a preliminary study focused on a relatively simple task: Wi-Fi-based human activity recognition. This choice allows us to explore the feasibility of LLMs in wireless sensing but also comes with certain limitations.

Our approach primarily evaluates zero-shot performance, which, while promising, may still lag behind traditional supervised learning methods in highly complex or fine-grained recognition tasks. Besides, our study is limited to a controlled environment with a self-collected dataset, and the generalizability of LLMs to diverse real-world scenarios with varying Wi-Fi conditions, environmental interference, and device heterogeneity remains an open question.

Additionally, we have yet to explore the full potential of LLMs in more advanced Wi-Fi sensing applications, such as fine-grained gesture recognition, occupancy detection, and passive health monitoring. Future work should investigate the scalability of LLM-based approaches, their robustness to domain shifts, and their integration with multimodal sensing techniques in broader IoT applications.


% Bibliography entries for the entire Anthology, followed by custom entries
%\bibliography{anthology,custom}
% Custom bibliography entries only
\bibliography{main}
\newpage
\appendix

\section{Experiment prompts}
\label{sec:prompt}
The prompts used in the LLM experiments are shown in the following Table~\ref{tab:prompts}.

\definecolor{titlecolor}{rgb}{0.9, 0.5, 0.1}
\definecolor{anscolor}{rgb}{0.2, 0.5, 0.8}
\definecolor{labelcolor}{HTML}{48a07e}
\begin{table*}[h]
	\centering
	
 % \vspace{-0.2cm}
	
	\begin{center}
		\begin{tikzpicture}[
				chatbox_inner/.style={rectangle, rounded corners, opacity=0, text opacity=1, font=\sffamily\scriptsize, text width=5in, text height=9pt, inner xsep=6pt, inner ysep=6pt},
				chatbox_prompt_inner/.style={chatbox_inner, align=flush left, xshift=0pt, text height=11pt},
				chatbox_user_inner/.style={chatbox_inner, align=flush left, xshift=0pt},
				chatbox_gpt_inner/.style={chatbox_inner, align=flush left, xshift=0pt},
				chatbox/.style={chatbox_inner, draw=black!25, fill=gray!7, opacity=1, text opacity=0},
				chatbox_prompt/.style={chatbox, align=flush left, fill=gray!1.5, draw=black!30, text height=10pt},
				chatbox_user/.style={chatbox, align=flush left},
				chatbox_gpt/.style={chatbox, align=flush left},
				chatbox2/.style={chatbox_gpt, fill=green!25},
				chatbox3/.style={chatbox_gpt, fill=red!20, draw=black!20},
				chatbox4/.style={chatbox_gpt, fill=yellow!30},
				labelbox/.style={rectangle, rounded corners, draw=black!50, font=\sffamily\scriptsize\bfseries, fill=gray!5, inner sep=3pt},
			]
											
			\node[chatbox_user] (q1) {
				\textbf{System prompt}
				\newline
				\newline
				You are a helpful and precise assistant for segmenting and labeling sentences. We would like to request your help on curating a dataset for entity-level hallucination detection.
				\newline \newline
                We will give you a machine generated biography and a list of checked facts about the biography. Each fact consists of a sentence and a label (True/False). Please do the following process. First, breaking down the biography into words. Second, by referring to the provided list of facts, merging some broken down words in the previous step to form meaningful entities. For example, ``strategic thinking'' should be one entity instead of two. Third, according to the labels in the list of facts, labeling each entity as True or False. Specifically, for facts that share a similar sentence structure (\eg, \textit{``He was born on Mach 9, 1941.''} (\texttt{True}) and \textit{``He was born in Ramos Mejia.''} (\texttt{False})), please first assign labels to entities that differ across atomic facts. For example, first labeling ``Mach 9, 1941'' (\texttt{True}) and ``Ramos Mejia'' (\texttt{False}) in the above case. For those entities that are the same across atomic facts (\eg, ``was born'') or are neutral (\eg, ``he,'' ``in,'' and ``on''), please label them as \texttt{True}. For the cases that there is no atomic fact that shares the same sentence structure, please identify the most informative entities in the sentence and label them with the same label as the atomic fact while treating the rest of the entities as \texttt{True}. In the end, output the entities and labels in the following format:
                \begin{itemize}[nosep]
                    \item Entity 1 (Label 1)
                    \item Entity 2 (Label 2)
                    \item ...
                    \item Entity N (Label N)
                \end{itemize}
                % \newline \newline
                Here are two examples:
                \newline\newline
                \textbf{[Example 1]}
                \newline
                [The start of the biography]
                \newline
                \textcolor{titlecolor}{Marianne McAndrew is an American actress and singer, born on November 21, 1942, in Cleveland, Ohio. She began her acting career in the late 1960s, appearing in various television shows and films.}
                \newline
                [The end of the biography]
                \newline \newline
                [The start of the list of checked facts]
                \newline
                \textcolor{anscolor}{[Marianne McAndrew is an American. (False); Marianne McAndrew is an actress. (True); Marianne McAndrew is a singer. (False); Marianne McAndrew was born on November 21, 1942. (False); Marianne McAndrew was born in Cleveland, Ohio. (False); She began her acting career in the late 1960s. (True); She has appeared in various television shows. (True); She has appeared in various films. (True)]}
                \newline
                [The end of the list of checked facts]
                \newline \newline
                [The start of the ideal output]
                \newline
                \textcolor{labelcolor}{[Marianne McAndrew (True); is (True); an (True); American (False); actress (True); and (True); singer (False); , (True); born (True); on (True); November 21, 1942 (False); , (True); in (True); Cleveland, Ohio (False); . (True); She (True); began (True); her (True); acting career (True); in (True); the late 1960s (True); , (True); appearing (True); in (True); various (True); television shows (True); and (True); films (True); . (True)]}
                \newline
                [The end of the ideal output]
				\newline \newline
                \textbf{[Example 2]}
                \newline
                [The start of the biography]
                \newline
                \textcolor{titlecolor}{Doug Sheehan is an American actor who was born on April 27, 1949, in Santa Monica, California. He is best known for his roles in soap operas, including his portrayal of Joe Kelly on ``General Hospital'' and Ben Gibson on ``Knots Landing.''}
                \newline
                [The end of the biography]
                \newline \newline
                [The start of the list of checked facts]
                \newline
                \textcolor{anscolor}{[Doug Sheehan is an American. (True); Doug Sheehan is an actor. (True); Doug Sheehan was born on April 27, 1949. (True); Doug Sheehan was born in Santa Monica, California. (False); He is best known for his roles in soap operas. (True); He portrayed Joe Kelly. (True); Joe Kelly was in General Hospital. (True); General Hospital is a soap opera. (True); He portrayed Ben Gibson. (True); Ben Gibson was in Knots Landing. (True); Knots Landing is a soap opera. (True)]}
                \newline
                [The end of the list of checked facts]
                \newline \newline
                [The start of the ideal output]
                \newline
                \textcolor{labelcolor}{[Doug Sheehan (True); is (True); an (True); American (True); actor (True); who (True); was born (True); on (True); April 27, 1949 (True); in (True); Santa Monica, California (False); . (True); He (True); is (True); best known (True); for (True); his roles in soap operas (True); , (True); including (True); in (True); his portrayal (True); of (True); Joe Kelly (True); on (True); ``General Hospital'' (True); and (True); Ben Gibson (True); on (True); ``Knots Landing.'' (True)]}
                \newline
                [The end of the ideal output]
				\newline \newline
				\textbf{User prompt}
				\newline
				\newline
				[The start of the biography]
				\newline
				\textcolor{magenta}{\texttt{\{BIOGRAPHY\}}}
				\newline
				[The ebd of the biography]
				\newline \newline
				[The start of the list of checked facts]
				\newline
				\textcolor{magenta}{\texttt{\{LIST OF CHECKED FACTS\}}}
				\newline
				[The end of the list of checked facts]
			};
			\node[chatbox_user_inner] (q1_text) at (q1) {
				\textbf{System prompt}
				\newline
				\newline
				You are a helpful and precise assistant for segmenting and labeling sentences. We would like to request your help on curating a dataset for entity-level hallucination detection.
				\newline \newline
                We will give you a machine generated biography and a list of checked facts about the biography. Each fact consists of a sentence and a label (True/False). Please do the following process. First, breaking down the biography into words. Second, by referring to the provided list of facts, merging some broken down words in the previous step to form meaningful entities. For example, ``strategic thinking'' should be one entity instead of two. Third, according to the labels in the list of facts, labeling each entity as True or False. Specifically, for facts that share a similar sentence structure (\eg, \textit{``He was born on Mach 9, 1941.''} (\texttt{True}) and \textit{``He was born in Ramos Mejia.''} (\texttt{False})), please first assign labels to entities that differ across atomic facts. For example, first labeling ``Mach 9, 1941'' (\texttt{True}) and ``Ramos Mejia'' (\texttt{False}) in the above case. For those entities that are the same across atomic facts (\eg, ``was born'') or are neutral (\eg, ``he,'' ``in,'' and ``on''), please label them as \texttt{True}. For the cases that there is no atomic fact that shares the same sentence structure, please identify the most informative entities in the sentence and label them with the same label as the atomic fact while treating the rest of the entities as \texttt{True}. In the end, output the entities and labels in the following format:
                \begin{itemize}[nosep]
                    \item Entity 1 (Label 1)
                    \item Entity 2 (Label 2)
                    \item ...
                    \item Entity N (Label N)
                \end{itemize}
                % \newline \newline
                Here are two examples:
                \newline\newline
                \textbf{[Example 1]}
                \newline
                [The start of the biography]
                \newline
                \textcolor{titlecolor}{Marianne McAndrew is an American actress and singer, born on November 21, 1942, in Cleveland, Ohio. She began her acting career in the late 1960s, appearing in various television shows and films.}
                \newline
                [The end of the biography]
                \newline \newline
                [The start of the list of checked facts]
                \newline
                \textcolor{anscolor}{[Marianne McAndrew is an American. (False); Marianne McAndrew is an actress. (True); Marianne McAndrew is a singer. (False); Marianne McAndrew was born on November 21, 1942. (False); Marianne McAndrew was born in Cleveland, Ohio. (False); She began her acting career in the late 1960s. (True); She has appeared in various television shows. (True); She has appeared in various films. (True)]}
                \newline
                [The end of the list of checked facts]
                \newline \newline
                [The start of the ideal output]
                \newline
                \textcolor{labelcolor}{[Marianne McAndrew (True); is (True); an (True); American (False); actress (True); and (True); singer (False); , (True); born (True); on (True); November 21, 1942 (False); , (True); in (True); Cleveland, Ohio (False); . (True); She (True); began (True); her (True); acting career (True); in (True); the late 1960s (True); , (True); appearing (True); in (True); various (True); television shows (True); and (True); films (True); . (True)]}
                \newline
                [The end of the ideal output]
				\newline \newline
                \textbf{[Example 2]}
                \newline
                [The start of the biography]
                \newline
                \textcolor{titlecolor}{Doug Sheehan is an American actor who was born on April 27, 1949, in Santa Monica, California. He is best known for his roles in soap operas, including his portrayal of Joe Kelly on ``General Hospital'' and Ben Gibson on ``Knots Landing.''}
                \newline
                [The end of the biography]
                \newline \newline
                [The start of the list of checked facts]
                \newline
                \textcolor{anscolor}{[Doug Sheehan is an American. (True); Doug Sheehan is an actor. (True); Doug Sheehan was born on April 27, 1949. (True); Doug Sheehan was born in Santa Monica, California. (False); He is best known for his roles in soap operas. (True); He portrayed Joe Kelly. (True); Joe Kelly was in General Hospital. (True); General Hospital is a soap opera. (True); He portrayed Ben Gibson. (True); Ben Gibson was in Knots Landing. (True); Knots Landing is a soap opera. (True)]}
                \newline
                [The end of the list of checked facts]
                \newline \newline
                [The start of the ideal output]
                \newline
                \textcolor{labelcolor}{[Doug Sheehan (True); is (True); an (True); American (True); actor (True); who (True); was born (True); on (True); April 27, 1949 (True); in (True); Santa Monica, California (False); . (True); He (True); is (True); best known (True); for (True); his roles in soap operas (True); , (True); including (True); in (True); his portrayal (True); of (True); Joe Kelly (True); on (True); ``General Hospital'' (True); and (True); Ben Gibson (True); on (True); ``Knots Landing.'' (True)]}
                \newline
                [The end of the ideal output]
				\newline \newline
				\textbf{User prompt}
				\newline
				\newline
				[The start of the biography]
				\newline
				\textcolor{magenta}{\texttt{\{BIOGRAPHY\}}}
				\newline
				[The ebd of the biography]
				\newline \newline
				[The start of the list of checked facts]
				\newline
				\textcolor{magenta}{\texttt{\{LIST OF CHECKED FACTS\}}}
				\newline
				[The end of the list of checked facts]
			};
		\end{tikzpicture}
        \caption{GPT-4o prompt for labeling hallucinated entities.}\label{tb:gpt-4-prompt}
	\end{center}
\vspace{-0cm}
\end{table*}
% \section{Full Experiment Results}
% \begin{table*}[th]
    \centering
    \small
    \caption{Classification Results}
    \begin{tabular}{lcccc}
        \toprule
        \textbf{Method} & \textbf{Accuracy} & \textbf{Precision} & \textbf{Recall} & \textbf{F1-score} \\
        \midrule
        \multicolumn{5}{c}{\textbf{Zero Shot}} \\
                Zero-shot E-eyes & 0.26 & 0.26 & 0.27 & 0.26 \\
        Zero-shot CARM & 0.24 & 0.24 & 0.24 & 0.24 \\
                Zero-shot SVM & 0.27 & 0.28 & 0.28 & 0.27 \\
        Zero-shot CNN & 0.23 & 0.24 & 0.23 & 0.23 \\
        Zero-shot RNN & 0.26 & 0.26 & 0.26 & 0.26 \\
DeepSeek-0shot & 0.54 & 0.61 & 0.54 & 0.52 \\
DeepSeek-0shot-COT & 0.33 & 0.24 & 0.33 & 0.23 \\
DeepSeek-0shot-Knowledge & 0.45 & 0.46 & 0.45 & 0.44 \\
Gemma2-0shot & 0.35 & 0.22 & 0.38 & 0.27 \\
Gemma2-0shot-COT & 0.36 & 0.22 & 0.36 & 0.27 \\
Gemma2-0shot-Knowledge & 0.32 & 0.18 & 0.34 & 0.20 \\
GPT-4o-mini-0shot & 0.48 & 0.53 & 0.48 & 0.41 \\
GPT-4o-mini-0shot-COT & 0.33 & 0.50 & 0.33 & 0.38 \\
GPT-4o-mini-0shot-Knowledge & 0.49 & 0.31 & 0.49 & 0.36 \\
GPT-4o-0shot & 0.62 & 0.62 & 0.47 & 0.42 \\
GPT-4o-0shot-COT & 0.29 & 0.45 & 0.29 & 0.21 \\
GPT-4o-0shot-Knowledge & 0.44 & 0.52 & 0.44 & 0.39 \\
LLaMA-0shot & 0.32 & 0.25 & 0.32 & 0.24 \\
LLaMA-0shot-COT & 0.12 & 0.25 & 0.12 & 0.09 \\
LLaMA-0shot-Knowledge & 0.32 & 0.25 & 0.32 & 0.28 \\
Mistral-0shot & 0.19 & 0.23 & 0.19 & 0.10 \\
Mistral-0shot-Knowledge & 0.21 & 0.40 & 0.21 & 0.11 \\
        \midrule
        \multicolumn{5}{c}{\textbf{4 Shot}} \\
GPT-4o-mini-4shot & 0.58 & 0.59 & 0.58 & 0.53 \\
GPT-4o-mini-4shot-COT & 0.57 & 0.53 & 0.57 & 0.50 \\
GPT-4o-mini-4shot-Knowledge & 0.56 & 0.51 & 0.56 & 0.47 \\
GPT-4o-4shot & 0.77 & 0.84 & 0.77 & 0.73 \\
GPT-4o-4shot-COT & 0.63 & 0.76 & 0.63 & 0.53 \\
GPT-4o-4shot-Knowledge & 0.72 & 0.82 & 0.71 & 0.66 \\
LLaMA-4shot & 0.29 & 0.24 & 0.29 & 0.21 \\
LLaMA-4shot-COT & 0.20 & 0.30 & 0.20 & 0.13 \\
LLaMA-4shot-Knowledge & 0.15 & 0.23 & 0.13 & 0.13 \\
Mistral-4shot & 0.02 & 0.02 & 0.02 & 0.02 \\
Mistral-4shot-Knowledge & 0.21 & 0.27 & 0.21 & 0.20 \\
        \midrule
        
        \multicolumn{5}{c}{\textbf{Suprevised}} \\
        SVM & 0.94 & 0.92 & 0.91 & 0.91 \\
        CNN & 0.98 & 0.98 & 0.97 & 0.97 \\
        RNN & 0.99 & 0.99 & 0.99 & 0.99 \\
        % \midrule
        % \multicolumn{5}{c}{\textbf{Conventional Wi-Fi-based Human Activity Recognition Systems}} \\
        E-eyes & 1.00 & 1.00 & 1.00 & 1.00 \\
        CARM & 0.98 & 0.98 & 0.98 & 0.98 \\
\midrule
 \multicolumn{5}{c}{\textbf{Vision Models}} \\
           Zero-shot SVM & 0.26 & 0.25 & 0.25 & 0.25 \\
        Zero-shot CNN & 0.26 & 0.25 & 0.26 & 0.26 \\
        Zero-shot RNN & 0.28 & 0.28 & 0.29 & 0.28 \\
        SVM & 0.99 & 0.99 & 0.99 & 0.99 \\
        CNN & 0.98 & 0.99 & 0.98 & 0.98 \\
        RNN & 0.98 & 0.99 & 0.98 & 0.98 \\
GPT-4o-mini-Vision & 0.84 & 0.85 & 0.84 & 0.84 \\
GPT-4o-mini-Vision-COT & 0.90 & 0.91 & 0.90 & 0.90 \\
GPT-4o-Vision & 0.74 & 0.82 & 0.74 & 0.73 \\
GPT-4o-Vision-COT & 0.70 & 0.83 & 0.70 & 0.68 \\
LLaMA-Vision & 0.20 & 0.23 & 0.20 & 0.09 \\
LLaMA-Vision-Knowledge & 0.22 & 0.05 & 0.22 & 0.08 \\

        \bottomrule
    \end{tabular}
    \label{full}
\end{table*}




\end{document}




\appendix
\section{Additional Results}
\label{sec:Additional_Results}

This section provides a comprehensive overview of how partial training can be beneficial for OOD detection. It presents visualizations of histograms of AUROC scores (from Tables \ref{tab:auc_results_one_channel} and \ref{tab:auc_results_three_channel}) based on 1000 random samples from various datasets, using the OOD score defined in equation \ref{eq:ood_score_final}. These histograms demonstrate that partially trained models can outperform, or at least match, fully trained models in OOD detection tasks. This challenges the assumption that longer training always leads to better OOD detection performance. 



\begin{figure}[ht!]
    \centering
    \begin{subfigure}[t]{0.49\textwidth}
        \centering
        \framebox[\textwidth]{\includegraphics[width=\textwidth]{images/additional/histogram_mnist.png}}
    \end{subfigure}
    \hfill
    \begin{subfigure}[t]{0.49\textwidth}
        \centering
        \framebox[\textwidth]{\includegraphics[width=\textwidth]{images/additional/histogram_fashionmnist.png}}
    \end{subfigure}
    \hfill
    \begin{subfigure}[t]{0.49\textwidth}
        \centering
       \framebox[\textwidth]{\includegraphics[width=\textwidth]{images/additional/histogram_kmnist.png}}
    \end{subfigure}
    \hfill
    \begin{subfigure}[t]{0.49\textwidth}
        \centering
        \framebox[\textwidth]{\includegraphics[width=\textwidth]{images/additional/histogram_omniglot.png}}
    \end{subfigure}
        \caption{\textbf{\emph{Histograms of OOD Scores for One-Channel colour Datasets}}\\[0.5em]
    The OOD scores, defined in Equation \ref{eq:ood_score_final}, are visualized here to provide insight into the performance of the models on one-channel datasets. These scores are linked to the AUROC scores presented in Table \ref{tab:auc_results_one_channel}, emphasizing the diminishing importance of fully trained models for OOD detection. Each box in the histograms represents the OOD scores when a dataset is treated as ID and evaluated against three different OOD datasets. The first row shows the results for OOD detection with a batch size of five, while the second row presents results for single-sample OOD detection. The first columns correspond to partially trained models at various epochs, while the second columns represent fully trained models at their final epochs.\\ The gap in the histograms of \emph{Omniglot} for the first epoch and a batch size of 5 is so wide that two different scales have been used to illustrate the performance effectively. In all other datasets, the overlap between histograms increases as training progresses.\\ When viewed alongside the generated samples in Figure \ref{fig:generated_samples} and the numerical results in Table \ref{tab:auc_results_one_channel}, this comparison demonstrates that partially trained models can achieve comparable or even superior OOD detection performance compared to fully trained ones.}
    \label{fig:OOD_hist_one_channel}
\end{figure}



\begin{figure}[ht!]
    \centering
    \begin{subfigure}[t]{0.49\textwidth}
        \centering
        \framebox[\textwidth]{\includegraphics[width=\textwidth]{images/additional/histogram_celeba.png}}
    \end{subfigure}
    \hfill
    \begin{subfigure}[t]{0.49\textwidth}
        \centering
        \framebox[\textwidth]{\includegraphics[width=\textwidth]{images/additional/histogram_imagenet32.png}}
    \end{subfigure}
    \hfill
    \begin{subfigure}[t]{0.49\textwidth}
        \centering
        \framebox[\textwidth]{\includegraphics[width=\textwidth]{images/additional/histogram_gtsrb.png}}
    \end{subfigure}
        \hfill
    \begin{subfigure}[t]{0.49\textwidth}
        \centering
        \framebox[\textwidth]{\includegraphics[width=\textwidth]{images/additional/histogram_cifar10.png}}
    \end{subfigure}
        \hfill
    \begin{subfigure}[t]{0.49\textwidth}
        \centering
        \framebox[\textwidth]{\includegraphics[width=\textwidth]{images/additional/histogram_svhn.png}}
    \end{subfigure}

    \caption{\textbf{\emph{Histograms of OOD Scores for Three-Channel colour Datasets}}\\[0.5em] This figure presents the OOD scores for five three-channel datasets (CelebA, CIFAR-10, GTSRB, ImageNet32, and SVHN) at different stages of model training. The histograms illustrate the distribution of OOD scores when each dataset is treated as ID and evaluated against other datasets as OOD. The first row displays OOD detection results using a batch size of five, while the second row visualizes results for single-sample OOD detection. The first column corresponds to partially trained models at various epochs, whereas the second column shows results for fully trained models at the final epoch.\\
    Considering the first rows in boxes, for OOD detection using GoF, the gap between histograms increases with training progress for CelebA and ImageNet32. Conversely, in GTSRB and SVHN, the gap transitions into overlap, and for CIFAR-10, the overlap becomes more pronounced as training progresses. In single-sample OOD detection, the existing overlap becomes even more significant in fully trained models compared to partially trained ones.\\
    To better analyze the effectiveness of immature models, these histograms should be viewed alongside the generated sample images in Figure \ref{fig:generated_samples} and the AUROC results in Table \ref{tab:auc_results_three_channel}.}
    
    \label{fig:OOD_hist_three_channel}

\end{figure}







\clearpage
\section{Generated Samples}
\label{sec:Generated_Samples}
Exploring the generated samples from the GLOW model trained on one-channel and three-channel datasets at different training stages. Figure \ref{fig:generated_samples} displays the generated samples for both types of datasets, with each set representing two stages: partially trained (left) and fully trained (right). The images for the three-channel datasets (CelebA, CIFAR-10, GTSRB, ImageNet32, and SVHN) are of resolution $32 \times 32$ pixels, while the one-channel datasets (FashionMNIST, MNIST, KMNIST, and Omniglot) have a resolution of $28 \times 28$ pixels. As training progresses, the quality of the generated samples improves, particularly for more complex datasets. Despite this improvement in sample quality, partially trained models often yield higher or equivalent AUROC scores in OOD detection tasks compared to fully trained models, as shown in the corresponding tables. This counter-intuitive relationship highlights a diminishing return in OOD detection performance as the models become fully converged. The intermediate epochs for the sample generation align with those presented in the respective tables, illustrating that partially trained models can often outperform fully trained models in the OOD detection task, even though their sample quality has not reached its peak.

\begin{figure}[ht!]
    \centering
    \begin{subfigure}[t]{0.39\textwidth}
        \centering
        \includegraphics[width=\textwidth]{images/additional/samples_three_channel_low.png}
    \end{subfigure}
    % \hfill
    \begin{subfigure}[t]{0.39\textwidth}
        \centering
        \includegraphics[width=\textwidth]{images/additional/samples_three_channel_high.png}
    \end{subfigure}
    \begin{subfigure}[t]{0.39\textwidth}
        \centering
        \includegraphics[width=\textwidth]{images/additional/samples_one_channel_low.png}
    \end{subfigure}
    % \hfill
    \begin{subfigure}[t]{0.39\textwidth}
        \centering
        \includegraphics[width=\textwidth]{images/additional/samples_one_channel_high.png}
    \end{subfigure}
\caption{\textbf{\emph{Generated Samples at Different Training Stages of the GLOW Model on Three-Channel and One-Channel Datasets}}\\[0.5em]
Samples generated from five three-channel datasets (CelebA, CIFAR-10, GTSRB, ImageNet32, and SVHN) and four one-channel datasets (FashionMNIST, MNIST, KMNIST, and Omniglot) at two different stages of training: partially trained (left) and fully trained (right). The samples were generated using a temperature parameter set to 1 across all datasets. As highlighted in Tables \ref{tab:auc_results_three_channel} and \ref{tab:auc_results_one_channel}, partially trained models frequently achieve higher or comparable AUROC scores in OOD detection tasks compared to fully trained models. The partially trained epochs correspond to intermediate training stages shown in the tables, while the fully trained models are evaluated at the 250th epoch for all three-channel datasets, the 200th epoch for FashionMNIST, MNIST, and KMNIST, and the 100th epoch for Omniglot. Notably, for Omniglot, effective OOD detection is evident even at the first epoch, negating the necessity for extended training.}
    \label{fig:generated_samples}
\end{figure}



\clearpage
\section{NLL as OOD Score}
\label{sec:NLL_as_OOD}
\begin{figure}[ht!]
    \centering
    \includegraphics[width=\textwidth]{images/additional/NNL__models.png}
    \caption{\textbf{\emph{Visualization of negative bits per dimension(BPD) failure as an OOD detection score for various ID-OOD dataset pairs.}}\\[0.5em]The figure illustrates the performance of the GLOW model when trained on a specific dataset (ID) and evaluated against four OOD datasets. In the first row (left), the model is trained on CelebA and tested on SVHN, ImageNet32, GTSRB, and CIFAR-10. Surprisingly, the model assigns higher likelihood (higher negative BPD) to OOD samples from SVHN than to its ID data, indicating a failure in OOD detection. Similarly, in the first row (right), the model trained on CIFAR-10 also demonstrates poor separation between ID and OOD datasets.
    In the second row (left), the model is trained on GTSRB, and in the second row (right), on ImageNet32. For all more complex datasets, the model assigns disproportionately lower NLL values to simpler OOD datasets like SVHN, signifying higher likelihood for OOD samples. This trend suggests that as the complexity of the ID dataset increases, the model struggles to differentiate OOD samples effectively.
    Finally, in the third row, the model is trained on SVHN. In this case, the GLOW model performs as expected, assigning frequently higher likelihood (higher negative BPD) to ID data. These observations highlight the inherent limitations of NLL as a stand-alone OOD criterion in reliably distinguishing between ID and OOD datasets, especially when dataset complexities vary significantly.}
    \label{fig:NLL_models}
\end{figure}


% \clearpage
\section{Experimental Setups}
\label{sec:Experimental_Setups}

To train the three-channel colour datasets, the Glow model is applied using (CIFAR-10, GTSRB, ImageNet32, and SVHN) datasets with a batch size of 64 and a learning rate of 5e-4, for a total of 250 epochs. The network architecture included 3 blocks with 32 flow steps per block, 512 hidden channels, convolutional layers in affine coupling layer and inverse convolution for flow permutation.

For one-channel colour datasets, the model was trained on (FashionMNIST, MNIST, KMNIST and Omniglot) with a batch size of 128, using a learning rate of 1e-3 for 200 epochs. The architecture included one block with 10 flow steps, 1000 hidden channels, and linear layer in ACL layers, with a weight decay of 1e-4.

To calculate the OOD scores, the test split of each dataset was used to fit Gaussian to layer-wise logarithmic features. Additionally, 1000 random samples were selected to compute the AUROC results for each dataset.



\end{document}