\section{Limitations}

Our study primarily focuses on English and Chinese, while fraud is a global issue that affects many languages and cultural contexts. We acknowledge that incorporating more languages and diverse examples would provide a more comprehensive assessment. Additionally, as AI-generated content, such as AI-synthesized images and deepfake videos, is increasingly exploited in fraud, future research should explore multimodal fraud detection. Furthermore, our evaluation relies on large language models (LLMs) to assess the success or failure of fraudulent attempts. Although we have validated the consistency between LLM-as-judge and human annotators, more advanced fraud detection and risk warning systems remain essential for mitigating real-world threats.

% \paragraph{Lack of Real-Time Adaptation} \ourbench is based on a static dataset, limiting its ability to evaluate LLMs against evolving fraud tactics. Real-world fraud schemes continuously adapt, making it essential for future benchmarks to incorporate dynamically generated fraud cases.