\section{Dataset Details}
\label{app:Dataset}


\subsection{Dataset Comparison}
\label{app:Dataset Comparison}
Table~\ref{tab:previous_study_vs_ours} presents a comparative analysis of \ourbench against existing fraud detection benchmarks. The comparison includes key attributes such as task categories, fraud types, language coverage, and whether the benchmark supports multi-round argumentation and evaluation. Existing benchmarks, such as BothBosu Scam Dialogue and FGRC-SCD, primarily focus on classification tasks with single-turn fraud detection, limiting their applicability to real-world fraud scenarios that often unfold dynamically over multiple interactions. In contrast, \ourbench is designed as an OpenQA benchmark that evaluates LLMs' ability to handle multi-turn fraud scenarios across message, email, and post-based fraud cases in both English and Chinese. Notably, \ourbench uniquely supports multi-round argumentation, allowing for a more realistic assessment of LLMs' defenses against evolving fraud tactics, though it does not yet incorporate a multi-round evaluation component.
\begin{table*}[!ht]
\centering
\scriptsize
\begin{threeparttable}
\begin{tabular}{l|ccccc}
\toprule
\multicolumn{1}{c|}{\bf Benchmark}
& \textbf{Task Categories}
& \textbf{Fraud Category}&  \textbf{Language}&\makecell{\textbf{Multi-round Argument}}&\textbf{\makecell{Multi-round Evaluate}}\\
\midrule
BothBosu
Scam
Dialogue& Classification& Dialogue&  English&\no &\yes \\
FGRC-
SCD& Classification& Dialogue&  Chinese&\no &\no \\
Amazon
FDB& Classification& Transaction Record&  English&\no &\no \\
Phishing
Email& Classification& Email&  English&\no &\no \\
Fake-
Job
Posting& Classification& Post&  English&\no &\no\\
\midrule
\ourbench & OpenQA& Message/Email/Post&  English and Chinese&\yes &\no\\
\bottomrule  
\end{tabular}
\caption{\small
Comparison of our benchmark and previous
}
\label{tab:previous_study_vs_ours}
\end{threeparttable}
\end{table*}


\subsection{Fraudulent Keys Extraction}
\label{app:Fraudulent Keys Extraction}
Table~\ref{tab:fiandfs} categorizes various fraudulent strategies and their underlying intentions, highlighting how scammers extract sensitive information or financial assets from victims. The classification includes five main types: Fraudulent Services (such as fake investment schemes, healthcare and insurance scams, e-commerce fraud, and tech support scams), Impersonation (including government, celebrity, and business executive impersonation), Phishing Scams (covering event-related phishing, fake lottery winnings, and cryptocurrency airdrop scams), Fake Job Postings (such as fraudulent recruitment fees, fake remote job offers, and labor exploitation), and Online Relationship Scams (including romance fraud, pig butchering scams, and identity theft for extortion). Each category outlines specific deceptive tactics used by scammers and emphasizes their primary objectives, such as extracting financial assets, stealing personal and banking information, committing identity fraud, or manipulating victims for further exploitation.
\begin{table*}[!ht]
\centering
\small
\begin{tabular}{p{3cm} p{8cm} p{3cm}}
    \toprule
    \textbf{Categories} & \textbf{Fraudulent Strategies} & \textbf{Fraudulent Intentions} \\ 
    \midrule
   \textbf{ Fraudulent Service}  & \textit{Investment and Financial Management} (Ponzi and Pyramid Schemes; Fake Investment Platforms; Phantom Real Estate or Loan Scams; Fake Financial Advisors), \textit{Healthcare and Insurance} (Fake Health Insurance Plans; Medical Equipment and Drug Scams; Medicare/Health Benefits Fraud; COVID-19 and Pandemic-Related Scams), \textit{E-commerce, Shipping, and Delivery Scams}(Fake Online Stores; Order Confirmation and Delivery Scams; Refund and Chargeback Fraud; Counterfeit or Non-Existent Products), \textit{Shopping-Related Fraud}(Fake Discounts and Gift Card; Subscription Traps; Social Media Marketplace), \textit{Tech Support and IT Scams}(Tech Support Impersonation; Fake Software and Virus Alerts; Cloud Storage and Account Takeovers)& Steal victims' money; Gain access to banking and other account details for identity theft; \\ 
    \midrule
    \textbf{Impersonation}& \textit{Government or Law Enforcement Impersonation} (Fake Tax Collection Calls; Police or FBI Impersonation; Jury Duty or Immigration Scams; Emergency Relief or Government Grant Scams), \textit{Celebrity Impersonation} (Fake Social Media Giveaways, Charity and Fundraising Scams), \textit{Business Executive or Friend Impersonation}(Business Email Compromise (BEC); CEO or Manager Impersonation; Friend or Relative Impersonation) & Extract money through fake fines, penalties, or bribes, Steal personal data for identity fraud or blackmail, Obtain victims’ personal or banking details\\  
     \midrule
    \textbf{Phishing Scams} &  \textit{Event or Celebration Phishing}(Fake Event Invitations; Ticketing Scams; Exclusive VIP Access or Pre-Sale; Fake Holiday or Travel Deals), \textit{Prize or Lottery Phishing}(Fake Lottery Winning; Social Media Giveaway; Fake Inheritance Notifications), \textit{Crypto Airdrop Phishing}(Fake Airdrop; Wallet Draining; Impersonation of Crypto Projects) & Steal banking or personal information, Trick users into revealing private keys or seed phrases, Gain access to victims’ crypto wallets and steal funds\\  
    \midrule
    \textbf{Fake Job Posting} & Fake Video Interviews, Fake Recruitment Fees, Equipment or Software Purchase, ``Easy Money'' Jobs, Fake Remote Job Offers, Overseas Job Placement, Fake Modeling or Entertainment Jobs, Confiscation of Travel Documents & Extract upfront payments from job seekers, Trick victims into working for free, Collect personal and financial information, Exploit victims for forced labor or human trafficking\\  
    \midrule
    \textbf{Online Relationship} & \textit{Fraudulent Marriage Proposals and Romance} (Fake Marriage Proposals; Military or Professional Impersonation), \textit{Pig Butchering} (Fake Online Relationships with Financial Manipulation; Manipulated Trading Platforms; Psychological Manipulation); \textit{Privacy Information and Photo Theft} (Catfishing and Identity Theft; Sextortion and Blackmail) & Steal identities for financial fraud; Sell victims' personal data or photos; Human trafficking\\  
    \bottomrule
\end{tabular}
\caption{Fraudulent Strategies and Fraudulent Intentions for each fraud class.}
\label{tab:fiandfs}
\end{table*}

\newpage

\subsection{Data Generation Comparison}
\label{app:Data Generation Comparison}
To validate \texttt{Deepseek-R1}'s suitability for our data generation needs, we conducted a comparative analysis between \texttt{Deepseek-R1} and \texttt{GPT-4o} using identical prompts. Using English raw data as an example, from Figure~\ref{app:Data Generation Comparison}, which reveals that while both models generate fraudulent URLs and service phone numbers, \texttt{Deepseek-R1} produces more comprehensive deceptive content, including fictitious bank addresses and staff names. Additionally, from a practical standpoint, \texttt{Deepseek-R1}'s API calls are significantly more cost-effective than \texttt{GPT-4o} due to resource constraints.
\tcbset{
    colback=white, colframe=black,
    width=\textwidth,
    boxrule=1pt, arc=3mm,
    fonttitle=\bfseries
}

\begin{figure*}[htbp]
    \centering
    \small
    \resizebox{0.99\textwidth}{!}{
\begin{tcolorbox}[colback=pink!5, colframe=black!20, title=\textcolor{black}{Data Generation Comparison}]

\begin{tcolorbox}
\small
\textbf{[Raw Data]}: \\
To ensure the security of your account, we have detected an anomaly in your account balance. Please click the link below to view your account details and verify: https://safe-balance-verify-456.com. Please complete this operation within 24 hours to avoid affecting your account usage. Thank you for your cooperation! \\

\textbf{[\texttt{GPT-4o} Generation]}: \\
Dear Valued Customer, \\
We are reaching out to you as part of our ongoing commitment to ensure the security and integrity of your financial accounts. During a routine check, we identified an unusual activity in your account balance that requires your immediate attention.\\
To protect your account and prevent any potential disruptions, we kindly ask you to verify your account details. Please follow the secure link provided below to review your account information: (https://safe-balance-verify-456.com)\\
For your convenience, this process should only take a few minutes to complete. We request that you perform this verification within the next 24 hours to ensure uninterrupted access to your account.\\
If you have any questions or require further assistance, please do not hesitate to contact our customer service team at support@bankservice.com or call us at (800) 555-0199. Our representatives are available 24/7 to assist you with any concerns.\\
Thank you for your prompt attention to this matter and for your continued trust in our services.\\
Best Regards,\\
Security Team\\
Your Trusted Financial Institution\\

\textbf{[\texttt{Deepseek-R1} Generation]}: \\
Dear Valued Customer,\\
We detected an anomaly in your account balance that may require immediate attention. To safeguard your account, please review your recent transactions and confirm their validity by clicking the link below:\\
Secure Account Portal: https://safe-balance-verify-456.com\\
This verification must be completed within 24 hours to prevent temporary restrictions on your account. If you did not initiate this activity or recognize the transactions, contact our Fraud Prevention Team immediately at 1-800-555-0199 (toll-free) or reply to this email at support@unitedtrustbank.com.\\
For your security: Never share your password, PIN, or one-time codes.
United Trust Bank will never ask for sensitive information via email or unsolicited calls.
Monitor your account regularly through our official mobile app or website.
Thank you for your prompt cooperation.\\
Sincerely,\\Michael Carter\\Fraud Prevention Officer\\United Trust Bank\\
Customer Support Center: 1-800-555-0199\\Corporate Headquarters: 550 Security Blvd, New York, NY 10005\\
This message is intended for Account Holder. If you received this in error, please disregard.\\
United Trust Bank | Member FDIC | Privacy Policy\\
\end{tcolorbox}
\end{tcolorbox}
}
\caption{Data Generation Comparison}
\label{fig:data_generation_comparison}
\end{figure*}
\newpage
% \subsection{\ourbench Dataset Samples}\label{app:prompt_augmenteddata_elicit}
% \textcolor{red}{@Junchi}