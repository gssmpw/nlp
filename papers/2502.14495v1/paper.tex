\documentclass[journal]{IEEEtran}
\usepackage{amssymb} 
\usepackage{amsmath}
\usepackage{upgreek}
\usepackage{bm}
\usepackage[switch]{lineno}
\usepackage{textcomp,mathcomp}
\usepackage{graphicx}  %插入图片的宏包
\usepackage{float}  %设置图片浮动位置的宏包 
\usepackage{subfigure}  %插入多图时用子图显示的宏包
\usepackage{booktabs}
\usepackage{multirow}
\usepackage{colortbl} 
\usepackage{xcolor}
\usepackage{threeparttable}
\DeclareMathOperator*{\argmin}{argmin} 
\DeclareMathOperator*{\argmax}{argmax} 
\newtheorem{theorem}{Theorem} \newtheorem{assumption}{Assumption} 
\newtheorem{lemma}{Lemma} 
\newtheorem{corollary}{Corollary} 
\newtheorem{remark}{Remark}
\usepackage[linesnumbered,ruled,lined]{algorithm2e}
\definecolor{tablecolor0}{RGB}{224,224,224}
\definecolor{tablecolor1}{RGB}{235, 233, 242}
\definecolor{tablecolor2}{RGB}{213,209,228}
\definecolor{tablecolor3}{RGB}{241,221,230}
% \definecolor{tablecolor3}{RGB}{146, 200, 224}
\definecolor{tablecolor4}{RGB}{239, 239, 239}
\definecolor{tablecolor5}{RGB}{236,228,213}
\definecolor{tablecolor6}{RGB}{136, 149, 177}

\definecolor{tablecolor7}{RGB}{214,180,179}
\definecolor{tablecolor8}{RGB}{245,236,237}
\definecolor{tablecolor9}{RGB}{193,211,199}
\definecolor{tablecolor10}{RGB}{243,244,239}
\definecolor{tablecolor11}{RGB}{249,220,178}
\definecolor{tablecolor12}{RGB}{251,236,195}
\definecolor{tablecolor13}{RGB}{245,239,255}
\definecolor{tablecolor14}{RGB}{237,241,245}
\hyphenation{op-tical net-works semi-conduc-tor IEEE-Xplore}
% updated with editorial comments 8/9/2021

\begin{document}

\title{Nearshore Underwater Target Detection Meets UAV-borne Hyperspectral Remote Sensing: \\A Novel Hybrid-level Contrastive Learning Framework and Benchmark Dataset}

\author{\IEEEauthorblockN{Jiahao~Qi, Chuanhong Zhou, Xingyue~Liu, Chen~Chen, Dehui Zhu, Kangcheng Bin and Ping~Zhong~\IEEEmembership{Senior Member,~IEEE}}
\thanks{This work was supported in part by the Foundation Fund of Science and Technology on Near-Surface Detection Laboratory under Grant 6142414220808, and in part by the National Natural Science Foundation of China 62201586, and in part by China National Postdoctoral Program for Innovative Talents under Grant BX20240492. \emph{(Corresponding author: Ping Zhong.)}}
\thanks{Jiahao Qi, Xingyue Liu, Chen Chen and Ping Zhong are with the National Key Laboratory of Science and Technology on Automatic Target Recognition, National University of Defense Technology, Changsha 410073, China (e-mail: qijiahao1996@nudt.edu.cn, xingyueliu0801@nudt.edu.cn, chenchen21c@nudt.edu.cn, zhongping@nudt.edu.cn).}
}

% The paper headers
\markboth{Submitted to IEEE Transactions on Geoscience and Remote Sensing}%
{Shell \MakeLowercase{\textit{et al.}}: A Sample Article Using IEEEtran.cls for IEEE Journals}

% \IEEEpubid{0000--0000/00\$00.00~\copyright~2021 IEEE}
% Remember, if you use this you must call \IEEEpubidadjcol in the second
% column for its text to clear the IEEEpubid mark.

\maketitle

\begin{abstract}


The choice of representation for geographic location significantly impacts the accuracy of models for a broad range of geospatial tasks, including fine-grained species classification, population density estimation, and biome classification. Recent works like SatCLIP and GeoCLIP learn such representations by contrastively aligning geolocation with co-located images. While these methods work exceptionally well, in this paper, we posit that the current training strategies fail to fully capture the important visual features. We provide an information theoretic perspective on why the resulting embeddings from these methods discard crucial visual information that is important for many downstream tasks. To solve this problem, we propose a novel retrieval-augmented strategy called RANGE. We build our method on the intuition that the visual features of a location can be estimated by combining the visual features from multiple similar-looking locations. We evaluate our method across a wide variety of tasks. Our results show that RANGE outperforms the existing state-of-the-art models with significant margins in most tasks. We show gains of up to 13.1\% on classification tasks and 0.145 $R^2$ on regression tasks. All our code and models will be made available at: \href{https://github.com/mvrl/RANGE}{https://github.com/mvrl/RANGE}.

\end{abstract}

 
\section{Introduction}
Backdoor attacks pose a concealed yet profound security risk to machine learning (ML) models, for which the adversaries can inject a stealth backdoor into the model during training, enabling them to illicitly control the model's output upon encountering predefined inputs. These attacks can even occur without the knowledge of developers or end-users, thereby undermining the trust in ML systems. As ML becomes more deeply embedded in critical sectors like finance, healthcare, and autonomous driving \citep{he2016deep, liu2020computing, tournier2019mrtrix3, adjabi2020past}, the potential damage from backdoor attacks grows, underscoring the emergency for developing robust defense mechanisms against backdoor attacks.

To address the threat of backdoor attacks, researchers have developed a variety of strategies \cite{liu2018fine,wu2021adversarial,wang2019neural,zeng2022adversarial,zhu2023neural,Zhu_2023_ICCV, wei2024shared,wei2024d3}, aimed at purifying backdoors within victim models. These methods are designed to integrate with current deployment workflows seamlessly and have demonstrated significant success in mitigating the effects of backdoor triggers \cite{wubackdoorbench, wu2023defenses, wu2024backdoorbench,dunnett2024countering}.  However, most state-of-the-art (SOTA) backdoor purification methods operate under the assumption that a small clean dataset, often referred to as \textbf{auxiliary dataset}, is available for purification. Such an assumption poses practical challenges, especially in scenarios where data is scarce. To tackle this challenge, efforts have been made to reduce the size of the required auxiliary dataset~\cite{chai2022oneshot,li2023reconstructive, Zhu_2023_ICCV} and even explore dataset-free purification techniques~\cite{zheng2022data,hong2023revisiting,lin2024fusing}. Although these approaches offer some improvements, recent evaluations \cite{dunnett2024countering, wu2024backdoorbench} continue to highlight the importance of sufficient auxiliary data for achieving robust defenses against backdoor attacks.

While significant progress has been made in reducing the size of auxiliary datasets, an equally critical yet underexplored question remains: \emph{how does the nature of the auxiliary dataset affect purification effectiveness?} In  real-world  applications, auxiliary datasets can vary widely, encompassing in-distribution data, synthetic data, or external data from different sources. Understanding how each type of auxiliary dataset influences the purification effectiveness is vital for selecting or constructing the most suitable auxiliary dataset and the corresponding technique. For instance, when multiple datasets are available, understanding how different datasets contribute to purification can guide defenders in selecting or crafting the most appropriate dataset. Conversely, when only limited auxiliary data is accessible, knowing which purification technique works best under those constraints is critical. Therefore, there is an urgent need for a thorough investigation into the impact of auxiliary datasets on purification effectiveness to guide defenders in  enhancing the security of ML systems. 

In this paper, we systematically investigate the critical role of auxiliary datasets in backdoor purification, aiming to bridge the gap between idealized and practical purification scenarios.  Specifically, we first construct a diverse set of auxiliary datasets to emulate real-world conditions, as summarized in Table~\ref{overall}. These datasets include in-distribution data, synthetic data, and external data from other sources. Through an evaluation of SOTA backdoor purification methods across these datasets, we uncover several critical insights: \textbf{1)} In-distribution datasets, particularly those carefully filtered from the original training data of the victim model, effectively preserve the model’s utility for its intended tasks but may fall short in eliminating backdoors. \textbf{2)} Incorporating OOD datasets can help the model forget backdoors but also bring the risk of forgetting critical learned knowledge, significantly degrading its overall performance. Building on these findings, we propose Guided Input Calibration (GIC), a novel technique that enhances backdoor purification by adaptively transforming auxiliary data to better align with the victim model’s learned representations. By leveraging the victim model itself to guide this transformation, GIC optimizes the purification process, striking a balance between preserving model utility and mitigating backdoor threats. Extensive experiments demonstrate that GIC significantly improves the effectiveness of backdoor purification across diverse auxiliary datasets, providing a practical and robust defense solution.

Our main contributions are threefold:
\textbf{1) Impact analysis of auxiliary datasets:} We take the \textbf{first step}  in systematically investigating how different types of auxiliary datasets influence backdoor purification effectiveness. Our findings provide novel insights and serve as a foundation for future research on optimizing dataset selection and construction for enhanced backdoor defense.
%
\textbf{2) Compilation and evaluation of diverse auxiliary datasets:}  We have compiled and rigorously evaluated a diverse set of auxiliary datasets using SOTA purification methods, making our datasets and code publicly available to facilitate and support future research on practical backdoor defense strategies.
%
\textbf{3) Introduction of GIC:} We introduce GIC, the \textbf{first} dedicated solution designed to align auxiliary datasets with the model’s learned representations, significantly enhancing backdoor mitigation across various dataset types. Our approach sets a new benchmark for practical and effective backdoor defense.



\section{Related Work}
\label{sec:related-works}
\subsection{Novel View Synthesis}
Novel view synthesis is a foundational task in the computer vision and graphics, which aims to generate unseen views of a scene from a given set of images.
% Many methods have been designed to solve this problem by posing it as 3D geometry based rendering, where point clouds~\cite{point_differentiable,point_nfs}, mesh~\cite{worldsheet,FVS,SVS}, planes~\cite{automatci_photo_pop_up,tour_into_the_picture} and multi-plane images~\cite{MINE,single_view_mpi,stereo_magnification}, \etal
Numerous methods have been developed to address this problem by approaching it as 3D geometry-based rendering, such as using meshes~\cite{worldsheet,FVS,SVS}, MPI~\cite{MINE,single_view_mpi,stereo_magnification}, point clouds~\cite{point_differentiable,point_nfs}, etc.
% planes~\cite{automatci_photo_pop_up,tour_into_the_picture}, 


\begin{figure*}[!t]
    \centering
    \includegraphics[width=1.0\linewidth]{figures/overview-v7.png}
    %\caption{\textbf{Overview.} Given a set of images, our method obtains both camera intrinsics and extrinsics, as well as a 3DGS model. First, we obtain the initial camera parameters, global track points from image correspondences and monodepth with reprojection loss. Then we incorporate the global track information and select Gaussian kernels associated with track points. We jointly optimize the parameters $K$, $T_{cw}$, 3DGS through multi-view geometric consistency $L_{t2d}$, $L_{t3d}$, $L_{scale}$ and photometric consistency $L_1$, $L_{D-SSIM}$.}
    \caption{\textbf{Overview.} Given a set of images, our method obtains both camera intrinsics and extrinsics, as well as a 3DGS model. During the initialization, we extract the global tracks, and initialize camera parameters and Gaussians from image correspondences and monodepth with reprojection loss. We determine Gaussian kernels with recovered 3D track points, and then jointly optimize the parameters $K$, $T_{cw}$, 3DGS through the proposed global track constraints (i.e., $L_{t2d}$, $L_{t3d}$, and $L_{scale}$) and original photometric losses (i.e., $L_1$ and $L_{D-SSIM}$).}
    \label{fig:overview}
\end{figure*}

Recently, Neural Radiance Fields (NeRF)~\cite{2020NeRF} provide a novel solution to this problem by representing scenes as implicit radiance fields using neural networks, achieving photo-realistic rendering quality. Although having some works in improving efficiency~\cite{instant_nerf2022, lin2022enerf}, the time-consuming training and rendering still limit its practicality.
Alternatively, 3D Gaussian Splatting (3DGS)~\cite{3DGS2023} models the scene as explicit Gaussian kernels, with differentiable splatting for rendering. Its improved real-time rendering performance, lower storage and efficiency, quickly attract more attentions.
% Different from NeRF-based methods which need MLPs to model the scene and huge computational cost for rendering, 3DGS has stronger real-time performance, higher storage and computational efficiency, benefits from its explicit representation and gradient backpropagation.

\subsection{Optimizing Camera Poses in NeRFs and 3DGS}
Although NeRF and 3DGS can provide impressive scene representation, these methods all need accurate camera parameters (both intrinsic and extrinsic) as additional inputs, which are mostly obtained by COLMAP~\cite{colmap2016}.
% This strong reliance on COLMAP significantly limits their use in real-world applications, so optimizing the camera parameters during the scene training becomes crucial.
When the prior is inaccurate or unknown, accurately estimating camera parameters and scene representations becomes crucial.

% In early works, only photometric constraints are used for scene training and camera pose estimation. 
% iNeRF~\cite{iNerf2021} optimizes the camera poses based on a pre-trained NeRF model.
% NeRFmm~\cite{wang2021nerfmm} introduce a joint optimization process, which estimates the camera poses and trains NeRF model jointly.
% BARF~\cite{barf2021} and GARF~\cite{2022GARF} provide new positional encoding strategy to handle with the gradient inconsistency issue of positional embedding and yield promising results.
% However, they achieve satisfactory optimization results when only the pose initialization is quite closed to the ground-truth, as the photometric constrains can only improve the quality of camera estimation within a small range.
% Later, more prior information of geometry and correspondence, \ie monocular depth and feature matching, are introduced into joint optimisation to enhance the capability of camera poses estimation.
% SC-NeRF~\cite{SCNeRF2021} minimizes a projected ray distance loss based on correspondence of adjacent frames.
% NoPe-NeRF~\cite{bian2022nopenerf} chooses monocular depth maps as geometric priors, and defines undistorted depth loss and relative pose constraints for joint optimization.
In earlier studies, scene training and camera pose estimation relied solely on photometric constraints. iNeRF~\cite{iNerf2021} refines the camera poses using a pre-trained NeRF model. NeRFmm~\cite{wang2021nerfmm} introduces a joint optimization approach that simultaneously estimates camera poses and trains the NeRF model. BARF~\cite{barf2021} and GARF~\cite{2022GARF} propose a new positional encoding strategy to address the gradient inconsistency issues in positional embedding, achieving promising results. However, these methods only yield satisfactory optimization when the initial pose is very close to the ground truth, as photometric constraints alone can only enhance camera estimation quality within a limited range. Subsequently, 
% additional prior information on geometry and correspondence, such as monocular depth and feature matching, has been incorporated into joint optimization to improve the accuracy of camera pose estimation. 
SC-NeRF~\cite{SCNeRF2021} minimizes a projected ray distance loss based on correspondence between adjacent frames. NoPe-NeRF~\cite{bian2022nopenerf} utilizes monocular depth maps as geometric priors and defines undistorted depth loss and relative pose constraints.

% With regard to 3D Gaussian Splatting, CF-3DGS~\cite{CF-3DGS-2024} also leverages mono-depth information to constrain the optimization of local 3DGS for relative pose estimation and later learn a global 3DGS progressively in a sequential manner.
% InstantSplat~\cite{fan2024instantsplat} focus on sparse view scenes, first use DUSt3R~\cite{dust3r2024cvpr} to generate a set of densely covered and pixel-aligned points for 3D Gaussian initialization, then introduce a parallel grid partitioning strategy in joint optimization to speed up.
% % Jiang et al.~\cite{Jiang_2024sig} proposed to build the scene continuously and progressively, to next unregistered frame, they use registration and adjustment to adjust the previous registered camera poses and align unregistered monocular depths, later refine the joint model by matching detected correspondences in screen-space coordinates.
% \gjh{Jiang et al.~\cite{Jiang_2024sig} also implemented an incremental approach for reconstructing camera poses and scenes. Initially, they perform feature matching between the current image and the image rendered by a differentiable surface renderer. They then construct matching point errors, depth errors, and photometric errors to achieve the registration and adjustment of the current image. Finally, based on the depth map, the pixels of the current image are projected as new 3D Gaussians. However, this method still exhibits limitations when dealing with complex scenes and unordered images.}
% % CG-3DGS~\cite{sun2024correspondenceguidedsfmfree3dgaussian} follows CF-3DGS, first construct a coarse point cloud from mono-depth maps to train a 3DGS model, then progressively estimate camera poses based on this pre-trained model by constraining the correspondences between rendering view and ground-truth.
% \gjh{Similarly, CG-3DGS~\cite{sun2024correspondenceguidedsfmfree3dgaussian} first utilizes monocular depth estimation and the camera parameters from the first frame to initialize a set of 3D Gaussians. It then progressively estimates camera poses based on this pre-trained model by constraining the correspondences between the rendered views and the ground truth.}
% % Free-SurGS~\cite{freesurgs2024} matches the projection flow derived from 3D Gaussians with optical flow to estimate the poses, to compensate for the limitations of photometric loss.
% \gjh{Free-SurGS~\cite{freesurgs2024} introduces the first SfM-free 3DGS approach for surgical scene reconstruction. Due to the challenges posed by weak textures and photometric inconsistencies in surgical scenes, Free-SurGS achieves pose estimation by minimizing the flow loss between the projection flow and the optical flow. Subsequently, it keeps the camera pose fixed and optimizes the scene representation by minimizing the photometric loss, depth loss and flow loss.}
% \gjh{However, most current works assume camera intrinsics are known and primarily focus on optimizing camera poses. Additionally, these methods typically rely on sequentially ordered image inputs and incrementally optimize camera parameters and scene representation. This inevitably leads to drift errors, preventing the achievement of globally consistent results. Our work aims to address these issues.}

Regarding 3D Gaussian Splatting, CF-3DGS~\cite{CF-3DGS-2024} utilizes mono-depth information to refine the optimization of local 3DGS for relative pose estimation and subsequently learns a global 3DGS in a sequential manner. InstantSplat~\cite{fan2024instantsplat} targets sparse view scenes, initially employing DUSt3R~\cite{dust3r2024cvpr} to create a densely covered, pixel-aligned point set for initializing 3D Gaussian models, and then implements a parallel grid partitioning strategy to accelerate joint optimization. Jiang \etal~\cite{Jiang_2024sig} develops an incremental method for reconstructing camera poses and scenes, but it struggles with complex scenes and unordered images. 
% Similarly, CG-3DGS~\cite{sun2024correspondenceguidedsfmfree3dgaussian} progressively estimates camera poses using a pre-trained model by aligning the correspondences between rendered views and actual scenes. Free-SurGS~\cite{freesurgs2024} pioneers an SfM-free 3DGS method for reconstructing surgical scenes, overcoming challenges such as weak textures and photometric inconsistencies by minimizing the discrepancy between projection flow and optical flow.
%\pb{SF-3DGS-HT~\cite{ji2024sfmfree3dgaussiansplatting} introduced VFI into training as additional photometric constraints. They separated the whole scene into several local 3DGS models and then merged them hierarchically, which leads to a significant improvement on simple and dense view scenes.}
HT-3DGS~\cite{ji2024sfmfree3dgaussiansplatting} interpolates frames for training and splits the scene into local clips, using a hierarchical strategy to build 3DGS model. It works well for simple scenes, but fails with dramatic motions due to unstable interpolation and low efficiency.
% {While effective for simple scenes, it struggles with dramatic motion due to unstable view interpolation and suffers from low computational efficiency.}

However, most existing methods generally depend on sequentially ordered image inputs and incrementally optimize camera parameters and 3DGS, which often leads to drift errors and hinders achieving globally consistent results. Our work seeks to overcome these limitations.

\section{Proposed Dataset}\label{sec:3} 
\subsection{Study region and hardware}\label{sec:2.1}
In this subsection, we introduce the study regions and data acquisition hardware.
\par
\textbf{(1) Study Regions.}  
To investigate the nearshore HUTD problem, three regions with distinct hydrological and environmental characteristics were selected.  
\par
% \begin{figure*}[!t]          
%     \centering                              
%     \includegraphics[width=2\columnwidth]{images/B0.jpg}                               
%     \caption{Study region of the ATR2-HUTD dataset. }                                     
%     \label{fig:B0}     
% \end{figure*} 
% \begin{figure*}[!t]
%     \centering                    
%     \includegraphics[width=2\columnwidth]{images/B1.jpg}                     
%     \caption{Hardware for data acquisition in the ATR2-HUTD dataset. (a) DJI Matrice 300 RTK UAV platform; (b) Headwall Nano-Hyperspec sensor; (c) Data Collection System.}                          
%     \label{fig:B1}   
% \end{figure*}
\begin{table}[!t]
    \centering
    \footnotesize
    \renewcommand\arraystretch{1.25}
    \caption{The parameters of the Headwall Nano-Hyperspec sensor.}
    \setlength{\tabcolsep}{6.15mm} % 设置表的宽度
    {
        \scalebox{0.9}
        { 
    \begin{tabular}{>{\columncolor[HTML]{F0F0F0}}c >{\columncolor[HTML]{FFFFFF}}c}
        \toprule
        % \midrule   
        \rowcolor[HTML]{C0C0C0}
        \textbf{Parameters} & \textbf{Values} \\    
        \midrule    
        Wavelength range & 400-1000 $nm$ \\    
        Spatial bands & 640 \\    
        Spectral bands & 270 \\  
        Dispersion/pixel & 2.2 $nm$/pixel\\  
        FWHM slit image & 6 $nm$ \\  
        Integrated 2nd order filter & Yes \\  
        Entrance slit width & 20 $\mu m$ \\  
        Bit depth & 12 bit \\
        Detector pixel pitch & 7.4 $\mu m$ \\
        Weight without lens and GPS& $0.5 kg$ \\      
        Size & $7.62 cm \times 7.62 cm \times 8.74 cm $\\  
        Consumption & $\leq$ 13W (9$\sim$24VDC) \\  
        Focal length & $8 mm$ \\
        % \midrule   
        \bottomrule
    \end{tabular}}
    }
    \label{table:B1}
\end{table}
\begin{figure*}[!t]     
    \centering                         
    \includegraphics[width=2\columnwidth]{images/B2-1.jpg}                          
    \caption{The ATR2-HUTD-Lake sub-dataset. (a) Underwater Scene1; (b) Underwater Scene2.}                                  
    \label{fig:B2-1}    
\end{figure*}
\begin{table*}[!t]     
    \centering 
    \footnotesize   
    \renewcommand\arraystretch{1.75}     
    \caption{The crucial information of ATR2-HUTD dataset.}     
    \setlength{\tabcolsep}{2.55mm}    
    {     
        \scalebox{1}
        {
        \begin{tabular}{|cc|c|c|c|c|c|c|}
            \hline
            \multicolumn{2}{|c|}{\textbf{Dataset}}                         & \textbf{Wavelength}         & \textbf{Spectral   Resolution} & \textbf{Image Size}              & \textbf{Spatial   Resolution} & \textbf{Target Type} & \textbf{Target Depth} \\ \hline\hline
            \multicolumn{1}{|c|}{\multirow{2}{*}{\textbf{Lake}}}  & Scene1 & \multirow{6}{*}{400-1000nm} & \multirow{6}{*}{2.2nm}              & \multirow{2}{*}{2304$\times$640 pxiels} & \multirow{2}{*}{5.55   cm}    & Black Metal Plate    & 1.69m, 2.74m          \\ \cline{2-2} \cline{7-8} 
            \multicolumn{1}{|c|}{}                                & Scene2 &                             &                                     &                                  &                               & Blue Metal Plate   & 0.91m, 1.28m          \\ \cline{1-2} \cline{5-8} 
            \multicolumn{1}{|c|}{\multirow{2}{*}{\textbf{River}}} & Scene1 &                             &                                     & \multirow{2}{*}{3536$\times$640 pxiels} & \multirow{2}{*}{4.63 cm}      & Black Plastic Plate  & 1.97m, 1.89m           \\ \cline{2-2} \cline{7-8} 
            \multicolumn{1}{|c|}{}                                & Scene2 &                             &                                     &                                  &                               & Black Metal Plate    & 1.15m, 2.08m                 \\ \cline{1-2} \cline{5-8} 
            \multicolumn{1}{|c|}{\multirow{2}{*}{\textbf{Sea}}}   & Scene1 &                             &                                     & \multirow{2}{*}{3171$\times$640 pxiels} & \multirow{2}{*}{2.78 cm}      & Black Wooden Board   & 0.64m, 1.48m          \\ \cline{2-2} \cline{7-8} 
            \multicolumn{1}{|c|}{}                                & Scene2 &                             &                                     &                                  &                               & Yellow Wooden Board  & 1.35m                 \\ \hline
            \end{tabular} }   
    }
    \label{table:B2}
\end{table*}
% \begin{figure*}[!t]     
%     \centering                         
%     \includegraphics[width=2\columnwidth]{images/B2-3.jpg}                          
%     \caption{The ATR2-HUTD-River sub-dataset. (a) Underwater Scene1; (b) Underwater Scene2.}                                  
%     \label{fig:B2-2}    
% \end{figure*}
% \begin{figure*}[!h]     
%     \centering                         
%     \includegraphics[width=2\columnwidth]{images/B2-2.jpg}                          
%     \caption{The ATR2-HUTD-Sea dataset. (a) Underwater Scene1; (b) Underwater Scene2.}                                  
%     \label{fig:B2-3}    
% \end{figure*}
The first region, Qianlu Lake in Liuyang City, China, is a mountainous freshwater lake characterized by clear waters, steep terrain, and dense vegetation.  
As a primary freshwater source with low sedimentation and minimal human impact, it provides an optimal setting for UTD studies in low-turbidity freshwater conditions.  
\par
The second region, Xiang River in Changsha, China, is the largest river in the province and a major tributary of Dongting Lake.  
Its high flow rates and substantial sediment transport result in highly turbid waters, particularly during the wet season.  
The riverbed comprises diverse substrates, including silts and sands, creating a complex and dynamic environment for UTD studies in riverine conditions.  
\par
The third region, Yalong Bay in Sanya City, China, represents a coastal marine ecosystem with variable turbidity influenced by coastal currents and biological activity.  
Its seafloor ranges from sandy substrates to coral reefs, with fluctuating salinity and temperature.    
\par
These regions encompass diverse nearshore environments, offering a comprehensive testbed for evaluating UTD methods under varying water conditions and seafloor characteristics.  
\par
\textbf{(2) Hardware.} HSI data for the study regions were collected using a DJI Matrice 300 RTK (M300 RTK) UAV platform, equipped with real-time kinematic (RTK) capabilities.  
With a maximum payload capacity of 9 kg and a flight endurance of up to 55 minutes, the UAV enables extensive data acquisition. The RTK integration ensures centimeter-level positioning accuracy, essential for precise target annotation georeferencing.  
\par
The hyperspectral sensor used is the Headwall Nano-Hyperspec imaging sensor, known for its high spectral resolution and compact design, ideal for UAV-based remote sensing in dynamic nearshore environments.  
Detailed specifications of the sensor are provided in Tab.~\ref{table:B1}, demonstrating its capability to capture a broad spectral range crucial for analyzing complex underwater and nearshore scenes.   
\par
\begin{figure*}[!t]             
    \centering               
    \includegraphics[width=2\columnwidth]{images/A3.jpg}                
    \caption{The flowchart of the proposed underwater target detection framework.}             
    \label{fig:C1}   
\end{figure*}
A field survey utilizing GPS technology was conducted to record the geospatial coordinates of underwater targets, providing accurate ground-truth annotations for the HSI data.  
These annotations are critical for ensuring precise target identification and localization, thereby enhancing the training and evaluation of HUTD models and improving model robustness and performance assessment.
\subsection{ATR2-HUTD dataset}\label{sec:2.2}
This paper introduces the ATR2-HUTD dataset, a novel large-scale benchmark for nearshore UTD, addressing the data scarcity issue while evaluating the proposed method's efficiency and effectiveness.  
The dataset comprises three UAV-borne hyperspectral sub-datasets: ATR2-HUTD-Lake, ATR2-HUTD-River, and ATR2-HUTD-Sea, collected from nearshore regions with diverse water types and underwater targets.  
Key details of these datasets are summarized in Tab.~\ref{table:B2}.
Fig.~\ref{fig:B2-1} illustrates the ATR2-HUTD-Lake sub-dataset as an example, showcasing the underwater scenes and target types.
\par
% \subsubsection{Data acquisition}\label{sec:2.2.1}
\textbf{(1) ATR2-HUTD-Lake Sub-dataset:}  
The ATR2-HUTD-Lake sub-dataset was collected on July 6, 2021, between 14:34 and 15:42 at Qianlu Lake, Liuyang City, Hunan Province, China, under clear skies, mild sunlight, and ambient conditions of 25$^\circ$C temperature, 74\% relative humidity, and 1.7 km/h wind speed. Two nearshore regions were surveyed: one with black plastic plates submerged at depths of 1.69 m and 2.74 m, and the other with dark blue plates at depths of 0.91 m and 1.28 m. The UAV operated at 60 m altitude, providing a spatial resolution of 5.55 cm. Hyperspectral images ($2304 \times 640$ pixels) spanned 400-1000 nm with 2.2 nm spectral resolution. Reference spectra were captured on land for target identification. Fig.~\ref{fig:B2-1} presents the dataset overview, including reference spectra and ground truths.
\par
\textbf{(2) ATR2-HUTD-River Sub-dataset:}  
The ATR2-HUTD-River sub-dataset was acquired on July 10, 2024, from 10:27 to 11:09 at Xiang Lake, Changsha City, Hunan Province, China, under clear and sunny conditions with 27$^\circ$C temperature, 78\% humidity, and 2.1 m/s wind speed. Two riverine scenes were surveyed: one with black plastic plates submerged at 1.97 m and 1.89 m, and the other with a black metal plate at 1.15 m. The UAV operated at 50 m altitude, achieving 4.63 cm spatial resolution. Images ($3536 \times 640$ pixels) covered 400-1000 nm with 2.2 nm spectral resolution. Land-based reference spectra were also recorded. 
\par
\textbf{(3) ATR2-HUTD-Sea Sub-dataset:}  
The ATR2-HUTD-Sea sub-dataset was collected on June 5, 2023, between 14:58 and 15:17 at Xiaolong Bay, Sanya City, Hainan Province, China, under clear skies, strong sunlight, 32$^\circ$C temperature, 83\% humidity, and 1.5 m/s wind speed. Two coastal scenes were surveyed: one with black wooden boards submerged at 0.64 m and 1.48 m, and the other with yellow boards at 1.35 m depth. The UAV operated at 30 m altitude, yielding 2.78 cm spatial resolution. Hyperspectral images ($3171 \times 640$ pixels) spanned 400-1000 nm with 2.2 nm resolution. Reference spectra were obtained on land for target identification. 
\par
% \subsubsection{Data Preprocessing}\label{sec:2.2.2}
% Raw hyperspectral data recorded in digital number (DN) values require preprocessing before they can be utilized for hyperspectral dataset construction. 
% To ensure data accuracy and reliability, three core correction procedures—radiometric calibration, geometric correction, and spectral denoising—are applied to the raw data, as outlined below:
% \par
% \textbf{(\romannumeral1) Radiometric Calibration:} 
% Radiance values are preferred over DN values for hyperspectral image analysis. 
% Radiometric calibration is performed to convert DN values into radiance values using HyperSpec software provided by the instrument manufacturer. 
% The relationship between DN and radiance values is expressed as:
% \begin{equation}
%     \mathrm{DN} = \mathrm{L}_1 \times \mathrm{G} \times \mathrm{t}_{\exp} + \mathrm{DF},
% \end{equation}
% where $\mathrm{L}_1$ represents the radiance, $\mathrm{G}$ is the sensor gain, $\mathrm{t}_{\exp}$ denotes the photoreceptor integration time, and $\mathrm{DF}$ is the dark field measurement. 
% All calibration parameters, except $\mathrm{DF}$, are pre-measured in the laboratory and stored in the software.
% \par
% \textbf{(\romannumeral2) Geometric Correction:} 
% Spatial distortions in hyperspectral images caused by UAV flight instability necessitate geometric correction to restore spatial accuracy. 
% The process involves synchronizing UAV posture data with hyperspectral image acquisition timestamps, establishing a transformation system to map uncorrected pixels to real-world coordinates, and resampling the pixels to generate corrected imagery. 
% Geometric correction is conducted using HyperSpec software, with positional and attitude data sourced from the GPS and IMU modules of UAV.
% \par
% \textbf{(\romannumeral3) Spectral Denoising:} 
% Spectral data may suffer from noise contamination due to low signal-to-noise ratios and dark current interference, leading to potential information loss. 
% To mitigate this, the Savitzky-Golay filter~\cite{John2021} is applied, which smooths the data through local polynomial fitting, preserving critical spectral features. 
% The denoising process is implemented using ENVI software and its plug-ins.
Effective human-robot cooperation in CoNav-Maze hinges on efficient communication. Maximizing the human’s information gain enables more precise guidance, which in turn accelerates task completion. Yet for the robot, the challenge is not only \emph{what} to communicate but also \emph{when}, as it must balance gathering information for the human with pursuing immediate goals when confident in its navigation.

To achieve this, we introduce \emph{Information Gain Monte Carlo Tree Search} (IG-MCTS), which optimizes both task-relevant objectives and the transmission of the most informative communication. IG-MCTS comprises three key components:
\textbf{(1)} A data-driven human perception model that tracks how implicit (movement) and explicit (image) information updates the human’s understanding of the maze layout.
\textbf{(2)} Reward augmentation to integrate multiple objectives effectively leveraging on the learned perception model.
\textbf{(3)} An uncertainty-aware MCTS that accounts for unobserved maze regions and human perception stochasticity.
% \begin{enumerate}[leftmargin=*]
%     \item A data-driven human perception model that tracks how implicit (movement) and explicit (image transmission) information updates the human’s understanding of the maze layout.
%     \item Reward augmentation to integrate multiple objectives effectively leveraging on the learned perception model.
%     \item An uncertainty-aware MCTS that accounts for unobserved maze regions and human perception stochasticity.
% \end{enumerate}

\subsection{Human Perception Dynamics}
% IG-MCTS seeks to optimize the expected novel information gained by the human through the robot’s actions, including both movement and communication. Achieving this requires a model of how the human acquires task-relevant information from the robot.

% \subsubsection{Perception MDP}
\label{sec:perception_mdp}
As the robot navigates the maze and transmits images, humans update their understanding of the environment. Based on the robot's path, they may infer that previously assumed blocked locations are traversable or detect discrepancies between the transmitted image and their map.  

To formally capture this process, we model the evolution of human perception as another Markov Decision Process, referred to as the \emph{Perception MDP}. The state space $\mathcal{X}$ represents all possible maze maps. The action space $\mathcal{S}^+ \times \mathcal{O}$ consists of the robot's trajectory between two image transmissions $\tau \in \mathcal{S}^+$ and an image $o \in \mathcal{O}$. The unknown transition function $F: (x, (\tau, o)) \rightarrow x'$ defines the human perception dynamics, which we aim to learn.

\subsubsection{Crowd-Sourced Transition Dataset}
To collect data, we designed a mapping task in the CoNav-Maze environment. Participants were tasked to edit their maps to match the true environment. A button triggers the robot's autonomous movements, after which it captures an image from a random angle.
In this mapping task, the robot, aware of both the true environment and the human’s map, visits predefined target locations and prioritizes areas with mislabeled grid cells on the human’s map.
% We assume that the robot has full knowledge of both the actual environment and the human’s current map. Leveraging this knowledge, the robot autonomously navigates to all predefined target locations. It then randomly selects subsequent goals to reach, prioritizing grid locations that remain mislabeled on the human’s map. This ensures that the robot’s actions are strategically focused on providing useful information to improve map accuracy.

We then recruited over $50$ annotators through Prolific~\cite{palan2018prolific} for the mapping task. Each annotator labeled three randomly generated mazes. They were allowed to proceed to the next maze once the robot had reached all four goal locations. However, they could spend additional time refining their map before moving on. To incentivize accuracy, annotators receive a performance-based bonus based on the final accuracy of their annotated map.


\subsubsection{Fully-Convolutional Dynamics Model}
\label{sec:nhpm}

We propose a Neural Human Perception Model (NHPM), a fully convolutional neural network (FCNN), to predict the human perception transition probabilities modeled in \Cref{sec:perception_mdp}. We denote the model as $F_\theta$ where $\theta$ represents the trainable weights. Such design echoes recent studies of model-based reinforcement learning~\cite{hansen2022temporal}, where the agent first learns the environment dynamics, potentially from image observations~\cite{hafner2019learning,watter2015embed}.

\begin{figure}[t]
    \centering
    \includegraphics[width=0.9\linewidth]{figures/ICML_25_CNN.pdf}
    \caption{Neural Human Perception Model (NHPM). \textbf{Left:} The human's current perception, the robot's trajectory since the last transmission, and the captured environment grids are individually processed into 2D masks. \textbf{Right:} A fully convolutional neural network predicts two masks: one for the probability of the human adding a wall to their map and another for removing a wall.}
    \label{fig:nhpm}
    \vskip -0.1in
\end{figure}

As illustrated in \Cref{fig:nhpm}, our model takes as input the human’s current perception, the robot’s path, and the image captured by the robot, all of which are transformed into a unified 2D representation. These inputs are concatenated along the channel dimension and fed into the CNN, which outputs a two-channel image: one predicting the probability of human adding a new wall and the other predicting the probability of removing a wall.

% Our approach builds on world model learning, where neural networks predict state transitions or environmental updates based on agent actions and observations. By leveraging the local feature extraction capabilities of CNNs, our model effectively captures spatial relationships and interprets local changes within the grid maze environment. Similar to prior work in localization and mapping, the CNN architecture is well-suited for processing spatially structured data and aligning the robot’s observations with human map updates.

To enhance robustness and generalization, we apply data augmentation techniques, including random rotation and flipping of the 2D inputs during training. These transformations are particularly beneficial in the grid maze environment, which is invariant to orientation changes.

\subsection{Perception-Aware Reward Augmentation}
The robot optimizes its actions over a planning horizon \( H \) by solving the following optimization problem:
\begin{subequations}
    \begin{align}
        \max_{a_{0:H-1}} \;
        & \mathop{\mathbb{E}}_{T, F} \left[ \sum_{t=0}^{H-1} \gamma^t \left(\underbrace{R_{\mathrm{task}}(\tau_{t+1}, \zeta)}_{\text{(1) Task reward}} + \underbrace{\|x_{t+1}-x_t\|_1}_{\text{(2) Info reward}}\right)\right] \label{obj}\\ 
        \subjectto \quad
        &x_{t+1} = F(x_t, (\tau_t, a_t)), \quad a_t\in\Ocal \label{const:perception_update}\\ 
        &\tau_{t+1} = \tau_t \oplus T(s_t, a_t), \quad a_t\in \Ucal\label{const:history_update}
    \end{align}
\end{subequations} 

The objective in~\eqref{obj} maximizes the expected cumulative reward over \( T \) and \( F \), reflecting the uncertainty in both physical transitions and human perception dynamics. The reward function consists of two components: 
(1) The \emph{task reward} incentivizes efficient navigation. The specific formulation for the task in this work is outlined in \Cref{appendix:task_reward}.
(2) The \emph{information reward} quantifies the change in the human’s perception due to robot actions, computed as the \( L_1 \)-norm distance between consecutive perception states.  

The constraint in~\eqref{const:history_update} ensures that for movement actions, the trajectory history \( \tau_t \) expands with new states based on the robot’s chosen actions, where \( s_t \) is the most recent state in \( \tau_t \), and \( \oplus \) represents sequence concatenation. 
In constraint~\eqref{const:perception_update}, the robot leverages the learned human perception dynamics \( F \) to estimate the evolution of the human’s understanding of the environment from perception state $x_t$ to $x_{t+1}$ based on the observed trajectory \( \tau_t \) and transmitted image \( a_t\in\Ocal \). 
% justify from a cognitive science perspective
% Cognitive science research has shown that humans read in a way to maximize the information gained from each word, aligning with the efficient coding principle, which prioritizes minimizing perceptual errors and extracting relevant features under limited processing capacity~\cite{kangassalo2020information}. Drawing on this principle, we hypothesize that humans similarly prioritize task-relevant information in multimodal settings. To accommodate this cognitive pattern, our robot policy selects and communicates high information-gain observations to human operators, akin to summarizing key insights from a lengthy article.
% % While the brain naturally seeks to gain information, the brain employs various strategies to manage information overload, including filtering~\cite{quiroga2004reducing}, limiting/working memory, and prioritizing information~\cite{arnold2023dealing}.
% In this context of our setup, we optimize the selection of camera angles to maximize the human operator's information gain about the environment. 

\subsection{Information Gain Monte Carlo Tree Search (IG-MCTS)}
IG-MCTS follows the four stages of Monte Carlo tree search: \emph{selection}, \emph{expansion}, \emph{rollout}, and \emph{backpropagation}, but extends it by incorporating uncertainty in both environment dynamics and human perception. We introduce uncertainty-aware simulations in the \emph{expansion} and \emph{rollout} phases and adjust \emph{backpropagation} with a value update rule that accounts for transition feasibility.

\subsubsection{Uncertainty-Aware Simulation}
As detailed in \Cref{algo:IG_MCTS}, both the \emph{expansion} and \emph{rollout} phases involve forward simulation of robot actions. Each tree node $v$ contains the state $(\tau, x)$, representing the robot's state history and current human perception. We handle the two action types differently as follows:
\begin{itemize}
    \item A movement action $u$ follows the environment dynamics $T$ as defined in \Cref{sec:problem}. Notably, the maze layout is observable up to distance $r$ from the robot's visited grids, while unexplored areas assume a $50\%$ chance of walls. In \emph{expansion}, the resulting search node $v'$ of this uncertain transition is assigned a feasibility value $\delta = 0.5$. In \emph{rollout}, the transition could fail and the robot remains in the same grid.
    
    \item The state transition for a communication step $o$ is governed by the learned stochastic human perception model $F_\theta$ as defined in \Cref{sec:nhpm}. Since transition probabilities are known, we compute the expected information reward $\bar{R_\mathrm{info}}$ directly:
    \begin{align*}
        \bar{R_\mathrm{info}}(\tau_t, x_t, o_t) &= \mathbb{E}_{x_{t+1}}\|x_{t+1}-x_t\|_1 \\
        &= \|p_\mathrm{add}\|_1 + \|p_\mathrm{remove}\|_1,
    \end{align*}
    where $(p_\mathrm{add}, p_\mathrm{remove}) \gets F_\theta(\tau_t, x_t, o_t)$ are the estimated probabilities of adding or removing walls from the map. 
    Directly computing the expected return at a node avoids the high number of visitations required to obtain an accurate value estimate.
\end{itemize}

% We denote a node in the search tree as $v$, where $s(v)$, $r(v)$, and $\delta(v)$ represent the state, reward, and transition feasibility at $v$, respectively. The visit count of $v$ is denoted as $N(v)$, while $Q(v)$ represents its total accumulated return. The set of child nodes of $v$ is denoted by $\mathbb{C}(v)$.

% The goal of each search is to plan a sequence for the robot until it reaches a goal or transmits a new image to the human. We initialize the search tree with the current human guidance $\zeta$, and the robot's approximation of human perception $x_0$. Each search node consists consists of the state information required by our reward augmentation: $(\tau, x)$. A node is terminal if it is the resulting state of a communication step, or if the robot reaches a goal location. 

% A rollout from the expanded node simulates future transitions until reaching a terminal state or a predefined depth $H$. Actions are selected randomly from the available action set $\mathcal{A}(s)$. If an action's feasibility is uncertain due to the environment's unknown structure, the transition occurs with probability $\delta(s, a)$. When a random number draw deems the transition infeasible, the state remains unchanged. On the other hand, for communication steps, we don't resolve the uncertainty but instead compute the expected information gain reward: \philip{TODO: adjust notation}
% \begin{equation}
%     \mathbb{E}\left[R_\mathrm{info}(\tau, x')\right] = \sum \mathrm{NPM(\tau, o)}.
% \end{equation}

\subsubsection{Feasibility-Adjusted Backpropagation}
During backpropagation, the rewards obtained from the simulation phase are propagated back through the tree, updating the total value $Q(v)$ and the visitation count $N(v)$ for all nodes along the path to the root. Due to uncertainty in unexplored environment dynamics, the rollout return depends on the feasibility of the transition from the child node. Given a sample return \(q'_{\mathrm{sample}}\) at child node \(v'\), the parent node's return is:
\begin{equation}
    q_{\mathrm{sample}} = r + \gamma \left[ \delta' q'_{\mathrm{sample}} + (1 - \delta') \frac{Q(v)}{N(v)} \right],
\end{equation}
where $\delta'$ represents the probability of a successful transition. The term \((1 - \delta')\) accounts for failed transitions, relying instead on the current value estimate.

% By incorporating uncertainty-aware rollouts and backpropagation, our approach enables more robust decision-making in scenarios where the environment dynamics is unknown and avoids simulation of the stochastic human perception dynamics.

\section{Experiments and analysis} \label{sec:5}
This section presents comprehensive experiments on the ATR2-HUTD dataset to evaluate the effectiveness of the proposed method. 
Section~\ref{sec:4.1} outlines the experimental metrics used. 
Section~\ref{sec:4.2} details the network architecture, comparison methods, experimental setup, and parameter configurations. 
To highlight the superiority of the proposed method, Section~\ref{sec:4.3} provides both quantitative analysis and visual evaluations across all comparison methods. 
Section~\ref{sec:4.4} includes ablation studies to assess the contributions of different model components, while Section~\ref{sec:4.5} presents a parameter sensitivity analysis.
\subsection{Evaluation Indicators}\label{sec:4.1}
To quantitatively assess the performance of the proposed method, we employ three widely recognized evaluation metrics in the HTD field.
\par
\textbf{(\romannumeral1) Receiver Operating Characteristic (ROC)~\cite{ROC, ROC3D}:} 
The ROC curve offers an unbiased, threshold-independent evaluation of detection performance. This paper presents three 2D ROC curves: $( \mathrm{P}_{\mathrm{d}}, \mathrm{P}_{\mathrm{f}})$, $( \mathrm{P}_{\mathrm{d}}, \tau)$, and $( \mathrm{P}_{\mathrm{f}}, \tau)$, along with a 3D ROC curve~\cite{ROC3D} of $(\tau, \mathrm{P}_{\mathrm{d}}, \mathrm{P}_{\mathrm{f}})$ for a comprehensive performance evaluation. A detector with ROC curves closer to the upper left, upper right, and lower left corners generally exhibits superior HTD performance.
\par
\textbf{(\romannumeral2) Area Under the ROC Curve (AUC)~\cite{Zhang2015}:} 
To address challenges in visually comparing ROC curves, we compute the area under each of the three 2D ROC curves: $\text{AUC}_{( \mathrm{P}_{\mathrm{d}}, \mathrm{P}_{\mathrm{f}})}$, $\text{AUC}_{( \mathrm{P}_{\mathrm{d}}, \tau)}$, and $\text{AUC}_{( \mathrm{P}_{\mathrm{f}}, \tau)}$. Larger AUC values indicate better performance, with $\text{AUC}_{( \mathrm{P}_{\mathrm{d}}, \mathrm{P}_{\mathrm{f}})} \to 1$, $\text{AUC}_{( \mathrm{P}_{\mathrm{d}}, \tau)} \to 1$, and $\text{AUC}_{( \mathrm{P}_{\mathrm{f}}, \tau)} \to 0$ signifying superior detection performance. Additionally, two AUC-based metrics are introduced for a more comprehensive evaluation:
\begin{equation}
    \mathrm{AUC}_{\mathrm{OA}} = \mathrm{AUC}_{\left(P_f, P_d\right)} + \mathrm{AUC}_{\left(\tau, P_d\right)} - \mathrm{AUC}_{\left(\tau, P_f\right)},
\end{equation}
\begin{equation}
    \mathrm{AUC}_{\mathrm{SNPR}} = \frac{\mathrm{AUC}_{\left(\tau, P_d\right)}}{\mathrm{AUC}_{\left(\tau, P_f\right)}},
\end{equation}
where higher values of $\mathrm{AUC}_{\mathrm{OA}} \to 2$ and $\mathrm{AUC}_{\mathrm{SNPR}} \to +\infty$ indicate improved detector performance.
% \textbf{(\romannumeral3) Separability Map~\cite{Liu2022}:} The degree of separation between the targets and backgrounds in the detection map is a critical performance indicator for UTD methods. 
% Thus, we also utilize the separability map for quantitative comparison in this study. 
% Specifically, the separability map uses green and blue boxes to represent the statistics of the target and background, respectively. 
% The horizontal line within each box indicates the median value, while the upper and lower whiskers denote the maximum and minimum values, providing a clear representation of the data range and central tendency. 
% \par
% A larger overlap between the two boxes suggests that the statistics of the target and background are similar, indicating poor separation between them. 
% Conversely, less overlap indicates better separation. 
% Moreover, background suppression is considered more effective when the blue box is closer to the ordinate 0, while higher target prominence is indicated when the green box is closer to ordinate 1.
% \clearpage
\subsection{Experimental Details and Settings}\label{sec:4.2}
\textbf{(\romannumeral1) Experimental Details:} 
The experimental setup and details of the proposed method are as follows. Unless otherwise specified, the parameters are applied consistently across all sub-datasets. The method consists of three core components: the RGC module, the HLCL module, and the SPL strategy, each contributing significantly to performance.

In the RGC module, unsupervised clustering is performed using the K-Means~\cite{Sinaga2020} algorithm, with cluster numbers set to 36, 39, and 42 for the lake, river, and sea sub-datasets, respectively, based on environmental complexity and waterbed characteristics.

The HLCL module employs the 3D-ResNet50~\cite{Jiang2019} network for spectral-spatial feature extraction. To enhance robustness and contrastive learning, untargeted FGSM~\cite{GoodfellowSS14} data augmentation is applied with a maximum perturbation of $\epsilon=0.1$ under the $l_{\infty}$ norm. The hybrid-level contrastive learning framework is trained for 50 epochs per SPL iteration. The Adam optimizer is used with a batch size of 256. The initial learning rate is $5\times10^{-3}$, decaying to $5\times10^{-5}$ through a cosine annealing schedule after 100 epochs, and a weight decay of $1\times10^{-4}$ is applied to reduce overfitting.

The SPL strategy is executed for 10 iterations across all sub-datasets to ensure convergence and computational efficiency.

For HUTD, as described in Section~\ref{sec3.4}, we use learned representations combined with basic hyperspectral detectors. To isolate the effect of detectors on performance, we employ two classic detectors, CEM~\cite{KRUSE1993145} and SAM~\cite{Manolakis2002}, as baseline methods.

\textbf{(\romannumeral2) Experimental Settings:} 
We compare the proposed method against several state-of-the-art (SOTA) HTD and HUTD methods, including two traditional HTD detectors (CEM and SAM), two advanced HTD methods (IEEPST~\cite{IEEPST} and MCLT~\cite{Wang2024}), and four HUTD methods (UTD-Net~\cite{Qi2021}, TUTDF~\cite{LiZheyong2023}, TDSS-UTD~\cite{Li2023}, and NUN-UTD~\cite{Liu2024}).

To ensure fairness, each method is executed with the original hyperparameter settings as specified in their respective publications. All experiments are conducted on a machine equipped with seven NVIDIA A6000 GPUs, an AMD Ryzen 5995WX CPU, and 128 GB of RAM, running Ubuntu 22.04.

\subsection{Main Results} \label{sec:4.3}
\textbf{(\romannumeral1) Detection Maps:} Figs.~\ref{fig:C1-1} to~\ref{fig:C1-2} present detection maps from the ATR2-HUTD-Lake sub-dataset, offering a qualitative comparison of the evaluated methods.
The detection maps of other sub-datasets are provided in the supplementary material.
\par
\begin{figure*}[!t]                 
    \centering                    
    \includegraphics[width=2\columnwidth]{images/C1-1.jpg}                     
    \caption{Detection maps of ATR2-HUTD Lake Scene1. (a) Pseudo-color image. (b) Ground truth. (c) CEM. (d) SAM. (e) IEEPST. (f) MCLT. (g) UTD-Net. (h) TUTDF. (i) TDSS-UTD. (j) NUN-UTD. (m) HUCLNet+CEM. (n) HUCLNet+SAM.}                  
    \label{fig:C1-1}    
\end{figure*}
\begin{figure*}[!t]                 
    \centering                    
    \includegraphics[width=2\columnwidth]{images/C1-2.jpg}                     
    \caption{Detection maps of ATR2-HUTD Lake Scene2. (a) Pseudo-color image. (b) Ground truth. (c) CEM. (d) SAM. (e) IEEPST. (f) MCLT. (g) UTD-Net. (h) TUTDF. (i) TDSS-UTD. (j) NUN-UTD. (m) HUCLNet+CEM. (n) HUCLNet+SAM.}                    
    \label{fig:C1-2}    
\end{figure*}
% \begin{figure*}[!t]                 
%     \centering                    
%     \includegraphics[width=2\columnwidth]{images/C1-3.jpg}                     
%     \caption{Detection maps of ATR2-HUTD River Scene1. (a) Pseudo-color image. (b) Ground truth. (c) CEM. (d) SAM. (e) IEEPST. (f) MCLT. (g) UTD-Net. (h) TUTDF. (i) TDSS-UTD. (j) NUN-UTD. (m) HUCLNet+CEM. (n) HUCLNet+SAM.}                      
%     \label{fig:C1-3}    
% \end{figure*}
% \begin{figure*}[!t]                 
%     \centering                    
%     \includegraphics[width=2\columnwidth]{images/C1-4.jpg}                     
%     \caption{Detection maps of ATR2-HUTD River Scene2. (a) Pseudo-color image. (b) Ground truth. (c) CEM. (d) SAM. (e) IEEPST. (f) MCLT. (g) UTD-Net. (h) TUTDF. (i) TDSS-UTD. (j) NUN-UTD. (m) HUCLNet+CEM. (n) HUCLNet+SAM.}                     
%     \label{fig:C1-4}    
% \end{figure*}
% \begin{figure*}[!t]                 
%     \centering                    
%     \includegraphics[width=2\columnwidth]{images/C1-5.jpg}                     
%     \caption{Detection maps of ATR2-HUTD Sea Scene1. (a) Pseudo-color image. (b) Ground truth. (c) CEM. (d) SAM. (e) IEEPST. (f) MCLT. (g) UTD-Net. (h) TUTDF. (i) TDSS-UTD. (j) NUN-UTD. (m) HUCLNet+CEM. (n) HUCLNet+SAM.}                 
%     \label{fig:C1-5}    
% \end{figure*}
% \begin{figure*}[!t]                 
%     \centering                    
%     \includegraphics[width=2\columnwidth]{images/C1-6.jpg}                     
%     \caption{Detection maps of ATR2-HUTD Sea Scene2. (a) Pseudo-color image. (b) Ground truth. (c) CEM. (d) SAM. (e) IEEPST. (f) MCLT. (g) UTD-Net. (h) TUTDF. (i) TDSS-UTD. (j) NUN-UTD. (m) HUCLNet+CEM. (n) HUCLNet+SAM.}                    
%     \label{fig:C1-2}    
% \end{figure*}
Traditional methods, such as CEM and SAM, exhibit significant limitations in underwater environments. CEM struggles with background noise suppression, resulting in false positives, while SAM fails to delineate target boundaries and often misses targets, especially in complex scenarios like the ATR2-HUTD River dataset. Its sensitivity to spectral noise and limited adaptability to spectral variations lead to incomplete detection and poor target-background separation.
\par
\begin{figure*}[!t]                 
    \centering                    
    \includegraphics[width=2\columnwidth]{images/C2-1.jpg}                     
    \caption{ROC curves comparison on ATR2-HUTD Lake Scene1. (a) 3-D ROC curve. (b) 2-D ROC curve of $(P_d, P_f)$. (c) 2-D ROC curve of $(P_f, \tau)$. (d) 2-D ROC curve of $(P_d, \tau)$.}                 
    \label{fig:C2-1}    
\end{figure*}
\begin{figure*}[!t]                 
    \centering                    
    \includegraphics[width=2\columnwidth]{images/C2-2.jpg}                     
    \caption{ROC curves comparison on ATR2-HUTD Lake Scene2. (a) 3-D ROC curve. (b) 2-D ROC curve of $(P_d, P_f)$. (c) 2-D ROC curve of $(P_f, \tau)$. (d) 2-D ROC curve of $(P_d, \tau)$.}                                  
    \label{fig:C2-2}    
\end{figure*}
Advanced land-cover detection methods, including IEEPST and MCLT, also underperform in underwater environments. IEEPST struggles to suppress background interference, particularly when water column spectral signatures overlap with target signatures in the ATR2-HUTD River sub-dataset. While MCLT leverages contrastive learning for feature enhancement, it shows reduced sensitivity to small or low-reflectance targets, hindered by the nonlinearities and spectral noise typical of underwater HSI data. These results underscore the necessity of specialized techniques for HUTD.

Among SOTA HUTD methods, UTD-Net demonstrates notable improvements by effectively unmixing target-water mixed pixels. However, it faces challenges with background interference in scenes with extensive non-target bottom areas, leading to high false positive rates. NUN-UTD improves target identification by preserving weak target spectral signals, yet remains susceptible to background interference when spectral characteristics of the background resemble those of the target, leading to false positives in spectrally overlapping environments.

Physical-based methods, such as TUTDF and TDSS-UTD, enhance background suppression using underwater imaging models and predicted depth values. However, TUTDF's performance declines in complex environments due to depth estimation inaccuracies, leading to inconsistent detection. Similarly, TDSS-UTD struggles in environments with substantial depth variation, such as the ATR2-HUTD River dataset, where depth errors degrade detection accuracy. Variations in underwater imaging mechanisms between deep and nearshore scenes further limit their effectiveness.

In contrast, HUCLNet-based methods consistently outperform the alternatives. By integrating instance-level and prototype-level contrastive learning, these methods effectively detect faint and deeply submerged targets with minimal false positives, enhancing background suppression and detection accuracy. HUCLNet+CEM and HUCLNet+SAM show resilience to spectral variability, capturing subtle target features while maintaining clear target-background separation, even under significant underwater bottom interference. These methods provide the most comprehensive target coverage and background suppression in challenging environments, such as the ATR2-HUTD River dataset, demonstrating the superior effectiveness of HUCLNet in mitigating spectral variability and improving detection accuracy.
\par 
\textbf{(\romannumeral2) ROC Curves:} Subjective analysis of detection maps may be insufficient for comprehensive evaluation. Therefore, 3-D ROC curves and their 2-D projections: ($P_d$, $P_f$), ($P_d$, $\tau$), and ($P_f$, $\tau$) were used to objectively assess detection performance on the ATR2-HUTD dataset, enabling a detailed evaluation of detection efficiency, target preservation, and background suppression. 
The ROC curves of ATR-HUTD-Lake sub-dataset are provided in Figs~\ref{fig:C2-1} to~\ref{fig:C2-2}, while those of the ATR-HUTD-River and ATR-HUTD-Sea sub-datasets are provided in the supplementary material.
\par
Figs.~\ref{fig:C2-1} (a) to~\ref{fig:C2-2} (a) show the 3-D ROC curves, highlighting the relationship between the true positive rate ($P_d$), false alarm probability ($P_f$), and detection threshold ($\tau$). HUCLNet+CEM and HUCLNet+SAM consistently outperform other methods, exhibiting higher $P_d$ and lower $P_f$ over a wide range of $\tau$, demonstrating superior adaptability.
\par
% \begin{figure*}[!t]                 
%     \centering                    
%     \includegraphics[width=2\columnwidth]{images/C2-3.jpg}                     
%     \caption{ROC curves comparison on ATR2-HUTD River Scene1. (a) 3-D ROC curve. (b) 2-D ROC curve of $(P_d, P_f)$. (c) 2-D ROC curve of $(P_f, \tau)$. (d) 2-D ROC curve of $(P_d, \tau)$.}                                   
%     \label{fig:C2-3}    
% \end{figure*}
% \begin{figure*}[!t]                 
%     \centering                    
%     \includegraphics[width=2\columnwidth]{images/C2-4.jpg}                     
%     \caption{ROC curves comparison on ATR2-HUTD River Scene2. (a) 3-D ROC curve. (b) 2-D ROC curve of $(P_d, P_f)$. (c) 2-D ROC curve of $(P_f, \tau)$. (d) 2-D ROC curve of $(P_d, \tau)$.}                                 
%     \label{fig:C2-4}    
% \end{figure*}
% \begin{figure*}[!t]                 
%     \centering                    
%     \includegraphics[width=2\columnwidth]{images/C2-5.jpg}                     
%     \caption{ROC curves comparison on ATR2-HUTD Sea Scene1. (a) 3-D ROC curve. (b) 2-D ROC curve of $(P_d, P_f)$. (c) 2-D ROC curve of $(P_f, \tau)$. (d) 2-D ROC curve of $(P_d, \tau)$.}                                   
%     \label{fig:C2-5}    
% \end{figure*}
% \begin{figure*}[!t]                 
%     \centering                    
%     \includegraphics[width=2\columnwidth]{images/C2-6.jpg}                     
%     \caption{ROC curves comparison on ATR2-HUTD Sea Scene2. (a) 3-D ROC curve. (b) 2-D ROC curve of $(P_d, P_f)$. (c) 2-D ROC curve of $(P_f, \tau)$. (d) 2-D ROC curve of $(P_d, \tau)$.}                                
%     \label{fig:C2-2}    
% \end{figure*}
Figs.~\ref{fig:C2-1} (b) to~\ref{fig:C2-2} (b) present the 2-D ROC curves of ($P_d$, $P_f$). HUCLNet-based methods occupy the top-left region, indicating superior detection accuracy. In contrast, traditional HTD methods, such as CEM and SAM, struggle to balance $P_d$ and $P_f$, particularly for targets with varying spectral properties. Although advanced HTD and SOTA HUTD methods show moderate performance, they fail to suppress false alarms in complex river environments, compromising detection accuracy.

Figs.~\ref{fig:C2-1}(c) to~\ref{fig:C2-2}(c) depict the 2-D ROC curves of ($P_f$, $\tau$), assessing background suppression. NUN-UTD shows high $P_f$ across thresholds, indicating poor background-target discrimination. While methods like MCLT and TUTDF show some improvement, they still struggle with high false alarm rates due to spectral overlap. \textbf{UTD-Net performs well in background suppression but largely by classifying all pixels as background}, as reflected in detection maps (Figs.~\ref{fig:C1-1} to~\ref{fig:C1-2}) and AUC$_{P_{d}, \tau}$ values (Tabs.~\ref{auc_lake} to~\ref{auc_sea}). In comparison, HUCLNet+CEM and HUCLNet+SAM exhibit superior background suppression with low $P_f$ and high AUC$_{P_{d}, \tau}$ values.

Figs.~\ref{fig:C2-1}(d) to~\ref{fig:C2-2}(d) present the 2-D ROC curves of ($P_d$, $\tau$), evaluating target preservation. Traditional methods, such as SAM, show significant drops in $P_d$ as $\tau$ increases, indicating poor target preservation. Advanced HTD and SOTA HUTD methods, such as MCLT and TDSS-UTD, show some improvement but still lag behind NUN-UTD and TUTDF. However, \textbf{the improved performance of NUN-UTD and TUTDF primarily results from misclassifying all pixels as targets}, as shown by high false alarm rates in detection maps (Figs.~\ref{fig:C1-1} to~\ref{fig:C1-2}) and increased AUC$_{P_{f}, \tau}$ values. In contrast, HUCLNet+CEM and HUCLNet+SAM maintain high $P_d$ at lower $\tau$, demonstrating robust and reliable target preservation.
\par
\begin{table*}[!t] 
    \centering
    \footnotesize   
    \caption{Quantitative comparison results on the ATR2-HUTD-Lake Sub-dataset. The best and second best results are in \textbf{bold} and with \underline{underline}.} \label{auc_lake}
    \renewcommand{\arraystretch}{1.5}
    \setlength{\tabcolsep}{1.85mm}
    \scalebox{0.875}
    {
        \begin{tabular}{ccccccccccc}
            \hline
            \multirow{2.4}{*}{\textbf{Method}} & \multicolumn{5}{c}{\cellcolor{tablecolor7!60}\textbf{ATR2-HUTD-Lake Scene1}}       & \multicolumn{5}{c}{\cellcolor{tablecolor8}\textbf{ATR2-HUTD-Lake Scene2}}       \\ \cmidrule(lr){2-6} \cmidrule(lr){7-11}
                                    & $\text{AUC}_{( \mathrm{P}_{\mathrm{d}},\mathrm{P}_{\mathrm{f}})}\textcolor{red}{\uparrow }$ & $\text{AUC}_{( \mathrm{P}_{\mathrm{f}}, \tau)}\textcolor{green}{\downarrow }$ & $\text{AUC}_{( \mathrm{P}_{\mathrm{d}},\tau)}\textcolor{red}{\uparrow }$ & $\mathrm{AUC}_{\mathrm{OA}} \textcolor{red}{\uparrow }$ & $\mathrm{AUC}_{\mathrm{SNPR}}\textcolor{red}{\uparrow }$ & $\text{AUC}_{( \mathrm{P}_{\mathrm{d}},\mathrm{P}_{\mathrm{f}})}\textcolor{red}{\uparrow }$ & $\text{AUC}_{( \mathrm{P}_{\mathrm{f}}, \tau)}\textcolor{green}{\downarrow }$ & $\text{AUC}_{( \mathrm{P}_{\mathrm{d}},\tau)}\textcolor{red}{\uparrow }$ & $\mathrm{AUC}_{\mathrm{OA}} \textcolor{red}{\uparrow }$ & $\mathrm{AUC}_{\mathrm{SNPR}}\textcolor{red}{\uparrow }$ \\ \hline
                                    CEM         & 0.671          & 0.250          & 0.258          & 0.678          & 1.028          & 0.489          & 0.524          & 0.520          & 0.485          & 0.994          \\
                                    SAM         & 0.670          & \underline{0.129}    & 0.151          & 0.692          & 1.170          & 0.480          & \underline{0.143}    & 0.025          & 0.362          & 0.172          \\
                                    IEEPST      & 0.424          & 0.204          & 0.075          & 0.295          & 0.369          & 0.417          & 0.187          & 0.036          & 0.266          & 0.193          \\
                                    MCLT        & 0.401          & 0.422          & 0.377          & 0.357          & 0.894          & 0.365          & 0.243          & 0.173          & 0.296          & 0.715          \\
                                    UTD-Net     & 0.846          & \textbf{0.013} & 0.019          & 0.853          & 1.510          & 0.944          & \textbf{0.041} & 0.073          & 0.976          & 1.773          \\
                                    TUTDF       & \underline{0.990}          & 0.634          & \underline{0.726}    & 1.081          & 1.145          & \underline{0.998}          & 0.461          & \underline{0.768}    & 1.306          & 1.667          \\
                                    TDSS-UTD    & 0.964          & 0.215          & 0.369          & 1.117          & 1.712          & \textbf{0.999} & 0.166          & 0.444          & 1.277          & 2.676          \\
                                    NUN-UTD     & 0.758          & 0.913          & \textbf{0.994} & 0.838          & 1.088          & 0.765          & 0.792          & \textbf{0.995} & 0.968          & 1.257          \\
            \rowcolor{tablecolor13!60}HUCLNet+CEM & 0.958    & 0.302          & 0.642          & \underline{1.298}    & \underline{2.126}    & 0.989    & 0.226          & 0.634          & \underline{1.397}    & \underline{2.805}    \\
            \rowcolor{tablecolor14!60}HUCLNet+SAM & \textbf{0.995} & 0.209          & 0.710          & \textbf{1.501} & \textbf{3.393} & \textbf{0.999} & 0.265          & 0.765          & \textbf{1.501} & \textbf{2.891} \\ \hline
        \end{tabular}}
\end{table*}
\begin{table*}[!t] 
    \centering
    \footnotesize   
    \caption{Quantitative comparison results on the ATR2-HUTD-River Sub-dataset. The best and second best results are in \textbf{bold} and with \underline{underline}.} \label{auc_river}
    \renewcommand{\arraystretch}{1.5}
    \setlength{\tabcolsep}{1.85mm}
    \scalebox{0.875}
    {
        \begin{tabular}{ccccccccccc}
            \hline
            \multirow{2.4}{*}{\textbf{Method}} & \multicolumn{5}{c}{\cellcolor{tablecolor9}\textbf{ATR2-HUTD-River Scene1}}       & \multicolumn{5}{c}{\cellcolor{tablecolor10}\textbf{ATR2-HUTD-River Scene2}}       \\ \cmidrule(lr){2-6} \cmidrule(lr){7-11}
                                    & $\text{AUC}_{( \mathrm{P}_{\mathrm{d}},\mathrm{P}_{\mathrm{f}})}\textcolor{red}{\uparrow }$ & $\text{AUC}_{( \mathrm{P}_{\mathrm{f}}, \tau)}\textcolor{green}{\downarrow }$ & $\text{AUC}_{( \mathrm{P}_{\mathrm{d}},\tau)}\textcolor{red}{\uparrow }$ & $\mathrm{AUC}_{\mathrm{OA}} \textcolor{red}{\uparrow }$ & $\mathrm{AUC}_{\mathrm{SNPR}}\textcolor{red}{\uparrow }$ & $\text{AUC}_{( \mathrm{P}_{\mathrm{d}},\mathrm{P}_{\mathrm{f}})}\textcolor{red}{\uparrow }$ & $\text{AUC}_{( \mathrm{P}_{\mathrm{f}}, \tau)}\textcolor{green}{\downarrow }$ & $\text{AUC}_{( \mathrm{P}_{\mathrm{d}},\tau)}\textcolor{red}{\uparrow }$ & $\mathrm{AUC}_{\mathrm{OA}} \textcolor{red}{\uparrow }$ & $\mathrm{AUC}_{\mathrm{SNPR}}\textcolor{red}{\uparrow }$ \\ \hline
                                    CEM         & 0.746          & 0.280          & 0.300          & 0.765          & 1.070          & 0.650          & 0.544          & 0.553          & 0.659          & 1.016          \\
                                    SAM         & 0.657          & 0.214          & 0.186          & 0.629          & 0.871          & 0.656          & \underline{0.078}    & 0.066          & 0.645          & 0.854          \\
                                    IEEPST      & 0.455          & 0.203          & 0.033          & 0.286          & 0.163          & 0.594          & 0.274          & 0.236          & 0.556          & 0.861          \\
                                    MCLT        & 0.550          & 0.989          & 0.990          & 0.552          & 1.001          & 0.531          & 0.970          & \underline{0.971}    & 0.533          & 1.002          \\
                                    UTD-Net     & \underline{0.843}          & \underline{0.080}    & 0.096          & 0.860          & 1.209          & \underline{0.889}          & \textbf{0.075} & 0.088          & \underline{0.903}          & 1.176          \\
                                    TUTDF       & 0.568          & 0.822          & \underline{0.824}    & 0.570          & 1.003          & 0.659          & 0.356          & 0.363          & 0.667          & 1.022          \\
                                    TDSS-UTD    & 0.402          & 0.438          & 0.415          & 0.379          & 0.948          & 0.539 & 0.179          & 0.174          & 0.534          & 0.974          \\
                                    NUN-UTD     & 0.632          & 0.968          & \textbf{0.999} & 0.663          & 1.032          & 0.503          & 0.977          & \textbf{0.980} & 0.505          & 1.002          \\
            \rowcolor{tablecolor13!60}HUCLNet+CEM & 0.794    & 0.353          & 0.518          & \underline{0.959}    & \underline{1.468}    & 0.753    & 0.354          & 0.481          & 0.880    & \underline{1.360}    \\
            \rowcolor{tablecolor14!60}HUCLNet+SAM & \textbf{0.966} & \textbf{0.055} & 0.175          & \textbf{1.086} & \textbf{3.206} & \textbf{0.924} & 0.178          & 0.327          & \textbf{1.073} & \textbf{1.837} \\ \hline
        \end{tabular}}
\end{table*}
\begin{table*}[!t] 
    \centering
    \footnotesize   
    \caption{Quantitative comparison results on the ATR2-HUTD-Sea Sub-dataset. The best and second best results are in \textbf{bold} and with \underline{underline}.} \label{auc_sea}
    \renewcommand{\arraystretch}{1.5}
    \setlength{\tabcolsep}{1.85mm}
    \scalebox{0.875}
    {
        \begin{tabular}{ccccccccccc}
            \hline
            \multirow{2.4}{*}{\textbf{Method}} & \multicolumn{5}{c}{\cellcolor{tablecolor11}\textbf{ATR2-HUTD-Sea Scene1}}       & \multicolumn{5}{c}{\cellcolor{tablecolor12!50}\textbf{ATR2-HUTD-Sea Scene2}}       \\ \cmidrule(lr){2-6} \cmidrule(lr){7-11}
                                    & $\text{AUC}_{( \mathrm{P}_{\mathrm{d}},\mathrm{P}_{\mathrm{f}})}\textcolor{red}{\uparrow }$ & $\text{AUC}_{( \mathrm{P}_{\mathrm{f}}, \tau)}\textcolor{green}{\downarrow }$ & $\text{AUC}_{( \mathrm{P}_{\mathrm{d}},\tau)}\textcolor{red}{\uparrow }$ & $\mathrm{AUC}_{\mathrm{OA}} \textcolor{red}{\uparrow }$ & $\mathrm{AUC}_{\mathrm{SNPR}}\textcolor{red}{\uparrow }$ & $\text{AUC}_{( \mathrm{P}_{\mathrm{d}},\mathrm{P}_{\mathrm{f}})}\textcolor{red}{\uparrow }$ & $\text{AUC}_{( \mathrm{P}_{\mathrm{f}}, \tau)}\textcolor{green}{\downarrow }$ & $\text{AUC}_{( \mathrm{P}_{\mathrm{d}},\tau)}\textcolor{red}{\uparrow }$ & $\mathrm{AUC}_{\mathrm{OA}} \textcolor{red}{\uparrow }$ & $\mathrm{AUC}_{\mathrm{SNPR}}\textcolor{red}{\uparrow }$ \\ \hline
                                    CEM         & 0.805          & 0.309          & 0.349          & 0.845          & 1.128           & 0.845          & 0.332          & 0.351          & 0.864          & 1.057          \\
                                    SAM         & 0.866          & 0.125    & 0.188          & 0.929          & 1.503           & 0.819          & 0.099          & 0.033          & 0.753          & 0.333          \\
                                    IEEPST      & 0.850          & 0.252          & 0.363          & 0.961          & 1.441           & 0.580          & 0.326          & 0.269          & 0.523          & 0.826          \\
                                    MCLT        & 0.895          & 0.980          & \underline{0.994} & 0.909          & 1.014           & 0.317          & 0.953          & \underline{0.944}    & 0.309          & 0.991          \\
                                    UTD-Net     & 0.762          & \underline{0.050}          & 0.083          & 0.796          & 1.682           & 0.774          & \textbf{0.043} & 0.070          & 0.801          & 1.634          \\
                                    TUTDF       & 0.952          & 0.841          & 0.872          & 0.984          & 1.037           & 0.903          & 0.426          & 0.482          & 0.959          & 1.131          \\
                                    TDSS-UTD    & 0.861          & 0.310          & 0.371          & 0.923          & 1.199           & 0.984          & 0.218          & 0.425          & 1.192          & 1.948          \\
                                    NUN-UTD     & \underline{0.979}    & 0.534          & \textbf{0.999} & \textbf{1.445} & 1.872           & 0.975          & 0.959          & \textbf{0.984} & 0.999          & 1.025          \\
            \rowcolor{tablecolor13!60}HUCLNet+CEM & 0.972          & 0.133          & 0.569          & \underline{1.409}    & \underline{4.284}     & \underline{0.987}    & 0.111          & 0.401          & \underline{1.287}    & \underline{3.620}    \\
            \rowcolor{tablecolor14!60}HUCLNet+SAM & \textbf{0.985} & \textbf{0.019} & 0.325          & 1.292          & \textbf{17.501} & \textbf{0.989} & \underline{0.053}    & 0.474          & \textbf{1.420} & \textbf{8.857} \\ \hline
        \end{tabular}}
\end{table*}
\textbf{(\romannumeral3) AUC Values:} The AUC values for each sub-dataset of the ATR2-HUTD dataset are computed using five key metrics: $\text{AUC}_{( \mathrm{P}_{\mathrm{d}}, \mathrm{P}_{\mathrm{f}})}$, $\text{AUC}_{( \mathrm{P}_{\mathrm{d}}, \tau)}$, $\text{AUC}_{( \mathrm{P}_{\mathrm{f}}, \tau)}$, $\text{AUC}_{SNPR}$, and $\text{AUC}_{OA}$, as detailed in Tabs.~\ref{auc_lake} to~\ref{auc_sea}. These metrics quantitatively assess detection accuracy, target preservation, background suppression, signal-to-noise ratio, and overall performance in varied underwater environments.
\par
\begin{table*}[!ht] 
    \centering
    \footnotesize   
    \caption{Quantitative results of ablation studies on the ATR2-HUTD dataset.} \label{ablation study}
    \renewcommand{\arraystretch}{2}
    \setlength{\tabcolsep}{2.5mm}
    \begin{threeparttable}
        \scalebox{0.975}
        { 
    \begin{tabular}{ccccccc}
        \hline
        \textbf{Module Name}                  & \textbf{Design}                                                      & $\text{AUC}_{( \mathrm{P}_{\mathrm{d}},\mathrm{P}_{\mathrm{f}})}\textcolor{red}{\uparrow }$ & $\text{AUC}_{( \mathrm{P}_{\mathrm{f}}, \tau)}\textcolor{green}{\downarrow }$ & $\text{AUC}_{( \mathrm{P}_{\mathrm{d}},\tau)}\textcolor{red}{\uparrow }$ & $\mathrm{AUC}_{\mathrm{OA}} \textcolor{red}{\uparrow }$ & $\mathrm{AUC}_{\mathrm{SNPR}}\textcolor{red}{\uparrow }$  \\ \hline
        \rowcolor{tablecolor0!50}
        \textbf{HUCLNet}                                          & N/A & 0.943 & 0.188 & 0.502 & 1.258 & 4.446 \\
        \rowcolor{tablecolor1!50}
        \cellcolor{tablecolor1!50}                             & w/o Cluster Refinement Strategy                             & 0.823 & 0.206 & 0.388 & 1.005 & 3.141 \\
        \rowcolor{tablecolor1!50}
        \multirow{-2}{*}{\cellcolor{tablecolor1!50}\textbf{RGC module}}  & w/o Reference Spectrum based Clustering Method & 0.737 & 0.211 & 0.375 & 0.901 & 2.616 \\
        \rowcolor{tablecolor2!50} 
        \cellcolor{tablecolor2!50}                              & w/o Instance-level Contrastive Learning                     & 0.878 & 0.199 & 0.438 & 1.117 & 3.513 \\
        \rowcolor{tablecolor2!50} 
        \cellcolor{tablecolor2!50}                              & w/o Prototype-level Contrastive Learning                    & 0.728 & 0.239 & 0.359 & 0.848 & 2.359 \\
        \rowcolor{tablecolor2!50}
        \cellcolor{tablecolor2!50}                              & w/o Hyperspectral-Oriented Data Augmentation                    & 0.883 & 0.195 & 0.452 & 1.165 & 3.584 \\
        \rowcolor{tablecolor2!50} 
        \multirow{-4}{*}{\cellcolor{tablecolor2!50}\textbf{HLCL module}} & w/o HLCL module$^{1}$                                             & 0.696 & 0.252 & 0.248 & 0.692 & 0.933 \\
        \rowcolor{tablecolor3!50} 
        \textbf{SPL Paradigm}                                          & w/o SPL Paradigm                                            & 0.743 & 0.217 & 0.388 & 0.914 & 2.864 \\ \hline
        \end{tabular}}
        \begin{tablenotes}
            \scriptsize
            \item[1] This experimental design is analogous to the baseline HTD methods, as the RGC module and SPL paradigm are dependent on the HLCL module for functionality.
        \end{tablenotes}
        \end{threeparttable}
\end{table*}
The $\text{AUC}_{( \mathrm{P}_{\mathrm{d}}, \mathrm{P}_{\mathrm{f}})}$ metric, which quantifies the trade-off between the true positive rate ($P_d$) and false alarm probability ($P_f$), is critical for evaluating detection performance. HUCLNet+SAM leads with an average score of 0.976, followed by HUCLNet+CEM at 0.909. Traditional methods, such as SAM (0.701) and MCLT (0.691), underperform significantly, while SOTA HUTD methods like TUTDF and NUN-UTD fall short of HUCLNet-based methods in detection capability.
\par
For background suppression, assessed by $\text{AUC}_{( \mathrm{P}_{\mathrm{f}}, \tau)}$, HUCLNet+SAM achieves the highest performance in the ATR2-HUTD-River Scene1 and ATR2-HUTD-Sea sub-datasets, the most complex nearshore environments. It also demonstrates robust performance across other sub-datasets. In contrast, SOTA HUTD methods, including TUTDF and NUN-UTD, show elevated values, suggesting overfitting due to high false positive rates.
\par
The $\text{AUC}_{( \mathrm{P}_{\mathrm{d}}, \tau)}$ metric, assessing target preservation, reveals HUCLNet-based methods performing well, though NUN-UTD leads. This may be attributed to the HLCL module in HUCLNet, which compromises target-background feature separation, impacting target preservation. Additionally, NUN-UTD's higher false positive rate boosts $P_d$ but hinders background suppression.
\par
The $\text{AUC}_{OA}$ metric, combining $\text{AUC}_{( \mathrm{P}_{\mathrm{d}}, \mathrm{P}_{\mathrm{f}})}$, $\text{AUC}_{( \mathrm{P}_{\mathrm{d}}, \tau)}$, and $\text{AUC}_{( \mathrm{P}_{\mathrm{f}}, \tau)}$, further emphasizes HUCLNet's superiority. HUCLNet+SAM achieves the highest average score of 1.312, with HUCLNet+CEM following at 1.205. In contrast, traditional and SOTA HUTD methods score between 0.492 and 0.928, underscoring HUCLNet's effectiveness in background suppression, target preservation, and detection accuracy in complex nearshore environments.
\par
Finally, the $\text{AUC}_{SNPR}$ metric, which measures robustness under varying signal-to-noise ratios, underscores HUCLNet+SAM's superior performance, achieving the highest scores across all sub-datasets, including 17.501 in ATR2-HUTD-Sea Scene1. HUCLNet+CEM consistently ranks second, while traditional HTD and SOTA HUTD methods show lower scores, indicating reduced robustness in fluctuating signal conditions.
\par
% \begin{figure*}[!t]                 
%     \centering                    
%     \includegraphics[scale=0.65]{images/C3-1.jpg}                     
%     \caption{Target-background separability boxplots for ATR2-HUTD Lake sub-dataset. (a) Scene1. (b) Scene2.}                    
%     \label{fig:C3-1}    
% \end{figure*}


% \begin{figure*}[!t]                 
%     \centering                    
%     \includegraphics[scale=0.65]{images/C3-2.jpg}                     
%     \caption{Target-background separability boxplots for ATR2-HUTD River sub-dataset. (a) Scene1. (b) Scene2.}                                 
%     \label{fig:C3-2}    
% \end{figure*}
% \begin{figure*}[!t]                 
%     \centering                    
%     \includegraphics[scale=0.65]{images/C3-3.jpg}                     
%     \caption{Target-background separability boxplots for ATR2-HUTD Sea sub-dataset. (a) Scene1. (b) Scene2.}                                      
%     \label{fig:C3-3}    
% \end{figure*}
% \textbf{(\romannumeral4) Separability Maps:} To assess the effectiveness of the comparison methods in distinguishing targets from the background, target-background separability is analyzed using boxplots, providing a clear visual representation of the separation. Figs.~\ref{fig:C3-1} to \ref{fig:C3-3} present these separability boxplots for all methods across the ATR2-HUTD sub-datasets.
% \par
% Traditional HTD methods, CEM and SAM, show limited separability, with SAM slightly outperforming CEM. In all sub-datasets, target boxes overlap with background boxes, despite some background suppression, indicating poor separation of targets from the background in underwater environments.
% Advanced HTD methods, IEEPST and MCLT, show minor improvement over traditional methods. 
% However, except for the ATR2-HUTD Sea sub-dataset (scene 1), target boxes still overlap with background boxes in most sub-datasets. 
% This suggests that even with advanced techniques, suppressing background noise and achieving clear target separation remains challenging in complex underwater environments.
% \par
% HUTD methods show improved separability. Specifically, UTD-Net achieves significant background suppression, though some overlap remains. 
% Additionally, UTD-Net exhibits a detection range near 0 in certain sub-datasets, indicating a high false positive rate. 
% NUN-UTD, an enhanced version of UTD-Net, improves target highlighting but still struggles with background noise suppression, leading to suboptimal performance in more complex scenes such as those in the ATR2-HUTD River and Sea sub-datasets.
% Compared to unmixing-based HUTD methods, TUTDF and TDSS-UTD demonstrate better separability, with detection ranges closer to 1, indicating more effective reduction of target-background correlation. 
% However, both methods still exhibit considerable target-background overlap and limited suppression in complex scenes, such as the ATR2-HUTD River sub-dataset.
% \par
% In contrast, the proposed HUCLNet-based methods, HUCLNet+CEM and HUCLNet+SAM, exhibit superior separability, with target boxes generally fully separated from the background. 
% These methods effectively suppress background noise, enabling reliable target detection in underwater environments. 
% The detection range for HUCLNet-based methods is close to 1, while the background range is near 0, indicating a low false positive rate. Compared to CEM and SAM, HUCLNet significantly enhances target-background separability, demonstrating its effectiveness in underwater hyperspectral target detection.
\subsection{Ablation Studies}\label{sec:4.4}
To evaluate the efficacy of each component in our method, we conducted ablation studies on the ATR2-HUTD dataset. These studies aim to confirm that the observed improvements stem not only from the increased number of parameters but also from the architectural design, which enhances the HUTD task. The HUCLNet framework is divided into three components for experimental validation. 
Corresponding results are presented in Tab.~\ref{ablation study}.
\par
\textbf{(\romannumeral1) Analysis of the RGC Module:} We validate the RGC with the following designs: 
\begin{itemize}
    \item \textbf{w/o Cluster Refinement Strategy:} This design excludes the cluster refinement strategy, relying solely on the reference spectrum-based clustering method. 
    \item \textbf{w/o Reference Spectrum-based Clustering:} This design omits the reference spectrum-based clustering approach from the RGC module.
\end{itemize}
\par
Without the cluster refinement strategy, the RGC module directly uses the original clustering results, often misclassifying pixels and negatively impacting prototype-level learning. As seen in Tab.~\ref{ablation study}, this leads to lower average metric values compared to the full HUCLNet-based methods, demonstrating the importance of refined pseudo-labels. Removing the RGC module entirely, the HLCL module uses pixel instances from the original HSI for instance-level contrastive learning, focusing on individual pixel spectra and neglecting the target-background relationships. Performance improves slightly over baseline HTD methods but remains significantly inferior to complete HUCLNet-based methods, highlighting the critical role of the RGC module in providing reliable pseudo-labels.
\par
\begin{figure*}[!t]                 
    \centering                    
    \includegraphics[width=2\columnwidth]{images/C4.jpg}                     
    \caption{Parameter analysis results on the ATR2-HUTD dataset. (a) Number of clusters in the RCG module; (b) Batch size in the HLCL module; (c) Attack method in the HLCL module. Red and yellow points indicate the maximum and minimum values, respectively.}                 
    \label{fig:C4-1}    
\end{figure*}
\textbf{(\romannumeral2) Analysis of the HLCL Module:} We evaluate the HLCL module with the following designs:
\begin{itemize}
    \item \textbf{w/o Instance-level Contrastive Learning:} This design removes instance-level contrastive learning, relying only on refined cluster labels from the RGC module.
    \item \textbf{w/o Prototype-level Contrastive Learning:} This design removes prototype-level contrastive learning, retaining only instance-level contrastive learning.
    \item \textbf{w/o Hyperspectral-Oriented Data Augmentation:} This design removes hyperspectral-specific data augmentation from the HLCL module.
    \item \textbf{w/o HLCL Module:} This design excludes the entire HLCL module.
\end{itemize}
\par
According to Tab.~\ref{ablation study}, we can draw the following conclusions.
When the HLCL module operates without instance-level contrastive learning, HUCLNet relies solely on the cluster labels, leading to performance degradation. However, prototype-level contrastive learning alone still outperforms baseline HTD methods, emphasizing the importance of target-background separability. The removal of prototype-level contrastive learning results in poorer performance compared to the instance-level design, indicating its greater impact on separability. When hyperspectral-oriented data augmentation is excluded, traditional augmentation methods lead to observable performance degradation, confirming the importance of hyperspectral-specific augmentation in enhancing feature discriminability and HUCLNet's performance. Finally, removing the HLCL module entirely reduces HUCLNet to baseline HTD methods, resulting in substantial performance loss, reinforcing the HLCL module's primary contribution to performance improvement.
\par
\textbf{(\romannumeral3) Analysis of the SPL Paradigm:} We evaluate the SPL paradigm with the following design: 
\begin{itemize}
    \item \textbf{w/o SPL Paradigm:} This design trains the model using the traditional self-supervised learning framework, which consists of a single reliable-guided clustering step followed by hybrid-level contrastive learning.
\end{itemize}
\par
Without the SPL paradigm, inaccurate clustering due to limited spectral discriminability hinders contrastive learning effectiveness, resulting in error propagation and performance degradation. Tab.~\ref{ablation study} confirms that the SPL paradigm significantly enhances HUCLNet's performance, underscoring the importance of the self-paced strategy in guiding model training and improving detection accuracy.
\par
\subsection{Parameter Analysis}\label{sec:4.5}
The key hyperparameters of the HUCLNet architecture, including the number of clusters in the RGC module, batch size, and attack method in the HLCL module, were analyzed through experiments on the ATR2-HUTD dataset. The results, primarily focusing on the $\text{AUC}_{\text{OA}}$ metric, are presented in Fig.~\ref{fig:C4-1}, as it is the most critical indicator of overall detection performance.
\par
\textbf{(\romannumeral1) Number of Clusters in the RGC Module:} The number of clusters in the RGC module plays a crucial role in clustering accuracy and overall HUCLNet performance. The number of clusters was varied between 30 and 48, with a step size of 3 (Fig.~\ref{fig:C4-1} (a)). Performance improves with an increasing number of clusters up to an optimal point, after which it deteriorates due to over-segmentation, where target pixels are fragmented into multiple clusters. This fragmentation hinders prototype-level contrastive learning, leading to inconsistent target representations. For the ATR2-HUTD Lake, River, and Sea sub-datasets, the optimal number of clusters was 36, 39, and 42, respectively. Even with suboptimal cluster numbers, HUCLNet outperforms baseline methods.

\textbf{(\romannumeral2) Batch Size in the HLCL Module:} The batch size in the HLCL module is another critical parameter affecting HUCLNet performance. Varying the batch size from 32 to 512 with a step size of 64, results (Fig.~\ref{fig:C4-1} (b)) show that larger batch sizes generally improve performance by increasing the number of negative samples, enhancing feature discriminability. This is consistent with prior work~\cite{Chen2020}, which indicates that larger batch sizes benefit contrastive learning. However, performance gains plateau at higher batch sizes, and larger sizes impose greater memory and computational demands. A batch size of 256 provides an optimal balance between performance and resource usage across all ATR2-HUTD sub-datasets.

\textbf{(\romannumeral3) Attack Method in the HLCL Module:} The choice of attack method in the HLCL module influences the generation of adversarial samples for contrastive learning. Four attack methods—FGSM~\cite{GoodfellowSS14}, PGD~\cite{MadryMSTV18}, FAB~\cite{Croce020}, and SPSA~\cite{SPSA}—were tested with a perturbation limit of $\epsilon = 0.1$. As shown in Fig.~\ref{fig:C4-1} (c), performance across attack methods is similar, suggesting that the specific choice of attack method has minimal impact, as long as the generated adversarial samples are effective. Given its computational efficiency and comparable performance, we adopt the FGSM attack method for HUCLNet.

% \subsection{Visualization of the effect of HUCLNet}\label{sec:4.5}
\paragraph{Summary}
Our findings provide significant insights into the influence of correctness, explanations, and refinement on evaluation accuracy and user trust in AI-based planners. 
In particular, the findings are three-fold: 
(1) The \textbf{correctness} of the generated plans is the most significant factor that impacts the evaluation accuracy and user trust in the planners. As the PDDL solver is more capable of generating correct plans, it achieves the highest evaluation accuracy and trust. 
(2) The \textbf{explanation} component of the LLM planner improves evaluation accuracy, as LLM+Expl achieves higher accuracy than LLM alone. Despite this improvement, LLM+Expl minimally impacts user trust. However, alternative explanation methods may influence user trust differently from the manually generated explanations used in our approach.
% On the other hand, explanations may help refine the trust of the planner to a more appropriate level by indicating planner shortcomings.
(3) The \textbf{refinement} procedure in the LLM planner does not lead to a significant improvement in evaluation accuracy; however, it exhibits a positive influence on user trust that may indicate an overtrust in some situations.
% This finding is aligned with prior works showing that iterative refinements based on user feedback would increase user trust~\cite{kunkel2019let, sebo2019don}.
Finally, the propensity-to-trust analysis identifies correctness as the primary determinant of user trust, whereas explanations provided limited improvement in scenarios where the planner's accuracy is diminished.

% In conclusion, our results indicate that the planner's correctness is the dominant factor for both evaluation accuracy and user trust. Therefore, selecting high-quality training data and optimizing the training procedure of AI-based planners to improve planning correctness is the top priority. Once the AI planner achieves a similar correctness level to traditional graph-search planners, strengthening its capability to explain and refine plans will further improve user trust compared to traditional planners.

\paragraph{Future Research} Future steps in this research include expanding user studies with larger sample sizes to improve generalizability and including additional planning problems per session for a more comprehensive evaluation. Next, we will explore alternative methods for generating plan explanations beyond manual creation to identify approaches that more effectively enhance user trust. 
Additionally, we will examine user trust by employing multiple LLM-based planners with varying levels of planning accuracy to better understand the interplay between planning correctness and user trust. 
Furthermore, we aim to enable real-time user-planner interaction, allowing users to provide feedback and refine plans collaboratively, thereby fostering a more dynamic and user-centric planning process.


% \begin{abstract}
% This document describes the most common article elements and how to use the IEEEtran class with \LaTeX \ to produce files that are suitable for submission to the IEEE.  IEEEtran can produce conference, journal, and technical note (correspondence) papers with a suitable choice of class options. 
% \end{abstract}

% \begin{IEEEkeywords}
% Article submission, IEEE, IEEEtran, journal, \LaTeX, paper, template, typesetting.
% \end{IEEEkeywords}

% \section{Introduction}
% \IEEEPARstart{T}{his} file is intended to serve as a ``sample article file''
% for IEEE journal papers produced under \LaTeX\ using
% IEEEtran.cls version 1.8b and later. The most common elements are covered in the simplified and updated instructions in ``New\_IEEEtran\_how-to.pdf''. For less common elements you can refer back to the original ``IEEEtran\_HOWTO.pdf''. It is assumed that the reader has a basic working knowledge of \LaTeX. Those who are new to \LaTeX \ are encouraged to read Tobias Oetiker's ``The Not So Short Introduction to \LaTeX ,'' available at: \url{http://tug.ctan.org/info/lshort/english/lshort.pdf} which provides an overview of working with \LaTeX.

% \section{The Design, Intent, and \\ Limitations of the Templates}
% The templates are intended to {\bf{approximate the final look and page length of the articles/papers}}. {\bf{They are NOT intended to be the final produced work that is displayed in print or on IEEEXplore\textsuperscript{\textregistered}}}. They will help to give the authors an approximation of the number of pages that will be in the final version. The structure of the \LaTeX\ files, as designed, enable easy conversion to XML for the composition systems used by the IEEE. The XML files are used to produce the final print/IEEEXplore pdf and then converted to HTML for IEEEXplore.

% \section{Where to Get \LaTeX \ Help --- User Groups}
% The following online groups are helpful to beginning and experienced \LaTeX\ users. A search through their archives can provide many answers to common questions.
% \begin{list}{}{}
% \item{\url{http://www.latex-community.org/}} 
% \item{\url{https://tex.stackexchange.com/} }
% \end{list}

% \section{Other Resources}
% See \cite{ref1,ref2,ref3,ref4,ref5} for resources on formatting math into text and additional help in working with \LaTeX .

% \section{Text}
% For some of the remainer of this sample we will use dummy text to fill out paragraphs rather than use live text that may violate a copyright.

% Itam, que ipiti sum dem velit la sum et dionet quatibus apitet voloritet audam, qui aliciant voloreicid quaspe volorem ut maximusandit faccum conemporerum aut ellatur, nobis arcimus.
% Fugit odi ut pliquia incitium latum que cusapere perit molupta eaquaeria quod ut optatem poreiur? Quiaerr ovitior suntiant litio bearciur?

% Onseque sequaes rectur autate minullore nusae nestiberum, sum voluptatio. Et ratem sequiam quaspername nos rem repudandae volum consequis nos eium aut as molupta tectum ulparumquam ut maximillesti consequas quas inctia cum volectinusa porrum unt eius cusaest exeritatur? Nias es enist fugit pa vollum reium essusam nist et pa aceaqui quo elibusdandis deligendus que nullaci lloreri bla que sa coreriam explacc atiumquos simolorpore, non prehendunt lam que occum\cite{ref6} si aut aut maximus eliaeruntia dia sequiamenime natem sendae ipidemp orehend uciisi omnienetus most verum, ommolendi omnimus, est, veni aut ipsa volendelist mo conserum volores estisciis recessi nveles ut poressitatur sitiis ex endi diti volum dolupta aut aut odi as eatquo cullabo remquis toreptum et des accus dolende pores sequas dolores tinust quas expel moditae ne sum quiatis nis endipie nihilis etum fugiae audi dia quiasit quibus.
% \IEEEpubidadjcol
% Ibus el et quatemo luptatque doluptaest et pe volent rem ipidusa eribus utem venimolorae dera qui acea quam etur aceruptat.
% Gias anis doluptaspic tem et aliquis alique inctiuntiur?

% Sedigent, si aligend elibuscid ut et ium volo tem eictore pellore ritatus ut ut ullatus in con con pere nos ab ium di tem aliqui od magnit repta volectur suntio. Nam isquiante doluptis essit, ut eos suntionsecto debitiur sum ea ipitiis adipit, oditiore, a dolorerempos aut harum ius, atquat.

% Rum rem ditinti sciendunti volupiciendi sequiae nonsect oreniatur, volores sition ressimil inus solut ea volum harumqui to see\eqref{deqn_ex1a} mint aut quat eos explis ad quodi debis deliqui aspel earcius.

% \begin{equation}
% \label{deqn_ex1a}
% x = \sum_{i=0}^{n} 2{i} Q.
% \end{equation}

% Alis nime volorempera perferi sitio denim repudae pre ducilit atatet volecte ssimillorae dolore, ut pel ipsa nonsequiam in re nus maiost et que dolor sunt eturita tibusanis eatent a aut et dio blaudit reptibu scipitem liquia consequodi od unto ipsae. Et enitia vel et experferum quiat harum sa net faccae dolut voloria nem. Bus ut labo. Ita eum repraer rovitia samendit aut et volupta tecupti busant omni quiae porro que nossimodic temquis anto blacita conse nis am, que ereperum eumquam quaescil imenisci quae magnimos recus ilibeaque cum etum iliate prae parumquatemo blaceaquiam quundia dit apienditem rerit re eici quaes eos sinvers pelecabo. Namendignis as exerupit aut magnim ium illabor roratecte plic tem res apiscipsam et vernat untur a deliquaest que non cus eat ea dolupiducim fugiam volum hil ius dolo eaquis sitis aut landesto quo corerest et auditaquas ditae voloribus, qui optaspis exero cusa am, ut plibus.


% \section{Some Common Elements}
% \subsection{Sections and Subsections}
% Enumeration of section headings is desirable, but not required. When numbered, please be consistent throughout the article, that is, all headings and all levels of section headings in the article should be enumerated. Primary headings are designated with Roman numerals, secondary with capital letters, tertiary with Arabic numbers; and quaternary with lowercase letters. Reference and Acknowledgment headings are unlike all other section headings in text. They are never enumerated. They are simply primary headings without labels, regardless of whether the other headings in the article are enumerated. 

% \subsection{Citations to the Bibliography}
% The coding for the citations is made with the \LaTeX\ $\backslash${\tt{cite}} command. 
% This will display as: see \cite{ref1}.

% For multiple citations code as follows: {\tt{$\backslash$cite\{ref1,ref2,ref3\}}}
%  which will produce \cite{ref1,ref2,ref3}. For reference ranges that are not consecutive code as {\tt{$\backslash$cite\{ref1,ref2,ref3,ref9\}}} which will produce  \cite{ref1,ref2,ref3,ref9}

% \subsection{Lists}
% In this section, we will consider three types of lists: simple unnumbered, numbered, and bulleted. There have been many options added to IEEEtran to enhance the creation of lists. If your lists are more complex than those shown below, please refer to the original ``IEEEtran\_HOWTO.pdf'' for additional options.\\

% \subsubsection*{\bf A plain  unnumbered list}
% \begin{list}{}{}
% \item{bare\_jrnl.tex}
% \item{bare\_conf.tex}
% \item{bare\_jrnl\_compsoc.tex}
% \item{bare\_conf\_compsoc.tex}
% \item{bare\_jrnl\_comsoc.tex}
% \end{list}

% \subsubsection*{\bf A simple numbered list}
% \begin{enumerate}
% \item{bare\_jrnl.tex}
% \item{bare\_conf.tex}
% \item{bare\_jrnl\_compsoc.tex}
% \item{bare\_conf\_compsoc.tex}
% \item{bare\_jrnl\_comsoc.tex}
% \end{enumerate}

% \subsubsection*{\bf A simple bulleted list}
% \begin{itemize}
% \item{bare\_jrnl.tex}
% \item{bare\_conf.tex}
% \item{bare\_jrnl\_compsoc.tex}
% \item{bare\_conf\_compsoc.tex}
% \item{bare\_jrnl\_comsoc.tex}
% \end{itemize}





% \subsection{Figures}
% Fig. 1 is an example of a floating figure using the graphicx package.
%  Note that $\backslash${\tt{label}} must occur AFTER (or within) $\backslash${\tt{caption}}.
%  For figures, $\backslash${\tt{caption}} should occur after the $\backslash${\tt{includegraphics}}.

% \begin{figure}[!t]
% \centering
% \includegraphics[width=2.5in]{fig1}
% \caption{Simulation results for the network.}
% \label{fig_1}
% \end{figure}

% Fig. 2(a) and 2(b) is an example of a double column floating figure using two subfigures.
%  (The subfig.sty package must be loaded for this to work.)
%  The subfigure $\backslash${\tt{label}} commands are set within each subfloat command,
%  and the $\backslash${\tt{label}} for the overall figure must come after $\backslash${\tt{caption}}.
%  $\backslash${\tt{hfil}} is used as a separator to get equal spacing.
%  The combined width of all the parts of the figure should do not exceed the text width or a line break will occur.
% %
% \begin{figure*}[!t]
% \centering
% \subfloat[]{\includegraphics[width=2.5in]{fig1}%
% \label{fig_first_case}}
% \hfil
% \subfloat[]{\includegraphics[width=2.5in]{fig1}%
% \label{fig_second_case}}
% \caption{Dae. Ad quatur autat ut porepel itemoles dolor autem fuga. Bus quia con nessunti as remo di quatus non perum que nimus. (a) Case I. (b) Case II.}
% \label{fig_sim}
% \end{figure*}

% Note that often IEEE papers with multi-part figures do not place the labels within the image itself (using the optional argument to $\backslash${\tt{subfloat}}[]), but instead will
%  reference/describe all of them (a), (b), etc., within the main caption.
%  Be aware that for subfig.sty to generate the (a), (b), etc., subfigure
%  labels, the optional argument to $\backslash${\tt{subfloat}} must be present. If a
%  subcaption is not desired, leave its contents blank,
%  e.g.,$\backslash${\tt{subfloat}}[].


 

% \section{Tables}
% Note that, for IEEE-style tables, the
%  $\backslash${\tt{caption}} command should come BEFORE the table. Table captions use title case. Articles (a, an, the), coordinating conjunctions (and, but, for, or, nor), and most short prepositions are lowercase unless they are the first or last word. Table text will default to $\backslash${\tt{footnotesize}} as
%  the IEEE normally uses this smaller font for tables.
%  The $\backslash${\tt{label}} must come after $\backslash${\tt{caption}} as always.
 
% \begin{table}[!t]
% \caption{An Example of a Table\label{tab:table1}}
% \centering
% \begin{tabular}{|c||c|}
% \hline
% One & Two\\
% \hline
% Three & Four\\
% \hline
% \end{tabular}
% \end{table}

% \section{Algorithms}
% Algorithms should be numbered and include a short title. They are set off from the text with rules above and below the title and after the last line.

% \begin{algorithm}[H]
% \caption{Weighted Tanimoto ELM.}\label{alg:alg1}
% \begin{algorithmic}
% \STATE 
% \STATE {\textsc{TRAIN}}$(\mathbf{X} \mathbf{T})$
% \STATE \hspace{0.5cm}$ \textbf{select randomly } W \subset \mathbf{X}  $
% \STATE \hspace{0.5cm}$ N_\mathbf{t} \gets | \{ i : \mathbf{t}_i = \mathbf{t} \} | $ \textbf{ for } $ \mathbf{t}= -1,+1 $
% \STATE \hspace{0.5cm}$ B_i \gets \sqrt{ \textsc{max}(N_{-1},N_{+1}) / N_{\mathbf{t}_i} } $ \textbf{ for } $ i = 1,...,N $
% \STATE \hspace{0.5cm}$ \hat{\mathbf{H}} \gets  B \cdot (\mathbf{X}^T\textbf{W})/( \mathbb{1}\mathbf{X} + \mathbb{1}\textbf{W} - \mathbf{X}^T\textbf{W} ) $
% \STATE \hspace{0.5cm}$ \beta \gets \left ( I/C + \hat{\mathbf{H}}^T\hat{\mathbf{H}} \right )^{-1}(\hat{\mathbf{H}}^T B\cdot \mathbf{T})  $
% \STATE \hspace{0.5cm}\textbf{return}  $\textbf{W},  \beta $
% \STATE 
% \STATE {\textsc{PREDICT}}$(\mathbf{X} )$
% \STATE \hspace{0.5cm}$ \mathbf{H} \gets  (\mathbf{X}^T\textbf{W} )/( \mathbb{1}\mathbf{X}  + \mathbb{1}\textbf{W}- \mathbf{X}^T\textbf{W}  ) $
% \STATE \hspace{0.5cm}\textbf{return}  $\textsc{sign}( \mathbf{H} \beta )$
% \end{algorithmic}
% \label{alg1}
% \end{algorithm}

% Que sunt eum lam eos si dic to estist, culluptium quid qui nestrum nobis reiumquiatur minimus minctem. Ro moluptat fuga. Itatquiam ut laborpo rersped exceres vollandi repudaerem. Ulparci sunt, qui doluptaquis sumquia ndestiu sapient iorepella sunti veribus. Ro moluptat fuga. Itatquiam ut laborpo rersped exceres vollandi repudaerem. 
% \section{Mathematical Typography \\ and Why It Matters}

% Typographical conventions for mathematical formulas have been developed to {\bf provide uniformity and clarity of presentation across mathematical texts}. This enables the readers of those texts to both understand the author's ideas and to grasp new concepts quickly. While software such as \LaTeX \ and MathType\textsuperscript{\textregistered} can produce aesthetically pleasing math when used properly, it is also very easy to misuse the software, potentially resulting in incorrect math display.

% IEEE aims to provide authors with the proper guidance on mathematical typesetting style and assist them in writing the best possible article. As such, IEEE has assembled a set of examples of good and bad mathematical typesetting \cite{ref1,ref2,ref3,ref4,ref5}. 

% Further examples can be found at \url{http://journals.ieeeauthorcenter.ieee.org/wp-content/uploads/sites/7/IEEE-Math-Typesetting-Guide-for-LaTeX-Users.pdf}

% \subsection{Display Equations}
% The simple display equation example shown below uses the ``equation'' environment. To number the equations, use the $\backslash${\tt{label}} macro to create an identifier for the equation. LaTeX will automatically number the equation for you.
% \begin{equation}
% \label{deqn_ex1}
% x = \sum_{i=0}^{n} 2{i} Q.
% \end{equation}

% \noindent is coded as follows:
% \begin{verbatim}
% \begin{equation}
% \label{deqn_ex1}
% x = \sum_{i=0}^{n} 2{i} Q.
% \end{equation}
% \end{verbatim}

% To reference this equation in the text use the $\backslash${\tt{ref}} macro. 
% Please see (\ref{deqn_ex1})\\
% \noindent is coded as follows:
% \begin{verbatim}
% Please see (\ref{deqn_ex1})\end{verbatim}

% \subsection{Equation Numbering}
% {\bf{Consecutive Numbering:}} Equations within an article are numbered consecutively from the beginning of the
% article to the end, i.e., (1), (2), (3), (4), (5), etc. Do not use roman numerals or section numbers for equation numbering.

% \noindent {\bf{Appendix Equations:}} The continuation of consecutively numbered equations is best in the Appendix, but numbering
%  as (A1), (A2), etc., is permissible.\\

% \noindent {\bf{Hyphens and Periods}}: Hyphens and periods should not be used in equation numbers, i.e., use (1a) rather than
% (1-a) and (2a) rather than (2.a) for subequations. This should be consistent throughout the article.

% \subsection{Multi-Line Equations and Alignment}
% Here we show several examples of multi-line equations and proper alignments.

% \noindent {\bf{A single equation that must break over multiple lines due to length with no specific alignment.}}
% \begin{multline}
% \text{The first line of this example}\\
% \text{The second line of this example}\\
% \text{The third line of this example}
% \end{multline}

% \noindent is coded as:
% \begin{verbatim}
% \begin{multline}
% \text{The first line of this example}\\
% \text{The second line of this example}\\
% \text{The third line of this example}
% \end{multline}
% \end{verbatim}

% \noindent {\bf{A single equation with multiple lines aligned at the = signs}}
% \begin{align}
% a &= c+d \\
% b &= e+f
% \end{align}
% \noindent is coded as:
% \begin{verbatim}
% \begin{align}
% a &= c+d \\
% b &= e+f
% \end{align}
% \end{verbatim}

% The {\tt{align}} environment can align on multiple  points as shown in the following example:
% \begin{align}
% x &= y & X & =Y & a &=bc\\
% x' &= y' & X' &=Y' &a' &=bz
% \end{align}
% \noindent is coded as:
% \begin{verbatim}
% \begin{align}
% x &= y & X & =Y & a &=bc\\
% x' &= y' & X' &=Y' &a' &=bz
% \end{align}
% \end{verbatim}





% \subsection{Subnumbering}
% The amsmath package provides a {\tt{subequations}} environment to facilitate subnumbering. An example:

% \begin{subequations}\label{eq:2}
% \begin{align}
% f&=g \label{eq:2A}\\
% f' &=g' \label{eq:2B}\\
% \mathcal{L}f &= \mathcal{L}g \label{eq:2c}
% \end{align}
% \end{subequations}

% \noindent is coded as:
% \begin{verbatim}
% \begin{subequations}\label{eq:2}
% \begin{align}
% f&=g \label{eq:2A}\\
% f' &=g' \label{eq:2B}\\
% \mathcal{L}f &= \mathcal{L}g \label{eq:2c}
% \end{align}
% \end{subequations}

% \end{verbatim}

% \subsection{Matrices}
% There are several useful matrix environments that can save you some keystrokes. See the example coding below and the output.

% \noindent {\bf{A simple matrix:}}
% \begin{equation}
% \begin{matrix}  0 &  1 \\ 
% 1 &  0 \end{matrix}
% \end{equation}
% is coded as:
% \begin{verbatim}
% \begin{equation}
% \begin{matrix}  0 &  1 \\ 
% 1 &  0 \end{matrix}
% \end{equation}
% \end{verbatim}

% \noindent {\bf{A matrix with parenthesis}}
% \begin{equation}
% \begin{pmatrix} 0 & -i \\
%  i &  0 \end{pmatrix}
% \end{equation}
% is coded as:
% \begin{verbatim}
% \begin{equation}
% \begin{pmatrix} 0 & -i \\
%  i &  0 \end{pmatrix}
% \end{equation}
% \end{verbatim}

% \noindent {\bf{A matrix with square brackets}}
% \begin{equation}
% \begin{bmatrix} 0 & -1 \\ 
% 1 &  0 \end{bmatrix}
% \end{equation}
% is coded as:
% \begin{verbatim}
% \begin{equation}
% \begin{bmatrix} 0 & -1 \\ 
% 1 &  0 \end{bmatrix}
% \end{equation}
% \end{verbatim}

% \noindent {\bf{A matrix with curly braces}}
% \begin{equation}
% \begin{Bmatrix} 1 &  0 \\ 
% 0 & -1 \end{Bmatrix}
% \end{equation}
% is coded as:
% \begin{verbatim}
% \begin{equation}
% \begin{Bmatrix} 1 &  0 \\ 
% 0 & -1 \end{Bmatrix}
% \end{equation}\end{verbatim}

% \noindent {\bf{A matrix with single verticals}}
% \begin{equation}
% \begin{vmatrix} a &  b \\ 
% c &  d \end{vmatrix}
% \end{equation}
% is coded as:
% \begin{verbatim}
% \begin{equation}
% \begin{vmatrix} a &  b \\ 
% c &  d \end{vmatrix}
% \end{equation}\end{verbatim}

% \noindent {\bf{A matrix with double verticals}}
% \begin{equation}
% \begin{Vmatrix} i &  0 \\ 
% 0 & -i \end{Vmatrix}
% \end{equation}
% is coded as:
% \begin{verbatim}
% \begin{equation}
% \begin{Vmatrix} i &  0 \\ 
% 0 & -i \end{Vmatrix}
% \end{equation}\end{verbatim}

% \subsection{Arrays}
% The {\tt{array}} environment allows you some options for matrix-like equations. You will have to manually key the fences, but there are other options for alignment of the columns and for setting horizontal and vertical rules. The argument to {\tt{array}} controls alignment and placement of vertical rules.

% A simple array
% \begin{equation}
% \left(
% \begin{array}{cccc}
% a+b+c & uv & x-y & 27\\
% a+b & u+v & z & 134
% \end{array}\right)
% \end{equation}
% is coded as:
% \begin{verbatim}
% \begin{equation}
% \left(
% \begin{array}{cccc}
% a+b+c & uv & x-y & 27\\
% a+b & u+v & z & 134
% \end{array} \right)
% \end{equation}
% \end{verbatim}

% A slight variation on this to better align the numbers in the last column
% \begin{equation}
% \left(
% \begin{array}{cccr}
% a+b+c & uv & x-y & 27\\
% a+b & u+v & z & 134
% \end{array}\right)
% \end{equation}
% is coded as:
% \begin{verbatim}
% \begin{equation}
% \left(
% \begin{array}{cccr}
% a+b+c & uv & x-y & 27\\
% a+b & u+v & z & 134
% \end{array} \right)
% \end{equation}
% \end{verbatim}

% An array with vertical and horizontal rules
% \begin{equation}
% \left( \begin{array}{c|c|c|r}
% a+b+c & uv & x-y & 27\\ \hline
% a+b & u+v & z & 134
% \end{array}\right)
% \end{equation}
% is coded as:
% \begin{verbatim}
% \begin{equation}
% \left(
% \begin{array}{c|c|c|r}
% a+b+c & uv & x-y & 27\\
% a+b & u+v & z & 134
% \end{array} \right)
% \end{equation}
% \end{verbatim}
% Note the argument now has the pipe "$\vert$" included to indicate the placement of the vertical rules.


% \subsection{Cases Structures}
% Many times cases can be miscoded using the wrong environment, i.e., {\tt{array}}. Using the {\tt{cases}} environment will save keystrokes (from not having to type the $\backslash${\tt{left}}$\backslash${\tt{lbrace}}) and automatically provide the correct column alignment.
% \begin{equation*}
% {z_m(t)} = \begin{cases}
% 1,&{\text{if}}\ {\beta }_m(t) \\ 
% {0,}&{\text{otherwise.}} 
% \end{cases}
% \end{equation*}
% \noindent is coded as follows:
% \begin{verbatim}
% \begin{equation*}
% {z_m(t)} = 
% \begin{cases}
% 1,&{\text{if}}\ {\beta }_m(t),\\ 
% {0,}&{\text{otherwise.}} 
% \end{cases}
% \end{equation*}
% \end{verbatim}
% \noindent Note that the ``\&'' is used to mark the tabular alignment. This is important to get  proper column alignment. Do not use $\backslash${\tt{quad}} or other fixed spaces to try and align the columns. Also, note the use of the $\backslash${\tt{text}} macro for text elements such as ``if'' and ``otherwise.''

% \subsection{Function Formatting in Equations}
% Often, there is an easy way to properly format most common functions. Use of the $\backslash$ in front of the function name will in most cases, provide the correct formatting. When this does not work, the following example provides a solution using the $\backslash${\tt{text}} macro:

% \begin{equation*} 
%   d_{R}^{KM} = \underset {d_{l}^{KM}} {\text{arg min}} \{ d_{1}^{KM},\ldots,d_{6}^{KM}\}.
% \end{equation*}

% \noindent is coded as follows:
% \begin{verbatim}
% \begin{equation*} 
%  d_{R}^{KM} = \underset {d_{l}^{KM}} 
%  {\text{arg min}} \{ d_{1}^{KM},
%  \ldots,d_{6}^{KM}\}.
% \end{equation*}
% \end{verbatim}

% \subsection{ Text Acronyms Inside Equations}
% This example shows where the acronym ``MSE" is coded using $\backslash${\tt{text\{\}}} to match how it appears in the text.

% \begin{equation*}
%  \text{MSE} = \frac {1}{n}\sum _{i=1}^{n}(Y_{i} - \hat {Y_{i}})^{2}
% \end{equation*}

% \begin{verbatim}
% \begin{equation*}
%  \text{MSE} = \frac {1}{n}\sum _{i=1}^{n}
% (Y_{i} - \hat {Y_{i}})^{2}
% \end{equation*}
% \end{verbatim}

% \section{Conclusion}
% The conclusion goes here.


% \section*{Acknowledgments}
% This should be a simple paragraph before the References to thank those individuals and institutions who have supported your work on this article.



% {\appendix[Proof of the Zonklar Equations]
% Use $\backslash${\tt{appendix}} if you have a single appendix:
% Do not use $\backslash${\tt{section}} anymore after $\backslash${\tt{appendix}}, only $\backslash${\tt{section*}}.
% If you have multiple appendixes use $\backslash${\tt{appendices}} then use $\backslash${\tt{section}} to start each appendix.
% You must declare a $\backslash${\tt{section}} before using any $\backslash${\tt{subsection}} or using $\backslash${\tt{label}} ($\backslash${\tt{appendices}} by itself
%  starts a section numbered zero.)}



% %{\appendices
% %\section*{Proof of the First Zonklar Equation}
% %Appendix one text goes here.
% % You can choose not to have a title for an appendix if you want by leaving the argument blank
% %\section*{Proof of the Second Zonklar Equation}
% %Appendix two text goes here.}

\bibliographystyle{IEEEtran}
\bibliography{IEEEabrv,ref}

% \section{References Section}
% You can use a bibliography generated by BibTeX as a .bbl file.
%  BibTeX documentation can be easily obtained at:
%  http://mirror.ctan.org/biblio/bibtex/contrib/doc/
%  The IEEEtran BibTeX style support page is:
%  http://www.michaelshell.org/tex/ieeetran/bibtex/
 
 % argument is your BibTeX string definitions and bibliography database(s)
%\bibliography{IEEEabrv,../bib/paper}
%
% \section{Simple References}
% You can manually copy in the resultant .bbl file and set second argument of $\backslash${\tt{begin}} to the number of references
%  (used to reserve space for the reference number labels box).

% \begin{thebibliography}{1}
% \bibliographystyle{IEEEtran}

% \bibitem{ref1}
% {\it{Mathematics Into Type}}. American Mathematical Society. [Online]. Available: https://www.ams.org/arc/styleguide/mit-2.pdf

% \bibitem{ref2}
% T. W. Chaundy, P. R. Barrett and C. Batey, {\it{The Printing of Mathematics}}. London, U.K., Oxford Univ. Press, 1954.

% \bibitem{ref3}
% F. Mittelbach and M. Goossens, {\it{The \LaTeX Companion}}, 2nd ed. Boston, MA, USA: Pearson, 2004.

% \bibitem{ref4}
% G. Gr\"atzer, {\it{More Math Into LaTeX}}, New York, NY, USA: Springer, 2007.

% \bibitem{ref5}M. Letourneau and J. W. Sharp, {\it{AMS-StyleGuide-online.pdf,}} American Mathematical Society, Providence, RI, USA, [Online]. Available: http://www.ams.org/arc/styleguide/index.html

% \bibitem{ref6}
% H. Sira-Ramirez, ``On the sliding mode control of nonlinear systems,'' \textit{Syst. Control Lett.}, vol. 19, pp. 303--312, 1992.

% \bibitem{ref7}
% A. Levant, ``Exact differentiation of signals with unbounded higher derivatives,''  in \textit{Proc. 45th IEEE Conf. Decis.
% Control}, San Diego, CA, USA, 2006, pp. 5585--5590. DOI: 10.1109/CDC.2006.377165.

% \bibitem{ref8}
% M. Fliess, C. Join, and H. Sira-Ramirez, ``Non-linear estimation is easy,'' \textit{Int. J. Model., Ident. Control}, vol. 4, no. 1, pp. 12--27, 2008.

% \bibitem{ref9}
% R. Ortega, A. Astolfi, G. Bastin, and H. Rodriguez, ``Stabilization of food-chain systems using a port-controlled Hamiltonian description,'' in \textit{Proc. Amer. Control Conf.}, Chicago, IL, USA,
% 2000, pp. 2245--2249.

% \end{thebibliography}


% \newpage

% \section{Biography Section}
% If you have an EPS/PDF photo (graphicx package needed), extra braces are
%  needed around the contents of the optional argument to biography to prevent
%  the LaTeX parser from getting confused when it sees the complicated
%  $\backslash${\tt{includegraphics}} command within an optional argument. (You can create
%  your own custom macro containing the $\backslash${\tt{includegraphics}} command to make things
%  simpler here.)
 
% \vspace{11pt}

% \bf{If you include a photo:}\vspace{-33pt}
% \begin{IEEEbiography}[{\includegraphics[width=1in,height=1.25in,clip,keepaspectratio]{fig1}}]{Michael Shell}
% Use $\backslash${\tt{begin\{IEEEbiography\}}} and then for the 1st argument use $\backslash${\tt{includegraphics}} to declare and link the author photo.
% Use the author name as the 3rd argument followed by the biography text.
% \end{IEEEbiography}

% \vspace{11pt}

% \bf{If you will not include a photo:}\vspace{-33pt}
% \begin{IEEEbiographynophoto}{John Doe}
% Use $\backslash${\tt{begin\{IEEEbiographynophoto\}}} and the author name as the argument followed by the biography text.
% \end{IEEEbiographynophoto}




% \vfill

\end{document}


