\section{Introduction}\label{sec:1} 
\IEEEPARstart{U}{nderwater} target detection (UTD)~\cite{Liu2024,Zhang2023,Li2023} aims to locate and identify underwater objects, providing essential data for ecosystem conservation and sustainable resource management to mitigate environmental threats.  
Despite its significance, effective UTD in nearshore regions remains challenging due to the dynamic and complex underwater environment, necessitating rapid, large-scale data acquisition.  
Remote sensing~\cite{8697135, 9174822} offers a promising solution by enabling extensive spatial data collection with high temporal resolution.  
\par
\begin{figure}[!t]                 
    \centering                    
    \includegraphics[width=1\columnwidth]{images/A0.jpg}                     
    \caption{Limitations of RGB imagery and advantages of hyperspectral imagery in underwater depth estimation. (a) In RGB images, the target and background have nearly identical spatial appearances; (b) In hyperspectral images, the target and background exhibit distinct spectral signatures.}                  
    \label{fig:A0}   
\end{figure} 
\begin{figure*}[!t]                 
    \centering                    
    \includegraphics[width=2\columnwidth]{images/A2.jpg}                     
    \caption{The illustration of opportunities and challenges in hyperspectral nearshore underwater target detection. (a) Opportunities; (b) Challenges.}                  
    \label{fig:A1}    
\end{figure*} 
RGB images are the most commonly used data type in remote sensing and play a crucial role in environmental monitoring~\cite{Ma2019}.  
They capture spatial features such as texture, shape, and color, which are essential for general environmental analysis.  
However, RGB imagery has significant limitations in nearshore UTD.  
Studies~\cite{10704737,10336777} show that radiation at 0.45$\upmu$m (blue) and 0.65$\upmu$m (red) is strongly absorbed by chlorophyll, reducing reflectance in these bands and limiting the capture of underwater scene details.  
Consequently, RGB-based spatial features often lack discriminability in nearshore environments.  
As illustrated in Fig.~\ref{fig:A0} (a), this limitation is further exacerbated by the restricted spatial resolution of remote sensing data, which reduces the distinctiveness of underwater targets against the background.  
Similar constraints exist in other spatial imaging modalities, such as infrared imagery, suggesting that spatial features alone are insufficient for nearshore UTD.  
\par  
In contrast, hyperspectral imagery (HSI) captures hundreds to thousands of narrow spectral bands, providing rich spectral information across the visible, near-infrared, and shortwave infrared regions.  
This enables precise target identification based on unique spectral signatures, even in complex optical conditions~\cite{Manolakis2014}.  
Unlike RGB imagery, which relies on spatial features, HSI offers fine-grained spectral features that enhance target-background differentiation.  
Its extensive spectral coverage mitigates water absorption effects, facilitating accurate modeling of underwater scenes.  
As shown in Fig.~\ref{fig:A0} (b), underwater targets and background regions exhibit distinct spectral characteristics despite their similar spatial appearances, making HSI well-suited for nearshore UTD.  
\par  
Recent advances in hyperspectral remote sensing have been driven by satellite, airborne, and UAV-based platforms.  
Among these, UAV-based HSI systems offer significant advantages for nearshore UTD, as illustrated in Fig.~\ref{fig:A1} (a).  
They provide high spatial resolution imagery, often at the centimeter scale, minimizing subpixel interference~\cite{Jiao2022}.  
Additionally, UAV-acquired hyperspectral data are less affected by atmospheric attenuation and environmental noise, ensuring higher image quality than other platforms~\cite{Phang2023, Gu2023}.  
With their flexibility, cost-effectiveness, and real-time data acquisition capabilities, UAVs are particularly suited for monitoring dynamic nearshore environments~\cite{Zhong2020}.  
These attributes position UAV-based HSI as a promising solution for addressing UTD challenges, forming the focus of this study.  
\subsection{UAV-borne Hyperspectral Underwater Target Detection}
Despite the advantages of hyperspectral target detection (HTD), its application in underwater environments is hindered by spectral distortions induced by the water column~\cite{Gillis2020}.  
Light absorption and scattering alter the spectral signatures of underwater targets, causing deviations from their reference spectra\footnote{
    Reference spectra denote the known spectral signatures of underwater targets, obtained either on land or in controlled settings.
}~\cite{Gillis2020}.  
Existing HTD methods assume that target spectra match their reference spectra~\cite{Manolakis2014}, an assumption frequently violated in underwater conditions.  
As shown in Fig.~\ref{fig:A1-2}, spectral distortions vary with depth, turbidity, and water composition, further degrading the accuracy of conventional HTD approaches.  
To address these challenges, prior studies~\cite{Liu2024,Gillis2020,Jay2012,LiZheyong2023,Qi2021} have explored two primary detection strategies.  
\par
\begin{figure}[!t]                  
    \centering                    
    \includegraphics[width=1\columnwidth]{images/A1.jpg}                     
    \caption{Illustration of spectral distortions induced by underwater conditions, with depth as an example. 
    The spectral signature of underwater target diverge from their reference spectra, and this deviation varies with depth.
    The similar situations can be observed for other underwater conditions~\cite{Gillis2020}.}           
    \label{fig:A1-2}    
\end{figure}
The first strategy predicts underwater spectral signatures from known reference spectra using a bathymetric model~\cite{Lee1998}, followed by target detection.  
Jay \emph{et al.}~\cite{Jay2012} propose a method that corrects water column distortions using a bathymetric model and detects targets with a GLRT-based adaptive filter, without requiring prior water parameter knowledge.  
Similarly, Gillis~\cite{Gillis2020} develops a framework that predicts submerged target spectra through radiative transfer modeling and nonlinear dimensionality reduction.  
More recently, Li \emph{et al.}~\cite{LiZheyong2023} introduce a transfer-based hyperspectral underwater target detection (HUTD) framework that synthesizes spectral data at various depths and employs domain adaptation for improved target detection.  

\par  
The second strategy restores reference spectra from observed underwater spectra using hyperspectral unmixing techniques.  
This approach is also based on the bathymetric model, which expresses the observed underwater spectrum as a linear combination of the reference spectrum and the water body spectrum.  
Qi \emph{et al.}~\cite{Qi2021} develop the first unmixing-based HUTD network, reconstructing reference spectra through hyperspectral unmixing.  
Liu \emph{et al.}~\cite{Liu2024} extend this approach with a nonlinear representation, adapting unmixing-based HUTD for nearshore scenarios.  

\par  
Despite promising results, several challenges remain in adapting these methods to nearshore environments:  
\begin{itemize}  
    \item \textbf{Dependency on the bathymetric model.}  
    Existing HUTD methods rely on the bathymetric model to describe underwater imaging mechanisms.  
    However, its assumptions, including linear mixing and uniform water column properties~\cite{Lee1998}, often do not reflect real-world conditions.  
    These limitations, illustrated in Fig.~\ref{fig:A1} (b, Top), undermine detection accuracy and generalizability.  
    Furthermore, effective use of the bathymetric model requires prior knowledge of target depth, suspended particle concentration, and water optical properties~\cite{Gillis2020}, which are challenging to obtain in dynamic nearshore environments.  

    \item \textbf{Limited target characterization.}  
    As shown in Fig.~\ref{fig:A1} (b, Bottom), spectral signatures of the same material vary significantly in nearshore environments due to factors such as water quality, turbidity, and light attenuation.  
    This spectral variability increases uncertainty, degrading the performance of existing HUTD methods.  
    Prediction-based approaches struggle with fluctuating environmental parameters, while restoration-based methods face additional distortions that complicate spectral unmixing.  
    These challenges stem from an overemphasis on spectral restoration or prediction rather than accurate target characterization, which is crucial for nearshore UTD.  
    In dynamic nearshore conditions, where spectral differences between targets and backgrounds are minimal, precise target characterization is essential for improving detection performance.  
\end{itemize}  

\subsection{UAV-borne Hyperspectral Underwater Target Detection Datasets}
Advancements in hyperspectral remote sensing depend on high-quality data.  
The performance of HUTD algorithms is strongly influenced by the diversity and comprehensiveness of training and evaluation datasets.  
UAV-based hyperspectral datasets are essential for assessing detection algorithms under realistic underwater conditions.  
Several HUTD datasets have been introduced, each contributing unique insights.  
\par  
Zhang \emph{et al.}~\cite{Zhang2023} collected two HUTD datasets using a Headwall Nano-Hyperspec sensor mounted on a DJI Matrice 600 Pro UAV at 40 m altitude over Qingdao and Liaocheng, China.  
The HNU-UTD dataset includes Tetrapods, Cement, and Plants as underwater targets.  
Li \emph{et al.}~\cite{LiZheyong2023} introduced the NPU-Pool dataset, acquired with a Gaia Field-V10 imager spanning 400--1000 nm and a spatial resolution of $100 \times 100$, with targets at depths of 0 to 3.1 m under controlled indoor lighting.  
Meanwhile, Li \emph{et al.}~\cite{Li2023} proposed the NPU-Sea dataset, captured in Sanya, Hainan Province, under real seawater conditions, where sea surface waves, water quality, and target movement influenced the data.  
This dataset includes iron plate targets at depths of 0.8 m and 3.0 m.  
More recently, Liu \emph{et al.}~\cite{Liu2024} introduced the ATR2-Lake dataset, collected at Qianlu Lake Reservoir, China, using a Headwall Nano-Hyperspec sensor mounted on a DJI Matrice 300 RTK UAV.  
The dataset includes black metal plates at depths of 1 to 3 m.  
Tab.~\ref{dataset} summarizes the main characteristics of these datasets.
\begin{table*}[!t] 
    \centering
    \renewcommand{\arraystretch}{2}
    \setlength{\tabcolsep}{1.9mm}
    \caption{Key characteristics of existing hyperspectral underwater target detection datasets.} \label{dataset}
    \begin{threeparttable}
    \scalebox{0.9}
    {   
        \begin{tabular}{ccccccccc}
            \toprule
            \multirow{2}{*}{\textbf{Dataset}} & \multicolumn{2}{c}{\cellcolor{tablecolor1}\textbf{Sensor-Related}} & \multicolumn{3}{c}{\cellcolor{tablecolor2}\textbf{Dataset-Related}}                                               & \multicolumn{2}{c}{\cellcolor{tablecolor3}\textbf{Target-Related}}       \\ \cmidrule(lr){2-3} \cmidrule(lr){4-6} \cmidrule(lr){7-8}
                                     & Wavelength  & Spectral Resolution  & Image Size                   & Scenario Type    & Accessible & Target Type               & Target Depth \\ \midrule
            \textbf{HNU-UTD}$^1$                  & 400-1000nm  & 2.2nm    & 560$\times$610, 250$\times$250   & Sea              & Yes        & Tetrapod, Cement, Plants & Unknown      \\
            \textbf{NPU-Pool}$^{2,3}$                 & 400-780 nm  & 3.5nm                & 100$\times$100                      & Anechoic pool    & No         & Iron, Stone, Rubber       & 0-3.1m       \\
            \textbf{NPU-Sea}$^3 $                  & 400-780 nm  & 3.5nm                & 350$\times$350                      & Sea              & No         & Iron                      & 0.8m, 3m     \\
            \textbf{ATR2-Lake}               & 400-1000nm  & 2.2nm               & 242$\times$341, 255$\times$261, 137$\times$178    & Lake             & Yes        & Metal                     & 1m-3m        \\
            \rowcolor{tablecolor5!50}\textbf{ATR2-HUTD}                & 400-1000nm  & 2.2nm               & 2304$\times$640, 3536$\times$640, 3171$\times$640 & Sea, Lake, River & Yes        & Metal, Wooden, Plastic    & 1m-3m       \\ \bottomrule
            \end{tabular}
    }
    \begin{tablenotes}
        \item[1] HNU-UTD dataset collected HSIs via both UAV and Satellite platforms, but we only list the UAV-related data. 
        \item[2] NPU-Pool dataset includes both outdoor and indoor sub-datasets, but the original paper only details the indoor sub-dataset, with no information on the \\ outdoor sub-dataset. As the dataset is not publicly accessible, only the indoor sub-dataset is included in the table.
        \item[3] NPU-Pool and NPU-Sea datasets were collected using tripods rather than UAVs, but are included here for comparison due to the limited availability of \\ HUTD datasets.
        % \item[4] ATR2-HUTD is the extension of the ATR2-Lake dataset, containing more diverse underwater debris targets and scenarios with larger image sizes.
    \end{tablenotes}
    \end{threeparttable}
\end{table*}
\par  
Despite their contributions, existing datasets have critical limitations that hinder the development of robust HUTD algorithms.  
Based on Tab.~\ref{dataset}, the key limitations can be summarized as follows:  
\begin{itemize}  
    \item \textbf{Limited data scale.}  
    Most HUTD datasets are relatively small, typically comprising only hundreds of pixels.  
    This restricted scale fails to capture the complexity of underwater scenes, limiting background variability and hindering generalization to diverse nearshore environments.  
    It also constrains the thorough evaluation of detection algorithms.  
    \item \textbf{Insufficient scene diversity.}  
    Many datasets focus on specific water types, such as seas~\cite{Zhang2023}, lakes~\cite{Liu2024}, or controlled environments~\cite{LiZheyong2023}, failing to represent the full optical variability of real-world underwater conditions.  
    Differences in turbidity, salinity, and light attenuation remain underrepresented, leading to potential overfitting and reduced model adaptability in dynamic aquatic environments.  
\end{itemize}  
\par
\subsection{Contributions of This Study}  
In this paper, we introduces a novel contrastive learning framework, \textbf{H}yperspectral \textbf{U}nderwater \textbf{C}ontrastive \textbf{L}earning \textbf{N}etwork (HUCLNet), which integrates a self-paced learning (SPL) paradigm to address key challenges in HUTD.  
Unlike conventional prediction- or restoration-based approaches, HUCLNet learns a semantically rich latent space, where underwater target spectra are closely aligned with reference spectra while remaining distinct from background spectra in a data-driven manner.  
\par
HUCLNet comprises two core modules: the reliability-guided clustering (RGC) module and the hybrid-level contrastive learning (HLCL) module.  
The RGC module assigns hyperspectral pixels to prototypes via unsupervised clustering, incorporating a fixed prototype derived from the reference spectrum.  
A novel reliability criterion is introduced to assess cluster trustworthiness, refining pixel assignments into reliable clusters and unreliable instances.  
The HLCL module processes unreliable instances via instance-level contrastive learning to enhance discriminative representation and clustering accuracy, while reliable clusters undergo prototype-level contrastive learning to align target spectra with references while maintaining separation from background spectra.  
To further enhance contrastive learning, we propose a hyperspectral-specific data augmentation strategy based on unsupervised adversarial training.  
The entire framework follows the SPL paradigm, progressively incorporating unreliable instances into reliable clusters as the HLCL module improves target characterization, thereby strengthening representation learning and improving HUTD performance.  
Experimental results demonstrate that HUCLNet significantly outperforms state-of-the-art (SOTA) HUTD methods, effectively addressing key methodological gaps in the field.  
\par
Beyond the proposed framework, this paper also introduces \textbf{ATR2-HUTD} dataset, a large-scale UAV-borne HUTD dataset designed to overcome the limitations of existing datasets.  
ATR2-HUTD comprises three sub-datasets—ATR2-HUTD-Lake, ATR2-HUTD-River, and ATR2-HUTD-Sea—collected from LiuYang, Changsha, and Sanya, China, respectively.  
By encompassing diverse lacustrine, riverine, and coastal environments with varying water conditions, ATR2-HUTD mitigates the scene diversity limitations in current datasets, improving model generalization to real-world aquatic settings.  
Additionally, ATR2-HUTD features larger image dimensions, ranging from $2304 \times 640$ to $3536 \times 640$ pixels, providing enhanced spatial details and greater background variability.  
This expanded scale surpasses most existing datasets, alleviating data size constraints and enriching the foundation for model training and evaluation.  
As summarized in Tab.~\ref{dataset}, ATR2-HUTD introduces greater realism and complexity, establishing a more rigorous benchmark for advancing robust and generalizable HUTD methodologies.  
\par

% \par
% In summary, the main contributions of this paper are:
% \begin{enumerate}
%     \item We propose a novel contrastive learning framework, HUCLNet, for precise HUTD in coastal areas. 
%     HUCLNet addresses the HUTD problem by leveraging the intrinsic structure and characteristics of HSI data, rather than relying on the bathymetric models, offering a more robust and generalizable solution.
%     \item We introduce a SPL paradigm to guide the training of HUCLNet, mitigating the noise in unsupervised clustering results and contributing to the convergence of the contrastive learning process.
%     \item We develop the ATR2-HUTD dataset, a large-scale UAV-borne HUTD dataset that covers diverse coastal underwater and nearshore environments, including various water types. The dataset serves as a comprehensive benchmark for evaluating HUTD methods and advancing the development of more robust and generalizable solutions.
%     \item Extensive experiments on the ATR2-HUTD dataset show that HUCLNet significantly outperforms state-of-the-art HUTD methods, effectively addressing key methodological gaps in the field.
% \end{enumerate}
The rest of this paper is structured as follows:  
Section~\ref{sec:3} details the ATR2-HUTD dataset.  
Section~\ref{sec:4} introduces and analyzes the proposed UTD framework.  
Section~\ref{sec:5} presents the experimental results and analysis.  
Section~\ref{sec:6} concludes the paper and discusses future research directions.  
