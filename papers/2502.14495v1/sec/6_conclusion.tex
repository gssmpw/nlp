\section{Conclusion} \label{sec:6}
This paper addresses the challenge of detecting underwater targets in nearshore environments, where severe water attenuation distorts spectral characteristics. We propose a UAV-borne hyperspectral target localization strategy, supported by the ATR2-HUTD benchmark dataset, specifically designed for accurate underwater detection. The dataset includes three UAV-borne hyperspectral sub-datasets, each representing distinct underwater scenarios.
To improve detection, we introduce HUCLNet, a hybrid-level contrastive learning framework that integrates reliability-guided clustering and a self-paced learning paradigm, optimized for UAV-borne hyperspectral imagery.
Extensive experiments on the ATR2-HUTD dataset demonstrate HUCLNet's superior performance across multiple evaluation metrics, including detection accuracy, target preservation, background suppression, signal-to-noise ratio, and overall detection effectiveness, outperforming both traditional and state-of-the-art methods.
Ablation studies and hyperparameter analyses confirm the contributions of each HUCLNet component, providing insights into optimal configurations for maximal performance. Future work will explore HUCLNet's application in more complex underwater environments and assess its generalization across diverse hyperspectral sensors.

% \section{Acknowledgement}
% This work was supported in part by the Foundation Fund of Science and Technology on Near-Surface Detection Laboratory under Grant 6142414220808, and in part by the National Natural Science Foundation of China 62201586, and in part by China National Postdoctoral Program for Innovative Talents under Grant BX20240492.

% \section{Declaration}
% During the preparation of this work the author(s) used ChatGPT in order to improve the academic tone and readability of the manuscript. 
% After using this tool/service, the author(s) reviewed and edited the content as needed and take(s) full responsibility for the content of the publication.