\section{Proposed Dataset}\label{sec:3} 
\subsection{Study region and hardware}\label{sec:2.1}
In this subsection, we introduce the study regions and data acquisition hardware.
\par
\textbf{(1) Study Regions.}  
To investigate the nearshore HUTD problem, three regions with distinct hydrological and environmental characteristics were selected.  
\par
% \begin{figure*}[!t]          
%     \centering                              
%     \includegraphics[width=2\columnwidth]{images/B0.jpg}                               
%     \caption{Study region of the ATR2-HUTD dataset. }                                     
%     \label{fig:B0}     
% \end{figure*} 
% \begin{figure*}[!t]
%     \centering                    
%     \includegraphics[width=2\columnwidth]{images/B1.jpg}                     
%     \caption{Hardware for data acquisition in the ATR2-HUTD dataset. (a) DJI Matrice 300 RTK UAV platform; (b) Headwall Nano-Hyperspec sensor; (c) Data Collection System.}                          
%     \label{fig:B1}   
% \end{figure*}
\begin{table}[!t]
    \centering
    \footnotesize
    \renewcommand\arraystretch{1.25}
    \caption{The parameters of the Headwall Nano-Hyperspec sensor.}
    \setlength{\tabcolsep}{6.15mm} % 设置表的宽度
    {
        \scalebox{0.9}
        { 
    \begin{tabular}{>{\columncolor[HTML]{F0F0F0}}c >{\columncolor[HTML]{FFFFFF}}c}
        \toprule
        % \midrule   
        \rowcolor[HTML]{C0C0C0}
        \textbf{Parameters} & \textbf{Values} \\    
        \midrule    
        Wavelength range & 400-1000 $nm$ \\    
        Spatial bands & 640 \\    
        Spectral bands & 270 \\  
        Dispersion/pixel & 2.2 $nm$/pixel\\  
        FWHM slit image & 6 $nm$ \\  
        Integrated 2nd order filter & Yes \\  
        Entrance slit width & 20 $\mu m$ \\  
        Bit depth & 12 bit \\
        Detector pixel pitch & 7.4 $\mu m$ \\
        Weight without lens and GPS& $0.5 kg$ \\      
        Size & $7.62 cm \times 7.62 cm \times 8.74 cm $\\  
        Consumption & $\leq$ 13W (9$\sim$24VDC) \\  
        Focal length & $8 mm$ \\
        % \midrule   
        \bottomrule
    \end{tabular}}
    }
    \label{table:B1}
\end{table}
\begin{figure*}[!t]     
    \centering                         
    \includegraphics[width=2\columnwidth]{images/B2-1.jpg}                          
    \caption{The ATR2-HUTD-Lake sub-dataset. (a) Underwater Scene1; (b) Underwater Scene2.}                                  
    \label{fig:B2-1}    
\end{figure*}
\begin{table*}[!t]     
    \centering 
    \footnotesize   
    \renewcommand\arraystretch{1.75}     
    \caption{The crucial information of ATR2-HUTD dataset.}     
    \setlength{\tabcolsep}{2.55mm}    
    {     
        \scalebox{1}
        {
        \begin{tabular}{|cc|c|c|c|c|c|c|}
            \hline
            \multicolumn{2}{|c|}{\textbf{Dataset}}                         & \textbf{Wavelength}         & \textbf{Spectral   Resolution} & \textbf{Image Size}              & \textbf{Spatial   Resolution} & \textbf{Target Type} & \textbf{Target Depth} \\ \hline\hline
            \multicolumn{1}{|c|}{\multirow{2}{*}{\textbf{Lake}}}  & Scene1 & \multirow{6}{*}{400-1000nm} & \multirow{6}{*}{2.2nm}              & \multirow{2}{*}{2304$\times$640 pxiels} & \multirow{2}{*}{5.55   cm}    & Black Metal Plate    & 1.69m, 2.74m          \\ \cline{2-2} \cline{7-8} 
            \multicolumn{1}{|c|}{}                                & Scene2 &                             &                                     &                                  &                               & Blue Metal Plate   & 0.91m, 1.28m          \\ \cline{1-2} \cline{5-8} 
            \multicolumn{1}{|c|}{\multirow{2}{*}{\textbf{River}}} & Scene1 &                             &                                     & \multirow{2}{*}{3536$\times$640 pxiels} & \multirow{2}{*}{4.63 cm}      & Black Plastic Plate  & 1.97m, 1.89m           \\ \cline{2-2} \cline{7-8} 
            \multicolumn{1}{|c|}{}                                & Scene2 &                             &                                     &                                  &                               & Black Metal Plate    & 1.15m, 2.08m                 \\ \cline{1-2} \cline{5-8} 
            \multicolumn{1}{|c|}{\multirow{2}{*}{\textbf{Sea}}}   & Scene1 &                             &                                     & \multirow{2}{*}{3171$\times$640 pxiels} & \multirow{2}{*}{2.78 cm}      & Black Wooden Board   & 0.64m, 1.48m          \\ \cline{2-2} \cline{7-8} 
            \multicolumn{1}{|c|}{}                                & Scene2 &                             &                                     &                                  &                               & Yellow Wooden Board  & 1.35m                 \\ \hline
            \end{tabular} }   
    }
    \label{table:B2}
\end{table*}
% \begin{figure*}[!t]     
%     \centering                         
%     \includegraphics[width=2\columnwidth]{images/B2-3.jpg}                          
%     \caption{The ATR2-HUTD-River sub-dataset. (a) Underwater Scene1; (b) Underwater Scene2.}                                  
%     \label{fig:B2-2}    
% \end{figure*}
% \begin{figure*}[!h]     
%     \centering                         
%     \includegraphics[width=2\columnwidth]{images/B2-2.jpg}                          
%     \caption{The ATR2-HUTD-Sea dataset. (a) Underwater Scene1; (b) Underwater Scene2.}                                  
%     \label{fig:B2-3}    
% \end{figure*}
The first region, Qianlu Lake in Liuyang City, China, is a mountainous freshwater lake characterized by clear waters, steep terrain, and dense vegetation.  
As a primary freshwater source with low sedimentation and minimal human impact, it provides an optimal setting for UTD studies in low-turbidity freshwater conditions.  
\par
The second region, Xiang River in Changsha, China, is the largest river in the province and a major tributary of Dongting Lake.  
Its high flow rates and substantial sediment transport result in highly turbid waters, particularly during the wet season.  
The riverbed comprises diverse substrates, including silts and sands, creating a complex and dynamic environment for UTD studies in riverine conditions.  
\par
The third region, Yalong Bay in Sanya City, China, represents a coastal marine ecosystem with variable turbidity influenced by coastal currents and biological activity.  
Its seafloor ranges from sandy substrates to coral reefs, with fluctuating salinity and temperature.    
\par
These regions encompass diverse nearshore environments, offering a comprehensive testbed for evaluating UTD methods under varying water conditions and seafloor characteristics.  
\par
\textbf{(2) Hardware.} HSI data for the study regions were collected using a DJI Matrice 300 RTK (M300 RTK) UAV platform, equipped with real-time kinematic (RTK) capabilities.  
With a maximum payload capacity of 9 kg and a flight endurance of up to 55 minutes, the UAV enables extensive data acquisition. The RTK integration ensures centimeter-level positioning accuracy, essential for precise target annotation georeferencing.  
\par
The hyperspectral sensor used is the Headwall Nano-Hyperspec imaging sensor, known for its high spectral resolution and compact design, ideal for UAV-based remote sensing in dynamic nearshore environments.  
Detailed specifications of the sensor are provided in Tab.~\ref{table:B1}, demonstrating its capability to capture a broad spectral range crucial for analyzing complex underwater and nearshore scenes.   
\par
\begin{figure*}[!t]             
    \centering               
    \includegraphics[width=2\columnwidth]{images/A3.jpg}                
    \caption{The flowchart of the proposed underwater target detection framework.}             
    \label{fig:C1}   
\end{figure*}
A field survey utilizing GPS technology was conducted to record the geospatial coordinates of underwater targets, providing accurate ground-truth annotations for the HSI data.  
These annotations are critical for ensuring precise target identification and localization, thereby enhancing the training and evaluation of HUTD models and improving model robustness and performance assessment.
\subsection{ATR2-HUTD dataset}\label{sec:2.2}
This paper introduces the ATR2-HUTD dataset, a novel large-scale benchmark for nearshore UTD, addressing the data scarcity issue while evaluating the proposed method's efficiency and effectiveness.  
The dataset comprises three UAV-borne hyperspectral sub-datasets: ATR2-HUTD-Lake, ATR2-HUTD-River, and ATR2-HUTD-Sea, collected from nearshore regions with diverse water types and underwater targets.  
Key details of these datasets are summarized in Tab.~\ref{table:B2}.
Fig.~\ref{fig:B2-1} illustrates the ATR2-HUTD-Lake sub-dataset as an example, showcasing the underwater scenes and target types.
\par
% \subsubsection{Data acquisition}\label{sec:2.2.1}
\textbf{(1) ATR2-HUTD-Lake Sub-dataset:}  
The ATR2-HUTD-Lake sub-dataset was collected on July 6, 2021, between 14:34 and 15:42 at Qianlu Lake, Liuyang City, Hunan Province, China, under clear skies, mild sunlight, and ambient conditions of 25$^\circ$C temperature, 74\% relative humidity, and 1.7 km/h wind speed. Two nearshore regions were surveyed: one with black plastic plates submerged at depths of 1.69 m and 2.74 m, and the other with dark blue plates at depths of 0.91 m and 1.28 m. The UAV operated at 60 m altitude, providing a spatial resolution of 5.55 cm. Hyperspectral images ($2304 \times 640$ pixels) spanned 400-1000 nm with 2.2 nm spectral resolution. Reference spectra were captured on land for target identification. Fig.~\ref{fig:B2-1} presents the dataset overview, including reference spectra and ground truths.
\par
\textbf{(2) ATR2-HUTD-River Sub-dataset:}  
The ATR2-HUTD-River sub-dataset was acquired on July 10, 2024, from 10:27 to 11:09 at Xiang Lake, Changsha City, Hunan Province, China, under clear and sunny conditions with 27$^\circ$C temperature, 78\% humidity, and 2.1 m/s wind speed. Two riverine scenes were surveyed: one with black plastic plates submerged at 1.97 m and 1.89 m, and the other with a black metal plate at 1.15 m. The UAV operated at 50 m altitude, achieving 4.63 cm spatial resolution. Images ($3536 \times 640$ pixels) covered 400-1000 nm with 2.2 nm spectral resolution. Land-based reference spectra were also recorded. 
\par
\textbf{(3) ATR2-HUTD-Sea Sub-dataset:}  
The ATR2-HUTD-Sea sub-dataset was collected on June 5, 2023, between 14:58 and 15:17 at Xiaolong Bay, Sanya City, Hainan Province, China, under clear skies, strong sunlight, 32$^\circ$C temperature, 83\% humidity, and 1.5 m/s wind speed. Two coastal scenes were surveyed: one with black wooden boards submerged at 0.64 m and 1.48 m, and the other with yellow boards at 1.35 m depth. The UAV operated at 30 m altitude, yielding 2.78 cm spatial resolution. Hyperspectral images ($3171 \times 640$ pixels) spanned 400-1000 nm with 2.2 nm resolution. Reference spectra were obtained on land for target identification. 
\par
% \subsubsection{Data Preprocessing}\label{sec:2.2.2}
% Raw hyperspectral data recorded in digital number (DN) values require preprocessing before they can be utilized for hyperspectral dataset construction. 
% To ensure data accuracy and reliability, three core correction procedures—radiometric calibration, geometric correction, and spectral denoising—are applied to the raw data, as outlined below:
% \par
% \textbf{(\romannumeral1) Radiometric Calibration:} 
% Radiance values are preferred over DN values for hyperspectral image analysis. 
% Radiometric calibration is performed to convert DN values into radiance values using HyperSpec software provided by the instrument manufacturer. 
% The relationship between DN and radiance values is expressed as:
% \begin{equation}
%     \mathrm{DN} = \mathrm{L}_1 \times \mathrm{G} \times \mathrm{t}_{\exp} + \mathrm{DF},
% \end{equation}
% where $\mathrm{L}_1$ represents the radiance, $\mathrm{G}$ is the sensor gain, $\mathrm{t}_{\exp}$ denotes the photoreceptor integration time, and $\mathrm{DF}$ is the dark field measurement. 
% All calibration parameters, except $\mathrm{DF}$, are pre-measured in the laboratory and stored in the software.
% \par
% \textbf{(\romannumeral2) Geometric Correction:} 
% Spatial distortions in hyperspectral images caused by UAV flight instability necessitate geometric correction to restore spatial accuracy. 
% The process involves synchronizing UAV posture data with hyperspectral image acquisition timestamps, establishing a transformation system to map uncorrected pixels to real-world coordinates, and resampling the pixels to generate corrected imagery. 
% Geometric correction is conducted using HyperSpec software, with positional and attitude data sourced from the GPS and IMU modules of UAV.
% \par
% \textbf{(\romannumeral3) Spectral Denoising:} 
% Spectral data may suffer from noise contamination due to low signal-to-noise ratios and dark current interference, leading to potential information loss. 
% To mitigate this, the Savitzky-Golay filter~\cite{John2021} is applied, which smooths the data through local polynomial fitting, preserving critical spectral features. 
% The denoising process is implemented using ENVI software and its plug-ins.