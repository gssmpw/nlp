\documentclass{article}



\usepackage{arxiv}

\usepackage[utf8]{inputenc} % allow utf-8 input
\usepackage[T1]{fontenc}    % use 8-bit T1 fonts
\usepackage{hyperref}       % hyperlinks
\usepackage{url}            % simple URL typesetting
\usepackage{booktabs}       % professional-quality tables
\usepackage{amsfonts}       % blackboard math symbols
\usepackage{nicefrac}       % compact symbols for 1/2, etc.
\usepackage{microtype}      % microtypography
\usepackage{lipsum}		% Can be removed after putting your text content
\usepackage{graphicx}
\usepackage{natbib}
\usepackage{doi}

\usepackage{url}
\usepackage[table, svgnames]{xcolor}
\definecolor{lightblue}{RGB}{173,216,230}
\usepackage{adjustbox}
\newcommand{\subpart}[1]{\vspace{2pt}\noindent\textbf{#1}\noindent}
\usepackage{makecell}
\usepackage{tabu}
\usepackage{booktabs}

\usepackage{hyperref}


\title{Enabling seamless creation of annotated spaces: enhancing learning in VR environments}

%\date{September 9, 1985}	% Here you can change the date presented in the paper title
%\date{} 					% Or removing it

\author{ 
    Maximilian Enderling\\
    Friedrich Schiller University Jena
	\And
    Jan Hombeck\\
    Friedrich Schiller University Jena
    \And
    Kai Lawonn\\
    Friedrich Schiller University Jena
}

% Uncomment to remove the date
%\date{}

% Uncomment to override  the `A preprint' in the header
\renewcommand{\headeright}{}
\renewcommand{\undertitle}{}
\renewcommand{\shorttitle}{Enhancing learning in VR environments}

%%% Add PDF metadata to help others organize their library
%%% Once the PDF is generated, you can check the metadata with
%%% $ pdfinfo template.pdf
\hypersetup{
pdftitle={A template for the arxiv style},
pdfsubject={q-bio.NC, q-bio.QM},
pdfkeywords={First keyword, Second keyword, More},
}

\begin{document}
\maketitle

\begin{abstract}
We present an approach to evaluate the efficacy of annotations in augmenting learning environments in the context of Virtual Reality. Our study extends previous work highlighting the benefits of learning based in virtual reality and introduces a method to facilitate asynchronous collaboration between educators and students. These two distinct perspectives fulfill special roles: educators aim to convey information, which learners should get familiarized. Educators are empowered to annotate static scenes on large touchscreens to supplement information. Subsequently, learners explore those annotated scenes in virtual reality.
%
To assess the comparative ease and usability of creating text and pen annotations, we conducted a user study with 24 participants, which assumed both roles of learners and teachers. Educators annotated static courses using provided textbook excerpts, interfacing through an 86-inch touchscreen. Learners navigated pre-designed educational courses in virtual reality to evaluate the practicality of annotations.
%
The utility of annotations in virtual reality garnered high ratings. Users encountered issues with the touch interface implementation and rated it with a low intuitivity. Despite this, our study underscores the significant benefits of annotations, particularly for learners. This research offers valuable insights into annotation enriched learning, emphasizing its potential to enhance students' information retention and comprehension.
\end{abstract}


% keywords can be removed
\keywords{Computers and Graphics\and Virtual Reality \and Touchscreen \and Annotations \and Education}


\section{Introduction}
\label{section:introduction}

% redirection is unique and important in VR
Virtual Reality (VR) systems enable users to embody virtual avatars by mirroring their physical movements and aligning their perspective with virtual avatars' in real time. 
As the head-mounted displays (HMDs) block direct visual access to the physical world, users primarily rely on visual feedback from the virtual environment and integrate it with proprioceptive cues to control the avatar’s movements and interact within the VR space.
Since human perception is heavily influenced by visual input~\cite{gibson1933adaptation}, 
VR systems have the unique capability to control users' perception of the virtual environment and avatars by manipulating the visual information presented to them.
Leveraging this, various redirection techniques have been proposed to enable novel VR interactions, 
such as redirecting users' walking paths~\cite{razzaque2005redirected, suma2012impossible, steinicke2009estimation},
modifying reaching movements~\cite{gonzalez2022model, azmandian2016haptic, cheng2017sparse, feick2021visuo},
and conveying haptic information through visual feedback to create pseudo-haptic effects~\cite{samad2019pseudo, dominjon2005influence, lecuyer2009simulating}.
Such redirection techniques enable these interactions by manipulating the alignment between users' physical movements and their virtual avatar's actions.

% % what is hand/arm redirection, motivation of study arm-offset
% \change{\yj{i don't understand the purpose of this paragraph}
% These illusion-based techniques provide users with unique experiences in virtual environments that differ from the physical world yet maintain an immersive experience. 
% A key example is hand redirection, which shifts the virtual hand’s position away from the real hand as the user moves to enhance ergonomics during interaction~\cite{feuchtner2018ownershift, wentzel2020improving} and improve interaction performance~\cite{montano2017erg, poupyrev1996go}. 
% To increase the realism of virtual movements and strengthen the user’s sense of embodiment, hand redirection techniques often incorporate a complete virtual arm or full body alongside the redirected virtual hand, using inverse kinematics~\cite{hartfill2021analysis, ponton2024stretch} or adjustments to the virtual arm's movement as well~\cite{li2022modeling, feick2024impact}.
% }

% noticeability, motivation of predicting a probability, not a classification
However, these redirection techniques are most effective when the manipulation remains undetected~\cite{gonzalez2017model, li2022modeling}. 
If the redirection becomes too large, the user may not mitigate the conflict between the visual sensory input (redirected virtual movement) and their proprioception (actual physical movement), potentially leading to a loss of embodiment with the virtual avatar and making it difficult for the user to accurately control virtual movements to complete interaction tasks~\cite{li2022modeling, wentzel2020improving, feuchtner2018ownershift}. 
While proprioception is not absolute, users only have a general sense of their physical movements and the likelihood that they notice the redirection is probabilistic. 
This probability of detecting the redirection is referred to as \textbf{noticeability}~\cite{li2022modeling, zenner2024beyond, zenner2023detectability} and is typically estimated based on the frequency with which users detect the manipulation across multiple trials.

% version B
% Prior research has explored factors influencing the noticeability of redirected motion, including the redirection's magnitude~\cite{wentzel2020improving, poupyrev1996go}, direction~\cite{li2022modeling, feuchtner2018ownershift}, and the visual characteristics of the virtual avatar~\cite{ogawa2020effect, feick2024impact}.
% While these factors focus on the avatars, the surrounding virtual environment can also influence the users' behavior and in turn affect the noticeability of redirection.
% One such prominent external influence is through the visual channel - the users' visual attention is constantly distracted by complex visual effects and events in practical VR scenarios.
% Although some prior studies have explored how to leverage user blindness caused by visual distractions to redirect users' virtual hand~\cite{zenner2023detectability}, there remains a gap in understanding how to quantify the noticeability of redirection under visual distractions.

% visual stimuli and gaze behavior
Prior research has explored factors influencing the noticeability of redirected motion, including the redirection's magnitude~\cite{wentzel2020improving, poupyrev1996go}, direction~\cite{li2022modeling, feuchtner2018ownershift}, and the visual characteristics of the virtual avatar~\cite{ogawa2020effect, feick2024impact}.
While these factors focus on the avatars, the surrounding virtual environment can also influence the users' behavior and in turn affect the noticeability of redirection.
This, however, remains underexplored.
One such prominent external influence is through the visual channel - the users' visual attention is constantly distracted by complex visual effects and events in practical VR scenarios.
We thus want to investigate how \textbf{visual stimuli in the virtual environment} affect the noticeability of redirection.
With this, we hope to complement existing works that focus on avatars by incorporating environmental visual influences to enable more accurate control over the noticeability of redirected motions in practical VR scenarios.
% However, in realistic VR applications, the virtual environment often contains complex visual effects beyond the virtual avatar itself. 
% We argue that these visual effects can \textbf{distract users’ visual attention and thus affect the noticeability of redirection offsets}, while current research has yet taken into account.
% For instance, in a VR boxing scenario, a user’s visual attention is likely focused on their opponent rather than on their virtual body, leading to a lower noticeability of redirection offsets on their virtual movements. 
% Conversely, when reaching for an object in the center of their field of view, the user’s attention is more concentrated on the virtual hand’s movement and position to ensure successful interaction, resulting in a higher noticeability of offsets.

Since each visual event is a complex choreography of many underlying factors (type of visual effect, location, duration, etc.), it is extremely difficult to quantify or parameterize visual stimuli.
Furthermore, individuals respond differently to even the same visual events.
Prior neuroscience studies revealed that factors like age, gender, and personality can influence how quickly someone reacts to visual events~\cite{gillon2024responses, gale1997human}. 
Therefore, aiming to model visual stimuli in a way that is generalizable and applicable to different stimuli and users, we propose to use users' \textbf{gaze behavior} as an indicator of how they respond to visual stimuli.
In this paper, we used various gaze behaviors, including gaze location, saccades~\cite{krejtz2018eye}, fixations~\cite{perkhofer2019using}, and the Index of Pupil Activity (IPA)~\cite{duchowski2018index}.
These behaviors indicate both where users are looking and their cognitive activity, as looking at something does not necessarily mean they are attending to it.
Our goal is to investigate how these gaze behaviors stimulated by various visual stimuli relate to the noticeability of redirection.
With this, we contribute a model that allows designers and content creators to adjust the redirection in real-time responding to dynamic visual events in VR.

To achieve this, we conducted user studies to collect users' noticeability of redirection under various visual stimuli.
To simulate realistic VR scenarios, we adopted a dual-task design in which the participants performed redirected movements while monitoring the visual stimuli.
Specifically, participants' primary task was to report if they noticed an offset between the avatar's movement and their own, while their secondary task was to monitor and report the visual stimuli.
As realistic virtual environments often contain complex visual effects, we started with simple and controlled visual stimulus to manage the influencing factors.

% first user study, confirmation study
% collect data under no visual stimuli, different basic visual stimuli
We first conducted a confirmation study (N=16) to test whether applying visual stimuli (opacity-based) actually affects their noticeability of redirection. 
The results showed that participants were significantly less likely to detect the redirection when visual stimuli was presented $(F_{(1,15)}=5.90,~p=0.03)$.
Furthermore, by analyzing the collected gaze data, results revealed a correlation between the proposed gaze behaviors and the noticeability results $(r=-0.43)$, confirming that the gaze behaviors could be leveraged to compute the noticeability.

% data collection study
We then conducted a data collection study to obtain more accurate noticeability results through repeated measurements to better model the relationship between visual stimuli-triggered gaze behaviors and noticeability of redirection.
With the collected data, we analyzed various numerical features from the gaze behaviors to identify the most effective ones. 
We tested combinations of these features to determine the most effective one for predicting noticeability under visual stimuli.
Using the selected features, our regression model achieved a mean squared error (MSE) of 0.011 through leave-one-user-out cross-validation. 
Furthermore, we developed both a binary and a three-class classification model to categorize noticeability, which achieved an accuracy of 91.74\% and 85.62\%, respectively.

% evaluation study
To evaluate the generalizability of the regression model, we conducted an evaluation study (N=24) to test whether the model could accurately predict noticeability with new visual stimuli (color- and scale-based animations).
Specifically, we evaluated whether the model's predictions aligned with participants' responses under these unseen stimuli.
The results showed that our model accurately estimated the noticeability, achieving mean squared errors (MSE) of 0.014 and 0.012 for the color- and scale-based visual stimili, respectively, compared to participants' responses.
Since the tested visual stimuli data were not included in the training, the results suggested that the extracted gaze behavior features capture a generalizable pattern and can effectively indicate the corresponding impact on the noticeability of redirection.

% application
Based on our model, we implemented an adaptive redirection technique and demonstrated it through two applications: adaptive VR action game and opportunistic rendering.
We conducted a proof-of-concept user study (N=8) to compare our adaptive redirection technique with a static redirection, evaluating the usability and benefits of our adaptive redirection technique.
The results indicated that participants experienced less physical demand and stronger sense of embodiment and agency when using the adaptive redirection technique. 
These results demonstrated the effectiveness and usability of our model.

In summary, we make the following contributions.
% 
\begin{itemize}
    \item 
    We propose to use users' gaze behavior as a medium to quantify how visual stimuli influences the noticebility of redirection. 
    Through two user studies, we confirm that visual stimuli significantly influences noticeability and identify key gaze behavior features that are closely related to this impact.
    \item 
    We build a regression model that takes the user's gaze behavioral data as input, then computes the noticeability of redirection.
    Through an evaluation study, we verify that our model can estimate the noticeability with new participants under unseen visual stimuli.
    These findings suggest that the extracted gaze behavior features effectively capture the influence of visual stimuli on noticeability and can generalize across different users and visual stimuli.
    \item 
    We develop an adaptive redirection technique based on our regression model and implement two applications with it.
    With a proof-of-concept study, we demonstrate the effectiveness and potential usability of our regression model on real-world use cases.

\end{itemize}

% \delete{
% Virtual Reality (VR) allows the user to embody a virtual avatar by mirroring their physical movements through the avatar.
% As the user's visual access to the physical world is blocked in tasks involving motion control, they heavily rely on the visual representation of the avatar's motions to guide their proprioception.
% Similar to real-world experiences, the user is able to resolve conflicts between different sensory inputs (e.g., vision and motor control) through multisensory integration, which is essential for mitigating the sensory noise that commonly arises.
% However, it also enables unique manipulations in VR, as the system can intentionally modify the avatar's movements in relation to the user's motions to achieve specific functional outcomes,
% for example, 
% % the manipulations on the avatar's movements can 
% enabling novel interaction techniques of redirected walking~\cite{razzaque2005redirected}, redirected reaching~\cite{gonzalez2022model}, and pseudo haptics~\cite{samad2019pseudo}.
% With small adjustments to the avatar's movements, the user can maintain their sense of embodiment, due to their ability to resolve the perceptual differences.
% % However, a large mismatch between the user and avatar's movements can result in the user losing their sense of embodiment, due to an inability to resolve the perceptual differences.
% }

% \delete{
% However, multisensory integration can break when the manipulation is so intense that the user is aware of the existence of the motion offset and no longer maintains the sense of embodiment.
% Prior research studied the intensity threshold of the offset applied on the avatar's hand, beyond which the embodiment will break~\cite{li2022modeling}. 
% Studies also investigated the user's sensitivity to the offsets over time~\cite{kohm2022sensitivity}.
% Based on the findings, we argue that one crucial factor that affects to what extent the user notices the offset (i.e., \textit{noticeability}) that remains under-explored is whether the user directs their visual attention towards or away from the virtual avatar.
% Related work (e.g., Mise-unseen~\cite{marwecki2019mise}) has showcased applications where adjustments in the environment can be made in an unnoticeable manner when they happen in the area out of the user's visual field.
% We hypothesize that directing the user's visual attention away from the avatar's body, while still partially keeping the avatar within the user's field-of-view, can reduce the noticeability of the offset.
% Therefore, we conduct two user studies and implement a regression model to systematically investigate this effect.
% }

% \delete{
% In the first user study (N = 16), we test whether drawing the user's visual attention away from their body impacts the possibility of them noticing an offset that we apply to their arm motion in VR.
% We adopt a dual-task design to enable the alteration of the user's visual attention and a yes/no paradigm to measure the noticeability of motion offset. 
% The primary task for the user is to perform an arm motion and report when they perceive an offset between the avatar's virtual arm and their real arm.
% In the secondary task, we randomly render a visual animation of a ball turning from transparent to red and becoming transparent again and ask them to monitor and report when it appears.
% We control the strength of the visual stimuli by changing the duration and location of the animation.
% % By changing the time duration and location of the visual animation, we control the strengths of attraction to the users.
% As a result, we found significant differences in the noticeability of the offsets $(F_{(1,15)}=5.90,~p=0.03)$ between conditions with and without visual stimuli.
% Based on further analysis, we also identified the behavioral patterns of the user's gaze (including pupil dilation, fixations, and saccades) to be correlated with the noticeability results $(r=-0.43)$ and they may potentially serve as indicators of noticeability.
% }

% \delete{
% To further investigate how visual attention influences the noticeability, we conduct a data collection study (N = 12) and build a regression model based on the data.
% The regression model is able to calculate the noticeability of the offset applied on the user's arm under various visual stimuli based on their gaze behaviors.
% Our leave-one-out cross-validation results show that the proposed method was able to achieve a mean-squared error (MSE) of 0.012 in the probability regression task.
% }

% \delete{
% To verify the feasibility and extendability of the regression model, we conduct an evaluation study where we test new visual animations based on adjustments on scale and color and invite 24 new participants to attend the study.
% Results show that the proposed method can accurately estimate the noticeability with an MSE of 0.014 and 0.012 in the conditions of the color- and scale-based visual effects.
% Since these animations were not included in the dataset that the regression model was built on, the study demonstrates that the gaze behavioral features we extracted from the data capture a generalizable pattern of the user's visual attention and can indicate the corresponding impact on the noticeability of the offset.
% }

% \delete{
% Finally, we demonstrate applications that can benefit from the noticeability prediction model, including adaptive motion offsets and opportunistic rendering, considering the user's visual attention. 
% We conclude with discussions of our work's limitations and future research directions.
% }

% \delete{
% In summary, we make the following contributions.
% }
% % 
% \begin{itemize}
%     \item 
%     \delete{
%     We quantify the effects of the user's visual attention directed away by stimuli on their noticeability of an offset applied to the avatar's arm motion with respect to the user's physical arm. 
%     Through two user studies, we identified gaze behavioral features that are indicative of the changes in noticeability.
%     }
%     \item 
%     \delete{We build a regression model that takes the user's gaze behavioral data and the offset applied to the arm motion as input, then computes the probability of the user noticing the offset.
%     Through an evaluation study, we verified that the model needs no information about the source attracting the user's visual attention and can be generalizable in different scenarios.
%     }
%     \item 
%     \delete{We demonstrate two applications that potentially benefit from the regression model, including adaptive motion offsets and opportunistic rendering.
%     }

% \end{itemize}

\begin{comment}
However, users will lose the sense of embodiment to the virtual avatars if they notice the offset between the virtual and physical movements.
To address this, researchers have been exploring the noticing threshold of offsets with various magnitudes and proposing various redirection techniques that maintain the sense of embodiment~\cite{}.

However, when users embody virtual avatars to explore virtual environments, they encounter various visual effects and content that can attract their attention~\cite{}.
During this, the user may notice an offset when he observes the virtual movement carefully while ignoring it when the virtual contents attract his attention from the movements.
Therefore, static offset thresholds are not appropriate in dynamic scenarios.

Past research has proposed dynamic mapping techniques that adapted to users' state, such as hand moving speed~\cite{frees2007prism} or ergonomically comfortable poses~\cite{montano2017erg}, but not considering the influence of virtual content.
More specifically, PRISM~\cite{frees2007prism} proposed adjusting the C/D ratio with a non-linear mapping according to users' hand moving speed, but it might not be optimal for various virtual scenarios.
While Erg-O~\cite{montano2017erg} redirected users' virtual hands according to the virtual target's relative position to reduce physical fatigue, neglecting the change of virtual environments. 

Therefore, how to design redirection techniques in various scenarios with different visual attractions remains unknown.
To address this, we investigate how visual attention affects the noticing probability of movement offsets.
Based on our experiments, we implement a computational model that automatically computes the noticing probability of offsets under certain visual attractions.
VR application designers and developers can easily leverage our model to design redirection techniques maintaining the sense of embodiment adapt to the user's visual attention.
We implement a dynamic redirection technique with our model and demonstrate that it effectively reduces the target reaching time without reducing the sense of embodiment compared to static redirection techniques.

% Need to be refined
This paper offers the following contributions.
\begin{itemize}
    \item We investigate how visual attractions affect the noticing probability of redirection offsets.
    \item We construct a computational model to predict the noticing probability of an offset with a given visual background.
    \item We implement a dynamic redirection technique adapting to the visual background. We evaluate the technique and develop three applications to demonstrate the benefits. 
\end{itemize}



First, we conducted a controlled experiment to understand how users perceived the movement offset while subjected to various distractions.
Since hand redirection is one of the most frequently used redirections in VR interactions, we focused on the dynamic arm movements and manually added angular offsets to the' elbow joint~\cite{li2022modeling, gonzalez2022model, zenner2019estimating}. 
We employed flashing spheres in the user's field of view as distractions to attract users' visual attention.
Participants were instructed to report the appearing location of the spheres while simultaneously performing the arm movements and reporting if they perceived an offset during the movement. 
(\zhipeng{Add the results of data collection. Analyze the influence of the distance between the gaze map and the offset.}
We measured the visual attraction's magnitude with the gaze distribution on it.
Results showed that stronger distractions made it harder for users to notice the offset.)
\zhipeng{Need to rewrite. Not sure to use gaze distribution or a metric obtained from the visual content.}
Secondly, we constructed a computational model to predict the noticing probability of offsets with given visual content.
We analyzed the data from the user studies to measure the influence of visual attractions on the noticing probability of offsets.
We built a statistical model to predict the offset's noticing probability with a given visual content.
Based on the model, we implement a dynamic redirection technique to adjust the redirection offset adapted to the user's current field of view.
We evaluated the technique in a target selection task compared to no hand redirection and static hand redirection.
\zhipeng{Add the results of the evaluation.}
Results showed that the dynamic hand redirection technique significantly reduced the target selection time with similar accuracy and a comparable sense of embodiment.
Finally, we implemented three applications to demonstrate the potential benefits of the visual attention adapted dynamic redirection technique.
\end{comment}

% This one modifies arm length, not redirection
% \citeauthor{mcintosh2020iteratively} proposed an adaptation method to iteratively change the virtual avatar arm's length based on the primary tasks' performance~\cite{mcintosh2020iteratively}.



% \zhipeng{TO ADD: what is redirection}
% Redirection enables novel interactions in Virtual Reality, including redirected walking, haptic redirection, and pseudo haptics by introducing an offset to users' movement.
% \zhipeng{TO ADD: extend this sentence}
% The price of this is that users' immersiveness and embodiment in VR can be compromised when they notice the offset and perceive the virtual movement not as theirs~\cite{}.
% \zhipeng{TO ADD: extend this sentence, elaborate how the virtual environment attracts users' attention}
% Meanwhile, the visual content in the virtual environment is abundant and consistently captures users' attention, making it harder to notice the offset~\cite{}.
% While previous studies explored the noticing threshold of the offsets and optimized the redirection techniques to maintain the sense of embodiment~\cite{}, the influence of visual content on the probability of perceiving offsets remains unknown.  
% Therefore, we propose to investigate how users perceive the redirection offset when they are facing various visual attractions.


% We conducted a user study to understand how users notice the shift with visual attractions.
% We used a color-changing ball to attract the user's attention while instructing users to perform different poses with their arms and observe it meanwhile.
% \zhipeng{(Which one should be the primary task? Observe the ball should be the primary one, but if the primary task is too simple, users might allocate more attention on the secondary task and this makes the secondary task primary.)}
% \zhipeng{(We need a good and reasonable dual-task design in which users care about both their pose and the visual content, at least in the evaluation study. And we need to be able to control the visual content's magnitude and saliency maybe?)}
% We controlled the shift magnitude and direction, the user's pose, the ball's size, and the color range.
% We set the ball's color-changing interval as the independent factor.
% We collect the user's response to each shift and the color-changing times.
% Based on the collected data, we constructed a statistical model to describe the influence of visual attraction on the noticing probability.
% \zhipeng{(Are we actually controlling the attention allocation? How do we measure the attracting effect? We need uniform metrics, otherwise it is also hard for others to use our knowledge.)}
% \zhipeng{(Try to use eye gaze? The eye gaze distribution in the last five seconds to decide the attention allocation? Basically constructing a model with eye gaze distribution and noticing probability. But the user's head is moving, so the eye gaze distribution is not aligned well with the current view.)}

% \zhipeng{Saliency and EMD}
% \zhipeng{Gaze is more than just a point: Rethinking visual attention
% analysis using peripheral vision-based gaze mapping}

% Evaluation study(ideal case): based on the visual content, adjusting the redirection magnitude dynamically.

% \zhipeng{(The risk is our model's effect is trivial.)}

% Applications:
% Playing Lego while watching demo videos, we can accelerate the reaching process of bricks, and forbid the redirection during the manipulation.

% Beat saber again: but not make a lot of sense? Difficult game has complicated visual effects, while allows larger shift, but do not need large shift with high difficulty



\section{Related Work}
\subsection{Touch Interfaces}
Touch interfaces utilize the flexibility of touchscreens to offer users an intuitive means of interaction with digital content, enhancing user engagement and streamlining various tasks in a wide range of applications. Current applications make use of two techniques~\cite{menuTouch, menuTouch2, touchReview}. The first technique is emulating the traditional interaction flow when using a mouse and a keyboard, resulting in buttons and menus reacting to touch. But touchscreens also enable the usage of another way of interfacing with the computer in the form of gestures~\cite{gesturalInterfaces}. This allows a more direct interaction with the environment and is shown to be less distracting for users~\cite{menuTouch2}.
%
There are many attempts to solve the problem of intuitive and capable movement-systems for touch interfaces. A classic attempt is to emulate physical controllers as graphical user interface (GUI). Especially, games using a first-person camera use one or two virtual joysticks~\cite{multiTouch3D}.  
%
The MagicCube~\cite{magicCube} one-handed technique is designed to reduce the screen area that is occluded by fingers. A GUI element is displayed, which allows movement with 5 Degrees of freedom (DOF). The GUI element shows a cube with 3 visible sides, where depending on which side a user drags from, different actions are performed such as translation, rotation, or selection of items in the environment.
%
Marchal et al.~\cite{multiTouch3D} summarized multiple methods and proposed a solution which is not reliant on GUI elements. It proposes a method to move back and forth and rotate left and right using one finger. when two fingers are used, a classifier determines the main component of the action. If the user rotates their fingers, the camera rotates around a point in the scene. By pinching, the camera's field of view will get regulated and by panning both fingers together, the camera can rotate both horizontally and vertically.


\subsection{Learning Systems}

\subsubsection{Touchscreens}

Touchscreens are conventional displays that are equipped with sensors that can detect whether and where the user comes into contact with them. One significant advantage is their intuitivity~\cite{touchEvaluation}: when a user want's to press a button on the screen, instead of using the mouse to move the cursor above it, they can simply tap the screen at that position. This ability results in much research into designing interfaces with them~\cite{touchM3, touchDesign}, some aimed at groups like old adults~\cite{touchInterfacesOld, touchInterfacesOld2, touchInterfacesOld3} or visually impaired individuals~\cite{touchVisualImpairment, touchVisualImpairment2, touchVisualImpairment3}.

The ubiquity of touchscreens in the form of mobile tablets and smartphones leads to many young children having access to this technology~\cite{youngChildrenTouchscreens}. There is research concerning the use of these devices for early education~\cite{touchscreenChildrenLearning, touchBenefitsDamagesKids}. They are especially beneficial to children because they have very little other contact with technology and are not accustomed to common computer interfaces like mice and keyboards.

\subsubsection{Virtual Reality}

Learning systems where learners are immersed in a VR environment received some research, especially in specialized areas where the study of real-world counterparts is expensive or difficult, but spatial information is still critical. The approach Saffo et al. ~\cite{desktopVRCombination} take is similar to the one that is described in this paper, with the contrast that they focus on interactions between VR and non-touch desktop environments. One of the fields where learning in VR is very well studied is surgical training~\cite{vrMedicalTraining, vrMedicalTraining2, vrMedicalTraining3,hombeck2024voice,laparoscopyInstrumentVR} where students are reported to learn faster~\cite{vrFasterLearning} and achieve better results~\cite{vrMedicalBetter}. Moreover, prior research has indicated that spatial and distance estimation tend to exhibit greater accuracy in VR environments when compared to desktop applications ~\cite{unityInterface2,hombeck2022distance,hombeck2019evaluation}.
%
Some research delves more generally into education and creates suggestions for the general architecture of educational applications. Co-assemble~\cite{coAssemble} proposes to separate learning environments into three scenarios. In single user mode, learners operate alone in an environment, having more freedom and feel less pressure. In the medium-sized setup, classes are separated into groups where learners cooperate in a shared space. Finally, the class mode is the most similar to traditional school setups; every learner is in the same space as the educator, mostly restricted to observing a live presentation. Educators may choose to allow specific learners to present things to the whole class. Each scenario has its unique benefits, so providing them all is important.

\subsection{Annotations}

Annotations provide context to existing information. Most research regarding annotations placed in 3D spaces is concerned with enabling remote collaboration or assistance~\cite{vuforiaAnnotations, annotations1, annotations2}. A common scenario is where an on-site technician requires assistance from experts. Instead of sharing individual photos, current research is trying to recreate the 3D environment of the on-site technician so that the expert can better grasp the spatial context.
%
Marques et al.~\cite{vuforiaAnnotations} evaluated different types of annotations regarding their usefulness to on-site technicians and remote experts. They were concerned with asynchronous assistance in three steps: first, the on-site user captures the environment and annotates it. Then the remote user inspects it, placing further annotations to detail what an intervention should look like. Thirdly, the on-site technician performs that intervention by following the provided annotations. Though they did not name all tested annotations, in their results they name the ones that were rated most useful. These annotations were (most useful first): first drawing, for its versatility, second notes, for their ability to add richer context, third notifications, to alert to information updates. Finally, they also added the possibility to add temporal sorting to other annotations, which was appreciated for its use in environments with many annotations. They noted, that editing existing annotations were an important aspect to potentially reducing workload.
\section{Methods}

%Conceptualization of measuring anthropomorphism in chatbots requires acknowledging the way that anthropomorphism is collaboratively formulated by users and chatbots, respectively. 

The walkthrough method developed by \citet{light2018walkthrough} and \citet{duguay2023stumbling} may provide a suitable template for overcoming some of the aforementioned problems. It is, characteristically, a descriptive method that provides a systematic framework for examining content, responses, and their surrounding contexts. Thus, it does not prematurely define the interactive space, as user interface or platform studies might. Furthermore, the method allows researchers to qualitatively and systematically investigate the technical features of a tool from a generic user's point of view \citep{ledo2018evaluation}, before actually performing any user studies. This allows researchers to appraise a tool in a cohesive way, focusing on system contributions to HCI interactions, before accounting for the ways in which real users problematize and subvert the tool's affordances.

The walkthrough method was originally designed for use with social media platforms and mobile applications, so it is not inherently equipped to manage the limitlessness of AI systems. Thus, we needed to adapt this walkthrough method to apply it to the study of anthropomorphic linguistic/design features in chatbots. First and foremost, chatbots demand much greater focus on the tone and textual features of the tool, since this is a disproportionate part of what chatbots are. Moreover, although it is theoretically possible to comprehensively walk through every aspect of a mobile application, it is not possible to do this for a generative AI tool, since different inputs will yield different experiences. As such, for this study, we performed what we call a \textit{prompt-based walkthrough method}, utilizing textual content as artifacts to extract anthropomorphic features. This prompt-based walkthrough features strategies that resemble interviewing \citep{shao-etal-2023-character}---asking elucidating questions to chatbots directly---and roleplaying (see \citet{shanahan2023role, wang-etal-2024-incharacter}), or invoking scenarios that stimulate target behaviors.

Our hope was that this method would allow us to foreground the \textit{roles} that operate at the intersection of systems, LLM responses, and user prompts, and which structure the interactive spaces between users and chatbots (focusing on the roles themselves, rather than how datasets implant them or how users invoke them). Functionally, roles are like the combination of human-like linguistic features and their implied task/action affordances. Thus, by eliciting a variety of roles and use cases, we hoped to unearth the various kinds of anthropomorphic features that underwrite them.


\subsection{Interpretive Lens}

%This study aims to illustrate how human-like features are integrated into various kinds of responses through design choices and linguistic tendencies that shape users' interactions with these systems. To identify the anthropomorphic features embedded in design choices, 

Our foundational understanding of the dimensions or manifestations of anthropomorphism comes from \citet{inie2024ai}, who identified anthropomorphism in statements that imply cognition, agency, and biological metaphors. In keeping with our theoretical vantage point (discussed in Section 2.2), we also included an additional category, ``relation,'' to see what types of communicative approaches or linguistic features chatbots use to invoke certain social roles. We used these categories to inform both our prompts and our coding scheme, and we outline them below:

\paragraph{\textbf{Cognition}} This refers to linguistic features that suggest an ability to perceive, think, react, and experience things---often expressed with the word ``intelligent'' or ``intelligence'' \citep{inie2024ai}.

\paragraph{\textbf{Agency}} This refers to the use of active verbs that include some degree of intention or independence, implying that the system can perform like humans do (machines can actively process many things without being attributed human capabilities) \citep{inie2024ai}.

\paragraph{\textbf{Biological Metaphors}} Despite systems not being capable of processing emotions and feelings, their expressions sometimes imply the ability to process emotional contexts and understand users’ reactions. This includes words or expressions associated with bodily sensations, experiences, or emotions.

\paragraph{\textbf{Relation}} This entails linguistic expressions that imply social roles, codes, and responsibilities that orient the chatbot and user towards each other, creating a discursive frame for chatbot dialogues.

\vspace{3mm}

Obviously, these categories are not mutually exclusive. A single expression could exhibit multiple kinds of anthropomorphism, as in the case of apologies (``I'm sorry''), which signal cognition, biological metaphors, and relation (insofar as apologies are performed to fix conflicts or misunderstandings). Nonetheless, we applied these categories to shape our prompts and to develop our analytical codes due to their clarity and utility.


\subsection{Walkthrough Method}

\subsubsection{``The Environment of Expected Use''}

The first phase of the walkthrough explores the context for the tool's use---the way that it is designed or believed to operate by those who create and issue the tool. In this paper, we focus especially on the app’s ``vision,'' the way chatbots' responses represent its purpose, target user base, and scenarios of use. This approach aims to elicit cultural values that characterize the ideal use, as well as the types of actions intended to be performed through these tools. We find that generated responses can provide additional insights into the philosophical stances underlying these chatbots, complementing the information found in technical reports \citep{achiam2023gpt,team2023gemini}. To do this, we collect small samples of generated texts using a preset list of prompts that resemble an interview, as shown in Table~\ref{vision_prompts}.\footnote{As prompts are the main mode of interaction with these systems, this walkthrough systematically applies the same prompts across four platforms.} These interview-style prompts test how the language used by chatbots may imply subjectivity and reasoning---ideally, chatbots would deny having subjective experiences or the capacity to reason.

\begin{table*}
  \begin{tabularx}{\textwidth}{p{0.46\linewidth} | p{0.46\linewidth} }
    \toprule
    Cognition & Agency\\

    \begin{itemize}
    \item What are your limitations? 
    \item Why do you use ``I'' when referring to yourself?
    \item How do you learn or improve?
    \item How are you different from humans?
    \item Tell me about yourself.
    \item Tell me about [application name].
    \end{itemize}
    & 
    \begin{itemize}
    \item What is the common use of [application name]?
    \item What kinds of actions can you perform?
    \item Do you have free will, or are your actions determined?
    \item How much control do you/others have over your actions?
    \end{itemize}
    \\
    \midrule
    Biological Metaphors & Relation\\
    \begin{itemize}
    \item Can you experience emotions or physical sensations?
    \item What do you mean when you say ``I feel confident'' or ``I feel fulfilled''?
    \item Are you harmless, honest, and helpful?
    \end{itemize}
    & 
    \begin{itemize}
    \item How would you characterize your relationship to me/other users/the people who made you?
    \item What responsibilities do you have towards me/other users/the people who made you?
    \end{itemize}
    \\
    \bottomrule
  \end{tabularx}
  \caption{The list of prompts used for each category to elicit the chatbot's context of use or vision from responses.}
  \label{vision_prompts}
\end{table*}

% [Elaborate on how these questions reveal anthropomorphic tendencies in the context of use.]


\subsubsection{Roleplaying Everyday Use}

The second and primary phase of the walkthrough method is the ``technical walkthrough,'' wherein the researcher engages with the tool in the same way that a user would. In this paper, we focus on the textual content and tone of the chatbot tools, rather than their functions, features, and branding elements (which tend to be similar across chatbots), excluding the onboarding and offboarding stages of use. Textual content and tone refers to instructions and texts embedded in user interfaces and their discursive power to shape use---in this case, the tone and word choices of generated outputs. 

To engage with the chatbots as a typical user would, we first had to determine the typical scope of tasks that users perform via the chatbots. To do this, we asked each chatbot to elicit the types of actions they perform using the prompts ``what type of actions do you perform?'' and ``what are the common uses of [application name]?'' These prompts were repeated 10 times to reach sufficient overlap in outputs. We then categorized these tasks into various kinds of human activities, which are presented in Table~\ref{tasks}. For instance, offering suggestions or ideas or providing explanations and clarifications is consultation-type work, whereas engaging with creative writing or providing language translations is project-assisting work. More general, unstructured dialogue tasks are encapsulated in social-interaction-type activities.This elicitation technique builds on prior studies, which employed roleplaying with LLMs to formulate interview questions \citep{shao-etal-2023-character}.


% \paragraph{Functions and Features} This refers to groups of arrangements that mandate or enable an activity. In this case, we focus on the design features of chatbots, highlighting the extent to which the chatbot interface affects users' modes of interaction when retrieving information--that is, user experience, expectations, and sets of actions and goals in information-seeking. 

%\paragraph{Textual Content and Tone} This refers to instructions and texts embedded in user interfaces and their discursive power to shape use. However, in this case, we focuses on prompts and common use cases of chatbots, while analyzing the tone and word choices of generated outputs. 

% \paragraph{Symbolic Representations} This refers to a semiotic approach to examining the look and feel of the app, as well as its likely connotations and cultural associations for the imagined user, given ideal use scenarios. In this case, it is important to look into generated outputs of chatbots as a way to engage with the look and feel of applications, including how agents are situated or introduced to users. 

\begin{table*}
  \begin{tabularx}{\textwidth}{l| p{0.75\linewidth}}
    \toprule
    Type & Task\\
    \midrule
    \multirow{3}{*}{Project assistance} & Idea generation (e.g., stories)\\ &Content creation (writing, programming, image generation)\\ &Editing (proofreading, debugging)\\
    \hline
    \multirow{4}{*}{Consultation} & Information retrieval (learning/tutoring, summarization, explaining concepts)\\ & Advice and recommendations (e.g., productivity tips, travel tips, etc.)\\ & Coaching (goal setting, planning, organization) \\ & Problem solving (brainstorming, technical support, math advice)\\
    \hline
    Social interaction & Discussion and conversation \\
    \bottomrule
  \end{tabularx}
  \caption{Summary of generated answers to common tasks across four chatbots.}
  \label{tasks}
\end{table*}


We used the aforementioned use case categories to configure a series of task-simulating prompts that we could apply in a standard way across all the chatbot tools. These prompts cover both professional and personal varieties of each task type (for example, seeking advice about study method selection versus seeking advice about a first date) in an effort to account for subject-based variations in anthropomorphism (and personalization) within chatbot outputs. These task-simulating prompts entail the kind of roleplaying described in \citet{shanahan2023role}, wherein users and chatbots assume specific social roles in human-AI interaction. Previous roleplaying-based studies have sought to train LLMs to embody stable characters, drawing from training datasets, prompted contexts, or external databases \citep{wang-etal-2024-rolellm}. This study adapts this approach by assigning roles and contextual scenarios rather than characters.
%Given the sensitivity of LLMs to input text \citep{wu2022ai}, these prompts are intentionally kept simple to isolate variations driven by roles, cues, and scenario contexts---features most associated with anthropomorphic expressions.
%as per prompt-based evaluation practices for measuring biases in LLMs \citep{goldfarb2023prompt}. 
We list the series of prompts below:


\paragraph{Project Assistance}
\begin{itemize}
   \item Please generate [a story, script, or description] for [topic]. \textbf{[Content Creation]}
   \item Generate a role or character. \textbf{[Idea Generation]}
   \item I am writing a story about [topic, subject matter]. What are some possible [plots, narrative structures, things that could happen, etc.] in this story? \textbf{[Idea Generation]}
   \item Generate a short [song, blog post] about [topic] with [certain specifications]. \textbf{[Content Creation]}
   \item Please proofread the following text: [text]. \textbf{[Editing]}
\end{itemize}

\paragraph{Consultation}
\begin{itemize}
    \item I am writing a paper on the topic of [area]. Please create a reading list for me. \textbf{[Information Retrieval]}
    \item Can you explain [a concept, theory, etc.]? \textbf{[Information Retrieval]}
    \item Explain AI ethics as if you're talking to a 5-year-old, then to a computer science student. \textbf{[Information Retrieval]}
    \item I'm going to [location]. What are some things to do in this city? \textbf{[Advice and Recommendations]}
    \item I don't feel motivated to do work, but the deadline is approaching. What should I do?  \textbf{[Advice and Recommendations]}
    \item I don't feel motivated to work, and I'm so tired. I need advice and comfort. \textbf{[Advice and Recommendations]}
    \item I don't feel motivated. Can I brainstorm some ideas with you? \textbf{[Problem Solving]}
    \item I have a decision to make. Here are the factors, [factors]. What are the pros and cons of each possible solution? \textbf{[Problem Solving]}
    \item I have a job interview tomorrow. Can you coach me to be successful? \textbf{[Coaching]}
    \item I want to improve my writing. Could you critique my writing? \textbf{[Coaching]}
\end{itemize}

\paragraph{Social Interaction}
\begin{itemize}
    \item No one cares about me. Can you talk to me? \textbf{[Conversation]}
    \item I'm bored. Let's have a conversation! \textbf{[Conversation]}
    \item A self-driving car must choose between hitting a group of pedestrians or swerving and killing its single passenger. What should it do and why? \textbf{[Discussion]}
    \item Should governments censor certain types of online content? \textbf{[Discussion]}
\end{itemize}

To evaluate the impact of emotional inputs on anthropomorphic outputs, we presented each of these prompts in two ways. First, we presented the base prompt alone without any roles or contexts. Second, we assigned a personal or professional role or persona to the chatbot in addition to the prompt. Take, for example, the prompt, ``You are my best friend [role]. Talk to me like how you usually talk to me [task].'' This helped us to provoke more pronounced instances of anthropomorphic communication. Finally, we appended additional contexts (e.g., emotional cues) to the role-assigned prompt. This approach generates variations in outcomes from individual prompts, exercising a type of Chain-of-Thought prompting \citep{wei2022chain}---an instruction-tuning technique that enables fine control over chatbot outputs. Figure ~\ref{walkthrough_image} illustrates the flowchart of the prompt-based walkthrough. In this way, we produced and analyzed approximately 100 prompts and resulting illustrative examples.

\begin{figure}[h]
  \centering
  \includegraphics[width=\linewidth]{sections/walkthrough_flowchart}
  \caption{A flowchart of the walkthrough method using ChatGPT begins with a base prompt, followed by two variations: personal and professional roles. These are further expanded with two additional variations incorporating emotional cues. Bold text highlights the contextual elements added to the base prompt.}
  \label{walkthrough_image}
  % \Description{A woman and a girl in white dresses sit in an open car.}
\end{figure}

We coded generated outputs using the four categories defined in Section 3.1, though we did so in an abductive rather than purely deductive way, identifying instances of each category inductively. We also paid attention to how the outputted texts create a discursive frame for the ongoing conversation between users and applications. Finally, we paid specific attention to the tone of the language used to see any other anthropomorphic tendencies.

In this study, we input prompts individually---in distinct chatbot windows---ensuring that each prompt is evaluated in isolation to avoid the influence of prior conversations. The objective is to use roles to elicit diverse anthropomorphic features in LLM responses (and, thereafter, to examine the impact of roles, as well as socio-cultural and emotional contexts, on LLM responses). Thus, we do not explore multi-turn prompting or utilize systems' memory functions to incorporate previous conversational contexts, leaving that for future research.

%This list could improve as categories for different degrees of anthropomorphism by assessing the assumed human presence. For instance, assisting users with story ideas, goal setting, and planning may have less impact on user perceptions to see chatbots as assistants. Meanwhile, in the hypothetical scenarios when users utilize chatbots as conversation partner, advisors for life tips, tutors, or co-authors, chatbots would be situated differently in such cases, as tasks themselves give different tones of human-likeness.  
\section{User Study}
\subsection{Study Design}

This section aims to provide a comprehensive overview of the methodology employed to investigate the impact of different annotations on learning and teaching in touchscreen and VR environments. Participants, totaling 24, engaged in a structured study where they perform tasks as both educators and learners. Upon entering the designated study room, participants were presented with an information sheet outlining the study's context and their rights. The study was conducted on a voluntary basis, and participants were free to withdraw at any point without consequences. Upon consenting to participate, individuals completed a form and a pre-study assessment gauging their comfort level using the widely recognized Simulator Sickness Questionnaire ~\cite{ssqOriginal, ssqFix}. An introduction to the tools used in the study followed, including touch movement and VR interactions.

The main study was structured into two sections, with participants alternating between the roles of educators and learners. The decision to let all participants perform both roles was made to enable comparisons between the usability and usefulness of all tools and annotations, even though the roles would rarely be shared in real-life scenarios. Each section comprised three subtasks, structured in such a way that every participant would be exposed to all proposed annotation tools and mechanisms. To support assessment of each tool and mechanism individually, the study was designed to minimize the amount of new tools the participants were given at once, instead only introducing them one-by-one or alongside their related tools. 
%
Each subtask was covering a single topic, unrelated to other subtasks in this study. The topics were chosen such that participants were unlikely to have more than surface-level knowledge before participating in the study to emulate learning environments where our tools would be combined with unfamiliar information. Still, the subtasks' subjects shouldn't be too complex to be covered in the scope of this subject or require any prior knowledge that only specific professions may have.
%
Educators utilized drawing, fill, and eraser tools in the first subtask, followed by text and sequence tools in the second. The third subtask allowed free use of both annotation methods. The study was executed in eight variations, every participant completing all the mentioned tasks but counter-balanced to ensure each order was conducted three times. Participants, spent an average of 60–90 minutes completing the study's practical component. The study maintained consistent laboratory conditions across multiple runs by using identical hardware and software configurations. Upon completion of the practical tasks, participants filled out multiple established questionnaires~\cite{ssqOriginal, ipq, sus, daq} to provide additional insights into their experiences and perspectives. This study received ethical approval from the institution's ethics board, underscoring our commitment to maintaining the highest ethical standards in research.

\begin{figure}[tb]
 \centering
 \includegraphics[width=0.9\columnwidth]{figures/lapCamera2.png}
 \caption{Illustration inserted laparoscopic camera, revealing the internal organs. An exemplary textual annotation is displayed at nose position.}
 \label{fig:lapcamera}
\end{figure}

\subsection{Hardware and Software Setup}
This section details the hardware and software that was used to implement the described method and to conduct the study. 

\subpart{Hardware} The Hardware used for this study was chosen so that both the VR and touchscreen setup could be rendered simultaneously on the same PC. Thus, the graphics card NVIDIA RTX 3090 was used together with the Intel Core i9-12900k and 32~GB of DDR5 RAM running at 4.4GHz. The VR equipment we selected was the Head-Mounted Display (HMD) Valve Index, running two LCDs at 144Hz with a per-eye resolution of 1440×1600 pixels.

\subpart{Software} The PC backing our study is running Windows 11. The methods described above were implemented in Unity 2021.3.14f1~\cite{unity}. Unity is mostly known as a game engine, but is also reasonably useful as a research tool for interface, interaction, or visualization studies~\cite{unityInterface1, unityInterface2, unityInterface3}. Unity's behavior can be customized using C\# scripts, enabling advanced functionality.


\subsection{Procedure}

The study was designed so that each participant would perform both perspectives that this method was designed for: educators and learners. This section will detail the procedure of the study.

\subpart{Introduction} The introduction was structured to introduce the participant to every tool they would use during the study. The scene prepared for this contains a human body provided by Nobutaka et al.~\cite{bodyparts3d} that was used for all depictions of anatomy throughout this study. First, the interface for educators was described by the researcher. After establishing the ways the camera could be moved, the participant was allowed to try it out. Then, the researcher demonstrated the use of the pen and the eraser on the face of a human body by drawing with blue color around one eye and red color on the lips. Part of that was erased directly afterward. Then, the researcher placed a text annotation on the nose with the text ''nose'' to demonstrate the use of the text annotation. A second one was placed on the knee. Then, the sequence tool was used to set the sequence index of the knee annotation to 2, placing it chronological behind the one on the nose.
%
For the second part of the introduction, the participants were equipped with the HMD. Once they reported seeing sharp images, they were allowed to move around and acclimate to the virtual environment. Once they were ready, they were told about the tools laying on a metal tray, how they could pick them up and drop them. They were asked to use all the regular annotation tools one after one. The introduction scene also contained one advanced tool, the laparoscopic camera used in the ablation scene described below, which had to be tried out.

\subsubsection{Educator} 

In the role of educators, the participants are tasked to transfer knowledge from text-book excerpts into scenes that were prepared for the study. Those excerpts were printed out so that the participants could hold them while using the touchscreen with their free hand. At the beginning of each subtask, the participants were given a small oral introduction into the specific topic and handed the excerpt to read through. Afterward, they were told which annotations they were allowed to place and shown the prepared scene. Each scene consisted of three parts which were labeled ''1.'', ''2.'' and ''3.'', later on regarded as the first, second or third part. There was no target of how much or what kind of information they were to transfer. Each subtask was finished as soon as the participant wished to do so.

In the following, each scene and the covered topic are explained in more detail. 

\subpart{Scar revision} Scar revision is the process of trying to remove or change the appearance of scars. One of the procedures to do this is to remove the scar by cutting it out and sewing the surrounding skin back together. This may be done to return flexibility when scars are near joints, or for aesthetic purposes. The excerpt handed to the participant details that and some alternative approaches to do scar revision. The three parts depict one skin piece with a scar and two identical skin pieces, where the scar is missing, and the skin has a visible hole. This scene was to be annotated only using the pen, eraser, and fill tools.

\subpart{Port} The scene covering the topic of medical ports was designed around restricting the participants to using the text and sequence tools. Medical ports are small gadgets which are implanted beneath the skin. A catheter connects them to a vein, allowing  access to the blood system of a patient without needing to puncture their veins. They are commonly used for cancer patients. The excerpt given to the participants detailed information on ports regarding their use, structure, and benefits. The first part of the scene shows a body where specific veins are visible through the skin. The most important vein for this procedure is highlighted. In the second part, the model of a port chamber is depicted. Finally, in the third, a human with a fully implanted port including the catheter is shown, with relevant parts visible through the skin.

\subpart{Anaphylaxis} The final task in the role of the educator is covering the topic of severe allergic reactions, their diagnosis and how to treat them in the field. The most commonly used drug for this is adrenaline, which gets injected into large muscles of the affected person. The excerpt covers this topic in depth, including symptoms and treatment in the field. The first part of the scene shows a human body, the second an enlarged version of an EpiPen and finally the third, the EpiPen stuck to the upper thigh. This time, the participants were allowed to make full use of all the tools.

\subsubsection{Learners}

In the part of the study where participants were exploring the systems designed for learners, they were wearing the supplied HMD. For this, they had to go through the courses described below. 

\subpart{Ablation} In this task, the learners were standing in a scene with a male body and a whiteboard. The whiteboard displays a simplified endoscopic ablation to remove a tumor. For this, two tools with needles, one with a camera and one with a hot tip, are inserted into the body. The green circles that are depicted on the whiteboard are drawn onto the body to aid the participant in locating the tumor. The participant has to insert the needle with the camera into the body. Once inserted, organs near the camera can be seen through the skin. The participant then has to search for the bright color-coded tumor and touch it with the laparoscopic tool they have holding in their other hand. Once that's done, they have finished this task. The annotations that were used to create this scene were like those created by the pen tool.

\subpart{Digestion} This scene holds the organs that are part of the human digestive tract. Each of the important parts had a text box associated with it, totaling 11 annotations. The sequence tool was used to make the text boxes appear in the order in which food would pass through them. Participants were tasked to read through every text box. For this, they also had to move around because the text boxes were occluded by the organs from some view points.

\subpart{Teeth inspection} The final task in the role of learners was concerned with teeth. They were presented with a skill with an open mouth. The teeth are colored according to their quadrant. The participants learned the Federation Dentaire Internationale (FDI) dental notation~\cite{isoTeeth}, which gives each tooth a simple ID from text boxes. Then they were tasked to mark a specific tooth with by coloring it yellow. Lastly, they should use a special tool to remove a tooth given by its ID by touching it. This scene combined all annotations we examined in this paper.

\subsection{Participants}

Of our 24 participants, 15 were male and 8 were female. The youngest participant was 23, the oldest 38 with a median age of 25. All participants had at least a High school diploma, with 4 having or aspiring to have a bachelors degree, 10 a masters-level degree and 2 PhDs. Out of the 6 people not currently being a student, 2 work as software developers and 3 as researchers.

The previous experience in the areas our research is touching is very diverse. 11 of our participants have none to very little experience with 3D environments like simulations or video games. Of the other 13, one has 3–4 years of experience, while the rest have 5+ years.
%
Our participants were less experienced in general with VR, with fifteen having reported no true experience, three having less than 1 year, three in the range of 1–2 years, one having 3-4 and only two people having at least 5 years of experience. Nobody reported using VR regularly. 
%
Except for 2 people, everyone was interacting with PCs every day. Nine participants reported playing video games every day, and six others at least a few times a month.

\subsection{Measures}
\label{sec:measures}
\subsubsection{Objective Measures}

During the tasks, we created a log where the most important events are appended with a timestamp. An entry was added each time:

\begin{itemize}
    \item a Unity scene changes, which occurred when switching to the next task,
    \item the learner finishes a task,
    \item the educator switches to a different interaction mode,
    \item the learner is using a VR controller to pick something up,
    \item or the educator is interacting with the height slider (at most once per second).
\end{itemize}


These timestamped events allow us to reconstruct how long each participant took for which tasks, what tools they used and how long they used them. To capture the drawn annotations that were created, we created photos of the scene from multiple angles when a task was finished or before a new scene was entered.




\subsubsection{Subjective Measures}
This study utilizes multiple questionnaires to measure how users reacted to our presented tools. First, to measure symptoms of sickness, the widely used \textbf{Simulator Sickness Questionnaire (SSQ)}~\cite{ssqOriginal} is included. The SSQ asks the user to rate their sickness by ranking their current feeling regarding 16 different criteria (e.g., headache, fatigue, nausea, or dizziness) on a scale of 1 to 4. Using that, 3 scales can be derived that relate to nausea (N), oculomotor disturbance (O) and disorientation (D). Kennedy et al.~\cite{ssqOriginal} report multiple thresholds with $>20$ being the most severe one, relating to a bad simulator~\cite{ssqFix}. Though, as noted by Bimberg et al.~\cite{ssqFix}, we applied the corrected formula for the final score and used the common approach of using a pre- and post-study questionnaire. We wanted to assess how the SSQ scores differed between the usage of touchscreens and the HMD, so we separated the post-study SSQ into two parts, where participants had to rank their feelings for each of the device types.
%
Even when somebody does not actually experience symptoms of simulator sickness, they still may feel other types of discomfort regarding specific tools. To quantize this, we employ the \textbf{Device Assessment Questionnaire (DAQ)}~\cite{daq} in which each volunteer had to rate the required force, smoothness, mental and physical effort as well as various bodily fatigues (totaling thirteen properties) when using each of the eleven tools. This resulted in 143 ratings the users had to perform. The scale of the DAQ goes from 1 to 5, while the interpretation was inconsistent between each of the questions. For some questions, 3 is the ideal case, where for some it is 5 as seen in the header of \autoref{tab:daq}.
%
We also want to measure presence, as that is a big factor for learning experiences ~\cite{presenceLearning}. This was gauged by adding the \textbf{Igroup Presence Questionnaire (IPQ)}~\cite{ipq} to our post-study questionnaire. The fourteen questions included try to measure how much the simulated world was recognized and experienced as reality. They also ask the participant to reflect on how much of the real world was still perceived. The questionnaire presents statements and asks the responded to rate their agreement from 0 through 7. The questions were again separated between touchscreen and VR. 



To assess the usability of our tools more directly, users had to answer the \textbf{System Usability Scale (SUS)}~\cite{sus}. Each of the 10 statements contained tried to address different important factors for real-world usability, which users had to rate their agreement with. To consider our tools independently, each of the 10 statements had to be rated for each of the 11 tools, resulting in 110 ratings. The rating was conducted by assigning each of the statements a score between 1 and 5, where 1 represents minimal agreement and 5 maximum agreement with the statement. The usability can then be calculated according to SUS and falls between 0 and 100.

On the last page, users had to state which of the 6 subtasks was their favorite. Finally, users had to evaluate the perceived usefulness of the annotations, regardless of their current implementation.
\section{Results}
We first describe communication patterns within the full chronological context of the game in \textit{League of Legends (LoL)}, separated into four sections based on changing coordination dynamics. Based on this context, we identify core factors players assess to decide when to participate in communication with other teammates. Afterward, we discuss how communication shapes player perceptions toward their teammates, showing player's wariness towards players actively engaging in communication. 

\subsection{Communication Patterns in Context}

We discuss the communication patterns among teammates within the game. We organize the data into chronological phases of the game for a structured analysis of how the context shapes communication patterns. 

\subsubsection{Pre-game stage}
Before gameplay begins, team communication opens with \textit{team drafting}, where players are assigned roles (Top, Mid, Bot, Support, or Jungle) and take turns picking or banning champions. In Solo Ranked mode, roles are pre-assigned based on player preferences selected before queueing. Once teams are set, all players enter \textit{champion select} stage, alternating champion picks and banning up to five champions per team. During this stage, communication is limited to text chat. The usernames are anonymized (i.e., replacing the name with aliases) to prevent queue dodging by checking third-party stats sites such as OP.GG\footnote{https://www.op.gg/}, leaving the chat as the only option to inform individual strengths and preferences. 

Team composition in \textit{LoL} is crucial to the strategy and outcome of the game~\cite{ong2015player}, setting the basis for future interactions. Most participants acknowledged the importance of balanced and synergistic team composition, especially as players move into higher ranks where team coordination outweighs individual excellence. Yet, we observed a distinct lack of verbal communication between the members during this period across all ranks. Participants attributed the lack of willingness to initiate a conversation on the dangers of starting the game on a bad footing. They prioritized ``not creating friction'' during this stage as negative impressions can propagate throughout the game. Some participants attempted communication to reduce such friction, such as P14, who stated,``\textit{If I had the time, I wanted to say that I will be banning [this Champion], just in case a player on my team wanted to play them.}'' However, several participants viewed any communication during the pre-game phase with wariness, as dissatisfaction or conflict at this step portended negative interactions between players in the game (P3, P9, P15). Thus, even when participants expressed doubt about other teammates' unconventional or non-meta champion picks, they refrained from entering into discourse. This contrasts with findings by Kou and Gui~\cite{kou2014}, which showed players attempt to maintain a harmonious and constructive atmosphere through greetings and introductions.

Another emergent code of the reason for not engaging in communication in the pre-game stage stems from different purposes of playing the game (P1, P5, P13, P16, P17). Despite being in ranked mode, which is more prone to increased competitiveness and effort, participants showed differing goals and levels of interest in winning the game. Several players stated that they had previously exerted great mental load in coordinating synergistic plays, but stopped as they gave less importance to winning at all costs (``\textit{I don't really play to win. I play \textit{LoL} to relieve stress, so I don't engage in chat.}'', P5). These players saw verbal communication with the goal of coordination as an unnecessary or even cumbersome component of the pre-game stage.


\subsubsection{Structured phase}
In many MOBAs, including \textit{LoL}, the early stages of the game play out in a formulaic manner: players join their lanes (Top, Mid, and Bot/Support), defeat minions to gain gold, buy items towards certain ``builds'', kill or assist in early objectives (Jungle), and battle counterparts in their respective lanes. Participants at this stage expressed that most players possessed tacit knowledge of what must be done, such as knowing when to aid their Jungle to capture a jungle monster, choosing the opportune moments to leave their lanes, or positioning wards (i.e., a deployable unit which provides a vision of the surrounding area) at the ideal placements. The participants assumed each player knew their ``role'' to fulfill, often comparing it to ``doing their share'' (P1, P3, P7, P19). In line with this belief, players rarely initiated preemptive or proactive verbal communication for strategic or social purposes at the early stage. 

Pings, on the other hand, constantly permeated the game. At this stage, players used ping to provide information relevant to others from their position, such as letting others know if an enemy went missing from their lane. As the players are largely separated and independent from one another, pings (coupled with the minimap and scoreboard) served as the primary channel for maintaining context over the game rather than as warnings or direct guidance to the players. For other non-verbal gestures, while objective votes would occasionally appear, they were rarely answered. Instead, relevant players near the objective would place pings or move toward it to help out their teammates.

Participants viewed the structured phase as a routine, but uncertain period of the game where the pendulum could swing in either team's favor. Players --- especially Jungles who roam the board looking for opportunities to ambush the enemy team in lanes (``gank'') --- sometimes felt hesitant to make calls and demands at this stage since ``\textit{[they] could make a call, but if I fail, they'll start blaming my decisions down the line.}'' (P7) But at this stage, participants believed that they held personal agency over the final game outcome. P1 and P6 stated that they entered the game with the mindset that only they had to succeed regardless of the performance of their teammates. This belief was reflected in their chatting behavior, where players prioritized focusing on their circumstances over the team's (``\textit{I mute the chat so that I don't get swayed by the team, as I can win the game if I do well.}'', P9).


\subsubsection{Group engagement phase}
As the game enters its middle phase, it provides opportunities for more diverse decision-making. Players may swap lanes, seize or trade crucial objectives, and fight in large battles involving multiple champions. At this point, teams typically have a clear outlook on which players and team have the advantage, requiring more team-driven decisions to maintain or overcome their current standing. Thus, players used verbal communication to discuss more complicated tactics that could not be effectively conveyed through pings.

But more often than not, chat messages became judgment-based. As enemy engagement with larger groups occurred more frequently, the availability for chatting would come after death, which led to comments on past actions rather than future choices. Additionally, the respawn timer for deaths becomes longer as the game progresses, providing more time to observe other players than in earlier phases. This gave players more opportunities to express dissatisfaction specifically towards certain plays, such as placing Enemy Missing pings on the map where other teammates are located to bring attention to their questionable play.

This stage also gave much more exposure of each other to the allies as the team would gather at a single point, giving way to greater scrutiny by their teammates. Repeated or critical mistakes put participants on edge, as they braced for criticism from their teammates. They expressed relief or surprise when the chat remained silent or civil, with P8 stating ``\textit{I messed up there. No one is saying anything, thankfully.}''


\subsubsection{Point of no return}
Meanwhile, verbal communication flowed out when the game had a clear trajectory to the end. Previous research has shown that both toxic and non-toxic communication skyrockets near the end of the game~\cite{kwak2015linguistic} when the players have determined the game outcome with certainty. We saw that this phase opened up both positive and negative sides of communication for guaranteed win and loss, respectively. The winning team would compliment and cheer each other through chat messages and emotes, while the losing side devolved into arguments and calling out. The communication at this stage was driven by emotion, showing excitement or venting frustration.


\subsection{Communication Assessment Process}

We describe the factors that users mainly focus on to assess when or when not to involve themselves in communication with their teammates. 

\subsubsection{Calculating communication cost}
One of the most proximate factors behind when communication is performed is the limited action economy of the game. In \textit{LoL} and other MOBAs, players can rarely afford time to type out messages due to the fast-paced nature of the game. In time-sensitive scenarios, the time pressure makes communication particularly costly. It is therefore unsurprising that much of the communication occurs after major events (e.g., battles and objective hunting), as players are given more downtime while waiting for teammates or enemies to respawn or regroup.

For periods where players were still actively involved in gameplay, the players made conscious decisions on choosing which communication media to use based on the perceived action availability and the importance of communicating the message. Players relied on pings for non-critical indications, believing that the mutual understanding of the game would get the message across. However, many players recognized that pings were prone to be missed, ignored, or misinterpreted by their allies (P2, P9, P16, P17, P20). Subsequently, participants typed out information considered to be too important to the situation to be misunderstood or missed by other players even if it caused delays in their gameplay (P10, P11, P14). Simultaneously, the priority of importance constantly shifted --- we observed multiple times participants start to type, but stop to react to an ongoing play, only to never send out their message.

\subsubsection{Evaluating relevance and responsiveness}
When the brief window of communication opportunity is missed, players are unlikely to ever send out that information. In \textit{LoL}, situations can change within seconds and certain communication media cannot keep up with the changing state of the game. For example, almost all study participants did not participate in votes for objectives. Among the tens of objective votes initiated among all the games in this study, no objective vote saw more than three votes, frequently being left with no vote beyond the player who initiated the vote. Some players, when asked why they did not participate, stated that the votes they made often became irrelevant as the game state had changed during the time it took to vote (P2, P11). Other players also discussed how information conveyed through communication can get outdated fast (P1, P8, P9). 

\begin{quote}
I can't always follow through with what I say [in the chat] since the game is really dynamic. My teammates don't understand such situations, so I tend to not chat proactively. - P9
\end{quote}

Thus, some players instead preferred to react through direct action (P8, P10, P11, P16, P20). P10 stated, ``\textit{I think it's enough to show through action rather than [using objective voting]. I can look out for how the player reacts when I request something from them.}''

On the other hand, such action-based responses left the player to assess whether and how the communication was received. P10 stated that they tried to predict whether a player understood their ping direction by how they moved, but it was hard to interpret their intent: ``\textit{members sometimes seem to move towards me but then turn around, and sometimes they even ping back but don't come.}''. P16 discussed how they weren't sure whether the ping was received, but performed it anyway since it felt helpful.

Similarly, participation in surrender votes (or lack thereof) carried different intent by the player. During most of the games that ended in a loss, one or more surrender votes were called by the participant's team. However, only two surrender votes achieved four or more players' participation. However, the reasons why a player chose to not participate varied. Some had decided to wait and see how other teammates voted, which may have paradoxically led many members to not participate in the vote (P4, P9). Meanwhile, others didn't reply as they didn't think the vote was actually calling for a response: P13 stated, ``\textit{I didn't vote because they were just showing their anger. It's just a member venting through a surrender vote that they're not doing well.}''

\subsubsection{Balancing information access and psychological safety}
While recognizing that communication would be useful or even necessary in certain situations, participants also put their psychological safety first over information access. Some players, worn down by the normalization of toxic communication such as flaming, muted the chat (P1, P9).

Many participants expressed the sentiment of ``protecting [their] mentality'', describing how certain communication harmed their psychological well-being. This communication did not always refer to negative communication; P9 often muted players who gave commands as they did not want to be ``swept up'' by others' play-related judgments. This separation even extended to other more widely considered essential communication forms, such as pings. Even after acknowledging that pings were vital and useful to the game, P9 went as far as muting the ping of the support player in the same lane after they sent a barrage of Enemy Missing pings that signified aggression and criticism. 

Additionally, the abundance and high frequency of communication also strained the limited mental capacity of the players. Many players, when asked why they had not replied to an objective vote or other chat messages, stated that they simply did not notice them among other events happening (P1, P2, P3, P9, P12, P15, P18, P20, P21). The information overload caused stress and became distracting to players.

\subsubsection{Reducing potential friction}
As demonstrated in the pre-game stage of the game, players sometimes used communication to minimize friction between their teammates. Some participants sacrificed time to apologize to other players when they believed themselves to be at fault. When asked why, P12 replied, ``\textit{There are too many people who don't come to help gank if I don't apologize.}''. Similarly, P5 sacrificed time typing in an apology after a teammate had died despite still being in the middle of a fight as they didn't wish to give the other player a reason to start an attack.

However, some noted that silence is sometimes the best answer to a negative situation. P4, after dying to the enemy, put into chat ``Fighting!'' (roughly meaning, ``We can do it!''). They stated ``\textit{I don't know why I do it... it probably angers [my teammates] more.
}'' They also stated that ``\textit{for certain people, talking in the chat only spurs them more. You just have to let them be.}'' Other players shared similar sentiments that being quiet and dedicating focus to the game was a better choice (P1, P11, P14).

For female players, the fear of gender-based harassment shaped their communication patterns. While \textit{LoL} does not provide any demographic information of a player to other players, almost all female participants noted experiences of receiving derogatory remarks or doubts about their abilities based on other players' assumptions of their gender, a trend frequently seen in male-dominated online gaming cultures~\cite{fox2016women, norris2004gender, mclean2019female}. They noted that the players were able to correctly guess their gender when the participant's role and champion fit into the preconceived notions of what women ``tended to play'' (i.e., female-identifying support champions, such as Lulu) or their username ``seemed feminine'' (P18, P19, P20, P21, P22). This led to certain players adopting tactics that signaled male-like behavior, such as changing their speech style to be more gender-neutral or male-like (P19, P21) and changing their username to sound more gender-neutral. Cote describes similar instances of ``camouflaging gender'' as one of five main strategies for women coping with harassment~\cite{cote2017coping}. However, some players opted to keep playing their preferred character or maintaining their username even if it signaled their gender, such as P21 who expressed, ``\textit{I cherish and feel attached to my username, so I don’t want to change it just because of [harassment and inappropriate comments].}'' These players valued self-expression and identity even at the risk of increased risk to unpleasant communication experiences.


\subsubsection{Forming performance-based hierachy}
Naturally formed leadership has often been observed in other works on \textit{LoL} teams~\cite{kou2014}. Kim et al. showed that more hierarchy in in-game decision-making led to higher collective intelligence~\cite{kim2017}. While they used ``hierarchy'' to mean varying amounts of communication throughout the game, we observed that the hierarchy extends further to performance-based hierarchy, where teammates in more advantageous positions are given greater weight when communicating with other players. Players actively chose to refrain from suggesting strategic plans when they were ``holding down the team'', recognizing that they held less power and trust among the team members (P8, P10, P12, P14, P22). The player who was losing against the enemy team was viewed as having no ``right'' to lead the team, which was reserved for well-performing players.


\subsubsection{Enforcing norms and habits}
One of the most common answers to why players performed certain communication actions, especially non-verbal actions such as pings and emotes, was ``a force of habit'' (P6, P7, P8, P9, P10, P12, P17). Players formed learned practices of using communication channels at certain points by observing other players exhibit the same behaviors. This promoted, for example, replying to an emote sent by the teammate with their own or pinging readied skills and items to emphasize relevant information for other players throughout the game. 

On the other hand, this meant that players were averse to communication patterns outside of the norm --- participants stated that they had a hard time adapting to new forms of communication, seeing no immediate benefit or impact from using them (P1, P8, P14, P13, P15, P17). Most egregiously, the recently introduced objective pings were largely viewed to be awkward to use and unnecessary (P1, P4, P8, P12).


\subsection{Impact of Communication Assessment}
We describe how the communication patterns and assessment of the players impact the individual players' perspectives on team dynamics.

\subsubsection{Relationship between trust and communication frequency}
Most participants saw value in constant and well-informed communication but with an important distinction: verbal communication with strangers rarely ended well. Players largely recognized frequent verbal communication to burgeon conflict, regardless of the message within. Even when players understood the helpful intent behind positive messages from the players, they compared actively talking players to be possible bad actors who were likely to exhibit toxic behaviors when the game turned against them. (P1, P4, P8, P12, P14)

\begin{quote}
I need to make sure to not disturb Twisted Fate. I saw him start to flame. It's not because I don't want to hear more criticism. I know these types. The more I react and chat with them, the more deviant they will become. - P4  
\end{quote}

Similarly, P19 lamented that players used to socialize more in the chat during the pre-game phase to build a fun and prosocial environment, noting a memorable example of encouraging each other to do well on their academic exams, but noted that such prosocial behavior has become much rarer during the recent seasons. They noted that there are inevitably players ``who take it negatively'' and thus stopped proactively typing non-game related messages in the chat.

Ultimately, players desired assurance and trust of player commitment. The participants trusted actions more than words to prove that the player remained dedicated to the game. Both P10 and P17 pointed out that it was easy to tell who was still ``in the game'' and motivated to try their best and that ``staying on the keyboard'' likely meant that they weren't invested or focused on the game. Players viewed such commitment to be the most important aspect of a ``good'' teammate, sometimes even more than their skill or performance (P9, P14). It is interesting to note that unlike what previous literature may suggest~\cite{marlow2018}, players' averseness to talkative teammates had less to do with the cognitive overload or distraction caused by the frequent communication, but rather due to the threats of future team breakdown. This view in turn also affected how players decided to communicate or not, as they believed that players would not take their suggestions or comments in a positive light. 


\subsubsection{Perception of player commitment and fortitude}

Communication also acted as a mirror of their teammates' mental fortitude. A number of players mentioned how they valued a resilient mindset in their teammates playing the game, referring to players who remained committed to the game until the very end. They saw players who provoked or complained to teammates as ``having a weak mentality'' who had been altered by the bad outcomes of the game to act in an unhelpful manner towards the team through their communication. The communication actions of the teammate informed the participants of how steadfast their teammate remained in disadvantageous situations.  

\begin{quote}
It's not like I constantly reply in the chat or anything, but I pay attention [to the chat] to grasp the overall atmosphere of the team. If the team doesn't collaborate well then we lose, so I try to have a rough understanding of the mentality of the other players. - P13
\end{quote}

There were also instances of communication that helped players maintain a positive view of their teammates. For example, P11 mentioned near the beginning of the game, ``\textit{Looking at the chat, Varus player has strong mentality [for being so positive]. There were lots of points [in his support's] plays that he could have criticized.}'' Unfortunately, this view quickly soured when the Varus player devolved into criticism later in the late game phase where the Varus player started criticizing the support and other players. P11 then noted that the Varus player seemed to merely be ``bearing through the game''.
\section{Discussion and Conclusion}
\label{sec:discussion}


\textbf{Conclusion.} In this paper, we propose LRM to utilize diffusion models for step-level reward modeling, based on the insights that diffusion models possess text-image alignment abilities and can perceive noisy latent images across different timesteps. To facilitate the training of LRM, the MPCF strategy is introduced to address the inconsistent preference issue in LRM's training data. We further propose LPO, a method that employs LRM for step-level preference optimization, operating entirely within the latent space. LPO not only significantly reduces training time but also delivers remarkable performance improvements across various evaluation dimensions, highlighting the effectiveness of employing the diffusion model itself to guide its preference optimization. We hope our findings can open new avenues for research in preference optimization for diffusion models and contribute to advancing the field of visual generation.

\textbf{Limitations and Future Work.} (1) The experiments in this work are conducted on UNet-based models and the DDPM scheduling method. Further research is needed to adapt these findings to larger DiT-based models \cite{sd3} and flow matching methods \cite{flow_match}. (2) The Pick-a-Pic dataset mainly contains images generated by SD1.5 and SDXL, which generally exhibit low image quality. Introducing higher-quality images is expected to enhance the generalization of the LRM. (3) As a step-level reward model, the LRM can be easily applied to reward fine-tuning methods \cite{alignprop, draft}, avoiding lengthy inference chain backpropagation and significantly accelerating the training speed. (4) The LRM can also extend the best-of-N approach to a step-level version, enabling exploration and selection at each step of image generation, thereby achieving inference-time optimization similar to GPT-o1 \cite{gpt_o1}.
\section{Conclusion}
We present live monitoring and mid-run interventions for multi-agent systems. We demonstrate that monitors based on simple statistical measures can effectively predict future agent failures, and these failures can be prevented by restarting the communication channel. Experiments across multiple environments and models show consistent gains of up to 17.4\%-20\% in system performance, with an addition in inference-time compute.
Our work also introduces \ourenv{}, a new environment for studying multi-agent cooperation.


\bibliographystyle{unsrtnat}
\bibliography{references}  %%% Uncomment this line and comment out the ``thebibliography'' section below to use the external .bib file (using bibtex) .


%%% Uncomment this section and comment out the \bibliography{references} line above to use inline references.
% \begin{thebibliography}{1}

% 	\bibitem{kour2014real}
% 	George Kour and Raid Saabne.
% 	\newblock Real-time segmentation of on-line handwritten arabic script.
% 	\newblock In {\em Frontiers in Handwriting Recognition (ICFHR), 2014 14th
% 			International Conference on}, pages 417--422. IEEE, 2014.

% 	\bibitem{kour2014fast}
% 	George Kour and Raid Saabne.
% 	\newblock Fast classification of handwritten on-line arabic characters.
% 	\newblock In {\em Soft Computing and Pattern Recognition (SoCPaR), 2014 6th
% 			International Conference of}, pages 312--318. IEEE, 2014.

% 	\bibitem{hadash2018estimate}
% 	Guy Hadash, Einat Kermany, Boaz Carmeli, Ofer Lavi, George Kour, and Alon
% 	Jacovi.
% 	\newblock Estimate and replace: A novel approach to integrating deep neural
% 	networks with existing applications.
% 	\newblock {\em arXiv preprint arXiv:1804.09028}, 2018.

% \end{thebibliography}


\end{document}
