\documentclass{article}



\usepackage{arxiv}

\usepackage[utf8]{inputenc} % allow utf-8 input
\usepackage[T1]{fontenc}    % use 8-bit T1 fonts
\usepackage{hyperref}       % hyperlinks
\usepackage{url}            % simple URL typesetting
\usepackage{booktabs}       % professional-quality tables
\usepackage{amsfonts}       % blackboard math symbols
\usepackage{nicefrac}       % compact symbols for 1/2, etc.
\usepackage{microtype}      % microtypography
\usepackage{lipsum}		% Can be removed after putting your text content
\usepackage{graphicx}
\usepackage{natbib}
\usepackage{doi}

\usepackage{url}
\usepackage[table, svgnames]{xcolor}
\definecolor{lightblue}{RGB}{173,216,230}
\usepackage{adjustbox}
\newcommand{\subpart}[1]{\vspace{2pt}\noindent\textbf{#1}\noindent}
\usepackage{makecell}
\usepackage{tabu}
\usepackage{booktabs}

\usepackage{hyperref}


\title{Enabling seamless creation of annotated spaces: enhancing learning in VR environments}

%\date{September 9, 1985}	% Here you can change the date presented in the paper title
%\date{} 					% Or removing it

\author{ 
    Maximilian Enderling\\
    Friedrich Schiller University Jena
	\And
    Jan Hombeck\\
    Friedrich Schiller University Jena
    \And
    Kai Lawonn\\
    Friedrich Schiller University Jena
}

% Uncomment to remove the date
%\date{}

% Uncomment to override  the `A preprint' in the header
\renewcommand{\headeright}{}
\renewcommand{\undertitle}{}
\renewcommand{\shorttitle}{Enhancing learning in VR environments}

%%% Add PDF metadata to help others organize their library
%%% Once the PDF is generated, you can check the metadata with
%%% $ pdfinfo template.pdf
\hypersetup{
pdftitle={A template for the arxiv style},
pdfsubject={q-bio.NC, q-bio.QM},
pdfkeywords={First keyword, Second keyword, More},
}

\begin{document}
\maketitle

\begin{abstract}
We present an approach to evaluate the efficacy of annotations in augmenting learning environments in the context of Virtual Reality. Our study extends previous work highlighting the benefits of learning based in virtual reality and introduces a method to facilitate asynchronous collaboration between educators and students. These two distinct perspectives fulfill special roles: educators aim to convey information, which learners should get familiarized. Educators are empowered to annotate static scenes on large touchscreens to supplement information. Subsequently, learners explore those annotated scenes in virtual reality.
%
To assess the comparative ease and usability of creating text and pen annotations, we conducted a user study with 24 participants, which assumed both roles of learners and teachers. Educators annotated static courses using provided textbook excerpts, interfacing through an 86-inch touchscreen. Learners navigated pre-designed educational courses in virtual reality to evaluate the practicality of annotations.
%
The utility of annotations in virtual reality garnered high ratings. Users encountered issues with the touch interface implementation and rated it with a low intuitivity. Despite this, our study underscores the significant benefits of annotations, particularly for learners. This research offers valuable insights into annotation enriched learning, emphasizing its potential to enhance students' information retention and comprehension.
\end{abstract}


% keywords can be removed
\keywords{Computers and Graphics\and Virtual Reality \and Touchscreen \and Annotations \and Education}


\begin{figure}[ht]
    \centering
    \includegraphics[width=0.8\linewidth]{graphs/greater_than_naive.pdf}
    \vspace{0.5cm}
    \includegraphics[width=0.8\linewidth]{graphs/p1_bottom.png}
    \vspace{-5pt}
    \caption{\textcolor{positional}{Positional} vs.\ \textcolor{nonpositional}{non-positional} circuits. In a \textcolor{nonpositional}{non-positional} circuit, the same edges must be included at all positions. A \textcolor{positional}{positional} circuit can distinguish between the same edge at different positions. This specificity yields better trade-offs between circuit size and faithfulness. It can also increase both precision and recall.}
    \label{fig:p1}
    \vspace{-5pt}
\end{figure}

\section{Introduction}

\looseness=-1
A primary goal of interpretability research is to characterize the internal mechanisms in language models (LMs) and other NLP models. 
A core approach in this area is \textbf{circuit discovery}---identifying the minimal subgraph within the model's computation graph that performs a specific task \citep{olah2021framework,olah-mech}.
Typically, the nodes of a circuit represent model components (e.g., attention heads, neurons, or layers).
While manual circuit discovery methods can yield position-specific insights \citep{wanginterpretability,goldowskydill2023localizingmodelbehaviorpath}, \emph{automatic methods often overlook positional information}, treating components as uniformly relevant across all input token positions \citep{conmytowards,syed2023attribution}. 
For instance, if an attention head is included in a circuit, it is assumed to contribute equally to the computation for every position in the input sequence.
The assumption that circuits are position-invariant ignores the fact that different positions often require distinct computations.
By ignoring positions, current methods limit their ability to capture mechanisms that operate across positions, such as interactions between attention heads across positions.

In this study, we start by demonstrating that positional agnosticism is a significant limitation (\S\ref{sec:motivating}). Then, to address these limitations, we introduce a new approach: position-aware edge attribution patching (PEAP; \S\ref{sec:full_circ_discovery}; Figure~\ref{fig:p1}). Current approaches  assume that if an edge is in a circuit, then the same edge will be in the circuit at all positions, thus leading to low precision. It is also assumed that an edge's importance should be aggregated across positions before deciding whether it should be included in the circuit; this can lead to cancellation effects, and thus low recall. PEAP instead allows us to compute the importance of cross-positional edges, and separately evaluates edge importance at each position. We show that this leads to smaller and more accurate circuits; see Figure~\ref{fig:p1}.

Incorporating positional information into circuit discovery is straightforward when inputs have the same length and structure across examples.

However, realistic datasets are not nearly this templatic.
How, then, can we incorporate positional information into automatic circuit discovery?
To address this challenge, we propose \textbf{schemas} (\S\ref{sec:schema}). 
Schemas assign semantic labels to spans of tokens, enabling information aggregation across examples even when the spans differ in length.

For example, in the input ``The \textcolor{positional}{war} lasted from 1453 to 14\underline{\hspace{1em}},'' the span ``\textcolor{positional}{war}'' could be labeled as ``\emph{Subject}''.
This enables handling spans with varying lengths: the phrase ``\textcolor{positional}{Black Plague}'' in another example can be treated as a single positional span with the same role as ``\textcolor{positional}{war}''.
In experiments with two LMs and three tasks, we find that circuits discovered using schemas achieve a better trade-off between circuit size and faithfulness to the model's behavior than position-agnostic circuits.
Importantly, position-aware circuits offer a more precise representation of the underlying mechanisms, providing a more concise foundation for mechanistic explanations.

We also present a fully automated pipeline for schema generation and application (\S\ref{sec:schema-generation}) using large language models (LLMs). 
We evaluate the quality of the generated schemas and their utility in discovering position-aware circuits (\S\ref{sec:schema-eval}).
Notably, circuits derived using automatically generated and applied schemas achieve comparable faithfulness scores to circuits discovered with human-designed and manually applied schemas.

We summarize our contributions as follows:
\begin{itemize}[noitemsep,leftmargin=*,topsep=1pt,parsep=1pt]
    \item Introduce a position-aware circuit discovery method, which obtains better faithfulness than position-agnostic discovery.  
    \item Introduce dataset schemas,  facilitating positional circuit discovery in more naturalistic settings. 
    \item Develop an automated schema generation and application pipeline with LLMs, yielding schemas that are comparable to manually-annotated ones.
\end{itemize}

\section{Related Work}
\subsection{Touch Interfaces}
Touch interfaces utilize the flexibility of touchscreens to offer users an intuitive means of interaction with digital content, enhancing user engagement and streamlining various tasks in a wide range of applications. Current applications make use of two techniques~\cite{menuTouch, menuTouch2, touchReview}. The first technique is emulating the traditional interaction flow when using a mouse and a keyboard, resulting in buttons and menus reacting to touch. But touchscreens also enable the usage of another way of interfacing with the computer in the form of gestures~\cite{gesturalInterfaces}. This allows a more direct interaction with the environment and is shown to be less distracting for users~\cite{menuTouch2}.
%
There are many attempts to solve the problem of intuitive and capable movement-systems for touch interfaces. A classic attempt is to emulate physical controllers as graphical user interface (GUI). Especially, games using a first-person camera use one or two virtual joysticks~\cite{multiTouch3D}.  
%
The MagicCube~\cite{magicCube} one-handed technique is designed to reduce the screen area that is occluded by fingers. A GUI element is displayed, which allows movement with 5 Degrees of freedom (DOF). The GUI element shows a cube with 3 visible sides, where depending on which side a user drags from, different actions are performed such as translation, rotation, or selection of items in the environment.
%
Marchal et al.~\cite{multiTouch3D} summarized multiple methods and proposed a solution which is not reliant on GUI elements. It proposes a method to move back and forth and rotate left and right using one finger. when two fingers are used, a classifier determines the main component of the action. If the user rotates their fingers, the camera rotates around a point in the scene. By pinching, the camera's field of view will get regulated and by panning both fingers together, the camera can rotate both horizontally and vertically.


\subsection{Learning Systems}

\subsubsection{Touchscreens}

Touchscreens are conventional displays that are equipped with sensors that can detect whether and where the user comes into contact with them. One significant advantage is their intuitivity~\cite{touchEvaluation}: when a user want's to press a button on the screen, instead of using the mouse to move the cursor above it, they can simply tap the screen at that position. This ability results in much research into designing interfaces with them~\cite{touchM3, touchDesign}, some aimed at groups like old adults~\cite{touchInterfacesOld, touchInterfacesOld2, touchInterfacesOld3} or visually impaired individuals~\cite{touchVisualImpairment, touchVisualImpairment2, touchVisualImpairment3}.

The ubiquity of touchscreens in the form of mobile tablets and smartphones leads to many young children having access to this technology~\cite{youngChildrenTouchscreens}. There is research concerning the use of these devices for early education~\cite{touchscreenChildrenLearning, touchBenefitsDamagesKids}. They are especially beneficial to children because they have very little other contact with technology and are not accustomed to common computer interfaces like mice and keyboards.

\subsubsection{Virtual Reality}

Learning systems where learners are immersed in a VR environment received some research, especially in specialized areas where the study of real-world counterparts is expensive or difficult, but spatial information is still critical. The approach Saffo et al. ~\cite{desktopVRCombination} take is similar to the one that is described in this paper, with the contrast that they focus on interactions between VR and non-touch desktop environments. One of the fields where learning in VR is very well studied is surgical training~\cite{vrMedicalTraining, vrMedicalTraining2, vrMedicalTraining3,hombeck2024voice,laparoscopyInstrumentVR} where students are reported to learn faster~\cite{vrFasterLearning} and achieve better results~\cite{vrMedicalBetter}. Moreover, prior research has indicated that spatial and distance estimation tend to exhibit greater accuracy in VR environments when compared to desktop applications ~\cite{unityInterface2,hombeck2022distance,hombeck2019evaluation}.
%
Some research delves more generally into education and creates suggestions for the general architecture of educational applications. Co-assemble~\cite{coAssemble} proposes to separate learning environments into three scenarios. In single user mode, learners operate alone in an environment, having more freedom and feel less pressure. In the medium-sized setup, classes are separated into groups where learners cooperate in a shared space. Finally, the class mode is the most similar to traditional school setups; every learner is in the same space as the educator, mostly restricted to observing a live presentation. Educators may choose to allow specific learners to present things to the whole class. Each scenario has its unique benefits, so providing them all is important.

\subsection{Annotations}

Annotations provide context to existing information. Most research regarding annotations placed in 3D spaces is concerned with enabling remote collaboration or assistance~\cite{vuforiaAnnotations, annotations1, annotations2}. A common scenario is where an on-site technician requires assistance from experts. Instead of sharing individual photos, current research is trying to recreate the 3D environment of the on-site technician so that the expert can better grasp the spatial context.
%
Marques et al.~\cite{vuforiaAnnotations} evaluated different types of annotations regarding their usefulness to on-site technicians and remote experts. They were concerned with asynchronous assistance in three steps: first, the on-site user captures the environment and annotates it. Then the remote user inspects it, placing further annotations to detail what an intervention should look like. Thirdly, the on-site technician performs that intervention by following the provided annotations. Though they did not name all tested annotations, in their results they name the ones that were rated most useful. These annotations were (most useful first): first drawing, for its versatility, second notes, for their ability to add richer context, third notifications, to alert to information updates. Finally, they also added the possibility to add temporal sorting to other annotations, which was appreciated for its use in environments with many annotations. They noted, that editing existing annotations were an important aspect to potentially reducing workload.
\section{Proposed Method}
\label{sec:method}

In this section, we introduce our homotopy-based multi-objective framework for face parsing and its integration with both \textbf{GAN-based} and \textbf{diffusion-based} face editing models. We outline the problem formulation, dataset preparation, model architecture, training strategy, and evaluation pipeline, emphasizing \textbf{fairness}, \textbf{robustness}, and \textbf{semantic alignment}.

\subsection{Problem Formulation}
\label{subsec:problem_formulation}

We define the dataset \(\mathbf{X} = \{x_i\}\), where each face image is paired with a segmentation mask \( y_i \in \mathbf{Y} \), mapping to 19 facial components (e.g., hair, eyes, mouth). Demographic attributes are denoted as \(\mathbf{a}\) (e.g., \texttt{Male}, \texttt{Young}, \texttt{Wearing Hat}). Our objective is to train a segmentation function \( f_\theta(\cdot) \) that predicts \(\hat{y}_i\) while optimizing for accuracy, fairness, and robustness. Accuracy is maximized by aligning \(\hat{y}_i\) with \(y_i\) using Dice loss~\cite{sudre2017generalised}. Fairness is enforced by minimizing variance \(\mathrm{Var}(\mathrm{mIoU}_g)\) across demographic groups, ensuring equitable segmentation quality. Robustness is maintained by penalizing performance degradation (\(\mathrm{mIoU}\) drop) under input perturbations such as noise and occlusion.

\begin{algorithm}[h][t]
\caption{Multi-Objective Face Parsing (Pseudo-code)}
\label{alg:multi_objective_pseudocode}
\begin{algorithmic}[1]
\Require Homotopy function \(h(t)\) providing \((\alpha, \beta, \gamma)\) for epoch \(t\)
\For{epoch \(t = 1 \dots T\)}
    \State \((\alpha, \beta, \gamma) = h(t)\)
    \For{each batch in DataLoader}
        \State \textbf{Load} images \(\{x\}\), masks \(\{m\}\), attributes \(\{a\}\)
        \State outputs \(= f_{\theta}(x)\) \Comment{U-Net forward pass}
        \State \(\mathcal{L}_{\mathrm{acc}} = \mathrm{DiceLoss}(outputs, m)\)
        \State outputs\(_{\mathrm{noisy}} = outputs + \text{random\_noise}()\)
        \State \(\mathcal{L}_{\mathrm{rob}} = -\mathrm{mIoU}(\mathrm{softmax}(outputs_{\mathrm{noisy}}), m)\)
        \State \(\mathcal{L}_{\mathrm{fair}} = \mathrm{Var}\left[\mathrm{mIoU}_g\right]\)
        \State \(\mathcal{L}_{\text{total}} = \alpha\,\mathcal{L}_{\mathrm{acc}} + \beta\,\mathcal{L}_{\mathrm{rob}} + \gamma\,\mathcal{L}_{\mathrm{fair}}\)
        \State \textbf{Backward} and \textbf{update} \(\theta\)
    \EndFor
\EndFor
\end{algorithmic}
\end{algorithm}

\subsection{Dataset Preparation}
\label{subsec:dataset_preparation}

We employ the CelebAMask-HQ dataset \cite{CelebAMask-HQ}, divided into training, validation, and test sets. Each image and mask are resized to \(256 \times 256\) for compatibility with our U-Net architecture. Demographic attributes are extracted from annotations to compute fairness metrics.

\subsection{Model Architecture}
\label{subsec:model_architecture}

Our segmentation model utilizes a U-Net architecture with a ResNet-34 encoder pre-trained on ImageNet. It outputs 19 channels corresponding to distinct facial regions, balancing computational efficiency with high segmentation accuracy.

\subsection{Multi-Objective Training}
\label{subsec:multi_objective}

We train the U-Net segmentation models by optimizing a weighted sum of accuracy, fairness, and robustness losses, dynamically adjusted using homotopy-based scheduling. The training process is outlined in Algorithm~\ref{alg:multi_objective_pseudocode}.

\paragraph{Loss Components}
\begin{itemize}
    \item \textbf{Accuracy Loss (\(\mathcal{L}_{\mathrm{acc}}\)):} Dice loss measures the overlap between predicted and ground truth masks.
    \item \textbf{Robustness Loss (\(\mathcal{L}_{\mathrm{rob}}\)):} Negative \(\mathrm{mIoU}\) under perturbed predictions to ensure stability.
    \item \textbf{Fairness Loss (\(\mathcal{L}_{\mathrm{fair}}\)):} Variance of \(\mathrm{mIoU}\) across demographic groups to promote equitable performance.
\end{itemize}

\textbf{Alternative Fairness Computation:} We also compute per-group \(\mathrm{mIoU}\) for each demographic attribute, enabling detailed analysis of performance disparities (see Section~\ref{subsec:fairness_comparison}).

\subsection{Homotopy-Based Loss Scheduling}
\label{subsec:homotopy}

We dynamically balance the three loss components using epoch-dependent weights \(\alpha(t)\), \(\beta(t)\), and \(\gamma(t)\), ensuring \(\alpha(t) + \beta(t) + \gamma(t) = 1\). Initially, accuracy is prioritized, with weights shifting towards robustness and fairness over time. We explore three scheduling strategies:

\begin{itemize}
    \item \textbf{Linear:} \(\alpha(t)\) decreases linearly, while \(\beta(t)\) and \(\gamma(t)\) increase proportionally.
    \item \textbf{Sigmoid:} Smooth logistic transitions for gradual emphasis shifts.
    \item \textbf{Piecewise:} Abrupt changes in weight distribution at predefined training stages.
\end{itemize}

\begin{figure}[t]
    \centering
    \includegraphics[width=\columnwidth]{figures/parameters_by_homotopy-crop.pdf}
    \caption{Comparison of \(\alpha\), \(\beta\), and \(\gamma\) schedules across three homotopy methods (Linear, Sigmoid, and Piecewise) over 30 epochs. Each subplot illustrates the evolution of a parameter (\(\alpha\), \(\beta\), or \(\gamma\)) as it adapts during training, highlighting the differences in transition dynamics across homotopy strategies. The legend below the figure identifies the homotopy method for each curve.}
    \label{fig:homotopy-schedules}
\end{figure}


Figure~\ref{fig:homotopy-schedules} illustrates the evolution of these weights across training epochs for each homotopy method.

\subsection{Integration with Generative Models}

\subsubsection{GAN-Based Face Editing}
\label{subsec:gan_integration}

We utilize the trained U-Nets to generate segmentation maps for the training and validation sets, which are then used to train a Pix2PixHD GAN. The GAN architecture comprises:

\begin{itemize}
    \item \textbf{Generator} \(G\): Transforms segmentation maps into RGB images.
    \item \textbf{Discriminator} \(D\): Distinguishes real images from generated ones.
\end{itemize}

The GAN training involves a combination of adversarial loss and pixel-level \(L_1\) reconstruction loss:
\[
\mathcal{L}_{\mathrm{GAN}} = \mathcal{L}_{\mathrm{adv}}(G, D) + \lambda \, \|\hat{x} - x\|_1,
\]
where \(\hat{x} = G(\text{segmentation\_map})\) and \(x\) is the real image.

During testing, the GAN generates images using segmentation maps from the test set produced by both single-objective and multi-objective U-Nets, enabling evaluation of how segmentation quality impacts generative performance.

\subsubsection{ControlNet-Based Face Editing}
\label{subsec:controlnet_integration}

In addition to GANs, we integrate \textbf{ControlNet} \cite{zhang2023adding} for diffusion-based face editing. ControlNet leverages segmentation maps to guide the diffusion process, enhancing image fidelity and semantic alignment. Our setup includes:

\begin{itemize}
    \item \textbf{ControlNet Model:} Pre-trained on Stable Diffusion, fine-tuned on our segmentation maps.
    \item \textbf{Diffusion Pipeline:} Combines ControlNet with a text encoder and U-Net backbone to generate photorealistic faces conditioned on segmentation maps.
\end{itemize}

\textbf{Training Procedure:} ControlNet is fine-tuned for a single epoch using segmentation maps from the training set. In diffusion-based experiments, we compare only the single-objective model with the multi-objective linear homotopy model to manage computational resources effectively. The training minimizes the standard denoising loss:
\[
\mathcal{L}_{\mathrm{ControlNet}} = \mathcal{L}_{\mathrm{denoise}},
\]
where \(\mathcal{L}_{\mathrm{denoise}}\) is the Mean Squared Error between predicted and actual noise. During testing, ControlNet generates images using test set segmentation maps from both U-Net models, allowing assessment of segmentation quality's effect on diffusion-based generation.

\subsection{Evaluation Metrics and Setup}
\label{subsec:evaluation}

\paragraph{Segmentation Metrics}  
We evaluate segmentation performance using the mean Intersection-over-Union (\(\mathrm{mIoU}\)) across 19 facial classes. Fairness is quantified by the variance \(\mathrm{Var}(\mathrm{mIoU}_g)\) across demographic groups, and robustness is assessed through performance under Gaussian noise, occlusions, and blur.

\paragraph{Generative Metrics}  
For GAN outputs, we evaluate image quality using \textbf{Fréchet Inception Distance (FID)}, which quantifies realism by comparing feature distributions between generated and real images. Additionally, \textbf{Learned Perceptual Image Patch Similarity (LPIPS)} measures perceptual similarity, where lower scores indicate greater visual resemblance to real images.

\paragraph{Implementation Details}  
All experiments are implemented in PyTorch and trained on four NVIDIA A10 GPUs using the Adam optimizer with a learning rate of \(10^{-4}\). For ControlNet, we fine-tune the pre-trained \texttt{control\_v11p\_sd15\_seg} model based on Stable Diffusion v1.5. Our pipeline supports gradient accumulation and mixed precision (FP16) for computational efficiency. Homotopy-based loss scheduling is configurable (\texttt{linear}, \texttt{sigmoid}, \texttt{piecewise}). Detailed training configurations will be released alongside our code and models to ensure reproducibility.

\paragraph{Workflow Summary}  
\begin{enumerate}
    \item \textbf{Train U-Nets:} Train single-objective and multi-objective U-Nets on the training set, validate on the validation set.
    \item \textbf{Generate Segmentation Maps:} Use trained U-Nets to produce segmentation maps for training, validation, and test sets.
    \item \textbf{Train GAN:} Train the Pix2PixHD GAN using segmentation maps from the training and validation sets.
    \item \textbf{Fine-Tune ControlNet:} Fine-tune ControlNet on training set segmentation maps for one epoch.
    \item \textbf{Generate and Evaluate Images:} Generate images using GAN and ControlNet with test set segmentation maps from both U-Net models; evaluate using FID and LPIPS.
\end{enumerate}

In the following section, we present quantitative and qualitative results demonstrating the effectiveness of our approach across various conditions and demographic groups.

\section{User Study}
\subsection{Study Design}

This section aims to provide a comprehensive overview of the methodology employed to investigate the impact of different annotations on learning and teaching in touchscreen and VR environments. Participants, totaling 24, engaged in a structured study where they perform tasks as both educators and learners. Upon entering the designated study room, participants were presented with an information sheet outlining the study's context and their rights. The study was conducted on a voluntary basis, and participants were free to withdraw at any point without consequences. Upon consenting to participate, individuals completed a form and a pre-study assessment gauging their comfort level using the widely recognized Simulator Sickness Questionnaire ~\cite{ssqOriginal, ssqFix}. An introduction to the tools used in the study followed, including touch movement and VR interactions.

The main study was structured into two sections, with participants alternating between the roles of educators and learners. The decision to let all participants perform both roles was made to enable comparisons between the usability and usefulness of all tools and annotations, even though the roles would rarely be shared in real-life scenarios. Each section comprised three subtasks, structured in such a way that every participant would be exposed to all proposed annotation tools and mechanisms. To support assessment of each tool and mechanism individually, the study was designed to minimize the amount of new tools the participants were given at once, instead only introducing them one-by-one or alongside their related tools. 
%
Each subtask was covering a single topic, unrelated to other subtasks in this study. The topics were chosen such that participants were unlikely to have more than surface-level knowledge before participating in the study to emulate learning environments where our tools would be combined with unfamiliar information. Still, the subtasks' subjects shouldn't be too complex to be covered in the scope of this subject or require any prior knowledge that only specific professions may have.
%
Educators utilized drawing, fill, and eraser tools in the first subtask, followed by text and sequence tools in the second. The third subtask allowed free use of both annotation methods. The study was executed in eight variations, every participant completing all the mentioned tasks but counter-balanced to ensure each order was conducted three times. Participants, spent an average of 60–90 minutes completing the study's practical component. The study maintained consistent laboratory conditions across multiple runs by using identical hardware and software configurations. Upon completion of the practical tasks, participants filled out multiple established questionnaires~\cite{ssqOriginal, ipq, sus, daq} to provide additional insights into their experiences and perspectives. This study received ethical approval from the institution's ethics board, underscoring our commitment to maintaining the highest ethical standards in research.

\begin{figure}[tb]
 \centering
 \includegraphics[width=0.9\columnwidth]{figures/lapCamera2.png}
 \caption{Illustration inserted laparoscopic camera, revealing the internal organs. An exemplary textual annotation is displayed at nose position.}
 \label{fig:lapcamera}
\end{figure}

\subsection{Hardware and Software Setup}
This section details the hardware and software that was used to implement the described method and to conduct the study. 

\subpart{Hardware} The Hardware used for this study was chosen so that both the VR and touchscreen setup could be rendered simultaneously on the same PC. Thus, the graphics card NVIDIA RTX 3090 was used together with the Intel Core i9-12900k and 32~GB of DDR5 RAM running at 4.4GHz. The VR equipment we selected was the Head-Mounted Display (HMD) Valve Index, running two LCDs at 144Hz with a per-eye resolution of 1440×1600 pixels.

\subpart{Software} The PC backing our study is running Windows 11. The methods described above were implemented in Unity 2021.3.14f1~\cite{unity}. Unity is mostly known as a game engine, but is also reasonably useful as a research tool for interface, interaction, or visualization studies~\cite{unityInterface1, unityInterface2, unityInterface3}. Unity's behavior can be customized using C\# scripts, enabling advanced functionality.


\subsection{Procedure}

The study was designed so that each participant would perform both perspectives that this method was designed for: educators and learners. This section will detail the procedure of the study.

\subpart{Introduction} The introduction was structured to introduce the participant to every tool they would use during the study. The scene prepared for this contains a human body provided by Nobutaka et al.~\cite{bodyparts3d} that was used for all depictions of anatomy throughout this study. First, the interface for educators was described by the researcher. After establishing the ways the camera could be moved, the participant was allowed to try it out. Then, the researcher demonstrated the use of the pen and the eraser on the face of a human body by drawing with blue color around one eye and red color on the lips. Part of that was erased directly afterward. Then, the researcher placed a text annotation on the nose with the text ''nose'' to demonstrate the use of the text annotation. A second one was placed on the knee. Then, the sequence tool was used to set the sequence index of the knee annotation to 2, placing it chronological behind the one on the nose.
%
For the second part of the introduction, the participants were equipped with the HMD. Once they reported seeing sharp images, they were allowed to move around and acclimate to the virtual environment. Once they were ready, they were told about the tools laying on a metal tray, how they could pick them up and drop them. They were asked to use all the regular annotation tools one after one. The introduction scene also contained one advanced tool, the laparoscopic camera used in the ablation scene described below, which had to be tried out.

\subsubsection{Educator} 

In the role of educators, the participants are tasked to transfer knowledge from text-book excerpts into scenes that were prepared for the study. Those excerpts were printed out so that the participants could hold them while using the touchscreen with their free hand. At the beginning of each subtask, the participants were given a small oral introduction into the specific topic and handed the excerpt to read through. Afterward, they were told which annotations they were allowed to place and shown the prepared scene. Each scene consisted of three parts which were labeled ''1.'', ''2.'' and ''3.'', later on regarded as the first, second or third part. There was no target of how much or what kind of information they were to transfer. Each subtask was finished as soon as the participant wished to do so.

In the following, each scene and the covered topic are explained in more detail. 

\subpart{Scar revision} Scar revision is the process of trying to remove or change the appearance of scars. One of the procedures to do this is to remove the scar by cutting it out and sewing the surrounding skin back together. This may be done to return flexibility when scars are near joints, or for aesthetic purposes. The excerpt handed to the participant details that and some alternative approaches to do scar revision. The three parts depict one skin piece with a scar and two identical skin pieces, where the scar is missing, and the skin has a visible hole. This scene was to be annotated only using the pen, eraser, and fill tools.

\subpart{Port} The scene covering the topic of medical ports was designed around restricting the participants to using the text and sequence tools. Medical ports are small gadgets which are implanted beneath the skin. A catheter connects them to a vein, allowing  access to the blood system of a patient without needing to puncture their veins. They are commonly used for cancer patients. The excerpt given to the participants detailed information on ports regarding their use, structure, and benefits. The first part of the scene shows a body where specific veins are visible through the skin. The most important vein for this procedure is highlighted. In the second part, the model of a port chamber is depicted. Finally, in the third, a human with a fully implanted port including the catheter is shown, with relevant parts visible through the skin.

\subpart{Anaphylaxis} The final task in the role of the educator is covering the topic of severe allergic reactions, their diagnosis and how to treat them in the field. The most commonly used drug for this is adrenaline, which gets injected into large muscles of the affected person. The excerpt covers this topic in depth, including symptoms and treatment in the field. The first part of the scene shows a human body, the second an enlarged version of an EpiPen and finally the third, the EpiPen stuck to the upper thigh. This time, the participants were allowed to make full use of all the tools.

\subsubsection{Learners}

In the part of the study where participants were exploring the systems designed for learners, they were wearing the supplied HMD. For this, they had to go through the courses described below. 

\subpart{Ablation} In this task, the learners were standing in a scene with a male body and a whiteboard. The whiteboard displays a simplified endoscopic ablation to remove a tumor. For this, two tools with needles, one with a camera and one with a hot tip, are inserted into the body. The green circles that are depicted on the whiteboard are drawn onto the body to aid the participant in locating the tumor. The participant has to insert the needle with the camera into the body. Once inserted, organs near the camera can be seen through the skin. The participant then has to search for the bright color-coded tumor and touch it with the laparoscopic tool they have holding in their other hand. Once that's done, they have finished this task. The annotations that were used to create this scene were like those created by the pen tool.

\subpart{Digestion} This scene holds the organs that are part of the human digestive tract. Each of the important parts had a text box associated with it, totaling 11 annotations. The sequence tool was used to make the text boxes appear in the order in which food would pass through them. Participants were tasked to read through every text box. For this, they also had to move around because the text boxes were occluded by the organs from some view points.

\subpart{Teeth inspection} The final task in the role of learners was concerned with teeth. They were presented with a skill with an open mouth. The teeth are colored according to their quadrant. The participants learned the Federation Dentaire Internationale (FDI) dental notation~\cite{isoTeeth}, which gives each tooth a simple ID from text boxes. Then they were tasked to mark a specific tooth with by coloring it yellow. Lastly, they should use a special tool to remove a tooth given by its ID by touching it. This scene combined all annotations we examined in this paper.

\subsection{Participants}

Of our 24 participants, 15 were male and 8 were female. The youngest participant was 23, the oldest 38 with a median age of 25. All participants had at least a High school diploma, with 4 having or aspiring to have a bachelors degree, 10 a masters-level degree and 2 PhDs. Out of the 6 people not currently being a student, 2 work as software developers and 3 as researchers.

The previous experience in the areas our research is touching is very diverse. 11 of our participants have none to very little experience with 3D environments like simulations or video games. Of the other 13, one has 3–4 years of experience, while the rest have 5+ years.
%
Our participants were less experienced in general with VR, with fifteen having reported no true experience, three having less than 1 year, three in the range of 1–2 years, one having 3-4 and only two people having at least 5 years of experience. Nobody reported using VR regularly. 
%
Except for 2 people, everyone was interacting with PCs every day. Nine participants reported playing video games every day, and six others at least a few times a month.

\subsection{Measures}
\label{sec:measures}
\subsubsection{Objective Measures}

During the tasks, we created a log where the most important events are appended with a timestamp. An entry was added each time:

\begin{itemize}
    \item a Unity scene changes, which occurred when switching to the next task,
    \item the learner finishes a task,
    \item the educator switches to a different interaction mode,
    \item the learner is using a VR controller to pick something up,
    \item or the educator is interacting with the height slider (at most once per second).
\end{itemize}


These timestamped events allow us to reconstruct how long each participant took for which tasks, what tools they used and how long they used them. To capture the drawn annotations that were created, we created photos of the scene from multiple angles when a task was finished or before a new scene was entered.




\subsubsection{Subjective Measures}
This study utilizes multiple questionnaires to measure how users reacted to our presented tools. First, to measure symptoms of sickness, the widely used \textbf{Simulator Sickness Questionnaire (SSQ)}~\cite{ssqOriginal} is included. The SSQ asks the user to rate their sickness by ranking their current feeling regarding 16 different criteria (e.g., headache, fatigue, nausea, or dizziness) on a scale of 1 to 4. Using that, 3 scales can be derived that relate to nausea (N), oculomotor disturbance (O) and disorientation (D). Kennedy et al.~\cite{ssqOriginal} report multiple thresholds with $>20$ being the most severe one, relating to a bad simulator~\cite{ssqFix}. Though, as noted by Bimberg et al.~\cite{ssqFix}, we applied the corrected formula for the final score and used the common approach of using a pre- and post-study questionnaire. We wanted to assess how the SSQ scores differed between the usage of touchscreens and the HMD, so we separated the post-study SSQ into two parts, where participants had to rank their feelings for each of the device types.
%
Even when somebody does not actually experience symptoms of simulator sickness, they still may feel other types of discomfort regarding specific tools. To quantize this, we employ the \textbf{Device Assessment Questionnaire (DAQ)}~\cite{daq} in which each volunteer had to rate the required force, smoothness, mental and physical effort as well as various bodily fatigues (totaling thirteen properties) when using each of the eleven tools. This resulted in 143 ratings the users had to perform. The scale of the DAQ goes from 1 to 5, while the interpretation was inconsistent between each of the questions. For some questions, 3 is the ideal case, where for some it is 5 as seen in the header of \autoref{tab:daq}.
%
We also want to measure presence, as that is a big factor for learning experiences ~\cite{presenceLearning}. This was gauged by adding the \textbf{Igroup Presence Questionnaire (IPQ)}~\cite{ipq} to our post-study questionnaire. The fourteen questions included try to measure how much the simulated world was recognized and experienced as reality. They also ask the participant to reflect on how much of the real world was still perceived. The questionnaire presents statements and asks the responded to rate their agreement from 0 through 7. The questions were again separated between touchscreen and VR. 



To assess the usability of our tools more directly, users had to answer the \textbf{System Usability Scale (SUS)}~\cite{sus}. Each of the 10 statements contained tried to address different important factors for real-world usability, which users had to rate their agreement with. To consider our tools independently, each of the 10 statements had to be rated for each of the 11 tools, resulting in 110 ratings. The rating was conducted by assigning each of the statements a score between 1 and 5, where 1 represents minimal agreement and 5 maximum agreement with the statement. The usability can then be calculated according to SUS and falls between 0 and 100.

On the last page, users had to state which of the 6 subtasks was their favorite. Finally, users had to evaluate the perceived usefulness of the annotations, regardless of their current implementation.
\begin{table*}[t]
    \centering
    \resizebox{\textwidth}{!}{
\begin{tabular}{l|rrllrrll}
\toprule
\textbf{Dataset} & \multicolumn{4}{c}{\textbf{GSM8K}} & \multicolumn{4}{c}{\textbf{MATH}} \\
\cmidrule(lr){1-1} \cmidrule(lr){2-5} \cmidrule(lr){6-9}
\textbf{Method} & Acc & Len & Rel. Acc & Rel. Len & Acc & Len & Rel. Acc & Rel. Len \\
\midrule
\multicolumn{9}{l}{\textit{Zero-Shot Prompting}} \\
\midrule
\hspace{12pt}Baseline & 78.06 & 241.87 & 100.00 \small{(0.00)} & 100.00 \small{(0.00)} & 46.40 & 480.37 & 100.00 \small{(0.00)} & 100.00 \small{(0.00)} \\
\hspace{12pt}Be Concise & 77.98 & 214.87 & 99.85 \small{(1.18)} & 88.46 \small{(10.37)} & 47.76 & 446.09 & 102.71 \small{(7.59)} & 92.66 \small{(7.46)} \\
\hspace{12pt}Hand Crafted 2 (ours) & 76.72 & 184.13 & 98.27 \small{(3.67)} & 77.10 \small{(22.27)} & 46.84 & 404.85 & 101.62 \small{(4.79)} & 85.26 \small{(15.97)} \\
\midrule
\multicolumn{9}{l}{\textit{FT - External Data}} \\
\midrule
\hspace{12pt}Direct Answer & 19.70 & 3.17 & 24.88 \small{(5.03)} & 1.36 \small{(0.40)} & 15.08 & 6.98 & 35.16 \small{(10.34)} & 1.44 \small{(0.73)} \\
\hspace{12pt}Human CoT & 65.73 & 127.85 & 83.82 \small{(7.28)} & 54.95 \small{(13.17)} & 33.88 & 243.54 & 75.61 \small{(13.56)} & 53.14 \small{(13.87)} \\
\hspace{12pt}GPT4o CoT & 76.36 & 156.24 & 97.65 \small{(3.63)} & 67.60 \small{(16.70)} & 40.44 & 399.80 & 90.52 \small{(15.07)} & 87.21 \small{(22.22)} \\
\midrule
\multicolumn{9}{l}{\textit{FT - Best-of-N Self-Generation}} \\
\midrule
\hspace{12pt}Naive BoN & 77.12 & 214.22 & 98.79 \small{(1.64)} & 87.17 \small{(8.79)} & 47.64 & 433.26 & 101.74 \small{(7.04)} & 89.89 \small{(3.99)} \\
\hspace{12pt}Rational Metareasoning & 76.15 & 207.49 & 97.21 \small{(5.74)} & 84.93 \small{(5.09)} & 47.56 & 432.56 & 103.02 \small{(6.56)} & 90.56 \small{(5.25)} \\
\midrule
\multicolumn{9}{l}{\textit{FT - Few-Shot Conditioned Self-Generation (ours)}} \\
\midrule
\hspace{12pt}FS-Human & 76.66 & 161.72 & 98.06 \small{(3.28)} & 67.96 \small{(16.62)} & 46.44 & 421.54 & 99.69 \small{(6.97)} & 87.78 \small{(5.98)} \\
\hspace{12pt}FS-GPT4o & 78.07 & 175.54 & 99.94 \small{(1.69)} & 73.15 \small{(13.49)} & 47.36 & 421.21 & 101.87 \small{(5.33)} & 87.58 \small{(6.60)} \\
\hspace{12pt}FS-Self & 77.27 & 190.03 & 98.86 \small{(2.51)} & 77.51 \small{(9.18)} & 48.00 & 426.67 & 102.67 \small{(5.24)} & 88.50 \small{(4.49)} \\
\midrule
\multicolumn{9}{l}{\textit{FT - Few-Shot Conditioned Best-of-N Self-Generation (ours)}} \\
\midrule
% GPT4o Best-of-16 (Naive) & 75.48 & 153.51 & 96.56 \small{(3.79)} & 64.12 \small{(16.35)} & 47.28 & 367.49 & 101.50 \small{(9.81)} & 76.96 \small{(11.42)} \\
\hspace{12pt}FS-GPT4o-BoN & 75.88 & 153.38 & 97.00 \small{(4.11)} & 64.25 \small{(16.66)} & 47.36 & 364.33 & 102.56 \small{(6.24)} & 76.30 \small{(10.56)} \\
\hspace{24pt}\raisebox{0.5ex}{$\llcorner$}\hspace{4pt}\textit{Budget-Matched} & 76.24 & 160.59 & 97.44 \small{(3.67)} & 67.15 \small{(16.41)} & 47.52 & 384.43 & 101.58 \small{(9.53)} & 80.43 \small{(9.04)} \\

\bottomrule
\end{tabular}
    }
    \caption{
        Evaluation of zero-shot prompting, fine-tuning (FT), and our proposed methods on GSM8K and MATH datasets, averaged across five main models.
        We report absolute accuracy (\%) and length (tokens), along with relative accuracy (\%) and length (\%) compared to baseline prompting. Values in parentheses show standard deviations.
    }
    \label{tab:main_results}
\end{table*}

\section{Results}
\label{results}

% \begin{table*}[t]
% \resizebox{\textwidth}{!}{
% \centering
% \small
% \begin{tabular}{m{2.5cm}|m{15cm}}
% \toprule
% \textbf{Method} & \textbf{Content} \\ %& \# \textbf{Tokens}\\
% \midrule
% \midrule[0.05em]
% \multicolumn{2}{c}{Question} \\ 
% \midrule
% - & A robe takes 2 bolts of blue fiber and half that much white fiber.  How many bolts in total does it take?  \\
% \midrule
% \midrule[0.05em]
% \multicolumn{2}{c}{Reasoning Paths} \\
% \midrule
% Zero-Shot & To find the total number of bolts needed, we need to calculate the amount of white fiber first, since it's half the amount of blue fiber. Step 1: Determine the amount of blue fiber needed. The robe takes 2 bolts of blue fiber. Step 2: Calculate the amount of white fiber needed. The robe takes half the amount of white fiber as blue fiber. Half of 2 bolts is $2/2$ = 1 bolt. Step 3: Add the amount of blue fiber and white fiber to find the total. Total bolts = blue fiber + white fiber. Total bolts = 2 bolts + 1 bolt. Total bolts = 3 bolts. The answer is \textbf{\textit{3}}. \\ %& 213 \\
% \midrule
% FS-GPT4o-BoN & The robe takes 2 bolts of blue fiber. It takes half that much white fiber, which is $2 / 2$ = 1 bolt. Add the blue and white fiber together: 2 + 1 = 3 bolts. The answer is \textbf{\textit{3}}. \\ %57 \\
% \bottomrule
% \end{tabular}
% }
% \caption{Example rationales generated from the original Llama-3.1-8B model and the model fine-tuned using our method. The question is randomly sampled from GSM8K. For the same question, one can observe that our rationale is much more concise, removing redundant parts that explain the question.}
% \label{tab:samples}
% \vspace{-0.1in}
% \end{table*}


\begin{table*}[t]
\resizebox{\textwidth}{!}{
\centering
\normalsize
\begin{tabular}{m{17.5cm}}
\toprule
\multicolumn{1}{c}{\textbf{Question}} \\ 
\midrule
A robe takes 2 bolts of blue fiber and half that much white fiber.  How many bolts in total does it take? \\
\midrule
\multicolumn{1}{c}{\textbf{Reasoning Paths}} \\
\midrule
\textbf{[Zero-Shot]} 
To find the total number of bolts needed, we need to calculate the amount of white fiber first, since it's half the amount of blue fiber. Step 1: Determine the amount of blue fiber needed. The robe takes 2 bolts of blue fiber. Step 2: Calculate the amount of white fiber needed. The robe takes half the amount of white fiber as blue fiber. Half of 2 bolts is $2/2$ = 1 bolt. Step 3: Add the amount of blue fiber and white fiber to find the total. Total bolts = blue fiber + white fiber. Total bolts = 2 bolts + 1 bolt. Total bolts = 3 bolts. The answer is 3. \\
\midrule
\textbf{[FS-GPT4o-BoN]} 
The robe takes 2 bolts of blue fiber. It takes half that much white fiber, which is $2 / 2$ = 1 bolt. Add the blue and white fiber together: 2 + 1 = 3 bolts. The answer is 3. \\
\bottomrule
\end{tabular}
}
\caption{Example rationales generated from the original Llama-3.1-8B model (\textbf{Zero-Shot}) and the model fine-tuned using our method (\textbf{FS-GPT4o-BoN}). The question is randomly sampled from GSM8K. For the same question, one can observe that our rationale is much more concise, removing redundant parts that explain the question.}
\label{tab:samples}
\vspace{-0.1in}
\end{table*}


\subsection{Main results}

Our main results, presented in \autoref{tab:main_results} and \autoref{fig:main_methods_comparison}, demonstrate the performance of our self-training methods against baseline approaches.
% We highlight key observations from these results below.

\paragraph{Naive BoN fine-tuning is effective but sample inefficient.}
Naive BoN fine-tuning effectively reduces output length without significantly degrading model performance. 
This also holds true for Qwen2.5-Math-1.5B and DeepSeekMath-7B (\autoref{tab:main_results_full_gsm8k} and \autoref{tab:main_results_full_math}), which failed to achieve length reduction through zero-shot prompting.
% However, while naive BoN does reduce output length, the reduction is limited to 12\%.
However, the length reduction from naive BoN with $N=16$ is limited to 12\% on average.
Furthermore, as illustrated in Figure~\ref{fig:bon_sample_efficiency}, achieving more compression with BoN becomes progressively less efficient.

\paragraph{Iterative baseline yields similar results as naive BoN fine-tuning.}
% We compare our single-step naive BoN approach with Rational Metareasoning \cite{de2024rational}, an iterative approach using expert iteration \cite{zelikman2022star}  which incorporates an additional \textit{utility reward} to balance efficiency and accuracy in BoN sampling.
Rational Metareasoning, an iterative baseline, yields similar relative length reduction and relative accuracy to BoN fine-tuning. 
This suggests that the utility reward proposed by \citet{de2024rational} may not effectively achieve both accuracy gains and token length reduction.

\begin{figure}[t] % "h" places the figure roughly here
    \centering
    \includegraphics[width=\columnwidth]{figures/main_methods_comparison.pdf} % Adjust width as needed
    \caption{Tradeoff between relative accuracy and length reduction for main methods. Results are averaged over GSM8K and MATH across five main models. Matching colors and shapes indicate the same FS prompt. FS conditioning without augmentation (†) are marked with lighter colors. 
    Relative length reduction refers to 100 - relative length (\%).}
    \label{fig:main_methods_comparison} % Label for referencing in text
\end{figure}
% \red{TODO - shorten this}

\paragraph{Few-shot conditioning outperforms BoN in length reduction.}
The results demonstrate that few-shot conditioning achieves a greater relative length reduction compared to naive BoN, including math-specialized models (\autoref{tab:main_results_full_gsm8k} and \autoref{tab:main_results_full_math}).
% This reduction is attributed to the fact that the fine-tuning datasets generated through few-shot conditioning contain shorter reasoning paths compared to those generated by naive BoN, as illustrated in \autoref{fig:bon_sample_efficiency}.  % too long
This is in line with the superior length reduction of few-shot conditioning, compared to naive BoN as shown in \autoref{fig:bon_sample_efficiency}.
Notably, self-training on generations conditioned on human-annotated examples (FS-Human) achieves an average relative length of 67.96\% on GSM8K, compared to 87.17\% with naive BoN.  % good to have some specific numbers in the text
% We further analyze the effect of fine-tuning on length reduction in \autoref{analysis}.



\paragraph{Self-training better preserves accuracy than training with external data.} 
\autoref{tab:main_results} shows fine-tuning with external data (\textit{FT-External Data}) leads to a significant reduction in relative length but causes a severe drop in relative accuracy. 
% \autoref{fig:main_methods_comparison} further highlights that while fine-tuning with GPT-4o CoT (FT-GPT4o) achieves slightly greater reduction in relative length than fine-tuning with self-generated data using few-shots from GPT-4o (FS-GPT4o), it results in substantially lower relative accuracy.  % a bit complicated / not concrete (conrete evidence = one where we beat external FT in both accuracy and reduction)
\autoref{fig:main_methods_comparison} further highlights the accuracy preservation of self-training: fine-tuning with external concise reasoning supervision from GPT-4o (FT-GPT4o) lies below the Pareto-curve of relative accuracy and relative length reduction, established by our self-training methods.
% NAMGYU - TODO add some commentary

\paragraph{Few-shot conditioned BoN achieves best length reduction while maintaining accuracy.}
% Few-shot conditioned BoN enables substantial length reduction compared to all other BoN and few-shot methods while maintaining relative accuracy.
FS-BoN elicits the largest length reduction among our self-training methods, while maintaining relative accuracy, on average.
Notably, for math-specialized models, FS-GPT4o-BoN achieves the greatest reduction among all methods, except those fine-tuned on external data which greatly sacrifice the accuracy (\autoref{tab:main_results_full_gsm8k} and \autoref{tab:main_results_full_math}). 
% This result reflects how applying BoN to few-shot conditioning further reduces the relative length of the training data while also increasing the proportion of correct samples.  % unnecessary

\paragraph{Augmentation boosts accuracy for few-shot conditioning.}
\autoref{fig:main_methods_comparison} compares few-shot conditioning, i.e., FS and FS-BoN, with and without augmentation (†). 
Augmentation improves accuracy by providing solutions for previously unsolvable hard questions as discussed in \autoref{sample_augmentation}. 
While augmentation may slightly affect reduction rates, they remain superior to naive BoN and RM.
% Similar effect is observed for augmentation in FS-BoN.
% Even when matching the budget (\textit{Budget-Matched}) with other fine-tuning methods using self-generated data in \autoref{tab:main_results}, it achieves the greatest length reduction among them with minimal accuracy degradation.
Even when matching the budget (\textit{Budget-Matched}) with other self-training methods in \autoref{tab:main_results}, it achieves the greatest length reduction among them with minimal accuracy degradation.
The effect of augmentation on training data length is analyzed in \autoref{appx_augmentation_length}.
% Furthermore, as shown in Figure \ref{fig:main_methods_comparison}, augmentation on few-shot conditioned BoN enhances accuracy similar to naive BoN and Meta-Reasoning while achieving greater length reduction.

\begin{figure}[t]
    \centering
    \includegraphics[width=\columnwidth]{figures/length_by_difficulty.pdf} % Adjust width as needed
    \caption{\textbf{Tokens are reduced adaptively according to question difficulty.} 
    Token reduction rate for each difficulty level on MATH, for 4 models individually and averaged.
    % Higher difficulty levels show lower reduction rates.
    Relative length reduction refers to 100 - relative length (\%).
    }
    \label{fig:length_difficulty} % Label for referencing in text
\end{figure}

\subsection{Analysis}
\label{analysis}
% This section analyzes length reduction: transfer from generation to fine-tuning, reduction by question difficulty, qualitative analysis, and consistency across model sizes. DeepSeekMath-7B is excluded from quantitative analysis due to cost.
% let's keep this short
In this section, we analyze the length reduction effects in depth.
We exclude DeepSeekMath-7B from quantiative analysis due to cost.


% \paragraph{Analysis on sample efficiency}
% As shown in \autoref{fig:bon_sample_efficiency}, best-of-n (BoN) sampling requires a substantial number of samples to be generated to achieve a level of reasoning length reduction comparable to that achievable through few-shot conditioning.
% In other words, it is infeasible to reach the reasoning length reduction performance of few-shot conditioning using BoN alone, without generating a prohibitively large number of samples.
% However, our experiments consistently demonstrate that combining few-shot conditioning with BoN sampling is more effective in reducing reasoning length than using either technique in isolation.
% Specifically, few-shot conditioning helps to guide the model towards generating more concise reasoning paths, while BoN sampling allows us to select the shortest and most accurate path from a diverse set of candidates.
% This synergistic effect results in a more efficient and effective approach to concise reasoning.


% \paragraph{FT can reduce generation length effectively.}
% As shown in \autoref{fig:ft_length_scatter}, after fine-tuning, the models tend to follow the length of the training data, suggesting that reasoning length reduction can be achieved through simple supervised fine-tuning on short reasoning samples.
% Note that test generation length is relatively longer than the training data length, as the models can generate lengthy incorrect answers, while the training data consists of correct answers.
% Correctly generated answers align more closely with training data length as shown in (Appendix~\ref{appx_length_scatter_correct}).

% \paragraph{Length reduction through generation and fine-tuning}
% Our method reduces reasoning length in two stages: generation and fine-tuning.
% First, as shown in \autoref{fig:ft_length_scatter}, 
% % generation length for training data varies depending on the method. 
% few-shot conditioning methods produce shorter outputs than naive BoN, with few-shot conditioned BoN achieving the shortest. 
% Second, fine-tuning with shorter rationales results in shorter model outputs, showing a strong correlation between test and training lengths\footnote{Test generation lengths are generally longer than training data lengths due to the possibility of lengthy incorrect answers during testing. Test outputs that are correct align more closely with training data lengths, as shown in Appendix~\ref{appx_length_scatter_correct}.}.
% Overall, FS-GPT4o-BoN effectively generates and trains for shorter reasoning paths.
% Additionally, unlike zero-shot methods, our approach significantly reduces token length in math-tuned models like Qwen2.5-Math-1.5B with FS-GPT4o-BoN, achieving 54.7\% relative length after fine-tuning. (See \autoref{tab:main_results_full_gsm8k} and \autoref{tab:main_results_full_math}).

\paragraph{Tokens are reduced adaptively according to question complexity.} 
The MATH dataset's difficulty levels range from 1 (basic algebra) to 5 (advanced calculus and complex mathematical reasoning).
As shown in \autoref{fig:length_difficulty}, our method adaptively reduces tokens based on question difficulty, with higher difficulty leading to less reduction.
% Most models achieve their peak reduction (around 20\%--40\%) at difficulty levels 1-2, where simple concepts allow for more concise explanations.
% The reduction rate gradually declines at levels 3-5, indicating our method's ability to preserve necessary details for complex problems automatically.
%  -> not precise. simple concepts allow for more concise explanations *in absolute terms*, but this does not necessarily mean that length reduction *relative to the default* should be high. E.g., if the model already uses very few tokens for easy questions, then relative reduction would be low.
The higher reduction (20\%--40\%) at easier difficulty levels (1--2) suggests that the original model outputs for these easier questions contained unnecessary tokens.
This reveals a gap in current models' ability to tailor their inference budget to problem complexity.
Our method effectively closes this gap by reducing redundancy, allowing for more precise token allocation based on question difficulty.

\begin{figure}[t] % "h" places the figure roughly here
    \centering
    \includegraphics[width=\columnwidth]{figures/scaling_methods_comparison.pdf} % Adjust width as needed
    \caption{Scaling study on baseline and few-shot conditioned self-training methods. Results are averaged over GSM8K and MATH for Llama 1B, 3B, and 8B.
    % Accuracy tends to be maintained, with greater length reduction using our FS-GPT4o(-BoN) method.
    Relative length reduction refers to 100 - relative length (\%).
    }
    \label{fig:scaling_methods_comparison} % Label for referencing in text
\end{figure}

\paragraph{Self-training maintains consistency across model scales.}
We conduct a scaling study on Llama-3.2-1B, 3B, and Llama-3.1-8B to examine consistency across different model sizes (\autoref{fig:scaling_methods_comparison}). 
Overall, token reduction increases as the model size increases, while the maintenance of accuracy does not show a strong correlation with model size. 
RM exhibits lower reduction rates compared to our few-shot conditioned self-training methods across all models and shows a decrease in accuracy with increasing model size. 
% The few-shot method also shows a similar trend in length reduction, but it achieves the best relative accuracy in the 3B model.
Our standalone few-shot conditioning method (FS-GPT4o) also shows a similar trend in length reduction, but better preserves accuracy.
Our joint FS-GPT4o-BoN method achieves the greatest reduction across all models, maintaining relative accuracy across different model sizes, especially in the largest 8B model.



\paragraph{Sample study}
\autoref{tab:samples} presents qualitative examples of reasoning paths generated by the model before and after fine-tuning with our method. 
The original reasoning exhibits verbosity, containing redundant processes such as question confirmation and repeated instructions. 
In contrast, the reasoning generated by our method includes only the necessary steps, significantly reducing the number of tokens while still arriving at the correct answer. 
% These examples demonstrate the effectiveness of our method in reducing token count. 
More examples are provided in the \autoref{appx_sample_studies}.

\begin{figure}[t]
    \centering
    \includegraphics[width=\columnwidth]{figures/both_length_scatter.pdf} % Adjust width as needed
    \caption{\textbf{Fine-tuning effectively transfers the length reduction to the model.} Correlation between the relative length of train data and test output averaged over GSM8K and MATH across 4 models. Training length includes only correct solutions. Solid points represent test lengths including all generated outputs, while lighter points indicate test lengths of correct solutions only.}
    \label{fig:ft_length_scatter} % Label for referencing in text
\end{figure}

\paragraph{Length reduction is transferred through fine-tuning.}
As shown in \autoref{fig:ft_length_scatter}, fine-tuning with shorter rationales results in shorter model outputs, showing a strong correlation between test and training lengths.
% Test generation lengths (solid datapoints) are generally longer than training data lengths due to the possibility of lengthy incorrect answers during testing.
% However, when comparing with test generation lengths that are correct (lighter datapoints), they align more closely with training data lengths.
We note that the length of test outputs (incorrect and correct) are longer than the length of training samples (only correct) on average.
This is mainly because incorrect paths are generally longer than correct ones.
We find a closer correspondence between train length and test length of correct samples only, indicated by the lighter datapoints.
This discrepancy suggests the need to terminate incorrect paths early to minimize redundant inference overhead.
We consider this for future work.

\section{Discussion}

In this paper, we explored the relationship between human evaluations and NLP benchmarks of chat-finetuned language models (chat LMs). Our work is motivated by the recent shift towards human evaluations as the primary means of assessing chat LM performance, and the desire to determine the role that NLP benchmarks should play.

Through a large-scale study of the Chat Llama 2 model family on a diverse set of human and NLP evaluations, we demonstrated that NLP benchmarks are generally well-correlated with human judgments of chat LM quality. However, our analysis also reveals some notable exceptions to this overall trend. In particular, we find that adversarial and safety-focused evaluations, as well as language assistance and open question answering tasks, exhibit weaker or negative correlations respectively with NLP benchmarks. We also explored predicting human evaluation scores from NLP evaluation scores using overparameterized linear regression models. Our results suggest that NLP benchmarks can indeed be used to predict aggregate human preferences, although we caution that the limited sample size and the assumptions of our models may limit the generalizability of these findings. Our results suggest that NLP benchmarks can serve as fast and cheap proxies of slower and expensive human evaluations in assessing chat LMs.

Additionally, our work highlights the need for further research into NLP evaluations that can effectively capture important aspects of LM behavior, such as safety, robustness to adversarial inputs, and performance on complex, open-ended tasks. It is possible that new NLP benchmarks can provide signals on these topics, e.g., \citep{wang2023decodingtrust}. Of particular interest is developing human-interpretable and scaling-predictable evaluation processes, e.g., \citep{schaeffer2024emergent, ruan2024observational,schaeffer2024predictingdownstreamcapabilitiesfrontier}. Developing and refining such evaluation methods \citep{madaan2024quantifyingvarianceevaluationbenchmarks}, as well as detecting whether evaluations scores faithfully capture models' true performance \citep{oren2023proving,schaeffer2023pretrainingtestsetneed,roberts2023cutoff,jiang2024investigatingdatacontaminationpretraining,zhang2024careful,duan2024uncoveringlatentmemoriesassessing} will be crucial for ensuring that LMs are safe, reliable, and beneficial as they become increasingly integrated into society.

% In conclusion, our study provides insights into the relationship between human evaluations and NLP benchmarks of chat language models. By leveraging the complementary strengths of both human and NLP benchmarks, we can build a more complete understanding of LM capabilities and behaviors, ultimately enabling the development of models more capable, trustworthy, and beneficial to society.

\section{Discussion and Conclusion}
\label{sec:discuss}

We presented \bench, the first framework  and experimental platform to benchmark AI Agents for IT automation tasks. \bench strives to capture the complexity of modern IT systems and the diversity of IT tasks. The reproducibility of \bench ensures the community-driven effort despite inherent nondeterminism of large-scale IT systems. 

One of the key design principles of \bench is ensuring its flexibility to support diverse areas of different IT systems
and its extensibility to new scenarios. While current scope of \bench is comprehensive and representative, we plan to further enrich the benchmark suites by adding other important processes essential to modern IT automation. Furthermore, we plan to expand our benchmarking beyond event-triggered scenarios. 
We are actively working to expand scenario coverage for the supported processes and promote growth through open-community contributions.
 We invite the community to reproduce their real-world-inspired incidents in a synthetic sandboxed environment leveraging the \bench. We expect that everyone contributing can bring their expertise to the table.

We expect \bench to drive the innovations of AI agent-based techniques with a direct impact on the safety, efficiency, and intelligence of today’s IT infrastructures. 
With \bench, we are starting to explore many deep, exciting open problems: How to develop domain-specific AI agents that specialize in certain types of IT tasks? How to orchestrate multiple agents with various expertise to collaborate on bigger projects? How can we ensure safety of agent-driven solutions? How can we effectively use human-in-the-loop while developing diverse adaptive agents? We invite everyone to participate in answering these questions and realizing the vision of using AI agents to automate critical IT tasks.



\bibliographystyle{unsrtnat}
\bibliography{references}  %%% Uncomment this line and comment out the ``thebibliography'' section below to use the external .bib file (using bibtex) .


%%% Uncomment this section and comment out the \bibliography{references} line above to use inline references.
% \begin{thebibliography}{1}

% 	\bibitem{kour2014real}
% 	George Kour and Raid Saabne.
% 	\newblock Real-time segmentation of on-line handwritten arabic script.
% 	\newblock In {\em Frontiers in Handwriting Recognition (ICFHR), 2014 14th
% 			International Conference on}, pages 417--422. IEEE, 2014.

% 	\bibitem{kour2014fast}
% 	George Kour and Raid Saabne.
% 	\newblock Fast classification of handwritten on-line arabic characters.
% 	\newblock In {\em Soft Computing and Pattern Recognition (SoCPaR), 2014 6th
% 			International Conference of}, pages 312--318. IEEE, 2014.

% 	\bibitem{hadash2018estimate}
% 	Guy Hadash, Einat Kermany, Boaz Carmeli, Ofer Lavi, George Kour, and Alon
% 	Jacovi.
% 	\newblock Estimate and replace: A novel approach to integrating deep neural
% 	networks with existing applications.
% 	\newblock {\em arXiv preprint arXiv:1804.09028}, 2018.

% \end{thebibliography}


\end{document}
