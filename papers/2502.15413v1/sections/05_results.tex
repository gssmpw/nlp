\section{Results}
\begin{figure}[tb]
    \centering
    \begin{minipage}{0.49\columnwidth}
        \centering
        \includegraphics[width=\linewidth]{figures/teacherTool.png}
        \caption{The time in minutes the participants used each tool on the touchscreen.}
        \label{fig:touchToolUsage}
    \end{minipage}
    \hfill
    \begin{minipage}{0.49\columnwidth}
        \centering
        \includegraphics[width=\linewidth]{figures/vrTool.png}
        \caption{The time in minutes the participants used each tool in VR.}
        \label{fig:vrToolUsage}
    \end{minipage}
\end{figure}

\begin{figure}[t]
    \centering
    \begin{minipage}{0.49\columnwidth}
        \centering
        \includegraphics[width=\linewidth]{figures/usability.png}
        \caption{Box plot of the SUS~\cite{sus} usability scores for the different tools. TS = Touchscreen.}
        \label{fig:usability}
    \end{minipage}
    \hfill
    \begin{minipage}{0.49\columnwidth}
        \centering
        \includegraphics[width=\linewidth]{figures/touchSickness.png}
        \vspace{1em}
        \caption{Box plot of the recorded SSQ~\cite{ssqOriginal, ssqFix} scores for the touchscreen.}
        \label{fig:touchSickness}
    \end{minipage}
\end{figure}

\subsection{Objective Measures}

We recorded the time the users spent during each scene. It is important to note that, in most cases, the time that the user needed to complete the pre-study questionnaire is included in the initial scene length. Moreover, the clock was running while the users read the excerpt that was given to them for the scenes where they had to act as an educator. This was performed because multiple participants did not read the whole text at the beginning, which would have enabled simple measurement of that duration. The average time spent on the learner tasks, Ablation ($1.7\pm{}0.7 min$), Digestion ($3.4\pm{}1.4 min$), Teeth inspection ($3.2\pm{}1.3 min$) were significantly shorter than the ones spent on annotating scenes as an educator, which were Scar revision ($11.4\pm{}4.3 min$), Port ($14.4\pm{}6.7 min$) and Anaphylaxis ($11.5\pm{}5.1 min$). There were 3 guided scenes, namely all 3 learner-tasks, where users completed tasks without applying much creativity. Still, the coefficient of variation ($standard deviation / mean$) normalizes the standard deviation and is around $0.37$ to $0.46$ on all scene lengths.
%
On average, the participants spent 37.4 minutes in the educator's role and only 8.3 minutes as learners. This, combined with the average 18 minutes which the introduction and pre-study assessment took, participants spent on average 63 minutes and 39 minutes until the end of the practical part of the study. Afterward, they were still required to complete the post-study questionnaire, which was not time-limited.
%
As our study log included details regarding which tool was switched, we could reconstruct the time each tool was active. This is not necessarily the amount of time the tools were in use because idling would not be excluded. The mean average use time for the tools available on the touch screen were: Movement ($1055.9\pm{}383.0s$), Text ($667.0\pm{}384.8s$), Draw ($339.2\pm{}156.2s$), Eraser ($67.9\pm{}57.9s$), Sequence ($124.7\pm{}98.7s$). In total, the users spent 46.8\% of their time as an educator using the movement mechanics. A detailed distribution is depicted in \autoref{fig:touchToolUsage}.


The tools held in VR were also tracked and can be observed in \autoref{fig:vrToolUsage}. While the timer ran while the participants were just holding the tools, research personnel observed very little unnecessary retention of tools because users placed an emphasis on dropping tools as soon as they didn't need them anymore. Some users ended up automatically grabbing their controllers, which lead to them picking up tools without then noticing. Mean usage times for the tools: laparoscopic camera ($58.5\pm{}23.0s$), laparoscopic cautery ($54.5\pm{}22.9s$),  eraser ($2.5\pm{}7.9s$), marker ($14.3\pm{}19.6s$), fill tool ($18.4\pm{}23.2s$), fill-undo tool ($6.0\pm{}11.6s$), Forceps ($26.8\pm{}25.7s$).

\begin{figure}[t]
    \centering
    \begin{minipage}[t]{0.49\columnwidth}
        \centering
        \raisebox{0pt}[\dimexpr\height+1\baselineskip\relax]{
            \includegraphics[width=\linewidth]{figures/presence.png}
        }
        \caption{The IPQ presence scores~\cite{ipq}, calculated from the responses given by the participants. TS = Touchscreen.}
        \label{fig:presence}
    \end{minipage}
    \hfill
    \begin{minipage}[t]{0.49\columnwidth}
        \centering
        \raisebox{26pt}[\dimexpr\height+1\baselineskip\relax]{
            \includegraphics[width=\linewidth]{figures/vrSickness.png}
        }
        \caption{Box plot for the recorded SSQ~\cite{ssqOriginal, ssqFix} scores for VR.}
        \label{fig:vrSickness}
    \end{minipage}
\end{figure}

\begin{table*}[b]
    \centering
        \caption{Mean ± standard deviation for the results of the DAQ~\cite{daq}. Cell color intensity indicates the magnitude of difference to the optimum. opt.=optimum; fat. = fatigue}
    \label{tab:daq}
     \begin{adjustbox}{width=\textwidth}
    \begin{tabular}{rlllllllllllll}
Tool & \makecell{Smoothness\\opt.:5}  & \makecell{Accuracy\\opt.:5}  & \makecell{Comfort\\opt.:5}  & \makecell{Usability\\opt.:5} & \makecell{Finger fat.\\opt.:1} & \makecell{Wrist fat.\\opt.:1} & \makecell{Arm fat.\\opt.:1} & \makecell{Shoulder fat.\\opt.:1} & \makecell{Neck fat.\\opt.:1} & \makecell{Mental effort\\opt.:3}  & \makecell{Physical effort\\opt.:3} & \makecell{Force required \\opt.:3}& \makecell{Speed\\opt.:3} \\
\toprule
Touch Move & 2.75±1.22\cellcolor{lightblue!56} & 2.75±1.45\cellcolor{lightblue!56} & 2.50±1.06\cellcolor{lightblue!62} & 2.58±1.10\cellcolor{lightblue!60} & 2.25±1.59\cellcolor{lightblue!31} & 1.42±0.97\cellcolor{lightblue!10} & 1.83±1.27\cellcolor{lightblue!20} & 1.62±1.01\cellcolor{lightblue!15} & 1.21±0.51\cellcolor{lightblue!5} & 3.75±0.74\cellcolor{lightblue!37} & 3.42±0.88\cellcolor{lightblue!20} & 3.29±0.62\cellcolor{lightblue!14} & 3.17±0.70\cellcolor{lightblue!8} \\
Touch Pen & 2.43±1.44\cellcolor{lightblue!64} & 2.70±1.52\cellcolor{lightblue!57} & 2.87±1.06\cellcolor{lightblue!53} & 3.00±1.17\cellcolor{lightblue!50} & 1.78±1.09\cellcolor{lightblue!19} & 1.26±0.62\cellcolor{lightblue!6} & 1.43±0.66\cellcolor{lightblue!10} & 1.39±0.72\cellcolor{lightblue!9} & 1.17±0.49\cellcolor{lightblue!4} & 3.09±0.29\cellcolor{lightblue!4} & 3.13±0.76\cellcolor{lightblue!6} & 3.22±0.90\cellcolor{lightblue!10} & 3.26±0.62\cellcolor{lightblue!13} \\
Touch Fill & 4.00±1.15\cellcolor{lightblue!25} & 4.21±1.08\cellcolor{lightblue!19} & 3.89±1.20\cellcolor{lightblue!27} & 4.26±0.87\cellcolor{lightblue!18} & 1.26±0.56\cellcolor{lightblue!6} & 1.16±0.50\cellcolor{lightblue!3} & 1.21±0.54\cellcolor{lightblue!5} & 1.21±0.54\cellcolor{lightblue!5} & 1.16±0.50\cellcolor{lightblue!3} & 2.84±0.37\cellcolor{lightblue!7} & 2.79±0.42\cellcolor{lightblue!10} & 3.05±0.52\cellcolor{lightblue!2} & 2.79±0.54\cellcolor{lightblue!10} \\
Touch Erase & 2.32±1.17\cellcolor{lightblue!67} & 2.82±1.53\cellcolor{lightblue!54} & 3.00±1.07\cellcolor{lightblue!50} & 3.45±1.10\cellcolor{lightblue!38} & 1.59±0.96\cellcolor{lightblue!14} & 1.41±0.96\cellcolor{lightblue!10} & 1.73±1.03\cellcolor{lightblue!18} & 1.41±0.67\cellcolor{lightblue!10} & 1.18±0.50\cellcolor{lightblue!4} & 3.09±0.29\cellcolor{lightblue!4} & 3.18±0.80\cellcolor{lightblue!9} & 3.41±0.67\cellcolor{lightblue!20} & 3.32±0.65\cellcolor{lightblue!15} \\
Touch Text & 3.46±1.28\cellcolor{lightblue!38} & 3.92±1.38\cellcolor{lightblue!27} & 3.38±1.28\cellcolor{lightblue!40} & 3.79±1.35\cellcolor{lightblue!30} & 1.67±1.09\cellcolor{lightblue!16} & 1.17±0.56\cellcolor{lightblue!4} & 1.33±0.56\cellcolor{lightblue!8} & 1.25±0.53\cellcolor{lightblue!6} & 1.29±0.62\cellcolor{lightblue!7} & 3.25±0.68\cellcolor{lightblue!12} & 3.25±0.79\cellcolor{lightblue!12} & 3.04±0.36\cellcolor{lightblue!2} & 3.25±0.68\cellcolor{lightblue!12} \\
Touch Sequence & 3.86±1.13\cellcolor{lightblue!28} & 4.23±1.02\cellcolor{lightblue!19} & 3.68±1.29\cellcolor{lightblue!32} & 4.27±0.94\cellcolor{lightblue!18} & 1.32±0.65\cellcolor{lightblue!7} & 1.14±0.47\cellcolor{lightblue!3} & 1.23±0.53\cellcolor{lightblue!5} & 1.18±0.50\cellcolor{lightblue!4} & 1.18±0.50\cellcolor{lightblue!4} & 3.14±0.77\cellcolor{lightblue!6} & 3.05±0.58\cellcolor{lightblue!2} & 3.05±0.38\cellcolor{lightblue!2} & 3.09±0.43\cellcolor{lightblue!4} \\
\midrule
VR Move & 4.13±1.25\cellcolor{lightblue!21} & 4.52±0.99\cellcolor{lightblue!11} & 4.09±1.24\cellcolor{lightblue!22} & 4.52±0.99\cellcolor{lightblue!11} & 1.13±0.46\cellcolor{lightblue!3} & 1.09±0.42\cellcolor{lightblue!2} & 1.22±0.60\cellcolor{lightblue!5} & 1.17±0.49\cellcolor{lightblue!4} & 1.43±0.66\cellcolor{lightblue!10} & 2.96±0.37\cellcolor{lightblue!2} & 2.87±0.46\cellcolor{lightblue!6} & 3.00±0.52\cellcolor{lightblue!0} & 2.87±0.46\cellcolor{lightblue!6} \\
VR Pen & 3.75±1.36\cellcolor{lightblue!31} & 4.21±1.14\cellcolor{lightblue!19} & 3.83±1.34\cellcolor{lightblue!29} & 4.29±0.91\cellcolor{lightblue!17} & 1.38±0.71\cellcolor{lightblue!9} & 1.29±0.75\cellcolor{lightblue!7} & 1.42±0.72\cellcolor{lightblue!10} & 1.29±0.62\cellcolor{lightblue!7} & 1.29±0.62\cellcolor{lightblue!7} & 2.96±0.46\cellcolor{lightblue!2} & 2.88±0.45\cellcolor{lightblue!6} & 3.08±0.78\cellcolor{lightblue!4} & 2.92±0.50\cellcolor{lightblue!4} \\
VR Fill & 4.27±0.94\cellcolor{lightblue!18} & 4.55±0.86\cellcolor{lightblue!11} & 4.00±1.23\cellcolor{lightblue!25} & 4.64±0.49\cellcolor{lightblue!9} & 1.23±0.61\cellcolor{lightblue!5} & 1.18±0.50\cellcolor{lightblue!4} & 1.14±0.47\cellcolor{lightblue!3} & 1.18±0.50\cellcolor{lightblue!4} & 1.27±0.63\cellcolor{lightblue!6} & 2.95±0.38\cellcolor{lightblue!2} & 2.82±0.39\cellcolor{lightblue!9} & 3.05±0.49\cellcolor{lightblue!2} & 2.91±0.29\cellcolor{lightblue!4} \\
VR Erase & 3.77±1.19\cellcolor{lightblue!30} & 4.00±1.31\cellcolor{lightblue!25} & 3.82±1.26\cellcolor{lightblue!29} & 4.27±0.88\cellcolor{lightblue!18} & 1.36±0.73\cellcolor{lightblue!9} & 1.45±0.96\cellcolor{lightblue!11} & 1.45±0.86\cellcolor{lightblue!11} & 1.32±0.65\cellcolor{lightblue!7} & 1.36±0.66\cellcolor{lightblue!9} & 2.95±0.38\cellcolor{lightblue!2} & 3.00±0.62\cellcolor{lightblue!0} & 3.00±0.62\cellcolor{lightblue!0} & 2.95±0.58\cellcolor{lightblue!2} \\
VR Text& 3.50±1.14\cellcolor{lightblue!37} & 3.55±0.96\cellcolor{lightblue!36} & 3.77±1.19\cellcolor{lightblue!30} & 3.95±0.72\cellcolor{lightblue!26} & 1.23±0.61\cellcolor{lightblue!5} & 1.09±0.43\cellcolor{lightblue!2} & 1.27±0.63\cellcolor{lightblue!6} & 1.27±0.63\cellcolor{lightblue!6} & 1.41±0.67\cellcolor{lightblue!10} & 3.14±0.64\cellcolor{lightblue!6} & 3.05±0.58\cellcolor{lightblue!2} & 3.27±0.63\cellcolor{lightblue!13} & 3.09±0.43\cellcolor{lightblue!4} \\
\bottomrule
\end{tabular}
\end{adjustbox}
\end{table*}


\subsection{Subjective Measures}

The SSQ scores for VR in \autoref{fig:vrSickness} and the touchscreen in \autoref{fig:touchSickness} are represented as box plots. Here, higher scores are undesirable and values above 20 are typically interpreted as bad~\cite{ssqOriginal}. It can be observed that the median for all sickness values lies at 0. The box contains every value that falls in the second and third quartile. Thus, every point above the upper end of the box falls within the worst 25\% of reported values. The whiskers above and below each pox extend to include the minimum and maximum values. Anything outside the whiskers was detected as an outlier. It should be noted that values below 0 are achieved because participants reported more sickness pre-study when compared to post-study reporting. This should not be interpreted that our simulator is helpful in reducing sickness, but rather the sickness diminishing over time~\cite{ssqFix}.
% 
The results of the DAQ are reported in \autoref{tab:daq}. Since the optimal values differ, we colored each cell to indicate how unfavorable the judgements of the participants were. For the last 4 columns, the reported value can either be too high (above 3), or it can be too low (below 3). This allows the participants, e.g., the option to describe the mental effort as too little, which could lead to slips, or too high, which tires users more than necessary.
%
We applied the IPQ to assess the experienced presence of our volunteers. The users gave a list of statements a score from 0 to 7. Different statements pertain to distinct categories within the IPQ. The resulting scores can be found in \autoref{fig:presence}. While presence is higher in VR than when using the touchscreen, the general presence is the highest for both.
%
The last questionnaire provided the users with a direct opportunity to rate the tools. The resulting SUS scores which are depicted in \autoref{fig:usability} range from 0 to 100. In general, VR tools were rated the best.
%
Users also had to choose their favorite task. From the 24 participants, 14 chose the teeth inspection, 5 the digestion and 4 the ablation task. Only one participant liked any of the educators tasks the best, which was the port in this case. At last, participants rated each of the annotations regarding their utility, as seen in \autoref{fig:annotationsUse}.
