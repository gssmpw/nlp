\section{Conclusion}
In conclusion, our quantitative user study, involving 24 participants, assessed the impact of different annotations on learning and teaching within touchscreen and VR environments. Through this investigation, we aimed to explore the usability of annotations and interfaces designed as outlined in this paper.

The study revealed valuable insights into the usability of annotations in mixed environments. Participants, tasked with annotating scenes using the touchscreen interface, demonstrated the ability to transfer medical textbook material. While acknowledging the potential for refinement in the touch interface, the study affirmed its utility and showcased a broad range of possible applications. As learners immersed in the VR environment, participants expressed a strong affinity for the appeal of our VR setting. Their feedback provided valuable suggestions for diverse areas of application within the virtual realm. This positive reception underscores the potential of our work in creating engaging and effective learning experiences in virtual reality mixed with touchscreen educators.

Our work contributes to the establishment of a foundational framework for the utilization of annotations in both learning and teaching contexts within mixed environments that integrate touchscreens and virtual reality technology. By bridging the gap between these two modalities, our research opens avenues for innovative educational applications. While our study demonstrated promising results, we acknowledge certain limitations, such as the need for further refinement of the touch interface. Future research could explore enhancements to address these limitations and delve into specific areas suggested by participants for VR applications. 

In summary, this study advances our understanding of the practical implementation of annotations in mixed reality educational settings. The positive feedback from participants and the demonstrated usability of our proposed interface underscore the potential for enriching learning and teaching experiences through the synergy of touchscreens and virtual reality.

