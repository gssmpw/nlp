\section{Introduction}
The advent of Virtual Reality (VR) devices has ushered in a new era, marked by the growing popularity of VR applications across diverse domains ~\cite{vrPopularity}. In particular, the realm of education and training has witnessed a surge in innovation, with an array of applications leveraging the immersive potential of VR ~\cite{reviewVRTrainings, vrELearning}. Amidst this burgeoning landscape, our focus lies on harnessing the unique capabilities of VR for educational purposes, emphasizing a mixed environment that intertwines virtual reality with tangible interactions facilitated by large touchscreens.

The gamut of applications in VR education ranges from purely virtual worlds to hybrid setups incorporating both virtual and physical elements ~\cite{arClimbing, arAmbulance, laparoscopyInstrumentVR, weaponTrainingVR}. A common thread among these applications is the creation of virtual training environments, each adopting diverse approaches with varying levels of complexity ~\cite{vrTrainingRecording, coAssemble}. In our research, we advocate for a mixed environment where students are immersed in VR, while educators interface through a large touchscreen, similar to other existing research ~\cite{largeDisplays1, largeDisplays2}. This approach further enables the future creation of systems where educators can harness the benefits of VR in teaching, without encountering some of its problems. For instance, educators would still have the possibility to observe and engage with students in the real time, fostering a holistic understanding of student activities and enhancing assistance during the learning process.
%
The incorporation of large touchscreens into the educational paradigm introduces a dynamic layer to the teaching and learning process. During synchronous classes, educators gain the ability to closely observe real-world student actions, a perspective not afforded when immersed in virtual reality. This close monitoring enhances educators' grasp of students' activities, enabling more effective assistance when queries arise ~\cite{gesturalInterfaces}. Additionally, educators can maintain interaction with the physical environment, accessing various supplementary sources, including printed materials. To address the occasional need for additional information beyond the virtual training environment, educators can view a replicated representation of learners' perspectives on the touchscreen interface, leveraging its expansive display capabilities to accommodate multiple views simultaneously.

While touch interfaces are commonplace and well-studied, especially in the context of smartphones ~\cite{gesturalInterfaces}, their application in the realm of 3D controls is an active area of research ~\cite{multiTouch3D}. Our contribution extends this line of inquiry by proposing novel techniques for effective multitouch recognition on touchscreens within mixed educational spaces.
%
On the side of the learners, the benefits of immersive VR experiences are manifold. Research indicates that VR aids users in gaining insights across diverse fields ~\cite{vrLearningSerious}. Specialized applications targeting language learning ~\cite{vrLearningLanguage} and spatial relationship comprehension ~\cite{vrLearningSpatial} showcase the versatility of VR in enhancing the learning process ~\cite{kowalewski2018laptrain,yiannakopoulou2015virtual,vrFasterLearning}. Our work builds upon these foundations, presenting an alternative approach to generating training scenarios that balances ease of use with educational efficacy.
%
To achieve this balance, we empower educators to work with carefully crafted static scenes enriched with annotations ~\cite{annotations1, annotations2}. Unlike conventional approaches, our method draws inspiration from the annotation framework proposed by Marques et al.~\cite{vuforiaAnnotations}, adapting it to the unique demands of classroom and training course settings. These annotations serve as dynamic elements within the virtual space, facilitating asynchronous interactions between educators and learners.
%
Our empirical evaluation delves into the effectiveness of asynchronous interactions facilitated by our proposed mixed environment. The study, involving 24 participants, scrutinizes the ease of creating courses and the utility of generated annotations. This assessment focuses on the educators' effectiveness in conveying information and learners' ability to comprehend the annotated content.

In summary, our contributions encompass the design of a versatile mixed space compatible with different device types, the proposal of a method for rapidly enriching scenes with annotations, and the results of a quantitative study. Through these contributions, we aim to pave the way for the development of educational applications wherein a single teacher can interact synchronously with learners in virtual spaces.
In the subsequent sections, we delve into the intricacies of our mixed environment design, the annotation methodology, and the study's outcomes, providing valuable insights for the advancement of educational technology. This paper is accompanied by a video where each unique feature of the system we will propose is visible.
\begin{figure}
 \centering
 \includegraphics[width=\textwidth]{figures/Teaser2.png}
 \caption{Views when using the system we propose in this paper. The two roles (left: educator on touchscreen, right: learner in VR) share the same environment but see different versions of user interfaces.}
 \label{fig:teaser}
\end{figure}