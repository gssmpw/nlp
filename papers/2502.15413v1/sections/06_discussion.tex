\section{Discussion}


\subsection{Objective Measures}

We recorded the duration each participant spent in the different scenarios, which revealed greatly varying timescales, with the teachers' tasks taking around 3–5 times longer than the learners' tasks. While they not directly comparable, they were designed to be of approximately equal complexity. When reviewing the coefficients of variation, the time needed to plan lessons appears to be equally predictable, indicating that while planning a lesson for learners to complete, the duration most students will take is not too variable. The quick completion times for learners also hint at good comprehensibility of the annotations. None of the participants were observed to be confused for more than a few seconds while being in the learners' role. The educators' role proved more challenging, which is reflected in the higher duration per task. Reading through textbook material of a different specialty proved to be challenging, as participants reported.

As discussed in the results, educators spend 46.8\% percent of their time using the movement tool. Ideally, this would be used only very briefly between the use of other tools. This statistic indicates that the movement techniques utilized in this paper can still be optimized. Next in line is the text tool, which is explained by the fact that entering text takes time. It should also be considered, that only a few of the participants were already familiar with the topics covered. Participants frequently consulted the textbook excerpt again while already in text mode. Tool use in VR was very limited. This was intentional, since even while learners had access to the tools, they mostly observed already existing annotations by reading through text.

\subsection{Subjective Measures}

The SSQ provides us with measurements regarding simulator sickness. Only a few of our test subjects experienced simulator sickness. The median value for every scale in the SSQ is observed to be 0. Values below 0 should be considered as 0 when interpreting the SSQ~\cite{ssqFix}. Though, the disorientation subscale shows that more than 25\% of our test subjects at least some disorientation experienced. As was to be expected, the touchscreen was not at all sickening, according to the SSQ. This gives us assurance, that the sickness experienced using VR was not induced from problems with the virtual environment we designed, but rather from the use of VR. Only 2 out of our 24 participants experienced a total sickness score higher than 20, both of which reported having little to no prior experience with VR, with which they could have compared their experiences during the study. Their sickness could be explained by the human factor ~\cite{vrSicknessReasons}.

In \autoref{tab:daq} we observe a clear divide between the VR and touchscreen tools. Users clearly favored the VR tools, while the three touch tools to move, paint and erase seem to have issues in their current implementation. The move tool was reported to cause finger fatigue and require too much mental and physical effort. Most of the touch tools were reported to cause slight arm and shoulder fatigue, which is probably caused by the fact that gestures require you to move your whole arms. The effort, force, and speed were considered excellent for the VR tools.

To measure presence, we consider the IPQ scores. When comparing each of the scores between VR and the touchscreen, VR provides more presence. With a median of 6, the general presence reported for VR is very high. As discussed earlier, this was desired since it is shown to aid in learning~\cite{presenceLearning}. When using the touchscreen, participants experienced less presence. This ensures that they are still aware of their surroundings and, later on, aware of the learners in their class.

The SUS scores once again gave participants the option to share their experiences with the tools. The divide between VR and touchscreen tools is a bit less clear now. Users communicated orally during their study the reasons for dissatisfaction. Regarding the movement on the touchscreen, many had issues with its complexity. Participants used the provided height-slider sporadically. It is possible, that after a longer acclimatization period, they would adapt to it better. Many expressed wishes to rotate the camera around a point in space, teleport to specific points, use pinch to move forward or add GUI-based directional pads. Compared to the simplicity of moving in VR, this underlines the need for more research into 3D touch movement controls. Issues with the touch pen and erase mostly stem from inconsistent pen sizes and problems with the library causing the pen to draw through body parts. While some volunteers disliked having to use the touchscreen for keyboard input when writing text, others found it dynamic and adapted the size and layout of the keys to their needs. Based on this situation, we recommend providing a physical wireless keyboard when users wish to enter more than a few characters. When participants were in the role of the learner, they sometimes observed text boxes to be occluded by other parts of the scene but could mitigate this issue by moving around, which caused the text box to rotate. Still, other solutions like text boxes that draw on top of other geometry, possibly only when pointed at, come to mind and could be explored in future research.

Users enjoyed experiencing the courses in the role of the learner. This was reflected in the choice of their favorite task, where 23 out of 24 chose one of the learner tasks. The study was concluded with \autoref{fig:annotationsUse}, which demonstrates the perceived utility of the provided annotation techniques for learning contexts. For each of the annotations, at least a simple majority agreed that the annotation was very useful. The text and sequence tools were seen even more favorable, with more than two out of three agreeing that they are very useful. While some could argue that this is to be expected, since text is the basis of communicating in most education, it also adds certainty that our proposed text system is at least adequate.
\begin{figure}[tb]
 \centering
 \includegraphics[width=0.8\columnwidth]{figures/annotationRating.png}
 \caption{Percentage of user reports for each of the utility ratings of specific annotations in the context of learning.}
 \label{fig:annotationsUse}
\end{figure}


\subsection{Limitations}

While this study provides valuable insights into the utilization of annotations in virtual reality environments, it is crucial to acknowledge some considerations that may have influenced the outcomes.
%
One aspect is the measurement of durations during the study. The method employed involved capturing the start and end times of events, which, in some cases, included passive phases. For instance, reported times for tool usage on the touchscreen encompassed periods when participants had the tool active without utilizing it. This method might have introduced variations in the recorded durations. 
%
An additional challenge surfaced regarding time constraints. Although the study was scheduled for two hours per participant, some individuals required more time. Consequently, this occasionally led to brief interruptions when accommodating the next participant. While this dynamic could potentially impact the continuity of the study, it underscores the engagement and thoroughness of participants.
%
A potential factor influencing participant experiences was the number and length of forms they were required to complete. Feedback from some participants indicated that the forms were perceived as lengthy, potentially contributing to moments of fatigue. It is essential to recognize that participant fatigue may have influenced the quality of responses in later stages of the study.