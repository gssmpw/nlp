\newcommand{\stylefeaturecategoriesfig}{
    \begin{figure*}[ht!]
        \centering
        \resizebox{0.95\textwidth}{!}{
            \begin{tikzpicture}[
                font=\scriptsize,
                category/.style={rectangle, draw, minimum width=4cm, inner sep=10pt, align=left, rounded corners=3pt}, % Same width within columns
                feature/.style={align=left}
            ]
            
            \definecolor{color1}{RGB}{230,230,250}  % Light lavender
            \definecolor{color2}{RGB}{255,239,213}  % Light peach
            \definecolor{color3}{RGB}{240,255,240}  % Light mint green
            \definecolor{color4}{RGB}{255,250,205}  % Light yellow
            \definecolor{color5}{RGB}{224,255,255}  % Light cyan
            \definecolor{color6}{RGB}{255,228,225}  % Light pink
            \definecolor{color7}{RGB}{245,245,220}  % Light beige
            \definecolor{color8}{RGB}{255,245,238}  % Light peach puff
            
            \node[category, fill=color1] (syntactic) at (-3, 0) {
                \textbf{Syntactic Features} \\ 
                Usage of Conjunctions \\
                Usage of Articles \\
                Frequent Usage of Function Words \\
                Usage of Personal Pronouns \\
                Usage of Pronouns \\
                Usage of Active Voice \\
                Usage of Contractions \\
                Frequent Usage of Determiners \\
                Usage of Prepositions
            };
            
            \node[category, fill=color2, below=0.25cm of syntactic] (graphical) {
                \textbf{Graphical and Digital Features} \\ 
                Usage of Numerical Substitution \\
                Usage of Uppercase Letters \\
                Usage of Text Emojis \\
                Usage of Emojis \\
                Presence of Misspelled Words \\
                Usage of Only Uppercase Letters \\
                Usage of Only Lowercase Letters \\
                Usage of Numerical Digits \\
                Frequent Usage of Punctuation
            };
            
            
            \node[category, fill=color3, right=0.5cm of syntactic] (emotional) at (-1, 0.5) {
                \textbf{Emotional and Cognitive Features} \\ 
                Positive Sentiment Expression \\
                Usage of Words Indicating Affective vs. Perceptual Processes \\
                Usage of Words Indicating Cognitive vs. Perceptual Processes \\
                Usage of Words Indicating Affective vs. Cognitive Processes \\
                Usage of Certain Tone
            };
    
            
            \node[category, fill=color5, right=0.5cm of emotional] (stylistic) {
                \textbf{Stylistic and Aesthetic Features} \\ 
                Usage of Metaphors \\
                Usage of Formal Tone \\
                Incorporation of Humor \\
                Fluency in Sentence Construction \\
                Complex Sentence Structure \\
                Usage of Sarcasm
            };
            
            \node[category, fill=color4, below=0.25cm of emotional] (social) {
                \textbf{Social and Interpersonal Features} \\ 
                Usage of Offensive Language \\
                Usage of Self-Focused Language vs. You-Focused \\
                Usage of Self-Focused Perspective vs. Third-person Singular\\
                Usage of Self-Focused Language vs. Inclusive-focused \\
                Usage of Self-Focused Language vs. Audience-focused  \\
                Usage of Polite Tone
            };
            
            
            \node[category, fill=color6, below=0.25cm of stylistic] (lexical) {
                \textbf{Lexical Features} \\ 
                Usage of Long Words \\
                Usage of Nominalizations \\
                Frequent Usage of Common Verbs
            };
            
            \node[category, fill=color7, text width=5cm] (temporal) at (5.4, -4.7) {
                \textbf{Temporal and Aspectual Features} \\ 
                Usage of Present-focused vs. Future-focused \\
                Usage of Present-focused vs. Past-focused
            };
            
            \end{tikzpicture}
        }
        \caption{We generate synthetic parallel examples to train \textsc{StyleDistance} for a wide range of style features in seven linguistic and stylistic categories. Further details on these features can be found in Appendix \ref{sec:appendix:stylefeatures}.}
        \label{fig:stylefeaturecategories}
    \end{figure*}
}
