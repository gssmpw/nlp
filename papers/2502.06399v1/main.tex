\documentclass{article}
\usepackage[utf8]{inputenc}

%\usepackage{amsaddr}
\usepackage{subcaption}
\usepackage{physics}
\usepackage{amsmath,graphicx}
\usepackage{mathtools}
\mathtoolsset{showonlyrefs=true}
%\usepackage{times}
%\usepackage{comment}
\usepackage{float}
\usepackage{amsmath}
\usepackage{amsfonts}
\usepackage{amsthm}
\usepackage{enumitem}  
\usepackage{xfrac}
\usepackage[ruled,vlined]{algorithm2e}
%\usepackage{hyperref}
% \hypersetup{
%     colorlinks=true,
%     linkcolor=blue,
%     filecolor=blue,
%     citecolor = blue,      
%     urlcolor=blue,
%     }
\usepackage{authblk}

\usepackage[backref=page]{hyperref}
%%%% these patches ensure that the backrefs point to the actual occurrences of the citations in the text, not just the page or section in which they appeared
%%%% https://tex.stackexchange.com/questions/54541/precise-back-reference-target-with-hyperref-and-backref
%%%% BEGIN BACKREF DIRECT PATCH, apply these AFTER loading hyperref package with appropriate backref option
% The following options are provided for the patch, currently with a poor interface!
% * If there are multiple cites on the same (page|section) (depending on backref mode),
%   should we show only the first one or should we show them all?
\newif\ifbackrefshowonlyfirst
\backrefshowonlyfirstfalse
%\backrefshowonlyfirsttrue
%%%% end of options
%
% hyperref is essential for this patch to make any sense, so it is not unreasonable to request it be loaded before applying the patch
\makeatletter
% 1. insert a phantomsection before every cite, so hyperref has something to target
%    * in case natbib is loaded. hyperref provides an appropriate hook so this should be safe, and we don't even need to check if natbib is loaded!
\let\BR@direct@old@hyper@natlinkstart\hyper@natlinkstart
\renewcommand*{\hyper@natlinkstart}{\phantomsection\BR@direct@old@hyper@natlinkstart}% note that the anchor will appear after any brackets at the start of the citation, but that's not really a big issue?
%    * if natbib isn't used, backref lets \@citex to \BR@citex during \AtBeginDocument
%      so just patch \BR@citex
\let\BR@direct@oldBR@citex\BR@citex
\renewcommand*{\BR@citex}{\phantomsection\BR@direct@oldBR@citex}%

% 2. if using page numbers, show the page number but still hyperlink to the phantomsection instead of just the page!
\long\def\hyper@page@BR@direct@ref#1#2#3{p. \hyperlink{#3}{#1}}

% check which package option the user loaded (pages (hyperpageref) or sections (hyperref)?)
\ifx\backrefxxx\hyper@page@backref
    % they wanted pages! make sure they get our re-definition
    \let\backrefxxx\hyper@page@BR@direct@ref
    \ifbackrefshowonlyfirst
        %\let\backrefxxxdupe\hyper@page@backref% test only the page number
        \newcommand*{\backrefxxxdupe}[3]{#1}% test only the page number
    \fi
\else
    \ifbackrefshowonlyfirst
        \newcommand*{\backrefxxxdupe}[3]{#2}% test only the section name
    \fi
\fi

% 3. now make sure that even if there is no numbered section, the hyperref's still work instead of going to the start of the document!
\RequirePackage{etoolbox}
\patchcmd{\Hy@backout}{Doc-Start}{\@currentHref}{}{\errmessage{I can't seem to patch backref}}
\makeatother
%%%% END BACKREF PATCHES

\usepackage[numbers]{natbib}
\bibliographystyle{abbrvnat}

\usepackage{blindtext} % for dummy text

\makeatletter
    \setlength\@fptop{0\p@}
\makeatother

\usepackage{fourier}

\newtheorem{theorem}{Theorem}[section]

\newtheorem{definition}{Definition}[section]

\newtheorem{assumption}{Assumption}[section]
\newtheorem{lemma}[theorem]{Lemma}
\newtheorem{corollary}[theorem]{Corollary}
\newtheorem{proposition}{Proposition}[section]
\newtheorem{remark}{Remark}[section]

\newcommand{\inp}[0]{{z}}
\newcommand{\inpS}[0]{{\mathcal{Z}}}
%\newcommand{\inner}[2]{ \left\langle #1 \; , \; #2 \right\rangle }
% by Yen-Huan
\usepackage{interval}
\intervalconfig{soft open fences}
\usepackage{xcolor}
\usepackage{braket}
\usepackage{changes}
% \newcommand{\norm}[1]{\Vert #1 \Vert}
\DeclareMathOperator{\inte}{Int}
\DeclareMathOperator{\dom}{dom}
\DeclareMathOperator{\supp}{supp}
\DeclareMathOperator*{\argmin}{ \arg \min }

\renewcommand{\intercal}{\mathsf{\scriptscriptstyle{T}}}

\title{A Linearly Convergent Algorithm for Computing the Petz-Augustin Information}

\author[1]{Chun-Neng~Chu}
\author[2]{Wei-Fu~Tseng}
\author[1,2,3]{Yen-Huan Li}

\affil[1]{Department of Computer Science and Information Engineering,\protect\\National Taiwan University}
\affil[2]{Department of Mathematics, National Taiwan University}
\affil[3]{Center for Quantum Science and Engineering, \protect\\ National Taiwan University}

\date{}
\begin{document}

\maketitle

\begin{abstract}
We propose an iterative algorithm for computing the Petz-Augustin information of order $\alpha\in(1/2,1)\cup(1,\infty)$. 
The optimization error is guaranteed to converge at a rate of $O\left(\abs{1-1/\alpha}^T\right)$, where $T$ is the number of iterations. 
Let $n$ denote the cardinality of the input alphabet of the classical-quantum channel, and $d$ the dimension of the quantum states. 
The algorithm has an initialization time complexity of $O\left(n d^{3}\right)$ 
%. 
%, 
%where $n$ is the cardinality of the input alphabet of the classical-quantum channel, and $d$ is the dimension of the quantum state. 
% The per-iteration time complexity of the algorithm is $O\left(n d^{2}+d^3\right)$.  
and a per-iteration time complexity of $O\left(n d^{2}+d^3\right)$. 
To the best of our knowledge, this is the first algorithm for computing the Petz-Augustin information with a non-asymptotic convergence guarantee. 
\end{abstract}

%\section{Introduction}
%
%Let $\alpha \in ( 0, 1 ) \cup ( 1, \infty )$. 
%We consider the problem of computing the order-$\alpha$ \emph{Petz-Augustin information}, given by
%\begin{equation}
%f_\star = \min_{\sigma \in \mathcal{D}_d} f ( \sigma ), \quad f ( \sigma ) \coloneqq \mathsf{E}_\omega \left[ D_\alpha ( \omega \Vert \sigma )  \right] . \label{eq_problom} 
%\end{equation}
%Here, $\mathcal{D}_d$ denotes the set of quantum density matrices--Hermitian positive semi-definite matrices of unit traces---in $\mathbb{C}^{d \times d}$; 
%$\omega$ is a random variable taking values in a finite set $\Omega \subset \mathcal{D}_d$; 
%for any two density matrices $\rho$ and $\sigma$, the notation $D_\alpha ( \rho \Vert \sigma )$ denotes the order-$\alpha$ Petz-R\'enyi divergence \citep{Petz1986}: 
%\[
%D_\alpha ( \rho \Vert \sigma ) \coloneqq \frac{1}{1 - \alpha} \log \tr ( \rho^\alpha \sigma^{1 - \alpha} ) , 
%\]
%where $\sigma$ is allowed to have trace not equal to $1$. 
%We assume that the union of $\supp \omega$ equals $\mathbb{C}^d$; 
%otherwise, one can consider a lower-dimensional subspace of $\mathbb{C}^d$. 
%The Petz-Augustin information is crucial in, for example, characterizing the error exponents for quantum source and channel coding as well as privacy amplification \cite{Cheng2024a,Cheng2022,Cheng2022a,Mosonyi2017,Shen2023}. 
%% Remarkably, the Petz-Augustin information converges to the quantum relative entropy when $\alpha \to 1$. 
%
%We will focus on the case where $\alpha \in ( 0, 1 ) \cup ( 1, 2 )$, as then the optimization problem \eqref{eq_problom} is convex \cite{Mosonyi2017}. 
%However, despite the convexity, \citet{You2022} have shown that neither the objective function $f$ nor its gradient is Lipschitz continuous, violating standard assumptions in analyzing first-order optimization algorithms. 
%As a result, even the convergence of gradient descent, arguably the most basic first-order algorithm, for the optimization problem \eqref{eq_problom} has been unclear.  
%Although optimization algorithms exploiting higher-order information may work \citep{Nesterov2018a}, their computational complexities grow fast with increasing dimension $d$, making them prohibitive for quantum information applications.  
%
%In this paper, we propose a simple first-order algorithm for solving the optimization problem \eqref{eq_problom} with $\alpha \in ( 1, 2 ]$. 
%
%We prove that the algorithm outputs an $\varepsilon$-approximate solution in $O ( c_\alpha \log ( 1 / \varepsilon ) )$ iterations, where $c_\alpha \coloneqq \alpha / ( \alpha - 1 )$. 
%% Its per-iteration time complexity is $O ( \abs{ \Omega } d^2 + d^3 )$, where $\abs{ \Omega }$ denotes the cardinality of the set $\Omega$.
%To the best of our knowledge, this is the first algorithm for computing the Petz-Augustin information that possesses a non-asymptotic error guarantee. 
%
%\section{Related Work}
%
%We recover the classical Augustin information when all matrices in the optimization problem \eqref{eq_problom} share the same eigenbasis. 
%The Augustin information was proposed by \citet{Augustin1969}.
%\citet{Augustin1978} also proposed a fixed point iteration, which was later proved to asymptotically converge 

\section{Introduction}
Several fundamental quantities in quantum information theory, including the quantum channel capacity, the relative entropy of entanglement, and the quantum R\'{e}nyi information, are formulated as optimization problems and lack closed-form expressions \cite{Hayashi2017,Wilde2017}. 
%These two are added by Prof. Li.
%p383 (8.77) entanglement
%p393  Further, Donald and Horodecki [33] showed Condition E3 for entanglement of relative entropy Er,S(ρ)(They call it relative entropy of entanglement)
%p159 (4.9) cq channel capacity
While their operational meanings and mathematical properties have been explored in depth, much less is known %regarding 
about
how to efficiently compute these quantities. 
Recent works 
% have focused 
have been primarily focusing 
on 
the computation of quantities involving 
the quantum relative entropy. 
For instance, the self-concordance property of the quantum relative entropy has been exploited to facilitate the use of interior-point methods \citep{Fawzi2023,Faybusovich2017,He2024b}. 
Various 
representations of the matrix logarithm and quantum relative entropy
have been established, enabling numerous optimization tasks to be solved by semidefinite programming \cite{Brown2024,Chandrasekaran2017,Fawzi2022,Fawzi2017a,Fawzi2019,Frenkel2023,He2024b,Huang2024,Jencova2024,Kossmann2024a,Kossmann2024b}. 
The classical Blahut-Arimoto algorithm has been generalized and quite well studied for computing quantum channel capacities \cite{Hayashi2024,He2024a,Li2019b,Nagaoka1998,Ramakrishnan2021}. 

A natural generalization of the quantum relative entropy is the quantum R\'{e}nyi divergence. 
Due to the non-commutative nature of the quantum setup, there are %several 
multiple notions of quantum R\'{e}nyi divergences, 
such as the Petz-R{\'e}nyi divergence \cite{Petz1986}, the sandwiched R{\'e}nyi divergence \cite{Muller2013,Wilde2014}, the geometric R\'{e}nyi divergence \cite{Fang2021,Matsumoto2018}, and the \#-R{\'e}nyi divergence \cite{Fawzi2021}.
Unfortunately, research on optimization involving quantum R{\'e}nyi divergences remains limited. 
\citet{Fawzi2021}, 
%p2 Using the joint concavity of the matrix geometric mean, the optimization program in (5) is convex and for rational values of α it can be expressed as a semidefinite program [14, 33
%p20 Using the fact that Deregα (N kM) ≤ D#α (N kM) and the fact that the set of channels VΘ is representable by a semidefinite program, we obtain efficiently computable bound
as well as 
\citet{Fang2021}, 
%p4 any optimization minM∈V Dbα(N kM) can be computed as a semidefinite program if V is a set of channels characterized by semidefinite conditions.
formulated computational tasks involving the \#-R{\'e}nyi divergence and the geometric R{\'e}nyi divergence as semidefinite programs.
\citet{Liu2024}, by utilizing a specific approximation of the matrix geometric mean,  proposed a new class of algorithms that can be used to compute the geometric R\'{e}nyi divergence. 
%p5 "We introduce quantum algorithms for learning the metric in machine learning, by phrasing this as an optimisation problem using a geometric perspective."
%p6 the first quantum algorithm for computing the geometric α-R´enyi relative entropy,
%p1 For example, we show how to use them in the estimation of geometric ´enyi relative entropies
% \citet{You2022} proposed entropic mirror descent with the Polyak step size to compute various forms of the quantum R\'{e}nyi information and the quantum Augustin information. 
% Notably, the algorithm proposed by \citet{You2022} is a general-purpose convex optimization algorithm that does not exploit the specific structures of quantum R\'{e}nyi divergences and lacks a non-asymptotic convergence guarantee.
\citet{You2022} proposed computing various forms of the quantum R\'{e}nyi information and the quantum Augustin information via mirror descent with a Polyak-type step size. 
Notably, their algorithm guarantees only asymptotic convergence and lacks a complexity characterization. 

We are particularly interested in computing the Petz-Augustin information of order $\alpha$, which is defined as the minimum of an expected Petz-R\'{e}nyi divergence of order 
$\alpha$ over the set of quantum density matrices \eqref{eq:Petz_Augustin}.
The Petz-Augustin information is a generalization of the quantum mutual information and has an application in characterizing the sphere-packing exponent of classical-quantum channel coding \cite{Dalai2014, Cheng2019}.
Unlike the quantum mutual information, the Petz-Augustin information does not have a closed-form expression.
The optimization problem defining the Petz-Augustin information is convex for orders $\alpha \in (0, 1) \cup (1, 2]$ \cite{Mosonyi2017}. 
% , and we focus on the regime $\alpha\in\left(1,2\right]$.
However, the gradients and Hessians of R{\'e}nyi divergences are unbounded \cite[Propositions 3.1 and 3.2]{You2022}, 
violating standard assumptions in convex optimization literature. 
As a result, standard first-order optimization algorithms and their theoretical guarantees do not directly apply.
To the best of our knowledge, no existing first-order optimization method for computing the Petz-Augustin information of any order 
has a non-asymptotic convergence guarantee.  
Although second-order methods \cite{Nesterov2018a} may be applicable, their per-iteration time complexities grow rapidly with the dimension of the quantum state, and hence do not scale well with the number of qubits.

We propose a simple iteration rule in Section \ref{eq:CohenIteration}, tailored for computing the Petz-Augustin information.
We prove that the optimization error of the proposed iteration rule converges at a rate of $O\left(\abs{1-1/\alpha}^T\right)$ for all $\alpha \in (1/2,1)\cup(1,\infty)$, where $T$ denotes the number of iterations (Theorem \ref{ineq:FvalConv}).
This result is achieved by establishing the contractive property of the iterates with respect to the Thompson metric (Lemma \ref{ineq:Contract}).
Notably, our algorithm is computationally cheaper than standard first-order methods, as
the time complexity of computing a gradient is $O\left(nd^2+d^4\right)$ \cite{You2022}, whereas the per-iteration time complexity of our algorithm is only $O\left(n d^2+d^3\right)$, where $n$ is the cardinality of the input alphabet of the classical-quantum channel, and $d$ is the dimension of the quantum state.

%We shall note that our iteration rule is inspired by an efficient algorithm proposed by \citet{Cohen2015} for computing the \emph{$\ell_p$-Lewis weights}.
%The $\ell_p$-Lewis weights, along with their variants, have applications in $\ell_1$-regression \cite{Parulekar2021,Durfee2018} and linear programming \cite{Lee2020}.
%For a comprehensive treatment of the $\ell_p$-Lewis weights, we refer readers to \citet{Cohen2015}.
%In Section \ref{subsec:RelationWithCP}, we discuss the relation between the $\ell_p$-Lewis weights and the Petz-Augustin information.
%Although there are similarities, we clarify in Section \ref{subsec:RelationWithCP} that our results cannot be trivially derived from those of \citet{Cohen2015}.
%In Section \ref{subsec:Conv}, we address the difficulties encountered in adapting the algorithm proposed by \citet{Cohen2015} to compute the Petz-Augustin information, with the hope that further studies on more general optimization problems involving quantum R\'{e}nyi divergences will be inspired by our analysis.

%Finally, we note that our algorithm does not cover the case where $\alpha\in(0,1)$, which appears in characterizing the sphere-packing exponent of classical-quantum channel coding \cite{Dalai2014, Cheng2019}. % See (22) in Cheng2019 and (3) in Dalai2014
%However, Lemma \ref{ineq:TraceBoundViolation} indicates that our main theorem cannot be directly extended to orders less than $1$. 
%Generalizing our algorithm for orders less than $1$ remains a direction for future research.




%Although the Petz-Augustin information of order $\alpha \in (1,2]$ currently lacks an operational meaning, it still offers valuable implications. 
%Specifically, there exists another type of quantum Augustin information, known as the \emph{sandwiched Augustin information}, which is upper-bounded by the Petz-Augustin information for any shared order $\alpha > 1$ \cite{Mosonyi2017}. 
%The sandwiched Augustin information of order $\alpha \in (1,2]$ is crucial for characterizing achievability bounds in quantum soft covering and privacy amplification \cite{Cheng2022b,Shen2023}, suggesting that the Petz-Augustin information provides a looser bound.

 
%in
%of %https://link.springer.com/article/10.1007/s00220-017-2928-4
%classical-quantum channel coding. 
%Looking ahead, the Petz-Augustin information has broader applications for orders less than $1$ than for those in $\left(1,2\right]$.
%For example, it is used in characterizing the sphere-packing exponent of classical-quantum channel coding \cite{Dalai2014, Cheng2019}. % See (22) in Cheng2019 and (3) in Dalai2014
%Looking ahead, the Petz-Augustin information of order $\alpha\in(0,1)$ is used to characterize the sphere-packing exponent of classical-quantum channel coding \cite{Dalai2014, Cheng2019}, % See (22) in Cheng2019 and (3) in Dalai2014
%whereas for $\alpha\in\left(1,2\right]$, it currently lacks an operational meaning.
%Our algorithm does not cover the case where $\alpha\in(0,1)$, which appears in characterizing the sphere-packing exponent of classical-quantum channel coding \cite{Dalai2014, Cheng2019}. % See (22) in Cheng2019 and (3) in Dalai2014
%However, Lemma \ref{ineq:TraceBoundViolation} indicates that our main theorem cannot be directly extended to orders less than $1$. 
%Generalizing our algorithm for orders less than $1$ remains a direction for future research.


\section{Related Work}
\label{sec:related_work}
\subsection{Computing Classical Augustin information}
The classical Augustin information \cite{Augustin1978,Csiszar1995} can be viewed as a special case of the Petz-Augustin information 
where all the matrices commute. 
Augustin \cite{Augustin1978} proposed a fixed-point iteration, which we 
refer to as the Augustin iteration, for computing the Augustin information. 
The Augustin iteration
is proved to converge asymptotically for $\alpha\in(0,1)$ by 
\citet{Karakos2008} and 
\citet{Nakiboglu2019}. 
Recently, \citet{Tsai2024} proved that the Augustin iteration converges 
%linearly 
at a linear rate
with respect to the Hilbert projective metric for $\alpha\in(1/2,1)\cup(1,3/2)$.

Our proposed method can be viewed as a generalization of the Augustin iteration for the classical Augustin information.
Denote the function to be minimized in computing the Petz-Augustin information as $f_{\alpha}(\sigma_B)$, which is given in Section \ref{Sec:PetzAug}.
The Augustin iteration can be written as\footnote{Although originally proposed for the classical case, where matrices reduce to vectors, we present it in matrix form, as our algorithm is designed for the quantum case.}
\begin{align}
    \sigma_B^{(t+1)}=\sigma_B^{(t)} \left(-\nabla f_{\alpha}\left(\sigma_B^{(t)}\right)\right),
\end{align}
where $\sigma_B^{(t)}$ denotes the $t^{\text{th}}$ iterate of the algorithm.
On the other hand, our method can be expressed as 
\begin{align}
    \sigma_B^{(t+1)}=\sigma_B^{(t)} \left(-\nabla f_{\alpha}\left(\sigma_B^{(t)}\right)\right)^{1/\alpha},
\end{align}
if all the matrices commute. 
Thus, our method can be viewed as a generalization of the Augustin iteration with an additional parameter $1/\alpha$ in the exponent, which is analogous to the step size in first-order optimization methods. 

For algorithms different from the Augustin iteration, an alternating minimization method \cite{Kamatsuka2024} converges at a rate of $O(1/T)$ for $\alpha\in(1,\infty)$ \cite{Tsai2024}, where $T$ denotes the number of iterations. 
Riemannian gradient descent with respect to the Poincar{\'e} metric %\cite{Wang2024} 
also converges at a rate of $O(1/T)$ for all $\alpha\in(0,1)\cup(1,\infty)$ \cite{Wang2024}.

\subsection{Computing Petz-Augustin information}
The optimization problem defining the Petz-Augustin information of order $\alpha$ is known to be convex for $\alpha\in(0,1)\cup\left(1,2\right]$ \cite{Mosonyi2017}.
Since the objective function has a locally bounded gradient, entropic mirror descent with Armijo line search \cite{Li2019a} or with the Polyak step size \cite{You2022} is applicable for $\alpha\in(0,1)\cup\left(1,2\right]$. 
However, these two algorithms only guarantee asymptotic convergence. 
To the best of our knowledge, there is currently no 
% algorithm 
first-order method
that guarantees a non-asymptotic convergence rate for any $\alpha\in(0,1)\cup(1,\infty)$.

Our algorithm (Equation \eqref{eq:DiscreteCP}) is inspired by an algorithm proposed by \citet{Cohen2015} for computing the $\ell_p$-Lewis weights, which has been proven to converge linearly with respect to the Thompson metric.
The $\ell_p$-Lewis weights, along with their variants, have applications in $\ell_1$-regression \cite{Durfee2018,Parulekar2021} and linear programming \cite{Lee2020}.
%For a comprehensive discussion of the $\ell_p$-Lewis weights, we refer readers to the paper of \citet{Cohen2015}.
%Note that our results 
%% cannot be trivially derived 
%do not immediately follow
%from those of \citet{Cohen2015}.
%See Section \ref{subsec:RelationWithCP} for a detailed discussion.
Despite the conceptual connection, our results do not immediately follow from those of \citet{Cohen2015}, as discussed in Section \ref{subsec:RelationWithCP}. 

After completing this work, we noticed a recent study by \citet{Cheng2024}, which also
considers an iterative algorithm that coincides with ours.
However, their derivation and analysis adopt a different perspective and only guarantee asymptotic convergence for $\alpha\in(1,\infty)$.
In contrast, we prove that %our proposed 
the 
algorithm converges at a rate of $O\left(\left|1-1/\alpha\right|^T\right)$ for $\alpha\in(1/2,1)\cup(1,\infty)$, 
not only extending the range of $\alpha$ but also providing a non-asymptotic guarantee. 
%, where $T$ denotes the number of iterations.

%For related works that did not focus on the computation, \citet{Cheng2019}, \citet{Cheng2024}, as well as \citet{Mosonyi2021}, proposed several operators and analyzed their fixed-point properties at the minimizers of the convex optimization problem that defines the Petz-Augustin information.
%In particular, our proposed iteration rule in Section \ref{subsec:CohenIteration} involves repeatedly applying an operator  $T_{f_{\alpha}}$ to the iterates.
%The operator $T_{f_{\alpha}}$ coincides with the one proposed by \citet[Equation (32)]{Cheng2024}, where it is referred to as the \emph{(adjusted) power mean}.
%However, they did not design algorithms based on their proposed operators.

%To design an efficient algorithm for computing the Petz-Augustin information, we observe that computing the classical Augustin information of order $\alpha$ is similar to the task of computing the $\ell_p$-Lewis weights.
%Specifically, this connection holds when $\alpha=2/p$. 
%For a comprehensive treatment of computing the $\ell_p$-Lewis weights, we refer readers to \citet{Cohen2015} as well as \citet{Lee2020}. 
%\citet{Cohen2015} also proposed a fixed-point iteration, which we refer to as the \textit{CP-iteration}. They proved the linear convergence of the CP-iteration by leveraging the contractive property of the iteration mapping.
%The relationship between the computation of the Augustin information and the $\ell_p$-Lewis weights is briefly discussed in Section \ref{subsec:CohenIteration}. 
%Our proposed algorithm can be seen as a quantum generalization of the CP-iteration. 
%We also prove a linear convergence rate for the optimization error, which was not considered in \citet{Cohen2015}.


\section{Preliminaries}
\label{sec:problem}
\subsection{Notations}
We denote the sets of vectors in $\mathbb{R}^d$ with nonnegative entries and
strictly 
positive entries by $\mathbb{R}_{+}^d$ and $\mathbb{R}_{++}^d$, respectively.
We denote $\mathbb{C}^d$ by $\mathcal{H}_B$. 
We denote the set of all Hermitian matrices in $\mathbb{C}^{d\times d}$ by $\mathcal{B}\left(\mathcal{H}_B\right)$.
%We denote the Hilbert-Schmidt inner product by $\braket{\cdot , \cdot}_{\text{H}}$. 

For any $\sigma_B \in \mathcal{B}(\mathcal{H}_B)$, we denote its support by 
\begin{align}
    \supp(\sigma_B)\coloneqq \Set{ u \in \mathcal{H}_B \mid \sigma_B u \neq 0 }.
\end{align}

We define $A_i$ to be the $i^{\text{th}}$ row of a matrix $A$ and $v[i]$ to be the $i^{\text{th}}$ entry of a vector $v$. 
For any vector $v$, we denote by $\mathrm{Diag}(v)$ the diagonal matrix whose $i$-th diagonal element is $v[i]$. 
For any $v\in\mathbb{C}^d$, we denote its conjugate transpose by $v^*$.
For any function \( f: \mathbb{R} \to \mathbb{R} \) and 
% a 
vector \( v \in \mathbb{R}^d \), we define \( f(v) \) 
% to be 
as 
% a 
the 
\( d \)-dimensional vector where \( f(v)[i] = f(v[i]) \). 
Similarly, for such a function \( f \) and \( \sigma \in \mathcal{B}(\mathcal{H}_B) \), we define \( f(\sigma) \) as \( \sum_{i=1}^d f(\lambda_i)  u_i u_i^* \), where \( \sigma = \sum_{i=1}^d \lambda_i   u_i u_i^* \) is the eigendecomposition of \( \sigma \). 


We denote the probability simplex in $\mathbb{R}^d$ by $\Delta_{d-1}$, i.e.,
\begin{align}
    \Delta_{d-1}\coloneqq % \left\{ v \in \mathbb{R}_{+}^d \middle\vert\ \sum_{i=1}^d v[i] = 1 \right\}.
    \Set{ v \in \mathbb{R}_{+}^d | \sum_{i=1}^d v[i] = 1 }.
\end{align}

For any \( \sigma_1, \sigma_2 \in \mathcal{B}(\mathcal{H}_B) \), we write \( \sigma_1 \leq \sigma_2 \) if and only if \( \sigma_2 - \sigma_1 \) is positive semi-definite. Similarly, we write \( \sigma_1 < \sigma_2 \) if and only if \( \sigma_2 - \sigma_1 \) is positive definite.
We define $\mathcal{B}\left(\mathcal{H}_B\right)_{+}$ and $\mathcal{B}\left( \mathcal{H}_B\right)_{++}$ as the nonnegative cone and the positive cone in \( \mathcal{B}(\mathcal{H}_B) \), respectively. 
Specifically,
\begin{equation} 
\mathcal{B}\left(\mathcal{H}_B\right)_{+} \coloneqq \set{ \sigma_B \in \mathcal{B}(\mathcal{H}_B) | \sigma_B \geq 0},
\end{equation} 
and
\begin{equation} 
\mathcal{B}\left(\mathcal{H}_B\right)_{++} \coloneqq \set{ \sigma_B \in \mathcal{B}(\mathcal{H}_B) | \sigma_B > 0}.
\end{equation}
We denote the set of quantum density matrices in $\mathcal{B}\left(\mathcal{H}_B\right)_{+}$ by $\mathcal{D}(\mathcal{H}_B)$, i.e.,
\begin{equation} \label{eq:set_B}
\mathcal{D}(\mathcal{H}_B) \coloneqq \set{ \sigma_B \in \mathcal{B}\left(\mathcal{H}_B\right)_{+} | \operatorname{Tr}[\sigma_B] = 1}.
\end{equation} 
For any $\sigma\in\mathcal{B}\left(\mathcal{H}_B\right)_{+}$, we denote its $i^{\text{th}}$ eigenvalue, ordered in decreasing order, by $\lambda_i(\sigma)$. 


%The idea of considering the Thompson metric is inspired by the work of \citet{Tsai2024}, in which they proved that the iterates of \emph{Augustin's iteration} for computing the Petz-Augustin information, under the condition that all matrices commute, converges linearly with respect to \emph{Hilbert's projective metric}.
%Notably, the algorithm we propose in Section \ref{subsec:CohenIteration} iteratively applies an operator  $T_{f_{\alpha}}$ to update the iterates.
%In Section \ref{subsec:Conv}, we analyze the convergence of our proposed iteration rule by proving the contractive property of the iterates with respect to the Thompson metric.  
%Then, we show in Lemma \ref{ineq:SmallThomp2SmallOptError} and Theorem \ref{ineq:FvalConv} that the convergence guarantee of the iterates with respect to the Thompson metric can be translated into that of the function values.


\subsection{Thompson metric}
The Thompson metric is a useful tool to study the behavior of dynamical systems \citep{Krause2015,Lemmens2012,Nussbaum1988,Thompson1963}. 
% The Thompson metric 
It 
can be defined on the interiors of any normal cone in real Banach spaces, such as $\mathbb{R}_{++}^d$ and $\mathcal{B}\left(\mathcal{H}_B\right)_{++}$ . 
% \cite{Nussbaum1988}.
%\citep{Krause2015,Lemmens2012,Nussbaum1988,Thompson1963}. 
%Nussbaum1988,p17
%cone imply close and convex, p9
Here, we are only interested in the following definition of the Thompson metric specialized for $\mathcal{B}\left(\mathcal{H}_B\right)_{++}$.
\begin{definition}[{\cite{Thompson1963}}]
    \label{eq:Thompson}
    The Thompson metric between
	any     
    $U,V\in\mathcal{B}\left(\mathcal{H}_B\right)_{++}$ % from \citet{Thompson1963} is equivalently defined as
    is given by 
    \begin{equation}
        d_{\mathrm{T}}(V,U)\coloneqq \inf %\left\{r\geq0|\exp(-r) V\leq U \leq \exp(r)V\right\}.
        \Set{ r\geq0 | \exp(-r) V\leq U \leq \exp(r)V }.
    \end{equation}
\end{definition}

%The Thompson metric is commonly used to study the existence and uniqueness of fixed points of nonlinear operators via 
%% a 
%contraction arguments \cite{Lim2009,Montrucchio1998}.

We
will 
analyze the convergence of our proposed iteration rule by proving a contractive property of the iterates with respect to the Thompson metric (see Lemma \ref{ineq:Contract} and Theorem \ref{ineq:FvalConv}). 
This analysis relies on the following lemmas concerning the Thompson metric, specialized for $\mathcal{B}\left(\mathcal{H}_B\right)_{++}$.

\begin{lemma}[{\cite[Lemma 3]{Thompson1963}}]
    \label{eq:ThompIsMetric}
    The Thompson metric is a well-defined metric on $\mathcal{B}\left(\mathcal{H}_B\right)_{++}$, and $\mathcal{B}\left(\mathcal{H}_B\right)_{++}$ is complete with respect to this metric.
\end{lemma}
\begin{lemma}[{\cite[Proposition 1.5]{Nussbaum1988}}]% page 19
    \label{ineq:NegCurveProp}
    For any $U,V\in\mathcal{B}\left(\mathcal{H}_B\right)_{++}$, 
    we have
    \begin{align}
        d_{\mathrm{T}}(U^r,V^r)\leq |r|d_{\mathrm{T}}(U,V),\quad\forall r\in[-1,1].
    \end{align}
\end{lemma}
%\begin{lemma}[{\cite{Thompson1963}}]
%    \label{eq:Brouwer}
%    Let $T:\mathcal{B}\left(\mathcal{H}_B\right)_{++}\mapsto\mathcal{B}\left(\mathcal{H}_B\right)_{++}$ be an operator.
%    Suppose there exists a scalar $r$ with $0\leq r<1$ such that
%    \begin{align}
%        d_{\mathrm{T}}(T(U),T(V))
%        \leq r d_{\mathrm{T}}(U,V),\quad\forall U,V\in\mathcal{B}\left(\mathcal{H}_B\right)_{++}.
%    \end{align}
%    Then, there exists a unique fixed point $\sigma\in\mathcal{B}\left(\mathcal{H}_B\right)_{++}$ of the operator $T$.
    %Moreover, let $\sigma_1\in\mathcal{B}\left(\mathcal{H}_B\right)_{++}$, and define the sequence  $\sigma_{t+1}=T(\sigma_{t})$ for all $t\in\mathbb{N}$.
    %Then, the sequence $\left\{\sigma_t\right\}_t$ converges in norm to $\sigma$.
%\end{lemma}

\begin{lemma}
    \label{ineq:ThompsonLogHomo}
    For any $U,V\in\mathcal{B}\left( \mathcal{H}_B\right)_{++}$ and $r>0$, we have
    \begin{align}
        d_{\mathrm{T}}\left(U,rV\right)
        \leq d_{\mathrm{T}}\left(U,V\right)+\abs{\log(r)}.
    \end{align}
\end{lemma}

\begin{proof}
    By Definition \ref{eq:Thompson}, we write
    \begin{align}
        \exp\left(-d_{\mathrm{T}}\left(V,U\right)\right)V
        \leq U
        \leq \exp\left(d_{\mathrm{T}}\left(V,U\right)\right)V.
    \end{align}
    It follows that
    \begin{align}
        \exp\left(-d_{\mathrm{T}}\left(V,U\right)-\abs{\log(r)}\right)V
        &\leq\exp\left(-d_{\mathrm{T}}\left(V,U\right)+\log(r)\right)V\nonumber\\
        &\leq rU\nonumber\\
        &\leq \exp\left(d_{\mathrm{T}}\left(V,U\right)+\log(r)\right)V\nonumber\\
        &\leq \exp\left(d_{\mathrm{T}}\left(V,U\right)+\abs{\log(r)}\right)V.\nonumber
    \end{align}
    By Definition \ref{eq:Thompson}, the inequalities above imply that 
    \begin{align}
        d_{\mathrm{T}}\left(U,rV\right)
        \leq d_{\mathrm{T}}\left(V,U\right)+\abs{\log(r)}.
    \end{align}
    This concludes the proof.
\end{proof}
 
\subsection{Petz-Augustin Information}
\label{Sec:PetzAug}
Let 
$\mathcal{X}=
%\{1,2,...,n\}$ 
\Set{1,2,\dots,n}$
denote the input alphabet, and let  $P_{X}$ denote the probability distribution of 
a
%the 
random variable $X$ defined on $\mathcal{X}$. 
Let $\mathcal{W}:\mathcal{X}\mapsto\mathcal{D}(\mathcal{H}_B)$ be 
a
%the 
classical-quantum channel that maps an element $x\in \mathcal{X}$ to a density matrix $\rho_B^x$.
We assume that $\supp\left(\sum_{x:P_X(x)>0}\rho_B^x\right) = \mathcal{H}_B \setminus \set{0}$.
If the assumption does not hold, we may project all the matrices to a lower-dimensional space.
%Let $\rho_{{X}B} = \sum_{x\in \mathcal{X}}P_{X}(x)E_x\otimes\rho_B^x$ be the classical-quantum state, where $E_x$ is the $n$-by-$n$ diagonal matrix with the $x^{\text{th}}$ diagonal element equal to $1$ and all other diagonal elements equal to 0.


Given \(\rho \in \mathcal{D}(\mathcal{H}_B)\) and \(\sigma \in \mathcal{B}\left(\mathcal{H}_B\right)_{+}\), the Petz-R\'{e}nyi divergence of order \(\alpha\) is given by \cite{Petz1986}  
\[
D_{\alpha}(\rho \| \sigma) \coloneqq  
\begin{cases}  
\frac{1}{\alpha - 1} \log \operatorname{Tr}\left[\rho^{\alpha} \sigma^{1-\alpha}\right], & \text{if } \left(\supp(\rho) \subseteq \supp(\sigma)\right)\\
&\text{ or }\left(\alpha\in(0,1)
\text{ and }\supp(\rho)\cap\supp(\sigma)\neq \emptyset\right), \\  
\infty, &\text{otherwise },%& \text{if } \supp(\rho) \not\subseteq \supp(\sigma)\text{ and }\alpha\in(1,\infty),  
\end{cases}   
\]
for \(\alpha \in (0, 1) \cup (1, \infty)\), where we allow \(\sigma\) to have a trace not equal to 1.
%The function above is just the \emph{Petz-R\'enyi divergence} \cite{Petz1986}, where we allow $\sigma$ to have a trace not equal to $1$.
% Under the assumption that $\cup_{x:P_X(x)>0} \supp\left(\rho_B^x\right) = \mathcal{H}_B$,
The Petz-Augustin information of order $\alpha$ 
%for $\rho_{{X}B}$ 
is given by 
\begin{equation} \label{eq:Petz_Augustin}
    \min_{\sigma_B \in \mathcal{D}(\mathcal{H}_B)} f_{\alpha}(\sigma_B), \quad f_{\alpha} ( \sigma_B ) \coloneqq \mathbb{E}_{P_{X}}\left[{D_{\alpha}\left(\rho_B^{X} \vert\vert\sigma_B\right)}\right].
\end{equation}
Under the assumption that $\supp\left(\sum_{x:P_X(x)>0}\rho_B^x\right) = \mathcal{H}_B\setminus\set{0}$, the minimizer of this minimization problem exists \cite[Lemma IV.8]{Mosonyi2021} and is 
% a full-rank density matrix 
full-rank \cite[Lemma IV.11]{Mosonyi2021}.



%The reason why we consider the  Thompson metric instead of Hilbert's projective metric is that the convergence of the function values is more important than that of the iterates.
%Specifically, We show in Lemma \ref{ineq:SmallThomp2SmallOptError} and Theorem \ref{ineq:FvalConv} that the convergence guarantee of the iterates with respect to the Thompson metric can be translated into that of the function values.

%We only focus on the Thompson metric defined on the positive definite matrices.




\section{A Simple Iteration Rule and Its Convergence}
\label{sec:alg}
\subsection{A Simple Iteration Rule}
\label{subsec:CohenIteration}


Let $\alpha\in(0,1)\cup(1,\infty)$. 
Given $f_{\alpha}$, we define the operator
\begin{equation}
    \label{eq:CohenIteration}
    T_{f_{\alpha}} \colon \mathcal{B}\left(\mathcal{H}_B\right)_{++}\mapsto\mathcal{B}\left(\mathcal{H}_B\right)_{++}:U\mapsto\left(\mathbb{E}_{P_X}\left[\frac{ \left(\rho_B^X\right)^\alpha}{ \Tr\left[ \left(\rho_B^X\right)^\alpha U \right] }\right]\right)^{(1-\alpha)/\alpha}.
\end{equation}
We propose the following simple iteration rule for solving the optimization problem \eqref{eq:Petz_Augustin}:  
\begin{itemize}
\item Let $\sigma_B^{(1)}$ be a full-rank density matrix. 
\item For every $t \in \mathbb{N}$, compute $\sigma_B^{(t + 1)} = T_{f_{\alpha}} \left(\left( \sigma_B^{(t)}\right)^{1-\alpha} \right)^{1/(1-\alpha)}$, and output $\frac{\sigma_B^{(t + 1)}}{\Tr\left[\sigma_B^{(t + 1)}\right]}$. 
\end{itemize} 

To verify the well-definedness of the operator $T_{f_{\alpha}}$, we first observe that
for each $x\in\mathcal{X}$, since $\rho_B^x\neq 0$, the denominator $\Tr\left[ \left(\rho_B^x\right)^\alpha U \right]$ is finite and positive 
% , i.e.,
% $\Tr\left[ \left(\rho_B^x\right)^\alpha U \right]\in(0,\infty)$,
for any $U\in\mathcal{B}\left(\mathcal{H}_B\right)_{++}$.
Consequently, we have 
\begin{align}
    \supp\left(\frac{\left(\rho_B^x\right)^{\alpha}}{\Tr\left[ \left(\rho_B^x\right)^\alpha U \right]}\right)
    =\supp\left(\rho_B^x\right).
\end{align}
Given the assumption $\supp\left(\sum_{x:P_X(x)>0}\rho_B^x\right) = \mathcal{H}_B \setminus \set{0}$, it follows that the support of $T_{f_{\alpha}}(U)$ is equal to $\mathcal{H}_B\setminus\set{0}$. 
This ensures that the operator $T_{f_{\alpha}}$ is well-defined. 
To understand why we restrict the domain of the operator $T_{f_{\alpha}}$ to $\mathcal{B}\left(\mathcal{H}_B\right)_{++}$, suppose $U\in\mathcal{B}\left(\mathcal{H}_B\right)_{+}$ is not full rank.
In this case, the denominator $\Tr\left[ \left(\rho_B^x\right)^\alpha U \right]$ may become zero, rendering the definition of the operator $T_{f_{\alpha}}$ ill-defined.


% Recall that we are given $d$-by-$d$ density matrices $\rho_B^x$ for $x=1,2,\dots n$.
To evaluate the time complexity of the proposed iteration rule, we 
% write the iteration rule 
express it 
explicitly as follows:
\begin{align}
    \label{eq:DiscreteCP}
    \sigma_B^{(t + 1)} = \left(\sum_{x=1}^n P_X(x)\frac{ \left(\rho_B^x\right)^\alpha}{ \Tr\left[ \left(\rho_B^x\right)^\alpha \left(\sigma_B^{(t)}\right)^{1-\alpha} \right] }\right)^{1/\alpha}.
\end{align}
% For each $x\in\{1,2,\cdots,n\}$, t
The matrix powers $\left(\rho_B^x\right)^\alpha$ can be computed and stored before the first iteration begins.
%Thus, the per-iteration time complexity is dominated by the computation of the matrix power $(\sigma_B^{(t)})^{1-\alpha}$, 
%$n$ traces of matrix multiplications $\Tr\left[ (\rho_B^x)^\alpha (\sigma_B^{(t)})^{1-\alpha} \right]$, and the other matrix power $\left(\cdot\right)^{1/\alpha}$ taken outside the summation.
% For any $\sigma,\rho\in\mathcal{B}\left(\mathcal{H}_B\right)_{++}$ and $r\in\mathbb{R}$, t
Given $\left(\rho_B^x\right)^\alpha$ for all $x \in \mathcal{X}$, each $\Tr\left[ \left(\rho_B^x\right)^\alpha \left(\sigma_B^{(t)}\right)^{1-\alpha} \right]$ can be computed in $O ( d^2 )$ time. 
Computing $\left(\sigma_B^{(t)}\right)^{1-\alpha}$ and raising a matrix to the power $(1 / \alpha)$ each require $O ( d^3 )$ time. 
%The computational complexities of computing 
%a matrix power 
%% the matrix power $\rho^r$ 
%and the trace of
%a 
%matrix multiplication 
%% $\Tr\left[\rho\sigma\right]$ 
%are $O(d^3)$ and $O(d^2)$, respectively.
%%Consequently, the initialization and per-iteration time complexities are $O\left(n d^{3}\right)$ and $O\left(d^{3}+n d^{2}\right)$, respectively.
Consequently, the initialization time complexity is $O\left(n d^{3}\right)$, and the per-iteration time complexity is $O\left(d^{3} + n d^{2}\right)$.

\subsection{Convergence Analysis}
\label{subsec:Conv}
%We note that, for $\alpha>1$, to the best of our knowledge, it is currently unknown whether the \emph{Augustin mean} is unique.
%Therefore, to 

Below, we present our main theorem.
\begin{theorem}
    \label{ineq:FvalConv}
    For any $\alpha\in(1/2,1)\cup(1,\infty)$, let $\Set{\sigma_B^{(t)}}_{t\in\mathbb{N}}$ be the sequence of iterates generated by our proposed iteration rule.
    Then, we have 
    \begin{align}
        d_{\mathrm{T}}\left(\left(\sigma_B^{\star}\right)^{1-\alpha},\left(\frac{\sigma_B^{(T+1)}}{\Tr\left[\sigma_B^{(T+1)}\right]}\right)^{1-\alpha}\right)
        \leq 2\left|1-\frac{1}{\alpha}\right|^{T} d_{\mathrm{T}}\left(\left(\sigma_B^{\star}\right)^{1-\alpha},\left(\sigma_B^{(1)}\right)^{1-\alpha}\right),
    \end{align}
    and
    \begin{align}
        f_{\alpha}\left(\frac{\sigma_B^{(T+1)}}{\Tr\left[\sigma_B^{(T+1)}\right]}\right)-f_{\alpha}\left(\sigma_B^{\star}\right)
        \leq \left|\frac{2}{\alpha-1}\right|\cdot\left|1-\frac{1}{\alpha}\right|^{T} d_{\mathrm{T}}\left(\left(\sigma_B^{\star}\right)^{1-\alpha},\left(\sigma_B^{(1)}\right)^{1-\alpha}\right),
    \end{align}
    where $\sigma_B^{\star}$ is the minimizer of the optimization problem \eqref{eq:Petz_Augustin}.
    Moreover, for $\alpha > 1$, the function values are non-increasing, i.e.,
    \begin{align}
        f_{\alpha}\left(\frac{\sigma_B^{(t+1)}}{\Tr\left[\sigma_B^{(t+1)}\right]}\right)
        \leq
        f_{\alpha}\left(\frac{\sigma_B^{(t)}}{\Tr\left[\sigma_B^{(t)}\right]}\right),\quad\forall t\in\mathbb{N}.
    \end{align}
    Furthermore, the quantity $d_{\mathrm{T}}\left(\left(\sigma_B^{\star}\right)^{1-\alpha},\left(\sigma_B^{(1)}\right)^{1-\alpha}\right)$ is bounded above.
\end{theorem}

% The proof of Theorem \ref{ineq:FvalConv} in Section \ref{subsec:PfMainThm} is based on the following key observations:
The proof of Theorem \ref{ineq:FvalConv}, which we defer to Section \ref{subsec:PfMainThm}, relies on the following observations.
\begin{itemize}
    \item % The operator $T_{f_{\alpha}}$ is contractive with respect to the Thompson metric (Section \ref{subsec:Contract}). 
    The operator $T_{f_{\alpha}}$ is contractive with a ratio of $| 1 - 1 / \alpha |$ in the Thompson metric (Section \ref{subsec:Contract}). 
    As a result, it has a unique fixed point, and the iterates $\sigma_B^{(t)}$ converge to this fixed point at a rate of $O\left(|1-1/\alpha|^T\right)$
    in the Thompson metric. 
    \item The unique fixed point of $T_{f_{\alpha}}$ 
    % is also 
    coincides with 
    the minimizer of the optimization problem \eqref{eq:Petz_Augustin} (see Section \ref{subsec:FixPt}).
    \item % The iterates $\sigma_B^{(t)}$ may fall outside the constraint set $\mathcal{D}\left(\mathcal{H}_B\right)$ of the optimization problem \eqref{eq:Petz_Augustin}.
%    Fortunately, we show that Thompson metric is stable under
%    trace 
%    normalization (Section \ref{subsec:NormalStable}).
	The iterates $\sigma_B^{(t)}$ may not be ``physical,'' in the sense that they may not have unit traces.
	Fortunately, we show that the Thompson metrics between the iterates and the minimizer are preserved under trace normalization, up to a multiplicative constant. 
    Therefore, the trace-normalized iterates still converge 
    to the minimizer
    at a rate of $O\left(|1-1/\alpha|^T\right)$.
    \item % The variation in the function values can be upper-bounded by the Thompson metric (Section \ref{subsec:ControlFval}); hence, the convergence guarantee of the iterates $\sigma_B^{(t)}/\Tr\left[\sigma_B^{(t)}\right]$ with respect to the Thompson metric can be translated into the convergence guarantee of the function values $f_{\alpha}\left(\sigma_B^{(t)}/\Tr\left[\sigma_B^{(t)}\right]\right)$.
    % The optimization error in function value can be upper-bounded by the Thompson metric between the iterates and the minimizer (see Section \ref{subsec:ControlFval}). 
    The variation in function values can be upper-bounded by the Thompson metric between the iterates and the minimizer (see Section \ref{subsec:ControlFval}). 
    Consequently, the above error bound in the Thompson metric translates into an error bound in function value.
\end{itemize}
\subsubsection{Contractivity of $T_{f_{\alpha}}$}
\label{subsec:Contract}
%We first prove that the operator $T_{f_{\alpha}}$ is contractive.
%\begin{lemma}(Fej\'{e}r Monotonicity)
\begin{lemma}[{Contractive Property}]
    \label{ineq:Contract}
    Let $\alpha\in(1/2,1)\cup(1,\infty)$. 
    For any $U,V\in\mathcal{B}\left(\mathcal{H}_B\right)_{++}$, we have
    \begin{align}
        d_{\mathrm{T}}\left(T_{f_{\alpha}}(V),T_{f_{\alpha}}(U)\right)
        \leq \left|1-\frac{1}{\alpha}\right| d_{\mathrm{T}}\left(V,U\right).
    \end{align}
\end{lemma}
\begin{proof}
     By Definition \ref{eq:Thompson}, we have
    \begin{align}
        \exp\left(-d_{\mathrm{T}}\left(V,U\right)\right)V
        \leq U
        \leq \exp\left(d_{\mathrm{T}}\left(V,U\right)\right)V.\nonumber
    \end{align}
    Since $\rho_{B}^{X}\in\mathcal{D}\left(\mathcal{H}_B\right)$, we write
    \begin{align}
        T_{f_{\alpha}}\left(U\right)^{\alpha/(1-\alpha)}
        &=\mathbb{E}_{P_X}\left[\frac{ \left(\rho_B^X\right)^{\alpha}}{\operatorname{Tr}\left[ \left(\rho_B^X\right)^{\alpha}U\right]}\right]\nonumber\\
        &\geq\exp\left(- d_{\mathrm{T}}\left(V,U\right)\right)\mathbb{E}_{P_X}\left[\frac{ \left(\rho_B^X\right)^{\alpha}}{\operatorname{Tr}\left[ \left(\rho_B^X\right)^{\alpha}V\right]}\right]\nonumber\\
        &=\exp\left(-d_{\mathrm{T}}\left(V,U\right)\right)T_{f_{\alpha}}\left(V\right)^{\alpha/(1-\alpha)}.\nonumber
    \end{align}
    Similarly, we write
    \begin{align}
        T_{f_{\alpha}}\left(U\right)^{\alpha/(1-\alpha)}
        &\leq\exp\left(d_{\mathrm{T}}\left(V,U\right)\right)\mathbb{E}_{P_X}\left[\frac{ \left(\rho_B^X\right)^{\alpha}}{\operatorname{Tr}\left[ \left(\rho_B^X\right)^{\alpha}V\right]}\right]\nonumber\\
        &=\exp\left(d_{\mathrm{T}}\left(V,U\right)\right)T_{f_{\alpha}}\left(V\right)^{\alpha/(1-\alpha)}.\nonumber
    \end{align}
    By Definition \ref{eq:Thompson}, the two inequalities above imply that 
    \begin{align}
        d_{\mathrm{T}}\left(T_{f_{\alpha}}\left(V\right)^{\alpha/(1-\alpha)},T_{f_{\alpha}}\left(U\right)^{\alpha/(1-\alpha)}\right) \leq d_{\mathrm{T}} ( V, U ).
    \end{align}
    Then, by Lemma \ref{ineq:NegCurveProp}, we have
    \begin{align}
        d_{\mathrm{T}}\left(T_{f_{\alpha}}\left(V\right),T_{f_{\alpha}}\left(U\right)\right)
        &=d_{\mathrm{T}}\left(\left(T_{f_{\alpha}}\left(V\right)^{\alpha/(1-\alpha)}\right)^{(1-\alpha)/\alpha},\left(T_{f_{\alpha}}\left(U\right)^{\alpha/(1-\alpha)}\right)^{(1-\alpha)/\alpha}\right)\nonumber\\
        &\leq\left|\frac{1-\alpha}{\alpha}\right|d_{\mathrm{T}}\left(T_{f_{\alpha}}\left(V\right)^{\alpha/(1-\alpha)},T_{f_{\alpha}}\left(U\right)^{\alpha/(1-\alpha)}\right)\nonumber\\
        &\leq\left|1-\frac{1}{\alpha}\right|d_{\mathrm{T}}\left(V,U\right).\nonumber
    \end{align}
\end{proof}
\subsubsection{Fixed-Point Property of $T_{f_{\alpha}}$}
\label{subsec:FixPt}
\begin{lemma}
    \label{eq:FixPtIsMin}
    For any $\alpha  \in (0, 1) \cup (1, \infty)$, there exists a unique minimizer $\sigma_B^{\star}$ of the optimization problem \eqref{eq:Petz_Augustin}.
    Moreover, for the same $\sigma_B^{\star}$, $\left(\sigma_B^{\star}\right)^{1-\alpha}$ is the unique fixed point of the operator $T_{f_{\alpha}}$ for $\alpha\in(0,1)\cup(1,\infty)$.
\end{lemma}
%\begin{corollary}
%    \label{ineq:Fejer}
%    Consider $\alpha\in(1/2,1)\cup(1,\infty)$.
%    Let $\sigma_B^{\star}$ and $\sigma_B^{(t)}$ be as defined in Theorem \ref{ineq:FvalConv}.
%    Then, we have
%    \begin{align}
%        d_{\mathrm{T}}\left(\left(\sigma_B^{\star}\right)^{1-\alpha},\left(\sigma_B^{(T+1)}\right)^{1-\alpha}\right)
%        \leq\left|1-\frac{1}{\alpha}\right|^{T} d_{\mathrm{T}}\left(\left(\sigma_B^{\star}\right)^{1-\alpha},\left(\sigma_B^{1}\right)^{1-\alpha}\right).
%    \end{align}
%\end{corollary}
The remainder of this section is devoted to the proof of Lemma \ref{eq:FixPtIsMin}.
For $\alpha\in(0,1)$, Lemma \ref{eq:FixPtIsMin} has been 
% proved 
proven 
by \citet[Proposition 2(b)]{Cheng2019}.
For $\alpha\in(1,\infty)$, the proof of Lemma \ref{eq:FixPtIsMin} relies on the following observations:
\begin{itemize}
    \item The traces of the iterates are always less than or equal to $1$ (Lemma \ref{ineq:TraceBound}).
    \item The function values are non-increasing (Lemma \ref{ineq:FvalDecrease}).
\end{itemize}
%\begin{lemma}[{\cite[Proposition 2]{Cheng2019}}]
%    \label{eq:ChengFix}
%    Let $\alpha\in(0,1)$.
%    There exists a unique minimizer $\sigma_B^{\star}$ of the optimization problem \eqref{eq:Petz_Augustin}.
%    Moreover, the following equation holds:
%    \begin{align}
%        \sigma_B^{\star}=\left(\mathbb{E}_{P_X}\left[\frac{\left(\rho_B^{X}\right)^{\alpha}}{\Tr\left[\left(\rho_B^{X}\right)^{\alpha}\left(\sigma_B^{\star}\right)^{1-\alpha}\right]}\right]\right)^{1/\alpha}.
%    \end{align}
%\end{lemma}

We will use Lemma \ref{ineq:Araki} and Lemma \ref{ineq:HolderEq} to prove Lemma \ref{ineq:TraceBound}.
% , both of which were established in earlier research. 
% Subsequently, using Lemma \ref{ineq:TraceBound}, we can establish Lemma \ref{ineq:FvalDecrease}.
\begin{lemma}[{Araki-Lieb-Thirring Inequality \cite{Araki1990}}]
\label{ineq:Araki}
For any $U,V\in\mathcal{B}\left(\mathcal{H}_B\right)_{++}$, we have 
\begin{align}
    \Tr\left[\left(V^{1/2}UV^{1/2}\right)^{sr}\right]\leq\Tr\left[\left(V^{r/2} U^r V^{r/2}\right)^{s}\right],
\end{align}
for all $s>0$ and $r\geq 1$.
\end{lemma}
\begin{lemma}[{H\"older Inequality \cite{Larotonda2018}}]
    \label{ineq:HolderEq}
    For any $U,V\in\mathcal{B}\left(\mathcal{H}_B\right)_{++}$ and $p>1$, we have
    \begin{align}
        \Tr[UV]\leq\Tr[U^{p}]^{1/p}\Tr[V^{p/(p-1)}]^{1-1/p}.
    \end{align}
    Moreover, equality holds if and only if 
    \begin{align}
        \frac{U^p}{\Tr[U^p]}=\frac{V^{p/(p-1)}}{\Tr[V^{p/(p-1)}]}.
    \end{align}
\end{lemma}
% \begin{lemma}[{Bound of the Trace}]
\begin{lemma}[{Bound of Trace}]
\label{ineq:TraceBound}
For any $\alpha\in(1,\infty)$ and $\sigma_B\in\mathcal{B}\left(\mathcal{H}_B\right)_{++}$ such that $\Tr\left[\sigma_B\right]\leq 1$, we have 
\begin{align}
    \Tr\left[T_{f_{\alpha}}\left(\sigma_B^{1-\alpha}\right)^{1/(1-\alpha)}\right]\leq 1.
\end{align}
Moreover, equality holds if and only if 
$\sigma_B$ is a fixed point of $T_{f_\alpha}\left((\cdot)^{1-\alpha}\right)^{1/(1-\alpha)}$ on $\mathcal{D} ( \mathcal{H}_B )$.
\end{lemma}

\begin{proof}
% Let $\sigma_B$ be as defined in Lemma \ref{ineq:TraceBound}.
Let $\sigma_B \in \mathcal{B}\left(\mathcal{H}_B\right)_{++}$ such that $\Tr\left[\sigma_B\right]\leq 1$. 
Let $U = T_{f_{\alpha}}\left(\sigma_B^{1-\alpha}\right)^{\alpha/(1-\alpha)}$ and $V = \sigma_B^{1-\alpha}$.
Then, both $U$ and $V$ are positive definite, and we have 
% .
% We have 
\begin{align}
    \Tr \left[ UV \right]
    = \Tr\left[ \mathbb{E}_{P_X}\left[\frac{ (\rho_B^X)^\alpha}{ \Tr\left[ (\rho_B^X)^\alpha \sigma_B^{1-\alpha} \right] }\right] \sigma_B^{1-\alpha} \right]
    =1.\nonumber
\end{align}
Then, we write 
    \begin{align}
        \Tr\left[\left(T_{f_{\alpha}}\left(\sigma_B^{1-\alpha}\right)\right)^{1/(1-\alpha)}\right]
        &= \Tr\left[ U^{ 1/\alpha }\right]\\
        &= \Tr\left[ \left( V^{ 1/(2\alpha)} U^{ 1/\alpha } V^{ 1/(2\alpha) }\right) V^{ - 1/\alpha } \right]\\
        &\leq\Tr\left[ \left( V^{ 1/(2\alpha) } U^{ 1/\alpha } V^{ 1/(2\alpha) }\right)^\alpha \right]^{ 1/\alpha }  \Tr\left[ \left( V^{ -1/\alpha } \right)^{\alpha/(\alpha-1)} \right]^{ 1- 1/\alpha } \\
        &=\Tr\left[ \left( V^{  1/(2\alpha) } U^{1/\alpha } V^{ 1/(2\alpha) }\right)^\alpha \right]^{ 1/\alpha }  \Tr\left[ \sigma_B \right]^{ 1- 1/\alpha }\\
        &\leq \Tr\left[ \left( V^{ 1/(2\alpha)} U^{ 1/\alpha } V^{ 1/(2\alpha) }\right)^\alpha \right]^{ 1/\alpha },
    \end{align}
    where the first inequality follows from the H\"older inequality (Lemma \ref{ineq:HolderEq}), 
    the third equality follows from the definition of $V$,
    and the last inequality follows from the assumption that $\Tr[\sigma_B] \leq 1$.

    Then, by the Araki-Lieb-Thirring inequality (Lemma \ref{ineq:Araki}), we have 
    \begin{align}
        \Tr\left[ \left( V^{ 1/(2\alpha) } U^{ 1/\alpha } V^{ 1/(2\alpha) }\right)^\alpha \right] 
        \leq \Tr\left[ V^{1/2 } U V^{ 1/2 }\right]
        = \Tr\left[UV\right]= 1.
    \end{align} 
    Therefore, we conclude that 
    \begin{align}
        \Tr\left[\left(T_{f_{\alpha}}\left(\sigma_B^{1-\alpha}\right)\right)^{1/(1-\alpha)}\right]\leq 1.
    \end{align}
    
    We proceed to prove the if and only if condition. 
    Note that the ``if'' direction holds trivially. 
    It remains to prove the ``only if'' direction. 
%    To prove that the equality
%    holds if and only if
%    $\sigma_B$ is a fixed point of $T_{f_\alpha}\left((\cdot)^{1-\alpha}\right)^{1/(1-\alpha)}$ on $\mathcal{D} ( \mathcal{H}_B )$,
%    it suffices to verify the ``only if'' direction.
    Suppose that $\Tr\left[T_{f_{\alpha}}\left(\sigma_B^{1-\alpha}\right)^{1/(1-\alpha)}\right]=1$.
    Let $U$ and $V$ be defined as above. 
    Recall that we have proved
    \begin{align}
        \Tr\left[T_{f_{\alpha}}\left(\sigma_B^{1-\alpha}\right)^{1/(1-\alpha)}\right]
        &= \Tr\left[ \left( V^{ 1/(2\alpha)} U^{ 1/\alpha } V^{ 1/(2\alpha) }\right) V^{ - 1/\alpha } \right]\\
        &\leq\Tr\left[ \left( V^{ 1/(2\alpha) } U^{ 1/\alpha } V^{ 1/(2\alpha) }\right)^\alpha \right]^{ 1/\alpha }  \Tr\left[ \left( V^{ -1/\alpha } \right)^{\alpha/(\alpha-1)} \right]^{ 1- 1/\alpha } \\
        &=\Tr\left[ \left( V^{ 1/(2\alpha) } U^{ 1/\alpha } V^{ 1/(2\alpha) }\right)^\alpha \right]^{ 1/\alpha }  \Tr\left[ \sigma_B \right]^{ 1- 1/\alpha },
    \end{align}
    and
    \begin{align}
        \Tr\left[ \left( V^{ 1/(2\alpha) } U^{ 1/\alpha } V^{ 1/(2\alpha) }\right)^\alpha \right]\leq 1.
    \end{align}
    Since we have assumed that $\Tr\left[T_{f_{\alpha}}\left(\sigma_B^{1-\alpha}\right)^{1/(1-\alpha)}\right]=1$ and $\Tr\left[ \sigma_B \right]\leq 1$,
    it must be the case that
    \begin{align}
        \Tr\left[ \left( V^{ 1/(2\alpha) } U^{ 1/\alpha } V^{ 1/(2\alpha) }\right)^\alpha \right]
        =\Tr\left[ \left( V^{ -1/\alpha } \right)^{\alpha/(\alpha-1)} \right]
        =\Tr[\sigma_B]
        =1,
    \end{align}
    and
    \begin{align}
        &\Tr\left[ \left( V^{ 1/(2\alpha)} U^{ 1/\alpha } V^{ 1/(2\alpha) }\right) V^{ - 1/\alpha } \right]\\
        &\quad=\Tr\left[ \left( V^{ 1/(2\alpha) } U^{ 1/\alpha } V^{ 1/(2\alpha) }\right)^\alpha \right]^{ 1/\alpha }  \Tr\left[ \left( V^{ -1/\alpha } \right)^{\alpha/(\alpha-1)} \right]^{ 1- 1/\alpha }.
    \end{align}    
    %It remains to verify that $T_{f_{\alpha}}\left(\sigma_B^{1-\alpha}\right)^{1/(1-\alpha)}=\sigma_B$.
    Using the equality condition of the H\"{o}lder inequality (Lemma \ref{ineq:HolderEq}), the above equality implies
    \begin{align}
        \frac{\left( V^{ 1/(2\alpha) } U^{ 1/\alpha } V^{ 1/(2\alpha) }\right)^\alpha}{\Tr\left[\left( V^{ 1/(2\alpha) } U^{ 1/\alpha } V^{ 1/(2\alpha) }\right)^\alpha\right]}
        =\frac{\left( V^{ -1/\alpha } \right)^{\alpha/(\alpha-1)}}{\Tr\left[ \left( V^{ -1/\alpha } \right)^{\alpha/(\alpha-1)} \right]},
    \end{align}
    where the denominators on both sides, as concluded above, are equal to $1$.
    Therefore, we have  
    \[
    \left( V^{ 1/(2\alpha) } U^{ 1/\alpha } V^{ 1/(2\alpha) }\right)^\alpha=\left( V^{ -1/\alpha } \right)^{\alpha/(\alpha-1)} . 
    \]
    Plugging in the definitions of $U$ and $V$, we get 
    \[
    T_{f_{\alpha}}\left(\sigma_B^{1-\alpha}\right)^{1/(1-\alpha)}=\sigma_B . 
    \]
%    Moreover, by the definitions of $U$ and $V$, the following conditions are equivalent:
%    \begin{itemize}
%        \item $\left( V^{ 1/(2\alpha) } U^{ 1/\alpha } V^{ 1/(2\alpha) }\right)^\alpha=\left( V^{ -1/\alpha } \right)^{\alpha/(\alpha-1)}$
%        \item $T_{f_{\alpha}}\left(\sigma_B^{1-\alpha}\right)^{1/(1-\alpha)}=\sigma_B$
%    \end{itemize}
    This completes the proof. 
    % Hence, we conclude that $T_{f_{\alpha}}\left(\sigma_B^{1-\alpha}\right)^{1/(1-\alpha)}=\sigma_B$, provided that $\Tr\left[T_{f_{\alpha}}\left(\sigma_B^{1-\alpha}\right)^{1/(1-\alpha)}\right]=1$.
    %Thus, we conclude that 
    %\begin{align}
    %    \Tr\left[T_{f_{\alpha}}\left(\sigma_B\right)\right]\leq 1.
    %\end{align}
\end{proof}


%Second, we prove that the function values $f_{\alpha}\left(\sigma_B^{(t)}\right)$ are non-increasing.
Next, using Lemma \ref{ineq:TraceBound}, we prove that the function values are non-increasing , as stated in Lemma \ref{ineq:FvalDecrease}.
\begin{lemma}[{Monotonicity of the Function Value}]
    \label{ineq:FvalDecrease}
    For any $\alpha\in(1,\infty)$ and $\sigma_B\in\mathcal{B}\left(\mathcal{H}_B\right)_{++}$ such that $\Tr\left[\sigma_B\right]\leq 1$, we have  
    \begin{align}
        f_{\alpha}\left(T_{f_{\alpha}}\left(\sigma_B^{1-\alpha}\right)^{1/(1-\alpha)}\right)
        \leq f_{\alpha}(\sigma_B).
    \end{align}
    Moreover, equality holds if and only if $\sigma_B^{1 - \alpha}$ is a fixed point of $T_{f_\alpha}$.
\end{lemma}
\begin{proof}
    % Suppose that
    % $\sigma_B^{1 - \alpha}$ is not a fixed point of $T_{f_\alpha}$. 
    Let $\sigma_B \in \mathcal{B}\left(\mathcal{H}_B\right)_{++}$ such that $\Tr\left[\sigma_B\right]\leq 1$. 
    We write
    \begin{align}
        &f_{\alpha}\left(T_{f_{\alpha}}\left(\sigma_B^{1-\alpha}\right)^{1/(1-\alpha)}\right)- f_{\alpha}(\sigma_B)\nonumber\\
        &\quad =\frac{1}{\alpha-1}\mathbb{E}_{P_X}\left[\log\left(\frac{ \Tr\left[ \left(\rho_B^X\right)^{\alpha}T_{f_{\alpha}}\left(\sigma_B^{1-\alpha}\right)\right]}{\Tr\left[ \left(\rho_B^X\right)^{\alpha}\sigma_B^{1-\alpha}\right]}\right)\right]\nonumber\\
        &\quad\leq\frac{1}{\alpha-1}\mathbb{E}_{P_X}\left[\frac{ \Tr\left[ \left(\rho_B^X\right)^{\alpha}T_{f_{\alpha}}\left(\sigma_B^{1-\alpha}\right)\right]}{\Tr\left[ \left(\rho_B^X\right)^{\alpha}\sigma_B^{1-\alpha}\right]}-1\right]\nonumber\\
        &\quad =\frac{1}{\alpha-1}\left(\Tr\left[T_{f_{\alpha}}\left(\sigma_B^{1-\alpha}\right)^{\alpha/(1-\alpha)}T_{f_{\alpha}}\left(\sigma_B^{1-\alpha}\right)\right]-1\right),\nonumber %\\
        % &\quad< 0,
    \end{align}
    where the first inequality exploits the fact that $\log x \leq x - 1$, 
    and 
    the second equality follows from the definition of the operator $T_{f_{\alpha}}$.
    % , and the second inequality follows from Lemma \ref{ineq:TraceBound}.
    % This concludes the proof.
    The lemma then follows from Lemma \ref{ineq:TraceBound}. 
\end{proof}

%Third, we prove that the optimal set $\mathcal{P}_{f_{\alpha}}$ is closed under the operator $T_{f_{\alpha}}$.
%\begin{lemma}[{Closedness of the Optimal Set}]
%    \label{eq:CloseOptSet}
%    Consider $\alpha\in(1,\infty)$.
%    For all $\sigma_B^{\star}$ in the optimal set $\mathcal{P}_{f_{\alpha}}$, we have
%    \begin{align}
%        T_{f_{\alpha}}\left(\left(\sigma_B^{\star}\right)^{1-\alpha}\right)^{1/(1-\alpha)}\in\mathcal{P}_{f_{\alpha}}.
%    \end{align}
%\end{lemma}
Finally, we prove Lemma \ref{eq:FixPtIsMin} by showing that the unique fixed point of the operator $T_{f_{\alpha}}$ is also the minimizer of the optimization problem \eqref{eq:Petz_Augustin}.
\begin{proof}(Lemma \ref{eq:FixPtIsMin})
    By Lemma \ref{eq:ThompIsMetric},  Lemma \ref{ineq:Contract}, and the Banach fixed point theorem, there exists a unique $\sigma_B^{\star}\in\mathcal{B}\left(\mathcal{H}_B\right)_{++}$ such that $\left(\sigma_B^{\star}\right)^{1-\alpha}$ is the fixed point of the operator $T_{f_{\alpha}}$ for $\alpha\in(1/2,1)\cup(1,\infty)$.
    %By Lemma \ref{eq:ChengFix} and the definition of the operator $T_{f_{\alpha}}$, $\sigma_B^{\star}$ is also the unique minimizer of the optimization problem \eqref{eq:Petz_Augustin} for $\alpha\in(1/2,1)$.
    We recall that for $\alpha\in(0,1)$, Lemma \ref{eq:FixPtIsMin} has already been proved by \citet[Proposition 2(b)]{Cheng2019}.
    For $\alpha\in(1,\infty)$, let $\tilde{\sigma}_B^{\star}$ be the minimizer of the optimization problem \eqref{eq:Petz_Augustin}.
    Suppose that $\left(\tilde{\sigma}_B^{\star}\right)^{1-\alpha}$ is not the fixed point of $T_{f_{\alpha}}$.
    Then, the equality conditions in Lemma \ref{ineq:TraceBound} and Lemma \ref{ineq:FvalDecrease} do not hold, and we have 
    \begin{align}
        &f_{\alpha}\left(\frac{T_{f_{\alpha}}\left(\left(\tilde{\sigma}_B^{\star}\right)^{1-\alpha}\right)^{1/(1-\alpha)}}{\Tr\left[T_{f_{\alpha}}\left(\left(\tilde{\sigma}_B^{\star}\right)^{1-\alpha}\right)^{1/(1-\alpha)}\right]}\right)\\
        &\quad=f_{\alpha}\left(T_{f_{\alpha}}\left(\left(\tilde{\sigma}_B^{\star}\right)^{1-\alpha}\right)^{1/(1-\alpha)}\right)+\log\left(\Tr\left[T_{f_{\alpha}}\left(\left(\tilde{\sigma}_B^{\star}\right)^{1-\alpha}\right)^{1/(1-\alpha)}\right]\right)\\
        &\quad<f_{\alpha}\left(\tilde{\sigma}_B^{\star}\right).
    \end{align}
    %where the inequality follows from Lemma \ref{ineq:FvalDecrease} and Lemma \ref{ineq:TraceBound}.
    This inequality contradicts the optimality of $\tilde{\sigma}_B^{\star}$.
    Therefore, we conclude that $\sigma_B^{\star}=\tilde{\sigma}_B^{\star}$.
\end{proof}
\begin{remark}
    After completing our work, we happened to find that the fixed-point property proven by \citet[Proposition 4(c)]{Cheng2018}\footnote{Note that this result does not appear in the journal version \cite{Cheng2022}.} can also lead to the conclusion in Lemma \ref{eq:FixPtIsMin}.
    However, our proof strategy differs from theirs.
    Furthermore, for $\alpha > 1$, our proof strategy yields an additional useful property for implementing the proposed iteration rule: the function values are non-increasing along the iteration path (Lemma \ref{ineq:FvalDecrease}).
\end{remark}


%Second, we show that the unique fixed point of the operator $T_{\alpha}$ is the minimizer of the optimization problem \eqref{eq:Petz_Augustin}.
%\begin{lemma}[{\citet[Remark IV.15]{Mosonyi2021}}]
%    (Suspected! In \cite[Theorem IV.14]{Mosonyi2021}, they only prove it for $\alpha\in\left(0,2\right]$. They claim that hao-chung proved it in 2019. However, \citet{Cheng2024} claim that it was proven by \citet{Mosonyi2021}. Anyhow, I first trust them.)
%    Consider $\alpha\in(0,1)\cup(1,\infty)$.
%    Then, any $\sigma_B^{\star}$ is a minimizer of the optimization problem \eqref{eq:Petz_Augustin} if and only if it satisfies the following equation:
%    \begin{align}
%        \sigma_B^{\star}=\mathbb{E}_{P_X}\left[\frac{\left(\sigma_B^{\star}\right)^{(1-\alpha)/2}\left(\rho_B^{X}\right)^{\alpha}\left(\sigma_B^{\star}\right)^{(1-\alpha)/2}}{\Tr\left[\left(\rho_B^{X}\right)^{\alpha}\left(\sigma_B^{\star}\right)^{1-\alpha}\right]}\right].
%    \end{align}
%\end{lemma}

%\begin{corollary}(Fixed-Point Property)
%    \label{eq:FixedPt}
%    Consider $\alpha\in(1/2,1)\cup(1,\infty)$.
%    Then, minimizer $\sigma_B^{\star}$ of the optimization problem \eqref{eq:Petz_Augustin} uniquely exists, and satisfies the following equation:
%    \begin{align}
%        \left(\sigma_B^{\star}\right)^{1-\alpha}=T_{\alpha}\left(\left(\sigma_B^{\star}\right)^{1-\alpha}\right).
%    \end{align}
%\end{corollary}
% \subsubsection{Stability of Thompson Metric under Normalization}
\subsubsection{Preservation of Thompson Metric under Trace-Normalization}
\label{subsec:NormalStable}
The constraint set of the optimization problem \eqref{eq:Petz_Augustin} is the set of density matrices $\mathcal{D}\left(\mathcal{H}_B\right)$, whereas the traces of the iterates $\sigma_B^{(t)}$ may not equal $1$.
To address this, we show in Lemma \ref{ineq:SmallNormalThompson} that the Thompson metric is 
% stable 
preserved
under 
% normalization.
trace-normalization, up to a multiplicative constant of $2$. 

\begin{lemma}
    \label{ineq:SmallNormalThompson}
    Let $\alpha\in(0,1)\cup(1,\infty)$.
    For any $U,V\in\mathcal{B}\left( \mathcal{H}_B\right)_{++}$ such that $\Tr[V]=1$, we have
    \begin{align}
        d_{\mathrm{T}}\left(V^{1-\alpha},\left(\frac{U}{\Tr[U]}\right)^{1-\alpha}\right)
        \leq 2 d_{\mathrm{T}}\left(V^{1-\alpha},U^{1-\alpha}\right).
    \end{align}
\end{lemma}

% To prove Lemma \ref{ineq:SmallNormalThompson}, we rely on Lemma \ref{ineq:Lidskii}, which was established in earlier research.
We will use 
% Lemma \ref{ineq:ThompsonLogHomo}  and 
Lemma \ref{ineq:Lidskii} to prove Lemma \ref{ineq:SmallNormalThompson}. 

\begin{lemma}[{\cite[Corollary 7.7.4(c)]{Horn2013}}]
    \label{ineq:Lidskii}
    For any $U,V\in\mathcal{B}\left(\mathcal{H}_B\right)_{++}$ such that $U\leq V$, we have
    \begin{align}
        \lambda_i(U)\leq\lambda_i(V),\quad\forall i\in\set{1,2,\dots,d}.
    \end{align}
\end{lemma}
\begin{proof}(Lemma \ref{ineq:SmallNormalThompson})
    Let $U,V$ be as defined in Lemma \ref{ineq:SmallNormalThompson}.
    By Lemma \ref{ineq:ThompsonLogHomo}, we write 
    \begin{align}
        d_{\mathrm{T}}\left(V^{1-\alpha},\left(\frac{U}{\Tr[U]}\right)^{1-\alpha}\right)\leq d_{\mathrm{T}}\left(V^{1-\alpha},U^{1-\alpha}\right)+|\alpha-1|\cdot\left|\log\left(\Tr[U]\right)\right|.
    \end{align}
    It remains to bound the quantity $\Tr[U]$, which can be written as follows:
    \begin{align}
        \Tr[U]
        =\sum_{i=1}^{d}\left(\lambda_i(U)^{1-\alpha}\right)^{1/(1-\alpha)}
        =\sum_{i=1}^{d}\left(\lambda_i(U^{1-\alpha})\right)^{1/(1-\alpha)}.
    \end{align}
    On the other hand, by Lemma \ref{ineq:Lidskii}, for each $i\in\set{1,2,\dots,d}$, we have
    \begin{align}
        \exp\left(- d_{\mathrm{T}}
        \left(V^{1-\alpha},U^{1-\alpha}\right)\right)\lambda_i\left(V^{1-\alpha}\right)
        \leq \lambda_i\left(U^{1-\alpha}\right)
        \leq\exp\left( d_{\mathrm{T}}
        \left(V^{1-\alpha},U^{1-\alpha}\right)\right)\lambda_i\left(V^{1-\alpha}\right).
    \end{align}
    Consequently, we obtain
    \begin{align}
        \exp\left( \frac{-d_{\mathrm{T}}
        \left(V^{1-\alpha},U^{1-\alpha}\right)}{|1-\alpha|}\right)\Tr[V]
        \leq\Tr[U]
        \leq \exp\left( \frac{d_{\mathrm{T}}
        \left(V^{1-\alpha},U^{1-\alpha}\right)}{|1-\alpha|}\right)\Tr[V].
    \end{align}
    Since we assume that $\Tr[V]=1$, it follows that
    \begin{align}
        %\left|\log\left(\Tr[U]\right)\right|
        \abs{\log\left(\Tr[U]\right)}
        \leq\left|\frac{1}{\alpha-1}\right|d_{\mathrm{T}}
        \left(V^{1-\alpha},U^{1-\alpha}\right).
    \end{align}
    This concludes the proof.
\end{proof}



% \subsubsection{Optimization Error in Function Value}
\subsubsection{Bounding Variation in Function Values}
\label{subsec:ControlFval}
Finally, it remains to translate the convergence guarantee of the iterates into that of the function values.
We prove that the difference between the function values is bounded above by the Thompson metric.
\begin{lemma}
    \label{ineq:SmallThomp2SmallOptError}
    Let $\alpha\in(0,1)\cup(1,\infty)$.
    For any $U,V\in\mathcal{B}\left( \mathcal{H}_B\right)_{++}$, we have
    \begin{align}
        f_{\alpha}\left(U\right)-f_{\alpha}\left(V\right)\leq \left|\frac{1}{\alpha-1}\right|d_{\mathrm{T}}\left(V^{1-\alpha},U^{1-\alpha}\right).
    \end{align}
\end{lemma}
\begin{proof}(Lemma \ref{ineq:SmallThomp2SmallOptError})
    %We first assume $\alpha>1$.
    Let $U,V$ be as defined in Lemma \ref{ineq:SmallThomp2SmallOptError}.
    By Definition \ref{eq:Thompson}, we have
    \begin{align}
        \exp\left(- d_{\mathrm{T}}
        \left(V^{1-\alpha},U^{1-\alpha}\right)\right)V^{1-\alpha}
        \leq U^{1-\alpha}
        \leq\exp\left( d_{\mathrm{T}}
        \left(V^{1-\alpha},U^{1-\alpha}\right)\right)V^{1-\alpha}.
    \end{align}
    Therefore, we write
    \begin{align}
        f_{\alpha}\left(U\right)
        &=\mathbb{E}_{P_X}\left[\frac{1}{\alpha-1}\log\operatorname{Tr}\left[ \left(\rho_B^X\right)^{\alpha} U^{1-\alpha}\right]\right]\nonumber\\
        &\leq\mathbb{E}_{P_X}\left[\frac{1}{\alpha-1}\log\operatorname{Tr}\left[ \left(\rho_B^X\right)^{\alpha}\left(\exp\left( d_{\mathrm{T}}\left(V^{1-\alpha},U^{1-\alpha}\right)\right)V^{1-\alpha}\right)\right]\right]\nonumber\\
        &=f_{\alpha}\left(V\right)+\frac{1}{\alpha-1}d_{\mathrm{T}}\left(V^{1-\alpha},U^{1-\alpha}\right)\nonumber
    \end{align}
    for $\alpha>1$, and
    \begin{align}
        f_{\alpha}\left(U\right)
        &\leq\mathbb{E}_{P_X}\left[\frac{1}{\alpha-1}\log\operatorname{Tr}\left[ \left(\rho_B^X\right)^{\alpha}\left(\exp\left( -d_{\mathrm{T}}\left(V^{1-\alpha},U^{1-\alpha}\right)\right)V^{1-\alpha}\right)\right]\right]\nonumber\\
        &=f_{\alpha}\left(V\right)+\frac{1}{1-\alpha} d_{\mathrm{T}}\left(V^{1-\alpha},U^{1-\alpha}\right)\nonumber
    \end{align}
    for $\alpha<1$.
    %where the last equality follows from the log-homogeneity of the function $f_{\alpha}$.
    This concludes the proof.
    %for $\alpha>1$.
    
    %By the same argument, the inequality also holds for $\alpha\in(0,1)$.
\end{proof}

\subsubsection{Proof of the Main Theorem}
\label{subsec:PfMainThm}
\begin{proof}(Theorem \ref{ineq:FvalConv})
    Let $\sigma_B^{\star}$ and $\sigma_B^{(t)}$ be defined as in Theorem \ref{ineq:FvalConv}.
    By Lemma \ref{ineq:Contract}, Lemma \ref{eq:FixPtIsMin}, and induction, we write
    \begin{align}
        d_{\mathrm{T}}\left(\left(\sigma_B^{\star}\right)^{1-\alpha},\left(\sigma_B^{(T+1)}\right)^{1-\alpha}\right) 
        \leq\left|1-\frac{1}{\alpha}\right|^T d_{\mathrm{T}}\left(\left(\sigma_B^{\star}\right)^{1-\alpha},\left(\sigma_B^{(1)}\right)^{1-\alpha}\right).
    \end{align}
    Consequently, by Lemma \ref{ineq:SmallNormalThompson}, we obtain
    \begin{align}
        d_{\mathrm{T}}\left(\left(\sigma_B^{\star}\right)^{1-\alpha},\left(\frac{\sigma_B^{(T+1)}}{\Tr\left[\sigma_B^{(T+1)}\right]}\right)^{1-\alpha}\right) 
        \leq 2\left|1-\frac{1}{\alpha}\right|^T d_{\mathrm{T}}\left(\left(\sigma_B^{\star}\right)^{1-\alpha},\left(\sigma_B^{(1)}\right)^{1-\alpha}\right).
    \end{align}
    By Lemma \ref{ineq:SmallThomp2SmallOptError}, it follows that
    \begin{align}
        f_{\alpha}\left(\frac{\sigma_B^{(T+1)}}{\Tr\left[\sigma_B^{(T+1)}\right]}\right)-f_{\alpha}\left(\sigma_B^{\star}\right)
        \leq \left|\frac{2}{\alpha-1}\right|\cdot \left|1-\frac{1}{\alpha}\right|^T d_{\mathrm{T}}\left(\left(\sigma_B^{\star}\right)^{1-\alpha},\left(\sigma_B^{(1)}\right)^{1-\alpha}\right).
    \end{align}
    It remains to prove that
    \begin{align}
        f_{\alpha}\left(\frac{\sigma_B^{(t+1)}}{\Tr\left[\sigma_B^{(t+1)}\right]}
        \right)
        \leq f_{\alpha}\left(\frac{\sigma_B^{(t)}}{\Tr\left[\sigma_B^{(t)}\right]}\right),\quad\forall\alpha>1.
    \end{align} 
    By the monotonicity of the function value (Lemma \ref{ineq:FvalDecrease}) and the bound of the trace (Lemma \ref{ineq:TraceBound}), we write
    \begin{align}
        &f_{\alpha}\left(\frac{T_{f_{\alpha}}\left(\left(\frac{\sigma_B^{(t)}}{\Tr\left[\sigma_B^{(t)}\right]}\right)^{1-\alpha}\right)^{1/(1-\alpha)}}{\Tr\left[T_{f_{\alpha}}\left(\left(\frac{\sigma_B^{(t)}}{\Tr\left[\sigma_B^{(t)}\right]}\right)^{1-\alpha}\right)^{1/(1-\alpha)}\right]}\right)\\
        &\quad=f_{\alpha}\left(T_{f_{\alpha}}\left(\left(\frac{\sigma_B^{(t)}}{\Tr\left[\sigma_B^{(t)}\right]}\right)^{1-\alpha}\right)^{1/(1-\alpha)}\right)+\log\left(\Tr\left[T_{f_{\alpha}}\left(\left(\frac{\sigma_B^{(t)}}{\Tr\left[\sigma_B^{(t)}\right]}\right)^{1-\alpha}\right)^{1/(1-\alpha)}\right]\right)\\
        &\quad\leq f_{\alpha}\left(\frac{\sigma_B^{(t)}}{\Tr\left[\sigma_B^{(t)}\right]}\right)+0,\quad\forall\alpha>1.
    \end{align}
    It remains to prove that
    \begin{align}
        \frac{\sigma_B^{(t+1)}}{\Tr\left[\sigma_B^{(t+1)}\right]}
        = \frac{T_{f_{\alpha}}\left(\left(\frac{\sigma_B^{(t)}}{\Tr\left[\sigma_B^{(t)}\right]}\right)^{1-\alpha}\right)^{1/(1-\alpha)}}{\Tr\left[T_{f_{\alpha}}\left(\left(\frac{\sigma_B^{(t)}}{\Tr\left[\sigma_B^{(t)}\right]}\right)^{1-\alpha}\right)^{1/(1-\alpha)}\right]}.
    \end{align}
    Note that for any $\sigma\in\mathcal{B}\left(\mathcal{H}_B\right)_{++}$ and $\gamma>0$, we have
    \begin{align}
        T_{f_{\alpha}}\left(\left(\gamma\sigma\right)^{1-\alpha}\right)^{1/(1-\alpha)}
        =\gamma^{(\alpha-1)/\alpha} T_{f_{\alpha}}\left(\sigma^{1-\alpha}\right)^{1/(1-\alpha)},
    \end{align}
    and thus,
    \begin{align}
        \frac{T_{f_{\alpha}}\left(\left(\gamma\sigma\right)^{1-\alpha}\right)^{1/(1-\alpha)}}{\Tr\left[T_{f_{\alpha}}\left(\left(\gamma\sigma\right)^{1-\alpha}\right)^{1/(1-\alpha)}\right]}
        =\frac{T_{f_{\alpha}}\left(\sigma^{1-\alpha}\right)^{1/(1-\alpha)}}{\Tr\left[T_{f_{\alpha}}\left(\sigma^{1-\alpha}\right)^{1/(1-\alpha)}\right]}.
    \end{align}
    This implies that, by taking $\sigma=\sigma_B^{(t)}$ and $\gamma=1/\Tr\left[\sigma_B^{(t)}\right]$,
    \begin{align}
        \frac{\sigma_B^{(t+1)}}{\Tr\left[\sigma_B^{(t+1)}\right]}
        =\frac{T_{f_{\alpha}}\left(\left(\sigma_B^{(t)}\right)^{1-\alpha}\right)^{1/(1-\alpha)}}{\Tr\left[T_{f_{\alpha}}\left(\left(\sigma_B^{(t)}\right)^{1-\alpha}\right)^{1/(1-\alpha)}\right]}
        = \frac{T_{f_{\alpha}}\left(\left(\frac{\sigma_B^{(t)}}{\Tr\left[\sigma_B^{(t)}\right]}\right)^{1-\alpha}\right)^{1/(1-\alpha)}}{\Tr\left[T_{f_{\alpha}}\left(\left(\frac{\sigma_B^{(t)}}{\Tr\left[\sigma_B^{(t)}\right]}\right)^{1-\alpha}\right)^{1/(1-\alpha)}\right]}.
    \end{align}
    
    Finally, since $\left(\sigma_B^{(1)}\right)^{1-\alpha}$ and $\left(\sigma_B^{\star}\right)^{1-\alpha}$ are full-rank density matrices, and the Thompson metric is a metric on $\mathcal{B}\left(\mathcal{H}_B\right)_{++}$ (Lemma \ref{eq:ThompIsMetric}), the quantity $d_{\mathrm{T}}\left(\left(\sigma_B^{\star}\right)^{1-\alpha},\left(\sigma_B^{(1)}\right)^{1-\alpha}\right)$ is finite. 
    This concludes the proof.
\end{proof}

\section{Interesting Connection with $\ell_p$-Lewis Weights}
\label{subsec:RelationWithCP}
Our iteration rule, introduced in Section \ref{subsec:CohenIteration}, is inspired by the iteration rule 
%due to 
proposed by
\citet{Cohen2015} for computing the $\ell_p$-Lewis weights. 
% the computation of $\ell_p$-Lewis weight in \cite{Cohen2015}. % which involves computing
To this end, we introduce the $\ell_p$-Lewis weights and discuss their connection to our work.
%The following paragraphs in this section are organized as follows:
%\begin{itemize}
%    \item First, we introduce the optimization problem defining the $\ell_p$-Lewis weights (Problem \eqref{eq:LewisDef}), along with the iteration rule proposed by \citet{Cohen2015} for solving it (Equation \eqref{eq:CPIterDef}).
%    \item Second, to demonstrate the similarity between the $\ell_p$-Lewis weights and the Petz-Augustin information, we reformulate a special case of the optimization problem defining the $\ell_p$-Lewis weights (Problem \eqref{eq:ReWrittenLewis}) and the corresponding iteration rule (Equation \eqref{eq:RawCohenIteration}).
%    \item Finally, we highlight the challenges in adapting the algorithm proposed by \citet{Cohen2015} to compute the Petz-Augustin information.
%\end{itemize}

The $\ell_p$-Lewis weights are given by \cite{Lee2020}
\begin{equation}
    \label{eq:LewisDef}
    w^{\star}\in\argmin_{w\in m \Delta_{d-1}}\frac{-1}{1-\frac{2}{p}}\log\det\left(A^{\intercal}\mathrm{Diag}\left(w^{1-2/p}\right)A\right),
\end{equation}
for $p\in(0,\infty)$, where $A$ is a $d$-by-$m$ real matrix and $A^\intercal$ denotes the transpose of $A$.
\citet{Cohen2015} proposed the following iteration rule: 
\begin{equation}
    \label{eq:CPIterDef}
    w^{(t+1)}=T_{\mathrm{CP}}\left(w^{(t)}\right),
\end{equation}
where
\begin{align}
    T_{\mathrm{CP}}\left(w^{(t)}\right)[i]\coloneqq\left(A_i^{\intercal}\left(A^{\intercal}\mathrm{Diag}\left(\left(w^{(t)}\right)^{1-2/p}\right)A\right)^{-1}A_i\right)^{p/2},
\end{align}
for each $i\in\set{1,2,\dots,d}$.
They proved that 
% their 
the 
iterates 
% $\mathrm{Diag}\left(w_t\right)$ 
$w^{(t)}$
converge linearly to 
% $\mathrm{Diag}\left(w_{\star}\right)$ 
$w^{\star}$
with respect to the Thompson metric for $p\in(0,4)$.

To demonstrate the similarity between their proposed iteration rule and ours (Equation \eqref{eq:DiscreteCP}), we note that, when $m=1$, the $\ell_p$-Lewis weights can be written as 
\begin{equation}
    \label{eq:ReWrittenLewis}
    w^{\star}\in\argmin_{w\in \Delta_{d-1}}\frac{-1}{1-\frac{2}{p}}\log\left(\Tr\left[\mathrm{Diag}\left(a\right)^{2/p}\mathrm{Diag}\left(w\right)^{1-2/p}\right]\right),
\end{equation}
where $a$ is a $d$-dimensional vector with $a[i]=|A_i|^p$ for each $i\in\set{1,2,\dots,d}$.
Let $\alpha=2/p$. 
Then, the problem corresponds to a special case of the optimization problem \eqref{eq:Petz_Augustin} where all 
% the 
matrices commute.
%, and the cardinality of the input alphabet is $1$. 
In this case, 
% their algorithm 
the iteration rule proposed by \citet{Cohen2015}
can be expressed as
\begin{equation}
    \label{eq:RawCohenIteration}
    \mathrm{Diag}\left(w^{(t+1)}\right)=\left(\frac{\mathrm{Diag}\left(a\right)^{\alpha}}{\Tr\left[\mathrm{Diag}\left(a\right)^{\alpha}\mathrm{Diag}\left(w^{(t)}\right)^{1-\alpha}\right]}\right)^{1/\alpha} , 
\end{equation}
where we deliberately write the iterates as matrices to illustrate the similarity with our iteration rule \eqref{eq:DiscreteCP}. 
%Therefore, our algorithm can be viewed as a generalization of the algorithm proposed by \citet{Cohen2015}.


Despite the similarity in algorithms, our results cannot be trivially derived from the work of \citet{Cohen2015}.
Specifically, there are two main challenges in adapting their proposed algorithm to compute the Petz–Augustin information:
\begin{itemize}
    \item The optimization problem \eqref{eq:Petz_Augustin} defining the Petz–Augustin information includes an additional expectation term, $\mathbb{E}_{P_X}\left[\cdot\right]$.
    %, taken over the input alphabet $X$.
    Consequently, even when all the matrices commute, the computation of the Petz–Augustin information cannot be reduced to that of the $\ell_p$-Lewis weights.
    \item \citet{Cohen2015} established the contractive property of their proposed algorithm with respect to the Thompson metric by proving
    \begin{align}
    \log\left(\max\Set{\frac{T_{\mathrm{CP}}(u)[i]}{T_{\mathrm{CP}}(v)[i]}, \frac{T_{\mathrm{CP}}(v)[i]}{T_{\mathrm{CP}}(u)[i]}}\right)\leq\left|1-\frac{p}{2}\right|\log\left(\max\Set{\frac{u[i]}{v[i]}, \frac{v[i]}{u[i]}}\right),
    \end{align}
    for each $1 \leq i \leq d$.
    This can be viewed as comparing the eigenvalues of the commuting matrices $\mathrm{Diag}\left(u\right)$ and $\mathrm{Diag}\left(v\right)$.
    However, in our case, the iterates $\sigma_B^{(t)}$ and the minimizer $\sigma_B^{\star}$, as defined in Theorem \ref{ineq:FvalConv}, may not commute, making it infeasible to compare their eigenvalues within matched eigenspaces.
    %\item In our case, we aim to analyze the convergence of the function values $f_{\alpha}(\sigma_B^{(t)})$, but it is unclear whether this can be achieved based on the convergence guarantee of the iterates $\sigma_B^{(t)}$.
\end{itemize}

For the first challenge, we identify an appropriate generalization of the algorithm proposed by \citet{Cohen2015}, leading to a new algorithm introduced in Section \ref{subsec:CohenIteration} for computing the Petz-Augustin information.

For the second challenge, we leverage the properties of the Thompson metric specialized for $\mathcal{B}\left(\mathcal{H}_B\right)_{++}$ (Lemma \ref{eq:ThompIsMetric}, \ref{ineq:NegCurveProp} and \ref{ineq:ThompsonLogHomo}).
Specifically, in Lemma \ref{ineq:Contract}, we show that our proposed algorithm is a contraction with respect to the Thompson metric for $\alpha\in(1/2,1)\cup(1,\infty)$.

%For the third challenge, we prove in Lemma \ref{ineq:SmallThomp2SmallOptError} and Theorem \ref{ineq:FvalConv} that the convergence guarantee of the iterates $\sigma_B^{(t)}$ with respect to the Thompson metric can be translated into the convergence guarantee of the function values $f_{\alpha}(\sigma_B^{(t)})$.


\section{Numerical Results} \label{sec:numerics}

% \begin{figure}[H]
\begin{figure}[ht]
    \begin{subfigure}{.5\textwidth}
      \centering
      \includegraphics[width=.8\linewidth]{figure_alpha_0.8.png}
      \caption{$\alpha=0.8$}
      \label{fig:sfig1}
    \end{subfigure}%
    \begin{subfigure}{.5\textwidth}
      \centering
      \includegraphics[width=.8\linewidth]{figure_alpha_1.5.png}
      \caption{$\alpha=1.5$}
      \label{fig:sfig2}
    \end{subfigure}
    \begin{subfigure}{.5\textwidth}
      \centering
      \includegraphics[width=.8\linewidth]{figure_alpha_3.png}
      \caption{$\alpha=3$}
      \label{fig:sfig3}
    \end{subfigure}
    \begin{subfigure}{.5\textwidth}
      \centering
      \includegraphics[width=.8\linewidth]{figure_alpha_5.png}
      \caption{$\alpha=5$}
      \label{fig:sfig4}
    \end{subfigure}
    \caption{Approximate optimization error versus the number of iterations for $\alpha>0.5$}
    \label{fig1}
\end{figure}

% \begin{figure}[H]
\begin{figure}[ht]
    \begin{subfigure}{.5\textwidth}
      \centering
      \includegraphics[width=.8\linewidth]{figure_alpha_0.2.png}
      \caption{$\alpha=0.2$}
      \label{fig:sfig5}
    \end{subfigure}
    \begin{subfigure}{.5\textwidth}
      \centering
      \includegraphics[width=.8\linewidth]{figure_alpha_0.4.png}
      \caption{$\alpha=0.4$}
      \label{fig:sfig6}
    \end{subfigure}
    \caption{Approximate optimization error versus the number of iterations for $\alpha\leq 0.5$}
    \label{fig2}
\end{figure}
We implement our proposed 
% algorithm 
iteration rule 
described in Section \ref{subsec:CohenIteration} to compute the Petz-Augustin information for $\alpha$ in $\set{0.2,0.4,0.8,1.5,3,5}$. 
The source code is available on GitHub\footnote{\url{https://github.com/chunnengchu/PetzAugustin/}}.
Throughout the experiments for $\alpha>0.5$, we set the cardinality of $\mathcal{X}$ to $2^{5}$ and the dimension of $\mathcal{H}_B$ to $2^{7}$. 
The quantum states $ \rho_B^x $ are generated using the 
% $\href{https://qutip.org/docs/3.1.0/guide/guide-random.html}{rand\_dm}$ 
\texttt{rand\_dm}
function from the 
% PYTHON 
Python 
package QuTiP \citep{Johansson2012}.
Since the exact solution of the optimization problem \eqref{eq:Petz_Augustin} is unavailable, we compute an approximate optimization error instead. 
The approximate optimization error is defined as $f_{\alpha}\left(\sigma_B^{(t)}/\Tr\left[\sigma_B^{(t)}\right]\right)-\hat{f}^{\star}$, where $\hat{f}^\star$ denotes the function value of the last iterate after $30$ iterations of our proposed algorithm.
%The approximate optimization error at the $t^{\text{th}}$ iteration is computed as 
%% $f_{\alpha}\left(\sigma_B^{(t)}/\Tr\left[\sigma_B^{(t)}\right]\right)-f^{\star}$, 
%$f_{\alpha}\left(\sigma_B^{(t)}/\Tr\left[\sigma_B^{(t)}\right]\right)-\hat{f}_{\star}$, 
%where 
%% $f^{\star}$ 
%$\hat{f}_\star$
%denotes the function value of the last iterate after $30$ iterations of our proposed algorithm. 

% For $\alpha\leq 0.5$, 
For experiments with $\alpha \leq 0.5$, we manually design a challenging instance of the optimization problem \eqref{eq:Petz_Augustin}, 
% which is described in detail in \texttt{README.md} in the 
detailed in the \texttt{README.md} file in the 
% same 
aforementioned 
GitHub repository. 
The approximate optimization error is defined similarly as above. 
%Similarly, the approximate optimization error at the $t^{\text{th}}$ iteration is computed as $f_{\alpha}\left(\sigma_B^{(t)}/\Tr\left[\sigma_B^{(t)}\right]\right)-f^{\star}$.
However, our proposed algorithm is not guaranteed to converge for $\alpha\leq 0.5$.
Hence, for $\alpha\leq 0.5$, we 
% denote $f^{\star}$ as 
replace $\hat{f}^{\star}$ with 
the function value of the best iterate over $30$ iterations of entropic mirror descent with the Polyak step size \cite{You2022}, as this method is guaranteed to converge asymptotically and is known to converge quickly in practice.
% And the 


In Figure \ref{fig1}, we observe linear convergence rates for the optimization error when $\alpha>0.5$.
Notably, Theorem \ref{ineq:FvalConv} establishes that the exponent of the linear convergence rate is bounded above by $\left|1 - \frac{1}{\alpha}\right|$.
Consistent with this result, Figure \ref{fig1} demonstrates a similar relationship between $\alpha$ and the
empirical 
convergence rate.

%Our convergence guarantee in Theorem \ref{ineq:FvalConv} does not cover the case where $\alpha\in(0,1)$, which appears in characterizing the sphere-packing exponent of classical-quantum channel coding \cite{Dalai2014, Cheng2019}.
Since Lemma \ref{eq:FixPtIsMin} implies that our algorithm functions as a fixed-point iteration for $\alpha \in (0, 1)\cup(1,\infty)$, despite the fact that our convergence guarantee in Theorem \ref{ineq:FvalConv} does not cover the case where $\alpha \leq 0.5$, we present in Figure \ref{fig2} the experimental results for $\alpha \in \set{0.2, 0.4}$. 
Numerical experiments suggest that our proposed algorithm seems to diverge for $\alpha\in\set{0.2,0.4}$ on the aforementioned synthetic instance of the optimization problem \eqref{eq:Petz_Augustin}.
%Additionally, we observe that the algorithm generates iterates with extremely large or small entries after a few iterations. 
%This behavior causes the function values to exceed Python's representable numerical range.
Since no existing algorithm for computing the Petz-Augustin information of order $\alpha\in\left(0,1/2\right]$ has a non-asymptotic convergence guarantee, developing a rigorous algorithm for this purpose remains an open direction for future research.
%Additionally, the valleys in the middle of Figures \ref{fig:sfig5} and \ref{fig:sfig6} indicate that the smallest function values are encountered in the first few iterations, respectively.

\section{Acknowledgements}

This work is supported by the Young Scholar Fellowship (Einstein Program) of the National Science and Technology Council (NSTC) of Taiwan under grant number NSTC 112-2636-E-002-003; the 2030 Cross-Generation Young Scholars Program (Excellent Young Scholars) of the NSTC under grant number NSTC 112-2628-E-002-019-MY3; the research project “Geometry of Quantum Learning and Optimization” of National Taiwan University under grant number NTU-CC-114L895006; and the Academic Career Development Research Program (Laurel Research Project) of National Taiwan University under grant number NTU-CDP-114L7744.

%\bibliographystyle{alpha}
\bibliography{refs}
%% This must be in the first 5 lines to tell arXiv to use pdfLaTeX, which is strongly recommended.
\pdfoutput=1
% In particular, the hyperref package requires pdfLaTeX in order to break URLs across lines.

\documentclass[11pt]{article}

% Change "review" to "final" to generate the final (sometimes called camera-ready) version.
% Change to "preprint" to generate a non-anonymous version with page numbers.
\usepackage{acl}

% Standard package includes
\usepackage{times}
\usepackage{latexsym}

% Draw tables
\usepackage{booktabs}
\usepackage{multirow}
\usepackage{xcolor}
\usepackage{colortbl}
\usepackage{array} 
\usepackage{amsmath}

\newcolumntype{C}{>{\centering\arraybackslash}p{0.07\textwidth}}
% For proper rendering and hyphenation of words containing Latin characters (including in bib files)
\usepackage[T1]{fontenc}
% For Vietnamese characters
% \usepackage[T5]{fontenc}
% See https://www.latex-project.org/help/documentation/encguide.pdf for other character sets
% This assumes your files are encoded as UTF8
\usepackage[utf8]{inputenc}

% This is not strictly necessary, and may be commented out,
% but it will improve the layout of the manuscript,
% and will typically save some space.
\usepackage{microtype}
\DeclareMathOperator*{\argmax}{arg\,max}
% This is also not strictly necessary, and may be commented out.
% However, it will improve the aesthetics of text in
% the typewriter font.
\usepackage{inconsolata}

%Including images in your LaTeX document requires adding
%additional package(s)
\usepackage{graphicx}
% If the title and author information does not fit in the area allocated, uncomment the following
%
%\setlength\titlebox{<dim>}
%
% and set <dim> to something 5cm or larger.

\title{Wi-Chat: Large Language Model Powered Wi-Fi Sensing}

% Author information can be set in various styles:
% For several authors from the same institution:
% \author{Author 1 \and ... \and Author n \\
%         Address line \\ ... \\ Address line}
% if the names do not fit well on one line use
%         Author 1 \\ {\bf Author 2} \\ ... \\ {\bf Author n} \\
% For authors from different institutions:
% \author{Author 1 \\ Address line \\  ... \\ Address line
%         \And  ... \And
%         Author n \\ Address line \\ ... \\ Address line}
% To start a separate ``row'' of authors use \AND, as in
% \author{Author 1 \\ Address line \\  ... \\ Address line
%         \AND
%         Author 2 \\ Address line \\ ... \\ Address line \And
%         Author 3 \\ Address line \\ ... \\ Address line}

% \author{First Author \\
%   Affiliation / Address line 1 \\
%   Affiliation / Address line 2 \\
%   Affiliation / Address line 3 \\
%   \texttt{email@domain} \\\And
%   Second Author \\
%   Affiliation / Address line 1 \\
%   Affiliation / Address line 2 \\
%   Affiliation / Address line 3 \\
%   \texttt{email@domain} \\}
% \author{Haohan Yuan \qquad Haopeng Zhang\thanks{corresponding author} \\ 
%   ALOHA Lab, University of Hawaii at Manoa \\
%   % Affiliation / Address line 2 \\
%   % Affiliation / Address line 3 \\
%   \texttt{\{haohany,haopengz\}@hawaii.edu}}
  
\author{
{Haopeng Zhang$\dag$\thanks{These authors contributed equally to this work.}, Yili Ren$\ddagger$\footnotemark[1], Haohan Yuan$\dag$, Jingzhe Zhang$\ddagger$, Yitong Shen$\ddagger$} \\
ALOHA Lab, University of Hawaii at Manoa$\dag$, University of South Florida$\ddagger$ \\
\{haopengz, haohany\}@hawaii.edu\\
\{yiliren, jingzhe, shen202\}@usf.edu\\}



  
%\author{
%  \textbf{First Author\textsuperscript{1}},
%  \textbf{Second Author\textsuperscript{1,2}},
%  \textbf{Third T. Author\textsuperscript{1}},
%  \textbf{Fourth Author\textsuperscript{1}},
%\\
%  \textbf{Fifth Author\textsuperscript{1,2}},
%  \textbf{Sixth Author\textsuperscript{1}},
%  \textbf{Seventh Author\textsuperscript{1}},
%  \textbf{Eighth Author \textsuperscript{1,2,3,4}},
%\\
%  \textbf{Ninth Author\textsuperscript{1}},
%  \textbf{Tenth Author\textsuperscript{1}},
%  \textbf{Eleventh E. Author\textsuperscript{1,2,3,4,5}},
%  \textbf{Twelfth Author\textsuperscript{1}},
%\\
%  \textbf{Thirteenth Author\textsuperscript{3}},
%  \textbf{Fourteenth F. Author\textsuperscript{2,4}},
%  \textbf{Fifteenth Author\textsuperscript{1}},
%  \textbf{Sixteenth Author\textsuperscript{1}},
%\\
%  \textbf{Seventeenth S. Author\textsuperscript{4,5}},
%  \textbf{Eighteenth Author\textsuperscript{3,4}},
%  \textbf{Nineteenth N. Author\textsuperscript{2,5}},
%  \textbf{Twentieth Author\textsuperscript{1}}
%\\
%\\
%  \textsuperscript{1}Affiliation 1,
%  \textsuperscript{2}Affiliation 2,
%  \textsuperscript{3}Affiliation 3,
%  \textsuperscript{4}Affiliation 4,
%  \textsuperscript{5}Affiliation 5
%\\
%  \small{
%    \textbf{Correspondence:} \href{mailto:email@domain}{email@domain}
%  }
%}

\begin{document}
\maketitle
\begin{abstract}
Recent advancements in Large Language Models (LLMs) have demonstrated remarkable capabilities across diverse tasks. However, their potential to integrate physical model knowledge for real-world signal interpretation remains largely unexplored. In this work, we introduce Wi-Chat, the first LLM-powered Wi-Fi-based human activity recognition system. We demonstrate that LLMs can process raw Wi-Fi signals and infer human activities by incorporating Wi-Fi sensing principles into prompts. Our approach leverages physical model insights to guide LLMs in interpreting Channel State Information (CSI) data without traditional signal processing techniques. Through experiments on real-world Wi-Fi datasets, we show that LLMs exhibit strong reasoning capabilities, achieving zero-shot activity recognition. These findings highlight a new paradigm for Wi-Fi sensing, expanding LLM applications beyond conventional language tasks and enhancing the accessibility of wireless sensing for real-world deployments.
\end{abstract}

\section{Introduction}

In today’s rapidly evolving digital landscape, the transformative power of web technologies has redefined not only how services are delivered but also how complex tasks are approached. Web-based systems have become increasingly prevalent in risk control across various domains. This widespread adoption is due their accessibility, scalability, and ability to remotely connect various types of users. For example, these systems are used for process safety management in industry~\cite{kannan2016web}, safety risk early warning in urban construction~\cite{ding2013development}, and safe monitoring of infrastructural systems~\cite{repetto2018web}. Within these web-based risk management systems, the source search problem presents a huge challenge. Source search refers to the task of identifying the origin of a risky event, such as a gas leak and the emission point of toxic substances. This source search capability is crucial for effective risk management and decision-making.

Traditional approaches to implementing source search capabilities into the web systems often rely on solely algorithmic solutions~\cite{ristic2016study}. These methods, while relatively straightforward to implement, often struggle to achieve acceptable performances due to algorithmic local optima and complex unknown environments~\cite{zhao2020searching}. More recently, web crowdsourcing has emerged as a promising alternative for tackling the source search problem by incorporating human efforts in these web systems on-the-fly~\cite{zhao2024user}. This approach outsources the task of addressing issues encountered during the source search process to human workers, leveraging their capabilities to enhance system performance.

These solutions often employ a human-AI collaborative way~\cite{zhao2023leveraging} where algorithms handle exploration-exploitation and report the encountered problems while human workers resolve complex decision-making bottlenecks to help the algorithms getting rid of local deadlocks~\cite{zhao2022crowd}. Although effective, this paradigm suffers from two inherent limitations: increased operational costs from continuous human intervention, and slow response times of human workers due to sequential decision-making. These challenges motivate our investigation into developing autonomous systems that preserve human-like reasoning capabilities while reducing dependency on massive crowdsourced labor.

Furthermore, recent advancements in large language models (LLMs)~\cite{chang2024survey} and multi-modal LLMs (MLLMs)~\cite{huang2023chatgpt} have unveiled promising avenues for addressing these challenges. One clear opportunity involves the seamless integration of visual understanding and linguistic reasoning for robust decision-making in search tasks. However, whether large models-assisted source search is really effective and efficient for improving the current source search algorithms~\cite{ji2022source} remains unknown. \textit{To address the research gap, we are particularly interested in answering the following two research questions in this work:}

\textbf{\textit{RQ1: }}How can source search capabilities be integrated into web-based systems to support decision-making in time-sensitive risk management scenarios? 
% \sq{I mention ``time-sensitive'' here because I feel like we shall say something about the response time -- LLM has to be faster than humans}

\textbf{\textit{RQ2: }}How can MLLMs and LLMs enhance the effectiveness and efficiency of existing source search algorithms? 

% \textit{\textbf{RQ2:}} To what extent does the performance of large models-assisted search align with or approach the effectiveness of human-AI collaborative search? 

To answer the research questions, we propose a novel framework called Auto-\
S$^2$earch (\textbf{Auto}nomous \textbf{S}ource \textbf{Search}) and implement a prototype system that leverages advanced web technologies to simulate real-world conditions for zero-shot source search. Unlike traditional methods that rely on pre-defined heuristics or extensive human intervention, AutoS$^2$earch employs a carefully designed prompt that encapsulates human rationales, thereby guiding the MLLM to generate coherent and accurate scene descriptions from visual inputs about four directional choices. Based on these language-based descriptions, the LLM is enabled to determine the optimal directional choice through chain-of-thought (CoT) reasoning. Comprehensive empirical validation demonstrates that AutoS$^2$-\ 
earch achieves a success rate of 95–98\%, closely approaching the performance of human-AI collaborative search across 20 benchmark scenarios~\cite{zhao2023leveraging}. 

Our work indicates that the role of humans in future web crowdsourcing tasks may evolve from executors to validators or supervisors. Furthermore, incorporating explanations of LLM decisions into web-based system interfaces has the potential to help humans enhance task performance in risk control.






\section{Related Work}
\label{sec:relatedworks}

% \begin{table*}[t]
% \centering 
% \renewcommand\arraystretch{0.98}
% \fontsize{8}{10}\selectfont \setlength{\tabcolsep}{0.4em}
% \begin{tabular}{@{}lc|cc|cc|cc@{}}
% \toprule
% \textbf{Methods}           & \begin{tabular}[c]{@{}c@{}}\textbf{Training}\\ \textbf{Paradigm}\end{tabular} & \begin{tabular}[c]{@{}c@{}}\textbf{$\#$ PT Data}\\ \textbf{(Tokens)}\end{tabular} & \begin{tabular}[c]{@{}c@{}}\textbf{$\#$ IFT Data}\\ \textbf{(Samples)}\end{tabular} & \textbf{Code}  & \begin{tabular}[c]{@{}c@{}}\textbf{Natural}\\ \textbf{Language}\end{tabular} & \begin{tabular}[c]{@{}c@{}}\textbf{Action}\\ \textbf{Trajectories}\end{tabular} & \begin{tabular}[c]{@{}c@{}}\textbf{API}\\ \textbf{Documentation}\end{tabular}\\ \midrule 
% NexusRaven~\citep{srinivasan2023nexusraven} & IFT & - & - & \textcolor{green}{\CheckmarkBold} & \textcolor{green}{\CheckmarkBold} &\textcolor{red}{\XSolidBrush}&\textcolor{red}{\XSolidBrush}\\
% AgentInstruct~\citep{zeng2023agenttuning} & IFT & - & 2k & \textcolor{green}{\CheckmarkBold} & \textcolor{green}{\CheckmarkBold} &\textcolor{red}{\XSolidBrush}&\textcolor{red}{\XSolidBrush} \\
% AgentEvol~\citep{xi2024agentgym} & IFT & - & 14.5k & \textcolor{green}{\CheckmarkBold} & \textcolor{green}{\CheckmarkBold} &\textcolor{green}{\CheckmarkBold}&\textcolor{red}{\XSolidBrush} \\
% Gorilla~\citep{patil2023gorilla}& IFT & - & 16k & \textcolor{green}{\CheckmarkBold} & \textcolor{green}{\CheckmarkBold} &\textcolor{red}{\XSolidBrush}&\textcolor{green}{\CheckmarkBold}\\
% OpenFunctions-v2~\citep{patil2023gorilla} & IFT & - & 65k & \textcolor{green}{\CheckmarkBold} & \textcolor{green}{\CheckmarkBold} &\textcolor{red}{\XSolidBrush}&\textcolor{green}{\CheckmarkBold}\\
% LAM~\citep{zhang2024agentohana} & IFT & - & 42.6k & \textcolor{green}{\CheckmarkBold} & \textcolor{green}{\CheckmarkBold} &\textcolor{green}{\CheckmarkBold}&\textcolor{red}{\XSolidBrush} \\
% xLAM~\citep{liu2024apigen} & IFT & - & 60k & \textcolor{green}{\CheckmarkBold} & \textcolor{green}{\CheckmarkBold} &\textcolor{green}{\CheckmarkBold}&\textcolor{red}{\XSolidBrush} \\\midrule
% LEMUR~\citep{xu2024lemur} & PT & 90B & 300k & \textcolor{green}{\CheckmarkBold} & \textcolor{green}{\CheckmarkBold} &\textcolor{green}{\CheckmarkBold}&\textcolor{red}{\XSolidBrush}\\
% \rowcolor{teal!12} \method & PT & 103B & 95k & \textcolor{green}{\CheckmarkBold} & \textcolor{green}{\CheckmarkBold} & \textcolor{green}{\CheckmarkBold} & \textcolor{green}{\CheckmarkBold} \\
% \bottomrule
% \end{tabular}
% \caption{Summary of existing tuning- and pretraining-based LLM agents with their training sample sizes. "PT" and "IFT" denote "Pre-Training" and "Instruction Fine-Tuning", respectively. }
% \label{tab:related}
% \end{table*}

\begin{table*}[ht]
\begin{threeparttable}
\centering 
\renewcommand\arraystretch{0.98}
\fontsize{7}{9}\selectfont \setlength{\tabcolsep}{0.2em}
\begin{tabular}{@{}l|c|c|ccc|cc|cc|cccc@{}}
\toprule
\textbf{Methods} & \textbf{Datasets}           & \begin{tabular}[c]{@{}c@{}}\textbf{Training}\\ \textbf{Paradigm}\end{tabular} & \begin{tabular}[c]{@{}c@{}}\textbf{\# PT Data}\\ \textbf{(Tokens)}\end{tabular} & \begin{tabular}[c]{@{}c@{}}\textbf{\# IFT Data}\\ \textbf{(Samples)}\end{tabular} & \textbf{\# APIs} & \textbf{Code}  & \begin{tabular}[c]{@{}c@{}}\textbf{Nat.}\\ \textbf{Lang.}\end{tabular} & \begin{tabular}[c]{@{}c@{}}\textbf{Action}\\ \textbf{Traj.}\end{tabular} & \begin{tabular}[c]{@{}c@{}}\textbf{API}\\ \textbf{Doc.}\end{tabular} & \begin{tabular}[c]{@{}c@{}}\textbf{Func.}\\ \textbf{Call}\end{tabular} & \begin{tabular}[c]{@{}c@{}}\textbf{Multi.}\\ \textbf{Step}\end{tabular}  & \begin{tabular}[c]{@{}c@{}}\textbf{Plan}\\ \textbf{Refine}\end{tabular}  & \begin{tabular}[c]{@{}c@{}}\textbf{Multi.}\\ \textbf{Turn}\end{tabular}\\ \midrule 
\multicolumn{13}{l}{\emph{Instruction Finetuning-based LLM Agents for Intrinsic Reasoning}}  \\ \midrule
FireAct~\cite{chen2023fireact} & FireAct & IFT & - & 2.1K & 10 & \textcolor{red}{\XSolidBrush} &\textcolor{green}{\CheckmarkBold} &\textcolor{green}{\CheckmarkBold}  & \textcolor{red}{\XSolidBrush} &\textcolor{green}{\CheckmarkBold} & \textcolor{red}{\XSolidBrush} &\textcolor{green}{\CheckmarkBold} & \textcolor{red}{\XSolidBrush} \\
ToolAlpaca~\cite{tang2023toolalpaca} & ToolAlpaca & IFT & - & 4.0K & 400 & \textcolor{red}{\XSolidBrush} &\textcolor{green}{\CheckmarkBold} &\textcolor{green}{\CheckmarkBold} & \textcolor{red}{\XSolidBrush} &\textcolor{green}{\CheckmarkBold} & \textcolor{red}{\XSolidBrush}  &\textcolor{green}{\CheckmarkBold} & \textcolor{red}{\XSolidBrush}  \\
ToolLLaMA~\cite{qin2023toolllm} & ToolBench & IFT & - & 12.7K & 16,464 & \textcolor{red}{\XSolidBrush} &\textcolor{green}{\CheckmarkBold} &\textcolor{green}{\CheckmarkBold} &\textcolor{red}{\XSolidBrush} &\textcolor{green}{\CheckmarkBold}&\textcolor{green}{\CheckmarkBold}&\textcolor{green}{\CheckmarkBold} &\textcolor{green}{\CheckmarkBold}\\
AgentEvol~\citep{xi2024agentgym} & AgentTraj-L & IFT & - & 14.5K & 24 &\textcolor{red}{\XSolidBrush} & \textcolor{green}{\CheckmarkBold} &\textcolor{green}{\CheckmarkBold}&\textcolor{red}{\XSolidBrush} &\textcolor{green}{\CheckmarkBold}&\textcolor{red}{\XSolidBrush} &\textcolor{red}{\XSolidBrush} &\textcolor{green}{\CheckmarkBold}\\
Lumos~\cite{yin2024agent} & Lumos & IFT  & - & 20.0K & 16 &\textcolor{red}{\XSolidBrush} & \textcolor{green}{\CheckmarkBold} & \textcolor{green}{\CheckmarkBold} &\textcolor{red}{\XSolidBrush} & \textcolor{green}{\CheckmarkBold} & \textcolor{green}{\CheckmarkBold} &\textcolor{red}{\XSolidBrush} & \textcolor{green}{\CheckmarkBold}\\
Agent-FLAN~\cite{chen2024agent} & Agent-FLAN & IFT & - & 24.7K & 20 &\textcolor{red}{\XSolidBrush} & \textcolor{green}{\CheckmarkBold} & \textcolor{green}{\CheckmarkBold} &\textcolor{red}{\XSolidBrush} & \textcolor{green}{\CheckmarkBold}& \textcolor{green}{\CheckmarkBold}&\textcolor{red}{\XSolidBrush} & \textcolor{green}{\CheckmarkBold}\\
AgentTuning~\citep{zeng2023agenttuning} & AgentInstruct & IFT & - & 35.0K & - &\textcolor{red}{\XSolidBrush} & \textcolor{green}{\CheckmarkBold} & \textcolor{green}{\CheckmarkBold} &\textcolor{red}{\XSolidBrush} & \textcolor{green}{\CheckmarkBold} &\textcolor{red}{\XSolidBrush} &\textcolor{red}{\XSolidBrush} & \textcolor{green}{\CheckmarkBold}\\\midrule
\multicolumn{13}{l}{\emph{Instruction Finetuning-based LLM Agents for Function Calling}} \\\midrule
NexusRaven~\citep{srinivasan2023nexusraven} & NexusRaven & IFT & - & - & 116 & \textcolor{green}{\CheckmarkBold} & \textcolor{green}{\CheckmarkBold}  & \textcolor{green}{\CheckmarkBold} &\textcolor{red}{\XSolidBrush} & \textcolor{green}{\CheckmarkBold} &\textcolor{red}{\XSolidBrush} &\textcolor{red}{\XSolidBrush}&\textcolor{red}{\XSolidBrush}\\
Gorilla~\citep{patil2023gorilla} & Gorilla & IFT & - & 16.0K & 1,645 & \textcolor{green}{\CheckmarkBold} &\textcolor{red}{\XSolidBrush} &\textcolor{red}{\XSolidBrush}&\textcolor{green}{\CheckmarkBold} &\textcolor{green}{\CheckmarkBold} &\textcolor{red}{\XSolidBrush} &\textcolor{red}{\XSolidBrush} &\textcolor{red}{\XSolidBrush}\\
OpenFunctions-v2~\citep{patil2023gorilla} & OpenFunctions-v2 & IFT & - & 65.0K & - & \textcolor{green}{\CheckmarkBold} & \textcolor{green}{\CheckmarkBold} &\textcolor{red}{\XSolidBrush} &\textcolor{green}{\CheckmarkBold} &\textcolor{green}{\CheckmarkBold} &\textcolor{red}{\XSolidBrush} &\textcolor{red}{\XSolidBrush} &\textcolor{red}{\XSolidBrush}\\
API Pack~\cite{guo2024api} & API Pack & IFT & - & 1.1M & 11,213 &\textcolor{green}{\CheckmarkBold} &\textcolor{red}{\XSolidBrush} &\textcolor{green}{\CheckmarkBold} &\textcolor{red}{\XSolidBrush} &\textcolor{green}{\CheckmarkBold} &\textcolor{red}{\XSolidBrush}&\textcolor{red}{\XSolidBrush}&\textcolor{red}{\XSolidBrush}\\ 
LAM~\citep{zhang2024agentohana} & AgentOhana & IFT & - & 42.6K & - & \textcolor{green}{\CheckmarkBold} & \textcolor{green}{\CheckmarkBold} &\textcolor{green}{\CheckmarkBold}&\textcolor{red}{\XSolidBrush} &\textcolor{green}{\CheckmarkBold}&\textcolor{red}{\XSolidBrush}&\textcolor{green}{\CheckmarkBold}&\textcolor{green}{\CheckmarkBold}\\
xLAM~\citep{liu2024apigen} & APIGen & IFT & - & 60.0K & 3,673 & \textcolor{green}{\CheckmarkBold} & \textcolor{green}{\CheckmarkBold} &\textcolor{green}{\CheckmarkBold}&\textcolor{red}{\XSolidBrush} &\textcolor{green}{\CheckmarkBold}&\textcolor{red}{\XSolidBrush}&\textcolor{green}{\CheckmarkBold}&\textcolor{green}{\CheckmarkBold}\\\midrule
\multicolumn{13}{l}{\emph{Pretraining-based LLM Agents}}  \\\midrule
% LEMUR~\citep{xu2024lemur} & PT & 90B & 300.0K & - & \textcolor{green}{\CheckmarkBold} & \textcolor{green}{\CheckmarkBold} &\textcolor{green}{\CheckmarkBold}&\textcolor{red}{\XSolidBrush} & \textcolor{red}{\XSolidBrush} &\textcolor{green}{\CheckmarkBold} &\textcolor{red}{\XSolidBrush}&\textcolor{red}{\XSolidBrush}\\
\rowcolor{teal!12} \method & \dataset & PT & 103B & 95.0K  & 76,537  & \textcolor{green}{\CheckmarkBold} & \textcolor{green}{\CheckmarkBold} & \textcolor{green}{\CheckmarkBold} & \textcolor{green}{\CheckmarkBold} & \textcolor{green}{\CheckmarkBold} & \textcolor{green}{\CheckmarkBold} & \textcolor{green}{\CheckmarkBold} & \textcolor{green}{\CheckmarkBold}\\
\bottomrule
\end{tabular}
% \begin{tablenotes}
%     \item $^*$ In addition, the StarCoder-API can offer 4.77M more APIs.
% \end{tablenotes}
\caption{Summary of existing instruction finetuning-based LLM agents for intrinsic reasoning and function calling, along with their training resources and sample sizes. "PT" and "IFT" denote "Pre-Training" and "Instruction Fine-Tuning", respectively.}
\vspace{-2ex}
\label{tab:related}
\end{threeparttable}
\end{table*}

\noindent \textbf{Prompting-based LLM Agents.} Due to the lack of agent-specific pre-training corpus, existing LLM agents rely on either prompt engineering~\cite{hsieh2023tool,lu2024chameleon,yao2022react,wang2023voyager} or instruction fine-tuning~\cite{chen2023fireact,zeng2023agenttuning} to understand human instructions, decompose high-level tasks, generate grounded plans, and execute multi-step actions. 
However, prompting-based methods mainly depend on the capabilities of backbone LLMs (usually commercial LLMs), failing to introduce new knowledge and struggling to generalize to unseen tasks~\cite{sun2024adaplanner,zhuang2023toolchain}. 

\noindent \textbf{Instruction Finetuning-based LLM Agents.} Considering the extensive diversity of APIs and the complexity of multi-tool instructions, tool learning inherently presents greater challenges than natural language tasks, such as text generation~\cite{qin2023toolllm}.
Post-training techniques focus more on instruction following and aligning output with specific formats~\cite{patil2023gorilla,hao2024toolkengpt,qin2023toolllm,schick2024toolformer}, rather than fundamentally improving model knowledge or capabilities. 
Moreover, heavy fine-tuning can hinder generalization or even degrade performance in non-agent use cases, potentially suppressing the original base model capabilities~\cite{ghosh2024a}.

\noindent \textbf{Pretraining-based LLM Agents.} While pre-training serves as an essential alternative, prior works~\cite{nijkamp2023codegen,roziere2023code,xu2024lemur,patil2023gorilla} have primarily focused on improving task-specific capabilities (\eg, code generation) instead of general-domain LLM agents, due to single-source, uni-type, small-scale, and poor-quality pre-training data. 
Existing tool documentation data for agent training either lacks diverse real-world APIs~\cite{patil2023gorilla, tang2023toolalpaca} or is constrained to single-tool or single-round tool execution. 
Furthermore, trajectory data mostly imitate expert behavior or follow function-calling rules with inferior planning and reasoning, failing to fully elicit LLMs' capabilities and handle complex instructions~\cite{qin2023toolllm}. 
Given a wide range of candidate API functions, each comprising various function names and parameters available at every planning step, identifying globally optimal solutions and generalizing across tasks remains highly challenging.



\section{Preliminaries}
\label{Preliminaries}
\begin{figure*}[t]
    \centering
    \includegraphics[width=0.95\linewidth]{fig/HealthGPT_Framework.png}
    \caption{The \ourmethod{} architecture integrates hierarchical visual perception and H-LoRA, employing a task-specific hard router to select visual features and H-LoRA plugins, ultimately generating outputs with an autoregressive manner.}
    \label{fig:architecture}
\end{figure*}
\noindent\textbf{Large Vision-Language Models.} 
The input to a LVLM typically consists of an image $x^{\text{img}}$ and a discrete text sequence $x^{\text{txt}}$. The visual encoder $\mathcal{E}^{\text{img}}$ converts the input image $x^{\text{img}}$ into a sequence of visual tokens $\mathcal{V} = [v_i]_{i=1}^{N_v}$, while the text sequence $x^{\text{txt}}$ is mapped into a sequence of text tokens $\mathcal{T} = [t_i]_{i=1}^{N_t}$ using an embedding function $\mathcal{E}^{\text{txt}}$. The LLM $\mathcal{M_\text{LLM}}(\cdot|\theta)$ models the joint probability of the token sequence $\mathcal{U} = \{\mathcal{V},\mathcal{T}\}$, which is expressed as:
\begin{equation}
    P_\theta(R | \mathcal{U}) = \prod_{i=1}^{N_r} P_\theta(r_i | \{\mathcal{U}, r_{<i}\}),
\end{equation}
where $R = [r_i]_{i=1}^{N_r}$ is the text response sequence. The LVLM iteratively generates the next token $r_i$ based on $r_{<i}$. The optimization objective is to minimize the cross-entropy loss of the response $\mathcal{R}$.
% \begin{equation}
%     \mathcal{L}_{\text{VLM}} = \mathbb{E}_{R|\mathcal{U}}\left[-\log P_\theta(R | \mathcal{U})\right]
% \end{equation}
It is worth noting that most LVLMs adopt a design paradigm based on ViT, alignment adapters, and pre-trained LLMs\cite{liu2023llava,liu2024improved}, enabling quick adaptation to downstream tasks.


\noindent\textbf{VQGAN.}
VQGAN~\cite{esser2021taming} employs latent space compression and indexing mechanisms to effectively learn a complete discrete representation of images. VQGAN first maps the input image $x^{\text{img}}$ to a latent representation $z = \mathcal{E}(x)$ through a encoder $\mathcal{E}$. Then, the latent representation is quantized using a codebook $\mathcal{Z} = \{z_k\}_{k=1}^K$, generating a discrete index sequence $\mathcal{I} = [i_m]_{m=1}^N$, where $i_m \in \mathcal{Z}$ represents the quantized code index:
\begin{equation}
    \mathcal{I} = \text{Quantize}(z|\mathcal{Z}) = \arg\min_{z_k \in \mathcal{Z}} \| z - z_k \|_2.
\end{equation}
In our approach, the discrete index sequence $\mathcal{I}$ serves as a supervisory signal for the generation task, enabling the model to predict the index sequence $\hat{\mathcal{I}}$ from input conditions such as text or other modality signals.  
Finally, the predicted index sequence $\hat{\mathcal{I}}$ is upsampled by the VQGAN decoder $G$, generating the high-quality image $\hat{x}^\text{img} = G(\hat{\mathcal{I}})$.



\noindent\textbf{Low Rank Adaptation.} 
LoRA\cite{hu2021lora} effectively captures the characteristics of downstream tasks by introducing low-rank adapters. The core idea is to decompose the bypass weight matrix $\Delta W\in\mathbb{R}^{d^{\text{in}} \times d^{\text{out}}}$ into two low-rank matrices $ \{A \in \mathbb{R}^{d^{\text{in}} \times r}, B \in \mathbb{R}^{r \times d^{\text{out}}} \}$, where $ r \ll \min\{d^{\text{in}}, d^{\text{out}}\} $, significantly reducing learnable parameters. The output with the LoRA adapter for the input $x$ is then given by:
\begin{equation}
    h = x W_0 + \alpha x \Delta W/r = x W_0 + \alpha xAB/r,
\end{equation}
where matrix $ A $ is initialized with a Gaussian distribution, while the matrix $ B $ is initialized as a zero matrix. The scaling factor $ \alpha/r $ controls the impact of $ \Delta W $ on the model.

\section{HealthGPT}
\label{Method}


\subsection{Unified Autoregressive Generation.}  
% As shown in Figure~\ref{fig:architecture}, 
\ourmethod{} (Figure~\ref{fig:architecture}) utilizes a discrete token representation that covers both text and visual outputs, unifying visual comprehension and generation as an autoregressive task. 
For comprehension, $\mathcal{M}_\text{llm}$ receives the input joint sequence $\mathcal{U}$ and outputs a series of text token $\mathcal{R} = [r_1, r_2, \dots, r_{N_r}]$, where $r_i \in \mathcal{V}_{\text{txt}}$, and $\mathcal{V}_{\text{txt}}$ represents the LLM's vocabulary:
\begin{equation}
    P_\theta(\mathcal{R} \mid \mathcal{U}) = \prod_{i=1}^{N_r} P_\theta(r_i \mid \mathcal{U}, r_{<i}).
\end{equation}
For generation, $\mathcal{M}_\text{llm}$ first receives a special start token $\langle \text{START\_IMG} \rangle$, then generates a series of tokens corresponding to the VQGAN indices $\mathcal{I} = [i_1, i_2, \dots, i_{N_i}]$, where $i_j \in \mathcal{V}_{\text{vq}}$, and $\mathcal{V}_{\text{vq}}$ represents the index range of VQGAN. Upon completion of generation, the LLM outputs an end token $\langle \text{END\_IMG} \rangle$:
\begin{equation}
    P_\theta(\mathcal{I} \mid \mathcal{U}) = \prod_{j=1}^{N_i} P_\theta(i_j \mid \mathcal{U}, i_{<j}).
\end{equation}
Finally, the generated index sequence $\mathcal{I}$ is fed into the decoder $G$, which reconstructs the target image $\hat{x}^{\text{img}} = G(\mathcal{I})$.

\subsection{Hierarchical Visual Perception}  
Given the differences in visual perception between comprehension and generation tasks—where the former focuses on abstract semantics and the latter emphasizes complete semantics—we employ ViT to compress the image into discrete visual tokens at multiple hierarchical levels.
Specifically, the image is converted into a series of features $\{f_1, f_2, \dots, f_L\}$ as it passes through $L$ ViT blocks.

To address the needs of various tasks, the hidden states are divided into two types: (i) \textit{Concrete-grained features} $\mathcal{F}^{\text{Con}} = \{f_1, f_2, \dots, f_k\}, k < L$, derived from the shallower layers of ViT, containing sufficient global features, suitable for generation tasks; 
(ii) \textit{Abstract-grained features} $\mathcal{F}^{\text{Abs}} = \{f_{k+1}, f_{k+2}, \dots, f_L\}$, derived from the deeper layers of ViT, which contain abstract semantic information closer to the text space, suitable for comprehension tasks.

The task type $T$ (comprehension or generation) determines which set of features is selected as the input for the downstream large language model:
\begin{equation}
    \mathcal{F}^{\text{img}}_T =
    \begin{cases}
        \mathcal{F}^{\text{Con}}, & \text{if } T = \text{generation task} \\
        \mathcal{F}^{\text{Abs}}, & \text{if } T = \text{comprehension task}
    \end{cases}
\end{equation}
We integrate the image features $\mathcal{F}^{\text{img}}_T$ and text features $\mathcal{T}$ into a joint sequence through simple concatenation, which is then fed into the LLM $\mathcal{M}_{\text{llm}}$ for autoregressive generation.
% :
% \begin{equation}
%     \mathcal{R} = \mathcal{M}_{\text{llm}}(\mathcal{U}|\theta), \quad \mathcal{U} = [\mathcal{F}^{\text{img}}_T; \mathcal{T}]
% \end{equation}
\subsection{Heterogeneous Knowledge Adaptation}
We devise H-LoRA, which stores heterogeneous knowledge from comprehension and generation tasks in separate modules and dynamically routes to extract task-relevant knowledge from these modules. 
At the task level, for each task type $ T $, we dynamically assign a dedicated H-LoRA submodule $ \theta^T $, which is expressed as:
\begin{equation}
    \mathcal{R} = \mathcal{M}_\text{LLM}(\mathcal{U}|\theta, \theta^T), \quad \theta^T = \{A^T, B^T, \mathcal{R}^T_\text{outer}\}.
\end{equation}
At the feature level for a single task, H-LoRA integrates the idea of Mixture of Experts (MoE)~\cite{masoudnia2014mixture} and designs an efficient matrix merging and routing weight allocation mechanism, thus avoiding the significant computational delay introduced by matrix splitting in existing MoELoRA~\cite{luo2024moelora}. Specifically, we first merge the low-rank matrices (rank = r) of $ k $ LoRA experts into a unified matrix:
\begin{equation}
    \mathbf{A}^{\text{merged}}, \mathbf{B}^{\text{merged}} = \text{Concat}(\{A_i\}_1^k), \text{Concat}(\{B_i\}_1^k),
\end{equation}
where $ \mathbf{A}^{\text{merged}} \in \mathbb{R}^{d^\text{in} \times rk} $ and $ \mathbf{B}^{\text{merged}} \in \mathbb{R}^{rk \times d^\text{out}} $. The $k$-dimension routing layer generates expert weights $ \mathcal{W} \in \mathbb{R}^{\text{token\_num} \times k} $ based on the input hidden state $ x $, and these are expanded to $ \mathbb{R}^{\text{token\_num} \times rk} $ as follows:
\begin{equation}
    \mathcal{W}^\text{expanded} = \alpha k \mathcal{W} / r \otimes \mathbf{1}_r,
\end{equation}
where $ \otimes $ denotes the replication operation.
The overall output of H-LoRA is computed as:
\begin{equation}
    \mathcal{O}^\text{H-LoRA} = (x \mathbf{A}^{\text{merged}} \odot \mathcal{W}^\text{expanded}) \mathbf{B}^{\text{merged}},
\end{equation}
where $ \odot $ represents element-wise multiplication. Finally, the output of H-LoRA is added to the frozen pre-trained weights to produce the final output:
\begin{equation}
    \mathcal{O} = x W_0 + \mathcal{O}^\text{H-LoRA}.
\end{equation}
% In summary, H-LoRA is a task-based dynamic PEFT method that achieves high efficiency in single-task fine-tuning.

\subsection{Training Pipeline}

\begin{figure}[t]
    \centering
    \hspace{-4mm}
    \includegraphics[width=0.94\linewidth]{fig/data.pdf}
    \caption{Data statistics of \texttt{VL-Health}. }
    \label{fig:data}
\end{figure}
\noindent \textbf{1st Stage: Multi-modal Alignment.} 
In the first stage, we design separate visual adapters and H-LoRA submodules for medical unified tasks. For the medical comprehension task, we train abstract-grained visual adapters using high-quality image-text pairs to align visual embeddings with textual embeddings, thereby enabling the model to accurately describe medical visual content. During this process, the pre-trained LLM and its corresponding H-LoRA submodules remain frozen. In contrast, the medical generation task requires training concrete-grained adapters and H-LoRA submodules while keeping the LLM frozen. Meanwhile, we extend the textual vocabulary to include multimodal tokens, enabling the support of additional VQGAN vector quantization indices. The model trains on image-VQ pairs, endowing the pre-trained LLM with the capability for image reconstruction. This design ensures pixel-level consistency of pre- and post-LVLM. The processes establish the initial alignment between the LLM’s outputs and the visual inputs.

\noindent \textbf{2nd Stage: Heterogeneous H-LoRA Plugin Adaptation.}  
The submodules of H-LoRA share the word embedding layer and output head but may encounter issues such as bias and scale inconsistencies during training across different tasks. To ensure that the multiple H-LoRA plugins seamlessly interface with the LLMs and form a unified base, we fine-tune the word embedding layer and output head using a small amount of mixed data to maintain consistency in the model weights. Specifically, during this stage, all H-LoRA submodules for different tasks are kept frozen, with only the word embedding layer and output head being optimized. Through this stage, the model accumulates foundational knowledge for unified tasks by adapting H-LoRA plugins.

\begin{table*}[!t]
\centering
\caption{Comparison of \ourmethod{} with other LVLMs and unified multi-modal models on medical visual comprehension tasks. \textbf{Bold} and \underline{underlined} text indicates the best performance and second-best performance, respectively.}
\resizebox{\textwidth}{!}{
\begin{tabular}{c|lcc|cccccccc|c}
\toprule
\rowcolor[HTML]{E9F3FE} &  &  &  & \multicolumn{2}{c}{\textbf{VQA-RAD \textuparrow}} & \multicolumn{2}{c}{\textbf{SLAKE \textuparrow}} & \multicolumn{2}{c}{\textbf{PathVQA \textuparrow}} &  &  &  \\ 
\cline{5-10}
\rowcolor[HTML]{E9F3FE}\multirow{-2}{*}{\textbf{Type}} & \multirow{-2}{*}{\textbf{Model}} & \multirow{-2}{*}{\textbf{\# Params}} & \multirow{-2}{*}{\makecell{\textbf{Medical} \\ \textbf{LVLM}}} & \textbf{close} & \textbf{all} & \textbf{close} & \textbf{all} & \textbf{close} & \textbf{all} & \multirow{-2}{*}{\makecell{\textbf{MMMU} \\ \textbf{-Med}}\textuparrow} & \multirow{-2}{*}{\textbf{OMVQA}\textuparrow} & \multirow{-2}{*}{\textbf{Avg. \textuparrow}} \\ 
\midrule \midrule
\multirow{9}{*}{\textbf{Comp. Only}} 
& Med-Flamingo & 8.3B & \Large \ding{51} & 58.6 & 43.0 & 47.0 & 25.5 & 61.9 & 31.3 & 28.7 & 34.9 & 41.4 \\
& LLaVA-Med & 7B & \Large \ding{51} & 60.2 & 48.1 & 58.4 & 44.8 & 62.3 & 35.7 & 30.0 & 41.3 & 47.6 \\
& HuatuoGPT-Vision & 7B & \Large \ding{51} & 66.9 & 53.0 & 59.8 & 49.1 & 52.9 & 32.0 & 42.0 & 50.0 & 50.7 \\
& BLIP-2 & 6.7B & \Large \ding{55} & 43.4 & 36.8 & 41.6 & 35.3 & 48.5 & 28.8 & 27.3 & 26.9 & 36.1 \\
& LLaVA-v1.5 & 7B & \Large \ding{55} & 51.8 & 42.8 & 37.1 & 37.7 & 53.5 & 31.4 & 32.7 & 44.7 & 41.5 \\
& InstructBLIP & 7B & \Large \ding{55} & 61.0 & 44.8 & 66.8 & 43.3 & 56.0 & 32.3 & 25.3 & 29.0 & 44.8 \\
& Yi-VL & 6B & \Large \ding{55} & 52.6 & 42.1 & 52.4 & 38.4 & 54.9 & 30.9 & 38.0 & 50.2 & 44.9 \\
& InternVL2 & 8B & \Large \ding{55} & 64.9 & 49.0 & 66.6 & 50.1 & 60.0 & 31.9 & \underline{43.3} & 54.5 & 52.5\\
& Llama-3.2 & 11B & \Large \ding{55} & 68.9 & 45.5 & 72.4 & 52.1 & 62.8 & 33.6 & 39.3 & 63.2 & 54.7 \\
\midrule
\multirow{5}{*}{\textbf{Comp. \& Gen.}} 
& Show-o & 1.3B & \Large \ding{55} & 50.6 & 33.9 & 31.5 & 17.9 & 52.9 & 28.2 & 22.7 & 45.7 & 42.6 \\
& Unified-IO 2 & 7B & \Large \ding{55} & 46.2 & 32.6 & 35.9 & 21.9 & 52.5 & 27.0 & 25.3 & 33.0 & 33.8 \\
& Janus & 1.3B & \Large \ding{55} & 70.9 & 52.8 & 34.7 & 26.9 & 51.9 & 27.9 & 30.0 & 26.8 & 33.5 \\
& \cellcolor[HTML]{DAE0FB}HealthGPT-M3 & \cellcolor[HTML]{DAE0FB}3.8B & \cellcolor[HTML]{DAE0FB}\Large \ding{51} & \cellcolor[HTML]{DAE0FB}\underline{73.7} & \cellcolor[HTML]{DAE0FB}\underline{55.9} & \cellcolor[HTML]{DAE0FB}\underline{74.6} & \cellcolor[HTML]{DAE0FB}\underline{56.4} & \cellcolor[HTML]{DAE0FB}\underline{78.7} & \cellcolor[HTML]{DAE0FB}\underline{39.7} & \cellcolor[HTML]{DAE0FB}\underline{43.3} & \cellcolor[HTML]{DAE0FB}\underline{68.5} & \cellcolor[HTML]{DAE0FB}\underline{61.3} \\
& \cellcolor[HTML]{DAE0FB}HealthGPT-L14 & \cellcolor[HTML]{DAE0FB}14B & \cellcolor[HTML]{DAE0FB}\Large \ding{51} & \cellcolor[HTML]{DAE0FB}\textbf{77.7} & \cellcolor[HTML]{DAE0FB}\textbf{58.3} & \cellcolor[HTML]{DAE0FB}\textbf{76.4} & \cellcolor[HTML]{DAE0FB}\textbf{64.5} & \cellcolor[HTML]{DAE0FB}\textbf{85.9} & \cellcolor[HTML]{DAE0FB}\textbf{44.4} & \cellcolor[HTML]{DAE0FB}\textbf{49.2} & \cellcolor[HTML]{DAE0FB}\textbf{74.4} & \cellcolor[HTML]{DAE0FB}\textbf{66.4} \\
\bottomrule
\end{tabular}
}
\label{tab:results}
\end{table*}
\begin{table*}[ht]
    \centering
    \caption{The experimental results for the four modality conversion tasks.}
    \resizebox{\textwidth}{!}{
    \begin{tabular}{l|ccc|ccc|ccc|ccc}
        \toprule
        \rowcolor[HTML]{E9F3FE} & \multicolumn{3}{c}{\textbf{CT to MRI (Brain)}} & \multicolumn{3}{c}{\textbf{CT to MRI (Pelvis)}} & \multicolumn{3}{c}{\textbf{MRI to CT (Brain)}} & \multicolumn{3}{c}{\textbf{MRI to CT (Pelvis)}} \\
        \cline{2-13}
        \rowcolor[HTML]{E9F3FE}\multirow{-2}{*}{\textbf{Model}}& \textbf{SSIM $\uparrow$} & \textbf{PSNR $\uparrow$} & \textbf{MSE $\downarrow$} & \textbf{SSIM $\uparrow$} & \textbf{PSNR $\uparrow$} & \textbf{MSE $\downarrow$} & \textbf{SSIM $\uparrow$} & \textbf{PSNR $\uparrow$} & \textbf{MSE $\downarrow$} & \textbf{SSIM $\uparrow$} & \textbf{PSNR $\uparrow$} & \textbf{MSE $\downarrow$} \\
        \midrule \midrule
        pix2pix & 71.09 & 32.65 & 36.85 & 59.17 & 31.02 & 51.91 & 78.79 & 33.85 & 28.33 & 72.31 & 32.98 & 36.19 \\
        CycleGAN & 54.76 & 32.23 & 40.56 & 54.54 & 30.77 & 55.00 & 63.75 & 31.02 & 52.78 & 50.54 & 29.89 & 67.78 \\
        BBDM & {71.69} & {32.91} & {34.44} & 57.37 & 31.37 & 48.06 & \textbf{86.40} & 34.12 & 26.61 & {79.26} & 33.15 & 33.60 \\
        Vmanba & 69.54 & 32.67 & 36.42 & {63.01} & {31.47} & {46.99} & 79.63 & 34.12 & 26.49 & 77.45 & 33.53 & 31.85 \\
        DiffMa & 71.47 & 32.74 & 35.77 & 62.56 & 31.43 & 47.38 & 79.00 & {34.13} & {26.45} & 78.53 & {33.68} & {30.51} \\
        \rowcolor[HTML]{DAE0FB}HealthGPT-M3 & \underline{79.38} & \underline{33.03} & \underline{33.48} & \underline{71.81} & \underline{31.83} & \underline{43.45} & {85.06} & \textbf{34.40} & \textbf{25.49} & \underline{84.23} & \textbf{34.29} & \textbf{27.99} \\
        \rowcolor[HTML]{DAE0FB}HealthGPT-L14 & \textbf{79.73} & \textbf{33.10} & \textbf{32.96} & \textbf{71.92} & \textbf{31.87} & \textbf{43.09} & \underline{85.31} & \underline{34.29} & \underline{26.20} & \textbf{84.96} & \underline{34.14} & \underline{28.13} \\
        \bottomrule
    \end{tabular}
    }
    \label{tab:conversion}
\end{table*}

\noindent \textbf{3rd Stage: Visual Instruction Fine-Tuning.}  
In the third stage, we introduce additional task-specific data to further optimize the model and enhance its adaptability to downstream tasks such as medical visual comprehension (e.g., medical QA, medical dialogues, and report generation) or generation tasks (e.g., super-resolution, denoising, and modality conversion). Notably, by this stage, the word embedding layer and output head have been fine-tuned, only the H-LoRA modules and adapter modules need to be trained. This strategy significantly improves the model's adaptability and flexibility across different tasks.


\section{Experiment}
\label{s:experiment}

\subsection{Data Description}
We evaluate our method on FI~\cite{you2016building}, Twitter\_LDL~\cite{yang2017learning} and Artphoto~\cite{machajdik2010affective}.
FI is a public dataset built from Flickr and Instagram, with 23,308 images and eight emotion categories, namely \textit{amusement}, \textit{anger}, \textit{awe},  \textit{contentment}, \textit{disgust}, \textit{excitement},  \textit{fear}, and \textit{sadness}. 
% Since images in FI are all copyrighted by law, some images are corrupted now, so we remove these samples and retain 21,828 images.
% T4SA contains images from Twitter, which are classified into three categories: \textit{positive}, \textit{neutral}, and \textit{negative}. In this paper, we adopt the base version of B-T4SA, which contains 470,586 images and provides text descriptions of the corresponding tweets.
Twitter\_LDL contains 10,045 images from Twitter, with the same eight categories as the FI dataset.
% 。
For these two datasets, they are randomly split into 80\%
training and 20\% testing set.
Artphoto contains 806 artistic photos from the DeviantArt website, which we use to further evaluate the zero-shot capability of our model.
% on the small-scale dataset.
% We construct and publicly release the first image sentiment analysis dataset containing metadata.
% 。

% Based on these datasets, we are the first to construct and publicly release metadata-enhanced image sentiment analysis datasets. These datasets include scenes, tags, descriptions, and corresponding confidence scores, and are available at this link for future research purposes.


% 
\begin{table}[t]
\centering
% \begin{center}
\caption{Overall performance of different models on FI and Twitter\_LDL datasets.}
\label{tab:cap1}
% \resizebox{\linewidth}{!}
{
\begin{tabular}{l|c|c|c|c}
\hline
\multirow{2}{*}{\textbf{Model}} & \multicolumn{2}{c|}{\textbf{FI}}  & \multicolumn{2}{c}{\textbf{Twitter\_LDL}} \\ \cline{2-5} 
  & \textbf{Accuracy} & \textbf{F1} & \textbf{Accuracy} & \textbf{F1}  \\ \hline
% (\rownumber)~AlexNet~\cite{krizhevsky2017imagenet}  & 58.13\% & 56.35\%  & 56.24\%& 55.02\%  \\ 
% (\rownumber)~VGG16~\cite{simonyan2014very}  & 63.75\%& 63.08\%  & 59.34\%& 59.02\%  \\ 
(\rownumber)~ResNet101~\cite{he2016deep} & 66.16\%& 65.56\%  & 62.02\% & 61.34\%  \\ 
(\rownumber)~CDA~\cite{han2023boosting} & 66.71\%& 65.37\%  & 64.14\% & 62.85\%  \\ 
(\rownumber)~CECCN~\cite{ruan2024color} & 67.96\%& 66.74\%  & 64.59\%& 64.72\% \\ 
(\rownumber)~EmoVIT~\cite{xie2024emovit} & 68.09\%& 67.45\%  & 63.12\% & 61.97\%  \\ 
(\rownumber)~ComLDL~\cite{zhang2022compound} & 68.83\%& 67.28\%  & 65.29\% & 63.12\%  \\ 
(\rownumber)~WSDEN~\cite{li2023weakly} & 69.78\%& 69.61\%  & 67.04\% & 65.49\% \\ 
(\rownumber)~ECWA~\cite{deng2021emotion} & 70.87\%& 69.08\%  & 67.81\% & 66.87\%  \\ 
(\rownumber)~EECon~\cite{yang2023exploiting} & 71.13\%& 68.34\%  & 64.27\%& 63.16\%  \\ 
(\rownumber)~MAM~\cite{zhang2024affective} & 71.44\%  & 70.83\% & 67.18\%  & 65.01\%\\ 
(\rownumber)~TGCA-PVT~\cite{chen2024tgca}   & 73.05\%  & 71.46\% & 69.87\%  & 68.32\% \\ 
(\rownumber)~OEAN~\cite{zhang2024object}   & 73.40\%  & 72.63\% & 70.52\%  & 69.47\% \\ \hline
(\rownumber)~\shortname  & \textbf{79.48\%} & \textbf{79.22\%} & \textbf{74.12\%} & \textbf{73.09\%} \\ \hline
\end{tabular}
}
\vspace{-6mm}
% \end{center}
\end{table}
% 

\subsection{Experiment Setting}
% \subsubsection{Model Setting.}
% 
\textbf{Model Setting:}
For feature representation, we set $k=10$ to select object tags, and adopt clip-vit-base-patch32 as the pre-trained model for unified feature representation.
Moreover, we empirically set $(d_e, d_h, d_k, d_s) = (512, 128, 16, 64)$, and set the classification class $L$ to 8.

% 

\textbf{Training Setting:}
To initialize the model, we set all weights such as $\boldsymbol{W}$ following the truncated normal distribution, and use AdamW optimizer with the learning rate of $1 \times 10^{-4}$.
% warmup scheduler of cosine, warmup steps of 2000.
Furthermore, we set the batch size to 32 and the epoch of the training process to 200.
During the implementation, we utilize \textit{PyTorch} to build our entire model.
% , and our project codes are publicly available at https://github.com/zzmyrep/MESN.
% Our project codes as well as data are all publicly available on GitHub\footnote{https://github.com/zzmyrep/KBCEN}.
% Code is available at \href{https://github.com/zzmyrep/KBCEN}{https://github.com/zzmyrep/KBCEN}.

\textbf{Evaluation Metrics:}
Following~\cite{zhang2024affective, chen2024tgca, zhang2024object}, we adopt \textit{accuracy} and \textit{F1} as our evaluation metrics to measure the performance of different methods for image sentiment analysis. 



\subsection{Experiment Result}
% We compare our model against the following baselines: AlexNet~\cite{krizhevsky2017imagenet}, VGG16~\cite{simonyan2014very}, ResNet101~\cite{he2016deep}, CECCN~\cite{ruan2024color}, EmoVIT~\cite{xie2024emovit}, WSCNet~\cite{yang2018weakly}, ECWA~\cite{deng2021emotion}, EECon~\cite{yang2023exploiting}, MAM~\cite{zhang2024affective} and TGCA-PVT~\cite{chen2024tgca}, and the overall results are summarized in Table~\ref{tab:cap1}.
We compare our model against several baselines, and the overall results are summarized in Table~\ref{tab:cap1}.
We observe that our model achieves the best performance in both accuracy and F1 metrics, significantly outperforming the previous models. 
This superior performance is mainly attributed to our effective utilization of metadata to enhance image sentiment analysis, as well as the exceptional capability of the unified sentiment transformer framework we developed. These results strongly demonstrate that our proposed method can bring encouraging performance for image sentiment analysis.

\setcounter{magicrownumbers}{0} 
\begin{table}[t]
\begin{center}
\caption{Ablation study of~\shortname~on FI dataset.} 
% \vspace{1mm}
\label{tab:cap2}
\resizebox{.9\linewidth}{!}
{
\begin{tabular}{lcc}
  \hline
  \textbf{Model} & \textbf{Accuracy} & \textbf{F1} \\
  \hline
  (\rownumber)~Ours (w/o vision) & 65.72\% & 64.54\% \\
  (\rownumber)~Ours (w/o text description) & 74.05\% & 72.58\% \\
  (\rownumber)~Ours (w/o object tag) & 77.45\% & 76.84\% \\
  (\rownumber)~Ours (w/o scene tag) & 78.47\% & 78.21\% \\
  \hline
  (\rownumber)~Ours (w/o unified embedding) & 76.41\% & 76.23\% \\
  (\rownumber)~Ours (w/o adaptive learning) & 76.83\% & 76.56\% \\
  (\rownumber)~Ours (w/o cross-modal fusion) & 76.85\% & 76.49\% \\
  \hline
  (\rownumber)~Ours  & \textbf{79.48\%} & \textbf{79.22\%} \\
  \hline
\end{tabular}
}
\end{center}
\vspace{-5mm}
\end{table}


\begin{figure}[t]
\centering
% \vspace{-2mm}
\includegraphics[width=0.42\textwidth]{fig/2dvisual-linux4-paper2.pdf}
\caption{Visualization of feature distribution on eight categories before (left) and after (right) model processing.}
% 
\label{fig:visualization}
\vspace{-5mm}
\end{figure}

\subsection{Ablation Performance}
In this subsection, we conduct an ablation study to examine which component is really important for performance improvement. The results are reported in Table~\ref{tab:cap2}.

For information utilization, we observe a significant decline in model performance when visual features are removed. Additionally, the performance of \shortname~decreases when different metadata are removed separately, which means that text description, object tag, and scene tag are all critical for image sentiment analysis.
Recalling the model architecture, we separately remove transformer layers of the unified representation module, the adaptive learning module, and the cross-modal fusion module, replacing them with MLPs of the same parameter scale.
In this way, we can observe varying degrees of decline in model performance, indicating that these modules are indispensable for our model to achieve better performance.

\subsection{Visualization}
% 


% % 开始使用minipage进行左右排列
% \begin{minipage}[t]{0.45\textwidth}  % 子图1宽度为45%
%     \centering
%     \includegraphics[width=\textwidth]{2dvisual.pdf}  % 插入图片
%     \captionof{figure}{Visualization of feature distribution.}  % 使用captionof添加图片标题
%     \label{fig:visualization}
% \end{minipage}


% \begin{figure}[t]
% \centering
% \vspace{-2mm}
% \includegraphics[width=0.45\textwidth]{fig/2dvisual.pdf}
% \caption{Visualization of feature distribution.}
% \label{fig:visualization}
% % \vspace{-4mm}
% \end{figure}

% \begin{figure}[t]
% \centering
% \vspace{-2mm}
% \includegraphics[width=0.45\textwidth]{fig/2dvisual-linux3-paper.pdf}
% \caption{Visualization of feature distribution.}
% \label{fig:visualization}
% % \vspace{-4mm}
% \end{figure}



\begin{figure}[tbp]   
\vspace{-4mm}
  \centering            
  \subfloat[Depth of adaptive learning layers]   
  {
    \label{fig:subfig1}\includegraphics[width=0.22\textwidth]{fig/fig_sensitivity-a5}
  }
  \subfloat[Depth of fusion layers]
  {
    % \label{fig:subfig2}\includegraphics[width=0.22\textwidth]{fig/fig_sensitivity-b2}
    \label{fig:subfig2}\includegraphics[width=0.22\textwidth]{fig/fig_sensitivity-b2-num.pdf}
  }
  \caption{Sensitivity study of \shortname~on different depth. }   
  \label{fig:fig_sensitivity}  
\vspace{-2mm}
\end{figure}

% \begin{figure}[htbp]
% \centerline{\includegraphics{2dvisual.pdf}}
% \caption{Visualization of feature distribution.}
% \label{fig:visualization}
% \end{figure}

% In Fig.~\ref{fig:visualization}, we use t-SNE~\cite{van2008visualizing} to reduce the dimension of data features for visualization, Figure in left represents the metadata features before model processing, the features are obtained by embedding through the CLIP model, and figure in right shows the features of the data after model processing, it can be observed that after the model processing, the data with different label categories fall in different regions in the space, therefore, we can conclude that the Therefore, we can conclude that the model can effectively utilize the information contained in the metadata and use it to guide the model for classification.

In Fig.~\ref{fig:visualization}, we use t-SNE~\cite{van2008visualizing} to reduce the dimension of data features for visualization.
The left figure shows metadata features before being processed by our model (\textit{i.e.}, embedded by CLIP), while the right shows the distribution of features after being processed by our model.
We can observe that after the model processing, data with the same label are closer to each other, while others are farther away.
Therefore, it shows that the model can effectively utilize the information contained in the metadata and use it to guide the classification process.

\subsection{Sensitivity Analysis}
% 
In this subsection, we conduct a sensitivity analysis to figure out the effect of different depth settings of adaptive learning layers and fusion layers. 
% In this subsection, we conduct a sensitivity analysis to figure out the effect of different depth settings on the model. 
% Fig.~\ref{fig:fig_sensitivity} presents the effect of different depth settings of adaptive learning layers and fusion layers. 
Taking Fig.~\ref{fig:fig_sensitivity} (a) as an example, the model performance improves with increasing depth, reaching the best performance at a depth of 4.
% Taking Fig.~\ref{fig:fig_sensitivity} (a) as an example, the performance of \shortname~improves with the increase of depth at first, reaching the best performance at a depth of 4.
When the depth continues to increase, the accuracy decreases to varying degrees.
Similar results can be observed in Fig.~\ref{fig:fig_sensitivity} (b).
Therefore, we set their depths to 4 and 6 respectively to achieve the best results.

% Through our experiments, we can observe that the effect of modifying these hyperparameters on the results of the experiments is very weak, and the surface model is not sensitive to the hyperparameters.


\subsection{Zero-shot Capability}
% 

% (1)~GCH~\cite{2010Analyzing} & 21.78\% & (5)~RA-DLNet~\cite{2020A} & 34.01\% \\ \hline
% (2)~WSCNet~\cite{2019WSCNet}  & 30.25\% & (6)~CECCN~\cite{ruan2024color} & 43.83\% \\ \hline
% (3)~PCNN~\cite{2015Robust} & 31.68\%  & (7)~EmoVIT~\cite{xie2024emovit} & 44.90\% \\ \hline
% (4)~AR~\cite{2018Visual} & 32.67\% & (8)~Ours (Zero-shot) & 47.83\% \\ \hline


\begin{table}[t]
\centering
\caption{Zero-shot capability of \shortname.}
\label{tab:cap3}
\resizebox{1\linewidth}{!}
{
\begin{tabular}{lc|lc}
\hline
\textbf{Model} & \textbf{Accuracy} & \textbf{Model} & \textbf{Accuracy} \\ \hline
(1)~WSCNet~\cite{2019WSCNet}  & 30.25\% & (5)~MAM~\cite{zhang2024affective} & 39.56\%  \\ \hline
(2)~AR~\cite{2018Visual} & 32.67\% & (6)~CECCN~\cite{ruan2024color} & 43.83\% \\ \hline
(3)~RA-DLNet~\cite{2020A} & 34.01\%  & (7)~EmoVIT~\cite{xie2024emovit} & 44.90\% \\ \hline
(4)~CDA~\cite{han2023boosting} & 38.64\% & (8)~Ours (Zero-shot) & 47.83\% \\ \hline
\end{tabular}
}
\vspace{-5mm}
\end{table}

% We use the model trained on the FI dataset to test on the artphoto dataset to verify the model's generalization ability as well as robustness to other distributed datasets.
% We can observe that the MESN model shows strong competitiveness in terms of accuracy when compared to other trained models, which suggests that the model has a good generalization ability in the OOD task.

To validate the model's generalization ability and robustness to other distributed datasets, we directly test the model trained on the FI dataset, without training on Artphoto. 
% As observed in Table 3, compared to other models trained on Artphoto, we achieve highly competitive zero-shot performance, indicating that the model has good generalization ability in out-of-distribution tasks.
From Table~\ref{tab:cap3}, we can observe that compared with other models trained on Artphoto, we achieve competitive zero-shot performance, which shows that the model has good generalization ability in out-of-distribution tasks.


%%%%%%%%%%%%
%  E2E     %
%%%%%%%%%%%%


\section{Conclusion}
In this paper, we introduced Wi-Chat, the first LLM-powered Wi-Fi-based human activity recognition system that integrates the reasoning capabilities of large language models with the sensing potential of wireless signals. Our experimental results on a self-collected Wi-Fi CSI dataset demonstrate the promising potential of LLMs in enabling zero-shot Wi-Fi sensing. These findings suggest a new paradigm for human activity recognition that does not rely on extensive labeled data. We hope future research will build upon this direction, further exploring the applications of LLMs in signal processing domains such as IoT, mobile sensing, and radar-based systems.

\section*{Limitations}
While our work represents the first attempt to leverage LLMs for processing Wi-Fi signals, it is a preliminary study focused on a relatively simple task: Wi-Fi-based human activity recognition. This choice allows us to explore the feasibility of LLMs in wireless sensing but also comes with certain limitations.

Our approach primarily evaluates zero-shot performance, which, while promising, may still lag behind traditional supervised learning methods in highly complex or fine-grained recognition tasks. Besides, our study is limited to a controlled environment with a self-collected dataset, and the generalizability of LLMs to diverse real-world scenarios with varying Wi-Fi conditions, environmental interference, and device heterogeneity remains an open question.

Additionally, we have yet to explore the full potential of LLMs in more advanced Wi-Fi sensing applications, such as fine-grained gesture recognition, occupancy detection, and passive health monitoring. Future work should investigate the scalability of LLM-based approaches, their robustness to domain shifts, and their integration with multimodal sensing techniques in broader IoT applications.


% Bibliography entries for the entire Anthology, followed by custom entries
%\bibliography{anthology,custom}
% Custom bibliography entries only
\bibliography{main}
\newpage
\appendix

\section{Experiment prompts}
\label{sec:prompt}
The prompts used in the LLM experiments are shown in the following Table~\ref{tab:prompts}.

\definecolor{titlecolor}{rgb}{0.9, 0.5, 0.1}
\definecolor{anscolor}{rgb}{0.2, 0.5, 0.8}
\definecolor{labelcolor}{HTML}{48a07e}
\begin{table*}[h]
	\centering
	
 % \vspace{-0.2cm}
	
	\begin{center}
		\begin{tikzpicture}[
				chatbox_inner/.style={rectangle, rounded corners, opacity=0, text opacity=1, font=\sffamily\scriptsize, text width=5in, text height=9pt, inner xsep=6pt, inner ysep=6pt},
				chatbox_prompt_inner/.style={chatbox_inner, align=flush left, xshift=0pt, text height=11pt},
				chatbox_user_inner/.style={chatbox_inner, align=flush left, xshift=0pt},
				chatbox_gpt_inner/.style={chatbox_inner, align=flush left, xshift=0pt},
				chatbox/.style={chatbox_inner, draw=black!25, fill=gray!7, opacity=1, text opacity=0},
				chatbox_prompt/.style={chatbox, align=flush left, fill=gray!1.5, draw=black!30, text height=10pt},
				chatbox_user/.style={chatbox, align=flush left},
				chatbox_gpt/.style={chatbox, align=flush left},
				chatbox2/.style={chatbox_gpt, fill=green!25},
				chatbox3/.style={chatbox_gpt, fill=red!20, draw=black!20},
				chatbox4/.style={chatbox_gpt, fill=yellow!30},
				labelbox/.style={rectangle, rounded corners, draw=black!50, font=\sffamily\scriptsize\bfseries, fill=gray!5, inner sep=3pt},
			]
											
			\node[chatbox_user] (q1) {
				\textbf{System prompt}
				\newline
				\newline
				You are a helpful and precise assistant for segmenting and labeling sentences. We would like to request your help on curating a dataset for entity-level hallucination detection.
				\newline \newline
                We will give you a machine generated biography and a list of checked facts about the biography. Each fact consists of a sentence and a label (True/False). Please do the following process. First, breaking down the biography into words. Second, by referring to the provided list of facts, merging some broken down words in the previous step to form meaningful entities. For example, ``strategic thinking'' should be one entity instead of two. Third, according to the labels in the list of facts, labeling each entity as True or False. Specifically, for facts that share a similar sentence structure (\eg, \textit{``He was born on Mach 9, 1941.''} (\texttt{True}) and \textit{``He was born in Ramos Mejia.''} (\texttt{False})), please first assign labels to entities that differ across atomic facts. For example, first labeling ``Mach 9, 1941'' (\texttt{True}) and ``Ramos Mejia'' (\texttt{False}) in the above case. For those entities that are the same across atomic facts (\eg, ``was born'') or are neutral (\eg, ``he,'' ``in,'' and ``on''), please label them as \texttt{True}. For the cases that there is no atomic fact that shares the same sentence structure, please identify the most informative entities in the sentence and label them with the same label as the atomic fact while treating the rest of the entities as \texttt{True}. In the end, output the entities and labels in the following format:
                \begin{itemize}[nosep]
                    \item Entity 1 (Label 1)
                    \item Entity 2 (Label 2)
                    \item ...
                    \item Entity N (Label N)
                \end{itemize}
                % \newline \newline
                Here are two examples:
                \newline\newline
                \textbf{[Example 1]}
                \newline
                [The start of the biography]
                \newline
                \textcolor{titlecolor}{Marianne McAndrew is an American actress and singer, born on November 21, 1942, in Cleveland, Ohio. She began her acting career in the late 1960s, appearing in various television shows and films.}
                \newline
                [The end of the biography]
                \newline \newline
                [The start of the list of checked facts]
                \newline
                \textcolor{anscolor}{[Marianne McAndrew is an American. (False); Marianne McAndrew is an actress. (True); Marianne McAndrew is a singer. (False); Marianne McAndrew was born on November 21, 1942. (False); Marianne McAndrew was born in Cleveland, Ohio. (False); She began her acting career in the late 1960s. (True); She has appeared in various television shows. (True); She has appeared in various films. (True)]}
                \newline
                [The end of the list of checked facts]
                \newline \newline
                [The start of the ideal output]
                \newline
                \textcolor{labelcolor}{[Marianne McAndrew (True); is (True); an (True); American (False); actress (True); and (True); singer (False); , (True); born (True); on (True); November 21, 1942 (False); , (True); in (True); Cleveland, Ohio (False); . (True); She (True); began (True); her (True); acting career (True); in (True); the late 1960s (True); , (True); appearing (True); in (True); various (True); television shows (True); and (True); films (True); . (True)]}
                \newline
                [The end of the ideal output]
				\newline \newline
                \textbf{[Example 2]}
                \newline
                [The start of the biography]
                \newline
                \textcolor{titlecolor}{Doug Sheehan is an American actor who was born on April 27, 1949, in Santa Monica, California. He is best known for his roles in soap operas, including his portrayal of Joe Kelly on ``General Hospital'' and Ben Gibson on ``Knots Landing.''}
                \newline
                [The end of the biography]
                \newline \newline
                [The start of the list of checked facts]
                \newline
                \textcolor{anscolor}{[Doug Sheehan is an American. (True); Doug Sheehan is an actor. (True); Doug Sheehan was born on April 27, 1949. (True); Doug Sheehan was born in Santa Monica, California. (False); He is best known for his roles in soap operas. (True); He portrayed Joe Kelly. (True); Joe Kelly was in General Hospital. (True); General Hospital is a soap opera. (True); He portrayed Ben Gibson. (True); Ben Gibson was in Knots Landing. (True); Knots Landing is a soap opera. (True)]}
                \newline
                [The end of the list of checked facts]
                \newline \newline
                [The start of the ideal output]
                \newline
                \textcolor{labelcolor}{[Doug Sheehan (True); is (True); an (True); American (True); actor (True); who (True); was born (True); on (True); April 27, 1949 (True); in (True); Santa Monica, California (False); . (True); He (True); is (True); best known (True); for (True); his roles in soap operas (True); , (True); including (True); in (True); his portrayal (True); of (True); Joe Kelly (True); on (True); ``General Hospital'' (True); and (True); Ben Gibson (True); on (True); ``Knots Landing.'' (True)]}
                \newline
                [The end of the ideal output]
				\newline \newline
				\textbf{User prompt}
				\newline
				\newline
				[The start of the biography]
				\newline
				\textcolor{magenta}{\texttt{\{BIOGRAPHY\}}}
				\newline
				[The ebd of the biography]
				\newline \newline
				[The start of the list of checked facts]
				\newline
				\textcolor{magenta}{\texttt{\{LIST OF CHECKED FACTS\}}}
				\newline
				[The end of the list of checked facts]
			};
			\node[chatbox_user_inner] (q1_text) at (q1) {
				\textbf{System prompt}
				\newline
				\newline
				You are a helpful and precise assistant for segmenting and labeling sentences. We would like to request your help on curating a dataset for entity-level hallucination detection.
				\newline \newline
                We will give you a machine generated biography and a list of checked facts about the biography. Each fact consists of a sentence and a label (True/False). Please do the following process. First, breaking down the biography into words. Second, by referring to the provided list of facts, merging some broken down words in the previous step to form meaningful entities. For example, ``strategic thinking'' should be one entity instead of two. Third, according to the labels in the list of facts, labeling each entity as True or False. Specifically, for facts that share a similar sentence structure (\eg, \textit{``He was born on Mach 9, 1941.''} (\texttt{True}) and \textit{``He was born in Ramos Mejia.''} (\texttt{False})), please first assign labels to entities that differ across atomic facts. For example, first labeling ``Mach 9, 1941'' (\texttt{True}) and ``Ramos Mejia'' (\texttt{False}) in the above case. For those entities that are the same across atomic facts (\eg, ``was born'') or are neutral (\eg, ``he,'' ``in,'' and ``on''), please label them as \texttt{True}. For the cases that there is no atomic fact that shares the same sentence structure, please identify the most informative entities in the sentence and label them with the same label as the atomic fact while treating the rest of the entities as \texttt{True}. In the end, output the entities and labels in the following format:
                \begin{itemize}[nosep]
                    \item Entity 1 (Label 1)
                    \item Entity 2 (Label 2)
                    \item ...
                    \item Entity N (Label N)
                \end{itemize}
                % \newline \newline
                Here are two examples:
                \newline\newline
                \textbf{[Example 1]}
                \newline
                [The start of the biography]
                \newline
                \textcolor{titlecolor}{Marianne McAndrew is an American actress and singer, born on November 21, 1942, in Cleveland, Ohio. She began her acting career in the late 1960s, appearing in various television shows and films.}
                \newline
                [The end of the biography]
                \newline \newline
                [The start of the list of checked facts]
                \newline
                \textcolor{anscolor}{[Marianne McAndrew is an American. (False); Marianne McAndrew is an actress. (True); Marianne McAndrew is a singer. (False); Marianne McAndrew was born on November 21, 1942. (False); Marianne McAndrew was born in Cleveland, Ohio. (False); She began her acting career in the late 1960s. (True); She has appeared in various television shows. (True); She has appeared in various films. (True)]}
                \newline
                [The end of the list of checked facts]
                \newline \newline
                [The start of the ideal output]
                \newline
                \textcolor{labelcolor}{[Marianne McAndrew (True); is (True); an (True); American (False); actress (True); and (True); singer (False); , (True); born (True); on (True); November 21, 1942 (False); , (True); in (True); Cleveland, Ohio (False); . (True); She (True); began (True); her (True); acting career (True); in (True); the late 1960s (True); , (True); appearing (True); in (True); various (True); television shows (True); and (True); films (True); . (True)]}
                \newline
                [The end of the ideal output]
				\newline \newline
                \textbf{[Example 2]}
                \newline
                [The start of the biography]
                \newline
                \textcolor{titlecolor}{Doug Sheehan is an American actor who was born on April 27, 1949, in Santa Monica, California. He is best known for his roles in soap operas, including his portrayal of Joe Kelly on ``General Hospital'' and Ben Gibson on ``Knots Landing.''}
                \newline
                [The end of the biography]
                \newline \newline
                [The start of the list of checked facts]
                \newline
                \textcolor{anscolor}{[Doug Sheehan is an American. (True); Doug Sheehan is an actor. (True); Doug Sheehan was born on April 27, 1949. (True); Doug Sheehan was born in Santa Monica, California. (False); He is best known for his roles in soap operas. (True); He portrayed Joe Kelly. (True); Joe Kelly was in General Hospital. (True); General Hospital is a soap opera. (True); He portrayed Ben Gibson. (True); Ben Gibson was in Knots Landing. (True); Knots Landing is a soap opera. (True)]}
                \newline
                [The end of the list of checked facts]
                \newline \newline
                [The start of the ideal output]
                \newline
                \textcolor{labelcolor}{[Doug Sheehan (True); is (True); an (True); American (True); actor (True); who (True); was born (True); on (True); April 27, 1949 (True); in (True); Santa Monica, California (False); . (True); He (True); is (True); best known (True); for (True); his roles in soap operas (True); , (True); including (True); in (True); his portrayal (True); of (True); Joe Kelly (True); on (True); ``General Hospital'' (True); and (True); Ben Gibson (True); on (True); ``Knots Landing.'' (True)]}
                \newline
                [The end of the ideal output]
				\newline \newline
				\textbf{User prompt}
				\newline
				\newline
				[The start of the biography]
				\newline
				\textcolor{magenta}{\texttt{\{BIOGRAPHY\}}}
				\newline
				[The ebd of the biography]
				\newline \newline
				[The start of the list of checked facts]
				\newline
				\textcolor{magenta}{\texttt{\{LIST OF CHECKED FACTS\}}}
				\newline
				[The end of the list of checked facts]
			};
		\end{tikzpicture}
        \caption{GPT-4o prompt for labeling hallucinated entities.}\label{tb:gpt-4-prompt}
	\end{center}
\vspace{-0cm}
\end{table*}
% \section{Full Experiment Results}
% \begin{table*}[th]
    \centering
    \small
    \caption{Classification Results}
    \begin{tabular}{lcccc}
        \toprule
        \textbf{Method} & \textbf{Accuracy} & \textbf{Precision} & \textbf{Recall} & \textbf{F1-score} \\
        \midrule
        \multicolumn{5}{c}{\textbf{Zero Shot}} \\
                Zero-shot E-eyes & 0.26 & 0.26 & 0.27 & 0.26 \\
        Zero-shot CARM & 0.24 & 0.24 & 0.24 & 0.24 \\
                Zero-shot SVM & 0.27 & 0.28 & 0.28 & 0.27 \\
        Zero-shot CNN & 0.23 & 0.24 & 0.23 & 0.23 \\
        Zero-shot RNN & 0.26 & 0.26 & 0.26 & 0.26 \\
DeepSeek-0shot & 0.54 & 0.61 & 0.54 & 0.52 \\
DeepSeek-0shot-COT & 0.33 & 0.24 & 0.33 & 0.23 \\
DeepSeek-0shot-Knowledge & 0.45 & 0.46 & 0.45 & 0.44 \\
Gemma2-0shot & 0.35 & 0.22 & 0.38 & 0.27 \\
Gemma2-0shot-COT & 0.36 & 0.22 & 0.36 & 0.27 \\
Gemma2-0shot-Knowledge & 0.32 & 0.18 & 0.34 & 0.20 \\
GPT-4o-mini-0shot & 0.48 & 0.53 & 0.48 & 0.41 \\
GPT-4o-mini-0shot-COT & 0.33 & 0.50 & 0.33 & 0.38 \\
GPT-4o-mini-0shot-Knowledge & 0.49 & 0.31 & 0.49 & 0.36 \\
GPT-4o-0shot & 0.62 & 0.62 & 0.47 & 0.42 \\
GPT-4o-0shot-COT & 0.29 & 0.45 & 0.29 & 0.21 \\
GPT-4o-0shot-Knowledge & 0.44 & 0.52 & 0.44 & 0.39 \\
LLaMA-0shot & 0.32 & 0.25 & 0.32 & 0.24 \\
LLaMA-0shot-COT & 0.12 & 0.25 & 0.12 & 0.09 \\
LLaMA-0shot-Knowledge & 0.32 & 0.25 & 0.32 & 0.28 \\
Mistral-0shot & 0.19 & 0.23 & 0.19 & 0.10 \\
Mistral-0shot-Knowledge & 0.21 & 0.40 & 0.21 & 0.11 \\
        \midrule
        \multicolumn{5}{c}{\textbf{4 Shot}} \\
GPT-4o-mini-4shot & 0.58 & 0.59 & 0.58 & 0.53 \\
GPT-4o-mini-4shot-COT & 0.57 & 0.53 & 0.57 & 0.50 \\
GPT-4o-mini-4shot-Knowledge & 0.56 & 0.51 & 0.56 & 0.47 \\
GPT-4o-4shot & 0.77 & 0.84 & 0.77 & 0.73 \\
GPT-4o-4shot-COT & 0.63 & 0.76 & 0.63 & 0.53 \\
GPT-4o-4shot-Knowledge & 0.72 & 0.82 & 0.71 & 0.66 \\
LLaMA-4shot & 0.29 & 0.24 & 0.29 & 0.21 \\
LLaMA-4shot-COT & 0.20 & 0.30 & 0.20 & 0.13 \\
LLaMA-4shot-Knowledge & 0.15 & 0.23 & 0.13 & 0.13 \\
Mistral-4shot & 0.02 & 0.02 & 0.02 & 0.02 \\
Mistral-4shot-Knowledge & 0.21 & 0.27 & 0.21 & 0.20 \\
        \midrule
        
        \multicolumn{5}{c}{\textbf{Suprevised}} \\
        SVM & 0.94 & 0.92 & 0.91 & 0.91 \\
        CNN & 0.98 & 0.98 & 0.97 & 0.97 \\
        RNN & 0.99 & 0.99 & 0.99 & 0.99 \\
        % \midrule
        % \multicolumn{5}{c}{\textbf{Conventional Wi-Fi-based Human Activity Recognition Systems}} \\
        E-eyes & 1.00 & 1.00 & 1.00 & 1.00 \\
        CARM & 0.98 & 0.98 & 0.98 & 0.98 \\
\midrule
 \multicolumn{5}{c}{\textbf{Vision Models}} \\
           Zero-shot SVM & 0.26 & 0.25 & 0.25 & 0.25 \\
        Zero-shot CNN & 0.26 & 0.25 & 0.26 & 0.26 \\
        Zero-shot RNN & 0.28 & 0.28 & 0.29 & 0.28 \\
        SVM & 0.99 & 0.99 & 0.99 & 0.99 \\
        CNN & 0.98 & 0.99 & 0.98 & 0.98 \\
        RNN & 0.98 & 0.99 & 0.98 & 0.98 \\
GPT-4o-mini-Vision & 0.84 & 0.85 & 0.84 & 0.84 \\
GPT-4o-mini-Vision-COT & 0.90 & 0.91 & 0.90 & 0.90 \\
GPT-4o-Vision & 0.74 & 0.82 & 0.74 & 0.73 \\
GPT-4o-Vision-COT & 0.70 & 0.83 & 0.70 & 0.68 \\
LLaMA-Vision & 0.20 & 0.23 & 0.20 & 0.09 \\
LLaMA-Vision-Knowledge & 0.22 & 0.05 & 0.22 & 0.08 \\

        \bottomrule
    \end{tabular}
    \label{full}
\end{table*}




\end{document}


%\appendix


\end{document}