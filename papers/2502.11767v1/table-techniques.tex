
\newcommand\TE{\rule{0pt}{2.0ex}}
\newcommand\BE{\rule[-1.1ex]{0pt}{0pt}}

{
\newcolumntype{C}{ >{\centering\arraybackslash} m{4cm} } 
\providecommand{\rotateDeg}{90}
\setlength{\tabcolsep}{1.8pt}
\providecommand{\rotDeg}{70}
\definecolor{verylightgreennew}{RGB}	{220,255,220}
\definecolor{verylightrednew}{RGB}		{255, 230, 230}
\definecolor{verylightreddarker}{HTML} {FFCBCB} 
\definecolor{verylightrednew}{RGB}		{255, 230, 230}
\definecolor{verylightrednewlighter}{RGB}		{255, 229, 239}
\definecolor{lightgraynew}{rgb}{0.95,0.95,0.95}
\definecolor{newgray}{RGB}{0.3,0.3,0.3}
\providecommand{\cellsz}{0.40cm} 
\providecommand{\cellszlg}{0.50cm} 
\providecommand{\cellszsm}{0.40cm} 
\renewcommand{\cm}{{\color{greencm}\normalsize\cmark}}
\renewcommand{\cmgray}{{\color{lightgraynew}\normalsize\cmark}}
\renewcommand{\xm}{{\color{verylightreddarker}\normalsize\xmark}}
\newcommand\BBBBB{\rule[1.6ex]{0pt}{1.6ex}}
\newcommand\BBBnew{\rule[-2.5ex]{0pt}{0pt}} 
\newcommand\BBBBBB{\rule[-1.1ex]{0pt}{0pt}} 
\newcommand{\sysName}[1]{{\sf
\BBBBBB
#1
}}
\providecommand{\cellsomewhat}{
\BBBBB
\cmgray
\cellcolor{verylightgreennew}
}
\providecommand{\cellno}{
\BBBBB
\xm
\cellcolor{verylightrednew}}
\providecommand{\cellyes}{
\BBBBB
\cm
\cellcolor{verylightgreennew}
}


\newcolumntype{H}{>{\setbox0=\hbox\bgroup}c<{\egroup}@{}} 




\begin{table*}[t!]
\vspace{3mm}
\centering
\def\arraystretch{1.2} 
\scriptsize
% \footnotesize
% \small
\begin{minipage}{1.0\linewidth}
% \columnwidth}
{\begin{center}
\begin{tabular}
{P{2mm}
l
c 
!{\vrule width 0.8pt} 
% QUERYING
P{\cellszlg} P{\cellszlg} P{\cellszlg} 
P{\cellszlg} 
% @{}
!{\vrule width 0.6pt}
% ANNOTATION
P{\cellszlg} 
P{\cellszlg} 
P{\cellszlg}  
% @{}
!{\vrule width 0.6pt}
% APPLICATIONS
P{\cellszsm} 
P{\cellszsm} 
P{\cellszsm} 
P{\cellszsm} 
P{\cellszsm} 
P{\cellszsm} 
P{\cellszsm} 
P{\cellszsm} 
P{\cellszsm}
% P{\cellszsm} 
!{\vrule width 0.8pt}
@{}
}

& & 
& \multicolumn{4}{c!{\vrule width 0.6pt}}{\textcolor{googlegreen}{\textsc{\bfseries\scshape Querying}} 
% (\textbf{\S\ref{sec:querying}})
}
& \multicolumn{3}{c!{\vrule width 0.6pt}}{\textcolor{googleblue}{
\textsc{\bfseries\scshape Annotation}}
% (\textbf{\S\ref{sec:ann}})
% \textbf{\makecell{Annotation\\\textcolor{black}{(\textbf{Section~\ref{sec:ann}})}}}
}
& \multicolumn{9}{c!{\vrule width 0.6pt}}{\textcolor{googlered}{\textsc{\bfseries\scshape Applications}}} 
\\

& & 
\BBBnew
& \multicolumn{4}{c!{\vrule width 0.6pt}}{(\textbf{Section~\ref{sec:querying}})}
\BBBnew
& \multicolumn{3}{c!{\vrule width 0.6pt}}{(\textbf{Section~\ref{sec:ann}})}
& \multicolumn{9}{c!{\vrule width 0.6pt}}{(\textbf{Section~\ref{sec:apps}})} 
\\



% & 
% % \multicolumn{1}{l}{\textbf{\bfseries\scshape \small Method}} 
% & 
% & 
% % QUERYING
% \rotatebox{\rotateDeg}{\textbf{\S\ref{sec:querying-selection}~~Traditional Selection}} & 
% \rotatebox{\rotateDeg}{\textbf{\S\ref{sec:querying-llm-based-selection}~~LLM-based Selection}} &
% \rotatebox{\rotateDeg}{\textbf{LLM-based Generation (\S\ref{sec:querying-llm-based-gen})}} &
% \rotatebox{\rotateDeg}{\textbf{Hybrid (\S\ref{sec:querying-llm-based-selection-and-gen})}} &
% % ANNOTATION
% \rotatebox{\rotateDeg}{\textbf{Human Annotation (\S\ref{sec:ann-human})}} &
% \rotatebox{\rotateDeg}{\textbf{LLM-based Annotation (\S\ref{sec:ann-llm-based})}} &  
% \rotatebox{\rotateDeg}{\textbf{Hybrid (\S\ref{sec:ann-hybrid-human-and-llm})}} & 
% % ==========
% %  APPLICATIONS
% % ==========
% \rotatebox{\rotateDeg}{\textbf{Hybrid}} & 
% \rotatebox{\rotateDeg}{\textbf{Link prediction}} &
% \rotatebox{\rotateDeg}{\textbf{Link classification}} &
% \rotatebox{\rotateDeg}{\textbf{Node classification}} &
% \rotatebox{\rotateDeg}{\textbf{Clustering}} &
% \rotatebox{\rotateDeg}{\textbf{Other}} 
% \\
% \noalign{\hrule height 0.8pt}



& 
% \multicolumn{1}{l}{\textbf{\bfseries\scshape \small Method}} 
& 
& 
% QUERYING
\rotatebox{\rotateDeg}{\textbf{Traditional Selection (\S\ref{sec:querying-selection})}} & 
\rotatebox{\rotateDeg}{\textbf{LLM-based Selection (\S\ref{sec:querying-llm-based-selection})}} &
\rotatebox{\rotateDeg}{\textbf{LLM-based Generation (\S\ref{sec:querying-llm-based-gen})}} &
\rotatebox{\rotateDeg}{\textbf{Hybrid (\S\ref{sec:querying-llm-based-selection-and-gen})}} &
% ANNOTATION
\rotatebox{\rotateDeg}{\textbf{Human Annotation (\S\ref{sec:ann-human})}} &
\rotatebox{\rotateDeg}{\textbf{LLM-based Annotation (\S\ref{sec:ann-llm-based})}} &  
\rotatebox{\rotateDeg}{\textbf{Hybrid (\S\ref{sec:ann-hybrid-human-and-llm})}} & 
% ==========
%  APPLICATIONS
% ==========
\rotatebox{\rotateDeg}{\textbf{Text Classification}} & 
\rotatebox{\rotateDeg}{\textbf{Text Summarization}} &
\rotatebox{\rotateDeg}{\textbf{Classification}} &
\rotatebox{\rotateDeg}{\textbf{Question Answering}} &
\rotatebox{\rotateDeg}{\textbf{Entity Matching}} &
\rotatebox{\rotateDeg}{\textbf{Debiasing}} &
\rotatebox{\rotateDeg}{\textbf{Translation}} &
\rotatebox{\rotateDeg}{\textbf{Sentiment Analysis}} &
% translation, sentiment analysis
\rotatebox{\rotateDeg}{\textbf{Other}} 
\\
\noalign{\hrule height 0.8pt}


% An approach called APE~\cite{qian2024ape} focuses on an active prompt engineering approach for entity matching where at each iteration, a set of prompts are derived, and then evaluated by a committee of models, where the best is selected, and then the approach repeats.
& \sysName{$\mathsf{\sf \bf APE}$}~\cite{qian2024ape}
&
% QUERYING
& \cellno % Traditional Selection
& \cellno % LLM Selection
& \cellyes % LLM Gen
& \cellno % Hybrid
% ANNOTATION
& \cellno % human
& \cellyes % LLM
& \cellno % Hybrid
% APPLICATIONS
& \cellno 
& \cellno
& \cellno
& \cellno
& \cellyes 
& \cellno
& \cellno
& \cellno 
& \cellno 
\\
\hline


% \citet{li2024active} proposes LLM-Determined Curriculum Active Learning (LDCAL) that improves the stability of the active learner by selecting instances from easy to hard by using LLMs to determine the difficulty of a document.
& \sysName{$\mathsf{\sf \bf LDCAL}$}~\cite{li2024active}
&
% QUERYING
& \cellyes % Traditional Selection
& \cellyes % LLM Selection
& \cellno % LLM Gen
& \cellno % Hybrid
% ANNOTATION
& \cellyes % human
& \cellno % LLM
& \cellno % Hybrid
% APPLICATIONS
& \cellno 
& \cellyes 
& \cellno
& \cellno
& \cellno
& \cellno
& \cellno
& \cellno 
& \cellno 
\\
\hline



% ActiveLLM~\cite{bayer2024activellm} uses LLMs to select instances for the few-shot and model mismatch setting. Importantly, ActiveLLM can estimate uncertainty and diversity without any annotated data, and does not require training during the annotation process.
& \sysName{$\mathsf{\sf \bf ActiveLLM}$}~\cite{bayer2024activellm}
&
% QUERYING
& \cellno % Traditional Selection
& \cellyes % LLM Selection
& \cellno % LLM Gen
& \cellno % Hybrid
% ANNOTATION
& \cellno % human
& \cellno % LLM
& \cellno % Hybrid
% APPLICATIONS
& \cellyes 
& \cellno 
& \cellno
& \cellno
& \cellno
& \cellno
& \cellno
& \cellno 
& \cellno 
\\
\hline



% ActivePrune~\cite{azeemi2024language} introduces a language model approach for pruning unlabeled instances for active learning settings where the unlabeled instance pool is large, and thus, computationally costly for an acquisition function to search over.
% Experiments on translation, sentiment analysis, topic classification, and summarization tasks
& \sysName{$\mathsf{\sf \bf ActivePrune}$}~\cite{azeemi2024language}
&
% QUERYING
& \cellyes % Traditional Selection
& \cellyes % LLM Selection
& \cellno % LLM Gen
& \cellno % Hybrid
% ANNOTATION
& \cellyes % human
& \cellno % LLM
& \cellno % Hybrid
% APPLICATIONS
& \cellyes 
& \cellyes 
& \cellno
& \cellno
& \cellno
& \cellno
& \cellyes
& \cellyes 
& \cellno 
\\
\hline


% \citet{ming2024autolabel} proposed AutoLabel
& \sysName{$\mathsf{\sf \bf AutoLabel}$}~\cite{ming2024autolabel}
&
% QUERYING
& \cellyes % Traditional Selection
& \cellno % LLM Selection
& \cellno % LLM Gen
& \cellno % Hybrid
% ANNOTATION
& \cellno % human
& \cellno % LLM
& \cellyes % Hybrid
% APPLICATIONS
& \cellno 
& \cellno 
& \cellno
& \cellno
& \cellyes
& \cellno
& \cellno
& \cellno 
& \cellno 
\\
\hline

% \citet{zhang2023llmaaa} proposed LLMaAA that leverages LLMs for annotation in an active learning loop. LLMaAA is used for both named entity recognition and relation extraction.
& \sysName{$\mathsf{\sf \bf LLMaAA}$}~\cite{zhang2023llmaaa}
&
% QUERYING
& \cellyes % Traditional Selection
& \cellno % LLM Selection
& \cellno % LLM Gen
& \cellno % Hybrid
% ANNOTATION
& \cellno % human
& \cellyes % LLM
& \cellno % Hybrid
% APPLICATIONS
& \cellno 
& \cellno 
& \cellno
& \cellno
& \cellyes
& \cellno
& \cellno
& \cellno 
& \cellyes % Other (RELATION EXTRACTION)
\\
\hline

% \cite{diao2023active} uses LLMs to generate $k$ answers to a question that are then used to measure the uncertainty for selection.
% In this work, they simply leverage LLMs to generate answers which are used to estimate the uncertainty of a given question, however, they do not generate new questions not in the initial dataset.
& \sysName{$\mathsf{\sf \bf Active\text{-}Prompt}$}~\cite{diao2023active}
&
% QUERYING
& \cellno % Traditional Selection
& \cellno % LLM Selection
& \cellyes % LLM Gen
& \cellno % Hybrid
% ANNOTATION
& \cellyes % human
& \cellno % LLM
& \cellno % Hybrid
% APPLICATIONS
& \cellno 
& \cellno 
& \cellno
& \cellyes
& \cellno
& \cellno
& \cellno
& \cellno 
& \cellno % Other (RELATION EXTRACTION)
\\
\hline




% \cite{rouzegar2024enhancing} investigates using LLMs with human annotations for text classification to achieve lower costs while maintaining accuracy. 
& \sysName{$\mathsf{\sf \bf HybridAL}$}~\cite{rouzegar2024enhancing}
&
% QUERYING
& \cellyes % Traditional Selection
& \cellno % LLM Selection
& \cellno % LLM Gen
& \cellno % Hybrid
% ANNOTATION
& \cellno % human
& \cellno % LLM
& \cellyes % Hybrid
% APPLICATIONS
& \cellyes 
& \cellno 
& \cellno
& \cellno
& \cellno
& \cellno
& \cellno
& \cellno 
& \cellno % Other 
\\
\hline

% ACL 2024
% https://aclanthology.org/2024.acl-long.592.pdf
& \sysName{$\mathsf{\sf \bf NoiseAL}$}~\cite{yuan2024hide} & 
% QUERYING
& \cellno % Traditional Selection
& \cellno % LLM Selection
& \cellyes % LLM Gen
& \cellyes % Hybrid
% ANNOTATION
& \cellno % human
& \cellyes % LLM
& \cellno % Hybrid
% APPLICATIONS
& \cellyes 
& \cellno 
& \cellno
& \cellno
& \cellno
& \cellno
& \cellno
& \cellno 
& \cellyes 
\\
\hline

% ACL 2024
% https://aclanthology.org/2024.acl-long.778.pdf
& \sysName{$\mathsf{\sf \bf CAL}$}~\cite{du2024causal} & 
% QUERYING
& \cellno % Traditional Selection
& \cellno % LLM Selection
& \cellno % LLM Gen
& \cellyes % Hybrid
% ANNOTATION
& \cellyes % human
& \cellno % LLM
& \cellno % Hybrid
% APPLICATIONS
& \cellno 
& \cellno 
& \cellno
& \cellno
& \cellno
& \cellyes
& \cellno
& \cellno 
& \cellno 
\\
\hline

% ICML 2024
% https://openreview.net/pdf?id=CTgEV6qgUy
& \sysName{$\mathsf{\sf \bf APL}$}~\cite{muldrew2024active} & 
% QUERYING
& \cellyes % Traditional Selection
& \cellno % LLM Selection
& \cellno % LLM Gen
& \cellno % Hybrid
% ANNOTATION
& \cellyes % human
& \cellno % LLM
& \cellno % Hybrid
% APPLICATIONS
& \cellno 
& \cellyes 
& \cellno
& \cellno
& \cellno
& \cellno
& \cellno
& \cellno 
& \cellyes
\\
\hline


% PKDD 2024
% https://arxiv.org/pdf/2404.02261
& \sysName{$\mathsf{\sf \bf AL\text{-}Loop}$}~\cite{kholodna2024llms} & 
% QUERYING
& \cellyes % Traditional Selection
& \cellno % LLM Selection
& \cellno % LLM Gen
& \cellno % Hybrid
% ANNOTATION
& \cellno % human
& \cellyes % LLM
& \cellno % Hybrid
% APPLICATIONS
& \cellno 
& \cellno 
& \cellno
& \cellno
& \cellyes
& \cellno
& \cellno
& \cellno 
& \cellno 
\\
\hline

% Neurips 2024
% https://arxiv.org/pdf/2406.10023
& \sysName{$\mathsf{\sf \bf BAL\text{-}PM}$}~\cite{melo2024deep} & 
% QUERYING
& \cellno % Traditional Selection
& \cellyes % LLM Selection
& \cellno % LLM Gen
& \cellno % Hybrid
% ANNOTATION
& \cellyes % human
& \cellno % LLM
& \cellno % Hybrid
% APPLICATIONS
& \cellno 
& \cellyes 
& \cellno
& \cellno
& \cellno
& \cellno
& \cellno
& \cellno 
& \cellno 
\\
\hline

% EMNLP 2023
% https://arxiv.org/pdf/2311.15614
& \sysName{$\mathsf{\sf \bf FreeAL}$}~\cite{xiao2023freeal} & 
% QUERYING
& \cellno % Traditional Selection
& \cellno % LLM Selection
& \cellno % LLM Gen
& \cellyes % Hybrid
% ANNOTATION
& \cellno % human
& \cellyes % LLM
& \cellno % Hybrid
% APPLICATIONS
& \cellyes 
& \cellno 
& \cellno
& \cellno
& \cellyes
& \cellno
& \cellno
& \cellyes 
& \cellno \\
\hline

% EMNLP Finding 2023 
% https://arxiv.org/pdf/2305.14264
& \sysName{$\mathsf{\sf \bf AL\text{-}Principle}$}~\cite{margatina2023active} & 
% QUERYING
& \cellyes % Traditional Selection
& \cellno % LLM Selection
& \cellno % LLM Gen
& \cellno % Hybrid
% ANNOTATION
& \cellyes % human
& \cellno % LLM
& \cellno % Hybrid
% APPLICATIONS
& \cellyes 
& \cellno 
& \cellyes
& \cellyes  % multiple-choice tasks
& \cellno
& \cellno
& \cellno
& \cellyes 
& \cellno 
\\
\hline

% EMNLP Finding 2023 
% https://arxiv.org/pdf/2305.12710

& \sysName{$\mathsf{\sf \bf Beyond\text{-}Labels}$}~\cite{yao2023beyond}
&
% QUERYING
& \cellno % Traditional Selection
& \cellno % LLM Selection
& \cellyes  % LLM Gen
& \cellno % Hybrid
% ANNOTATION
& \cellyes % human
& \cellno % LLM
& \cellno % Hybrid
% APPLICATIONS
& \cellyes 
& \cellno 
& \cellno
& \cellno
& \cellno
& \cellno
& \cellno
& \cellno 
& \cellyes 
\\
\hline


% Other potentially related works
% CVPR 2024 Active Prompt Learning in Vision Language Models
% https://ieeexplore.ieee.org/document/10655710

% arxiv 2023 Large Language Models as Annotators: Enhancing Generalization of NLP
% https://arxiv.org/pdf/2306.15766



% & \sysName{$\mathsf{\sf \bf TODO}$}~\cite{todo}
% &
% % QUERYING
% & \cellno % Traditional Selection
% & \cellno % LLM Selection
% & \cellno % LLM Gen
% & \cellno % Hybrid
% % ANNOTATION
% & \cellno % human
% & \cellno % LLM
% & \cellno % Hybrid
% % APPLICATIONS
% & \cellno 
% & \cellno 
% & \cellno
% & \cellno
% & \cellno
% & \cellno
% & \cellno
% & \cellno 
% & \cellno 
% \\
% \hline




\noalign{\hrule height 0.7pt}
\end{tabular}
\end{center}
}

\end{minipage}
\vspace{-1mm}
\caption{%
Overview of the proposed taxonomy for LLM-based active learning techniques and their applications.
% Techniques are categorized by the type of input graph supported, whether the approach focuses on pre-processing, in-processing (in-training), or post-processing, as well as the task the technique was designed.
Using this taxonomy, we provide a qualitative and quantitative comparison of LLM-based active learning methods.
% \hongjie{
% There are 3 rows I want someone to verify and confirm the taxonomy is accurate.
% CAL: this work utilizes AL to debias LLMs. The querying stage seems to depend on LLMs, but there is no annotation stage. 
% APL: not using LLMs for querying or annotation, but use it to rank selected samples; 
% AL-Principles: use LLM models for evaluation; Beyond-labels: Please verify LLM Generation is accurate, it is used for explanation generation, but not sample generation.
% }
% \namyong{
% I'll check if the check marks for the three papers (CAL, AL-Principle, Beyond-Labels) are accurate.
% }
% \namyong{
% Marked LDCAL as (1) using both traditional and LLM-based selection, and as (2) using human annotation. In LDCAL, LLMs are used for curriculum learning (evaluating the difficulty of training instances), but not for selection. However, I suppose its use for curriculum learning can be seen as an indirect form of selection.
% }
}
\label{table:qual-and-quant-comparison}
% \vspace{-3.5mm}
% \vspace{-6mm}
\vspace{-3mm}
\end{table*}
}








