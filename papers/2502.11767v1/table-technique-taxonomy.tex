


\definecolor{googleblue}{HTML}{4285F4}
\definecolor{googlered}{HTML}{DB4437}
\definecolor{googlepurple}{HTML}{A142F4} % New purple color
\definecolor{googlegreen}{HTML}{0F9D58}


\begin{table*}[t!]
\centering
\small
\renewcommand{\arraystretch}{1.30} 
% \begin{tabular}{cr Xl}
\begin{tabularx}{1.0\linewidth}{
>{\centering\arraybackslash}m{18mm} 
>{\RaggedLeft\arraybackslash}m{44mm} 
X
% >{\arraybackslash}m{90mm} H
}
\toprule
\textbf{Class} & \textbf{General Mechanism} & 
\textbf{Description}
% \textbf{Example Models and Methods} 
\\ 
\midrule


\multirow{5}{*}{\textcolor{googlegreen}{\textbf{\makecell{Querying\\\textcolor{black}{(\textbf{Section~\ref{sec:querying}})}}}}} 
& Traditional Selection (Sec.~\ref{sec:querying-selection}) &  
This class of techniques uses traditional selection such as uncertainty sampling, disagreement, gradient-based sampling, and so on.
\\

& LLM-based Selection (Sec.~\ref{sec:querying-llm-based-selection}) &  
The class of LLM-based selection techniques focus on using LLMs to select the instances. 
% These techniques can be used 
\\

& LLM-based Generation (Sec.~\ref{sec:querying-llm-based-gen}) &  
The class of LLM-based generation techniques focus on generating novel instances.  \\

& Hybrid (Sec.~\ref{sec:querying-llm-based-selection-and-gen}) &  
Combines advantages of both LLM-based selection and generation
\\

\midrule

\multirow{5}{*}{\textcolor{googleblue}{\textbf{\makecell{Annotation\\\textcolor{black}{(\textbf{Section~\ref{sec:ann}})}}}}}
& Human Annotation (Sec.~\ref{sec:ann-human}) &  
Traditional human annotation simply refers to using humans to annotate the selected or generated instances, which is costly.
\\

& LLM-based Annotation (Sec.~\ref{sec:ann-llm-based}) &  
The class of LLM-based annotation techniques focus on leveraging LLMs for annotation and evaluation. This class of techniques are far cheaper than human annotation.
\\

& Hybrid (Sec.~\ref{sec:ann-hybrid-human-and-llm}) &  
This class of techniques aim to leverage the advantages of both humans and LLMs for optimal annotations while minimizing cost 
% (monetary, time, etc).
\\
% \midrule

% \multirow{2}{*}{\textcolor{googlegreen}{\textbf{\makecell{Hybrid\\\textcolor{black}{(\textbf{Section~\ref{sec:hybrid}})}}}}} &  TODO~\cite{} \\ 
% % \multirow{4}{*}{(\textbf{Section~\ref{sec:ret}})}
% & Alignment (Sec.~\ref{sec:ret-alignment}) &  TODO~\cite{} \\
% % & Generation (Sec.~\ref{sec:ret-generation}) & \citet{ye2024contemporary},Yo'LLaVA~\cite{nguyen2024yo} \\ 
% % & Fine-tuning (Sec.~\ref{sec:ret-finetune}) & FedPAM \cite{feng2024fedpam}, VITR \cite{gong2023vitr}\\ 


\bottomrule
\end{tabularx}
\caption{Taxonomy of LLM-based Active Learning Techniques (\Cref{sec:querying,sec:ann}).
% \ryan{%
% Active learning for LLMs vs. LLM for Active Learning
% }%
}
\label{tab:taxonomy-techniques}
\vspace{-3mm}
\end{table*}
