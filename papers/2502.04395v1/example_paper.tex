%%%%%%%% ICML 2025 EXAMPLE LATEX SUBMISSION FILE %%%%%%%%%%%%%%%%%

\documentclass{article}

% Recommended, but optional, packages for figures and better typesetting:
\usepackage{microtype}
\usepackage{graphicx}
\usepackage{subfigure}
\usepackage{booktabs} % for professional tablesx

% hyperref makes hyperlinks in the resulting PDF.
% If your build breaks (sometimes temporarily if a hyperlink spans a page)
% please comment out the following usepackage line and replace
% \usepackage{icml2025} with \usepackage[nohyperref]{icml2025} above.
\usepackage{hyperref}
\usepackage{makecell}
\usepackage{xcolor}

\usepackage{threeparttable}

% Attempt to make hyperref and algorithmic work together better:
\newcommand{\theHalgorithm}{\arabic{algorithm}}

% Use the following line for the initial blind version submitted for review:
% \usepackage{icml2025}

% If d, instead use the following line for the camera-ready submission:
\usepackage[accepted]{icml2025}

% For theorems and such
\usepackage{amsmath}
\usepackage{amssymb}
\usepackage{caption}
\usepackage{mathtools}
\usepackage{amsthm}
\usepackage{enumitem}
\usepackage{multirow}
\usepackage{xspace}
\usepackage{wrapfig}

% if you use cleveref..
\usepackage[capitalize,noabbrev]{cleveref}

%%%%%%%%%%%%%%%%%%%%%%%%%%%%%%%%
% THEOREMS
%%%%%%%%%%%%%%%%%%%%%%%%%%%%%%%%
\theoremstyle{plain}
\newtheorem{theorem}{Theorem}[section]
\newtheorem{proposition}[theorem]{Proposition}
\newtheorem{lemma}[theorem]{Lemma}
\newtheorem{corollary}[theorem]{Corollary}
\theoremstyle{definition}
\newtheorem{definition}[theorem]{Definition}
\newtheorem{assumption}[theorem]{Assumption}
\theoremstyle{remark}
\newtheorem{remark}[theorem]{Remark}

% Todonotes is useful during development; simply uncomment the next line
%    and comment out the line below the next line to turn off comments
%\usepackage[disable,textsize=tiny]{todonotes}
\usepackage[textsize=tiny]{todonotes}

\newcommand{\method}{\text{Time-VLM}\xspace}
\newcommand{\update}[1]{{\textcolor{black}{#1}}}
\newcommand{\boldres}[1]{{\textbf{\textcolor{red}{#1}}}}
\newcommand{\secondres}[1]{{\underline{\textcolor{blue}{#1}}}}

\definecolor{pink}{rgb}{1, 0, 0.5}
\newcommand{\revision}[1]{{\textcolor{black}{#1}}}

\newcommand*{\shortautoref}[1]{%
  \begingroup
    \def\sectionautorefname{Section}%
    \def\subsectionautorefname{Section}%
    \def\figureautorefname{Figure}%
    \def\tableautorefname{Table}%
    \def\equationautorefname{Equation}%
    \autoref{#1}%
  \endgroup
}


% The \icmltitle you define below is probably too long as a header.
% Therefore, a short form for the running title is supplied here:
% \icmltitlerunning{Submission and Formatting Instructions for ICML 2025}

\begin{document}

\twocolumn[
\icmltitle{\method: Exploring Multimodal Vision-Language Models for Augmented Time Series Forecasting}

% It is OKAY to include author information, even for blind
% submissions: the style file will automatically remove it for you
% unless you've provided the [accepted] option to the icml2025
% package.

% List of affiliations: The first argument should be a (short)
% identifier you will use later to specify author affiliations
% Academic affiliations should list Department, University, City, Region, Country
% Industry affiliations should list Company, City, Region, Country

% You can specify symbols, otherwise they are numbered in order.
% Ideally, you should not use this facility. Affiliations will be numbered
% in order of appearance and this is the preferred way.
% \icmlsetsymbol{equal}{*}

\begin{icmlauthorlist}
\icmlauthor{Siru Zhong}{hkustgz}
\icmlauthor{Weilin Ruan}{hkustgz}
\icmlauthor{Ming Jin}{gu}
\icmlauthor{Huan Li}{zju}
\icmlauthor{Qingsong Wen}{sa}
\icmlauthor{Yuxuan Liang}{hkustgz}
\end{icmlauthorlist}

\icmlaffiliation{hkustgz}{The Hong Kong University of Science and Technology (Guangzhou)}
\icmlaffiliation{gu}{Griffith University}
\icmlaffiliation{zju}{Zhejiang University}
\icmlaffiliation{sa}{Squirrel AI}

\icmlcorrespondingauthor{Yuxuan Liang}{yuxliang@outlook.com}

% You may provide any keywords that you
% find helpful for describing your paper; these are used to populate
% the "keywords" metadata in the PDF but will not be shown in the document
% \icmlkeywords{Machine Learning, ICML}

\vskip 0.3in
]

% this must go after the closing bracket ] following \twocolumn[ ...

% This command actually creates the footnote in the first column
% listing the affiliations and the copyright notice.
% The command takes one argument, which is text to display at the start of the footnote.
% The \icmlEqualContribution command is standard text for equal contribution.
% Remove it (just {}) if you do not need this facility.

%\printAffiliationsAndNotice{}  % leave blank if no need to mention equal contribution
\printAffiliationsAndNotice{} % otherwise use the standard text.

\begin{abstract}
Multi-modal models, such as CLIP, have demonstrated strong performance in aligning visual and textual representations, excelling in tasks like image retrieval and zero-shot classification. Despite this success, the mechanisms by which these models utilize training data, particularly the role of memorization, remain unclear. In uni-modal models, both supervised and self-supervised, memorization has been shown to be essential for generalization. However, it is not well understood how these findings would apply to CLIP, which incorporates elements from both supervised learning via captions that provide a supervisory signal similar to labels, and from self-supervised learning via the contrastive objective.
To bridge this gap in understanding, we propose a formal definition of memorization in CLIP (CLIPMem) and use it to quantify memorization in CLIP models. Our results indicate that CLIP’s memorization behavior falls between the supervised and self-supervised paradigms, with "mis-captioned" samples exhibiting highest levels of memorization. 
Additionally, we find that the text encoder contributes more to memorization than the image encoder, suggesting that mitigation strategies should focus on the text domain. 
Building on these insights, we propose multiple strategies to reduce memorization while at the same time improving utility---something that had not been shown before for traditional learning paradigms where reducing memorization typically results in utility decrease.
%, some of our proposed mitigations for CLIP can reduce memorization while improving downstream utility.\todo{This needs to be reworked: our CLIPMem has the practical application to identify "miscaptioned" samples, such that we can remove them from the training, and then get better results. This is particularly important given that CLIP is trained on large amounts of uncurated data from the internet, and one cannot review all these image pairs. With out metric, it becomes possible.
%aybe also include the risk of exposure for these data points that otherwise arise.
%}
% Multi-modal models, such as CLIP, exhibit strong performance in aligning visual and textual representations, thereby achieving remarkable performance in tasks like image retrieval and zero-shot classification. 
% While the models have a strong generalization ability, it is not fully understood how the models leverage their training data to achieve this.
% One factor that is often linked to a model's generalization ability is memorization. For uni-modal models, both in supervised and self-supervised, it has been shown that memorization is required for generalization.
% Yet, it is unclear how the findings will translate to CLIP because in CLIP captions as supervisory signals, somewhat akin to traditional labels, but also employs self-supervised contrastive learning. Hence, CLIP is in between both paradigms.
% To bridge this gap, we propose a formal definition of memorization in CLIP (CLIPMem) and use it to quantify memorization in CLIP models. 
% Our results show that CLIP's memorization behavior indeed falls between supervised and self-supervised paradigms. Notably, "mis-captioned" samples exhibit high levels of memorization.
% Additionally, we find that the the text encoder has a higher impact on memorization than the image encoder.
% Based on these findings, we find some effective mitigation strategies for memorization in CLIP that focus more on the text domain to maintain model performance while reducing memorization.
% Indeed, unlike in traditional supervised or self-supervised learning, where reducing memorization often reduces utility, we empirically find that some mitigations in CLIP not only reduce memorization but at the same time improve downstream utility.
\end{abstract}

Large language models (LLMs) show significant performance in various downstream
tasks~\citep{brown_language_2020,openai_gpt-4_2024,dubey_llama_2024}. Studies
have found that training on high quality corpus improves the ability of LLMs
to solve different problems such as writing code, doing math exercises, and
answering logic questions~\citep{cai_internlm2_2024,deepseek-ai_deepseek-v3_2024,qwen_qwen25_2024}.
Therefore, effectively selecting high-quality text data is an important subject for
training LLM.

\begin{figure}[t]
    \centering
    \includegraphics[width=\linewidth]{figures/head.pdf}
    \caption{The overview of CritiQ. We (1) employ human annotators to annotate $\sim$30
    pairwise quality comparisons, (2) use CritiQ Flow to mine quality criteria, (3)
    use the derived criteria to annotate 25k pairs, and (4) train the CritiQ Scorer to
    perform efficient data selection.}
    \label{fig:overview}
\end{figure}

To select high-quality data from a large corpus, researchers manually design heuristics~\citep{dubey_llama_2024,rae_scaling_2022},
calculate perplexity using existing LLMs~\citep{marion2023moreinvestigatingdatapruning,wenzek2019ccnetextractinghighquality},
train classifiers~\citep{brown_language_2020,dubey_llama_2024,xie_data_2023} and
query LLMs for text quality through careful prompt engineering~\citep{gunasekar_textbooks_2023,wettig_qurating_2024,sachdeva_how_2024}.
Large-scale human annotation and prompt engineering require a lot of human
effort. Giving a comprehensive description of what high-quality data is like is also
challenging. As a result, manually designing heuristics lacks robustness and introduces
biases to the data processing pipeline, potentially harming model performance
and generalization. In addition, quality standards vary across different
domains. These methods can not be directly applied to other domains without significant
modifications.

To address these problems, we introduce CritiQ, a novel method to automatically
and effectively capture human preferences for data quality and perform efficient data
selection. Figure~\ref{fig:overview} gives an overview of CritiQ, comprising an agent
workflow, CritiQ Flow, and a scoring model, CritiQ Scorer. Instead of manually describing
how high quality is defined, we employ LLM-based agents to summarize quality
criteria from only $\sim$30 human-annotated pairs.

CritiQ Flow starts from a knowledge base of data quality criteria. The worker
agents are responsible to perform pairwise judgment under a given
criterion. The manager agent generates new criteria and refines them through reflection
on worker agents' performance. The final judgment is made by majority voting among
all worker agents, which gives a multi-perspective view of data quality.

To perform efficient data selection, we employ the worker agents to annotate a randomly
selected pairwise subset, which is ~1000x larger than the human-annotated one.
Following \citet{korbak_pretraining_2023,wettig_qurating_2024}, we train CritiQ
Scorer, a lightweight Bradley-Terry model~\citep{bradley_rank_1952} to convert
pairwise preferences into numerical scores for each text. We use CritiQ Scorer to
score the entire corpus and sample the high-quality subset.

For our experiments, we established human-annotated test sets to quantitatively
evaluate the agreement rate with human annotators on data quality preferences. We implemented the manager agent by \texttt{GPT-4o} and the worker
agent by \texttt{Qwen2.5-72B-Insruct}. We conducted experiments on different
domains including code, math, and logic, in which CritiQ Flow shows a consistent
improvement in the accuracies on the test sets, demonstrating the effectiveness
of our method in capturing human preferences for data quality. To validate the quality
of the selected dataset, we continually train \texttt{Llama 3.1}~\citep{dubey_llama_2024}
models and find that the models achieve better performance on downstream tasks
compared to models trained on the uniformly sampled subsets.

We highlight our contributions as follows. We will release the code to facilitate
future research.

\begin{itemize}
    \item We introduce CritiQ, a method that captures human preferences for data
        quality and performs efficient data selection at little cost of human
        annotation effort.

    \item Continual pretraining experiments show improved model performance in code,
        math, and logic tasks trained on our selected high-quality subset compared to the raw dataset.

    \item Ablation studies demonstrate the effectiveness of the knowledge base and
        the the reflection process.
\end{itemize}

\begin{figure*}[t]
    \centering
    \includegraphics[width=\linewidth]{figures/method.pdf}
    \caption{CritiQ Flow comprises two major components: multi-criteria pairwise
    judgment and the criteria evolution process. The multi-criteria pairwise
    judgment process employs a series of worker agents to make quality
    comparisons under a certain criterion. The criteria evolution process aims to
    obtain data quality criteria that highly align with human judgment through
    an iterative evolution. The initial criteria are retrieved from the
    knowledge base. After evolution, we select the final criteria to annotate
    the dataset for training CritiQ Scorer.}
    \label{fig:method}
\end{figure*}

\section{Related Work}

\begin{figure*}[t!]
    \centering
    \includegraphics[width=0.99\textwidth]{figures/framework.pdf}
    \caption{Overview of the \method framework.}
    \label{fig:framework}
    \vspace{-1em}
\end{figure*}

\noindent\textbf{Text-Augmented Models for Time Series Forecasting.} 

The success of LLMs inspires their application to time series tasks. Methods like LLMTime~\cite{gruver2023large} and LLM4TS~\cite{chang2023llm4ts} tokenize time series data for autoregressive prediction but inherit LLMs' limitations, such as poor arithmetic and recursive capabilities. Recent approaches, including GPT4TS~\cite{zhou2023one} and TimeLLM~\cite{jin2023time}, project time series into textual representations to leverage LLMs' reasoning abilities. However, they face challenges like the modality gap and lack of time series-optimized word embeddings, leading to potential information loss. UniTime~\cite{liu2024unitime} and TimeFFM~\cite{liu2024time} incorporate domain-specific instructions and federated learning, respectively, but remain constrained by their reliance on text alone.

\noindent\textbf{Vision-Augmented Models for Time Series Forecasting.} 

Vision emerges as a natural way to preserve temporal patterns. Early approaches use CNNs for matrix-formed time series~\cite{li2020forecasting, sood2021visual}, while TimesNet~\cite{wu2023timesnet} introduces multi-periodic decomposition for unified 2D modeling. VisionTS~\cite{chen2024visiontsvisualmaskedautoencoders} pioneers pre-trained visual encoders with grayscale time series images, and TimeMixer++~\cite{wang2024timemixer++} advances the field with multi-scale frequency-based time-image transformations. Despite their effectiveness in temporal modeling, these methods often lack semantic context, hard to use high-level contextual information for prediction.

\noindent\textbf{Vision-Language Models.} 

VLMs like ViLT~\cite{kim2021vilt}, CLIP~\cite{radford2021learning}, and ALIGN~\cite{jia2021scaling} transform multimodal understanding by aligning visual and textual representations. Recent advancements, like BLIP-2~\cite{li2022blip2} and LLaVA~\cite{liu2023visual}, further enhance multimodal reasoning. However, VLMs remain underexplored for time series analysis. Our work bridges this gap by leveraging VLMs to integrate temporal, visual, and textual modalities, addressing the limitations of unimodal approaches.
\section{Method}
\label{sec:method}
In this section, we propose a neuroscience-informed fMRI encoder designed to achieve high-performance, subject-agnostic decoding. To further enable versatile decoding, we introduce the construction of a brain instruction tuning dataset, which captures diverse semantic representations encoded in fMRI data.

\subsection{Method Overview}
As illustrated in Figure~\ref{fig:arch}, our model consists of an fMRI encoder $f_\theta$ and an off-the-shelf LLM. In practice, we use Vicuna-7b \cite{zheng2023judging} as our LLM to maintain consistency with our baseline \cite{xia2024umbrae}. For each sample, let $\boldsymbol{v} = [v_1, v_2, \cdots, v_N]\in \mathbb{R}^N$ be the fMRI signals of input voxels, where $N$ is the number of voxels. Note that $N$ varies between different subjects, ranging from $12,682$ to $17,907$ in the dataset we use \cite{allen2022massive}.

The fMRI encoder $f_\theta$, featuring a neuroscience-informed attention layer, encodes $\boldsymbol{v}$ to fMRI tokens $X_v = [\boldsymbol{x}_{v,1}, \boldsymbol{x}_{v,2}, \cdots, \boldsymbol{x}_{v,L}] \in \mathbb{R}^{d\times L}$, where $L$ is the number of tokens and $d$ is the dimension of token embeddings. We then prepend these learned fMRI tokens to the language tokens in the BIT dataset we propose.

\subsection{fMRI Encoder}
As mentioned before, currently most models for fMRI decoding can not handle varying input shapes and are not subject-agnostic, with only a few exceptions \cite{mai2023unibrain}. However, these exceptions still suffer from information loss and uneven representations of certain brain areas. To this end, we propose a novel neuroscience-informed attention mechanism to accommodate varying voxel numbers across subjects, enabling a subject-agnostic encoding strategy. Below we talk about the design of \textit{queries} $\{\boldsymbol{q}_i\}$, \textit{keys} $\{\boldsymbol{k}_i\}$ and \textit{values} $\{\boldsymbol{v}_i\}$ in the attention layer. For \textit{values}, we directly use the fMRI signal of each voxel, which means $\boldsymbol{v_i} = v_i \in \mathbb{R}$. Making each voxel a \textit{value} token maximally prevents information loss compared to pooling- \cite{wang2024mindbridge} or sampling-based \cite{mai2023unibrain} methods. The \textit{queries} are randomly initialized and learnable. We expect each \textit{query} to represent a certain pattern of the brain (refer to visualizations in Section \ref{sec:vis}). The design of \textit{keys} will be discussed below.

\noindent\textbf{Exclude fMRI values from \textit{keys}}
The vanilla cross attention \cite{zhu2020deformable,vaswani2017attention} derives both \textit{keys} and \textit{values} from the same input source. However, we found this would lead to poor performance in fMRI. We argue the reason: different from images or text, which are usually considered translation-invariant, the positions of voxels carry specific brain \textit{functional information}, as voxels in different areas are associated with distinct brain functions. Consequently, a voxel's position alone can theoretically serve as effective \textit{keys} for attention weight computation. Including fMRI values into \textit{keys}, however, introduces additional noise instead of valuable information, thus resulting in poorer performance. Moreover, since brain regions tend to serve similar functions across individuals, decoupling voxel positions from fMRI signals can facilitate the sharing of priors across subjects, potentially improving generalization to unseen subjects.

In light of this, instead of the vanilla cross attention, which derives the \textit{keys} and \textit{values} from the same inputs, we exclude the fMRI value of each voxel and use its positional information alone as its \textit{key} embedding. The positional information is encoded from the coordinates of each voxel, i.e. $\boldsymbol{k}_i^{\text{pos}} = \operatorname{PE}(\boldsymbol{c}_i)$ for the $i$-th voxel, where $\boldsymbol{c}_i \in \mathbb{R}^3$ denotes the coordinates of the voxel. In practice, we use the Fourier positional encoding proposed in \cite{tancik2020fourier} due to its superiority in encoding coordinate information.

\noindent\textbf{Incorporation of Brain Parcellations}
% \noindent\textbf{Incorporation of Brain Parcellations}
While positional encoding alone improves performance, it lacks inherent neuroscientific grounding, potentially making it challenging for the model to efficiently learn representations aligned with established principles of brain function. To overcome this, we incorporate existing brain region parcellations \cite{glasser2016multi,rolls2020automated} into the \textit{key} embeddings. Formally, given a parcellation $\mathcal{P}$, with regions indexed by $1, \cdots, N_\mathcal{P}$. Let $\mathcal{P}(i) \in [1, 2, \cdots, N_\mathcal{P}]$ be the region that the $i$-th voxel belongs to, and $E[\mathcal{P}(i)] \in \mathbb{R}^d$ be the corresponding learnable embedding of the region, which will be incorporated in the \textit{key} embeddings as $\boldsymbol{k}_i^{\text{reg}, \mathcal{P}} = E[\mathcal{P}(i)] \in \mathbb{R}^d$.

\noindent\textbf{Combining Multiple Parcellations}
It is crucial to choose an appropriate brain region parcellation. Previous region-based methods \cite{qiu2023learning,li2021braingnn, kan2022brain} can usually only choose one arbitrarily. In contrast, our model design allows us to combine multiple parcellations $\mathcal{P}^1, \mathcal{P}^2, \cdots$ by concatenating their respective region encodings to the \textit{key} embeddings. In conclusion, the final \textit{key} embeddings are the concatenation by the positional encoding and multiple region encodings,
\begin{equation}
    \boldsymbol{k}_i = \boldsymbol{k}_i^\text{pos} \| \boldsymbol{k}_i^{\text{reg}, \mathcal{P}^1} \|  \boldsymbol{k}_i^{\text{reg}, \mathcal{P}^2} \| \cdots
\end{equation}
where $\|$ denotes the concatenation operation. This process is illustrated in Figure~\ref{fig:arch}'s lower right part.

The positional and region encodings complement each other: The region encodings serve as coarse-scale features, providing a neuroscientific-grounded basis, while the fine-scale positional encoding allows our model to learn finer-grained information directly from the data.

This attention design separates a voxel's \textit{functional information}—which is largely consistent across individuals—from its fMRI value, thereby enhancing generalization. Instead of relying on pooling or sampling, the attention mechanism employs learnable aggregation, while the integration of positional encoding and neuroscientifically informed region encodings further ensures high performance.

After the attention layer, we obtain the hidden representations $\boldsymbol{z}_q \in \mathbb{R}^{N_q} $ where $N_q$ is the number of query embeddings. We then employ an MLP and a reshape operation to map the hidden representations to $L$ fMRI tokens, i.e., $   X_v = \operatorname{reshape}\left( \operatorname{MLP}(
    \{\boldsymbol{z}_q\}
    ) \right) \in \mathbb{R}^{L \times d}$.

The process of the fMRI encoder is illustrated in Figure~\ref{fig:arch}. The obtained fMRI tokens are then prepended to the language tokens in conversations.
\begin{figure}
    \centering
    \includegraphics[width=\linewidth]{figures/arch.pdf}
    % \vspace{-2.2em}
    \caption{Model Architecture. The fMRI encoder maps fMRI to a series of fMRI tokens through our proposed neuroscience-informed attention. The large language model, with both fMRI and text tokens, will be trained by brain instruction tuning.}
    \label{fig:arch}
    \vspace{-1em}
\end{figure}

\subsection{Brain Instruction Tuning (BIT)}
To enable versatile fMRI-to-text decoding, an appropriate BIT dataset is required, yet no such dataset currently exists. To bridge this gap, we construct one based on the fact: MSCOCO images \cite{chen2015microsoft} serve as stimuli for fMRI recordings in the fMRI study \cite{allen2022massive}, and an abundance of datasets provide text annotations (e.g., VQA) for MSCOCO images. Using the images as intermediaries, we select those relevant to brain functions and pair the fMRI data with corresponding text annotations. For example, given an image of a billboard with annotated textual content, we can reasonably infer that when a subject perceives textual information (e.g., contents on the billboard), corresponding representations are encoded in the brain. This suggests the possibility of extracting such information from fMRI signals. We select datasets to fulfill various purposes, enabling the model to capture diverse aspects of semantic information embedded in fMRI signals, including visual perception \& scene understanding, language \& symbolic processing, memory \& knowledge retrieval and complex reasoning, which are considered among most fundamental and essential properties of human brains \cite{robertson2002memory,stenning2012human,wade2013visual,friederici2017language}.

\begin{figure}[h]
% \vspace{-0.5em}
    \centering
    \includegraphics[width=\linewidth]{figures/bit.pdf}
\vspace{-1.8em}
    \caption{Dataset Taxonomy in Brain Instruction Tuning.}
    \label{fig:bit}
% \vspace{-1em}
\end{figure}

\noindent\textbf{Perception \& Scene Understanding} As illustrated in Figure~\ref{fig:bit}, we begin by using caption tasks at both coarse and fine-grained levels to train the model’s ability to understand and summarize what the subject perceives visually \cite{chen2015microsoft,krause2017hierarchical}. Additionally, we incorporate QA tasks \cite{ren2015exploring,krishna2017visual,acharya2019tallyqa} to enhance the model's ability to retrieve and reason about visually perceived content.

\noindent\textbf{Memory \& Knowledge Retrieval} To go beyond tasks directly related to present visual perception, we construct the \emph{previous captioning} task, a memory-oriented task that challenges the model to caption images that the subject previously viewed, simulating memory recall processes. Furthermore, we aim to encode knowledge structures in human brains. The OK-VQA \cite{marino2019ok} and A-OKVQA \cite{schwenk2022okvqa} datasets include questions requiring external knowledge that is not present in the image but resides in human brains. For example, A photo of a hydrant may prompt the answer "firetruck," even though the firetruck is absent in the image. This association also reflects the way human cognition operates through a network of interconnected meanings, where one concept unconsciously triggers another. Such a process, which is called "slippage of the signifier" \cite{lacan2001ecrits, lacan1988seminar, miller2018four}, highlights the symbolic processes through which the brain constructs and retrieves meaning. 

\noindent\textbf{Language \& Symbolic Processing} In addition to the aforementioned OK-VQA and A-OKVQA datasets, which are also related to symbolic process, we further combine datasets of text recognition \cite{biten2019scene} and numerical reasoning \cite{acharya2019tallyqa} to facilitate this aspect.

\noindent\textbf{Complex Reasoning} Finally, we try to approximate the reasoning process that happens in human brains with datasets \cite{liu2023visual,wang2023see,li2018vqa} that require intricate logical and inferential processes. We expect these datasets to challenge the model to extract the reasoning process, drawing upon both visual understanding and abstract problem-solving, thus bridging perception, memory, and knowledge into a cohesive cognitive framework.

We ended up with a brain instruction tuning dataset consisting of $980,610$ conversations associated with fMRI recordings from $15$ datasets. Appendix~\ref{app:dataset} lists the instructions and other details for each dataset. The instruction tuning enables versatile fMRI-to-text decoding. In particular, the introduction of tasks like \textit{previous caption} empowers the model to perform a broader range of tasks beyond vision-related ones, which the previous model \cite{xia2024umbrae} fails.

\begingroup
\sisetup{
  table-format=2.2,  % 3 digits before the decimal, 2 after
  table-align-text-pre=false,
  propagate-math-font=true,
  table-number-alignment=center,
  detect-weight=true,detect-inline-weight=math
}
\begin{table*}[bp]
    \centering
    \vspace{-1.7em}
    \caption{Results of brain captioning. The CIDEr metric is scaled by a factor of 100 for consistency with Table~\ref{tab:caption} and baselines.}
    \label{tab:caption}
\vspace{0.1in}
    \resizebox{\linewidth}{!}{
    \begin{tabular}{lcSSSSSSSS}
    \toprule
 % \multirow{2}{*}{Method}&  \multirow{2}{*}{cross-subject}&\multicolumn{5}{c}{fMRI caption} & &  &\\
   {Method} & {\makecell{subject\\agnostic}}  &{{BLEU-1} $\uparrow$} & {BLEU-2 $\uparrow$} & {BLEU-3 $\uparrow$} & {{BLEU-4} $\uparrow$} &{METEOR $\uparrow$}&{ROUGE $\uparrow$}& {CIDEr $\uparrow$}&{SPICE $\uparrow$}\\
    \midrule
    SDRecon \cite{takagi2023high}    & {\xmark} &36.21 & 17.11 & 7.22 & 3.43   &10.03&  25.13&13.83 &5.02 \\
    OneLLM  \cite{han2024onellm}  & {\xmark} &47.04 & 26.97 & 15.49 & 9.51   &13.55&  35.05&22.99 & 6.26\\
    UniBrain \cite{mai2023unibrain}   & {\xmark} & {$-$}   & {$-$}    & {$-$}  & {$-$}     &16.90&  22.20& {$-$} & {$-$}\\
    BrainCap \cite{ferrante2023brain}  & {\xmark} &55.96 & 36.21 & 22.70 & 14.51   &16.68& 40.69&41.30 & 9.06\\
     BrainChat \cite{huang2024brainchat} & {\xmark}   &52.30& 29.20& 17.10& 10.70 &14.30& 45.70&26.10 & {$-$}\\
    UMBRAE \cite{xia2024umbrae}    & {\xmark} &59.44& 40.48& 27.66&19.03&19.45&  43.71&61.06&12.79\\
    \name{} (Ours)  & {\cmark} & \bfseries 61.75 &  \bfseries42.84 & \bfseries29.86&\bfseries21.24  & 
\bfseries 19.54 &\bfseries45.82 & 60.97  & 11.79\\
    \bottomrule
    \end{tabular}}
\end{table*}
\endgroup


To train the model with the BIT dataset, for each sample $\boldsymbol{v}$, we sample a multi-run conversation $X_t = (X_u^1, X_a^1, \cdots, X_u^T, X_a^T)$ from all conversations associated with it, where $T \geq 1$ represents the number of turns. $a$ indicates the message from the assistant and $u$ indicates the message is from the user. The training objective is to maximize the probability of the assistant's response only
$$
\arg\max_\theta p(X_a | X_v, X_{\text{inst}}) = \prod_{t=1}^T p({\color{magenta}X_a^t} | X_u^{\leq t}, X_a^{\le t }, X_\text{inst}, X_v)
$$
Figure~\ref{fig:chat} illustrates the chat template and the training objective. We freeze the weights of the LLM and only train the fMRI encoder since we want to preserve the LLM's language modeling prior and ensure a fair comparison with baselines such as \citet{xia2024umbrae}.

\noindent\textbf{Computational Complexity} According to the analysis in Appendix~\ref{app:complexity}, our model does not introduce additional complexity compared to previous methods \cite{scotti2024mindeye2, wang2024mindbridge}.


\begin{figure}[htbp]
\vspace{-0.8em}
\centering
\begin{minipage}{0.99\columnwidth}\vspace{0mm}    \centering
\begin{tcolorbox}[colback=white,colframe=gray,left=1pt,top=1pt,bottom=1pt]
\sffamily
\footnotesize	
  \texttt{<system message>}\\
  user: $X_v$, $X_\text{inst}$, $X_1^u$ \\
  assistant: {\color{magenta}$X_1^a$}\\
user: $X_2^u$\\
  assistant: {\color{magenta}$X_2^a$}\\
  $\cdots\cdots$
\end{tcolorbox}
\end{minipage}
\caption{The chat template used during instruction tuning, illustrating two turns of conversations. Two turns of conversations are shown. Tokens highlighted in {\color{magenta}magenta} are used for next-token prediction loss computation.}
\label{fig:chat}
\vspace{-1.2em}
\end{figure}


\section{Experiments}
In this section, we first evaluate our model on various downstream tasks, demonstrating its versatile decoding capabilities. Next, we assess its generalizability to novel subjects and its adaptability to real-world applications. Finally, we analyze the functions of queries in our neuroscience-informed attention mechanism.

\subsection{Settings}

\noindent\textbf{fMRI Datasets} 
We use the widely used Natural Scenes Dataset (NSD) \cite{allen2022massive}, a large-scale dataset consisting of fMRI measurements of $8$ healthy adult subjects. During data collection, subjects viewed images from the MS-COCO dataset \cite{lin2014microsoft} and were instructed to press buttons to indicate whether they had previously seen each image.

% \textit{Instruction datasets}
% COCO Caption \cite{chen2015microsoft}, Paragraph Captioning \cite{krause2017hierarchical},
% COCO QA \cite{ren2015exploring}, 
% VQAv2 \cite{goyal2017making}, Visual Genome \cite{krishna2017visual}, A-OKVQA \cite{schwenk2022okvqa}, ST-VQA \cite{biten2019scene}, OK-VQA \cite{marino2019ok}, \cite{acharya2019tallyqa}, VQA-E \cite{li2018vqa}, FSVQA \cite{shin2016color}
% VisDial \cite{murahari2019improving},
% LVIS-instruct4v \cite{wang2023see}, LLaVA Instruct 150K \cite{liu2023visual}

\noindent\textbf{Downstream Datasets} The downstream dataset will be discussed within each experiment section. See examples and a short description for all dataset we will use in Appendex~\ref{app:dataset}.
Implementation details could be found in Appendix~\ref{app:impl}.

\begingroup
\sisetup{
  table-format=3.2,  % 3 digits before the decimal, 2 after
  table-align-text-pre=false,
  propagate-math-font=true,
  table-number-alignment=center,
  detect-weight=true,detect-inline-weight=math
}
\def\Uline#1{#1\llap{\uline{\phantom{#1}}}}
\begin{table*}[t]\centering
\vspace{-1em}
\caption{Versatile decoding. A dash $-$ means the model could not perform this task. The superscript $^\circ$ means the model is trained from scratch in contrast to their BIT version. The CIDEr metric is scaled by a factor of 100 for consistency with Table~\ref{tab:caption} and baselines.}
\vskip 0.1in
\label{tab:versatile}
\resizebox{\linewidth}{!}{
\begin{tabular}{cc|SSS|SSS|SSS}\toprule
& & {OneLLM} & {UMBRAE} &{BrainChat} & {MindBridge$^\circ$} & {UniBrain$^\circ$} & {\makecell{\name{}}$^\circ$} &{MindBridge} & {UniBrain} & {\makecell{\name{}}} \\
\midrule
subj-agnostic &  & {\xmark}& {\xmark} & {\xmark} & {\cmark} & {\cmark} & {\cmark} &{\cmark} & {\cmark} & {\cmark} \\
% subj-agnostic &  & {\xmark}& {\xmark} & {\xmark} & {\xmark} & {\cmark} & {\cmark} &\xmark & {\cmark} & {\cmark} \\
\midrule
COCO-QA &Accuracy$\uparrow$ &11.09\% & 22.23\% & 39.44\% &40.19\% &38.38\% &42.09\% &\Uline{45.33\%} &42.00\% & \bfseries 48.19\% \\
\midrule
VG-QA &Accuracy$\uparrow$ & 8.76\% & 19.67\% & 21.00\% &20.84\% &21.27\% &21.68\% &23.53\% & \Uline{24.02\%} &\bfseries 24.06\% \\
\midrule
VQA-v2 &Accuracy $\uparrow$& 33.68\% & \Uline{51.23}\% & 40.02\% &43.25\% &46.04\% &44.13\% &47.91\% &48.58\% &\bfseries 52.14\% \\
\midrule
A-OKVQA &Accuracy $\uparrow$&25.23\%& 43.24\% &20.52\% &22.12\% &19.47\% &29.20\% & \Uline{50.44\%} &43.36\% &\bfseries 52.21\% \\
\midrule
ST-VQA &ANLS $\uparrow$& 5.74\%& 5.46\% &9.58\% &10.20\% &7.01\% &\Uline{12.76}\% &11.64\% &8.76\% &\bfseries 12.92\% \\
\midrule
OK-VQA &Accuracy $\uparrow$&22.98\% & 10.35\% &17.22\%&27.63\% &18.63\% &27.70\% &32.13\% &\Uline{32.30\%} & \bfseries 33.33\% \\
\midrule
\multirow{2}{*}{TallyQA} &Accuracy $\uparrow$& 8.34\% & 44.10\% & 43.22\%&43.49\% &44.83\% &43.75\% &49.46\% & \Uline{53.77\%} &\bfseries 54.76\% \\
&RMSE $\downarrow$&7.45 & 3.94 & 1.90 &2.03 &1.83 &2.04 &1.86 &\bfseries 1.67 &\Uline{1.76} \\
\midrule
\multirow{6}{*}{Paragraph Caption} &BLEU-1$\uparrow$ &0.26 & \bfseries 29.82 & 22.21&21.82 &25.69 &26.49 &25.69 &28.28 & \Uline{29.43} \\
&BLEU-2 $\uparrow$ &0.08 & 14.26 &10.23 &10.47 &12.62 &12.48 &13.00 &\Uline{15.47} & \bfseries 15.78 \\
&BLEU-3$\uparrow$ &0.03 & 6.52 &6.38 &5.58 &6.70 &6.43 &7.10 &\Uline{8.90} & \bfseries 9.14 \\
&BLEU-4$\uparrow$ &0.01 & 2.95 &2.12 &3.14 &3.81 &3.63 &4.22 &\bfseries5.60 & \Uline{5.51} \\
&METEOR $\uparrow$&2.36 &12.60 &9.10 &10.95 &11.13 &2.44 &3.56 &\bfseries13.50 & \Uline{13.18} \\
&CIDEr $\uparrow$&0.00 & 7.39 &6.02&7.50 &3.92 &\Uline{10.71} &\bfseries11.39 &1.82 & 7.80 \\
\midrule
\multirow{8}{*}{VQA-E} &Accuracy $\uparrow$ &19.60\%& 47.84\% &46.20\% &45.40\% &44.42\% &44.55\% &\Uline{48.48\%} &48.39\% &\bfseries 50.95\% \\
&BLEU-1 $\uparrow$&17.32& 29.83 & 35.99 &35.63 &35.30 &35.08 &36.18 &\Uline{37.26} &\bfseries 37.70 \\
&BLEU-2 $\uparrow$&7.44& 14.76 & 18.33  &18.27 &18.04 &17.82 &19.38 &\Uline{20.41} &\bfseries 20.56 \\
&BLEU-3 $\uparrow$&3.62& 8.17 & 10.01 &10.32 &10.20 &10.05 &11.30 &\Uline{12.25} &\bfseries 12.34 \\
&BLEU-4 $\uparrow$&1.82& 4.87 &6.60 &6.27 &6.14 &6.00 &7.00 &\Uline{7.83} &\bfseries 7.92 \\
&CIDEr $\uparrow$&19.32& 63.26 &78.33 &79.05 &77.31 &76.80 &86.62 &\Uline{92.09} & \bfseries 93.60 \\
&METEOR $\uparrow$&6.69 & 12.25 &13.64 &14.13 &13.89 &13.96 &14.81 &\Uline{15.51} &\bfseries 15.62 \\
&ROUGE $\uparrow$&16.84& 28.38 & 32.82&33.78 &33.25 &33.11 &34.56 &\Uline{35.87} &\bfseries 35.88 \\
\midrule
\multirow{8}{*}{FSVQA} &VQA Acc. $\uparrow$& 31.44\% & 40.67\% &36.30\% &42.00\% &37.05\% &42.53\% &\Uline{45.95\%} &44.58\% &\bfseries 48.03\% \\
&FSVQA Acc. $\uparrow$& 21.02\% & 0.00\% & 30.22\% &37.40\% &32.30\% &38.50\% &\Uline{40.97\%} &37.87\% &\bfseries 43.00\% \\
&BLEU-1 $\uparrow$& 37.42 & 23.11 & 83.99&85.68 &83.84 &85.88 &\Uline{86.52} &85.10 &\bfseries 87.10 \\
&BLEU-2 $\uparrow$ & 31.72 & 5.86&78.50 &81.27 &78.81 &81.62 &\Uline{82.28} &80.01 &\bfseries 83.03 \\
&BLEU-3 $\uparrow$& 26.95 & 2.10&73.00 &77.10 &73.97 &77.62 &\Uline{78.34} &75.49 &\bfseries 79.27 \\
&BLEU-4 $\uparrow$& 22.48 & 1.04 &69.73 &72.89 &68.91 &73.56 &\Uline{74.35} &70.73 &\bfseries 75.50 \\
&METEOR $\uparrow$& 26.35 & 8.93 &44.76 &47.59 &45.94 &47.96 &\Uline{48.63} &46.89 &\bfseries 49.05 \\
&CIDEr $\uparrow$& 312.75 & 4.07 &600.00 &636.40 &609.00 &646.26 &\Uline{657.02} &628.83 &\bfseries 666.26 \\
\midrule
\multirow{8}{*}{Previous Caption} & BLEU-1 $\uparrow$ & 41.86  & {$-$} & 21.19 & 21.17 & 24.84 & \Uline{44.52} & 42.45 & 43.01 & \bfseries 47.20 \\
&BLEU-2 $\uparrow$ & 19.44 & {$-$} & 8.00 & 7.57 & 9.70 & \Uline{22.46} & 20.04 & 20.03 & \bfseries 25.16\\
&BLEU-3 $\uparrow$ & 9.25 & {$-$} & 1.98 & 2.85 & 3.40 & \Uline{10.39} & 9.61 & 9.19 & \bfseries 12.95\\
&BLEU-4 $\uparrow$ & 3.67 & {$-$} & 1.02& 1.28 & 1.46 & \Uline{5.45} &  5.31 & 4.58 & \bfseries 7.49 \\
&METEOR $\uparrow$ & 10.14 & {$-$} & 6.55& 6.46 & 7.20 & \Uline{11.00} & 10.83 & 10.81 & \bfseries 11.96 \\
&ROUGE $\uparrow$ & 30.19 & {$-$} & 21.23& 20.88 & 23.04 & \Uline{33.20} &32.38 & 31.99 & \bfseries 34.58 \\
&CIDEr $\uparrow$ & 6.65 & {$-$} & 9.21& 8.83 & \Uline{11.73} & 9.39 & 7.89 & 7.53 & \bfseries 16.02 \\
&SPICE $\uparrow$ & 2.49 & {$-$} & 2.44& 2.56 & 2.78 & \Uline{3.07} & 2.80 & 2.92 & \bfseries 3.93 \\
\bottomrule
\end{tabular}}
\vspace{-1em}
\end{table*}
\endgroup

\subsection{Brain Captioning}
To evaluate the model's performance on downstream tasks, we start with the widely used brain captioning benchmark \cite{xia2024umbrae}. The task, built upon COCO Caption \cite{chen2015microsoft} requires the model to predict captions of given images as fMRI stimuli.

\noindent\textbf{Baselines}
The following baselines are considered in this experiment: SDRecon \cite{takagi2023high}, UniBrain \cite{mai2023unibrain}, and BrainCap \cite{ferrante2023brain} employs a linear regression, mapping the fMRI to the inputs of an image caption model \cite{li2023blip}. OneLLM \cite{han2024onellm} is a multimodal large language models that align $8$ modalities (including fMRI) with language all in one model. For fair and efficient comparison, we only finetune the encoder, given that we freeze the LLM in our method as well. UMBRAE learns an encoder that maps fMRIs to images through an encoder similar to the MLP mixer \cite{tolstikhin2021mlp}. BrainChat \cite{huang2024brainchat} segments the flattened voxels into 16 patches and employs a transformer to decode text conditioned on the patches.
It is worth noting that all of these baselines require subject-specific layers or parameters. In contrast, our model is subject-agnostic, thus with the potential to generalize on novel subjects.

\noindent\textbf{Metric} Following previous works, we use five standard metrics for text generation: BLEU-$k$ \cite{papineni2002bleu}, ROUGE-L \cite{lin2004rouge}, CIDEr \cite{vedantam2015cider}, SPICE \cite{anderson2016spice}, METEOR \cite{banerjee2005meteor}.

Table \ref{tab:caption} shows that our model outperforms baselines in terms of most metrics, with an average improvement of $3.32\%$, even if our model does not have any subject-specific layers. We argue that this is attributed to both the novel architecture design and the introduction of BIT, which will be evident in the next experiment.

\begingroup
\sisetup{
  table-format=3.2,  % 3 digits before the decimal, 2 after
  table-align-text-pre=false,
  table-number-alignment=center,
detect-weight=true,detect-inline-weight=math
}
\begin{table}[t]
\vspace{-1em}
\centering
\caption{Model generalization, compared with subject-agnostic model. We train the models on subject $1-7$ and evaluate on subject $8$, which is the held-out subject.}
\label{tab:subj8}
\vskip 0.1in
\resizebox{\linewidth}{!}{
\begin{tabular}{ccSSS}\toprule
& & {MindBridge} & {UniBrain} & {\name{}} \\
\midrule
COCO-QA &Accuracy $\uparrow$&35.88 &24.95 &\bfseries 38.75 \\
\midrule
VG-QA &Accuracy $\uparrow$&\bfseries 20.56 &16.23 & 18.81 \\
\midrule
VQA-v2 &Accuracy $\uparrow$&42.80 &40.16 &\bfseries 44.69 \\
\midrule
A-OKVQA &Accuracy $\uparrow$&44.55 &28.71 &\bfseries 45.54 \\
\midrule
ST-VQA &ANLS $\uparrow$&9.33 &9.30 &\bfseries 10.97 \\
\midrule
OK-VQA &Accuracy $\uparrow$&21.94 &17.09 &\bfseries 24.45 \\
\midrule
\multirow{2}{*}{TallyQA} &Accuracy $\uparrow$&38.92 &32.51 &\bfseries 41.28 \\
&RMSE $\downarrow$&2.12 &\bfseries 2.02 &2.16 \\
\midrule
\multirow{8}{*}{COCO-Caption} &BLEU-1 $\uparrow$&39.84 &41.90 &\bfseries 47.3 \\
&BLEU-2 $\uparrow$&19.55 &19.67 &\bfseries 25.35 \\
&BLEU-3 $\uparrow$&9.29 &8.89 &\bfseries 13.61 \\
&BLEU-4 $\uparrow$&5.24 &4.33 &\bfseries 8.15 \\
&METEOR $\uparrow$&10.39 &10.80 &\bfseries 11.4 \\
&ROUGE $\uparrow$&31.10 &31.54 &\bfseries 34.64 \\
&CIDEr $\uparrow$&\bfseries 8.70 &6.40 &6.41 \\
&SPICE $\uparrow$&2.67 &2.39 &\bfseries 3.61 \\
\midrule
\multirow{6}{*}{Paragraph Caption} &BLEU-1 $\uparrow$&23.18 &21.73 &\bfseries 27.21 \\
&BLEU-2 $\uparrow$&10.71 &8.94 &\bfseries 12.48 \\
&BLEU-3 $\uparrow$&4.61 &3.72 &\bfseries 5.81 \\
&BLEU-4 $\uparrow$&2.22 &1.92 &\bfseries 3.01 \\
&METEOR $\uparrow$&9.99 &9.47 &\bfseries 10.24 \\
&CIDEr $\uparrow$&0.71 &1.56 &\bfseries 4.05 \\
\midrule
\multirow{8}{*}{VQA-E} &Accuracy $\uparrow$&41.78 &38.53 &\bfseries 44.81 \\
&BLEU-1 $\uparrow$&32.54 &32.86 &\bfseries 34.46 \\
&BLEU-2 $\uparrow$&16.13 &15.48 &\bfseries 17.86 \\
&BLEU-3 $\uparrow$&8.82 &7.98 &\bfseries 10.23 \\
&BLEU-4 $\uparrow$&5.16 &4.42 &\bfseries 6.19 \\
&CIDEr $\uparrow$&68.13 &58.79 &\bfseries 77.36 \\
&METEOR $\uparrow$&12.74 &12.26 &\bfseries 13.56 \\
&ROUGE $\uparrow$&30.63 &29.38 &\bfseries 32.99 \\
\midrule
\multirow{8}{*}{FSVQA} &VQA Acc. $\uparrow$&42.33 &37.92 &\bfseries 43.65 \\
&FSVQA Acc. $\uparrow$&37.16 &30.83 &\bfseries 38.41 \\
&BLEU-1 $\uparrow$&75.81 &82.94 &\bfseries 85.69 \\
&BLEU-2 $\uparrow$&71.03 &77.24 &\bfseries 81.18 \\
&BLEU-3 $\uparrow$&66.40 &71.94 &\bfseries 77.06 \\
&BLEU-4 $\uparrow$&61.55 &66.54 &\bfseries 72.86 \\
&METEOR $\uparrow$&45.84 &45.24 &\bfseries 47.87 \\
&CIDEr $\uparrow$&428.39 &587.78 &\bfseries 641.11 \\
\bottomrule
\end{tabular}
}
\vspace{-2em}
\end{table}
\endgroup

\subsection{Versatile Decoding}
The purpose of experiments in this section is two-fold: 1) To investigate the impact of our model design and the introduction of BIT on performance improvement. 2) To evaluate the versatility of the model, i.e., its performance on various downstream tasks.

\noindent\textbf{Baselines} Besides baselines that could be adapted to this experiment from the previous one, we further consider the following subject-agnostic models as baselines.
1) MindBridge \cite{wang2024mindbridge} flatten the voxels and adaptively adjust the padding and stride to pool the voxels into a fixed dimension. The original implementation of MindBridge has subject-specific parameters. However, since those parameters are of the same size, we make them shared across subjects and thus make the model subject-agnostic.
2) UniBrain \cite{wang2024unibrain} samples voxels into a fixed number of groups and employs a transformer where groups are treated as tokens. This UniBrain is unrelated to the UniBrain in the previous section; they just share the same name.

\noindent\textbf{Datasets \& Metric}
We use the test split of all QA \& caption datasets in the BIT dataset. We strictly adhere to the official metrics on all datasets. In summary, for sentence generation, we use BLEU-$k$~\cite{papineni2002bleu}, ROUGE-L~\cite{lin2004rouge}, CIDEr~\cite{vedantam2015cider}, SPICE~\cite{anderson2016spice}, METEOR~\cite{banerjee2005meteor}. For QA-related tasks, we use VQA accuracy~\cite{antol2015vqa} as well as special metrics proposed in the original paper (e.g. ANLS for ST-VQA~\cite{biten2019scene}).

The results are shown in Table~\ref{tab:versatile}. Our model outperforms baselines, with an average improvement of $12.0\%$. Further, by comparing instruction tuning and from-scratch models, we find that instruction tuning has a significant positive effect, with an average improvement of $28.0\%$. The results remain stable across different random seeds; for instance, according to our observations, the BLEU-1 score for paragraph captioning exhibits a maximum of $\pm 0.3$ variance.

\begingroup
\sisetup{
  table-format=2.2,  % 3 digits before the decimal, 2 after
  table-align-text-pre=false,
  table-number-alignment=center,
detect-weight=true,detect-inline-weight=math
}
\begin{table}[htbp]
\vspace{-1.5em}
    \caption{Model adaptation to new tasks. \textit{sentiment understanding} and \textit{utility/affordance} are sub-datasets from TDIUC that are particularly relevant to BCI applications.}
    \label{tab:new_task}
\vskip 0.1in
    \centering
    \resizebox{\linewidth}{!}{
    \begin{tabular}{cSSSS}
    \toprule
    \multirow{2}{*}{Method} &  \multicolumn{2}{c}{Overall} & {Sentiment Understanding} & {Utility/Affordance} \\
    \cmidrule{2-3} \cmidrule{4-4} \cmidrule{5-5}
    & {A-MPT} & {H-MPT} & {Accuracy} & {Accuracy} \\
    \midrule
    \name{}$^\circ$ & 41.09\% & 19.38\% & 70.00\% & 0.00\%\\
    \midrule
         MindBridge & 49.77\% & 39.88\% & 80.00\% & 14.29\%\\
         UniBrain & 51.50\% & 36.76\% & 80.00\% & 28.57\%\\
         \name{}& \bfseries 54.08\% &  \bfseries 45.43\% & \bfseries 80.77\% & \bfseries 50.00\% \\
    \bottomrule
    \end{tabular}
    }
    \vspace{-1em}
\end{table}
\endgroup


\subsection{Unseen Subject Generalization}
Our neuroscience-informed, subject-agnostic design enhances generalization to novel subjects, a crucial factor in real-world applications where training a model for each individual is impractical. To evaluate it, we perform instruction tuning on $7$ out of the $8$ subjects in the natural scene dataset \cite{allen2022massive}, and evaluate generalization on the held-out subject. Table~\ref{tab:subj8} shows our model outperforms two other subject-agnostic baselines in most cases, with an average improvement of $16.4\%$ compared to the second-best model.

\subsection{Adapting to New Tasks}
It is common that users want to adapt the \name{} to their own specific use cases. To this end, we aim to assess our model's adaptability to new tasks.

\noindent\textbf{Dataset \& Metrics} We use TDIUC \cite{kafle2017analysis}, a QA dataset consisting of $12$ types of questions, as a benchmark to evaluate the model's various capabilities comprehensively. Additionally, we further select $2$ task types-\textit{sentiment understanding} and \textit{utility/affordance} tasks, that are particularly relevant to BCI applications as sub-datasets. The \textit{utility/affordance} task, for instance, enables the model to identify useful objects in a given scene and autonomously decide whether to utilize them. Following their paper, we compute the accuracy of each type and report the arithmetic mean-per-type (A-MPT) and the harmonic mean-per-type (H-MPT). For the $2$ selected types, we report the accuracy respectively. 

Table~\ref{tab:new_task} shows our model achieves balanced (high harmonic mean) and consistently improved performances with an average of $13.5\%$. We could also observe the performance benefits from BIT, with $25.0\%$ absolute improvement.

\begin{figure*}[t]
    \begin{subfigure}[c]{0.15\textwidth}
        \centering
        \includegraphics[width=\textwidth]{figures/brain_query/brain_query_ppa.pdf}
        % \caption{Query: PPA}
        \vspace{-1.5em}
        \caption{}
        \label{sub:ppa}
    \end{subfigure}
    \hfill
    \begin{subfigure}[c]{0.15\textwidth}
        \centering
        \includegraphics[width=\textwidth]{figures/brain_query/brain_query_ffa_1.pdf}
        % \caption{Query: FFA-1}
        \vspace{-1.5em}
                \caption{}
        \label{sub:ffa-1}
    \end{subfigure}
    \hfill
    \begin{subfigure}[c]{0.15\textwidth}
        \centering
        \includegraphics[width=\textwidth]{figures/brain_query/brain_query_opa_ofa.pdf}
        % \caption{Query: FFA-1}
        \vspace{-1.5em}
        \caption{}
        \label{sub:opa_ofa}
    \end{subfigure}
    \hfill
    \begin{subfigure}[c]{0.15\textwidth}
        \centering
        \includegraphics[width=\textwidth]{figures/brain_query/brain_query_earlyvis_eba.pdf}
        % \caption{Query: FFA-1}
        \vspace{-1.5em}
        \caption{}
        \label{sub:ev_eba}
    \end{subfigure}
    \hfill
    \begin{subfigure}[c]{0.15\textwidth}
        \centering
        \includegraphics[width=\textwidth]{figures/brain_query/brain_query_multi_low_high.pdf}
        % \caption{Query: FFA-1}
        \vspace{-1.5em}
        \caption{}
        \label{sub:low_high}
    \end{subfigure}
    \hfill
    \begin{subfigure}[c]{0.15\textwidth}
        \centering
        \includegraphics[width=\textwidth]{figures/brain_query/brain_query_multi_high_high.pdf}
        % \caption{Query: FFA-1}
        \vspace{-1.5em}
        \caption{}
        \label{sub:high_high}
    \end{subfigure}
    \hfill
    \begin{subfigure}[c]{0.05\textwidth}
        \centering
        \includegraphics[width=\textwidth]{figures/brain_query/colorbar.png}
    \end{subfigure}
    \vspace{-1.4em}
    \caption{Visaulization of attention weights between \textit{queries} and brain voxels. Each subfigure represents a \textit{query} token, and the strength of color indicates its attention weight (after min-max normalization) to each voxel. 
    }
    \label{fig:query}
    \vspace{-1.2em}
\end{figure*}

\subsection{Ablation Study}
We conduct ablation studies on the design of key embeddings in the neuroscience-informed attention module in Figure~\ref{fig:ablation}. The results strongly validate our design. The vanilla cross attention (\textit{Pos Enc.+fMRI}) leads to poor performance while removing fMRI values from the key embeddings (\textit{Pos Enc.}) yields a significant improvement. Replacing positional encoding with region encodings (\textit{Reg. Enc.}) accelerates convergence in the early stages since it is grounded by neuroscientific principles. However, it is eventually outperformed by \textit{Pos Enc.} due to the lack of finer-grained information. Combining the positional encoding and region encodings (\textit{Pos Enc.+Reg Enc.}) achieves the best model design. In addition, replacing positional encoding with an MLP that maps coordinates to embeddings results in poor performance (\textit{(x,y,z)+MLP}), which indicates the amount of high-frequency spatial information in fMRI signals.

\begin{figure}[h]
\vspace{-0.8em}
    \centering
    \includegraphics[width=\linewidth]{figures/ablation.pdf}
    \vspace{-2em}
    \caption{Ablation study of the key embedding design. Pos Enc. stands for positional encoding. Reg Enc. stands for multiple region encodings. \textit{(x, y, z)+MLP} means we employ an MLP to map the coordinates to the embeddings instead of positional encoding.}
    \label{fig:ablation}
    \vspace{-1.3em}
\end{figure}
\subsection{Visualizations and Interpretations}
\label{sec:vis}



% \begin{figure}
%     \centering
%     \includegraphics[width=\linewidth]{figures/fig_brain.pdf}
%     \caption{Voxel-Level Brain Mapping of Model Attention.(a)-(c) present the attention map on a flattened brain with deeper blue indicating higher value. (d) gives an average value rank of different brain regions from (a).}
%     \label{fig:brain}
% \end{figure}

% Interpretability is of high importance in brain-inspired research, providing critical cues for how to leverage brain information more efficiently in the future and how to correctly transfer models into the real-world application \cite{fellous2019explainable,farahani2022explainable}. Other than compared advanced brain-decoding methods in this paper, our neuroscience-informed framework is interpretable on the brain-level. Accordingly, we conducted analysis in this section to show our model’s capability in distilling complex voxel-level brain signals and locating informative regions for high-level semantic tasks. We extracted the average attention feature from $\mathbf{Q}$ and $\mathbf{K}$ before it interacted with voxel values $\mathbf{V}$ and mapped it back to the original 3D brain cortex, as Figure \ref{fig:brain} shows. Notably, compared to early visual areas, brain regions from higher-level information processing stages caught much more attention of the model. This pattern is different from previous brain-imaging decoding methods with interpretations, such as \cite{takagi2023high} and (), in which early visual regions were among the most notable brain areas. Early visual regions are important, initially processing visual information, passing it to next stages, yet not crucial for high-level cognitive conceptualization. MindLLM recognized a map mainly focusing on Parahippocampal Place Area (PPA), Fusiform Face Area (FFA) and Extrastriate Body Area (EBA). PPA locates in parahippocampal gyrus, related to conceptual association, semantic processing and enviornmental memory \cite{epstein1999parahippocampal,kohler2002differential,bar2008scenes,epstein2010reliable}. FFA is known for its critical role in expertise recognition, social cognition and identity memory \cite{schultz2003role,tsantani2021ffa,xu2005revisiting}. And EBA involves body perception and contextual reasoning \cite{urgesi2007representation,carey2019distinct}. Together, they form a pattern of processing and extracting distilled sophisticated objects and spatial information from outside world, producing conceptual semantic cognition, and even inferring hidden messages behind the visual scenes from the information human gets \cite{amoruso2011beyond}. This revealed pattern proves model’s ability in getting high-level accurate information from the brain and provides evidence about potential brain regions for multitask brain decoding as well as the possibility of implementing better brain decoding model design with richer brain signals.

% The design of the learnable \textit{queries} provides a way to dynamically integrate massive brain-anatomical information from the \textit{keys}. 
Unlike previous deep learning models \cite{scotti2024mindeye2,mai2023unibrain}, our model allows interpretations by investigating how \textit{queries} work in the neuroscience-informed attention layer. We inspect the attention weights between queries and voxels in Figure~\ref{fig:query}. 
% Weight values are scaled by min-max normalization and results are presented in Figure \ref{fig:query}. 

We found that some queries primarily focus on processing single brain regions, such as Parahippocampal Place Area (PPA) (Figure~\ref{sub:ppa}) and Fusiform Face Area (FFA) (\ref{sub:ffa-1}). As previous research has shown, PPA is related to conceptual association, semantic processing and environmental memory \cite{epstein1999parahippocampal,kohler2002differential,bar2008scenes,epstein2010reliable} and FFA is known for its critical role in expertise recognition, social cognition and identity memory \cite{schultz2003role,tsantani2021ffa,xu2005revisiting}. Both are important brain regions for the conceptualization of visual information and are responsible for the interaction between real-time stimulus and past memory \cite{brewer1998making,ranganath2004category,golarai2007differential}. 

Moreover, there are some queries that attend to multiple brain regions, revealing the information transmission between low- and high-level brain regions. For instance, interactions between early visual areas and higher-level regions like PPA and IntraParietal Sulcus (IPS) (Figure~\ref{sub:low_high}), revealing a potential pattern for human attention-guided actions \cite{tunik2007beyond,connolly2016coding}. Additionally, queries are also found responsible for communications between high-level brain regions (Figure \ref{sub:opa_ofa},\ref{sub:high_high}). Together, these findings indicate that the learnable queries may reflect the dynamics of human brain activities in the visual task, from seeing and thinking about the image to pressing the button for the visual recall task in NSD \cite{allen2022massive}. 

% This emphasizes the advantage of modeling interactions between brain-informed keys and queries in the context of parallel multitask working\rex{what is this multitask working?}.

% As brain regions defined in neuroscience research likely to be processed individually and interactively in our brain-decoding pipeline\rex{i don't understand. doesn't seem like a useful sentence}, our model is potentially aware of the biological properties of the human brain rather than mechanically mixing all values up in the black box, like \rex{we haven't compared. how do we support such claims?} \cite{scotti2024reconstructing,scotti2024mindeye2,jiang2024mindshot}.

We also provide qualitative analysis of model responses in Appendix~\ref{app:qual}.

% \begin{figure}
%     \centering
%     \includegraphics[width=\linewidth]{figures/brain_query_slice.pdf}
%     \caption{Slices of Model Attention by Query Index.}
%     \label{fig:query}
% \end{figure}
\section{Conclusion}

We presented \method, a novel framework leveraging pretrained VLMs to unify temporal, visual, and textual modalities for time series forecasting. By integrating the RAL, VAL, and TAL, \method bridges modality gaps, enabling rich cross-modal interactions. Extensive experiments demonstrate state-of-the-art performance across various datasets, especially in few-shot and zero-shot scenarios, outperforming existing methods while maintaining efficiency. Our work establishes a new direction for multimodal time series forecasting, highlighting the potential of VLMs in capturing temporal dynamics and semantic context.

Notably, \method operates can solely on original time series data without external information, ensuring fair comparisons and showcasing its ability to generate textual and visual representations directly from the data for self-augmentation. This design not only enhances accuracy but also emphasizing the framework's robustness, particularly in domains where external data is scarce or unavailable.

Future work may explore adaptive visual transformations for complex patterns, enhancing text utilization, extending to multi-task, and developing more efficient multimodal time series foundation models. For details, see \shortautoref{appx:future_work}.


\section*{Impact Statement}

This paper presents work whose goal is to advance the field of Machine Learning by integrating temporal, visual, and textual modalities for time series forecasting. While our approach improves accuracy and cross-domain generalization, we acknowledge potential risks such as data privacy concerns, algorithmic bias, and increased computational costs. We encourage further research into mitigating these risks to ensure responsible deployment in high-stakes applications.


% In the unusual situation where you want a paper to appear in the
% references without citing it in the main text, use \nocite
% \nocite{langley00}

\bibliography{example_paper}
\bibliographystyle{icml2025}


%%%%%%%%%%%%%%%%%%%%%%%%%%%%%%%%%%%%%%%%%%%%%%%%%%%%%%%%%%%%%%%%%%%%%%%%%%%%%%%
%%%%%%%%%%%%%%%%%%%%%%%%%%%%%%%%%%%%%%%%%%%%%%%%%%%%%%%%%%%%%%%%%%%%%%%%%%%%%%%
% APPENDIX
%%%%%%%%%%%%%%%%%%%%%%%%%%%%%%%%%%%%%%%%%%%%%%%%%%%%%%%%%%%%%%%%%%%%%%%%%%%%%%%
%%%%%%%%%%%%%%%%%%%%%%%%%%%%%%%%%%%%%%%%%%%%%%%%%%%%%%%%%%%%%%%%%%%%%%%%%%%%%%%
\newpage
\appendix
\onecolumn

\section{Experimental Details}
\subsection{Dataset Details}
\label{appx:dataset_details} 
\begin{table}[htbp]
  \caption{Summary of the benchmark datasets. Each dataset contains multiple time series (Dim.) with different sequence lengths, and is split into training, validation and testing sets. The data are collected at different frequencies across various domains.}
  \label{tab:dataset}
  \centering
  \scalebox{0.99}{
  \begin{tabular}{l|c|c|c|c|c}
    \toprule
Dataset & Dim. & Series Length & Dataset Size & Frequency & Domain \\
\toprule
ETTm1 & 7 & \{96, 192, 336, 720\} & (34465, 11521, 11521)  & 15 min & Temperature \\
ETTm2 & 7 & \{96, 192, 336, 720\} & (34465, 11521, 11521)  & 15 min & Temperature \\
ETTh1 & 7 & \{96, 192, 336, 720\} & (8545, 2881, 2881) & 1 hour & Temperature \\
ETTh2 & 7 & \{96, 192, 336, 720\} & (8545, 2881, 2881) & 1 hour & Temperature \\ 
Electricity & 321 & \{96, 192, 336, 720\} & (18317, 2633, 5261) & 1 hour & Electricity \\ 
Traffic & 862 & \{96, 192, 336, 720\} & (12185, 1757, 3509) & 1 hour & Transportation \\ 
Weather & 21 & \{96, 192, 336, 720\} & (36792, 5271, 10540) & 10 min 
& Weather \\

\bottomrule
\end{tabular}
}
\end{table}


% \begin{table}[htbp]
%   \caption{The dimension indicates the number of time series (i.e., channels)}
%   \vspace{1em}
%   \label{tab:dataset}
%   \centering
%   \begin{small}
%   \scalebox{0.9}{
%   \begin{tabular}{c|l|c|c|c|c|c}
%     \toprule
%     Tasks & Dataset & Dim. & Series Length & Dataset Size & Frequency & Domain \\
%     \toprule
%      & ETTm1 & 7 & \{96, 192, 336, 720\} & (34465, 11521, 11521)  & 15 min & Temperature \\
%     \cmidrule{2-7}
%     Long-term & ETTm2 & 7 & \{96, 192, 336, 720\} & (34465, 11521, 11521)  & 15 min & Temperature \\
%      \cmidrule{2-7}
%      Forecasting & ETTh1 & 7 & \{96, 192, 336, 720\} & (8545, 2881, 2881) & 1 hour & Temperature \\
%      \cmidrule{2-7}
%      & ETTh2 & 7 & \{96, 192, 336, 720\} & (8545, 2881, 2881) & 1 hour & Temperature \\ 
%      \cmidrule{2-7}
%      & Electricity & 321 & \{96, 192, 336, 720\} & (18317, 2633, 5261) & 1 hour & Electricity \\ 
%      \cmidrule{2-7}
%      & Traffic & 862 & \{96, 192, 336, 720\} & (12185, 1757, 3509) & 1 hour & Transportation \\ 
%      \cmidrule{2-7}
%      & Weather & 21 & \{96, 192, 336, 720\} & (36792, 5271, 10540) & 10 min & Weather \\
%      \cmidrule{2-7}
%      & Illness & 7 & \{24, 36, 48, 60\} & (617, 74, 170) & 1 week & Illness \\
%     \midrule
%     & M3-Quarterly & 1 & 8 & (756, 0, 756) & Quarterly & Multiple \\
%     \cmidrule{2-7}
%     & M4-Yearly & 1 & 6 & (23000, 0, 23000) & Yearly & Demographic \\
%     \cmidrule{2-7}
%     & M4-Quarterly & 1 & 8 & (24000, 0, 24000) & Quarterly & Finance \\
%     \cmidrule{2-7}
%     Short-term & M4-Monthly & 1 & 18 & (48000, 0, 48000) & Monthly & Industry \\
%      \cmidrule{2-7}
%     Forecasting & M4-Weakly & 1 & 13 & (359, 0, 359) & Weekly & Macro \\
%      \cmidrule{2-7}
%      & M4-Daily & 1 & 14 & (4227, 0, 4227) & Daily & Micro \\
%      \cmidrule{2-7}
%      & M4-Hourly & 1 & 48 & (414, 0, 414) & Hourly & Other \\
%     \bottomrule
%     \end{tabular}
%     }
%     \end{small}
% \end{table}

We conduct experiments on the above real-world datasets to evaluate the performance of our proposed model and follow the same data processing and train-validation-test set split protocol used in TimesNet benchmark~\cite{wu2022timesnet}, ensuring a strict chronological order to prevent data leakage. Different datasets require specific adjustments to accommodate their unique characteristics:
\label{appx:dataset_configurations}

\paragraph{ETT Dataset~\cite{kim2021reversible}} The Electricity Transformer Temperature (ETT) dataset consists of both hourly (ETTh) and 15-minute (ETTm) frequency data, with 7 variables ($enc\_{in}$ = $dec\_{in}$ = $c\_{out}$ = 7) measuring transformer temperatures and related factors. For ETTh data, we set periodicity to 24 with hourly frequency, while ETTm data uses a periodicity of 96 with 15-minute intervals. Standard normalization is applied to each feature independently, and the model maintains the same architectural configuration across both temporal resolutions.

\paragraph{Traffic Dataset~\cite{wu2022timesnet}} The traffic flow dataset represents a high-dimensional scenario with 862 variables capturing traffic movements across different locations. To handle this scale, we implement gradient checkpointing and efficient attention mechanisms, complemented by progressive feature loading. The batch size is dynamically adjusted based on available GPU memory, and we maintain a periodicity of 24 to capture daily patterns. Our model employs specialized memory optimization techniques to process this large feature space efficiently.

\paragraph{ECL Dataset~\cite{wu2021autoformer}} The electricity consumption dataset contains 321 variables monitoring power usage patterns. We employ robust scaling techniques to handle outliers and implement sophisticated missing value imputation strategies. The model incorporates adaptive normalization layers to address the varying scales of electricity consumption across different regions and time periods. The daily periodicity is preserved through careful temporal encoding, while the high feature dimensionality is managed through efficient attention mechanisms.

\paragraph{Weather Dataset~\cite{wu2021autoformer}} This multivariate dataset encompasses 21 weather-related variables, each with distinct physical meanings and scale properties. Our approach implements feature-specific normalization to handle the diverse variable ranges while maintaining their physical relationships. The model captures both daily and seasonal patterns through enhanced temporal encoding, with special attention mechanisms designed to model the complex interactions between different weather variables. We maintain consistent prediction quality across all variables through carefully calibrated cross-attention mechanisms.


\subsection{Optimization Settings}
\label{appx:optimization_settings}

\subsubsection{Model Architecture Parameters}
\label{appx:model_parameters}
\begin{table}[htbp]
\centering
\caption{Default Model Architecture Parameters}
\begin{tabular}{|l|l|p{5.5cm}|}
\hline
\textbf{Parameter} & \textbf{Default Value} & \textbf{Description} \\
\hline
\multicolumn{3}{|l|}{\textit{Visual Representation Parameters}} \\
\hline
image\_size & 64 & Size of generated image representation \\
patch\_size & 16 & Size of patches for input processing \\
grayscale & True & Whether to use grayscale images \\
\hline
\multicolumn{3}{|l|}{\textit{Diffusion Process Parameters}} \\
\hline
num\_timesteps & 300 & Number of diffusion training steps \\
inference\_steps & 50 & Number of inference steps \\
beta\_start & 0.00085 & Initial value of noise schedule \\
beta\_end & 0.012 & Final value of noise schedule \\
use\_ddim & True & Whether to use DDIM sampler \\
unet\_layers & 1 & Number of layers in UNet \\
\hline
\multicolumn{3}{|l|}{\textit{Model Architecture Parameters}} \\
\hline
d\_model & 256 & Dimension of model hidden states \\
d\_ldm & 256 & Hidden dimension of LDM \\
d\_fusion & 256 & Dimension of gated fusion module \\
e\_layers & 2 & Number of encoder layers \\
d\_layers & 1 & Number of decoder layers \\
\hline
\multicolumn{3}{|l|}{\textit{Training Configuration}} \\
\hline
freeze\_ldm & True & Whether to freeze LDM parameters \\
save\_images & False & Whether to save generated images \\
output\_type & full & Type of output for ablation study \\
\hline
\end{tabular}
\end{table}
The core architecture of our diffusion-based model consists of several key components, each with specific parameter settings. The autoencoder pathway is configured with an image size of $64\times64$ and a patch size of $16$, providing an efficient latent representation while maintaining temporal information. The diffusion process uses $1000$ timesteps with carefully tuned noise scheduling ($\beta_{start} = 0.00085, \beta_{end} = 0.012$) to ensure stable training.

For the transformer backbone, we employ a configuration with $d_model = 256$ and $8$ attention heads, which empirically shows strong performance across different datasets. The encoder-decoder structure uses $2$ encoder layers and $1$ decoder layer, with a feed-forward dimension of $768$, striking a balance between model capacity and computational efficiency.

\subsubsection{Training Parameters}
\label{appx:training_settings}
\begin{table}[htbp]
\centering
\caption{Default Training Parameters}
\begin{tabular}{|l|l|p{5.5cm}|}
\hline
\textbf{Parameter} & \textbf{Default Value} & \textbf{Description} \\
\hline
\multicolumn{3}{|l|}{\textit{Basic Training Parameters}} \\
\hline
batch\_size & 32 & Number of samples per training batch \\
learning\_rate & 0.001 & Initial learning rate for optimization \\
train\_epochs & 10 & Total number of training epochs \\
patience & 3 & Epochs before early stopping \\
loss & MSE & Type of loss function \\
label\_len & 48 & Length of start token sequence \\
seq\_len & 96 & Length of input sequence \\
norm\_const & 0.4 & Coefficient for normalization \\
padding & 8 & Size of sequence padding \\
stride & 8 & Step size for sliding window \\
pred\_len & \makecell[l]{96/192/336/720} & Available prediction horizons \\
\hline
\multicolumn{3}{|l|}{\textit{Dataset-specific Parameters}} \\
\hline
c\_out & \makecell[l]{7 (ETTh1/h2/m1/m2) \\ 21 (Weather) \\ 321 (Electricity) \\ 862 (Traffic)} & Dataset-specific output dimensions \\
\hline
periodicity & \makecell[l]{24 (ETTh1/h2/Electricity/Traffic) \\ 96 (ETTm1/m2) \\ 144 (Weather)} & Natural cycle length per dataset \\
\hline
\end{tabular}
\end{table}
We adopt a comprehensive training strategy with both general and task-specific parameters. The model is trained with a batch size of $32$ and an initial learning rate of $0.001$, using the \textit{AdamW} optimizer. Early stopping with a patience of $3$ epochs is implemented to prevent over-fitting. For time series processing, we use a sequence length of $96$ and a prediction length of $96$, with a label length of 48 for teacher forcing during training.

The training process employs automatic mixed precision (AMP) when available to accelerate training while maintaining numerical stability. We use MSE as the primary loss function, supplemented by additional regularization terms for specific tasks.

\subsection{Evaluation Metrics}
\label{appx:metric}
For evaluation metrics, we utilize the mean square error (MSE) and mean absolute error (MAE) for long-term forecasting. 
% In terms of the short-term forecasting on M4 benchmark, we adopt the symmetric mean absolute percentage error (SMAPE), mean absolute scaled error (MASE), and overall weighted average (OWA) as in N-BEATS \citep{oreshkin2019n}. Note that OWA is a specific metric utilized in the M4 competition. 
The calculations of these metrics are as follows:
\begin{align*} \label{equ:metrics}
    \text{MSE} &= \frac{1}{H}\sum_{h=1}^T (\mathbf{Y}_{h} - \Hat{\mathbf{Y}}_{h})^2,
    &
    \text{MAE} &= \frac{1}{H}\sum_{h=1}^H|\mathbf{Y}_{h} - \Hat{\mathbf{Y}}_{h}|,\\
    % \text{SMAPE} &= \frac{200}{H} \sum_{h=1}^H \frac{|\mathbf{Y}_{h} - \Hat{\mathbf{Y}}_{h}|}{|\mathbf{Y}_{h}| + |\Hat{\mathbf{Y}}_{h}|},
    % &
    % \text{MAPE} &= \frac{100}{H} \sum_{h=1}^H \frac{|\mathbf{Y}_{h} - \Hat{\mathbf{Y}}_{h}|}{|\mathbf{Y}_{h}|}, \\
    % \text{MASE} &= \frac{1}{H} \sum_{h=1}^H \frac{|\mathbf{Y}_{h} - \Hat{\mathbf{Y}}_{h}|}{\frac{1}{H-s}\sum_{j=s+1}^{H}|\mathbf{Y}_j - \mathbf{Y}_{j-s}|},
    % &
    % \text{OWA} &= \frac{1}{2} \left[ \frac{\text{SMAPE}}{\text{SMAPE}_{\textrm{Naïve2}}}  + \frac{\text{MASE}}{\text{MASE}_{\textrm{Naïve2}}}  \right],
\end{align*}
where $s$ is the periodicity of the time series data. $H$ denotes the number of data points (i.e., prediction horizon in our cases). $\mathbf{Y}_{h}$ and $\Hat{\mathbf{Y}}_{h}$ are the $h$-th ground truth and prediction where $h \in \{1, \cdots, H\}$.

\section{Details of Baseline Methods}
\label{appx:baselines}
We compare our approach with three categories of baseline methods used for comparative evaluation: transformer-based architectures, diffusion-based models, and other competitive approaches for time series forecasting.
\paragraph{Transformer-based Models:}
\textbf{FEDformer~\cite{zhou2022fedformer}} integrates wavelet decomposition with a Transformer architecture to efficiently capture multi-scale temporal dependencies by processing both time and frequency domains. 
\textbf{Autoformer~\cite{wu2021autoformer}} introduces a decomposing framework that separates the time series into trend and seasonal components, employing an autocorrelation mechanism for periodic pattern extraction.
\textbf{ETSformer~\cite{woo2022etsformer}} extends the classical exponential smoothing method with a Transformer architecture, decomposing time series into level, trend, and seasonal components while learning their interactions through attention mechanisms.
\textbf{Informer~\cite{zhou2021informer}} addresses the quadratic complexity issue of standard attention mechanisms through ProbSparse self-attention, which enables efficient handling of long input sequences.
\textbf{Reformer~\cite{kitaev2020reformer}} optimizes attention computation via Locality-Sensitive Hashing (LSH) and reversible residual networks, significantly reducing memory and computational costs.
\paragraph{Diffusion-based Models:}
\textbf{CSDI~\cite{tashiro2021csdi}} is tailored for irregularly-spaced time series, learning a score function of noise distribution under given conditions to generate samples for forecasting.
\textbf{ScoreGrad~\cite{song2020score}} utilizes a continuous-time framework for progressive denoising from Gaussian noise to reconstruct the original signal, allowing for adjustable step sizes during the denoising process.
\paragraph{Other Competitive Models:}
\textbf{DLinear~\cite{zeng2023transformers}} proposes a linear transformation approach directly on time series data, simplifying the prediction process under the assumption of linear changes over time.
\textbf{TimesNet~\cite{wu2022timesnet}} focuses on multi-scale feature extraction using various convolution kernels to capture temporal dependencies of different lengths, automatically selecting the most suitable feature scales.
\textbf{LightTS~\cite{campos2023lightts}} aims to build lightweight time series forecasting models, streamlining structures and parameters to reduce computational resource requirements while maintaining high predictive performance.

Each baseline method represents distinct paradigms within probabilistic generative modeling, attention-based architectures, and linear models, providing a comprehensive benchmark against which to evaluate LDM4TS.
\section{Complete results}

\subsection{Few-shot Forecasting}
\label{appx:few-shot}
\begin{table*}[h!]
\captionsetup{font=small} 
\caption{Full few-shot learning results on 5\% training data with forecasting horizons $H \in $\{96, 192, 336, 720\}. A lower value indicates better performance. '-' means that 5\% time series is not sufficient to constitute a training set. {\boldres{Red}}: the best, \secondres{Blue}: the second best}
\vspace{-1em}
\label{tab:few-shot-forecasting-5per-full}
\begin{center}
\begin{small}
\scalebox{0.65}{
\setlength\tabcolsep{3pt}
\begin{tabular}{c|c|cc|cc|cc|cc|cc|cc|cc|cc|cc|cc|cc|cc|cc}
\toprule

\multicolumn{2}{c|}{Methods}&\multicolumn{2}{c|}{\method{}}&\multicolumn{2}{c|}{Time-LLM}&\multicolumn{2}{c|}{GPT4TS}&\multicolumn{2}{c|}{DLinear}&\multicolumn{2}{c|}{PatchTST}&\multicolumn{2}{c|}{TimesNet}&\multicolumn{2}{c|}{FEDformer}&\multicolumn{2}{c|}{Autoformer}&\multicolumn{2}{c|}{Stationary}&\multicolumn{2}{c|}{ETSformer}&\multicolumn{2}{c|}{LightTS}&\multicolumn{2}{c|}{Informer}&\multicolumn{2}{c}{Reformer} \\

\midrule

\multicolumn{2}{c|}{Metric} & MSE  & MAE & MSE & MAE& MSE & MAE& MSE  & MAE& MSE  & MAE& MSE  & MAE& MSE  & MAE& MSE  & MAE& MSE  & MAE& MSE  & MAE& MSE  & MAE& MSE  & MAE& MSE  & MAE\\
\midrule

\multirow{5}{*}{\rotatebox{90}{$ETTh1$}}
& 96  & \boldres{0.417} & \boldres{0.435} & \secondres{0.483} & \secondres{0.464} & 0.543 & 0.506 & 0.547 & 0.503 & 0.557 & 0.519 & 0.892 & 0.625 & 0.593 & 0.529 & 0.681 & 0.570 & 0.952 & 0.650 & 1.169 & 0.832 & 1.483 & 0.910 & 1.225 & 0.812 & 1.198 & 0.795\\
& 192 & \boldres{0.450} & \boldres{0.458} & \secondres{0.629} & \secondres{0.540} & 0.748 & 0.580 & 0.720 & 0.604 & 0.711 & 0.570 & 0.940 & 0.665 & 0.652 & 0.563 & 0.725 & 0.602 & 0.943 & 0.645 & 1.221 & 0.853 & 1.525 & 0.930 & 1.249 & 0.828 & 1.273 & 0.853\\
& 336 & \boldres{0.460} & \boldres{0.465} & 0.768 & 0.626 & 0.754 & 0.595 & 0.984 & 0.727 & 0.816 & 0.619 & 0.945 & 0.653 & \secondres{0.731} & \secondres{0.594} & 0.761 & 0.624 & 0.935 & 0.644 & 1.179 & 0.832 & 1.347 & 0.870 & 1.202 & 0.811 & 1.254 & 0.857\\
& 720 & - & - & - & - & - & - & - & - & - & - & - & - & - & - & - & - & - & - & - & - & - & - & - & - & - & -\\
& Avg & \boldres{0.442} & \boldres{0.453} & 0.627 & 0.543 & 0.681 & \secondres{0.560} & 0.750 & 0.611 & 0.694 & 0.569 & 0.925 & 0.647 & \secondres{0.658} & 0.562 & 0.722 & 0.598 & 0.943 & 0.646 & 1.189 & 0.839 & 1.451 & 0.903 & 1.225 & 0.817 & 1.241 & 0.835\\
\midrule

\multirow{5}{*}{\rotatebox{90}{$ETTh2$}}
& 96  & \boldres{0.302} & \boldres{0.365} & \secondres{0.336} & \secondres{0.397} & 0.376 & 0.421 & 0.442 & 0.456 & 0.401 & 0.421 & 0.409 & 0.420 & 0.390 & 0.424 & 0.428 & 0.468 & 0.408 & 0.423 & 0.678 & 0.619 & 2.022 & 1.006 & 3.837 & 1.508 & 3.753 & 1.518\\
& 192 & \boldres{0.361} & \boldres{0.406} & \secondres{0.406} & \secondres{0.425} & 0.418 & 0.441 & 0.617 & 0.542 & 0.452 & 0.455 & 0.483 & 0.464 & 0.457 & 0.465 & 0.496 & 0.504 & 0.497 & 0.468 & 0.845 & 0.697 & 3.534 & 1.348 & 3.975 & 1.933 & 3.516 & 1.473\\
& 336 & \boldres{0.398} & \boldres{0.434} & \secondres{0.405} & \secondres{0.432} & 0.408 & 0.439 & 1.424 & 0.849 & 0.464 & 0.469 & 0.499 & 0.479 & 0.477 & 0.483 & 0.486 & 0.496 & 0.507 & 0.481 & 0.905 & 0.727 & 4.063 & 1.451 & 3.956 & 1.520 & 3.312 & 1.427\\
& 720 & - & - & - & - & - & - & - & - & - & - & - & - & - & - & - & - & - & - & - & - & - & - & - & - & - & -\\
& Avg & \boldres{0.354} & \boldres{0.402} & \secondres{0.382} & \secondres{0.418} & 0.400 & 0.433 & 0.694 & 0.577 & 0.827 & 0.615 & 0.439 & 0.448 & 0.463 & 0.454 & 0.441 & 0.457 & 0.470 & 0.489 & 0.809 & 0.681 & 3.206 & 1.268 & 3.922 & 1.653 & 3.527 & 1.472\\
\midrule

\multirow{5}{*}{\rotatebox{90}{$ETTm1$}}
& 96  & \boldres{0.314} & \boldres{0.357} & \secondres{0.316} & 0.377 & 0.386 & 0.405 & 0.332 & \secondres{0.374} & 0.399 & 0.414 & 0.606 & 0.518 & 0.628 & 0.544 & 0.726 & 0.578 & 0.823 & 0.587 & 1.031 & 0.747 & 1.048 & 0.733 & 1.130 & 0.775 & 1.234 & 0.798\\
& 192 & \boldres{0.343} & \boldres{0.373} & 0.450 & 0.464 & 0.440 & 0.438 & 0.358 & 0.390 & 0.441 & \secondres{0.436} & 0.681 & 0.539 & 0.666 & 0.566 & 0.750 & 0.591 & 0.844 & 0.591 & 1.087 & 0.766 & 1.097 & 0.756 & 1.150 & 0.788 & 1.287 & 0.839\\
& 336 & \boldres{0.373} & \boldres{0.391} & 0.450 & 0.424 & 0.485 & 0.459 & \secondres{0.402} & \secondres{0.416} & 0.499 & 0.467 & 0.786 & 0.597 & 0.807 & 0.628 & 0.851 & 0.659 & 0.870 & 0.603 & 1.138 & 0.787 & 1.147 & 0.775 & 1.198 & 0.809 & 1.288 & 0.842\\
& 720 & \boldres{0.425} & \boldres{0.420} & \secondres{0.483} & \secondres{0.471} & 0.577 & 0.499 & 0.511 & 0.489 & 0.767 & 0.587 & 0.796 & 0.593 & 0.822 & 0.633 & 0.857 & 0.655 & 0.893 & 0.611 & 1.245 & 0.831 & 1.200 & 0.799 & 1.175 & 0.794 & 1.247 & 0.828\\
& Avg & \boldres{0.364} & \boldres{0.385} & 0.425 & 0.434 & 0.472 & 0.450 & \secondres{0.400} & \secondres{0.417} & 0.526 & 0.476 & 0.717 & 0.561 & 0.730 & 0.592 & 0.796 & 0.620 & 0.857 & 0.598 & 1.125 & 0.782 & 1.123 & 0.765 & 1.163 & 0.791 & 1.264 & 0.826\\
\midrule

\multirow{5}{*}{\rotatebox{90}{$ETTm2$}}
& 96  & \boldres{0.169} & \boldres{0.260} & \secondres{0.174} & \secondres{0.261} & 0.199 & 0.280 & 0.236 & 0.326 & 0.206 & 0.288 & 0.220 & 0.299 & 0.229 & 0.320 & 0.232 & 0.322 & 0.238 & 0.316 & 0.404 & 0.485 & 1.108 & 0.772 & 3.599 & 1.478 & 3.883 & 1.545\\
& 192 & \boldres{0.224} & \secondres{0.298} & \secondres{0.215} & \boldres{0.287} & 0.256 & 0.316 & 0.306 & 0.373 & 0.264 & 0.324 & 0.311 & 0.361 & 0.394 & 0.361 & 0.291 & 0.357 & 0.298 & 0.349 & 0.479 & 0.521 & 1.317 & 0.850 & 3.578 & 1.475 & 3.553 & 1.484\\
& 336 & \boldres{0.282} & \secondres{0.338} & \secondres{0.273} & \boldres{0.330} & 0.318 & 0.353 & 0.380 & 0.423 & 0.334 & 0.367 & 0.338 & 0.366 & 0.378 & 0.427 & 0.478 & 0.517 & 0.353 & 0.380 & 0.552 & 0.555 & 1.415 & 0.879 & 3.561 & 1.473 & 3.446 & 1.460\\
& 720 & \boldres{0.375} & \boldres{0.397} & \secondres{0.433} & \secondres{0.412} & 0.460 & 0.436 & 0.674 & 0.583 & 0.454 & 0.432 & 0.509 & 0.465 & 0.523 & 0.510 & 0.553 & 0.538 & 0.475 & 0.445 & 0.701 & 0.627 & 1.822 & 0.984 & 3.896 & 1.533 & 3.445 & 1.460\\
& Avg & \boldres{0.262} & \boldres{0.323} & \secondres{0.274} & \boldres{0.323} & 0.308 & \secondres{0.346} & 0.399 & 0.426 & 0.314 & 0.352 & 0.344 & 0.372 & 0.381 & 0.404 & 0.388 & 0.433 & 0.341 & 0.372 & 0.534 & 0.547 & 1.415 & 0.871 & 3.658 & 1.489 & 3.581 & 1.487\\
\midrule

\multirow{5}{*}{\rotatebox{90}{$\revision{Weather}$}}
& 96  & 0.176 & \secondres{0.231} & \secondres{0.172} & 0.263 & 0.175 & 0.230 & 0.184 & 0.242 & \boldres{0.171} & \boldres{0.224} & 0.207 & 0.253 & 0.229 & 0.309 & 0.227 & 0.299 & 0.215 & 0.252 & 0.218 & 0.295 & 0.230 & 0.285 & 0.497 & 0.497 & 0.406 & 0.435\\
& 192 & \boldres{0.216} & \boldres{0.263} & \secondres{0.224} & \secondres{0.271} & 0.227 & 0.276 & 0.228 & 0.283 & 0.230 & 0.277 & 0.272 & 0.307 & 0.265 & 0.317 & 0.278 & 0.333 & 0.290 & 0.307 & 0.294 & 0.331 & 0.274 & 0.323 & 0.620 & 0.545 & 0.446 & 0.450\\
& 336 & \boldres{0.264} & \boldres{0.298} & 0.282 & \secondres{0.321} & 0.286 & 0.322 & \secondres{0.279} & 0.322 & 0.294 & 0.326 & 0.313 & 0.328 & 0.353 & 0.392 & 0.351 & 0.393 & 0.353 & 0.348 & 0.359 & 0.398 & 0.318 & 0.355 & 0.649 & 0.547 & 0.465 & 0.459\\
& 720 & \boldres{0.327} & \boldres{0.342} & 0.366 & 0.381 & 0.366 & \secondres{0.379} & \secondres{0.364} & 0.388 & 0.384 & 0.387 & 0.400 & 0.385 & 0.391 & 0.394 & 0.387 & 0.389 & 0.452 & 0.407 & 0.461 & 0.461 & 0.401 & 0.418 & 0.570 & 0.522 & 0.471 & 0.468\\
& Avg & \boldres{0.246} & \boldres{0.284} & \secondres{0.260} & 0.309 & 0.263 & \secondres{0.301} & 0.263 & 0.308 & 0.269 & 0.303 & 0.298 & 0.318 & 0.309 & 0.353 & 0.310 & 0.353 & 0.327 & 0.328 & 0.333 & 0.371 & 0.305 & 0.345 & 0.584 & 0.527 & 0.447 & 0.453\\
\midrule

\multirow{5}{*}{\rotatebox{90}{$\revision{Electricity}$}}
& 96  & 0.185 & 0.296 & 0.147 & \secondres{0.242} & \boldres{0.143} & \boldres{0.241} & 0.150 & 0.251 & \secondres{0.145} & 0.244 & 0.315 & 0.389 & 0.235 & 0.322 & 0.297 & 0.367 & 0.484 & 0.518 & 0.697 & 0.638 & 0.639 & 0.609 & 1.265 & 0.919 & 1.414 & 0.855\\
& 192 & 0.194 & 0.302 & \boldres{0.158} & \boldres{0.241} & \secondres{0.159} & \secondres{0.255} & 0.163 & 0.263 & 0.163 & 0.260 & 0.318 & 0.396 & 0.247 & 0.341 & 0.308 & 0.375 & 0.501 & 0.531 & 0.718 & 0.648 & 0.772 & 0.678 & 1.298 & 0.939 & 1.240 & 0.919\\
& 336 & 0.210 & 0.315 & \secondres{0.178} & \secondres{0.277} & 0.179 & \boldres{0.274} & \boldres{0.175} & 0.278 & 0.183 & 0.281 & 0.340 & 0.415 & 0.267 & 0.356 & 0.354 & 0.411 & 0.574 & 0.578 & 0.758 & 0.667 & 0.901 & 0.745 & 1.302 & 0.942 & 1.253 & 0.921\\
& 720 & 0.251 & 0.346 & \secondres{0.224} & \secondres{0.312} & 0.233 & 0.323 & \boldres{0.219} & \boldres{0.311} & 0.233 & 0.323 & 0.635 & 0.613 & 0.318 & 0.394 & 0.426 & 0.466 & 0.952 & 0.786 & 1.028 & 0.788 & 1.200 & 0.871 & 1.259 & 0.919 & 1.249 & 0.921\\
& Avg & 0.218 & 0.315 & \secondres{0.179} & \boldres{0.268} & \boldres{0.178} & \secondres{0.273} & 0.176 & 0.275 & 0.181 & 0.277 & 0.402 & 0.453 & 0.266 & 0.353 & 0.346 & 0.404 & 0.627 & 0.603 & 0.800 & 0.685 & 0.878 & 0.725 & 1.281 & 0.929 & 1.289 & 0.904\\
\midrule

\multirow{5}{*}{\rotatebox{90}{$\revision{Traffic}$}}
& 96  & 0.550 & 0.408 & \secondres{0.414} & \secondres{0.291} & 0.419 & 0.298 & 0.427 & 0.304 & \boldres{0.404} & \boldres{0.286} & 0.854 & 0.492 & 0.670 & 0.421 & 0.795 & 0.481 & 1.468 & 0.821 & 1.643 & 0.855 & 1.157 & 0.636 & 1.557 & 0.821 & 1.586 & 0.841\\
& 192 & 0.552 & 0.408 & \secondres{0.419} & \boldres{0.291} & 0.434 & 0.305 & 0.447 & 0.315 & \boldres{0.412} & \secondres{0.294} & 0.894 & 0.517 & 0.653 & 0.405 & 0.837 & 0.503 & 1.509 & 0.838 & 1.856 & 0.928 & 1.688 & 0.848 & 1.596 & 0.834 & 1.602 & 0.844\\
& 336 & 0.572 & 0.414 & \boldres{0.437} & \secondres{0.314} & 0.449 & 0.313 & 0.478 & 0.333 & \secondres{0.439} & \boldres{0.310} & 0.853 & 0.471 & 0.707 & 0.445 & 0.867 & 0.523 & 1.602 & 0.860 & 2.080 & 0.999 & 1.826 & 0.903 & 1.621 & 0.841 & 1.668 & 0.868\\
& 720 & - & - & - & - & - & - & - & - & - & - & - & - & - & - & - & - & - & - & - & - & - & - & - & - & - & -\\
& Avg & 0.558 & 0.410 & \secondres{0.423} & \secondres{0.298} & 0.434 & 0.305 & 0.450 & 0.317 & \boldres{0.418} & \boldres{0.296} & 0.867 & 0.493 & 0.676 & 0.423 & 0.833 & 0.502 & 1.526 & 0.839 & 1.859 & 0.927 & 1.557 & 0.795 & 1.591 & 0.832 & 1.618 & 0.851\\

\bottomrule
\end{tabular}
}
\end{small}
\end{center}
\vspace{-1em}
\end{table*}
\begin{table*}[h!]
\captionsetup{font=small} 
\caption{Full few-shot learning results on 10\% training data.}
\label{tab:few-shot-forecasting-10per-full}
\begin{center}
\begin{small}
\vspace{-1em}
\scalebox{0.65}{
\setlength\tabcolsep{3pt}
\begin{tabular}{c|c|cc|cc|cc|cc|cc|cc|cc|cc|cc|cc|cc|cc|cc}
\toprule

\multicolumn{2}{c|}{Methods}&\multicolumn{2}{c|}{\method{}}&\multicolumn{2}{c|}{Time-LLM}&\multicolumn{2}{c|}{GPT4TS}&\multicolumn{2}{c|}{DLinear}&\multicolumn{2}{c|}{PatchTST}&\multicolumn{2}{c|}{TimesNet}&\multicolumn{2}{c|}{FEDformer}&\multicolumn{2}{c|}{Autoformer}&\multicolumn{2}{c|}{Stationary}&\multicolumn{2}{c|}{ETSformer}&\multicolumn{2}{c|}{LightTS}&\multicolumn{2}{c|}{Informer}&\multicolumn{2}{c}{Reformer} \\

\midrule

\multicolumn{2}{c|}{Metric} & MSE  & MAE & MSE & MAE& MSE & MAE& MSE  & MAE& MSE  & MAE& MSE  & MAE& MSE  & MAE& MSE  & MAE& MSE  & MAE& MSE  & MAE& MSE  & MAE& MSE  & MAE& MSE  & MAE\\
\midrule

\multirow{5}{*}{\rotatebox{90}{$ETTh1$}}
& 96  & \boldres{0.391} & \boldres{0.404} & \secondres{0.448} & 0.460 & 0.458 & \secondres{0.456} & 0.492 & 0.495 & 0.516 & 0.485 & 0.861 & 0.628 & 0.512 & 0.499 & 0.613 & 0.552 & 0.918 & 0.639 & 1.112 & 0.806 & 1.298 & 0.838 & 1.179 & 0.792 & 1.184 & 0.790\\
& 192 & \boldres{0.420} & \boldres{0.431} & \secondres{0.484} & \secondres{0.483} & 0.570 & 0.516 & 0.565 & 0.538 & 0.598 & 0.524 & 0.797 & 0.593 & 0.624 & 0.555 & 0.722 & 0.598 & 0.915 & 0.629 & 1.155 & 0.823 & 1.322 & 0.854 & 1.199 & 0.806 & 1.295 & 0.850\\
& 336 & \boldres{0.439} & \boldres{0.448} & \secondres{0.589} & 0.540 & 0.608 & \secondres{0.535} & 0.721 & 0.622 & 0.657 & 0.550 & 0.941 & 0.648 & 0.691 & 0.574 & 0.750 & 0.619 & 0.939 & 0.644 & 1.179 & 0.832 & 1.347 & 0.870 & 1.202 & 0.811 & 1.294 & 0.854\\
& 720 & \boldres{0.476} & \boldres{0.484} & \secondres{0.700} & \secondres{0.604} & 0.725 & 0.591 & 0.986 & 0.743 & 0.762 & 0.610 & 0.877 & 0.641 & 0.728 & 0.614 & 0.721 & 0.616 & 0.887 & 0.645 & 1.273 & 0.874 & 1.534 & 0.947 & 1.217 & 0.825 & 1.223 & 0.838\\
& Avg & \boldres{0.431} & \boldres{0.442} & \secondres{0.556} & \secondres{0.522} & 0.590 & 0.525 & 0.691 & 0.600 & 0.633 & 0.542 & 0.869 & 0.628 & 0.639 & 0.561 & 0.702 & 0.596 & 0.915 & 0.639 & 1.180 & 0.834 & 1.375 & 0.877 & 1.199 & 0.809 & 1.249 & 0.833\\
\midrule

\multirow{5}{*}{\rotatebox{90}{$ETTh2$}}
& 96  & \secondres{0.284} & \secondres{0.347} & \boldres{0.275} & \boldres{0.326} & 0.331 & 0.374 & 0.357 & 0.411 & 0.353 & 0.389 & 0.378 & 0.409 & 0.382 & 0.416 & 0.413 & 0.451 & 0.389 & 0.411 & 0.678 & 0.619 & 2.022 & 1.006 & 3.837 & 1.508 & 3.788 & 1.533\\
& 192 & \boldres{0.349} & \secondres{0.398} & \secondres{0.374} & \boldres{0.373} & 0.402 & 0.411 & 0.569 & 0.519 & 0.403 & 0.414 & 0.490 & 0.467 & 0.478 & 0.474 & 0.474 & 0.477 & 0.473 & 0.455 & 0.785 & 0.666 & 2.329 & 1.104 & 3.856 & 1.513 & 3.552 & 1.483\\
& 336 & \boldres{0.370} & \boldres{0.412} & \secondres{0.406} & \secondres{0.429} & \secondres{0.406} & 0.433 & 0.671 & 0.572 & 0.426 & 0.441 & 0.537 & 0.494 & 0.504 & 0.501 & 0.547 & 0.543 & 0.507 & 0.480 & 0.839 & 0.694 & 2.453 & 1.122 & 3.952 & 1.526 & 3.395 & 1.526\\
& 720 & \boldres{0.441} & \secondres{0.466} & \secondres{0.427} & \boldres{0.449} & 0.449 & 0.464 & 0.824 & 0.648 & 0.477 & 0.480 & 0.510 & 0.491 & 0.499 & 0.509 & 0.516 & 0.523 & 0.477 & 0.472 & 1.273 & 0.874 & 3.816 & 1.407 & 3.842 & 1.503 & 3.205 & 1.401\\
& Avg & \boldres{0.361} & \secondres{0.405} & \secondres{0.370} & \boldres{0.394} & 0.397 & 0.421 & 0.605 & 0.538 & 0.415 & 0.431 & 0.479 & 0.465 & 0.466 & 0.475 & 0.488 & 0.499 & 0.462 & 0.455 & 0.894 & 0.713 & 2.655 & 1.160 & 3.872 & 1.513 & 3.485 & 1.486\\
\midrule

\multirow{5}{*}{\rotatebox{90}{$ETTm1$}}
& 96  & \boldres{0.310} & \boldres{0.354} & \secondres{0.346} & \secondres{0.388} & 0.390 & 0.404 & 0.352 & 0.392 & 0.410 & 0.419 & 0.583 & 0.501 & 0.578 & 0.518 & 0.774 & 0.614 & 0.761 & 0.568 & 0.911 & 0.688 & 0.921 & 0.682 & 1.162 & 0.785 & 1.442 & 0.847\\
& 192 & \boldres{0.340} & \boldres{0.370} & \secondres{0.373} & 0.416 & 0.429 & 0.423 & 0.382 & \secondres{0.412} & 0.437 & 0.434 & 0.630 & 0.528 & 0.617 & 0.546 & 0.754 & 0.592 & 0.781 & 0.574 & 0.955 & 0.703 & 0.957 & 0.701 & 1.172 & 0.793 & 1.444 & 0.862\\
& 336 & \boldres{0.369} & \boldres{0.387} & \secondres{0.413} & \secondres{0.426} & 0.469 & 0.439 & 0.419 & 0.434 & 0.476 & 0.454 & 0.725 & 0.568 & 0.998 & 0.775 & 0.869 & 0.677 & 0.803 & 0.587 & 0.991 & 0.719 & 0.998 & 0.716 & 1.227 & 0.908 & 1.450 & 0.866\\
& 720 & \boldres{0.423} & \boldres{0.417} & \secondres{0.485} & 0.476 & 0.569 & 0.498 & 0.490 & \secondres{0.477} & 0.681 & 0.556 & 0.769 & 0.549 & 0.693 & 0.579 & 0.810 & 0.630 & 0.844 & 0.581 & 1.062 & 0.747 & 1.007 & 0.719 & 1.207 & 0.797 & 1.366 & 0.850\\
& Avg & \boldres{0.360} & \boldres{0.382} & \secondres{0.404} & \secondres{0.427} & 0.464 & 0.441 & 0.411 & 0.429 & 0.501 & 0.466 & 0.677 & 0.537 & 0.722 & 0.605 & 0.802 & 0.628 & 0.797 & 0.578 & 0.980 & 0.714 & 0.971 & 0.705 & 1.192 & 0.821 & 1.426 & 0.856\\
\midrule

\multirow{5}{*}{\rotatebox{90}{$ETTm2$}}
& 96  & \boldres{0.169} & \boldres{0.260} & \secondres{0.177} & \secondres{0.261} & 0.188 & 0.269 & 0.213 & 0.303 & 0.191 & 0.274 & 0.212 & 0.285 & 0.291 & 0.399 & 0.352 & 0.454 & 0.229 & 0.308 & 0.331 & 0.430 & 0.813 & 0.688 & 3.203 & 1.407 & 4.195 & 1.628\\
& 192 & \boldres{0.222} & \boldres{0.296} & \secondres{0.241} & \secondres{0.314} & 0.251 & 0.309 & 0.278 & 0.345 & 0.252 & 0.317 & 0.270 & 0.323 & 0.307 & 0.379 & 0.694 & 0.691 & 0.291 & 0.343 & 0.400 & 0.464 & 1.008 & 0.768 & 3.112 & 1.387 & 4.042 & 1.601\\
& 336 & \secondres{0.278} & \secondres{0.335} & \boldres{0.274} & \boldres{0.327} & 0.307 & 0.346 & 0.338 & 0.385 & 0.306 & 0.353 & 0.323 & 0.353 & 0.543 & 0.559 & 2.408 & 1.407 & 0.348 & 0.376 & 0.469 & 0.498 & 1.031 & 0.775 & 3.255 & 1.421 & 3.963 & 1.585\\
& 720 & \boldres{0.381} & \secondres{0.401} & \secondres{0.417} & \boldres{0.390} & 0.426 & 0.417 & 0.436 & 0.440 & 0.433 & 0.427 & 0.474 & 0.449 & 0.712 & 0.614 & 1.913 & 1.166 & 0.461 & 0.438 & 0.589 & 0.557 & 1.096 & 0.791 & 3.909 & 1.543 & 3.711 & 1.532\\
& Avg & \boldres{0.263} & \boldres{0.323} & \secondres{0.277} & \boldres{0.323} & 0.293 & \secondres{0.335} & 0.316 & 0.368 & 0.296 & 0.343 & 0.320 & 0.353 & 0.463 & 0.488 & 1.342 & 0.930 & 0.332 & 0.366 & 0.447 & 0.487 & 0.987 & 0.756 & 3.370 & 1.440 & 3.978 & 1.587\\
\midrule

\multirow{5}{*}{\rotatebox{90}{$\revision{Weather}$}}
& 96  & 0.174 & 0.228 & \boldres{0.161} & \boldres{0.210} & \secondres{0.163} & \secondres{0.215} & 0.171 & 0.224 & 0.165 & 0.215 & 0.184 & 0.230 & 0.188 & 0.253 & 0.221 & 0.297 & 0.192 & 0.234 & 0.199 & 0.272 & 0.217 & 0.269 & 0.374 & 0.401 & 0.335 & 0.380\\
& 192 & 0.217 & 0.262 & \boldres{0.204} & \boldres{0.248} & \secondres{0.210} & \secondres{0.254} & 0.215 & 0.263 & 0.210 & 0.257 & 0.245 & 0.283 & 0.250 & 0.304 & 0.270 & 0.322 & 0.269 & 0.295 & 0.279 & 0.332 & 0.259 & 0.304 & 0.552 & 0.478 & 0.522 & 0.462\\
& 336 & 0.263 & \secondres{0.296} & 0.261 & 0.302 & \boldres{0.256} & \boldres{0.292} & \secondres{0.258} & 0.299 & 0.259 & 0.297 & 0.305 & 0.321 & 0.312 & 0.346 & 0.320 & 0.351 & 0.370 & 0.357 & 0.356 & 0.386 & 0.303 & 0.334 & 0.724 & 0.541 & 0.715 & 0.535\\
& 720 & 0.326 & 0.340 & \boldres{0.309} & \boldres{0.332} & 0.321 & \secondres{0.339} & \secondres{0.320} & 0.346 & 0.332 & 0.346 & 0.381 & 0.371 & 0.387 & 0.393 & 0.390 & 0.396 & 0.441 & 0.405 & 0.437 & 0.448 & 0.377 & 0.382 & 0.739 & 0.558 & 0.611 & 0.500\\
& Avg & 0.245 & 0.282 & \boldres{0.234} & \boldres{0.273} & \secondres{0.238} & \secondres{0.275} & 0.241 & 0.283 & 0.242 & 0.279 & 0.279 & 0.301 & 0.284 & 0.324 & 0.300 & 0.342 & 0.318 & 0.323 & 0.318 & 0.360 & 0.289 & 0.322 & 0.597 & 0.495 & 0.546 & 0.469\\
\midrule

\multirow{5}{*}{\rotatebox{90}{$\revision{Electricity}$}}
& 96  & 0.160 & 0.269 & \boldres{0.139} & 0.241 & \boldres{0.139} & \boldres{0.237} & \secondres{0.150} & 0.253 & 0.140 & \secondres{0.238} & 0.299 & 0.373 & 0.231 & 0.323 & 0.261 & 0.348 & 0.420 & 0.466 & 0.599 & 0.587 & 0.350 & 0.425 & 1.259 & 0.919 & 0.993 & 0.784\\
& 192 & 0.174 & 0.279 & \boldres{0.151} & \boldres{0.248} & \secondres{0.156} & \secondres{0.252} & 0.164 & 0.264 & 0.160 & 0.255 & 0.305 & 0.379 & 0.261 & 0.356 & 0.338 & 0.406 & 0.411 & 0.459 & 0.620 & 0.598 & 0.376 & 0.448 & 1.160 & 0.873 & 0.938 & 0.753\\
& 336 & 0.190 & 0.294 & \boldres{0.169} & \boldres{0.270} & \secondres{0.175} & \boldres{0.270} & 0.181 & 0.282 & 0.180 & \secondres{0.276} & 0.319 & 0.391 & 0.360 & 0.445 & 0.410 & 0.474 & 0.434 & 0.473 & 0.662 & 0.619 & 0.428 & 0.485 & 1.157 & 0.872 & 0.925 & 0.745\\
& 720 & \boldres{0.229} & \secondres{0.323} & 0.240 & 0.322 & \secondres{0.233} & \boldres{0.317} & 0.223 & 0.321 & 0.241 & 0.323 & 0.369 & 0.426 & 0.530 & 0.585 & 0.715 & 0.685 & 0.510 & 0.521 & 0.757 & 0.664 & 0.611 & 0.597 & 1.203 & 0.898 & 1.004 & 0.790\\
& Avg & 0.198 & 0.291 & \boldres{0.175} & \secondres{0.270} & \secondres{0.176} & \boldres{0.269} & 0.180 & 0.280 & 0.180 & 0.273 & 0.323 & 0.392 & 0.346 & 0.427 & 0.431 & 0.478 & 0.444 & 0.480 & 0.660 & 0.617 & 0.441 & 0.489 & 1.195 & 0.891 & 0.965 & 0.768\\
\midrule

\multirow{5}{*}{\rotatebox{90}{$\revision{Traffic}$}}
& 96  & 0.465 & 0.349 & 0.418 & \secondres{0.291} & \secondres{0.414} & 0.297 & 0.419 & 0.298 & \boldres{0.403} & \boldres{0.289} & 0.719 & 0.416 & 0.639 & 0.400 & 0.672 & 0.405 & 1.412 & 0.802 & 1.643 & 0.855 & 1.157 & 0.636 & 1.557 & 0.821 & 1.527 & 0.815\\
& 192 & 0.468 & 0.350 & \boldres{0.414} & \boldres{0.296} & \secondres{0.426} & 0.301 & 0.434 & 0.305 & \secondres{0.415} & \boldres{0.296} & 0.748 & 0.428 & 0.637 & 0.416 & 0.727 & 0.424 & 1.419 & 0.806 & 1.641 & 0.854 & 1.207 & 0.661 & 1.454 & 0.765 & 1.538 & 0.817\\
& 336 & 0.483 & 0.356 & \boldres{0.421} & 0.311 & 0.434 & \boldres{0.303} & 0.449 & 0.313 & 0.426 & 0.304 & 0.853 & 0.471 & 0.655 & 0.427 & 0.749 & 0.454 & 1.443 & 0.815 & 1.711 & 0.878 & 1.334 & 0.713 & 1.521 & 0.812 & 1.550 & 0.819\\
& 720 & 0.520 & 0.373 & \boldres{0.462} & \boldres{0.327} & 0.487 & 0.337 & 0.484 & 0.336 & \secondres{0.474} & \secondres{0.331} & 1.485 & 0.825 & 0.722 & 0.456 & 0.847 & 0.499 & 1.539 & 0.837 & 2.660 & 1.157 & 1.292 & 0.726 & 1.605 & 0.846 & 1.588 & 0.833\\
& Avg & 0.484 & 0.357 & \boldres{0.429} & \secondres{0.306} & 0.440 & 0.310 & 0.447 & 0.313 & \secondres{0.430} & \boldres{0.305} & 0.951 & 0.535 & 0.663 & 0.425 & 0.749 & 0.446 & 1.453 & 0.815 & 1.914 & 0.936 & 1.248 & 0.684 & 1.534 & 0.811 & 1.551 & 0.821\\

\bottomrule
\end{tabular}}
\end{small}
\end{center}
\vspace{-1em}
\end{table*}

\subsection{Zero-shot Forecasting}
\label{appx:zero-shot}
\begin{table}[h!]
\begin{center}
\captionsetup{font=small} 
\caption{\revision{Full zero-shot learning results on ETT datasets. A lower value indicates better performance.}}
\label{tab:zero-shot-forecasting}
\begin{small}
\setlength\tabcolsep{3pt}
\scalebox{0.80}{
\begin{tabular}{c|c|cc|cc|cc|cc|cc|cc|cc|cc}
\toprule
\multicolumn{2}{c|}{Methods}&\multicolumn{2}{c|}{\method}&\multicolumn{2}{c|}{\revision{Time-LLM}}&\multicolumn{2}{c|}{\revision{LLMTime}}&\multicolumn{2}{c|}{GPT4TS}&\multicolumn{2}{c|}{DLinear}&\multicolumn{2}{c|}{PatchTST}&\multicolumn{2}{c|}{TimesNet}&\multicolumn{2}{c}{Autoformer}\\
\midrule
\multicolumn{2}{c|}{Metric} & MSE & MAE & \revision{MSE} & \revision{MAE} & MSE & MAE & MSE & MAE & MSE & MAE& MSE & MAE & MSE & MAE & MSE & MAE\\
\midrule
\multirow{5}{*}{\rotatebox{0}{$ETTh1$} $\rightarrow$ \rotatebox{0}{$ETTh2$}} 
& 96  &\boldres{0.277} & \boldres{0.338} & \secondres{0.279} & \secondres{0.337} & 0.510 & 0.576 & 0.335 & 0.374 & 0.347 & 0.400 & 0.304 & 0.350 & 0.358 & 0.387 & 0.469 & 0.486\\
& 192 & \boldres{0.333} & \boldres{0.378} & \secondres{0.351} & \secondres{0.374} & 0.523 & 0.586 & 0.412 & 0.417 & 0.447 & 0.460 & 0.386 & 0.400 & 0.427 & 0.429 & 0.634 & 0.567\\
& 336 & \boldres{0.360} & \boldres{0.399} & \secondres{0.388} & 0.415 & 0.640 & 0.637 & 0.441 & 0.444 & 0.515 & 0.505 & 0.414 & 0.428 & 0.449 & 0.451 & 0.655 & 0.588\\
& 720 & \boldres{0.383} & \boldres{0.425} & \secondres{0.391} & \secondres{0.420} & 2.296 & 1.034 & 0.438 & 0.452 & 0.665 & 0.589 & 0.419 & 0.443 & 0.448 & 0.458 & 0.570 & 0.549\\
& Avg & \boldres{0.338} & \boldres{0.385} & \secondres{0.353} & \secondres{0.387} & 0.992 & 0.708 & 0.406 & 0.422 & 0.493 & 0.488 & 0.380 & 0.405 & 0.421 & 0.431 & 0.582 & 0.548\\
\midrule
\multirow{5}{*}{\rotatebox{0}{$ETTh1 $} $\rightarrow$ \rotatebox{0}{$ETTm2 $}}
& 96  &\secondres{0.207} & \secondres{0.297} & \boldres{0.189} & \boldres{0.293} & 0.646 & 0.563 & 0.236 & 0.315 & 0.255 & 0.357 & 0.215 & 0.304 & 0.239 & 0.313 & 0.352 & 0.432\\
& 192 & \secondres{0.258} & \secondres{0.329} & \boldres{0.237} & \boldres{0.312} & 0.934 & 0.654 & 0.287 & 0.342 & 0.338 & 0.413 & 0.275 & 0.339 & 0.291 & 0.342 & 0.413 & 0.460\\
& 336 & \secondres{0.310} & \secondres{0.360} & \boldres{0.291} & \boldres{0.365} & 1.157 & 0.728 & 0.341 & 0.374 & 0.425 & 0.465 & 0.334 & 0.373 & 0.342 & 0.371 & 0.465 & 0.489\\
& 720 & \secondres{0.398} & \secondres{0.412} & \boldres{0.372} & \boldres{0.390} & 4.730 & 1.531 & 0.435 & 0.422 & 0.640 & 0.573 & 0.431 & 0.424 & 0.434 & 0.419 & 0.599 & 0.551\\
& Avg & \secondres{0.293} & \secondres{0.350} & \boldres{0.273} & \boldres{0.340} & 1.867 & 0.869 & 0.325 & 0.363 & 0.415 & 0.452 & 0.314 & 0.360 & 0.327 & 0.361 & 0.457 & 0.483\\
\midrule
\multirow{5}{*}{\rotatebox{0}{$ETTh2 $} $\rightarrow$ \rotatebox{0}{$ETTh1 $}}
& 96  &\boldres{0.434} & \boldres{0.441} & \secondres{0.450} & \secondres{0.452} & 1.130 & 0.777 & 0.732 & 0.577 & 0.689 & 0.555 & 0.485 & 0.465 & 0.848 & 0.601 & 0.693 & 0.569\\
& 192 & \boldres{0.464} & \boldres{0.454} & \secondres{0.465} & \secondres{0.461} & 1.242 & 0.820 & 0.758 & 0.559 & 0.707 & 0.568 & 0.565 & 0.509 & 0.860 & 0.610 & 0.760 & 0.601\\
& 336 & \boldres{0.489} & \boldres{0.481} & \secondres{0.501} & \secondres{0.482} & 1.328 & 0.864 & 0.759 & 0.578 & 0.710 & 0.577 & 0.581 & 0.515 & 0.867 & 0.626 & 0.781 & 0.619\\
& 720 & \secondres{0.595} & \secondres{0.543} & \boldres{0.501} & \boldres{0.502} & 4.145 & 1.461 & 0.781 & 0.597 & 0.704 & 0.596 & 0.628 & 0.561 & 0.887 & 0.648 & 0.796 & 0.644\\
& Avg & \secondres{0.496} & \secondres{0.480} & \boldres{0.479} & \boldres{0.474} & 1.961 & 0.981 & 0.757 & 0.578 & 0.703 & 0.574 & 0.565 & 0.513 & 0.865 & 0.621 & 0.757 & 0.608\\
\midrule
\multirow{5}{*}{\rotatebox{0}{$ETTh2 $} $\rightarrow$ \rotatebox{0}{$ETTm2 $}}
& 96  &\secondres{0.204} & \secondres{0.297} & \boldres{0.174} & \boldres{0.276} & 0.646 & 0.563 & 0.253 & 0.329 & 0.240 & 0.336 & 0.226 & 0.309 & 0.248 & 0.324 & 0.263 & 0.352\\
& 192 & \secondres{0.255} & \secondres{0.328} & \boldres{0.233} & \boldres{0.315} & 0.934 & 0.654 & 0.293 & 0.346 & 0.295 & 0.369 & 0.289 & 0.345 & 0.296 & 0.352 & 0.326 & 0.389\\
& 336 & \secondres{0.311} & \secondres{0.362} & \boldres{0.291} & \boldres{0.337} & 1.157 & 0.728 & 0.347 & 0.376 & 0.345 & 0.397 & 0.348 & 0.379 & 0.353 & 0.383 & 0.387 & 0.426\\
& 720 & \secondres{0.420} & \secondres{0.425} & \boldres{0.392} & \boldres{0.417} & 4.730 & 1.531 & 0.446 & 0.429 & 0.432 & 0.442 & 0.439 & 0.427 & 0.471 & 0.446 & 0.487 & 0.478\\
& Avg & \secondres{0.297} & \secondres{0.353} & \boldres{0.272} & \boldres{0.341} & 1.867 & 0.869 & 0.335 & 0.370 & 0.328 & 0.386 & 0.325 & 0.365 & 0.342 & 0.376 & 0.366 & 0.411\\
\midrule
\multirow{5}{*}{\rotatebox{0}{$ETTm1 $} $\rightarrow$ \rotatebox{0}{$ETTh2 $}}
& 96  &\boldres{0.297} & \boldres{0.356} & \secondres{0.321} & \secondres{0.369} & 0.510 & 0.576 & 0.353 & 0.392 & 0.365 & 0.415 & 0.354 & 0.385 & 0.377 & 0.407 & 0.435 & 0.470\\
& 192 & \boldres{0.349} & \boldres{0.388} & \secondres{0.389} & \secondres{0.410} & 0.523 & 0.586 & 0.443 & 0.437 & 0.454 & 0.462 & 0.447 & 0.434 & 0.471 & 0.453 & 0.495 & 0.489\\
& 336 & \boldres{0.374} & \boldres{0.409} & \secondres{0.408} & \secondres{0.433} & 0.640 & 0.637 & 0.469 & 0.461 & 0.496 & 0.494 & 0.481 & 0.463 & 0.472 & 0.484 & 0.470 & 0.472\\
& 720 & \boldres{0.396} & \boldres{0.433} & \secondres{0.406} & \secondres{0.436} & 2.296 & 1.034 & 0.466 & 0.468 & 0.541 & 0.529 & 0.474 & 0.471 & 0.495 & 0.482 & 0.480 & 0.485\\
& Avg & \boldres{0.354} & \boldres{0.397} & \secondres{0.381} & \secondres{0.412} & 0.992 & 0.708 & 0.433 & 0.439 & 0.464 & 0.475 & 0.439 & 0.438 & 0.457 & 0.454 & 0.470 & 0.479\\
\midrule
\multirow{5}{*}{\rotatebox{0}{$ETTm1 $} $\rightarrow$ \rotatebox{0}{$ETTm2 $}}
& 96  &\secondres{0.178} & \secondres{0.264} & \boldres{0.169} & \boldres{0.257} & 0.646 & 0.563 & 0.217 & 0.294 & 0.221 & 0.314 & 0.195 & 0.271 & 0.222 & 0.295 & 0.385 & 0.457\\
& 192 & \boldres{0.226} & \boldres{0.298} & \secondres{0.227} & \secondres{0.318} & 0.934 & 0.654 & 0.277 & 0.327 & 0.286 & 0.359 & 0.258 & 0.311 & 0.288 & 0.337 & 0.433 & 0.469\\
& 336 & \boldres{0.279} & \boldres{0.329} & \secondres{0.290} & \secondres{0.338} & 1.157 & 0.728 & 0.331 & 0.360 & 0.357 & 0.406 & 0.317 & 0.348 & 0.341 & 0.367 & 0.476 & 0.477\\
& 720 & \boldres{0.373} & \secondres{0.385} & \secondres{0.375} & \boldres{0.367} & 4.730 & 1.531 & 0.429 & 0.413 & 0.476 & 0.476 & 0.416 & 0.404 & 0.436 & 0.418 & 0.582 & 0.535\\
& Avg & \boldres{0.264} & \boldres{0.319} & \secondres{0.268} & \secondres{0.320} & 1.867 & 0.869 & 0.313 & 0.348 & 0.335 & 0.389 & 0.296 & 0.334 & 0.322 & 0.354 & 0.469 & 0.484\\
\midrule
\multirow{5}{*}{\rotatebox{0}{$ETTm2 $} $\rightarrow$ \rotatebox{0}{$ETTh2 $}}
& 96  &\boldres{0.285} & \boldres{0.347} & \secondres{0.298} & \secondres{0.356} & 0.510 & 0.576 & 0.360 & 0.401 & 0.333 & 0.391 & 0.327 & 0.367 & 0.360 & 0.401 & 0.353 & 0.393\\
& 336 & \secondres{0.380} & \secondres{0.415} & \boldres{0.367} & \boldres{0.412} & 0.640 & 0.637 & 0.460 & 0.459 & 0.505 & 0.503 & 0.439 & 0.447 & 0.460 & 0.459 & 0.452 & 0.459\\
& 720 & \secondres{0.424} & \secondres{0.451} & \boldres{0.393} & \boldres{0.434} & 2.296 & 1.034 & 0.485 & 0.477 & 0.543 & 0.534 & 0.459 & 0.470 & 0.485 & 0.477 & 0.453 & 0.467\\
& Avg & \secondres{0.359} & \boldres{0.399} & \boldres{0.354} & \secondres{0.400} & 0.992 & 0.708 & 0.435 & 0.443 & 0.455 & 0.471 & 0.409 & 0.425 & 0.435 & 0.443 & 0.423 & 0.439\\
\midrule
\multirow{5}{*}{\rotatebox{0}{$ETTm2 $} $\rightarrow$ \rotatebox{0}{$ETTm1 $}}
& 96  &\secondres{0.370} & \secondres{0.390} & \boldres{0.359} & \boldres{0.397} & 1.179 & 0.781 & 0.747 & 0.558 & 0.570 & 0.490 & 0.491 & 0.437 & 0.747 & 0.558 & 0.735 & 0.576\\
& 192 & \secondres{0.400} & \boldres{0.409} & \boldres{0.390} & \secondres{0.420} & 1.327 & 0.846 & 0.781 & 0.560 & 0.590 & 0.506 & 0.530 & 0.470 & 0.781 & 0.560 & 0.753 & 0.586\\
& 336 & \secondres{0.426} & \boldres{0.420} & \boldres{0.421} & \secondres{0.445} & 1.478 & 0.902 & 0.778 & 0.578 & 0.706 & 0.567 & 0.565 & 0.497 & 0.778 & 0.578 & 0.750 & 0.593\\
& 720 & \secondres{0.531} & \boldres{0.487} & \boldres{0.487} & \secondres{0.488} & 3.749 & 1.408 & 0.769 & 0.573 & 0.731 & 0.584 & 0.686 & 0.565 & 0.769 & 0.573 & 0.782 & 0.609\\
& Avg & \secondres{0.432} & \boldres{0.426} & \boldres{0.414} & \secondres{0.438} & 1.933 & 0.984 & 0.769 & 0.567 & 0.649 & 0.537 & 0.568 & 0.492 & 0.769 & 0.567 & 0.755 & 0.591\\
\bottomrule
\end{tabular}}
\end{small}
\end{center}
\vspace{-1em}
\end{table}

\subsection{Short-term Forecasting}
\label{appx:short-term}
\begin{table}[h!]
\renewcommand\arraystretch{1.2}
\captionsetup{font=small} 
\caption{Full short-term time series forecasting results. The forecasting horizons are in [6, 48] and the last three rows are weighted averaged from all datasets under different sampling intervals. A lower value indicates better performance.}
\label{tab:short-term-forecasting-full}
\begin{center}
\begin{small}
\scalebox{0.70}{
\setlength\tabcolsep{3pt}
\begin{tabular}{cc|ccccccccccccccc}
\toprule

\multicolumn{2}{c|}{Methods}& \method &Time-LLM &GPT4TS&TimesNet&PatchTST&N-HiTS&N-BEATS& ETSformer& LightTS& DLinear &FEDformer &Stationary &Autoformer  &Informer&Reformer \\

\midrule
\multirow{3}{*}{\rotatebox{90}{Yearly}}
&SMAPE&\boldres{13.285}&\secondres{13.419}&15.110&15.378&13.477&13.422&13.487&18.009&14.247&16.965&14.021&13.717&13.974&14.727&16.169\\
&MASE&\boldres{2.993}&\secondres{3.005}&3.565&3.554&3.019&3.056&3.036&4.487&3.109&4.283&3.036&3.078&3.134&3.418&3.800\\
&OWA&\boldres{0.783}&\secondres{0.789}&0.911&0.918&0.792&0.795&0.795&1.115&0.827&1.058&0.811&0.807&0.822&0.881&0.973\\
\midrule

\multirow{3}{*}{\rotatebox{90}{Quarterly}}
&SMAPE&\boldres{10.218}&\secondres{10.110}&10.597&10.465&10.380&10.185&10.564&13.376&11.364&12.145&11.100&10.958&11.338&11.360&13.313\\
&MASE&\boldres{1.203}&\secondres{1.178}&1.253&1.227&1.233&1.180&1.252&1.906&1.328&1.520&1.350&1.325&1.365&1.401&1.775\\
&OWA&\boldres{0.903}&\secondres{0.889}&0.938&0.923&0.921&0.893&0.936&1.302&1.000&1.106&0.996&0.981&1.012&1.027&1.252\\
\midrule

\multirow{3}{*}{\rotatebox{90}{Monthly}}
&SMAPE&\boldres{12.788}&\secondres{12.980}&13.258&13.513&12.959&13.059&13.089&14.588&14.014&13.514&14.403&13.917&13.958&14.062&20.128\\
&MASE&\boldres{0.942}&\secondres{0.963}&1.003&1.039&0.970&1.013&0.996&1.368&1.053&1.037&1.147&1.097&1.103&1.141&2.614\\
&OWA&\boldres{0.886}&\secondres{0.903}&0.931&0.957&0.905&0.929&0.922&1.149&0.981&0.956&1.038&0.998&1.002&1.024&1.927\\
\midrule

\multirow{3}{*}{\rotatebox{90}{Others}}
&SMAPE&\boldres{4.945}&\secondres{4.795}&6.124&6.913&4.952&4.711&6.599&7.267&15.880&6.709&7.148&6.302&5.485&24.460&32.491\\
&MASE&\boldres{3.257}&\secondres{3.178}&4.116&4.507&3.347&3.054&4.430&5.240&11.434&4.953&4.041&4.064&3.865&20.960&33.355\\
&OWA&\boldres{1.034}&\secondres{1.006}&1.259&1.438&1.049&0.977&1.393&1.591&3.474&1.487&1.389&1.304&1.187&5.879&8.679\\
\midrule

\multirow{3}{*}{\rotatebox{90}{Average}}
&SMAPE&\boldres{11.894}&\secondres{11.983}&12.690&12.880&12.059&12.035&12.250&14.718&13.525&13.639&13.160&12.780&12.909&14.086&18.200\\
&MASE&\boldres{1.592}&\secondres{1.595}&1.808&1.836&1.623&1.625&1.698&2.408&2.111&2.095&1.775&1.756&1.771&2.718&4.223\\
&OWA&\boldres{0.855}&\secondres{0.859}&0.940&0.955&0.869&0.869&0.896&1.172&1.051&1.051&0.949&0.930&0.939&1.230&1.775\\

\bottomrule

\end{tabular}
}
\end{small}
\end{center}
\end{table}

\subsection{Long-term Forecasting}
\label{appx:long-term}
\begin{table}[h!]
\captionsetup{font=small}
\caption{Full long-term forecasting results. We use the same protocol as in \shortautoref{tab:few-shot-forecasting-5per}.}
\label{tab:long-term-forecasting-full}
\begin{center}
\begin{small}
\scalebox{0.65}{
\setlength\tabcolsep{3pt}
\begin{tabular}{c|c|cc|cc|cc|cc|cc|cc|cc|cc|cc|cc|cc|cc|cc}
\toprule

\multicolumn{2}{c|}{Methods}&\multicolumn{2}{c|}{\method{}}&\multicolumn{2}{c|}{Time-LLM}&\multicolumn{2}{c|}{GPT4TS}&\multicolumn{2}{c|}{DLinear}&\multicolumn{2}{c|}{PatchTST}&\multicolumn{2}{c|}{TimesNet}&\multicolumn{2}{c|}{FEDformer}&\multicolumn{2}{c|}{Autoformer}&\multicolumn{2}{c|}{Stationary}&\multicolumn{2}{c|}{ETSformer}&\multicolumn{2}{c|}{LightTS}&\multicolumn{2}{c|}{Informer}&\multicolumn{2}{c}{Reformer} \\

\midrule

\multicolumn{2}{c|}{Metric} & MSE  & MAE & MSE & MAE& MSE & MAE& MSE  & MAE& MSE  & MAE& MSE  & MAE& MSE  & MAE& MSE  & MAE& MSE  & MAE& MSE  & MAE& MSE  & MAE& MSE  & MAE& MSE  & MAE\\
\midrule

\multirow{5}{*}{\rotatebox{90}{$ETTh1$}}
& 96  & \boldres{0.361} & \boldres{0.386} & \secondres{0.362} & \secondres{0.392} & 0.376 & 0.397 & 0.375 & 0.399 & 0.370 & 0.399 & 0.384 & 0.402 & 0.376 & 0.419 & 0.449 & 0.459 & 0.513 & 0.491 & 0.494 & 0.479 & 0.424 & 0.432 & 0.865 & 0.713 & 0.837 & 0.728\\
& 192 & \boldres{0.397} & \boldres{0.415} & \secondres{0.398} & \secondres{0.418} & 0.416 & 0.418 & 0.405 & 0.416 & 0.413 & 0.421 & 0.436 & 0.429 & 0.420 & 0.448 & 0.500 & 0.482 & 0.534 & 0.504 & 0.538 & 0.504 & 0.475 & 0.462 & 1.008 & 0.792 & 0.923 & 0.766\\
& 336 & \boldres{0.420} & \boldres{0.421} & 0.430 & \secondres{0.427} & 0.442 & 0.433 & 0.439 & 0.443 & \secondres{0.422} & 0.436 & 0.491 & 0.469 & 0.459 & 0.465 & 0.521 & 0.496 & 0.588 & 0.535 & 0.574 & 0.521 & 0.518 & 0.488 & 1.107 & 0.809 & 1.097 & 0.835\\
& 720 & \boldres{0.441} & \boldres{0.458} & \secondres{0.442} & \secondres{0.457} & 0.477 & 0.456 & 0.472 & 0.490 & 0.447 &  0.466 & 0.521 & 0.500 & 0.506 & 0.507 & 0.514 & 0.512 & 0.643 & 0.616 & 0.562 & 0.535 & 0.547 & 0.533 & 1.181 & 0.865 & 1.257 & 0.889\\
& Avg & \boldres{0.405} & \boldres{0.420} & \secondres{0.408} & \secondres{0.423} & 0.465 & 0.455 & 0.422 & 0.437 & 0.413 & 0.430 & 0.458 & 0.450 & 0.440 & 0.460 & 0.496 & 0.487 & 0.570 & 0.537 & 0.542 & 0.510 & 0.491 & 0.479 & 1.040 & 0.795 & 1.029 & 0.805\\
\midrule

\multirow{5}{*}{\rotatebox{90}{$ETTh2$}}
& 96  & \boldres{0.267} & 0.335 & \secondres{0.268} & \boldres{0.328} & 0.285 & 0.342 & 0.289 & 0.353 & 0.274 & 0.336 & 0.340 & 0.374 & 0.358 & 0.397 & 0.346 & 0.388 & 0.476 & 0.458 & 0.340 & 0.391 & 0.397 & 0.437 & 3.755 & 1.525 & 2.626 & 1.317\\
& 192 & \boldres{0.326} & \boldres{0.373} & \secondres{0.329} & \secondres{0.375} & 0.354 & 0.389 & 0.383 & 0.418 & 0.339 & \secondres{0.379} & 0.402 & 0.414 & 0.429 & 0.439 & 0.456 & 0.452 & 0.512 & 0.493 & 0.430 & 0.439 & 0.520 & 0.504 & 5.602 & 1.931 & 11.120 & 2.979\\
& 336 & \secondres{0.357} & \secondres{0.406} & 0.368 & 0.409 & 0.373 & 0.407 & 0.448 & 0.465 & \boldres{0.329} & \boldres{0.380} & 0.452 & 0.452 & 0.496 & 0.487 & 0.482 & 0.486 & 0.552 & 0.551 & 0.485 & 0.479 & 0.626 & 0.559 & 4.721 & 1.835 & 9.323 & 2.769\\
& 720 & 0.412 & 0.449 & \boldres{0.372} & \boldres{0.420} & 0.406 & 0.441 & 0.605 & 0.551 & \secondres{0.379} & \secondres{0.422} & 0.462 & 0.468 & 0.463 & 0.474 & 0.515 & 0.511 & 0.562 & 0.560 & 0.500 & 0.497 & 0.863 & 0.672 & 3.647 & 1.625 & 3.874 & 1.697\\
& Avg & 0.341 & 0.391 & \secondres{0.334} & \secondres{0.383} & 0.381 & 0.412 & 0.431 & 0.446 & \boldres{0.330} & \boldres{0.379} & 0.414 & 0.427 & 0.437 & 0.449 & 0.450 & 0.459 & 0.526 & 0.516 & 0.439 & 0.452 & 0.602 & 0.543 & 4.431 & 1.729 & 6.736 & 2.191\\
\midrule

\multirow{5}{*}{\rotatebox{90}{$ETTm1$}}
& 96  & 0.304 & 0.346 & \boldres{0.272} & \boldres{0.334} & 0.292 & 0.346 & 0.299 & 0.343 & \secondres{0.290} & \secondres{0.342} & 0.338 & 0.375 & 0.379 & 0.419 & 0.505 & 0.475 & 0.386 & 0.398 & 0.375 & 0.398 & 0.374 & 0.400 & 0.672 & 0.571 & 0.538 & 0.528\\
& 192 & 0.332 & \secondres{0.366} & \boldres{0.310} & \boldres{0.358} & 0.332 & 0.372 & 0.335 & 0.365 & \secondres{0.332} & 0.369 & 0.374 & 0.387 & 0.426 & 0.441 & 0.553 & 0.496 & 0.459 & 0.444 & 0.408 & 0.410 & 0.400 & 0.407 & 0.795 & 0.669 & 0.658 & 0.592\\
& 336 & 0.364 & \boldres{0.383} & \boldres{0.352} & \secondres{0.384} & 0.366 & 0.394 & 0.369 & 0.386 & 0.366 & 0.392 & 0.410 & 0.411 & 0.445 & 0.459 & 0.621 & 0.537 & 0.495 & 0.464 & 0.435 & 0.428 & 0.438 & 0.438 & 1.212 & 0.871 & 0.898 & 0.721\\
& 720 & 0.402 & \boldres{0.410} & \boldres{0.383} & \secondres{0.411} & 0.417 & 0.421 & 0.425 & 0.421 & 0.416 & 0.420 & 0.478 & 0.450 & 0.543 & 0.490 & 0.671 & 0.561 & 0.585 & 0.516 & 0.499 & 0.462 & 0.527 & 0.502 & 1.166 & 0.823 & 1.102 & 0.841\\
& Avg & \secondres{0.350} & \secondres{0.377} & \boldres{0.329} & \boldres{0.372} & 0.388 & 0.403 & 0.357 & 0.378 & 0.351 & 0.380 & 0.400 & 0.406 & 0.448 & 0.452 & 0.588 & 0.517 & 0.481 & 0.456 & 0.429 & 0.425 & 0.435 & 0.437 & 0.961 & 0.734 & 0.799 & 0.671\\
\midrule

\multirow{5}{*}{\rotatebox{90}{$ETTm2$}}
& 96  & \boldres{0.160} & \boldres{0.250} & \secondres{0.161} & \secondres{0.253} & 0.173 & 0.262 & 0.167 & 0.269 & 0.165 & 0.255 & 0.187 & 0.267 & 0.203 & 0.287 & 0.255 & 0.339 & 0.192 & 0.274 & 0.189 & 0.280 & 0.209 & 0.308 & 0.365 & 0.453 & 0.658 & 0.619\\
& 192 & \boldres{0.215} & \boldres{0.291} & \secondres{0.219} & \secondres{0.293} & 0.229 & \secondres{0.301} & \secondres{0.224} & 0.303 & 0.220 & 0.292 & 0.249 & 0.309 & 0.269 & 0.328 & 0.281 & 0.340 & 0.280 & 0.339 & 0.253 & 0.319 & 0.311 & 0.382 & 0.533 & 0.563 & 1.078 & 0.827\\
& 336 & \boldres{0.270} & \boldres{0.325} & \secondres{0.271} & \secondres{0.329} & 0.286 & 0.341 & 0.281 & 0.342 & 0.274 & 0.329 & 0.321 & 0.351 & 0.325 & 0.366 & 0.339 & 0.372 & 0.334 & 0.361 & 0.314 & 0.357 & 0.442 & 0.466 & 1.363 & 0.887 & 1.549 & 0.972\\
& 720 & \boldres{0.348} & \boldres{0.378} & \secondres{0.352} & \secondres{0.379} & 0.378 & 0.401 & 0.397 & 0.421 & 0.362 & 0.385 & 0.408 & 0.403 & 0.421 & 0.415 & 0.433 & 0.432 & 0.417 & 0.413 & 0.414 & 0.413 & 0.675 & 0.587 & 3.379 & 1.338 & 2.631 & 1.242\\
& Avg & \boldres{0.248} & \boldres{0.311} & \secondres{0.251} & \secondres{0.313} & 0.284 & 0.339 & 0.267 & 0.333 & 0.255 & 0.315 & 0.291 & 0.333 & 0.305 & 0.349 & 0.327 & 0.371 & 0.306 & 0.347 & 0.293 & 0.342 & 0.409 & 0.436 & 1.410 & 0.810 & 1.479 & 0.915\\
\midrule

\multirow{5}{*}{\rotatebox{90}{$\revision{Weather}$}}
& 96  & \secondres{0.148} & \secondres{0.200} & \boldres{0.147} & 0.201 & 0.162 & 0.212 & 0.176 & 0.237 & 0.149 & \boldres{0.198} & 0.172 & 0.220 & 0.217 & 0.296 & 0.266 & 0.336 & 0.173 & 0.223 & 0.197 & 0.281 & 0.182 & 0.242 & 0.300 & 0.384 & 0.689 & 0.596\\
& 192 & \secondres{0.193} & \secondres{0.240} & \boldres{0.189} & \boldres{0.234} & 0.204 & 0.248 & 0.220 & 0.282 & 0.194 & 0.241 & 0.219 & 0.261 & 0.276 & 0.336 & 0.307 & 0.367 & 0.245 & 0.285 & 0.237 & 0.312 & 0.227 & 0.287 & 0.598 & 0.544 & 0.752 & 0.638\\
& 336 & \boldres{0.243} & \secondres{0.281} & 0.262 & \boldres{0.279} & 0.254 & 0.286 & 0.265 & 0.319 & \secondres{0.245} & 0.282 & 0.280 & 0.306 & 0.339 & 0.380 & 0.359 & 0.395 & 0.321 & 0.338 & 0.298 & 0.353 & 0.282 & 0.334 & 0.578 & 0.523 & 0.639 & 0.596\\
& 720 & \secondres{0.312} & \secondres{0.332} & \boldres{0.304} & \boldres{0.316} & 0.326 & 0.337 & 0.333 & 0.362 & 0.314 & 0.334 & 0.365 & 0.359 & 0.403 & 0.428 & 0.419 & 0.428 & 0.414 & 0.410 & 0.352 & 0.288 & 0.352 & 0.386 & 1.059 & 0.741 & 1.130 & 0.792\\
& Avg & \boldres{0.224} & \secondres{0.263} & \secondres{0.225} & \boldres{0.257} & 0.237 & 0.270 & 0.248 & 0.300 & 0.225 & 0.264 & 0.259 & 0.287 & 0.309 & 0.360 & 0.338 & 0.382 & 0.288 & 0.314 & 0.271 & 0.334 & 0.261 & 0.312 & 0.634 & 0.548 & 0.803 & 0.656\\
\midrule

\multirow{5}{*}{\rotatebox{90}{$\revision{Electricity}$}}
& 96  & 0.142 & 0.245 & \secondres{0.131} & \secondres{0.224} & 0.139 & 0.238 & 0.140 & 0.237 & \boldres{0.129} & \boldres{0.222} & 0.168 & 0.272 & 0.193 & 0.308 & 0.201 & 0.317 & 0.169 & 0.273 & 0.187 & 0.304 & 0.207 & 0.307 & 0.274 & 0.368 & 0.312 & 0.402\\
& 192 & 0.157 & 0.260 & \boldres{0.152} & \secondres{0.241} & 0.153 & 0.251 & 0.153 & 0.249 & \secondres{0.157} & \boldres{0.240} & 0.184 & 0.289 & 0.201 & 0.315 & 0.222 & 0.334 & 0.182 & 0.286 & 0.199 & 0.315 & 0.213 & 0.316 & 0.296 & 0.386 & 0.348 & 0.433\\
& 336 & 0.174 & 0.276 & \boldres{0.160} & \boldres{0.248} & 0.169 & 0.266 & 0.169 & 0.267 & \secondres{0.163} & \secondres{0.259} & 0.198 & 0.300 & 0.214 & 0.329 & 0.231 & 0.338 & 0.200 & 0.304 & 0.212 & 0.329 & 0.230 & 0.333 & 0.300 & 0.394 & 0.350 & 0.433\\
& 720 & 0.214 & 0.308 & \boldres{0.192} & \secondres{0.298} & 0.206 & 0.297 & 0.203 & 0.301 & \secondres{0.197} & \boldres{0.290} & 0.220 & 0.320 & 0.246 & 0.355 & 0.254 & 0.361 & 0.222 & 0.321 & 0.233 & 0.345 & 0.265 & 0.360 & 0.373 & 0.439 & 0.340 & 0.420\\
& Avg & 0.172 & 0.273 & \boldres{0.158} & \boldres{0.252} & 0.167 & \secondres{0.263} & 0.166 & \secondres{0.263} & \secondres{0.161} & \boldres{0.252} & 0.192 & 0.295 & 0.214 & 0.327 & 0.227 & 0.338 & 0.193 & 0.296 & 0.208 & 0.323 & 0.229 & 0.329 & 0.311 & 0.397 & 0.338 & 0.422\\
\midrule

\multirow{5}{*}{\rotatebox{90}{$\revision{Traffic}$}}
& 96  & 0.393 & 0.290 & \secondres{0.362} & \boldres{0.248} & 0.388 & 0.282 & 0.410 & 0.282 & \boldres{0.360} & \secondres{0.249} & 0.593 & 0.321 & 0.587 & 0.366 & 0.613 & 0.388 & 0.612 & 0.338 & 0.607 & 0.392 & 0.615 & 0.391 & 0.719 & 0.391 & 0.732 & 0.423\\
& 192 & 0.405 & 0.296 & \boldres{0.374} & \boldres{0.247} & 0.407 & 0.290 & 0.423 & 0.287 & \secondres{0.379} & \secondres{0.256} & 0.617 & 0.336 & 0.604 & 0.373 & 0.616 & 0.382 & 0.613 & 0.340 & 0.621 & 0.399 & 0.601 & 0.382 & 0.696 & 0.379 & 0.733 & 0.420\\
& 336 & 0.420 & 0.305 & \boldres{0.385} & \secondres{0.271} & 0.412 & 0.294 & 0.436 & 0.296 & \secondres{0.392} & \boldres{0.264} & 0.629 & 0.336 & 0.621 & 0.383 & 0.622 & 0.337 & 0.618 & 0.328 & 0.622 & 0.396 & 0.613 & 0.386 & 0.777 & 0.420 & 0.742 & 0.420\\
& 720 & 0.459 & 0.323 & \boldres{0.430} & \secondres{0.288} & 0.450 & 0.312 & 0.466 & 0.315 & \secondres{0.432} & \boldres{0.286} & 0.640 & 0.350 & 0.626 & 0.382 & 0.660 & 0.408 & 0.653 & 0.355 & 0.632 & 0.396 & 0.658 & 0.407 & 0.864 & 0.472 & 0.755 & 0.423\\
& Avg & 0.419 & 0.303 & \boldres{0.388} & \secondres{0.264} & 0.414 & 0.294 & 0.433 & 0.295 & \secondres{0.390} & \boldres{0.263} & 0.620 & 0.336 & 0.610 & 0.376 & 0.628 & 0.379 & 0.624 & 0.340 & 0.621 & 0.396 & 0.622 & 0.392 & 0.764 & 0.416 & 0.741 & 0.422\\

\bottomrule
\end{tabular}
}
\end{small}
\end{center}
\end{table}

\section{Visualizations}
\label{appx:visualizations}

\subsection{Visualization of Generated Time Series Images}

The image generation module employs advanced techniques—frequency and periodicity Encoding, multi-scale convolution, interpolation and normalization—to create informative and discriminative image representations of time series data. These representations enhance downstream VLMs for improved forecasting. As shown in Figure~\ref{fig:time_series_images}, the generated images capture key temporal characteristics through the following features:

\begin{figure}[h!]
    \centering
    \includegraphics[width=1\textwidth]{figures/time_series_images.pdf}
    \caption{Time series transformed images, capturing key temporal characteristics, including trends, stationarity, seasonality, sudden changes, and frequency-domain patterns.}
    \label{fig:time_series_images}
\end{figure}

\begin{itemize}[leftmargin=*, itemsep=0pt]
    \item \textbf{Frequency-Domain Information}: FFT integration captures frequency-domain characteristics, visualized as distinct textures—fine-grained for high-frequency components and broader color regions for low-frequency components.

    \item \textbf{Multi-scale Periodic Encoding}: Temporal dependencies at multiple scales (e.g., daily, weekly) are encoded, visible as regular patterns such as repeating vertical bands for daily cycles or broader horizontal patterns for weekly cycles.
    
    \item \textbf{Image Interpolation}: Bilinear interpolation ensures smooth and coherent images, preserving essential time series characteristics through seamless transitions between color intensities.

    \item \textbf{Color Trends}: Color intensity corresponds to time series values—darker regions (e.g., deep blue) indicate lower values, while brighter regions (e.g., yellow) represent higher values, enabling easy identification of trends.

    \item \textbf{Abrupt Changes and Anomalies}: Sudden shifts in color intensity (e.g., sharp transitions from dark to bright) highlight abrupt changes or anomalies, crucial for identifying irregular events like traffic spikes or weather shifts.
\end{itemize}    


\subsection{Visualization of prediction results}

The prediction results in Figures \ref{fig:vis_forecast_96}, \ref{fig:vis_forecast_192}, \ref{fig:vis_forecast_336}, and \ref{fig:vis_forecast_720} demonstrate \method's ability to accurately forecast time series across diverse datasets and prediction horizons. For datasets with clear periodic structures, such as the daily cycles in ETTh1 and ETTm1, \method captures both global trends and fine-grained temporal patterns effectively. This is evident in the close alignment between the true values (solid lines) and predicted values (dashed lines) across all horizons. Similarly, for the ECL dataset, which exhibits regular consumption patterns, \method delivers highly accurate forecasts, showcasing its strength in handling structured environments.

\begin{figure*}[h!]
    \centering
    \includegraphics[width=1\textwidth]{figures/forecasting_result_96.pdf}
    \caption{Prediction results visualization for ETTh1, ETTm1, ECL, and Traffic datasets at 96 prediction lengths. True values (solid line) and predicted values (dashed line) are shown for each dataset and horizon.}
    \label{fig:vis_forecast_96}
\end{figure*}

\begin{figure*}[h!]
    \centering
    \includegraphics[width=1\textwidth]{figures/forecasting_result_192.pdf}
    \caption{Prediction results visualization for ETTh1, ETTm1, ECL, and Traffic datasets at 192 prediction lengths. True values (solid line) and predicted values (dashed line) are shown for each dataset and horizon.}
    \label{fig:vis_forecast_192}
\end{figure*}

However, performance varies for datasets with irregular or abrupt changes. On the Traffic dataset, which is characterized by non-stationary patterns, \method shows slight deviations in capturing sudden fluctuations, particularly at longer horizons (e.g., 336 and 720). These deviations highlight the challenges of modeling highly irregular data and suggest opportunities for refining the time series-to-image transformation process to better handle such scenarios.

\begin{figure*}[h!]
    \centering
    \includegraphics[width=1\textwidth]{figures/forecasting_result_336.pdf}
    \caption{Prediction results visualization for ETTh1, ETTm1, ECL, and Traffic datasets at 336 prediction lengths. True values (solid line) and predicted values (dashed line) are shown for each dataset and horizon.}
    \label{fig:vis_forecast_336}
\end{figure*}

\begin{figure*}[h!]
    \centering
    \includegraphics[width=1\textwidth]{figures/forecasting_result_720.pdf}
    \caption{Prediction results visualization for ETTh1, ETTm1, ECL, and Traffic datasets at 720 prediction lengths. True values (solid line) and predicted values (dashed line) are shown for each dataset and horizon.}
    \label{fig:vis_forecast_720}
\end{figure*}
\section{Future Work}
\label{appx:future_work}

\subsection{Limitations}

While \method demonstrates significant improvements in time series forecasting by integrating temporal, visual, and textual modalities, it has some limitations.

First, the framework performs less robustly on datasets with highly volatile or irregular patterns, such as those with sudden changes or non-stationary trends, compared to datasets with periodic structures. This limitation may arise from the current visual transformation techniques, which may not adequately capture abrupt temporal dynamics or sudden shifts. Future work could refine these transformations to better handle such irregularities.

Second, the current implementation relies on pre-trained VLMs like ViLT and CLIP, which are optimized for natural vision-language tasks rather than time series forecasting. While these models excel in visual understanding, their textual capabilities are limited, often supporting only shorter text inputs and lacking domain-specific knowledge relevant to time series. This restricts their ability to fully utilize textual context for forecasting. Future work could involve developing larger, domain-specific VLMs trained on multimodal time series datasets to address these limitations.

\subsection{Future Work}

Building on the current framework, several promising directions for future research emerge:

\vspace{-1em}
\begin{itemize}[leftmargin=*, itemsep=0pt]
    \item \textbf{Optimizing Visual Transformations:} Future work could focus on developing adaptive visual transformation techniques that better preserve temporal dynamics, especially for datasets with irregular or non-stationary patterns, to more effectively highlight sudden changes and complex trends.

    \item \textbf{Scaling Multimodal VLMs for Enhanced Forecasting:}  While the current framework uses smaller pre-trained Vision-Language Models (VLMs), scaling to larger models could improve forecasting accuracy. Investigating trade-offs between model size, computational efficiency, and performance is a promising direction for future research. Additionally, studying different VLM architectures could identify optimal designs for temporal modeling.  

    \item \textbf{Interpretable Multimodal Learning for Time Series Analysis:}  Understanding the contributions of visual and textual modalities in time series forecasting is crucial for improving model transparency. Future work could explore the interpretability of multimodal features, analyzing how different types of information contribute to performance gains. This would provide deeper insights into temporal dependencies and enhance trust in multimodal forecasting models.  

    \item \textbf{Pre-training Multimodal Foundation Models for Time Series Analysis:} Existing VLMs are not designed to handle time series data, limiting their ability to capture domain-specific temporal context. Future research could focus on constructing large-scale multimodal datasets that pair time series data with rich textual and visual annotations, enabling the development of models specifically optimized for time series forecasting. Additionally, this multimodal framework could be extended to support multi-task learning, enhancing the model's versatility for tasks such as anomaly detection, classification, or imputation. This would allow the model to capture a broader range of temporal patterns and dependencies, improving its applicability across various domains.

\end{itemize}

By addressing these directions, future research can build on the foundation laid by \method, advancing the field of multimodal time series forecasting while ensuring responsible and ethical deployment in real-world applications.

%%%%%%%%%%%%%%%%%%%%%%%%%%%%%%%%%%%%%%%%%%%%%%%%%%%%%%%%%%%%%%%%%%%%%%%%%%%%%%%
%%%%%%%%%%%%%%%%%%%%%%%%%%%%%%%%%%%%%%%%%%%%%%%%%%%%%%%%%%%%%%%%%%%%%%%%%%%%%%%


\end{document}


% This document was modified from the file originally made available by
% Pat Langley and Andrea Danyluk for ICML-2K. This version was created
% by Iain Murray in 2018, and modified by Alexandre Bouchard in
% 2019 and 2021 and by Csaba Szepesvari, Gang Niu and Sivan Sabato in 2022.
% Modified again in 2023 and 2024 by Sivan Sabato and Jonathan Scarlett.
% Previous contributors include Dan Roy, Lise Getoor and Tobias
% Scheffer, which was slightly modified from the 2010 version by
% Thorsten Joachims & Johannes Fuernkranz, slightly modified from the
% 2009 version by Kiri Wagstaff and Sam Roweis's 2008 version, which is
% slightly modified from Prasad Tadepalli's 2007 version which is a
% lightly changed version of the previous year's version by Andrew
% Moore, which was in turn edited from those of Kristian Kersting and
% Codrina Lauth. Alex Smola contributed to the algorithmic style files.
