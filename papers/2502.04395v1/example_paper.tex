%%%%%%%% ICML 2025 EXAMPLE LATEX SUBMISSION FILE %%%%%%%%%%%%%%%%%

\documentclass{article}

% Recommended, but optional, packages for figures and better typesetting:
\usepackage{microtype}
\usepackage{graphicx}
\usepackage{subfigure}
\usepackage{booktabs} % for professional tablesx

% hyperref makes hyperlinks in the resulting PDF.
% If your build breaks (sometimes temporarily if a hyperlink spans a page)
% please comment out the following usepackage line and replace
% \usepackage{icml2025} with \usepackage[nohyperref]{icml2025} above.
\usepackage{hyperref}
\usepackage{makecell}
\usepackage{xcolor}

\usepackage{threeparttable}

% Attempt to make hyperref and algorithmic work together better:
\newcommand{\theHalgorithm}{\arabic{algorithm}}

% Use the following line for the initial blind version submitted for review:
% \usepackage{icml2025}

% If d, instead use the following line for the camera-ready submission:
\usepackage[accepted]{icml2025}

% For theorems and such
\usepackage{amsmath}
\usepackage{amssymb}
\usepackage{caption}
\usepackage{mathtools}
\usepackage{amsthm}
\usepackage{enumitem}
\usepackage{multirow}
\usepackage{xspace}
\usepackage{wrapfig}

% if you use cleveref..
\usepackage[capitalize,noabbrev]{cleveref}

%%%%%%%%%%%%%%%%%%%%%%%%%%%%%%%%
% THEOREMS
%%%%%%%%%%%%%%%%%%%%%%%%%%%%%%%%
\theoremstyle{plain}
\newtheorem{theorem}{Theorem}[section]
\newtheorem{proposition}[theorem]{Proposition}
\newtheorem{lemma}[theorem]{Lemma}
\newtheorem{corollary}[theorem]{Corollary}
\theoremstyle{definition}
\newtheorem{definition}[theorem]{Definition}
\newtheorem{assumption}[theorem]{Assumption}
\theoremstyle{remark}
\newtheorem{remark}[theorem]{Remark}

% Todonotes is useful during development; simply uncomment the next line
%    and comment out the line below the next line to turn off comments
%\usepackage[disable,textsize=tiny]{todonotes}
\usepackage[textsize=tiny]{todonotes}

\newcommand{\method}{\text{Time-VLM}\xspace}
\newcommand{\update}[1]{{\textcolor{black}{#1}}}
\newcommand{\boldres}[1]{{\textbf{\textcolor{red}{#1}}}}
\newcommand{\secondres}[1]{{\underline{\textcolor{blue}{#1}}}}

\definecolor{pink}{rgb}{1, 0, 0.5}
\newcommand{\revision}[1]{{\textcolor{black}{#1}}}

\newcommand*{\shortautoref}[1]{%
  \begingroup
    \def\sectionautorefname{Section}%
    \def\subsectionautorefname{Section}%
    \def\figureautorefname{Figure}%
    \def\tableautorefname{Table}%
    \def\equationautorefname{Equation}%
    \autoref{#1}%
  \endgroup
}


% The \icmltitle you define below is probably too long as a header.
% Therefore, a short form for the running title is supplied here:
% \icmltitlerunning{Submission and Formatting Instructions for ICML 2025}

\begin{document}

\twocolumn[
\icmltitle{\method: Exploring Multimodal Vision-Language Models for Augmented Time Series Forecasting}

% It is OKAY to include author information, even for blind
% submissions: the style file will automatically remove it for you
% unless you've provided the [accepted] option to the icml2025
% package.

% List of affiliations: The first argument should be a (short)
% identifier you will use later to specify author affiliations
% Academic affiliations should list Department, University, City, Region, Country
% Industry affiliations should list Company, City, Region, Country

% You can specify symbols, otherwise they are numbered in order.
% Ideally, you should not use this facility. Affiliations will be numbered
% in order of appearance and this is the preferred way.
% \icmlsetsymbol{equal}{*}

\begin{icmlauthorlist}
\icmlauthor{Siru Zhong}{hkustgz}
\icmlauthor{Weilin Ruan}{hkustgz}
\icmlauthor{Ming Jin}{gu}
\icmlauthor{Huan Li}{zju}
\icmlauthor{Qingsong Wen}{sa}
\icmlauthor{Yuxuan Liang}{hkustgz}
\end{icmlauthorlist}

\icmlaffiliation{hkustgz}{The Hong Kong University of Science and Technology (Guangzhou)}
\icmlaffiliation{gu}{Griffith University}
\icmlaffiliation{zju}{Zhejiang University}
\icmlaffiliation{sa}{Squirrel AI}

\icmlcorrespondingauthor{Yuxuan Liang}{yuxliang@outlook.com}

% You may provide any keywords that you
% find helpful for describing your paper; these are used to populate
% the "keywords" metadata in the PDF but will not be shown in the document
% \icmlkeywords{Machine Learning, ICML}

\vskip 0.3in
]

% this must go after the closing bracket ] following \twocolumn[ ...

% This command actually creates the footnote in the first column
% listing the affiliations and the copyright notice.
% The command takes one argument, which is text to display at the start of the footnote.
% The \icmlEqualContribution command is standard text for equal contribution.
% Remove it (just {}) if you do not need this facility.

%\printAffiliationsAndNotice{}  % leave blank if no need to mention equal contribution
\printAffiliationsAndNotice{} % otherwise use the standard text.

Vision-Language Models (VLMs) occasionally generate outputs that contradict input images, constraining their reliability in real-world applications. While visual prompting is reported to suppress hallucinations by augmenting prompts with relevant area inside an image, the effectiveness in terms of the area remains uncertain. This study analyzes success and failure cases of Attention-driven visual prompting in object hallucination, revealing that preserving background context is crucial for mitigating object hallucination.

Large language models (LLMs) show significant performance in various downstream
tasks~\citep{brown_language_2020,openai_gpt-4_2024,dubey_llama_2024}. Studies
have found that training on high quality corpus improves the ability of LLMs
to solve different problems such as writing code, doing math exercises, and
answering logic questions~\citep{cai_internlm2_2024,deepseek-ai_deepseek-v3_2024,qwen_qwen25_2024}.
Therefore, effectively selecting high-quality text data is an important subject for
training LLM.

\begin{figure}[t]
    \centering
    \includegraphics[width=\linewidth]{figures/head.pdf}
    \caption{The overview of CritiQ. We (1) employ human annotators to annotate $\sim$30
    pairwise quality comparisons, (2) use CritiQ Flow to mine quality criteria, (3)
    use the derived criteria to annotate 25k pairs, and (4) train the CritiQ Scorer to
    perform efficient data selection.}
    \label{fig:overview}
\end{figure}

To select high-quality data from a large corpus, researchers manually design heuristics~\citep{dubey_llama_2024,rae_scaling_2022},
calculate perplexity using existing LLMs~\citep{marion2023moreinvestigatingdatapruning,wenzek2019ccnetextractinghighquality},
train classifiers~\citep{brown_language_2020,dubey_llama_2024,xie_data_2023} and
query LLMs for text quality through careful prompt engineering~\citep{gunasekar_textbooks_2023,wettig_qurating_2024,sachdeva_how_2024}.
Large-scale human annotation and prompt engineering require a lot of human
effort. Giving a comprehensive description of what high-quality data is like is also
challenging. As a result, manually designing heuristics lacks robustness and introduces
biases to the data processing pipeline, potentially harming model performance
and generalization. In addition, quality standards vary across different
domains. These methods can not be directly applied to other domains without significant
modifications.

To address these problems, we introduce CritiQ, a novel method to automatically
and effectively capture human preferences for data quality and perform efficient data
selection. Figure~\ref{fig:overview} gives an overview of CritiQ, comprising an agent
workflow, CritiQ Flow, and a scoring model, CritiQ Scorer. Instead of manually describing
how high quality is defined, we employ LLM-based agents to summarize quality
criteria from only $\sim$30 human-annotated pairs.

CritiQ Flow starts from a knowledge base of data quality criteria. The worker
agents are responsible to perform pairwise judgment under a given
criterion. The manager agent generates new criteria and refines them through reflection
on worker agents' performance. The final judgment is made by majority voting among
all worker agents, which gives a multi-perspective view of data quality.

To perform efficient data selection, we employ the worker agents to annotate a randomly
selected pairwise subset, which is ~1000x larger than the human-annotated one.
Following \citet{korbak_pretraining_2023,wettig_qurating_2024}, we train CritiQ
Scorer, a lightweight Bradley-Terry model~\citep{bradley_rank_1952} to convert
pairwise preferences into numerical scores for each text. We use CritiQ Scorer to
score the entire corpus and sample the high-quality subset.

For our experiments, we established human-annotated test sets to quantitatively
evaluate the agreement rate with human annotators on data quality preferences. We implemented the manager agent by \texttt{GPT-4o} and the worker
agent by \texttt{Qwen2.5-72B-Insruct}. We conducted experiments on different
domains including code, math, and logic, in which CritiQ Flow shows a consistent
improvement in the accuracies on the test sets, demonstrating the effectiveness
of our method in capturing human preferences for data quality. To validate the quality
of the selected dataset, we continually train \texttt{Llama 3.1}~\citep{dubey_llama_2024}
models and find that the models achieve better performance on downstream tasks
compared to models trained on the uniformly sampled subsets.

We highlight our contributions as follows. We will release the code to facilitate
future research.

\begin{itemize}
    \item We introduce CritiQ, a method that captures human preferences for data
        quality and performs efficient data selection at little cost of human
        annotation effort.

    \item Continual pretraining experiments show improved model performance in code,
        math, and logic tasks trained on our selected high-quality subset compared to the raw dataset.

    \item Ablation studies demonstrate the effectiveness of the knowledge base and
        the the reflection process.
\end{itemize}

\begin{figure*}[t]
    \centering
    \includegraphics[width=\linewidth]{figures/method.pdf}
    \caption{CritiQ Flow comprises two major components: multi-criteria pairwise
    judgment and the criteria evolution process. The multi-criteria pairwise
    judgment process employs a series of worker agents to make quality
    comparisons under a certain criterion. The criteria evolution process aims to
    obtain data quality criteria that highly align with human judgment through
    an iterative evolution. The initial criteria are retrieved from the
    knowledge base. After evolution, we select the final criteria to annotate
    the dataset for training CritiQ Scorer.}
    \label{fig:method}
\end{figure*}

\section{Related Work}

\begin{figure*}[t!]
    \centering
    \includegraphics[width=0.99\textwidth]{figures/framework.pdf}
    \caption{Overview of the \method framework.}
    \label{fig:framework}
    \vspace{-1em}
\end{figure*}

\noindent\textbf{Text-Augmented Models for Time Series Forecasting.} 

The success of LLMs inspires their application to time series tasks. Methods like LLMTime~\cite{gruver2023large} and LLM4TS~\cite{chang2023llm4ts} tokenize time series data for autoregressive prediction but inherit LLMs' limitations, such as poor arithmetic and recursive capabilities. Recent approaches, including GPT4TS~\cite{zhou2023one} and TimeLLM~\cite{jin2023time}, project time series into textual representations to leverage LLMs' reasoning abilities. However, they face challenges like the modality gap and lack of time series-optimized word embeddings, leading to potential information loss. UniTime~\cite{liu2024unitime} and TimeFFM~\cite{liu2024time} incorporate domain-specific instructions and federated learning, respectively, but remain constrained by their reliance on text alone.

\noindent\textbf{Vision-Augmented Models for Time Series Forecasting.} 

Vision emerges as a natural way to preserve temporal patterns. Early approaches use CNNs for matrix-formed time series~\cite{li2020forecasting, sood2021visual}, while TimesNet~\cite{wu2023timesnet} introduces multi-periodic decomposition for unified 2D modeling. VisionTS~\cite{chen2024visiontsvisualmaskedautoencoders} pioneers pre-trained visual encoders with grayscale time series images, and TimeMixer++~\cite{wang2024timemixer++} advances the field with multi-scale frequency-based time-image transformations. Despite their effectiveness in temporal modeling, these methods often lack semantic context, hard to use high-level contextual information for prediction.

\noindent\textbf{Vision-Language Models.} 

VLMs like ViLT~\cite{kim2021vilt}, CLIP~\cite{radford2021learning}, and ALIGN~\cite{jia2021scaling} transform multimodal understanding by aligning visual and textual representations. Recent advancements, like BLIP-2~\cite{li2022blip2} and LLaVA~\cite{liu2023visual}, further enhance multimodal reasoning. However, VLMs remain underexplored for time series analysis. Our work bridges this gap by leveraging VLMs to integrate temporal, visual, and textual modalities, addressing the limitations of unimodal approaches.
\begin{figure}[t!]
    \centering
    \includegraphics[width=0.45\textwidth]{images/method_visual-crop.pdf}
    \caption{Process of Visual Attention Evaluation.}
    \label{img1}
\end{figure}

\subsection{API Prompting}
API Prompting~(sketched in Figure~\ref{api}) is a Visual Prompting method that highlights important parts in an image using a Visual Attention Heatmap derived from a Vision-Language Model~\citep{api}. The Attribution Map, representing the contribution of image tokens to model outputs, is extracted from a VLM (referred to as Heatmap VLM or H-VLM), convolved, resized to match the image size, and then overlaid on the original image.

Following the study by \citet{api}, Vision-Transformer-based CLIP and LLaVA are employed as Heatmap VLMs. The methods for extracting Visual Attention Attribution Maps from each model are described below.

\paragraph{CLIP Attribution Map}
CLIP computes similarity between text and image representations, and the Attribution Map \( \Psi \) is obtained by decomposing the similarity function \( \text{sim}(\hat{I}, \hat{T}) \). %The image representation \( \hat{I} \) is expressed as:
% \begin{equation}
%    \begin{aligned}
%        \hat{I} &= \mathcal{L}([Z^{0}_{\text{cls}}])
%         + \sum_{\ell=1}^{L} \mathcal{L}([\text{MSA}^{\ell}(Z^{\ell-1})]_{\text{cls}}) \\
%        &\quad + \sum_{\ell=1}^{L} \mathcal{L}([\text{MLP}^{\ell}(\hat{Z}^{\ell})]_{\text{cls}}).
%    \end{aligned}
% \end{equation}

Since later MSA layers greatly impact image representation~\citep{clipdec}, the similarity function is approximated as:
\begin{equation}
   \begin{aligned}
\text{sim}(\hat{I}, \hat{T}) \approx \text{sim}\left(\sum_
{\ell=L'}^{L} \mathcal{L}(\text{MSA}^{\ell}([Z^{\ell-1}]))_{\text{cls}}, \hat{T}\right).
\end{aligned}
\end{equation}
To filter out irrelevant regions, a complementary Attribution Map \( \Psi^{\text{comp}} \) is introduced:
\begin{equation}
   \begin{aligned}
   \Psi_{i,j}^{\text{comp}} &\triangleq 1 - \text{sim}(\mathcal{L}(Z_{t}^{L}), \hat{T}), \\
&\quad \text{where} \quad t = 1 + j + P \cdot (i - 1).
\end{aligned}
\end{equation}
Combining both maps, the final CLIP Attribution Map is defined as:
\begin{equation}
   \begin{aligned}
   \Psi = \Psi^{\text{cls}} + \Psi^{\text{comp}} - \Psi^{\text{comp}} \odot \Psi^{\text{cls}}.
\end{aligned}
\end{equation}

\paragraph{LLaVA Attribution Map}
LLaVA can provide an Attribution Map \( \Psi \) using Multi-Head Self-Attention (MSA) weights between output text tokens and image tokens. The Attribution Map is computed by averaging over all output tokens and attention heads:
\begin{equation}
   \begin{aligned}
\Psi_{i,j} &\triangleq \frac{1}{MH} 
  \sum_{m=1}^{M} 
  \sum_{h=1}^{H} 
  A_{m,t}^{(\bar{L},h)}, \\
&\quad \text{where} \quad 
t = j + P \cdot (i - 1).
\end{aligned}
\end{equation}
Here, \(M\) is the number of output tokens, \(H\) is the number of attention heads, \(P\) is the number of patches per image side, and \(A^{(\bar{L},h)}\) represents cross-attention between output text and image tokens at layer \(\bar{L}\) and attention head \(h\).

\subsection{Background Role Examination}
To assess the necessity of background information for object recognition, ground truth segmentation data is used as a Heatmap during API Prompting and the accuracy of output is evaluated (hereafter referred to as API - Seg.). Binary segmentation masks, overlaid in gray are input into VLMs to evaluate their impact on output accuracy. If POPE response accuracy remains unchanged, background information is deemed unnecessary.

\subsection{Minimum Cutoff}
Minimum cutoff redefines the minimum value in segmentation or Visual Attention Heatmap based on a threshold. Since a threshold of 0.5 showed improvement in Table~\ref{table2}, values below 0.5 in the cutoff are replaced with 0.5, refining segmentation granularity.

\begin{table*}[t]
\centering
\begin{tabular}{lllrrrrr}
\toprule
\textbf{Dataset} & \textbf{Model} & \textbf{Prompting} & \textbf{Acc.} & \textbf{Prec.} & \textbf{Rec.} & \textbf{TNR} & \textbf{F1} %& \textbf{Yes (\%)} 
\\ \cmidrule(lr){1-8}
\multirow{7}{*}{\textbf{MSCOCO}} & \multirow{7}{*}{LLaVA} & w/o prpt. & 86.23 & 84.21  & 89.19 & 83.27 & 86.63 \\ %& 52.96 \\
& & API~(CLIP) & \ensuremath{\blacktriangle}86.52 & \ensuremath{\blacktriangle}84.78 & \ensuremath{\triangledown}89.02 & \ensuremath{\blacktriangle}84.02 & \ensuremath{\blacktriangle}86.85 \\ %& \ensuremath{\blacktriangle}52.50  \\
& & API~(CLIP) w Cutoff & \ensuremath{\blacktriangle}88.59 & \ensuremath{\blacktriangle}85.84 & \ensuremath{\blacktriangle}92.43 & \ensuremath{\blacktriangle}84.75 & \ensuremath{\blacktriangle}89.01 \\
 & & API~(LLaVA) & \ensuremath{\triangledown}86.11 & \ensuremath{\blacktriangle}84.72 & \ensuremath{\triangledown}88.12 & \ensuremath{\blacktriangle}84.10 & \ensuremath{\triangledown}86.39 \\ %&  \ensuremath{\blacktriangle}52.01 \\
 & & API~(LLaVA) w Cutoff & \ensuremath{\blacktriangle}87.98 & \ensuremath{\blacktriangle}85.10 & \ensuremath{\blacktriangle}92.09 & \ensuremath{\blacktriangle}83.88 & \ensuremath{\blacktriangle}88.46 \\
  & & API - Seg. & - & - & \ensuremath{\triangledown}71.78 & - &  - \\ %& - \\
  & & API - Seg. w Cutoff & - & - & \ensuremath{\blacktriangle}89.24 & - &  - \\ %& - \\
 \bottomrule
\end{tabular}
\caption{POPE results on MSCOCO datasets with API Prompting.}
\label{table1}
\end{table*}

\begin{table*}[h!]
\centering
\begin{tabular}{llrrrr}
\toprule
%\toprule
\multirow{2}{*}{\textbf{H-VLM}} & \multirow{2}{*}{\textbf{Output}} & \multicolumn{4}{c}{\textbf{Visual Attention Alignment}} \\ %\cline{2-9} 
& & \textbf{Prec.} & \textbf{Rec.}  & \textbf{IoU}    & \textbf{MSE}   \\ 
%\cmidrule(lr){1-9}
\cmidrule(lr){1-6}
%Overall   & 0.1270   & 0.8318 & 0.1092 & 0.2920  & 0.1010   & 0.6523 & 0.0767 & 0.1337 \\ \hline
%\multirow{3}{*}{CLIP} & - & 12.70 & 83.18 & 10.92 & 29.20 \\
\multirow{2}{*}{CLIP}& Correct~(87\%)  &\ensuremath{\blacktriangle}13.52   & \ensuremath{\blacktriangle}83.94 & \ensuremath{\blacktriangle}11.68 & \ensuremath{\blacktriangle}28.46  \\
& Incorrect~(13\%) & 6.09   & 77.02 & 4.78 & 35.22 \\
%\multirow{3}{*}{LLaVA} & - & 10.10 & 65.23 & 7.67 & 13.37 \\
\multirow{2}{*}{LLaVA}& Correct~(86\%)  & \ensuremath{\blacktriangle}10.62  & 64.62 & \ensuremath{\blacktriangle}8.10 & 13.60 \\ 
& Incorrect~(14\%) & 6.17   & \ensuremath{\blacktriangle}69.78 & 4.44 & \ensuremath{\blacktriangle}11.60 \\
\bottomrule
%\bottomrule
\end{tabular}
\caption{Alignment of CLIP/LLaVA Visual Attention.}
\label{table3}
\end{table*}

\subsection{Evaluation of Visual Attention Alignment}
To assess how well visual attention focuses on target objects, Precision, Recall, Intersection over Union~(IoU), and Mean Squared Error~(MSE) are computed between the object segmentation data and the Visual Attention Heatmap as depicted in Figure~\ref{img1}. 
Visual Attention Heatmaps are converted into binary arrays using thresholds set to the average value of each heatmap.


% \begin{figure}[t]
%      \centering
%     \includegraphics[width=0.47\textwidth]{images/cutoff-crop.pdf}
%      \caption{Example of Cutoff Segmentation Annotation}    \label{clip}
%  \end{figure}
\section{Experiments}

\input{tables/few-shot-forecasting-5}
\input{tables/few-shot-forecasting-10}
\begin{table}[!h]
\renewcommand\arraystretch{1.2}
\begin{center}
\captionsetup{font=small}
\caption{\revision{Zero-shot learning results. Full results see \shortautoref{appx:zero-shot}.}}
\label{tab:zero-shot-forecasting-brief}
\vspace{-1em}
\begin{small}
\scalebox{0.56}{
\setlength\tabcolsep{2.5pt}
\begin{tabular}{c|cc|cc|cc|cc|cc|cc}
\toprule
\multicolumn{1}{c|}{\multirow{2}{*}{Methods}}
&\multicolumn{2}{c|}{\method\textcolor{green!60!black}{\textsubscript{\textbf{143M}}}}&\multicolumn{2}{c|}{Time-LLM\textcolor{orange}{\textsubscript{\textbf{3405M}}}}&\multicolumn{2}{c|}{LLMTime}&\multicolumn{2}{c|}{GPT4TS}&\multicolumn{2}{c|}{DLinear}&\multicolumn{2}{c}{PatchTST}\\

\multicolumn{1}{c|}{} & \multicolumn{2}{c}{\scalebox{0.99}{(\textbf{Ours})}} & 
\multicolumn{2}{|c|}{\scalebox{0.99}{\citeyearpar{jin2023time}}} &
\multicolumn{2}{c|}{\scalebox{0.99}{\citeyearpar{gruver2023large}}} &
\multicolumn{2}{c|}{\scalebox{0.99}{\citeyearpar{zhou2023one}}} & \multicolumn{2}{c|}{\scalebox{0.99}{\citeyearpar{zeng2023transformers}}} & \multicolumn{2}{c}{\scalebox{0.99}{\citeyearpar{nie2022time}}}  \\

\midrule

\multicolumn{1}{c|}{Metric} & MSE & MAE & MSE & MAE & MSE & MAE & MSE & MAE & MSE & MAE& MSE & MAE \\
\midrule
\multirow{1}{*}{\rotatebox{0}{$ETTh1$} $\rightarrow$ \rotatebox{0}{$ETTh2$}}  
& \boldres{0.338} & \boldres{0.385} & \secondres{0.353} & \secondres{0.387} & 0.992 & 0.708 & 0.406 & 0.422 & 0.493 & 0.488 & 0.380 & 0.405 \\
\midrule
\multirow{1}{*}{\rotatebox{0}{$ETTh1 $} $\rightarrow$ \rotatebox{0}{$ETTm2 $}}
& \secondres{0.293} & \secondres{0.350} & \boldres{0.273} & \boldres{0.340} & 1.867 & 0.869 & 0.325 & 0.363 & 0.415 & 0.452 & 0.314 & 0.360 \\
\midrule
\multirow{1}{*}{\rotatebox{0}{$ETTh2 $} $\rightarrow$ \rotatebox{0}{$ETTh1 $}}
& \secondres{0.496} & \secondres{0.480} & \boldres{0.479} & \boldres{0.474} & 1.961 & 0.981 & 0.757 & 0.578 & 0.703 & 0.574 & 0.565 & 0.513 \\
\midrule
\multirow{1}{*}{\rotatebox{0}{$ETTh2 $} $\rightarrow$ \rotatebox{0}{$ETTm2 $}}
& \secondres{0.297} & \secondres{0.353} & \boldres{0.272} & \boldres{0.341} & 1.867 & 0.869 & 0.335 & 0.370 & 0.328 & 0.386 & 0.325 & 0.365 \\
\midrule
\multirow{1}{*}{\rotatebox{0}{$ETTm1 $} $\rightarrow$ \rotatebox{0}{$ETTh2 $}}
& \boldres{0.354} & \boldres{0.397} & \secondres{0.381} & \secondres{0.412} & 0.992 & 0.708 & 0.433 & 0.439 & 0.464 & 0.475 & 0.439 & 0.438 \\
\midrule
\multirow{1}{*}{\rotatebox{0}{$ETTm1 $} $\rightarrow$ \rotatebox{0}{$ETTm2 $}}
& \boldres{0.264} & \boldres{0.319} & \secondres{0.268} & \secondres{0.320} & 1.867 & 0.869 & 0.313 & 0.348 & 0.335 & 0.389 & 0.296 & 0.334 \\
\midrule
\multirow{1}{*}{\rotatebox{0}{$ETTm2 $} $\rightarrow$ \rotatebox{0}{$ETTh2 $}}
& \secondres{0.359} & \boldres{0.399} & \boldres{0.354} & \secondres{0.400} & 0.992 & 0.708 & 0.435 & 0.443 & 0.455 & 0.471 & 0.409 & 0.425 \\
\midrule
\multirow{1}{*}{\rotatebox{0}{$ETTm2 $} $\rightarrow$ \rotatebox{0}{$ETTm1 $}}
& \secondres{0.432} & \boldres{0.426} & \boldres{0.414} & \secondres{0.438} & 1.933 & 0.984 & 0.769 & 0.567 & 0.649 & 0.537 & 0.568 & 0.492 \\

\bottomrule
\end{tabular}
}
\end{small}
\end{center}
\vspace{-2em}
\end{table}

\noindent\textbf{Datasets and Metrics.} We evaluate \method on seven widely-used time series datasets spanning diverse domains, including energy consumption (ETTh1, ETTh2, ETTm1, ETTm2), weather forecasting, electricity load prediction (ECL, 321 variables), and traffic flow estimation (Traffic, 862 variables)~\cite{zhou2021informer, lai2018modeling}. These datasets, extensively adopted for benchmarking long-term forecasting models~\cite{wu2022timesnet}, exhibit varying characteristics in sampling frequency, dimensionality, and temporal patterns. For short-term forecasting, we utilize the M4 benchmark~\citep{makridakis2018m4}, which encompasses marketing data at various sampling frequencies. Forecasting performance is evaluated using Mean Absolute Error (MAE) and Mean Squared Error (MSE), following standard practices in the field. Additional details on datasets and metrics are provided in Appendix~\ref{appx:dataset_details} and~\ref{appx:evaluation_metric}.

% \noindent\textbf{Baselines.} We compare \method with state-of-the-art time series models, including text-augmented methods like TimeLLM \citeyearpar{jin2023time}, GPT4TS \citeyearpar{zhou2023one}, and LLMTime \citeyearpar{gruver2023large}; vision-augmented methods like TimesNet \citeyearpar{wu2023timesnet}; traditional deep models like PatchTST \citeyearpar{nie2022time}, ESTformer \citeyearpar{woo2022etsformer}, Non-Stationary Transformer \citeyearpar{liu2022non}, FEDformer \citeyearpar{zhou2022fedformer}, Autoformer \citeyearpar{wu2021autoformer}, Informer \citeyearpar{zhou2021informer}, and Reformer \citeyearpar{kitaev2020reformer}; and recent competitive models like DLinear \citeyearpar{zeng2023transformers}, LightTS \citeyearpar{zhang2022less}, N-HiTS \citeyearpar{challu2023nhits}, and N-BEATS \citeyearpar{oreshkin2019n}. Performance results for some baselines are cited from \cite{liu2024time} where applicable.


\noindent\textbf{Baselines.} We compare \method with state-of-the-art time series models, including text-augmented methods like TimeLLM \citeyearpar{jin2023time}, GPT4TS \citeyearpar{zhou2023one}, and LLMTime \citeyearpar{gruver2023large}; vision-augmented methods like TimesNet \citeyearpar{wu2023timesnet}; traditional deep models like PatchTST \citeyearpar{nie2022time}, ESTformer \citeyearpar{woo2022etsformer}, Non-Stationary Transformer \citeyearpar{liu2022non}, FEDformer \citeyearpar{zhou2022fedformer}, Autoformer \citeyearpar{wu2021autoformer}, Informer \citeyearpar{zhou2021informer}, and Reformer \citeyearpar{kitaev2020reformer}; and recent competitive models like DLinear \citeyearpar{zeng2023transformers}, LightTS \citeyearpar{zhang2022less}, N-HiTS \citeyearpar{challu2023nhits}, and N-BEATS \citeyearpar{oreshkin2019n}. Notably, \method is the first framework combining three modalities for time series forecasting. Performance results for some baselines are cited from \citeyearpar{liu2024time} where applicable.


\noindent\textbf{Implementation Details.} We compare \method against state-of-the-art models using a unified evaluation pipeline, following the configurations in \citep{wu2022timesnet} for fair comparison. ViLT \citep{kim2021vilt} is the default backbone, with \texttt{"vilt-b32-finetuned-coco"}. Other VLMs like CLIP and BLIP-2 are also supported. All models are trained with the Adam optimizer (learning rate $10^{-3}$, halved every epoch), a batch size of 32, and a maximum of 10 epochs with early stopping. Experiments are conducted on an Nvidia RTX A6000 GPU with 48GB memory. Additional optimization details are in Appendix~\ref{appx:optimization_settings}.

\begin{table*}[h!]
\renewcommand\arraystretch{1.2}
\captionsetup{font=small} 
\caption{Short-term time series forecasting results on M4. The forecasting horizons are in [6, 48] and the three rows provided are weighted averaged from all datasets under different sampling intervals. Full results see \shortautoref{appx:short-term}.}
\label{tab:short-term-forecasting}
\vspace{-1em}
\begin{center}
\begin{small}
\scalebox{0.69}{
\setlength\tabcolsep{2.5pt}
\begin{tabular}{cc|ccccccccccccccc}
\toprule

\multicolumn{2}{c|}{\multirow{2}{*}{Methods}}
& \multicolumn{1}{c|}{\method{}\textcolor{green!60!black}{\textsubscript{\textbf{143M}}}} &\multicolumn{1}{c|}{Time-LLM\textcolor{orange}{\textsubscript{\textbf{3405M}}}}&\multicolumn{1}{c|}{GPT4TS} &\multicolumn{1}{c|}{TimesNet}&\multicolumn{1}{c|}{PatchTST}&\multicolumn{1}{c|}{N-HiTS}&\multicolumn{1}{c|}{N-BEATS}& \multicolumn{1}{c|}{ETSformer}& \multicolumn{1}{c|}{LightTS}& \multicolumn{1}{c|}{DLinear} &\multicolumn{1}{c|}{FEDformer} &\multicolumn{1}{c|}{Stationary} &\multicolumn{1}{c|}{Autoformer}  &\multicolumn{1}{c|}{Informer} &\multicolumn{1}{c}{Reformer} \\

\multicolumn{2}{c|}{} & \multicolumn{1}{c|}{\scalebox{0.99}{(\textbf{Ours})}} & \multicolumn{1}{c|}{\scalebox{0.99}{\citeyearpar{jin2023time}}} & \multicolumn{1}{c|}{\scalebox{0.99}{\citeyearpar{zhou2023one}}} & \multicolumn{1}{c|}{\scalebox{0.99}{\citeyearpar{wu2022timesnet}}} &
\multicolumn{1}{c|}{\scalebox{0.99}{\citeyearpar{nie2022time}}} &
\multicolumn{1}{c|}{\scalebox{0.99}{\citeyearpar{challu2023nhits}}} &
\multicolumn{1}{c|}{\scalebox{0.99}{\citeyearpar{oreshkin2019n}}} &
\multicolumn{1}{c|}{\scalebox{0.99}{\citeyearpar{woo2022etsformer}}} &
\multicolumn{1}{c|}{\scalebox{0.99}{\citeyearpar{zhang2022less}}} &
\multicolumn{1}{c|}{\scalebox{0.99}{\citeyearpar{zeng2023transformers}}} &
\multicolumn{1}{c|}{\scalebox{0.99}{\citeyearpar{zhou2022fedformer}}} &
\multicolumn{1}{c|}{\scalebox{0.99}{\citeyearpar{liu2022non}}} &
\multicolumn{1}{c|}{\scalebox{0.99}{\citeyearpar{wu2021autoformer}}} &
\multicolumn{1}{c|}{\scalebox{0.99}{\citeyearpar{zhou2021informer}}} &
\multicolumn{1}{c}{\scalebox{0.99}{\citeyearpar{kitaev2020reformer}}} \\

\midrule

\multirow{3}{*}{\rotatebox{90}{Average}}
&SMAPE &\boldres{11.894} &\secondres{11.983} &12.690 &12.880 &12.059 &12.035 &12.250 &14.718 &13.525 &13.639 &13.160 &12.780 &12.909 &14.086 &18.200 \\
&MASE &\boldres{1.592} &\secondres{1.595} &1.808 &1.836 &1.623 &1.625 &1.698 &2.408 &2.111 &2.095 &1.775 &1.756 &1.771 &2.718 &4.223 \\
&OWA &\boldres{0.855} &\secondres{0.859} &0.940 &0.955 &0.869 &0.869 &0.896 &1.172 &1.051 &1.051 &0.949 &0.930 &0.939 &1.230 &1.775 \\

\bottomrule
\end{tabular}
}
\end{small}
\end{center}
\vspace{-1em}
\end{table*}
\input{tables/long-term-forecasting}

\subsection{Few-shot Forecasting}

We evaluate the few-shot capabilities of \method by testing its performance using only 5\% or 10\% of training data. This assesses its ability to combine pre-trained multimodal knowledge from VLM with time series-specific features for effective forecasting under minimal task-specific data.

As shown in \shortautoref{tab:few-shot-forecasting-5per} and \shortautoref{tab:few-shot-forecasting-10per}, \method consistently outperforms most baselines across datasets. For example, on ETTh1 with 5\% training data, \method reduces MSE by 29.5\% and MAE by 16.6\% compared to the second-best model, TimeLLM. On ETTm1 with 10\% data, it surpasses TimeLLM by 11.1\% in MSE and 10.5\% in MAE. On Weather with 5\% data, \method outperforms TimeLLM by 7.7\% in MSE and 9.4\% in MAE.

The performance gap between \method and traditional models (e.g., PatchTST, FEDformer) is more pronounced in few-shot settings, demonstrating the superiority of multi-modality in data-scarce scenarios. Notably, \method achieves this with only 143M parameters, significantly fewer than TimeLLM's 3405M, highlighting its efficiency.

\subsection{Zero-shot Forecasting}

We evaluate the zero-shot capability of \method in cross-domain settings, where the model predicts on unseen datasets by effectively leveraging knowledge from unrelated domains. To ensure a more comprehensive and rigorous comparison, we use the ETT datasets as Time-LLM \cite{jin2023time}, with results summarized in \shortautoref{tab:zero-shot-forecasting-brief}.

\method demonstrates strong zero-shot generalization, consistently outperforming or matching state-of-the-art baselines with fewer parameters. For example, in $ETTh1 \rightarrow ETTh2$, \method surpasses TimeLLM with a 4.2\% lower MSE and 0.5\% lower MAE. In $ETTm1 \rightarrow ETTh2$, it outperforms TimeLLM by 7.1\% in MSE and 3.6\% in MAE. In $ETTm2 \rightarrow ETTh2$, \method achieves competitive performance, closely matching TimeLLM with only a 1.4\% difference in MSE and 0.3\% in MAE.

\subsection{Short-term Forecasting}

For short-term forecasting, we evaluate \method on the M4 benchmark, which includes marketing data at various sampling frequencies. Performance is measured using SMAPE, MASE, and OWA metrics, averaged across datasets and sampling intervals (see \shortautoref{tab:short-term-forecasting}).

\method demonstrates strong performance, consistently outperforming state-of-the-art baselines across all metrics. For instance, it surpasses the second-best model, Time-LLM, with improvements of 0.7\% in SMAPE, 0.2\% in MASE, and 0.5\% in OWA, all while utilizing significantly fewer parameters and computational resources. Compared to traditional models like PatchTST and N-HiTS, the performance gains more, highlighting the benefit of multimodal knowledge in short-term forecasting. These gains stem from \method's integration of temporal, visual, and textual data, capturing richer features for improved accuracy.

\subsection{Long-term Forecasting}

We evaluate the long-term forecasting capabilities of \method across diverse temporal horizons and datasets.

As shown in \shortautoref{tab:long-term-forecasting}, \method achieves competitive performance compared to state-of-the-art baselines. For example, on ETTh1, \method surpasses TimeLLM with 0.7\% improvements in MSE and MAE. On ETTm2, it outperforms TimeLLM by 1.2\% in MSE and 0.6\% in MAE. However, on Weather, \method slightly trails TimeLLM with a 0.4\% higher MSE and 2.3\% higher MAE.

Overall, \method demonstrates robust performance across diverse tasks and datasets, highlighting its generalization and efficiency. By leveraging multimodal knowledge, it consistently outperforms state-of-the-art baselines with significantly fewer parameters (143M vs. TimeLLM's 3405M), making it a practical solution for real-world applications.

\subsection{Model Analysis}

\noindent\textbf{Ablation Studies:} \autoref{tab:multimodal_ablation} evaluates the contributions of key components of \method, including the RAL, VAL, and TAL. Results are averaged across forecasting horizons $H \in \{96, 192, 336, 720\}$ on the Weather dataset, with performance degradation (\textit{\%Deg}) measured for each variant.

\vspace{-0.5em}
\begin{table}[h!]
\renewcommand\arraystretch{1.2}
\captionsetup{font=small} 
\caption{Ablation study on multimodal components.}
\vspace{-1em}
\label{tab:multimodal_ablation}
\begin{center}
\begin{small}
\scalebox{0.75}{
\setlength\tabcolsep{4pt}
\begin{tabular}{@{}ccccccccc@{}}
\toprule
\multirow{2}{*}{Horizon} 
& \multicolumn{2}{c}{Full} 
& \multicolumn{2}{c}{w/o RAL} 
& \multicolumn{2}{c}{w/o VAL} 
& \multicolumn{2}{c}{w/o TAL} \\

\cmidrule(lr){2-3} \cmidrule(lr){4-5} \cmidrule(lr){6-7} \cmidrule(lr){8-9}
& MSE & MAE & MSE & MAE & MSE & MAE & MSE & MAE \\
\midrule
96  & \boldres{0.160} & \boldres{0.213} & 0.273 & 0.324 & 0.213 & 0.266 & \secondres{0.165} & \secondres{0.218} \\
192 & \boldres{0.203} & \boldres{0.252} & 0.297 & 0.338 & 0.237 & 0.298 & \secondres{0.208} & \secondres{0.257} \\
336 & \boldres{0.253} & \boldres{0.291} & 0.325 & 0.354 & 0.255 & 0.302 & \secondres{0.258} & \secondres{0.295} \\
720 & \boldres{0.317} & \boldres{0.340} & 0.369 & 0.383 & 0.309 & 0.357 & \secondres{0.322} & \secondres{0.345} \\
\midrule
Avg & \boldres{0.233} & \boldres{0.274} & 0.316 & 0.350 & 0.254 & 0.306 & \secondres{0.238} & \secondres{0.279} \\
\%Deg & -- & -- & $35.6\%\uparrow$ & $27.7\%\uparrow$ & $9.0\%\uparrow$ & $11.7\%\uparrow$ & $2.1\%\uparrow$ & $1.8\%\uparrow$ \\
\bottomrule
\end{tabular}
}
\end{small}
\end{center}
\vspace{-1em}
\end{table}

The study highlights the critical role of each component. Removing the RAL causes the largest performance drop (\(35.6\%\) in MSE and \(27.7\%\) in MAE), emphasizing its importance in capturing temporal dependencies through memory bank interactions. The VAL, which transforms time series into visual representations, is essential, with its exclusion leading to significant degradation (\(9.0\%\) in MSE and \(11.7\%\) in MAE). This underscores its ability to preserve fine-grained temporal patterns using VLM vision encoder. In contrast, removing the TAL results in minor degradation (\(2.1\%\) in MSE and \(1.8\%\) in MAE), likely due to sparse textual tokens in the VLM output (e.g., 11 out of 156 in ViLT). While the TAL provides valuable semantic context, its impact is limited by the VLM's temporal understanding. Future work could explore larger VLMs with extended textual inputs to enhance temporal-semantic alignment.

\noindent\textbf{Multimodal and Few/Zero-shot Analysis:} \method's few-shot and zero-shot capabilities arise from its integration of temporal, visual, and textual modalities. The RAL models temporal dependencies through memory bank interactions, ensuring robust feature extraction with limited data. The VAL captures visually interpretable features (e.g., trend, seasonality, periodicity) in domain-agnostic visual representations, while the TAL generates contextual descriptions, providing semantic insights for better generalization. Together, these components enable \method to leverage pre-trained multimodal knowledge, making it highly adaptable to new tasks and domains with minimal training data.

\vspace{-0.5em}
\begin{figure}[h!]
    \centering
    \includegraphics[width=1\linewidth]{figures/multimodal_effectiveness.pdf}
    \caption{2D UMAP visualization (Left) and Gate weight distributions (Right) of multimodal and temporal memory embeddings, highlighting their complementary behavior.}
    \label{fig:fusion_analysis}
\end{figure}
\vspace{-0.5em}

To validate the adaptation of VLM capabilities to time series, we analyze the similarity between RAL (temporal) and TAL/VAL (multimodal) embeddings. Figure~\ref{fig:fusion_analysis} visualizes their complementary behavior. The left panel shows balanced gate weight distributions, indicating effective fusion of multimodal and temporal representations. The right panel's UMAP visualization reveals distinct yet overlapping clusters, confirming successful integration of multimodal information while preserving unique characteristics. This demonstrates \method's ability to adapt VLM-derived embeddings for robust  time series analysis.

\textbf{Computation Studies:} \method demonstrates strong computational efficiency, as shown in \autoref{tab:computational-efficiency}. With only 143.6M parameters (1/20 of Time-LLM's 3404.6M), memory usage scales from 1968 MiB (Weather) to 24916 MiB (Traffic), adapting to dataset complexity. Inference speed ranges from 0.2057s/iter (ECL) to 0.4809s/iter (ETTh1), efficiently handling varying loads. In contrast, Time-LLM requires over 37GB of memory even for smaller datasets like ETTh1 and ETTh2, making it infeasible for larger datasets such as Weather, ECL, and Traffic. This highlights \method's lightweight design and practical scalability.

\begin{table}[h!]
\captionsetup{font=small} 
  \caption{Computational efficiency comparison between \method and Time-LLM across datasets. ``-'' denotes memory exceeds 49GB, infeasible on a single GPU. Results are averaged over multiple prediction steps under consistent conditions.}
  \vspace{-0.5em}
  \centering
  \label{tab:computational-efficiency}
  \begin{threeparttable}
  \begin{small}
  \scalebox{0.63}{
  \renewcommand{\multirowsetup}{\centering}
  \setlength{\tabcolsep}{5pt}
  \begin{tabular}{l|l|ccccccc}
    \toprule
    Method & Metric & ETTh1 & ETTh2 & ETTm1 & ETTm2 & Weather & ECL & Traffic \\
    \midrule
    \multirow{3}{*}{\method} 
    & Param. (M) & \boldres{143.6} & \boldres{143.6} & \boldres{143.6} & \boldres{143.6} & \boldres{143.6} & \boldres{143.6} & \boldres{143.6} \\
    & Mem. (MiB) & \boldres{2630} & \boldres{2630} & \boldres{2640} & \boldres{2640} & \boldres{1968} & \boldres{10818} & \boldres{24916} \\
    & Speed (s/iter) & \boldres{0.481} & \boldres{0.438} & \boldres{0.277} & \boldres{0.210} & \boldres{0.296} & \boldres{0.206} & \boldres{0.323} \\
    \midrule
    \multirow{3}{*}{Time-LLM} 
    & Param. (M) & \secondres{3404.6} & \secondres{3404.6} & \secondres{3404.6} & \secondres{3404.6} & \secondres{-} & \secondres{-} & \secondres{-} \\
    & Mem. (MiB) & \secondres{37723} & \secondres{37723} & \secondres{37849} & \secondres{37849} & \secondres{-} & \secondres{-} & \secondres{-} \\
    & Speed (s/iter) & \secondres{0.607} & \secondres{0.553} & \secondres{0.349} & \secondres{0.265} & \secondres{-} & \secondres{-} & \secondres{-} \\
    \bottomrule
  \end{tabular}
  }
  \end{small}
  \end{threeparttable}
\end{table}


\noindent\textbf{Hyperparameter Studies:} We analyze key hyperparameters' impact on performance, as shown in Figure~\ref{fig:hyperparameters}. The sequence length performs best between 96 and 1024 timesteps, with 512 being optimal for most datasets. Longer sequences introduce noise without significant gains. The normalization constant peaks at 0.4, while the model dimension performs best at 128 for simpler datasets (e.g., ETTh1, ETTh2) and larger values for complex ones (e.g., Traffic, Weather). The gate network dimension, controlling multimodal fusion, achieves optimal results at 256 across most datasets.

\vspace{-0.5em}
\begin{figure}[h!]
    \centering
    \includegraphics[width=0.48\textwidth]{figures/hyperparameters.pdf}
    \caption{Hyperparameters sensitivity analysis on input length, normalization constant, dimension of model and dimension of gate network, reflected by MAE.}
    \label{fig:hyperparameters}
\end{figure}
\vspace{-0.5em}

\section{Conclusion}

We presented \method, a novel framework leveraging pretrained VLMs to unify temporal, visual, and textual modalities for time series forecasting. By integrating the RAL, VAL, and TAL, \method bridges modality gaps, enabling rich cross-modal interactions. Extensive experiments demonstrate state-of-the-art performance across various datasets, especially in few-shot and zero-shot scenarios, outperforming existing methods while maintaining efficiency. Our work establishes a new direction for multimodal time series forecasting, highlighting the potential of VLMs in capturing temporal dynamics and semantic context.

Notably, \method operates can solely on original time series data without external information, ensuring fair comparisons and showcasing its ability to generate textual and visual representations directly from the data for self-augmentation. This design not only enhances accuracy but also emphasizing the framework's robustness, particularly in domains where external data is scarce or unavailable.

Future work may explore adaptive visual transformations for complex patterns, enhancing text utilization, extending to multi-task, and developing more efficient multimodal time series foundation models. For details, see \shortautoref{appx:future_work}.


\section*{Impact Statement}

This paper presents work whose goal is to advance the field of Machine Learning by integrating temporal, visual, and textual modalities for time series forecasting. While our approach improves accuracy and cross-domain generalization, we acknowledge potential risks such as data privacy concerns, algorithmic bias, and increased computational costs. We encourage further research into mitigating these risks to ensure responsible deployment in high-stakes applications.


% In the unusual situation where you want a paper to appear in the
% references without citing it in the main text, use \nocite
% \nocite{langley00}

\bibliography{example_paper}
\bibliographystyle{icml2025}


%%%%%%%%%%%%%%%%%%%%%%%%%%%%%%%%%%%%%%%%%%%%%%%%%%%%%%%%%%%%%%%%%%%%%%%%%%%%%%%
%%%%%%%%%%%%%%%%%%%%%%%%%%%%%%%%%%%%%%%%%%%%%%%%%%%%%%%%%%%%%%%%%%%%%%%%%%%%%%%
% APPENDIX
%%%%%%%%%%%%%%%%%%%%%%%%%%%%%%%%%%%%%%%%%%%%%%%%%%%%%%%%%%%%%%%%%%%%%%%%%%%%%%%
%%%%%%%%%%%%%%%%%%%%%%%%%%%%%%%%%%%%%%%%%%%%%%%%%%%%%%%%%%%%%%%%%%%%%%%%%%%%%%%
\newpage
\onecolumn
\newpage
\appendix

\section{Experimental Details}
\subsection{Dataset Details}
\label{appx:dataset_details} 
\begin{table}[htbp]
  \caption{Summary of the benchmark datasets. Each dataset contains multiple time series (Dim.) with different sequence lengths, and is split into training, validation and testing sets. The data are collected at different frequencies across various domains.}
  \label{tab:dataset}
  \centering
  \scalebox{0.99}{
  \begin{tabular}{l|c|c|c|c|c}
    \toprule
Dataset & Dim. & Series Length & Dataset Size & Frequency & Domain \\
\toprule
ETTm1 & 7 & \{96, 192, 336, 720\} & (34465, 11521, 11521)  & 15 min & Temperature \\
ETTm2 & 7 & \{96, 192, 336, 720\} & (34465, 11521, 11521)  & 15 min & Temperature \\
ETTh1 & 7 & \{96, 192, 336, 720\} & (8545, 2881, 2881) & 1 hour & Temperature \\
ETTh2 & 7 & \{96, 192, 336, 720\} & (8545, 2881, 2881) & 1 hour & Temperature \\ 
Electricity & 321 & \{96, 192, 336, 720\} & (18317, 2633, 5261) & 1 hour & Electricity \\ 
Traffic & 862 & \{96, 192, 336, 720\} & (12185, 1757, 3509) & 1 hour & Transportation \\ 
Weather & 21 & \{96, 192, 336, 720\} & (36792, 5271, 10540) & 10 min 
& Weather \\

\bottomrule
\end{tabular}
}
\end{table}


% \begin{table}[htbp]
%   \caption{The dimension indicates the number of time series (i.e., channels)}
%   \vspace{1em}
%   \label{tab:dataset}
%   \centering
%   \begin{small}
%   \scalebox{0.9}{
%   \begin{tabular}{c|l|c|c|c|c|c}
%     \toprule
%     Tasks & Dataset & Dim. & Series Length & Dataset Size & Frequency & Domain \\
%     \toprule
%      & ETTm1 & 7 & \{96, 192, 336, 720\} & (34465, 11521, 11521)  & 15 min & Temperature \\
%     \cmidrule{2-7}
%     Long-term & ETTm2 & 7 & \{96, 192, 336, 720\} & (34465, 11521, 11521)  & 15 min & Temperature \\
%      \cmidrule{2-7}
%      Forecasting & ETTh1 & 7 & \{96, 192, 336, 720\} & (8545, 2881, 2881) & 1 hour & Temperature \\
%      \cmidrule{2-7}
%      & ETTh2 & 7 & \{96, 192, 336, 720\} & (8545, 2881, 2881) & 1 hour & Temperature \\ 
%      \cmidrule{2-7}
%      & Electricity & 321 & \{96, 192, 336, 720\} & (18317, 2633, 5261) & 1 hour & Electricity \\ 
%      \cmidrule{2-7}
%      & Traffic & 862 & \{96, 192, 336, 720\} & (12185, 1757, 3509) & 1 hour & Transportation \\ 
%      \cmidrule{2-7}
%      & Weather & 21 & \{96, 192, 336, 720\} & (36792, 5271, 10540) & 10 min & Weather \\
%      \cmidrule{2-7}
%      & Illness & 7 & \{24, 36, 48, 60\} & (617, 74, 170) & 1 week & Illness \\
%     \midrule
%     & M3-Quarterly & 1 & 8 & (756, 0, 756) & Quarterly & Multiple \\
%     \cmidrule{2-7}
%     & M4-Yearly & 1 & 6 & (23000, 0, 23000) & Yearly & Demographic \\
%     \cmidrule{2-7}
%     & M4-Quarterly & 1 & 8 & (24000, 0, 24000) & Quarterly & Finance \\
%     \cmidrule{2-7}
%     Short-term & M4-Monthly & 1 & 18 & (48000, 0, 48000) & Monthly & Industry \\
%      \cmidrule{2-7}
%     Forecasting & M4-Weakly & 1 & 13 & (359, 0, 359) & Weekly & Macro \\
%      \cmidrule{2-7}
%      & M4-Daily & 1 & 14 & (4227, 0, 4227) & Daily & Micro \\
%      \cmidrule{2-7}
%      & M4-Hourly & 1 & 48 & (414, 0, 414) & Hourly & Other \\
%     \bottomrule
%     \end{tabular}
%     }
%     \end{small}
% \end{table}

We conduct experiments on the above real-world datasets to evaluate the performance of our proposed model and follow the same data processing and train-validation-test set split protocol used in TimesNet benchmark~\cite{wu2022timesnet}, ensuring a strict chronological order to prevent data leakage. Different datasets require specific adjustments to accommodate their unique characteristics:
\label{appx:dataset_configurations}

\paragraph{ETT Dataset~\cite{kim2021reversible}} The Electricity Transformer Temperature (ETT) dataset consists of both hourly (ETTh) and 15-minute (ETTm) frequency data, with 7 variables ($enc\_{in}$ = $dec\_{in}$ = $c\_{out}$ = 7) measuring transformer temperatures and related factors. For ETTh data, we set periodicity to 24 with hourly frequency, while ETTm data uses a periodicity of 96 with 15-minute intervals. Standard normalization is applied to each feature independently, and the model maintains the same architectural configuration across both temporal resolutions.

\paragraph{Traffic Dataset~\cite{wu2022timesnet}} The traffic flow dataset represents a high-dimensional scenario with 862 variables capturing traffic movements across different locations. To handle this scale, we implement gradient checkpointing and efficient attention mechanisms, complemented by progressive feature loading. The batch size is dynamically adjusted based on available GPU memory, and we maintain a periodicity of 24 to capture daily patterns. Our model employs specialized memory optimization techniques to process this large feature space efficiently.

\paragraph{ECL Dataset~\cite{wu2021autoformer}} The electricity consumption dataset contains 321 variables monitoring power usage patterns. We employ robust scaling techniques to handle outliers and implement sophisticated missing value imputation strategies. The model incorporates adaptive normalization layers to address the varying scales of electricity consumption across different regions and time periods. The daily periodicity is preserved through careful temporal encoding, while the high feature dimensionality is managed through efficient attention mechanisms.

\paragraph{Weather Dataset~\cite{wu2021autoformer}} This multivariate dataset encompasses 21 weather-related variables, each with distinct physical meanings and scale properties. Our approach implements feature-specific normalization to handle the diverse variable ranges while maintaining their physical relationships. The model captures both daily and seasonal patterns through enhanced temporal encoding, with special attention mechanisms designed to model the complex interactions between different weather variables. We maintain consistent prediction quality across all variables through carefully calibrated cross-attention mechanisms.


\subsection{Optimization Settings}
\label{appx:optimization_settings}

\subsubsection{Model Architecture Parameters}
\label{appx:model_parameters}
\begin{table}[htbp]
\centering
\caption{Default Model Architecture Parameters}
\begin{tabular}{|l|l|p{5.5cm}|}
\hline
\textbf{Parameter} & \textbf{Default Value} & \textbf{Description} \\
\hline
\multicolumn{3}{|l|}{\textit{Visual Representation Parameters}} \\
\hline
image\_size & 64 & Size of generated image representation \\
patch\_size & 16 & Size of patches for input processing \\
grayscale & True & Whether to use grayscale images \\
\hline
\multicolumn{3}{|l|}{\textit{Diffusion Process Parameters}} \\
\hline
num\_timesteps & 300 & Number of diffusion training steps \\
inference\_steps & 50 & Number of inference steps \\
beta\_start & 0.00085 & Initial value of noise schedule \\
beta\_end & 0.012 & Final value of noise schedule \\
use\_ddim & True & Whether to use DDIM sampler \\
unet\_layers & 1 & Number of layers in UNet \\
\hline
\multicolumn{3}{|l|}{\textit{Model Architecture Parameters}} \\
\hline
d\_model & 256 & Dimension of model hidden states \\
d\_ldm & 256 & Hidden dimension of LDM \\
d\_fusion & 256 & Dimension of gated fusion module \\
e\_layers & 2 & Number of encoder layers \\
d\_layers & 1 & Number of decoder layers \\
\hline
\multicolumn{3}{|l|}{\textit{Training Configuration}} \\
\hline
freeze\_ldm & True & Whether to freeze LDM parameters \\
save\_images & False & Whether to save generated images \\
output\_type & full & Type of output for ablation study \\
\hline
\end{tabular}
\end{table}
The core architecture of our diffusion-based model consists of several key components, each with specific parameter settings. The autoencoder pathway is configured with an image size of $64\times64$ and a patch size of $16$, providing an efficient latent representation while maintaining temporal information. The diffusion process uses $1000$ timesteps with carefully tuned noise scheduling ($\beta_{start} = 0.00085, \beta_{end} = 0.012$) to ensure stable training.

For the transformer backbone, we employ a configuration with $d_model = 256$ and $8$ attention heads, which empirically shows strong performance across different datasets. The encoder-decoder structure uses $2$ encoder layers and $1$ decoder layer, with a feed-forward dimension of $768$, striking a balance between model capacity and computational efficiency.

\subsubsection{Training Parameters}
\label{appx:training_settings}
\begin{table}[htbp]
\centering
\caption{Default Training Parameters}
\begin{tabular}{|l|l|p{5.5cm}|}
\hline
\textbf{Parameter} & \textbf{Default Value} & \textbf{Description} \\
\hline
\multicolumn{3}{|l|}{\textit{Basic Training Parameters}} \\
\hline
batch\_size & 32 & Number of samples per training batch \\
learning\_rate & 0.001 & Initial learning rate for optimization \\
train\_epochs & 10 & Total number of training epochs \\
patience & 3 & Epochs before early stopping \\
loss & MSE & Type of loss function \\
label\_len & 48 & Length of start token sequence \\
seq\_len & 96 & Length of input sequence \\
norm\_const & 0.4 & Coefficient for normalization \\
padding & 8 & Size of sequence padding \\
stride & 8 & Step size for sliding window \\
pred\_len & \makecell[l]{96/192/336/720} & Available prediction horizons \\
\hline
\multicolumn{3}{|l|}{\textit{Dataset-specific Parameters}} \\
\hline
c\_out & \makecell[l]{7 (ETTh1/h2/m1/m2) \\ 21 (Weather) \\ 321 (Electricity) \\ 862 (Traffic)} & Dataset-specific output dimensions \\
\hline
periodicity & \makecell[l]{24 (ETTh1/h2/Electricity/Traffic) \\ 96 (ETTm1/m2) \\ 144 (Weather)} & Natural cycle length per dataset \\
\hline
\end{tabular}
\end{table}
We adopt a comprehensive training strategy with both general and task-specific parameters. The model is trained with a batch size of $32$ and an initial learning rate of $0.001$, using the \textit{AdamW} optimizer. Early stopping with a patience of $3$ epochs is implemented to prevent over-fitting. For time series processing, we use a sequence length of $96$ and a prediction length of $96$, with a label length of 48 for teacher forcing during training.

The training process employs automatic mixed precision (AMP) when available to accelerate training while maintaining numerical stability. We use MSE as the primary loss function, supplemented by additional regularization terms for specific tasks.

\subsection{Evaluation Metrics}
\label{appx:metric}
For evaluation metrics, we utilize the mean square error (MSE) and mean absolute error (MAE) for long-term forecasting. 
% In terms of the short-term forecasting on M4 benchmark, we adopt the symmetric mean absolute percentage error (SMAPE), mean absolute scaled error (MASE), and overall weighted average (OWA) as in N-BEATS \citep{oreshkin2019n}. Note that OWA is a specific metric utilized in the M4 competition. 
The calculations of these metrics are as follows:
\begin{align*} \label{equ:metrics}
    \text{MSE} &= \frac{1}{H}\sum_{h=1}^T (\mathbf{Y}_{h} - \Hat{\mathbf{Y}}_{h})^2,
    &
    \text{MAE} &= \frac{1}{H}\sum_{h=1}^H|\mathbf{Y}_{h} - \Hat{\mathbf{Y}}_{h}|,\\
    % \text{SMAPE} &= \frac{200}{H} \sum_{h=1}^H \frac{|\mathbf{Y}_{h} - \Hat{\mathbf{Y}}_{h}|}{|\mathbf{Y}_{h}| + |\Hat{\mathbf{Y}}_{h}|},
    % &
    % \text{MAPE} &= \frac{100}{H} \sum_{h=1}^H \frac{|\mathbf{Y}_{h} - \Hat{\mathbf{Y}}_{h}|}{|\mathbf{Y}_{h}|}, \\
    % \text{MASE} &= \frac{1}{H} \sum_{h=1}^H \frac{|\mathbf{Y}_{h} - \Hat{\mathbf{Y}}_{h}|}{\frac{1}{H-s}\sum_{j=s+1}^{H}|\mathbf{Y}_j - \mathbf{Y}_{j-s}|},
    % &
    % \text{OWA} &= \frac{1}{2} \left[ \frac{\text{SMAPE}}{\text{SMAPE}_{\textrm{Naïve2}}}  + \frac{\text{MASE}}{\text{MASE}_{\textrm{Naïve2}}}  \right],
\end{align*}
where $s$ is the periodicity of the time series data. $H$ denotes the number of data points (i.e., prediction horizon in our cases). $\mathbf{Y}_{h}$ and $\Hat{\mathbf{Y}}_{h}$ are the $h$-th ground truth and prediction where $h \in \{1, \cdots, H\}$.

\section{Details of Baseline Methods}
\label{appx:baselines}
We compare our approach with three categories of baseline methods used for comparative evaluation: transformer-based architectures, diffusion-based models, and other competitive approaches for time series forecasting.
\paragraph{Transformer-based Models:}
\textbf{FEDformer~\cite{zhou2022fedformer}} integrates wavelet decomposition with a Transformer architecture to efficiently capture multi-scale temporal dependencies by processing both time and frequency domains. 
\textbf{Autoformer~\cite{wu2021autoformer}} introduces a decomposing framework that separates the time series into trend and seasonal components, employing an autocorrelation mechanism for periodic pattern extraction.
\textbf{ETSformer~\cite{woo2022etsformer}} extends the classical exponential smoothing method with a Transformer architecture, decomposing time series into level, trend, and seasonal components while learning their interactions through attention mechanisms.
\textbf{Informer~\cite{zhou2021informer}} addresses the quadratic complexity issue of standard attention mechanisms through ProbSparse self-attention, which enables efficient handling of long input sequences.
\textbf{Reformer~\cite{kitaev2020reformer}} optimizes attention computation via Locality-Sensitive Hashing (LSH) and reversible residual networks, significantly reducing memory and computational costs.
\paragraph{Diffusion-based Models:}
\textbf{CSDI~\cite{tashiro2021csdi}} is tailored for irregularly-spaced time series, learning a score function of noise distribution under given conditions to generate samples for forecasting.
\textbf{ScoreGrad~\cite{song2020score}} utilizes a continuous-time framework for progressive denoising from Gaussian noise to reconstruct the original signal, allowing for adjustable step sizes during the denoising process.
\paragraph{Other Competitive Models:}
\textbf{DLinear~\cite{zeng2023transformers}} proposes a linear transformation approach directly on time series data, simplifying the prediction process under the assumption of linear changes over time.
\textbf{TimesNet~\cite{wu2022timesnet}} focuses on multi-scale feature extraction using various convolution kernels to capture temporal dependencies of different lengths, automatically selecting the most suitable feature scales.
\textbf{LightTS~\cite{campos2023lightts}} aims to build lightweight time series forecasting models, streamlining structures and parameters to reduce computational resource requirements while maintaining high predictive performance.

Each baseline method represents distinct paradigms within probabilistic generative modeling, attention-based architectures, and linear models, providing a comprehensive benchmark against which to evaluate LDM4TS.
\newpage
\subsection{Long-term Forecasting}
\label{appx:long_term_details}
\begin{table}[!ht]
\centering
\caption{Details of long-term forecasting results.}
\vspace{-2mm}
\scalebox{0.78}{
\begin{tabular}{c|c|cc|cc|cc|cc|cc|cc|cc|cc|cc|cc}
\toprule
\multicolumn{2}{c|}{Methods} & \multicolumn{2}{c|}{LDM4TS} & \multicolumn{2}{c|}{CSDI} & \multicolumn{2}{c|}{ScoreGrad} & \multicolumn{2}{c|}{Autoformer} & \multicolumn{2}{c|}{FEDformer} & \multicolumn{2}{c|}{DLinear} & \multicolumn{2}{c|}{Informer} & \multicolumn{2}{c|}{TimesNet} & \multicolumn{2}{c|}{LightTS} & \multicolumn{2}{c}{Reformer} \\
\multicolumn{2}{c|}{Metric} & MSE & MAE & MSE & MAE & MSE & MAE & MSE & MAE & MSE & MAE & MSE & MAE & MSE & MAE & MSE & MAE & MSE & MAE & MSE & MAE \\ \midrule
\multirow{5}{*}{ETTh1} & 96 & 0.388 & 0.411 & 0.910 & 0.708 & 1.030 & 0.702 & 0.505 & 0.482 & 0.377 & 0.418 & 0.396 & 0.411 & 0.953 & 0.773 & 0.389 & 0.412 & 0.412 & 0.422 & 0.852 & 0.680 \\
& 192 & 0.412 & 0.430 & 0.975 & 0.738 & 1.033 & 0.705 & 0.475 & 0.470 & 0.420 & 0.444 & 0.445 & 0.440 & 1.016 & 0.790 & 0.439 & 0.442 & 0.581 & 0.545 & 1.008 & 0.742 \\
& 336 & 0.471 & 0.473 & 0.977 & 0.739 & 1.025 & 0.707 & 0.524 & 0.508 & 0.459 & 0.467 & 0.487 & 0.465 & 1.030 & 0.782 & 0.494 & 0.471 & 0.740 & 0.637 & 0.934 & 0.726 \\
& 720 & 0.501 & 0.502 & 1.086 & 0.783 & 1.036 & 0.721 & 0.512 & 0.509 & 0.502 & 0.503 & 0.513 & 0.510 & 1.161 & 0.853 & 0.517 & 0.494 & 0.625 & 0.574 & 1.229 & 0.831 \\
& \multicolumn{1}{c|}{Avg} & 0.443 & 0.454 & 0.987 & 0.742 & 1.031 & 0.709 & 0.504 & 0.492 & 0.439 & 0.458 & 0.460 & 0.457 & 1.040 & 0.799 & 0.460 & 0.455 & 0.590 & 0.544 & 1.006 & 0.745 \\ \midrule
\multirow{5}{*}{ETTh2} & 96 & 0.316 & 0.378 & 0.455 & 0.455 & 0.502 & 0.497 & 0.375 & 0.409 & 0.347 & 0.386 & 0.341 & 0.395 & 3.314 & 1.456 & 0.337 & 0.371 & 0.394 & 0.431 & 1.732 & 1.064 \\
& 192 & 0.356 & 0.404 & 0.493 & 0.506 & 0.509 & 0.501 & 0.463 & 0.461 & 0.422 & 0.437 & 0.482 & 0.479 & 6.750 & 2.137 & 0.405 & 0.415 & 0.635 & 0.550 & 2.662 & 1.305 \\
& 336 & 0.438 & 0.461 & 0.540 & 0.532 & 0.506 & 0.503 & 0.538 & 0.506 & 0.495 & 0.485 & 0.593 & 0.542 & 4.594 & 1.784 & 0.451 & 0.449 & 0.454 & 0.463 & 2.593 & 1.267 \\
& 720 & 0.436 & 0.465 & 0.668 & 0.588 & 0.531 & 0.520 & 0.491 & 0.498 & 0.493 & 0.494 & 0.840 & 0.661 & 3.545 & 1.592 & 0.435 & 0.449 & 3.556 & 1.267 & 3.135 & 1.339 \\
& \multicolumn{1}{c|}{Avg} & 0.387 & 0.427 & 0.539 & 0.520 & 0.512 & 0.505 & 0.467 & 0.468 & 0.439 & 0.451 & 0.564 & 0.519 & 4.551 & 1.742 & 0.407 & 0.421 & 1.260 & 0.678 & 2.531 & 1.244 \\ \midrule
\multirow{5}{*}{ETTm1} & 96 & 0.331 & 0.373 & 0.771 & 0.628 & 1.001 & 0.668 & 0.489 & 0.476 & 0.401 & 0.428 & 0.346 & 0.374 & 0.634 & 0.563 & 0.413 & 0.405 & 0.344 & 0.382 & 0.879 & 0.662 \\
& 192 & 0.346 & 0.382 & 0.789 & 0.640 & 1.008 & 0.673 & 0.599 & 0.532 & 0.465 & 0.462 & 0.382 & 0.391 & 0.729 & 0.617 & 0.481 & 0.436 & 0.348 & 0.381 & 0.963 & 0.736 \\
& 336 & 0.371 & 0.394 & 0.812 & 0.652 & 1.020 & 0.680 & 0.609 & 0.545 & 0.496 & 0.485 & 0.415 & 0.415 & 1.176 & 0.862 & 0.496 & 0.456 & 0.438 & 0.446 & 1.064 & 0.771 \\
& 720 & 0.362 & 0.397 & 0.852 & 0.675 & 1.033 & 0.691 & 0.608 & 0.551 & 0.521 & 0.504 & 0.473 & 0.451 & 0.929 & 0.717 & 0.518 & 0.475 & 0.576 & 0.539 & 1.147 & 0.777 \\
& \multicolumn{1}{c|}{Avg} & 0.352 & 0.387 & 0.806 & 0.649 & 1.016 & 0.678 & 0.576 & 0.526 & 0.471 & 0.470 & 0.404 & 0.408 & 0.867 & 0.690 & 0.477 & 0.443 & 0.427 & 0.437 & 1.013 & 0.737 \\ \midrule
\multirow{5}{*}{ETTm2} & 96 & 0.184 & 0.274 & 0.316 & 0.392 & 0.390 & 0.418 & 0.212 & 0.295 & 0.214 & 0.300 & 0.210 & 0.293 & 0.525 & 0.568 & 0.192 & 0.267 & 0.210 & 0.304 & 0.822 & 0.682 \\
& 192 & 0.334 & 0.382 & 0.349 & 0.408 & 0.418 & 0.433 & 0.279 & 0.338 & 0.286 & 0.355 & 0.269 & 0.328 & 1.051 & 0.761 & 0.253 & 0.304 & 0.419 & 0.459 & 1.465 & 0.916 \\
& 336 & 0.376 & 0.398 & 0.388 & 0.426 & 0.454 & 0.451 & 0.322 & 0.360 & 0.331 & 0.375 & 0.321 & 0.361 & 1.550 & 0.945 & 0.327 & 0.355 & 0.612 & 0.589 & 2.092 & 1.101 \\
& 720 & 0.436 & 0.465 & 0.473 & 0.476 & 0.521 & 0.486 & 0.414 & 0.411 & 0.439 & 0.435 & 0.415 & 0.413 & 3.247 & 1.359 & 0.422 & 0.405 & 2.077 & 1.102 & 3.116 & 1.335 \\
& \multicolumn{1}{c|}{Avg} & 0.333 & 0.380 & 0.381 & 0.425 & 0.446 & 0.447 & 0.307 & 0.351 & 0.318 & 0.366 & 0.304 & 0.349 & 1.593 & 0.908 & 0.299 & 0.333 & 0.830 & 0.614 & 1.874 & 1.009 \\ \midrule
\multirow{5}{*}{Weather} & 96 & 0.166 & 0.220 & 0.222 & 0.288 & 0.268 & 0.300 & 0.241 & 0.320 & 0.258 & 0.334 & 0.150 & 0.220 & 0.150 & 0.220 & 0.169 & 0.218 & 0.169 & 0.241 & 0.660 & 0.617 \\
& 192 & 0.214 & 0.262 & 0.267 & 0.319 & 0.317 & 0.334 & 0.305 & 0.363 & 0.303 & 0.356 & 0.203 & 0.270 & 0.203 & 0.270 & 0.227 & 0.265 & 0.243 & 0.304 & 0.951 & 0.768 \\
& 336 & 0.262 & 0.298 & 0.314 & 0.347 & 0.366 & 0.364 & 0.346 & 0.384 & 0.352 & 0.385 & 0.253 & 0.318 & 0.253 & 0.318 & 0.280 & 0.300 & 0.271 & 0.328 & 1.474 & 0.966 \\
& 720 & 0.341 & 0.352 & 0.381 & 0.387 & 0.434 & 0.405 & 0.423 & 0.434 & 0.419 & 0.427 & 0.374 & 0.415 & 0.374 & 0.415 & 0.386 & 0.371 & 0.355 & 0.386 & 1.832 & 1.082 \\
& \multicolumn{1}{c|}{Avg} & 0.245 & 0.283 & 0.296 & 0.335 & 0.347 & 0.351 & 0.329 & 0.375 & 0.333 & 0.375 & 0.245 & 0.306 & 0.245 & 0.306 & 0.265 & 0.288 & 0.259 & 0.315 & 1.229 & 0.858 \\ \midrule
\multirow{5}{*}{ECL} & 96 & 0.173 & 0.272 & 0.983 & 0.802 & 1.240 & 0.877 & 0.199 & 0.315 & 0.579 & 0.359 & 0.210 & 0.302 & 0.380 & 0.436 & 0.170 & 0.271 & 0.216 & 0.319 & 0.295 & 0.382 \\
& 192 & 0.182 & 0.283 & 0.997 & 0.805 & 1.248 & 0.880 & 0.276 & 0.358 & 0.619 & 0.381 & 0.210 & 0.305 & 0.352 & 0.426 & 0.193 & 0.288 & 0.227 & 0.329 & 0.334 & 0.412 \\
& 336 & 0.203 & 0.306 & 0.976 & 0.800 & 1.258 & 0.885 & 0.243 & 0.353 & 0.601 & 0.370 & 0.223 & 0.319 & 0.405 & 0.454 & 0.201 & 0.297 & 0.247 & 0.349 & 0.355 & 0.425 \\
& 720 & 0.236 & 0.334 & 1.015 & 0.810 & 1.286 & 0.895 & 0.293 & 0.380 & 0.651 & 0.399 & 0.258 & 0.350 & 0.378 & 0.436 & 0.266 & 0.355 & 0.283 & 0.376 & 0.319 & 0.396 \\
& \multicolumn{1}{c|}{Avg} & 0.199 & 0.299 & 0.993 & 0.804 & 1.258 & 0.884 & 0.253 & 0.352 & 0.612 & 0.377 & 0.225 & 0.319 & 0.378 & 0.438 & 0.208 & 0.303 & 0.243 & 0.343 & 0.326 & 0.404 \\ \midrule
\multirow{5}{*}{Traffic} & 96 & 0.529 & 0.315 & 1.697 & 0.859 & 2.077 & 1.016 & 0.334 & 0.528 & 0.579 & 0.359 & 0.649 & 0.396 & 0.774 & 0.432 & 0.600 & 0.316 & 0.619 & 0.401 & 0.736 & 0.418 \\
& 192 & 0.534 & 0.313 & 1.732 & 0.869 & 2.089 & 1.019 & 0.697 & 0.452 & 0.619 & 0.381 & 0.598 & 0.370 & 0.746 & 0.421 & 0.626 & 0.332 & 0.635 & 0.550 & 0.712 & 0.396 \\
& 336 & 0.541 & 0.317 & 1.753 & 0.873 & 2.108 & 1.023 & 0.754 & 0.492 & 0.601 & 0.370 & 0.605 & 0.373 & 0.826 & 0.463 & 0.638 & 0.340 & 0.454 & 0.463 & 0.712 & 0.393 \\
& 720 & 0.594 & 0.339 & 1.765 & 0.874 & 2.124 & 1.024 & 0.777 & 0.497 & 0.651 & 0.399 & 0.646 & 0.395 & 0.980 & 0.544 & 0.678 & 0.354 & 3.556 & 1.267 & 0.709 & 0.389 \\
& \multicolumn{1}{c|}{Avg} & 0.550 & 0.321 & 1.736 & 0.869 & 2.099 & 1.020 & 0.641 & 0.492 & 0.612 & 0.377 & 0.624 & 0.384 & 0.832 & 0.465 & 0.636 & 0.336 & 1.260 & 0.678 & 0.717 & 0.399 \\
\bottomrule
\end{tabular}
}
\end{table}

\newpage
\subsection{Few-shot Forecasting}
\label{appx:few_shot_details}
\begin{table}[!ht]
\centering
\caption{Details of few-shot forecasting results on 10\% training data.}
\vspace{-2mm}
\scalebox{0.78}{
\begin{tabular}{c|c|cc|cc|cc|cc|cc|cc|cc|cc|cc|cc}
\toprule
\multicolumn{2}{c|}{Methods} & \multicolumn{2}{c|}{LDM4TS} & \multicolumn{2}{c|}{CSDI} & \multicolumn{2}{c|}{ScoreGrad} & \multicolumn{2}{c|}{Autoformer} & \multicolumn{2}{c|}{FEDformer} & \multicolumn{2}{c|}{DLinear} & \multicolumn{2}{c|}{Informer} & \multicolumn{2}{c|}{TimesNet} & \multicolumn{2}{c|}{LightTS} & \multicolumn{2}{c}{Reformer} \\
\multicolumn{2}{c|}{Metric} & MSE & MAE & MSE & MAE & MSE & MAE & MSE & MAE & MSE & MAE & MSE & MAE & MSE & MAE & MSE & MAE & MSE & MAE & MSE & MAE \\ \midrule
\multirow{5}{*}{ETTh1} & 96 & 0.410 & 0.418 & 0.817 & 0.645 & 1.032 & 0.703 & 0.613 & 0.552 & 0.612 & 0.499 & 0.492 & 0.495 & 1.179 & 0.792 & 0.861 & 0.628 & 1.298 & 0.838 & 1.184 & 0.790 \\
& 192 & 0.443 & 0.443 & 0.835 & 0.653 & 1.033 & 0.705 & 0.722 & 0.598 & 0.624 & 0.555 & 0.565 & 0.538 & 1.199 & 0.806 & 0.797 & 0.593 & 1.322 & 0.854 & 1.295 & 0.850 \\
& 336 & 0.481 & 0.479 & 0.849 & 0.666 & 1.024 & 0.707 & 0.750 & 0.619 & 0.691 & 0.574 & 0.721 & 0.622 & 1.202 & 0.811 & 0.941 & 0.648 & 1.347 & 0.870 & 1.294 & 0.854 \\
& 720 & 0.549 & 0.534 & 0.895 & 0.698 & 1.036 & 0.721 & 0.721 & 0.616 & 0.728 & 0.614 & 0.986 & 0.743 & 1.217 & 0.825 & 0.877 & 0.641 & 1.534 & 0.947 & 1.223 & 0.838 \\
& \multicolumn{1}{c|}{Avg} & 0.471 & 0.468 & 0.849 & 0.665 & 1.031 & 0.709 & 0.702 & 0.596 & 0.639 & 0.561 & 0.691 & 0.600 & 1.199 & 0.809 & 0.869 & 0.628 & 1.375 & 0.877 & 1.249 & 0.833 \\ \midrule
\multirow{5}{*}{ETTh2} & 96 & 0.355 & 0.413 & 0.481 & 0.501 & 0.501 & 0.497 & 0.413 & 0.451 & 0.382 & 0.416 & 0.357 & 0.411 & 3.837 & 1.508 & 0.378 & 0.409 & 2.022 & 1.006 & 3.788 & 1.533 \\
& 192 & 0.406 & 0.435 & 0.535 & 0.527 & 0.509 & 0.502 & 0.474 & 0.477 & 0.478 & 0.474 & 0.569 & 0.519 & 3.856 & 1.513 & 0.430 & 0.467 & 2.329 & 1.104 & 3.552 & 1.483 \\
& 336 & 0.463 & 0.486 & 0.533 & 0.529 & 0.506 & 0.503 & 0.547 & 0.543 & 0.504 & 0.501 & 0.671 & 0.572 & 3.952 & 1.526 & 0.537 & 0.494 & 2.453 & 1.122 & 3.395 & 1.526 \\
& 720 & 0.583 & 0.505 & 0.558 & 0.537 & 0.531 & 0.520 & 0.516 & 0.523 & 0.499 & 0.509 & 0.824 & 0.648 & 3.842 & 1.503 & 0.510 & 0.491 & 3.816 & 1.407 & 3.205 & 1.401 \\
& \multicolumn{1}{c|}{Avg} & 0.452 & 0.460 & 0.527 & 0.523 & 0.512 & 0.505 & 0.488 & 0.499 & 0.466 & 0.475 & 0.605 & 0.538 & 3.872 & 1.513 & 0.479 & 0.465 & 2.655 & 1.160 & 3.485 & 1.486 \\ \midrule
\multirow{5}{*}{ETTm1} & 96 & 0.305 & 0.356 & 0.755 & 0.589 & 1.001 & 0.668 & 0.774 & 0.614 & 0.578 & 0.518 & 0.352 & 0.392 & 1.162 & 0.785 & 0.583 & 0.501 & 0.921 & 0.682 & 1.442 & 0.847 \\
& 192 & 0.345 & 0.378 & 0.771 & 0.596 & 1.008 & 0.673 & 0.754 & 0.592 & 0.617 & 0.556 & 0.382 & 0.412 & 1.172 & 0.793 & 0.630 & 0.528 & 0.957 & 0.701 & 1.444 & 0.862 \\
& 336 & 0.383 & 0.399 & 0.786 & 0.609 & 1.020 & 0.680 & 0.869 & 0.677 & 0.998 & 0.775 & 0.419 & 0.434 & 1.227 & 0.908 & 0.725 & 0.568 & 0.998 & 0.716 & 1.450 & 0.866 \\
& 720 & 0.452 & 0.439 & 0.825 & 0.629 & 1.033 & 0.691 & 0.810 & 0.630 & 0.693 & 0.579 & 0.490 & 0.477 & 1.207 & 0.797 & 0.769 & 0.549 & 1.007 & 0.719 & 1.366 & 0.850 \\
& \multicolumn{1}{c|}{Avg} & 0.371 & 0.393 & 0.784 & 0.606 & 1.015 & 0.678 & 0.802 & 0.628 & 0.722 & 0.635 & 0.411 & 0.429 & 1.192 & 0.821 & 0.677 & 0.537 & 0.971 & 0.705 & 1.426 & 0.856 \\ \midrule
\multirow{5}{*}{ETTm2} & 96 & 0.201 & 0.294 & 0.242 & 0.338 & 0.390 & 0.418 & 0.352 & 0.454 & 0.291 & 0.399 & 0.213 & 0.303 & 3.203 & 1.407 & 0.212 & 0.285 & 0.813 & 0.688 & 4.195 & 1.628 \\
& 192 & 0.288 & 0.342 & 0.296 & 0.364 & 0.418 & 0.433 & 0.694 & 0.691 & 0.307 & 0.379 & 0.278 & 0.345 & 3.112 & 1.387 & 0.270 & 0.323 & 1.008 & 0.768 & 4.042 & 1.601 \\
& 336 & 0.411 & 0.419 & 0.350 & 0.395 & 0.454 & 0.451 & 2.408 & 1.407 & 0.543 & 0.559 & 0.338 & 0.385 & 3.255 & 1.421 & 0.323 & 0.353 & 1.031 & 0.775 & 3.963 & 1.585 \\
& 720 & 0.443 & 0.435 & 0.447 & 0.446 & 0.521 & 0.488 & 1.913 & 1.166 & 0.712 & 0.614 & 0.436 & 0.440 & 3.909 & 1.543 & 0.474 & 0.449 & 1.096 & 0.791 & 3.711 & 1.532 \\
& \multicolumn{1}{c|}{Avg} & 0.336 & 0.373 & 0.334 & 0.385 & 0.446 & 0.447 & 1.342 & 0.930 & 0.463 & 0.488 & 0.316 & 0.368 & 3.370 & 1.440 & 0.320 & 0.353 & 0.987 & 0.756 & 3.978 & 1.587 \\ \midrule
\multirow{5}{*}{Weather} & 96 & 0.151 & 0.239 & 0.221 & 0.280 & 0.268 & 0.300 & 0.221 & 0.297 & 0.188 & 0.253 & 0.171 & 0.224 & 0.374 & 0.401 & 0.184 & 0.230 & 0.217 & 0.269 & 0.355 & 0.380 \\
& 192 & 0.187 & 0.247 & 0.265 & 0.317 & 0.317 & 0.334 & 0.270 & 0.322 & 0.250 & 0.304 & 0.215 & 0.263 & 0.552 & 0.478 & 0.245 & 0.283 & 0.259 & 0.304 & 0.522 & 0.462 \\
& 336 & 0.251 & 0.269 & 0.312 & 0.347 & 0.366 & 0.364 & 0.320 & 0.351 & 0.312 & 0.346 & 0.258 & 0.299 & 0.724 & 0.541 & 0.305 & 0.321 & 0.303 & 0.334 & 0.715 & 0.535 \\
& 720 & 0.328 & 0.349 & 0.381 & 0.387 & 0.434 & 0.405 & 0.390 & 0.396 & 0.387 & 0.393 & 0.320 & 0.346 & 0.739 & 0.558 & 0.381 & 0.371 & 0.377 & 0.382 & 0.511 & 0.500 \\
& \multicolumn{1}{c|}{Avg} & 0.229 & 0.276 & 0.295 & 0.333 & 0.347 & 0.351 & 0.300 & 0.342 & 0.284 & 0.324 & 0.241 & 0.283 & 0.597 & 0.495 & 0.279 & 0.301 & 0.289 & 0.322 & 0.526 & 0.469 \\ \midrule
\multirow{5}{*}{ECL} & 96 & 0.141 & 0.236 & 0.889 & 0.779 & 1.240 & 0.877 & 0.261 & 0.348 & 0.231 & 0.323 & 0.150 & 0.253 & 1.259 & 0.919 & 0.299 & 0.373 & 0.350 & 0.425 & 0.993 & 0.764 \\
& 192 & 0.158 & 0.260 & 0.896 & 0.781 & 1.248 & 0.880 & 0.338 & 0.406 & 0.261 & 0.356 & 0.164 & 0.264 & 1.160 & 0.873 & 0.305 & 0.379 & 0.376 & 0.448 & 0.998 & 0.795 \\
& 336 & 0.179 & 0.289 & 0.915 & 0.788 & 1.258 & 0.885 & 0.410 & 0.474 & 0.360 & 0.445 & 0.181 & 0.282 & 1.157 & 0.872 & 0.319 & 0.391 & 0.428 & 0.485 & 0.923 & 0.745 \\
& 720 & 0.209 & 0.314 & 0.936 & 0.792 & 1.286 & 0.895 & 0.715 & 0.685 & 0.530 & 0.585 & 0.223 & 0.321 & 1.203 & 0.898 & 0.369 & 0.426 & 0.611 & 0.597 & 1.004 & 0.790 \\
& \multicolumn{1}{c|}{Avg} & 0.172 & 0.275 & 0.909 & 0.785 & 1.258 & 0.884 & 0.431 & 0.478 & 0.346 & 0.427 & 0.180 & 0.280 & 1.195 & 0.891 & 0.323 & 0.392 & 0.441 & 0.489 & 0.980 & 0.769 \\ \midrule
\multirow{5}{*}{Traffic} & 96 & 0.615 & 0.360 & 1.701 & 0.868 & 2.078 & 1.016 & 0.672 & 0.405 & 0.639 & 0.416 & 0.942 & 0.571 & 1.557 & 0.821 & 0.719 & 0.416 & 1.157 & 0.836 & 1.527 & 0.815 \\
& 192 & 0.595 & 0.340 & 1.740 & 0.870 & 2.089 & 1.019 & 0.727 & 0.424 & 0.637 & 0.416 & 0.924 & 0.564 & 1.454 & 0.765 & 0.748 & 0.428 & 1.207 & 0.661 & 1.550 & 0.817 \\
& 336 & 0.611 & 0.352 & 1.746 & 0.871 & 2.107 & 1.023 & 0.749 & 0.454 & 0.655 & 0.427 & 0.941 & 0.569 & 1.521 & 0.812 & 0.853 & 0.471 & 1.334 & 0.713 & 1.550 & 0.819 \\
& 720 & 0.661 & 0.375 & 1.789 & 0.874 & 2.124 & 1.024 & 0.847 & 0.499 & 0.722 & 0.456 & 0.975 & 0.578 & 1.605 & 0.846 & 1.485 & 0.825 & 1.292 & 0.726 & 1.580 & 0.833 \\
& \multicolumn{1}{c|}{Avg} & 0.621 & 0.357 & 1.744 & 0.871 & 2.100 & 1.020 & 0.749 & 0.446 & 0.663 & 0.425 & 0.945 & 0.570 & 1.534 & 0.811 & 0.951 & 0.535 & 1.248 & 0.684 & 1.551 & 0.821 \\
\bottomrule
\end{tabular}
}
\label{tab:few_shot_detail}
\end{table}
\begin{table}[!ht]
\centering
\caption{Details of few-shot results on 5\% training data. '-' means that 5\% data is not sufficient to constitute a training set.}
\vspace{-2mm}
\scalebox{0.78}{
\begin{tabular}{c|c|cc|cc|cc|cc|cc|cc|cc|cc|cc|cc}
\toprule
\multicolumn{2}{c|}{Methods} & \multicolumn{2}{c|}{LDM4TS} & \multicolumn{2}{c|}{CSDI} & \multicolumn{2}{c|}{ScoreGrad} & \multicolumn{2}{c|}{Autoformer} & \multicolumn{2}{c|}{FEDformer} & \multicolumn{2}{c|}{DLinear} & \multicolumn{2}{c|}{Informer} & \multicolumn{2}{c|}{TimesNet} & \multicolumn{2}{c|}{LightTS} & \multicolumn{2}{c}{Reformer} \\
\multicolumn{2}{c|}{Metric} & MSE & MAE & MSE & MAE & MSE & MAE & MSE & MAE & MSE & MAE & MSE & MAE & MSE & MAE & MSE & MAE & MSE & MAE & MSE & MAE \\ \midrule
\multirow{5}{*}{ETTh1} & 96 & 0.403 & 0.420 & 0.825 & 0.645 & 1.017 & 0.678 & 0.681 & 0.570 & 0.593 & 0.529 & 0.547 & 0.503 & 1.225 & 0.812 & 0.892 & 0.625 & 1.483 & 0.910 & 1.198 & 0.795 \\
& 192 & 0.464 & 0.458 & 0.842 & 0.659 & 1.032 & 0.690 & 0.725 & 0.602 & 0.652 & 0.563 & 0.720 & 0.604 & 1.249 & 0.828 & 0.940 & 0.665 & 1.525 & 0.930 & 1.273 & 0.853 \\
& 336 & 0.509 & 0.489 & 0.883 & 0.683 & 1.028 & 0.695 & 0.761 & 0.624 & 0.731 & 0.594 & 0.984 & 0.727 & 1.202 & 0.811 & 0.945 & 0.653 & 1.347 & 0.870 & 1.254 & 0.857 \\
& 720 & - & - & - & - & - & - & - & - & - & - & - & - & - & - & - & - & - & - & - & - \\
& \multicolumn{1}{c|}{Avg} & 0.458 & 0.456 & 0.850 & 0.662 & 1.026 & 0.687 & 0.722 & 0.599 & 0.659 & 0.562 & 0.750 & 0.611 & 1.225 & 0.817 & 0.926 & 0.648 & 1.452 & 0.903 & 1.242 & 0.835 \\
\midrule
\multirow{5}{*}{ETTh2} & 96 & 0.347 & 0.385 & 0.486 & 0.504 & 0.446 & 0.445 & 0.428 & 0.468 & 0.390 & 0.424 & 0.442 & 0.456 & 3.837 & 1.508 & 0.409 & 0.420 & 2.022 & 1.006 & 3.753 & 1.518 \\
& 192 & 0.538 & 0.510 & 0.521 & 0.520 & 0.520 & 0.483 & 0.496 & 0.504 & 0.457 & 0.465 & 0.617 & 0.542 & 3.975 & 1.933 & 0.483 & 0.464 & 3.534 & 1.348 & 3.516 & 1.473 \\
& 336 & 0.604 & 0.515 & 0.558 & 0.542 & 0.539 & 0.502 & 0.486 & 0.496 & 0.477 & 0.483 & 1.424 & 0.849 & 3.956 & 1.520 & 0.499 & 0.479 & 4.063 & 1.451 & 3.312 & 1.427 \\
& 720 & - & - & - & - & - & - & - & - & - & - & - & - & - & - & - & - & - & - & - & - \\
& \multicolumn{1}{c|}{Avg} & 0.496 & 0.470 & 0.522 & 0.522 & 0.502 & 0.477 & 0.470 & 0.489 & 0.441 & 0.457 & 0.828 & 0.616 & 3.923 & 1.654 & 0.464 & 0.454 & 3.206 & 1.268 & 3.527 & 1.473 \\
\midrule
\multirow{5}{*}{ETTm1} & 96 & 0.368 & 0.384 & 0.763 & 0.592 & 0.984 & 0.642 & 0.726 & 0.578 & 0.628 & 0.544 & 0.332 & 0.374 & 1.130 & 0.775 & 0.606 & 0.518 & 1.048 & 0.733 & 1.234 & 0.798 \\
& 192 & 0.384 & 0.400 & 0.772 & 0.594 & 1.003 & 0.653 & 0.750 & 0.591 & 0.666 & 0.566 & 0.358 & 0.390 & 1.150 & 0.788 & 0.681 & 0.539 & 1.097 & 0.756 & 1.287 & 0.839 \\
& 336 & 0.405 & 0.411 & 0.782 & 0.603 & 1.013 & 0.662 & 0.851 & 0.659 & 0.807 & 0.628 & 0.402 & 0.416 & 1.198 & 0.809 & 0.786 & 0.597 & 1.147 & 0.775 & 1.288 & 0.842 \\
& 720 & 0.471 & 0.452 & 0.819 & 0.623 & 1.038 & 0.679 & 0.857 & 0.655 & 0.822 & 0.633 & 0.511 & 0.489 & 1.175 & 0.794 & 0.796 & 0.593 & 1.200 & 0.799 & 1.247 & 0.828 \\
& \multicolumn{1}{c|}{Avg} & 0.407 & 0.412 & 0.784 & 0.603 & 1.009 & 0.659 & 0.796 & 0.621 & 0.731 & 0.593 & 0.401 & 0.417 & 1.163 & 0.792 & 0.717 & 0.562 & 1.123 & 0.766 & 1.264 & 0.827 \\
\midrule
\multirow{5}{*}{ETTm2} & 96 & 0.196 & 0.282 & 0.242 & 0.337 & 0.287 & 0.351 & 0.232 & 0.322 & 0.229 & 0.320 & 0.236 & 0.326 & 3.599 & 1.478 & 0.220 & 0.299 & 1.108 & 0.772 & 3.883 & 1.545 \\
& 192 & 0.277 & 0.331 & 0.298 & 0.367 & 0.342 & 0.381 & 0.291 & 0.357 & 0.394 & 0.381 & 0.306 & 0.373 & 3.578 & 1.475 & 0.311 & 0.361 & 1.317 & 0.850 & 3.553 & 1.484 \\
& 336 & 0.332 & 0.371 & 0.351 & 0.396 & 0.396 & 0.411 & 0.478 & 0.517 & 0.378 & 0.427 & 0.380 & 0.423 & 3.561 & 1.473 & 0.338 & 0.366 & 1.415 & 0.879 & 3.446 & 1.460 \\
& 720 & 0.441 & 0.427 & 0.451 & 0.448 & 0.491 & 0.458 & 0.553 & 0.538 & 0.523 & 0.510 & 0.674 & 0.583 & 3.896 & 1.533 & 0.509 & 0.465 & 1.822 & 0.984 & 3.445 & 1.460 \\
& \multicolumn{1}{c|}{Avg} & 0.311 & 0.353 & 0.335 & 0.387 & 0.379 & 0.400 & 0.388 & 0.433 & 0.381 & 0.405 & 0.399 & 0.426 & 3.659 & 1.490 & 0.345 & 0.373 & 1.416 & 0.871 & 3.582 & 1.487 \\
\midrule
\multirow{5}{*}{Weather} & 96 & 0.178 & 0.230 & 0.221 & 0.280 & 0.268 & 0.300 & 0.227 & 0.299 & 0.229 & 0.309 & 0.184 & 0.242 & 0.497 & 0.497 & 0.207 & 0.253 & 0.230 & 0.285 & 0.406 & 0.435 \\
& 192 & 0.223 & 0.270 & 0.265 & 0.317 & 0.317 & 0.334 & 0.278 & 0.333 & 0.265 & 0.317 & 0.228 & 0.283 & 0.620 & 0.545 & 0.272 & 0.307 & 0.274 & 0.323 & 0.446 & 0.450 \\
& 336 & 0.280 & 0.314 & 0.312 & 0.347 & 0.366 & 0.364 & 0.351 & 0.393 & 0.353 & 0.392 & 0.279 & 0.322 & 0.649 & 0.547 & 0.313 & 0.328 & 0.318 & 0.355 & 0.465 & 0.459 \\
& 720 & 0.350 & 0.363 & 0.381 & 0.387 & 0.434 & 0.405 & 0.387 & 0.389 & 0.391 & 0.394 & 0.364 & 0.388 & 0.570 & 0.522 & 0.400 & 0.385 & 0.401 & 0.418 & 0.471 & 0.468 \\
& \multicolumn{1}{c|}{Avg} & 0.258 & 0.294 & 0.295 & 0.333 & 0.347 & 0.351 & 0.311 & 0.354 & 0.310 & 0.353 & 0.264 & 0.309 & 0.584 & 0.528 & 0.298 & 0.318 & 0.306 & 0.345 & 0.447 & 0.453 \\
\midrule
\multirow{5}{*}{ECL} & 96 & 0.190 & 0.283 & 0.893 & 0.777 & 1.231 & 0.871 & 0.297 & 0.367 & 0.235 & 0.322 & 0.231 & 0.325 & 1.265 & 0.919 & 0.315 & 0.389 & 0.639 & 0.609 & 1.414 & 0.855 \\
& 192 & 0.193 & 0.287 & 0.906 & 0.782 & 1.233 & 0.872 & 0.308 & 0.375 & 0.247 & 0.341 & 0.231 & 0.329 & 1.298 & 0.939 & 0.318 & 0.396 & 0.772 & 0.678 & 1.240 & 0.919 \\
& 336 & 0.208 & 0.303 & 0.919 & 0.788 & 1.244 & 0.876 & 0.354 & 0.411 & 0.267 & 0.356 & 0.243 & 0.342 & 1.302 & 0.942 & 0.340 & 0.415 & 0.901 & 0.745 & 1.253 & 0.921 \\
& 720 & 0.242 & 0.333 & 0.947 & 0.796 & 1.274 & 0.888 & 0.426 & 0.466 & 0.318 & 0.394 & 0.277 & 0.370 & 1.269 & 0.919 & 0.635 & 0.613 & 1.200 & 0.871 & 1.249 & 0.921 \\
& \multicolumn{1}{c|}{Avg} & 0.208 & 0.301 & 0.916 & 0.786 & 1.245 & 0.877 & 0.346 & 0.405 & 0.267 & 0.353 & 0.246 & 0.342 & 1.281 & 0.930 & 0.402 & 0.453 & 0.878 & 0.726 & 1.289 & 0.904 \\
\midrule
\multirow{5}{*}{Traffic} & 96 & 0.642 & 0.376 & 1.701 & 0.868 & 2.078 & 1.016 & 0.795 & 0.481 & 0.670 & 0.421 & 0.748 & 0.459 & 1.557 & 0.821 & 0.854 & 0.492 & 1.157 & 0.636 & 1.586 & 0.841 \\
& 192 & 0.626 & 0.360 & 1.740 & 0.870 & 2.089 & 1.019 & 0.837 & 0.503 & 0.653 & 0.405 & 0.705 & 0.442 & 1.596 & 0.834 & 0.894 & 0.517 & 1.688 & 0.848 & 1.602 & 0.844 \\
& 336 & 0.627 & 0.362 & 1.746 & 0.871 & 2.107 & 1.023 & 0.867 & 0.523 & 0.707 & 0.445 & 0.635 & 0.397 & 1.621 & 0.841 & 0.853 & 0.471 & 1.826 & 0.903 & 1.668 & 0.868 \\
& 720 & - & - & - & - & - & - & - & - & - & - & - & - & - & - & - & - & - & - & - & - \\
& \multicolumn{1}{c|}{Avg} & 0.632 & 0.366 & 1.729 & 0.870 & 2.092 & 1.019 & 0.833 & 0.502 & 0.677 & 0.424 & 0.696 & 0.433 & 1.591 & 0.832 & 0.867 & 0.493 & 1.557 & 0.796 & 1.619 & 0.851 \\
\bottomrule
\end{tabular}
}
\label{tab:few_shot_5p_detail}
\end{table}

\subsection{Zero-shot Forecasting}
\label{appx:zero_shot_details}
\begin{table}[!ht]
\centering
\caption{Details of Zero-shot forecasting results.}
\vspace{-2mm}
\scalebox{0.78}{
\begin{tabular}{c|c|cc|cc|cc|cc|cc|cc|cc|cc|cc|cc}
\toprule
\multicolumn{2}{c|}{Methods} & \multicolumn{2}{c|}{LDM4TS} & \multicolumn{2}{c|}{CSDI} & \multicolumn{2}{c|}{ScoreGrad} & \multicolumn{2}{c|}{Autoformer} & \multicolumn{2}{c|}{FEDformer} & \multicolumn{2}{c|}{DLinear} & \multicolumn{2}{c|}{Informer} & \multicolumn{2}{c|}{ETSformer} & \multicolumn{2}{c|}{LightTS} & \multicolumn{2}{c}{Reformer} \\
\multicolumn{2}{c|}{Metric} & MSE & MAE & MSE & MAE & MSE & MAE & MSE & MAE & MSE & MAE & MSE & MAE & MSE & MAE & MSE & MAE & MSE & MAE & MSE & MAE \\ \midrule
\multirow{5}{*}{\makecell{ETTh1$\rightarrow$\\ETTh2}} & 96 & 0.349 & 0.383 & 0.463 & 0.489 & 0.502 & 0.497 & 0.469 & 0.486 & 0.414 & 0.447 & 0.347 & 0.400 & 1.834 & 1.058 & 0.381 & 0.427 & 0.376 & 0.424 & 1.833 & 1.060 \\
& 192 & 0.435 & 0.434 & 0.483 & 0.501 & 0.509 & 0.502 & 0.456 & 0.567 & 0.514 & 0.508 & 0.447 & 0.460 & 1.738 & 1.038 & 0.427 & 0.603 & 0.681 & 0.577 & 1.872 & 1.080 \\
& 336 & 0.478 & 0.465 & 0.477 & 0.559 & 0.506 & 0.503 & 0.655 & 0.588 & 0.533 & 0.525 & 0.515 & 0.505 & 2.513 & 1.212 & 0.658 & 0.615 & 1.169 & 0.795 & 2.004 & 1.114 \\
& 720 & 0.572 & 0.527 & 0.577 & 0.559 & 0.531 & 0.520 & 0.570 & 0.549 & 0.520 & 0.522 & 0.665 & 0.589 & 3.082 & 1.369 & 0.889 & 0.713 & 2.075 & 1.000 & 2.766 & 1.245 \\
& \multicolumn{1}{c|}{Avg} & 0.458 & 0.452 & 0.500 & 0.527 & 0.512 & 0.505 & 0.582 & 0.548 & 0.495 & 0.501 & 0.493 & 0.488 & 2.292 & 1.169 & 0.589 & 0.589 & 1.075 & 0.699 & 2.119 & 1.125 \\ \midrule
\multirow{5}{*}{\makecell{ETTh1 $\rightarrow$\\ ETTm2}} & 96 & 0.227 & 0.316 & 0.343 & 0.410 & 0.390 & 0.418 & 0.352 & 0.432 & 0.277 & 0.367 & 0.255 & 0.357 & 1.803 & 1.055 & 0.391 & 0.482 & 0.307 & 0.402 & 1.963 & 1.107 \\
& 192 & 0.312 & 0.373 & 0.383 & 0.430 & 0.418 & 0.433 & 0.413 & 0.460 & 0.329 & 0.399 & 0.338 & 0.413 & 1.682 & 1.017 & 0.514 & 0.514 & 0.614 & 0.560 & 2.007 & 1.131 \\
& 336 & 0.368 & 0.407 & 0.417 & 0.444 & 0.454 & 0.451 & 0.465 & 0.489 & 0.400 & 0.447 & 0.425 & 0.465 & 2.228 & 1.128 & 0.514 & 0.575 & 1.184 & 0.812 & 2.121 & 1.151 \\
& 720 & 0.569 & 0.505 & 0.499 & 0.489 & 0.521 & 0.486 & 0.599 & 0.551 & 0.485 & 0.482 & 0.640 & 0.573 & 2.954 & 1.297 & 0.857 & 0.699 & 2.128 & 1.028 & 2.822 & 1.272 \\
& \multicolumn{1}{c|}{Avg} & 0.369 & 0.400 & 0.410 & 0.444 & 0.446 & 0.447 & 0.457 & 0.483 & 0.373 & 0.424 & 0.415 & 0.452 & 2.167 & 1.124 & 0.569 & 0.568 & 1.058 & 0.700 & 2.228 & 1.165 \\ \midrule
\multirow{5}{*}{\makecell{ETTh2 $\rightarrow$\\ ETTh1}} & 96 & 0.706 & 0.540 & 1.318 & 0.879 & 1.033 & 0.703 & 0.693 & 0.569 & 2.310 & 1.238 & 0.689 & 0.555 & 1.157 & 0.781 & 0.863 & 0.651 & 0.528 & 0.486 & 0.902 & 0.689 \\
& 192 & 0.691 & 0.564 & 1.361 & 0.891 & 1.034 & 0.705 & 0.760 & 0.601 & 0.793 & 0.622 & 0.707 & 0.568 & 1.673 & 0.995 & 0.822 & 0.663 & 0.554 & 0.503 & 0.903 & 0.717 \\
& 336 & 0.697 & 0.576 & 1.431 & 0.901 & 1.025 & 0.707 & 0.781 & 0.619 & 0.883 & 0.662 & 0.710 & 0.577 & 1.730 & 0.978 & 0.883 & 0.696 & 0.598 & 0.535 & 0.933 & 0.713 \\
& 720 & 0.796 & 0.630 & 1.555 & 0.920 & 1.037 & 0.721 & 0.796 & 0.644 & 0.906 & 0.692 & 0.704 & 0.596 & 2.291 & 1.158 & 0.886 & 0.690 & 0.586 & 0.547 & 0.895 & 0.700 \\
& \multicolumn{1}{c|}{Avg} & 0.723 & 0.577 & 1.416 & 0.898 & 1.032 & 0.709 & 0.757 & 0.608 & 1.423 & 0.803 & 0.703 & 0.574 & 1.713 & 0.978 & 0.864 & 0.675 & 0.567 & 0.518 & 0.909 & 0.705 \\ \midrule
\multirow{5}{*}{\makecell{ETTh2 $\rightarrow$\\ ETTm2}} & 96 & 0.286 & 0.373 & 0.323 & 0.397 & 0.390 & 0.418 & 0.263 & 0.352 & 0.480 & 0.518 & 0.240 & 0.336 & 2.789 & 1.353 & 0.646 & 0.622 & 0.285 & 0.385 & 2.115 & 1.141 \\
& 192 & 0.326 & 0.391 & 0.363 & 0.418 & 0.418 & 0.433 & 0.326 & 0.389 & 0.328 & 0.383 & 0.295 & 0.369 & 8.210 & 2.073 & 1.370 & 0.965 & 0.482 & 0.499 & 2.830 & 1.282 \\
& 336 & 0.473 & 0.461 & 0.399 & 0.434 & 0.454 & 0.451 & 0.387 & 0.426 & 0.427 & 0.459 & 0.345 & 0.397 & 5.368 & 1.877 & 1.893 & 1.158 & 0.799 & 0.656 & 2.702 & 1.242 \\
& 720 & 0.644 & 0.551 & 0.502 & 0.497 & 0.521 & 0.486 & 0.467 & 0.478 & 0.496 & 0.481 & 0.432 & 0.442 & 4.054 & 1.660 & 1.371 & 0.921 & 1.246 & 0.801 & 3.257 & 1.363 \\
& \multicolumn{1}{c|}{Avg} & 0.432 & 0.444 & 0.397 & 0.437 & 0.446 & 0.447 & 0.386 & 0.411 & 0.433 & 0.460 & 0.328 & 0.386 & 4.606 & 1.763 & 1.320 & 0.917 & 0.703 & 0.585 & 2.726 & 1.257 \\ \midrule
\multirow{5}{*}{\makecell{ETTm1 $\rightarrow$\\ ETTh2}} & 96 & 0.390 & 0.402 & 0.466 & 0.492 & 0.502 & 0.497 & 0.435 & 0.470 & 0.573 & 0.556 & 0.365 & 0.415 & 1.377 & 0.894 & 0.474 & 0.500 & 0.388 & 0.444 & 1.621 & 1.008 \\
& 192 & 0.471 & 0.439 & 0.484 & 0.504 & 0.508 & 0.501 & 0.495 & 0.489 & 0.582 & 0.551 & 0.454 & 0.462 & 1.715 & 1.011 & 0.778 & 0.666 & 0.517 & 0.523 & 2.130 & 1.177 \\
& 336 & 0.496 & 0.460 & 0.506 & 0.514 & 0.506 & 0.503 & 0.470 & 0.472 & 0.578 & 0.563 & 0.496 & 0.494 & 1.519 & 0.940 & 0.668 & 0.601 & 0.576 & 0.575 & 1.844 & 1.056 \\
& 720 & 0.450 & 0.436 & 0.561 & 0.549 & 0.503 & 0.520 & 0.480 & 0.485 & 0.619 & 0.582 & 0.541 & 0.529 & 1.492 & 0.936 & 0.894 & 0.712 & 0.797 & 0.682 & 1.056 & 1.081 \\
& \multicolumn{1}{c|}{Avg} & 0.452 & 0.434 & 0.504 & 0.515 & 0.505 & 0.505 & 0.470 & 0.479 & 0.587 & 0.565 & 0.464 & 0.475 & 1.526 & 0.945 & 0.704 & 0.620 & 0.572 & 0.556 & 1.663 & 1.081 \\ \midrule
\multirow{5}{*}{\makecell{ETTm1 $\rightarrow$\\ ETTm2}} & 96 & 0.220 & 0.287 & 0.341 & 0.409 & 0.390 & 0.418 & 0.385 & 0.457 & 0.335 & 0.425 & 0.221 & 0.314 & 1.333 & 0.883 & 0.348 & 0.433 & 0.260 & 0.357 & 1.568 & 0.991 \\
& 192 & 0.284 & 0.335 & 0.374 & 0.424 & 0.418 & 0.433 & 0.433 & 0.469 & 0.384 & 0.439 & 0.286 & 0.359 & 1.617 & 0.985 & 0.651 & 0.625 & 0.366 & 0.442 & 2.271 & 1.223 \\
& 336 & 0.385 & 0.391 & 0.418 & 0.445 & 0.454 & 0.451 & 0.476 & 0.477 & 0.434 & 0.472 & 0.357 & 0.406 & 1.519 & 0.953 & 0.551 & 0.551 & 0.467 & 0.517 & 1.964 & 1.090 \\
& 720 & 0.574 & 0.484 & 0.485 & 0.484 & 0.521 & 0.486 & 0.582 & 0.535 & 0.543 & 0.516 & 0.476 & 0.476 & 1.615 & 0.985 & 0.863 & 0.701 & 0.772 & 0.666 & 2.264 & 1.142 \\
& \multicolumn{1}{c|}{Avg} & 0.366 & 0.374 & 0.405 & 0.440 & 0.446 & 0.447 & 0.469 & 0.484 & 0.424 & 0.463 & 0.335 & 0.389 & 1.521 & 0.951 & 0.603 & 0.578 & 0.466 & 0.495 & 2.017 & 1.111 \\ \midrule
\multirow{5}{*}{\makecell{ETTm2 $\rightarrow$\\ ETTh2}} & 96 & 0.368 & 0.416 & 0.425 & 0.467 & 0.502 & 0.497 & 0.353 & 0.393 & 0.773 & 0.637 & 0.333 & 0.391 & 1.877 & 0.699 & 0.608 & 0.608 & 0.381 & 0.433 & 1.006 & 0.750 \\
& 192 & 0.503 & 0.475 & 0.454 & 0.485 & 0.509 & 0.502 & 0.432 & 0.437 & 0.455 & 0.457 & 0.441 & 0.458 & 1.364 & 0.912 & 0.551 & 0.567 & 0.620 & 0.589 & 1.561 & 0.914 \\
& 336 & 0.595 & 0.529 & 0.475 & 0.493 & 0.507 & 0.503 & 0.452 & 0.459 & 0.482 & 0.485 & 0.505 & 0.503 & 1.449 & 0.929 & 0.886 & 0.770 & 0.911 & 0.733 & 2.243 & 1.121 \\
& 720 & 0.653 & 0.556 & 0.576 & 0.547 & 0.531 & 0.520 & 0.453 & 0.467 & 0.470 & 0.486 & 0.543 & 0.534 & 2.922 & 1.282 & 4.727 & 1.887 & 2.292 & 1.164 & 3.413 & 1.385 \\
& \multicolumn{1}{c|}{Avg} & 0.535 & 0.494 & 0.482 & 0.498 & 0.512 & 0.505 & 0.423 & 0.439 & 0.545 & 0.516 & 0.455 & 0.471 & 1.663 & 0.955 & 1.693 & 0.958 & 1.051 & 0.730 & 2.056 & 1.043 \\ \midrule
\multirow{5}{*}{\makecell{ETTm2 $\rightarrow$\\ ETTm1}} & 96 & 0.488 & 0.431 & 0.986 & 0.743 & 1.000 & 0.668 & 0.735 & 0.576 & 0.761 & 0.595 & 0.570 & 0.490 & 0.899 & 0.623 & 0.736 & 0.601 & 0.835 & 0.551 & 0.848 & 0.637 \\
& 192 & 0.591 & 0.497 & 1.057 & 0.764 & 1.009 & 0.673 & 0.753 & 0.586 & 0.857 & 0.632 & 0.590 & 0.506 & 0.816 & 0.621 & 0.703 & 0.591 & 0.773 & 0.554 & 0.915 & 0.687 \\
& 336 & 0.640 & 0.503 & 1.004 & 0.754 & 1.020 & 0.680 & 0.750 & 0.593 & 0.779 & 0.600 & 0.706 & 0.567 & 0.823 & 0.634 & 0.731 & 0.613 & 0.592 & 0.508 & 1.090 & 0.777 \\
& 720 & 0.631 & 0.515 & 1.107 & 0.791 & 1.033 & 0.691 & 0.782 & 0.609 & 0.879 & 0.645 & 0.731 & 0.584 & 0.877 & 0.672 & 0.740 & 0.624 & 0.665 & 0.587 & 0.910 & 0.690 \\
& \multicolumn{1}{c|}{Avg} & 0.588 & 0.487 & 1.039 & 0.763 & 1.016 & 0.678 & 0.755 & 0.591 & 0.819 & 0.618 & 0.649 & 0.537 & 0.854 & 0.637 & 0.728 & 0.607 & 0.716 & 0.550 & 0.941 & 0.698 \\
\bottomrule
\end{tabular}
}
\end{table}

% \subsection{Ablation Studies}
% \label{appx:ablation_details}
\newpage
\section{Interpretability of LDM4TS}
\subsection{Showcases}
\label{appx:showcases}
\begin{figure}[!ht]
  \centering
  \includegraphics[width=\linewidth]{figure/showcases_96.pdf}
\caption{Visualization of ETTh1 predictions by different models under the input-96-predict-96 setting. The orange lines stand for the ground truth and the blue lines stand for predicted values.}
\label{fig:showcases_96}
\end{figure}

\begin{figure}[!ht]
  \centering
  \includegraphics[width=\linewidth]{figure/showcases_192.pdf}
\caption{Visualization of ETTh1 predictions by different models under the input-96-predict-192 setting. The orange lines stand for the ground truth and the blue lines stand for predicted values.}
\label{fig:showcases_192}
\end{figure}

\begin{figure}[!ht]
  \centering
  \includegraphics[width=\linewidth]{figure/showcases_336.pdf}
\caption{Visualization of ETTh1 predictions by different models under the input-96-predict-336 setting. The orange lines stand for the ground truth and the blue lines stand for predicted values.}
\label{fig:showcases_336}
\end{figure}

\begin{figure}[!ht]
  \centering
  \includegraphics[width=\linewidth]{figure/showcases_720.pdf}
\caption{Visualization of ETTh1 predictions by different models under the input-96-predict-720 setting. The orange lines stand for the ground truth and the blue lines stand for predicted values.}
\label{fig:showcases_720}
\end{figure}

\newpage
\subsection{Visualization of Pixel Space}
\label{appx:visualization_picel_space}
\begin{figure*}[!ht]
  \centering
  \includegraphics[width=\linewidth]{figure/visualization_of_vision_encoder.pdf}
\caption{Visualization of different time series encoding methods for the vision encoder. We show three approaches: segmentation-based methods (SEG), Gramian Angular Field (GAF), and Recurrence Plot (RP). All methods transform raw time series into image representations with dimensions $\mathbb{R}^{B\times3 \times H\times W}$, where $B$ is the batch size, 3 represents RGB channels, $H$ and $W$ denote the height and width of the generated images.}
\label{fig:visualization_ve}
\end{figure*}
\newpage
\section{Prerequisites of Latent Diffusion Models}
\label{appx:ldm_algorithm}

\subsection{Autoencoder Framework}
Latent Diffusion Models (LDMs) leverage the autoencoder architecture to facilitate efficient learning in the latent space. An autoencoder comprises two primary components: an encoder and a decoder. The encoder $\mathcal{E}$ compresses high-dimensional input data $x \in \mathbb{R}^D$ into a lower-dimensional latent representation $z \in \mathbb{R}^d$, where $d \ll D$. This compression not only reduces the computational complexity but also captures the essential features of the data. In our implementation, we utilize the pre-trained AutoencoderKL from the \textbf{\textit{stable-diffusion-v1-4}}, which has demonstrated remarkable capabilities in image compression and reconstruction. Mathematically, this process is described as:
\begin{equation}
    z = \mathcal{E}(x)
\end{equation}

\paragraph{Latent Space Scaling}
\label{latent_space_scaling}
In practice, the latent representations produced by the encoder are typically scaled by a factor $s = 0.18215$ to ensure numerical stability and optimal distribution characteristics:
\begin{equation}
    z_{scaled} = s \cdot \mathcal{E}(x)
\end{equation}
This specific scaling factor originates from the VAE design in Stable Diffusion and is derived through empirical optimization. The value is calculated to minimize the KL divergence between the scaled latent distribution and the standard normal distribution:
\begin{equation}
    s^* = \operatorname*{argmin}_{s} \mathbb{E}_{x \sim p_{data}}[D_{KL}(s \cdot \mathcal{E}(x) \| \mathcal{N}(0, 1))]
\end{equation}
where $D_{KL}$ represents the Kullback-Leibler divergence. In our framework, this scaling operation serves multiple critical purposes. It ensures numerical stability during the diffusion process by maintaining consistent value ranges while facilitating better optimization dynamics by bringing the latent distribution closer to the standard normal. This operation also maintains compatibility with the pre-trained weights while allowing for efficient processing of our visual time series representations.

The optimization process involves collecting latent representations $z = \mathcal{E}(x)$ from a large dataset, computing their empirical statistics $\mu_z$ and $\sigma_z^2$, and determining the optimal scaling factor $s$ such that $s\sigma_z \approx 1$ to match the target standard deviation. This process has been extensively validated in the context of both image generation and, in our case, time series visual representations. During decoding, the inverse scaling is applied to restore the original magnitude:
\begin{equation}
    \hat{x} = \mathcal{D}(z_{scaled}/s)
\end{equation}
The autoencoder is trained to minimize the reconstruction loss:
\begin{equation}
    \mathcal{L}_{\text{AE}} = \|\mathcal{D}(\mathcal{E}(x)) - x\|_2^2
\end{equation}
However, in the context of LDMs, the autoencoder enables operations to be performed in the compressed latent space, thereby enhancing efficiency without significant loss of information. In our implementation, we freeze the pret-rained autoencoder parameters in the LDM4TS model during training, focusing the optimization process on diffusion dynamics and temporal feature extraction. This design choice significantly reduces computational overhead while maintaining the benefits of well-learned representations from the compressed latent space.

\subsection{Foundations of Diffusion Models}
Diffusion models define a principled framework for generative modeling through gradual noise addition and removal. In our LDM4TS framework, we adapt this process specifically for time series visual representations while maintaining the fundamental probabilistic structure.

\paragraph{Forward Process}
The forward diffusion process follows a Markov chain that progressively adds Gaussian noise:
\begin{align}
    q(x_t|x_{t-1}) &= \mathcal{N}(x_t; \sqrt{1-\beta_t}x_{t-1}, \beta_t\mathbf{I}) \\
    q(x_t|x_0) &= \mathcal{N}(x_t; \sqrt{\bar{\alpha}_t}x_0, (1-\bar{\alpha}_t)\mathbf{I})
\end{align}
Here, $q(x_t|x_{t-1})$ describes the transition from step $t-1$ to $t$, where $\beta_t$ controls the noise schedule. In our implementation, we adopt a linear noise schedule with carefully tuned parameters $\beta_{start}=0.00085$ and $\beta_{end}=0.012$. The second equation gives the direct relationship between any noisy sample $x_t$ and the original data $x_0$, where $\bar{\alpha}_t = \prod_{s=1}^t (1-\beta_s)$ represents the cumulative product of noise levels.

\paragraph{Reverse Process}
The reverse process learns to gradually denoise data through:
\begin{equation}
    p_\theta(x_{t-1}|x_t) = \mathcal{N}(x_{t-1}; \mu_\theta(x_t,t), \Sigma_\theta(x_t,t))
\end{equation}
where the mean and variance are parameterized as:
\begin{align}
    \mu_\theta(x_t,t) &= \frac{1}{\sqrt{\alpha_t}}(x_t - \frac{\beta_t}{\sqrt{1-\bar{\alpha}_t}}\epsilon_\theta(x_t,t)) \\
    \Sigma_\theta(x_t,t) &= \frac{1-\bar{\alpha}_{t-1}}{1-\bar{\alpha}_t}\beta_t
\end{align}
In our framework, we modify the noise prediction network $\epsilon_\theta$ to accept additional conditioning information, transforming the reverse process into:
\begin{equation}
    p_\theta(x_{t-1}|x_t,c) = \mathcal{N}(x_{t-1}; \mu_\theta(x_t,t,c), \Sigma_\theta(x_t,t))
\end{equation}
where $c$ represents the concatenated frequency domain embeddings and encoded textual descriptions. This modification allows the model to leverage both spectral and semantic information during the denoising process while maintaining the same variance schedule.

\paragraph{Score-based Generation}
The score function represents the gradient of the log-density:
\begin{equation}
    s_\theta(x_t,t) = \nabla_{x_t} \log p_\theta(x_t) = -\frac{\epsilon_\theta(x_t,t)}{\sqrt{1-\bar{\alpha}_t}}
\end{equation}
This formulation enables training through denoising score matching:
\begin{equation}
    \mathcal{L}_{\text{score}} = \mathbb{E}_{t,x_0,\epsilon}[\|\epsilon - \epsilon_\theta(x_t,t)\|_2^2]
\end{equation}

\paragraph{Sampling Methods}
Different sampling strategies offer various trade-offs between generation quality and computational efficiency. In our implementation, we primarily utilize DDIM for its deterministic nature and faster sampling capabilities, though both approaches are supported:
\begin{itemize}
    \item \textbf{DDPM}: Uses the full chain of $T$ steps with stochastic sampling:
    \begin{equation}
        x_{t-1} = \mu_\theta(x_t,t) + \sigma_t\epsilon, \quad \epsilon \sim \mathcal{N}(0,\mathbf{I})
    \end{equation}
    \item \textbf{DDIM}: Enables faster sampling through deterministic trajectories:
    \begin{equation}
        x_{t-1} = \sqrt{\bar{\alpha}_{t-1}}\left(\frac{x_t - \sqrt{1-\bar{\alpha}_t}\epsilon_\theta(x_t,t)}{\sqrt{\bar{\alpha}_t}}\right) + \sqrt{1-\bar{\alpha}_{t-1}}\epsilon_\theta(x_t,t)
    \end{equation}
\end{itemize}

\subsection{U-Net Architecture}
The U-Net architecture serves as the backbone for noise prediction in our framework, combining multi-view processing with skip connections specifically designed for time series visual patterns. Our implementation modifies the standard U-Net structure to better handle temporal dependencies while maintaining spatial coherence.

\paragraph{Encoder-Decoder Structure}
The architecture consists of multiple resolution levels:
\begin{itemize}
    \item \textbf{Downsampling path}: Progressive feature compression
    \begin{equation}
        h_l = \text{ResBlock}(\text{Down}(h_{l-1})), \quad l = 1,\ldots,L
    \end{equation}
    
    \item \textbf{Upsampling path}: Gradual feature reconstruction
    \begin{equation}
        h'_l = \text{ResBlock}(\text{Up}(h'_{l+1})) + h_l, \quad l = L,\ldots,1
    \end{equation}
    
    \item \textbf{Skip connections}: Feature preservation across scales
    \begin{equation}
        h'_l = h'_l + \text{Project}(h_l)
    \end{equation}
\end{itemize}

\paragraph{Feature Extraction}
Each resolution level processes features through a sequence of operations:
\begin{equation}
    F_l = \text{Conv}(\text{GroupNorm}(\text{Attention}(h_l)))
\end{equation}
These operations are augmented with timestep embeddings, which provide temporal information to guide the denoising process:
\begin{equation}
    \gamma_t = \text{MLP}(\text{SinusoidalPos}(t))
\end{equation}
In our implementation, the timestep embedding is projected through a two-layer MLP with SiLU activation, following the design choices in Stable Diffusion for consistency and stability.

\subsection{Attention Mechanisms}
Our model employs attention mechanisms to capture both local and global dependencies in the visual representations:
\paragraph{Self-Attention}
Computes interactions within a feature set through scaled dot-product attention:
\begin{equation}
    \text{Attention}(Q,K,V) = \text{softmax}(\frac{QK^T}{\sqrt{d_k}})V
\end{equation}
where $Q,K,V \in \mathbb{R}^{N \times d_k}$ are query, key, and value matrices.

\paragraph{Cross-Attention}
Enables conditioning through external information:
\begin{equation}
    \text{CrossAttn}(Q,K,V) = \text{softmax}(\frac{QK^T}{\sqrt{d_k}})V
\end{equation}
where $Q$ comes from the main branch and $K,V$ from conditioning information. Multi-head attention extends this:
\begin{equation}
    \text{MultiHead}(Q,K,V) = \text{Concat}(\text{head}_1,\ldots,\text{head}_h)W^O
\end{equation}

\subsection{Conditional Generation}
\label{appx:conditional_generation}
Our framework implements a sophisticated dual-conditioning mechanism that leverages both frequency domain features and semantic descriptions to guide the diffusion process. This multi-modal approach enables robust capture of both temporal patterns and contextual information:

\paragraph{Frequency Conditioning}
To effectively encode the rich spectral information inherent in time series data, we implement a sophisticated frequency domain transformation pipeline. This process begins with the application of a Hann window function, which is crucial for minimizing spectral leakage and enhancing frequency resolution:

\begin{equation}
    w_t = 0.5(1 - \cos(\frac{2\pi t}{L-1}))
\end{equation}
The frequency features are then systematically extracted through a carefully designed three-step process. First, we apply the window function to the input sequence:
\begin{equation}
    X_{win} = X \odot w
\end{equation}

Next, we transform the windowed signal into the frequency domain using the Fast Fourier Transform:
\begin{equation}
    X_{fft} = \text{FFT}(X_{win}) = \sum_{t=0}^{L-1} X_{win}(t) \cdot e^{-2\pi i k t / L}
\end{equation}
Finally, to preserve the complete spectral information, we concatenate the real and imaginary components of the FFT output:
\begin{equation}
    c_{freq} = \text{Concat}[X_{fft_{real}}, X_{fft_{imag}}] \in \mathbb{R}^{B \times (2DL+2)}
\end{equation}
where $L$ denotes the sequence length, $w$ represents the Hann window function, and $\odot$ indicates element-wise multiplication. The terms $X_{fft_{real}}$ and $X_{fft_{imag}}$ correspond to the real and imaginary components of the Fourier transform respectively. This comprehensive encoding strategy enables our model to capture both amplitude and phase information across multiple frequency bands, while maintaining computational efficiency through strategic dimensionality reduction.

\paragraph{Text Conditioning}
To provide semantic guidance for the diffusion process, we automatically generate descriptive prompts by extracting key characteristics from the input time series. The prompt generation function $d_{prompt}(X)$ captures the following statistical properties:
\vspace{-1em}
\begin{itemize}[leftmargin=*, itemsep=0pt]
    \item Statistical measures: minimum, maximum, and median values
    \item Temporal dynamics: overall trend direction and lag patterns
    \item Context information: prediction length and historical window size
    \item Domain knowledge: dataset-specific descriptions
\end{itemize}
\vspace{-1em}
These features are combined into a structured prompt template. A typical generated prompt follows the format:
\begin{quote}
\textit{\textcolor{gray}{"$<|start_prompt|>$Dataset description: \{description\}. Task: forecast the next \{pred\_len\} steps given the previous \{seq\_len\} steps. Input statistics: min value \{min\}, max value \{max\}, median value \{median\}, trend is \{trend\_direction\}, top-5 lags are \{lags\}.$<|<end_prompt>|>$"}}
\end{quote}

The prompts are then processed through a frozen \textit{\textbf{BERT-base-uncased}} model (110M parameters) to extract semantic features. Specifically, each prompt is first tokenized using BERT's WordPiece tokenizer with a maximum sequence length of 77 tokens:
\begin{equation}
    h_{token} = \text{BERT}(d_{prompt}(X)) \in \mathbb{R}^{B \times L_{seq} \times d_{ff}}
\end{equation}
where $L_{seq}$ is the sequence length after tokenization and $d_{ff}=768$ is BERT's hidden dimension. The token-level features are aggregated through mean pooling to obtain a sequence-level representation:
\begin{equation}
    h_{pool} = \frac{1}{L_{seq}}\sum_{i=1}^{L_{seq}} h_{token}[:,i,:] \in \mathbb{R}^{B \times d_{ff}}
\end{equation}
The pooled features are then projected to match the latent dimension through a learnable transformation:
\begin{equation}
    c_{text} = \text{TextEncoder}(X)= \text{TextProj}(h_{pool}) \in \mathbb{R}^{B \times d_{model}}
\end{equation}
where $\text{TextProj}(\cdot)$ consists of a linear layer that projects from $d_{ff}$ to $d_{model}$, followed by layer normalization and ReLU activation to enhance feature expressiveness.

The frequency and text conditions are fused through a cross-modal attention mechanism:
\begin{equation}
    c = \text{CrossAttn}(\text{MLP}([c_{\text{text}}; c_{\text{freq}}])) \in \mathbb{R}^{B \times d_{model}}
\end{equation}
where the MLP first projects the concatenated features to a higher dimension for better feature interaction, and the cross-attention layer enables dynamic feature selection based on the latent representation. This combined conditioning signal guides the diffusion process by injecting both semantic and frequency information into each denoising step through the attention blocks of the U-Net architecture.
\newpage
\section{Pseudo of LDM4TS framework}
\begin{algorithm}[H]
\caption{Training Process of LDM4TS}
\begin{algorithmic}[1]
\REQUIRE Time series data $\{\mathbf{X}, \mathbf{Y}\}$, where $\mathbf{X} \in \mathbb{R}^{B\times L\times D}$, $\mathbf{Y} \in \mathbb{R}^{B\times pred\_len\times D}$
\ENSURE Trained model parameters $\Theta$

\STATE Initialize VAE encoder $\mathcal{E}$, decoder $\mathcal{D}$, UNet $\mathcal{U}$, \hfill $\triangleright$ Model components
\STATE Initialize diffusion steps $T$, noise schedule $\{\beta_t\}_{t=1}^T$ \hfill $\triangleright$ Diffusion parameters
\STATE $\alpha_t = 1 - \beta_t$, $\bar{\alpha}_t = \prod_{s=1}^t \alpha_s$ \hfill $\triangleright$ Compute coefficients

\WHILE{not converged}
    \STATE Sample mini-batch $\{\mathbf{X}_b, \mathbf{Y}_b\}$ \hfill $\triangleright$ Get training batch
    
    \STATE // Data Preprocessing
    \STATE $means, stdev \gets \text{ComputeStats}(\mathbf{X}_b)$ \hfill $\triangleright$ Calculate statistics
    \STATE $\mathbf{X}_{norm} \gets (\mathbf{X}_b - means) / stdev$ \hfill $\triangleright$ Normalize input
    
    \STATE // Vision Transformation for Time Series
    \FOR{$i \in \{1,\ldots,D\}$}
        \STATE $\mathbf{X}_i \gets \mathbf{X}_{norm}[:,:,i]$ \hfill $\triangleright$ Extract dimension $i$ \;
        \STATE $\mathbf{I}_{seg,i} \gets \text{Segmentation}(\mathbf{X}_i)$ \hfill $\triangleright$ Time series to image \;
        \STATE $\mathbf{I}_{gaf,i} \gets \text{GramianAngularField}(\mathbf{X}_i)$ \hfill $\triangleright$ Polar encoding \;
        \STATE $\mathbf{I}_{rp,i} \gets \text{RecurrencePlot}(\mathbf{X}_i)$ \hfill $\triangleright$ Distance matrix
    \ENDFOR
    
    \STATE $\mathbf{I} \gets \text{Concat}([\mathbf{I}_{seg}, \mathbf{I}_{gaf}, \mathbf{I}_{rp}])$ \hfill $\triangleright$ Combine all views
    
    \STATE // Conditional Controls
    \STATE $\mathbf{f} \gets \text{FreqEncoder}(\text{FFT}(\mathbf{X}_{norm}))$ \hfill $\triangleright$ Extract frequency features
    \STATE $\mathbf{c} \gets \text{TextEncoder}(\text{GeneratePrompt}(\mathbf{X}_b))$ \hfill $\triangleright$ Generate text embedding
    \STATE $\mathbf{cond} \gets \text{FusionLayer}([\mathbf{f}, \mathbf{c}])$ \hfill $\triangleright$ Fuse conditions

    \STATE // Forward Diffusion Process
    \STATE $\epsilon \sim \mathcal{N}(0, \mathbf{I})$ \hfill $\triangleright$ Sample random noise
    \STATE $\mathbf{z}_t \gets \sqrt{\bar{\alpha}_t}\mathbf{z}_0 + \sqrt{1-\bar{\alpha}_t}\epsilon$ \hfill $\triangleright$ Noisy latent
    
    \STATE // Reverse Diffusion Process
    \STATE $\mathbf{z}_0 \gets (\mathbf{z}_t - \sqrt{1-\bar{\alpha}_t}\hat{\epsilon})/\sqrt{\bar{\alpha}_t}$ \hfill $\triangleright$ Denoised latent
    
    % \STATE // Latent Diffusion Process
    % \STATE $\epsilon \sim \mathcal{N}(0, \mathbf{I})$ \hfill $\triangleright$ Sample random noise
    % \STATE $\mathbf{z}_t \gets \sqrt{\bar{\alpha}_t}\mathbf{z}_0 + \sqrt{1-\bar{\alpha}_t}\epsilon$ \hfill $\triangleright$ Noisy latent
    
    % \FUNCTION{ReverseDiffusion}{$\mathbf{z}_t, \hat{\epsilon}, t$} \hfill $\triangleright$ Remove noise
    %     \RETURN $(\mathbf{z}_t - \sqrt{1-\bar{\alpha}_t}\hat{\epsilon})/\sqrt{\bar{\alpha}_t}$ \hfill $\triangleright$ Denoised latent
    % \ENDFUNCTION
    
    \STATE $\mathbf{z}_0 \gets \mathcal{E}(\mathbf{I})$ \hfill $\triangleright$ Encode to latent space
    \STATE Sample $t \sim \text{Uniform}\{1,...,T\}$ \hfill $\triangleright$ Random timestep
    \STATE $\mathbf{z}_t, \epsilon \gets \text{ForwardDiffusion}(\mathbf{z}_0, t)$ \hfill $\triangleright$ Forward process
    \STATE $\hat{\epsilon} \gets \mathcal{U}(\mathbf{z}_t, t, \mathbf{cond})$ \hfill $\triangleright$ Predict noise
    \STATE $\mathbf{z}_{rec} \gets \text{ReverseDiffusion}(\mathbf{z}_t, \hat{\epsilon}, t)$ \hfill $\triangleright$ Reverse process
    \STATE $\mathbf{I}_{rec} \gets \mathcal{D}(\mathbf{z}_{rec})$ \hfill $\triangleright$ Decode image
    
    \STATE // Feature Extraction and Fusion
    \STATE $\mathbf{h}_v \gets \text{VisionEncoder}(\mathbf{I}_{rec})$ \hfill $\triangleright$ Visual features
    \STATE $\mathbf{h}_t \gets \text{TemporalEncoder}(\text{PatchEmbed}(\mathbf{X}_{norm}))$ \hfill $\triangleright$ Temporal features
    \STATE $\alpha \gets \text{Softmax}(\text{MLP}([\mathbf{h}_v, \mathbf{h}_t]))$ \hfill $\triangleright$ Compute weights
    \STATE $\hat{\mathbf{Y}} \gets \alpha_1\text{VisionHead}(\mathbf{h}_v) + \alpha_2\text{TemporalHead}(\mathbf{h}_t)$ \hfill $\triangleright$ Fuse predictions
    
    \STATE // Optimization
    \STATE $\mathcal{L}_{diff} \gets \|\epsilon - \hat{\epsilon}\|_2^2$ \hfill $\triangleright$ Diffusion loss
    \STATE $\mathcal{L}_{pred} \gets \|\hat{\mathbf{Y}} - \mathbf{Y}_b\|_2^2$ \hfill $\triangleright$ Prediction loss
    \STATE $\mathcal{L} \gets \lambda_1\mathcal{L}_{diff} + \lambda_2\mathcal{L}_{pred}$ \hfill $\triangleright$ Total loss
    \STATE $\Theta \gets \text{Adam}(\Theta, \nabla_{\Theta}\mathcal{L})$ \hfill $\triangleright$ Update parameters
\ENDWHILE

\STATE \textbf{return} $\Theta$ \hfill $\triangleright$ Return trained model
\end{algorithmic}
\end{algorithm}
\newpage
\section{Analysis of Vision Transformation Methods}
\label{appx:transformation}

Time series analysis faces the fundamental challenge of capturing complex temporal dynamics that manifest simultaneously across multiple scales. While traditional methods excel at specific temporal resolutions, they often struggle to comprehensively capture the full spectrum of patterns ranging from rapid local variations to gradual global trends. This limitation motivates our investigation into vision transformation techniques that can effectively encode rich temporal information into spatial patterns, making them amenable to powerful vision-based processing approaches.

Our framework introduces a systematic approach to time series visualization through three theoretically-grounded transformation methods. Each method targets distinct yet complementary aspects of temporal dynamics, providing a comprehensive representation of the underlying time series structure:

\subsection{Segmentation Representation (SEG)}
The SEG transformation addresses the challenge of preserving local temporal structures while enabling efficient detection of periodic patterns. This method operates by restructuring a time series $x \in \mathbb{R}^L$ into a matrix $M \in \mathbb{R}^{\lceil L/T \rceil \times T}$, where T represents the period length. The transformation process can be formally expressed as:

\begin{equation}
    M_{i,j} = x_{(i-1)T + j}, \quad \text{for } i=1,\dots,\lceil L/T \rceil, j=1,\dots,T
\end{equation}

This segmentation approach offers several theoretical and practical advantages:

\begin{itemize}
    \item \textbf{Local Structure Preservation:} Each row in the matrix represents a complete segment of length T, maintaining the original temporal relationships at the finest granularity
    \item \textbf{Periodic Pattern Detection:} The vertical alignment of segments facilitates the identification of recurring patterns across different time periods
    \item \textbf{Multi-scale Analysis:} By varying the period length T, the transformation can capture patterns at different temporal scales, enabling hierarchical pattern discovery
\end{itemize}

The optimal period length T is determined through an optimization process that maximizes temporal correlation:
\begin{equation}
    T = \arg\max_{k} \sum_{i=1}^{\lceil L/k \rceil} \sum_{j=1}^{k-1} \text{Corr}(M_{i,j}, M_{i,j+1})
\end{equation}

where $\text{Corr}(\cdot,\cdot)$ denotes the correlation between adjacent columns. This optimization ensures optimal alignment of periodic patterns while maintaining temporal fidelity.

\subsection{Gramian Angular Field (GAF)}
The GAF transformation provides a sophisticated approach to encoding temporal relationships through polar coordinate mapping and trigonometric relationships. This method preserves both magnitude and temporal correlation information through a series of carefully designed transformations.

First, the time series is normalized to a bounded interval $[-1,1]$ or $[0,1]$:
\begin{equation}
    \tilde{x}_i = \frac{x_i - \min(x)}{\max(x) - \min(x)}
\end{equation}

The normalized values are then encoded in a polar coordinate system:
\begin{equation}
    \phi = \arccos(\tilde{x}_i), \quad r = \frac{t_i}{N}
\end{equation}

where $t_i$ represents temporal position and $N$ serves as a scaling factor. The final Gramian matrix is constructed through:
\begin{equation}
    G_{i,j} = \cos(\phi_i - \phi_j)
\end{equation}

This transformation offers several key advantages:
\begin{itemize}
    \item \textbf{Scale Invariance:} The polar encoding ensures that the representation is robust to amplitude variations
    \item \textbf{Temporal Correlation Preservation:} The Gramian matrix captures both local and global temporal dependencies
    \item \textbf{Dimensionality Reduction:} The transformation provides a compact representation while preserving essential temporal information
\end{itemize}

\subsection{Recurrence Plot (RP)}
The RP transformation leverages phase space reconstruction to visualize the recurrent behavior in dynamical systems. Based on Taken's embedding theorem, this method first reconstructs the phase space trajectory:

\begin{equation}
    \vec{x}_i = (x_i, x_{i+\tau}, ..., x_{i+(m-1)\tau})
\end{equation}

where $m$ is the embedding dimension and $\tau$ is the time delay. The recurrence matrix is then constructed as:

\begin{equation}
    R_{i,j} = \Theta(\epsilon - \|\vec{x}_i - \vec{x}_j\|)
\end{equation}

where $\Theta$ is the Heaviside function and $\epsilon$ is a threshold distance. This transformation reveals fundamental dynamical properties through several characteristic patterns:

\begin{itemize}
    \item \textbf{Diagonal Lines:} Parallel to the main diagonal, indicating similar evolution of trajectories and revealing deterministic structures
    \item \textbf{Vertical/Horizontal Lines:} Representing periods of state stagnation or laminar phases
    \item \textbf{Complex Patterns:} Non-uniform structures indicating chaos or non-linear dynamics
\end{itemize}

\subsection{Theoretical Integration and Complementarity}
The integration of these three transformations provides a comprehensive framework for time series analysis, offering several theoretical and practical advantages:

\begin{itemize}
    \item \textbf{Multi-scale Pattern Capture:} Each transformation targets different temporal scales - SEG preserves local structures, GAF encodes global correlations, and RP reveals system dynamics
    \item \textbf{Theoretical Complementarity:} The methods maintain theoretical orthogonality while targeting distinct aspects of temporal information
    \item \textbf{Robustness through Diversity:} The combination of different representations provides natural redundancy, mitigating individual limitations
    \item \textbf{Computational Efficiency:} All transformations leverage efficient matrix operations, making them practical for large-scale applications
\end{itemize}

Empirical evidence supports the effectiveness of this multi-view approach, demonstrating superior performance across diverse datasets and prediction horizons compared to single-transformation methods. This success validates our theoretical framework and suggests that comprehensive temporal feature extraction benefits significantly from the synergistic combination of complementary visual representations.
\section{Discussion on Applicability and Limitations}
\label{appx:discussion}

\subsection{Theoretical Foundations and Motivations}
The application of Latent Diffusion Models (LDM) to time series forecasting represents a fundamental shift from traditional approaches. This section examines the theoretical foundations and practical motivations behind our framework. Traditional time series forecasting methods often struggle with three key challenges: capturing complex distributions, modeling uncertainty, and handling long-range dependencies. Our LDM-based approach addresses these challenges through its principled probabilistic framework:
\begin{equation}
p(x_{t+1:t+H}|x_{1:t}) = \int p(z)p_\theta(x_{t+1:t+H}|z,x_{1:t})dz
\end{equation}

This formulation offers several advantages. First, it naturally models the full conditional distribution of future values rather than point estimates, enabling robust uncertainty quantification. Second, the iterative denoising process allows the model to capture temporal dependencies at multiple scales. Third, the latent space representation provides an efficient mechanism for learning compressed temporal features.

The integration of vision transformations with LDM is motivated by both theoretical insights and empirical observations. Time series data, when properly transformed into visual representations, exhibits several important properties:

\begin{enumerate}
\item \textbf{Pattern Preservation}: Visual encodings preserve crucial temporal structures including periodicity, trends, and seasonality through spatial arrangements. This is formally expressed as:
\begin{equation}
d(X_1, X_2) \propto d(\mathcal{V}(X_1), \mathcal{V}(X_2))
\end{equation}
where $\mathcal{V}$ represents our vision transformation processing.

\item \textbf{Information Complementarity}: Different visual encodings capture complementary aspects of temporal dynamics:
\begin{equation}
\mathcal{I}(X;Z) \geq \mathcal{I}(X;Z_{visual}) + \mathcal{I}(X;Z_{temporal})
\end{equation}
This property ensures that no critical temporal information is lost during transformation.

\item \textbf{Transfer Learning Potential}: The visual domain enables leveraging powerful pre-trained vision models and their hierarchical feature extraction capabilities.
\end{enumerate}

\subsection{Current Limitations and Future Directions}
The primary limitation of our framework centers on the computational efficiency of the diffusion process. The iterative nature of denoising introduces significant computational overhead, expressed as $T_{\text{total}} = T_{\text{visual}} + t \cdot T_{\text{diffusion}} + T_{\text{projection}}$, which poses challenges for real-time applications and resource-constrained environments. While we have implemented several efficiency-enhancing strategies, such as gradient-free vision transformations and component freezing (including text and vision encoders), the sequential nature of the diffusion process remains a fundamental constraint. The model also exhibits sensitivity to hyperparameter choices in the diffusion schedule, as reflected in the gradient behavior $\sigma(\nabla_\theta \mathcal{L}) \propto \lambda_1\beta_t + \lambda_2\alpha_t$, though this sensitivity is primarily confined to the diffusion component.

Our framework's modular architecture provides clear pathways for future improvements. The design allows for easy replacement and enhancement of individual components, from vision transformation methods (We were able to employ more than the three strategies in the article) to temporal encoder architectures. Future research could focus on developing more efficient training algorithms and incorporating adaptive computation mechanisms while maintaining the model's predictive power. The integration of sophisticated scheduling mechanisms or alternative diffusion formulations could address current computational constraints and improve training stability. Additionally, the framework's extensibility enables adaptation to specific domain requirements and integration of emerging techniques in both visual representation learning and diffusion modeling. These potential improvements, combined with the framework's demonstrated strengths in probabilistic modeling and multi-scale feature capture, suggest promising directions for advancing time series forecasting capabilities.
\newpage
\section{Efficiency Analysis}
\label{appx:efficiency}
\begin{table}[h]
\centering
\caption{Computational efficiency comparison on ETTh1 dataset. We report numbers of trainable parameters and inference time (in milliseconds) across different prediction horizons (H).}
\label{tab:efficiency}
\begin{tabular}{r|c|cccc}
\toprule
\multirow{2}{*}{Model} & \multirow{2}{*}{\# Parameters} & \multicolumn{4}{c}{Inference Time (ms)} \\
\cmidrule{3-6}
& & H = 96 & H = 192 & H = 336 & H = 720 \\
\midrule
LDM4TS (Ours) & 5.4M & 76.88 & 80.31 & 193.44 & 192.19 \\
\midrule
TimeGrad & 3.1M & 870.2 & 1854.5 & 3119.7 & 6724.1 \\
ScoreGrad & 4.65M & 3.44 & 4.53 & 4.22 & 7.81 \\ 
CSDI & 10M & 90.4 & 142.8 & 398.9 & 513.1 \\
SSSD & 32M & 418.6 & 645.4 & 1054.2 & 2516.9 \\
\bottomrule
\end{tabular}
\end{table}

We conduct a comprehensive efficiency analysis of LDM4TS, focusing on computational costs and model complexity through extensive experiments on the ETTh1 dataset. Our analysis encompasses both inference time measurements across various prediction horizons and parameter count comparisons with other popular diffusion-based models.

As shown in Table~\ref{tab:efficiency}, LDM4TS demonstrates substantial improvements in inference efficiency over most diffusion-based competitors. For shorter horizons (H=96,192), our model achieves inference times of 76.88ms and 80.31ms respectively, significantly outperforming TimeGrad (870.2ms, 1854.5ms) and SSSD (418.6ms, 645.4ms). Note that for H=720, we observe a slightly faster inference time than H=336, which may be attributed to varying GPU resource availability during our experiments. While ScoreGrad shows faster inference times, our extensive experiments demonstrate that LDM4TS achieves superior forecasting accuracy, offering a better trade-off between efficiency and performance.

The current results indicate that LDM4TS successfully mitigates the computational bottleneck typical of diffusion-based forecasting through our innovative architectural design, including VAE encoding and efficient cross-modal fusion mechanisms. Though traditional models like transformers and linear models maintain faster inference speeds due to their simpler architectures, our ongoing work focuses on further optimization to make diffusion-based forecasting more practical while preserving its superior modeling capabilities.

%%%%%%%%%%%%%%%%%%%%%%%%%%%%%%%%%%%%%%%%%%%%%%%%%%%%%%%%%%%%%%%%%%%%%%%%%%%%%%%
%%%%%%%%%%%%%%%%%%%%%%%%%%%%%%%%%%%%%%%%%%%%%%%%%%%%%%%%%%%%%%%%%%%%%%%%%%%%%%%


\end{document}


% This document was modified from the file originally made available by
% Pat Langley and Andrea Danyluk for ICML-2K. This version was created
% by Iain Murray in 2018, and modified by Alexandre Bouchard in
% 2019 and 2021 and by Csaba Szepesvari, Gang Niu and Sivan Sabato in 2022.
% Modified again in 2023 and 2024 by Sivan Sabato and Jonathan Scarlett.
% Previous contributors include Dan Roy, Lise Getoor and Tobias
% Scheffer, which was slightly modified from the 2010 version by
% Thorsten Joachims & Johannes Fuernkranz, slightly modified from the
% 2009 version by Kiri Wagstaff and Sam Roweis's 2008 version, which is
% slightly modified from Prasad Tadepalli's 2007 version which is a
% lightly changed version of the previous year's version by Andrew
% Moore, which was in turn edited from those of Kristian Kersting and
% Codrina Lauth. Alex Smola contributed to the algorithmic style files.
