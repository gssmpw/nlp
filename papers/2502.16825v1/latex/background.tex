\section{Background}

% \subsection{Reward Models}
% The reward model is a fundamental component of the conventional reinforcement learning from human feedback pipeline.
% Two responses of a prompt $x$, $y_w$ and $y_l$, represent the preferred and dispreferred completions, respectively. 
% It is assumed that these preferences are governed by a latent reward model $r^*(x, y)$, which is not directly accessible. 
% Among the many methods for modeling preferences, the Bradley-Terry (BT) framework is a popular choice. 
% This framework, along with more general approaches like the Plackett-Luce ranking model, can be extended to handle multiple answer ranking issues if needed. 
% The BT model defines the human preference probability $p^*$ as:
% \begin{align}
% p^*(y_w \succ y_l | x) &= \notag \\
% &\frac{\exp(r^*(x, y_w))}
% {\exp(r^*(x, y_w)) + \exp(r^*(x, y_l))} \nonumber
% \end{align}

% Given a dataset composed of pairs $D = \{(x^{(i)}, y_w^{(i)}, y_l^{(i)})\}_{i=1}^N$, a reward model $r_\phi(x, y)$ can be parameterized and trained using maximum likelihood. 
% Formulating this task as binary classification, the negative log-likelihood loss is given by:
% \begin{align}
% & \mathcal{L}_R(r_\phi, D) = \notag \\
% & -\mathbb{E}_{(x, y_w, y_l) \sim D} \left[ \log \sigma(r_\phi(x, y_w) - r_\phi(x, y_l)) \right] \nonumber
% \end{align}

% where $\sigma$ is the sigmoid function.

% In this paper, we focus on the application of reward models. We employ it to score completions of instructions and select appropriate responses according to the reward.

% Reduce spacing around subfigure captions
\captionsetup[subfigure]{aboveskip=0pt, belowskip=2pt}
% Reduce spacing around the main figure caption
\captionsetup[figure]{aboveskip=2pt, belowskip=0pt}

\begin{figure*}[!t]
\centering
\subfloat[Llama]{%
\begin{minipage}[t]{0.5\linewidth} % Group 1: First two images
    \centering
    \includegraphics[width=0.49\linewidth]{figs/llamabase.pdf}
    \hfill
    \includegraphics[width=0.49\linewidth]{figs/llamainstruct.pdf}
\end{minipage}%
}
\hfill
\subfloat[Mistral]{%
\begin{minipage}[t]{0.5\linewidth} % Group 2: Second two images
    \centering
    \includegraphics[width=0.49\linewidth]{figs/mistralbase.pdf}
    \hfill
    \includegraphics[width=0.49\linewidth]{figs/mistralinstruct.pdf}
\end{minipage}%
}
\caption{Alpaca evaluation results. The conventional approach which selects the response with the highest reward as the chosen response and the response with the lowest reward as the rejected response for DPO fails to improve the performance of models when we increase the number of samples. The x-axis represents the number of samples, while the y-axis shows the score (\%).}
\label{dy_wr}
\vspace{-1em}
\end{figure*}


\subsection{Direct Preference Optimization}
Different from conventional RLHF which first compresses human preferences into reward models, direct preference optimization is a RL-free algorithm for training language models to align with human preferences.
DPO is recognized as one of the most widely used methods for preference optimization. 
It reformulates the reward function $r$ into a closed-form expression aligned with the optimal policy:
\begin{equation}
r(x, y) = \beta \log \frac{\pi_\theta(y | x)}{\pi_{\text{ref}}(y | x)} + \beta \log Z(x) \nonumber
\end{equation}

where $\pi_\theta$ denotes the policy model, $\pi_{\text{ref}}$ represents the reference model (usually the supervised fine-tuned model) and $Z(x)$ is the partition function. 
By embedding this reward formulation into the Bradley-Terry (BT) ranking framework, the probability of preference $p(y_w > y_l | x)$ is computed as $\sigma(r(x, y_w) - r(x, y_l))$, where $\sigma$ is the sigmoid function. Therefore, DPO replaces the reliance on a reward model with the policy model, resulting in the following objective:
\begin{align}
& \mathcal{L}_{\text{DPO}}(\pi_\theta; \pi_{\text{ref}}) = \notag \\
& -\mathbb{E}_{(x, y_w, y_l) \sim \mathcal{D}} \Big[ \log \sigma(r(x, y_w) - r(x, y_l)) \Big] \notag
\end{align}

where $r(x, y) = \beta \log \frac{\pi_\theta(y \mid x)}{\pi_{\text{ref}}(y \mid x)}$.


\subsection{Preference Data Construction}
\label{conven_pipe}
Recently, a method for constructing preference pairs without relying on human annotations has been gaining popularity~\cite{dong2023raft, meng2024simpo}.
As shown in Figure~\ref{pipe}, given a language model policy $\pi_{\theta}$,  a reward function $r$ and $k$ prompts $\left\{x_i\right\}_{i=1}^k$, we sample $n$ generations $\left\{y_{ij}\right\}_{j=1}^n$ for the $i$-th prompt from $\pi_{\theta}$. 
The given reward model will be used to score the sampled generations.
The reward of $n$ candidate samples of $i$-th prompt is $\left\{r_{ij}\right\}_{j=1}^n$.
Afterwards, the completion of the highest reward score $\max_{j=1}^{n} \{r_{ij}\}$
is selected as the chosen response, while the completion of the lowest reward score $\min_{j=1}^{n} \{r_{ij}\}$ is selected as the rejected response to construct a preference pair $(y_w^{(i)}, y_l^{(i)})$ for $x_i$. 
In practice, $n=5$ can achieve significant performance gains~\cite{meng2024simpo}.
In this work, we explore the effects of increasing the number of samples, $n$.