% !TeX root = ../main.tex

\section{Conclusions}\label{Section8}

We conduct the first systematic literature review of OSF. Overall, we classify the literature according the four OS layers, \ie kernels, file systems, drivers, and hypervisors. We first summarize the general workflow of OSF, and then elaborate the details of each step of OSF. Further, we summarize unique fuzzing challenges for different OS layers. Based on the findings from our systematic survey, we discuss the future research directions in OSF. We hope our work will encourage further research in OSF and provide valuable guidance to newcomers in this field.

%We believe that exploring emerging technologies can effectively address challenges such as automated seed generation. Moreover, attention should be given to other new OS domains, such as fuzzing for embedded OS or for OS rewritten in new secure languages. 

% is an automated technique for testing operating system kernels, file systems, drivers, and hypervisors. Due to the vast codebase, the complexity of these components often exceeds that of traditional software fuzzing, making them more prone to unexpected crashes. This complexity is typically reflected in various aspects such as seed generation, selection, trimming, mutation, feedback calculation, and vulnerability monitoring. Additionally, different OS layers present unique challenges for fuzzing tasks. For example, kernel fuzzing focuses on generating valid syscall sequences and handling their complex argument interfaces; file system fuzzing emphasizes efficient mutation of image files; driver fuzzing requires support for a wide range of drivers to accommodate new code commits; and hypervisor fuzzing prioritizes discovering effective methods for fuzzing virtual device communication.

%Finally, conducting a trustworthy evaluation is crucial, as it helps prevent future researchers from being misled by erroneous results. 
