% !TeX root = ../main.tex

\section{Future Research Directions}\label{Section7}

\subsection{LLM-Based Test Case Generation}

Despite the widespread recognition of Syzkaller in kernel fuzzing, its support for rapidly iterating and continuously integrating open-source OSs exhibits significant limitations. Particularly,~for~systems such as Linux, with an average of 200 commits to the mainline per day \cite{Hao2022DemystifyingTD}, Syzlang necessitates continuous manual updates and enhancements. This is not only inefficient, but also leads to limited code coverage due to incomplete system call descriptions. The advent of large language models (LLMs) for automatic test case generation proposes a potential solution, leveraging the formidable language processing capabilities of LLMs for adapting OS specifications to Syzlang language learning and fine-tuning. This approach not only paves new avenues for the automatic generation of Syzlang, but also aims to enhance the accuracy and coverage of test cases. Furthermore, the linguistic processing prowess of LLMs could be employed to precisely analyze complex driver modules, such as DMA communications, thereby facilitating the automatic generation of test cases for Driver and Hypervisor layer I/O interfaces. Research in this direction could encompass developing novel model architectures, optimizing LLM training processes, and validating the efficacy and accuracy of LLM-generated test cases in practical OSF tasks.

\subsection{Dependency-Enhanced Fuzzing Solution}

Dependency issues are a key challenge in improving the effectiveness of OSF. Current approaches use program analysis techniques to extract the control flow and data flow of the PUT to establish dependency relationships between seeds. Although these approaches can generate semantically correct seed sequence, they often struggle to trigger deep vulnerabilities with state dependencies due to the lack of contextual information in static control flow dependencies. For example, \texttt{setsockopt} must be called twice in a row to trigger a specific vulnerability \cite{Xu2024MOCKOK}. Additionally, while the latest approach, Mock, dynamically learns state dependencies using neural network models, it affects fuzzing efficiency and introduces a degree of randomness. To simultaneously address both control flow and state dependencies, future research could explore minimal common seed sequences by mining sources such as CVE reports (state dependencies), PoCs (state dependencies), and real-world applications (control flow dependencies) before fuzzing is executed. Furthermore, by introducing a domain-specific language (DSL) in Syzlang to describe the dependency relationships of seeds, an end-to-end OSF solution with dependency-enhanced handling could be achieved. Although DSL cannot cover all dependencies, it provides a solid starting point for OSF, enabling seamless integration with learning models, dynamic program flow analysis, and other techniques during fuzzing to continuously strengthen the expressiveness and capture of dependencies.

\subsection{Embedded Operating System Fuzzing}

With the widespread deployment of embedded OSs (\eg RT-Linux, FreeRTOS, Zephyr, \etc)~in~industrial software infrastructure, \eg autonomous driving, ensuring their security and reliability is of utmost importance. However, vulnerabilities within the code of these embedded open-source OSs may present significant long-term security and safety threats to these industries. Furthermore, the application and effectiveness of fuzzing techniques in embedded devices are significantly~constrained by the limited computational resources and the high costs of experimental equipment. While simulation environments offer a compromise between flexibility and realism, they fall short~of accurately replicating the intricate interplay of hardware and software found in embedded systems. This limitation becomes particularly acute in fuzzing OSs for intelligent vehicles, where not all aspects of embedded hardware and software can be effectively simulated. Consequently, a significant research direction for the future involves identifying and developing methods to improve both the execution efficiency and the code coverage of fuzzing techniques, specifically within the practical constraints of embedded environments.

\subsection{Rust-Based Kernel Fuzzing}

With the increasing utilization of the Rust language in the development of open-source OS kernels due to its memory safety features, fuzzing for Rust-based kernels has become critically important. Although Syzlang is a widely used to generate initial seeds in Syzlang, its construction~in~C~poses challenges for direct application to Rust-based environments. Research efforts could begin~by~attempting to port Syzkaller to Rust-based kernels, focusing on analyzing the relationships between system calls and initially setting aside the complexities of Syzkaller's construction. The Rust community is still in the process of developing features analogous to the Sanitizer and KCOV modules, which are crucial for advanced fuzzing tasks. The creation of Rust equivalents for these tools, along with a new system call description language tailored for Rust, could significantly advance the field. Furthermore, by leveraging LLMs to accelerate the generation of automatic test cases, a higher degree of automation and efficiency can be achieved in fuzzing for Rust-based kernels.

%\textbf{Direction 4: Enhance the dependency in RTOS Fuzzing.}