% !TeX root = ../main.tex

\section{Introduction}

Operating systems (OSs) play a crucial role in the operation of modern computer systems.~They~are responsible for managing the hardware and software resources of a computer. 
%They provide a unified abstraction interface and resource management capability to application-level software systems. They also offer a highly compatible middle layer and efficient resource scheduling mechanisms for heterogeneous hardware, 
They serve~as~the cornerstone for maximizing hardware resource utilization and ensuring software system stability. Specifically, open-source OSs, such as Linux, Android, FreeRTOS, Zephyr, \etc, have become~particularly favored in most fields like autonomous driving \cite{Apollo,Chen2022ParallelDO}, robotics \cite{ROS2}, cloud services~\cite{Li2015LeveragingLC},~and Web of Things \cite{Liu2023LeveragingAO,Sun2023SCTAPSS} for the advantages of transparency, customizability, and community-driven innovation, leading to their anticipated dominance in future computer systems.

However, as the scale of open-source OS codebases continues to expand, the security threats they pose have become increasingly alarming, raising significant concerns about the secure operation of software systems. According to National Vulnerability Database (NVD) \cite{NVD},~more~than~9,300~vulnerabilities have been discovered in the kernels of mainstream open-source OSs~(including Linux, Android, FreeBSD, OpenBSD and Zephyr) since 2004. In particular, the number of vulnerabilities surges to 3,300 in the single year of 2024, an increase that is 10 times higher than the previous~year. % and more than 5 times the number of vulnerabilities in 2017 (which previously held the highest record for vulnerabilities). 
Similarly, the Common Vulnerabilities and Exposures (CVE) \cite{cve} has documented at least 7,600 vulnerabilities in the Linux kernel to date. Among them, NVD has disclosed at least 1,566 high-severity vulnerabilities and 156 critical vulnerabilities that require immediate remediation. 
% Additionally, syzbot \cite{Syzbot} has disclosed that it has already fixed over 5,600 Linux kernel vulnerabilities. 

%For example, Stoep \cite{stoep2016android} reported that the number of vulnerabilities in the Android kernel increased from 4\% in 2014 to 39\% in 2016.

OS-level vulnerabilities are particularly concerning when compared to those in user-level applications. This is because such vulnerabilities can be further exploited, potentially allowing attackers to gain complete control over the OSs \cite{kernel-exploits, Xu2015FromCT, Zhang2015AndroidRA,chen2019slake}, leading to irreversible consequences. The staggering number of these vulnerabilities and the malicious outcomes they have caused have~attracted significant attention from security researchers. As a result, there is a strong and growing~interest in developing effective and efficient techniques to identify and mitigate these potential vulnerabilities, thereby aiding the continuous evolution of open-source OSs. One of the widely used techniques is fuzzing, which was first introduced by Miller et al. \cite{Miller1990AnES} in 1990, and has achieved notable~success across various domains, \eg compilers \cite{Yang2023WhiteboxCF}, interpreters \cite{Holler2012FuzzingWC}, and open-source software~\cite{Serebryany2017OSSFuzzG}.

Fuzzing, also known as fuzz testing, is a technique that involves feeding semi-randomly generated test cases as inputs to the program under test (PUT) to trigger program paths that may contain software vulnerabilities. Over the past decade, fuzzing has become capable of effectively testing complex OS code. This progress has received widespread attention from researchers, who aim to enhance the depth and breadth of OS fuzzing ({OSF}) by incorporating cutting-edge techniques~such~as program analysis and deep learning. However, compared to traditional fuzzing, the complex domain knowledge involved in OS makes developing effective and efficient OSF particularly~challenging. It not only requires focusing on advancements in fuzzing techniques, but also demands consideration of the inherent complexity and multi-layered interaction of OS. Typically, survey papers play~a~key role in advancing this field by providing a comprehensive review of the OSF techniques as well as summarizing existing challenges and pinpointing potential directions.

However, to date, there has been no systematic review of OSF. While there are some survey~papers on traditional software fuzzing \cite{Eisele2022EmbeddedFA,Yun2022FuzzingOE,Zhu2022FuzzingAS,Mans2018TheAS,Li2018FuzzingAS,Liang2018FuzzingSO,Zhang2018SurveyOD,Mallissery2023DemystifyTF,bohme2020fuzzing,Godefroid2020}, they do not systematically introduce the general steps of OSF, and they also do not highlight the unique challenges that~OSF faces compared to traditional software fuzzing. Therefore, to bridge the gap, a systematic review of the state-of-the-art OSF is essential, aiming to provide a comprehensive guideline on this topic.

% For the scope of this survey, after conducting a systematic search of state-of-the-art OSF, we found that in addition to \textbf{Kernel}, \textbf{File System}, and \textbf{Driver}, there are also papers focusing on \textbf{Hypervisor} fuzzing. This is because OSF requires support from hypervisors as a running environment, and vulnerabilities in hypervisors themselves could potentially lead to security risks for the OS. Therefore, it is reasonable to include fuzzing papers or tools targeting these four OS layers in our review taxonomy to ensure the completeness of our survey objectives.

For the scope of this survey, after conducting a systematic search of state-of-the-art OSF, we observed that the PUT is typically classified into \textit{kernel}, \textit{file system}, \textit{driver}, and \textit{hypervisor} because each of these OS layers present distinct challenges in fuzzing. Specifically, in a highly heterogeneous hardware environment (\eg for intelligent vehicles), achieving unified resource management within the OS requires the support from \textit{hypervisor}; and vulnerabilities from \textit{hypervisor} can be exploited~to launch malicious attacks against the host OS. Therefore, to ensure the completeness of our survey, we also included \textit{hypervisor}, as a component of a generalized OS, within our scope. 

%Regarding the survey process, our core approach was to examine all collected OSF studies with the goal of addressing the general procedures of OSF, specific problems in fuzzing different OS layers, and considerations for credible evaluation.

To systematically review OSF, we designed a comprehensive search process to identify existing high-quality research materials (see Section \ref{Section2}), including \todo{58} state-of-the-art OSF papers and~\todo{4}~open-source OSF tools. Through a thorough analysis of these materials, we uncovered the rising trend in OSF research and explained the reasons behind it. Then, we introduced the distinctive features~of fuzzing techniques for the four OS layers (\ie kernel, file system, driver, and hypervisor) from~a high-level perspective, and summarized a general workflow of OSF, consisting of three core modules, \ie \textit{input}, \textit{fuzzing engine}, and \textit{running~environment} (see Section \ref{Section3}). By analyzing the control flow and data flow between these modules,~we~further clarified the fundamental differences between OSF and traditional software fuzzing. Next, we conducted a comprehensive review of the state-of-the-art OSF with respect to seven key steps in the three core modules (see Section \ref{Section4}), aiming~to~systematically sort out the technological advancements in OSF and the associated complex domain knowledge. Meanwhile, we summarized the unique problems encountered by the four OS layers during fuzzing as well as the corresponding solutions (see Section \ref{Section5}). 
% Additionally, we identify potential trustworthy issues in the experimental evaluations of existing OSF research in Section \ref{Section6}. In response, we conduct a trustworthy analysis of current OSF experiments across three dimensions: experimental setting, effectiveness, and performance, and proposed five suggestions to help researchers improve the trustworthy of their experiments. 
Finally, based on the findings of our review, we provided \todo{four} future research directions in the OSF field (see Section \ref{Section7}).

%To systematically review OSF, we designed a comprehensive search process to identify existing high-quality research materials, including papers and open-source tools. Through a thorough analysis of these materials, we uncovered the rising trend in OSF research and explained the reasons behind it. Before diving into a detailed technical analysis, we first outlined the three core modules of OSF: \textbf{Input}, \textbf{Fuzzing Engine}, and \textbf{Running Environment}. By analyzing the control flow and data flow between these modules, we further clarified the fundamental differences between OSF and traditional fuzzing techniques. To systematically sort out the technological advancements in OSF and the associated complex domain knowledge, we refined the core modules into eight key steps and conducted a comprehensive review of the latest research outcomes for each step. At the same time, we summarized the unique \todo{difficults/challenges} encountered by four OS layers (Kernel, File System, Driver, and Hypervisor) during fuzzing, as well as the corresponding solutions. Additionally, we identified potential trustworthy issues in the experimental evaluations of existing OSF research. In response, we conducted a trustworthy analysis of current OSF experiments across three dimensions: experimental setting, effectiveness, and performance, and proposed five suggestions to help researchers improve the trustworthy of their experiments. Finally, based on the findings of our review and discussions, we outlined \todo{xxx} future research directions in the OSF field and concluded the paper with a summary.

%The article is structured as follows. Section \ref{Section2} presents a methodology to gathering 54 state-of-the-art OSF papers and 3 open-source tools. Section \ref{Section3} provides an overview of the current state of OSF research, comparing the distinctive features of 4 OS layer fuzzing techniques from a high-level perspective. Section \ref{Section4} discusses the general steps of executing OSF, highlighting the advantages and disadvantages of different techniques. Section \ref{Section5} delves into the specific problems of fuzzing at each OS layer, summarizing the research progress on these problems. Section \ref{Section6} addresses how to conduct a trustworthy evaluation, outlining key considerations. Section \ref{Section7} reflects on future research trends in OSF based on the findings of this survey. Finally, section \ref{Section8} concludes the article.

