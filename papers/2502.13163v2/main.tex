%%
%% This is file `sample-acmsmall.tex',
%% generated with the docstrip utility.
%%
%% The original source files were:
%%
%% samples.dtx  (with options: `all,journal,bibtex,acmsmall')
%% 
%% IMPORTANT NOTICE:
%% 
%% For the copyright see the source file.
%% 
%% Any modified versions of this file must be renamed
%% with new filenames distinct from sample-acmsmall.tex.
%% 
%% For distribution of the original source see the terms
%% for copying and modification in the file samples.dtx.
%% 
%% This generated file may be distributed as long as the
%% original source files, as listed above, are part of the
%% same distribution. (The sources need not necessarily be
%% in the same archive or directory.)
%%
%%
%% Commands for TeXCount
%TC:macro \cite [option:text,text]
%TC:macro \citep [option:text,text]
%TC:macro \citet [option:text,text]
%TC:envir table 0 1
%TC:envir table* 0 1
%TC:envir tabular [ignore] word
%TC:envir displaymath 0 word
%TC:envir math 0 word
%TC:envir comment 0 0
%%
%%
%% The first command in your LaTeX source must be the \documentclass
%% command.
%%
%% For submission and review of your manuscript please change the
%% command to \documentclass[manuscript, screen, review]{acmart}.
%%
%% When submitting camera ready or to TAPS, please change the command
%% to \documentclass[sigconf]{acmart} or whichever template is required
%% for your publication.
%%
%%

\documentclass[acmsmall, screen]{acmart}
%%
%% \BibTeX command to typeset BibTeX logo in the docs
\AtBeginDocument{%
  \providecommand\BibTeX{{%
    Bib\TeX}}}

%% Rights management information.  This information is sent to you
%% when you complete the rights form.  These commands have SAMPLE
%% values in them; it is your responsibility as an author to replace
%% the commands and values with those provided to you when you
%% complete the rights form.
\setcopyright{acmlicensed}
% \copyrightyear{2018}
% \acmYear{2018}
% \acmDOI{XXXXXXX.XXXXXXX}


%%
%% These commands are for a JOURNAL article.
%\acmJournal{CSUR}
% \acmVolume{37}
% \acmNumber{4}
% \acmArticle{111}
% \acmMonth{8}

%%
%% Submission ID.
%% Use this when submitting an article to a sponsored event. You'll
%% receive a unique submission ID from the organizers
%% of the event, and this ID should be used as the parameter to this command.
%%\acmSubmissionID{123-A56-BU3}

%%
%% For managing citations, it is recommended to use bibliography
%% files in BibTeX format.
%%
%% You can then either use BibTeX with the ACM-Reference-Format style,
%% or BibLaTeX with the acmnumeric or acmauthoryear sytles, that include
%% support for advanced citation of software artefact from the
%% biblatex-software package, also separately available on CTAN.
%%
%% Look at the sample-*-biblatex.tex files for templates showcasing
%% the biblatex styles.
%%

%%
%% The majority of ACM publications use numbered citations and
%% references.  The command \citestyle{authoryear} switches to the
%% "author year" style.
%%
%% If you are preparing content for an event
%% sponsored by ACM SIGGRAPH, you must use the "author year" style of
%% citations and references.
%% Uncommenting
%% the next command will enable that style.
%%\citestyle{acmauthoryear}

\usepackage{longtable}
\let\Bbbk\relax
\usepackage{amssymb} % For checkmark symbol
\usepackage{array} % 提供额外的列定义选项

\usepackage{xspace}
\usepackage{setspace}
\usepackage{tcolorbox}
\usepackage{booktabs} % 导入三线表需要的宏包
\usepackage{ragged2e}
\usepackage{multirow}
\usepackage{graphicx}
\usepackage{adjustbox}
\usepackage{tabularx}
\newcolumntype{C}[1]{>{\centering\arraybackslash}p{#1}}
\usepackage{colortbl}
\usepackage{graphicx}  %插入图片的宏包
\usepackage{float}  %设置图片浮动位置的宏包
\usepackage{subfigure}  %插入多图时用子图显示的宏包
\usepackage{color}
\usepackage{arydshln} % 负责画虚线的包
\usepackage{caption}
\usepackage{wasysym}
\usepackage{tikz}
\usepackage{makecell}
\usepackage{enumitem}

% save space
\usepackage{microtype}
\setlength\floatsep{0.4\baselineskip plus 3pt minus 2pt} % distance between two floats
\setlength\textfloatsep{0.4\baselineskip plus 3pt minus 2pt} % distance between floats on the top or the bottom and the text
\setlength\intextsep{0.4\baselineskip plus 3pt minus 2pt} % distance between floats inserted inside the text (using h) and the text
\setlength\dbltextfloatsep{0.4\baselineskip plus 3pt minus 2pt} % distance between a float spanning both columns and the text
\setlength\dblfloatsep{0.4\baselineskip plus 3pt minus 2pt} % distance between two floats spanning both columns.

\newcommand*\emptycirc[1][1ex]{\tikz\draw (0,0) circle (#1);} 
\newcommand*\halfcirchor[1][1ex]{%
	\begin{tikzpicture}
	\draw[fill] (0,0)-- (90:#1) arc (90:270:#1) -- cycle ;
	\draw (0,0) circle (#1);
	\end{tikzpicture}}

\newcommand*\halfcircver[1][1ex]{%
  \begin{tikzpicture}
    \fill[black] (0,0) -- (#1,0) arc (0:180:#1) -- cycle; % Fill top half with black
    \fill[white] (0,0) -- (#1,0) arc (0:-180:#1) -- cycle; % Fill bottom half with white
    \draw (0,0) circle (#1); % Draw the circle outline
  \end{tikzpicture}}

\newcommand*\fullcirc[1][1ex]{\tikz\fill (0,0) circle (#1);} 

% Define a new command for circled numbers
\newcommand*\circled[1]{\tikz[baseline=(char.base)]{
            \node[shape=circle,draw,inner sep=0.5pt] (char) {#1};}}

% Define a new command for rotated table headers
\newcommand{\rotatedheader}[1]{\rotatebox{90}{#1}}
% 定义一个新的列类型,命名为 'M',这将创建一个居中对齐的段落列
\newcolumntype{M}[1]{>{\centering\arraybackslash}m{#1}}

\newcommand{\ie}[0]{\textit{i.e.,}\xspace}
\newcommand{\eg}[0]{\textit{e.g.,}\xspace}
\newcommand{\etc}[0]{\textit{etc.}\xspace}

\newcommand{\todo}[1]{\textcolor{black}{#1}}


%%
%% end of the preamble, start of the body of the document source.
\begin{document}

%%
%% The "title" command has an optional parameter,
%% allowing the author to define a "short title" to be used in page headers.
\title{A Survey of Fuzzing Open-Source Operating Systems}

\author{Kun Hu}
\email{huk23@m.fudan.edu.cn}
\affiliation{%
  \department{School of Computer Science and Shanghai Key Laboratory of Data Science}
  \institution{Fudan University}
  \state{Shanghai}
  \country{China}
  \postcode{200433}
}

\author{Qicai Chen}
\email{qcchen23@m.fudan.edu.cn}
\affiliation{%
  \department{School of Computer Science and Shanghai Key Laboratory of Data Science}
  \institution{Fudan University}
  \state{Shanghai}
  \country{China}
  \postcode{200433}
}

\author{Zilong Lu}
\email{zllu23@m.fudan.edu.cn}
\affiliation{%
  \department{School of Computer Science and Shanghai Key Laboratory of Data Science}
  \institution{Fudan University}
  \state{Shanghai}
  \country{China}
  \postcode{200433}
}

\author{Wenzhuo Zhang}
\email{wzzhang24@m.fudan.edu.cn}
\affiliation{%
  \department{School of Computer Science and Shanghai Key Laboratory of Data Science}
  \institution{Fudan University}
  \state{Shanghai}
  \country{China}
  \postcode{200433}
}

\author{Bihuan Chen}
\email{bhchen@fudan.edu.cn}
\affiliation{%
  \department{School of Computer Science and Shanghai Key Laboratory of Data Science}
  \institution{Fudan University}
  \state{Shanghai}
  \country{China}
  \postcode{200433}
}

\author{You Lu}
\email{ylu24@m.fudan.edu.cn}
\affiliation{%
  \department{School of Computer Science and Shanghai Key Laboratory of Data Science}
  \institution{Fudan University}
  \state{Shanghai}
  \country{China}
  \postcode{200433}
}

\author{Haowen Jiang}
\email{hwjiang23@m.fudan.edu.cn}
\affiliation{%
  \department{School of Computer Science and Shanghai Key Laboratory of Data Science}
  \institution{Fudan University}
  \state{Shanghai}
  \country{China}
  \postcode{200433}
}

\author{Bingkun Sun}
\email{bksun21@m.fudan.edu.cn}
\affiliation{%
  \department{School of Computer Science and Shanghai Key Laboratory of Data Science}
  \institution{Fudan University}
  \state{Shanghai}
  \country{China}
  \postcode{200433}
}

\author{Xin Peng}
\email{pengxin@fudan.edu.cn}
\affiliation{%
  \department{School of Computer Science and Shanghai Key Laboratory of Data Science}
  \institution{Fudan University}
  \state{Shanghai}
  \country{China}
  \postcode{200433}
}

\author{Wenyun Zhao}
\email{wyzhao@fudan.edu.cn}
\affiliation{%
  \department{School of Computer Science and Shanghai Key Laboratory of Data Science}
  \institution{Fudan University}
  \state{Shanghai}
  \country{China}
  \postcode{200433}
}

%%
%% The "author" command and its associated commands are used to define
%% the authors and their affiliations.
%% Of note is the shared affiliation of the first two authors, and the
%% "authornote" and "authornotemark" commands
%% used to denote shared contribution to the research.
% \author{Ben Trovato}
% \authornote{Both authors contributed equally to this research.}
% \email{trovato@corporation.com}
% \orcid{1234-5678-9012}
% \author{G.K.M. Tobin}
% \authornotemark[1]
% \email{webmaster@marysville-ohio.com}
% \affiliation{%
%   \institution{Institute for Clarity in Documentation}
%   \streetaddress{P.O. Box 1212}
%   \city{Dublin}
%   \state{Ohio}
%   \country{USA}
%   \postcode{43017-6221}
% }

% \author{Lars Th{\o}rv{\"a}ld}
% \affiliation{%
%   \institution{The Th{\o}rv{\"a}ld Group}
%   \streetaddress{1 Th{\o}rv{\"a}ld Circle}
%   \city{Hekla}
%   \country{Iceland}}
% \email{larst@affiliation.org}

%%
%% By default, the full list of authors will be used in the page
%% headers. Often, this list is too long, and will overlap
%% other information printed in the page headers. This command allows
%% the author to define a more concise list
%% of authors' names for this purpose.
\renewcommand{\shortauthors}{Hu et al.}

%%
%% The abstract is a short summary of the work to be presented in the
%% article.


\begin{abstract}
  % !TeX root = ../main.tex

Vulnerabilities in open-source operating systems (OSs)  pose substantial security risks to software systems, making their detection crucial. While fuzzing has been an effective vulnerability detection technique in various domains, OS fuzzing (OSF) faces unique challenges due to OS complexity and multi-layered interaction, and has not been comprehensively reviewed. Therefore, this work systematically surveys the state-of-the-art OSF techniques, categorizes them based on the general fuzzing process, and investigates challenges specific to kernel, file system, driver, and hypervisor fuzzing. Finally, future research directions for OSF are discussed. GitHub: \href{https://github.com/pghk13/Survey-OSF}{\textcolor{blue}{https://github.com/pghk13/Survey-OSF}}.

\end{abstract}

%%
%% The code below is generated by the tool at http://dl.acm.org/ccs.cfm.
%% Please copy and paste the code instead of the example below.
%
\begin{CCSXML}
<ccs2012>
 <concept>
  <concept_id>00000000.0000000.0000000</concept_id>
  <concept_desc>Do Not Use This Code, Generate the Correct Terms for Your Paper</concept_desc>
  <concept_significance>500</concept_significance>
 </concept>
 <concept>
  <concept_id>00000000.00000000.00000000</concept_id>
  <concept_desc>Do Not Use This Code, Generate the Correct Terms for Your Paper</concept_desc>
  <concept_significance>300</concept_significance>
 </concept>
 <concept>
  <concept_id>00000000.00000000.00000000</concept_id>
  <concept_desc>Do Not Use This Code, Generate the Correct Terms for Your Paper</concept_desc>
  <concept_significance>100</concept_significance>
 </concept>
 <concept>
  <concept_id>00000000.00000000.00000000</concept_id>
  <concept_desc>Do Not Use This Code, Generate the Correct Terms for Your Paper</concept_desc>
  <concept_significance>100</concept_significance>
 </concept>
</ccs2012>
\end{CCSXML}

\ccsdesc[500]{Security and privacy~Systems security}
\ccsdesc[500]{Software and its engineering~Software testing and debugging}


%%
%% Keywords. The author(s) should pick words that accurately describe
%% the work being presented. Separate the keywords with commas.
\keywords{Operating System Fuzzing, Kernel Fuzzing, File System Fuzzing, Driver Fuzzing, Hypervisor Fuzzing}

% \received{20 February 2007}
% \received[revised]{12 March 2009}
% \received[accepted]{5 June 2009}

%%
%% This command processes the author and affiliation and title
%% information and builds the first part of the formatted document.
\maketitle

\section{Introduction}
\label{sec:intro}
%
The development of Generative AI (GenAI) and Large Language Models (LLMs) signifies an advancement in artificial intelligence, distinguished by its capability to generate diverse content, including texts, images, and code.
This capability has brought GenAI tools into the spotlight, facilitating their integration into daily life to address various pertinent issues and tasks.
Given their utility in tasks such as data analysis and content generation, GenAI and LLMs are actively explored for their potential in more complex applications. 
GenAI has garnered significant attention in the field of cyber security, with recent studies underscoring its potential to enhance security measures, simulate attacks for training and testing, and refine threat detection systems through advanced data analytics.
Examples include studies conducted by~\citet{9105926},~\citet{10198233}, and~\citet{hassanin2024comprehensiveoverviewlargelanguage}, all of which focus on the potential advantages that GenAI could offer as a tool to automate various complex security tasks.

Internet of Things (IoT) is increasingly recognized as a critical area requiring detailed attention and innovative approaches, as IoT devices become more integrated into daily life and industrial systems.
As IoT devices are heterogeneous in nature, the security of these devices requires specialized knowledge and expertise.
GenAI has the potential to enhance existing methods or develop new approaches for IoT security, thereby reducing the need for specialist knowledge to implement advanced security solutions.
Consequently, GenAI represents a promising tool for future IoT security research, which could improve both the security and usability of IoT systems.
In the coming years, significant research is anticipated to be conducted on the use of GenAI to improve IoT security.

This paper presents a comprehensive survey of current state-of-the-art work on the application of GenAI to IoT security.
We begin by providing a foundational understanding of IoT systems along with the core principles of Generative AI (GenAI), with a primary focus on LLMs.
As part of our analysis of the current use of GenAI in enhancing IoT security, we explore the application of each model.
Subsequently, we analyze potential further applications, identifying areas where GenAI could be beneficial with three case studies.
Our evaluation is articulated through the use of the MITRE ATT\&CK Mitigation framework for Industrial Control Systems (ICS).
%
\subsection{Internet of Things (IoT)}
%
IoT is a transformative concept in connectivity, where an extensive network enables devices ranging from household appliances to medical equipment to connect directly to the Internet, facilitating seamless data exchange without human intervention.
This innovation has broad applications in smart homes, healthcare, transportation and urban development, significantly improving operational efficiency~\citet{kimani2019cyber}.
~\citet{alwahedi2024machine, chui2023survey} present a similar perspective, who describe the IoT as a network that connects physical objects through embedded sensors and software.
This configuration not only facilitates the exchange of real-time data, but also transforms physical data into digital information, providing a comprehensive means of managing diverse systems.
The framework underscores the ability of the IoT to digitalize physical entities, fostering an intelligent and interconnected environment.

As described in a study by~\citet{hassija2019survey}, the IoT ecosystem consists of four essential layers: the foundational layer that uses sensors and actuators for data collection, followed by a communication layer that transmits the data.
The middleware layer then bridges the data flow between the network and application layers, allowing processing and integration.
The final layer hosts various IoT applications, such as smart grids and smart factories, demonstrating the structured and integrated approach of IoT systems in various sectors.
%
\subsection{IoT Security Challenges}
%
The potential vulnerabilities of IoT devices pose a significant security threat to IoT ecosystems.
Their interconnected nature exposes them to a variety of cyber threats, data breaches, and privacy violations~\citet{hassija2019survey}.
Insufficient security updates, inadequate security measures, and difficulties in managing dynamic device configurations are among the most common security issues. 
These vulnerabilities mainly consist of communication vulnerabilities, operating system vulnerabilities, and software vulnerabilities. 
There is an ongoing research effort to enhance the overall security of IoT to effectively address these issues.

The diverse application of IoT devices, from home automation to medical systems, makes them an attractive target for malicious activity.
Therefore, it is imperative to implement protective measures such as authentication protocols, intrusion detection systems, and machine learning algorithms to fortify these networks against potential threats.
Various methods have been employed to address these issues, including deep learning~\citet{9060970} and blockchain technology~\citet{10.1145/3320154.3320163}.
%
\subsection{Generative AI and Large Language Models}
%
In the context of GenAI development, LLMs could be viewed as a breakthrough in AI innovation due to their ability to generate, classify and reason based on the datasets with which they are trained~\citet{jo2023promise}.
Through advancement in algorithms and computational power, GenAI has become increasingly important in a wide range of domains, including cyber security.
With the capability to generate novel data instances from learned patterns, this technology offers a revolutionary approach to data analysis and simulation, demonstrating its potential for transformative applications in the digital world.
LLMs, such as ChatGPT~\citet{openai2024gpt4} and Gemini~\citet{10113601}, represent a significant advancement in GenAI, particularly in the processing and generation of natural language.
These models have evolved to understand context, generate coherent responses, and even detect anomalies in text, making them invaluable tools that extend beyond simple communication.
LLMs demonstrate the increasing sophistication of AI's ability to handle complex, nuanced tasks, mirroring human-like understanding and interaction with large volumes of data.

Digital defense strategies have been transformed by the integration of GenAI, specifically LLMs, into cyber security.
LLMs are well-positioned to enhance security for interconnected digital systems, including IoT.
The use of these technologies for security purposes is gaining traction, indicating a promising direction for addressing security threats.
The emerging field of GenAI, particularly through the lens of LLMs, represents not only a technological advancement but also a transformative force in cyber security.
These AI models generate realistic simulations and learn complex patterns providing significant benefits.

% !TeX root = ../main.tex
%\newpage

\section{Collection Strategy and Result}\label{Section2}

% To systematically review Operating System Fuzzing (OSF) and its related research, \cite{garousi2016systematic} suggest that both a search strategy and result analysis are necessary to assess the quality of the literature and research trends. In this section, we design a search strategy that involves determining how to find relevant literature, assess its quality, and keep updated with the latest publications. In results analysis, we conduct an assessment of the research trends in OSF over the past decade and examine the shifts in research specifics in recent years.

Following \cite{garousi2016systematic}, we use a scientific and effective collection strategy (see Section~\ref{Collection Methodology}) and present~a detailed analysis of the collection result (see Section~\ref{Results Analysis}) to systematically review OSF. 

% We introduce our collection methodology in \S \ref{Collection Methodology} and present the statistics and analysis of the collection results in \S \ref{Results Analysis}.



\subsection{Collection Strategy}\label{Collection Methodology}

Figure~\ref{img:collection_methodology} shows our strategy to collect relevant works, assess their quality, and keep updated with~the latest publications. We both review scientific literature and collect open-source tools for OSF. 

\subsubsection{Scientific Literature Review}

% To conduct a comprehensive review of OSF, we established a repository specifically for the accumulation and categorization of relevant literature. This initiative is aimed at facilitating a systematic review of publications within the OSF field. In addition, a closed-loop search strategy for literature search was designed and executed, ensuring a methodical approach in the assessment of relevant works. The temporal scope of our survey was set to encompass a decade, ranging from January 2013 through December 2024 (i.e. at least 10 years), thereby capturing the most recent and pertinent contributions to the OSF. The principal steps are outlined below:

%We establish a repository of scientific literature for comprehensive review of operating system fuzzing, ensuring a methodical approach in the assessment of relevant publications. We set the temporal scope of our survey to encompass a decade, ranging from January 2013 to December 2024 (i.e. at least 10 years). The detailed steps are as follows:

%We establish a repository of scientific literature for comprehensive review of operating system fuzzing, ensuring a methodical approach in the assessment of relevant publications. 
We set the temporal scope of our survey to cover the period~from the earliest relevant papers or tools in this field to the present, ranging from January 2015~to~August 2024 (\ie around 10 years). The detailed steps are as follows.

% \begin{figure}[!t]
%   \centering
%   \includegraphics[width=0.85\linewidth]{img/Statistic1.png}
%   \caption{Step 1 depicts the count of \textbf{selected papers}. Step 2 demonstrates the refinement and expansion of the paper count through \textbf{filtering and snowballing}. Step 3 adds \textbf{open-source tools} during the paper review.}
%   \label{statistic1}
% \end{figure}

\begin{figure}[!t]
  \includegraphics[width=0.9\linewidth]{img/methodology.pdf}
  % \vspace{-5pt}
  \caption{Our Collection Strategy}
  \label{img:collection_methodology}
\end{figure}

\textbf{Database Search.} 
% Prior to embarking on a literature survey, it is customary to use "OS Fuzzing" as the keyword to select appropriate papers. However, employing "OS Fuzzing" as the sole keyword failed to fully capture all existing literature on open-source OSF. As outlined in the "Introduction" section, it became evident that the keywords should encompass the primary test subjects within open-source operating systems, including: "Kernel/File system/Driver/Hypervisor Fuzzing," "OS Fuzzing," "Fuzzing Survey," and "open-source/Linux/Android/FreeBSD/OpenBSD Fuzzing." Therefore, all keywords were indexed in the well-known computer science databases such as ACM Digital Library, IEEE Xplore, DBLP, and Semantic Scholar. Fig.\ref{statistic1} illustrates the variation in the number of papers across the three stages of the search strategy execution. As shown in Fig.\ref{statistic1}, the final tally from step 1 revealed a total of 46 papers, indicating a sufficient corpus to support this survey effort. It is noteworthy that this stage only provided a preliminary confirmation. Consequently, there may still exist some issues, such as: 1) Some papers may not actually fall within the scope of the survey, leading to issues of paper irrelevance; and 2) there exists a relationship of inclusion and overlap among keywords, resulting in redundancy in the retrieved literature, where multiple keywords may point to the same paper, thereby causing issues with the completeness of the literature search.
This step aims to find the potential relevant papers by searching electronic databases. Specifically, we select ACM Digital Library, IEEE~Xplore, DBLP and Semantic Scholar~as our databases, which are popular bibliography databases containing a comprehensive list of research venues in computer science. 
Initially, employing ``Operating System Fuzzing'' as the sole keyword fails to fully capture the existing literature about OSF. Keywords need to focus on the application~of fuzzing to operating systems (\eg Linux, Android, FreeBSD, \etc).  In addition, keywords should cover the main tested objects in open-source operating systems, such as kernel, file system, driver, \etc We optimize the search keywords in an iterative manner for the purpose of collecting as many related papers as possible. Our final search keywords are reported as follows. Moreover, our search targets titles, abstracts, and keywords of the papers, since these parts often convey the theme of a paper. Finally, we obtain a total of \todo{56} candidate papers during \emph{database search}. 

\begin{tcolorbox}[size=fbox, opacityfill=0.15]
\small
  (``Linux'' OR ``Android'' OR ``FreeBSD'' OR ``OpenBSD'' OR ``Zephyr'' OR ``Open-source Operating System'')\\
 AND  (``Kernel'' OR ``File system'' OR ``Driver'' OR ``Hypervisor'')  AND  (``Fuzzing'' OR ``OSF'')
\end{tcolorbox}

% Note that, some papers may not actually fall within the scope of the survey, leading to issues of paper irrelevance. In addition, keyword coverage may still be insufficient, resulting in the omission of some related work. Next steps are needed to assess the quality of these papers and update the literature repository.

\textbf{Paper Filtering.} 
% To ensure the relevance and completeness of the literature, it is necessary to more precisely filter papers from the repository that fall within the scope of OSF and to continually collect the cited papers as exhaustively as possible. Therefore, the step 2 is divided into filtering and snowballing: 1) Filtering. One is that the venue of the paper originates from \textbf{top conferences or journals} in the computer science field to ensure the literature's high quality, such as USENIX Security Symposium (USEC), the ACM Conference on Computer and Communications Security (CCS), ISOC Network and Distributed System Security Symposium (NDSS), the IEEE Symposium on Security and Privacy (S\&P), \etc 
We perform a manual assessment on the \todo{56} candidate papers obtained from~our \emph{database search} to ensure their relevance and quality. Specifically, to determine whether each~candidate paper is relevant to OSF and has high quality, we analyze the abstracts and introductions of these papers, following the inclusion and exclusion criteria formulated as follows.

\begin{itemize}[leftmargin=*]
  %\setlength{\itemindent}{-6.5mm}
  \item \textbf{Inclusion Criteria}. \textbf{IC1}: papers that introduce the process of OSF; and \textbf{IC2}: papers that propose a technique of OSF.
  \item \textbf{Exclusion Criteria}. \textbf{EC1}: survey papers or summary papers; \textbf{EC2}: papers that do not target fuzzing; \textbf{EC3}: papers that do not focus on fuzzing operating system and its components; and~\textbf{EC4}: papers that have not been published in top-tier conferences or journals.
\end{itemize}

Specifically, for \textbf{EC1}, such survey papers are discussed in Section~\ref{Section3-1} for a comparison with~our survey; % for \textbf{EC3}, papers that do not target testing open-source operating system or its kernel, file system, driver and hypervisor are excluded; 
and for \textbf{EC4}, we discuss these top-tier conferences and journals in Section~\ref{Results Analysis}. Through our manual assessment, we remove \todo{13} papers, resulting in \todo{43} papers.

\textbf{Backward \& Forward Snowballing.} To reduce the risk of missing relevant papers, we perform both backward and forward snowballing \cite{wohlin2014guidelines} on the \todo{43} papers. In backward snowballing,~we~check the references in these papers to obtain candidate papers, while in forward snowballing, we use Google Scholar to locate candidate papers that cite these papers. For these candidate papers obtained by snowballing, we also apply the same inclusion and exclusion criteria to identify relevant papers. Finally, we add \todo{15} new relevant papers, resulting in a final set of \todo{58} papers.

\textbf{Full-Text Analysis.} We download all the resulting \todo{58} papers, and conduct a full-text analysis~to
%We further analysis the Introduction, Conclusion and other parts of the papers to determine whether a center paper is relevant with OSF. Specifically, we need to 
 identify the fuzzing target (\ie the operating system and its components) and the proposed fuzzing technique. After reading all these papers, we classify them to form our survey (see Section~\ref{Section4} and \ref{Section5}). 


\subsubsection{Open-Source Tools Collection} Some open-source tools of OSF have not been published~in~academic papers, and these tools should not be ignored. Therefore, we use the same search keywords to collect open-source tools whose stars are more than 400 stars on GitHub. We eliminate tools that have been published in academic papers, and select \todo{4} additional open-source tools.

% Through the above steps, it is confident that the collection methodology devised in this paper maximally covers OSF papers and open-source tools, with the next subsection presenting statistical results.

\begin{figure*}[!t]
  \centering
  \subfigure[Distribution across Publication Venues]{
    \includegraphics[width=0.42\linewidth]{fig/publication_venues.pdf}\label{fig:publication_venues}}
  \hspace{20pt}
  \subfigure[Distribution Across Publication Years and Layers]{
      \includegraphics[width=0.428\textwidth]{fig/publication_years.pdf}\label{fig:publication_years}}
  \caption{The Analysis Results of Paper Collection}
\end{figure*}


\subsection{Collection Result Analysis}\label{Results Analysis}

We analyze the collected papers from three perspectives, \ie the publication venues, the publication years, and the target OS layers. 

\textbf{Publication Venues.} Figure~\ref{fig:publication_venues} shows the distribution of all the papers across the publication venues. The \todo{58} papers are published across 17 top-tier venues in four domains, \ie security,~software engineering, computer architecture, and computer storage systems. Specifically, (\romannumeral1)~most~of~the papers, up to \todo{79\%}, are published in security venues such as \emph{USENIX Security Symposium}, \emph{ACM~Conference on Computer and Communications Security (CCS)}, \emph{IEEE Symposium on Security and Privacy (S\&P)}, and \emph{Network and Distributed System Security Symposium (NDSS)}; (\romannumeral2)~\todo{12\%} of the papers are published in software engineering venues such as \emph{ACM Symposium on Operating Systems Principles (SOSP)}, \emph{International Symposium on Software Testing and Analysis (ISSTA)}, and \emph{International Conference on Software Engineering (ICSE)}; (\romannumeral3)~there are \todo{4} papers, accounting for \todo{7\%}, published in computer architecture venues, one each in \emph{European Conference on Computer Systems (EuroSys)}, \emph{USENIX Annual Technical Conference (ATC)}, \emph{ACM Transactions on Embedded Computing Systems (TECS)}, and \emph{IEEE Transactions on Computer-Aided Design of Integrated Circuits and Systems (TCAD)}; and (\romannumeral4)~since operating system is related to storage system, there is \todo{1} paper published in computer storage system venues (\ie \emph{ACM Transactions on Storage (TOS)}). It can be conclude that~the~application of fuzzing techniques to operating systems spans multiple fields of computer science.

%{fig:publication_years}, we can see that the number of the papers related to operating system fuzzing shows a general ascending trend, from 2013 to 2023. It is evident that interest in OSF has been escalating year by year, which indicates heightened attention in leveraging fuzzing to uncover deeper security vulnerabilities on open-source operating system. In addition, we can also get from the figure that most of these papers focus on the fuzzing of kernel. Particularly, the AFL \cite{AFL} fuzzing framework, launched in 2013, caused a significant stir in kernel testing. The Syzkaller \cite{Syzkaller} fuzzing framework proposed by Google offers researchers a fundamental kernel fuzzing engine. Consequently, researches focused on kernel fuzzing has sprung up since the kernel boundary is broader and related to other layers. Furthermore, the advent of large language models (LLM) has enhanced software testing due to their advanced natural language processing capabilities \cite{yang2023kernelgpt}. Since 2023, notable contributions in the domain of ``LLM for Kernel Fuzzing'' have emerged. However, using kernel fuzzing interfaces (\ie between user and kernel space) fails to uncover deep bugs in other operating system layers (\eg verification chain checks in driver). Thus, an increasing number of researchers realized the necessity of testing other operating system layers such as the file system, drivers, and hypervisors. 


\textbf{Publication Years and Target OS Layers.}  Figure~\ref{fig:publication_years} presents the number of papers published in each year as well as the distribution across the target OS layers in each year. Overall, the~number of OSF papers shows a general ascending trend from 2015 to 2024. It is evident that interest~in~OSF has been escalating year by year, which indicates increased attention in leveraging fuzzing~to~uncover deeper security vulnerabilities in open-source OS. Notice that since the papers from 2024 have not been fully surveyed yet, Figure~\ref{fig:publication_years} only shows the number of OSF publications till~August~2024. Moreover, most papers focus on the fuzzing of kernel. Particularly, Google’s fuzzing framework Syzkaller~\cite{Syzkaller} launched in 2015. It provides researchers with a foundational kernel fuzzing engine, and has made a substantial impact on kernel fuzzing. However, using kernel fuzzing interfaces~(\ie between user and kernel space) fails to uncover deep security vulnerabilities in other OS layers (\eg verification chain checks in driver). Therefore, an increasing number of papers realized the necessity of fuzzing other OS layers (\ie file systems, drivers, and hypervisors) since 2020.

%Particularly, the AFL \cite{AFL} fuzzing framework, launched in 2013, caused a significant stir in kernel testing. The Syzkaller \cite{Syzkaller} fuzzing framework proposed by Google offers researchers a fundamental kernel fuzzing engine.

%Consequently, researches focused on kernel fuzzing has sprung up since the kernel boundary is broader and related to other layers. Furthermore, the advent of large language models (LLM) has enhanced software testing due to their advanced natural language processing capabilities \cite{yang2023kernelgpt}. Since 2023, notable contributions in the domain of ``LLM for Kernel Fuzzing'' have emerged.

%\newpage

% !TeX root = ../main.tex

\section{Overview of OSF}\label{Section3}
%By analyzing the limitations of existing fuzzing surveys, we propose a new classification dimension for a systematic review of OSF. Following this classification, we introduction the general workflow of OSF from the initial test case generation to the final bug analysis, and discuss the influence of the inherent complexity of operating systems on the fuzzing process.

After analyzing existing fuzzing surveys, we employ PUT-based classification for a systematic~review of OSF (Section~\ref{Section3-1}). Following this classification, we introduce the main tasks of the four OS layer fuzzing (Section~\ref{Section3-2}), and provide a high-level overview of the general workflow of OSF (Section~\ref{Section3-3}).

\subsection{Classification Dimension} \label{Section3-1}

%The existing classification method for fuzzing surveys is based on four dimensions: 1) the amount of information the fuzzer requires or uses (black-, white-, and gray-box), 2) the method of seed update (mutation- and generation-based), 3) research gaps, and 4) testing objects. The above taxonomy serves most fuzzing-related literature surveys, but only the last dimension applies to the open source OS fuzzing survey in this paper. The main reasons are explained below:

% The existing classification method for fuzzing surveys is based on four dimensions: 1) the amount of information the fuzzer requires or uses (black-, white-, and gray-box), 2) the method of seed update (mutation- and generation-based), 3) research gaps, and 4) PUT-based fuzzing method.
% \todo{The first three classifications typically apply to the study of fuzzing techniques themselves, while only the last one is relevant to the fuzzing of open-source OS discussed in this paper.} The main reasons are explained below:

% \begin{itemize}
%   \item [1)] 
%   The first dimension focused on the taxonomy of fuzzer concerning black-, white-, and grey-box with an insight toward the every stage of fuzzing techniques \cite{Mans2018TheAS,Li2018FuzzingAS,Liang2018FuzzingSO,Godefroid2020,bohme2020fuzzing}. This type of classification is useful for researchers to understand traditional fuzzing techniques, as there are significant distinctions between fuzzing approaches for open-source and closed-source applications. \todo{However, for the context of this paper's discussion on open-source OS fuzzing, we focus solely on how to conduct fuzzing trial on open-source OS, which are more complex than traditional software, and do not consider closed-source OS or their fuzzing methods.}
%   \item [2)]
%   The second dimension based on the seed updating approach can be divided in mutation- and generation-based fuzzer \cite{Saavedra2019ARO}. This taxonomy concentrate on the strengths and limitations of all the cutting edge techniques for seed updating approach such as random-oriented \cite{schumilo2020hyper}, grammar-aware \cite{schumilo2021nyx, myung2022mundofuzz}, semantic-aware \cite{pan2021V-shuttle, cesarano2023iris, bulekov2022morphuzz} and \etc The selection of these approaches is driven by application scenarios and are constructed to serve either mutation-based or generation-based fuzzer. Unfortunately, while these methods provide an introduction to the field with an idea of what seed updating methods are available and how to accurately update the seeds, they lack discussions on essential fuzzing components such as seed selection, seed trim, feedback, \etc As a result, they are not suitable for a systematic review of the complete OSF workflow.
%   \item [3)]
%   The third dimension provides a comprehensive review of the potential research challenges introduced by integrating other specific techniques into fuzzing \cite{Wang2019ASR, Zhang2018SurveyOD, Wang2020SoKTP, Zhu2022FuzzingAS, Mallissery2023DemystifyTF}. This type of taxonomy is indeed useful in analyzing the challenges of a particular technique, for example, Mallissery and Wu \cite{Mallissery2023DemystifyTF} focus on reviewing the research gaps of program analysis in fuzzing and tries to give the direction of feasibility of bridging these gaps. Since this type of taxonomy focuses on a comprehensive summary of a particular technique, it still does not provide a comprehensive review of all the technical details and difficulties involved in executing OSF.
%   \item [4)]
%   The last classification dimension categorizes fuzzing methods based on different modules or software layers within the PUT \cite{Yun2022FuzzingOE, Eisele2022EmbeddedFA}, for example, Eisele et al. \cite{Eisele2022EmbeddedFA} divide embedded system fuzzing into direct fuzzing, emulation-based fuzzing, and firmware analysis, and review the unique fuzzing technique in all state-of-the-art fuzzers for embedded system. This dimension surpasses the above classifications in terms of comprehensiveness considerations, due to the fact that different OS layer fuzzing interfaces inherently determine the overall approach to seed generation, selection, mutation, feedback, and vulnerability monitoring. Therefore, in order to fill the research gap in OSF survey and to comprehensively review the expertise and specifics required by the different OS layers fuzzing, this taxonomy is adopted as the guiding criterion for categorization in this paper.
% \end{itemize}

Existing fuzzing surveys classify the literature by one of the four dimensions, \ie
1) the amount~of~information the fuzzer requires or uses (black-, white-, and gray-box)~\cite{Mans2018TheAS, Li2018FuzzingAS, Liang2018FuzzingSO, Godefroid2020, bohme2020fuzzing}, 
2) the~strategy employed for seed update (mutation- and generation-based)~\cite{Saavedra2019ARO, schumilo2020hyper, schumilo2021nyx, myung2022mundofuzz, pan2021V-shuttle, cesarano2023iris, bulekov2022morphuzz},~3)~the research gaps of integrating advanced techniques into traditional fuzzing workflows~\cite{Wang2019ASR, Zhang2018SurveyOD, Wang2020SoKTP, Zhu2022FuzzingAS, Mallissery2023DemystifyTF}, and 
4) the PUT \cite{Yun2022FuzzingOE, Eisele2022EmbeddedFA}. 
Here, we adopt the fourth dimension as the guiding taxonomy, as the first three dimensions primarily focus on the general study of fuzzing techniques and lack the specificity required for analyzing OSF. In contrast, the PUT-based classification provides~a~comprehensive framework for systematically reviewing the general workflow of OSF while focusing on the unique challenges posed by the different OS layers (\ie different PUTs).


%\subsection{The Interface of Each OS Layer}
\subsection{OS Layers}\label{Section3-2}

%According to the classification in this paper, we categorize the current OSF into four categories, namely, kernel, driver, file system, and hypervisor. Reviewing the existing state-of-the-art OSFs, the four OS layers have significant differences in their respective fuzzing interfaces, fuzzing techniques, and research challenges. In the following, we offer an overview of the fuzzing interfaces across each OS layer. The details of the fuzzing techniques for each layer are discussed in Section \ref{Section4}. Additionally, the research challenges are comprehensively reviewed in Section \ref{Section5}.

According to the PUT-based classification we employed, OSF can be categorized into four types, \ie kernel fuzzing, file system fuzzing, driver fuzzing, and hypervisor fuzzing. Due to the distinct differences in fuzzing approaches across these four OS layers, it is necessary to outline the primary functions of each OS layer and the primary tasks of their respective fuzzing approaches.


\begin{figure}[!t]
  \centering
  \includegraphics[width=0.60\linewidth]{img/OSFuzzing.png}
  \caption{The Interfaces of Each OS Layer (\protect\circled{1}, \protect\circled{2}, \protect\circled{3} and \protect\circled{4} denote the fuzzing interfaces for Kernel, Driver, File System and Hypervisor respectively; and \protect\circled{1} is also the fuzzing interface for Driver and File System).}
  \label{OSFuzzing}
\end{figure}


%\textbf{Kernel}. Kernel fuzzing works by running executable entities on the OS that can transition into kernel space, and the logical sequence of syscalls \todo{(also known as system calls)} within program will determine which part of the kernel code branch is executed. If a crash occurs during execution, the crash needs to be analyzed to determine whether it represents an actual vulnerability. As shown in the Figure \ref{OSFuzzing}, a test case is a sequence of syscalls with sequential or logical relationships, and these collections can take various forms, such as executable programs or command-line scripts. If it is an executable program, it includes program variables and general functions beyond syscalls, but if it is a command-line script, it usually contains only syscalls and parameter values. When the first test case is ready, it is mounted in user space under the control of the fuzzing engine to execute the program logic of the test case. During this process, the operating system switches from the user space to the kernel space \circled{1} and monitors the behavior of the kernel by triggering code branches of specific modules. In Figure \ref{OSFuzzing}, the kernel space illustrates the activated code branches, where circles represent basic code blocks. The darker sequences indicate the branches activated by the test case, while lighter circles denote the yet-to-be-uncovered code blocks. To uncover potential vulnerabilities, fuzzing aims to generate effective syscall sequences that achieve comprehensive coverage of the kernel's code branches.

\textbf{Kernel}. Kernel is one of the most critical systems in an OS because it manages essential resources such as processes and memory for the entire system and provides a unified programming interface for user-space programs to interact with hardware resources. All other system types rely~on~and operate on top of the kernel. Consequently, vulnerabilities in the kernel can be maliciously exploited, potentially causing significant damage to the systems running on it. As illustrated in Figure \ref{OSFuzzing}, to perform kernel fuzzing, a fuzzer triggers the kernel's code paths by switching from user space~to kernel space through syscall \circled{1} (a.k.a. system call) \cite{Koopman1997ComparingOS}. Therefore, the core task of a kernel fuzzer is to generate various test cases (referred to as seeds) by combining syscalls~provided~by~the~kernel to continuously trigger the kernel's code paths. For example, the darker sequences in Figure \ref{OSFuzzing} denote branches covered by test cases, while the lighter circles denote code blocks yet to be covered.

%\textbf{Driver}. As kernel subsystem, device driver plays a crucial role in maintaining the security of the OS kernel \cite{Chou2001AnES, Palix2011FaultsIL, Stoep2018AndroidSecurity}. A primary reason is the breadth and diversity of driver implementations, which hampers scalable and coherent security analysis. Worse, they expose a richer attack surface than the kernel or other kernel subsystems: drivers expose peripheral attack surfaces in addition to system call attack surfaces. As a result, vulnerabilities in device drivers are often discovered and exploited by adversaries through both attack surfaces in local and remote attack scenarios \cite{Beniamini2017Part1, Beniamini2017Part2, Chang2017, Davis2011, NohlLell2014}. As Figure \ref{OSFuzzing} demonstrates, there are two attack surfaces in driver: syscalls \circled{1} that interact with the user-space, and peripheral interfaces \circled{2} that directly communicate with devices. The syscall-based interface aligns with kernel fuzzing, but driver fuzzing specifically focuses on mutating syscalls that manipulate device files, \eg read(), write(), seek(), ioctl() and \etc The peripheral interface can be implemented through either I/O interception or device configuration. I/O interception involves capturing access to I/O objects (\ie DMA, MMIO, and Port I/O) to mutate I/O data. This mutated data is then redirected to the target object to monitor the driver's behavior. In contrast, device configuration simulates a peripheral device's behavior, injecting the simulated data into the I/O channel to discover vulnerabilities in a specific device driver. The breadth and diversity of device drivers complicate the discovery of potential vulnerabilities. Consequently, current research on driver fuzzing exhibits significant technical variability. This requires careful consideration of various test interfaces and the functional characteristics of devices to effectively tailor fuzzing approaches.

\textbf{File System}. As a core system service within an OS, file system is essential for tasks such~as~reading, writing, managing, and scheduling files, as well as ensuring data consistency during system crashes. Most file systems, like ext4 \cite{Cao2007Ext4TN}, XFS \cite{XFS2018}, Btrfs \cite{Rodeh2013BTRFSTL} and F2FS \cite{Lee2015F2FSAN}, operate within~the~OS kernel. Therefore, approaches used in kernel fuzzing can be adapted for file system fuzzing. However, approaches based purely on syscalls often yield numerous invalid results because the system state is predominantly influenced by metadata in file system operations. In contrast, operations~on regular file data through syscalls like \texttt{read()} and \texttt{write()} contribute little to identifying file system vulnerabilities. Hence, an effective file system fuzzer typically combines sequences of file operation-related syscalls \circled{1} with images \circled{3} where metadata has been changed. As shown in Figure \ref{OSFuzzing}, the altered disk image is mounted to the file system's partition using privileged commands \cite{schumilo2017kafl, Hydra2020finding}, becoming the new target for fuzzing. Kernel-provided syscalls are then used to conduct read, write, management, and scheduling operations, facilitating a comprehensive fuzzing of the file system.

%\textbf{File System}.
%Since the primary functions of a file system involve reading, writing, managing, and scheduling files, it is challenging to continuously trigger file system code paths through random fuzzing within the broad functional boundaries of the kernel. Notably, the file system's critical resource characteristics make it more susceptible to unique data race vulnerabilities, necessitating the careful design of specialized fuzzing techniques aimed specifically at particular file systems and their unique bugs. In existing research on file system fuzzing, the approach remains to initiate file system actions by mutating file operation behaviors and data through syscalls \circled{1}. It is crucial for researchers to ensure that these system calls actually reach the targeted file system's code areas. Additionally, another input has emerged for focused fuzz testing of File Systems: file system images \cite{schumilo2017kafl, kim2020finding}. As illustrated in Fig.\ref{OSFuzzing}, an image file \circled{3} that includes the complete structure of a file system is simulated in user space. This file is recognized by the operating system as a volume partition and is used in conjunction with syscalls to perform concentrated file system fuzzing. The process involves initially mutating the metadata, directory structures, and actual data of the image file. Subsequently, various IOCTLS (Input/Output Control System Calls) are used to programmatically mount it to the file system’s partition volume as the new object of testing. Finally, exposed kernel system calls are utilized to perform operations such as reading, writing, managing, and scheduling on this volume to achieve the goals of concentrated file system fuzzing. Consequently, although the file system, as a subsystem of the kernel, can have its code areas partially triggered through kernel fuzz testing, its unique characteristics demand more than just system calls for comprehensive testing. Therefore, to effectively discover both potential and specific bugs, it is necessary to employ specialized interfaces related to file operations for concentrated testing.

\textbf{Driver}. Drivers are responsible for communication and control between the user/kernel and hardware devices. They act as the bridge between hardware and OS, managing tasks such~as~device initialization, data transfer, and interrupt handling, ensuring that user-space applications~can~interact with underlying hardware \cite{Chou2001AnES, Palix2011FaultsIL, Stoep2018AndroidSecurity}. In driver fuzzing, the primary objective is to uncover vulnerabilities in device drivers across the initialization, data communication, and control stages. Since drivers can receive operation requests from both user space and hardware devices, they expose a broader attack surface compared to the kernel or other kernel subsystems \cite{Beniamini2017Part1, Beniamini2017Part2, Chang2017, Davis2011, NohlLell2014}. As illustrated in Figure \ref{OSFuzzing}, the two main attack surfaces for driver fuzzing are syscalls \circled{1} and peripheral interfaces \circled{2}. Unlike kernel fuzzers, syscall-based driver fuzzers focus more on syscalls that directly operate on device files, \eg \texttt{read()}, \texttt{write()}, \texttt{seek()}, \texttt{ioctl()}, \etc

The peripheral interface can be exploited in two ways, \ie I/O interception and device configuration. I/O interception involves intercepting access to I/O objects (\eg DMA, MMIO, and Port I/O) to mutate I/O data, which is then redirected to the target to observe the driver's behavior.~Device configuration, on the other hand, simulates peripheral device behavior and injects the simulated data into the I/O channel. In addition, it is worth noting that drivers constitute the largest codebase among kernel subsystems, and exhibit significant variability due to implementations by different vendors. As a result, driver fuzzers often face larger challenges in achieving high coverage, generating diverse test cases, and ensuring fuzzing effectiveness given these characteristics.

%\textbf{Hypervisor}. In environments with highly heterogeneous hardware, OS inherently lack the capability to manage and schedule heterogeneous computing resources, such as those used in railways, avionics, and automotive industries \cite{Cinque2021VirtualizingMS}. To enable multiple operating systems to run in parallel within a heterogeneous resource environment while maintaining security isolation between them \cite{RTCA_DO_178C, ISO_26262_2011}, Hypervisors utilize virtualization technology to divide hardware resources (CPU, memory, disk space, \etc) into several virtual partitions \cite{Popek1974FormalRF, Cinque2021VirtualizingMS, Cilardo2021VirtualizationOM}, supporting the deployment and operation of various OSs. In such scenarios, the Hypervisor and OS collectively form what is termed a generalized operating system. Unlike traditional operating systems, Hypervisors play a critical role in resource management and security isolation (\eg memory corruptions might lead to virtual machine escapes), where such issues are nearly impossible to discover directly in the kernel, drivers, or file systems, necessitating focused testing of the Hypervisor to uncover these vulnerabilities. As illustrated in Figure \ref{OSFuzzing}, interaction with the Hypervisor typically includes I/O channels or triggering hypercalls from the guest OS, which execute hypercall handler programs within the Hypervisor \circled{4}. I/O channels are more extensively used in fully virtualized environments because this technology provides complete simulation of hardware, leading to more frequent I/O communication. In the fuzzing of fully virtualized hypervisor, three types of I/O interaction interfaces provided by virtual devices to the guest machines can be utilized (Port I/O instructions for in and out, MMIO for device initialization, and DMA for complex and massive data communications). These interfaces can be actively operated by fuzzers through specific APIs or intercepted and injected with mutated data through hooking. In para-virtualization technologies, specialized instructions like 'vmcall' are introduced to implement Hypercalls, allowing the OS to communicate directly with the Hypervisor in bypassing the virtual machine. Hence, fuzzers can still fuzz the Hypervisor's code logic through the Hypercall interface. In summary, the Hypervisor plays a crucial role in a generalized operating system, responsible for managing and allocating virtual devices. Therefore, it is essential to conduct focused fuzzing on the Hypervisor to thoroughly uncover vulnerabilities related to heterogeneous resource management and security isolation unique to the generalized operating system.


\textbf{Hypervisor}. In environments with highly heterogeneous hardware, OSs inherently lack~the~capability to manage and schedule heterogeneous computing resources, such as those used in industries like railways, avionics, and automotive systems \cite{Cinque2021VirtualizingMS}. To enable the concurrent execution of multiple OSs while maintaining secure isolation between them in such heterogeneous resource environments \cite{RTCA_DO_178C, ISO_26262_2011}, hypervisors, a.k.a. Virtual Machine Monitors (VMMs), leverage virtualization~technique to partition hardware resources (\eg CPU, memory, and disk space) into multiple virtual partitions \cite{Popek1974FormalRF, Cinque2021VirtualizingMS, Cilardo2021VirtualizationOM}. When a guest OS, which runs within a Virtual Machine (VM), accesses or operates hardware devices, it triggers a VM-exit event that transfers these privileged operations~to~the~hypervisor for hardware emulation. Therefore, the objective of hypervisor fuzzing is to accurately emulate the behavior of the guest OS when accessing and operating these virtualized hardware components, which serves as the primary entry point for implementing hypervisor fuzzing trials. As illustrated in Figure \ref{OSFuzzing}, the main interfaces involved include I/O channels and hypercalls \circled{4}. 

To ease the understanding of how to perform hypervisor fuzzing using these interfaces,~we~introduce the technical background of hypervisors. Technically, hypervisors can be categorized into full virtualization and para-virtualization. In full virtualization, the guest OS is unaware~of~its~virtualized environment. Consequently, when the guest OS attempts to access physical hardware (\eg memory access), the hypervisor intercepts the request. At this point, the VM triggers a trap~and~enters~a~VM-exit state, with the hypervisor assuming a full control of the VM's operations. Subsequently, the hypervisor forwards the memory access request to the Device Emulator, which provides virtualization for Port I/O (input/output port instructions), MMIO (Memory-Mapped I/O for direct memory access), and DMA (for complex and large-scale data transfers). Finally, the Device Emulator returns the virtual memory access interface to the guest OS. In a fully virtualized environment, the I/O channel is more extensively utilized because this technique offers complete hardware emulation. However, this can also lead to more frequent communication overhead.

% These I/O communication operations can be initiated using specific APIs or mutated by hooks intercepting the communication data.

To facilitate and accelerate communication between the guest OS and the hypervisor, modern hypervisors often support hardware-accelerated virtualization technologies. By introducing specialized instructions (such as \texttt{vmcall}) to perform hypercalls, para-virtualization allows OSs to bypass the VM and communicate directly with the hypervisor. Therefore, in addition to the I/O channel, the hypercall interface can also be employed to implement hypervisor fuzzing trials. As a result, it is necessary to conduct specialized fuzzing of hypervisors to uncover vulnerabilities related to resource management and security isolation in generalized OSs.


\subsection{OSF Workflow}\label{Section3-3}

We summarize the general workflow of OSF in Figure~\ref{OSF_WorkFlow}, which consists of three key modules, \ie \textit{input}, \textit{fuzzing engine}, and \textit{running environment}. Each module is described in detail as follows.

\begin{figure}[!t]
  \centering
  \includegraphics[width=0.66\linewidth]{img/OSF_WorkFlw.png}
  \caption{General Workflow of OSF (It has three modules, \ie input, fuzzing engine, and running environment).}
  \label{OSF_WorkFlow}
\end{figure}

\textbf{Input}. There are three types of inputs to be considered in the input module, \ie initial seed,~specification, and target OS. First, the initial seed is the raw material fed to the fuzzing engine, which~is initially deposited in a global corpus (\ie a repository for storing seed candidates) and subsequently used throughout the fuzzing workflow. Second, the specification, which describes the structure and syntax of the seed, is parsed by the fuzzing engine and used to generate new seeds or enhance~the quality of the seeds. It is important to note that the use of the specifications is not applicable~to~all scenarios; it mainly applies to highly structured seeds with documented descriptions of the seed structure \cite{Syzkaller, chen2020koobe, zhao2022statefuzz, sun2021healer, shen2021rtkaller, shen2022tardis, jeong2023segfuzz, xu2020krace, jeong2019razzer}. Third, the target OS requires to be instrumented to collect runtime information for feedback analysis and bug monitoring during fuzzing.

\textbf{Fuzzing Engine}. The main function of the fuzzing engine is to generate new seeds and wrap them into test cases for the input fed to the executor in the running environment. This module~go through a complete closed loop of seed selection from a corpus, seed mutation and/or generation,~and~seed~trim before storing to the corpus. Specifically, the corpus stores only seeds that have~been~verified~as ``high-quality'', where ``high-quality'' seeds can be defined as the seeds~that~have~triggered~bugs~\cite{zou2022syzscope, wu2018fuze,lin2022grebe,shen2021rtkaller}, the initial seed generated based on tailored rules \cite{shen2022drifuzz} or specifications \cite{Syzkaller},~or~the streamlined seeds after seed trim as they trigger new coverage \cite{pailoor2018moonshine, shen2021rtkaller}. Such a consideration stems from the empirical conclusion that \todo{mutations} based on high-quality seeds usually have more opportunities to move the execution of the PUT closer to trigger bugs~\cite{you2017semfuzz, shen2022drifuzz, lin2022grebe, Unicorefuzz2019}.~However, even if the corpus always stores high-quality seeds, it is necessary to consider which~seeds should be prioritized in order to improve the efficiency of \todo{seed mutation}. Therefore, seed selection~is~often adopted to prioritize the seeds in the corpus, and a computation method of seed priority is often devised to ensure that the selected seeds have more chances to reveal new bugs, \eg PageRank~\cite{lin2022grebe}, program analysis \cite{wu2018fuze, chen2022sfuzz}, evolutionary algorithm \cite{shen2022drifuzz, henderson2017vdf, xu2020krace}, empirical strategy \cite{xu2020krace, Syzkaller, you2017semfuzz, zou2022syzscope}, reinforcement learning \cite{wang2021syzvegas}, \etc Then, to continuously generate test cases~to~automate~the whole fuzzing process, \todo{seed mutation and/or generation}  transform and/or generate the seeds in some way and wrap them into test cases that can be executed by the target OS. For example, seed mutation employs parameter-level bit flips or byte-level replacement to generate seeds based on the high-quality seeds in the corpus. Seed generation builds test case models from the specification~to generate new seeds. Finally, to further improve the speed performance of OSF, the generated~seed usually requires to be trimmed to reduce its mutation space as well as its execution time, \eg greedy algorithm~\cite{pailoor2018moonshine} and stepwise filtering \cite{schumilo2020hyper, henderson2017vdf}. The trimmed seeds are then stored into the corpus, and the process is repeated to automatically drive new and high-quality seed generation.

\textbf{Running Environment}. The running environment is responsible for feeding the test cases wrapped by the fuzzing engine into the executor. Unlike traditional fuzzing techniques, OSF~usually relies on full-virtual environment to avoid as much as possible the effects propagated by~OS~crashes. In contrast, traditional fuzzing techniques do not fatally affect the entire fuzzing process even if a crash occurs. Therefore, the running environment of OSF is often designed as full-virtual~environment that can be quickly recovered, allowing the entire system to be quickly restarted to a known, clean state in the event of a crash \cite{shen2022tardis, lin2022grebe, schumilo2021nyx, pan2021V-shuttle, zhao2022semantic, song2020agamotto}. Such a design allows~a~fuzzer to run thousands of test cases continuously and automatically in an isolated environment without worrying about the potential for lasting damage from individual test cases. Subsequent execution results fall into two categories. If the executor does not crash, the feedback will be collected~by the instrumented code, which is used to guide \todo{seed mutation} of the fuzzing engine to steer~the~fuzzing towards uncovered program paths. If the executor triggers a crash, the instrumented code records the information and sends it to the bug analyzer. The bug analyzer identifies and categorizes errors, and generates detailed bug reports. Finally, if the fuzzing process is not yet finished, it should be restored to its pre-crash state, and then move on to explore other states of the target OS.

Although there are some similarities with traditional fuzzing in terms of the coarse-grained~fuzzing process, the technical challenges and implementation details of each step need to take into account the specificity and complexity of OS, which ultimately makes it show a big difference with traditional fuzzing in terms of the fine-grained \todo{details}. Therefore, we will elaborate on the technical details and technical distinctions of each step illustrated in Figure \ref{OSF_WorkFlow} in Section~\ref{Section4} and \ref{Section5}.


% !TeX root = ../main.tex

\section{OSF Steps In Detail} \label{Section4}
% In contrast to traditional fuzzing techniques, each module within OSF framework demonstrates significant distinctions. The discrepancies primarily stem from the system software layer’s role as a bridge between developers or users and the hardware, which has extensive interaction interfaces. However, the communication data of these interfaces differ from those of applications, with a highly nested structure and potential dependencies that can decrease the efficiency of OSF and even make it challenging to effectively trigger the target code paths. Accordingly, section \ref{Section4} will detail the key techniques of the three modules shown in Fig.\ref{OSF_WorkFlow} — Input, Fuzzing Engine, and Running Environment. Additionally, it provides a comprehensive summarization of the differences between various fuzzing techniques and their applicable scenarios.

OSF differs significantly from traditional fuzzing in technical approach, largely due to the OS's intricate modular design, its extensive concurrency, and the complex, diverse interaction interfaces connecting user space and hardware with kernel space. These structural complexities create unique challenges for fuzzing, impacting both its effectiveness and efficiency in practical implementations. Thus, we review the details of the three core modules—input, fuzzing engine, and runtime environment—shown in Figure \ref{OSF_WorkFlow}, and comprehensively compare the applicability scenarios of various fuzzing techniques.

%OSF in approach exhibits distinct technical detail differences from traditional fuzzing. These differences mainly stem from the high complexity of the OS and the diversity of interaction interfaces. These characteristics make fuzzing suffer from the challenges of effectiveness and efficiency in practical implementation. We review \todo{the details of the three modules} - input, fuzzing engine and runtime environment - shown in Figure \ref{OSF_WorkFlow}. In addition, \todo{we comprehensively summarize the differences between various fuzzing techniques and their applicability scenarios.}

% \begin{footnotesize}
%         \begin{longtable}{m{2.0cm}m{1.1cm}m{0.8cm}m{1.2cm}m{1.2cm}m{0.9cm}m{0.9cm}m{1.1cm}m{0.5cm}}
%                 \caption{OSF steps}\label{tab1}                                                                                                                                                                                                                                                                                                      \\

%                 \toprule
%                 \multirow{2.5}{*}{\textbf{Fuzzer}}                         & \multicolumn{3}{c}{\textbf{Input}} & \multicolumn{3}{c}{\textbf{Fuzzing Engine}} & \multicolumn{2}{c}{\textbf{Running Environment}}                                                                                                                                       \\
%                 \cmidrule(lr){2-4} \cmidrule(lr){5-7} \cmidrule(lr){8-9} & \textbf{OL}                  & \textbf{ISG}                  & \textbf{I.}                        & \textbf{SS} & \textbf{ST} & \textbf{U} & \textbf{F} & \textbf{VTs}
%                 \\
%                 \midrule
%                 \endfirsthead


%                 \toprule
%                 \multirow{2.5}{*}{\textbf{Fuzzer}}                         & \multicolumn{3}{c}{\textbf{Input}} & \multicolumn{3}{c}{\textbf{Fuzzing Engine}} & \multicolumn{2}{c}{\textbf{Running Environment}}                                                                                                                                       \\
%                 \cmidrule(lr){2-4} \cmidrule(lr){5-7} \cmidrule(lr){8-9} & \textbf{OL}                  & \textbf{ISG}                  & \textbf{I.}                        & \textbf{SS} & \textbf{ST} & \textbf{U.} & \textbf{F.} & \textbf{VTs}
%                 \\
%                 \midrule
%                 \endhead

%                 \bottomrule
%                 \endfoot


%                 Trinity\cite{Trinity}                                    & Kernel                             & Spec.                                       & -                                                & -                       & -                  & Gen.(1)         & *Code                                                              \\

%                 perf\_fuzzer\cite{Weaver2015perf}                        & Kernel                             & Spec.                                       & -                                                & -                       & -                  & Gen.(1)         & *Code                                                              \\

%                 Syzkaller\cite{Syzkaller}                                & Kernel                             & Spec.'                                      & Static                                           & -                       & Feedback           & Mutation        & *Code                                                              \\

%                 KernelFuzzer\cite{KernelFuzzer2016}                      & Kernel                             & Spec.                                       & Static                                           & -                       & -                  & Gen.(1)         & *Code                                                              \\

%                 DIFUZE\cite{2017DIFUZE}                                  & Driver                             & Spec.'                                      & Static                                           & -                       & -                  & Gen.(2)         & -                 & \fullcirc+\halfcirchor+\halfcircver            \\

%                 VDF\cite{henderson2017vdf}                               & Hypervisor                         & Trace                                       & Static                                           & -                       & Crash              & Mutation        & Code              & \fullcirc+\emptycirc+\halfcirchor              \\

%                 KAFL\cite{schumilo2017kafl}                              & Filesystem                         & -                                           & Static+ Dynamic                                  & -                       & -                  & Mutation        & *Code             & \fullcirc                                      \\

%                 Semfuzz\cite{you2017semfuzz}                             & Kernel                             & PoC                                         & Static                                           & Distance                & -                  & Mutation        & Code              & \fullcirc                                      \\

%                 usb-fuzzer\cite{Syzkaller}                               & Driver                             & Spec.'                                      & Static                                           & -                       & -                  & Mutation        & Code              & \fullcirc                                      \\

%                 Moonshine\cite{pailoor2018moonshine}                     & Kernel                             & Trace                                       & Static                                           & Feedback                & -                  & Mutation        & Code              & \fullcirc+\emptycirc                           \\

%                 FUZE\cite{wu2018fuze}                                    & Kernel                             & PoC                                         & Static                                           & -                       & Distance           & Mutation        & Code              & \fullcirc                                      \\

%                 Schwarz~et~al.\cite{schwarz2018automated}                & Kernel                             & Trace                                       & -                                                & -                       & -                  & Gen.(1)         & *Code             & \emptycirc                                     \\

%                 Chizpurfle\cite{Chizpurfle2019}                          & Kernel                             & Trace                                       & Dynamic                                          & -                       & -                  & Mutation        & Code              & \halfcirchor                                   \\

%                 Razzer\cite{jeong2019razzer}                             & Kernel                             & Spec.'                                      & Static                                           & -                       & -                  & Mutation        & Code              & \emptycirc                                     \\

%                 JANUS\cite{JANUS2019fuzzing}                             & Filesystem                         & -                                           & Static                                           & -                       & -                  & Mutation        & Code+ Custom      & \fullcirc                                      \\

%                 SLAKE\cite{chen2019slake}                                & Kernel                             & PoC                                         & Static                                           & -                       & Distance           & Mutation        & -                 & \fullcirc                                      \\

%                 Shi et al.\cite{shi2019industry}                         & Kernel                             & Spec.'                                      & Static                                           & -                       & -                  & Mutation        & -                 & \fullcirc+\emptycirc+\halfcircver              \\\

%                 Unicorefuzz\cite{Unicorefuzz2019}                        & Kernel                             & Template                                    & Static                                           & -                       & -                  & Mutation        & Code              & \fullcirc                                      \\

%                 KOOBE\cite{chen2020koobe}                                & Kernel                             & PoC                                         & Dynamic                                          & -                       & -                  & Mutation        & Code+ Custom      & \fullcirc                                      \\

%                 Krace\cite{xu2020krace}                                  & Filesystem                         & Spec.'                                      & Static                                           & Minimum                 & -                  & Mutation        & Code+ Thread      & \emptycirc                                     \\

%                 Hyper-CUBE\cite{schumilo2020hyper}                       & Hypervisor                         & Trace                                       & -                                                & -                       & Crash              & -               & -                 & \fullcirc+\emptycirc+\halfcircver+\halfcirchor \\

%                 Hydra\cite{Hydra2020finding}                             & Filesystem                         & -                                           & Static                                           & -                       & Crash              & Mutation        & Code+ Custom      & \fullcirc+\halfcirchor                         \\

%                 USBFuzz\cite{peng2020usbfuzz}                            & Driver                             & Template                                    & Static                                           & -                       & -                  & Mutation        & Code              & \fullcirc                                      \\

%                 Agamotto\cite{song2020agamotto}                          & Driver                             & Trace                                       & Static                                           & -                       & -                  & Mutation        & *Code             & -                                              \\

%                 HFL\cite{kim2020hfl}                                     & Kernel                             & Spec.'                                      & Static                                           & -                       & -                  & Mutation        & Code              & \fullcirc                                      \\

%                 X-AFL\cite{liang2020xafl}                                & Kernel                             & Trace                                       & Static                                           & -                       & -                  & Mutation        & Code              & \fullcirc                                      \\

%                 HEALER\cite{sun2021healer}                               & Kernel                             & Spec.'                                      & Static                                           & -                       & -                  & Gen.+Mut.       & Code              & \fullcirc+\emptycirc+\halfcirchor              \\

%                 Rtkaller\cite{shen2021rtkaller}                          & Kernel                             & Spec.'                                      & Static                                           & -                       & -                  & Mutation        & Code              & \fullcirc+\emptycirc                           \\

%                 NYX\cite{schumilo2021nyx}                                & Hypervisor                         & Spec.                                       & Dynamic                                          & -                       & -                  & Mutation        & Code              & \fullcirc                                      \\

%                 V-Shuttle\cite{pan2021V-shuttle}                         & Hypervisor                         & Trace                                       & Static                                           & -                       & -                  & Mutation        & Code              & \fullcirc+\halfcirchor                         \\

%                 BSOD\cite{maier2021bsod}                                 & Driver                             & Template                                    & Dynamic                                          & -                       & -                  & Gen.+Mut.       & Code              & \fullcirc                                      \\

%                 SyzVegas\cite{wang2021syzvegas}                          & Kernel                             & Spec.'                                      & Static                                           & Feedback                & -                  & Mutation        & Code              & \fullcirc                                      \\

%                 StateFuzz\cite{zhao2022statefuzz}                        & Driver                             & Spec.'                                      & Static                                           & Minimum                 & -                  & Mutation        & Code+ Custom      & \fullcirc                                      \\

%                 GREBE\cite{lin2022grebe}                                 & Kernel                             & PoC                                         & Static                                           & Minimum                 & -                  & Mutation        & Custom            & \fullcirc+\halfcircver                         \\

%                 MundoFuzz\cite{myung2022mundofuzz}                       & Hypervisor                         & Trace                                       & Static                                           & -                       & -                  & Mutation        & Code              & \fullcirc+\halfcirchor                         \\

%                 Morphuzz\cite{bulekov2022morphuzz}                       & Hypervisor                         & Trace                                       & Static                                           & -                       & -                  & Mutation        & Code              & \emptycirc+\halfcirchor                        \\

%                 CONZZER\cite{CONZZER2022context}                         & Filesystem                         & Spec.'                                      & Static                                           & Feedback                & -                  & Mutation        & Code+ Thread      & \emptycirc                                     \\

%                 Sfuzz\cite{chen2022sfuzz}                                & Kernel                             & -                                           & Static                                           & -                       & -                  & Mutation        & *Code             & \fullcirc                                      \\

%                 Dr.Fuzz\cite{zhao2022semantic}                           & Driver                             & Spec.'                                      & Static+ Dynamic                                  & -                       & -                  & Mutation        & *Code+ Custom     & \fullcirc+\halfcircver                         \\

%                 %36
%                 PrintFuzz\cite{ma2022printfuzz}                          & Driver                             & Spec'                                       & Static+ Dynamic                                  & -                       & -                  & Gen.+Mut.       & Code              & \fullcirc+\emptycirc                           \\

%                 DriFuzz\cite{shen2022drifuzz}                            & Driver                             & Trace                                       & Static                                           & -                       & -                  & Gen.+Mut.       & Code              & \fullcirc                                      \\

%                 Demystifying\cite{Hao2022DemystifyingTD}                 & Kernel                             & -                                           & Static                                           & -                       & -                  & -               & -                 & -                                              \\

%                 KSG\cite{sun2022ksg}                                     & Kernel                             & Spec.'                                      & Static                                           & -                       & -                  & -               & -                 & \fullcirc+\emptycirc+\halfcirchor              \\

%                 %43
%                 SyzScope\cite{zou2022syzscope}                           & Kernel                             & PoC                                         & Static                                           & Feedback                & -                  & Mutation        & -                 & \fullcirc                                      \\

%                 Tardis\cite{shen2022tardis}                              & Kernel                             & Spec                                        & Static                                           & -                       & -                  & Mutation        & Code              & \fullcirc+\halfcirchor                         \\

%                 Segfuzz\cite{jeong2023segfuzz}                           & Kernel                             & Spec.'                                      & Static                                           & -                       & -                  & Mutation        & Code+ Thread      & \fullcirc+\emptycirc+\halfcircver              \\

%                 IRIS\cite{cesarano2023iris}                              & Hypervisor                         & IRIS                                        & Static                                           & -                       & -                  & Mutation        & Code              & \fullcirc                                      \\

%                 FUZZNG\cite{bulekov2023FUZZNG}                           & Kernel                             & Spec.                                       & Static                                           & -                       & -                  & Mutation        & Code              & \fullcirc                                      \\

%                 ACTOR\cite{fleischer2023actor}                           & Kernel                             & Trace                                       & Static                                           & -                       & -                  & Gen.(2)         & Code              & \fullcirc+\halfcirchor                         \\

%                 SyzDescribe\cite{hao2023syzdescribe}                     & Driver                             & Spec.'                                      & -                                                & -                       & -                  & -               & -                 & -                                              \\

%                 DEVFUZZ\cite{wu2023devfuzz}                              & Driver                             & Template                                    & Static+ Dynamic                                  & -                       & -                  & Mutation        & *Code+ Custom     & \fullcirc+\halfcirchor+\halfcircver            \\

%                 Syzdirect\cite{tan2023syzdirect}                         & Kernel                             & Spec.'                                      & Static                                           & Distance                & Distance           & Mutation        & Code              & -                                              \\

%                 ReUSB\cite{Jang2023ReUSB}                                & Driver                             & Trace                                       & Dynamic                                          & -                       & -                  & Mutation        & Code              & \fullcirc                                      \\

%                 DDRace\cite{Yuan2023DDRace}                              & Driver                             & Spec.'                                      & Static                                           & Feedback                & -                  & Mutation        & Code+ Thread      & \fullcirc+\emptycirc                           \\

%                 VDGUARD\cite{Liu2023VDGuard}                             & Hypervisor                         & Trace                                       & Static                                           & -                       & -                  & Mutation        & Code              & \fullcirc+\halfcirchor                         \\

%                 Lfuzz\cite{Liu2023LFuzz}                                 & Filesystem                         & -                                           & Static                                           & -                       & -                  & Mutation        & Code              & \fullcirc+\halfcirchor                         \\

%                 KernelGPT\cite{yang2023kernelgpt}                        & Kernel                             & Spec.'                                      & Static                                           & -                       & -                  & Mutation        & *Code             & \fullcirc+\halfcircver                         \\
%         \end{longtable}
%         \begin{flushleft}
%         \justifying
%         \textbf{OL} is the abbreviation of OS Layer. 
%         \textbf{ISG} is the abbreviation of initial seed generation.
%         \textbf{I.} is the abbreviation of instrumentation.
%         \textbf{SS} is the abbreviation of seed selection.
%         \textbf{ST} is the abbreviation of seed trim.
%         \textbf{U.} is the abbreviation of update.
%         \textbf{F.} is the abbreviation of feedback.
%         \textbf{VTs} is the abbreviation of vulnerability types: ``\fullcirc''  means Memroy Violation Bug. ``\emptycirc'' means Concurrency Bug.        ``\halfcirchor'' means Logic Bug. ``\halfcircver'' means Access Violation      Bug.
%         \end{flushleft}
%         \end{footnotesize}

\begin{footnotesize}
        \begin{longtable}{m{1.8cm}m{1.3cm}m{0.6cm}m{1.6cm}m{1.0cm}m{0.8cm}m{0.9cm}m{1.3cm}m{0.5cm}}
        \caption{Overview of Fuzzers Sorted by Publication Year.}\label{tab1} \vspace{-0.3cm}\\
        
        \toprule
        \multirow{2.5}{*}{\textbf{Fuzzer}} & \multicolumn{3}{c}{\textbf{Input}} & \multicolumn{3}{c}{\textbf{Fuzzing Engine}} & \multicolumn{2}{c}{\textbf{Running Environment}} \\
        \cmidrule(lr){2-4} \cmidrule(lr){5-7} \cmidrule(lr){8-9} & \textbf{OSL} & \textbf{ISG} & \textbf{I.} & \textbf{SS} & \textbf{ST} & \textbf{U.} & \textbf{F.} & \textbf{VTs} \\
        \midrule
        \endfirsthead
        
        \toprule
        \multirow{2.5}{*}{\textbf{Fuzzer}} & \multicolumn{3}{c}{\textbf{Input}} & \multicolumn{3}{c}{\textbf{Fuzzing Engine}} & \multicolumn{2}{c}{\textbf{Running Environment}} \\
        \cmidrule(lr){2-4} \cmidrule(lr){5-7} \cmidrule(lr){8-9} & \textbf{OSL} & \textbf{ISG} & \textbf{I.} & \textbf{SS} & \textbf{ST} & \textbf{U.} & \textbf{F.} & \textbf{VTs} \\
        \midrule
        \endhead
        
        \bottomrule
        \endfoot
        
        Trinity\cite{Trinity} & Kernel & Spec. & - & - & - & Gen.(1) & Code & -\\
        
        %perf\_fuzzer\cite{Weaver2015perf} & Kernel & Spec. & - & - & - & Gen.(1) & *Code & -\\
        
        Syzkaller\cite{Syzkaller} & Kernel & Spec.' & Static & Feedback & Feedback & Mut. & Code & -\\
        
        KernelFuzzer\cite{KernelFuzzer2016} & Kernel & Spec. & Static & - & - & Gen.(1) & Code & -\\
        
        DIFUZE\cite{2017DIFUZE} & Driver & Spec.' & Static & - & - & Gen.(2) & - & M.,L.,O. \\
        
        VDF\cite{henderson2017vdf} & Hypervisor & Trace & Static & - & Crash & Mut. & Code & M.,C.,L. \\
        
        KAFL\cite{schumilo2017kafl} & File System & - & Static+Dynamic & - & - & Mut. & Code & M. \\
        
        Semfuzz\cite{you2017semfuzz} & Kernel & PoC & Static & Distance & - & Mut. & Code & M. \\
        
        usb-fuzzer\cite{Syzkaller} & Driver & Spec.' & Static & - & - & Mut. & Code & M. \\
        
        Moonshine\cite{pailoor2018moonshine} & Kernel & Trace & Static & Feedback & - & Mut. & Code & M.,C. \\
        
        FUZE\cite{wu2018fuze} & Kernel & PoC & Static & - & Distance & Mut. & Code & M. \\
        
        Schwarz~et~al.\cite{schwarz2018automated} & Kernel & Trace & - & - & - & Gen.(1) & Code & C. \\
        
        Razzer\cite{jeong2019razzer} & Kernel & Spec.' & Static & - & - & Mut. & Code & C. \\
        
        JANUS\cite{JANUS2019fuzzing} & File System & - & Static & - & - & Mut. & Code+Cust. & M.,L. \\
        
        SLAKE\cite{chen2019slake} & Kernel & PoC & Static & - & Distance & Mut. & - & M. \\
        
        Shi et al.\cite{shi2019industry} & Kernel & Spec.' & Static & - & - & Mut. & - & M.,C.,O. \\

        PeriScope\cite{Song2019PeriScopeAE} & Driver & Trace & Static & - & Feedback & Mut. & Code & M. \\
        
        Unicorefuzz\cite{Unicorefuzz2019} & Kernel & Pattern & Static & - & - & Mut. & Code & M. \\
        
        KOOBE\cite{chen2020koobe} & Kernel & PoC & Dynamic & - & - & Mut. & Code+Cust. & M. \\
        
        Krace\cite{xu2020krace} & File System & Spec.' & Static & Minimum & - & Mut. & Code+Thread & C. \\
        
        Hyper-CUBE\cite{schumilo2020hyper} & Hypervisor & Pattern & - & - & Crash & - & - & M.,C.,L.,O. \\
        
        Hydra\cite{Hydra2020finding} & File System & - & Static & - & Crash & Mut. & Code+Cust. & M.,L. \\
        
        USBFuzz\cite{peng2020usbfuzz} & Driver & Pattern & Static & - & - & Mut. & Code & M. \\
        
        Agamotto\cite{song2020agamotto} & Driver & Trace & Static & - & - & Mut. & Code & - \\

        Ex-vivo\cite{Pustogarov2020ExvivoDA} & Driver & Trace & Static & - & - & Mut. & Code & M. \\
        
        HFL\cite{kim2020hfl} & Kernel & Spec.' & Static & - & - & Mut. & Code & M. \\
        
        X-AFL\cite{liang2020xafl} & Kernel & Trace & Static & - & - & Mut. & Code & M. \\
        
        HEALER\cite{sun2021healer} & Kernel & Spec.' & Static & - & Feedback & Gen.+Mut. & Code & M.,C.,L. \\
        
        Rtkaller\cite{shen2021rtkaller} & Kernel & Spec.' & Static & - & - & Mut. & Code & M.,C. \\
        
        NYX\cite{schumilo2021nyx} & Hypervisor & Spec. & Dynamic & - & - & Mut. & Code & M. \\
        
        V-Shuttle\cite{pan2021V-shuttle} & Hypervisor & Trace & Static & - & - & Mut. & Code & M.,L. \\
        
        BSOD\cite{maier2021bsod} & Driver & Pattern & Dynamic & - & - & Gen.+Mut. & Code & M. \\
        
        SyzVegas\cite{wang2021syzvegas} & Kernel & Spec.' & Static & Feedback & - & Mut. & Code & M. \\
        
        StateFuzz\cite{zhao2022statefuzz} & Driver & Spec.' & Static & Similarity & - & Mut. & Code+Cust. & M. \\
        
        GREBE\cite{lin2022grebe} & Kernel & PoC & Static & Minimum & - & Mut. & Custom & M.,O. \\
        
        MundoFuzz\cite{myung2022mundofuzz} & Hypervisor & Trace & Static & - & - & Mut. & Code & M.,L. \\
        
        Morphuzz\cite{bulekov2022morphuzz} & Hypervisor & Trace & Static & - & - & Mut. & Code & C.,L. \\
        
        CONZZER\cite{CONZZER2022context} & File System & Pattern & Static & Feedback & - & Mut. & Code+Thread & C. \\
        
        %Sfuzz\cite{chen2022sfuzz} & Kernel & - & Static & - & - & Mut. & *Code & M. \\
        
        Dr.Fuzz\cite{zhao2022semantic} & Driver & Spec.' & Static+Dynamic & - & - & Mut. & Code+Cust. & M.,O. \\
        
        PrintFuzz\cite{ma2022printfuzz} & Driver & Spec.' & Static+Dynamic & - & - & Gen.+Mut. & Code & M.,C.\\
        
        DriFuzz\cite{shen2022drifuzz} & Driver & Trace & Static & - & - & Gen.+Mut. & Code & M. \\
        
        Hao~et~al.\cite{Hao2022DemystifyingTD} & Kernel & - & Static & - & - & - & - & - \\
        
        KSG\cite{sun2022ksg} & Kernel & Spec.' & Static & - & - & - & - & M.,C.,L. \\
        
        SyzScope\cite{zou2022syzscope} & Kernel & PoC & Static & Feedback & - & Mut. & - & M. \\
        
        Tardis\cite{shen2022tardis} & Kernel & Spec. & Static & - & - & Mut. & Code & M.,L. \\
        
        Segfuzz\cite{jeong2023segfuzz} & Kernel & Spec.' & Static & - & - & Mut. & Code+Thread & M.,C.,O. \\
        
        IRIS\cite{cesarano2023iris} & Hypervisor & Trace & Static & - & - & Mut. & Code & M. \\
        
        FUZZNG\cite{bulekov2023FUZZNG} & Kernel & Pattern & Static & - & - & Mut. & Code & M. \\
        
        ACTOR\cite{fleischer2023actor} & Kernel & Trace & Static & - & - & Gen.(2) & Code & M.,L. \\
        
        SyzDescribe\cite{hao2023syzdescribe} & Driver & Spec.' & - & - & - & - & - & - \\
        
        DEVFUZZ\cite{wu2023devfuzz} & Driver & Pattern & Static+Dynamic & - & - & Mut. & Code+Cust. & M.,L.,O. \\
        
        Syzdirect\cite{tan2023syzdirect} & Kernel & Spec.' & Static & Distance & Distance & Mut. & Code & - \\
        
        ReUSB\cite{Jang2023ReUSB} & Driver & Trace & Dynamic & - & - & Mut. & Code & M. \\
        
        DDRace\cite{Yuan2023DDRace} & Driver & Spec.' & Static & Feedback & - & Mut. & Code+Thread & M.,C. \\
        
        VDGUARD\cite{Liu2023VDGuard} & Hypervisor & Trace & Static & - & - & Mut. & Code & M.,L. \\

        ViDeZZo\cite{Liu2023ViDeZZoDV} & Hypervisor & Trace & Static & - & - & Gen.+Mut. & Code & M.\\
        
        Lfuzz\cite{Liu2023LFuzz} & File System & - & Static & - & - & Mut. & Code & M.,L. \\

        KernelGPT\cite{yang2023kernelgpt} & Kernel & Spec.' & Static & - & - & Mut. & Code & M.,O. \\

        BRF \cite{Hung2024BRFFT} & Kernel & Spec.' & Static & - & - & Gen.+Mut. & Code & M.,C.\\

        MOCK \cite{Xu2024MOCKOK} & Kernel & Spec.' & Static & Feedback & Feedback & Gen.+Mut. & Code & M.,C.\\

        SATURN\cite{Xu2024Saturn} & Driver & Spec.' & Static & Similarity & - & Gen.+Mut. & Code & M.,C.,I.\\

        Syzgen++\cite{Chen2024SyzGen++} & Driver & Spec.' & Static & - & - & Gen. & Code & M.,C.,I.\\

        VIRTFUZZ\cite{Huster2024ToBoldly} & Driver & Trace & Static & - & - & Mut. & Code & M.,L.\\

        HYPERPILL\cite{Bulekov2024HYPERPILLFF} & Hypervisor & Pattern & Static & - & - & Mut. & Code & M. \\

        %SyzBridge \cite{Zou2024SyzBridgeBT} & Kernel & \\
        
        %KernelGPT\cite{yang2023kernelgpt} & Kernel & Spec.' & Static & - & - & Mutation & *Code & \fullcirc+\halfcircver \\
        \end{longtable}
        \vspace{-0.4cm}
        \begin{flushleft}
        \justifying
        A “-” means it is irrelevant, not mentioned, or unclear in detail.
        \textbf{OSL} is the abbreviation of OS Layer. 
        \textbf{ISG} is the abbreviation of initial seed generation.
        \textbf{I.} is the abbreviation of instrumentation.
        \textbf{SS} is the abbreviation of seed selection.
        \textbf{ST} is the abbreviation of seed trim.
        \textbf{U.} is the abbreviation of update.
        \textbf{F.} is the abbreviation of feedback.
        \textbf{VTs} is the abbreviation of vulnerability types: \textbf{ M.} is the abbreviation of Memory Violation Bug. \textbf{C.} is the abbreviation of Concurrency Bug. \textbf{L.} is the abbreviation of Logic Bug. \textbf{O.} means Privilege Protection Bug, Processor Exception, or Data Integrity Bug.
        \end{flushleft}
        \end{footnotesize}
        
%``\fullcirc'' is the abbreviation of M.. ``\emptycirc'' means Concurrency Bug. ``\halfcirchor'' means Logic Bug. ``\halfcircver'' means Access Violation Bug.

% subsection 3
\subsection{Initial Seed Generation}

Initial seed generation focuses on how to generate seeds at the beginning of OSF, which can be categorized into four types based on the input provided to the fuzzer: Pattern-, Specification-, Trace-, and Poc-based initial seed generation approach. As shown in Table \ref{tab1}, these categories are listed under the column ``ISG''.

\subsubsection{Pattern-based}

%Template-Based generation involves providing the fuzzer with a template file that interacts with the actual PUT. The fuzzer then generates the initial seed sequence based on this template's content. This method includes early open-source fuzzing frameworks such as AFL \cite{AFL}, Honggfuzz \cite{Honggfuzz}, TriforceAFL \cite{TriforceAFL}, and libFuzzer \cite{libfuzzer}. Although these frameworks are not specifically designed for OS fuzzing, they can be adapted for OSF seed generation by compiling custom initial input files to run on the target operating system.
\todo{In early research, traditional fuzzer tools or random number generator like AFL \cite{AFL}, Honggfuzz \cite{Honggfuzz}, TriforceAFL \cite{TriforceAFL}, libFuzzer \cite{libfuzzer}, and PRNG (a pseudo-random-number generator) \cite{PRNG2020} were extended to a initial seed generator for OSF. Their approaches in initial seed generation are classified as pattern-based methods because they involve defining patterns of data structures, interaction behaviors, or input formats for another new PUT within their original fuzzing framework. The process typically requires creating a file that these tools can recognize, leveraging domain knowledge to define the patterns used for interacting with the new PUT. For instance, in AFL, this could involve a predefined input format for kernel fuzzing (an initial syscall sequence) or I/O operations that simulate device behavior in driver fuzzing.} 

%Although extending and customizing these open-source fuzzer tools can enable OSF, these methods often suffer from high randomness and low efficiency.}

\todo{Therefore, to fuzz driver, fuzzers like USBFuzz \cite{peng2020usbfuzz}, DEVFUZZ \cite{wu2023devfuzz}, and BSOD \cite{maier2021bsod} have extended AFL to generate device inputs (such as vendor IDs and communication data). These fuzzers generate initial seeds based on interaction patterns between drivers and devices, which are then passed to the respective drivers to simulate realistic hardware behavior in response to driver read requests. For kernel fuzzing, other fuzzers (such as CONZZER \cite{CONZZER2022context} and TriforceAFL) extend AFL to generate random syscall sequences, aiming to uncover vulnerabilities within the kernel and its subsystems. FUZZNG \cite{bulekov2023FUZZNG} utilizes libFuzzer to generate random bytes, translating them into valid syscall sequences. Unicorefuzz \cite{Unicorefuzz2019} uses AFL to produce binary instructions to fuzz kernel ported to user space. In hypervisor fuzzing, fuzzers such as \cite{schumilo2020hyper,Bulekov2024HYPERPILLFF,schumilo2021nyx,Liu2023VDGuard,bulekov2022morphuzz} utilize libFuzzer, PRNG, or AFL to generate initial bytecode sequences (initial seed). Although OSF can be achieved through extensions and customizations of open-source fuzzing tools, these methods often neglect syntactic and semantic constraints between seeds, leading to the problem of high randomness and inefficiency in fuzzing.}

%To overcome inefficiencies caused by random input from devices and the physical actions of attaching and detaching devices in driver fuzzing, frameworks, such as USBFuzz \cite{peng2020usbfuzz}, Devfuzz \cite{wu2023devfuzz}, and BSOD \cite{maier2021bsod} extend AFL to generate device inputs (\eg vendor IDs, communication data). They use a file to communicate fuzzer-generated inputs to the target program, which the device uses to respond to read requests from device drivers during IO operations.

%\todo{To overcome the inefficiencies caused by random inputs from devices and the physical operations of connecting and disconnecting devices during driver fuzzing, frameworks such as USBFuzz \cite{peng2020usbfuzz}, Devfuzz \cite{wu2023devfuzz}, and BSOD \cite{maier2021bsod} have extended AFL to generate device inputs (\eg vendor IDs, communication data). These fuzzers generate initial seeds based on these patterns and pass them to the PUT, simulating real hardware behavior in response to read requests from device drivers.}

%Additionally, to apply fuzzing to kernel, one kind of fuzzers such as CONZZER \cite{CONZZER2022context} and TriforceAFL extend AFL to find kernel and subsystem bugs by invoking randomly generated syscalls. FUZZNG \cite{bulekov2023FUZZNG} uses libFuzzer to first generate random byte buffers, which are then mapped into multiple valid syscall sequences. The other type of fuzzer, Unicorefuzz \cite{Unicorefuzz2019}, employs AFL to generate binary instructions, which are then used for fuzzing kernel modules that are ported to userland.

%\todo{Additionally, to apply fuzz testing to the kernel, certain fuzzers, such as CONZZER \cite{CONZZER2022context} and TriforceAFL, have also extended AFL to find kernel and subsystem bugs by invoking randomly generated syscalls. FUZZNG \cite{bulekov2023FUZZNG} utilizes libFuzzer to first generate random byte buffers, which are then translated into valid syscall sequences. Another type of fuzzer, Unicorefuzz \cite{Unicorefuzz2019}, uses AFL to generate binary instructions that are subsequently used for fuzzing kernel modules ported to user space.}

\subsubsection{Specification-based}
%Specification-Based initial seed generation requires constructing a specification for the PUT to create initial seed sequences based on both syntax and semantics. Early popular kernel fuzzing frameworks like Trinity \cite{Trinity} and KernelFuzz \cite{KernelFuzzer2016} use this approach. Due to their semi-random seed generation strategy and limited extensibility, these frameworks are often used as foundational infrastructure for syscall generation \cite{Weaver2015perf,pailoor2018moonshine, jeong2023segfuzz,xu2020krace}.
\todo{Specifications are a type of document that describes meta information for seeds, including details such as seed types, names, parameter types, and value ranges. Therefore, initial seed generation methods based on specifications require the construction of a dedicated specification for the PUT to guide the generation of the initial seed, ensuring it meet basic syntactic rules. Since the data structures for I/O communications are implemented by third-party vendors and are commercially protected, and existing fuzzing tools only construct specifications for syscalls, current specification-based initial seed generation methods are primarily applicable to scenarios where syscalls are used as seeds \cite{pailoor2018moonshine, jeong2023segfuzz,xu2020krace}.}

%Since this approach can clearly define the constraints and input-output requirements for function-specific seeds, it is commonly employed in the initial seed generation for syscall-based fuzzing. \cite{pailoor2018moonshine, jeong2023segfuzz,xu2020krace}.}

Early popular kernel fuzzing frameworks like Trinity \cite{Trinity} and KernelFuzz \cite{KernelFuzzer2016} use this approach. Trinity relies on hard-coded rules to produce initial syscall sequences, such as creating a list of file descriptors and annotating arguments with valid or near-valid data types and values. It is suitable for scenarios focusing solely on random syscalls generation \cite{xu2020krace, schwarz2018automated}, but it is less adaptable to kernel changes \cite{pailoor2018moonshine}. Although KernelFuzz considers detailed syscall specifications during fuzzing process, its adaptability and extensibility are challenged by \todo{random mutation strategies.}

To build diverse and accurate syscalls, Syzkaller \cite{Syzkaller} uses a structured description language called syzlang \cite{syzlang} to record syscall declarations, providing more semantic information during initial seed generation. As shown in Table \ref{tab1}, fuzzers with spec.'-based inital seed generation use Syzkaller as foundational infrastructure for initial seed generation, allowing them to focus on enhancing the effectiveness of other steps, such as seed selection \cite{wang2021syzvegas,Yuan2023DDRace}, seed trim \cite{shen2021rtkaller,zhao2022semantic,hao2023syzdescribe}, mutation strategy \cite{xu2020krace,sun2021healer,shen2021rtkaller,zhao2022semantic}, feedback optimization \cite{JANUS2019fuzzing,shi2019industry,xu2020krace,shen2021rtkaller,wang2021syzvegas,zhao2022statefuzz,CONZZER2022context,zhao2022semantic,jeong2023segfuzz}, and monitor enhancement \cite{jeong2019razzer,xu2020krace,CONZZER2022context,jeong2023segfuzz}. \todo{Additionally,
to improve the quality of the initial seed generated by Syzkaller, \cite{2017DIFUZE,kim2020hfl,ma2022printfuzz,sun2022ksg,hao2023syzdescribe,tan2023syzdirect,Yuan2023DDRace,Hung2024BRFFT,Xu2024MOCKOK} extract entry points and infer dependencies of syscall using program analysis techniques, manual analysis based on domain knowledge, and neural network model.} Another work, KernelGPT \cite{yang2023kernelgpt}, employs LLM to construct syzlang, enabling Syzkaller to generate Syzlang language for the newly merged code in the Linux mainline.


\subsubsection{Trace-Based}
To avoid the complex construction of specifications, Trace-Based generation uses real applications or devices as trigger engines to create initial seed sequences based on intercepted data structures. Some studies intercept syscall sequences \cite{pailoor2018moonshine,schwarz2018automated,liang2020xafl,fleischer2023actor,Jang2023ReUSB}, I/O communication \cite{song2020agamotto,Pustogarov2020ExvivoDA,pan2021V-shuttle,myung2022mundofuzz,bulekov2022morphuzz,shen2022drifuzz,Liu2023VDGuard,Song2019PeriScopeAE,Liu2023ViDeZZoDV,Huster2024ToBoldly}, or driver operations \cite{henderson2017vdf,cesarano2023iris,Jang2023ReUSB} to generate inital seed. This approach ensures that the initial seeds are real and effective, enhancing the fuzzer's flexibility and portability. However, a key limitation is the lack of usage specifications for critical structure of seed, which may render them ineffective during arbitrary seed mutations, thereby limiting the depth of testing. Trace-based initial seed generation techniques can be implemented through software, hardware instrumentation, or third-party tool like STRACE \cite{Strace}, Wireshark \cite{Wireshark} and USBMON \cite{usbmon}, and Flush+Reload \cite{Yarom2014FLUSHRELOADAH, 2011IS}.

\subsubsection{PoC-Based}

In addition to the previously mentioned methods, PoC-based approach generate the initial seed through taking a PoC or bug report as input, where the PoC or bug report can be acquire in CVE \cite{cve}, Linux git logs \cite{linuxkernel}, and bug description posted on forums and blogs \cite{fulldisclosure, sans, krebsonsecurity}. This method generally focuses on the field of OS bug exploitation or \todo{homogeneous vulnerability exploration}, such as FUZE \cite{wu2018fuze}, KOOBE \cite{chen2020koobe}, Syzscope \cite{zou2022syzscope}, GREBE \cite{lin2022grebe}, SLAKE \cite{chen2019slake} and SemFuzz \cite{you2017semfuzz}. The fuzzers employ program analysis techniques, such as taint analysis \cite{zou2022syzscope,lin2022grebe}, static analysis \cite{chen2019slake} and symbolic execution \cite{wu2018fuze, chen2020koobe}, to identify critical objects in the kernel that behave similarly to the bug. These techniques then generate initial seeds using syscalls and parameters that can reach these critical objects. Another study, SemFuzz \cite{you2017semfuzz}, uses Natural Language Processing (NLP) to extract vulnerable functions, vulnerability types, critical variables, and syscalls from CVEs and bug reports. It then uses Syzkaller to precisely generate syscall sequences that bring the execution of the target kernel closer to the vulnerable function.

% subsection 4
\subsection{Instrumentation}
\label{subsection4.2}
\todo{Instrumentation inserts probes into a program to collect runtime information from the PUT while preserving its original functionality and logical structure \cite{Instrumentation_Huang1978, ibm2020}.} By analyzing and processing this runtime data, insights into the program's control and data flow can be acquired. Consequently, this allows for intercepting data sending to PUT, the calculation of coverage metric and the monitoring of bugs. Such data then guides the fuzzing process towards more sensitive and potentially critical code paths.

Unlike userland fuzzing, performing instrumentation in kernel space using built-in compiler tools such as GCC Coverage (Gcov) \cite{gcov} and Address Sanitizer (Asan) \cite{ASAN} in gcc or clang/llvm is impractical due to the large scale and complexity of the modern OS kernel. Consequently, researchers have developed two instrumentation approaches for operating systems.

\subsubsection{Static Instrumentation}

Static instrumentation typically involves inserting probe programs into the source code or intermediate representation code. Open-source operating systems offer a rich set of tools for static instrumentation, such as KCOV \cite{vyukov2018kcov}, KASAN \cite{kernel2018kasan}, \etc These tools are well-suited for OSF due to their advantages of being easy to use, having relatively low overhead, being customizable, and providing strong support for Unix-like operating systems. For instance, Syzkaller \cite{Syzkaller} compiles KCOV and KASAN into the kernel to guide OSF based on collected code coverage and detects memory-related bugs. Note that KCOV and KASAN were introduced in Linux versions 4.6 and 4.0, respectively. For older versions of Linux, SemFuzz \cite{you2017semfuzz} implemented the porting of these tools.

However, built-in kernel instrumentation tools are not always effective in certain scenarios. For example, challenges arise in thoroughly exploring the program state space, handling thread interleaving for concurrency errors, and performing directed fuzzing. Some studies have addressed these specific challenges by extending existing compiler tools to implement fine-grained program analysis and targeted instrumentation. For example, coverage-guided fuzzers often discard test cases useful for exploring potential program states (\ie values of all program variables, virtual memory, and registers) if these test cases do not trigger new code paths. To overcome this challenge, StateFuzz \cite{zhao2022statefuzz} introduces a new feedback metric called state coverage and uses the static instrumentation tool LLVM SanCov to collect program state information, and then fine-tune the execution direction of coverage-guided fuzzers. In the context of concurrency error detection, fuzzers such as DDRace \cite{Yuan2023DDRace}, Krace \cite{xu2020krace}, SegFuzz \cite{jeong2023segfuzz}, CONZZER \cite{CONZZER2022context}, and Razzer identify customized thread coverage metrics to explore potential code areas for data races through the LLVM suite. For directed fuzzing, GREBE \cite{lin2022grebe} uses LLVM Analysis and Pass to track taint propagation paths in the program to identify critical objects. Tardis \cite{shen2022tardis} addresses the unavailability of KCOV in embedded operating systems (\eg \textit{UC/OS}, \textit{FreeRTOS}, \etc) by leveraging Clang's SanitizerCoverage and thus proposes an efficient coverage collection callback.

%MundoFuzz \cite{myung2022mundofuzz} utilizes AFL's instrumentation compiler (afl-clang-fast) to obtain branch coverage metrics and monitors logic-related vulnerabilities in a hypervisor with Msan \cite{MemorySanitizer}.

\subsubsection{Dynamic Instrumentation}
Dynamic Instrumentation, or Dynamic Binary Instrumentation (DBI), happens while the PUT is running. Although DBI have a higher runtime overhead compared to static instrumentation, its increased flexibility makes it useful for kernels without built-in instrumentation tools or for earlier versions of the Linux kernel. DBI solutions are generally categorized into hardware-assisted and software-assisted techniques. We review and summarize the application scenarios for both types of solutions below:

\textbf{Hardware-Assisted Solution.}
Hardware-assisted DBI (\eg Intel Processor Trace \cite{kleen2015,simplept} and ARM CoreSight \cite{coresight2017}) leverages special CPU features to trace the execution and branch information of a program. Note that hardware-assisted DBI records the execution paths rather than code coverage, but the detailed execution path information can be used to infer runtime coverage. Intel and ARM offer similar features; the former is suitable for testing kernels, drivers, file systems, and hypervisors in virtualized environments, while the latter is mainly used in embedded systems and mobile devices. However, ARM's tracing functionality is not mandatory for ARM CPUs, making it inapplicable to commercial Android devices \cite{Chizpurfle2019}.

In the collection of driver fuzzing feedback, KCOV fails to gather complete execution information for the driver validation chain because the \texttt{debugfs} files that expose KCOV coverage information are not yet available. This step typically includes hardware detection, resource allocation, and initial setup. Although it may not be as complex or extensive as the fully operational kernel code, it remains critical for system security and stability because attackers might exploit them before the system fully boots.

To overcome this problem, Dr.Fuzz \cite{zhao2022semantic}, PrIntFuzz \cite{ma2022printfuzz}, DEVFUZZ \cite{wu2023devfuzz}, and KAFL \cite{schumilo2017kafl} leverage Intel PT to track the execution flow during the driver initialization phase. They then switch back to KCOV to obtain precise coverage information, which can effectively enhances the depth of driver fuzzing through combining the strangths of both methods.


\textbf{Software-Assisted Solution.}
Software-assisted DBI refers to the dynamic injection of binary instructions during program execution using software breakpoints or binary rewriting techniques. This strategy is more flexible and generally applicable to devices without specific hardware support. For example, while ARM CoreSight is effective, it is difficult to apply to the commercial Android OS. Thus, this strategy should be treated with a grain of salt due to concerns about throughput and flexibility, as software-assisted DBI typically incurs a higher runtime overhead.

DBI software frameworks commonly used for OSF include Frida \cite{frida2017}, Valgrind \cite{Valgrind2007} , and \todo{INT 3 software breakpoints \cite{2006Intel6}.} Frida replaces an instruction in the debugged program with another instruction that stops the program and triggers a breakpoint handler function, allowing tracking of executed code blocks. Valgrind monitors applications on Linux through a virtual machine environment for focusing on analyzing memory usage and race condition errors. For instance, Chizpurfle \cite{Chizpurfle2019} leverages the DBI framework Frida and the Linux syscall ptrace to collect basic block coverage on real Android devices and uses Valgrind to monitor memory leaks and race conditions. INT 3 is an instruction in the x86 and x86-64 architectures that is used to trigger debug interrupts. Specifically, it obtains the execution flow information of the PUT by replacing the basic block jump instruction or the conditional control flow instruction after disassembling the binary PUT (\eg Capstone \cite{Capstone}) into a control flow graph. Eventually, the code coverage calculation for binary PUTs can be realized by integrating a modified Syzkaller's KCOV module or AFL's coverage calculation module. For example, BSOD \cite{maier2021bsod} connects to the VM using the introspection APIs, collects program control flow information by pausing the VM and replacing the first byte of the control flow instruction with the \texttt{0xcc} instruction (INT 3) .

% subsection 5
\subsection{Seed Selection}
\label{Section4.3}

%Seed selection \todo{selects the most useful seeds from the corpus for subsequent mutation.} This step is particularly important \todo{because the selected seeds directly determine the subsequent direction of the fuzzer. Hence, most research focuses on how to reduce the search space for seed selection and how to choose effective seeds.} Although there has been some optimization work on seed selection in userland fuzzing \cite{Rebert2014OptimizingSS, Woo2013SchedulingBM, Zhao2019SendHP}, these methods cannot be directly applied to operating systems due to the huge code base and \todo{various} interaction interfaces. To achieve directed fuzzing in OSF through seed selection, some fuzzers specifically designed for OSF (especially those using syscalls as the interaction interface) have proposed various seed selection strategies.

Seed selection \todo{selects the relatively ``valuable'' seeds from the corpus for subsequent mutation.} This step is particularly important \todo{because the selected seeds directly determine the subsequent direction of the fuzzer. Hence, most research focuses on how to reduce the search space for seed selection and how to choose effective seeds.} Although there has been some optimization work on seed selection in userland fuzzing \cite{Rebert2014OptimizingSS, Woo2013SchedulingBM, Zhao2019SendHP}, these methods cannot be directly applied to operating systems due to the huge code base and \todo{various} interaction interfaces. \todo{To obtain the most ``valuable'' seeds each time, some fuzzers specifically designed for OSF (especially those using syscalls as the interaction interface) have proposed various seed selection strategies.}

\subsubsection{Minimum Frequency.}
Past practices have proven that executing rarer code paths helps test extreme situations and makes it easier to expose bugs \cite{Bx00F6hme2016CoverageBasedGF, Lemieux2017FairFuzzAT}. Therefore, the Minimum Frequency principle refers to selecting the least frequently used seeds for the next round of fuzzing. Although this strategy is not optimal, its simplicity and effectiveness can achieve fuzzing objectives. For example, Krace \cite{xu2020krace} selects the two least used seeds each time and merges them into two different threads to explore data race errors. GREBE \cite{lin2022grebe} employs the PageRank \cite{Brin1998PageRank} algorithm to eliminate popular kernel objects, as testing these extensively explored objects often makes it harder to find bugs. \todo{This method is simple and efficient, and it can further reduce the search space for seed selection. However, seeds selected based on the minimum principle are not suitable for directed fuzzing of specific OS objects or vulnerabilities, as directed fuzzing typically focuses on a narrower range of code paths rather than rare paths.}

%However, when performing fuzzing targeted at a specific OS object or vulnerability (\ie directed fuzzing), the effectiveness of such methods may be limited.}



\subsubsection{Feedback-guided.}
A more straightforward principle is to select seeds that contribute the most to overall code coverage. For example, Syzkaller \cite{Syzkaller} is the first kernel fuzzer to guide seed selection using code coverage feedback, and it selects the seed with the largest relative coverage increase from corpus each time to be used in the next round of seed mutation. It is worth noting that the Syzkaller-based fuzzers in Table \ref{tab1} imply that they inherit Syzkaller's strategy if they innovate for seed selection and seed trim. Moonshine \cite{pailoor2018moonshine} prioritizes seed selection based on code coverage in descending order, ensuring that each selected seed is most beneficial for improving global coverage. HEALER \cite{sun2021healer} selects a syscall from the corpus that is `relevance'' to the current syscall sequence and inserts it into the sequence. This ``relevance'' is determined based on a syscall relation table preconstructed using coverage growth information from historical seeds, where an entry of 0 indicates no relation and 1 indicates a relevant relationship. During selection, HEALER leverages a predefined randomness parameter $\alpha$ to balance between exploitation (weighted selection based on the relation table, prioritizing syscalls that influence the current sequence) and exploration (random selection, disregarding the relation table). Syzvegas \cite{wang2021syzvegas} proposed an novel reward mechanism and used the Multi-Armed Bandit (MAB) algorithm to dynamically adjust the selection probability of seeds, prioritizing those that bring higher coverage and lower time costs for mutation, thus ensuring the selected seeds contribute to overall coverage improvement. \todo{Similarly, MAB is also utilized by MOCK \cite{Xu2024MOCKOK} to dynamically schedule the selection of seeds with higher coverage and within the time overhead. In addition, MOCK combines the context-aware dependency relation constructed by the neural network model on the basis of these seeds that trigger high coverage to select more compact seeds from the corpus each time, resulting in the selection of seeds with both kernel state and the ability to improve the coverage metric.}

Futhermore, in order to better guide fuzzing towards the target site \todo{(\ie specific OS object or vulnerability)}, one principle is to customize feedback metrics for the target site to direct the evolution of seed selection. For instance, CONZZER \cite{CONZZER2022context} and DDRace \cite{Yuan2023DDRace} designed thread interleaving feedback metrics tailored for data race bugs. They select seeds from the corpus that trigger more of these custom feedback metrics for each new round of mutation. Syzscope \cite{zou2022syzscope} defines high-risk impact and prioritizes selecting candidate seeds that expose this impact through KASAN. This type of seed selection strategy can constrain the evolution direction of the seeds to always point towards the target site. The principle of tailored feedback can effectively guide the direction of seed selection, but it requires researchers to carefully design and accurately calculate feedback metrics. Otherwise, the fuzzer may struggle to approach the target site, resulting in ineffective overhead.

%To better guide fuzzing towards the target site \todo{(\ie specific OS object or vulnerability)}, one principle is to customize feedback metrics for the target site to direct the evolution of seed selection. For instance, CONZZER \cite{CONZZER2022context} and DDRace \cite{Yuan2023DDRace} designed thread interleaving feedback metrics tailored for data race bugs. They select seeds from the corpus that trigger more of these custom feedback metrics for each new round of mutation. Syzscope \cite{zou2022syzscope} defines high-risk impact and prioritizes selecting candidate seeds that expose this impact through KASAN. This type of seed selection strategy can constrain the evolution direction of the seeds to always point towards the target site.

%It is noteworthy that while using only general code coverage (such as branch coverage and block coverage) as a guiding mechanism for seed selection contributes little to directed fuzzing, it is still useful for improving overall code coverage. Therefore, we still review this method. For example, Moonshine \cite{pailoor2018moonshine} prioritizes seed selection based on code coverage in descending order, ensuring that each selected seed is most beneficial for improving global coverage. Syzvegas \cite{wang2021syzvegas} proposed an novel reward mechanism and used the Multi-Armed Bandit (MAB) algorithm to dynamically adjust the selection probability of seeds, prioritizing those that bring higher coverage and lower time costs for mutation, thus ensuring the selected seeds contribute to overall coverage improvement. \todo{Similarly, MAB is also utilized by MOCK \cite{Xu2024MOCKOK} to dynamically schedule the selection of seeds with higher coverage and within the time overhead. In addition, MOCK combines the context-aware dependency relation constructed by the neural network model on the basis of these seeds that trigger high coverage to select more compact seeds from the corpus each time, resulting in the selection of seeds with both kernel state and the ability to improve the coverage metric.}

\subsubsection{Shortest Distance.}
Another seed selection strategy for directed fuzzing is the shortest distance principle. Specifically, this strategy requires researchers to first construct a call graph for kernel objects and then use the distance to the target site as the primary criterion for seed selection. Semfuzz \cite{you2017semfuzz} is the first fuzzer in the OSF to use the shortest distance principle for seed selection. It constructs the call graph by modifying GCC to collect call information during kernel compilation, and it uses the inverse of the distance from each candidate input's reachable functions to the vulnerable function as the priority. It selects the highest priority input for mutation each time. However, since Semfuzz's target site is limited to PoC-related vulnerable functions, it cannot fully execute OSF. SyzDirect \cite{tan2023syzdirect} employs static analysis to comprehensively identify interesting target sites within the kernel, constructs seed templates for reaching these sites, and combines the shortest distance principle to guide seed selection for directed fuzzing.

Seed selection based on the shortest distance principle is the most straightforward way to achieve directed fuzzing, but it also has the problem of not accurately determining which inputs can actually reach the target location. Consequently, it still wastes time on test cases that do not reach the target, leading to a waste of resources \cite{Huang2022BEACONDG, Zong2020FuzzGuardFO}. A good practice, Syzdirect \cite{tan2023syzdirect}, is to construct a template that describes the details of the seed composition for the target site, including function types, parameters, \etc This template can be used to verify the direction of seed evolution. Therefore, when the fuzzer gets stuck in ineffective local mutations, the template can guide the fuzzer to directly remove the seed from the corpus, helping it to escape the local mutation predicament. Hence, regardless of which of the above principles is adopted, it is best to consider a direction correction method when designing the seed selection strategy, continuously verifying whether the selected seeds are effectively moving towards the target site.

\subsubsection{Similarity Clustering.}
The similarity clustering approach groups or classifies seeds based on their shared features and assigns selection probabilities to each category according to fuzzing target. Methods such as minimum frequency, feedback-guided, and shortest distance inherently belong to metric feedback-based seed selection strategies, which prioritize early exploration of seeds that enhance specific metrics. However, these methods are prone to local optima and struggle to adapt to collaborative fuzzing scenarios requiring integrated multi-input interactions. To mitigate these issues, StateFuzz \cite{zhao2022statefuzz} clusters seeds with similar characteristics and ensures equal probability for selecting seeds across different clusters, thereby alleviating local optima. Furthermore, SATURN \cite{Xu2024Saturn} categorizes the corpus by device functionality (\eg printers, keyboards, storage devices) and dynamically selects seeds that enable host-device interaction based on the currently attached device type, preventing potential seed interaction conflicts during collaborative testing. Existing similarity clustering-based methods focus predominantly on clustering or classification, often employing random selection probabilities across categories. While they address challenges like local optima and collaborative testing, they still risk overlooking "valuable" seeds. 

% subsection 6
\subsection{Seed Trim}
\todo{Seed trim, also known as seed minimization, removes parts of the seed that do not contribute to the fuzzing objectives—such as researcher-defined feedback, vulnerability discovery, or proximity to the target site—while maintaining stable coverage \cite{Abdelnur2010SpectralFE,AFL,Peachtech}. For example, removing syscalls in a syscall sequence that do not contribute to defined targets or shortening the length of argument.} Seed trim is considered a critical step in ensuring fuzzing efficiency, as redundant seeds waste computational resources that could be used to thoroughly explore code regions. Since the problem of seed minimization has been proven to be NP-hard \cite{Rebert2014OptimizingSS}, existing OSF approaches typically use three types of heuristic principles to address the seed minimization problem: feedback-based, distance-based, and crash-based principles. Table \ref{tab1} shows fuzzers that optimize Seed trim, with `-' symbols indicating studies that do not explicitly mention optimization strategies.

%Seed Trim refers to removing redundant seeds and preserving them in the corpus \cite{Abdelnur2010SpectralFE,AFL,Peachtech}. Seed Trim is considered as a seed minimization problem because redundant seeds waste computational resources that could be used to thoroughly explore code areas. This problem has been proven to be NP-hard \cite{Rebert2014OptimizingSS}. Similar to seed selection, existing OSF approaches typically use three types of heuristic principles to address the seed minimization problem: feedback-based, distance-based, and crash-based principles. Table \ref{tab1} shows fuzzers that optimize Seed Trim, with `-' symbols indicating studies that do not explicitly mention optimization strategies.



%\textbf{The feedback-based principle} involves adding a seed to the corpus only after its coverage is stable and thoroughly minimized when a new coverage-triggering seed is discovered \cite{Syzkaller}. This principle is intuitive but often closely tied to the mutation phase, leading to significant computational resource waste \cite{wang2021syzvegas,Jang2023ReUSB}.

\textbf{The feedback-based principle} \todo{focuses on identifying seed sets that enhance feedback metrics, such as coverage, while simultaneously removing ineffective seeds. This approach requires consideration of two key aspects: 1) promptly eliminating seed sets that do not contribute to coverage improvement; and 2) simplifying the contributing seed sets by removing individual seeds that do not aid in coverage enhancement. Some works \cite{Syzkaller,Song2019PeriScopeAE,sun2021healer,Xu2024MOCKOK} evaluate the "value" of seeds by executing the seed set (a syscall sequence) in the corpus. If the syscall sequence does not contribute to coverage improvement, it is directly removed from the corpus; if it does improve coverage, the shortest contributing sequence is further extracted (\ie by removing individual non-contributing syscalls). While the principle behind this method is straightforward and intuitive, its efficiency is a significant concern due to the vast search space and the current lack of a method to quickly identify individual ineffective seeds \cite{wang2021syzvegas,Jang2023ReUSB}.}

\textbf{The distance-based principle} aims to address the challenges of directed fuzzing \cite{wu2018fuze,chen2019slake,hao2023syzdescribe}. It computes the reachable distance to the target site by constructing call graphs or creates seed templates that trigger the target site to filter out irrelevant seeds. This principle can reduce unnecessary exploration space due to the higher directional. However, it may not fully capture dynamic execution paths and runtime behaviors, affecting the accuracy of the filtering results. Additionally, the paths to the target site can be highly complex and dependent on multiple uncertain factors, causing some important seeds to be mistakenly filtered out \cite{Huang2022BEACONDG, Zong2020FuzzGuardFO}. Therefore, this method requires removing kernel objects outside the target site range to improve call graph construction efficiency and carefully extracting dependencies between seeds to ensure the accuracy of the minimization process.

\textbf{The crash-based principle} is applied after a crash is discovered, iteratively removing seeds that do not contribute to the crash by preserving the state of each seed generation and using seed replay. This method aims to increase the probability of crash reproduction but can also remove seeds with potential dependencies, leading to inconsistencies between the paths triggered by the minimized seeds and the original ones \cite{Jang2023ReUSB}.

% subsection 7
\subsection{Seed Update}
\label{Update}
Generation- and mutation-based are two common methods for seed updating \cite{Mans2018TheAS,Sutton2007FuzzingBF}. In OSF, seed update strategies focus more on mutation-based methods or a combination of both to enhance the depth of OS fuzzing. In this section, we will outline and review the application of these three strategies in OSF.

\subsubsection{Generation}
%The use of generation-based seed updating strategies in kernel fuzzing can be traced back to the seminal work of Koopman et al. \cite{Koopman1997ComparingOS}, who compared the robustness of operating systems with a limited set of manually selected syscall test cases. In OSF, generation-based OS fuzzers \cite{Trinity,KernelFuzzer2016,Weaver2015perf,schwarz2018automated} define input models through syscall grammar (\eg Gen.(1) in Table \ref{tab1}). These grammars define the parameter types and value ranges required for each syscall in a template form, thus providing the fundamental elements for seed updating.

\todo{The seed update strategy based on generation requires predefining seed usage rules to guide the generation of at least syntactically correct seeds. This method is particularly suitable for seeds that are structured, limited in number, or challenging to mutate accurately. Consequently, it is often applied in syscall-based (kernel) and I/O data-based (driver) fuzzing. In kernel fuzzing, generation-based operating system fuzzers \cite{Trinity,KernelFuzzer2016,schwarz2018automated} define a seed usage template based on the syntax of syscalls (\eg Gen.(1) in Table \ref{tab1}). This template specifies the name, type, argument types, and value ranges for each syscall, thereby providing the fundamental elements for seed updates.}

Another generation-based fuzzers ((\eg Gen.(2) in Table \ref{tab1})) extracts seed models that trigger target program sites to enhance the accuracy of directed fuzzing. For example, DIFUZE \cite{2017DIFUZE} uses static analysis to extract structural models of user space and driver interactions, enabling the construction of precise syscalls and parameter data structures for the target driver code. ACTOR \cite{fleischer2023actor} designed a flexible domain-specific language (DSL) to express and encode various vulnerability templates. These templates describe the triggering conditions for specific types of vulnerabilities, such as memory access errors and reference counting errors. Specifically, ACTOR records memory operations (referred to as ``actions'') during the fuzzing process based on vulnerability templates and attempts to recombine and rearrange these actions to generate new seeds that are more likely to trigger vulnerabilities.

% \begin{figure}[!t]
%         \includegraphics[width=0.99\linewidth]{img/mutation.pdf}
%         % \vspace{-5pt}
%         \caption{Mutation}
%         \label{img:mutation}
% \end{figure}

\subsubsection{Mutation}
Mutation-based strategies require designing multiple mutation algorithms (mutators) to update seeds. Unlike userland fuzzing, constructing complete syscall templates and accurate kernel seed models is often time-consuming and error-prone. Therefore, most operating system fuzzer tend to adopt mutation-based update strategies to explore deeper code paths. Mutation object can be classified into four categories: Syscall, Argument, Thread, and Other.

\textbf{Syscall.}
Syscall mutations are used in fuzzers that interact via syscalls, including adding, deleting, replacing, and reordering syscalls. Adding involves inserting an additional syscall into an existing syscall sequence, such as inserting a \texttt{read} between \texttt{ioctl} and \texttt{write}. Deleting involves removing a syscall from the sequence, for example, removing a \texttt{write} after a \texttt{read} instead of removing the \texttt{read} before the \texttt{write}. Replacing refers to substituting a commonly used resource, such as replacing the \texttt{ioctl} interface. Reordering involves randomly changing the order of one or more syscalls. Fuzzers (such as Razzer \cite{jeong2019razzer}, HEALER \cite{sun2021healer}, \etc) use Syzkaller as a mutation engine typically employ this method as a fundamental mutation operator. Early Syzkaller mutated syscalls in a randomized manner, which tended to ignore the dependencies between syscalls, resulting in the mutated syscall sequences becoming invalid. Therefore, some fuzzers \cite{pailoor2018moonshine,sun2021healer,Hao2022DemystifyingTD} focus on analyzing and extracting dependencies between syscalls to reduce the likelihood of syscall sequences becoming invalid after mutation.

\textbf{Argument.}
Argument mutation refers to performing mutation operations such as bit flip, byte reservation, byte replacement, and buffer fulfillment on syscall arguments and API arguments provided by the Hypervisor for testing (\eg the QTest framework used to support QEMU unit testing \cite{QTest}). Bit flip is a commonly used mutator for argument mutation, enabling quick argument changes by flipping specified or multiple random bits. Another mutator performs reservation and replacement on random bytes or variables \cite{JANUS2019fuzzing,Hydra2020finding,Liu2023LFuzz}. Despite the simplicity and ease of use of this method, its randomness limits the effectiveness of the fuzzer. Therefore, a more stable approach is to retain or completely replace the values of known mutable bytes or variables. For instance, GREBE \cite{lin2022grebe} reserve or replace the argument based on the original PoC in order to trigger more similar vulnerabilities. In addition, buffer fulfillment can be used for the variable-length argument in network syscall. For example, SemFuzz \cite{you2017semfuzz} caused an Use-After-Free vulnerability in kernel code that would otherwise handle skb.len properly by filling the \texttt{buf} argument of the \texttt{sendto} syscall to more than 512 bytes of data.

\textbf{Thread Interleaving.}
Data race is a common vulnerability in the kernel and kernel file system, and it is difficult to trigger concurrent access to shared data using only syscall- or argument-based mutators.  In order to identify data race behavior in the kernel or kernel file system, it is necessary to analyze the interleaving between two or more threads. Previous thread interleaving algorithms on userland testing use SKI \cite{Fonseca2014SKIEK} or PCT algorithms \cite{Burckhardt2010PCT} to schedule threads. The principle is to use hardware breakpoints to hang memory access threads, and subsequently randomly select a thread or schedule a high-priority thread to perform thread interleaving. This approach focuses only on user threads and has the disadvantages of missing race conditions and exploding search paths. To solve this problem, kernel fuzzer uses different techniques to improve kernel thread interleaving:

For random thread interleaving, Razzer \cite{jeong2019razzer} uses hardware breakpoints to interleave threads by adding two new hypercalls to the Hypervisor: \texttt{hcall\_set\_bp} and \texttt{hcall\_set\_order}, where \texttt{hcall\_set\_bp} instructs to set the breakpoint's virtual CPU and the breakpoint's address, and \texttt{hcall\_set\_order} sets the order in which threads are executed. Razzer uses these two hypercalls to control the order in which threads are executed, thus enabling random interleaving of threads. Krace \cite{xu2020krace} employs a low-invasion method for thread interleaving. First, \todo{syscall sequences} are generated for multiple different threads, and then these \todo{syscall sequences} are combined in an interleaved manner without disrupting the relative order of the \todo{syscall sequences} within individual threads to protect the original dependencies. Finally, delay injection is used to suspend the current thread at a memory access point for a period of time (picked at a random time from the designed ring buffer structure) to mutate the execution order of the multiple threads. CONZZER \cite{CONZZER2022context} takes a similar approach, exploring different interleaving possibilities by a random run of two functions specified to contain memory access instructions.

Random thread interleaving does not systematically search for interleaving possibilities and tends to perform redundant interleaving. To overcome this challenge, Segfuzz \cite{jeong2023segfuzz} disassembles a group of interleaved threads into multiple segments, subsequently performs syscall or argument mutators on the instructions within each segment, and finally merges them into new thread interleavings to fully explore the interleaving possibilities of each group of threads. It is worth noting that each segment consists of up to four memory instructions (\cite{Lu2008LearningFM} statistics that 92.4\% of concurrency bugs are due to the execution order of up to four memory accesses), and the relative order of the instructions remains fixed after decomposition. This approach improves the depth of thread interleaving exploration and but is limited to detecting data race vulnerabilities triggered by 4 memory access instructions \cite{Lu2008LearningFM}.

In addition, a different way of thread interleaving is adopted for directed fuzzing. As described in Section \ref{Section4.3}, directed fuzzing guides the testing path by constructing the distance between the current execution path and the target site. For example, DDRace \cite{Yuan2023DDRace} models control flow distance and data flow distance feedback metrics to guide the mutator in generating single-threaded syscall sequences accessing shared data and uses the proposed Race Pair Interleaving Path (RPIP) metrics as a priority to schedule multi-thread delays. In contrast to the above thread interleaving approaches, such approaches usually dedicate efforts to a specific vulnerability model, depending on program analysis techniques or by proposing feedback metrics related to the vulnerability to guide the thread interleaving.

\begin{figure*}[!t]
        \centering
        \subfigure[The definition of a task.]{
                \includegraphics[width=0.33\linewidth]{img/task_definition.pdf}\label{img:task_definition}}
        \hspace{20pt}\vline\hspace{20pt}
        \subfigure[An instantiated task.]{
                \includegraphics[width=0.51\textwidth]{img/task_instance.pdf}\label{img:task_instance}}
        \caption{A task for fuzzing RTOS}
\end{figure*}

\textbf{Other.}
The above mutation methods mainly update for syscall. However, syscall is unable to convey some detailed information, such as priority, execution state, and also unable to simulate the behavior of device drivers. For example, Rtkaller \cite{shen2021rtkaller} first generates a new syscall sequence using syscall- and argument-based mutator, and then constructs the basic execution unit-task of real time os (RTOS). As shown in Figure \ref{img:task_definition}, a task refers to execution units in rtos, comprising multiple syscall sequence accompanied with runtime priority (programs' execution order) and concurrency intensity (the number of the syscall sequences). According to the task definition in Fig.\ref{img:task_instance} first a new syscall sequence is generated using syscall and argument mutation and then Rtkaller replaces the priority of the four programs using random numbers to monitor the kernel functional modules related to real-time. \todo{Similarly}, Hydra \cite{Hydra2020finding} and janus \cite{JANUS2019fuzzing} employ argument-based mutators (\eg bit flip, \etc) to change meta data extracted from File system images and syscall-based mutators to generate manipulation of file system syscall sequence. VD-Guard \cite{Liu2023VDGuard}, Hypercube \cite{schumilo2020hyper}, V-shuttle \cite{pan2021V-shuttle}, VDF \cite{henderson2017vdf}, VIRTFUZZ \cite{Huster2024ToBoldly} and HYPERPILL \cite{Bulekov2024HYPERPILLFF} use argument-based mutators to alter I/O channel data to generate test cases for simulating the behavior of device drivers.

Seed update in OSF mainly focuses on mutation-based approaches. In addition, a strategy combining generation and mutaion expects to perform mutations directly on the basis of generated high-quality seeds to enhance the effectiveness of the seed renewal process.

% \begin{table} % 表格开始
%         \centering % 表格居中显示
%         \footnotesize
%         \caption{Feedback Categories in OSF.} % 表头标题
%         \vspace{-0.3cm}
%         \label{tab2} % 表格标签,便于引用

%         \begin{tabular}{ccl} % c表示单元格内容居中,l表示靠左,这里有两列,所以cc,如果是三列就是ccc或cll,根据自己

%                 \toprule % 顶部的线,这里可以定义粗细、toprule{1.5ptx}
%                 \multicolumn{1}{m{3cm}}{\centering Feedback} &                                                                                                                                                                                                                         % 中间的1.5cm表示第列宽度
%                 \multicolumn{1}{m{4cm}}{\centering Metric}   &
%                 \multicolumn{1}{m{4cm}}{\centering Fuzzers}                                                                                                                                                                                                                            \\

%                 \midrule % 中间的线
%                 \multirow{9}*{code coverage}                 & statement coverage                  & \cite{jeong2019razzer}, \cite{song2020agamotto}, \cite{zhao2022semantic}                                                                                                          \\

%                 \cline{2-3}

%                 ~                                            & \multirow{2}*{basic block coverage} & \cite{cesarano2023iris}, \cite{fleischer2023actor}, \cite{Syzkaller}, \cite{ma2022printfuzz}, \cite{maier2021bsod}, \cite{song2020agamotto}, \cite{Hung2024BRFFT},              \\
%                 ~                                            & ~                                   & \cite{shen2022drifuzz}, \cite{kim2020hfl}, \cite{you2017semfuzz}, \cite{schwarz2018automated}, \cite{Jang2023ReUSB}\\

%                 \cline{2-3}

%                 ~                                            & \multirow{2}*{edge coverage}        & \cite{zhao2022statefuzz}, \cite{shen2022tardis}, \cite{pan2021V-shuttle}, \cite{bulekov2023FUZZNG}, \cite{peng2020usbfuzz}, \cite{Syzkaller}, \cite{maier2021bsod},               \\
%                 ~                                            & ~                                   & \cite{wang2021syzvegas}, \cite{tan2023syzdirect}, \cite{pailoor2018moonshine}, \cite{Liu2023VDGuard}, \cite{Liu2023LFuzz}                                                         \\

%                 \cline{2-3}

%                 ~                                            & path coverage                       & \cite{chen2022sfuzz}, \cite{schumilo2017kafl}                                                                                                                                     \\

%                 \cline{2-3}

%                 ~                                            & \multirow{3}*{branch coverage}      & \cite{chen2020koobe}, \cite{sun2021healer}, \cite{shen2021rtkaller}, \cite{jeong2023segfuzz}, \cite{xu2020krace}, \cite{schumilo2021nyx}, \cite{henderson2017vdf},                \\
%                 ~                                            & ~                                   & \cite{myung2022mundofuzz}, \cite{bulekov2022morphuzz}, \cite{CONZZER2022context}, \cite{Hydra2020finding}, \cite{JANUS2019fuzzing}, \cite{shen2022drifuzz}, \cite{liang2020xafl}, \\
%                 ~                                            & ~                                   & \cite{wu2018fuze}, \cite{Xu2024MOCKOK}, \cite{Bulekov2024HYPERPILLFF}\\

%                 \midrule
%                 \multirow{4}*{thread coverage}               & alias coverage                      & \cite{xu2020krace}                                                                                                                                                                \\

%                 \cline{2-3}

%                 ~                                            & interleaving segment coverage       & \cite{jeong2023segfuzz}                                                                                                                                                           \\

%                 \cline{2-3}

%                 ~                                            & concurrent call pair coverage       & \cite{CONZZER2022context}                                                                                                                                                         \\

%                 \cline{2-3}

%                 ~                                            & race pair interleaving path         & \cite{Yuan2023DDRace}                                                                                                                                                             \\

%                 \midrule
%                 \multirow{6}*{custom}                        & critical object coverage            & \cite{lin2022grebe}                                                                                                                                                               \\

%                 \cline{2-3}

%                 ~                                            & state coverage                      & \cite{Hydra2020finding}, \cite{JANUS2019fuzzing}, \cite{zhao2022statefuzz}, \cite{zhao2022semantic}                                                                               \\

%                 \cline{2-3}

%                 ~                                            & action feedback                     & \cite{fleischer2023actor}                                                                                                                                                         \\

%                 \cline{2-3}

%                 ~                                            & probe model feedback                & \cite{wu2023devfuzz}                                                                                                                                                              \\

%                 \cline{2-3}

%                 ~                                            & capability feedback                 & \cite{chen2020koobe}                                                                                                                                                              \\ % &用来区分列,\\用来区分行

%                 \cline{2-3}

%                 ~                                            & DMA operation feedback              & \cite{Liu2023VDGuard}                                                                                                                                                             \\

%                 \bottomrule
%         \end{tabular}
% \end{table}

%The combination of generation and mutation aims to generate valid seeds with generation and then collaboratively update the seeds utilizing Mutation. For instance, Hydra \cite{Hydra2020finding} first uses the semantic assistance of argument types to generate syscalls (\eg \texttt{open}, \texttt{write}, \etc) and valid argument values (\eg integer values representing ranges, enumeration-type variables, \etc) specific to filesystem operations, which avoids being rejected by error-checking code at an early stage. Based on the generated syscall sequence, Hydra uses the argument-based mutator to update the arguments and add a random syscall at the end of the sequence, which ensures the state continuity of the updated syscall sequence as well as increasing the diversity of the seeds. Healer \cite{sun2021healer} utilizes static analysis to construct a relation table of syscalls from syzlang, which is used to guide the generation of new useful syscalls after a decrease in the gain of mutation. Another work that assists in syscall generation using static analysis is PrIntFuzz \cite{ma2022printfuzz}, which generates a syscall template with more information than syzlang. This template adds information for fault injection (data, fault codes, and interrupt signals) on top of syzlang, with the aim of generating syscall sequences specialized in handling faults for the mutation phase. Additionally, DriFuzz \cite{shen2022drifuzz} combines concolic and force execution to generate golden seeds for driver fuzzing in the first place, solving the problem that seeds based on mutation updates are difficult to pass through the full driver validation chain.
    

\subsubsection{Gen.+Mut.}
Recent advancements in fuzzing research have adopted a synergistic hybrid methodology integrating both generation-based and mutation-based strategies to address two critical challenges: (1) resolving efficiency bottlenecks caused by invalid initial seeds in domain-specific configurations, and (2) consistently producting variants of these effective seeds. This dual-mechanism framework demonstrates particular efficacy in directed fuzzing scenarios requiring targeted seed construction. The operational pipeline consists of two stages: the generation component synthesizes structurally valid seeds by leveraging domain-specific knowledge, while the mutation component systematically explores adjacent input spaces through combinatorial mutators.

For instance, Hydra \cite{Hydra2020finding} first uses the semantic assistance of argument types to generate syscalls (\eg \texttt{open}, \texttt{write}, \etc) and valid argument values (\eg integer values representing ranges, enumeration-type variables, \etc) specific to file system operations, which avoids being rejected by error-checking code at an early stage. Based on the generated syscall sequence, Hydra uses the argument-based mutator to update the arguments and add a random syscall at the end of the sequence, which ensures the state continuity of the updated syscall sequence as well as increasing the diversity of the seeds. \todo{HEALER \cite{sun2021healer} utilizes static analysis to construct a syscall relationship table from syzlang, which is used to directly generate syscalls that have dependencies on the current syscall sequence after mutation gains diminish, rather than continuing to execute low-yield mutation methods. Another work that employs static analysis to assist in generating syscalls is PrIntFuzz \cite{ma2022printfuzz}. It extends syzlang by adding fault injection information (such as data, fault codes, and interrupt signals) to generate syscall sequences related to fault handling. This allows the mutation algorithm to be directly applied to useful test cases, avoiding prolonged trial and error caused by relying solely on mutations. Additionally, DriFuzz \cite{shen2022drifuzz} combines concolic execution and forced execution to generate golden seeds that can pass the data validation phases in driver code. This approach addresses the issue of input rejection caused by seed update strategies based solely on mutation when faced with complex validation code logic. VideZZo \cite{Liu2023ViDeZZoDV} generates context-dependent I/O messages based on the proposed lightweight grammars, and applies the proposed 3 types of mutators with different granularities to extend the diversity of messages, \ie intra-message mutators, inter-message mutators, and group-level mutators. These 3 classes of mutators are similar to the syscall, argument-based mutation, which essentially wraps and applies its basic mutators (\eg delete, change order, bit flip) to a specific domain. BRF \cite{Hung2024BRFFT} extracts eBPF domain knowledge (\eg validator rules) and syscall dependencies to generate semantically correct eBPF programs, and then mutates the syscalls on top of the generated programs, solving the problem that seeds generated by previous kernel fuzzers are difficult to pass the eBPF validator efficiently. MOCK \cite{Xu2024MOCKOK} employs a neural network model to dynamically learn syscall sequences with context-aware dependencies to guide guide generation and mutation, which solves the problem of lack of context in the seeds updated by previous work. SATURN \cite{Xu2024Saturn} first dynamically extracts the file\_operations structure of device drivers using kallsyms and kcov, achieving precise mapping of syscall sequences (\eg \texttt{ioctl\$printer}) and their parameters (\eg valid file paths like \texttt{/dev/device\_name}). This approach addresses the inefficiency of initial seeds in USB driver fuzzing. During the mutation phase, SATURN products variants of these sequences through syscall-based and argument-based mutators, thereby exploring the input space for a class of usb driver fuzzing.}


The major overhead of the combined generation and mutation approach comes from preliminary seed analysis. This overhead is worth investing if the goal of fuzzing is to focus on specific issues in the OS, such as a single vulnerability type, a limited number of functional modules, and file operations or driver verification chains with continuous state features.


% subsection 9
\subsection{Feedback Mechanism}
Feedback is used to guide the direction of seed evolution. In our classification, feedback is categorized into code coverage, thread coverage, and custom coverage (as shown in Table \ref{tab2}). Since code coverage has been exhaustively reviewed and summarized in existing surveys \cite{Li2018FuzzingAS, Mans2018TheAS, Liang2018FuzzingSO}, we briefly discuss the acquisition of code coverage in OS and focus on kernel-related feedbacks.

\subsubsection{Code Coverage.}
Code coverage acquisition in OSF relies mainly on kcov (described in subsection \ref{subsection4.2}), as its task-level execution-based nature allows it to support more accurate collection of single syscall coverage \cite{vyukov2018kcov}. The use of kcov requires Linux version 4.6, gcc 6.1.0 and higher, or any version of Clang supported by the kernel. Therefore, the fuzzer must consider additional instrumentation methods for out-of-scope kernel versions or port kcov directly to the target version of the kernel \cite{shi2019industry}. It is worth noting that kcov is not capable of collecting all kernel coverage, and there are problems such as the kernel not being fully booted during the driver boot enumeration phase and the lack of support for soft/hard interrupt coverage acquisition.
To address this problem, \cite{schumilo2017kafl,wu2023devfuzz,ma2022printfuzz, zhao2022semantic,schumilo2021nyx} compute code coverage by using execution information such as branch instruction jumps, calls, and destination addresses recorded by the Intel PT, thus capturing instruction-level trace information during program execution.

% \begin{table} % 表格开始
% \centering % 表格居中显示
% \caption{Feedback in OSF.} % 表头标题
% \label{tab2} % 表格标签,便于引用

% \begin{tabular}{ccl} % c表示单元格内容居中,l表示靠左,这里有两列,所以cc,如果是三列就是ccc或cll,根据自己

% \toprule % 顶部的线,这里可以定义粗细、toprule{1.5ptx}
% \multicolumn{1}{m{3cm}}{\centering Feedback} &  % 中间的1.5cm表示第列宽度
% \multicolumn{1}{m{4cm}}{\centering Metric} &
% \multicolumn{1}{m{4cm}}{\centering Fuzzers}\\

% \midrule % 中间的线
% \multirow{11}*{code coverage} & statement coverage & Razzer, Agomotto, Dr.Fuzz. \\
% ~ & \multirow{3}*{basic block coverage} & Chizpurfle, IRIS, ACTOR, usb-fuzzer, \\
% ~ & ~ & PrintFuzz, BSOD, Agomotto, DriFuzz, HFL, \\
% ~ & ~ & Semfuzz, Automated Detection, KernelGPT. \\

% ~ & \multirow{3}*{edge coverage} & StateFuzz, Tardis, V-Shuttle, FUZZNG, USBFuzz, \\
% ~ & ~ & usb-fuzzer, BSOD, SyzVegas, Syzdirect, \\
% ~ & ~ & Moonshine. \\

% ~ & path coverage & Sfuzz, KAFL \\
% ~ & \multirow{3}*{branch coverage} & Koobe, HEALER, Rtkaller, Segfuzz, Krace, NYX, \\
% ~ & ~ & VDF, MundoFuzz, Morphuzz, CONZZER, Hydra, \\
% ~ & ~ & JANUS, DriFuzz, X-AFL, FUZE. \\

% \midrule
% \multirow{3}*{thread coverage} & alias coverage & Krace \\
% ~ & interleaving segment coverage & Segfuzz \\
% ~ & concurrent call pair coverage & CONZZER \\
% \midrule
% \multirow{5}*{custom coverage} & critical object coverage & GREBE \\
% ~ & state coverage & Hydra, JANUS, StateFuzz, Dr.Fuzz \\
% ~ & action feedback & ACTOR \\
% ~ & probe model feedback & DEVFUZZ \\
% ~ & capability feedback & KOOBE \\ % &用来区分列,\\用来区分行

% \bottomrule
% \end{tabular}
% \end{table}

% \begin{table}[!h] % 表格开始,添加[!h]使表格更紧凑地贴近上下文
%         \centering
%         \scriptsize
%         \caption{Feedback Categories in OSF.}
%         \vspace{-0.3cm} % 减少caption和表格之间的空隙
%         \label{tab2}
    
%         \begin{tabular}{ccl}
%             \toprule
%             \centering Feedback & Metric & Fuzzers \\
    
%             \midrule
%             \multirow{9}{*}{code coverage} 
%             & statement coverage & \cite{jeong2019razzer}, \cite{song2020agamotto}, \cite{zhao2022semantic} \\ 
%             \cline{2-3}
    
%             & \multirow{2}{*}{basic block coverage} 
%             & \cite{cesarano2023iris}, \cite{fleischer2023actor}, \cite{Syzkaller}, \cite{ma2022printfuzz}, \cite{maier2021bsod}, \cite{song2020agamotto}, \cite{Hung2024BRFFT} \\ 
%             & ~ & \cite{shen2022drifuzz}, \cite{kim2020hfl}, \cite{you2017semfuzz}, \cite{schwarz2018automated}, \cite{Jang2023ReUSB} \\ 
%             \cline{2-3}
    
%             & \multirow{2}{*}{edge coverage} 
%             & \cite{zhao2022statefuzz}, \cite{shen2022tardis}, \cite{pan2021V-shuttle}, \cite{bulekov2023FUZZNG}, \cite{peng2020usbfuzz}, \cite{Syzkaller}, \cite{maier2021bsod} \\ 
%             & ~ & \cite{wang2021syzvegas}, \cite{tan2023syzdirect}, \cite{pailoor2018moonshine}, \cite{Liu2023VDGuard}, \cite{Liu2023LFuzz} \\ 
%             \cline{2-3}
    
%             & path coverage & \cite{chen2022sfuzz}, \cite{schumilo2017kafl} \\ 
%             \cline{2-3}
    
%             & \multirow{3}{*}{branch coverage} 
%             & \cite{chen2020koobe}, \cite{sun2021healer}, \cite{shen2021rtkaller}, \cite{jeong2023segfuzz}, \cite{xu2020krace}, \cite{schumilo2021nyx}, \cite{henderson2017vdf} \\ 
%             & ~ & \cite{myung2022mundofuzz}, \cite{bulekov2022morphuzz}, \cite{CONZZER2022context}, \cite{Hydra2020finding}, \cite{JANUS2019fuzzing}, \cite{shen2022drifuzz}, \cite{liang2020xafl} \\ 
%             & ~ & \cite{wu2018fuze}, \cite{Xu2024MOCKOK}, \cite{Bulekov2024HYPERPILLFF} \\ 
%             \midrule
    
%             \multirow{4}{*}{thread coverage} 
%             & alias coverage & \cite{xu2020krace} \\ 
%             \cline{2-3}
    
%             & interleaving segment coverage & \cite{jeong2023segfuzz} \\ 
%             \cline{2-3}
    
%             & concurrent call pair coverage & \cite{CONZZER2022context} \\ 
%             \cline{2-3}
    
%             & race pair interleaving path & \cite{Yuan2023DDRace} \\ 
%             \midrule
    
%             \multirow{6}{*}{custom} 
%             & critical object coverage & \cite{lin2022grebe} \\ 
%             \cline{2-3}
    
%             & state coverage & \cite{Hydra2020finding}, \cite{JANUS2019fuzzing}, \cite{zhao2022statefuzz}, \cite{zhao2022semantic} \\ 
%             \cline{2-3}
    
%             & action feedback & \cite{fleischer2023actor} \\ 
%             \cline{2-3}
    
%             & probe model feedback & \cite{wu2023devfuzz} \\ 
%             \cline{2-3}
    
%             & capability feedback & \cite{chen2020koobe} \\ 
%             \cline{2-3}
    
%             & DMA operation feedback & \cite{Liu2023VDGuard} \\ 
%             \bottomrule
%         \end{tabular}
%     \end{table}

\begin{table}
    \centering
    \scriptsize % 减小字体大小
    \caption{Feedback Categories in OSF.}
    \label{tab2}
    \vspace{-0.3cm}
    \begin{tabular}{lll}
    \toprule
    \multicolumn{1}{m{1.5cm}}{\raggedright Feedback} &  % 中间的1.5cm表示第列宽度
    \multicolumn{1}{m{2.5cm}}{\raggedright Metric} &
    \multicolumn{1}{m{6.5cm}}{\raggedright Fuzzers}\\
    %Fuzzers & File System & Input Type & Status \\
    \midrule
    \multirow{4}*{Code Cov.}
    & Statement Cov. & \cite{jeong2019razzer,song2020agamotto,song2020agamotto} \\

    ~ & Basic Block Cov. & \cite{cesarano2023iris,fleischer2023actor,Syzkaller,ma2022printfuzz,maier2021bsod,song2020agamotto,Hung2024BRFFT,shen2022drifuzz,kim2020hfl,you2017semfuzz,schwarz2018automated,Pustogarov2020ExvivoDA,yang2023kernelgpt,Jang2023ReUSB, Huster2024ToBoldly} \\

    ~ & Edge Cov. & \cite{zhao2022statefuzz,shen2022tardis,Song2019PeriScopeAE,pan2021V-shuttle,bulekov2023FUZZNG,peng2020usbfuzz,Syzkaller,maier2021bsod,wang2021syzvegas,Pustogarov2020ExvivoDA,tan2023syzdirect,pailoor2018moonshine,Liu2023VDGuard,Liu2023LFuzz}\\ 

    ~ & Path Cov. &  \cite{schumilo2017kafl} \\ 

    ~ & Branch Cov. & \cite{chen2020koobe,sun2021healer,shen2021rtkaller,jeong2023segfuzz,xu2020krace,schumilo2021nyx,henderson2017vdf,myung2022mundofuzz,bulekov2022morphuzz,CONZZER2022context,Hydra2020finding,JANUS2019fuzzing,shen2022drifuzz,liang2020xafl,wu2018fuze,Xu2024MOCKOK,Xu2024Saturn,Bulekov2024HYPERPILLFF} \\ 
    \hline

    \multirow{4}*{Thread Cov.}
    & Alias Cov. & \cite{xu2020krace} \\

    ~ & Interleaving Segment Cov. & \cite{jeong2023segfuzz} \\ 

    ~ & Concurrent Call Pair Cov. & \cite{CONZZER2022context} \\

    ~ & Race Pair Interleaving Path & \cite{Yuan2023DDRace} \\
    \hline
    
    \multirow{6}*{Custom}
    & Critical Object Cov. & \cite{lin2022grebe} \\

    ~ & State Cov. &  \cite{Hydra2020finding,JANUS2019fuzzing,zhao2022statefuzz,zhao2022semantic} \\

    ~ & Action Feedback & \cite{fleischer2023actor} \\

    ~ & Probe Model Feedback & \cite{wu2023devfuzz} \\

    ~ & Capability Feedback & \cite{chen2020koobe} \\

    ~ & DMA Operation Feedback & \cite{Liu2023VDGuard} \\

    \bottomrule
    \end{tabular}
\end{table}


\begin{figure*}[!t]
        \centering
        \subfigure[No AIP exists in this thread interleaving context.]{
                \includegraphics[width=0.45\linewidth]{img/AIP_1.pdf}\label{img:AIP_1}}
        \hspace{10pt}\vline\hspace{10pt}
        \subfigure[An AIP ($wi\_3 \rightarrow ri\_2$) exists in this thread interleaving context.]{
                \includegraphics[width=0.45\textwidth]{img/AIP_2.pdf}\label{img:AIP_2}}
        \caption{An example of alias instruction pair coverage.}
\end{figure*}

\subsubsection{Thread Coverage}
Similar to thread mutation discussed in Subsection \ref{Update}, the discovery of concurrency vulnerabilities using solely code coverage as the only feedback mechanism is also limited. Therefore, it is necessary to dedicate a feedback mechanism related to thread Interleaving to guide the fuzzer in generating the seed that triggers the data race vulnerability. Existing works \cite{xu2020krace,CONZZER2022context,jeong2023segfuzz,Yuan2023DDRace} present distinct thread-related coverage metrics:

\cite{xu2020krace} proposes alias instruction pair coverage, or called AIP coverage (the first feedback mechanism related to thread coverage in OSF), as a metric of the degree of thread interleaving. An alias instruction pair is defined as a pair of instructions to read or write the same memory address in two concurrent threads, which is formalized using $A\leftarrow(wi\_x,t\_1)$ to indicate that there exists a $wi\_x$ instruction to write memory $A$ in thread $t\_1$, and that if there exists a $ri\_x$ instruction to read memory $A$ in another thread, \ie $A\leftarrow(ri\_y,t\_2)$, then ($wi\_x \rightarrow ri\_y$) denotes an alias instruction pair. For example, Figure \ref{img:AIP_1} and \ref{img:AIP_2} illustrates a group of memory access instructions and 2 threads, where the data race vulnerability to the shared data \texttt{V[length]} is triggered if and only if \texttt{A=C}. Figure \ref{img:AIP_1} and \ref{img:AIP_2} represent 2 different thread interleaving contexts, where (a) indicates that no alias instruction pair is covered due to the thread interleaving process does not read or write to the same memory address sequentially, and (b) indicates that the instruction pair ($wi\_3 \rightarrow ri\_2$) is covered due to the $ri\_3$ and \texttt{B=A+2} ($ri\_2$) read or write to the memory pointed to by \texttt{B} at the same time. A new syscall sequence is generated when the AIP coverage is no longer growing, allowing the fuzzer to focus on exploring the thread concurrency of the new syscall sequence at each time instead of focusing only on the sequential execution of the syscall.

However, although \cite{xu2020krace} is possible to fully interleave AIPs, it is possible that race condition misses due to ignoring the runtime context (\eg call stack, state of variables and data structures, \etc) of the instruction. For instance, the value of \texttt{A} could be dynamically determined by the context, \ie the instruction $wi\_1$ in Figure \ref{img:AIP_1} has the possibility of writing 2 to the memory address of \texttt{A}. Therefore, the opportunity to discover the data race vulnerability is missed by using AIP coverage. To solve this problem, \cite{CONZZER2022context} measures the thread interleaving using concurrency call pair coverage (CCP coverage) that records the call contexts of two function. Specifically, a CCP contains \texttt{func a} and \texttt{func b} call contexts $(CallCtx\_a1, CallCtx\_b1)$. For example, in thread 1 of Figure \ref{img:AIP_1}, $CallCtx\_a1 = (main \rightarrow ModuleX \rightarrow FuncA)$ determines $A =1$. If there is also another $CCP=(CallCtx\_a2, CallCtx\_b2)$ where $CallCtx\_a2 = (main \rightarrow ModuleZ \rightarrow FuncA)$, it could be explored for a different value of \texttt{A}, \eg $A=2$. In this case, CPP coverage explores more interleaved combinations of instructions through the call context, thus reducing the probability of a data race vulnerability.


\begin{figure*}[!t]
        \centering
        \subfigure[A pair of concurrent threads.]{
                \includegraphics[width=0.30\linewidth]{img/IS_1.pdf}\label{img:IS_1}}
        \hspace{10pt}\vline\hspace{10pt}
        \subfigure[An instance of the interleaving segment of (a).]{
                \includegraphics[width=0.23\textwidth]{img/IS_2.pdf}\label{img:IS_2}}
        \hspace{10pt}\vline\hspace{10pt}
        \subfigure[The possible thread interleaving using IS.]{
                \includegraphics[width=0.23\linewidth]{img/IS_3.pdf}\label{img:IS_3}}
        \caption{An example of interleaving segment coverage.}
\end{figure*}

Interleaving segment coverage (IS coverage) \cite{jeong2023segfuzz} reduces the exponential search space of thread interleaving and concentrates on exploring interleaved accesses to a memory address by narrowing the instruction set of interleaving to segments. Figure \ref{img:IS_1} and \ref{img:IS_2} demonstrate an uninitialized access bug of \texttt{\&v} through the interleaving segment. In contrast to \cite{xu2020krace} and \cite{CONZZER2022context}, IS focuses on a minimum instruction tuple each time, \ie performs \texttt{(read,read,write)} operations on the same address. As shown in Figure \ref{img:IS_3}, in this minimum instruction tuple, the uninitialized access bug is missed (failing to cover \texttt{\#S2}) because AIP incorrectly infers that $(wi\_3 \rightarrow ri\_1)$ and $(wi\_3 \rightarrow ri\_2)$ are identical in \texttt{\#S1}. Likewise, function-level based CCP could also miss \texttt{\#S2}, this is because CCP keeps track of simultaneously executing function pairs and is not aware of fine-grained interleaving at the instruction level. Overall, IS coverage reveals a systematic and efficient thread coverage. In the discussion of Subsection \ref{Update}, however, IS coverage could miss more complex concurrency scenarios due to its limitation of interleaving up to 4 instructions.

In addition, in the study of fuzzing special concurrency vulnerabilities (\eg concurrency UAF vulnerability), \cite{Yuan2023DDRace} proposes a new metric based on AIP, Race Pair Interleaving Path (RPIP), to trace the target side (\ie the potential UAF code areas) in multiple race pairs and value changes. Specifically, RPIP complements AIP by adding detail information, \ie an instruction tuple \texttt{(instr. type, thread no., shared var. value)}, to trace the accesses of multiple shared variables simultaneously. Meanwhile, all traced AIPs are categorized by shared variables and merged into the interleaving path of shared variables, which is used to measure the interleaving situation of shared variables in the target side. Although RPIP is used to guide fuzzing to uncover concurrency UAF vulnerability, the metric is still applicable to other concurrency vulnerabilities such as concurrency null pointer deference vulnerability. It is important to note that RPIP could have path explosion problem in the huge kernel code base, so this type of vulnerability usually relies on program analysis to extract the potential vulnerability side in advance.


%Since traditional code-coverage-driven fuzzing suffers from inefficiency and ineffectiveness in directed fuzzing (especially kernel bug exploitation), OSF at this stage aims to customize a new feedback metric to address these challenges. Prior to designing feedback mechanism, it is necessary to extract its structural information, which can be implemented using existing tools \cite{Bai2019EffectiveSA, Caballero2012UndangleED, Chipounov2011S2EAP, Shoshitaishvili2016SOKO}. As shown in Table \ref{tab2}, the existing custom feedback in the OSF includes both custom coverage and model-driven feedback.
\subsubsection{Custom Coverage.}
\todo{Traditional code coverage-driven fuzzing faces inefficiency and ineffectiveness in directed fuzzing (\eg kernel vulnerability exploitation fuzzing). To address these issues, existing OSF approaches have introduced various custom feedback mechanisms. Generally, custom feedback is highly correlated with the characteristics of the fuzzing task, which can be enhanced by incorporating other techniques, such as program analysis \cite{Bai2019EffectiveSA, Caballero2012UndangleED, Chipounov2011S2EAP, Shoshitaishvili2016SOKO}, into the fuzzing process. As shown in Table \ref{tab2}, the custom feedback mechanisms in OSF include custom coverage and domain-aware feedback.}

Custom coverage is a quantifiable feedback metric involving critical object (CO) coverage and State coverage, where CO coverage refers to the rate of hitting critical kernel objects in the crash report that can trigger a vulnerability (CO can be identified by backward taint analysis), which addresses the problem that code-coverage-driven feedback is limited to discovering the multiple error behaviors of a bug. State coverage instantiates the context in which the PUT is running, \eg Monitor's feedback \cite{Hydra2020finding}, file system metadata properties \cite{JANUS2019fuzzing}, state space of variables \cite{zhao2022statefuzz}, and function error codes \cite{zhao2022semantic}. These works enable directed fuzzing by quantifying the context related to the target under test as state coverage.

\todo{In contrast to coverage mechanisms, there is a category of feedback mechanisms called domain-aware feedback. Domain-aware refers to a domain knowledge built specifically to tackle a type of challenge, and is used to enhance the depth of fuzzing on such challenges.} For example, Action \cite{fleischer2023actor} refers to a 3-tuples \texttt{(Action type, addr., size)} dedicated to recording memory operations, revealing the specific syscall sequence that triggers a memory-related vulnerability by focusing on higher semantic memory operations. Probe model \cite{wu2023devfuzz} automatically generates seeds that can guide a fuzzer to properly initialize a device driver under test (DUT) through symbolic execution, which in turn effectively reach and test the functional code of the driver. Capability \cite{chen2020koobe} refers to the attributes of a specific vulnerability (\eg memory violation vulnerability), including: write address, size, and value. Specifically, Capability-driven fuzzing takes a PoC as input and analyzes the path and trigger conditions leading to an Out-Of-Bound (OOB) vulnerability via binary symbolic execution \cite{Chipounov2011S2EAP} to guide the seed mutation and generation. DMA operation feedback \cite{Liu2023VDGuard} indicates whether a DMA-related function is hit to guide a seed mutation, which is because vulnerabilities in hypervisor are mainly distributed in DMA-related operations. Therefore, DMA operation feedback is also typically an important feedback indicator for hypervisor fuzzing.

In summary, while code-coverage-driven fuzzing has achieved significant success, relying solely on code coverage as feedback is insufficient for OSF to effectively reach the expected testing code areas. Therefore, designing feedback relevant to the research question is an essential step. Last but not least, it does not mean that code coverage loses its significance, but instead remaining it as an infrastructure for OSF and fine-tuning it in combination with the use of new coverage metrics is the dominant research trend at this stage.

\subsection{Vulnerability Analysis}
\label{Vulnerability}

\subsubsection{Types}

Vulnerabilities found in OSFs are categorized into four types, as shown in Table \ref{tab1}, including memory violation, concurrency bug, logic bug, and \todo{other}.

Memory violation involves illegal access and operation of memory, including spatial and temporal vulnerabilities. Spatial memory errors refer to memory accesses that are outside the scope of their allocation, such as OOB and buffer overflow, \etc These errors occur when a program attempts to access a memory address that is outside the legal memory zone, which can cause the program to crash or incur undefined behavior. Temporal memory errors occur when memory is used at an incorrect moment, \eg UAF. Memory violation vulnerabilities are the most common type of error in OSFs, primarily because the C/C++ language provides direct control over the underlying memory management but lacks automated security checking mechanisms \cite{Unicorefuzz2019,chen2020koobe}. Thus, memory violation is also a major target of the current OSF and exploitation research.

Concurrency bug refers to the lack of proper synchronization mechanisms in multi-threaded or multi-process environments, which can lead to data inconsistency, deadlock, data race, double-fetch, and other race conditions, \etc Unlike userland fuzzing, the efficiency of exploring concurrency bugs in OS is limited by the large code base and complexity of the kernel, which makes it difficult to sufficiently detect various types of concurrency bugs, which could lead to memory violation errors (\eg traditional buffer overflows, use-after-free, \etc), resulting in serious kernel security attacks \cite{jeong2019razzer}.

Another category of vulnerabilities are kernel logic errors, which refer to bugs caused by incorrect coding logic that can stem from programmer error, incorrect algorithms, or incomplete, and may lead to problems such as outputting incorrect results, entering infinite loops, or skipping critical operations. For example, virtual devices entering infinite loops when receiving invalid data (Invalid Data Transfer) \cite{henderson2017vdf,pan2021V-shuttle,Liu2023VDGuard}, functions being called multiple times when not designed to be reentrant resulting in data inconsistencies (Reentrant Problems) \cite{myung2022mundofuzz}, multiple releases or incorrect formatting in interrupt request handling (Double IRQ Free and request irq format error) \cite{wu2023devfuzz}, attempts to disable already disabled devices \cite{wu2023devfuzz}, divide-by-zero operations \cite{pan2021V-shuttle}, and assertion errors in the \texttt{BUG\_ON} macro \cite{wu2023devfuzz,2017DIFUZE,pailoor2018moonshine,schumilo2020hyper}.

\todo{In addition to the aforementioned three categories of vulnerabilities, there are other relatively rare but equally critical OS vulnerabilities.}

\todo{\textbf{Privilege Protection Vulnerabilities} involve unauthorized operations or illegal access to protected resources, potentially leading to sensitive data leakage or compromising system integrity. Typical examples include vulnerabilities where user space attempts to access high-privilege memory regions within the kernel (User Memory Access \cite{wu2023devfuzz,2017DIFUZE}), denial of service vulnerabilities that prevent the kernel from providing its normal services (Kernel DoS \cite{schumilo2020hyper}), and vulnerabilities related to illegal writes to non-volatile memory (Writing to Non-Volatile Memory \cite{2017DIFUZE}).}

\todo{\textbf{Processor Exceptions} occur when the processor attempts to execute undefined instructions or when the instruction encoding or structure does not conform to processor specifications, leading to abnormal system termination. Common examples of such vulnerabilities include Invalid Opcode \cite{wu2023devfuzz} and General Protection Faults (GPFs) \cite{sun2021healer,lin2022grebe,maier2021bsod,zhao2022semantic}.}

\todo{\textbf{Data Integrity Vulnerabilities} refer to instances where data in the system becomes corrupted or inconsistent. This typically occurs when data is inadvertently modified during storage or transmission, resulting in unusable or incorrect data states, as seen in cases of Data Corruption \cite{pailoor2018moonshine}.}

\subsubsection{Monitor}
In terms of vulnerability monitoring, Muench et al. \cite{Muench2018Avatar2AM} track and categorize observable crashes and hangs into different classes, but this approach makes it difficult to detect vulnerabilities that do not immediately trigger a crash, such as buffer overflows. Another solution typically uses existing sanitizer\cite{Song2018SoKSF}tools for monitoring, such as KASAN\cite{KernelAddressSanitizer2019}, UBSAN \cite{UBsan}, ASan \cite{ASAN}, MSan \cite{MemorySanitizer2020}, \etc It is worth noting that KASAN may miss out-of-bounds accesses between neighboring legal memory regions or memory objects. For the situation where the sanitizer is not available (\eg Hypervisor \cite{henderson2017vdf,schumilo2020hyper,schumilo2021nyx}, Android OS \cite{2017DIFUZE}), it is generally required to record a sequence of crash-causing test cases entered into the target OS, and later on analysts debug to identify and categorize the bugs of the type.

For concurrency bug monitoring, the strategy of prior works \cite{Chen2020MUZZTG,Johansson2018RandomTW,Vinesh2019ConFuzzACF} is to integrate third-party checkers (\eg TSan \cite{TSan} and KCSAN \cite{KCSAN}) in fuzzing. However, such third-party checkers have been reported with many false positives \cite{CONZZER2022context} due to the omission of special synchronization primitives such as message queues and conditional variables \cite{ThreadSanitizerDR}. To improve the precision of monitor for concurrency-related vulnerabilities, customization of monitor using dynamic lockset analysis \cite{Savage1997EraserAD,jeong2019razzer,xu2020krace,jeong2023segfuzz}, happens-before analysis \cite{happensbefore}, and lockdep \cite{lockdep} to customize the monitor is the current mainstream solution.

%For the detection of such errors, Muench et al. \cite{Muench2018Avatar2AM} track and categorize observable crashes and hangs into different classes, but this approach makes it difficult to detect vulnerabilities that do not immediately trigger a crash, such as buffer overflows. Another solutions typically utilizes sanitizers \cite{Song2018SoKSF} to enable monitoring of memory violation vulnerabilities, but sanitizers such as ASAN \cite{ASAN}, MSAN \cite{MemorySanitizer2020}, \etc require special modifications for use in the kernel. Therefore, the dominant approach is to use KASAN \cite{KernelAddressSanitizer2019} embedded in the kernel. It is worth noting that KASAN may miss out-of-bounds accesses between neighboring legal memory regions or memory objects. For situations where KASAN is not available (\eg Hypervisor \cite{henderson2017vdf,schumilo2020hyper,schumilo2021nyx}, Android OS \cite{2017DIFUZE,Chizpurfle2019}), it is generally required to record a sequence of crash-causing test cases entered into the target OS, and later on analysts debug to identify and categorize the bugs of the type.



% !TeX root = ../main.tex

\section{Distinctive Fuzzing Characteristics among OS Layers} \label{Section5}
In this section, we discuss the characteristics of fuzzing in different system layers of an operating system (\ie kernel, file system, driver, and hypervisor) by using the PUT's interaction interface as a focal point. This discussion provides practitioners with a quick overview of the inherent problems and potential solutions associated with the adoption of fuzzing techniques in different system layers.


\subsection{Kernel Fuzzing}
Kernel fuzzer utilizes syscall as the interaction interface. \todo{Since the number of kernel functions far exceeds the number of syscalls and there is no one-to-one mapping between them, different uses of syscalls and their arguments can trigger entirely different kernel functions. In other words, improper use of syscalls is likely to deviate from the intended target, leading to unexpected crashes or producing invalid test results. To conduct an effective kernel fuzzing campaign, it is crucial to ensure the syntactical correctness of syscall sequences and argument usage, while also ensuring semantic validity. Therefore, kernel fuzzing typically encounter the problems of dependency inference (syntax), API polymorphism (semantics), and argument inference (semantics).} In Table \ref{tab_kernel}, we indicate which kernel fuzzers have specifically focused on these three problems, and we discuss them in detail as follows.


%Kernel fuzzer utilizes syscall as the interaction interface, thus the use of syscall is the key influence on the effectiveness. Typically, the problems of dependency inference, API polymorphism, and argument inference are encountered when performing kernel fuzzing. In Table \ref{tab_kernel}, we indicate which kernel fuzzers have specifically focused on these three problems, and we discuss them in detail as follows.

% \begin{table}
%   \footnotesize % 减小字体大小
%   \caption{Kernel Fuzzers.}
%   \label{tab_kernel}
%   \begin{tabular}{ccccc}
%   \toprule
%   Fuzzers & OS & Dependency Inference & API Polymorphism & Argument Inference \\
%   \midrule
%   Trinity\cite{Trinity} & L & - & - & -\\
%   PERF\cite{Weaver2015perf} & L & - & - & -\\
%   Syzkaller\cite{Syzkaller} & L & \fullcirc(1) & - & - \\
%   KernelFuzzer\cite{KernelFuzzer2016} & L & - & - & - \\
%   Moonshine\cite{pailoor2018moonshine} & L & \fullcirc(1) & \fullcirc & - \\
%   FUZE\cite{wu2018fuze} & L & - & - & - \\
%   Schwarz et al.\cite{schwarz2018automated} & L & - & - & ? \\
%   Razzer\cite{CONZZER2022context} & L & - & - & - \\
%   SLAKE\cite{Liu2023LFuzz} & L & - & - & -\\
%   Shi et al.\cite{shi2019industry} & L & - & - & - \\
%   Unicorefuzz\cite{Unicorefuzz2019} & L & - & - & - \\
%   KOOBE\cite{chen2020koobe} & L & - & - & - \\
%   HFL\cite{kim2020hfl} & L & \fullcirc(1) \fullcirc(2) & \fullcirc & \fullcirc \\
%   X-AFL\cite{liang2020xafl} & A & \fullcirc(1) \fullcirc(2) & - & - \\
%   HEALER\cite{sun2021healer} & L & \fullcirc(1) & - & - \\
%   Rtkaller\cite{shen2021rtkaller} & R & - & - & -\\
%   SyzVegas\cite{wang2021syzvegas} & L & - & - & -\\
%   GREBE\cite{lin2022grebe} & L & - & ? & - \\
%   Hao et al.\cite{Hao2022DemystifyingTD} & L & \fullcirc(1) \fullcirc(2) & - & - \\
%   KSG\cite{sun2022ksg} & L & - & \fullcirc & \fullcirc \\
%   Syzscope\cite{zou2022syzscope} & L & - & - & - \\
%   Tardis\cite{shen2022tardis} & U F R Z & - & - & - \\
%   Segfuzz\cite{jeong2023segfuzz} & L & - & - & - \\
%   FUZZNG\cite{bulekov2023FUZZNG} & L & - & - & \fullcirc \\
%   ACTOR\cite{fleischer2023actor} & L & - & - & - \\
%   Syzdirect\cite{tan2023syzdirect} & L & - & \fullcirc & - \\
%   KernelGPT\cite{yang2023kernelgpt} & L & - & -& \fullcirc\\
%   \bottomrule 
%   \end{tabular}
%   \\ An empty cell means that this aspect was not mentioned in the paper;  a ``?'' means that the element was mentioned, but the details are unclear. \textbf{OS}: ``A'' means Android, ``L'' means Linux, ``U'' means UC/OS, ``F'' means FreeRTOS, ``R'' means Rt-Thread, ``Z'' means Zephyr. \textbf{Dependency Inference}: \fullcirc(1) means considering explicit dependency, \fullcirc(2) means considering implicit dependency. \textbf{API Polymorphism}: \fullcirc means considering API Polymorphism. \textbf{Argument Inference}: \fullcirc means considering Argument Inference.
% \end{table}
% \begin{table}
%   \footnotesize % 减小字体大小
%   \caption{Kernel Fuzzers.}
%   \label{tab_kernel} 
%   \vspace{-0.3cm}
%   \begin{tabular}{ccccc}
%   \toprule
%   Fuzzers & OS & Dependency Inference & API Polymorphism & Argument Inference \\
%   \midrule
%   Trinity\cite{Trinity} & L &  &  & \\
%   PERF\cite{Weaver2015perf} & L &  & & \\
%   Syzkaller\cite{Syzkaller} & L & Explicit & & \\
%   KernelFuzzer\cite{KernelFuzzer2016} & L & Explicit &  &  \\
%   Moonshine\cite{pailoor2018moonshine} & L & Explicit & \fullcirc &  \\
%   FUZE\cite{wu2018fuze} & L & - & - & - \\
%   Schwarz et al.\cite{schwarz2018automated} & L &  &  & ? \\
%   Razzer\cite{CONZZER2022context} & L &  &  &  \\
%   SLAKE\cite{Liu2023LFuzz} & L &  &  & \\
%   Shi et al.\cite{shi2019industry} & L &  &  &  \\
%   Unicorefuzz\cite{Unicorefuzz2019} & L &  & &  \\
%   KOOBE\cite{chen2020koobe} & L &  &  &  \\
%   HFL\cite{kim2020hfl} & L & Explicit, Implicit & \fullcirc & \fullcirc \\
%   X-AFL\cite{liang2020xafl} & A & Explicit, Implicit &  &  \\
%   HEALER\cite{sun2021healer} & L & Explicit &  &  \\
%   Rtkaller\cite{shen2021rtkaller} & R &  &  & \\
%   SyzVegas\cite{wang2021syzvegas} & L & - & - & -\\
%   GREBE\cite{lin2022grebe} & L & - & ? & - \\
%   Hao et al.\cite{Hao2022DemystifyingTD} & L & Explicit, Implicit &  &  \\
%   KSG\cite{sun2022ksg} & L & - & \fullcirc & \fullcirc \\
%   Syzscope\cite{zou2022syzscope} & L & - & - & - \\
%   Tardis\cite{shen2022tardis} & U F R Z &  &  &  \\
%   Segfuzz\cite{jeong2023segfuzz} & L &  &  &  \\
%   FUZZNG\cite{bulekov2023FUZZNG} & L &  &  & \fullcirc \\
%   ACTOR\cite{fleischer2023actor} & L &  &  &  \\
%   Syzdirect\cite{tan2023syzdirect} & L &  & \fullcirc &  \\
%   BRF\cite{Hung2024BRFFT} & L & Dependency & & \fullcirc\\
%   MOCK\cite{Xu2024MOCKOK} & L\\
%   \bottomrule 
%   \end{tabular}
%   \begin{flushleft}
%   \justifying % 使脚注文字左右对齐
%   An empty cell means that this aspect was not mentioned in the paper; a ``-'' means it is not relevant; a ``?'' means that the element was mentioned, but the details are unclear. \textbf{OS}: ``A'' means Android, ``L'' means Linux, ``U'' means UC/OS, ``F'' means FreeRTOS, ``R'' means Rt-Thread, ``Z'' means Zephyr. \textbf{Dependency Inference}: \fullcirc(1) means considering explicit dependency, \fullcirc(2) means considering implicit dependency. \textbf{API Polymorphism}: \fullcirc means considering API Polymorphism. \textbf{Argument Inference}: \fullcirc means considering Argument Inference.
%   \end{flushleft}
% \end{table}

\begin{table}
  \scriptsize % 减小字体大小
  \caption{Kernel Fuzzers Sorted by Publication Year.}
  \label{tab_kernel} 
  \vspace{-0.3cm}
  \begin{tabular}{ccccc}
  \toprule
  Fuzzers & OS & Dependency Inference & API Polymorphism & Argument Inference \\
  \midrule
  Trinity\cite{Trinity} & Linux & - & - & -\\
  %PERF\cite{Weaver2015perf} & Linux & - & - & - \\
  Syzkaller\cite{Syzkaller} & Linux & DSL(ED) & - & - \\
  KernelFuzzer\cite{KernelFuzzer2016} & Linux & DSL(ED) & - & - \\
  Moonshine\cite{pailoor2018moonshine} & Linux & G\&CFA(ED,ID) & - & - \\
  FUZE\cite{wu2018fuze} & Linux & - & - & - \\
  Schwarz et al.\cite{schwarz2018automated} & Linux & - & - & - \\
  Razzer\cite{CONZZER2022context} & Linux & - & - & - \\
  SLAKE\cite{Liu2023LFuzz} & Linux & - & - & -\\
  Shi et al.\cite{shi2019industry} & Linux & - & - & - \\
  Unicorefuzz\cite{Unicorefuzz2019} & Linux & - & - & - \\
  KOOBE\cite{chen2020koobe} & Linux & - & - & - \\
  HFL\cite{kim2020hfl} & Linux & PTA(ED,ID) & DFA\&PTA & CE(NS) \\
  X-AFL\cite{liang2020xafl} & Android & - & - & - \\
  HEALER\cite{sun2021healer} & Linux & G(ED) & - & - \\
  Rtkaller\cite{shen2021rtkaller} & RT-Linux & - & - & -\\
  SyzVegas\cite{wang2021syzvegas} & Linux & - & - & -\\
  GREBE\cite{lin2022grebe} & Linux & - & - & - \\
  Hao et al.\cite{Hao2022DemystifyingTD} & Linux & M(ED,ID) & - &- \\
  KSG\cite{sun2022ksg} & Linux & - & DH & SE(TC) \\
  Syzscope\cite{zou2022syzscope} & Linux & - & - & - \\
  Tardis\cite{shen2022tardis} & \makecell{UC/OS, FreeRTOS, \\ RT-Thread, Zephyr} & - & - & - \\
  Segfuzz\cite{jeong2023segfuzz} & Linux & - & - & - \\
  FUZZNG\cite{bulekov2023FUZZNG} & Linux & - & - & H\&M(NS) \\
  ACTOR\cite{fleischer2023actor} & Linux & - & - & - \\
  Syzdirect\cite{tan2023syzdirect} & Linux & - & SA\&ICA & - \\
  KernelGPT\cite{yang2023kernelgpt} & Linux & - & - & LLM(TC) \\
  BRF\cite{Hung2024BRFFT} & Linux & H(ED,ID) & - & -\\
  MOCK\cite{Xu2024MOCKOK} & Linux & NN(ED,ID) & - & -\\
  \bottomrule 
  \end{tabular}
  \begin{flushleft}
  \justifying % 使脚注文字左右对齐
  A ``-'' means it is irrelevant, not mentioned, or unclear in detail. \todo{\textbf{Abbreviation}: In the form ``A(B)'', the fuzzer resolves a B-type issue using the A technique. In the form ``A'', the fuzzer employ A technique address the issue of API Polymorphism.\textbf{``A''} in ``A(B)'': \textbf{DSL}: Domain Specific Language, \textbf{G\&CF}: Graph Structure and Control Flow Analysis, \textbf{PTA}: Points-to Analysis, \textbf{G}: Graph Structure, \textbf{M}: Manual Checking, \textbf{H}: Heuristc Method, \textbf{NN}: Neural Network, \textbf{CE}: Concolic Execution, \textbf{SE}: Symbolic Execution, \textbf{H\&M}: Hooking and Mapping, \textbf{LLM}: Large Language Model. \textbf{B} in ``A(B)'': \textbf{ED}: Explicit Dependency, \textbf{ID}: Implicit Dependency, \textbf{NS}: Nested Structure, \textbf{TC}: Type Casting. \textbf{``A''}: \textbf{SA}: Static Analysis, \textbf{DFA\&PTA}: Data Flow Analysis and Points-to Analysis, \textbf{DH}: Dynamic Hooking, \textbf{SA\&ICA}: Static Analysis and Indirect Call Analysis.}
  \end{flushleft}
\end{table}

%A -'' means it is irrelevant, not mentioned, or unclear in detail. \textbf{Abbreviation}: the form ``A(B)'' indicates that fuzzer solves the B-type dependency using the ``A'' technique. \textbf{``A''} in ``A(B)'': ``DSL'' is the abbreviation of Domain Specific Language. ``T'' is marked as Trace. ``PTA'' is marked as Points-to Analysis. ``H'' means heuristic method. ``RG'' means Relation Graph. ``M'' is marked as manual checking. ``DFA\&PTA'' means Data Flow Analysis and Points-to Analy. ``DH'' is marked as Dynamic Hooking. ``SA\&ICA'' means Static Analysis and Indirect Call Analysis. ``CE'' is abbreviation of Concolic Execution. ``SE'' is marked as Symbolic Execution. ``H\&M'' means Hooking and Mapping. \textbf{``B''} in ``A(B)'': ``ED'' and ``ID'' is short for Explicit Dependency and Implicit Dependency, respectively. ``NS'' is marked as Nested Structure. ``TC'' is abbreviation of Type Casting.

%An empty cell means that this aspect was not mentioned in the paper; a ``-'' means it is not relevant; a ``?'' means that the element was mentioned, but the details are unclear. \textbf{OS}: ``A'' means Android, ``L'' means Linux, ``U'' means UC/OS, ``F'' means FreeRTOS, ``R'' means Rt-Thread, ``Z'' means Zephyr. \textbf{Dependency Inference}: \fullcirc(1) means considering explicit dependency, \fullcirc(2) means considering implicit dependency. \textbf{API Polymorphism}: \fullcirc means considering API Polymorphism. \textbf{Argument Inference}: \fullcirc means considering Argument Inference.


\subsubsection{Dependency inference.}
Kernel fuzzers typically start with a set of syscall sequences and continuously change the arguments and order of these syscalls using random mutations. In order to reduce randomness and enhance the effectiveness of kernel fuzzers, it is essential to infer potential dependencies between syscalls to construct correct sequences. These dependencies include explicit and implicit dependency. \todo{Explicit dependence refers to the relationship that the result produced by one syscall \( S_i \) is used directly as input to another syscall \( S_j \). For example, a \texttt{write} syscall must be preceded by an \texttt{open} syscall, or the return value of one syscall may serve as the input argument for another. Implicit dependency is defined as the another relationship that a syscall \( S_i \) affecting the execution of a syscall \( S_j \) through some shared data structure in the kernel, even though there is no direct result transfer between \( S_i \) and \( S_j \). A classic example of an implicit dependency is when \( S_i \) operates on a shared kernel variable, affecting the outcome of subsequent syscalls, \( S_j \), where i and j can be any value.}

\todo{Handling explicit dependencies is typically a straightforward task. Syzkaller \cite{Syzkaller}, KernelFuzzer \cite{KernelFuzzer2016} and HEALER \cite{sun2021healer} effectively address explicit dependencies by employing hardcoded rules to capture the return results generated by syscalls. However, these approaches only describe the correct syntax of individual syscalls, lacking information about interactions between multiple syscalls. As a result, they exhibit inherent limitations in recognizing implicit dependencies.}

%Syzkaller \cite{Syzkaller} can address the problem of explicit dependencies, this is because syzlang captures these potential relationships in hard-coded rules. In order to not rely on manual work, some works focusing on automatic seed generation must proactively address this problem. 

\todo{To simultaneously identify both explicit and implicit dependencies, Moonshine \cite{pailoor2018moonshine} constructs a dependency graph consisting of two types of nodes: syscall return values and arguments (implemented using customized Strace \cite{Strace} to trace syscall sequences). For example, an edge from a argument node \(a\) to a result node \(r\) represents that the value of \(a\) depends on the syscall that produced \(r\). To capture implicit dependencies, Moonshine performs control flow analysis to examine the read and write operations on global variables between syscalls, thereby identifying implicit dependencies. HFL \cite{kim2020hfl} uses points to analysis to obtain candidate explicit and implicit dependency pairs, and identifies data flow by symbolic execution, which reduces Moonshine's problem of missing implicit dependencies due to aliased variables. Hao et al. \cite{Hao2022DemystifyingTD} develop a measurement pipeline that quantifies the severity of dependencies and reveals the multiple causes of dependencies through manual analysis. Although effective, the manual-based analysis only focuses on specific linux kernel modules and fails to cover other critical modules such as drivers, file systems, \etc To extract the dependency relationships between eBPF programs and related system calls (such as \texttt{BPF\_MAP\_CREATE}), BRF \cite{Hung2024BRFFT} employs manual source code analysis to build an automated dependency parsing script with domain-specific knowledge. This script identifies both explicit dependencies (\eg argument passing between syscalls and eBPF helper functions) and implicit dependencies (\eg modifications to eBPF maps across multiple syscalls). Based on the extracted dependencies, BRF generates system calls and arguments tailored to the requirements of different eBPF programs, ensuring their correct loading and execution. }

%For implicit dependencies, HFL \cite{kim2020hfl} uses points to analysis to obtain candidate explicit and implicit dependency pairs, and identifies data flow by symbolic execution, which reduces Moonshine's problem of missing implicit dependencies due to aliased variables, but it is not known what types of dependencies are resolved and what percentage they account for due to the lack of in-depth dependency understanding. To capture both explicit and implicit dependencies, X-AFL \cite{liang2020xafl} determines explicit dependencies by invariant arguments value and implicit dependencies using the longest common subsequence. This approach has the advantage of simplicity and efficiency, but it has poor generalization capability and is prone to lose infrequent syscall sequences with implicit dependencies. Hao et al. \cite{Hao2022DemystifyingTD} develop a measurement pipeline that quantifies the severity of dependencies and reveals the multiple causes of dependencies through manual analysis. Although effective, the manual-based analysis only focuses on specific linux kernel modules and fails to cover other critical modules such as drivers, file systems, \etc To extract the dependency relationships between eBPF programs and related system calls (such as \texttt{BPF\_MAP\_CREATE}), BRF \cite{Hung2024BRFFT} employs manual source code analysis and verifier error feedback to build an automated dependency parsing script with domain-specific knowledge. This script identifies both explicit dependencies (\eg argument passing between syscalls and eBPF helper functions) and implicit dependencies (\eg modifications to eBPF maps across multiple syscalls). Based on the extracted dependencies, BRF generates system calls and parameters tailored to the requirements of different eBPF programs, ensuring their correct loading and execution. 

However, although previous approaches have constructed explicit and implicit dependencies, these dependencies are static and context-independent. Context-aware dependencies refer to relationships that exhibit different behaviors as the kernel state evolves, rather than remaining as unchanging, static explicit or implicit syscall sequences. For instance, two consecutive calls to \texttt{setsockopt} can configure a socket differently, altering the socket's state or the behavior of the protocol stack in the kernel, thereby placing it into a specific context that triggers deeper kernel code paths. In contrast, context-independent dependencies assume that a \texttt{write} call should follow the \texttt{setsockopt} call for data transmission, neglecting the need for further socket configuration to reach the correct state for triggering a particular vulnerability. This oversight can result in missed opportunities to exploit specific vulnerabilities. To address this issue, Mock \cite{Xu2024MOCKOK} introduces a neural network model that iteratively learns from refined syscall sequences. This approach captures both longer-range dependencies and changes in kernel state, enabling more accurate modeling of dynamic dependencies.

\subsubsection{API polymorphism.}
Polymorphism (or entry points) in the context of kernel syscalls refers to the kernel's ability to support various operations for different devices and features through indirect control transfer mechanisms, such as function pointer tables. Let \( S \) be a set containing function pointers \( f \) and let \( I \) be a set of syscall inputs. Polymorphism can be described as a mapping \( P \) such that for every \( i \in I \), there exists a function \( f_i \in S \) where \( P(i) = f_i \). The function \( f_i \) can vary based on the type or value of \( i \), thus enabling different syscall behaviors: \( P: I \to S \).

Due to the nearly 4,200 syscall variants in the open-source operating system Linux \cite{Syzkaller}, ignoring polymorphism can easily lead to the problem where test inputs fail to reach the expected code paths \cite{jeong2019razzer,schumilo2017kafl}. Some kernel fuzzers \cite{pailoor2018moonshine,han2017imf,2017DIFUZE} only extract function entry information through static analysis; however, this method encounters challenges in precise data flow analysis of deep code due to numerous indirect calls, linked list operations, nested data structures, and multi-level pointer dereferencing. 

To address this issue, HFL \cite{kim2020hfl} uses inter-procedural data flow analysis and static points-to analysis to determine which index variables of function pointer tables originate from syscall arguments. Based on this, HFL implements an offline converter that expands the calls of function pointer tables into explicit conditional branches. Finally, HFL employs symbolic execution engine , S2E \cite{Chipounov2011S2EAP}, on the converted explicit conditional branch paths to explore the syscalls and argument values that trigger the target functions. However, this method may lead to path explosion issues due to kernel uncertainties (such as symbolic execution needing to consider all possibilities in the function pointer table). Additionally, function pointer tables may be dynamically registered during kernel module initialization, dynamic configuration by other syscalls, and device hot-swapping (\eg different protocols like TCP can register specific operations at various times), making static analysis unable to capture all control flow paths completely. To address the issue of dynamic registration, KSG leverages eBPF \cite{eBPF} and Kprobe \cite{Kprobe} to dynamically hook multiple probes before and after specific kernel functions. By scanning device files and network protocols, KSG triggers the execution of hooked kernel functions to extract the syscall entry points of triggered submodules. This method introduces additional runtime overhead, especially when handling a large number of syscalls and frequent dynamic registrations. Moreover, since KSG relies on runtime behavior, it may miss some unregistered function pointers, resulting in incomplete analysis. Another more precise method, Syzdirect \cite{tan2023syzdirect}, focuses on using static analysis to identify and locate anchor functions to reduce the substantial overhead of modeling all functions and combines the state-of-the-art type-based indirect call analysis \cite{Lu2019WhereDI} to trace indirect call chains to determine syscall variants and arguments. Unlike KSG, Syzdirect directly uses PoC as input (including the target kernel functions and corresponding syscall variants) to evaluate the reproduction and triggering conditions of specific code locations. Therefore, this method is characterized by a lower false positive rate but has weaker generalizability.

  \begin{figure*}[!t]
    \centering
    \subfigure[An example presenting argument type casting at linux/net/ipv4/ip\_sockglue.c.]{
      \includegraphics[width=0.47\linewidth]{img/arg_type.pdf}\label{img:arg_type}}
    \hspace{5pt}\vline\hspace{5pt}
    \subfigure[An example presenting nested argument copying at linux/drivers/usb/core/devio.c.]{
        \includegraphics[width=0.47\textwidth]{img/arg_nested.pdf}\label{img:arg_nested}}
    \caption{An example presenting type casting and nested structure copying. The red line indicates the name of target function, the green line represents the argument that determines the conditional branching. The blue line in (a) involves type casting, while the blue line in (b) also includes nested structure copying.}
  \end{figure*}

\subsubsection{Argument inference.}

In kernel fuzzing, the inference of syscall argument types and structures can be crucial. If these inferences are inaccurate, the generated syscall arguments are probable to be invalid, consequently failing to fuzz the expected execution paths. In particular, argument inference involves two primary tasks: argument type inference and nested structure inference.

\textbf{Argument type casting} tends to make it hard for analysts to determine the type of arguments due to variable alias passing issues. Figure \ref{img:arg_type} shows the kernel API \texttt{do\_ip\_setsockopt} for syscall \texttt{setsockopt}, where \texttt{optname} determines the conditional branch (switch-case) to be executed, and the type of \texttt{optval} varies with the values of \texttt{optname} and \texttt{optlen}. For example, when \texttt{optname=IP\_OPTIONS}, \texttt{optval} is a pointer to structure \texttt{ip\_options\_rcu}. For this problem, the analyst needs to generate an appropriate value for the syscall's argument based on the code execution path, otherwise the execution of the fuzzing program will be blocked due to a failed type casting. To address this issue, KSG \cite{sun2022ksg} utilizes the symbolic execution of the Clang Static Analyzer (CSA) to perform path-sensitive analysis on the target function to check for the presence of type casting operations on the arguments (scalar-to-pointer and pointer-to-pointer casting). Also, a global mapping table is constructed for symbols and memory regions, which solves the problem that type casting is difficult to inference due to alias propagation. On the other hand, KernelGPT \cite{yang2023kernelgpt} utilizes GPT4 and designs a complete prompt chain to automatically infer the unknown types of ioctl's arguments and the specific values of the \texttt{cmd} arguments. However, the problem of nested structure pointer inference is not solved from these fuzzers. Also, such approaches are limited to the simple case of argument type inference for ioctl syscalls, which is insufficient to address complex scenarios where there is a complex chain of indirect calls from syscall to kernel source code.

\textbf{A nested structure} refers to a structure whose field members point to another structure that is dynamically allocated and initialized at runtime. As shown in Figure \ref{img:arg_type}, the \texttt{proc\_submiturb} function is triggered by the function \texttt{usbdev\_ioctl} derived from syscall \texttt{ioctl}, and is used to copy the user-space address \texttt{arg} into the nested structure. Such nested structures are usually frequent in special kernel APIs like \texttt{copy\_from\_user} and \texttt{copy\_to\_user} that are used for memory address copying, and traditional fuzzing have a hard time inferring such complex argument format due to dynamic allocation of addresses in such cases. Similarly, symbolic execution is based only on static analysis and explicitly symbolized input space; it cannot predict and handle dynamic memory regions pointed to by nested pointers. In order to accurately trace the memory operations of nested structure pointers, HFL \cite{kim2020hfl} specifically instruments the \texttt{copy\_from\_user} and \texttt{copy\_to\_user} functions and captures their arguments and return values (\ie the addresses and sizes of the source and target buffers). Additionally, the intercepted arguments are labeled as symbolic variables, and the buffer addresses of the nested structures as well as the data lengths are inferred using concolic execution. It is worth noting that these approaches focusing on argument inference typically rely on static analysis, symbolic execution methods to trace the propagation paths of arguments to constrain syscall and its arguments mutations, and rely specifically on handwritten work in syzlang, but these approaches cover limited syscall and requires a lot of manual work.

\todo{To address this problem}, a state-of-the-art approach, FUZZNG \cite{bulekov2023FUZZNG}, performs checksum corrections on mutated syscalls to ensure the validity of the syscall sequence. Specifically, FUZZNG developed a kernel module mod-NG that hooks the kernel's APIs related to handling file descriptor allocation and pointer arguments to reshape the input space of syscalls. For file descriptors, mod-NG hooks the kernel's \texttt{alloc\_fd} API, which is used to allocate new file descriptors, and the \texttt{fdget} API, which is used to retrieve file objects via file descriptors. By intercepting these APIs, mod-NG maps invalid file descriptors generated by mutations to existing file objects within the kernel. For pointer parameters, mod-NG hooks the \texttt{copy\_from\_user} function, allowing the kernel to populate the structure pointed to by the pointer into a valid user-space memory region when accessing user-space memory. Through this approach, FUZZNG reshapes the input space of syscalls, ensuring that even if the mutated syscall arguments are invalid, they can still be mapped to valid file objects and memory regions.


\subsection{File System Fuzzing}

File system is one of the basic system services of an operating system. Mainstream file systems in open source OS include: ext4 \cite{Cao2007Ext4TN}, XFS \cite{XFS2018}, Brtfs \cite{Rodeh2013BTRFSTL} and F2FS \cite{Lee2015F2FSAN}. The special issues considered during file system fuzzing differ from kernel fuzzing, and this difference is reflected in three dimensions: input type, status, and specific bugs in the file system. Therefore, through insights into the file system fuzzing process, we review and summarize the evolution of methods for these 3 types of problems in Table \ref{tab_filesystem}.

\subsubsection{Input type}
File system fuzzers can be categorized into syscall- and syscall+metadata-based according to input type. In modern operating systems, file systems are mounted to the kernel as files. \todo{Therefore, file system fuzzer can fuzz file system-related source code in the kernel by generating only syscall sequences that specialize in operating on files (\eg Syzkaller \cite{Syzkaller}, KAFL \cite{schumilo2017kafl} Krace \cite{xu2020krace}, CONZZER \cite{CONZZER2022context}).} Unfortunately, using a single syscall sequence as input for a file system fuzzer has its disvantages. First, it can lose the runtime state of the file system, making subsequent file operations contextually irrelevant and incurring the fuzzing experiment ineffective (\eg repeatedly issuing \texttt{read()} or \texttt{unlink()} operations on files that have been performed \texttt{rename()}). \todo{Second, only mutating the metadata effectively triggers the code responsible for handling file-related operations in the file system. Using syscalls as the sole test input generally alters user data, which is irrelevant to the testing process and thus ineffective. Furthermore, metadata accounts for only about 1\% of the entire file system image \cite{JANUS2019fuzzing}, meaning that mutation operations based solely on syscalls are likely to be mostly ineffective. To address this problem, another approaches, such as, JANUS \cite{JANUS2019fuzzing}, Hydra \cite{Hydra2020finding}, Lfuzz \cite{Liu2023LFuzz}, are to use the file system image and the syscalls operating on this image simultaneously as inputs, applying appropriate mutators to update them for continuous context-aware file system fuzzing.}

%Secondly, typically only metadata affects file system operations, while user data is useless. Furthermore, metadata only accounts for about 1\% of the whole file system image \cite{JANUS2019fuzzing}. This means that most mutation operations on the input during fuzz testing may be ineffective. Another approaches, such as, JANUS \cite{JANUS2019fuzzing}, Hydra \cite{Hydra2020finding}, Lfuzz \cite{Liu2023LFuzz}, are to use the file system image and the syscalls operating on this image simultaneously as inputs, applying appropriate mutators to update them for continuous context-aware file system fuzzing.

\subsubsection{Status} The status of a file system typically refers to the values of its metadata, such as file open status, file paths, and file byte lengths. Traditional fuzzers focus only on the initial status of the metadata, making it difficult to reach deeper region of the code under test. Therefore, effective file system fuzzing generally maintains the intermediate status of the metadata throughout the fuzzing process. This approach is able to generated seeds with contextual information, enabling them to explore deeper code regions more effectively. JANUS \cite{JANUS2019fuzzing} is the first file system fuzzer to correlate status with file operations. By constructing multiple structures to maintain the intermediate status of metadata, it can independently mutate syscall sequence and the metadata extracted from file system image, thereby generating context-aware hybrid seeds. \todo{However, since this approach performs random mutations only on a clustered and localized metadata region, the fuzzing process is constrained to a limited area of the image. Moreover, this localized operation hotspot reduces the value of saving and restoring image states, leading to significant performance and storage overhead.} Lfuzz \cite{Liu2023LFuzz} builds on the JANUS framework but maintains a location tracking table that records the accessed image locations and their vicinity over a period. This approach reduces the search space by up to 8 times compared to JANUS, significantly enhancing performance.

\todo{Unlike JANUS and Lfuzz, Hydra \cite{Hydra2020finding} focuses on addressing the issues of fuzzing inefficiency and non-reproducible bugs caused by the accumulation of non-continuous status. Conceptually, metadata represents the status of the initial image, and when new syscalls are frequently used to significantly alter the image status, status fragmentation occurs. This means that mutations targeting metadata have a diminishing effect on file operations, thereby reducing the overall fuzzing efficiency. For example, syscalls that operate on files, such as \texttt{open} and \texttt{write}, can change file permissions and locations, thereby altering or negating the image status determined by the metadata. To address the inefficiencies caused by state accumulation and the issue of non-reproducible bugs, Hydra implements several key strategies. First, it uses a Library OS-based executor to create a fresh execution instance for each test case, allowing for quick initialization of both the file system and kernel logic. This ensures that every test case runs in a clean status. Second, Hydra prioritizes mutating metadata to maintain the continuity of the image status. Lastly, Hydra focuses on mutating existing syscalls rather than prematurely introducing new ones, preventing an excessively large mutation space while maintaining control over the current image status.}

%However, since the metadata locations within the file system are concentrated and localized, JANUS incurs high performance and storage overheads by frequently saving and restoring the entire file system image each time it randomly flips bits in the metadata. Lfuzz \cite{Liu2023LFuzz} builds on the JANUS framework but maintains a location tracking table that records the accessed image locations and their vicinity over a period. This approach reduces the search space by up to 8 times compared to JANUS, significantly enhancing performance.

\begin{table}
    \centering
    \footnotesize % 减小字体大小
    \caption{File System Fuzzers Sorted by Publication Year.}
    \label{tab_filesystem}
    \vspace{-0.3cm}
    \begin{tabular}{ccccc}
    \toprule
    \multicolumn{1}{m{2cm}}{\centering Fuzzers} &  % 中间的1.5cm表示第列宽度
    \multicolumn{1}{m{4cm}}{\centering File System} &
    \multicolumn{1}{m{3cm}}{\centering Input Type} &
    \multicolumn{1}{m{0.5cm}}{\centering Status} &
    \multicolumn{1}{m{1.8cm}}{\centering Specific Bugs} \\
    %Fuzzers & File System & Input Type & Status \\
    \midrule
    Syzkaller\cite{Syzkaller} & ext4, Btrfs, F2FS, jfs, xfs, reiserFS & syscall & $\times$ & C.\\
    KAFL\cite{schumilo2017kafl} & ext4 & syscall & $\times$ & -\\
    JANUS\cite{JANUS2019fuzzing} & ext4, Btrfs, F2FS& syscall+image & \checkmark & L.\\
    Krace\cite{xu2020krace} & ext4, Btrfs & syscall & $\times$ & C.\\
    Hydra\cite{Hydra2020finding} & ext4, Btrfs, F2FS & syscall+image & \checkmark & I.,S.,C.,L.\\
    CONZZER\cite{CONZZER2022context} & Btrfs, jfs, xfs, reiserFS & syscall & $\times$ & C.\\
    Lfuzz\cite{Liu2023LFuzz} & ext4, Btrfs, F2FS & syscall+image & \checkmark & L.\\
    \bottomrule
    \end{tabular}
    \begin{flushleft}
      \justifying % 使脚注文字左右对齐
      A ``-'' means it is irrelevant, not mentioned, or unclear in detail. \todo{\textbf{Abbreviation}: \textbf{I.}: Crash Inconsistency. \textbf{S.}: Specification Violation. \textbf{C.}: Concurrency Bug. \textbf{L.}: Logic Bug.}
      \end{flushleft}
\end{table}

\subsubsection{Specific bugs}
\todo{Compared to other OS Layers, bugs in the file system are characterized by diversity and specificity \cite{Lu2013ASO}.} In addition to mainstream memory violation errors, bugs specific to file systems include crash inconsistency, concurrency bug, and logic bug due to their intrinsic data persistence and concurrency characteristics. Section \ref{Vulnerability} summarizes the types of vulnerabilities and the corresponding monitors, hence, we only examine the vulnerabilities related to file system and the possible impact.
%An approximate 40\% of patches in file systems are fixes for various bugs, which reflects the diversity and specificity of bugs in file systems \cite{Lu2013ASO}. In addition to mainstream memory violation errors, bugs specific to file systems include crash inconsistency, concurrency bug, and logic bug due to their intrinsic data persistence and concurrency characteristics. Section \ref{Vulnerability} summarizes the types of vulnerabilities and the corresponding monitors, hence, we only examine the vulnerabilities related to file system and the possible impact.

\textbf{Crash inconsistency} is the most typical vulnerability of file system, which refers to the situation when the state of the file system is not as expected after handling a crash (\eg a sudden power failure). This is typically due to the file system not correctly managing its metadata when writing, updating, or deleting data leading to catastrophic consequences: permanent loss or corruption of files. For example, using \texttt{pwrite64()} on a file does not persistently modify the length of the file. \todo{\textbf{Specification violation} is also common bug in file system. This type of bug occurs when the file system violates the standards or specifications it is supposed to follow during operations. For example, file system operations are expected to adhere to specific standards, such as the POSIX standard or Linux man pages, which define the allowed behaviors and error codes for file operations. For instance, the POSIX standard specifies that when executing the \texttt{unlink} syscall, the only permissible error code is \texttt{EPERM}, yet some implementations return \texttt{EISDIR}, which constitutes a violation of the POSIX standard \cite{Hydra2020finding}. However, such errors are not specifically addressed by existing file system fuzzers. The reason is that triggering these violations is unlikely to cause kernel panics or crashes. Furthermore, the test cases and bug monitor tools are not designed to target such bugs, meaning that file system fuzzers, from their initial design, have overlooked the opportunity to capture specification violation bugs.} Another category of vulnerabilities is \textbf{concurrency bug}. Although these bugs are not specific to file systems, modern file systems, which inherently involve shared data regions, also introduce a series of programming paradigms to leverage multi-core computing \cite{fstests}. These paradigms, such as read-copy-update (RCU) and asynchronous work queues, improve performance but significantly increase the likelihood of writing concurrent error-prone code. Furthermore, unlike other types of bugs, file system-specific \textbf{logic bug} do not adhere to any general pattern for definition (\eg F2FS requires its own concept of rb-tree consistency \cite{Hydra2020finding}). As discussed in Section \ref{Vulnerability}, logic bugs do not cause immediate crashes but can lead to undefined behavior over time, affecting both performance and reliability.

\subsection{Driver Fuzzing}
Since drivers interact directly with hardware and are the most extensive subsystem in the kernel, they possess several unique fuzzing characteristics. \todo{Table \ref{tab_driver} demonstrates these three characteristics}, first, they expose an additional hardware-side attack surface compared to the kernel and file systems, making it clear that syscalls alone are insufficient to uncover hardware-related vulnerabilities. Second, the validation chains in drivers increase the complexity of fuzzing. Furthermore, driver fuzzing has a stronger demand for device-free (\ie scalability) because previous work assumed that each driver under test had an actual hardware peripheral \cite{Song2019PeriScopeAE,Talebi2018CharmFD}, which limited the testing scope of driver fuzzers.


\subsubsection{Input type}
\todo{As a subsystem of the kernel, a driver fuzzer can also test drivers by fuzzing syscalls. For example, in Linux, each file under the /dev directory represents a hardware device. User-space applications can obtain a file descriptor for a device and interact with it using syscalls such as \texttt{read}, \texttt{write}, or \texttt{ioctl} to perform specific hardware operations. Existing works, such as usb-fuzzer \cite{Syzkaller} (the driver fuzzing module of Syzkaller), DIFUZE \cite{2017DIFUZE}, BSOD \cite{maier2021bsod}, StateFuzz \cite{zhao2022statefuzz}, SyzDescribe \cite{hao2023syzdescribe} SATURN \cite{Xu2024Saturn}, and Syzgen++ \cite{Chen2024SyzGen++}, are all based on Syzkaller to generate syscall sequences that interact with drivers, successfully achieving driver fuzzing. Notably, syscall-based driver fuzzers face inherent challenges related to complex argument structures and manually constructing domain-specific languages (DSLs). For instance, usb-fuzzer injects random data into the USB stack via syscalls to test USB drivers. DIFUZE uses static analysis to extract correct type and argument structure information from driver interfaces, helping to generate valid test cases. BSOD developed a virtual device, BSOD-fakedev, to record interactions (commands and data) between the graphics card and its driver and generates test cases that reproduce these interactions using AFL++ and Syzkaller, alleviating the difficulty of manually building a DSL. StateFuzz introduces a novel driver state feedback mechanism that guides Syzkaller in generating state-continuous test cases, addressing the shortcomings of Syzlang syntax and the code coverage feedback mechanism in state awareness. This significantly enhances the depth and effectiveness of Syzkaller in driver fuzzing. SyzDescribe combines static analysis to automatically generate Syzlang for newly merged driver code in the Linux mainline.} \todo{Another syscall-based driver fuzzer, Ex-vivo \cite{Pustogarov2020ExvivoDA}, uses AFL to generate arguments for the \texttt{ioctl} syscall, rather than Syzkaller. The reason is that Syzkaller typically requires a custom kernel for fuzzing, but many Android devices can only boot signed kernels (AFL only generates test cases without modifying the kernel). In other words, Syzkaller collects runtime information by customizing the kernel, which invalidates the kernel signature. As a result, Syzkaller becomes difficult to use for driver fuzzing on mobile devices.}

%\todo{Historically, hardware has often been regarded as trustworthy and difficult to tamper with, which led to its attack surface being overlooked during testing. Another key challenge is that drivers typically rely on real physical devices for operation, and the ``hardware-in-the-loop'' testing approach limits flexibility. To address these issues, early efforts utilized specialized programmable hardware devices (such as FaceDancer \cite{FaceDancer}) to simulate physical devices and create a closed testing loop. However, these devices are expensive, lack scalability (as they can only fuzz a limited number of drivers associated with specific hardware at a time), and depend on physical actions (such as manually connecting and disconnecting devices) to trigger tests.} These limitations have resulted in hardware-related testing being insufficiently covered and prioritized in fuzzing efforts. For example, the kernel fuzzer syzkaller \cite{Syzkaller} includes a usb-fuzzer extension, which injects random data into the USB stack through extended syscalls. DIFUZE \cite{2017DIFUZE} employs static analysis to extract supported request types and related parameters from custom interfaces in device drivers. This method was widely adopted early on due to its simplicity and ease of use. However, the complexity and variety of syscall parameters not only limit the fuzzer's flexibility but also make it challenging to detect driver vulnerabilities caused by device behavior.

\todo{While existing driver fuzzers have created a closed-loop testing approach via syscalls, it is insufficient to consider syscalls as the sole entry point for testing. This approach implies an assumption that driver fuzzing is similar to kernel fuzzing, where comprehensive fuzzing can be achieved through syscalls alone. However, this assumption overlooks the complexity and potential vulnerabilities in hardware-related code, resulting in inadequate coverage and attention to hardware-level vulnerabilities \cite{peng2020usbfuzz}.} Therefore, another fuzzing interface involves injecting malicious inputs from the hardware side through device configuration or I/O channels such as Port I/O, MMIO, and DMA. USBFuzz \cite{peng2020usbfuzz} uses a simulated USB device to match with kernel drivers and injects malicious USB descriptors. \todo{Fuzzers such as \cite{Song2019PeriScopeAE,song2020agamotto,zhao2022semantic,ma2022printfuzz,shen2022drifuzz,wu2023devfuzz,Jang2023ReUSB, Huster2024ToBoldly} intercept the I/O access of target devices (including the target address and data content). After mutating the I/O data, the fuzzer resends the mutated data to the corresponding I/O channels (such as MMIO, PIO, or DMA).} This approach enhances the flexibility of driver fuzzers and improves their ability to discover hardware-related vulnerabilities, such as those associated with hardware initialization, interrupt handling, direct DMA operations, and hardware error handling. It is worth noting that this approach has the requirement of verifying the difference between the simulated behavior and the behavior of the real hardware, as well as the need for an in-depth understanding of the communication protocols.

\begin{table} % 表格开始
    \centering % 表格居中显示
    \scriptsize % 减小字体大小
    \caption{Driver Fuzzers Sorted by Publication Year.} % 表头标题
    \label{tab_driver} % 表格标签,便于引用
    \vspace{-0.3cm}
    
    \begin{tabular}{ccccc} % c表示单元格内容居中,l表示靠左,这里有两列,所以cc,如果是三列就是ccc或cll,根据自己
    
    \toprule % 顶部的线,这里可以定义粗细、toprule{1.5ptx}
    \multicolumn{1}{m{2cm}}{\centering Fuzzers} &  % 中间的1.5cm表示第列宽度
    \multicolumn{1}{m{3.5cm}}{\centering Device Type} &
    \multicolumn{1}{m{2cm}}{\centering Input Type} &
    \multicolumn{1}{m{2cm}}{\centering Stage} &
    \multicolumn{1}{m{1.5cm}}{\centering Device-Free}\\
    
    \midrule % 中间的线
    % \midrule
    \multirow{1}*{usb-fuzzer\cite{Syzkaller}} & USB & syscall & 3 & $\checkmark$  \\

    % \midrule
    \multirow{1}*{DIFUZE\cite{2017DIFUZE}} & PCI/USB/I2C/Others & syscall & 3 & $\times$  \\

    % \midrule
    \multirow{1}*{Periscope\cite{Song2019PeriScopeAE}} & PCI/USB/I2C/Others & I/O & 3 & $\times$  \\

    % \midrule
    \multirow{1}*{USBFuzz\cite{peng2020usbfuzz}} & USB & device configuration & 3 & $\checkmark$ \\

    % \midrule
    \multirow{1}*{Agamotto\cite{song2020agamotto}} & PCI/USB & syscall+I/O & 3 & $\times$ \\

    % \midrule
    \multirow{1}*{Ex-vivo\cite{Pustogarov2020ExvivoDA}} & PCI/USB/I2C/Others & syscall & 3 & $\checkmark$ \\

    % \midrule
    \multirow{1}*{BSOD\cite{maier2021bsod}} & PCI & syscall & 3 & $\times$ \\

    % \midrule
    \multirow{1}*{StateFuzz\cite{zhao2022statefuzz}} & others & syscall & 3 & $\times$ \\

    % \midrule
    \multirow{1}*{Dr.Fuzz\cite{zhao2022semantic}} & PCI/USB/I2C/Others & I/O & 1, 2, 3 & $\checkmark$ \\

    % \midrule
    \multirow{1}*{PrintFuzz\cite{ma2022printfuzz}} & PCI/USB/I2C/Others & syscall+I/O & 1, 2, 3 & $\checkmark$ \\

    % \midrule
    \multirow{1}*{DriFuzz\cite{shen2022drifuzz}} & PCI/USB & I/O & 1, 2, 3 & $\checkmark$ \\

    % \midrule
    \multirow{1}*{SyzDescribe\cite{hao2023syzdescribe}} & PCI/USB/I2C/Others & syscall & 3 & $\times$ \\

    % \midrule
    \multirow{1}*{DEVFUZZ\cite{wu2023devfuzz}} & PCI/USB/I2C/Others & I/O & 1, 2, 3 & $\checkmark$ \\

    % \midrule
    \multirow{1}*{ReUSB\cite{Jang2023ReUSB}} & USB & syscall+I/O & 1, 2, 3 & $\times$ \\
    
    % \midrule
    \multirow{1}*{SATURN\cite{Xu2024Saturn}} & USB & syscall & 3 & $\checkmark$ \\

    % \midrule
    \multirow{1}*{Syzgen++\cite{Chen2024SyzGen++}} & Others & syscall & 3 & $\times$ \\

    % \midrule
    \multirow{1}*{VIRTFUZZ\cite{Huster2024ToBoldly}} & PCI & I/O & 3 & $\times$ \\
    
    \bottomrule
    \end{tabular}
    \end{table}

\subsubsection{The three-stage of driver fuzzing}
A driver contains a complete sequence of operations. In order to clearly represent the main features of the different stages in driver fuzzing, we categorize the transition of the driver from its initial state to the ready state into three main stages: device enumeration, probe execution, and device communication. The first two stages require enforcing a series of validations on the input, such as checking the chip version number, determining the I/O method used, and verifying the vendor and product IDs. This series of checks is referred to as the validation chain. Therefore, to effectively fuzz the entire driver code, the input must successfully pass through the entire validation chain. Failing to do so results in the generated seeds remaining stuck in the driver's initial stage, severely impairing the fuzzer's performance.

The first stage, device enumeration, refers to the complete process from the operating system detecting a device to binding the corresponding driver. Device detection can only be triggered during the OS boot phase by scanning the bus or through hot-plugging events. To cover the code paths of this stage, DEVFUZZ \cite{wu2023devfuzz} simulates device hot-plugging behavior through run the command-line ``\texttt{echo 1 > /sys/bus/pci}'', thereby repeatedly triggering the device enumeration process. Specifically, for devices on dynamically probed buses (such as PCI/PCIe), DEVFUZZ repeatedly executes the \texttt{echo} command to scan the PCI bus. For devices on buses using static enumeration (such as I2C), a new device is created at a specified I2C bus address, indirectly triggering the system's bus scan operation. It is important to note that the code coverage of this stage might not be directly obtained using the ``kcov'' integrated in the kernel because the driver code is enabled during kernel boot, and ``kcov'' debugfs is not ready for reading at that time. To overcome this issue, Wu et al. and Zhao et al. \cite{zhao2022semantic,wu2023devfuzz} combine Intel PT to trace execution during the boot process.

The second stage, probe execution, verifies whether the test case meets the conditions required for subsequent device initialization. \todo{Thus, this stage contains numerous specialized code segments, such as operations that validate magic values by reading specific registers or perform polling verification. These pose significant challenges to the quality of test cases.} To correctly pass the validation chain, Dr.Fuzz \cite{zhao2022semantic} uses error codes returned by the program as a new feedback mechanism to guide mutations. However, this method's performance might be limited by the large search space. To improve the accuracy of input generation, DEVFUZZ \cite{wu2023devfuzz} employs symbolic execution to build a probe model, symbolizing the identifiers corresponding to the magic values to solve the constraints. Meanwhile, DriFuzz \cite{shen2022drifuzz} employs concolic execution to reduce the path explosion problem that DEVFUZZ \cite{wu2023devfuzz} might encounter when handling polling verification code segments. Additionally, ReUSB \cite{Jang2023ReUSB} high-fidelity record-and-replay syscalls and USB device requests and responses, thereby avoiding the complex symbolic solving tasks during the validation chain process.

Upon successfully passing the validation chain, the process enters the third stage: device communication (also known as post-probing). This stage signifies that the device is in a ready state and primarily involves data transfer between the device and driver through Memory-Mapped I/O (MMIO) and Direct Memory Access (DMA) mechanisms. Syscall-based driver fuzzers \cite{Syzkaller, 2017DIFUZE, Pustogarov2020ExvivoDA, maier2021bsod, zhao2022statefuzz, hao2023syzdescribe, Xu2024Saturn, Chen2024SyzGen++} can trigger communication-related driver code using only syscalls such as \texttt{open}, \texttt{write}, \texttt{close}, and \texttt{ioctl}. For fuzzers that utilize I/O as input, PeriScope \cite{Song2019PeriScopeAE} employs a page-fault-based kernel monitoring mechanism to capture device drivers' access to MMIO and DMA regions. This allows the fuzzer to dynamically intervene in the communication process between the driver and the device, enabling fuzz testing and analysis. DEVFUZZ \cite{wu2023devfuzz} employs static and dynamic program analysis to construct device models for MMIO, Port I/O (PIO), and DMA, thereby generating the fields of the structures. Dr.Fuzz \cite{zhao2022semantic} reduces the number of related data structures and the possible values of their fields by constructing I/O dependency graphs using static analysis, effectively narrowing the I/O input space. In addition, VIRTFUZZ \cite{Huster2024ToBoldly} takes the monitored communication data of real devices interacting with the driver (\eg Bluetooth HCI packets, WLAN frames) as the initial seed and transfers the mutated I/O data to the device driver under test via VirtIO's virtqueue (a highly efficient ring-buffer queue used for data transfer between VMs and hosts) to achieve the communication stage of fuzzing.

\todo{Although syscall-based driver fuzzers can directly focus on the post-probing stage, this approach also means that potential vulnerabilities in the first and second stages cannot be discovered. Therefore, to thoroughly fuzz a driver, it is essential to combine lower-level interaction interfaces (\ie device configuration or I/O channels).}

\subsubsection{Device-free}
\todo{The execution of device drivers typically relies on real physical hardware, a “hardware-in-the-loop” testing approach that, while ensuring high fidelity, significantly limits testing flexibility and increases dependence on hardware resources. According to statistics, the ``\texttt{drivers/}'' directory in Ubuntu Linux 20.04 contains approximately 13 million lines of code, accounting for 64.8\% of the entire Linux source code \cite{wu2023devfuzz}. To address the complexity and cost associated with such testing, a notable characteristic driver fuzzing is the adoption of ``device-free fuzzing,'' where device behaviors are simulated to reduce reliance on physical hardware, thereby lowering costs and improving generalization.} Existing driver fuzzers have implemented device-free testing through emulated hardware abstraction frameworks. For example, the usb-fuzzer in syzkaller uses the DUMMY HCD \cite{DummyHCD}, a virtual USB host controller driver in the Linux kernel, to inject generated data into the USB stack, thereby simulating the behavior of an actual USB host controller. SATURN \cite{Xu2024Saturn} leverages the Linux kernel's Gadget subsystem—a framework enabling device emulation as USB peripherals (\eg virtual printers or keyboards)—to dynamically generate USB-compliant device configurations, including vendor IDs, product IDs, and endpoint descriptors. By dynamically loading/unloading Gadget kernel modules to simulate hot-plug behaviors, SATURN triggers bus rescan operations, thereby eliminating hardware dependencies and achieving comprehensive coverage of device enumeration paths. However, this method is limited to USB bus drivers and fails to trigger vulnerabilities in physical host controller drivers.

%\todo{Driver running typically relies on real physical devices, and this 'hardware-in-the-loop' testing approach limits flexibility. Therefore, a notable feature of driver fuzzing is the strong demand for device-free testing, where device behavior is simulated to reduce reliance on physical hardware, thereby lowering costs and improving generalization.} For instance, in the Linux ecosystem, there are over 13,000 PCI devices, constituting 64.8\% of the entire Linux source code \cite{Syzkaller}. Existing driver fuzzers have implemented device-free testing through various approaches. For example, the usb-fuzzer in syzkaller uses the DUMMY HCD \cite{DummyHCD}, a virtual USB host controller driver in the Linux kernel, to inject generated data into the USB stack, thereby simulating the behavior of an actual USB host controller. However, this method is limited to USB bus drivers and fails to trigger vulnerabilities in physical host controller drivers.

To achieve broader testing across different device drivers, hypervisors like QEMU can be employed to intercept the read/write requests of the target kernel, constructing simulated devices \cite{peng2020usbfuzz}. Although this approach offers readily available virtual devices, reducing the dependence on real hardware, it has limited simulation capabilities, and building simulators for unsupported devices requires extensive manual effort. For instance, QEMU supports fewer than 130 PCI devices \cite{zhao2022semantic}. Additionally, these software-based simulators may generate overly standardized inputs, which are insufficient for triggering vulnerabilities that rely on malformed input \cite{Markettos2019ThunderclapEV}.

\todo{Device emulation follows specific device models. For instance, in the Linux Kernel Device Model (LKDM), the driver must successfully initialize the relevant data structures before running. Once these data structures pass the second-stage validation chain, the fuzzer can manipulate these device structures (such as I/O ports, MMIO regions, \etc) to communicate with the driver (third stage), simulating behavior similar to real hardware. Thus, the key to device-free fuzzing lies in ensuring that the generated inputs mimic the response data of real devices, correctly initializing key device data structures to help the driver pass the validation chain checks. After entering the communication stage, the driver interacts with the device through low-level operations (such as \texttt{in}, \texttt{out}, \texttt{readl}, \texttt{writel}), which still rely on virtualization platforms (\eg QEMU) to intercept and emulate, achieving complete device-free fuzzing. Ex-vivo \cite{Pustogarov2020ExvivoDA} captures the driver’s data structures through memory snapshots, thereby bypassing all validation chain checks and eliminating dependency on the device. It is worth noting that while this approach achieves device-free fuzzing, it is not sufficiently comprehensive in covering the first two stages of testing. Therefore, recent fuzzers such as \cite{zhao2022semantic,ma2022printfuzz,shen2022drifuzz,wu2023devfuzz} achieve device-free fuzzing by extracting the intrinsic semantic information of drivers (\eg error codes) or employing program analysis techniques to infer device structures and simulate device behavior, enabling device-free fuzzing.}

%\todo{As a result, recent fuzzers such as \cite{zhao2022semantic,ma2022printfuzz,shen2022drifuzz,wu2023devfuzz} achieve device-free fuzzing by extracting the intrinsic semantic information of drivers (\eg error codes) or employing program analysis techniques to infer and simulate device input data in a 'reverse engineering' manner.}

%As a result, recent fuzzers such as \cite{zhao2022semantic,ma2022printfuzz,shen2022drifuzz,wu2023devfuzz} have leveraged intrinsic semantic information within drivers (\eg error codes) to guide seed generation or have utilized sophisticated program analysis techniques to model the driver workflow, particularly the verification chain, to enable device-free testing. However, it is undeniable that these approaches also introduce greater complexity.

\subsection{Hypervisor Fuzzing}
Hypervisors implement multi-domain deployment and resource isolation through emulated virtual devices, which are a major source of vulnerabilities \cite{CVE-2014-2894,CVE-2015-3456,CVE-2015-5279,CVE-2015-6855} and thus receive more attention in hypervisor testing activities \cite{QTest,Cong2013SymbolicEO}. \todo{Compared to the other OS layers, hypervisors have more attack surfaces and the details of implementing fuzzing for these interfaces are more complex.} Therefore, in addition to focusing on these attack surfaces, it is crucial to consider how to effectively conduct fuzzing using these interfaces. As shown in Table \ref{tab_hypervisor}, we summarize the input types and methodologies used by existing hypervisor fuzzers, as well as the hypervisors they have been tested on.

% \begin{table} % 表格开始
%     \centering % 表格居中显示
%     \footnotesize
%     \caption{Hypervisor Fuzzers.} % 表头标题
%     \label{tab4} % 表格标签,便于引用
    
%     \begin{tabular}{cccc} % c表示单元格内容居中,l表示靠左,这里有两列,所以cc,如果是三列就是ccc或cll,根据自己
    
%     \toprule % 顶部的线,这里可以定义粗细、toprule{1.5ptx}
%     \multicolumn{1}{m{2cm}}{\centering Fuzzers} &  % 中间的1.5cm表示第列宽度
%     \multicolumn{1}{m{2cm}}{\centering Hypervisor} &
%     \multicolumn{1}{m{2cm}}{\centering Interface} &
%     \multicolumn{1}{m{4cm}}{\centering How} \\
    
%     \midrule % 中间的线
%     \multirow{1}*{VDF\cite{henderson2017vdf}} & QEMU & PIO, MMIO & Qtest \\

%     \midrule
%     \multirow{1}*{Hyper-Cube\cite{schumilo2020hyper}} & QEMU, Bhyve, ACRN, VirtualBox, Vmware Fusion & PIO, MMIO, DMA, Hypercall, Instruction & Bytecode translation \\

%     \midrule
%     \multirow{1}*{NYX\cite{schumilo2021nyx}} & QEMU, Bhyve & PIO, MMIO, DMA, Hypercall, Instruction & Bytecode translation\\

%     \midrule
%     \multirow{1}*{V-Shuttle\cite{pan2021V-shuttle}} & QEMU, Virtual Box & DNA & Interception and Redirection\\

%     \midrule
%     \multirow{1}*{MundoFuzz\cite{myung2022mundofuzz}}  & QEMU, Bhyve & PIO, MMIO, DNA & Interception and Redirection\\

%     \midrule
%     \multirow{1}*{Morphuzz\cite{bulekov2022morphuzz}} & QEMU, Bhyve & PIO, MMIO, DNA & Qtest\\

%     \midrule
%     \multirow{1}*{IRIS\cite{cesarano2023iris}}  & Xen & Instruction & Interception and Redirection\\

%     \midrule
%     \multirow{1}*{VD-Guard\cite{Liu2023VDGuard}}  & QEMU, VirtualBox & PIO, MMIO, DNA & Bytecode translation\\
    
%     \bottomrule
%     \end{tabular}
%     \end{table}

\begin{table}
    \centering
    \scriptsize % 减小字体大小
    \caption{Hypervisor Fuzzers Sorted by Publication Year.}
    \label{tab_hypervisor}
    \vspace{-0.3cm}
    \begin{tabular}{cccc}
    \toprule
    Fuzzers & Hypervisor & Input Type & Input Method \\
    \midrule
    VDF\cite{henderson2017vdf} & QEMU & PIO, MMIO & Qtest \\
    Hyper-Cube\cite{schumilo2020hyper} & \makecell{QEMU, Bhyve, ACRN, \\ VirtualBox, Vmware Fusion} & PIO, MMIO, DMA, Hypercall, Instruction & Bytecode translation \\
    NYX\cite{schumilo2021nyx} & QEMU, Bhyve & PIO, MMIO, DMA, Hypercall, Instruction & Bytecode translation \\
    V-Shuttle\cite{pan2021V-shuttle} & QEMU, Virtual Box & DMA & Interception and Redirection \\
    MundoFuzz\cite{myung2022mundofuzz} & QEMU, Bhyve & PIO, MMIO, DMA & Interception and Redirection \\
    Morphuzz\cite{bulekov2022morphuzz} & QEMU, Bhyve & PIO, MMIO, DMA & Qtest \\
    IRIS\cite{cesarano2023iris} & Xen & Instruction & Interception and Redirection \\
    VD-Guard\cite{Liu2023VDGuard} & QEMU, VirtualBox & PIO, MMIO, DMA & Bytecode translation \\
    ViDeZZo\cite{Liu2023ViDeZZoDV} & QEMU, VirtualBox & PIO, MMIO, DMA & Interception and Redirection\\
    HYPERPILL\cite{Bulekov2024HYPERPILLFF} & QEMU & PIO, MMIO, DMA, Hypercall & Bytecode translation\\
    \bottomrule
    \end{tabular}
\end{table}



\subsubsection{Input type}
Unlike the other OS layers, hypervisor fuzzing requires interacting with multiple interfaces. Schumilo et al. \cite{schumilo2020hyper} identified the attack surfaces of hypervisors, including Port I/O, MMIO, DMA, hypercalls, and privileged instructions. Port I/O, MMIO, and DMA are primarily concerned with fuzzing virtual devices related to PCI/PCIe, ISA, and other bus interfaces. For instance, VDF \cite{henderson2017vdf} uses a recording and playback approach to focus specifically on MMIO-related activities. \todo{Hypercalls are specially designed interfaces that allows virtual machines to actively communicate with the hypervisor.} \todo{Similar to how a syscall allows a switch from user mode to kernel mode, a hypercall enables the guest operating system (guest OS) in a virtual machine to trigger a VM-exit. VM-exit is a virtualization mechanism that temporarily transfers control from the virtual machine to the hypervisor, ensuring that the virtual machine cannot directly access or modify the host's resources. After handling the operations triggered by the hypercall, the hypervisor returns control back to the virtual machine.} For example, the \texttt{vmcall} instruction in Intel VT-x is used to write the hypervisor's processing results back to the guest's memory. Privileged instructions refer to commands used for hardware resource management, such as accessing and modifying control registers, managing memory paging, and configuring interrupt controllers. For example, \todo{testers can use the \texttt{MOV CR3} instruction to trigger the hypervisor’s interception and emulation of the memory paging mechanism.} \todo{The difference between hypercalls and privileged instructions lies in their roles: hypercalls handle tasks related to virtualization (\eg the hypervisor creating and attaching a virtual network interface to a virtual machine), while privileged instructions involve more fundamental hardware operations (\eg modifying registers). Specifically, when privileged instructions are issued, the virtual machine is unaware that it is operating in a virtualized environment, and control over hardware is passively handed to the hypervisor for emulation before being mapped back to the virtualized environment. To improve hypervisor performance, developers aim to avoid running complex hypervisor emulation for tasks that do not involve low-level hardware operations. Hence, hypercalls were introduced to the hypervisor, allowing the virtual machine to recognize its virtualized environment and proactively request the hypervisor to perform virtualization-related tasks, thereby avoiding complex hardware emulation. It is worth noting that while hypercalls improve the performance of virtualization in the hypervisor, this does not imply that testing should focus solely on hypercall interfaces. On the contrary, the broader attack surface requires testers to consider the tasks associated with different interfaces within the hypervisor and effectively organize fuzzing efforts to ensure comprehensive testing.}



%Hypercalls are specialized software interrupts designed to enhance virtualization performance. Similar to how syscalls switch from userspace to kernel, hypercalls enable the guest OS to trigger a VM-exit, which directly transitions control to the hypervisor. For example, the \texttt{vmcall} instruction in Intel VT-x is used to write the hypervisor's processing results back to the guest's memory. Privileged instructions refer to commands used for hardware resource management, such as accessing and modifying control registers, managing memory paging, and configuring interrupt controllers. For example, a hypervisor might intercept and emulate high-privilege instructions like \texttt{MOV CR3}.

\subsubsection{Input method}
We summarize that existing fuzzers exploit the attack surface exposed by the hypervisor for fuzzing activities mainly through Qtest, bytecode translation, interception and redirection techniques.

\textbf{Qtest} is a lightweight framework provided by QEMU for testing virtual devices. With Qtest, testers can conveniently perform memory I/O read and write operations using APIs, bypassing the need to rely on combined CPU instructions. This means that Qtest allows direct interaction with virtual devices without depending on CPU instructions such as \texttt{inb \%Port\_Number} or \texttt{movl \%MMIO\_Address, value} \cite{bulekov2022morphuzz}. 

Based on this, \todo{fuzzing engine} can be automated by controlling Qtest to test the handling of virtual devices. For instance, VDF \cite{henderson2017vdf} records each MMIO operation's type, base offset, and the content of the data read or written, and it uses a custom Qtest API to replay mutated MMIO operations. However, this approach is limited to MMIO-related I/O activities and overlooks more frequent virtual device I/O operations testing, \ie DMA. This limitation arises because the location of the DMA buffer is dynamic and can be anywhere within the guest memory. Additionally, when the CPU accesses these buffers, the hypervisor does not proactively capture these accesses. 

\sloppy
To address this issue, Morphuzz \cite{bulekov2022morphuzz} intercepts guest-issued DMA operations by hooking into the DMA-access API provided by the hypervisor. Specifically, Morphuzz uses Qtest to send PIO/MMIO instructions generated by libfuzzer to the virtual device, which may trigger consecutive DMA operations. By intercepting the DMA access API, Morphuzz can intervene in the hypervisor's DMA access and inject specific patterns (\eg pointer rings with multiple unique addresses) into the corresponding guest memory regions to facilitate the fuzzing process. It is also worth mentioning that Morphuzz manually implemented a Qtest-like testing framework to perform fuzzing on the hypervisor Bhyve. 
However, it is important to note that this type of Qtest-based fuzzer is only applicable to QEMU and cannot be used for fuzzing different hypervisors. More critically, without carefully designing complex DMA data structures, the exploration of DMA-related testing remains limited.

\textbf{Bytecode translation} refers to a method where I/O request types are represented by predefined opcodes composed of different byte lengths. An interpreter then translates this bytecode into a compilable and executable program, hypercall, or privileged instruction. Additionally, this approach requires the fuzzer to provide an Agent OS to execute these specific I/O requests. Representative works in this domain include Hyper-Cube \cite{schumilo2020hyper} and NYX \cite{schumilo2021nyx} (a successor to Hyper-Cube). Hyper-Cube leveraged this technique to fuzz various attack surfaces, but it is considered a blind fuzzer, \todo{as the randomly generated bytecode leads to violations of resource usage protocols (such as variable states and function call timing relationships). NYX addresses this issue by leveraging affine types (a type system where variables can be used only once, or at most once, helping to prevent certain types of errors like double-free or double-close). NYX implements its own compiler and interpreter, which ultimately generates executable programs that perform I/O requests on the agent OS.} In addition, in order to prevent the potential for address errors caused by the previous use of static translation, VD-Guard \cite{Liu2023VDGuard} examines the current memory layout during translation via QEMU to obtain the memory region index corresponding to the target memory address. \todo{HYPERPILL~\cite{Bulekov2024HYPERPILLFF} first injects PRNG-generated values into relevant registers to trigger hypercalls, then utilizes runtime feedback to identify values that significantly affect the hypercall execution path. By further refining these critical inputs, HYPERPILL effectively avoids unexpected errors caused by invalid data.} This bytecode translation approach is not tied to any specific hypervisor platform, offering greater flexibility. However, it also introduces higher complexity, especially since manually written specifications are prone to errors, potentially limiting the effectiveness of the fuzzing process.

%HYPERPILL~\cite{Bulekov2024HYPERPILLFF} first injects PRNG-generated values into registers that can issue hypercalls, and subsequently extracts values that may have a significant impact on the hypercall execution path through runtime feedback. By combining a variety of different registers, the test coverage of the hypercall is extended. This bytecode translation approach is not tied to any specific hypervisor platform, offering greater flexibility. However, it also introduces higher complexity, especially since manually written specifications are prone to errors, potentially limiting the effectiveness of the fuzzing process.

%NYX addressed this limitation by employing affine types and using a custom compiler and interpreter to ultimately generate C program code that executes I/O requests within the Agent OS. In addition, in order to prevent the potential for address errors caused by the previous use of static translation, VD-Guard \cite{Liu2023VDGuard} examines the current memory layout during translation via QEMU to obtain the memory region index corresponding to the target memory address. HYPERPILL~\cite{Bulekov2024HYPERPILLFF} first injects PRNG-generated values into registers that can issue hypercalls, and subsequently extracts values that may have a significant impact on the hypercall execution path through runtime feedback. By combining a variety of different registers, the test coverage of the hypercall is extended. This bytecode translation approach is not tied to any specific hypervisor platform, offering greater flexibility. However, it also introduces higher complexity, especially since manually written specifications are prone to errors, potentially limiting the effectiveness of the fuzzing process.

\textbf{Interception and redirection } involves hooking critical function calls that trigger VM Exit events, such as kernel-specific I/O APIs or the VM Exit handlers within the hypervisor. This method allows for the collection of intermediate information, which is then mutated and passed to the hypervisor for fuzzing. The advantage of this approach is that it does not require manual seed specification; instead, it mutates and tests based on actual, effective seeds. Existing works such as MundoFuzz \cite{myung2022mundofuzz}, V-Shuttle \cite{pan2021V-shuttle}, IRIS \cite{cesarano2023iris}, and VideZZo \cite{Liu2023ViDeZZoDV} have successfully utilized this approach to fuzz hypervisors.

MundoFuzz hooks guest kernel-specific APIs that handle PIO, MMIO, and DMA operations, such as the Linux kernel's \texttt{inl()/outl()} for PIO access, \texttt{readl()/writel()} for MMIO access, and \texttt{dma\_map\_single()} for DMA buffer allocation. It then uses a customized Agent OS kernel to invoke these APIs to set register values. V-Shuttle, on the other hand, hooks DMA-related APIs within the hypervisor, such as \texttt{pci\_dma\_read}. This allows DMA operations to be "hijacked" and redirected to a fuzzer-generated seed file, replacing actual guest memory data with the mutated seed.

Similar to V-Shuttle, but different in that IRIS hooks the VM Exit handler within the hypervisor and uses a dummy VM to pass the mutated VM seed, including the Virtual Machine Control Structure (VMCS) and General-Purpose Registers (GPR), to the hypervisor. Unlike other hypervisor fuzzers, IRIS primarily aims to test the hypervisor's potential vulnerabilities when handling VM Exit events triggered by virtual machines.

Unlike the previous approaches, VideZZo only hooks the DMA API to collect information about DMA-access patterns, but does not perform the redirection, which is intended to automatically construct grammars with context-dependent properties based on domain knowledge.

% \input{src/sec06-trustworthy evaluation.tex}

% !TeX root = ../main.tex

\section{Future Research Directions}\label{Section7}

\subsection{LLM-Based Test Case Generation}

Despite the widespread recognition of Syzkaller in kernel fuzzing, its support for rapidly iterating and continuously integrating open-source OSs exhibits significant limitations. Particularly,~for~systems such as Linux, with an average of 200 commits to the mainline per day \cite{Hao2022DemystifyingTD}, Syzlang necessitates continuous manual updates and enhancements. This is not only inefficient, but also leads to limited code coverage due to incomplete system call descriptions. The advent of large language models (LLMs) for automatic test case generation proposes a potential solution, leveraging the formidable language processing capabilities of LLMs for adapting OS specifications to Syzlang language learning and fine-tuning. This approach not only paves new avenues for the automatic generation of Syzlang, but also aims to enhance the accuracy and coverage of test cases. Furthermore, the linguistic processing prowess of LLMs could be employed to precisely analyze complex driver modules, such as DMA communications, thereby facilitating the automatic generation of test cases for Driver and Hypervisor layer I/O interfaces. Research in this direction could encompass developing novel model architectures, optimizing LLM training processes, and validating the efficacy and accuracy of LLM-generated test cases in practical OSF tasks.

\subsection{Dependency-Enhanced Fuzzing Solution}

Dependency issues are a key challenge in improving the effectiveness of OSF. Current approaches use program analysis techniques to extract the control flow and data flow of the PUT to establish dependency relationships between seeds. Although these approaches can generate semantically correct seed sequence, they often struggle to trigger deep vulnerabilities with state dependencies due to the lack of contextual information in static control flow dependencies. For example, \texttt{setsockopt} must be called twice in a row to trigger a specific vulnerability \cite{Xu2024MOCKOK}. Additionally, while the latest approach, Mock, dynamically learns state dependencies using neural network models, it affects fuzzing efficiency and introduces a degree of randomness. To simultaneously address both control flow and state dependencies, future research could explore minimal common seed sequences by mining sources such as CVE reports (state dependencies), PoCs (state dependencies), and real-world applications (control flow dependencies) before fuzzing is executed. Furthermore, by introducing a domain-specific language (DSL) in Syzlang to describe the dependency relationships of seeds, an end-to-end OSF solution with dependency-enhanced handling could be achieved. Although DSL cannot cover all dependencies, it provides a solid starting point for OSF, enabling seamless integration with learning models, dynamic program flow analysis, and other techniques during fuzzing to continuously strengthen the expressiveness and capture of dependencies.

\subsection{Embedded Operating System Fuzzing}

With the widespread deployment of embedded OSs (\eg RT-Linux, FreeRTOS, Zephyr, \etc)~in~industrial software infrastructure, \eg autonomous driving, ensuring their security and reliability is of utmost importance. However, vulnerabilities within the code of these embedded open-source OSs may present significant long-term security and safety threats to these industries. Furthermore, the application and effectiveness of fuzzing techniques in embedded devices are significantly~constrained by the limited computational resources and the high costs of experimental equipment. While simulation environments offer a compromise between flexibility and realism, they fall short~of accurately replicating the intricate interplay of hardware and software found in embedded systems. This limitation becomes particularly acute in fuzzing OSs for intelligent vehicles, where not all aspects of embedded hardware and software can be effectively simulated. Consequently, a significant research direction for the future involves identifying and developing methods to improve both the execution efficiency and the code coverage of fuzzing techniques, specifically within the practical constraints of embedded environments.

\subsection{Rust-Based Kernel Fuzzing}

With the increasing utilization of the Rust language in the development of open-source OS kernels due to its memory safety features, fuzzing for Rust-based kernels has become critically important. Although Syzlang is a widely used to generate initial seeds in Syzlang, its construction~in~C~poses challenges for direct application to Rust-based environments. Research efforts could begin~by~attempting to port Syzkaller to Rust-based kernels, focusing on analyzing the relationships between system calls and initially setting aside the complexities of Syzkaller's construction. The Rust community is still in the process of developing features analogous to the Sanitizer and KCOV modules, which are crucial for advanced fuzzing tasks. The creation of Rust equivalents for these tools, along with a new system call description language tailored for Rust, could significantly advance the field. Furthermore, by leveraging LLMs to accelerate the generation of automatic test cases, a higher degree of automation and efficiency can be achieved in fuzzing for Rust-based kernels.

%\textbf{Direction 4: Enhance the dependency in RTOS Fuzzing.}

% !TeX root = ../main.tex

\section{Conclusions}\label{Section8}

We conduct the first systematic literature review of OSF. Overall, we classify the literature according the four OS layers, \ie kernels, file systems, drivers, and hypervisors. We first summarize the general workflow of OSF, and then elaborate the details of each step of OSF. Further, we summarize unique fuzzing challenges for different OS layers. Based on the findings from our systematic survey, we discuss the future research directions in OSF. We hope our work will encourage further research in OSF and provide valuable guidance to newcomers in this field.

%We believe that exploring emerging technologies can effectively address challenges such as automated seed generation. Moreover, attention should be given to other new OS domains, such as fuzzing for embedded OS or for OS rewritten in new secure languages. 

% is an automated technique for testing operating system kernels, file systems, drivers, and hypervisors. Due to the vast codebase, the complexity of these components often exceeds that of traditional software fuzzing, making them more prone to unexpected crashes. This complexity is typically reflected in various aspects such as seed generation, selection, trimming, mutation, feedback calculation, and vulnerability monitoring. Additionally, different OS layers present unique challenges for fuzzing tasks. For example, kernel fuzzing focuses on generating valid syscall sequences and handling their complex argument interfaces; file system fuzzing emphasizes efficient mutation of image files; driver fuzzing requires support for a wide range of drivers to accommodate new code commits; and hypervisor fuzzing prioritizes discovering effective methods for fuzzing virtual device communication.

%Finally, conducting a trustworthy evaluation is crucial, as it helps prevent future researchers from being misled by erroneous results. 


% % !TeX root = ../main.tex
\newpage
\section*{APPENDIX}

\subsection*{A \hspace{0.1cm} ACRONYMS AND ABBREVIATIONS}

% \noindent OSF \hspace{0.35cm} Operating System Fuzzing\\
% API \hspace{0.35cm} Application Programming Interface\\
% DBI \hspace{0.35cm} Dynamic Binary Instrumentation\\
% SBI \hspace{0.35cm} Static Binary Instrumentation\\
% PUT \hspace{0.35cm} Program Under Test\\
% MAB \hspace{0.35cm} Multi-Armed-Bandit\\
% DSL \hspace{0.35cm} Domain-Specific language\\
% AIP \hspace{0.35cm} Alias Instruction Pair\\
% CCP \hspace{0.35cm} Concurrency Call Pair\\
% IS  \hspace{0.35cm} Interleaving Segment\\
% RPIP \hspace{0.35cm} Race Pair Interleaving Path\\
% UAF \hspace{0.35cm} Use-After-Free\\
% CO \hspace{0.35cm} Critical Object\\
% OOB \hspace{0.35cm} Out-Of-Bound\\
% DUT \hspace{0.35cm} Driver Under Test\\
% CSA \hspace{0.35cm} Clang Static Analyzer\\
% PIO \hspace{0.35cm} Port I/O\\
% MMIO \hspace{0.35cm} Memory-Mapped I/O\\
% DMA \hspace{0.35cm} Direct Memory Access\\
% VMCS \hspace{0.35cm} Virtual Machine Control Structure\\
% GPR \hspace{0.35cm} General-Purpose Registers\\

\begin{tabbing}
    OSF \hspace{0.5cm} \= Operating System Fuzzing \\
    API \> Application Programming Interface \\
    VM   \> Virtual Machine \\
    VMMs \> Virtual Machine Monitors \\
    DBI \> Dynamic Binary Instrumentation \\
    SBI \> Static Binary Instrumentation \\
    PUT \> Program Under Test \\
    MAB \> Multi-Armed-Bandit \\
    DSL \> Domain-Specific language \\
    AIP \> Alias Instruction Pair \\
    CCP \> Concurrency Call Pair \\
    IS  \> Interleaving Segment \\
    RPIP \> Race Pair Interleaving Path \\
    UAF \> Use-After-Free \\
    GPFs \> General Protection Faults \\
    CO \> Critical Object \\
    OOB \> Out-Of-Bound \\
    DUT \> Driver Under Test \\
    CSA \> Clang Static Analyzer \\
    PIO \> Port I/O \\
    MMIO \> Memory-Mapped I/O \\
    DMA \> Direct Memory Access \\
    VMCS \> Virtual Machine Control Structure \\
    GPR \> General-Purpose Registers \\
    DSL \> domain-specific language \\
    LKDM \> Linux Kernel Device Model \\
    \end{tabbing}


\newpage

\bibliographystyle{ACM-Reference-Format}
\bibliography{src/reference}

\end{document}
\endinput

%%
%% End of file `sample-acmsmall.tex'.
