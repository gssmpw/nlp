\begin{figure*}[!htb]
\centering\small
\begin{tcolorbox}[colback=blue!5!white, colframe=blue!75!black, boxrule=0.5mm, arc=4mm, boxsep=5pt, width=\textwidth]
Is text\_a and text\_b similar in meaning? Respond with yes, related, or no. \\
\\
Here are a few examples. \\
\\
Example 1: \\
text\_a: has a topic of protecting the environment \\
text\_b: has a topic of environmental protection and sustainability \\
output: yes \\
\\
Example 2: \\
text\_a: has a language of German \\
text\_b: has a language of Deutsch \\
output: yes \\
\\
Example 3: \\
text\_a: has a topic of the relation between political figures \\
text\_b: has a topic of international diplomacy \\
output: related \\
\\
Example 4: \\
text\_a: has a topic of the sports \\
text\_b: has a topic of sports team recruiting new members \\
output: related \\
\\
Example 5: \\
text\_a: has a named language of Korean \\
text\_b: uses archaic and poetic diction \\
output: no \\
\\
Example 6: \\
text\_a: has a named language of Korean \\
text\_b: has a named language of Japanese \\
output: no \\
\\
Example 7: \\
text\_a: describes an important 20th century historical event \\
text\_b: describes a 20th century European politician \\
output: no \\
\\
Target: \\
text\_a: \{text\_a\} \\
text\_b: \{text\_b\} \\
output:
\end{tcolorbox}
\caption{The prompt we use to evaluate surface similarity between the inferred hypotheses and reference topics. The prompt is lightly edited from \citet{zhong_explaining_2024}.
The few-shot examples demonstrate three levels of similarity: equivalent meaning (yes), related concepts (related), and unrelated concepts (no).}
\label{fig:surface_similarity_prompt}
\end{figure*}
