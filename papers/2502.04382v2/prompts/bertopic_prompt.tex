\begin{figure*}[!htb]
\centering\small
\begin{tcolorbox}[colback=blue!5!white, colframe=blue!75!black, boxrule=0.5mm, arc=4mm, boxsep=5pt, width=\textwidth]
You are a machine learning researcher analyzing clusters from a topic model. You need to identify the key distinguishing features of documents in a specific topic cluster. \\
Example features could be: \\
- ``uses multiple adjectives to describe colors'' \\
- ``describes a patient experiencing seizures or epilepsy'' \\
- ``contains multiple single-digit numbers'' \\
\\
These are the top keywords describing the topic: \{topic\_keywords\}. \\
\\
You will also see a sample of documents belonging to this topic and a random sample of documents from other topics from the same dataset. \\
\\
TOPIC DOCUMENTS (a sample of documents belonging to this topic): \\
---------------- \\
\{topic\_documents\} \\
---------------- \\
\\
RANDOM DOCUMENTS (a sample of documents from other topics): \\
---------------- \\
\{random\_documents\} \\
---------------- \\
\\
Your goal is to create a short, specific label that captures what makes the topic documents different from the contrasting documents. \\
\\
The label should be: \\
1. As specific as possible, while still applying to all topic documents \\
2. Focused on concrete features rather than abstract concepts \\
3. Based on characteristics present in the topic documents but not in the contrasting documents \\
\\
Do not output anything else. Your response should be a single topic label starting with ``-'' and surrounded by quotes ``''. Your response is: \\
- ``
\end{tcolorbox}
\caption{The prompt we use to generate interpretable topic labels for \bertopic. 
The LLM is prompted with the keywords that distinguish the topic from others, and also with two sets of documents: 10 documents belonging to the target topic, and a random sample of 10 documents from other topics.
This labeling approach is inspired by \citet{grootendorst_llm_2024}, but we designed the prompt ourselves (intending it to be similar, where possible, to our neuron interpretation prompt).
}
\label{fig:bertopic_prompt}
\end{figure*}
