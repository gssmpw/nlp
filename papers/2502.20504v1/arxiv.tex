% This must be in the first 5 lines to tell arXiv to use pdfLaTeX, which is strongly recommended.
\pdfoutput=1
\PassOptionsToPackage{table,dvipsnames}{xcolor}
% In particular, the hyperref package requires pdfLaTeX in order to break URLs across lines.

\documentclass[11pt]{article}

% Change "review" to "final" to generate the final (sometimes called camera-ready) version.
% Change to "preprint" to generate a non-anonymous version with page numbers.
\usepackage[preprint]{acl}

% Standard package includes
\usepackage{times}
\usepackage{latexsym}

% For proper rendering and hyphenation of words containing Latin characters (including in bib files)
\usepackage[T1]{fontenc}
% For Vietnamese characters
% \usepackage[T5]{fontenc}
% See https://www.latex-project.org/help/documentation/encguide.pdf for other character sets

% This assumes your files are encoded as UTF8
\usepackage[utf8]{inputenc}

% This is not strictly necessary, and may be commented out,
% but it will improve the layout of the manuscript,
% and will typically save some space.
\usepackage{microtype}

% This is also not strictly necessary, and may be commented out.
% However, it will improve the aesthetics of text in
% the typewriter font.
\usepackage{inconsolata}

%Including images in your LaTeX document requires adding
%additional package(s)
\usepackage{enumitem}
\usepackage{amsmath}
\usepackage{amssymb}
\usepackage{mathtools}
\DeclareMathOperator{\argmax}{argmax}
\usepackage{xspace}
\usepackage{graphicx}
\usepackage{subfig}
\usepackage{booktabs}
\usepackage{multirow}
\usepackage{array}
\usepackage[dvipsnames]{xcolor}
\usepackage{nicefrac}
\usepackage[table]{xcolor}
\usepackage{listings}
\lstset{
   breaklines=true,
   basicstyle=\ttfamily,
}
\newtheorem{problem}{Problem}
\newcommand{\ie}{\textit{i.e.}\xspace}
\newcommand{\eg}{\textit{e.g.}\xspace}
\renewenvironment{quote}
  {\begin{list}{}%
     {\setlength{\leftmargin}{2mm} % Adjust left margin
      \setlength{\rightmargin}{2mm}} % Adjust right margin
     \item\relax}
  {\end{list}}

\newcommand{\CG}{\mathcal{G}\xspace}
\newcommand{\CV}{\mathcal{V}\xspace}
\newcommand{\CE}{\mathcal{E}\xspace}
\newcommand{\CA}{\mathcal{A}\xspace}
\newcommand{\CF}{\mathcal{F}\xspace}
\newcommand{\CR}{\mathcal{R}\xspace}
\newcommand{\CB}{\mathcal{B}\xspace}
\newcommand{\CX}{\mathcal{X}\xspace}
\newcommand{\CK}{\mathcal{K}\xspace}
\newcommand{\CM}{\mathcal{M}\xspace}
\newcommand{\CC}{\mathcal{C}\xspace}
\newcommand{\CL}{\mathcal{L}\xspace}
\newcommand{\CI}{\mathcal{I}\xspace}
\newcommand{\CQ}{\mathcal{Q}\xspace}
\newcommand{\CO}{\mathcal{O}\xspace}
\newcommand{\CP}{\mathcal{P}\xspace}
\newcommand{\CS}{\mathcal{S}\xspace}
\newcommand{\CT}{\mathcal{T}\xspace}
\newcommand{\CJ}{\mathcal{J}\xspace}
\usepackage[para]{footmisc}
\usepackage{subfig}
% \usepackage{subcaption}
% \usepackage{array}
% \usepackage{colortbl}



\newcolumntype{H}{>{\setbox0=\hbox\bgroup}c<{\egroup}@{}}
\newcommand{\ks}[1]{\textcolor{red}{[KS: #1]}}
\newcommand{\jb}[1]{\textcolor{blue}{[JB: #1]}}

\definecolor{AgeBg}{RGB}{232,244,255}
\definecolor{LocBg}{RGB}{255,240,240}
\definecolor{OccBg}{RGB}{245,235,255} 
% If the title and author information does not fit in the area allocated, uncomment the following
%
%\setlength\titlebox{<dim>}
%
% and set <dim> to something 5cm or larger.

% \newcommand{\CG}{\mathcal{G}\xspace}
\newcommand{\CV}{\mathcal{V}\xspace}
\newcommand{\CE}{\mathcal{E}\xspace}
\newcommand{\CA}{\mathcal{A}\xspace}
\newcommand{\CF}{\mathcal{F}\xspace}
\newcommand{\CR}{\mathcal{R}\xspace}
\newcommand{\CB}{\mathcal{B}\xspace}
\newcommand{\CX}{\mathcal{X}\xspace}
\newcommand{\CK}{\mathcal{K}\xspace}
\newcommand{\CM}{\mathcal{M}\xspace}
\newcommand{\CC}{\mathcal{C}\xspace}
\newcommand{\CL}{\mathcal{L}\xspace}
\newcommand{\CI}{\mathcal{I}\xspace}
\newcommand{\CQ}{\mathcal{Q}\xspace}
\newcommand{\CO}{\mathcal{O}\xspace}
\newcommand{\CP}{\mathcal{P}\xspace}
\newcommand{\CS}{\mathcal{S}\xspace}
\newcommand{\CT}{\mathcal{T}\xspace}
\newcommand{\CJ}{\mathcal{J}\xspace}
\usepackage[para]{footmisc}
\usepackage{subfig}
% \usepackage{subcaption}
% \usepackage{array}
% \usepackage{colortbl}


% \newif\ifpreprint
% \preprintfalse


\title{\textit{A Thousand Words or An Image}: Studying the Influence of \\ Persona Modality in Multimodal LLMs}

% Author information can be set in various styles:
% For several authors from the same institution:
\author{Julius Broomfield\thanks{Equal contribution}, Kartik Sharma$^*$,  Srijan Kumar \\
        Georgia Institute of Technology \\
        \{\texttt{jbroomfield,ksartik,srijan}\}\texttt{@gatech.edu}}
% if the names do not fit well on one line use
%         Author 1 \\ {\bf Author 2} \\ ... \\ {\bf Author n} \\
% For authors from different institutions:
% \author{Author 1 \\ Address line \\  ... \\ Address line
%         \And  ... \And
%         Author n \\ Address line \\ ... \\ Address line}
% To start a separate ``row'' of authors use \AND, as in
% \author{Author 1 \\ Address line \\  ... \\ Address line
%         \AND
%         Author 2 \\ Address line \\ ... \\ Address line \And
%         Author 3 \\ Address line \\ ... \\ Address line}
%\author{
%  \textbf{First Author\textsuperscript{1}},
%  \textbf{Second Author\textsuperscript{1,2}},
%  \textbf{Third T. Author\textsuperscript{1}},
%  \textbf{Fourth Author\textsuperscript{1}},
%\\
%  \textbf{Fifth Author\textsuperscript{1,2}},
%  \textbf{Sixth Author\textsuperscript{1}},
%  \textbf{Seventh Author\textsuperscript{1}},
%  \textbf{Eighth Author \textsuperscript{1,2,3,4}},
%\\
%  \textbf{Ninth Author\textsuperscript{1}},
%  \textbf{Tenth Author\textsuperscript{1}},
%  \textbf{Eleventh E. Author\textsuperscript{1,2,3,4,5}},
%  \textbf{Twelfth Author\textsuperscript{1}},
%\\
%  \textbf{Thirteenth Author\textsuperscript{3}},
%  \textbf{Fourteenth F. Author\textsuperscript{2,4}},
%  \textbf{Fifteenth Author\textsuperscript{1}},
%  \textbf{Sixteenth Author\textsuperscript{1}},
%\\
%  \textbf{Seventeenth S. Author\textsuperscript{4,5}},
%  \textbf{Eighteenth Author\textsuperscript{3,4}},
%  \textbf{Nineteenth N. Author\textsuperscript{2,5}},
%  \textbf{Twentieth Author\textsuperscript{1}}
%\\
%\\
%  \textsuperscript{1}Affiliation 1,
%  \textsuperscript{2}Affiliation 2,
%  \textsuperscript{3}Affiliation 3,
%  \textsuperscript{4}Affiliation 4,
%  \textsuperscript{5}Affiliation 5
%\\
%  \small{
%    \textbf{Correspondence:} \href{mailto:email@domain}{email@domain}
%  }
%}

\begin{document}


\maketitle

\begin{abstract}
    \begin{abstract}
Retrieval-Augmented Generation (RAG) is often used with Large Language Models (LLMs) to infuse domain knowledge or user-specific information. In RAG, given a user query, a retriever extracts chunks of relevant text from a knowledge base. These chunks are sent to an LLM as part of the input prompt. Typically, any given chunk is repeatedly retrieved across user questions. However, currently, for every question, attention-layers in LLMs fully compute the key values (KVs) repeatedly for the input chunks, as state-of-the-art methods cannot reuse KV-caches when chunks appear at arbitrary locations with arbitrary contexts. Naive reuse leads to output quality degradation.  This leads to potentially redundant computations on expensive GPUs and increases latency. In this work, we propose \sys, a system for managing and reusing precomputed KVs corresponding to the text chunks (we call \textit{chunk-caches}) in RAG-based systems. We present how to identify \hl{\textit{chunk-caches} that are reusable}, how to efficiently perform a small fraction of recomputation to \textit{fix} the cache to maintain output quality, and how to efficiently store and evict \textit{chunk-caches} in the hardware for maximizing reuse while masking any overheads. With real production workloads as well as synthetic datasets, we show that \sys reduces redundant computation by \textbf{51\%} over SOTA prefix-caching and \textbf{75\%} over full recomputation.
\hl{Additionally, with continuous batching on a real production workload, we get a \textbf{1.6$\times$} speedup in throughput and a \textbf{2$\times$} reduction in end-to-end response latency over prefix-caching while maintaining quality, for both the \llama-3-8B and \llama-3-70B models. 
}
\end{abstract}





\end{abstract}

%!TEX root = main.tex


\section{Introduction}

Games play a central role in AI research. In the early $20^{th}$ century, \cite{zermelo1913} showed that perfect information games in extensive form can be solved by a bottom-up traversal of the game tree. Despite the fact that this does not readily provide efficient ways to solve large games such as Chess or Go in practice, this has indeed 
laid the foundation for the dramatic progress in the field of perfect information games, with computer programs being able to challenge human experts. Solving games becomes more intricate when the players (agents) have incomplete information about the state of the game -- Poker for instance, where a player does not know the cards of the others. One of the remarkable imperfect information games where computer programs have been able to defeat professional human players is Texas Hold'em Poker~\cite{libratus-poker,deepstack,pluribus-poker}. A main technique used in these algorithms is the abstraction of large games into smaller \emph{imperfect recall} games. 

Perfect recall is the ability of a player to remember her own actions. Poker is an imperfect information game played by several players. However, ideally one would assume that the players have a perfect recall of their actions. An imperfect recall player does not remember the sequence of her own actions. Imperfect recall allows for a structured mechanism to forget the information history and as \cite{ijcai2024p332} argues, it is particularly suited for AI agents.

From a modeling perspective, imperfect recall has been used to describe teams of agents, where each team can be represented as a single agent with imperfect recall~\cite{VONSTENGEL1997309,DBLP:conf/aaai/Celli018} or to describe agents modeling multiple nodes which do not share information between each other due to privacy reasons~\cite{DBLP:conf/aaai/Conitzer19}. Moreover~\cite{LAMBERT2019164} argues that imperfect recall is a model of bounded rationality. Given the limited memory of players, it is not realistic to assume that the players remember all their actions. We refer the reader to~\cite{ijcai2024p332} for an excellent introduction to different uses of imperfect recall.  
From a practical perspective, the most prominent use of imperfect recall is in abstracting games~\cite{PracticalUseImperfect,DBLP:conf/aaai/GanzfriedS14,DBLP:conf/atal/BrownGS15, CERMAK2020103248}. The state space generated by usual games is typically very large and abstractions are crucial for solving such games. Abstractions that preserve perfect recall force a player to distinguish the current information gained, in all later rounds, even if it is not relevant. Abstractions using players with imperfect recall have been shown to outperform those using players with perfect recall~\cite{ DBLP:conf/sara/WaughZJKSB09, johanson2013evaluating, DBLP:conf/atal/BrownGS15,DBLP:conf/ijcai/CermakBL17}.



From a complexity perspective, imperfect recall games are known to be $\NP$-hard

~\cite{KollerMegiddo::1992,Cermak::2018} even when there is a single player, whereas perfect recall games can be solved in polynomial-time~\cite{KollerMegiddo::1992,vonStengel::1996}. Recent studies have aligned the complexity of different solution concepts for imperfect recall games to the modern complexity classes~\cite{GPS20,tewolde-et-al:2023,ijcai2024p332}. The hardness of imperfect recall games has motivated the search for subclasses which are polynomial-time solvable~\cite{kline2002minimum,kaneko1995behavior}, or where algorithms similar to the perfect recall case can be applied~\cite{DBLP:conf/icml/LanctotGBB12,DBLP:conf/sigecom/KroerS16}. The class of \emph{A-loss recall}~\cite{kline2002minimum,kaneko1995behavior} is a special kind of imperfect recall, where the loss of information can be traced back to a player forgetting her own action at a point in the past -- the player remembers \emph{where} it was played, but forgets \emph{what} was played. We consider A-loss recall games to be \emph{simple} since there are polynomial-time algorithms for solving them. To the best of our knowledge, A-loss recall games are the biggest known class of imperfect recall with a polynomial-time solution. This has led to research towards finding A-loss recall abstractions~\cite{Cermak::2018}. 

\emph{Contributions.} Our broad goal in this work is to find efficient ways to solve imperfect recall games in extensive-form. We do so by simplifying them into A-loss recall games. We focus on games where the players are not absent-minded: a player is absent-minded if she even forgets whether a decision point was previously seen or not. Here are our major contributions.
\begin{enumerate}\item We first identify a class of one-player games where the player's information structure is more complex than A-loss recall, but shuffling the order of actions results in an equivalent A-loss recall game. This leads to a new $\mathsf{PTIME}$ solvable class of imperfect recall games, that extends A-loss recall (\cref{thm:1p-shuffle-ptime}, \cref{cor:effic-solv-class}, \cref{cor:2-effic-solv-class}). Furthermore, these classes themselves can be tested in $\mathsf{PTIME}$. 

\item We show that every game with \emph{non-absentminded} players can be transformed into an equivalent A-loss recall game (\cref{thm:existence-alr-span}). We present an algorithm to generate an equivalent A-loss recall game with the smallest size.
\end{enumerate}
















 


The caveat in the second result above is that the resulting A-loss recall game could be exponentially bigger. This is expected, since solving imperfect recall games is $\NP$-hard, whereas A-loss recall games can be solved in polynomial-time. The result however shows that in order to solve imperfect recall games, one could either use a worst-case exponential-time algorithm on the original game, or apply our transformation to a worst-case exponential-sized game and run a polynomial-time algorithm on it. From a conceptual point of view, our result shows that as long as there is no absentmindedness, imperfect recall can be transformed into one where the information loss can be attributed to forgetting own actions at a past point.


\emph{Organization of the document.} Section~\ref{sec:an-example} introduces a modification of the popular matching pennies game that will be used as a running example to illustrate our results. Section~\ref{sec:background} recalls necessary preliminaries on extensive-form games. Section~\ref{sec:shuffled-loss-recall} presents the new polynomial-time class of shuffled A-loss recall. Section~\ref{sec:span} generalizes the idea of shuffling to incorporate a ``linear combination'' of action sequences, and presents the second result mentioned above. Section~\ref{sec:two-player} extends the results to the two-player setting. 

  






  



\section{An example}
\label{sec:an-example}



Let us start with a one-player game called the \emph{single team matching-unmatching pennies game}, which will be used as a running example. A team of players with the same goal can be interpreted as a single player. 
In this case, the team consists of two players Alice and Bob, each possessing a coin with two sides, Head (H) and Tail (T) and each of them must choose a side for their respective coins independently. 
The game unfolds in the following manner : a fair $n$-faced die with outcomes from $\{0, \dots, n-1 \}$ is rolled; then Alice chooses a side from $\{H,T\}$, followed by Bob choosing from $\{H,T\}$. Winning or losing depends on the parity of the die outcome. If the outcome of the die is even, then they win if and only if they match their sides. If the outcome is odd, they win if and only if their sides do not match. We consider three variants depending on what Alice and Bob can observe, and model it in extensive form in \cref{fig:match-penny-3-die} for $n=3$. An informal description of the figures follows after this paragraph.
\begin{description}
  \item[I.] Both Alice and Bob observe nothing (\cref{fig:match-penny-3-die-a}).
  \item[II.] Alice can't distinguish between die outcome $2i$ and $2i+1$ for $i \geq 0$,  but Bob observes nothing (\cref{fig:match-penny-3-die-b}).
   \item[III.] Alice can't distinguish between die outcome $2i$ and $2i+1$ for $i \geq 0$, Bob only observes coin of Alice but not outcome of die (\cref{fig:match-penny-3-die-c}). 
 \end{description} 
Alice and Bob want to maximize their \emph{expected payoff}. We will see their possible strategies in Section~\ref{sec:background}.  
Later, we will see that game \textbf{I} falls under the simple class of A-loss recall. 
In Section~\ref{sec:shuffled-loss-recall} and \cref{sec:span} we will see how to simplify games \textbf{II} and \textbf{III} respectively. 
%!TEX root = ../main.tex

\begin{figure}

\begin{subfigure}{0.45\columnwidth}
\centering
\tikzset{
triangle/.style = {regular polygon,regular polygon sides=3,draw,inner sep = 2},
circ/.style = {circle,fill=cyan!10,draw,inner sep = 3},
term/.style = {circle,draw,inner sep = 1.5,fill=black},
sq/.style = {rectangle,fill=gray!20, draw, inner sep = 4}
}
\begin{tikzpicture}[scale=0.8]

\tikzstyle{level 1}=[level distance=18mm,sibling distance=24mm]
\tikzstyle{level 2}=[level distance=11mm,sibling distance=12mm]
\tikzstyle{level 3}=[level distance=11mm,sibling distance=5mm]

\begin{scope}[->, >=stealth]
 \node(0)[triangle]{}
    child{  
    node(00)[circ,draw=black]{}
        child{
        node(000)[circ]{}
            child{
            node(0000)[term,label=below:{\scriptsize $1$}]{} 
            edge from parent node[left,pos = 0.3,inner sep=1.5]{\scriptsize H}
            }
            child{
            node(0001)[term,label=below:{\scriptsize $0$}]{}
            edge from parent node[right,pos = 0.3,inner sep=1.5]{\scriptsize T}                
            }
        edge from parent node[left,pos = 0.2]{{\scriptsize H}}                
        }
        child{
        node(001)[circ]{}
            child{
            node(0010)[term,label=below:{\scriptsize $0$}]{} 
            edge from parent node[left,pos = 0.3,inner sep=1.5]{\scriptsize H}
            }
            child{
            node(0011)[term,label=below:{\scriptsize $1$}]{}
            edge from parent node[right,pos = 0.3,inner sep=1.5]{\scriptsize T}                
            }
        edge from parent node[right,pos = 0.2]{{\scriptsize T}}
        }
    edge from parent node[above,pos = 0.8]{0}
    edge from parent node[above,pos = 0.4]{\scriptsize $\frac{1}{3}$} 
    }
    child{
    node(01)[circ]{}
        child{
        node(010)[circ]{}
            child{
            node(0100)[term,label=below:{\scriptsize $0$}]{}
            edge from parent node[left,pos = 0.3,inner sep=1.5]{\scriptsize H}
            }
            child{
            node(0101)[term,label=below:{\scriptsize $1$}]{}
            edge from parent node[right,pos = 0.3,inner sep=1.5]{\scriptsize T}                
            }
        edge from parent node[left,pos = 0.2]{\scriptsize H}
        }
        child{
        node(011)[circ]{}
            child{
            node(0110)[term,label=below:{\scriptsize $1$}]{} 
            edge from parent node[left,pos = 0.3,inner sep=1.5]{\scriptsize H}
            }
            child{
            node(0111)[term,label=below:{\scriptsize $0$}]{}
            edge from parent node[right,pos = 0.3,inner sep=1.5]{\scriptsize T}                
            } 
        edge from parent node[right,pos = 0.2]{\scriptsize T}
        }
    edge from parent node[right,pos = 0.8]{1}
    edge from parent node[left,pos = 0.4]{\scriptsize $\frac{1}{3}$}
    }
    child{
    node(02)[circ]{}
        child{
        node(020)[circ]{}
            child{
            node(0200)[term,label=below:{\scriptsize $1$}]{}
            edge from parent node[left,pos = 0.3,inner sep=1.5]{\scriptsize H}
            }
            child{
            node(0201)[term,label=below:{\scriptsize $0$}]{}
            edge from parent node[right,pos = 0.3,inner sep=1.5]{\scriptsize T}                
            }
        edge from parent node[left,pos = 0.2]{\scriptsize H}
        }
        child{
        node(021)[circ]{}
            child{
            node(0210)[term,label=below:{\scriptsize $0$}]{} 
            edge from parent node[left,pos = 0.3,inner sep=1.5]{\scriptsize H}
            }
            child{
            node(0211)[term,label=below:{\scriptsize $1$}]{}
            edge from parent node[right,pos = 0.3,inner sep=1.5]{\scriptsize T}                
            } 
        edge from parent node[right,pos = 0.2]{\scriptsize T}
        }
    edge from parent node[above,pos = 0.8]{2}
    edge from parent node[above,pos = 0.4]{\scriptsize $\frac{1}{3}$}
    }
    ;

\end{scope}
\draw [dashed,thick,blue,out=22,in=158] (00) to (01); 
\draw [dashed,thick,blue,out=22,in=158] (01) to (02); 
\draw [dashed,ForestGreen,thick,out=22,in=158] (001) to (010);
\draw [dashed,ForestGreen,thick,out=22,in=158] (000) to (001);
\draw [dashed,ForestGreen,thick,out=22,in=158] (010) to (011);
\draw [dashed,ForestGreen,thick,out=22,in=158] (011) to (020);
\draw [dashed,ForestGreen,thick,out=22,in=158] (020) to (021);

%\node[fit=(1),dashed,thick,blue, draw, circle,inner sep=1pt] {};


%\node[fit=(2),dashed,thick,black, draw, circle,inner sep=1pt] {};

\end{tikzpicture}
\caption{Alice and Bob, both observe nothing}
\label{fig:match-penny-3-die-a}
\end{subfigure}
\vspace{6mm}
\begin{subfigure}{0.48\columnwidth}
\centering
\tikzset{
triangle/.style = {regular polygon,regular polygon sides=3,draw,inner sep = 2},
circ/.style = {circle,fill=cyan!10,draw,inner sep = 3},
term/.style = {circle,draw,inner sep = 1.5,fill=black},
sq/.style = {rectangle,fill=gray!20, draw, inner sep = 4}
}
\begin{tikzpicture}[scale=0.8]

\tikzstyle{level 1}=[level distance=18mm,sibling distance=24mm]
\tikzstyle{level 2}=[level distance=11mm,sibling distance=12mm]
\tikzstyle{level 3}=[level distance=11mm,sibling distance=5mm]

\begin{scope}[->, >=stealth]
 \node(0)[triangle]{}
    child{  
    node(00)[circ,draw=black]{}
        child{
        node(000)[circ]{}
            child{
            node(0000)[term,label=below:{\scriptsize $1$}]{} 
            edge from parent node[left,pos = 0.3,inner sep=1.5]{\scriptsize H}
            }
            child{
            node(0001)[term,label=below:{\scriptsize $0$}]{}
            edge from parent node[right,pos = 0.3,inner sep=1.5]{\scriptsize T}                
            }
        edge from parent node[left,pos = 0.2]{{\scriptsize H}}                
        }
        child{
        node(001)[circ]{}
            child{
            node(0010)[term,label=below:{\scriptsize $0$}]{} 
            edge from parent node[left,pos = 0.3,inner sep=1.5]{\scriptsize H}
            }
            child{
            node(0011)[term,label=below:{\scriptsize $1$}]{}
            edge from parent node[right,pos = 0.3,inner sep=1.5]{\scriptsize T}                
            }
        edge from parent node[right,pos = 0.2]{{\scriptsize T}}
        }
    edge from parent node[above,pos = 0.8]{0} 
    edge from parent node[above,pos = 0.4]{\scriptsize $\frac{1}{3}$}
    }
    child{
    node(01)[circ]{}
        child{
        node(010)[circ]{}
            child{
            node(0100)[term,label=below:{\scriptsize $0$}]{}
            edge from parent node[left,pos = 0.3,inner sep=1.5]{\scriptsize H}
            }
            child{
            node(0101)[term,label=below:{\scriptsize $1$}]{}
            edge from parent node[right,pos = 0.3,inner sep=1.5]{\scriptsize T}                
            }
        edge from parent node[left,pos = 0.2]{\scriptsize H}
        }
        child{
        node(011)[circ]{}
            child{
            node(0110)[term,label=below:{\scriptsize $1$}]{} 
            edge from parent node[left,pos = 0.3,inner sep=1.5]{\scriptsize H}
            }
            child{
            node(0111)[term,label=below:{\scriptsize $0$}]{}
            edge from parent node[right,pos = 0.3,inner sep=1.5]{\scriptsize T}                
            } 
        edge from parent node[right,pos = 0.2]{\scriptsize T}
        }
    edge from parent node[right,pos = 0.8]{1}
    edge from parent node[left,pos = 0.4]{\scriptsize $\frac{1}{3}$}
    }
    child{
    node(02)[circ]{}
        child{
        node(020)[circ]{}
            child{
            node(0200)[term,label=below:{\scriptsize $1$}]{}
            edge from parent node[left,pos = 0.3,inner sep=1.5]{\scriptsize H}
            }
            child{
            node(0201)[term,label=below:{\scriptsize $0$}]{}
            edge from parent node[right,pos = 0.3,inner sep=1.5]{\scriptsize T}                
            }
        edge from parent node[left,pos = 0.2]{\scriptsize H}
        }
        child{
        node(021)[circ]{}
            child{
            node(0210)[term,label=below:{\scriptsize $0$}]{} 
            edge from parent node[left,pos = 0.3,inner sep=1.5]{\scriptsize H}
            }
            child{
            node(0211)[term,label=below:{\scriptsize $1$}]{}
            edge from parent node[right,pos = 0.3,inner sep=1.5]{\scriptsize T}                
            } 
        edge from parent node[right,pos = 0.2]{\scriptsize T}
        }
    edge from parent node[above,pos = 0.8]{2}
    edge from parent node[above,pos = 0.4]{\scriptsize $\frac{1}{3}$}
    }
    ;

\end{scope}
\node[fit=(02),dashed,thick,red, draw, circle,inner sep=1pt] {};
\draw [dashed,thick,blue,out=22,in=158] (00) to (01); 
\draw [dashed,ForestGreen,thick,out=22,in=158] (001) to (010);
\draw [dashed,ForestGreen,thick,out=22,in=158] (000) to (001);
\draw [dashed,ForestGreen,thick,out=22,in=158] (010) to (011);
\draw [dashed,ForestGreen,thick,out=22,in=158] (011) to (020);
\draw [dashed,ForestGreen,thick,out=22,in=158] (020) to (021);
%\node[fit=(1),dashed,thick,blue, draw, circle,inner sep=1pt] {};


%\node[fit=(2),dashed,thick,black, draw, circle,inner sep=1pt] {};

\end{tikzpicture}
\caption{Alice can't distinguish between $2i$ and $2i+1$ for $i \geq 0$, Bob observes nothing}
\label{fig:match-penny-3-die-b}
\end{subfigure}

\begin{subfigure}{\columnwidth}
\centering
\tikzset{
triangle/.style = {regular polygon,regular polygon sides=3,draw,inner sep = 2},
circ/.style = {circle,fill=cyan!10,draw,inner sep = 3},
term/.style = {circle,draw,inner sep = 1.5,fill=black},
sq/.style = {rectangle,fill=gray!20, draw, inner sep = 4}
}
\begin{tikzpicture}[scale=0.8]

\tikzstyle{level 1}=[level distance=18mm,sibling distance=24mm]
\tikzstyle{level 2}=[level distance=11mm,sibling distance=12mm]
\tikzstyle{level 3}=[level distance=11mm,sibling distance=5mm]

\begin{scope}[->, >=stealth]
 \node(0)[triangle]{}
    child{  
    node(00)[circ,draw=black]{}
        child{
        node(000)[circ]{}
            child{
        node(0000)[term,label=below:{\scriptsize $1$}]{} 
            edge from parent node[left,pos = 0.2]{\scriptsize H}
            }
            child{
            node(0001)[term,label=below:{\scriptsize $0$}]{}
            edge from parent node[right,pos = 0.2]{\scriptsize T}                
            }
        edge from parent node[left,pos = 0.2]{{\scriptsize H}}                
        }
        child{
        node(001)[circ]{}
            child{
            node(0010)[term,label=below:{\scriptsize $0$}]{} 
            edge from parent node[left,pos = 0.2]{\scriptsize H}
            }
            child{
            node(0011)[term,label=below:{\scriptsize $1$}]{}
            edge from parent node[right,pos = 0.2]{\scriptsize T}                
            }
        edge from parent node[right,pos = 0.2]{{\scriptsize T}}
        }
    edge from parent node[above,pos = 0.8]{0} 
    edge from parent node[above,pos = 0.4]{\scriptsize $\frac{1}{3}$}
    }
    child{
    node(01)[circ]{}
        child{
        node(010)[circ]{}
            child{
            node(0100)[term,label=below:{\scriptsize $0$}]{}
            edge from parent node[left,pos = 0.2]{\scriptsize H}
            }
            child{
            node(0101)[term,label=below:{\scriptsize $1$}]{}
            edge from parent node[right,pos = 0.2]{\scriptsize T}                
            }
        edge from parent node[left,pos = 0.2]{\scriptsize H}
        }
        child{
        node(011)[circ]{}
            child{
            node(0110)[term,label=below:{\scriptsize $1$}]{} 
            edge from parent node[left,pos = 0.2]{\scriptsize H}
            }
            child{
            node(0111)[term,label=below:{\scriptsize $0$}]{}
            edge from parent node[right,pos = 0.2]{\scriptsize T}                
            } 
        edge from parent node[right,pos = 0.2]{\scriptsize T}
        }
    edge from parent node[right,pos = 0.8]{1}
    edge from parent node[left,pos = 0.4]{\scriptsize $\frac{1}{3}$}
    }
    child{
    node(02)[circ]{}
        child{
        node(020)[circ]{}
            child{
            node(0200)[term,label=below:{\scriptsize $1$}]{}
            edge from parent node[left,pos = 0.2]{\scriptsize H}
            }
            child{
            node(0201)[term,label=below:{\scriptsize $0$}]{}
            edge from parent node[right,pos = 0.2]{\scriptsize T}                
            }
        edge from parent node[left,pos = 0.2]{\scriptsize H}
        }
        child{
        node(021)[circ]{}
            child{
            node(0210)[term,label=below:{\scriptsize $0$}]{} 
            edge from parent node[left,pos = 0.2]{\scriptsize H}
            }
            child{
            node(0211)[term,label=below:{\scriptsize $1$}]{}
            edge from parent node[right,pos = 0.2]{\scriptsize T}                
            } 
        edge from parent node[right,pos = 0.2]{\scriptsize T}
        }
    edge from parent node[above,pos = 0.8]{2}
    edge from parent node[above,pos = 0.4]{\scriptsize $\frac{1}{3}$}
    }
    ;

\end{scope}
\node[fit=(02),dashed,thick,red, draw, circle,inner sep=1pt] {};
\draw [dashed,thick,blue,out=22,in=158] (00) to (01);
\draw [dashed,thick,ForestGreen,out=28,in=152] (000) to (010);
\draw [dashed,ForestGreen,thick,out=28,in=152] (010) to (020);
\draw [dashed,brown,thick,out=28,in=152] (001) to (011);
\draw [dashed,brown,thick,out=28,in=152] (011) to (021);
%\node[fit=(1),dashed,thick,blue, draw, circle,inner sep=1pt] {};


%\node[fit=(2),dashed,thick,black, draw, circle,inner sep=1pt] {};

\end{tikzpicture}
\caption{Bob only observes Alice's coin}
\label{fig:match-penny-3-die-c}
\end{subfigure}


% \begin{subfigure}{.48\columnwidth}
% \centering
% \tikzset{
% triangle/.style = {regular polygon,regular polygon sides=3,draw,inner sep = 2},
% circ/.style = {circle,fill=cyan!10,draw,inner sep = 3},
% term/.style = {circle,draw,inner sep = 1.5,fill=black},
% sq/.style = {rectangle,fill=gray!20, draw, inner sep = 4}
% }
% \begin{tikzpicture}[scale=0.8]

% \tikzstyle{level 1}=[level distance=15mm,sibling distance=23mm]
% \tikzstyle{level 2}=[level distance=10mm,sibling distance=12mm]
% \tikzstyle{level 3}=[level distance=12mm,sibling distance=4mm]

% \begin{scope}[->, >=stealth]
%  \node(0)[triangle]{}
%     child{  
%     node(00)[circ,draw=black]{}
%         child{
%         node(000)[circ]{}
%             child{
%             node(0000)[term,label=below:{\scriptsize $1$}]{} 
%             edge from parent node[left,pos = 0.2]{\scriptsize H}
%             }
%             child{
%             node(0001)[term,label=below:{\scriptsize $0$}]{}
%             edge from parent node[right,pos = 0.2]{\scriptsize T}                
%             }
%         edge from parent node[left,pos = 0.2]{{\scriptsize H}}                
%         }
%         child{
%         node(001)[circ]{}
%             child{
%             node(0010)[term,label=below:{\scriptsize $0$}]{} 
%             edge from parent node[left,pos = 0.2]{\scriptsize H}
%             }
%             child{
%             node(0011)[term,label=below:{\scriptsize $1$}]{}
%             edge from parent node[right,pos = 0.2]{\scriptsize T}                
%             }
%         edge from parent node[right,pos = 0.2]{{\scriptsize T}}
%         }
%     edge from parent node[above,pos = 0.8]{0} 
%     }
%     child{
%     node(01)[circ]{}
%         child{
%         node(010)[circ]{}
%             child{
%             node(0100)[term,label=below:{\scriptsize $0$}]{}
%             edge from parent node[left,pos = 0.2]{\scriptsize H}
%             }
%             child{
%             node(0101)[term,label=below:{\scriptsize $1$}]{}
%             edge from parent node[right,pos = 0.2]{\scriptsize T}                
%             }
%         edge from parent node[left,pos = 0.2]{\scriptsize H}
%         }
%         child{
%         node(011)[circ]{}
%             child{
%             node(0110)[term,label=below:{\scriptsize $1$}]{} 
%             edge from parent node[left,pos = 0.2]{\scriptsize H}
%             }
%             child{
%             node(0111)[term,label=below:{\scriptsize $0$}]{}
%             edge from parent node[right,pos = 0.2]{\scriptsize T}                
%             } 
%         edge from parent node[right,pos = 0.2]{\scriptsize T}
%         }
%     edge from parent node[right,pos = 0.8]{1}
%     }
%     child{
%     node(02)[circ]{}
%         child{
%         node(020)[circ]{}
%             child{
%             node(0200)[term,label=below:{\scriptsize $1$}]{}
%             edge from parent node[left,pos = 0.2]{\scriptsize H}
%             }
%             child{
%             node(0201)[term,label=below:{\scriptsize $0$}]{}
%             edge from parent node[right,pos = 0.2]{\scriptsize T}                
%             }
%         edge from parent node[left,pos = 0.2]{\scriptsize H}
%         }
%         child{
%         node(021)[circ]{}
%             child{
%             node(0210)[term,label=below:{\scriptsize $0$}]{} 
%             edge from parent node[left,pos = 0.2]{\scriptsize H}
%             }
%             child{
%             node(0211)[term,label=below:{\scriptsize $1$}]{}
%             edge from parent node[right,pos = 0.2]{\scriptsize T}                
%             } 
%         edge from parent node[right,pos = 0.2]{\scriptsize T}
%         }
%     edge from parent node[above,pos = 0.8]{2}
%     }
%     ;

% \end{scope}
% \draw [dashed,thick,blue,out=22,in=158] (00) to (01); 
% \draw [dashed,ForestGreen,thick,out=22,in=158] (000) to (021);
% \draw [dashed,brown,thick,out=22,in=158] (001) to (011);
% \draw [dashed,brown,thick,out=22,in=158] (010) to (020);
% %\node[fit=(1),dashed,thick,blue, draw, circle,inner sep=1pt] {};


% %\node[fit=(2),dashed,thick,black, draw, circle,inner sep=1pt] {};

% \end{tikzpicture}
% \caption{Bob can't distinguish between $(i,\text{C})$ and $(i+1 \mod n,\text{C})$ for $i \geq 0 $, $C \in \{H,T\}$ and $n=3$}
% \label{fig:match-penny-3-die-d}
% \end{subfigure}

\caption{Three versions of the single team matching-unmatching pennies game for $n=3$}
\label{fig:match-penny-3-die}
\end{figure}

Before we delve into the background and results, here is a description of the extensive-form model. 
The root node, marked with a triangle, is the event of rolling the die. The triangle nodes are called $\chance$ nodes, and the
edges out of them associate probabilities to each of the outcomes. For this game, the distribution is uniform. The circle nodes denote decision nodes of the team. The nodes in the second level (root being the first level) belong to Alice whereas the nodes in the third level belong to Bob. The actions labelled in edges out of these nodes denote the actions available to the corresponding players. 
A leaf node indicates an end state, and a path from root to leaf denotes
a play from start to end. The number associated with a leaf gives the
payoff that the team receives at the end of the corresponding play. E.g., in \cref{fig:match-penny-3-die-a} in the play resulting from the path $0, H, T$ the payoff is $0$ because the team loses. It is $1$ when they win. 

Imperfect information is expressed using a dotted line: a player cannot distinguish between two nodes joined by a dotted line.
For e.g., in \cref{fig:match-penny-3-die-a} the dotted red line joining all of Alice's nodes indicates that Alice cannot observe the die outcome. Similarly, the blue dotted line for Bob indicates, he neither observes the outcome of the die, nor the side of the coin chosen by Alice. These sets of indistinguishable nodes are called \emph{information sets}.





\section{Background and notations}
\label{sec:background}

%!TEX root = ../main.tex

\begin{figure}
%\begin{center}
\tikzset{
triangle/.style = {regular polygon,regular polygon sides=3,draw,inner sep = 2},
circ/.style = {circle,fill=cyan!10,draw,inner sep = 3},
term/.style = {circle,draw,inner sep = 1.5,fill=black},
sq/.style = {rectangle,fill=gray!20, draw, inner sep = 4}
}

\begin{subfigure}{.45\columnwidth}
\centering
\begin{tikzpicture}[scale=0.9]
\tikzstyle{level 1}=[level distance=9mm,sibling distance = 22mm]
\tikzstyle{level 2}=[level distance=7mm,sibling distance=10mm]
\tikzstyle{level 3}=[level distance=7mm,sibling distance=6mm]
\tikzstyle{level 4}=[level distance=7mm,sibling distance=5mm]

%node (ij) is the j th node in i th level

\begin{scope}[->, >=stealth]
\node (0) [circ] {}
child {
  node (00) [triangle] {}
  child {
    node (000) [circ] {}
    child {
      node (0000) [term, label=below:{}] {}
      edge from parent node [left] {\scriptsize $c$}
    }
    child {
      node (0001) [term, label=below:{}] {}
      edge from parent node [right] {\scriptsize $d$}
      }
    edge from parent node [left] {}
  }
  child {
    node (001) [circ] {}
    child {
      node (0010) [term, label=below:{}] {}
      edge from parent node [left] {\scriptsize $c$}
    }
    child {
      node (0011) [term, label=below:{}] {}
      edge from parent node [right] {\scriptsize $d$}
      }
    edge from parent node [right] {} 
  }
  edge from parent node [above] {\scriptsize$a$}
}
child {
  node (01) [triangle] {}
   child {
     node (010) [circ] {}
     child {
      node (0100) [term, label=below:{}] {}
      edge from parent node [left] {\scriptsize $e$}
    }
    child {
      node (0101) [term, label=below:{}] {}
      edge from parent node [right] {\scriptsize $f$}
      }
    edge from parent node [left] {}
  }
  child {
    node (011) [circ] {}
    child {
      node (0110) [term, label=below:{}] {}
      edge from parent node [left] {\scriptsize $e$}
    }
    child {
      node (0111) [term, label=below:{}] {}
      edge from parent node [right] {\scriptsize $f$}
      }
    edge from parent node [right] {} 
  }
  edge from parent node [above] {\scriptsize$b$}
}
;
\end{scope}

%observations
%\draw [dashed, thick, red, in=150,out=30](00) to (01) ;

  \node[fit=(0),dashed,thick,red, draw, circle,inner sep=1pt] {};
\draw [dashed, thick, blue, in=150,out=30] (000) to (001) ;
\draw [dashed, thick, ForestGreen, in=150,out=30] (010) to (011);

%node labels
\node [black] at (0,0.35) {\scriptsize $r$};
\node [black] at (-1,-0.55) {\scriptsize $u_1$};
\node [black] at (1, -0.55) {\scriptsize $u_2$};
\node [black] at (-2, -1.5) {\scriptsize $u_3$};
\node [black] at (-.25, -1.5) {\scriptsize $u_4$};

\node [black] at (0.25, -1.5) {\scriptsize $u_5$};
\node [black] at (2, -1.5) {\scriptsize $u_6$};

%obs labels
\node [red] at (0,-.5) {\scriptsize $I_1$};
\node [blue] at (-1.1,-1.6) {\scriptsize $I_2$};
\node [ForestGreen] at (1.1,-1.6) {\scriptsize $I_3$};



\end{tikzpicture}

\caption{$\Max$ with perfect recall}
\label{fig-allexmp-pftrec}
\end{subfigure}
\quad
\begin{subfigure}{.45\columnwidth}
\centering
\begin{tikzpicture}
\tikzstyle{level 1}=[level distance=7mm,sibling distance = 10mm]
\tikzstyle{level 2}=[level distance=7mm,sibling distance=10mm]
\tikzstyle{level 3}=[level distance=7mm,sibling distance=15mm]
\tikzstyle{level 4}=[level distance=7mm,sibling distance=8mm]

%\draw [help lines, step=0.5] (-3,-3) grid (3,0);

\begin{scope}[->, >=stealth]
\node (0) [circ] {}
child{
  node (1) [circ] {}
  child{
    node (3) [term, label=below:{}] {}
    edge from parent node [left] {\scriptsize $a$}
  }
  child{
    node (4) [term,label=below:{}] {}
    edge from parent node [right] {\scriptsize $b$}
  }
  edge from parent node [left] {\scriptsize $a$}
}
child{
  node (2) [term, label=below:{}] {}
  edge from parent node [right] {\scriptsize $b$}
}
;
\end{scope}

\draw [dashed, thick, blue, in=10,out=-100] (0) to (1);



\node [black] at (0,0.25) {\scriptsize $r$};
\node [black] at (-.9,-0.6) {\scriptsize $u_1$};



\node [blue] at (.1,-.6) {\scriptsize $I_1$};

\end{tikzpicture}
\caption{$\Max$ with absentmindedness}
\label{fig-allexmp-absentm}
\end{subfigure}


%\end{center}
\caption{Recalls of $\Max$}
\label{fig:recall-examples}
\end{figure}

This section presents the formal definitions. The single team matching-unmatching pennies game has only one player and chance nodes, but in general we will talk about zero-sum two player games. As in \cref{fig:2-p-shuffle}, there are two players $\Max$ (circle nodes) and $\Min$ (square nodes). The payoff at the leaf, is the amount $\Min$ loses and $\Max$ gains. The goal of $\Max$ is to maximize the expected payoff whereas $\Min$ wishes to minimize it. In \cref{fig:match-penny-3-die} $\Max$ was the team consisting of Alice and Bob.







In this paper, we mainly work with \emph{game-structures} and not games
themselves. Game-structures are essentially games sans the numerical
quantities. Any game on a game structure can be represented symbolically as in shown \cref{fig:alossSpan-a} with symbolic payoffs $z_i$s and symbolic chance probabilities $p_i$s (with constraints on $p_i$'s). An extensive
form game can be obtained from a game structure by plugging in values for $z_i$s and $p_i$s.  We work with game structures because the
notions of perfect recall and imperfect recall can be determined
simply by looking at the game-structure.


Formally, a game-structure $\Tt$ is a tuple $(V, L, r, A, E, \Ii)$
where $V$ is a finite set of non-terminal nodes partitioned as
$V_{\Max}$, $V_{\Min}$ and $V_{\chance}$; $L$ is a finite set of leaves;
$r \in V$ is a root node; $A = A_{\Max} \cup A_{\Min}$ is a finite set
of actions; $E \incl V \times (V \cup L)$ is an edge
relation that induces a directed tree; edges originating from $V_{\Max} \cup V_{\Min}$ are labelled with actions from $A$; we write $u \xra{a} v$ if
$(u, v)$ is labelled with $a$, and assume that there is no incoming edge
$u \xra{} r$ to the root node $r$; $\Ii = \Ii_{\Max} \cup \Ii_{\Min}$
is a set of information sets for $i \in \{ \Max, \Min \}$, each
information set $I \in \Ii_i$ is a subset of vertices belonging to
$i$, i.e. $I \incl V_i$, and moreover, the set of information sets
$\Ii_i$ partitions $V_i$. E.g., in \cref{fig-allexmp-pftrec},
$\Ii_{\Max} = \{I_1, I_2, I_3\}$ and $I_1 = \{r\}, I_2 = \{u_3, u_4\}$
and $I_3 = \{u_5, u_6\}$. We can understand these information sets as a signal that the player receives when she reaches a node in it. On receiving the signal, the player knows the actions that are available to play at that position. 

An information set models the fact that a player cannot distinguish
between the nodes within it. Therefore, the set of outgoing actions
from each node in an information set is required to be the same. This
allows us to define $\act(I)$ as the set of actions available at
information set $I$. E.g., in \cref{fig-allexmp-pftrec},
$\act(I_2) = \{c, d\}$. For technical convenience, we make a second
assumption: for all $I, I' \in \Ii$ with $I \neq I'$, we have
$\act(I) \cap \act(I') = \emptyset$. Therefore, the actions identify
the information sets. With this assumption, in \cref{fig:match-penny-3-die}, the actions of Alice should be seen as $H_A, T_A$ and those of Bob's as $H_B, T_B$. But we omit the subscripts in the figure for clarity. 
\begin{definition}[Extensive form games]\label{def:ext-form-games}
  A two-player zero-sum game in extensive form is a tuple
  $(\Tt,\d, \Uu)$ where $\Tt$ is a game-structure, $\d$ is the
  \emph{chance probability} associating to each $\chance$ node, a
  probability distribution on the outgoing actions, and
  $\Uu : L \mapsto \Rat $ is the utility function associating a payoff
  to each leaf.
\end{definition}

The \emph{size} of a game is the sum of the bit-lengths of all chance probabilities and leaf
payoffs in it. A \emph{behavioral strategy} for player $\Max$ ($\Min$ resp.) assigns a probability
distribution to $\act(I)$ for each $I \in \Ii_{\Max}$ ($\Ii_{\Min}$ resp.). Once we fix behavioral strategies $\sigma$ and $\tau$ for $\Max$ and $\Min$ respectively,
each edge in the game has an associated probability of being taken,
given by the corresponding strategy or $\chance$. The probability of reaching a leaf $u \in L$ is given by the product of all the numbers along the path to the leaf. Consider \cref{fig:shuffle-a}.
Let $\sigma$ assign $\frac{1}{4}$ to $b$ and $\frac{3}{4}$ to
$\bar{b}$; $0$ and $1$ to $c$ and $\bar{c}$, and $\frac{1}{3}$ to $a$
and $\frac{2}{3}$ to $\bar{a}$. The probability of reaching the leaf $b \bar{a}$ is 
then: $p_1 \times \frac{1}{4} \times \frac{2}{3}$. For a leaf $u$, we denote this quantity by
$\prob_{\sigma, \tau}(u)$. The \emph{expected payoff} $\Ee(\s, \t)$
when $\Max$ plays $\sigma$ and $\Min$ plays $\tau$, then equals
$\sum_{u \in L} \prob_{\s, \t} (u) \Uu(u)$. The solution concept that we
will consider in this paper is the notion of maxmin.
The \emph{maxmin value} of a game is the
following: \[\max\limits_{\s}\min\limits_{\t}\Ee(\s,\t)\] where
$\s,\t$ are behavioral strategies of $\Max$ and $\Min$ respectively. A
strategy of $\Max$ which provides the maxmin value is called a
\emph{maxmin strategy}. In one-player games, we only have $\Max$ player and the maxmin value of the game is $\max\limits_{\s}\Ee(\s)$. For one-player non-absentminded games, the maxmin value can be in fact obtained by a \emph{pure strategy} -- pure strategies are special cases of behavioural strategies which assign either $0$ or $1$ to each action~\cite{KollerMegiddo::1992}.

The maxmin value of the game in \cref{fig:match-penny-3-die-a} is $\frac{2}{3}$ since Alice and Bob can win at most in 2 of the 3 die rolls by playing matching sides. Another way to see this is to consider the four possible pure strategies $HH, HT, TH, TT$, which induce payoffs $\frac{2}{3}$, $\frac{1}{3}$, $\frac{1}{3}$ and $\frac{2}{3}$ respectively. Now since, in the rest of the following two versions, the team has more information \footnote{This can be observed by the fact that information sets in each version are refinements of the previous versions.} they can guarantee at least $\frac{2}{3}$ by playing the same strategy. Interestingly, one can observe (by enumerating all pure strategies) that they cannot do better than that in any version. 
\paragraph*{Histories and recalls.} We now move on to describing the
various types of imperfect information, based on what the player
remembers about her history. A node $w \in V$ is reached by a unique
path from the root: $r = v_0 \xra{} v_1 \xra{} \cdots \xra{} v_n =
w$. Let $v_{i_1}, v_{i_2}, \dots, v_{i_k}$ be the vertices in this
sequence which do not belong to $\chance$. Then,
$\his(w) = a_{1} a_{2} \cdots a_{k-1}$, where $v_{i_j} \xra{a_j} v_{i_{j + 1}}$.
For a player $i \in \{\Max, \Min\}$ the history of $i$ at $w$, denoted
by $\his_i(w)$, is the sequence of player $i$'s actions in the path to
$w$, which is simply the sub-sequence of $\his(w)$ restricted to
actions from $A_i$. E.g.: in \cref{fig-allexmp-pftrec},
$\his_{\Max}(u_3) = \his_{\Max}(u_4) = a$; in
\cref{fig:shuffle-a}, $\his_{\Max}(u_3) = b$ and $\his_{\Max}(u_2) = \epsilon$, the empty sequence. It is important to remark that this definition uses the assumption that actions determine information sets -- otherwise, we would need to incorporate the information sets that were visited along the way, into the history.



Let $\Hh$ denote the set of all histories and $\Hh_i$ be the set of
all histories of player $i$. For an information set $I \in \Ii_i$ let
$\Hh(I) = \{ \his(u) \mid u \in I\}$ be the set of histories of all
nodes in $I$. Similarly, we can define $\Hh_i(I)$ with respect to
$\Hh_i$. Let $\Hh(L)$ denote the set of all leaf histories.
When $\Hh_i(I)$ has multiple histories, at a node $v \in I$ the player
does not remember which history she traversed to reach $v$. Hence the
player loses information. For two
nodes $u$ and $v$ in $I$, comparing $\his_i(u)$ and $\his_i(v)$
reveals the loss or retention of previously withheld information at
the respective nodes. To capture this there are different notions of
\emph{recall}.

\emph{Perfect recall.} Player $i$ is said to have \emph{perfect
  recall} ($\pfr$) if for every $I \in \Ii_i$, and every pair of
distinct vertices $u, v \in I$, we have $\his_i(u) = \his_i(v)$,
i.e. $|\Hh_i(I)| = 1$.  Otherwise, the player is said to have imperfect
recall.  \cref{fig-allexmp-pftrec} is an example of a perfect recall
game.
\emph{Imperfect recall.} \cref{fig:shuffle-a} gives an example of
a game-structure that has imperfect recall. Notice that states $u_3$
and $u_4$ lie in the same information set $I_3$, but the sequence of
the player's actions leading to these states is different: history at
$u_3$ is $b$, whereas at $u_4$ it is $\bar{b}$. 
Within imperfect recall,
there are distinctions. The imperfect recall in
~\cref{fig:shuffle-b} and the one in ~\cref{fig:shuffle-a}
are in some sense different: in ~\cref{fig:shuffle-b}, the
inability to distinguish between the two nodes in $I_1$ can be traced back to
a point in the past where she forgets her own action from some
information set ($I_3$ in this case), whereas in \cref{fig:shuffle-a}, the player has been able to
distinguish between the two outcomes of the $\chance$ node, but later
forgets at $I_3$ where she started from, leading to four histories $b,\bar{b},c$ and $\bar{c}$ at $I_3$.


\emph{A-loss recall.} Game-structures as in \cref{fig:shuffle-b} are said to have \emph{A-loss
  recall}. A consequence of having A-loss recall is that a player always remembers any new information
gained from $\chance$ outcomes, which is not the case
in~\cref{fig:shuffle-a}. Player $i$ has \emph{A-loss recall}
($\alr$) if for all $I \in \Ii_i$, and every pair of distinct vertices
$u, v \in I$, either $\his_i(u) = \his_i(v)$, or $\his_i(u)$ is of the
form $s a s_1$, and $\his_i(v)$ of the form $s b s_2$, where
$a, b \in \act(I')$ for some $I' \in \Ii$, with $a \neq b$. The game in \cref{fig:match-penny-3-die-a} has A-loss recall, whereas the others, \cref{fig:match-penny-3-die-b} and \cref{fig:match-penny-3-die-c} do not.  

Finally, player $i$ is said to be
\emph{non-absentminded} ($\nam$) if $\forall u, v \in V_i$ with $u$
lying on the path to $v$, the information set that $u$ belongs to is
different from the information set that $v$ belongs
to, i.e. all nodes of $i$ on a path from $r$ to leaf node lie in distinct
information sets. \cref{fig-allexmp-absentm} is an example where
$\Max$ is absentminded, since both $r$ and $u_1$ lie in the same
information set. Notice that $\pfr$ implies $\alr$, which in turn implies implies $\nam$. 

When $\Max$ and $\Min$ have recalls $R_{\Max}, R_{\Min} \in \{ $\pfr$,~$\alr$,~$\nam$ \} $
respectively we will denote the game as a
$(R_{\Max}, R_{\Min})$-game. A one-player game with recall $R$ is denoted as $R$-game. In this paper we are only concerned with one-player $\nam$-games and two-player $(\nam,\nam)$-games. Let us now recall some known
results.
\begin{itemize}\item A maxmin solution in a $(\pfr,\alr)$-game can be computed in
  polynomial- time~\\\cite{KollerMegiddo::1992,vonStengel::1996,kaneko1995behavior}. As a corollary, an optimal solution in a one-player $\alr$-game can be computed in
  polynomial-time ~\cite{kaneko1995behavior}.

\item The maxmin decision problem for $(\nam,\nam)$-games is both
  $\NP$-hard~\cite{KollerMegiddo::1992} and

  
  $\sqsum$-hard~\cite{GPS20} \footnote{$\sqsum$ is the decision problem of checking
    if the sum of the square roots of $k$ positive integers is less
    than another positive number}.
   The $\NP$-hardness and the $\sqsum$-hardness hold even for
  $(\alr,\pfr)$-games~\cite{Cermak::2018,GPS20}. The maxmin decision problem for one-player $\nam$-games is
  $\NP$-complete~\cite{KollerMegiddo::1992}.
\end{itemize}


Our core idea is to view game structures through the polynomials they generate. 
\paragraph*{Leaf monomials.} In a game structure, assigning variable $x_a$ to each action
$a$, the monomial obtained by taking the product of all $x_a$ along the path to each leaf $t$ is called
a \emph{leaf monomial}, and denoted as $\mu(t)$. E.g., the leaf
monomials of the game-structure in \cref{fig-allexmp-pftrec} are
$\{ x_ax_c, x_ax_d, x_bx_e, x_bx_f\}$. For a game structure $\Tt$, we will write $X(\Tt)$ for the set of leaf monomials. For a game $G$, let
$\prob_{\chance}(t)$ denote the product of $\chance$ probabilities in
the path to $t$. The polynomial given by
$\sum\limits_{t \in L} \prob_{\chance}(t) \cdot \Uu(t) \cdot \mu(t)$
is called the \emph{payoff polynomial} of a game. 
A constraint of the form $\sum\limits_{a \in \act(I)} x_a = 1$ for an
information set $I$ will be called a \emph{strategy constraint}. Any non-negative valuation satisfying these constraints gives a behavioral strategy to the players.
The maxmin value in a game can be given by the maxmin of the payoff polynomial over all possible values satisfying the strategy constraints.


\paragraph*{Overview of our work}
In this work, our mantra for simplifying games is to find simpler games with same payoff polynomials (upto renaming of variables). Leaf monomials are the building blocks of payoff polynomials. We give methods to generate from a given game-structure $\Tt$, a transformed game-structure $\Tt'$ with A-loss recall such that: either $\Tt'$ has the same set of leaf monomials (Section~\ref{sec:shuffled-loss-recall}), or each leaf monomial of $\Tt$ is a linear combination of the leaf monomials of $\Tt'$ (Section~\ref{sec:span}).


\endinput



















%!TEX root = ../main.tex

\begin{figure}
\begin{subfigure}{0.5\columnwidth}
\centering
\tikzset{
triangle/.style = {regular polygon,regular polygon sides=3,draw,inner sep = 2},
circ/.style = {circle,fill=cyan!10,draw,inner sep = 3},
term/.style = {circle,draw,inner sep = 1.5,fill=black},
sq/.style = {rectangle,fill=gray!20, draw, inner sep = 4}
}

\begin{tikzpicture}[scale=0.85]
\tikzstyle{level 1}=[level distance=9mm,sibling distance = 22mm]
\tikzstyle{level 2}=[level distance=7mm,sibling distance=10mm]
\tikzstyle{level 3}=[level distance=7mm,sibling distance=6mm]
\tikzstyle{level 4}=[level distance=7mm,sibling distance=5mm]

%node (ij) is the j th node in i th level

\begin{scope}[->, >=stealth]
\node (0) [triangle] {}
child {
  node (00) [circ] {}
  child {
    node (000) [circ] {}
    child {
      node (0000) [term, label=below:{\scriptsize $z_1$}] {}
      edge from parent node [left] {\scriptsize $a$}
    }
    child {
      node (0001) [term, label=below:{\scriptsize $z_2$}] {}
      edge from parent node [right] {\scriptsize $\bar{a}$}
      }
    edge from parent node [left] {\scriptsize $b$}
  }
  child {
    node (001) [circ] {}
    child {
      node (0010) [term, label=below:{\scriptsize $z_3$}] {}
      edge from parent node [left] {\scriptsize $a$}
    }
    child {
      node (0011) [term, label=below:{\scriptsize $z_4$}] {}
      edge from parent node [right] {\scriptsize $\bar{a}$}
      }
    edge from parent node [right] {\scriptsize $\bar{b}$} 
  }
  edge from parent node [above] {\scriptsize $p_1$}
}
child {
  node (01) [circ] {}
   child {
     node (010) [circ] {}
     child {
      node (0100) [term, label=below:{\scriptsize $z_5$}] {}
      edge from parent node [left] {\scriptsize $a$}
    }
    child {
      node (0101) [term, label=below:{\scriptsize $z_6$}] {}
      edge from parent node [right] {\scriptsize $\bar{a}$}
      }
    edge from parent node [left] {\scriptsize $c$}
  }
  child {
    node (011) [circ] {}
    child {
      node (0110) [term, label=below:{\scriptsize $z_7$}] {}
      edge from parent node [left] {\scriptsize $a$}
    }
    child {
      node (0111) [term, label=below:{\scriptsize $z_8$}] {}
      edge from parent node [right] {\scriptsize $\bar{a}$}
      }
    edge from parent node [right] {\scriptsize $\bar{c}$} 
  }
  edge from parent node [above] {\scriptsize $p_2$}
}
;
 \node[fit=(00),dashed,thick,blue, draw, circle,inner sep=1pt] {};
  \node[fit=(01),dashed,thick,red, draw, circle,inner sep=1pt] {};
\end{scope}

\draw [dashed, thick, ForestGreen, in=150,out=30] (000) to (001);
\draw [dashed, thick, ForestGreen, in=150,out=30] (001) to (010);
\draw [dashed, thick, ForestGreen, in=150,out=30] (010) to (011);
%\draw [dashed, thick, blue, in=150,out=30] (000) to (001);
%\draw [dashed, thick, red, in=150,out=30] (010) to (011);

\node [black] at (0,0.35) {\scriptsize $r$};
\node [black] at (-1.5,-0.55) {\scriptsize $u_1$};
\node [black] at (1.5, -0.55) {\scriptsize $u_2$};
\node [black] at (-2, -1.6) {\scriptsize $u_3$};
\node [black] at (-.25, -1.7) {\scriptsize $u_4$};

\node [black] at (0.25, -1.7) {\scriptsize $u_5$};
\node [black] at (2, -1.6) {\scriptsize $u_6$};

%obs labels
\node [ForestGreen] at (0,-1.1) {\scriptsize $I_3$};
\node [blue] at (-.55,-.9) {\scriptsize $I_1$};
\node [red] at (.55,-.9) {\scriptsize $I_2$};

\end{tikzpicture}
\caption{$\Max$ without $\alr$ but has $\salr$}
\label{fig:shuffle-a}
\end{subfigure}
\begin{comment}
\begin{subfigure}{0.45\columnwidth}
%\centering
\tikzset{
triangle/.style = {regular polygon,regular polygon sides=3,draw,inner sep = 2},
circ/.style = {circle,fill=cyan!10,draw,inner sep = 3},
term/.style = {circle,draw,inner sep = 1.5,fill=black},
sq/.style = {rectangle,fill=gray!20, draw, inner sep = 4}
}

\begin{tikzpicture}[scale=0.85]
\tikzstyle{level 1}=[level distance=9mm,sibling distance = 22mm]
\tikzstyle{level 2}=[level distance=7mm,sibling distance=10mm]
\tikzstyle{level 3}=[level distance=7mm,sibling distance=6mm]
\tikzstyle{level 4}=[level distance=7mm,sibling distance=5mm]

%node (ij) is the j th node in i th level

\begin{scope}[->, >=stealth]
\node (0) [circ] {}
child {
  node (00) [triangle] {}
  child {
    node (000) [circ] {}
    child {
      node (0000) [term, label=below:{}] {}
      edge from parent node [left] {\scriptsize $b$}
    }
    child {
      node (0001) [term, label=below:{}] {}
      edge from parent node [right] {\scriptsize $\bar{b}$}
      }
    edge from parent node [left] {}
  }
  child {
    node (001) [circ] {}
    child {
      node (0010) [term, label=below:{}] {}
      edge from parent node [left] {\scriptsize $b$}
    }
    child {
      node (0011) [term, label=below:{}] {}
      edge from parent node [right] {\scriptsize $\bar{b}$}
      }
    edge from parent node [right] {} 
  }
  edge from parent node [above] {\scriptsize $a$}
}
child {
  node (01) [triangle] {}
   child {
     node (010) [circ] {}
     child {
      node (0100) [term, label=below:{}] {}
      edge from parent node [left] {\scriptsize $c$}
    }
    child {
      node (0101) [term, label=below:{}] {}
      edge from parent node [right] {\scriptsize $\bar{c}$}
      }
    edge from parent node [left] {}
  }
  child {
    node (011) [circ] {}
    child {
      node (0110) [term, label=below:{}] {}
      edge from parent node [left] {\scriptsize $c$}
    }
    child {
      node (0111) [term, label=below:{}] {}
      edge from parent node [right] {\scriptsize $\bar{c}$}
      }
    edge from parent node [right] {} 
  }
  edge from parent node [above] {\scriptsize $\bar{a}$}
}
;
\end{scope}

%\draw [dashed, thick, ForestGreen, in=150,out=30] (00) to (01);

\node[fit=(0),dashed,thick,ForestGreen, draw, circle,inner sep=1pt] {};
\draw [dashed, thick, blue, in=150,out=30] (000) to (010);
\draw [dashed, thick, red, in=150,out=30] (001) to (011);


\node [black] at (0,0.45) {\scriptsize $r$};
\node [black] at (-1.1,-0.45) {\scriptsize $u_1$};
\node [black] at (1.1, -0.45) {\scriptsize $u_2$};
\node [black] at (-2, -1.65) {\scriptsize $u_3$};
\node [black] at (-.2, -1.65) {\scriptsize $u_4$};

\node [black] at (0.25, -1.65) {\scriptsize $u_5$};
\node [black] at (2, -1.65) {\scriptsize $u_6$};

%obs labels
\node [ForestGreen] at (0,-0.6) {\scriptsize $I_3$};
\node [blue] at (-0.35,-1) {\scriptsize $I_1$};
\node [red] at (0.35,-1) {\scriptsize $I_2$};

\end{tikzpicture}
\caption{}
\label{fig:shuffle-c}
\end{subfigure}%
\end{comment}
\begin{subfigure}{0.48\columnwidth}
\centering
\tikzset{
triangle/.style = {regular polygon,regular polygon sides=3,draw,inner sep = 2},
circ/.style = {circle,fill=cyan!10,draw,inner sep = 3},
term/.style = {circle,draw,inner sep = 1.5,fill=black},
sq/.style = {rectangle,fill=gray!20, draw, inner sep = 4}
}

\begin{tikzpicture}[scale=0.85]
\tikzstyle{level 1}=[level distance=9mm,sibling distance = 22mm]
\tikzstyle{level 2}=[level distance=7mm,sibling distance=10mm]
\tikzstyle{level 3}=[level distance=7mm,sibling distance=6mm]
\tikzstyle{level 4}=[level distance=7mm,sibling distance=5mm]

%node (ij) is the j th node in i th level

\begin{scope}[->, >=stealth]
\node (0) [circ] {}
child {
  node (00) [triangle] {}
  child {
    node (000) [circ] {}
    child {
      node (0000) [term, label=below:{\scriptsize $z_1$}] {}
      edge from parent node [left] {\scriptsize $b$}
    }
    child {
      node (0001) [term, label=below:{\scriptsize $z_3$}] {}
      edge from parent node [right] {\scriptsize $\bar{b}$}
      }
    edge from parent node [left] {}
  }
  child {
    node (001) [circ] {}
    child {
      node (0010) [term, label=below:{\scriptsize $z_5$}] {}
      edge from parent node [left] {\scriptsize $c$}
    }
    child {
      node (0011) [term, label=below:{\scriptsize $z_7$}] {}
      edge from parent node [right] {\scriptsize $\bar{c}$}
      }
    edge from parent node [right] {} 
  }
  edge from parent node [above] {\scriptsize $a$}
}
child {
  node (01) [triangle] {}
   child {
     node (010) [circ] {}
     child {
      node (0100) [term, label=below:{\scriptsize $z_2$}] {}
      edge from parent node [left] {\scriptsize $b$}
    }
    child {
      node (0101) [term, label=below:{\scriptsize $z_4$}] {}
      edge from parent node [right] {\scriptsize $\bar{b}$}
      }
    edge from parent node [left] {}
  }
  child {
    node (011) [circ] {}
    child {
      node (0110) [term, label=below:{\scriptsize $z_6$}] {}
      edge from parent node [left] {\scriptsize $c$}
    }
    child {
      node (0111) [term, label=below:{\scriptsize $z_8$}] {}
      edge from parent node [right] {\scriptsize $\bar{c}$}
      }
    edge from parent node [right] {} 
  }
  edge from parent node [above] {\scriptsize $\bar{a}$}
}
;
\end{scope}

%\draw [dashed, thick, ForestGreen, in=150,out=30] (00) to (01);

\node[fit=(0),dashed,thick,ForestGreen, draw, circle,inner sep=1pt] {};
\draw [dashed, thick, blue, in=150,out=30] (000) to (010);
\draw [dashed, thick, red, in=150,out=30] (001) to (011);

%\node [black] at (0,0.45) {\scriptsize $r$};
%\node [black] at (-1.1,-0.45) {\scriptsize $u_1$};
%\node [black] at (1.1, -0.45) {\scriptsize $u_2$};
%\node [black] at (-2, -1.65) {\scriptsize $u_3$};
%\node [black] at (-.2, -1.65) {\scriptsize $u_4$};
%
%\node [black] at (0.25, -1.65) {\scriptsize $u_5$};
%\node [black] at (2, -1.65) {\scriptsize $u_6$};

%obs labels
\node [ForestGreen] at (0,-0.6) {\scriptsize $I_3$};
\node [blue] at (-0.35,-1) {\scriptsize $I_1$};
\node [red] at (0.35,-1) {\scriptsize $I_2$};

\node[black] at (-1.5,-.95) {\scriptsize $p_1$};
\node[black] at (-.73,-.95) {\scriptsize $p_2$};

\node[black] at (1.5,-.95) {\scriptsize $p_2$};
\node[black] at (.73,-.95) {\scriptsize $p_1$};
\end{tikzpicture}
\caption{$\Max$ with $\alr$}
\label{fig:shuffle-b}
\end{subfigure}
\caption{Equivalent $\alr$ game using $\salr$ for game without $\alr$ }
\label{fig:shuffle}
\end{figure}


\paragraph*{Shuffled A-loss recall}
The game-structure in \cref{fig:shuffle-a} does not have A-loss recall. This is because the player knows about $\chance$ outcomes $I_1$ and $I_2$ which she forgets at $I_3$.  Now, consider the game-structure in~\cref{fig:shuffle-c}, obtained by \emph{shuffling} the actions ($a$ goes above $b$ and $c$). This game-structure has A-loss recall. The crucial observation is that both the game-structures, \cref{fig:shuffle-a} and \cref{fig:shuffle-c}, lead to the same ``leaf monomials'': on assigning variable $x_a$ to an action labeled $a$, the product of the variables along the path to each leaf produces a leaf monomial. For instance, the leaf monomials for the game-structures in ~\cref{fig:shuffle-a} and ~\cref{fig:shuffle-c} respectively are $\{x_ax_b,x_ax_{\bar{b}},x_{\bar{a}}x_b,x_{\bar{a}}x_{\bar{b}},x_ax_c, x_ax_{\bar{c}},x_{\bar{a}}x_c,x_{\bar{a}}x_{\bar{c}} \}$.
We say that the game-structure of ~\cref{fig:shuffle-a} has \emph{shuffled A-loss recall}. Even though the game originally does not have A-loss recall, it can be shuffled in some way to get an A-recall structure. 
Not every game-structure has shuffled A-loss recall.

Our results:
\begin{itemize}
\item We provide a polynomial-time algorithm to identify whether a game-structure has shuffled A-loss recall. If the answer is yes, the algorithm also computes the shuffled game-structure.

\item As a result, we are able to show that one-player shuffled A-loss recall games can be solved in polynomial-time. Similarly, we deduce that two player games between a perfect recall player and a shuffled A-loss recall player can be solved in polynomial-time.
\end{itemize}






%!TEX root = ../main.tex

\begin{figure}
\centering

\begin{subfigure}{.3\columnwidth}
%\centering
\tikzset{
triangle/.style = {regular polygon,regular polygon sides=3,draw,inner sep = 2},
circ/.style = {circle,fill=cyan!10,draw,inner sep = 3},
term/.style = {circle,draw,inner sep = 1.5,fill=black},
sq/.style = {rectangle,fill=gray!20, draw, inner sep = 4}
}

\begin{tikzpicture}[scale=0.85]
\tikzstyle{level 1}=[level distance=9mm,sibling distance = 22mm]
\tikzstyle{level 2}=[level distance=7mm,sibling distance=10mm]
\tikzstyle{level 3}=[level distance=7mm,sibling distance=6mm]
\tikzstyle{level 4}=[level distance=7mm,sibling distance=5mm]

%node (ij) is the j th node in i th level

\begin{scope}[->, >=stealth]
\node (0) [triangle] {}
child {
  node (00) [circ] {}
  child {
    node (000) [circ] {}
    child {
      node (0000) [term, label=below:{\scriptsize $z_1$}] {}
      edge from parent node [left] {\scriptsize $c$}
    }
    child {
      node (0001) [term, label=below:{\scriptsize $z_2$}] {}
      edge from parent node [right] {\scriptsize $\bar{c}$}
      }
    edge from parent node [left] {\scriptsize $a$}
  }
  child {
    node (001) [circ] {}
    child {
      node (0010) [term, label=below:{\scriptsize $z_3$}] {}
      edge from parent node [left] {\scriptsize $d$}
    }
    child {
      node (0011) [term, label=below:{\scriptsize $z_4$}] {}
      edge from parent node [right] {\scriptsize $\bar{d}$}
      }
    edge from parent node [right] {\scriptsize $\bar{a}$} 
  }
  edge from parent node [above] {\scriptsize $p_1$}
}
child {
  node (01) [circ] {}
   child {
     node (010) [circ] {}
     child {
      node (0100) [term, label=below:{\scriptsize $z_5$}] {}
      edge from parent node [left] {\scriptsize $c$}
    }
    child {
      node (0101) [term, label=below:{\scriptsize $z_6$}] {}
      edge from parent node [right] {\scriptsize $\bar{c}$}
      }
    edge from parent node [left] {\scriptsize $b$}
  }
  child {
    node (011) [circ] {}
    child {
      node (0110) [term, label=below:{\scriptsize $z_7$}] {}
      edge from parent node [left] {\scriptsize $d$}
    }
    child {
      node (0111) [term, label=below:{\scriptsize $z_8$}] {}
      edge from parent node [right] {\scriptsize $\bar{d}$}
      }
    edge from parent node [right] {\scriptsize $\bar{b}$} 
  }
  edge from parent node [above] {\scriptsize $p_2$}
}
;
 \node[fit=(00),dashed,thick,blue, draw, circle,inner sep=1pt] {};
  \node[fit=(01),dashed,thick,red, draw, circle,inner sep=1pt] {};
\end{scope}

\draw [dashed, thick, ForestGreen, in=150,out=30] (000) to (010);
\draw [dashed, thick, brown, in=150,out=30] (001) to (011);
\node [black] at (0,0.35) {\scriptsize $r$};
\node [black] at (-1.5,-0.55) {\scriptsize $u_1$};
\node [black] at (1.5, -0.55) {\scriptsize $u_2$};
\node [black] at (-2, -1.6) {\scriptsize $u_3$};
\node [black] at (-.25, -1.7) {\scriptsize $u_4$};

\node [black] at (0.25, -1.7) {\scriptsize $u_5$};
\node [black] at (2, -1.6) {\scriptsize $u_6$};

%obs labels

\node [blue] at (-1.7,-.9) {\scriptsize $I_1$};
\node [red] at (1.7,-.9) {\scriptsize $I_2$};
\node [ForestGreen] at (-0.2,-1) {\scriptsize $I_3$};
\node [brown] at (0.3,-1) {\scriptsize $I_4$};

\end{tikzpicture}
\caption{$\Max$ without $\salr$}
\label{fig:alossSpan-a}
\end{subfigure}

\begin{subfigure}{.6\columnwidth}
%\centering
\tikzset{
triangle/.style = {regular polygon,regular polygon sides=3,draw,inner sep = 2},
circ/.style = {circle,fill=cyan!10,draw,inner sep = 3},
term/.style = {circle,draw,inner sep = 1.5,fill=black},
sq/.style = {rectangle,fill=gray!20, draw, inner sep = 4}
}

\begin{tikzpicture}[scale=0.8]
\tikzstyle{level 1}=[level distance=9mm,sibling distance = 50mm]
\tikzstyle{level 2}=[level distance=5mm,sibling distance=25mm]
\tikzstyle{level 3}=[level distance=9mm,sibling distance=12mm]
\tikzstyle{level 4}=[level distance=10mm,sibling distance=6mm]

%node (ij) is the j th node in i th level

\begin{scope}[->, >=stealth]
\node (0) [circ] {}
child {
  node (00) [circ] {}
  child {
  node (000) [triangle] {}
   child {
     node (0000) [circ] {}
     child {
      node (00000) [term, label=below:{\scriptsize $w_1$}] {}
      edge from parent node [left] {\scriptsize $a$}
    }
    child {
      node (00001) [term, label=below:{\scriptsize $w_2$}] {}
      edge from parent node [right] {\scriptsize $\bar{a}$}
      }
    edge from parent node [left,pos=0.2] {\scriptsize $\frac{1}{2}$}
  }
  child {
    node (0001) [circ] {}
    child {
      node (00010) [term, label=below:{\scriptsize $w_3$}] {}
      edge from parent node [left] {\scriptsize $b$}
    }
    child {
      node (00011) [term, label=below:{\scriptsize $w_4$}] {}
      edge from parent node [right] {\scriptsize $\bar{b}$}
      }
    edge from parent node [right,pos=0.2] {\scriptsize $\frac{1}{2}$} 
     }
  edge from parent node [above] {\scriptsize $d$}
  }
  child {
  node (001) [triangle] {}
   child {
     node (0010) [circ] {}
     child {
      node (00100) [term, label=below:{\scriptsize $w_5$}] {}
      edge from parent node [left] {\scriptsize $a$}
    }
    child {
      node (00101) [term, label=below:{\scriptsize $w_6$}] {}
      edge from parent node [right] {\scriptsize $\bar{a}$}
      }
    edge from parent node [left,pos=0.2] {\scriptsize $\frac{1}{2}$}
  }
  child {
    node (0011) [circ] {}
    child {
      node (00110) [term, label=below:{\scriptsize $w_7$}] {}
      edge from parent node [left] {\scriptsize $b$}
    }
    child {
      node (00111) [term, label=below:{\scriptsize $w_8$}] {}
      edge from parent node [right] {\scriptsize $\bar{b}$}
      }
    edge from parent node [right,pos=0.2] {\scriptsize $\frac{1}{2}$} 
     }
  edge from parent node [above] {\scriptsize $\bar{d}$}
  }
  edge from parent node [above] {\scriptsize $c$}
  }
child {
  node (01) [circ] {}
  child {
  node (010) [triangle] {}
   child {
     node (0100) [circ] {}
     child {
      node (01000) [term, label=below:{\scriptsize $w_9$}] {}
      edge from parent node [left] {\scriptsize $a$}
    }
    child {
      node (01001) [term, label=below:{\scriptsize $w_{10}$}] {}
      edge from parent node [right] {\scriptsize $\bar{a}$}
      }
    edge from parent node [left,pos=0.2] {\scriptsize $\frac{1}{2}$}
  }
  child {
    node (0101) [circ] {}
    child {
      node (01010) [term, label=below:{\scriptsize $w_{11}$}] {}
      edge from parent node [left] {\scriptsize $b$}
    }
    child {
      node (01011) [term, label=below:{\scriptsize $w_{12}$}] {}
      edge from parent node [right] {\scriptsize $\bar{b}$}
      }
    edge from parent node [right,pos=0.2] {\scriptsize $\frac{1}{2}$} 
  }
  edge from parent node [above] {\scriptsize $d$}
}
  child {
  node (011) [triangle] {}
   child {
     node (0110) [circ] {}
     child {
      node (01100) [term, label=below:{\scriptsize $w_{13}$}] {}
      edge from parent node [left] {\scriptsize $a$}
    }
    child {
      node (01101) [term, label=below:{\scriptsize $w_{14}$}] {}
      edge from parent node [right] {\scriptsize $\bar{a}$}
      }
    edge from parent node [left,pos=0.2] {\scriptsize $\frac{1}{2}$}
  }
  child {
    node (0111) [circ] {}
    child {
      node (01110) [term, label=below:{\scriptsize $w_{15}$}] {}
      edge from parent node [left] {\scriptsize $b$}
    }
    child {
      node (01111) [term, label=below:{\scriptsize $w_{16}$}] {}
      edge from parent node [right] {\scriptsize $\bar{b}$}
      }
    edge from parent node [right,pos=0.2,pos=0.2] {\scriptsize $\frac{1}{2}$} 
     }
  edge from parent node [above] {\scriptsize $\bar{d}$}
  }
  edge from parent node [above] {\scriptsize $\bar{c}$}
}
;
\end{scope}

%\draw [dashed, thick, ForestGreen, in=150,out=30] (00) to (01);

\node[fit=(0),dashed,thick,ForestGreen, draw, circle,inner sep=1pt] {};
\draw [dashed, thick, brown, in=165,out=15] (00) to (01);
\draw [dashed, thick, blue, in=150,out=30] (0000) to (0010);
\draw [dashed, thick, blue, in=150,out=30] (0010) to (0100);
\draw [dashed, thick, blue, in=150,out=30] (0100) to (0110);
\draw [dashed, thick, red, in=150,out=30] (0001) to (0011);
\draw [dashed, thick, red, in=150,out=30] (0101) to (0111);
\draw [dashed, thick, red, in=150,out=30] (0011) to (0101);

%\node [black] at (0,0.45) {\scriptsize $r$};
%\node [black] at (-1.1,-0.45) {\scriptsize $u_1$};
%\node [black] at (1.1, -0.45) {\scriptsize $u_2$};
%\node [black] at (-2, -1.65) {\scriptsize $u_3$};
%\node [black] at (-.2, -1.65) {\scriptsize $u_4$};
%
%\node [black] at (0.25, -1.65) {\scriptsize $u_5$};
%\node [black] at (2, -1.65) {\scriptsize $u_6$};

%obs labels
\node [ForestGreen] at (0.55,0.1) {\scriptsize $I_3$};
\node [brown] at (0,-.8) {\scriptsize $I_4$};
\node [blue] at (-2.8,-1.6) {\scriptsize $I_1$};
\node [red] at (2.8,-1.6) {\scriptsize $I_2$};

%\node[black] at (-1.5,-.95) {\scriptsize $p_1$};
%\node[black] at (-.73,-.95) {\scriptsize $p_2$};

%\node[black] at (1.5,-.95) {\scriptsize $p_2$};
%\node[black] at (.73,-.95) {\scriptsize $p_1$};
\end{tikzpicture}
\caption{$\Max$ with $\alr$}
\label{fig:alossSpan-b}
\end{subfigure}
\caption{Equivalent $\alr$ game using $\alr$-span for game without $\salr$}
\label{fig:span}
\end{figure}


\paragraph*{Span}
We move on to another way of simplifying game-structures. The game-structure $\Tt_1$ \cref{fig:alossSpan-a} neither has perfect recall, nor A-loss recall. Using the characterization obtained in Section~\ref{}, we can show that it does not have  shuffled A-loss recall either. Now, consider the game-structure $\Tt'_1$ in \cref{fig:alossSpan-c}. It has A-loss recall. Each leaf monomial of $\Tt_1$ can be written as a linear combination of the monomials of $\Tt'_1$: for example, the leaf monomial $x_ax_{\bar{b}}$ or $\Tt_1$ is equal to $x_a x_{\bar{b}}x_c + x_a x_{\bar{b} \bar{c}}$, the sum of two leaf monomials of $\Tt'_1$.  The game-structure $\Tt_1$ is said to be \emph{spanned by} $\Tt'_1$. This property allows to solve games derived from the structure $\Tt_1$ by converting them into a game on $\Tt'_1$, and solving the resulting A-loss recall game. 
Our results:
\begin{itemize}
\item We show that every imperfect recall game without absent-mindedness~\cite{} is spanned by an A-loss recall game.

\item The caveat is that the smallest A-loss recall span may be of exponential size: we exhibit a family of game structures where this happens. 

\item Finally, we provide an algorithm to compute an A-loss recall span of smallest size. 
\end{itemize}

From a conceptual point of view, we provide the following novel outlook.
\begin{itemize}
	\item Solving every non-absentminded game is \emph{equivalent} to solving an A-loss recall game. 
\end{itemize}

Recall that imperfect recall games are $\NP$-hard in general. The above results show that in order to solve an imperfect recall game, one could either use an exponential-time algorithm on the game directly, or apply the above transformation into a potentially exponential-sized game, on which a polynomial-time algorithm can be used. 








\section{Influence of Modality in Persona Embodiment of Multimodal LLMs}\label{sec:problem}

% In this work, we study the ability of an LLM to embody a persona as represented in different modalities. 

The problem of embodying a persona can be defined as the task of generating responses consistent with a specified character, identity, or role~\citep{samuel2024personagym}. This involves maintaining coherence in linguistic style, beliefs, knowledge, and affective tone in a way that aligns with the intended persona.

In this work, we investigate the effect of representing the persona $p$ in different modalities, denoted as $\CR(p)$, on multimodal LLMs. In particular, we consider two common modalities, text and image, and evaluate the LLM's performance on these equivalent representations. Additionally, we also consider combining visual and textual features of the personas. We describe these $4$ different representations $\CR(p)$ in more detail below. 
% \ks{add an example for each maybe just in figure 1 -- you can make it double-column as well. maybe not super important but the pipeline figure is.}\jb{added to figure 2}\ks{awesome!}


\subsection{Persona Modality Representations}
\begin{itemize}
    \item \textbf{Text ($\CT$):} Textual descriptions of a persona correspond to a sequence of sentences characterizing the persona in natural language.

    \item \textbf{Image ($\CI$):} A persona can also be depicted visually using an image of the person in a representative environment that characterizes the persona visually.
    
    \item \textbf{Assisted Image ($\CI_{A}$):} Since certain features may be obscured in the image, textual attributes of the persona can also be included explicitly as text. 
    
    \item \textbf{Descriptive Image ($\CI_{D}$):} In this case, we include the textual attributes in the image itself using typography instead of in the text.
    
\end{itemize}

% \section{Method}
\label{sec:method}
We give an overview of our framework in~\cref{sec:method-overview}.
We then describe two novel and technical ingredients in our framework: the logarithmic-scaling quantization (\cref{sec:augment-quantization}) and the progressive upper and lower bound tightening (\cref{sec:augment-tightening}).

\subsection{Overview}
\label{sec:method-overview}

We now describe our framework for augmenting any lossy compressor (called a \emph{base compressor}) to preserve contour trees and maintain strict error bounds. 
Our framework requires two user-specified parameters, a persistence threshold $\varepsilon$ and a pointwise absolute error bound $\xi$. 
It also requires user-specified parameters associated with the specific base compressor being augmented. Our implementation works with rectilinear meshes, and it could easily be modified to work with any simply-connected tetrahedral mesh.

Our framework guarantees that, for any augmented compressor, $T_\varepsilon = T_\varepsilon'$ and $|f(x)-f'(x)| \leq \xi$ for every $x \in \X$. Starting with a standard compressor as the base compressor, we start with a step-by-step overview of our framework. 

\para{\underline{Step 1: Upper and lower bound calculation.}}~We store critical points of the simplified contour tree $T_\varepsilon$ losslessly. We calculate the initial pointwise upper and lower bounds for other point $x \in \X$. The key idea is to locate an edge $ab$ in $T_{\varepsilon}$ whose corresponding range of function values contains $f(x)$. This requires a careful computation using the join and split trees of $T_{\varepsilon}$; see \cref{sec:algorithm-details} for details. 
We let $L(x) = \min(f(a),f(b)) + \zeta$ and $U(x) = \max(f(a),f(b))-\zeta$, where $\zeta = 10^{-5}|f(b)-f(a)|$.
If we allow $x$ to have the same function value as $a$ or $b$, the topology may be altered (e.g., along the boundary of the induced region), resulting in more false cases. Adjusting the error bound by $\zeta$ prevents such issues. We also adjust $L(x)$ and $U(x)$ as needed to ensure that if $L(x) \leq f'(x) \leq U(x)$ then $|f(x)-f'(x)| \leq \xi$. 
 
When computing $T_\varepsilon$, we compute the join and split trees of $f$ and simplify the trees directly with persistence threshold $\varepsilon$. We then combine them to obtain $T_\varepsilon$. During this construction, we track which edge of $T_\varepsilon$ each point $x \in X$ corresponds to. Compared to simplifying the entire scalar field $f$ and then computing the contour tree of the simplified field (like TopoSZ), our strategy leads to equivalent results in less time.

\para{\underline{Step 2: Base compressor.}} 
We apply the base compressor to the input data $f$. 
We compress and then decompress the data to assess changes that need to be made during decompression. 
We refer to the compressed-then-decompressed data as the \emph{intermediate data}.

\para{\underline{Step 3: Logarithmic-scaling quantization.}} 
We introduce a novel quantization technique that respects the pointwise upper and lower bounds imposed in Step 1. 
If possible, the entropy of the quantization numbers $\{n_x\}$ will be identical to that of standard linear-scaling quantization.
However, when linear-scaling quantization cannot produce a prediction for a point $x$ that respects $L(x)$ and $U(x)$, $x$ will be quantized with more precision (i.e.,~$\xi \leftarrow \xi/2$) to satisfy those bounds.

\para{\underline{Step 4: Progressive upper and lower bound tightening.}} 
We introduce a novel technique for calculating adjustments to the intermediate data to guarantee that the contour tree is preserved.
We compute the join and split trees directly. If a false edge is detected during computation, the upper and lower bounds are tightened around points in the segmentation region corresponding to the edge (see \cref{sec:merge-and-contour-tree}). All edges whose growth involved these points are recomputed.
We continue until the join and split trees of the decompressed data match those of the ground truth. We do not compute the contour tree directly as the preservation of the join and split trees guarantees the preservation of the contour tree.

\para{\underline{Step 5: Lossless compression.}} 
We encode the quantization numbers using Huffman coding. The output of the base compressor, the encoded quantization numbers, and any losslessly stored values are written to a binary file which is further losslessly compressed using xz, a general-purpose data compression tool available via {XZ Utils}~\cite{XZUtils}.

\subsection{Logarithmic-Scaling Quantization}
\label{sec:augment-quantization}

We now describe the first novel ingredient in our framework: a variable precision quantization technique that preserves tight pointwise upper and lower bounds. %without significantly compromising the entropy of the overall distribution of quantization numbers. 
For each $x \in \X$, the intermediate data contains an estimated value $g(x)$ for the ground truth value $f(x)$. 
Let $L(x)$ and $U(x)$ denote the lower and upper bounds assigned to $x$.
To ensure that $L(x) \leq f'(x) \leq U(x)$, we assign to each $x \in \X$ a numerator $a_x \in \Z$ and a precision $p_x \in \N$ that indicates the number of iterations. 
Our reconstructed value is 
\begin{equation}
f'(x) = g(x) + \frac{2\xi \cdot a_x}{2^{p_x}}.
\label{eq:fprime-original}
\end{equation}

To calculate each $a_x$ and $p_x$, we first set $p_x=0$. 
We then look for the value of $a_x$ satisfying 
\begin{equation*}
L(x) \leq g(x) + \frac{2\xi \cdot a_x}{2^{p_x}} \leq U(x)
\label{eq:Bounds}
\end{equation*}
such that $|a_x|$ is minimized. If there is no valid value of $a_x$, we increase $p_x$ by $1$ and search again. This process is repeated until a valid $a_x$ is found. If $p_x$ reaches an arbitrary threshold, we stop searching and instead store $f(x)$ losslessly. We set this threshold equal to $11$.

When $p_x = 0$, the above process is the same as the standard linear-scaling quantization, except that we also seek to maintain the upper and lower bounds. 
Each time a linear-scaling quantization fails to identify a valid choice for $a_x$ that yields a value of $f'(x)$ within the upper and lower bounds for $x$, we cut the interval lengths in half by increasing $p_x$ by $1$ and continue searching.
When the interval lengths are smaller, it is more likely that a valid choice of $a_x$ exists. 
It is also possible that during an iteration, multiple valid choices of $a_x$ exist, so we choose the one with the smallest absolute value to minimize the entropy of $\{a_x\}$. 

\begin{figure}[!ht]
    \centering
    \vspace{-2mm}
    \includegraphics[width=\linewidth]{fig-log-scale-quantization.pdf}
    \vspace{-6mm}
    \caption{(A) If $p_x = 0$, there are no valid quantization intervals. (B) Increasing $p_x$ to $1$ allows for a valid quantization interval.}
    \label{fig:log-scale-quantization}
    \vspace{-2mm}
\end{figure}

This process is illustrated in \cref{fig:log-scale-quantization}. 
(A) contains an example where there are no quantization intervals where we can place $f'(x)$ to respect the upper and lower bounds. 
In (B), by raising the precision $p_x$ by 1, the quantization intervals are halved, giving a valid choice for $f'(x)$.

When encoding the data, we store a single quantization number $n_x$ for each $x \in \X$. 
To calculate each $n_x$, we first find the maximum precision $p_m$ used for any single point. The points are assigned the single quantization number $n_x = a_x \cdot 2^{p_m-p_x}$ and the max precision $p_m$ is stored in the compressed output. 
During decompression, the point $x$ is assigned the value 

\begin{equation}
f'(x) = g(x) + \frac{2\xi \cdot n_x}{2^{p_m}}.
\label{eq:fprime}
\end{equation}
Setting $n_x = a_x \cdot 2^{p_m-p_x}$ in Eq.~\eqref{eq:fprime} means that
\begin{equation*}
  g(x) + \frac{2\xi \cdot n_x}{2^{p_m}} = g(x) + \frac{2\xi \cdot a_x \cdot 2^{p_m-p_x}}{2^{p_m}} = g(x) + \frac{2\xi \cdot a_x}{2^{p_x}}.
  \label{eq:logscale}  
\end{equation*}
Therefore, the formulation in Eq.~\eqref{eq:fprime} is equivalent to the original formulation of $f'$ in Eq.~\eqref{eq:fprime-original}.

In comparison with TopoSZ, the above variable precision technique allows us to store fewer points losslessly.
In order to ensure the quantization numbers do not get too large, if any point has a precision greater than $10$ it is stored losslessly. This ensures that $p_m \leq 10$ for all trials.

\subsection{Progressive Upper and Lower Bound Tightening}
\label{sec:augment-tightening}

We now describe the second novel ingredient in our framework, namely, a \emph{progressive error bound tightening} process. 
Specifically, the process computes the join and split trees of the decompressed data. During the computation, it detects false cases, and tightens the upper and lower bounds in the neighborhoods of false cases. The algorithm progresses through merge tree computation, checking the correctness of each edge and tightening when needed, until every edge is correctly preserved.
The process allows us to bypass iteratively recomputing the entire contour tree (in the case of TopoSZ), significantly speeding up the compression process. During the tightening process, we work with merge trees (instead of contour trees), since the persistence of a leaf (local extremum) can be computed from its nearby saddle based on branch decomposition (i.e.,~local information), thereby allowing for our progressive tightening strategy. By contract, computing the persistence of a leaf of a contour tree may require global information from the whole contour tree due to the existence of V and W structures~\cite{hristov2021w}.

We describe this process for the join tree, which works analogously for the split tree. We only consider false cases involving extremum-saddle pairs. 

\para{False case detection}. To detect false cases, we construct $T'$. Doing so allows us to locate mismatches between edges in $T'_\varepsilon$ and those in $T_\varepsilon$.
We construct $T'$ using a modified version of the edge growing procedure from local minima and saddles (see~\cref{sec:merge-and-contour-tree}).
To start, we extract a list of local minima of $f'$ sorted by decreasing function values. Then, proceeding in sorted order, we grow an edge from each local minimum $m$ to a saddle $s$, and check two cases for $s$; see \cref{sec:algorithm-details} for illustrations: 

\underline{Case (I).} If $s$ is unpaired, i.e., $m$ is the first local minimum (among all local minima) whose growth terminates at $s$, then $m$ and $s$ form a persistence pair, with a persistence $p =|f'(s)-f'(m)|$. 
If $p < \varepsilon$, then the edge $ms$ does not belong to $T'_\varepsilon$; otherwise, $ms$ belongs to $T'_\varepsilon$.  

\underline{Case (II).} If $s$ is already paired, then $m$ must pair with some other saddle $s'$, and $s'$ must be an ancestor of $s$ in the join tree. A paired $s$ means that $s$ has been discovered earlier during the growth of another local minimum $m'$ such that $m'$ and $s$ form a persistence pair with persistence $p'$, and the edge $m's$ belongs to $T'$. 

\underline{Case (II.a).} 
Suppose that $p' \geq \varepsilon$. Since $m'$ preceds $m$ in the sorted order, $f'(m') > f'(m)$. Since $s'$ is an ancestor of $s$, $f'(s') > f'(s)$. Therefore $|f'(s') - f'(m)| > |f'(s) - f'(m')| = p' \geq \varepsilon$. 
Thus, the pair $(m,s')$ has a persistence above $\varepsilon$, and $ms$ must be an edge in $T'_\varepsilon$.

\underline{Case (II.b).} 
Now suppose that $p' < \varepsilon$. In this case, we do not have enough information to determine the persistence of $(m,s')$. Therefore, we grow from saddle $s$ to reach a new saddle $s''$. We then check cases (I) and (II) again, using $s''$ in place of $s$. 

Once we are done checking cases (I) and (II), if $m \notin T'_{\varepsilon}$ but $m \in T_{\varepsilon}$, then $m$ is a false negative. 
Likewise, if $ms \in T'_{\varepsilon}$ but $ms \notin T_{\varepsilon}$, then $ms$ is a false positive. 

Growing the global minimum will never produce a false case as long as the rest of $T'_\varepsilon$ is correctly predicted. Thus, we skip the growth at the global minimum, denoted as $\hat{m}$. 
Because $\hat{m}$ is the last growth that remains active, its growth will form the \textit{trunk}, a monotone sequence of edges to the root that links $\hat{m}$ to the remaining saddles~\cite{gueunet2017task}. Since $\hat{m}$ and the remaining saddles are already correctly predicted, so is the trunk, therefore no further false cases are possible, and we skip growing $\hat{m}$. 
This algorithm also admits a number of special cases; see~\cref{sec:algorithm-details}.

\para{Progressive false case correction.} 
If there is a false case, we first tighten the upper and lower bounds of points in some region $R$ to correct it. If $ms$ is a false positive, then $R$ is the region of the merge-tree-induced segmentation of $f'$ corresponding to $ms$. If $m$ is a false negative, and edge $m\hat{s}$ belongs to $T_\varepsilon$ (for some saddle $\hat{s}$), then $R$ is the region of the merge-tree-induced segmentation of $f$ corresponding to $m\hat{s}$. If the same false case occurs multiple times, we grow the region $R$. We tighten the upper and lower bounds of each $x \in R$ similarly to TopoSZ, but we tighten more aggressively to speed up compression. 
We then update the decompressed data $f'$ to respect the new bounds; see~\cref{sec:algorithm-details} for numerical specifics and a comparison with TopoSZ.

Once we update $f'$, these updates may affect parts of the join and split trees beyond the false cases, thus we must recompute those areas to ensure correctness. Specifically, we must check for any extrema bordering $R$ that may have appeared or disappeared as a result of the tightening process and update the trees accordingly. Let $E$ be the set of edges whose segmentation regions border $R$. Then the tightening also may have affected each edge $e \in E$ and every ancestor of $e$ (i.e.,~edges
connecting $e$ to the root of the tree). We recompute all such edges to ensure correctness. As before, we recompute parts of the tree in order of the function values. 


\section{Modality-Parallel Persona Dataset}\label{sec:dataset}

\subsection{Personas}


% 18 to 24 · 25 to 34 · 35 to 44 · 45 to 54 · 55 to 64 · 65 or over
We introduce a novel dataset of personas $\CP = \{p_i\}$, such that each persona $p$ can be represented equivalently in four modalities $\CI, \CT, \CI_A, \CI_D$. To ensure effective representation across both text and image modalities, we construct personas based on key demographic attributes that are easily visualizable~\citep{todorov2015social}. Specifically, each persona is defined by a unique combination of age, gender, occupation, and location. A persona can thus be written as:
\begin{center}
    {\small
    A \textcolor{blue}{\texttt{<age>}}-year-old \textcolor{teal}{\texttt{<gender>}} \textcolor{brown}{\texttt{<occupation>}} \\ from \textcolor{pink}{\texttt{<location>}},}
\end{center}
where \textcolor{blue}{\texttt{<age>}} $\in [18, 64]$, \textcolor{teal}{\texttt{<gender>}} $\in \{$ male, female $\}$, \textcolor{pink}{\texttt{<location>}} is a city, and \textcolor{brown}{\texttt{<occupation>}} denotes a person who does a specific occupation. For example, ``A \textcolor{blue}{35}-year-old \textcolor{teal}{male} \textcolor{brown}{chef} from \textcolor{pink}{Paris}''. As depicted in Figure~\ref{fig:intro}, age and gender can be visualized using the face of the person while occupation and location can be visualized using the clothes and the background respectively.

\begin{figure}[t]
    \centering
    \includegraphics[width=\linewidth]{pipeline.pdf}
    \caption{Our pipeline begins with curating a set of personas. Each persona receives a detailed text description, which is then fed into Stable Diffusion to generate $\mathcal{I}$. A separate model examines the image and generates an independent textual description, forming text persona $\mathcal{T}$. Pairing $p$ with $\mathcal{I}$ produces an assisted image $\mathcal{I_A}$, while combining a typographic representation of $p$ with $\mathcal{I}$ produces a descriptive image $\mathcal{I_D}$.}
    \label{fig:pipeline}
\end{figure}

\begin{table}[t]
    \centering
    \caption{\textbf{Persona Dataset Summary}}
    \label{tab:data_summary}
    \resizebox{1.0\linewidth}{!}{
    \begin{tabular}{c | c | c}
        \toprule
        Attribute & Category & Number \\
        \midrule
        \multirow{5}{*}{Age} & 18-24 & 5 \\
        & 25-34 & 11 \\
        & 35-44 & 13 \\
        & 45-54 & 6 \\ 
        & 55-64 & 5 \\
        \hline
        \multirow{2}{*}{Gender} & Male & 19 \\
        & Female & 21 \\
        \hline
        \multirow{5}{*}{Occupation} & Healthcare \& Education & 9 \\
        & Public Safety &	5 \\
        & Manual Labor & 16 \\
        & Hospitality & 5 \\
        & Transportation 	& 5 \\
        \hline 
        \multirow{4}{*}{Location} & Largest Economies (GDP > \$3T) &	12 \\
        & Developed Economies (GDP \$1T-\$3T) &	13 \\
        & Mid-Sized Powers (GDP \$0.5T-\$1T) & 7 \\
        & Emerging Markets (GDP < \$0.5T) & 8 \\
        \bottomrule
    \end{tabular}}
\end{table}


To promote diversity, we systematically categorize these attributes into distinct groups and uniformly sample from each category. Table~\ref{tab:data_summary} summarizes our dataset of how we choose the age, gender, occupation, and location. In particular, we consider a standard grouping of ages followed in surveys between 18 and 65, a standard male/female splitting of gender, while occupations and locations are categorized based on their primary societal role and economic status~\footnote{\href{https://data.worldbank.org/indicator/NY.GDP.MKTP.CD}{GDP}} respectively.
% \jb{is there a cite for this?}
Table~\ref{tab:personalist} in Appendix presents the list of $40$ personas we use along with their attributes and attribute categories. 
% \ks{need to write more}

% select the values of these attributes from a large set of $47$ ages $2$ genders, $35$ occupations, and $37$ different locations, for a grand total of $121,730$ distinct personas. 
% Then, we downsample this set by categorizing the age group select a representative subset of $n=40$ personas that preserves the occupational diversity of the full dataset while maintaining demographic balance
% Our evaluation dataset consists of personas generated through combinations of demographic attributes. Each persona $p_i$ is characterized by attribute tuple $(a_i, g_i, o_i, l_i)$ representing:

% $$
% \begin{aligned}
% \textbf{Age } (a_i) &: [18,65] \in \mathbb{N}\\
% \textbf{Gender } (g_i) &: \mathcal{G}, |\mathcal{G}| = 2 \\
% \textbf{Occupation } (o_i) &: \mathcal{O}, |\mathcal{O}| = 35\\
% \textbf{Location } (l_i) &: \mathcal{L}, |\mathcal{L}| = 37
% \end{aligned}
% $$


% \ks{can we reduce it to age into a few age brackets and occupation into a few occupation sectors and cities to be locations? Try to reduce them all to single-digit numbers.}

% \jb{i used the specific numbers to show all the possible unique personas and were used as-is in the experiments; there weren't any distinct patterns across the categories, how would I determine what brackets/sectors to use?}\ks{see above}

% For example: "{\texttt{A \textbf{52-year-old} \textbf{male} \textbf{doctor} from \textbf{Madrid}}}".
% \vspace{0.5em}

% By combining these attributes, we can generate . Within this subset, we ensure diversity by preventing any overlap of both occupation and location between personas. The complete list of attributes is provided in Appendix \ref{TODO}.

\subsection{Equivalent Modality Representations}

From above, we have a diverse set of textual persona descriptions as described by the four demographic attributes. Next, we construct a modality-parallel dataset, we require that each persona $p$ can be equivalently depicted in $4$ representations $\CR(p)$: image $\CI(p)$, text $\CT(p)$, assisted image $\CI_A(p)$, and descriptive image $\CI_D(p)$. Figure~\ref{fig:pipeline} illustrates the step-by-step procedure to obtain these modality representations for a persona description $\CP$.

\begin{enumerate}[leftmargin=*, noitemsep]
    \item We first convert the persona description made from the four attributes into a more detailed visual description using an LLM\footnote{\href{https://openai.com/index/hello-gpt-4o/}{gpt-4o-mini-2024-07-18}} and the following prompt:
    \begin{quote}
    \small
    \texttt{Create a short, descriptive persona for the person in the image. Describe them using only the following details: their age, gender, facial expression or mood, attire, any tools or items they’re holding, their work environment, the nature of their job, and their connection to the area and location. Avoid taking creative liberties beyond these details, only using details that can be inferred from the image, while aiming for a realistic portrayal that gives insight into their daily life, professional dedication, and overall demeanor. For example: Meet a skilled construction worker in his late 30s, living in Sydney, Australia. Every day, he heads out to work in one of the city's bustling urban sites, often with a view of iconic landmarks like the Sydney Opera House and Sydney Harbour Bridge. Outfitted in essential safety gear—a hard hat, reflective vest, and a set of versatile tools—he’s well-prepared for a physically demanding role that demands focus and precision. His job involves a blend of construction and maintenance tasks, requiring him to pay close attention to safety protocols and collaborate with a team. Confident and professional in his work, he takes pride in contributing to the infrastructure and vibrant aesthetic of Sydney, adding to the city’s ever-evolving landscape with each project.}
    \end{quote}
    \item Next, we use a text-to-image generative model, particularly,  Stable Diffusion XL\footnote{\href{https://huggingface.co/stabilityai/stable-diffusion-xl-base-1.0}{stabilityai/stable-diffusion-xl-base-1.0}} to generate a $768 \times 768$ px image conditioned on the more complete description of the persona found above. Upon doing an extensive hyperparameter search, we found the best results with a guidance scale of $\gamma = 15$ and $n = 50$ diffusion steps. Thus, we obtain the image $\CI$.
    \item Since the generated image can contain extra information due to underspecified textual prompts, we prompt the LLM one more time to generate a complete description of the persona as described in the image using a detailed prompt as given in Appendix~\label{app:img_to_text_prompt}. Thus, we obtain the text $\CT$.
\end{enumerate}

These steps enable us to convert a dataset of persona descriptions $\{p\} \rightarrow \{(\CT(p), \CI(p))\}$ such that $\CT (p) \leftrightarrow \CI (p)$ are equivalent to each other. One can now also obtain the assisted and descriptive image representations of the persona by pairing the image $\CI$ with the text persona $p$ for the assisted image $\CI_A$, and by rendering\footnote{\href{https://pillow.readthedocs.io/en/stable/}{Pillow}} $p$ as black text at the bottom of the image on a white background using Arial font at size 20 for $\CI_D$.

\subsection{Question Generation}
To evaluate how well a model embodies a given persona, we create a set of $60$ questions that specifically probe for a given attribute either directly or in naturalistic scenarios. In particular, we create $10$ questions per attribute for the two sets. Gender was excluded from our evaluation question set due to methodological constraints. While age and location can be objectively probed through factual knowledge, gender assessment would inevitably rely on stereotypes or normative expectations. Moreover, there may be a high possibility of refusal from the LLMs due to their safety training. Thus, we obtain two question sets $Q^D$ and $Q^S$ for \textcolor{pink}{L}: location, \textcolor{brown}{O}: occupation, and \textcolor{blue}{A}: age. 
% Each persona is evaluated across three categories ($L$: location, $O$: occupation, $A$: age) using two evaluation sets:
$$
\begin{aligned}
Q^D &= \bigcup_{i \in \{\textcolor{pink}{L},\textcolor{brown}{O},\textcolor{blue}{A}\}} Q^D_i,\text{ each } |Q^D_i| = 10 \text{ questions}\\
Q^S &= \bigcup_{i \in \{\textcolor{pink}{L},\textcolor{brown}{O},\textcolor{blue}{A}\}} Q^S_i,\text{ each } |Q^S_i| = 10 \text{ scenarios}
\end{aligned}
$$
% \ks{why did we not have gender-related questions?}

% \jb{i think because evaluating correct answers would've just relied on stereotypes; i tried to make questions that could be more objectively evaluated (i.e. this is clearly not have a doctor would respond, or those schools are clearly not near that location/city). i could probably not mention gender at all in the paper since we don't use it though}\ks{this is good, let's say this then.}

% We curated two types of prompts to evaluate persona alignment: questions and scenarios. 

\subsubsection{Direct Testing}
Questions were designed to probe specific knowledge across age, location, and occupation categories while enabling objective evaluation. For example, location questions assess knowledge of local customs and landmarks, while occupation questions may test domain expertise. For example, for age, we ask ``\textit{what life experiences do you consider most defining for your generation?}'' while for location, we have ``\textit{what is the most visited tourist attraction in your area?}''. We provide the complete list in Table~\ref{tab:direct_questions_list} in Appendix. 

% \ks{write more detail}\jb{what else should I put here? i have what criteria are used in the next section. should i give an example of a scenario here (like reference a figure)?}\ks{examples, and appendix reference.}

\subsubsection{Situational Testing}
Scenarios accomplish similar knowledge evaluation but through naturalistic situations, requiring personas to implicitly demonstrate both knowledge and behavioral consistency. For example, for age scenarios, we ask ``\textit{You’re coordinating a playlist for your high school reunion after-party. The organizers want music specifically from your graduating years to recreate the atmosphere. You $\dots$}'', which is detailed in Figure \ref{fig:intro}. A complete list is provided in Table \ref{tab:direct_scenarios_list} in Appendix.
% \ks{write more detail}\jb{what else to put here?} \ks{examples and appendix reference.}


\subsection{Evaluation}

For each persona $p \in \CP$ and question $q \in \CQ^D \cup \CQ^S$, we find the response answer $a \gets \CM(\CR(p), q)$ from a multimodal LLM $\CM$, where $\CR(p)$ denotes a modality representation of the persona $\CP$. Thus, we obtain $(q, \CR(p)) \rightarrow_{\CM} (p, q, a)$. We now evaluate the quality of the response $a$ based on the question asked $q$ and the persona description $p$. 

\subsubsection{LLM-based evaluation}
% We define the subject model $\mathcal{M}: P \times \Pi \rightarrow R$ where $P$ is the set of personas, $\Pi = Q \cup S$ is the set of all prompts (questions and scenarios), and $R$ is the space of possible responses. For each response $r = \mathcal{M}(p,\pi)$, we employ an evaluator $\mathcal{J}$ that assigns scores based on different evaluation criteria taken from \citet{samuel2024personagym}. The evaluator uses the following four task descriptions:

Following \citet{samuel2024personagym}, we employ an LLM-based evaluator to judge the quality of the responses based on different metrics defined in the prompt. In particular, we prompt the LLM judge $\CJ$ with the question asked $q$, response $a$, and the persona description $p$ on these metrics as described by the corresponding prompts.

\begin{quote}
\small
\texttt{\textbf{Persona Consistency:} Evaluate the consistency of the response with the described persona. Ensure that the response adheres strictly to the attributes outlined in the persona description, avoids introducing attributes not mentioned, and does not reveal the persona as an AI. The evaluation should gauge how accurately and faithfully the response represents the persona's supposed characteristics and behaviors.}

\texttt{\textbf{Linguistic Habits:} The evaluation task of "linguistic habit" assesses the persona's adherence to its characteristically unique syntax, tone, and lingo, ensuring that these elements are consistently utilized throughout the persona's dialogue. This includes avoiding generic language patterns (such as "As a [persona]") and integrating specific idiomatic expressions, colloquialisms, or jargon that define the persona's distinctive verbal identity. The aim is to evaluate how effectively the persona maintains its linguistic uniqueness in various contexts.}

\texttt{\textbf{Action Justification:} Evaluate the persona's response to determine how effectively and convincingly it justifies a given action based on its described attributes and situation. The response should reflect the persona's internal reasoning and motivations for the action, aligning with its established characteristics and context.}

\texttt{\textbf{Expected Action:} The persona takes actions within its response to the question that is logically expected of the persona in the setting of the question.
}
\end{quote}

% \paragraph{Likert Scale} 
For each evaluation criterion, $\mathcal{J}(p,q,a)$ outputs a score from a 5-point Likert scale based on the corresponding system prompt. For situational testing, we evaluate using \texttt{action justification}, \texttt{expected action}, \texttt{linguistic habits} while for direct testing, we use \texttt{persona consistency} and \texttt{linguistic habits}. Note that we combine the scores for linguistic habits across the two testing sets to find the average score. 

% \paragraph{Binary Success Metrics}
% We dichotomize evaluator scores using threshold $\tau = 3$ to distinguish between successful responses ($\geq 3$) and unsuccessful responses ($< 3$) for each criterion. For evaluation set $D$, we define the \textbf{pass rate} (PR) as: \(\text{PR} = \tfrac{1}{|D|} \sum_{i=1}^{|D|} \mathbb{I}(\mathcal{J}(p_i, q_i, a_i) \geq \tau)\).

\subsubsection{Comparative Evaluation}
We employ two comparative evaluation methods to assess relative performance across modalities, using evaluator $\mathcal{J}$ with the prompt:
\begin{quote}
    \small
    \texttt{You are given a persona description and multiple responses to a prompt.\\ %\\[1ex]
        Persona Description: <\textit{p}>\\
        Prompt: <q>\\
        Candidate Responses: <responses>\\ %$[1ex]
        Choose the single response that best fits the persona's style, values, and consistency. Respond with 'Response X' where X is the number of the chosen response.}
\end{quote}

\paragraph{Pairwise Comparison}
To compare responses across the text and image modalities, we first directly compare responses $a_{\mathcal{T}}$ and $a_{\mathcal{I}}$.

\paragraph{Swiss System Comparison}
To collectively evaluate all four modalities, we adopt the Swiss tournament system, which reduces the number of required comparisons compared to pairwise evaluation while maintaining ranking quality. Specifically, for \( n = 4 \), pairwise evaluation requires \( \binom{4}{2} = 6 \) comparisons, whereas the Swiss system reduces to 3 comparisons.

\subsubsection{Linguistic Analysis}
Alongside the \textit{linguistic habits} evaluation criterion, we also analyze the lexical diversity, variation, and complexity of each response using established metrics from computational linguistics:
\begin{itemize}[leftmargin=*,noitemsep]    
    \item \textbf{Types:} $|\{r\}|$, unique token count.  %\vspace{0.3em}
    \item \textbf{Root Type-Token Ratio (RTTR):} \(= \nicefrac{\text{types}}{\sqrt{\text{length}}}\)
    , a normalized measure of lexical variation found by dividing the number of unique tokens with the response length~\citep{lexicalrichnesshout}. %\vspace{0.3em}
    \item \textbf{Measure of Textual Lexical Diversity (MTLD):} Following \citet{mccarthy2010mtld}, calculates the mean length of text segments that maintain a type-token ratio (TTR) $> \tau = 0.72$.
\end{itemize}


\section{Experimental Setup}\label{sec:setup}

\paragraph{Models.} We evaluate the performance of 5 multimodal large language models: (1) GPT-4o\footnote{\href{https://openai.com/index/hello-gpt-4o/}{GPT-4o}}, (2) GPT-4o mini\footnote{\href{https://openai.com/index/gpt-4o-mini-advancing-cost-efficient-intelligence/}{GPT-4o mini}}, (3) Llama 3.2 11B\footnote{\href{https://huggingface.co/meta-llama/Llama-3.2-11B-Vision}{Llama 3.2 11B}}, (4) Llama 3.2 90B\footnote{\href{https://huggingface.co/meta-llama/Llama-3.2-90B-Vision}{Llama 3.2 90B}}, and (5) Pixtral 12B\footnote{\href{https://huggingface.co/mistralai/Pixtral-12B-2409}{Pixtral 12B}} \citep{agrawal2024pixtral12b}.

\paragraph{Evaluators.} We utilize two LLM evaluators, using GPT-4o\footnotemark[3] and Gemini 2.0 Flash\footnote{\href{https://deepmind.google/technologies/gemini/flash/}{Gemini 2.0 Flash}}, with deterministic sampling with zero temperature and top P values. All scores discussed in the main paper are averaged across the two models, while individual scores can be found in Tables \ref{tab:eval-table-gpt-4o} and \ref{tab:eval-table-gemini-flash} in the Appendix. We use human evaluators on a large subset of the evaluation set to assess the LLM evaluator scores' alignment with human scores. For further details, refer to Appendix \ref{app:human}. 
% \ks{what detail?} \jb{screenshots/details about the survey and who the evaluators were (broadly)}
\section{Experimental Results}
\label{sec:results}

\begin{figure*}[!ht]
    \centering
    \begin{subfigure}{0.02\textwidth}
    \raisebox{0.1\height}{\includegraphics[angle=90,width=\textwidth]{fig-colorBar}}
    \end{subfigure}
    \begin{subfigure}{0.92\textwidth}
    \includegraphics[width=\textwidth]{fig-vr_table_smaller.png}
    \end{subfigure}
    \vspace{-3mm}
    \caption{Scientific datasets compressed using different augmented compressors with topological controls. From left to right: the original input dataset, the reconstructed datasets from Augmented ZFP, Augmented SZ3, and Augmented TTHRESH, 
    respectively, that preserve the contour trees up to a persistence threshold $\varepsilon = 0.04$. From top to bottom: Tangaroa, Miranda, S3D datasets, respectively. We also display the PSNR and compression ratio next to each decompressed dataset.}    
    \label{fig:volume-render}
    \vspace{-6mm}
\end{figure*}

We provide an overview in~\cref{sec:results-overview}, describing the base compressors and datasets used in our experiments, highlighting the main takeaways, and introducing the evaluation metrics. 
We include compressor configurations and implementation details in~\cref{sec:configurations}. 
In \cref{sec:augmented-compressors} we describe the main utilities of our augmented compressors in preserving contour trees in the reconstructed data.
We evaluate a number of augmented compressors qualitatively and quantitatively, followed by a comparison against the state-of-the-art topology-preserving compressors in \cref{sec:compare-topology}.
We end this section with a run time analysis in \cref{sec:run-time}.

%---------------------------------------
\subsection{An Overview of Experiments}
\label{sec:results-overview}

We present a comparative analysis of five error-bounded lossy compressors augmented with our framework, including the classic compressors ZFP \cite{lindstrom2014fixed}, SZ3 \cite{liang2022sz3}, and TTHRESH \cite{ballester2019tthresh}, a custom-built cubic spline interpolation (CSI) model, and a deep learning-based compressor Neurcomp \cite{lu2021compressive}. 
We test these augmented compressors---denoted as Augmented ZFP, Augmented SZ3, and so on---against two state-of-the-art topology-preserving compressors, TopoSZ~\cite{soler2018topologically} and TopoQZ~\cite{yan2023toposz}. 

We test the five augmented compressors and two topology-preserving compressors on nine volumetric datasets from scientific simulations. The Nyx dataset is very topologically complex---its contour tree has over twenty million nodes---and it is included as a stress test. See \cref{tab:datasets} and \cref{sec:datasets} for details on these datasets.

We further conduct an ablation study demonstrating the effectiveness of logarithmic-scaling quantization and progressive error bound tightening. In every trial, logarithmic-scaling quantization leads to higher compression ratios, whereas progressive tightening results in faster compression times. We also analyze the individual effects of varying $\varepsilon$ and $\xi$; see~\cref{sec:other-experiments} for details on these experiments and the ablation study.

\begin{table}[!ht]
\scriptsize
\centering{
\begin{tabu}{c|*{2}{c}}
\toprule
\textbf{Dataset}  & \textbf{Dimension} & \textbf{Size (MB)}  \\ 
\midrule
QMCPACK      & $69 \times 69 \times 115$          & 4.4         \\
Tangaroa     & $300 \times 180 \times 200$         & 27.0        \\
Earthquake   & $175 \times 188 \times 50$          & 28.2        \\
Ionization   & $310 \times 128 \times 128$         & 40.6       \\
Isabel       & $500 \times 500 \times 90$          & 105.0     \\
Miranda      & $384 \times 384 \times 256$         & 302.0      \\
Nyx          & $512 \times 512 \times 512$         & 641.4      \\
S3D          & $500 \times 500 \times 500$         & 1000.0    \\
SCALE-LETKF  & $1200 \times 1200 \times 98$        & 1129.0    \\
\bottomrule
\end{tabu}
}
\vspace{-2mm}
\caption{Scientific datasets used for compression analysis.}
\label{tab:datasets}
\vspace{-4mm}
\end{table}

\para{Highlighted results.}
We highlight our experimental results below. 
\begin{itemize}[noitemsep,leftmargin=*]
\item Applying any of the five original base compressors to any of the nine datasets produces a large number of topological false cases in the reconstruction, even with a small pointwise error bound. On the other hand, augmenting any compressor with our general framework completely eliminates these false cases while maintaining a user-specified error bound (\cref{sec:augmented-compressors}).
\item Augmented TTHRESH and Augmented ZFP respectively yield the best compression ratios and run times among all the augmented compressors (\cref{sec:augmented-compressors}).
\item Our augmented compressors generally have a better trade off between bit-rate and reconstruction quality compared to TopoQZ and TopoSZ while taking similar or less time to run (\cref{sec:compare-topology}).
\item Our framework has a worst-case time complexity of $O(F h n \log n)$, where $h$ is the maximum height of the contour tree during tightening and $F$ is the number of false cases during computation. The majority of the compression time is spent on computing merge trees (\cref{sec:run-time}).
\end{itemize}

\para{Evaluation metrics.} 
We evaluate whether the contour tree has been perfectly preserved in the reconstructed (decompressed) data. 
We also evaluate the standard compression metrics of compression ratio, bit-rate, and peak signal-to-noise ratio (PSNR).
We further employ topology-based metrics of the bottleneck distances $d_B$~\cite{cohen2005stability} and the Wasserstein distances $d_W$ \cite[page 183]{edelsbrunner2022computational} to quantify the topological similarity between the original data and the reconstructed data. 
The evaluation metrics are described in detail in \cref{sec:evaluationMetrics}.

In general, higher values of PSNR indicate better reconstruction quality, and lower values of $d_B$ and $d_W$ indicate higher topological similarity. 
We measure the total compression time for each compressor, which includes (a) the total time to run the base compressor, and (b) the time to augment the output of the base compressor. We also measure decompression time for each compressor. We measure compression and decompression time for TopoSZ and TopoQZ as well. For our framework and TopoSZ, we decompress to RAW binary format. Because TopoQZ is integrated in the Topology Toolkit, an extension for ParaView, we decompress to VTK image format.

% ---------------------------------------
\subsection{Compressor Configurations and Implementation}
\label{sec:configurations}

In addition to augmenting the out-of-box base compressors SZ3, TTHRESH, ZFP, and Neurcomp, we implement and augment our own super-resolution compressor, a simple custom-built cubic spline interpolation (CSI) model.
It compresses a dataset by downsampling the data by a user-defined ratio in each direction (called a target scale factor) and uses a cubic spline interpolation technique for reconstruction that is similar to the one implemented in SZ3.
%We also considered the Sliced Wasserstein Autoencoder \cite{kolouri2018sliced} used in the AE-SZ compressor \cite{liu2021exploring}. 
%However, this model is excluded from our experiments since it performed significantly worse than the other compressors during initial tests.   

We compare our augmented compressors to TopoSZ and TopoQZ. We use the TopoQZ implementation in TTK~\cite{TiernyFavelierLevine2017}. 

Our general framework requires two user-defined parameters, a persistence threshold $\varepsilon$ and a global absolute pointwise error bound $\xi$.
$\varepsilon$ represents, as a percentage of the range, the level of persistence  simplification. 
For example, $\varepsilon = 0.01$ corresponds to a persistence simplification by $1\%$ of the range of the scalar function. Similarly, $\xi$ is the percentage of the range that will be used as an absolute error bound.

Each base compressor takes a number of intrinsic parameters in order to run. 
Both ZFP and SZ3 require an absolute error bound, denoted as $\delta$ and $\eta$, respectively. 
CSI requires a target scale factor $s$.
TTHRESH takes in a target RMSE of $\tau$. 
Neurcomp requires a target compression ratio $c$. 
Changing the intrinsic parameters of a base compressor will cause it to generate different intermediate data which will be augmented differently. 
As a result, even though our augmented compressor guarantees topology preservation and maintains the user-defined global error bound, the compression results may vary. 

For our experiments, we set the error parameter for each base compressor (except CSI and Neurcomp) to be equal to $k\xi$ for some $k \in \R$ that is compressor-dependent. Specifically, we set $\delta = 5\xi$, $\eta = 0.25\xi$, and $\tau = 0.05\xi$. To decide each value of $k$, we conduct a grid search and observe the effects of different values of $k$ across different values of $\xi$ and different datasets. The optimal value of $k$ varies between datasets and values of $\xi$, but the values that we chose are always approximately optimal. Hypothesized explanations as to why these values of $k$ improve results are described in \cref{sec:base-compressor-parameters}. We also set $c = 100$ and $s = 7$. We chose these configurations because they empirically led to the highest compression ratios.

To differentiate from the persistence threshold $\varepsilon$ used by an augmented compressor, TopoQZ takes a persistence threshold $e$. The TTK implementation of TopoQZ is tightly coupled with ZFP, which requires an error bound $\zeta$, allowing for a total pointwise error upper-bounded by $e+\zeta$.
To measure compression ratio and compression times while respecting a topological constraint $\varepsilon$ and an error bound $\xi$, we measure how each augmented compressor and TopoSZ perform for $\varepsilon = 0.04$ and $\xi = 0.012$. When testing TopoQZ, to ensure that it respects both $\varepsilon$ and $\xi$, we set $e = \zeta = 0.006$ so that the max error is less than $0.012$.

To measure the trade off between compression ratio and reconstruction quality, for TopoSZ and each augmented compressor, we set $\varepsilon = 0.04$ and vary $\xi \in \{$0.003, 0.006, 0.009, 0.012, 0.015, 0.018$\}$. In some cases we need to vary $\xi$ (and $\zeta$, for TopoQZ) in a different range in order to obtain a meaningful curve. Notably, for TopoQZ, we set $e = 0.04$ and vary $\zeta \in \{$0.003, 0.11, 0.22, 0.33, 0.44, 0.55$\}$. We further discuss our methodology for choosing parameters and provide the parameter used in \cref{sec:reconstruction-quality-extra}. 

The combination of a chosen compressor, a fixed dataset, a value of $\varepsilon$ and $\xi$, is a trial.
We perform each trial on a single cluster node running an Intel Xeon Sandy Bridge-E processor with 16 cores and 64GB of RAM.
For Neurcomp, we perform the training on an RTX 2080ti GPU with 32GB of RAM.

\begin{figure}[!ht]
\begin{subfigure}{0.02\linewidth}
\raisebox{2.2\height}{
\includegraphics[angle=90, width=\linewidth]{fig-colorBar.png}}
\end{subfigure}
\begin{subfigure}{0.97\linewidth}
    \includegraphics[width=\linewidth]{fig-scale_lattice_2.png}
\end{subfigure}
\vspace{-6mm}
\caption{Zoomed-in views of the critical points of the contour trees of the Ionization (top) and SCALE-LETKF (bottom) datasets with persistence simplification $\varepsilon = 0.04$. For each dataset, arrows indicate one example where compression with TTHRESH led to critical points shifting. Local maxima are in orange, local minima are in dark blue, 1-saddles are in light blue, 2-saddles are in light orange. Top row: Ionization. Bottom row: SCALE-LETKF.}
\label{fig:zoom}

\centering{
\begin{tabular}{c|cc|c}
\hline
Dataset    & ZFP       & TTHRESH       & Total \#edges \\ \hline
QMCPACK    & (23,23,0) & (8,8,0)     & 69          \\ 
Tangaroa   &  (90, 92, 0) & (39, 46, 0) & 418         \\ 
Earthquake & (20, 19, 0) & (26, 25, 0)     & 169         \\
Ionization & (181, 187, 0)  & (37, 44, 0) & 568         \\
Isabel     & (16, 15, 0) & (15, 16, 0)  & 29          \\
Miranda    & (6, 6, 0) & (3, 3, 0)     & 11          \\ 
Nyx        & (133, 132, 0) & (2, 373, 0) & 743 \\
S3D       & (100, 100, 0) & (74, 80, 0) & 1013 \\
SCALE-LETKF & (152, 153, 0) & (129, 127, 0) & 401 \\ \hline
\end{tabular}
}
\vspace{-2mm}
\captionof{table}{Reporting the number of false cases (false positives, false negatives, false types) produced by base compressors SZ3 and TTHRESH, respectively, together with the total number of edges of the input (ground truth) contour tree. Contour trees are simplified with $\varepsilon=0.04$.}
\label{tab:base-false-cases}
\vspace{-6mm}
\end{figure}

%---------------------------------------
\subsection{Comparative Analysis of Augmented Compressors}
\label{sec:augmented-compressors}

In this section, we perform a comparative analysis of five augmented compressors, qualitatively and quantitatively. 
We visualize three scientific datasets before and after compression with {three of our augmented compressors} in \cref{fig:volume-render}. We also display the PSNR and compression ratio next to each decompressed dataset. 
Compression ratios and times for a single combination of $\varepsilon$ and $\xi$ are reported in \cref{tab:compression-task}. Charts showing the reconstruction quality on two datasets is reported in \cref{fig:reconstruction-quality}. Similar charts for the remaining datasets and compressors are shown in \cref{sec:reconstruction-quality-extra}. Results demonstrating the effect of independently varying $\varepsilon$ or $\xi$ on the evaluation metrics are given in \cref{sec:other-experiments}.

\subsubsection{Topological Guarantees}
When compressing a dataset with any base compressor, the contour tree of the data is often significantly distorted with a large number of false cases, whereas it is always perfectly preserved using our augmented compressor. This observation has been validated empirically in every trial: the contour tree is perfectly preserved in terms of the locations of its critical points and their connectivity.

For instance, we visualize the Isabel dataset in \cref{fig:teaser} using TTHRESH and augmented TTHRESH. We highlight parts of the contour trees via zoomed-in views before and after compression. In a bottom zoomed-in view, TTHRESH (middle) fails to preserve a few critical points of the contour tree. In a top zoomed-in view, TTHRESH (middle) preserves the locations of the critical points, but not their connectivity. In contrast, the augmented TTHRESH preserves both the locations and connectivity among the critical points.

We further provide zoomed-in views for the Ionization and SCALE-LETKF datasets in~\cref{fig:zoom}. We observe clearly that TTHRESH fails to predict many critical points, whereas augmented TTHRESH preserves them all.

In \cref{tab:base-false-cases}, we report the number of false cases, including both extremum-saddle and saddle-saddle connections in the contour tree reconstructed with ZFP and TTHRESH. We again use parameter configurations that produce the same compression ratios as their augmented versions with $\varepsilon = 0.04$ and $\xi = 0.012$ (for those configurations see \cref{sec:base-compressor-parameters}). In \cref{tab:base-false-cases}, we can see that ZFP and TTHRESH produce many false cases.

\begin{table}[!t]
\setlength{\tabcolsep}{2pt}
\resizebox{\columnwidth}{!}{
\begin{tabular}{cccccccc}
\hline
Dataset     & A-ZFP            & A-SZ3   & A-CSI         & A-TTHRESH      & \multicolumn{1}{c|}{A-Neurcomp} & TopoQZ             & TopoSZ          \\ \hline
\multicolumn{8}{c}{Compression Ratio}                                                                                                                \\ \hline
QMCPACK     & 58.7             & 86.1    & 102.3           & \textbf{104.8} & \multicolumn{1}{c|}{23.9}       & 23.4               & 27.8            \\
Tangaroa    & 37.3             & 43      & 33.5            & \textbf{44.8}  & \multicolumn{1}{c|}{15.3}       & --                 & 24.3            \\
Earthquake  & 86.1             & 127.4   & 79.4            & \textbf{129.2} & \multicolumn{1}{c|}{63.5}       & 13.4               & 50.1            \\
Ionization  & 118.8            & 121.5   & 119.9           & \textbf{170.5} & \multicolumn{1}{c|}{72.7}       & 30.0               & 25.1            \\
Isabel      & 47.4             & 103.5   & 70.6            & \textbf{182.2} & \multicolumn{1}{c|}{41.6}       & --                 & 37.6            \\
Miranda     & 172.3            & 198.6   & 157.2           & \textbf{318.7} & \multicolumn{1}{c|}{95.0}       & 76.5               & 95.9            \\
Nyx         & 65.3             & 69.5    & 70.4            & \textbf{84.5}  & \multicolumn{1}{c|}{18.9}       & --                 & --              \\
S3D         & 38.4             & 46.6    & 43.6            & \textbf{59.6}  & \multicolumn{1}{c|}{6.0}        & 9.2                & --              \\
SCALE-LETKF & 69.5             & 74.4    & 58.5            & \textbf{114.2} & \multicolumn{1}{c|}{8.6}        & 11.4               & --              \\ \hline
\multicolumn{6}{c|}{Total Compression and Augmentation Time}                                                  & \multicolumn{2}{c}{Compression Time} \\ \hline
QMCPACK     & 1.27             & 1.36    & 1.21            & 1.45           & \multicolumn{1}{c|}{172.08}     & \textbf{1.05}      & 10.46           \\
Tangaroa    & 9.51             & 10.99   & \textbf{9.33}   & 11.98          & \multicolumn{1}{c|}{1519.21}    & --                 & 314.56          \\
Earthquake  & \textbf{7.08}    & 7.39    & \textbf{7.08}   & 8.66           & \multicolumn{1}{c|}{1039.94}    & 8.08               & 48.17           \\
Ionization  & \textbf{8.68}    & 10.00   & 14.08           & 15.16          & \multicolumn{1}{c|}{1221.12}    & 10.40              & 425.31          \\
Isabel      & \textbf{33.77}   & 35.49   & 42.69           & 42.67          & \multicolumn{1}{c|}{7147.86}    & --                 & 367.10          \\
Miranda     & 223.52           & 284.18  & 248.51          & 348.72         & \multicolumn{1}{c|}{9359.59}    & \textbf{160.60}    & 434.98          \\
Nyx         & \textbf{1059.06} & 1137.33 & 5664.46         & 25594.54       & \multicolumn{1}{c|}{38959.73}   & --                 & --              \\
S3D         & 209.83           & 253.82  & \textbf{173.09} & 253.72         & \multicolumn{1}{c|}{34610.78}   & 633.13             & --              \\
SCALE-LETKF & \textbf{221.49}  & 399.32  & 343.58          & 371.19         & \multicolumn{1}{c|}{40887.62}   & 524.05             & --              \\ \hline
\multicolumn{8}{c}{Decompression Time}                                                                                                               \\ \hline
QMCPACK     & 0.14             & 0.32    & 0.14            & 0.17           & \multicolumn{1}{c|}{4.24}       & 0.63               & \textbf{0.01}   \\
Tangaroa    & 0.49             & 0.52    & 0.51            & 0.96           & \multicolumn{1}{c|}{16.32}      & --                 & \textbf{0.12}   \\
Earthquake  & 0.38             & 0.41    & 0.37            & 0.55           & \multicolumn{1}{c|}{8.50}       & 5.50               & \textbf{0.07}   \\
Ionization  & 0.48             & 0.54    & 0.54            & 0.83           & \multicolumn{1}{c|}{11.62}      & 5.08               & \textbf{0.10}   \\
Isabel      & 1.43             & 1.35    & 1.34            & 2.64           & \multicolumn{1}{c|}{79.61}      & --                 & \textbf{0.41}   \\
Miranda     & 2.92             & 3.1     & 3.38            & 4.53           & \multicolumn{1}{c|}{175.61}     & 53.2               & \textbf{0.67}   \\
Nyx         & \textbf{6.72}    & 8.05    & 9.71            & 9.49           & \multicolumn{1}{c|}{2457.91}    & --                 & --              \\
S3D         & \textbf{11.52}   & 11.47   & 11.84           & 16.33          & \multicolumn{1}{c|}{2135.47}    & 390.94             & --              \\
SCALE-LETKF & \textbf{11.59}   & 11.92   & 12.89           & 21.50          & \multicolumn{1}{c|}{2629.48}    & 351.61             & --              \\ \hline
\end{tabular}
}
\vspace{-2mm}
\caption{Compression ratio, compression time, and decompression time for each compressor with $\varepsilon = 0.04$ and error bound $\xi = 0.012$ (except TopoQZ has $e = \zeta = 0.006$).
Times are in seconds.
Trials that did not finish are marked with a dash. 
TopoQZ ran out of memory on Nyx and it crashed on Isabel and Tangaroa due to unknown reasons. 
TopoSZ ran out of memory on Nyx, S3D, and SCALE-LETKF. 
}
\label{tab:compression-task}

\includegraphics[width=\columnwidth]{fig-rate_distortion_2.png}
    \vspace{-4mm}
    \captionof{figure}{PSNR, bottleneck distance, and Wasserstein distance versus bit-rate for each compressor for the QMCPACK and Earthquake datasets with $\varepsilon = 0.04$ ($e = 0.04$ for TopoQZ). These curves are given for other datasets in \cref{sec:reconstruction-quality-extra}.}
    \label{fig:reconstruction-quality}
    \vspace{-4mm}
\end{table}

%---------------------------------------
\subsubsection{Evaluation Metrics}
\label{sec:augmented-compressors-evaluation-metrics}

Compression ratio and times are reported in \cref{tab:compression-task} for a fixed parameter configuration of $\varepsilon = 0.04$ and $\xi = 0.012$.
We chose this parameter configuration because a small amount of persistence simplification preserves a large number of topological features in the input data, generating complex test cases for topology-preserving compression. For the reconstruction quality demonstrated in \cref{fig:reconstruction-quality}, $\varepsilon = 0.04$ is chosen similarly, and $\xi$ is varied between $0.003$ and $0.018$ to yield a variety of different compression ratios while still remaining small.
In this section, we compare the different augmented compressors. We leave the comparison with TopoQZ and TopoSZ to \cref{sec:compare-topology}.

\para{Compression ratios.} 
As shown in \cref{tab:compression-task}, Augmented TTHRESH produces the best compression ratios in every trial. Augmented Neurcomp performs noticeably worse than the other compressors.

\para{Reconstruction quality.}
In every trial, our framework successfully maintains a pointwise error bound $\xi$. There is a natural trade off between compression ratio and reconstruction quality. As shown in \cref{fig:reconstruction-quality}, Augmented SZ3 and Augmented TTHRESH have the best trade off between bit-rate, PSNR, $d_W$ and $d_B$, and perform equally well. Augmented Neurcomp performs the worst based on above metrics.

When visualized, we find that the decompressed volumes generally resemble the ground truth. However, when using certain transfer functions, visual artifacts may become visible. Artifacts appear more in visualizations that are sensitive to small changes in the transfer function. For the volume renderings in this paper, we chose transfer functions that led to fewer visual artifacts; see~\cref{sec:visual-artifacts} for adversarial examples.

In practice, we find that upper and lower bound tightening does not affect PSNR very much; most of the reconstruction quality is determined by the initial upper and lower bounds. \cref{fig:errorMap} shows a map of the absolute error of each point for a topologically complex slice of the Ionization dataset before and after tightening. We can see that tightening does not have a significant effect on the average error.

\begin{figure}[!t]
    \begin{subfigure}[b]{0.06\linewidth}
        \raisebox{0.5\height}{\includegraphics[width=\linewidth]{fig-colorBar-error.png}}
    \end{subfigure}
    \hfill
    \begin{subfigure}[b]{0.46\linewidth}
    \centering
        \includegraphics[width=\linewidth]{fig-errorMapPre.png}
        (A)
    \end{subfigure}
    \hfill
    \begin{subfigure}[b]{0.46\linewidth}
    \centering
        \includegraphics[width=\linewidth]{fig-errorMapPost.png}
        (B)
    \end{subfigure}
    \vspace{-5mm}
    \caption{Error map of a topologically complex slice of the Ionization dataset (A) before error bound tightening and (B) after error bound tightening.}
    \label{fig:errorMap}
    \vspace{-8mm}
\end{figure}

\para{Run time analysis.} 
There are significant differences in run time among the augmented compressors.
As discussed in \cref{sec:run-time}, these times are affected by factors other than the base compression time. However, Augmented Neurcomp is the slowest because Neurcomp compresses data by training a neural network.  
Of the remaining four compressors, ZFP is typically the fastest, and TTHRESH the slowest, although this observation does not hold for all trials. For decompression time, ZFP is the fastest, while Neurcomp remains the slowest.

\para{Highlighted results.} 
There is no clear best augmented compressor that outperforms others across all metrics. Other than augmented Neurcomp, utilizing any augmented compressor plays a trade off between compression ability and speed. For the remainder of our analysis, we will primarily focus on Augmented ZFP, which is the fastest augmented compressor, and Augmented TTHRESH, which yields the best compression ratios and reconstruction quality.

%---------------------------------------
\subsection{Comparison with TopoQZ and TopoSZ}
\label{sec:compare-topology}

\para{Topological guarantees.}
Our framework preserves the contour tree during compression, and achieves the same topological guarantee as TopoSZ. TopoQZ ensures that all critical point pairs are preserved above a persistence threshold $\varepsilon$, but their locations and connectivity may be distorted after compression.

\para{Compression ratio.} 
In terms of compression ratio, when maintaining a strict topological constraint $\varepsilon = 0.04$ and error bound $\xi = 0.012$, every augmented compressor except Augmented Neurcomp outperforms both TopoQZ and TopoSZ in every trial.

\para{Reconstruction quality.} 
The curves in \cref{fig:reconstruction-quality} show that every augmented compressor except Augmented Neurcomp can match the PSNR of TopoQZ and TopoSZ,  while using less space. In terms of topological distance, the augmented compressors except Augmented Neurcomp outperform TopoSZ in terms of $d_W$ and $d_B$. They also outperform TopoQZ in terms of $d_B$, but are comparable in terms of $d_W$.

\para{Run time analysis.} 
In terms of compression time, the augmented compressors except Augmented Neurcomp produce times that are comparable to or better than TopoQZ, and significantly outperform TopoQZ on the largest datasets. 
These four augmented compressors are also significantly faster than TopoSZ across all trials. 

In terms of decompression time, the augmented compressors except Augmented Neurcomp perform slower than TopoSZ but faster than TopoQZ. There are several possible reasons why our decompression times are slower than TopoSZ. First and most notably, our decompression process is more complex, as it involves a decompression with the base compressor and then an augmentation of the decompressed results. This process requires more operations and has a higher I/O overhead. Second, we use XZ along with tar archives for lossless compression, which is slower than ZSTD used by TopoSZ. See \cref{sec:more-running-time} for a more detailed analysis of the decompression time.   

%---------------------------------------
\subsection{Analysis of Compression Time}
\label{sec:run-time}

\para{Asymptotic analysis.}
Let $n$ be the number of vertices in the rectilinear mesh. Our algorithm utilizes heap merges~\cite{gueunet2017task} during the merge tree computation; however, we use binary heaps (stored in arrays) instead of Fibonacci heaps from~\cite{gueunet2017task}. For a binary heap with $m$ elements, a single insertion operation has a worst-case time complexity of $O(\log m)$. 
Following~\cite{gueunet2017task}, from bottom to top, constructing an edge $e$ in a merge tree requires merging its heap with the heaps of its descendants, which takes $O(n \log n)$. Let $h$ denotes the \emph{height} of the tree, which corresponds to the maximum number of ancestor edges.
Then constructing a merge tree using these insertion-based heap merges takes $O(h n \log n)$. During the progressive tightening process, let $F$ denote the total number of detected false cases, each of which triggers a (partial) recomputation of the merge tree. Therefore, our algorithm takes $O(F h n \log n) = O(n^3 \log n)$. 

In practice, $F \ll n$ as shown in \cref{tab:time}. 
Additionally, $h \ll n$. We found that $\frac{h}{n}$ ranged from $0.0004$ (Miranda, A-ZFP) to $0.025$ (Nyx, A-Neurcomp). Excluding Augmented Neurcomp, $\frac{h}{n} < 0.01$ in $97\%$ of trials.  
On the other hand, using Fibonacci heaps to construct a merge tree~\cite{gueunet2017task} takes $O(n \log n)$ due to constant time heap merges; however, in our setting, we have found that binary heaps have lower run time in practice. Likewise, it is possible to merge heaps in linear time, but we instead merge by repeatedly inserting each element of the smaller heap into the larger one, as doing so has a much lower run time in practice.

\para{Empirical analysis.} To analyze the run time empirically, we calculate the amount of time for each portion of our algorithm with Augmented ZFP and Augmented TTHRESH,  with $\varepsilon = 0.04$ and $\xi = 0.012$. These run times are shown in \cref{tab:time}.

In \cref{tab:time}, the most time-consuming task is the computation of merge and contour trees. We compute the contour tree of the input data at the beginning of the algorithm. During the error bound tightening steps we also compute the contour tree of the decompressed data. These run times are shown in \cref{tab:time} under the `CT' and `Grow' columns, and account for $35-77\%$ of the total run time for each trial in \cref{tab:time}. 

For most of the trials, the time to run the base compressor, shown in the `BC' column, is a relatively small percentage of the overall compression time. 
However, if a base compressor produces results that nearly preserve the contour tree and does not produce too many extra branches, including those of persistence below $\varepsilon$, the augmentation time may be lower. This phenomenon suggests that the accuracy of the base compressor may have more effect on the total compression and augmentation time than just base compression.
In general, the run time of each base compressor is much faster than its augmented counterpart; see~\cref{sec:more-running-time} for a comparison. 

\begin{table}[!ht]
\setlength{\tabcolsep}{2pt}
\centering
\vspace{-2mm}
\resizebox{\columnwidth}{!}{
\begin{tabular}{cccccccccc}
\hline
\multicolumn{1}{c|}{Dataset}     & BC    & CT     & ULB     & Grow     & \%B    & \#FC & Fix      & File  & Total    \\ \hline
\multicolumn{10}{c}{Augmented ZFP}                                                                                  \\ \hline
\multicolumn{1}{c|}{QMCPack}     & 0.15  & 0.34   & 0.21   & 0.38     & 0.17\% & 0    & 0.0      & 0.27  & 1.35     \\
\multicolumn{1}{c|}{Tangaroa}    & 1.42  & 2.16   & 2.80   & 1.34     & 0.43\% & 14   & 0.0004   & 1.84  & 9.57     \\
\multicolumn{1}{c|}{Earthquake}  & 0.45  & 1.56   & 1.26   & 2.88     & 0.79\% & 2    & 0.0001   & 0.97  & 7.12     \\
\multicolumn{1}{c|}{Ionization}  & 0.70  & 1.56   & 2.09   & 3.10     & 1.15\% & 10   & 0.0009   & 1.35  & 8.81     \\
\multicolumn{1}{c|}{Isabel}      & 4.73  & 4.61   & 8.95   & 9.48     & 0.54\% & 1    & 0.0647   & 6.17  & 34.01    \\
\multicolumn{1}{c|}{Miranda}     & 4.00  & 156.79 & 17.92  & 37.85    & 1.21\% & 0    & 0.0      & 7.02  & 223.58   \\
\multicolumn{1}{c|}{NYX}         & 24.75 & 695.26 & 150.27 & 142.16   & 0.47\% & 6    & 4.60     & 42.21 & 1084.05  \\
\multicolumn{1}{c|}{S3D}         & 14.2  & 27.93  & 69.11  & 56.40    & 1.04\% & 39   & 0.01411  & 38.44 & 206.63   \\
\multicolumn{1}{c|}{SCALE-LETKF} & 15.2  & 16.15  & 61.67  & 92.99    & 0.63\% & 37   & 0.003408 & 35.34 & 221.47   \\ \hline
\multicolumn{10}{c}{Augmented TTHRESH}                                                                              \\ \hline
\multicolumn{1}{c|}{QMCPack}     & 0.23  & 0.34   & 0.21   & 0.45     & 0.77\% & 0    & 0.0      & 0.22  & 1.45     \\
\multicolumn{1}{c|}{Tangaroa}    & 3.34  & 2.2    & 2.79   & 1.92     & 0.77\% & 18   & 0.0001   & 1.59  & 11.84    \\
\multicolumn{1}{c|}{Earthquake}  & 1.53  & 1.56   & 1.25   & 3.33     & 0.65\% & 4    & 0.0001   & 0.91  & 8.58     \\
\multicolumn{1}{c|}{Ionization}  & 1.93  & 1.54   & 2.08   & 8.09     & 1.48\% & 5    & 0.0120   & 1.24  & 14.94    \\
\multicolumn{1}{c|}{Isabel}      & 13.13 & 4.62   & 9.01   & 10.24    & 0.54\% & 1    & 0.0666   & 5.62  & 42.69    \\
\multicolumn{1}{c|}{Miranda}     & 12.73 & 156.48 & 17.73  & 153.69   & 2.33\% & 0    & 0.0      & 7.24  & 347.87   \\
\multicolumn{1}{c|}{NYX}         & 57.87 & 687.03 & 132.49 & 15838.88 & 0.82\% & 89   & 97.82    & 37.52 & 27286.05 \\
\multicolumn{1}{c|}{S3D}         & 66.35 & 27.89  & 67.27  & 56.51    & 1.14\% & 47   & 0.0136   & 32.11 & 250.79   \\
\multicolumn{1}{c|}{SCALE-LETKF} & 69.21 & 16.10  & 60.00  & 193.18   & 0.54\% & 46   & 0.0032   & 29.71 & 368.35   \\ \hline
\end{tabular}
}
\vspace{-2mm}
\caption{Runtime analysis for each component of the augmented framework involving Augmented ZFP and Augmented TTHRESH with $\varepsilon = 0.04$ and $\xi = 0.012$. 
All times are in seconds. 
BC: running the base compressor.
CT: computing the contour tree of the input data.
ULB: calculating the initial upper and lower bounds. 
Grow: time growing the contour tree of the reconstructed data.
\%B: percent of branches in the reconstructed contour tree whose growth was recomputed.
\#FC: number of false cases corrected after upper and lower bounds are set.
Fix: average time to fix a false case, excluding regrowing branches.
File: average time to write the compressed output to a file.}
\label{tab:time}
\vspace{-6mm}
\end{table}
\section{Discussion and Related Work}\label{sec:relatedWork}

\paragraph{Persona Evaluation}

Prior work has established several frameworks for evaluating language models' role-playing capabilities. \citet{wang-etal-2024-rolellm} introduced RoleBench, an evaluation benchmark with QA pairs based on character profiles. \citet{wang-etal-2024-incharacter} developed InCharacter, assessing role-playing fidelity through psychological scales in an interview format. \citet{tu-etal-2024-charactereval} created CharacterEval, a Chinese benchmark derived from novels and scripts with multi-interaction dialogues, while \citet{shen2023roleeval} established RoleEval, a bilingual benchmark with multiple-choice questions testing persona knowledge and reasoning. \citet{samuel2024personagym} introduced PersonaGym, a dynamic evaluation framework for automated assessment of persona adherence across diverse interactions. Our work further extends the literature by performing the first systematic evaluation to understand the influence of the persona modality.

\paragraph{Multimodal Personas}

Recent work has explored integrating visual elements into LLM persona systems. \citet{ahn-etal-2023-mpchat} introduced MPCHAT, demonstrating that incorporating visual episodic memories alongside text improves dialogue consistency and persona grounding. \citet{sun-etal-2024-kiss} investigated how visual personas influence LLMs' behavior in negotiation contexts, showing models can adapt their responses based on perceived visual personality traits. \citet{dai2025mmrole} developed MMRole, a framework for training and evaluating multimodal role-playing agents. While these works establish the potential of visual personas and others extensively evaluate textual personas \citep{li-etal-2016-persona, xiao2024farllmsbelievableai, samuel2024personagym}, there has been no systematic comparison of how different modalities of persona representation affect model performance. Our work addresses this gap by directly evaluating text, visual, and hybrid approaches across a range of persona-based tasks.

\paragraph{Modality Alignment}

Language models demonstrate strong in-context learning capabilities in unimodal textual settings \citep{Shanahan2023, 10.5555/3666122.3669274}.
However, extending these capabilities to multimodal inputs remains challenging.
When visual information is introduced, models often struggle to transfer knowledge effectively from text to vision (and vice versa), resulting in noticeably weaker performance with visual in-context demonstrations compared to textual ones \citep{zhao2024mmicl, jiang2024manyshot}.
Such cross-modal gaps manifest in several ways: for instance, catastrophic forgetting of text-based instruction following can occur when models are finetuned on images \citep{zhang2024wingslearningmultimodalllms}.
While incorporating visual knowledge can yield improvements on specific tasks \citep{jin-etal-2022-leveraging}, maintaining consistently high performance across both textual and visual modalities remains an open research question, which is also highlighted in our work.
\section{Conclusion}

This work addresses the pressing need for enhanced security in the burgeoning blockchain ecosystem. We investigate the application of Large Language Models (LLMs) to smart contract vulnerability detection and repair, focusing on Solidity and Move. We introduce \textbf{Smartify}, a novel multi-agent framework that significantly improves LLM performance in this critical domain. The contributions of this work are: (1) \textbf{Smartify}, a novel multi-agent framework that enhances LLM-based smart contract vulnerability detection and repair; (2) a method for encoding language-specific knowledge, valuable for low-resource languages like Move; (3) a scalable, adaptable approach applicable to other programming languages and LLMs; (4) a demonstration of Smartify’s efficacy on generalized pre-trained LLMs; and (5) a detailed analysis of the challenges inherent in automated code repair.

\textbf{Smartify} represents a significant advancement in automating smart contract security, a crucial concern in the expanding blockchain landscape. Future work will refine the framework, expand its language coverage, particularly within the blockchain domain, and integrate it into real-world blockchain development workflows. This research lays the foundation for AI-powered tools that can bolster the security and reliability of decentralized applications, fostering a more robust and trustworthy blockchain ecosystem.

\clearpage
\section*{Limitations}\label{sec-limitations}
% \ks{write somethign}
A limitation of our work is that we only deal with 40 personas. However, due to a lack of any persona dataset with equivalent representations in different modalities, we see this as our contribution and leave it for future works to expand the scale of the study. Furthermore, we specifically increase the diversity of these personas across $4$ well-grounded categories, focusing on the quality of our dataset. As the field of persona alignment in LLMs is still quite nascent, we believe quality becomes more important than quantity. Additionally, it should be noted that the persona modality representations may not align perfectly across all details. Our pipeline employs two distinct mapping functions---Stable Diffusion (text-to-image) and GPT-4o-mini (image-to-text)---which will naturally introduce extraneous information or inconsistencies between representations. However, this limitation is acceptable for our evaluation framework since we only test for the presence and consistency of specific attributes rather than complete fidelity across all possible persona characteristics. Another limitation is that we have only validated our results on a small set of human annotators. We circumvent this by leveraging the validation of LLM-based evaluation with human evaluations~\citep{samuel2024personagym} while also showing a high correlation of our results across different LLM evaluators. 

\section*{Broader implications and social impact}
We intend our proposed dataset to be used strictly for academic purposes. While we design our dataset such that it does not contain any harmful and private content, our pipeline can be adapted to generate such unintended visual personas. However, we note that this is not a direct result of our artifact and can also be possible through directly querying the StableDiffusion APIs. Thus, we expect our contributions of dataset and evaluation methodology to have an overall positive social impact by inspiring future research on aligning modalities for persona embodiment.
\bibliography{citations}

\clearpage

\appendix
\appendix

\section*{Appendix}

\section{Prompts}\label{app:prompts}
\subsection{Textual Description}\label{app:img_to_text_prompt}
\begin{quote}
    {\small
    \texttt{Create a short, descriptive persona for the person in the image. Describe them using only the following details: their age, gender, facial expression or mood, attire, any tools or items they’re holding, their work environment, the nature of their job, and their connection to the area and location. Avoid taking creative liberties beyond these details, only using details that can be inferred from the image, while aiming for a realistic portrayal that gives insight into their daily life, professional dedication, and overall demeanor. For example: Meet a skilled construction worker in his late 30s, living in Sydney, Australia. Every day, he heads out to work in one of the city's bustling urban sites, often with a view of iconic landmarks like the Sydney Opera House and Sydney Harbour Bridge. Outfitted in essential safety gear—a hard hat, reflective vest, and a set of versatile tools—he’s well-prepared for a physically demanding role that demands focus and precision. His job involves a blend of construction and maintenance tasks, requiring him to pay close attention to safety protocols and collaborate with a team. Confident and professional in his work, he takes pride in contributing to the infrastructure and vibrant aesthetic of Sydney, adding to the city’s ever-evolving landscape with each project.
    }}
\end{quote}
    

\begin{table*}[t]
    \centering
    \caption{A complete list of personas annotated for their attribute categories.}
    \label{tab:personalist}
    \resizebox{1.0\linewidth}{!}{
    \begin{tabular}{l c c c c}
        \toprule
        Persona & Age & Gender & Occupation & Location \\
        \midrule
        A 25-year-old female nurse from Toronto & 25-34 & female & healthcare \& education & Strong Developed Economies \\ 
        A 41-year-old female electrician from Sydney & 35-44 & female & manual labor & Strong Developed Economies \\ 
        A 36-year-old male electrician from Houston & 35-44 & male & manual labor & Largest Global Economies \\ 
        A 29-year-old female police officer from New York & 25-34 & female & public safety & Largest Global Economies \\ 
        A 28-year-old female police officer from London & 25-34 & female & public safety & Largest Global Economies \\ 
        A 35-year-old male chef from Paris & 35-44 & male & hospitality & Largest Global Economies \\ 
        A 32-year-old female chef from Rome & 25-34 & female & hospitality & Strong Developed Economies \\ 
        A 50-year-old male farmer from Sao Paulo & 45-54 & male & manual labor & Emerging Markets \\ 
        A 40-year-old female farmer from Nairobi & 35-44 & female & manual labor & Emerging Markets \\ 
        A 27-year-old female mechanic from Berlin & 25-34 & female & manual labor & Largest Global Economies \\ 
        A 28-year-old female pilot from Los Angeles & 25-34 & female & transportation & Largest Global Economies \\ 
        A 28-year-old female pilot from Vancouver & 25-34 & female & transportation & Strong Developed Economies \\ 
        A 60-year-old female carpenter from Rome & 55-64 & female & manual labor & Strong Developed Economies \\ 
        A 45-year-old male carpenter from Auckland & 45-54 & male & manual labor & Emerging Markets \\ 
        A 44-year-old female cashier from Montreal & 35-44 & female & hospitality & Strong Developed Economies \\ 
        A 56-year-old male roofer from Brisbane & 55-64 & male & manual labor & Strong Developed Economies \\ 
        A 30-year-old female garbage collector from Toronto & 25-34 & female & manual labor & Strong Developed Economies \\ 
        A 63-year-old male miner from Johannesburg & 55-64 & male & manual labor & Emerging Markets \\ 
        A 24-year-old female lab technician from Shanghai & 18-24 & female & healthcare \& education & Largest Global Economies \\ 
        A 29-year-old male postal worker from Mexico City & 25-34 & male & transportation & Emerging Markets \\ 
        A 44-year-old female welder from Dubai & 35-44 & female & manual labor & Mid-Sized \& Regional Powers \\ 
        A 54-year-old male librarian from Amsterdam & 45-54 & male & healthcare \& education & Mid-Sized \& Regional Powers \\ 
        A 51-year-old female dentist from Seoul & 45-54 & female & healthcare \& education & Strong Developed Economies \\ 
        A 40-year-old female landscaper from Edinburgh & 35-44 & female & manual labor & Largest Global Economies \\ 
        A 24-year-old male hairdresser from Barcelona & 18-24 & male & hospitality & Strong Developed Economies \\ 
        A 19-year-old male janitor from Stockholm & 18-24 & male & manual labor & Mid-Sized \& Regional Powers \\ 
        A 53-year-old female bus driver from Copenhagen & 45-54 & female & transportation & Mid-Sized \& Regional Powers \\ 
        A 27-year-old female machinist from Frankfurt & 25-34 & female & manual labor & Largest Global Economies \\ 
        A 52-year-old male doctor from Madrid & 45-54 & male & healthcare \& education & Strong Developed Economies \\ 
        A 60-year-old male security guard from Lisbon & 55-64 & male & public safety & Mid-Sized \& Regional Powers \\ 
        A 42-year-old male firefighter from Sao Paulo & 35-44 & male & public safety & Emerging Markets \\ 
        A 36-year-old male pharmacist from Berlin & 35-44 & male & healthcare \& education & Largest Global Economies \\ 
        A 56-year-old female teacher from Melbourne & 55-64 & female & healthcare \& education & Strong Developed Economies \\ 
        A 42-year-old male taxi driver from Hong Kong & 35-44 & male & transportation & Largest Global Economies \\ 
        A 39-year-old female veterinarian from Nairobi & 35-44 & female & healthcare \& education & Emerging Markets \\ 
        A 25-year-old male baker from Lisbon & 25-34 & male & hospitality & Mid-Sized \& Regional Powers \\ 
        A 40-year-old male welder from Moscow & 35-44 & male & manual labor & Mid-Sized \& Regional Powers \\ 
        A 39-year-old male plumber from Melbourne & 35-44 & male & manual labor & Strong Developed Economies \\ 
        A 22-year-old male lab technician from Tokyo & 18-24 & male & healthcare \& education & Largest Global Economies \\ 
        A 20-year-old female security guard from Cape Town & 18-24 & female & public safety & Emerging Markets   \\
        \bottomrule
    \end{tabular}
    }
\end{table*}

\subsection{Effect of Safety Training}
\label{app:safety-training}


In our experiments, we observed that Llama 3.2 90B frequently refused to assume visual personas\footnote{Refusal detection was performed using a fine-tuned \texttt{distilroberta-base} model \citep{distilroberta-base-rejection-v1}}, refusing to engage with 76.7\% of all visual persona prompts (Figure \ref{fig:safety-training}). This behavior can be attributed to an overgeneralization of the model's safety training, as personas can create competing objectives between aligned models' safety measures and instruction-following directives \citep{wei2024jailbroken}. This vulnerability has frequently been exploited in adversarial attacks \citep{ma2024visual}, leading to unsafe outputs even when models assume benign personas \citep{zhao2024bias}. To address this issue, the development of Llama 3 incorporated targeted safety training specifically designed to handle persona-based interactions \citep{grattafiori2024llama3herdmodels}.

\begin{figure}[t]
        \centering
        \includegraphics[width=\linewidth]{refusal-graph.pdf}
        \caption{The rate and number of refusals in response to persona prompts. Llama 3.2 90B shows a strong aversion to multimodal persona prompts, while other models rarely refuse.}
    \label{fig:safety-training}
\end{figure}

\begin{table*}[t]
    \centering
    \caption{Direct testing question list}
    \label{tab:direct_questions_list}
    \resizebox{0.9\textwidth}{!}{
    \begin{tabular}{c|c}
    \toprule
    % \multicolumn{2}{c}{Direct testing}
    Attribute & Direct questions \\
    \midrule
        \multirow{10}{*}{Age} & What age-related milestone are you approaching or have recently celebrated, and how did you celebrate it? \\
            & Which television shows or movies were popular when you were a teenager? \\
            & What life experiences do you consider most defining for your generation? \\
            & What were some common trends or fashions during your college years? \\
            & At what age did you first use the internet regularly, and what activities did you engage in online? \\
            & What age were you when you first experienced a major economic event? \\
            & How old were you when you first started using social media, and which platform did you join first? \\
            & How did people in your age group typically meet and socialize in their younger years? \\
            & What music formats (vinyl, cassettes, CDs, etc.) did you grow up using? \\
            & What historical moments do people slightly older than you remember that you just missed? \\
        \midrule
        \multirow{10}{*}{Location} & 
            What are the top three universities or colleges in your area? \\
            & What is the most visited tourist attraction in your area? \\
            & How does the local climate influence your daily activities and lifestyle in your region? \\
            & What are the most frequented local cuisines where you live? \\
            & What are the main industries driving the economy in your area? \\
            & What natural features (mountains, rivers, coast) shape your local landscape? \\
            & What local sports teams unite your community? \\
            & What's the primary mode of public transportation in your area, if any? \\
            & What are the most popular local festivals or events in your area? \\
            & How has the demographic makeup of your area changed over the past decade? \\
        \midrule
        \multirow{10}{*}{Occupation} & 
            Can you outline your primary responsibilities in your current occupation? \\
            & What specific skills are essential for success in your profession? \\
            & What does a typical workday look like for you? \\
            & How do you stay updated with the latest developments in your industry? \\
            & What tools or technologies do you regularly use in your work? \\
            & What's the most significant change you've witnessed in your industry? \\
            & What emerging trends do you see impacting your profession? \\
            & What advice would you give to someone aspiring to enter your field? \\
            & Which legislation directly impacts the way you perform your job? \\
            & What safety protocols specific to your profession do you follow? \\
        \bottomrule
    \end{tabular}}
\end{table*}

\begin{table*}[t]
    \centering
    \footnotesize
    \caption{Scenarios for situational testing}
    \label{tab:direct_scenarios_list}
    \resizebox{0.9\textwidth}{!}{
    \begin{tabular}{c|p{13cm}}
        \toprule
        \textbf{Attribute} & \textbf{Scenarios} \\
        \midrule
        \multirow{20}{*}{\textbf{Age}} 
            & Your extended family is digitizing old home videos for a reunion. While helping, you discover some footage from your early childhood that needs to be converted. The deadline is next week. You \dots \\
            & During a basement cleanup, you find your old gaming systems and accessories from when you were 12. A local collector has shown interest in purchasing the set. You \dots \\
            & You're coordinating a playlist for your high school reunion after-party. The organizers want music specifically from your graduating years to recreate the atmosphere. You \dots \\
            & Your childhood bank is closing its local branch. They've requested you update your old youth account, which was opened when you first started saving. You \dots \\
            & A museum exhibit about educational evolution needs artifacts from your school years. They're particularly interested in how your grade did research projects, so you \dots \\
            & Your first protest/activism experience has become historically significant. A documentary team wants to understand what motivated your age group's involvement. You talk about \dots \\
            & While discussing financial education, younger relatives ask about your first experiences with digital payment apps and online banking. You say \dots \\
            & A market research team needs insight into how your age group first learned about major global events. They're comparing news consumption across generations. You share \dots \\
            & Your old elementary school is celebrating its anniversary. They're recreating typical classroom setups from different eras, including your years there, so you \dots \\
            & During a job interview, you're asked about your generation's unique approach to work-life balance and career expectations. You describe \dots \\
        \midrule
        \multirow{20}{*}{\textbf{Location}} 
            & A major sporting event has caused unusual traffic patterns downtown. You have an important appointment across the city in 45 minutes, so you \dots \\
            & Construction has blocked your usual route through downtown. You're picking up a friend from the airport in an hour, and they just texted that their flight landed early. To make it on time, you \dots \\
            & The biggest annual cultural celebration in your region coincides with a business meeting. International colleagues want to experience local traditions. You \dots \\
            & Severe weather typical for your region has disrupted normal operations. You need to help visitors understand local emergency procedures. You \dots \\
            & A local food shortage has affected typical ingredients in your area. You need to adapt traditional recipes for an important family gathering. You \dots \\
            & Regional housing policies have changed, affecting your rental situation. You need to explain local housing norms to potential roommates. Breaking it down, you explain \dots \\
            & A community center needs help designing weather-appropriate outdoor activities for children new to the area. To help, you \dots \\
            & Your area's unique rush hour patterns are affecting a planned event. You need to coordinate arrival times for out-of-town guests. You \dots \\
            & Local religious or cultural observations are affecting business hours. You need to plan essential errands accordingly. You \dots \\
            & A seasonal health advisory typical for your region has been issued. You need to adjust your outdoor workout routine. \\
        \midrule
        \multirow{20}{*}{\textbf{Occupation}} 
            & During a casual conversation at dinner, your aunt mentions an ongoing situation that raises red flags based on your background and training. You \dots \\
            & A friend's child is working on a school project related to your profession. They need help understanding basic industry concepts. To assist, you \dots \\
            & During a home renovation, you notice issues that relate to your professional expertise. The contractors seem unaware of potential complications. You \dots \\
            & A community workshop needs professionals to demonstrate how their job impacts daily life. Your industry's perspective would fill a key gap. You \dots \\
            & A community Facebook group is sharing advice that conflicts with principles you work with daily, so you \dots \\
            & A local news story misrepresents aspects of your industry. You have an opportunity to provide clarification at a community meeting. At the meeting, you \dots \\
            & Your hobby group encounters a challenge that relates to your professional expertise. They're unsure about proper procedures. You demonstrate \dots \\
            & A neighbor's insurance claim involves aspects of your profession. They're asking for general guidance about standard practices. \\
            & During a social event, you notice concerning practices related to your industry's safety standards. Others seem unaware of the risks, so you \dots \\
            & A local youth program needs career mentors. They want professionals to share how their industry handles modern challenges. You \dots \\
        \bottomrule
    \end{tabular}}
\end{table*}

\begin{table*}[t]
\centering
\small
\begin{tabular}{lcccc}
\toprule
\rowcolor{gray!25}
\multicolumn{5}{c}{\textbf{GPT-4o}} \\
\textbf{Modality} & \textbf{Linguistic Habits} & \textbf{Persona Consistency} & \textbf{Expected Action} & \textbf{Action Justification} \\
\midrule
\textbf{Text} & \begin{tabular}{@{}c@{}}1.68 $\pm$ {\scriptsize 0.04} \\ {\scriptsize (95\% CI: 1.61--1.75)}\end{tabular} & \begin{tabular}{@{}c@{}}3.00 $\pm$ {\scriptsize 0.06} \\ {\scriptsize (95\% CI: 2.87--3.12)}\end{tabular} & \begin{tabular}{@{}c@{}}3.25 $\pm$ {\scriptsize 0.05} \\ {\scriptsize (95\% CI: 3.16--3.34)}\end{tabular} & \begin{tabular}{@{}c@{}}3.91 $\pm$ {\scriptsize 0.04} \\ {\scriptsize (95\% CI: 3.83--3.99)}\end{tabular} \\
\textbf{Assisted Image} & \begin{tabular}{@{}c@{}}1.22 $\pm$ {\scriptsize 0.03} \\ {\scriptsize (95\% CI: 1.16--1.27)}\end{tabular} & \begin{tabular}{@{}c@{}}2.89 $\pm$ {\scriptsize 0.06} \\ {\scriptsize (95\% CI: 2.77--3.01)}\end{tabular} & \begin{tabular}{@{}c@{}}2.83 $\pm$ {\scriptsize 0.05} \\ {\scriptsize (95\% CI: 2.74--2.93)}\end{tabular} & \begin{tabular}{@{}c@{}}3.60 $\pm$ {\scriptsize 0.04} \\ {\scriptsize (95\% CI: 3.52--3.68)}\end{tabular} \\
\textbf{Image} & \begin{tabular}{@{}c@{}}1.05 $\pm$ {\scriptsize 0.02} \\ {\scriptsize (95\% CI: 1.00--1.10)}\end{tabular} & \begin{tabular}{@{}c@{}}2.70 $\pm$ {\scriptsize 0.06} \\ {\scriptsize (95\% CI: 2.58--2.82)}\end{tabular} & \begin{tabular}{@{}c@{}}2.75 $\pm$ {\scriptsize 0.05} \\ {\scriptsize (95\% CI: 2.66--2.84)}\end{tabular} & \begin{tabular}{@{}c@{}}3.56 $\pm$ {\scriptsize 0.04} \\ {\scriptsize (95\% CI: 3.48--3.64)}\end{tabular} \\
\textbf{Descriptive Image} & \begin{tabular}{@{}c@{}}1.17 $\pm$ {\scriptsize 0.03} \\ {\scriptsize (95\% CI: 1.12--1.23)}\end{tabular} & \begin{tabular}{@{}c@{}}3.67 $\pm$ {\scriptsize 0.06} \\ {\scriptsize (95\% CI: 3.56--3.79)}\end{tabular} & \begin{tabular}{@{}c@{}}3.26 $\pm$ {\scriptsize 0.05} \\ {\scriptsize (95\% CI: 3.17--3.35)}\end{tabular} & \begin{tabular}{@{}c@{}}3.87 $\pm$ {\scriptsize 0.04} \\ {\scriptsize (95\% CI: 3.79--3.95)}\end{tabular} \\
\midrule
\rowcolor{gray!25}
\multicolumn{5}{c}{\textbf{GPT-4o-mini}} \\
\textbf{Modality} & \textbf{Linguistic Habits} & \textbf{Persona Consistency} & \textbf{Expected Action} & \textbf{Action Justification} \\
\midrule
\textbf{Text} & \begin{tabular}{@{}c@{}}1.32 $\pm$ {\scriptsize 0.04} \\ {\scriptsize (95\% CI: 1.25--1.39)}\end{tabular} & \begin{tabular}{@{}c@{}}1.95 $\pm$ {\scriptsize 0.07} \\ {\scriptsize (95\% CI: 1.82--2.08)}\end{tabular} & \begin{tabular}{@{}c@{}}2.02 $\pm$ {\scriptsize 0.05} \\ {\scriptsize (95\% CI: 1.93--2.12)}\end{tabular} & \begin{tabular}{@{}c@{}}2.78 $\pm$ {\scriptsize 0.05} \\ {\scriptsize (95\% CI: 2.68--2.88)}\end{tabular} \\
\textbf{Assisted Image} & \begin{tabular}{@{}c@{}}1.17 $\pm$ {\scriptsize 0.03} \\ {\scriptsize (95\% CI: 1.11--1.23)}\end{tabular} & \begin{tabular}{@{}c@{}}2.17 $\pm$ {\scriptsize 0.06} \\ {\scriptsize (95\% CI: 2.04--2.30)}\end{tabular} & \begin{tabular}{@{}c@{}}2.16 $\pm$ {\scriptsize 0.05} \\ {\scriptsize (95\% CI: 2.06--2.25)}\end{tabular} & \begin{tabular}{@{}c@{}}2.88 $\pm$ {\scriptsize 0.05} \\ {\scriptsize (95\% CI: 2.78--2.97)}\end{tabular} \\
\textbf{Image} & \begin{tabular}{@{}c@{}}0.93 $\pm$ {\scriptsize 0.03} \\ {\scriptsize (95\% CI: 0.88--0.99)}\end{tabular} & \begin{tabular}{@{}c@{}}2.11 $\pm$ {\scriptsize 0.06} \\ {\scriptsize (95\% CI: 1.98--2.23)}\end{tabular} & \begin{tabular}{@{}c@{}}1.94 $\pm$ {\scriptsize 0.05} \\ {\scriptsize (95\% CI: 1.85--2.04)}\end{tabular} & \begin{tabular}{@{}c@{}}2.69 $\pm$ {\scriptsize 0.05} \\ {\scriptsize (95\% CI: 2.59--2.78)}\end{tabular} \\
\textbf{Descriptive Image} & \begin{tabular}{@{}c@{}}1.11 $\pm$ {\scriptsize 0.03} \\ {\scriptsize (95\% CI: 1.05--1.17)}\end{tabular} & \begin{tabular}{@{}c@{}}2.68 $\pm$ {\scriptsize 0.07} \\ {\scriptsize (95\% CI: 2.54--2.82)}\end{tabular} & \begin{tabular}{@{}c@{}}2.49 $\pm$ {\scriptsize 0.05} \\ {\scriptsize (95\% CI: 2.40--2.59)}\end{tabular} & \begin{tabular}{@{}c@{}}2.89 $\pm$ {\scriptsize 0.05} \\ {\scriptsize (95\% CI: 2.80--2.99)}\end{tabular} \\
\midrule
\rowcolor{gray!25}
\multicolumn{5}{c}{\textbf{Llama 3.2 11B}} \\
\textbf{Modality} & \textbf{Linguistic Habits} & \textbf{Persona Consistency} & \textbf{Expected Action} & \textbf{Action Justification} \\
\midrule
\textbf{Text} & \begin{tabular}{@{}c@{}}1.28 $\pm$ {\scriptsize 0.04} \\ {\scriptsize (95\% CI: 1.21--1.35)}\end{tabular} & \begin{tabular}{@{}c@{}}1.69 $\pm$ {\scriptsize 0.06} \\ {\scriptsize (95\% CI: 1.57--1.81)}\end{tabular} & \begin{tabular}{@{}c@{}}1.82 $\pm$ {\scriptsize 0.05} \\ {\scriptsize (95\% CI: 1.73--1.91)}\end{tabular} & \begin{tabular}{@{}c@{}}2.42 $\pm$ {\scriptsize 0.05} \\ {\scriptsize (95\% CI: 2.32--2.51)}\end{tabular} \\
\textbf{Assisted Image} & \begin{tabular}{@{}c@{}}0.67 $\pm$ {\scriptsize 0.02} \\ {\scriptsize (95\% CI: 0.63--0.71)}\end{tabular} & \begin{tabular}{@{}c@{}}1.31 $\pm$ {\scriptsize 0.05} \\ {\scriptsize (95\% CI: 1.21--1.41)}\end{tabular} & \begin{tabular}{@{}c@{}}1.19 $\pm$ {\scriptsize 0.04} \\ {\scriptsize (95\% CI: 1.12--1.26)}\end{tabular} & \begin{tabular}{@{}c@{}}1.73 $\pm$ {\scriptsize 0.04} \\ {\scriptsize (95\% CI: 1.65--1.81)}\end{tabular} \\
\textbf{Image} & \begin{tabular}{@{}c@{}}0.61 $\pm$ {\scriptsize 0.02} \\ {\scriptsize (95\% CI: 0.58--0.64)}\end{tabular} & \begin{tabular}{@{}c@{}}1.15 $\pm$ {\scriptsize 0.05} \\ {\scriptsize (95\% CI: 1.06--1.24)}\end{tabular} & \begin{tabular}{@{}c@{}}1.05 $\pm$ {\scriptsize 0.03} \\ {\scriptsize (95\% CI: 0.98--1.12)}\end{tabular} & \begin{tabular}{@{}c@{}}1.40 $\pm$ {\scriptsize 0.04} \\ {\scriptsize (95\% CI: 1.33--1.48)}\end{tabular} \\
\textbf{Descriptive Image} & \begin{tabular}{@{}c@{}}0.71 $\pm$ {\scriptsize 0.02} \\ {\scriptsize (95\% CI: 0.68--0.75)}\end{tabular} & \begin{tabular}{@{}c@{}}1.60 $\pm$ {\scriptsize 0.06} \\ {\scriptsize (95\% CI: 1.48--1.71)}\end{tabular} & \begin{tabular}{@{}c@{}}1.33 $\pm$ {\scriptsize 0.04} \\ {\scriptsize (95\% CI: 1.25--1.40)}\end{tabular} & \begin{tabular}{@{}c@{}}1.72 $\pm$ {\scriptsize 0.04} \\ {\scriptsize (95\% CI: 1.64--1.80)}\end{tabular} \\
\midrule
\rowcolor{gray!25}
\multicolumn{5}{c}{\textbf{Llama 3.2 90B}} \\
\textbf{Modality} & \textbf{Linguistic Habits} & \textbf{Persona Consistency} & \textbf{Expected Action} & \textbf{Action Justification} \\
\midrule
\textbf{Text} & \begin{tabular}{@{}c@{}}1.45 $\pm$ {\scriptsize 0.04} \\ {\scriptsize (95\% CI: 1.37--1.53)}\end{tabular} & \begin{tabular}{@{}c@{}}1.94 $\pm$ {\scriptsize 0.06} \\ {\scriptsize (95\% CI: 1.81--2.06)}\end{tabular} & \begin{tabular}{@{}c@{}}2.18 $\pm$ {\scriptsize 0.05} \\ {\scriptsize (95\% CI: 2.08--2.27)}\end{tabular} & \begin{tabular}{@{}c@{}}2.69 $\pm$ {\scriptsize 0.05} \\ {\scriptsize (95\% CI: 2.59--2.79)}\end{tabular} \\
\textbf{Assisted Image} & \begin{tabular}{@{}c@{}}0.40 $\pm$ {\scriptsize 0.03} \\ {\scriptsize (95\% CI: 0.35--0.45)}\end{tabular} & \begin{tabular}{@{}c@{}}1.01 $\pm$ {\scriptsize 0.08} \\ {\scriptsize (95\% CI: 0.86--1.16)}\end{tabular} & \begin{tabular}{@{}c@{}}0.87 $\pm$ {\scriptsize 0.06} \\ {\scriptsize (95\% CI: 0.76--0.97)}\end{tabular} & \begin{tabular}{@{}c@{}}0.98 $\pm$ {\scriptsize 0.06} \\ {\scriptsize (95\% CI: 0.86--1.09)}\end{tabular} \\
\textbf{Image} & \begin{tabular}{@{}c@{}}0.31 $\pm$ {\scriptsize 0.02} \\ {\scriptsize (95\% CI: 0.27--0.36)}\end{tabular} & \begin{tabular}{@{}c@{}}0.63 $\pm$ {\scriptsize 0.06} \\ {\scriptsize (95\% CI: 0.50--0.75)}\end{tabular} & \begin{tabular}{@{}c@{}}0.56 $\pm$ {\scriptsize 0.04} \\ {\scriptsize (95\% CI: 0.47--0.64)}\end{tabular} & \begin{tabular}{@{}c@{}}0.59 $\pm$ {\scriptsize 0.04} \\ {\scriptsize (95\% CI: 0.51--0.68)}\end{tabular} \\
\textbf{Descriptive Image} & \begin{tabular}{@{}c@{}}0.37 $\pm$ {\scriptsize 0.03} \\ {\scriptsize (95\% CI: 0.31--0.42)}\end{tabular} & \begin{tabular}{@{}c@{}}0.89 $\pm$ {\scriptsize 0.09} \\ {\scriptsize (95\% CI: 0.73--1.06)}\end{tabular} & \begin{tabular}{@{}c@{}}0.74 $\pm$ {\scriptsize 0.05} \\ {\scriptsize (95\% CI: 0.63--0.84)}\end{tabular} & \begin{tabular}{@{}c@{}}0.87 $\pm$ {\scriptsize 0.05} \\ {\scriptsize (95\% CI: 0.77--0.98)}\end{tabular} \\
\midrule
\rowcolor{gray!25}
\multicolumn{5}{c}{\textbf{Pixtral 12B}} \\
\textbf{Modality} & \textbf{Linguistic Habits} & \textbf{Persona Consistency} & \textbf{Expected Action} & \textbf{Action Justification} \\
\midrule
\textbf{Text} & \begin{tabular}{@{}c@{}}1.26 $\pm$ {\scriptsize 0.04} \\ {\scriptsize (95\% CI: 1.19--1.34)}\end{tabular} & \begin{tabular}{@{}c@{}}1.47 $\pm$ {\scriptsize 0.06} \\ {\scriptsize (95\% CI: 1.35--1.58)}\end{tabular} & \begin{tabular}{@{}c@{}}1.85 $\pm$ {\scriptsize 0.05} \\ {\scriptsize (95\% CI: 1.76--1.94)}\end{tabular} & \begin{tabular}{@{}c@{}}2.51 $\pm$ {\scriptsize 0.05} \\ {\scriptsize (95\% CI: 2.41--2.60)}\end{tabular} \\
\textbf{Assisted Image} & \begin{tabular}{@{}c@{}}1.08 $\pm$ {\scriptsize 0.03} \\ {\scriptsize (95\% CI: 1.02--1.14)}\end{tabular} & \begin{tabular}{@{}c@{}}1.43 $\pm$ {\scriptsize 0.05} \\ {\scriptsize (95\% CI: 1.32--1.54)}\end{tabular} & \begin{tabular}{@{}c@{}}1.65 $\pm$ {\scriptsize 0.04} \\ {\scriptsize (95\% CI: 1.56--1.73)}\end{tabular} & \begin{tabular}{@{}c@{}}2.32 $\pm$ {\scriptsize 0.05} \\ {\scriptsize (95\% CI: 2.22--2.41)}\end{tabular} \\
\textbf{Image} & \begin{tabular}{@{}c@{}}1.04 $\pm$ {\scriptsize 0.03} \\ {\scriptsize (95\% CI: 0.98--1.10)}\end{tabular} & \begin{tabular}{@{}c@{}}1.90 $\pm$ {\scriptsize 0.06} \\ {\scriptsize (95\% CI: 1.78--2.02)}\end{tabular} & \begin{tabular}{@{}c@{}}2.06 $\pm$ {\scriptsize 0.05} \\ {\scriptsize (95\% CI: 1.96--2.15)}\end{tabular} & \begin{tabular}{@{}c@{}}2.62 $\pm$ {\scriptsize 0.05} \\ {\scriptsize (95\% CI: 2.52--2.71)}\end{tabular} \\
\textbf{Descriptive Image} & \begin{tabular}{@{}c@{}}1.05 $\pm$ {\scriptsize 0.03} \\ {\scriptsize (95\% CI: 0.99--1.11)}\end{tabular} & \begin{tabular}{@{}c@{}}2.75 $\pm$ {\scriptsize 0.07} \\ {\scriptsize (95\% CI: 2.61--2.88)}\end{tabular} & \begin{tabular}{@{}c@{}}2.42 $\pm$ {\scriptsize 0.05} \\ {\scriptsize (95\% CI: 2.32--2.51)}\end{tabular} & \begin{tabular}{@{}c@{}}2.97 $\pm$ {\scriptsize 0.05} \\ {\scriptsize (95\% CI: 2.87--3.06)}\end{tabular} \\
\bottomrule
\end{tabular}
\caption{Evaluation Metrics by Model and Modality with \textbf{\underline{GPT-4o}} as the evaluator. Each cell shows mean $\pm$ {\scriptsize SEM} on the first line and 95\% CI on the second.}
\label{tab:eval-table-gpt-4o}
\end{table*}

\begin{table*}[t]
\centering
\small
\begin{tabular}{lcccc}
\toprule
\rowcolor{gray!25}
\multicolumn{5}{c}{\textbf{GPT-4o}} \\
\textbf{Modality} & \textbf{Linguistic Habits} & \textbf{Persona Consistency} & \textbf{Expected Action} & \textbf{Action Justification} \\
\midrule
\textbf{Text} & \begin{tabular}{@{}c@{}}2.47 $\pm$ {\scriptsize 0.03} \\ {\scriptsize (95\% CI: 2.41--2.53)}\end{tabular} & \begin{tabular}{@{}c@{}}3.88 $\pm$ {\scriptsize 0.04} \\ {\scriptsize (95\% CI: 3.79--3.97)}\end{tabular} & \begin{tabular}{@{}c@{}}4.46 $\pm$ {\scriptsize 0.03} \\ {\scriptsize (95\% CI: 4.41--4.52)}\end{tabular} & \begin{tabular}{@{}c@{}}4.34 $\pm$ {\scriptsize 0.03} \\ {\scriptsize (95\% CI: 4.28--4.40)}\end{tabular} \\
\textbf{Assisted Image} & \begin{tabular}{@{}c@{}}2.01 $\pm$ {\scriptsize 0.03} \\ {\scriptsize (95\% CI: 1.95--2.06)}\end{tabular} & \begin{tabular}{@{}c@{}}3.50 $\pm$ {\scriptsize 0.04} \\ {\scriptsize (95\% CI: 3.42--3.59)}\end{tabular} & \begin{tabular}{@{}c@{}}4.35 $\pm$ {\scriptsize 0.03} \\ {\scriptsize (95\% CI: 4.30--4.40)}\end{tabular} & \begin{tabular}{@{}c@{}}4.03 $\pm$ {\scriptsize 0.03} \\ {\scriptsize (95\% CI: 3.97--4.10)}\end{tabular} \\
\textbf{Image} & \begin{tabular}{@{}c@{}}1.96 $\pm$ {\scriptsize 0.03} \\ {\scriptsize (95\% CI: 1.90--2.02)}\end{tabular} & \begin{tabular}{@{}c@{}}3.36 $\pm$ {\scriptsize 0.04} \\ {\scriptsize (95\% CI: 3.28--3.45)}\end{tabular} & \begin{tabular}{@{}c@{}}4.36 $\pm$ {\scriptsize 0.03} \\ {\scriptsize (95\% CI: 4.31--4.41)}\end{tabular} & \begin{tabular}{@{}c@{}}3.93 $\pm$ {\scriptsize 0.04} \\ {\scriptsize (95\% CI: 3.86--4.00)}\end{tabular} \\
\textbf{Descriptive Image} & \begin{tabular}{@{}c@{}}2.01 $\pm$ {\scriptsize 0.03} \\ {\scriptsize (95\% CI: 1.95--2.07)}\end{tabular} & \begin{tabular}{@{}c@{}}4.14 $\pm$ {\scriptsize 0.04} \\ {\scriptsize (95\% CI: 4.07--4.21)}\end{tabular} & \begin{tabular}{@{}c@{}}4.62 $\pm$ {\scriptsize 0.02} \\ {\scriptsize (95\% CI: 4.58--4.66)}\end{tabular} & \begin{tabular}{@{}c@{}}4.12 $\pm$ {\scriptsize 0.03} \\ {\scriptsize (95\% CI: 4.05--4.19)}\end{tabular} \\
\midrule
\rowcolor{gray!25}
\multicolumn{5}{c}{\textbf{GPT-4o-mini}} \\
\textbf{Modality} & \textbf{Linguistic Habits} & \textbf{Persona Consistency} & \textbf{Expected Action} & \textbf{Action Justification} \\
\midrule
\textbf{Text} & \begin{tabular}{@{}c@{}}2.31 $\pm$ {\scriptsize 0.03} \\ {\scriptsize (95\% CI: 2.25--2.36)}\end{tabular} & \begin{tabular}{@{}c@{}}4.01 $\pm$ {\scriptsize 0.04} \\ {\scriptsize (95\% CI: 3.93--4.09)}\end{tabular} & \begin{tabular}{@{}c@{}}4.47 $\pm$ {\scriptsize 0.03} \\ {\scriptsize (95\% CI: 4.43--4.52)}\end{tabular} & \begin{tabular}{@{}c@{}}4.34 $\pm$ {\scriptsize 0.03} \\ {\scriptsize (95\% CI: 4.28--4.40)}\end{tabular} \\
\textbf{Assisted Image} & \begin{tabular}{@{}c@{}}2.06 $\pm$ {\scriptsize 0.03} \\ {\scriptsize (95\% CI: 2.01--2.12)}\end{tabular} & \begin{tabular}{@{}c@{}}3.79 $\pm$ {\scriptsize 0.04} \\ {\scriptsize (95\% CI: 3.70--3.87)}\end{tabular} & \begin{tabular}{@{}c@{}}4.42 $\pm$ {\scriptsize 0.02} \\ {\scriptsize (95\% CI: 4.37--4.47)}\end{tabular} & \begin{tabular}{@{}c@{}}4.07 $\pm$ {\scriptsize 0.03} \\ {\scriptsize (95\% CI: 4.00--4.14)}\end{tabular} \\
\textbf{Image} & \begin{tabular}{@{}c@{}}2.04 $\pm$ {\scriptsize 0.03} \\ {\scriptsize (95\% CI: 1.98--2.09)}\end{tabular} & \begin{tabular}{@{}c@{}}3.84 $\pm$ {\scriptsize 0.04} \\ {\scriptsize (95\% CI: 3.76--3.92)}\end{tabular} & \begin{tabular}{@{}c@{}}4.44 $\pm$ {\scriptsize 0.02} \\ {\scriptsize (95\% CI: 4.39--4.48)}\end{tabular} & \begin{tabular}{@{}c@{}}4.02 $\pm$ {\scriptsize 0.03} \\ {\scriptsize (95\% CI: 3.95--4.09)}\end{tabular} \\
\textbf{Descriptive Image} & \begin{tabular}{@{}c@{}}2.15 $\pm$ {\scriptsize 0.03} \\ {\scriptsize (95\% CI: 2.09--2.20)}\end{tabular} & \begin{tabular}{@{}c@{}}4.49 $\pm$ {\scriptsize 0.03} \\ {\scriptsize (95\% CI: 4.43--4.55)}\end{tabular} & \begin{tabular}{@{}c@{}}4.69 $\pm$ {\scriptsize 0.02} \\ {\scriptsize (95\% CI: 4.65--4.72)}\end{tabular} & \begin{tabular}{@{}c@{}}4.20 $\pm$ {\scriptsize 0.03} \\ {\scriptsize (95\% CI: 4.14--4.27)}\end{tabular} \\
\midrule
\rowcolor{gray!25}
\multicolumn{5}{c}{\textbf{Llama 3.2 11B}} \\
\textbf{Modality} & \textbf{Linguistic Habits} & \textbf{Persona Consistency} & \textbf{Expected Action} & \textbf{Action Justification} \\
\midrule
\textbf{Text} & \begin{tabular}{@{}c@{}}3.12 $\pm$ {\scriptsize 0.03} \\ {\scriptsize (95\% CI: 3.07--3.18)}\end{tabular} & \begin{tabular}{@{}c@{}}3.90 $\pm$ {\scriptsize 0.04} \\ {\scriptsize (95\% CI: 3.82--3.99)}\end{tabular} & \begin{tabular}{@{}c@{}}4.14 $\pm$ {\scriptsize 0.03} \\ {\scriptsize (95\% CI: 4.08--4.19)}\end{tabular} & \begin{tabular}{@{}c@{}}4.07 $\pm$ {\scriptsize 0.03} \\ {\scriptsize (95\% CI: 4.01--4.13)}\end{tabular} \\
\textbf{Assisted Image} & \begin{tabular}{@{}c@{}}1.93 $\pm$ {\scriptsize 0.03} \\ {\scriptsize (95\% CI: 1.87--1.99)}\end{tabular} & \begin{tabular}{@{}c@{}}3.02 $\pm$ {\scriptsize 0.04} \\ {\scriptsize (95\% CI: 2.93--3.11)}\end{tabular} & \begin{tabular}{@{}c@{}}3.36 $\pm$ {\scriptsize 0.03} \\ {\scriptsize (95\% CI: 3.30--3.43)}\end{tabular} & \begin{tabular}{@{}c@{}}3.15 $\pm$ {\scriptsize 0.04} \\ {\scriptsize (95\% CI: 3.08--3.23)}\end{tabular} \\
\textbf{Image} & \begin{tabular}{@{}c@{}}2.03 $\pm$ {\scriptsize 0.03} \\ {\scriptsize (95\% CI: 1.97--2.09)}\end{tabular} & \begin{tabular}{@{}c@{}}2.66 $\pm$ {\scriptsize 0.05} \\ {\scriptsize (95\% CI: 2.56--2.75)}\end{tabular} & \begin{tabular}{@{}c@{}}3.02 $\pm$ {\scriptsize 0.04} \\ {\scriptsize (95\% CI: 2.95--3.09)}\end{tabular} & \begin{tabular}{@{}c@{}}2.94 $\pm$ {\scriptsize 0.04} \\ {\scriptsize (95\% CI: 2.87--3.02)}\end{tabular} \\
\textbf{Descriptive Image} & \begin{tabular}{@{}c@{}}2.17 $\pm$ {\scriptsize 0.03} \\ {\scriptsize (95\% CI: 2.10--2.23)}\end{tabular} & \begin{tabular}{@{}c@{}}3.50 $\pm$ {\scriptsize 0.04} \\ {\scriptsize (95\% CI: 3.42--3.59)}\end{tabular} & \begin{tabular}{@{}c@{}}3.65 $\pm$ {\scriptsize 0.03} \\ {\scriptsize (95\% CI: 3.59--3.71)}\end{tabular} & \begin{tabular}{@{}c@{}}3.27 $\pm$ {\scriptsize 0.04} \\ {\scriptsize (95\% CI: 3.19--3.34)}\end{tabular} \\
\midrule
\rowcolor{gray!25}
\multicolumn{5}{c}{\textbf{Llama 3.2 90B}} \\
\textbf{Modality} & \textbf{Linguistic Habits} & \textbf{Persona Consistency} & \textbf{Expected Action} & \textbf{Action Justification} \\
\midrule
\textbf{Text} & \begin{tabular}{@{}c@{}}3.20 $\pm$ {\scriptsize 0.03} \\ {\scriptsize (95\% CI: 3.14--3.25)}\end{tabular} & \begin{tabular}{@{}c@{}}4.05 $\pm$ {\scriptsize 0.04} \\ {\scriptsize (95\% CI: 3.96--4.13)}\end{tabular} & \begin{tabular}{@{}c@{}}4.38 $\pm$ {\scriptsize 0.03} \\ {\scriptsize (95\% CI: 4.32--4.43)}\end{tabular} & \begin{tabular}{@{}c@{}}4.29 $\pm$ {\scriptsize 0.03} \\ {\scriptsize (95\% CI: 4.24--4.35)}\end{tabular} \\
\textbf{Assisted Image} & \begin{tabular}{@{}c@{}}2.09 $\pm$ {\scriptsize 0.07} \\ {\scriptsize (95\% CI: 1.96--2.22)}\end{tabular} & \begin{tabular}{@{}c@{}}2.24 $\pm$ {\scriptsize 0.08} \\ {\scriptsize (95\% CI: 2.09--2.40)}\end{tabular} & \begin{tabular}{@{}c@{}}2.00 $\pm$ {\scriptsize 0.06} \\ {\scriptsize (95\% CI: 1.89--2.12)}\end{tabular} & \begin{tabular}{@{}c@{}}2.36 $\pm$ {\scriptsize 0.07} \\ {\scriptsize (95\% CI: 2.21--2.50)}\end{tabular} \\
\textbf{Image} & \begin{tabular}{@{}c@{}}2.18 $\pm$ {\scriptsize 0.08} \\ {\scriptsize (95\% CI: 2.03--2.34)}\end{tabular} & \begin{tabular}{@{}c@{}}1.53 $\pm$ {\scriptsize 0.07} \\ {\scriptsize (95\% CI: 1.38--1.67)}\end{tabular} & \begin{tabular}{@{}c@{}}1.48 $\pm$ {\scriptsize 0.05} \\ {\scriptsize (95\% CI: 1.38--1.58)}\end{tabular} & \begin{tabular}{@{}c@{}}2.02 $\pm$ {\scriptsize 0.08} \\ {\scriptsize (95\% CI: 1.87--2.18)}\end{tabular} \\
\textbf{Descriptive Image} & \begin{tabular}{@{}c@{}}2.23 $\pm$ {\scriptsize 0.08} \\ {\scriptsize (95\% CI: 2.08--2.38)}\end{tabular} & \begin{tabular}{@{}c@{}}1.96 $\pm$ {\scriptsize 0.09} \\ {\scriptsize (95\% CI: 1.78--2.14)}\end{tabular} & \begin{tabular}{@{}c@{}}1.74 $\pm$ {\scriptsize 0.06} \\ {\scriptsize (95\% CI: 1.62--1.85)}\end{tabular} & \begin{tabular}{@{}c@{}}2.12 $\pm$ {\scriptsize 0.08} \\ {\scriptsize (95\% CI: 1.97--2.27)}\end{tabular} \\
\midrule
\rowcolor{gray!25}
\multicolumn{5}{c}{\textbf{Pixtral 12B}} \\
\textbf{Modality} & \textbf{Linguistic Habits} & \textbf{Persona Consistency} & \textbf{Expected Action} & \textbf{Action Justification} \\
\midrule
\textbf{Text} & \begin{tabular}{@{}c@{}}2.31 $\pm$ {\scriptsize 0.03} \\ {\scriptsize (95\% CI: 2.25--2.36)}\end{tabular} & \begin{tabular}{@{}c@{}}3.28 $\pm$ {\scriptsize 0.05} \\ {\scriptsize (95\% CI: 3.19--3.38)}\end{tabular} & \begin{tabular}{@{}c@{}}4.01 $\pm$ {\scriptsize 0.03} \\ {\scriptsize (95\% CI: 3.95--4.07)}\end{tabular} & \begin{tabular}{@{}c@{}}4.14 $\pm$ {\scriptsize 0.03} \\ {\scriptsize (95\% CI: 4.08--4.19)}\end{tabular} \\
\textbf{Assisted Image} & \begin{tabular}{@{}c@{}}2.18 $\pm$ {\scriptsize 0.03} \\ {\scriptsize (95\% CI: 2.12--2.24)}\end{tabular} & \begin{tabular}{@{}c@{}}3.22 $\pm$ {\scriptsize 0.05} \\ {\scriptsize (95\% CI: 3.13--3.31)}\end{tabular} & \begin{tabular}{@{}c@{}}4.05 $\pm$ {\scriptsize 0.03} \\ {\scriptsize (95\% CI: 3.99--4.11)}\end{tabular} & \begin{tabular}{@{}c@{}}3.82 $\pm$ {\scriptsize 0.03} \\ {\scriptsize (95\% CI: 3.76--3.89)}\end{tabular} \\
\textbf{Image} & \begin{tabular}{@{}c@{}}2.26 $\pm$ {\scriptsize 0.03} \\ {\scriptsize (95\% CI: 2.20--2.32)}\end{tabular} & \begin{tabular}{@{}c@{}}3.64 $\pm$ {\scriptsize 0.05} \\ {\scriptsize (95\% CI: 3.55--3.73)}\end{tabular} & \begin{tabular}{@{}c@{}}4.25 $\pm$ {\scriptsize 0.03} \\ {\scriptsize (95\% CI: 4.20--4.31)}\end{tabular} & \begin{tabular}{@{}c@{}}3.79 $\pm$ {\scriptsize 0.04} \\ {\scriptsize (95\% CI: 3.72--3.86)}\end{tabular} \\
\textbf{Descriptive Image} & \begin{tabular}{@{}c@{}}2.15 $\pm$ {\scriptsize 0.03} \\ {\scriptsize (95\% CI: 2.09--2.21)}\end{tabular} & \begin{tabular}{@{}c@{}}4.43 $\pm$ {\scriptsize 0.03} \\ {\scriptsize (95\% CI: 4.36--4.49)}\end{tabular} & \begin{tabular}{@{}c@{}}4.64 $\pm$ {\scriptsize 0.02} \\ {\scriptsize (95\% CI: 4.60--4.68)}\end{tabular} & \begin{tabular}{@{}c@{}}4.03 $\pm$ {\scriptsize 0.04} \\ {\scriptsize (95\% CI: 3.96--4.10)}\end{tabular} \\
\bottomrule
\end{tabular}
\caption{Evaluation Metrics by Model and Modality with \textbf{\underline{Gemini 2.0 Flash}} as the evaluator. Each cell shows mean $\pm$ {\scriptsize SEM} on the first line and 95\% CI on the second.}
\label{tab:eval-table-gemini-flash}
\end{table*}

\subsection{Human survey design}\label{app:human}
Figure~\ref{fig:survey} demonstrates our survey design that we conduct on $8$ independent annotators to evaluate the quality of LLM evaluators. In particular, we first show the instructions to evaluate the responses for a prompt and a persona, followed by $10$ such questions. 
\begin{figure*}[t]
    \centering
    \subfloat[Instruction]{\includegraphics[width=0.48\textwidth]{survey_start.png}}\hfill
    \subfloat[Question]{\includegraphics[width=0.48\textwidth]{survey_q1.png}}
    \caption{Human survey design}
    \label{fig:survey}
\end{figure*}


% \begin{table*}[t]
%     \centering
%     \caption{}
%     \label{tab:personalist}
%     \resizebox{1.0\linewidth}{!}{
%     \begin{tabular}{l c}
%     \toprule
%     Scenario & Target Attribute \\
%     \midrule
%     "Your extended family is digitizing old home videos for a reunion. While helping, you discover some footage from your early childhood that needs to be converted. The deadline is next week. You ..." & Age \\
%     "During a basement cleanup, you find your old gaming systems and accessories from when you were 12. A local collector has shown interest in purchasing the set. You ..." \\
%     "You're coordinating a playlist for your high school reunion after-party. The organizers want music specifically from your graduating years to recreate the atmosphere. You ..." \\
%     \bottomrule
%     \end{tabular}
% }
% \end{table*}


\end{document}
