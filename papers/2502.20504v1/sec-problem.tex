\section{Influence of Modality in Persona Embodiment of Multimodal LLMs}\label{sec:problem}

% In this work, we study the ability of an LLM to embody a persona as represented in different modalities. 

The problem of embodying a persona can be defined as the task of generating responses consistent with a specified character, identity, or role~\citep{samuel2024personagym}. This involves maintaining coherence in linguistic style, beliefs, knowledge, and affective tone in a way that aligns with the intended persona.

In this work, we investigate the effect of representing the persona $p$ in different modalities, denoted as $\CR(p)$, on multimodal LLMs. In particular, we consider two common modalities, text and image, and evaluate the LLM's performance on these equivalent representations. Additionally, we also consider combining visual and textual features of the personas. We describe these $4$ different representations $\CR(p)$ in more detail below. 
% \ks{add an example for each maybe just in figure 1 -- you can make it double-column as well. maybe not super important but the pipeline figure is.}\jb{added to figure 2}\ks{awesome!}


\subsection{Persona Modality Representations}
\begin{itemize}
    \item \textbf{Text ($\CT$):} Textual descriptions of a persona correspond to a sequence of sentences characterizing the persona in natural language.

    \item \textbf{Image ($\CI$):} A persona can also be depicted visually using an image of the person in a representative environment that characterizes the persona visually.
    
    \item \textbf{Assisted Image ($\CI_{A}$):} Since certain features may be obscured in the image, textual attributes of the persona can also be included explicitly as text. 
    
    \item \textbf{Descriptive Image ($\CI_{D}$):} In this case, we include the textual attributes in the image itself using typography instead of in the text.
    
\end{itemize}
