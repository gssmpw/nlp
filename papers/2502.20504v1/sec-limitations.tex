\section*{Limitations}\label{sec-limitations}
% \ks{write somethign}
A limitation of our work is that we only deal with 40 personas. However, due to a lack of any persona dataset with equivalent representations in different modalities, we see this as our contribution and leave it for future works to expand the scale of the study. Furthermore, we specifically increase the diversity of these personas across $4$ well-grounded categories, focusing on the quality of our dataset. As the field of persona alignment in LLMs is still quite nascent, we believe quality becomes more important than quantity. Additionally, it should be noted that the persona modality representations may not align perfectly across all details. Our pipeline employs two distinct mapping functions---Stable Diffusion (text-to-image) and GPT-4o-mini (image-to-text)---which will naturally introduce extraneous information or inconsistencies between representations. However, this limitation is acceptable for our evaluation framework since we only test for the presence and consistency of specific attributes rather than complete fidelity across all possible persona characteristics. Another limitation is that we have only validated our results on a small set of human annotators. We circumvent this by leveraging the validation of LLM-based evaluation with human evaluations~\citep{samuel2024personagym} while also showing a high correlation of our results across different LLM evaluators. 

\section*{Broader implications and social impact}
We intend our proposed dataset to be used strictly for academic purposes. While we design our dataset such that it does not contain any harmful and private content, our pipeline can be adapted to generate such unintended visual personas. However, we note that this is not a direct result of our artifact and can also be possible through directly querying the StableDiffusion APIs. Thus, we expect our contributions of dataset and evaluation methodology to have an overall positive social impact by inspiring future research on aligning modalities for persona embodiment.