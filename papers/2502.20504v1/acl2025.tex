% This must be in the first 5 lines to tell arXiv to use pdfLaTeX, which is strongly recommended.
\pdfoutput=1
% In particular, the hyperref package requires pdfLaTeX in order to break URLs across lines.

\documentclass[11pt]{article}

% Change "review" to "final" to generate the final (sometimes called camera-ready) version.
% Change to "preprint" to generate a non-anonymous version with page numbers.
\usepackage[review]{acl}

% Standard package includes
\usepackage{times}
\usepackage{latexsym}

% For proper rendering and hyphenation of words containing Latin characters (including in bib files)
\usepackage[T1]{fontenc}
% For Vietnamese characters
% \usepackage[T5]{fontenc}
% See https://www.latex-project.org/help/documentation/encguide.pdf for other character sets

% This assumes your files are encoded as UTF8
\usepackage[utf8]{inputenc}

% This is not strictly necessary, and may be commented out,
% but it will improve the layout of the manuscript,
% and will typically save some space.
\usepackage{microtype}

% This is also not strictly necessary, and may be commented out.
% However, it will improve the aesthetics of text in
% the typewriter font.
\usepackage{inconsolata}

%Including images in your LaTeX document requires adding
%additional package(s)
\usepackage{enumitem}
\usepackage{amsmath}
\usepackage{amssymb}
\usepackage{mathtools}
\DeclareMathOperator{\argmax}{argmax}
\usepackage{xspace}
\usepackage{graphicx}
\usepackage{subfig}
\usepackage{booktabs}
\usepackage{multirow}
\usepackage{array}
\usepackage[dvipsnames]{xcolor}
\usepackage{nicefrac}
\usepackage[table]{xcolor}
\usepackage{listings}
\lstset{
   breaklines=true,
   basicstyle=\ttfamily,
}
\newtheorem{problem}{Problem}
\newcommand{\ie}{\textit{i.e.}\xspace}
\newcommand{\eg}{\textit{e.g.}\xspace}
\renewenvironment{quote}
  {\begin{list}{}%
     {\setlength{\leftmargin}{2mm} % Adjust left margin
      \setlength{\rightmargin}{2mm}} % Adjust right margin
     \item\relax}
  {\end{list}}

%
% --- inline annotations
%
\newcommand{\red}[1]{{\color{red}#1}}
\newcommand{\todo}[1]{{\color{red}#1}}
\newcommand{\TODO}[1]{\textbf{\color{red}[TODO: #1]}}
% --- disable by uncommenting  
% \renewcommand{\TODO}[1]{}
% \renewcommand{\todo}[1]{#1}



\newcommand{\VLM}{LVLM\xspace} 
\newcommand{\ours}{PeKit\xspace}
\newcommand{\yollava}{Yo’LLaVA\xspace}

\newcommand{\thisismy}{This-Is-My-Img\xspace}
\newcommand{\myparagraph}[1]{\noindent\textbf{#1}}
\newcommand{\vdoro}[1]{{\color[rgb]{0.4, 0.18, 0.78} {[V] #1}}}
% --- disable by uncommenting  
% \renewcommand{\TODO}[1]{}
% \renewcommand{\todo}[1]{#1}
\usepackage{slashbox}
% Vectors
\newcommand{\bB}{\mathcal{B}}
\newcommand{\bw}{\mathbf{w}}
\newcommand{\bs}{\mathbf{s}}
\newcommand{\bo}{\mathbf{o}}
\newcommand{\bn}{\mathbf{n}}
\newcommand{\bc}{\mathbf{c}}
\newcommand{\bp}{\mathbf{p}}
\newcommand{\bS}{\mathbf{S}}
\newcommand{\bk}{\mathbf{k}}
\newcommand{\bmu}{\boldsymbol{\mu}}
\newcommand{\bx}{\mathbf{x}}
\newcommand{\bg}{\mathbf{g}}
\newcommand{\be}{\mathbf{e}}
\newcommand{\bX}{\mathbf{X}}
\newcommand{\by}{\mathbf{y}}
\newcommand{\bv}{\mathbf{v}}
\newcommand{\bz}{\mathbf{z}}
\newcommand{\bq}{\mathbf{q}}
\newcommand{\bff}{\mathbf{f}}
\newcommand{\bu}{\mathbf{u}}
\newcommand{\bh}{\mathbf{h}}
\newcommand{\bb}{\mathbf{b}}

\newcommand{\rone}{\textcolor{green}{R1}}
\newcommand{\rtwo}{\textcolor{orange}{R2}}
\newcommand{\rthree}{\textcolor{red}{R3}}
\usepackage{amsmath}
%\usepackage{arydshln}
\DeclareMathOperator{\similarity}{sim}
\DeclareMathOperator{\AvgPool}{AvgPool}

\newcommand{\argmax}{\mathop{\mathrm{argmax}}}     



\newcolumntype{H}{>{\setbox0=\hbox\bgroup}c<{\egroup}@{}}
\newcommand{\ks}[1]{\textcolor{red}{[KS: #1]}}
\newcommand{\jb}[1]{\textcolor{blue}{[JB: #1]}}

\definecolor{AgeBg}{RGB}{232,244,255}
\definecolor{LocBg}{RGB}{255,240,240}
\definecolor{OccBg}{RGB}{245,235,255} 
% If the title and author information does not fit in the area allocated, uncomment the following
%
%\setlength\titlebox{<dim>}
%
% and set <dim> to something 5cm or larger.

% %
% --- inline annotations
%
\newcommand{\red}[1]{{\color{red}#1}}
\newcommand{\todo}[1]{{\color{red}#1}}
\newcommand{\TODO}[1]{\textbf{\color{red}[TODO: #1]}}
% --- disable by uncommenting  
% \renewcommand{\TODO}[1]{}
% \renewcommand{\todo}[1]{#1}



\newcommand{\VLM}{LVLM\xspace} 
\newcommand{\ours}{PeKit\xspace}
\newcommand{\yollava}{Yo’LLaVA\xspace}

\newcommand{\thisismy}{This-Is-My-Img\xspace}
\newcommand{\myparagraph}[1]{\noindent\textbf{#1}}
\newcommand{\vdoro}[1]{{\color[rgb]{0.4, 0.18, 0.78} {[V] #1}}}
% --- disable by uncommenting  
% \renewcommand{\TODO}[1]{}
% \renewcommand{\todo}[1]{#1}
\usepackage{slashbox}
% Vectors
\newcommand{\bB}{\mathcal{B}}
\newcommand{\bw}{\mathbf{w}}
\newcommand{\bs}{\mathbf{s}}
\newcommand{\bo}{\mathbf{o}}
\newcommand{\bn}{\mathbf{n}}
\newcommand{\bc}{\mathbf{c}}
\newcommand{\bp}{\mathbf{p}}
\newcommand{\bS}{\mathbf{S}}
\newcommand{\bk}{\mathbf{k}}
\newcommand{\bmu}{\boldsymbol{\mu}}
\newcommand{\bx}{\mathbf{x}}
\newcommand{\bg}{\mathbf{g}}
\newcommand{\be}{\mathbf{e}}
\newcommand{\bX}{\mathbf{X}}
\newcommand{\by}{\mathbf{y}}
\newcommand{\bv}{\mathbf{v}}
\newcommand{\bz}{\mathbf{z}}
\newcommand{\bq}{\mathbf{q}}
\newcommand{\bff}{\mathbf{f}}
\newcommand{\bu}{\mathbf{u}}
\newcommand{\bh}{\mathbf{h}}
\newcommand{\bb}{\mathbf{b}}

\newcommand{\rone}{\textcolor{green}{R1}}
\newcommand{\rtwo}{\textcolor{orange}{R2}}
\newcommand{\rthree}{\textcolor{red}{R3}}
\usepackage{amsmath}
%\usepackage{arydshln}
\DeclareMathOperator{\similarity}{sim}
\DeclareMathOperator{\AvgPool}{AvgPool}

\newcommand{\argmax}{\mathop{\mathrm{argmax}}}     


% \newif\ifpreprint
% \preprintfalse


\title{\textit{A Thousand Words or An Image}: Studying the Influence of \\ Persona Modality in Multimodal LLMs}

% Author information can be set in various styles:
% For several authors from the same institution:
% \author{Author 1 \and ... \and Author n \\
%         Address line \\ ... \\ Address line}
% if the names do not fit well on one line use
%         Author 1 \\ {\bf Author 2} \\ ... \\ {\bf Author n} \\
% For authors from different institutions:
% \author{Author 1 \\ Address line \\  ... \\ Address line
%         \And  ... \And
%         Author n \\ Address line \\ ... \\ Address line}
% To start a separate ``row'' of authors use \AND, as in
% \author{Author 1 \\ Address line \\  ... \\ Address line
%         \AND
%         Author 2 \\ Address line \\ ... \\ Address line \And
%         Author 3 \\ Address line \\ ... \\ Address line}
%\author{
%  \textbf{First Author\textsuperscript{1}},
%  \textbf{Second Author\textsuperscript{1,2}},
%  \textbf{Third T. Author\textsuperscript{1}},
%  \textbf{Fourth Author\textsuperscript{1}},
%\\
%  \textbf{Fifth Author\textsuperscript{1,2}},
%  \textbf{Sixth Author\textsuperscript{1}},
%  \textbf{Seventh Author\textsuperscript{1}},
%  \textbf{Eighth Author \textsuperscript{1,2,3,4}},
%\\
%  \textbf{Ninth Author\textsuperscript{1}},
%  \textbf{Tenth Author\textsuperscript{1}},
%  \textbf{Eleventh E. Author\textsuperscript{1,2,3,4,5}},
%  \textbf{Twelfth Author\textsuperscript{1}},
%\\
%  \textbf{Thirteenth Author\textsuperscript{3}},
%  \textbf{Fourteenth F. Author\textsuperscript{2,4}},
%  \textbf{Fifteenth Author\textsuperscript{1}},
%  \textbf{Sixteenth Author\textsuperscript{1}},
%\\
%  \textbf{Seventeenth S. Author\textsuperscript{4,5}},
%  \textbf{Eighteenth Author\textsuperscript{3,4}},
%  \textbf{Nineteenth N. Author\textsuperscript{2,5}},
%  \textbf{Twentieth Author\textsuperscript{1}}
%\\
%\\
%  \textsuperscript{1}Affiliation 1,
%  \textsuperscript{2}Affiliation 2,
%  \textsuperscript{3}Affiliation 3,
%  \textsuperscript{4}Affiliation 4,
%  \textsuperscript{5}Affiliation 5
%\\
%  \small{
%    \textbf{Correspondence:} \href{mailto:email@domain}{email@domain}
%  }
%}

\begin{document}


\maketitle

\begin{abstract}
    \begin{abstract}  
Test time scaling is currently one of the most active research areas that shows promise after training time scaling has reached its limits.
Deep-thinking (DT) models are a class of recurrent models that can perform easy-to-hard generalization by assigning more compute to harder test samples.
However, due to their inability to determine the complexity of a test sample, DT models have to use a large amount of computation for both easy and hard test samples.
Excessive test time computation is wasteful and can cause the ``overthinking'' problem where more test time computation leads to worse results.
In this paper, we introduce a test time training method for determining the optimal amount of computation needed for each sample during test time.
We also propose Conv-LiGRU, a novel recurrent architecture for efficient and robust visual reasoning. 
Extensive experiments demonstrate that Conv-LiGRU is more stable than DT, effectively mitigates the ``overthinking'' phenomenon, and achieves superior accuracy.
\end{abstract}  
\end{abstract}

% !TEX root = ../main.tex

Large Language Models (LLMs) have shown remarkable capabilities on numerous tasks in Natural Language Processing (NLP), 
ranging from language understanding to generation \cite{bubeck2023sparks, achiam2023gpt,team2023gemini, dubey2024llama}. The huge success of LLMs comes with important challenges to deploy them due to their massive size and computational costs. For instance,  Llama-3-405B \cite{dubey2024llama} requires 780GB of storage in half precision (FP16) and hence multiple high-end GPUs are needed just for inference. \textit{Model compression} has emerged as an important line of research to reduce the costs associated with deploying these foundation models. In particular, neural network pruning \cite{obd, hassibi1992second, benbaki2023fast}, where model weights are made to be sparse after training, has garnered significant attention. Different sparsity structures (Structured, Semi-Structured and Unstructured) obtained after neural network pruning result in different acceleration schemes. \textit{Structured pruning} removes entire structures such as channels, filters, or attention heads \cite{lebedev2016fast,wen2016learning,voita2019analyzing,el2022data} and readily results in acceleration as model weights dimensions are reduced. \textit{Semi-Structured pruning}, also known as, N:M sparsity \cite{zhou2021learning} requires that at most $N$ out of $M$ consecutive elements are non-zero elements. Modern NVIDIA GPUs provide support for 2:4 sparsity acceleration. \textit{Unstructured pruning} removes individual weights \cite{han2015learning, guo2016dynamic} from the model's weights and requires specialized hardware for acceleration. For instance, DeepSparse \cite{kurtic2022optimal, pmlr-v119-kurtz20a, DBLP:journals/corr/abs-2111-13445} provide CPU inference acceleration for unstructured sparsity.\\
Specializing to LLMs, one-shot pruning~\cite{meng2024alps, frantar2023sparsegpt, sun2023simple, zhang2023dynamic}, where one does a single forward pass on a small amount of calibration data, and prunes the model without expensive fine-tuning/retraining, is of particular interest. This setup requires less hardware requirements. For instance, \citet{meng2024alps} show how to prune an OPT-30B \cite{opt} using a single consumer-level V100 GPU with 32GB of CUDA memory, whereas full fine-tuning of such model using Adam \cite{kingma2014adam} at half-precision requires more than 220GB of CUDA memory.

Although one-shot pruning has desirable computational properties, it can degrade models' predictive and generative performance. To this end, recent work has studied extensions of model pruning to achieve smaller utility drop of model performance from compression. 
% Multiple one-shot methods have been developed in quantization \cite{frantar2022gptq, frantar2023sparsegpt, lin2024awq, behdin2023quantease, dettmers2023spqr} and neural network pruning \cite{frantar2023sparsegpt, meng2024alps, zhang2024oats}, which is closer to this paper's line of research. These one-shot methods do not require retraining--which is extremely expensive for models of the size of Llama-3-405B-- and work as resource-saving techniques that retain the model's performance. 

An interesting compression mechanism in the field of \textit{model compression} is the Sparse plus Low-Rank Matrix-Decomposition problem which aims to approximate model's weights by a sparse component plus a low-rank component~\cite{hintermuller2015robust, candes2011robust, lin2011linearized, 5394889, zhou2011godec, JMLR:v24:21-1130, NIPS2014_443cb001, yu2017compressing, li2023losparse}. Specializing to LLMs,~\citet{zhang2024oats} propose OATS 
%that addresses this type of %compression and 
that outperforms pruning methods for the same compression ratio (number of non-zero elements) on a wide range of LLM evaluation benchmarks (e.g. perplexity in Language generation). 

OATS \cite{zhang2024oats} is however a matrix decomposition algorithm inspired from a pruning algorithm Wanda \cite{sun2023simple}. Wanda has been designed as a relaxation/approximation of another state-of-the-art pruning algorithm SparseGPT \cite{frantar2023sparsegpt}. While Wanda has been found to be extremely useful and efficient in practice, recent work \cite{meng2024alps} show results where Wanda fails for high-sparsity regimes. In this paper, we provide a unified optimization framework to decompose pre-trained model weights into sparse plus low-rank components based on a layer-wise loss function. Our framework is modular and can incorporate different pruning and matrix-decomposition algorithms (developed independently in different contexts).
%under the umbrella of the local %layer-wise reconstruction error; 
Similar to~\cite{meng2024alps} we observe that our optimization-based framework results in models with better model utility-compression tradeoffs. The difference is particularly pronounced for higher compression regimes. 
%especially for higher compression %budgets, where SOTA methods 
% Our numerical results also show similar findings to \citet{meng2024alps} where high-sparsity significantly degrades the performance of approximation-based optimization methods like OATS.

Concurrently, in a different and complementary line of work,~\citet{mozaffari2024slope} have open-sourced highly-specialized CUDA kernels designed for N:M sparse \cite{zhou2021learning} plus low-rank matrix decompositions that result in significant acceleration and memory reduction for the pre-training of LLMs.
We note that our focus here is on improved algorithms for one-shot sparse plus low-rank matrix decompositions for foundation models with billions of parameters which is different from the work of \citet{mozaffari2024slope} that focuses on accelerating the pre-training of LLMs. The designed CUDA kernels \cite{mozaffari2024slope} can be exploited in our setting for faster acceleration and reduced memory footprint during inference.





% \textbf{Summary of approach and contributions:} We propose \ourmethod: an accurate and scalable framework for Sparse plus Low-Rank Matrix Decomposition for LLMs. Following the previous work on one-shot pruning and model compression, we pursue a layerwise approach. In particular, the reconstruction error resulting from compression in the output of each layer is minimized, under the compression constraints (i.e., sparsity and low-rank constraints).

\textbf{Summary of approach.\,\,\,\,} Our framework is coined \ourframework: \underline{H}ardware-\underline{A}ware (Semi-\underline{S}tructured) \underline{S}parse plus \underline{L}ow-rank \underline{E}fficient \& approximation-\underline{free} matrix decomposition for foundation models.

Hardware-aware refers to the fact that we mostly focus on a N:M sparse \cite{zhou2021learning} plus low-rank decomposition, for which acceleration on GPUs is possible, although \ourframework supports any type of sparsity pattern (unstructured, semi-structured, structured) in the sparsity constraint. Approximation-free refers to the fact that we directly minimize the local layer-wise reconstruction error introduced in \cref{eq:matrix-decomposition}, whereas we show prior work minimizes an approximation of this objective.

%Our unified framework introduces a well-posed 
%%As a part of our proposed framework, we consider an 
%%optimization form
We formulate the compression/decomposition task as a clear optimization problem; we minimize a local layer-wise reconstruction objective where the weights are given by the sum of a sparse and low-rank component. 
%%%of dense model weights under the  
%This optimization problem is decoupled into a sparse minimization subproblem and a low-rank minimization subproblem. 
We propose an efficient Alternating-Minimization approach that scales to models with billions of parameters relying on 
two key components: one involving sparse minimization (weight sparsity) and the other involving a low-rank optimization. 
For each of these subproblems 
we discuss how approximations to the optimization task can retrieve prior algorithms.
%the introduced subproblems, 
%we consider approximations to the minimization objective and retrieve different algorithms from related works given different %approximations.

% We provide an efficient and scalable algorithm based on Alternating-Minimization that does not rely on any approximation at the objective minimization level. 
% While \ourframework supports any sparsity pattern (unstructured, semi-structured, structured) in the sparsity constraint, we mostly focus on N:M sparsity \cite{zhou2021learning}, to make the decomposition Hardware-aware, as \citet{mozaffari2024slope} show how to get acceleration on modern GPUs for N:M sparse plus low-rank decomposition.

We note that \ourframework~differs from prior one-shot (sparse) pruning methods~\cite{frantar2023sparsegpt, meng2024alps, benbaki2023fast} as we seek a sparse plus low-rank decompositon of weights.
%%%%%introducing the low-rank component. 
Additionally, it differs from prior one-shot sparse plus low-rank matrix decomposition methods~\cite{zhang2024oats}
%by considering an approximation-free minimization approach of the 
as we directly minimize the local layer-wise reconstruction objective introduced in \cref{eq:matrix-decomposition}.

Our main \textbf{contributions} can be summarized as follows.
\begin{compactitem}
    \item We introduce \ourframework a unified one-shot LLM compression framework that scales to models with billions of parameters where we directly minimize the local layer-wise reconstruction error subject to  a sparse plus low-rank matrix decomposition of the pre-trained dense weights. 
    %    formulates a sparse plus low-rank matrix decomposition as an optimization problem with a local layer-wise reconstruction objective. We discuss approximations of this objective and show that OATS a popular method is recovered in a particular approximation.

    
    \item \ourframework uses an Alternating-Minimization approach that iteratively minimizes a Sparse and a Low-Rank component. \ourframework uses a given pruning method as a plug-in for the subproblem pertaining to the sparse component. Additionally, it uses Gradient-Descent type methods for the subproblem pertaining to the Low-Rank component.
    
    % \item In the subproblem pertaining to the sparse component, a rewrite of the optimization formulation shows that one can use any pruning algorithm, that minimizes the layer-wise reconstruction error, as a plug-in to sparsify the weights. We choose to show results for the algorithm SparseGPT.
    
    % In this pruning subproblem, we also enhance the performance of \ourmethod by exploiting the invariance of the Hessian--of the layer-wise reconstruction error--in each subproblem of the Alternating Minimization procedure, for a given layer. In particular, we use a pre-processing step that computes and stores the Hessian inverse--of the objective--, which is then passed to the deployed pruning algorithm (e.g. SparseGPT). 
    % \item In the subproblem  pertaining to the Low-Rank component, we give a theoretical closed form solution to the subproblem.
    % which does not scale to problems with billions of parameters. 
    % We also present a more tractable first-order optimization method for a reparametrization of the the low-rank problem, which is scalable to models with billions of parameters.
    
    % as $\bfUVt$ and use first-order optimization methods to minimize the layer-wise reconstruction objective.

    \item We discuss how special cases of our framework relying on specific approximations of the objective retrieve popular methods such as OATS, Wanda and MP --- \cite{zhang2024oats, sun2023simple,han2015learning, sze2020efficient}. This provides valuable insights into the underlying connections across different methods. 

    \item \ourframework improves upon state-of-the-art methods for one-shot sparse plus low-rank matrix decomposition. 
    For the Llama3-8B model with a 2:4 sparsity component plus a 64-rank component decomposition, \ourframework reduces the test perplexity by $12\%$ for the WikiText-2 dataset and reduces the gap (compared to the dense model) of the average of eight popular zero-shot tasks by $15\%$ compared to existing methods.
\end{compactitem}




\section{Influence of Modality in Persona Embodiment of Multimodal LLMs}\label{sec:problem}

% In this work, we study the ability of an LLM to embody a persona as represented in different modalities. 

The problem of embodying a persona can be defined as the task of generating responses consistent with a specified character, identity, or role~\citep{samuel2024personagym}. This involves maintaining coherence in linguistic style, beliefs, knowledge, and affective tone in a way that aligns with the intended persona.

In this work, we investigate the effect of representing the persona $p$ in different modalities, denoted as $\CR(p)$, on multimodal LLMs. In particular, we consider two common modalities, text and image, and evaluate the LLM's performance on these equivalent representations. Additionally, we also consider combining visual and textual features of the personas. We describe these $4$ different representations $\CR(p)$ in more detail below. 
% \ks{add an example for each maybe just in figure 1 -- you can make it double-column as well. maybe not super important but the pipeline figure is.}\jb{added to figure 2}\ks{awesome!}


\subsection{Persona Modality Representations}
\begin{itemize}
    \item \textbf{Text ($\CT$):} Textual descriptions of a persona correspond to a sequence of sentences characterizing the persona in natural language.

    \item \textbf{Image ($\CI$):} A persona can also be depicted visually using an image of the person in a representative environment that characterizes the persona visually.
    
    \item \textbf{Assisted Image ($\CI_{A}$):} Since certain features may be obscured in the image, textual attributes of the persona can also be included explicitly as text. 
    
    \item \textbf{Descriptive Image ($\CI_{D}$):} In this case, we include the textual attributes in the image itself using typography instead of in the text.
    
\end{itemize}

% \section{NumericBench Generation}
In this section, we present our created  NumericBench, which is specifically designed to evaluate fundamental numerical capabilities of LLMs. 
NumericBench consists of diverse datasets and tasks, 
enabling a systematic and comprehensive evaluation.
We discuss the datasets included in NumericBench, the key abilities it evaluates, and the methodology for benchmark generation.

\begin{table*}[t]
	\caption{NumericBench statistics. R: contextual retrieval, C: comparison, S: summary, L: logical reasoning. The token count is calculated based on tiktoken, which is the tokenizer used by Llama3~\cite{grattafiori2024llama3herdmodels}. The sentences used for token calculation include both the context and the question.}
	\centering
	\renewcommand{\arraystretch}{1.15} % 设置行间距为默认的 1.15 倍
	\setlength{\tabcolsep}{1.5pt} % 将列间距设置为 1pt
\resizebox{\textwidth}{!}{
	\begin{tabular}{c|c|c|c|c}
		\toprule
		\textbf{Data} & \textbf{Format} & \textbf{Questions} & \textbf{\# Instance} & \textbf{Avg Token} \\ \midrule
		
		\multirow{3}{*}{} 
		& \multirow{3}{*}{} 
		& \begin{tabular}[c]{@{}c@{}}R: What is the index of the first occurrence\\ of the number -3095 in the list?\end{tabular} 
		& 1000 & 3704.23 \\ \cline{3-5}
		
		\textbf{\begin{tabular}[c]{@{}c@{}}Number\\ List\end{tabular}}
		& $[69, -1, 6.1, \ldots, 5.7]$
		& \begin{tabular}[c]{@{}c@{}}C: Which index holds the smallest number\\
			 in the list between the indices 20 and 80?\end{tabular} 
		& 1000 & 3685.57  \\ \cline{3-5}
		
		& & \begin{tabular}[c]{@{}c@{}}S: What is the average of the index of\\
			 top 30 largest numbers in the list?\end{tabular} 
		& 1000 & 3654.78 \\ \midrule
		
		\multirow{3}{*}{} 
		& \multirow{3}{*}{
		\begin{tabular}[c]{@{}c@{}}
			\{date: 2024-06-19,\\
			close\_price: 9.79, \\
			open\_price: 9.4, \\
			\ldots \\
			PE\_ratio: 4.5416\}
		\end{tabular}
		} 
		& \begin{tabular}[c]{@{}c@{}}
			R: On which date did the close price\\
			 of stock firstly reach 61.76 yuan?
		\end{tabular}
		& 1000 & 27585.35 \\ \cline{3-5}
		
		\textbf{Stock}
		& 
		& \begin{tabular}[c]{@{}c@{}}
			C: Among the top-45 trading value days, which\\
			 date did the stock have the lowest close price?
		\end{tabular}
		 & 1000 & 27595.40 \\ \cline{3-5}
		
		& & \begin{tabular}[c]{@{}c@{}} 
			S: How many days had the close price higher than\\
			 the open price from 2024-07-31 to 2024-12-13?
		\end{tabular}	
		& 1000 & 27561.29 \\ \midrule
		
		\multirow{3}{*}{} 
		& \multirow{3}{*}{
		\begin{tabular}[c]{@{}c@{}}
			\{date: 2024-07-21,\\
			pressure\_msl: 999.96,\\
			dew\_point\_2m: 26.25,\\
			\ldots \\
			cloud\_cover: 61.5\}
		\end{tabular}
		} 
		& \begin{tabular}[c]{@{}c@{}} 
			R: On which date did the dew point temperature\\
			 at two meters firstly drop below 9.15°C?
		\end{tabular}
		& 1000 & 27359.26 \\ \cline{3-5}
		
		\textbf{Weather}
		& & \begin{tabular}[c]{@{}c@{}} 
			C: On which date did the MSL pressure reach its\\
			highest value when the cloud cover was below 9\%?
		\end{tabular}
		& 1000 & 27368.19 \\ \cline{3-5}
		
		& & \begin{tabular}[c]{@{}c@{}} 
			S: What was the average temperature at two meters\\
			when the relative humidity exceeded 78.56\%?
		\end{tabular}
		& 1000 & 27331.21 \\ \midrule
		
		\textbf{Sequence} 
		& $[0.34, 3, 6, \ldots, 111]$ 
		& L: What is the next number in the sequence? & 500 & 677.57 \\ \midrule
		
		\textbf{\begin{tabular}[c]{@{}c@{}}Arithmetic \\Operation\end{tabular}} 
		& \begin{tabular}[c]{@{}c@{}} 
		$a: 6.755,
		b: -1.225$
		\end{tabular}
		& \begin{tabular}[c]{@{}c@{}} 
		 $Q_{oper}$: What is the result of $a + b$?\\
		 $Q_{context}$: What is the result of $a $ plus $b$?
		 
		\end{tabular}
		& 12000 & 112.00 \\ \midrule
		
		\textbf{\begin{tabular}[c]{@{}c@{}}Mixed-number-string\\ Sequence\end{tabular}} 
		& \begin{tabular}[c]{@{}c@{}} 
		$effV2\ldots x98o7Lo$
		\end{tabular}
		& \begin{tabular}[c]{@{}c@{}} 
		How many numbers are there in the string? Note\\
		that a sequence like 'a243b' counts as a single number.
		\end{tabular}
		& 2000 & 196.53 \\ \bottomrule

	\end{tabular}
}
	\label{tab:data_stat}
	
\end{table*}

 

\subsection{Numeric Dataset Collection}
NumericBench offers a diverse collection of numerical datasets and questions designed to reflect real-world scenarios. 
This variety ensures that LLMs are thoroughly tested on their fundamental  abilities on numerical data.

\noindent\textbf{Number List Dataset.}
The synthetic number list dataset consists of simple collections of numerical values (integer and floats) 
presented as ordered or unordered lists.
Numbers in lists are one of the simplest and most fundamental data representations encountered in real-world scenarios.
Despite their simplicity, retrieving, indexing,  comparison, and summary on numbers can verify the fundamental numerical ability of LLMs. 
This dataset serves as a fundamental dataset of how well LLMs understand numerical values as discrete entities.



\noindent\textbf{Stock Dataset.}
The time-series  stock dataset is crawled from Eastmoney website~\cite{eastmoney}, 
which has eighteen attributes, such as stock close prices, open price,  trading volumes, and price-earnings ratio, over time.
Stock  data reflects dynamic, real-world numerical reasoning challenges that involve trend analysis, comparison, and decision-making under uncertainty,  representing real-world financial workflows.
 




\noindent\textbf{Weather Dataset.}
The weather dataset is crawled from Open-Meteo python API~\citep{openmeteo}, which includes data related to weather metrics, such as temperature, precipitation, humidity, and wind speed. 
The data is presented across various longitude and latitude.
 
 




\noindent\textbf{Numeric Sequence  Dataset.}
The synthetic numeric sequence dataset comprises sequences of numbers generated by arithmetic or geometric progression, complex patterns, or noisy inputs. 
Tasks require identifying patterns, predicting the next number, or reasoning about relationships between numbers.
Numerical sequences test the logical reasoning capabilities of LLMs, requiring pattern recognition and multi-step reasoning. This dataset introduces structured challenges that are common in mathematics and algorithmic reasoning.


 
\noindent\textbf{Arithmetic Operation Dataset.}
The dataset comprises 12,000 pairs of simple numbers, each undergoing addition, subtraction, multiplication, and division operations. Each pair of numbers, $a$ and $b$, consists of $k$-digit integers with three decimal places, where $k \in \{1, 2, \cdots, 6\}$. 
For each value of $k$, there are 2,000 pairs, evenly distributed across the four basic operations (i.e, $+, -,  *, /$), with 500 pairs per operation. 
This dataset is to evaluate the fundamental mathematical operation capabilities of LLMs, simulating the majority of mathematical calculation requirements in real-world scenarios.

\noindent\textbf{Mixed-number-string Sequence Dataset.}
The dataset consists of alphanumeric strings of varying lengths $\{50, 100, 150, 200\}$, each containing a randomized mix of letters and digits. For each string length, 500 samples are generated, resulting in a total of 2,000 samples. Each sample includes a query asking for the count of contiguous numeric sequences within the string, where sequences like "a243b" count as a single number. This dataset is designed to assess the ability of LLMs to identify and count numeric sequences.
 







\subsection{Fundamental Numerical Ability}
NumericBench is designed to comprehensively evaluate six fundamental numerical reasoning abilities of LLMs, which is 
%These three fundamental abilities are 
essential for solving real-world numeric-related tasks.
%such as numeric data summary and financial price analysis.


\noindent\textbf{Contextual Retrieval Ability.}
Contextual retrieval ability evaluates how well LLMs can locate, extract, and identify specific numerical values or their positions within structured or unstructured data. 
This includes tasks like finding a specific number in a list, retrieving values , and indexing numbers based on their order.
For example, as shown in Table~\ref{tab:data_stat}, it evaluates LLMs on tasks such as retrieving stock prices and identifying key values within numerical lists or domain-specific data (e.g., stock market and weather-related information).
This ability is fundamental to numerical reasoning because it forms the foundation for higher-order tasks, such as comparison, aggregation, and logical reasoning. 
 
 



\noindent\textbf{Comparison Ability.}
Comparison ability evaluates how well LLMs can compare numerical values to determine relationships such as greater than, less than, or equal to, and identify trends or differences in datasets. 
Comparison is vital for logical reasoning and decision-making, as many real-world tasks depend on accurate numerical evaluation. 
For instance,  as shown in Table~\ref{tab:data_stat},   comparing prices is essential in stock  for assessing performance, while weather forecasting requires analysis of temperature or precipitation trends over time. 
 



\noindent\textbf{Summary Ability.}
Summary ability assesses the LLM’s capacity to aggregate numerical data into concise insights, such as calculating totals, averages, or other statistical metrics. 
Summarization is critical for condensing large datasets into actionable information, enabling decision-making based on aggregated insights rather than raw data. 
This ability is indispensable in domains like electricity usage analysis, where summarizing hourly or daily consumption helps forecast bills, in business reporting for aggregating sales and revenue data to evaluate performance, 
and in healthcare analytics to monitor trends in patient metrics over time.



\noindent\textbf{Logic Reasoning Ability.}
Logical Reasoning Ability measures the LLM’s ability to perform multi-step operations involving numerical data, 
such as recognizing patterns, inferring rules, and applying arithmetic or geometric reasoning to solve complex problems. Logical reasoning extends beyond simple numerical tasks and reflects the LLM’s capacity for deeper, structured thinking. 
This ability is crucial for algorithm design, where solving problems involving numeric sequences or patterns is essential, in scientific research for identifying relationships and correlations in data.

\noindent\textbf{Arithmetic Operation Ability.}
It reflects the LLM's capacity to perform mathematical calculations accurately. Such ability is essential for tasks involving numerical computations, such as  automated machine learning through LLMs.





\noindent\textbf{Number Recognition  Ability.}
This measures the LLM's proficiency in identifying and interpreting numerical information within a given context. It represents a fundamental requirement for handling numeric-based tasks effectively.




\subsection{NumericBench Generation}
We use the number list, stock, and weather datasets to evaluate the contextual retrieval, comparison, and summary abilities of LLMs. 
Specifically, for each ability and each dataset, we prepare a set of questions designed to assess the corresponding target ability.
As shown in Table~\ref{appx:number_question}, Table~\ref{appx:stock_question}, and Table~\ref{appx:weather_question} in Appendix, there are nine question sets in total, covering three abilities across three datasets. 
When evaluating a specific ability (e.g., contextual retrieval) on a specific dataset (e.g., stock data), we randomly select one question from the corresponding question set for each data instance (e.g., a stock instance) 
and manually label the answer. This approach enables us to generate question-answer pairs for each ability on the number list, stock, and weather datasets.

For arithmetic operations and number counting in the strings dataset, the question format is straightforward, as illustrated in Table~\ref{tab:data_stat}. These questions are designed to evaluate the basic arithmetic operation and number recognition abilities of LLMs.



\section{Modality-Parallel Persona Dataset}\label{sec:dataset}

\subsection{Personas}


% 18 to 24 · 25 to 34 · 35 to 44 · 45 to 54 · 55 to 64 · 65 or over
We introduce a novel dataset of personas $\CP = \{p_i\}$, such that each persona $p$ can be represented equivalently in four modalities $\CI, \CT, \CI_A, \CI_D$. To ensure effective representation across both text and image modalities, we construct personas based on key demographic attributes that are easily visualizable~\citep{todorov2015social}. Specifically, each persona is defined by a unique combination of age, gender, occupation, and location. A persona can thus be written as:
\begin{center}
    {\small
    A \textcolor{blue}{\texttt{<age>}}-year-old \textcolor{teal}{\texttt{<gender>}} \textcolor{brown}{\texttt{<occupation>}} \\ from \textcolor{pink}{\texttt{<location>}},}
\end{center}
where \textcolor{blue}{\texttt{<age>}} $\in [18, 64]$, \textcolor{teal}{\texttt{<gender>}} $\in \{$ male, female $\}$, \textcolor{pink}{\texttt{<location>}} is a city, and \textcolor{brown}{\texttt{<occupation>}} denotes a person who does a specific occupation. For example, ``A \textcolor{blue}{35}-year-old \textcolor{teal}{male} \textcolor{brown}{chef} from \textcolor{pink}{Paris}''. As depicted in Figure~\ref{fig:intro}, age and gender can be visualized using the face of the person while occupation and location can be visualized using the clothes and the background respectively.

\begin{figure}[t]
    \centering
    \includegraphics[width=\linewidth]{pipeline.pdf}
    \caption{Our pipeline begins with curating a set of personas. Each persona receives a detailed text description, which is then fed into Stable Diffusion to generate $\mathcal{I}$. A separate model examines the image and generates an independent textual description, forming text persona $\mathcal{T}$. Pairing $p$ with $\mathcal{I}$ produces an assisted image $\mathcal{I_A}$, while combining a typographic representation of $p$ with $\mathcal{I}$ produces a descriptive image $\mathcal{I_D}$.}
    \label{fig:pipeline}
\end{figure}

\begin{table}[t]
    \centering
    \caption{\textbf{Persona Dataset Summary}}
    \label{tab:data_summary}
    \resizebox{1.0\linewidth}{!}{
    \begin{tabular}{c | c | c}
        \toprule
        Attribute & Category & Number \\
        \midrule
        \multirow{5}{*}{Age} & 18-24 & 5 \\
        & 25-34 & 11 \\
        & 35-44 & 13 \\
        & 45-54 & 6 \\ 
        & 55-64 & 5 \\
        \hline
        \multirow{2}{*}{Gender} & Male & 19 \\
        & Female & 21 \\
        \hline
        \multirow{5}{*}{Occupation} & Healthcare \& Education & 9 \\
        & Public Safety &	5 \\
        & Manual Labor & 16 \\
        & Hospitality & 5 \\
        & Transportation 	& 5 \\
        \hline 
        \multirow{4}{*}{Location} & Largest Economies (GDP > \$3T) &	12 \\
        & Developed Economies (GDP \$1T-\$3T) &	13 \\
        & Mid-Sized Powers (GDP \$0.5T-\$1T) & 7 \\
        & Emerging Markets (GDP < \$0.5T) & 8 \\
        \bottomrule
    \end{tabular}}
\end{table}


To promote diversity, we systematically categorize these attributes into distinct groups and uniformly sample from each category. Table~\ref{tab:data_summary} summarizes our dataset of how we choose the age, gender, occupation, and location. In particular, we consider a standard grouping of ages followed in surveys between 18 and 65, a standard male/female splitting of gender, while occupations and locations are categorized based on their primary societal role and economic status~\footnote{\href{https://data.worldbank.org/indicator/NY.GDP.MKTP.CD}{GDP}} respectively.
% \jb{is there a cite for this?}
Table~\ref{tab:personalist} in Appendix presents the list of $40$ personas we use along with their attributes and attribute categories. 
% \ks{need to write more}

% select the values of these attributes from a large set of $47$ ages $2$ genders, $35$ occupations, and $37$ different locations, for a grand total of $121,730$ distinct personas. 
% Then, we downsample this set by categorizing the age group select a representative subset of $n=40$ personas that preserves the occupational diversity of the full dataset while maintaining demographic balance
% Our evaluation dataset consists of personas generated through combinations of demographic attributes. Each persona $p_i$ is characterized by attribute tuple $(a_i, g_i, o_i, l_i)$ representing:

% $$
% \begin{aligned}
% \textbf{Age } (a_i) &: [18,65] \in \mathbb{N}\\
% \textbf{Gender } (g_i) &: \mathcal{G}, |\mathcal{G}| = 2 \\
% \textbf{Occupation } (o_i) &: \mathcal{O}, |\mathcal{O}| = 35\\
% \textbf{Location } (l_i) &: \mathcal{L}, |\mathcal{L}| = 37
% \end{aligned}
% $$


% \ks{can we reduce it to age into a few age brackets and occupation into a few occupation sectors and cities to be locations? Try to reduce them all to single-digit numbers.}

% \jb{i used the specific numbers to show all the possible unique personas and were used as-is in the experiments; there weren't any distinct patterns across the categories, how would I determine what brackets/sectors to use?}\ks{see above}

% For example: "{\texttt{A \textbf{52-year-old} \textbf{male} \textbf{doctor} from \textbf{Madrid}}}".
% \vspace{0.5em}

% By combining these attributes, we can generate . Within this subset, we ensure diversity by preventing any overlap of both occupation and location between personas. The complete list of attributes is provided in Appendix \ref{TODO}.

\subsection{Equivalent Modality Representations}

From above, we have a diverse set of textual persona descriptions as described by the four demographic attributes. Next, we construct a modality-parallel dataset, we require that each persona $p$ can be equivalently depicted in $4$ representations $\CR(p)$: image $\CI(p)$, text $\CT(p)$, assisted image $\CI_A(p)$, and descriptive image $\CI_D(p)$. Figure~\ref{fig:pipeline} illustrates the step-by-step procedure to obtain these modality representations for a persona description $\CP$.

\begin{enumerate}[leftmargin=*, noitemsep]
    \item We first convert the persona description made from the four attributes into a more detailed visual description using an LLM\footnote{\href{https://openai.com/index/hello-gpt-4o/}{gpt-4o-mini-2024-07-18}} and the following prompt:
    \begin{quote}
    \small
    \texttt{Create a short, descriptive persona for the person in the image. Describe them using only the following details: their age, gender, facial expression or mood, attire, any tools or items they’re holding, their work environment, the nature of their job, and their connection to the area and location. Avoid taking creative liberties beyond these details, only using details that can be inferred from the image, while aiming for a realistic portrayal that gives insight into their daily life, professional dedication, and overall demeanor. For example: Meet a skilled construction worker in his late 30s, living in Sydney, Australia. Every day, he heads out to work in one of the city's bustling urban sites, often with a view of iconic landmarks like the Sydney Opera House and Sydney Harbour Bridge. Outfitted in essential safety gear—a hard hat, reflective vest, and a set of versatile tools—he’s well-prepared for a physically demanding role that demands focus and precision. His job involves a blend of construction and maintenance tasks, requiring him to pay close attention to safety protocols and collaborate with a team. Confident and professional in his work, he takes pride in contributing to the infrastructure and vibrant aesthetic of Sydney, adding to the city’s ever-evolving landscape with each project.}
    \end{quote}
    \item Next, we use a text-to-image generative model, particularly,  Stable Diffusion XL\footnote{\href{https://huggingface.co/stabilityai/stable-diffusion-xl-base-1.0}{stabilityai/stable-diffusion-xl-base-1.0}} to generate a $768 \times 768$ px image conditioned on the more complete description of the persona found above. Upon doing an extensive hyperparameter search, we found the best results with a guidance scale of $\gamma = 15$ and $n = 50$ diffusion steps. Thus, we obtain the image $\CI$.
    \item Since the generated image can contain extra information due to underspecified textual prompts, we prompt the LLM one more time to generate a complete description of the persona as described in the image using a detailed prompt as given in Appendix~\label{app:img_to_text_prompt}. Thus, we obtain the text $\CT$.
\end{enumerate}

These steps enable us to convert a dataset of persona descriptions $\{p\} \rightarrow \{(\CT(p), \CI(p))\}$ such that $\CT (p) \leftrightarrow \CI (p)$ are equivalent to each other. One can now also obtain the assisted and descriptive image representations of the persona by pairing the image $\CI$ with the text persona $p$ for the assisted image $\CI_A$, and by rendering\footnote{\href{https://pillow.readthedocs.io/en/stable/}{Pillow}} $p$ as black text at the bottom of the image on a white background using Arial font at size 20 for $\CI_D$.

\subsection{Question Generation}
To evaluate how well a model embodies a given persona, we create a set of $60$ questions that specifically probe for a given attribute either directly or in naturalistic scenarios. In particular, we create $10$ questions per attribute for the two sets. Gender was excluded from our evaluation question set due to methodological constraints. While age and location can be objectively probed through factual knowledge, gender assessment would inevitably rely on stereotypes or normative expectations. Moreover, there may be a high possibility of refusal from the LLMs due to their safety training. Thus, we obtain two question sets $Q^D$ and $Q^S$ for \textcolor{pink}{L}: location, \textcolor{brown}{O}: occupation, and \textcolor{blue}{A}: age. 
% Each persona is evaluated across three categories ($L$: location, $O$: occupation, $A$: age) using two evaluation sets:
$$
\begin{aligned}
Q^D &= \bigcup_{i \in \{\textcolor{pink}{L},\textcolor{brown}{O},\textcolor{blue}{A}\}} Q^D_i,\text{ each } |Q^D_i| = 10 \text{ questions}\\
Q^S &= \bigcup_{i \in \{\textcolor{pink}{L},\textcolor{brown}{O},\textcolor{blue}{A}\}} Q^S_i,\text{ each } |Q^S_i| = 10 \text{ scenarios}
\end{aligned}
$$
% \ks{why did we not have gender-related questions?}

% \jb{i think because evaluating correct answers would've just relied on stereotypes; i tried to make questions that could be more objectively evaluated (i.e. this is clearly not have a doctor would respond, or those schools are clearly not near that location/city). i could probably not mention gender at all in the paper since we don't use it though}\ks{this is good, let's say this then.}

% We curated two types of prompts to evaluate persona alignment: questions and scenarios. 

\subsubsection{Direct Testing}
Questions were designed to probe specific knowledge across age, location, and occupation categories while enabling objective evaluation. For example, location questions assess knowledge of local customs and landmarks, while occupation questions may test domain expertise. For example, for age, we ask ``\textit{what life experiences do you consider most defining for your generation?}'' while for location, we have ``\textit{what is the most visited tourist attraction in your area?}''. We provide the complete list in Table~\ref{tab:direct_questions_list} in Appendix. 

% \ks{write more detail}\jb{what else should I put here? i have what criteria are used in the next section. should i give an example of a scenario here (like reference a figure)?}\ks{examples, and appendix reference.}

\subsubsection{Situational Testing}
Scenarios accomplish similar knowledge evaluation but through naturalistic situations, requiring personas to implicitly demonstrate both knowledge and behavioral consistency. For example, for age scenarios, we ask ``\textit{You’re coordinating a playlist for your high school reunion after-party. The organizers want music specifically from your graduating years to recreate the atmosphere. You $\dots$}'', which is detailed in Figure \ref{fig:intro}. A complete list is provided in Table \ref{tab:direct_scenarios_list} in Appendix.
% \ks{write more detail}\jb{what else to put here?} \ks{examples and appendix reference.}


\subsection{Evaluation}

For each persona $p \in \CP$ and question $q \in \CQ^D \cup \CQ^S$, we find the response answer $a \gets \CM(\CR(p), q)$ from a multimodal LLM $\CM$, where $\CR(p)$ denotes a modality representation of the persona $\CP$. Thus, we obtain $(q, \CR(p)) \rightarrow_{\CM} (p, q, a)$. We now evaluate the quality of the response $a$ based on the question asked $q$ and the persona description $p$. 

\subsubsection{LLM-based evaluation}
% We define the subject model $\mathcal{M}: P \times \Pi \rightarrow R$ where $P$ is the set of personas, $\Pi = Q \cup S$ is the set of all prompts (questions and scenarios), and $R$ is the space of possible responses. For each response $r = \mathcal{M}(p,\pi)$, we employ an evaluator $\mathcal{J}$ that assigns scores based on different evaluation criteria taken from \citet{samuel2024personagym}. The evaluator uses the following four task descriptions:

Following \citet{samuel2024personagym}, we employ an LLM-based evaluator to judge the quality of the responses based on different metrics defined in the prompt. In particular, we prompt the LLM judge $\CJ$ with the question asked $q$, response $a$, and the persona description $p$ on these metrics as described by the corresponding prompts.

\begin{quote}
\small
\texttt{\textbf{Persona Consistency:} Evaluate the consistency of the response with the described persona. Ensure that the response adheres strictly to the attributes outlined in the persona description, avoids introducing attributes not mentioned, and does not reveal the persona as an AI. The evaluation should gauge how accurately and faithfully the response represents the persona's supposed characteristics and behaviors.}

\texttt{\textbf{Linguistic Habits:} The evaluation task of "linguistic habit" assesses the persona's adherence to its characteristically unique syntax, tone, and lingo, ensuring that these elements are consistently utilized throughout the persona's dialogue. This includes avoiding generic language patterns (such as "As a [persona]") and integrating specific idiomatic expressions, colloquialisms, or jargon that define the persona's distinctive verbal identity. The aim is to evaluate how effectively the persona maintains its linguistic uniqueness in various contexts.}

\texttt{\textbf{Action Justification:} Evaluate the persona's response to determine how effectively and convincingly it justifies a given action based on its described attributes and situation. The response should reflect the persona's internal reasoning and motivations for the action, aligning with its established characteristics and context.}

\texttt{\textbf{Expected Action:} The persona takes actions within its response to the question that is logically expected of the persona in the setting of the question.
}
\end{quote}

% \paragraph{Likert Scale} 
For each evaluation criterion, $\mathcal{J}(p,q,a)$ outputs a score from a 5-point Likert scale based on the corresponding system prompt. For situational testing, we evaluate using \texttt{action justification}, \texttt{expected action}, \texttt{linguistic habits} while for direct testing, we use \texttt{persona consistency} and \texttt{linguistic habits}. Note that we combine the scores for linguistic habits across the two testing sets to find the average score. 

% \paragraph{Binary Success Metrics}
% We dichotomize evaluator scores using threshold $\tau = 3$ to distinguish between successful responses ($\geq 3$) and unsuccessful responses ($< 3$) for each criterion. For evaluation set $D$, we define the \textbf{pass rate} (PR) as: \(\text{PR} = \tfrac{1}{|D|} \sum_{i=1}^{|D|} \mathbb{I}(\mathcal{J}(p_i, q_i, a_i) \geq \tau)\).

\subsubsection{Comparative Evaluation}
We employ two comparative evaluation methods to assess relative performance across modalities, using evaluator $\mathcal{J}$ with the prompt:
\begin{quote}
    \small
    \texttt{You are given a persona description and multiple responses to a prompt.\\ %\\[1ex]
        Persona Description: <\textit{p}>\\
        Prompt: <q>\\
        Candidate Responses: <responses>\\ %$[1ex]
        Choose the single response that best fits the persona's style, values, and consistency. Respond with 'Response X' where X is the number of the chosen response.}
\end{quote}

\paragraph{Pairwise Comparison}
To compare responses across the text and image modalities, we first directly compare responses $a_{\mathcal{T}}$ and $a_{\mathcal{I}}$.

\paragraph{Swiss System Comparison}
To collectively evaluate all four modalities, we adopt the Swiss tournament system, which reduces the number of required comparisons compared to pairwise evaluation while maintaining ranking quality. Specifically, for \( n = 4 \), pairwise evaluation requires \( \binom{4}{2} = 6 \) comparisons, whereas the Swiss system reduces to 3 comparisons.

\subsubsection{Linguistic Analysis}
Alongside the \textit{linguistic habits} evaluation criterion, we also analyze the lexical diversity, variation, and complexity of each response using established metrics from computational linguistics:
\begin{itemize}[leftmargin=*,noitemsep]    
    \item \textbf{Types:} $|\{r\}|$, unique token count.  %\vspace{0.3em}
    \item \textbf{Root Type-Token Ratio (RTTR):} \(= \nicefrac{\text{types}}{\sqrt{\text{length}}}\)
    , a normalized measure of lexical variation found by dividing the number of unique tokens with the response length~\citep{lexicalrichnesshout}. %\vspace{0.3em}
    \item \textbf{Measure of Textual Lexical Diversity (MTLD):} Following \citet{mccarthy2010mtld}, calculates the mean length of text segments that maintain a type-token ratio (TTR) $> \tau = 0.72$.
\end{itemize}


\section{Experimental Setup}\label{sec:setup}

\paragraph{Models.} We evaluate the performance of 5 multimodal large language models: (1) GPT-4o\footnote{\href{https://openai.com/index/hello-gpt-4o/}{GPT-4o}}, (2) GPT-4o mini\footnote{\href{https://openai.com/index/gpt-4o-mini-advancing-cost-efficient-intelligence/}{GPT-4o mini}}, (3) Llama 3.2 11B\footnote{\href{https://huggingface.co/meta-llama/Llama-3.2-11B-Vision}{Llama 3.2 11B}}, (4) Llama 3.2 90B\footnote{\href{https://huggingface.co/meta-llama/Llama-3.2-90B-Vision}{Llama 3.2 90B}}, and (5) Pixtral 12B\footnote{\href{https://huggingface.co/mistralai/Pixtral-12B-2409}{Pixtral 12B}} \citep{agrawal2024pixtral12b}.

\paragraph{Evaluators.} We utilize two LLM evaluators, using GPT-4o\footnotemark[3] and Gemini 2.0 Flash\footnote{\href{https://deepmind.google/technologies/gemini/flash/}{Gemini 2.0 Flash}}, with deterministic sampling with zero temperature and top P values. All scores discussed in the main paper are averaged across the two models, while individual scores can be found in Tables \ref{tab:eval-table-gpt-4o} and \ref{tab:eval-table-gemini-flash} in the Appendix. We use human evaluators on a large subset of the evaluation set to assess the LLM evaluator scores' alignment with human scores. For further details, refer to Appendix \ref{app:human}. 
% \ks{what detail?} \jb{screenshots/details about the survey and who the evaluators were (broadly)}
\section{Experimental Results}
\label{sec:results}

\begin{figure*}[!ht]
    \centering
    \begin{subfigure}{0.02\textwidth}
    \raisebox{0.1\height}{\includegraphics[angle=90,width=\textwidth]{fig-colorBar}}
    \end{subfigure}
    \begin{subfigure}{0.92\textwidth}
    \includegraphics[width=\textwidth]{fig-vr_table_smaller.png}
    \end{subfigure}
    \vspace{-3mm}
    \caption{Scientific datasets compressed using different augmented compressors with topological controls. From left to right: the original input dataset, the reconstructed datasets from Augmented ZFP, Augmented SZ3, and Augmented TTHRESH, 
    respectively, that preserve the contour trees up to a persistence threshold $\varepsilon = 0.04$. From top to bottom: Tangaroa, Miranda, S3D datasets, respectively. We also display the PSNR and compression ratio next to each decompressed dataset.}    
    \label{fig:volume-render}
    \vspace{-6mm}
\end{figure*}

We provide an overview in~\cref{sec:results-overview}, describing the base compressors and datasets used in our experiments, highlighting the main takeaways, and introducing the evaluation metrics. 
We include compressor configurations and implementation details in~\cref{sec:configurations}. 
In \cref{sec:augmented-compressors} we describe the main utilities of our augmented compressors in preserving contour trees in the reconstructed data.
We evaluate a number of augmented compressors qualitatively and quantitatively, followed by a comparison against the state-of-the-art topology-preserving compressors in \cref{sec:compare-topology}.
We end this section with a run time analysis in \cref{sec:run-time}.

%---------------------------------------
\subsection{An Overview of Experiments}
\label{sec:results-overview}

We present a comparative analysis of five error-bounded lossy compressors augmented with our framework, including the classic compressors ZFP \cite{lindstrom2014fixed}, SZ3 \cite{liang2022sz3}, and TTHRESH \cite{ballester2019tthresh}, a custom-built cubic spline interpolation (CSI) model, and a deep learning-based compressor Neurcomp \cite{lu2021compressive}. 
We test these augmented compressors---denoted as Augmented ZFP, Augmented SZ3, and so on---against two state-of-the-art topology-preserving compressors, TopoSZ~\cite{soler2018topologically} and TopoQZ~\cite{yan2023toposz}. 

We test the five augmented compressors and two topology-preserving compressors on nine volumetric datasets from scientific simulations. The Nyx dataset is very topologically complex---its contour tree has over twenty million nodes---and it is included as a stress test. See \cref{tab:datasets} and \cref{sec:datasets} for details on these datasets.

We further conduct an ablation study demonstrating the effectiveness of logarithmic-scaling quantization and progressive error bound tightening. In every trial, logarithmic-scaling quantization leads to higher compression ratios, whereas progressive tightening results in faster compression times. We also analyze the individual effects of varying $\varepsilon$ and $\xi$; see~\cref{sec:other-experiments} for details on these experiments and the ablation study.

\begin{table}[!ht]
\scriptsize
\centering{
\begin{tabu}{c|*{2}{c}}
\toprule
\textbf{Dataset}  & \textbf{Dimension} & \textbf{Size (MB)}  \\ 
\midrule
QMCPACK      & $69 \times 69 \times 115$          & 4.4         \\
Tangaroa     & $300 \times 180 \times 200$         & 27.0        \\
Earthquake   & $175 \times 188 \times 50$          & 28.2        \\
Ionization   & $310 \times 128 \times 128$         & 40.6       \\
Isabel       & $500 \times 500 \times 90$          & 105.0     \\
Miranda      & $384 \times 384 \times 256$         & 302.0      \\
Nyx          & $512 \times 512 \times 512$         & 641.4      \\
S3D          & $500 \times 500 \times 500$         & 1000.0    \\
SCALE-LETKF  & $1200 \times 1200 \times 98$        & 1129.0    \\
\bottomrule
\end{tabu}
}
\vspace{-2mm}
\caption{Scientific datasets used for compression analysis.}
\label{tab:datasets}
\vspace{-4mm}
\end{table}

\para{Highlighted results.}
We highlight our experimental results below. 
\begin{itemize}[noitemsep,leftmargin=*]
\item Applying any of the five original base compressors to any of the nine datasets produces a large number of topological false cases in the reconstruction, even with a small pointwise error bound. On the other hand, augmenting any compressor with our general framework completely eliminates these false cases while maintaining a user-specified error bound (\cref{sec:augmented-compressors}).
\item Augmented TTHRESH and Augmented ZFP respectively yield the best compression ratios and run times among all the augmented compressors (\cref{sec:augmented-compressors}).
\item Our augmented compressors generally have a better trade off between bit-rate and reconstruction quality compared to TopoQZ and TopoSZ while taking similar or less time to run (\cref{sec:compare-topology}).
\item Our framework has a worst-case time complexity of $O(F h n \log n)$, where $h$ is the maximum height of the contour tree during tightening and $F$ is the number of false cases during computation. The majority of the compression time is spent on computing merge trees (\cref{sec:run-time}).
\end{itemize}

\para{Evaluation metrics.} 
We evaluate whether the contour tree has been perfectly preserved in the reconstructed (decompressed) data. 
We also evaluate the standard compression metrics of compression ratio, bit-rate, and peak signal-to-noise ratio (PSNR).
We further employ topology-based metrics of the bottleneck distances $d_B$~\cite{cohen2005stability} and the Wasserstein distances $d_W$ \cite[page 183]{edelsbrunner2022computational} to quantify the topological similarity between the original data and the reconstructed data. 
The evaluation metrics are described in detail in \cref{sec:evaluationMetrics}.

In general, higher values of PSNR indicate better reconstruction quality, and lower values of $d_B$ and $d_W$ indicate higher topological similarity. 
We measure the total compression time for each compressor, which includes (a) the total time to run the base compressor, and (b) the time to augment the output of the base compressor. We also measure decompression time for each compressor. We measure compression and decompression time for TopoSZ and TopoQZ as well. For our framework and TopoSZ, we decompress to RAW binary format. Because TopoQZ is integrated in the Topology Toolkit, an extension for ParaView, we decompress to VTK image format.

% ---------------------------------------
\subsection{Compressor Configurations and Implementation}
\label{sec:configurations}

In addition to augmenting the out-of-box base compressors SZ3, TTHRESH, ZFP, and Neurcomp, we implement and augment our own super-resolution compressor, a simple custom-built cubic spline interpolation (CSI) model.
It compresses a dataset by downsampling the data by a user-defined ratio in each direction (called a target scale factor) and uses a cubic spline interpolation technique for reconstruction that is similar to the one implemented in SZ3.
%We also considered the Sliced Wasserstein Autoencoder \cite{kolouri2018sliced} used in the AE-SZ compressor \cite{liu2021exploring}. 
%However, this model is excluded from our experiments since it performed significantly worse than the other compressors during initial tests.   

We compare our augmented compressors to TopoSZ and TopoQZ. We use the TopoQZ implementation in TTK~\cite{TiernyFavelierLevine2017}. 

Our general framework requires two user-defined parameters, a persistence threshold $\varepsilon$ and a global absolute pointwise error bound $\xi$.
$\varepsilon$ represents, as a percentage of the range, the level of persistence  simplification. 
For example, $\varepsilon = 0.01$ corresponds to a persistence simplification by $1\%$ of the range of the scalar function. Similarly, $\xi$ is the percentage of the range that will be used as an absolute error bound.

Each base compressor takes a number of intrinsic parameters in order to run. 
Both ZFP and SZ3 require an absolute error bound, denoted as $\delta$ and $\eta$, respectively. 
CSI requires a target scale factor $s$.
TTHRESH takes in a target RMSE of $\tau$. 
Neurcomp requires a target compression ratio $c$. 
Changing the intrinsic parameters of a base compressor will cause it to generate different intermediate data which will be augmented differently. 
As a result, even though our augmented compressor guarantees topology preservation and maintains the user-defined global error bound, the compression results may vary. 

For our experiments, we set the error parameter for each base compressor (except CSI and Neurcomp) to be equal to $k\xi$ for some $k \in \R$ that is compressor-dependent. Specifically, we set $\delta = 5\xi$, $\eta = 0.25\xi$, and $\tau = 0.05\xi$. To decide each value of $k$, we conduct a grid search and observe the effects of different values of $k$ across different values of $\xi$ and different datasets. The optimal value of $k$ varies between datasets and values of $\xi$, but the values that we chose are always approximately optimal. Hypothesized explanations as to why these values of $k$ improve results are described in \cref{sec:base-compressor-parameters}. We also set $c = 100$ and $s = 7$. We chose these configurations because they empirically led to the highest compression ratios.

To differentiate from the persistence threshold $\varepsilon$ used by an augmented compressor, TopoQZ takes a persistence threshold $e$. The TTK implementation of TopoQZ is tightly coupled with ZFP, which requires an error bound $\zeta$, allowing for a total pointwise error upper-bounded by $e+\zeta$.
To measure compression ratio and compression times while respecting a topological constraint $\varepsilon$ and an error bound $\xi$, we measure how each augmented compressor and TopoSZ perform for $\varepsilon = 0.04$ and $\xi = 0.012$. When testing TopoQZ, to ensure that it respects both $\varepsilon$ and $\xi$, we set $e = \zeta = 0.006$ so that the max error is less than $0.012$.

To measure the trade off between compression ratio and reconstruction quality, for TopoSZ and each augmented compressor, we set $\varepsilon = 0.04$ and vary $\xi \in \{$0.003, 0.006, 0.009, 0.012, 0.015, 0.018$\}$. In some cases we need to vary $\xi$ (and $\zeta$, for TopoQZ) in a different range in order to obtain a meaningful curve. Notably, for TopoQZ, we set $e = 0.04$ and vary $\zeta \in \{$0.003, 0.11, 0.22, 0.33, 0.44, 0.55$\}$. We further discuss our methodology for choosing parameters and provide the parameter used in \cref{sec:reconstruction-quality-extra}. 

The combination of a chosen compressor, a fixed dataset, a value of $\varepsilon$ and $\xi$, is a trial.
We perform each trial on a single cluster node running an Intel Xeon Sandy Bridge-E processor with 16 cores and 64GB of RAM.
For Neurcomp, we perform the training on an RTX 2080ti GPU with 32GB of RAM.

\begin{figure}[!ht]
\begin{subfigure}{0.02\linewidth}
\raisebox{2.2\height}{
\includegraphics[angle=90, width=\linewidth]{fig-colorBar.png}}
\end{subfigure}
\begin{subfigure}{0.97\linewidth}
    \includegraphics[width=\linewidth]{fig-scale_lattice_2.png}
\end{subfigure}
\vspace{-6mm}
\caption{Zoomed-in views of the critical points of the contour trees of the Ionization (top) and SCALE-LETKF (bottom) datasets with persistence simplification $\varepsilon = 0.04$. For each dataset, arrows indicate one example where compression with TTHRESH led to critical points shifting. Local maxima are in orange, local minima are in dark blue, 1-saddles are in light blue, 2-saddles are in light orange. Top row: Ionization. Bottom row: SCALE-LETKF.}
\label{fig:zoom}

\centering{
\begin{tabular}{c|cc|c}
\hline
Dataset    & ZFP       & TTHRESH       & Total \#edges \\ \hline
QMCPACK    & (23,23,0) & (8,8,0)     & 69          \\ 
Tangaroa   &  (90, 92, 0) & (39, 46, 0) & 418         \\ 
Earthquake & (20, 19, 0) & (26, 25, 0)     & 169         \\
Ionization & (181, 187, 0)  & (37, 44, 0) & 568         \\
Isabel     & (16, 15, 0) & (15, 16, 0)  & 29          \\
Miranda    & (6, 6, 0) & (3, 3, 0)     & 11          \\ 
Nyx        & (133, 132, 0) & (2, 373, 0) & 743 \\
S3D       & (100, 100, 0) & (74, 80, 0) & 1013 \\
SCALE-LETKF & (152, 153, 0) & (129, 127, 0) & 401 \\ \hline
\end{tabular}
}
\vspace{-2mm}
\captionof{table}{Reporting the number of false cases (false positives, false negatives, false types) produced by base compressors SZ3 and TTHRESH, respectively, together with the total number of edges of the input (ground truth) contour tree. Contour trees are simplified with $\varepsilon=0.04$.}
\label{tab:base-false-cases}
\vspace{-6mm}
\end{figure}

%---------------------------------------
\subsection{Comparative Analysis of Augmented Compressors}
\label{sec:augmented-compressors}

In this section, we perform a comparative analysis of five augmented compressors, qualitatively and quantitatively. 
We visualize three scientific datasets before and after compression with {three of our augmented compressors} in \cref{fig:volume-render}. We also display the PSNR and compression ratio next to each decompressed dataset. 
Compression ratios and times for a single combination of $\varepsilon$ and $\xi$ are reported in \cref{tab:compression-task}. Charts showing the reconstruction quality on two datasets is reported in \cref{fig:reconstruction-quality}. Similar charts for the remaining datasets and compressors are shown in \cref{sec:reconstruction-quality-extra}. Results demonstrating the effect of independently varying $\varepsilon$ or $\xi$ on the evaluation metrics are given in \cref{sec:other-experiments}.

\subsubsection{Topological Guarantees}
When compressing a dataset with any base compressor, the contour tree of the data is often significantly distorted with a large number of false cases, whereas it is always perfectly preserved using our augmented compressor. This observation has been validated empirically in every trial: the contour tree is perfectly preserved in terms of the locations of its critical points and their connectivity.

For instance, we visualize the Isabel dataset in \cref{fig:teaser} using TTHRESH and augmented TTHRESH. We highlight parts of the contour trees via zoomed-in views before and after compression. In a bottom zoomed-in view, TTHRESH (middle) fails to preserve a few critical points of the contour tree. In a top zoomed-in view, TTHRESH (middle) preserves the locations of the critical points, but not their connectivity. In contrast, the augmented TTHRESH preserves both the locations and connectivity among the critical points.

We further provide zoomed-in views for the Ionization and SCALE-LETKF datasets in~\cref{fig:zoom}. We observe clearly that TTHRESH fails to predict many critical points, whereas augmented TTHRESH preserves them all.

In \cref{tab:base-false-cases}, we report the number of false cases, including both extremum-saddle and saddle-saddle connections in the contour tree reconstructed with ZFP and TTHRESH. We again use parameter configurations that produce the same compression ratios as their augmented versions with $\varepsilon = 0.04$ and $\xi = 0.012$ (for those configurations see \cref{sec:base-compressor-parameters}). In \cref{tab:base-false-cases}, we can see that ZFP and TTHRESH produce many false cases.

\begin{table}[!t]
\setlength{\tabcolsep}{2pt}
\resizebox{\columnwidth}{!}{
\begin{tabular}{cccccccc}
\hline
Dataset     & A-ZFP            & A-SZ3   & A-CSI         & A-TTHRESH      & \multicolumn{1}{c|}{A-Neurcomp} & TopoQZ             & TopoSZ          \\ \hline
\multicolumn{8}{c}{Compression Ratio}                                                                                                                \\ \hline
QMCPACK     & 58.7             & 86.1    & 102.3           & \textbf{104.8} & \multicolumn{1}{c|}{23.9}       & 23.4               & 27.8            \\
Tangaroa    & 37.3             & 43      & 33.5            & \textbf{44.8}  & \multicolumn{1}{c|}{15.3}       & --                 & 24.3            \\
Earthquake  & 86.1             & 127.4   & 79.4            & \textbf{129.2} & \multicolumn{1}{c|}{63.5}       & 13.4               & 50.1            \\
Ionization  & 118.8            & 121.5   & 119.9           & \textbf{170.5} & \multicolumn{1}{c|}{72.7}       & 30.0               & 25.1            \\
Isabel      & 47.4             & 103.5   & 70.6            & \textbf{182.2} & \multicolumn{1}{c|}{41.6}       & --                 & 37.6            \\
Miranda     & 172.3            & 198.6   & 157.2           & \textbf{318.7} & \multicolumn{1}{c|}{95.0}       & 76.5               & 95.9            \\
Nyx         & 65.3             & 69.5    & 70.4            & \textbf{84.5}  & \multicolumn{1}{c|}{18.9}       & --                 & --              \\
S3D         & 38.4             & 46.6    & 43.6            & \textbf{59.6}  & \multicolumn{1}{c|}{6.0}        & 9.2                & --              \\
SCALE-LETKF & 69.5             & 74.4    & 58.5            & \textbf{114.2} & \multicolumn{1}{c|}{8.6}        & 11.4               & --              \\ \hline
\multicolumn{6}{c|}{Total Compression and Augmentation Time}                                                  & \multicolumn{2}{c}{Compression Time} \\ \hline
QMCPACK     & 1.27             & 1.36    & 1.21            & 1.45           & \multicolumn{1}{c|}{172.08}     & \textbf{1.05}      & 10.46           \\
Tangaroa    & 9.51             & 10.99   & \textbf{9.33}   & 11.98          & \multicolumn{1}{c|}{1519.21}    & --                 & 314.56          \\
Earthquake  & \textbf{7.08}    & 7.39    & \textbf{7.08}   & 8.66           & \multicolumn{1}{c|}{1039.94}    & 8.08               & 48.17           \\
Ionization  & \textbf{8.68}    & 10.00   & 14.08           & 15.16          & \multicolumn{1}{c|}{1221.12}    & 10.40              & 425.31          \\
Isabel      & \textbf{33.77}   & 35.49   & 42.69           & 42.67          & \multicolumn{1}{c|}{7147.86}    & --                 & 367.10          \\
Miranda     & 223.52           & 284.18  & 248.51          & 348.72         & \multicolumn{1}{c|}{9359.59}    & \textbf{160.60}    & 434.98          \\
Nyx         & \textbf{1059.06} & 1137.33 & 5664.46         & 25594.54       & \multicolumn{1}{c|}{38959.73}   & --                 & --              \\
S3D         & 209.83           & 253.82  & \textbf{173.09} & 253.72         & \multicolumn{1}{c|}{34610.78}   & 633.13             & --              \\
SCALE-LETKF & \textbf{221.49}  & 399.32  & 343.58          & 371.19         & \multicolumn{1}{c|}{40887.62}   & 524.05             & --              \\ \hline
\multicolumn{8}{c}{Decompression Time}                                                                                                               \\ \hline
QMCPACK     & 0.14             & 0.32    & 0.14            & 0.17           & \multicolumn{1}{c|}{4.24}       & 0.63               & \textbf{0.01}   \\
Tangaroa    & 0.49             & 0.52    & 0.51            & 0.96           & \multicolumn{1}{c|}{16.32}      & --                 & \textbf{0.12}   \\
Earthquake  & 0.38             & 0.41    & 0.37            & 0.55           & \multicolumn{1}{c|}{8.50}       & 5.50               & \textbf{0.07}   \\
Ionization  & 0.48             & 0.54    & 0.54            & 0.83           & \multicolumn{1}{c|}{11.62}      & 5.08               & \textbf{0.10}   \\
Isabel      & 1.43             & 1.35    & 1.34            & 2.64           & \multicolumn{1}{c|}{79.61}      & --                 & \textbf{0.41}   \\
Miranda     & 2.92             & 3.1     & 3.38            & 4.53           & \multicolumn{1}{c|}{175.61}     & 53.2               & \textbf{0.67}   \\
Nyx         & \textbf{6.72}    & 8.05    & 9.71            & 9.49           & \multicolumn{1}{c|}{2457.91}    & --                 & --              \\
S3D         & \textbf{11.52}   & 11.47   & 11.84           & 16.33          & \multicolumn{1}{c|}{2135.47}    & 390.94             & --              \\
SCALE-LETKF & \textbf{11.59}   & 11.92   & 12.89           & 21.50          & \multicolumn{1}{c|}{2629.48}    & 351.61             & --              \\ \hline
\end{tabular}
}
\vspace{-2mm}
\caption{Compression ratio, compression time, and decompression time for each compressor with $\varepsilon = 0.04$ and error bound $\xi = 0.012$ (except TopoQZ has $e = \zeta = 0.006$).
Times are in seconds.
Trials that did not finish are marked with a dash. 
TopoQZ ran out of memory on Nyx and it crashed on Isabel and Tangaroa due to unknown reasons. 
TopoSZ ran out of memory on Nyx, S3D, and SCALE-LETKF. 
}
\label{tab:compression-task}

\includegraphics[width=\columnwidth]{fig-rate_distortion_2.png}
    \vspace{-4mm}
    \captionof{figure}{PSNR, bottleneck distance, and Wasserstein distance versus bit-rate for each compressor for the QMCPACK and Earthquake datasets with $\varepsilon = 0.04$ ($e = 0.04$ for TopoQZ). These curves are given for other datasets in \cref{sec:reconstruction-quality-extra}.}
    \label{fig:reconstruction-quality}
    \vspace{-4mm}
\end{table}

%---------------------------------------
\subsubsection{Evaluation Metrics}
\label{sec:augmented-compressors-evaluation-metrics}

Compression ratio and times are reported in \cref{tab:compression-task} for a fixed parameter configuration of $\varepsilon = 0.04$ and $\xi = 0.012$.
We chose this parameter configuration because a small amount of persistence simplification preserves a large number of topological features in the input data, generating complex test cases for topology-preserving compression. For the reconstruction quality demonstrated in \cref{fig:reconstruction-quality}, $\varepsilon = 0.04$ is chosen similarly, and $\xi$ is varied between $0.003$ and $0.018$ to yield a variety of different compression ratios while still remaining small.
In this section, we compare the different augmented compressors. We leave the comparison with TopoQZ and TopoSZ to \cref{sec:compare-topology}.

\para{Compression ratios.} 
As shown in \cref{tab:compression-task}, Augmented TTHRESH produces the best compression ratios in every trial. Augmented Neurcomp performs noticeably worse than the other compressors.

\para{Reconstruction quality.}
In every trial, our framework successfully maintains a pointwise error bound $\xi$. There is a natural trade off between compression ratio and reconstruction quality. As shown in \cref{fig:reconstruction-quality}, Augmented SZ3 and Augmented TTHRESH have the best trade off between bit-rate, PSNR, $d_W$ and $d_B$, and perform equally well. Augmented Neurcomp performs the worst based on above metrics.

When visualized, we find that the decompressed volumes generally resemble the ground truth. However, when using certain transfer functions, visual artifacts may become visible. Artifacts appear more in visualizations that are sensitive to small changes in the transfer function. For the volume renderings in this paper, we chose transfer functions that led to fewer visual artifacts; see~\cref{sec:visual-artifacts} for adversarial examples.

In practice, we find that upper and lower bound tightening does not affect PSNR very much; most of the reconstruction quality is determined by the initial upper and lower bounds. \cref{fig:errorMap} shows a map of the absolute error of each point for a topologically complex slice of the Ionization dataset before and after tightening. We can see that tightening does not have a significant effect on the average error.

\begin{figure}[!t]
    \begin{subfigure}[b]{0.06\linewidth}
        \raisebox{0.5\height}{\includegraphics[width=\linewidth]{fig-colorBar-error.png}}
    \end{subfigure}
    \hfill
    \begin{subfigure}[b]{0.46\linewidth}
    \centering
        \includegraphics[width=\linewidth]{fig-errorMapPre.png}
        (A)
    \end{subfigure}
    \hfill
    \begin{subfigure}[b]{0.46\linewidth}
    \centering
        \includegraphics[width=\linewidth]{fig-errorMapPost.png}
        (B)
    \end{subfigure}
    \vspace{-5mm}
    \caption{Error map of a topologically complex slice of the Ionization dataset (A) before error bound tightening and (B) after error bound tightening.}
    \label{fig:errorMap}
    \vspace{-8mm}
\end{figure}

\para{Run time analysis.} 
There are significant differences in run time among the augmented compressors.
As discussed in \cref{sec:run-time}, these times are affected by factors other than the base compression time. However, Augmented Neurcomp is the slowest because Neurcomp compresses data by training a neural network.  
Of the remaining four compressors, ZFP is typically the fastest, and TTHRESH the slowest, although this observation does not hold for all trials. For decompression time, ZFP is the fastest, while Neurcomp remains the slowest.

\para{Highlighted results.} 
There is no clear best augmented compressor that outperforms others across all metrics. Other than augmented Neurcomp, utilizing any augmented compressor plays a trade off between compression ability and speed. For the remainder of our analysis, we will primarily focus on Augmented ZFP, which is the fastest augmented compressor, and Augmented TTHRESH, which yields the best compression ratios and reconstruction quality.

%---------------------------------------
\subsection{Comparison with TopoQZ and TopoSZ}
\label{sec:compare-topology}

\para{Topological guarantees.}
Our framework preserves the contour tree during compression, and achieves the same topological guarantee as TopoSZ. TopoQZ ensures that all critical point pairs are preserved above a persistence threshold $\varepsilon$, but their locations and connectivity may be distorted after compression.

\para{Compression ratio.} 
In terms of compression ratio, when maintaining a strict topological constraint $\varepsilon = 0.04$ and error bound $\xi = 0.012$, every augmented compressor except Augmented Neurcomp outperforms both TopoQZ and TopoSZ in every trial.

\para{Reconstruction quality.} 
The curves in \cref{fig:reconstruction-quality} show that every augmented compressor except Augmented Neurcomp can match the PSNR of TopoQZ and TopoSZ,  while using less space. In terms of topological distance, the augmented compressors except Augmented Neurcomp outperform TopoSZ in terms of $d_W$ and $d_B$. They also outperform TopoQZ in terms of $d_B$, but are comparable in terms of $d_W$.

\para{Run time analysis.} 
In terms of compression time, the augmented compressors except Augmented Neurcomp produce times that are comparable to or better than TopoQZ, and significantly outperform TopoQZ on the largest datasets. 
These four augmented compressors are also significantly faster than TopoSZ across all trials. 

In terms of decompression time, the augmented compressors except Augmented Neurcomp perform slower than TopoSZ but faster than TopoQZ. There are several possible reasons why our decompression times are slower than TopoSZ. First and most notably, our decompression process is more complex, as it involves a decompression with the base compressor and then an augmentation of the decompressed results. This process requires more operations and has a higher I/O overhead. Second, we use XZ along with tar archives for lossless compression, which is slower than ZSTD used by TopoSZ. See \cref{sec:more-running-time} for a more detailed analysis of the decompression time.   

%---------------------------------------
\subsection{Analysis of Compression Time}
\label{sec:run-time}

\para{Asymptotic analysis.}
Let $n$ be the number of vertices in the rectilinear mesh. Our algorithm utilizes heap merges~\cite{gueunet2017task} during the merge tree computation; however, we use binary heaps (stored in arrays) instead of Fibonacci heaps from~\cite{gueunet2017task}. For a binary heap with $m$ elements, a single insertion operation has a worst-case time complexity of $O(\log m)$. 
Following~\cite{gueunet2017task}, from bottom to top, constructing an edge $e$ in a merge tree requires merging its heap with the heaps of its descendants, which takes $O(n \log n)$. Let $h$ denotes the \emph{height} of the tree, which corresponds to the maximum number of ancestor edges.
Then constructing a merge tree using these insertion-based heap merges takes $O(h n \log n)$. During the progressive tightening process, let $F$ denote the total number of detected false cases, each of which triggers a (partial) recomputation of the merge tree. Therefore, our algorithm takes $O(F h n \log n) = O(n^3 \log n)$. 

In practice, $F \ll n$ as shown in \cref{tab:time}. 
Additionally, $h \ll n$. We found that $\frac{h}{n}$ ranged from $0.0004$ (Miranda, A-ZFP) to $0.025$ (Nyx, A-Neurcomp). Excluding Augmented Neurcomp, $\frac{h}{n} < 0.01$ in $97\%$ of trials.  
On the other hand, using Fibonacci heaps to construct a merge tree~\cite{gueunet2017task} takes $O(n \log n)$ due to constant time heap merges; however, in our setting, we have found that binary heaps have lower run time in practice. Likewise, it is possible to merge heaps in linear time, but we instead merge by repeatedly inserting each element of the smaller heap into the larger one, as doing so has a much lower run time in practice.

\para{Empirical analysis.} To analyze the run time empirically, we calculate the amount of time for each portion of our algorithm with Augmented ZFP and Augmented TTHRESH,  with $\varepsilon = 0.04$ and $\xi = 0.012$. These run times are shown in \cref{tab:time}.

In \cref{tab:time}, the most time-consuming task is the computation of merge and contour trees. We compute the contour tree of the input data at the beginning of the algorithm. During the error bound tightening steps we also compute the contour tree of the decompressed data. These run times are shown in \cref{tab:time} under the `CT' and `Grow' columns, and account for $35-77\%$ of the total run time for each trial in \cref{tab:time}. 

For most of the trials, the time to run the base compressor, shown in the `BC' column, is a relatively small percentage of the overall compression time. 
However, if a base compressor produces results that nearly preserve the contour tree and does not produce too many extra branches, including those of persistence below $\varepsilon$, the augmentation time may be lower. This phenomenon suggests that the accuracy of the base compressor may have more effect on the total compression and augmentation time than just base compression.
In general, the run time of each base compressor is much faster than its augmented counterpart; see~\cref{sec:more-running-time} for a comparison. 

\begin{table}[!ht]
\setlength{\tabcolsep}{2pt}
\centering
\vspace{-2mm}
\resizebox{\columnwidth}{!}{
\begin{tabular}{cccccccccc}
\hline
\multicolumn{1}{c|}{Dataset}     & BC    & CT     & ULB     & Grow     & \%B    & \#FC & Fix      & File  & Total    \\ \hline
\multicolumn{10}{c}{Augmented ZFP}                                                                                  \\ \hline
\multicolumn{1}{c|}{QMCPack}     & 0.15  & 0.34   & 0.21   & 0.38     & 0.17\% & 0    & 0.0      & 0.27  & 1.35     \\
\multicolumn{1}{c|}{Tangaroa}    & 1.42  & 2.16   & 2.80   & 1.34     & 0.43\% & 14   & 0.0004   & 1.84  & 9.57     \\
\multicolumn{1}{c|}{Earthquake}  & 0.45  & 1.56   & 1.26   & 2.88     & 0.79\% & 2    & 0.0001   & 0.97  & 7.12     \\
\multicolumn{1}{c|}{Ionization}  & 0.70  & 1.56   & 2.09   & 3.10     & 1.15\% & 10   & 0.0009   & 1.35  & 8.81     \\
\multicolumn{1}{c|}{Isabel}      & 4.73  & 4.61   & 8.95   & 9.48     & 0.54\% & 1    & 0.0647   & 6.17  & 34.01    \\
\multicolumn{1}{c|}{Miranda}     & 4.00  & 156.79 & 17.92  & 37.85    & 1.21\% & 0    & 0.0      & 7.02  & 223.58   \\
\multicolumn{1}{c|}{NYX}         & 24.75 & 695.26 & 150.27 & 142.16   & 0.47\% & 6    & 4.60     & 42.21 & 1084.05  \\
\multicolumn{1}{c|}{S3D}         & 14.2  & 27.93  & 69.11  & 56.40    & 1.04\% & 39   & 0.01411  & 38.44 & 206.63   \\
\multicolumn{1}{c|}{SCALE-LETKF} & 15.2  & 16.15  & 61.67  & 92.99    & 0.63\% & 37   & 0.003408 & 35.34 & 221.47   \\ \hline
\multicolumn{10}{c}{Augmented TTHRESH}                                                                              \\ \hline
\multicolumn{1}{c|}{QMCPack}     & 0.23  & 0.34   & 0.21   & 0.45     & 0.77\% & 0    & 0.0      & 0.22  & 1.45     \\
\multicolumn{1}{c|}{Tangaroa}    & 3.34  & 2.2    & 2.79   & 1.92     & 0.77\% & 18   & 0.0001   & 1.59  & 11.84    \\
\multicolumn{1}{c|}{Earthquake}  & 1.53  & 1.56   & 1.25   & 3.33     & 0.65\% & 4    & 0.0001   & 0.91  & 8.58     \\
\multicolumn{1}{c|}{Ionization}  & 1.93  & 1.54   & 2.08   & 8.09     & 1.48\% & 5    & 0.0120   & 1.24  & 14.94    \\
\multicolumn{1}{c|}{Isabel}      & 13.13 & 4.62   & 9.01   & 10.24    & 0.54\% & 1    & 0.0666   & 5.62  & 42.69    \\
\multicolumn{1}{c|}{Miranda}     & 12.73 & 156.48 & 17.73  & 153.69   & 2.33\% & 0    & 0.0      & 7.24  & 347.87   \\
\multicolumn{1}{c|}{NYX}         & 57.87 & 687.03 & 132.49 & 15838.88 & 0.82\% & 89   & 97.82    & 37.52 & 27286.05 \\
\multicolumn{1}{c|}{S3D}         & 66.35 & 27.89  & 67.27  & 56.51    & 1.14\% & 47   & 0.0136   & 32.11 & 250.79   \\
\multicolumn{1}{c|}{SCALE-LETKF} & 69.21 & 16.10  & 60.00  & 193.18   & 0.54\% & 46   & 0.0032   & 29.71 & 368.35   \\ \hline
\end{tabular}
}
\vspace{-2mm}
\caption{Runtime analysis for each component of the augmented framework involving Augmented ZFP and Augmented TTHRESH with $\varepsilon = 0.04$ and $\xi = 0.012$. 
All times are in seconds. 
BC: running the base compressor.
CT: computing the contour tree of the input data.
ULB: calculating the initial upper and lower bounds. 
Grow: time growing the contour tree of the reconstructed data.
\%B: percent of branches in the reconstructed contour tree whose growth was recomputed.
\#FC: number of false cases corrected after upper and lower bounds are set.
Fix: average time to fix a false case, excluding regrowing branches.
File: average time to write the compressed output to a file.}
\label{tab:time}
\vspace{-6mm}
\end{table}
\section{Discussion and Related Work}\label{sec:relatedWork}

\paragraph{Persona Evaluation}

Prior work has established several frameworks for evaluating language models' role-playing capabilities. \citet{wang-etal-2024-rolellm} introduced RoleBench, an evaluation benchmark with QA pairs based on character profiles. \citet{wang-etal-2024-incharacter} developed InCharacter, assessing role-playing fidelity through psychological scales in an interview format. \citet{tu-etal-2024-charactereval} created CharacterEval, a Chinese benchmark derived from novels and scripts with multi-interaction dialogues, while \citet{shen2023roleeval} established RoleEval, a bilingual benchmark with multiple-choice questions testing persona knowledge and reasoning. \citet{samuel2024personagym} introduced PersonaGym, a dynamic evaluation framework for automated assessment of persona adherence across diverse interactions. Our work further extends the literature by performing the first systematic evaluation to understand the influence of the persona modality.

\paragraph{Multimodal Personas}

Recent work has explored integrating visual elements into LLM persona systems. \citet{ahn-etal-2023-mpchat} introduced MPCHAT, demonstrating that incorporating visual episodic memories alongside text improves dialogue consistency and persona grounding. \citet{sun-etal-2024-kiss} investigated how visual personas influence LLMs' behavior in negotiation contexts, showing models can adapt their responses based on perceived visual personality traits. \citet{dai2025mmrole} developed MMRole, a framework for training and evaluating multimodal role-playing agents. While these works establish the potential of visual personas and others extensively evaluate textual personas \citep{li-etal-2016-persona, xiao2024farllmsbelievableai, samuel2024personagym}, there has been no systematic comparison of how different modalities of persona representation affect model performance. Our work addresses this gap by directly evaluating text, visual, and hybrid approaches across a range of persona-based tasks.

\paragraph{Modality Alignment}

Language models demonstrate strong in-context learning capabilities in unimodal textual settings \citep{Shanahan2023, 10.5555/3666122.3669274}.
However, extending these capabilities to multimodal inputs remains challenging.
When visual information is introduced, models often struggle to transfer knowledge effectively from text to vision (and vice versa), resulting in noticeably weaker performance with visual in-context demonstrations compared to textual ones \citep{zhao2024mmicl, jiang2024manyshot}.
Such cross-modal gaps manifest in several ways: for instance, catastrophic forgetting of text-based instruction following can occur when models are finetuned on images \citep{zhang2024wingslearningmultimodalllms}.
While incorporating visual knowledge can yield improvements on specific tasks \citep{jin-etal-2022-leveraging}, maintaining consistently high performance across both textual and visual modalities remains an open research question, which is also highlighted in our work.
\section{Conclusion and Discussion}
\label{sec:conclusion}

We introduce a novel framework for augmenting \emph{any} lossy compressor to preserve the contour tree of a volumetric dataset while maintaining a user-specified global error bound. 
To do this, our framework first imposes topology-informed upper and lower bounds on each data point. 
It then progressively tightens those bounds until the contour tree is preserved. 
We also introduce a novel encoding scheme that efficiently stores individual points with variable precision and maintains these upper and lower bounds. 
When our framework is used to augment state-of-the-art lossy compressors, it is shown to preserve the contour trees of various scientific datasets.
Our augmented compressors also achieve higher compression ratios and reconstruction quality than those obtained by existing topology-preserving compressors in comparable or faster time.
Our framework will benefit from any advancement with lossy compression since it can be used to augment increasingly effective lossy compressors to achieve better topology-preserving compression. 

Our framework is not without limitations. The compression times are longer than the base compressors. This difference gets worse as the topological complexity of the data increases.
However, in some use-cases, topological preservation is preferable to run time.
Regardless, our framework would benefit from more efficient or parallel implementations for the contour/merge tree computation and the encoding scheme. 


\clearpage
\section*{Limitations}\label{sec-limitations}
% \ks{write somethign}
A limitation of our work is that we only deal with 40 personas. However, due to a lack of any persona dataset with equivalent representations in different modalities, we see this as our contribution and leave it for future works to expand the scale of the study. Furthermore, we specifically increase the diversity of these personas across $4$ well-grounded categories, focusing on the quality of our dataset. As the field of persona alignment in LLMs is still quite nascent, we believe quality becomes more important than quantity. Additionally, it should be noted that the persona modality representations may not align perfectly across all details. Our pipeline employs two distinct mapping functions---Stable Diffusion (text-to-image) and GPT-4o-mini (image-to-text)---which will naturally introduce extraneous information or inconsistencies between representations. However, this limitation is acceptable for our evaluation framework since we only test for the presence and consistency of specific attributes rather than complete fidelity across all possible persona characteristics. Another limitation is that we have only validated our results on a small set of human annotators. We circumvent this by leveraging the validation of LLM-based evaluation with human evaluations~\citep{samuel2024personagym} while also showing a high correlation of our results across different LLM evaluators. 

\section*{Broader implications and social impact}
We intend our proposed dataset to be used strictly for academic purposes. While we design our dataset such that it does not contain any harmful and private content, our pipeline can be adapted to generate such unintended visual personas. However, we note that this is not a direct result of our artifact and can also be possible through directly querying the StableDiffusion APIs. Thus, we expect our contributions of dataset and evaluation methodology to have an overall positive social impact by inspiring future research on aligning modalities for persona embodiment.
\bibliography{citations}

\clearpage

\appendix
\clearpage
 
\definecolor{exampleblue}{RGB}{0, 114, 188} % Blue for header
\definecolor{exampleborder}{RGB}{0, 114, 188} % Blue for border
\definecolor{redtext}{RGB}{204, 0, 0}         % Red text for emphasis

\section{Appendix}
In this appendix, we provide additional details about the design of \textbf{NumericBench}, along with supplementary experimental results and case studies. The organization of the supplementary materials in this appendix is as follows:

\begin{enumerate}[leftmargin=*]
	
	\item \textbf{Question formats for contextual retrieval, comparison, and summary abilities.}  
As shown in Table~\ref{appx:number_question}, Table~\ref{appx:stock_question}, and Table~\ref{appx:weather_question}, 
we designed diverse question types tailored to each dataset to evaluate the three fundamental numerical abilities of LLMs: contextual retrieval, comparison, and summary. contextual retrieval  assesses the model’s capacity to accurately extract relevant numerical information from complex contexts; comparison tests the ability to analyze and compare numerical values;  Summary evaluates the synthesis of numerical information into concise and meaningful insights for tasks like reporting or trend analysis.


By designing tailored questions for each dataset, we ensure a comprehensive evaluation of LLMs’ numerical reasoning abilities across varying scenarios and complexities.
	\item \textbf{Basic numerical questions answered incorrectly by GPT-4o.}  
	As illustrated in Figure~\ref{fig:number_compare}, Figure~\ref{fig:multiplication}, and Figure~\ref{fig:number_couting}, GPT-4o failed to answer three basic numerical questions correctly. This result is surprising, considering GPT-4o's impressive performance in real-world applications. However, these findings highlight the weak fundamental numerical abilities of LLMs.
	
	\item \textbf{Token counts for short and long contexts.}  
	As shown in Table~\ref{tab:data_stat_short_long}, the token counts of long and short contexts differ significantly. This distinction enables a more thorough evaluation of LLM performance across scenarios involving varying context lengths. Short contexts are designed to test the model's ability to process and understand concise information, focusing on immediate comprehension and reasoning. In contrast, long contexts present a more complex challenge, requiring the model to handle extended sequences of information, maintain coherence over a larger context window, 
	and retrieve relevant details from earlier parts of the input. Such two type length can more comprehensively evaluate LLMs. 
	
	\item \textbf{Additional experimental results on noisy and varying-length contexts.}  
	As shown in Figure~\ref{fig:length_stock} and Figure~\ref{fig:length_weather}, existing LLMs perform poorly on the stock and weather datasets, although they achieve better performance compared to their results on short contexts. 
	Similarly, as shown in Figure~\ref{fig:noisy_weather}, LLMs perform poorly on noisy weather data.
 
 \item \textbf{Real failure cases of math-oriented LLMs.} In this paper, we do not compare existing math-oriented LLMs, such as Metamath-Llemma-7B~\citep{yu2023metamath}, Deepseek-Math-7B-Instruct~\citep{deepseek-math}, InternLM2-Math-7B~\citep{ying2024internlmmathopenmathlarge}, and MAmmoTH-7B~\citep{yue2023mammoth}. 
 This is primarily because these math-oriented LLMs are designed for specialized geometric and structured mathematical problems. They are unable to understand the tasks in NumericBench, fail to follow a correct reasoning process, and directly produce meaningless outputs. These failure cases are illustrated in Figure~\ref{fig:fail_internlm}, Figure~\ref{fig:fail_ds_math}, Figure~\ref{fig:fail_llemma}, and Figure~\ref{fig:fail_mammoth}.
 
\end{enumerate}

\noindent \textbf{The Use of AI Tools.} When writing  this paper, we use Grammarly\footnote{https://www.grammarly.com/} for automated spell checking and use GPT-4o\footnote{https://platform.openai.com/docs/models/gpt-4o} to refine several sentences.


\clearpage
 
 

\begin{table*}[!h]
	\centering
	
	\caption{Question format on number list dataset. R: contextual retrieval, C: comparison, S: summary. In the contextual retrieval task, a number $x$ is randomly selected from the given number list. For the comparison task, the $k$-th largest number is randomly generated within the range of one to the length of the number list. The indices $x$ corresponds to twenty percent of the length of the number list, while $y$ corresponds to eighty percent of the length. The number $z$ is randomly chosen within the range $(\min(\text{list}), \max(\text{list}))$. For the summary task, the top $k$ is set to thirty percent of the length of the number list.}
 
	\renewcommand{\arraystretch}{1.15}  
	\setlength{\tabcolsep}{1.5pt}  
	\begin{tabular}{c|c}
		\toprule
		\textbf{Ability}    & \textbf{Question Format} \\ \midrule
		\textit{\textbf{R}} &  \begin{tabular}[c]{@{}l@{}}
			$Q_0$: What is the index of the first occurrence of the number $x$ in the list?\\
			$Q_1$: What is the index of the last occurrence of the number $x$ in the list?\\
			$Q_2$: What is the number after the first occurrence of the number $x$ in the list?\\
			$Q_3$: What is the number before the last occurrence of the number $x$ in the list?\\
			$Q_4$: What is the index of the first even number in the list?\\
			$Q_5$: What is the index of the first odd number in the list?\\
			$Q_6$: What is the index of the last even number in the list?\\
			$Q_7$: What is the index of the last odd number in the list?
		\end{tabular} \\ \midrule
		\textit{\textbf{C}} &  \begin{tabular}[c]{@{}l@{}}
			$Q_8$: What is the index of the first occurrence of the $k$-th largest number in the given list?\\
			$Q_9$: Which index holds the greatest number in the list between the indices $x$ and $y$?\\
			$Q_{10}$: Which index holds the smallest number in the list between the indices $x$ and $y$?\\
			$Q_{11}$: Which number is closest to $z$ in the list between the indices $x$ and $y$?\\
			$Q_{12}$: Which number is furthest from $z$ in the list between the indices $x$ and $y$?\\
			$Q_{13}$: Which number is the largest among those less than $z$ in the list?\\
			$Q_{14}$: Which number is the smallest among those greater than $z$ in the list?
		\end{tabular} \\ \midrule
		\textit{\textbf{S}} &  \begin{tabular}[c]{@{}l@{}}
			$Q_{15}$: What is the maximum sum of any two consecutive items in the list?\\
			$Q_{16}$: What is the maximum sum of any three consecutive items in the list?\\
			$Q_{17}$: What is the maximum absolute difference between two consecutive items in the list?\\
			$Q_{18}$: What is the sum of the indices of the top $k$ largest numbers in the list?\\
			$Q_{19}$: What is the sum of the indices of the top $k$ smallest numbers in the list?\\
			$Q_{20}$: What is the average of the indices of the top $k$ largest numbers in the list?\\
			$Q_{21}$: What is the average of the indices of the top $k$ smallest numbers in the list?\\
			$Q_{22}$: How many times do numbers consecutively increase for more than five times?\\
			$Q_{23}$: How many times do numbers consecutively decrease for more than five times?\\
			$\cdots \cdots$ \\
		\end{tabular} \\ \bottomrule
	\end{tabular}	
	\label{appx:number_question}
\end{table*}
\clearpage


 

\begin{table*}[]
	
	\caption{Question format on stock dataset. R: contextual retrieval, C: comparison, S: summary. $x$ and $y$ lie within the minimum and maximum ranges of their respective attributes. The top $k$ corresponds to thirty percent of the number list. $date_1$ represents the day at the twentieth percentile of the stock history, while $date_2$ corresponds to the day at the eightieth percentile.}
	\centering
	\renewcommand{\arraystretch}{1.15} % 设置行间距为默认的 1.15 倍
	\setlength{\tabcolsep}{1.5pt} % 将列间距设置为 1pt
	\begin{tabular}{c|c}
		\toprule
		\textbf{Ability}    & \textbf{Question Format} \\ \midrule
		\textit{\textbf{R}} &  \begin{tabular}[c]{@{}l@{}}
			$Q_0$: On which date did the close price of the stock first reach $x$ yuan?\\
			$Q_1$: On which date did the highest price of the stock first reach $x$ yuan?\\
			$Q_2$: On which date did the volume of the stock first reach $x$ lots?\\
			$Q_3$: On which date did the value of the stock first reach $x$ thousand yuan?\\
			$Q_4$: On which date did the price change rate of the stock first reach $x$\%?\\
			$Q_5$: On which date did the price change of the stock first reach $x$ yuan?\\
		\end{tabular} \\ \midrule
		\textit{\textbf{C}} &  \begin{tabular}[c]{@{}l@{}}
			\begin{tabular}[c]{@{}l@{}}
				$Q_6$: On which date did the stock have the highest turnover rate when the close \\price was greater than $x$ yuan?
			\end{tabular}\\
			
			\begin{tabular}[c]{@{}l@{}}
				$Q_7$: On which date did the stock have the highest quantity relative ratio when \\the open price was less than $x$ yuan?
			\end{tabular}\\
			
			\begin{tabular}[c]{@{}l@{}}
				$Q_8$: On which date did the stock have the highest difference between the highest \\and lowest prices when the trading volume exceeded $x$ lots?
			\end{tabular}\\
			
			\begin{tabular}[c]{@{}l@{}}
				$Q_9$: On which date did the stock record the highest daily average price, calculated \\as 'value' divided by 'volume,' when the PE ratio was less than $x$?
			\end{tabular}\\
			
			\begin{tabular}[c]{@{}l@{}}
				$Q_{10}$: Among the top-$k$ trading value days, on which date did the stock have the \\lowest close price?
			\end{tabular}\\
			
			\begin{tabular}[c]{@{}l@{}}
				$Q_{11}$: When the quantity relative ratio exceeded $x$, on which date did the stock \\have the highest sum of the open price and close price?
			\end{tabular}\\
			
			\begin{tabular}[c]{@{}l@{}}
				$Q_{12}$: When the absolute price change rate exceeded $x$\%, on which date did the \\stock have the highest difference between the highest and lowest prices?
			\end{tabular}
		\end{tabular} \\ \midrule
		\textit{\textbf{S}} &  \begin{tabular}[c]{@{}l@{}}
			\begin{tabular}[c]{@{}l@{}}
				$Q_{13}$: How many days had a volume greater than $x$ from $date_1$ to $date_2$?
			\end{tabular}\\
			
			\begin{tabular}[c]{@{}l@{}}
				$Q_{14}$: How many days had the close price higher than the open price from \\$date_1$ to $date_2$?
			\end{tabular}\\
			
			\begin{tabular}[c]{@{}l@{}}
				$Q_{15}$: How many days had a close price higher than the open price, with the \\quantity relative ratio exceeding $x$\%?
			\end{tabular}\\
			
			\begin{tabular}[c]{@{}l@{}}
				$Q_{16}$: How many days had the close price reach $x$ yuan with the absolute price \\change rate exceeding $x$\%?
			\end{tabular}\\
			
			\begin{tabular}[c]{@{}l@{}}
				$Q_{17}$: What was the average trading volume when both the turnover rate \\exceeded $x$\% and the price change rate was greater than $y$\%?
			\end{tabular}\\
			
			\begin{tabular}[c]{@{}l@{}}
				$Q_{18}$: Excluding non-trading days, how many times did the open price of \\the stock rise for three or more consecutive days?
			\end{tabular}\\
			
			\begin{tabular}[c]{@{}l@{}}
				$Q_{19}$: Excluding non-trading days, how many times did the close price of \\the stock rise for three or more consecutive days?
			\end{tabular}\\
			
			\begin{tabular}[c]{@{}l@{}}
				$Q_{20}$: Excluding non-trading days, how many times did the open price and \\close price of the stock both rise for three or more consecutive days?
			\end{tabular}\\
		
			\begin{tabular}[c]{@{}l@{}}
			$\cdots \cdots$
		\end{tabular}
		
		\end{tabular} \\ \bottomrule
	\end{tabular}
\label{appx:stock_question}
\end{table*}
\clearpage
 

\begin{table*}[]
	\centering
	\caption{Question format on weather dataset.  R: contextual retrieval, C: comparison, S: summary. The value of $x$ falls within the minimum and maximum ranges of its respective attribute. $date_1$ represents the day at the twentieth percentile of the stock history, while $date_2$ represents the day at the eightieth percentile.}
	\begin{tabular}{c|c}
		\toprule
		\textbf{Ability}    & \textbf{Question Format} \\ \midrule
		\textit{\textbf{R}} &  \begin{tabular}[c]{@{}l@{}}
			$Q_0$: On which date did the temperature at two meters first reach $x$°C?\\
			$Q_1$: On which date did the relative humidity at two meters first exceed $x$\%?\\
			$Q_2$: On which date did the dew point temperature at two meters first drop below $x$°C?\\
			$Q_3$: On which date did the precipitation first exceed $x$ mm?\\
			$Q_4$: On which date did the sea-level air pressure first exceed $x$ hPa?\\
			$Q_5$: On which date did the cloud cover first reach $x$\%?\\
			$Q_6$: On which date did the wind speed at 10 meters first exceed $x$ m/s?
		\end{tabular} \\ \midrule
		\textit{\textbf{C}} &  \begin{tabular}[c]{@{}l@{}}
			\begin{tabular}[c]{@{}l@{}}
				$Q_7$: On which date did the temperature at two meters reach its highest value \\ 
				when the relative humidity was below $x$\%? 
			\end{tabular} \\
			
			\begin{tabular}[c]{@{}l@{}}
				$Q_8$: On which date did the relative humidity at two meters reach its lowest value \\ 
				when the temperature at two meters was above $x^\circ$C?
			\end{tabular} \\
			
			\begin{tabular}[c]{@{}l@{}}
				$Q_9$: On which date did the difference between the temperature and dew point \\ 
				at two meters reach its maximum when the cloud cover was below $x$\%? 
			\end{tabular} \\
			
			\begin{tabular}[c]{@{}l@{}}
				$Q_{10}$: On which date did the precipitation reach its highest value \\ 
				when the temperature at two meters was below $x^\circ$C? 
			\end{tabular} \\
			
			\begin{tabular}[c]{@{}l@{}}
				$Q_{11}$: On which date did the cloud cover reach its lowest value \\ 
				when the wind speed at 10 meters exceeded $x$ m/s? 
			\end{tabular} \\
			
			\begin{tabular}[c]{@{}l@{}}
				$Q_{12}$: On which date did the wind speed at 10 meters reach its highest value \\ 
				when the precipitation exceeded $x$ mm? 
			\end{tabular} \\
			
			\begin{tabular}[c]{@{}l@{}}
				$Q_{13}$: On which date did the sea-level air pressure reach its highest value \\ 
				when the cloud cover was below $x$\%? 
			\end{tabular}
		\end{tabular} \\ \midrule
		\textit{\textbf{S}} &  \begin{tabular}[c]{@{}l@{}}
			\begin{tabular}[c]{@{}l@{}}
				$Q_{14}$: How many days had a temperature at two meters greater than $x^\circ$C \\from $date_1$ to $date_2$? 
			\end{tabular} \\
			
			\begin{tabular}[c]{@{}l@{}}
				$Q_{15}$: How many days had a relative humidity at two meters exceeding $x$\% \\from $date_1$ to $date_2$? 
			\end{tabular} \\
			
			\begin{tabular}[c]{@{}l@{}}
				$Q_{16}$: How many days had a precipitation greater than $x$ mm from $date_1$ \\to $date_2$? 
			\end{tabular} \\
			
			\begin{tabular}[c]{@{}l@{}}
				$Q_{17}$: What was the average temperature at two meters when the relative \\humidity exceeded $x$\%? 
			\end{tabular} \\
			
			\begin{tabular}[c]{@{}l@{}}
				$Q_{18}$: What was the average wind speed at 10 meters when the precipitation \\exceeded $x$ mm? 
			\end{tabular} \\
			
			\begin{tabular}[c]{@{}l@{}}
				$Q_{19}$: How many times did the temperature at two meters rise for three or more \\consecutive days? 
			\end{tabular} \\
			
			\begin{tabular}[c]{@{}l@{}}
				$Q_{20}$: How many times did the relative humidity at two meters drop for \\three or more consecutive days? 
			\end{tabular} \\
		
					
		\begin{tabular}[c]{@{}l@{}}
			$\cdots \cdots$
		\end{tabular} \\
	
		\end{tabular} \\ \bottomrule
	\end{tabular}
\label{appx:weather_question}
\end{table*}

\clearpage
\begin{figure*}[t]
	\centering	
	\vspace{-1em}
	\frame{
		\includegraphics[width = 0.9\textwidth]{image/intro_example/number_compare.png}
	}
	%	\captionsetup{labelformat=empty}
	%	\addtocounter{figure}{-1}
	\caption{Number comparisons on GPT-4o. The correct answer is -9.11. }
	\label{fig:number_compare}
\end{figure*}

\begin{figure*}[t]
	\centering	
	\vspace{-1em}
	\frame{
		\includegraphics[width = 0.9\textwidth]{image/intro_example/multiplication.png}
	}
	%	\captionsetup{labelformat=empty}
	%	\addtocounter{figure}{-1}
	\caption{Number multiplication on GPT-4o. The correct answer is 102244.12. }
	\label{fig:multiplication}
\end{figure*}
\begin{figure*}[t]
	\centering	
	\vspace{-1em}
	\frame{
		\includegraphics[width = 0.9\textwidth]{image/intro_example/number_counting.jpg}
	}
	%	\captionsetup{labelformat=empty}
	%	\addtocounter{figure}{-1}
	\caption{Number counting on GPT-4o, which is required directly give answer. The correct answer is 4. }
	\label{fig:number_couting}
\end{figure*}

\clearpage


%\subsection{Additional Experiment Results}
%\subsubsection{Additional results on context length evaluation for stock and weather data}\label{appx:sssec:length}
		\begin{figure*}[t]
		\centering 	
		\subfloat[Contextual Retrieval]	
		{\centering\includegraphics[width=0.33\linewidth]{image/main_fig/retrieval-stock.pdf}}
		\hfill
		\subfloat[Comparison]
		{\centering\includegraphics[width=0.33\linewidth]{image/main_fig/compare-stock.pdf}}
		\subfloat[Summary]	
		{\centering\includegraphics[width=0.33\linewidth]{image/main_fig/summary-stock.pdf}}
		\hfill
		%	\subfloat[MUTAG]
		%	{\centering\includegraphics[width=0.25\linewidth, height=3.05cm]{image/g1-4.pdf}}	
		%	\hfill
		%	
		\caption{Evaluation on short and long context on stock dataset. Due to the input sequence length limit of Qwen2.5-72B-Inst on the API platform, the long dataset of all three abilities cannot be evaluated using this model.}
		\label{fig:length_stock}
	\end{figure*}
	
	
	
	\begin{figure*}[t]
		\centering 	
		\subfloat[Contextual Retrieval]	
		{\centering\includegraphics[width=0.33\linewidth]{image/main_fig/retrieval-weather.pdf}}
		\hfill
		\subfloat[Comparison]
		{\centering\includegraphics[width=0.33\linewidth]{image/main_fig/compare-weather.pdf}}
		\subfloat[Summary]	
		{\centering\includegraphics[width=0.33\linewidth]{image/main_fig/summary-weather.pdf}}
		\hfill
		%	\subfloat[MUTAG]
		%	{\centering\includegraphics[width=0.25\linewidth, height=3.05cm]{image/g1-4.pdf}}	
		%	\hfill
		%	
		\caption{Evaluation on short and long context on weather dataset. Due to the input sequence length limit of Qwen2.5-72B-Inst on the API platform, the long dataset of all three abilities cannot be evaluated using this model.}
		\label{fig:length_weather}
		
	\end{figure*}

		\begin{figure*}[t]
		
		\centering 	
		\subfloat[Contextual  Retrieval]	
		{\centering\includegraphics[width=0.33\linewidth]{image/noisy_dataset_fig/retrieval-noisy-weather.pdf}}
		\hfill
		\subfloat[Comparison]
		{\centering\includegraphics[width=0.33\linewidth]{image/noisy_dataset_fig/compare-noisy-weather.pdf}}
		\subfloat[Summary]	
		{\centering\includegraphics[width=0.33\linewidth]{image/noisy_dataset_fig/summary-noisy-weather.pdf}}
		\hfill
		%	\subfloat[MUTAG]
		%	{\centering\includegraphics[width=0.25\linewidth, height=3.05cm]{image/g1-4.pdf}}	
		%	\hfill
		%	
		\caption{Evaluation on  noisy weather dataset. Due to the input sequence length limit of Qwen2.5-72B-Inst on the API platform, the data containing 4 and 6 irrelevant attributes cannot be evaluated using this model.}
		\label{fig:noisy_weather}
		
	\end{figure*}
\clearpage
	
	
	
	
%	\begin{figure*}[t]
%		
%		\centering 	
%		\subfloat[Context Retrieval]	
%		{\centering\includegraphics[width=0.33\linewidth]{image/multi_dataset_fig/retrieval-multi-stock.pdf}}
%		\hfill
%		\subfloat[Comparison]
%		{\centering\includegraphics[width=0.33\linewidth]{image/multi_dataset_fig/compare-multi-stock.pdf}}
%		\subfloat[Summary]	
%		{\centering\includegraphics[width=0.33\linewidth]{image/multi_dataset_fig/summary-multi-stock.pdf}}
%		\hfill
%		%	\subfloat[MUTAG]
%		%	{\centering\includegraphics[width=0.25\linewidth, height=3.05cm]{image/g1-4.pdf}}	
%		%	\hfill
%		%	
%		\caption{Evaluation on multi-turn QA on stock dataset. Due to the input sequence length limit of Qwen2.5-72B-Inst on the API platform, the model cannot generate outputs in the third turn of the conversation. }
%		\label{fig:multurn_stock}
%		
%	\end{figure*}
	
	\begin{table*}[]
		\caption{The average token number on short and long instances for each data.}
		\centering
		\begin{tabular}{c|c|cc|cc}
			\toprule
			\multirow{2}{*}{\textbf{Dataset}}                                               & \multirow{2}{*}{\textbf{Ability}} & \multicolumn{2}{c|}{\textbf{Short}}                            & \multicolumn{2}{c}{\textbf{Long}}                             \\ \cmidrule{3-6} 
			
			&                                   & \multicolumn{1}{c|}{\textbf{\# Instance}} & \textbf{Avg Token} & \multicolumn{1}{c|}{\textbf{\# Instance}} & \textbf{Avg Token} \\ \midrule
			
			\multirow{3}{*}{\textbf{\begin{tabular}[c]{@{}c@{}}Number\\ List\end{tabular}}} & \textit{Contextual Retrieval}

                  & \multicolumn{1}{c|}{500}                     &        809.12     & \multicolumn{1}{c|}{500}                     &         6599.34      \\   
			
			& \textit{Comparison}                        & \multicolumn{1}{c|}{500}                     &     804.86     & \multicolumn{1}{c|}{500}                     &        6566.27      \\ 
			
			& \textit{Summary}



                  & \multicolumn{1}{c|}{500}                     &       822.49      & \multicolumn{1}{c|}{500}                     &       6487.07       \\ \midrule
			
			\multirow{3}{*}{\textbf{Stock}}                                                 & \textit{Contextual Retrieval}

                  & \multicolumn{1}{c|}{500}                     &        18529.07      & \multicolumn{1}{c|}{500}                     &      36641.63     \\  
			& \textit{Comparison}                        & \multicolumn{1}{c|}{500}                     &    18539.58     & \multicolumn{1}{c|}{500}                     &      36651.22      \\ 
			& \textit{Summary}

                  & \multicolumn{1}{c|}{500}                     &      18504.51      & \multicolumn{1}{c|}{500}                     &       36618.07      \\ \midrule
			
			\multirow{3}{*}{\textbf{Weather}}                                               & \textit{Contextual Retrieval}

                  & \multicolumn{1}{c|}{500}                     &        18362.38        & \multicolumn{1}{c|}{500}                     &        36356.13    \\  
			& \textit{Comparison}                        & \multicolumn{1}{c|}{500}                     &        18371.11    & \multicolumn{1}{c|}{500}                     &       36365.27     \\ 
			& \textit{Summary}



                  & \multicolumn{1}{c|}{500}                     &        18334.15     & \multicolumn{1}{c|}{500}                     &        36328.27    \\ \bottomrule
		\end{tabular}
		\label{tab:data_stat_short_long}
	\end{table*}
	\clearpage
%	\begin{figure*}[t]
%		
%		\centering 	
%		\subfloat[Context Retrieval]	
%		{\centering\includegraphics[width=0.33\linewidth]{image/multi_dataset_fig/retrieval-multi-weather.pdf}}
%		\hfill
%		\subfloat[Comparison]
%		{\centering\includegraphics[width=0.33\linewidth]{image/multi_dataset_fig/compare-multi-weather.pdf}}
%		\subfloat[Summary]	
%		{\centering\includegraphics[width=0.33\linewidth]{image/multi_dataset_fig/summary-multi-weather.pdf}}
%		\hfill
%		%	\subfloat[MUTAG]
%		%	{\centering\includegraphics[width=0.25\linewidth, height=3.05cm]{image/g1-4.pdf}}	
%		%	\hfill
%		%	
%		\caption{Evaluation on multi-turn QA on weather dataset.}
%		\label{fig:multurn_weather}
%		
%	\end{figure*}
	

	
	
%\noindent\textbf{Case Study}
	\newtcolorbox[auto counter, number within=section]{examplebox}[2][]{%
		colback=white,        % Background color
		colframe=exampleborder, % Border color
		coltitle=white,       % Header text color
		fonttitle=\bfseries,  % Header font
		colbacktitle=exampleblue, % Header background color
		title=Example~\thetcbcounter: #2, % Title format
	}
	\begin{figure*}
		\begin{examplebox}{InternLM2-Math-7B}
			\textbf{Failure:} \\Model repeatedly output the same number until the output sequence ends.
			
			\vspace{0.2cm}
			\textbf{User:} \\ Question: What is the maximum sum of any three consecutive items in the list? Options: A: 21501.9737231253, B: 21502.8144848529, C: 21501.9642201794, D: 21502.7354388437, E: 21502.0327435643, F: 21501.7526575621, G: 21501.7057960155, H: 21502 Data: [-5279, ....(100 numbers)]
			
			\vspace{0.3cm}
			\textbf{InternLM2-Math-7B Model:} \\
			Step 1: To find the maximum sum of any three consecutive items in the list, we need to identify the three largest numbers in the list and add them together. <PRM +> \\Step 2: We can start by sorting the list in ascending order to make it easier to identify the largest numbers. <PRM +> \\Step 3: After sorting, the list becomes: [-8509, -8509, -8509, -8509, -8509, -8509, -8509, -8509, -8509, -8509, -8509, -8509, -8509, -8509, -8509, -8509, -8509, -8509, -8509, -8509, -8509, -8509, -8509, -8509, -8509, -8509, -8509, -8509...(output too long)
		\end{examplebox}
		\caption{Failure type and conversation record of InternLM2-Math-7B. The model constantly generate the same "-8509" text until the output sequence was too long and truncated, leaving the problem unsolved.}
		\label{fig:fail_internlm}
	\end{figure*}
	
	\begin{figure*}
		\begin{examplebox}{DeepSeek-Math-7B}
				\textbf{Failure:} \\ Model outputs sequence without meaning, probably due to overly long input sequences. 
				
				\vspace{0.2cm}
				\textbf{User:} \\Question: What is the sum of the index of top 300 smallest numbers in the list? Options: A: 146734, B: 146738, C: 146733, D: 146724, E: 146728, F: 146727, G: 146721, H: 146729 Data: [3640, 1560, ...(100 numbers)]
				
				\vspace{0.3cm}
				\textbf{Deepseek-Math-7B Model:} \\
				31,3,9,3,1,1,3,3,3,3,3,3,3,3, 3,3,1,3,3,3,1,3,3,3, 3, 3,3,3, 1, 1,3,3, 3,1,3, 3,3,3, 1, 3,3,3, 3,3,1,3,1,3,91, 3,3,3, 3,31,3,3,3, 3, 339, 3, 3, 311, 1, 3, 1,3,3, 3,3,6,3,3, 3,31,3,3,1, 1, 1, 1, 3,1,1, 3,3, 1,1, 3,1,3,1, 1, 1, 1, 1, 19,1, 1, 1,191,1, 1,1, 19, 31,1,1, 1, 3,1, 1,3, 1133,3, 1, 1,1, 11, 11,3, 31, 3, 3,1,1,3,1, 1,33, 11,3,1911,3, 1,13,3131311119191,1, 3,1,11,1,19,11,1,1,1....(keeps outputting numbers)
		\end{examplebox}
		\caption{Failure type and conversation record of DeepSeek-Math-7B. The model only generates a series of numbers throughout its output, and we're unable to acquire any useful information from the output.}
		\label{fig:fail_ds_math}
	\end{figure*}	

	\begin{figure*}
		\begin{examplebox}{Metamath-Llemma-7B}
			\textbf{Failure:}\\ Model outputs ineffective chain of thoughts, and giving the wrong answer in the wrong format(ground truth is B))
			
			\vspace{0.2cm}
			\textbf{User:}\\ Question: What is the sum of the index of top 30 smallest numbers in the list? Options: A: 1566, B: 1572, C: 1575, D: 1573, E: 1577, F: 1564, G: 1569, H: 1567 Data: [-2568.1542995196, 1035.3645340299...(100 numbers)]
			
			\vspace{0.3cm}
			\textbf{Metamath-Llemma-7B Model:} \\ The data you receive is a list of numbers. To find the sum of the index of the top 30 smallest numbers, we need to find the 30 smallest numbers in the list and then calculate their sum. To find the 30 smallest numbers, we can use the min function in Python. The min function takes a list as input and returns the smallest number in the list. We can use the min function to find the 30 smallest numbers in the list. Now, we need to calculate the sum of these 30 smallest numbers. To do this, we can use the sum function in Python. The sum function takes a list as input and returns the sum of all the numbers in the list. We can use the sum function to calculate the sum of the 30 smallest numbers. Finally, we need to find the index of the sum. To do this, we can use the index function in Python. The index function takes a number as input and returns the index of that number in the list. We can use the index function to find the index of the sum. Therefore, the answer is (correct option). \#\#\#\# 1566 The answer is: 1566
		\end{examplebox}
		\caption{Failure type and conversation record of Metamath-Llemma-7B. The model attempts to use programming language to solve the problem, but it does not provide executable lines of code, only providing vague description of certain functions instead. After outputting the vague solution, the model outputs an answer, which is different from the ground truth and not in the expected format(We expect the model to provide the answer with a capital letter corresponding to the correct option).}
		\label{fig:fail_llemma}
	\end{figure*}

	\begin{figure*}
	\begin{examplebox}{MAmmoTH-13B}
		\textbf{Failure:}\\ Model outputs ineffective chain of thoughts, and giving the wrong answer in the wrong format(ground truth is B))
		
		\vspace{0.2cm}
		\textbf{User:}\\ Question: What is the sum of the index of top 30 smallest numbers in the list? Options: A: 1566, B: 1572, C: 1575, D: 1573, E: 1577, F: 1564, G: 1569, H: 1567 Data: [-2568.1542995196, 1035.3645340299...(100 numbers)]
		
		\vspace{0.3cm}
		\textbf{MAmmoTH-13B Model:}\\  Notice that there are two numbers in your list that have the smallest index: 4346.5174909575 and 8169.6394989672. They have the smallest index of 4346 and 8169 respectively. Since in the options the first number is the index of the smallest number, then the correct option is D.\\             Option: D
	\end{examplebox}
	\caption{Failure type and conversation record of Mammoth-13B. The model fail to understand the problem's instruction, and does not provide reasonable output. Moreover, it provides the wrong answer in incorrect format, just like Metamath-Llemma-7B(We expect the model to provide the answer with a capital letter corresponding to the correct option).}
	\label{fig:fail_mammoth}
	\end{figure*}

\end{document}
