% 
% 
The widespread integration of communication networks and smart devices in modern control systems has increased the vulnerability of industrial systems to online cyber-attacks, e.g., Industroyer, Blackenergy, etc \citep{osti_1505628}.
% Modern control systems have seen a large push to include communication networks and smart devices to increase performance, made possible by improvements in communication device cost and energy consumption. This trend has been coupled with the usage of open-standard communication protocols among industrial control systems, making them vulnerable to online cyber-attacks such as Industroyer, Blackenergy, etc \citep{osti_1505628}. 
To counter this, methods have been developed to improve security by achieving attack detection, mitigation, and monitoring, among others \citep{sandberg2022secure}. This paper focuses on active attack diagnosis to mitigate stealthy attacks. 
%
%\subsection{Literature review}

Active diagnosis techniques rely on the inclusion of additional moduli to control systems
% inclusion within the control system of additional moduli 
to alter the behavior of the system compared to information known by the attacker. 
For instance, the concept of additive watermarking was introduced in \cite{mo2015physical}, where noise signals of known mean and variance are added at the plant and compensated for it at the controller. 
This compensation, however, is not exact, causing some performance degradation. Thus, trade-offs between performance and detectability  are necessary \citep{zhu2023detection}.
% A later work \citep{zhu2023detection} designs the watermark signal by trading performance for detection. Thus, although additive watermarking serves as a good detection scheme, they endure performance losses even in the nominal case. 

In encrypted control \citep{darup2021encrypted}, the sensor data is encrypted, sent to the controller, and then operated on directly. Encrypted input signals are sent back to the plant for decryption. Although encryption is widespread in IT security, in control systems it presents some concerns, such as the introduction of time delays \citep{stabile2024verifiable}, while it may present inherent weaknesses \citep{alisic2023model}.
% they are not preferred as they introduce time delays \citep{stabile2024verifiable} which can cause instability, and some encryption schemes can be very weak  \citep{alisic2023model}. 

In moving target defense \citep{griffioen2020moving}, the plant is augmented with fictitious dynamics, known to the controller. The plant output is transmitted to the controller along with the fictitious states over a network under attack. 
The additional measurements then aide in the detection of attacks. 
This comes at the cost of higher communication bandwidth needs, which increases rapidly with the dimension of the augmented systems.
% Since the dynamics of the fictitious dynamics are exactly known to the controller, the attack is detected easily. However, when the scale of the system increases, the communication bandwidth used by moving the target defense approach increases rapidly. 

Other recently proposed works include two-way coding \citep{fang2019two}, a weak encryuption technique, and dynamic masking \citep{abdalmoaty2023privacy}, which enhances privacy as well as security, have been shown to be effective against zero-dynamics attacks.
% Two-way coding \citep{fang2019two} and dynamic masking \citep{abdalmoaty2023privacy} are other recently proposed approaches. Two-way coding is another form of weak encryption technique whilst dynamic masking proposes an architecture that enhances both privacy and security. These schemes are shown to be effective against zero dynamics attacks but remain to be studied for other classes of attacks. 
% Recent extensions include \citep{mukherjee2021secure,ramos2024privacy}.
% Some other works which are related are \citep{mukherjee2021secure}, an extension of \cite{fang2019two}. The work \citep{ramos2024privacy} is an extension of moving target defense for multi-agent systems. 
Furthermore, filtering techniques for attack detection are proposed by \cite{murguia2020security,hashemi2022codesign,escudero2023safety}, while not focusing on stealthy attacks.
% The works \citep{murguia2020security,hashemi2022codesign,escudero2023safety} develop filtering techniques to guarantee safety, without being focused on stealthy covert attacks.

Multiplicative watermarking (mWM) has been proposed by the authors as a diagnosis technique \citep{ferrari2020switching}. mWM consists of a pair of filters on each communication channel between the plant and its controller; the scheme is affine to weak encryption, whereby ``encoding'' and ``decoding'' are done by changing signals' dynamic characteristics through inverse pairs of filters. This enables original signals to be recovered exactly, and thus does not lead to performance degradation.
% A multiplicative watermark is an affine to a weak encryption technique, through which the signal is ``encoded'' by a filter, changing its dynamic behavior. The use of inverse pairs means that the original signal can be recovered, through ``decoding'' via an inverse filter. As such, differently to techniques based on additive watermarking, no performance is lost due to the injection of noise, and there are no bandwidth limitations.

%\subsection{Contributions}
One of the critical features of multiplicative watermarking is that to detect stealthy attacks, the mWM filter parameters must be switched over time. In this paper, an algorithm to optimally design the mWM parameters after a switching event is presented, enhancing detection performance, without changing the switching time.
% This is done without changing the switching time, which is taken as given.

\textcolor{black}{
To formalize the filter design problem, we suppose the defender is interested in optimal performance against adversaries injecting covert attacks with matched system parameters \citep{smith2015covert}, including the mWM parameters prior to the switch. This scenario represents a worst case where malicious agents can take full control of the system while remaining undetected.
Thus, the attack strategy is explicitly included within the formulation of the closed-loop system, and the mWM filters are chosen by solving an optimization problem minimizing the attack-energy-constrained output-to-output gain (AEC-OOG) \citep{anand2023risk}, a variation of the output-to-output gain proposed in  \cite{teixeira2015strategic}.
}
The main contributions of this paper are:
% We consider an adversary injecting a covert attack with matched system parameters \citep{smith2015covert}, i.e., an attacker with full knowledge of the control system parameters, including those of the mWM filters before the switch. This scenario is taken as a worst case, as it has been shown that this class of attacks can be made stealthy. To quantitatively define a cost, the output-to-output gain (OOG) \citep{teixeira2015strategic} is leveraged,
% a metric introduced to evaluate the impact of an additive attack in a control system. %Specifically, OOG evaluates the worst-case performance loss that an attacker injecting an undetectable attack can obtain. 
% Here, the maximum performance loss caused by a stealthy adversary with limited energy is taken, the attack-energy-constrained OOG (AEC-OOG) \citep{anand2023risk}. The main contributions of this paper are:
\begin{enumerate}
%[label=\alph*.]
\item The problem of optimally designing the switching mWM filters is formulated as an optimization problem, with the AEC-OOG is taken as the objective;%where the AEC-OOG is taken as the impact metric; 
\item The worst-case scenario of a covert attack with exact knowledge of plant and mWM filter parameters is embedded within the design problem;
% The optimization problem is defined to incorporate the worst-case scenario of a covert attack with exact knowledge of plant and mWM filter parameters;
\item The feasibility of the optimization problem is shown to be dependent only on stability conditions; 
\item A solution scheme is proposed to promote randomization of the mWM filter parameters such that an eavesdropping adversary cannot remain stealthy.
\end{enumerate} 

This builds on the results of \cite{ferrari2020switching}, where the focus was on the design of the switching protocols, rather than the parameters themselves.
Compared to previous work \citep{gallo2021design}, this paper introduces an optimization problem which is always feasible (thanks to the use of AEC-OOG in the objective), while also considering a more sophisticated class of covert attacks, where the presence of watermark is known to the adversary. 
Moreover, this paper poses a different objective than \citep{zhang2023hybrid}; indeed, while \citep{zhang2023hybrid} provided a design strategy to ensure certain privacy properties, in this paper we address the problem of optimal parameter design following a switching event.


%\subsection{Organization}
The rest of the paper is organized as follows. 
After formulating the problem in Section~\ref{sec:PF}, we propose our design algorithm in Section~\ref{sec:main}, and analyze its properties. It is then evaluated through a numerical example in Section~\ref{sec:NE}, and concluding remarks are given Section~\ref{sec:Con}.
% We provide the problem background in Section~\ref{sec:PF}. We formulate the design problem in Section~\ref{sec:main}, together with an analysis of its properties. The proposed algorithm is evaluated through a numerical example in Section \ref{sec:NE}. Concluding remarks are offered in Section \ref{sec:Con}.