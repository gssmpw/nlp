% autosam.tex
% Annotated sample file for the preparation of LaTeX files
% for the final versions of papers submitted to or accepted for 
% publication in AUTOMATICA.

% See also the Information for Authors.

% Make sure that the zip file that you send contains all the 
% files, including the files for the figures and the bib file.

% Output produced with the elsart style file does not imitate the
% AUTOMATICA style. The style file is generic for all Elsevier
% journals and the output is laid out for easy copy editing. The
% final document is produced from the source file in the
% AUTOMATICA style at Elsevier.

% You may use the style file autart.cls to obtain a two-column 
% document (see below) that more or less imitates the printed 
% Automatica style. This may helpful to improve the formatting 
% of the equations, tables and figures, and also serves to check 
% whether the paper satisfies the length requirements.

% Please note: Authors must not create their own macros.

% For further information regarding the preparation of LaTeX files 
% for Elsevier, please refer to the "Full Instructions to Authors" 
% from Elsevier's anonymous ftp server on ftp.elsevier.nl in the
% directory pub/styles, or from the internet (CTAN sites) on
% ftp.shsu.edu, ftp.dante.de and ftp.tex.ac.uk in the directory
% tex-archive/macros/latex/contrib/supported/elsevier.


%\documentclass{elsart}               % The use of LaTeX2e is preferred.

\documentclass[twocolumn]{autart}    % Enable this line and disable the 
                                     % preceding line to obtain a two-column 
                                     % document whose style resembles the
                                     % printed Automatica style.


\usepackage{graphicx}          % Include this line if your 
                               % document contains figures,
%\usepackage[dvips]{epsfig}    % or this line, depending on which
                               % you prefer.       
% \usepackage{amssymb}
% \usepackage{amsmath}
% \usepackage{enumitem}
% \usepackage{comment}
% \usepackage{mathtools}
% \usepackage{cuted}
% \usepackage{textcomp}
% \mathtoolsset{showonlyrefs}


\usepackage{cite}
\usepackage{amsmath,amssymb,amsfonts}
\usepackage{algorithmic}
\usepackage{graphicx}
\usepackage{cuted}
\usepackage{flushend}
\usepackage{mathtools}
\usepackage{array}
\usepackage{subcaption}
\mathtoolsset{showonlyrefs}
\usepackage{multirow}
\usepackage{multicol}
\usepackage{xcolor}
\usepackage{braket}
\usepackage{floatrow}
\usepackage{kantlipsum}
\usepackage{enumitem}
\usepackage{hyphenat}
\usepackage{enumitem}
\usepackage{wrapfig}

% \usepackage{algorithm}
% \usepackage{algpseudocode}
\floatsetup[table]{capposition=top}


\newtheorem{theorem}{Theorem}[section]
\newtheorem{definition}{Definition}[section]
\newtheorem{assumption}{Assumption}[section]
\newtheorem{corollary}{Corollary}[theorem]
\newtheorem{lemma}[theorem]{Lemma}
\newtheorem{remark}{Remark}
\newtheorem{problem}{Problem}
\DeclarePairedDelimiter{\ceil}{\lceil}{\rceil}

%%%%% Define symbols for estimates - we may want to change them, and this could make it easier
\newcommand{\est}[2]{\tilde{#1}_{[#2]}}
\newcommand{\err}[1]{\epsilon_{[#1]}}
\newcommand{\res}[1]{r_{[#1]}}
\newcommand{\thr}[2]{\bar{#1}_{[#2]}}

\newcommand{\E}[1]{\mathbf{E}\left[{#1}\right]}
\newcommand{\Evar}[1]{\mathbf{E}\left[{#1}{#1}^\top\right]}


% \definecolor{ajgCol}{rgb}{0.858, 0.188, 0.478}
\definecolor{ajgCol}{rgb}{0,0,0}
\newcommand{\ajg}[1]{{\color{ajgCol}#1}}

\definecolor{scaCol}{rgb}{0,0,0}
\newcommand{\sca}[1]{{\color{scaCol}#1}}

\newcommand{\PP}{\mathcal{P}}
\newcommand{\CC}{\mathcal{C}}
\newcommand{\WW}{\mathcal{W}}
\newcommand{\QQ}{\mathcal{Q}}
\newcommand{\HH}{\mathcal{H}}
\newcommand{\GG}{\mathcal{G}}

\hyphenation{con-strained}
\usepackage[round]{natbib}        % required for bibliography
\bibliographystyle{abbrvnat} % Choose a bibliography style that supports author-year
% \renewcommand{\cite}{\citep}
\allowdisplaybreaks


\begin{document}

\begin{frontmatter}
%\runtitle{Insert a suggested running title}  % Running title for regular 
                                              % papers but only if the title  
                                              % is over 5 words. Running title 
                                              % is not shown in output.

\title{Switching Multiplicative Watermark Design \\ Against Covert Attacks\thanksref{footnoteinfo}} % Title, preferably not more 
                                                % than 10 words.

\thanks[footnoteinfo]{This work has been partially supported by the Research Council of Norway through the project AIMWind (grant ID 312486), by the Swedish Research Council under the grant 2018-04396, and by the Swedish Foundation for Strategic Research. The material in this paper was partially presented at the 60th IEEE Conference on Decision and Control, Austin, Texas, 2021. Corresponding author Sribalaji. C. Anand. 
%Tel. +4618-471 7003.
}

\author[Paestum]{Alexander J. Gallo\thanksref{footnoteinfo2}}\ead{alexanderjulian.gallo@polimi.it},    % Add the 
\author[Rome]{Sribalaji C. Anand\thanksref{footnoteinfo2}}\ead{srca@kth.se},               % e-mail address 
\author[Baiae]{Andre M. H. Teixeira}\ead{andre.teixeira@it.uu.se},  % (ead) as shown
\author[Pompeii]{Riccardo M. G. Ferrari}\ead{r.ferrari@tudelft.nl}

\address[Paestum]{Department of Electronics, Information and Bioengineering, Politecnico di Milano, Milano, Italy.}  % Please supply                                              
\address[Rome]{School of Electrical Engineering and Computer Science and Digital Futures, KTH Royal Institute of Technology, Sweden}             % full addresses
\address[Baiae]{Department of Information Technology, Uppsala University, PO Box 337, SE-75105, Uppsala, Sweden.}        % here.

\address[Pompeii]{Delft Center for Systems and Control, Mechanical Engineering, TU Delft, Delft, Netherlands.}

\thanks[footnoteinfo2]{These authors contributed equally.}
          
\begin{keyword}                           % Five to ten keywords,  
Network security, Networked control systems, Fault detection and isolation 
%System security, Quadratic performance indices, Fault detection, $H_{\infty}$ control, Optimization.               % chosen from the IFAC 
\end{keyword}                             % keyword list or with the 
                                          % help of the Automatica 
                                          % keyword wizard


\begin{abstract}                          % Abstract of not more than 200 words.
\textit{Active techniques} have been introduced to give better detectability performance for cyber-attack diagnosis in cyber-physical systems (CPS). In this paper, switching multiplicative watermarking is considered, %where time-varying filters are defined to alter the dynamics of information transmitted over a communication channel between a plant and a controller in a CPS. 
% In this context, the objective of this paper is to 
whereby we propose an optimal design strategy to define switching filter parameters. 
Optimality is evaluated exploiting the so-called output-to-output gain of the closed loop system, including some supposed attack dynamics. 
A worst-case scenario of a matched covert attack is assumed, presuming that an attacker with full knowledge of the closed-loop system injects a stealthy attack of bounded energy. 
Our algorithm, given watermark filter parameters at some time instant, provides optimal next-step parameters.
Analysis of the algorithm is given, demonstrating its features, and demonstrating that through initialization of certain parameters outside of the algorithm, the parameters of the multiplicative watermarking can be randomized.
Simulation shows how, by adopting our method for parameter design, the attacker's impact on performance diminishes.
\end{abstract}

\end{frontmatter}

\section{Introduction}
% 
% 
The widespread integration of communication networks and smart devices in modern control systems has increased the vulnerability of industrial systems to online cyber-attacks, e.g., Industroyer, Blackenergy, etc \citep{osti_1505628}.
% Modern control systems have seen a large push to include communication networks and smart devices to increase performance, made possible by improvements in communication device cost and energy consumption. This trend has been coupled with the usage of open-standard communication protocols among industrial control systems, making them vulnerable to online cyber-attacks such as Industroyer, Blackenergy, etc \citep{osti_1505628}. 
To counter this, methods have been developed to improve security by achieving attack detection, mitigation, and monitoring, among others \citep{sandberg2022secure}. This paper focuses on active attack diagnosis to mitigate stealthy attacks. 
%
%\subsection{Literature review}

Active diagnosis techniques rely on the inclusion of additional moduli to control systems
% inclusion within the control system of additional moduli 
to alter the behavior of the system compared to information known by the attacker. 
For instance, the concept of additive watermarking was introduced in \cite{mo2015physical}, where noise signals of known mean and variance are added at the plant and compensated for it at the controller. 
This compensation, however, is not exact, causing some performance degradation. Thus, trade-offs between performance and detectability  are necessary \citep{zhu2023detection}.
% A later work \citep{zhu2023detection} designs the watermark signal by trading performance for detection. Thus, although additive watermarking serves as a good detection scheme, they endure performance losses even in the nominal case. 

In encrypted control \citep{darup2021encrypted}, the sensor data is encrypted, sent to the controller, and then operated on directly. Encrypted input signals are sent back to the plant for decryption. Although encryption is widespread in IT security, in control systems it presents some concerns, such as the introduction of time delays \citep{stabile2024verifiable}, while it may present inherent weaknesses \citep{alisic2023model}.
% they are not preferred as they introduce time delays \citep{stabile2024verifiable} which can cause instability, and some encryption schemes can be very weak  \citep{alisic2023model}. 

In moving target defense \citep{griffioen2020moving}, the plant is augmented with fictitious dynamics, known to the controller. The plant output is transmitted to the controller along with the fictitious states over a network under attack. 
The additional measurements then aide in the detection of attacks. 
This comes at the cost of higher communication bandwidth needs, which increases rapidly with the dimension of the augmented systems.
% Since the dynamics of the fictitious dynamics are exactly known to the controller, the attack is detected easily. However, when the scale of the system increases, the communication bandwidth used by moving the target defense approach increases rapidly. 

Other recently proposed works include two-way coding \citep{fang2019two}, a weak encryuption technique, and dynamic masking \citep{abdalmoaty2023privacy}, which enhances privacy as well as security, have been shown to be effective against zero-dynamics attacks.
% Two-way coding \citep{fang2019two} and dynamic masking \citep{abdalmoaty2023privacy} are other recently proposed approaches. Two-way coding is another form of weak encryption technique whilst dynamic masking proposes an architecture that enhances both privacy and security. These schemes are shown to be effective against zero dynamics attacks but remain to be studied for other classes of attacks. 
% Recent extensions include \citep{mukherjee2021secure,ramos2024privacy}.
% Some other works which are related are \citep{mukherjee2021secure}, an extension of \cite{fang2019two}. The work \citep{ramos2024privacy} is an extension of moving target defense for multi-agent systems. 
Furthermore, filtering techniques for attack detection are proposed by \cite{murguia2020security,hashemi2022codesign,escudero2023safety}, while not focusing on stealthy attacks.
% The works \citep{murguia2020security,hashemi2022codesign,escudero2023safety} develop filtering techniques to guarantee safety, without being focused on stealthy covert attacks.

Multiplicative watermarking (mWM) has been proposed by the authors as a diagnosis technique \citep{ferrari2020switching}. mWM consists of a pair of filters on each communication channel between the plant and its controller; the scheme is affine to weak encryption, whereby ``encoding'' and ``decoding'' are done by changing signals' dynamic characteristics through inverse pairs of filters. This enables original signals to be recovered exactly, and thus does not lead to performance degradation.
% A multiplicative watermark is an affine to a weak encryption technique, through which the signal is ``encoded'' by a filter, changing its dynamic behavior. The use of inverse pairs means that the original signal can be recovered, through ``decoding'' via an inverse filter. As such, differently to techniques based on additive watermarking, no performance is lost due to the injection of noise, and there are no bandwidth limitations.

%\subsection{Contributions}
One of the critical features of multiplicative watermarking is that to detect stealthy attacks, the mWM filter parameters must be switched over time. In this paper, an algorithm to optimally design the mWM parameters after a switching event is presented, enhancing detection performance, without changing the switching time.
% This is done without changing the switching time, which is taken as given.

\textcolor{black}{
To formalize the filter design problem, we suppose the defender is interested in optimal performance against adversaries injecting covert attacks with matched system parameters \citep{smith2015covert}, including the mWM parameters prior to the switch. This scenario represents a worst case where malicious agents can take full control of the system while remaining undetected.
Thus, the attack strategy is explicitly included within the formulation of the closed-loop system, and the mWM filters are chosen by solving an optimization problem minimizing the attack-energy-constrained output-to-output gain (AEC-OOG) \citep{anand2023risk}, a variation of the output-to-output gain proposed in  \cite{teixeira2015strategic}.
}
The main contributions of this paper are:
% We consider an adversary injecting a covert attack with matched system parameters \citep{smith2015covert}, i.e., an attacker with full knowledge of the control system parameters, including those of the mWM filters before the switch. This scenario is taken as a worst case, as it has been shown that this class of attacks can be made stealthy. To quantitatively define a cost, the output-to-output gain (OOG) \citep{teixeira2015strategic} is leveraged,
% a metric introduced to evaluate the impact of an additive attack in a control system. %Specifically, OOG evaluates the worst-case performance loss that an attacker injecting an undetectable attack can obtain. 
% Here, the maximum performance loss caused by a stealthy adversary with limited energy is taken, the attack-energy-constrained OOG (AEC-OOG) \citep{anand2023risk}. The main contributions of this paper are:
\begin{enumerate}
%[label=\alph*.]
\item The problem of optimally designing the switching mWM filters is formulated as an optimization problem, with the AEC-OOG is taken as the objective;%where the AEC-OOG is taken as the impact metric; 
\item The worst-case scenario of a covert attack with exact knowledge of plant and mWM filter parameters is embedded within the design problem;
% The optimization problem is defined to incorporate the worst-case scenario of a covert attack with exact knowledge of plant and mWM filter parameters;
\item The feasibility of the optimization problem is shown to be dependent only on stability conditions; 
\item A solution scheme is proposed to promote randomization of the mWM filter parameters such that an eavesdropping adversary cannot remain stealthy.
\end{enumerate} 

This builds on the results of \cite{ferrari2020switching}, where the focus was on the design of the switching protocols, rather than the parameters themselves.
Compared to previous work \citep{gallo2021design}, this paper introduces an optimization problem which is always feasible (thanks to the use of AEC-OOG in the objective), while also considering a more sophisticated class of covert attacks, where the presence of watermark is known to the adversary. 
Moreover, this paper poses a different objective than \citep{zhang2023hybrid}; indeed, while \citep{zhang2023hybrid} provided a design strategy to ensure certain privacy properties, in this paper we address the problem of optimal parameter design following a switching event.


%\subsection{Organization}
The rest of the paper is organized as follows. 
After formulating the problem in Section~\ref{sec:PF}, we propose our design algorithm in Section~\ref{sec:main}, and analyze its properties. It is then evaluated through a numerical example in Section~\ref{sec:NE}, and concluding remarks are given Section~\ref{sec:Con}.
% We provide the problem background in Section~\ref{sec:PF}. We formulate the design problem in Section~\ref{sec:main}, together with an analysis of its properties. The proposed algorithm is evaluated through a numerical example in Section \ref{sec:NE}. Concluding remarks are offered in Section \ref{sec:Con}.

\section{Problem description}\label{sec:PF}


We consider the Cyber-Physical System (CPS) in Fig.~\ref{fig:sys}. This includes plant $\PP$, controller and anomaly detector $\CC$, mWM filters $\WW,\QQ,\GG,\HH$, and the malicious agent $\mathcal{A}$. The mWM filters are defined pairwise, namely $\{\QQ,\WW\}$ and $\{\GG,\HH\}$ are referred to as, respectively, the output and input \textit{mWM filter pairs}. 
% In the following, we provide the description of the CPS, and define the problem.

\begin{figure}
    \centering
    \includegraphics[width=6.5cm]{NCS.eps}
    \caption{Block diagram of the closed-loop CPS including the plant $\PP$, controller $\CC$ and watermarking filters $\{\WW,\QQ,\GG,\HH\}$. The information transmitted between $\PP$ and $\CC$ is altered by the adversary $\mathcal A$. The dashed lines represent the network affected by the adversary.}
    \label{fig:sys}
\end{figure}

\subsection{Plant and controller}
Consider an LTI discrete-time (DT) plant modeled by: 
% whose dynamics are described as:
	\begin{equation}\label{eq:sys}
	    \PP: \left\{ \begin{aligned}
	        x_p[k+1] &= A_p x_p[k] + B_p u_h[k]\\
	        y_p[k] &= C_p x_p[k]\\
            y_J[k] &= C_J x_p[k] + D_J u_h[k]
	    \end{aligned}\right.
	\end{equation}
	where $x_p \in \mathbb R^n$ is the plant's state, $u_h \in \mathbb R^m$ its input, $y_p \in \mathbb R^p$ its measured output, and all the system's matrices are of the appropriate dimension. Furthermore, suppose a (possibly unmeasured) \textit{performance output} $y_J \in \mathbb{R}^{p_J}$ is defined, such that the performance of the system, evaluated over the interval $[k-N+1,k]$, for some $N \in \mathbb{N}$ \citep{zhou1996robust}, is given by:
	\begin{align}
	    J(x_p,u_h) &= %\| C_J x_p + D_J u_h \|^2_{\ell_2,[k-N+1,k]}= 
        \|y_J\|^2_{\ell_2,[k-N+1,k]}.
	\end{align}
	% Note that $y_J$ can be viewed as a \textit{virtual} output, i.e., it may not be measured directly. Next, we establish the following assumption.
    \begin{assumption}
    The tuples $(A_p,B_p)$ and $(C_p,A_p)$ are respectively, controllable and observable pairs.
		$\hfill\triangleleft$
	\end{assumption}
 \begin{assumption}\label{ass:stable}
     The plant $\mathcal{P}$ is stable and $x_p[0]=0$. $\hfill\triangleleft$
 \end{assumption}

Assumption~\ref{ass:stable}, necessary for the OOG to be meaningful \citep{teixeira2015secure}, does not reduce generality, as stability can be ensured by a local (non-networked) controller \citep{hu2007stability,lin2023secondary}, whilst $x_p[0] = 0$ can be considered because of linearity.

% \begin{remark}
%     Assumption~\ref{ass:stable} does not reduce generality, and is required for the output-to-output gain to be meaningful \citep{teixeira2015strategic}. The first statement can be achieved by introducing a two-level controller: a local feedback controller for stabilization, while a networked controller improves performance, changes setpoints, etc. \citep{hu2007stability,lin2023secondary}. The second statement follows from superposition.
%     $\hfill\triangleleft$
% \end{remark}

The plant is regulated by an observer-based dynamic controller $\CC$, described by:
\begin{equation}\label{eq:cntrl}
    \mathcal C: \left\{
	\begin{aligned}
		\hat{x}_p[k+1] &= A_p \hat{x}_p[k] + B_p u_c[k] + Ly_r[k]\\
		u_c[k] &= K\hat{x}_p[k]\\
		\hat{y}_p[k] &= C_p \hat{x}_p[k]\\
		y_r[k] &= y_q[k] - \hat{y}_p[k]
	\end{aligned}\right.
\end{equation}
where $\hat{x}_p \in \mathbb{R}^n, \hat{y}_p \in \mathbb{R}^p$ are the state and measurement estimates, $u_c \in \mathbb{R}^m$ the control input. The matrices $K$ and $L$ are the controller and observer gains respectively. Finally, the term $y_r$ in \eqref{eq:cntrl} is the residual output, used to detect the presence of an attack: given a threshold $\epsilon_r$, an attack is detected if the inequality $\|y_r\|_{\ell_2,[0,N_r]}^2 \leq \epsilon_r$ is falsified for any $N_r \in \mathbb{N}_+$. Note that in \eqref{eq:sys}-\eqref{eq:cntrl} $y_q$ and $u_h$, the outputs of $\QQ$ and $\HH$ (to be defined), are used as the input to the controller and the plant, respectively. 
% These are the output signals of watermark removers $\QQ$ and $\HH$, to be defined. 

\subsection{Multiplicative watermarking filters}
Consider mWM filters defined as follows
\begin{equation}\label{eq:sys:WM}
		\Sigma: \begin{cases}
			x_\sigma[k+1] = A_\sigma(\theta_\sigma[k]) x_\sigma [k] + B_\sigma(\theta_\sigma[k]) \nu_\sigma [k]\\
			\gamma_\sigma [k] = C_\sigma(\theta_\sigma[k]) x_\sigma [k] + D_\sigma(\theta_\sigma[k]) \nu_\sigma [k]
		\end{cases},
	\end{equation}
	with $\Sigma \in \{\GG,\HH,\WW,\QQ\}$, $\sigma \in \{g,h,w,q\}$, \ajg{where $g,h,w,q$ refer to variables pertaining to $\mathcal G, \mathcal H, \mathcal W, \mathcal Q$, respectively}\footnote{In the sequel whenever referring to the parameters of any one of the mWM filters, the subscript $\sigma$ is used. Conversely, if referring to all parameters, $\theta$ is used.}, $x_\sigma \in \mathbb{R}^{n_\sigma}$ the state of $\Sigma$, $\nu_\sigma \in \mathbb{R}^{m_\sigma}$ its input, $\gamma_\sigma \in \mathbb{R}^{p_\sigma}$ the output, and $\theta_{\sigma}[k]$ is a vector of parameters.

\begin{definition}[mWM filter parameters]
    The parameter $\theta_\sigma[k]$ is taken to be the concatenation of the vectorized form of all matrices $A_\sigma(\cdot), B_\sigma(\cdot), C_\sigma(\cdot), D_\sigma(\cdot)$.
    $\hfill\triangleleft$
\end{definition}

The parameter $\theta_{\sigma}$ is defined to be piecewise constant:
$$\theta_\sigma[k] = \bar{\theta}_\sigma[k_i], \forall k \in \{k_i, k_i+1, \dots, k_{i+1}-1\}$$
where $k_i, i = 0,1, \dots \in \mathbb{N}_+,$ are switching instants. In the following, with some abuse of notation, the time dependencies are dropped, with $\theta_\sigma$ and $\theta_\sigma^+$ used to define the parameters before and after a switching instant, 
% with $\theta_\sigma$ written instead of $\theta_\sigma[k_i]$, and $\theta_\sigma^+$ to define the filter parameters after a switching instant, 
i.e., $\theta_\sigma = \theta_\sigma[k_i]$, $\theta_\sigma^+ = \theta_\sigma[k_{i+1}]$.
% \input{mWM_gen_remov.tex}

% It is important to note that, because of this characterization and because of the non-uniqueness of state-space representation of transfer functions, the parametrization of the transfer functions of $\Sigma$ is not unique. Two parameters $\theta_{\sigma,1}$ and $\theta_{\sigma,2}$ are said to be \textit{equivalent} if they lead to the same transfer function. In the remainder of the paper, it is supposed that the selection of a specific parameter amongst all those equivalent can be defined by imposing some \textit{structure} onto $\theta_\sigma$. In the numerical results presented in Section~\ref{sec:NE}, for example, all matrices are defined to be diagonal.

% Defining the mWM parameters as time-varying has been shown to be critical against certain classes of attacks~\citep{ferrari2020switching}.
Furthermore, all filters are taken to be square systems, i.e., $m_\sigma = p_\sigma, \forall \sigma \in \{g,h,w,q\}$, and define $\nu_g \triangleq u_c, \nu_h \triangleq \tilde{u}_g, \nu_w \triangleq y_p, \nu_q \triangleq \tilde{y}_w, \gamma_g \triangleq u_g, \gamma_h \triangleq u_h, \gamma_w \triangleq y_w, \gamma_q \triangleq y_q.$
 %    \begin{equation}\label{eq:WM:input}
 %        \begin{array}{cccc}
 %            \nu_g \triangleq u_c, & \nu_h \triangleq \tilde{u}_g, & \nu_w \triangleq y_p, & \nu_q \triangleq \tilde{y}_w, \\
 %            \gamma_g \triangleq u_g, & \gamma_h \triangleq u_h, & \gamma_w \triangleq y_w, & \gamma_q \triangleq y_q. 
 %        \end{array}
	% \end{equation}
    Here, a \textit{tilde} is used to highlight that $\tilde u_g, \tilde y_w$ are received through the insecure communication network and as such may be affected by attacks. 
%
% This choice is made as a consequence of the fact that switching the parameters of the watermarking scheme is critical to enable the detection of attacks that would otherwise be stealthy, such as covert attacks with matched parameters \cite{ferrari2020switching}. 
%
%%%%%%% DONOT REMOVE THE FOLLOWING REMARK
\begin{remark}
    The objective of this paper is to \textit{optimally} design the successive parameters of the mWM filters $\theta_\sigma^+$, given their value $\theta_\sigma$. 
    It remains out of the scope of the paper to address other aspects of the switching mechanisms, such as determining the switching time, or defining the jump functions for the states. 
    Interested readers are referred to \citep{ferrari2020switching}.
    %for an example on its definition, and an analysis of the impact on stability.
    $\hfill \triangleleft$
\end{remark}

% The following defines what properties must be satisfied at any instant $k \in \mathbb{N}_+$ by two filters to be a valid mWM pair.
\begin{definition}[Watermarking pair]\label{def:WM}
Two systems $(\mathcal W,\mathcal Q)$ \eqref{eq:sys:WM}, are a \textit{watermarking pair} if:
\begin{enumerate}[label=\alph*.]
    \item \label{def:WM:inv} $\mathcal{W}$ and $\mathcal{Q}$ are stable and invertible, i.e., exists a positive definite matrix $Z_\sigma \succ 0, \sigma \in \{w,q\}$ such that %the following Lyapunov stability conditions hold 
    \begin{equation}\label{eq:WM:stab}
        A_\sigma^\top Z_\sigma A_\sigma - Z_\sigma \prec 0;
    \end{equation}
    \item \label{def:WM:eqParam} if $\theta_w[k] = \theta_q[k]$, $y_q[k] = y_p[k]$, i.e., 
    \begin{equation}\label{eq:WM:def}
	   \QQ \triangleq \WW^{-1}\,. \qquad \triangleleft
    \end{equation}
\end{enumerate}
\end{definition}
\begin{remark}\label{rem:stabSw}
    If $Z_\sigma$ in \eqref{eq:WM:stab} is the same for all $\theta_\sigma[k], k \in \mathbb N,\sigma\in\{g,h,w,q\}$, the mWM filters, on their own, are stable under arbitrary switching, as they all share a common Lyapunov function.
    $\hfill \triangleleft$
\end{remark}

\begin{definition}[{\cite[Lemma 3.15]{zhou1996robust}}]\label{def:inv:ss}
		Define the DT transfer function resulting from the system defined by the tuple $(A,B,C,D)$ as $			G(z) = \left[\begin{array}{c|c}
				A &B\\
				\hline
				C &D
			\end{array}
			\right],$
		and suppose that $D^{-1}$ exists. Then
		\begin{equation}\label{eq:sys:inv}
			G^{-1}(z) = \left[\begin{array}{c|c}
				A - BD^{-1}C    &BD^{-1}\\
				\hline
				-D^{-1}C        &D^{-1}
			\end{array}
			\right]
		\end{equation}
		is the inverse transfer function of $G(z)$.
		$\hfill\triangleleft$
	\end{definition}
% Definition~\ref{def:inv:ss} holds true for any equivalent state space transformation of the matrices defined in \eqref{eq:sys:inv}. Next we establish the following assumption. 
\begin{assumption}\label{ass:sync}
    The mWM parameters are matched, i.e., $\theta_w[k] = \theta_q[k]$ and $\theta_g[k] = \theta_h[k]\; \forall\;k \in \mathbb{N}$.
    $\hfill \triangleleft$
\end{assumption}
% Note that Assumption~\ref{ass:sync} is equivalent to saying that all mWM parameters' switching times $k_i, i \in \mathbb N$ are synchronized.

% From Definition~\ref{def:inv:ss}, under Assumption~\ref{ass:sync}, when no attack is present on the communication network between $\PP$ and $\CC$ (i.e., $\tilde{u}_g=u_g$ and $\tilde{y}_w = y_w$), so long as $x_w[0] = x_q[0]$ and $x_g[0] = x_h[0]$, $y_q[k] = y_p[k]$ and $u_h[k] = u_c[k]$ hold for all $k \in \mathbb{N}$, i.e., the mWM filters do not influence the closed-loop dynamics \citep{ferrari2020switching}.
% the following hold:
% \begin{align}\label{eq:sys:WM:inv}
% 	&\begin{array}{ccc}
% 		\mathcal{Q}(z)\mathcal{W}(z) = I_{p}\,,&  \mathcal{H}(z)\mathcal{G}(z) = I_{m}\,, 
%         &\forall z \in \mathbb C
% 	\end{array}\\
% 	&\begin{array}{ccc}
% 		y_q[k] = y_p[k]  \,, & u_h[k] = u_c[k]\,, &\forall k \in \mathbb{N}_+,
% 	\end{array}\label{eq:sys:WM:inv:2}
% \end{align}
% so long as $x_w[0] = x_q[0], x_g[0] = x_h[0]$. 
% In other words, when no attack is present, and if the mWM parameters remain synchronized, the mWM filters do not effect the closed-loop dynamics \citep{ferrari2020switching}. 

% \begin{figure*}[t]
%     \centering
%     {\footnotesize
%     \begin{align}
%         \left[
%         \begin{array}{c|c}
%         {A} & {B}\\
%         \hline
%         \\[\dimexpr-\normalbaselineskip+1pt] \bar{C}_r & 0\\
%         \hline
%         \\[\dimexpr-\normalbaselineskip+1pt] \bar{C}_J & \bar{D}_J
%         \end{array}
%         \right] &= 
%         \left[ 
%         \begin{array}{c|c}
%         \begin{matrix}
%             A_p & B_pC_h & B_pD_hC_g & B_pD_hD_gK & 0 & 0 & 0\\
%             0 & A_h & B_hC_g & B_hD_gK & 0 & 0 & 0\\
%             0 & 0 & A_g & B_gK & 0 & 0 & 0\\
%             LD_qD_wC_p & 0 &0 & A_p+B_pK-LC_p & LC_q & LD_qC_w & -LD_qC_a\\
%             B_qD_wC_p & 0 & 0 & 0 & A_q & B_qC_w & -B_qC_a\\
%             B_wC_p & 0 & 0 & 0 & 0 & A_w & 0\\
%             0 & 0 & 0 & 0 & 0 & 0 & A_a\\
%             \hline
%             D_qD_wC_p & 0 & 0 & -C_p & C_q & D_qC_w & -D_qC_a\\
%             \hline
%             C_J & D_JC_h & D_JD_hC_g & D_JD_hD_gK & 0 & 0 & 0
%         \end{matrix}     & \begin{matrix}
%         B_pD_h \\ B_h \\ 0 \\ 0 \\ 0 \\ 0 \\ B_a \\ \hline 0 \\ \hline D_JD_h
%         \end{matrix}
%         \end{array}
%         \right]\label{matrix_strip}
%     \end{align}
%     \hrule}
% \end{figure*}

\subsection{Attack model}\label{ch:probFor:atk}
Consider the malicious agent $\mathcal A$ located in the CPS as in Fig.~\ref{fig:sys}, capable of tampering with data transmitted between $\PP$ and $\CC$. 
% \ajg{The main motivation for introducing the mWM filters on the communication channels between the plant and the controller is the possibility of a malicious agent $\mathcal A$, located in the CPS such as the one in Fig.~\ref{fig:sys}. 
% In this context, $\mathcal A$ is thought to be capable of tampering with the transmitted data between $\PP$ and $\CC$.}
Without loss of generality, the injected attacks are modeled as additive signals:
 \begin{equation}\label{eq:atk}
     \tilde{u}_g[k] \triangleq u_g[k] + \varphi_u[k],\;\;\; \tilde{y}_w[k] \triangleq y_w[k] + \varphi_y[k],
 \end{equation}
where $\varphi_u[k]$ and $\varphi_y[k]$ are actuator and sensor attack signals designed by the adversary $\mathcal A$. To properly define our design algorithm in Section~\ref{sec:main}, an explicit strategy for the attack signals $\varphi_u$ and $\varphi_y$ must be defined by the defender. %, and are therefore interpreted as a design choice.
In this paper, we focus on covert attacks \citep{smith2015covert}, which remain undetected for passive diagnosis scheme.

% In particular, we consider an adversary that adopts a worst-case attack policy: covert attack with matched parameters. %In other words, we consider an adversary which has knowledge of the plant (common assumption in security literature), and the mWM parameters (which can be obtained through system identification techniques, for instance). Such an attack policy is adopted to mitigate the worst-case attack scenarios.
% Specifically, let $\theta_\sigma^a[k]$ be the parameters of the mWM filter known by the attacker\footnote{\ajg{Here, and throughout the paper, a super- or subscript $a$ is used to indicate that a variable pertains to $\mathcal A$.}}. Then, the following assumption is required.

The covert attack strategy, under Assumption~\ref{ass:param} and~\ref{ass:atkEng}, is as follows: the malicious agent $\mathcal A$ chooses $\varphi_u[k] \in \ell_{2e}$ freely, while $\varphi_y[k]$ satisfies:
\begin{equation}\label{eq:atk:cov}
	\mathcal A:
\left\{
\begin{aligned}
    x_a[k+1] &= A_a(\theta^a) x_a[k] + B_a(\theta^a) \varphi_u[k]\\
		y_a[k] &= C_a(\theta^a) x_a[k] + D_a(\theta^a) \varphi_u[k]\\
		\varphi_y[k] &= - y_a[k]
\end{aligned}
\right.
\end{equation}
where $x_a \triangleq [x_{h,a}^\top\;x_{p,a}^\top\;x_{w,a}^\top]^\top$ is the attacker's state, and its dynamics are the same as the cascade of $\HH, \PP, \WW$, parametrized\footnote{\ajg{Here, and throughout the paper, a super- or subscript $a$ is used to indicate that a variable pertains to $\mathcal A$.}} by $\theta^a_\sigma$.

\begin{assumption}\label{ass:param}
For all $k \in [k_{i+1},k_{i+2}], i \in \mathbb N_+$, the attacker parameters $\theta_\sigma^a[k]=\theta_\sigma[k_i]$, $\sigma \in \{h,w\}. \hfill\triangleleft$
\end{assumption}

\begin{assumption}\label{ass:atkEng}
    The attack energy is bounded and finite, i.e.,: $\Vert \varphi_u\Vert_{\ell_2}^2 \leq \epsilon_a$, with $\epsilon_a$ known to $\mathcal C$. $\hfill \triangleleft$
\end{assumption}

\begin{remark}
    Assumption~\ref{ass:atkEng} is introduced as it allows for guarantees that the algorithm proposed in Section~\ref{sec:main} always returns a feasible solution (see Theorem~\ref{thm:well:posed}).
    In general, %Assumption~\ref{ass:atkEng} is limiting: indeed, 
    while it may be that the adversary has limited energy \citep{djouadi2015finite}, it is a strong assumption that the bound $\epsilon_a$ is known to the defender.
    %
    Nonetheless, the attack energy bound $\epsilon_a$ may be seen as a design variable that, together with the chosen attack model \eqref{eq:atk:cov}, facilitates the definition of a systematic design of mWM filters by the defender.
    %
    % However, given the perspective that the attack policy \eqref{eq:atk:cov} is a design choice geared toward the definition of a systematic design of mWM filters, rather than an actual attack against the system, $\epsilon_a$ itself is a design variable, and thus its value is available to the defender.
    %
    Further remarks regarding the consequences of Assumption~\ref{ass:atkEng} not holding are postponed to Remark~\ref{rem:atkEng2}, following the formal definition of the attack-energy-constrained output-to-output gain in Definition~\ref{def:o2o}. 
    $\hfill \triangleleft$
\end{remark}

% In general, the adversary has limited energy \citep{djouadi2015finite}. Thus, the adversary injects an attack for a finite amount of time (say $T_a$). Although $T_a$ is unknown, we enforce that the adversary stops the attack eventually by a adding that the attack energy should be bounded. 

% \begin{remark}
%     In \eqref{p1}, $\epsilon_r$ and $\epsilon_a$ play in a critical role.
%     Firstly, the metric is always bounded, as is demonstrated in Theorem~\ref{thm:well:posed} and thus is well-posed for a design algorithm. 
%     %
%     Furthermore, it has explicit relations to both the $H_\infty$ metric (for increasing values of $\epsilon_r$) and to the original OOG (for increasing values of $\epsilon_a$ \citep[Prop.1]{anand2023risk}.
%     %
%     Finally, let us comment on the constraint on the attack energy, and argue that the choice of $\epsilon_a$ as a design parameter, to be chosen by the defender, rather than an assumpiton on the attack strategy is warranted. To do this, we consider \eqref{p1} (which is equivalent to \eqref{o1}, under Lemma~\ref{lem:sig_2_mat}) under increasing values of $\epsilon_a$, as well as the OOG as defined in \cite{teixeira2015strategic}.
%     If there are no zero dynamics in the system, the OOG is finite, and thus there is some value $\bar\epsilon_a$ such that, for all $\epsilon_a > \bar\epsilon_a$, the value of $\|y_r\|_{\ell_2}^2 = \epsilon_r$, and $\|y_J\|_{\ell_2}^2$ remains constant. Thus, defining $\epsilon_a \geq \bar \epsilon_a$, the optimal value $\theta^{+*}$ is not influenced by the constraint on the attack energy.
%     On the other hand, if there are zero dynamics that the attacker can exploit, the OOG is infinite, and $\|y_J\|_{\ell_2}^2$ grows unbounded as $\epsilon_a \rightarrow \infty$.
%     Thus, while $\theta^{+*}$, the solution to \eqref{o1}, is only optimal for $\|\varphi_u\| \leq \epsilon_a$, defining $\theta^*$ ensures the effect of $\varphi_u$ on $y_J$ is in some sense minimal if this is not the case.
%     % if the constraint is violated, defining $\theta^*$ ensures that the effect of an attack with unbounded energy on the performance output is in some sense minimal.
%     $\hfill\triangleleft$
% \end{remark}

% The AEC-OOG summarizes the adversary's objective of obtaining the maximum impact on the performance output, while remaining undetected. Compared to the output-to-output gain (OOG) proposed in \cite{teixeira2015strategic}, the attack signal energies are considered to always be bounded by $\epsilon_a$ -- a property exploited in Theorem~\ref{thm:well:posed}. This bound is treated as a \textit{design variable} in the hands of the defender, not as a constraint on the class of malicious signals that can be injected on the system. Further comments on this are given in Remark~\ref{rem:atkE}, in the following.

% \begin{align} 
% &\left[
% \begin{array}{c|c}
% {A}_a(\theta^a) & {B}_a(\theta^a)\\
% \hline
% \\[\dimexpr-\normalbaselineskip+2pt] {C}_a(\theta^a) & {D}_a(\theta^a)
% \end{array}
% \right] \triangleq\\
% &\left[ 
% \begin{array}{c|c}
% \begin{matrix}
% A_{h}(\theta_h^a) & 0 & 0 \\
% B_pC_h(\theta_h^a) & A_p & 0\\
% 0 & B_w(\theta_w^a)C_p & A_w(\theta_w^a)\\
% \hline 
% 0 & \quad D_w(\theta_w^a)C_p & C_w(\theta_w^a)
% \end{matrix}     & \begin{matrix}
% B_h(\theta_h^a) \\ B_pD_h(\theta_h^a) \\ 0 \\ \hline 0
% \end{matrix}
% \end{array}
% \right].
% \end{align} 
%In general, covert attacks with matched parameters remain undetectable to any diagnosis scheme by construction, while still having an impact on the performance of the system, because of the design of $\varphi_u$. \ajg{Thus, we consider the the worst-case covert attack policy, and provide a design algorithm to mitigate such attacks.}

\subsection{Problem formulation}\label{sec:PF:probFor}
The objective of this paper is to propose a design strategy capable of optimally designing the mWM filter parameters $\theta^+$, supposing a covert attack is present within the CPS. To formulate a metric to be used to define optimality, the closed-loop CPS dynamics must be defined. Under the attack strategy \eqref{eq:atk:cov}, the closed-loop system with the attack $\varphi_u$ as input and the performance and detection output as system outputs can be rewritten as
\begin{equation}\label{eq:S_cl}
		{\mathcal{S}}:\left\{
\begin{aligned}
{x}[k+1] &= A x[k] + B\varphi_u[k]\\
y_J[k] &= \bar{C}_J x[k] + \bar{D}_J \varphi_u[k]\\
y_r[k] &= \bar{C}_r x[k]
\end{aligned} \right.
	\end{equation}
where $x = \begin{bmatrix}x_p^\top, &x_h^\top, &x_g^\top, &x_c^\top, &x_q^\top, &x_w^\top, &x_a^\top\end{bmatrix}^\top$ is the closed-loop system state, while $y_r$ and $y_J$ remain the residual and performance outputs. All signals in \eqref{eq:S_cl} are also a function of the parameters $\theta^+$, but this dependence is dropped for clarity.
The definition of the matrices in \eqref{eq:S_cl} follow from \eqref{eq:sys}-\eqref{eq:sys:WM} and \eqref{eq:atk:cov}.
% All matrices in \eqref{eq:S_cl} are provided in \eqref{matrix_strip}.

% For the closed loop CPS dynamics in \eqref{eq:S_cl}, 
The defender aims to quantify (and later minimize) the maximum performance loss caused by a stealthy and bounded-energy adversary on \eqref{eq:S_cl}. 
This is done by exploiting the attack-energy-constrained output-to-output gain 
(AEC-OOG) \citep{anand2023risk}. %\footnote{The AEC-OOG is a restriction of the previously defined output-to-output gain (OOG) \citep{teixeira2015strategic}, by imposing that all attack signal energies are bounded.}
\begin{definition}[AEC-OOG]\label{def:o2o}
	The AEC-OOG
    % attack\hyp{}energy\hyp{}constrained output-to-output gain 
    of $\mathcal S$ in \eqref{eq:S_cl} is the value of the following optimization problem:
		\begin{equation}\label{eq:o2o}
			\begin{aligned}
				\sup_{\varphi_u\in\ell_{2e}} &\quad \|y_J\|_{\ell_2}^2 \\
				\text{s.t.}& \quad  \|y_r\|_{\ell_2}^2 \leq \epsilon_r,\; \|\varphi_u\|_{\ell_2}^2 \leq \epsilon_a,\;x[0] = 0.
			\end{aligned}
		\end{equation}
	where $\epsilon_a$ is the energy bound of the attack signal, $\epsilon_r$ is the detection threshold, and the value of \eqref{eq:o2o} denotes the performance loss caused by a stealthy adversary.	$\hfill\triangleleft$
	\end{definition}
% The AEC-OOG summarizes the adversary's objective of obtaining the maximum impact on the performance output, while remaining undetected. Compared to the output-to-output gain (OOG) proposed in \cite{teixeira2015strategic}, the attack signal energies are considered to always be bounded by $\epsilon_a$ -- a property exploited in Theorem~\ref{thm:well:posed}. This bound is treated as a \textit{design variable} in the hands of the defender, not as a constraint on the class of malicious signals that can be injected on the system. Further comments on this are given in Remark~\ref{rem:atkE}, in the following. 
%, to ensure our design algorithm is always feasible. %Further discussion regarding the AEC-OOG can be found in \cite{anand2023risk}.
% The AEC-OOG summarizes the following case: the goal of the adversary is to maximize the performance signal energy (as opposed to classical $\mathcal{H}_{\infty}$ control) while remaining undetected. The latter objective translates to constraining the detection output to remain under the predefined threshold value $\epsilon_r$. 
% \ajg{In this definition, we take the attack input to be bounded energy, where $\epsilon_a$ acts as this bound. The inclusion of this constraint in \eqref{eq:o2o} ensures that the AEC-OOG is always bounded, a property that is exploited in Theorem~\ref{thm:well:posed} to ensure that $\theta_\sigma^+$ can always be defined. 
% }
% Using the definition above, the main problem studied in this paper can be formalized.

\begin{problem}\label{problem_main}
Given $\theta_{\sigma}$ at some switching time $k_i, i \in \mathbb{N}_+$, find the optimal set of mWM filter parameters after a switching event $\theta_{\sigma}^+$, such that the AEC-OOG of the system $\mathcal{S}$ in \eqref{eq:S_cl} is minimized. $\hfill \triangleleft$
\end{problem}

\begin{remark}
    Because of its dependence on the AEC-OOG, the solution of Problem~\ref{problem_main} at time $k_i$ relies explicitly on the attack parameters $\theta^a[k_i]$. Given the malicious agent's strategy outlined in Section~\ref{ch:probFor:atk}, and Assumption~\ref{ass:param}, $\theta_\sigma^a[k_i] = \theta_\sigma^+[k_{i-1}]$ is known to $\CC$, without any additional knowledge required. %Thus, no additional knowledge is required by the defense mechanism in $\CC$ to define the closed-loop matrices in \eqref{eq:S_cl}.
    $\hfill \triangleleft$
\end{remark}

\begin{remark}\label{rem:atkEng2}
    We are now ready to formally treat the violation of Assumption~\ref{ass:atkEng}.
    To do this, let us first remark on some properties of the AEC-OOG, which follow from using finite bounds $\epsilon_r$ and $\epsilon_a$.
    Firstly, as demonstrated in Theorem~\ref{thm:well:posed}, the metric is always bounded, making it well suited for a design algorithm.
    Furthermore, it is explicitly related to both the $H_\infty$ metric and the original OOG proposed in \cite{teixeira2015strategic}, for increasing values of $\epsilon_r$ and $\epsilon_a$, respectively \citep[Prop.1]{anand2023risk}.
    Finally, we can comment on the constraint on the attack energy.
    Consider the value of \eqref{eq:o2o} under increasing values of $\epsilon_a$, as well the OOG as defined in \cite{teixeira2015strategic}.
    If the OOG is finite, there is some value $\bar\epsilon_a$ such that the AEC-OOG is the same as the OOG for all $\epsilon_a \geq \bar\epsilon_a$.
    If there are exploitable zero dynamcis, and the OOG is unbounded, $\|y_J\|_{\ell_2}^2$ grows unbounded as $\epsilon_a \rightarrow \infty$. Thus, while $\theta_\sigma^+$, the solution to Problem~\ref{problem_main}, is only optimal for covert attacks satisfying $\|\varphi_u\|_{\ell_2}^2 \leq \epsilon_a$, it ensures that the effect of $\varphi_u$ on $y_J$ is in some sense minimal if the attack energy constraint is violated.
    $\hfill\triangleleft$
\end{remark}

% The AEC-OOG summarizes the adversary's objective of obtaining the maximum impact on the performance output, while remaining undetected. Compared to the output-to-output gain (OOG) proposed in \cite{teixeira2015strategic}, the attack signal energies are considered to always be bounded by $\epsilon_a$ -- a property exploited in Theorem~\ref{thm:well:posed}. This bound is treated as a \textit{design variable} in the hands of the defender, not as a constraint on the class of malicious signals that can be injected on the system. Further comments on this are given in Remark~\ref{rem:atkE}, in the following.


% \begin{remark}
% \ajg{The AEC-OOG defined in \eqref{eq:o2o} can be related to other metrics in the literature namely the $H_{\infty}$ metric and the OOG. Let $\gamma_{\infty}$ and  $\gamma_{\text{OOG}}$ denote the $H_{\infty}$ gain, $\gamma_{OOG}$ of the closed loop system \eqref{eq:S_cl}, and $\gamma(\epsilon_a,\epsilon_r)$ denote the value of AEC-OOG for any given value of $\epsilon_a$ and $\epsilon_r$. Then it holds that $\lim_{\epsilon_a \to \infty} \gamma(\epsilon_a,\epsilon_r) = \gamma_{\text{OOG}}$ and $\lim_{\epsilon_r \to \infty} \gamma(\epsilon_a,\epsilon_r) = \gamma_{\infty}$} $\hfill \triangleleft$
% \end{remark}

%
\section{Optimal design of filters}\label{sec:main}
\section{Design}\label{sec:design}

%%%%%%%%%%%%%%%%%%%%%%%%%%%%%%


\begin{figure*}[t]
    \centering
    \includegraphics[trim = 15 530 15 15, width=1\textwidth]{Algorithm_drawio.pdf}
    \caption{Overview of KiSS}
    \label{fig:overview}
\end{figure*}


The results we gleaned from the previous section (see Section~\ref{sec:work_anly}) helped in developing our policy: KiSS. The KiSS or \textbf{Keep it Separated Serverless} policy aims to address critical challenges in Function-as-a-Service (FaaS) platforms, particularly in edge computing environments, by achieving the following objectives:

\begin{itemize}
    \item \textbf{Reduced Cold Start Latency:} Prioritizes high-frequency functions to minimize delays in real-time applications.
    \item \textbf{Improved Resource Efficiency:} Optimizes memory and compute usage while avoiding unnecessary overhead from static warm states.
    \item \textbf{Minimized Inter-Function Interference:} Enhances throughput and scalability through modular resource partitioning.
    \item \textbf{Improved Function Service Rate:} Adopts resource-aware policies to reduce dropped requests and maximize system reliability.
\end{itemize}


\subsection{KiSS Policy Overview}

KiSS introduces a modular, data-driven orchestration strategy designed to optimize serverless execution in resource-constrained environments, particularly at the edge. By leveraging our workload analysis (refer Section 2.5), our policy segments functions based on key metrics—memory footprint, invocation frequency, and execution time—to optimize performance across diverse workloads.

The edge computing context introduces unique challenges like limited memory, heterogeneous resources, and dynamic workloads. Generalized cloud strategies often fail to adapt to such constraints. KiSS addresses this gap by analyzing workload characteristics and implementing a resource-efficient, modular strategy that aligns with edge-specific demands.

\subsection{Components of KiSS Policy Design}
Figure~\ref{fig:overview} shows the overall architecture of KiSS. 
The incoming \textit{FaaS traffic} will include both small and large functions. 
The \textit{request handler} accepts the incoming functions and shares the function information to the workload analyzer. 
The \textit{workload analyser} processes the function information to profile the incoming function traffic information and generate data such as invocation frequency, memory footprint etc.
The \textit{KiSS policy} uses this data to estimate where this function will be placed between the two different warm pool partitions.

The \textit{load balancer} implements a partitioning logic where functions are allocated to distinct warm pools using (\textit{invoker 1} and \textit{invoker 2}) based on profiling thresholds:

(i)~Small Functions Pool: Dedicated to high-frequency, low-memory functions to ensure low latency, and (ii)~Large Functions Pool: Allocated for low-frequency, memory-intensive functions, minimizing contention with smaller containers.
Each warm pool operates autonomously achieving Policy Independence.
The \textit{Warm Pool Replacement Policy} for each warm container pool can independently implement different workload-specific strategies to reduce contention and enhance temporal locality.


These factors form the foundation of KiSS’s multi-tiered warm pool framework, allowing it to effectively manage serverless resources and enhance performance in edge computing. By addressing these challenges, KiSS positions itself as a practical and scalable solution for FaaS platforms in environments with diverse and demanding resource constraints.


\subsection{Innovations of KiSS Policy}

One of the most innovative features of KiSS is its multi-level warm pool partitioning, which isolates high- and low-frequency functions into separate pools. This design eliminates inefficiencies inherent in monolithic resource strategies by ensuring that small, frequently invoked functions are always ready to execute, while larger, less frequent functions remain accessible without competing for resources. This adaptability extends to the ability to add more pools as workload patterns evolve, making KiSS a flexible and future-proof solution. Moreover, its modular architecture supports diverse deployment scenarios, from centralized clouds to resource-constrained edge environments. Integration with traffic-aware schedulers ensures that KiSS maintains scalability and responsiveness even under fluctuating workloads.


\subsubsection{Advantages of KiSS}

The advantages of KiSS are particularly pronounced in edge environments. By keeping frequently accessed containers in warm states, it drastically reduces cold start latency, which is critical for real-time applications such as IoT and AI analytics. Static warm pool partitioning, based on workload analysis, optimizes memory usage by eliminating unnecessary overhead, ensuring that resources are used efficiently even in environments with stringent memory constraints. This strategy not only enhances performance but also reduces operational costs by consolidating memory usage and minimizing cold starts. KiSS’s platform-agnostic design further enhances its versatility, enabling seamless deployment across various serverless frameworks.


%
\section{Numerical example}\label{sec:NE}
% In this section, the effectiveness of the proposed algorithm is described through a numerical example. 
\subsection{Plant description}
Consider a power generating system \citep[Sec.4]{park2019stealthy} %which can be modeled as as shown in Fig. \ref{turbine} and 
modeled by the dynamics:
\begin{align}
\label{power_AB} \begin{bmatrix}
\dot{\eta}_1\\ \dot{\eta}_2 \\ \dot{\eta}_3
\end{bmatrix} &= 
\begin{bmatrix}
\frac{-1}{T_{lm}} & \frac{K_{lm}}{T_{lm}} & \frac{-2K_{lm}}{T_{lm}}\\
0 & \frac{-2}{T_h} & \frac{6}{T_h}\\
\frac{-1}{T_g R} & 0 & \frac{-1}{T_g}
\end{bmatrix}
\underbrace{\begin{bmatrix}
{\eta}_1\\ {\eta}_2 \\ {\eta}_3
\end{bmatrix}}_{\eta}
+ \begin{bmatrix}
0\\ 0 \\ \frac{1}{T_g}
\end{bmatrix}
{u}\\
\label{power_C} y_p &= \underbrace{ \begin{bmatrix}
1 & 0 & 0 
\end{bmatrix}}_{C_p}\eta,\;\;
y_J = \underbrace{
\begin{bmatrix}
0 & 1 & 0
\end{bmatrix}}_{C_J}\eta.
\end{align}
Here, $\eta \triangleq [df; dp + 2 dx; dx]$, where $df$ is the frequency deviation in \mbox{Hz}, $dp$ is the change in the generator output per unit (\mbox{p.u.}), and $dx$ is the change in the valve position \mbox{p.u.}. The parameters of the plant are listed in Table \ref{param}. 
% The constants $T_{lm}, T_h$, and $T_g$ represent the time constants of load and machine, hydro turbine, and governor, respectively, and $R \,\mathrm{(Hz/p.u.)}$ is the speed regulation due to the governor action. The constant $K_{lm}$ represents the steady-state gain of the load and machine. 
The Discrete-Time system matrices $(A_p,B_p,C_p,D_p)$ are obtained by discretizing the plant \eqref{power_AB}-\eqref{power_C} using zero-order hold with a sampling time $T_s=0.1\mbox{s}$. 

\begin{table}
\centering
\begin{tabular}{||c | c || c | c|| c | c ||} 
 \hline
 $K_{lm}$ & 1 & $T_{lm}$ & 6 &  $T_g$ & 0.2 \\
 \hline
 $T_{h}$ & $4$ & $T_s$ & 0.1 & $R$ & 0.05\\
 \hline
\end{tabular}
\caption{System Parameters}
\label{param}
\end{table}


The plant is stabilized locally with a static output feedback controller with constant gain $D_c=19$. The gains in \eqref{eq:cntrl} are obtained by minimizing a quadratic cost, using the MATLAB command \emph{dlqr}, resulting in:
\begin{align}
    K&=\begin{bmatrix}
        0.1986  & -0.0913  & -0.1143
    \end{bmatrix}\\
    L &= \begin{bmatrix}
       0.2735 &  -0.0509 & -0.2035
    \end{bmatrix}^\top.
\end{align}
% The controller (detector) gain is obtained by minimizing a quadratic cost, which can be done in MATLAB using the \emph{dlqr} command. 

% %%% You can delete the following figure to save space. 
% \begin{figure}
%     \centering
%     \includegraphics[width=8.4cm]{Turbine.eps}
%     \caption{\ajg{Pictorial representation of power generating system with a hydro turbine under covert attack. The plant consists of the power generating system (governer, hydro-turbine,machine) and a stabilizing controller. The plant output is transmitted over the network to a controller for process monitoring and tracking command. The solid/dashed/red lines represent physical/cyber and covert attack components respectively.}}
%     \label{turbine}
% \end{figure}
\subsection{Initializing the mWM design algorithm}
We consider a mWM filter of state dimension $n_{\sigma}=2$. The mWM filter parameters are initialized as $A_q = 0.2I_2$, $B_q=0.7e_{2 \times 1}$, $C_q = 0.1 e_{1 \times 2}$, $B_h=0.2e_{2 \times 1}$, $C_h=0.05 e_{1 \times 2}$, $A_h=0.3I_2$, $D_q=0.15$, $D_h=0.1$ where $e_{a \times b}$ represents a unit matrix of size $a \times b$. 
The other mWM matrices are derived such that they satisfy \eqref{eq:WM:def}. 
All unspecified matrices are zero. Following Assumption~\ref{ass:param}, it is assumed that the filter parameters $\theta$ are known by the adversary. 
% Thus, the aim is to find the next-step mWM parameters $\theta^+$ minimizing the AEC-OOG. 
To ensure randomization, as mentioned in Theorem~\ref{thm:nonRepeat}, the parameters $D_h$ and $D_q$ are initialized in Algorithm~\ref{algo2} as random numbers within the range $[0.1,0.15]$. We fix the parameters of all the mWM filter parameters at their initial value except for the matrix $A_{\sigma}, \sigma \in \{q,w,h,g\}$, i.e., our aim is to find a diagonal $A_{\sigma}$ that minimizes the value of the AEC-OOG.

As discussed in Section~\ref{ch:design:algo}, $A_q$ and $A_h$ are matrices whose diagonal elements take values in $(-1,1)$.
The exhaustive search is performed with a grid size of $n_s = 0.3$, and the search is initialized with ${\epsilon_r = 1}, {\epsilon_a = 50}$. Furthermore, for numerical stability, we modify the objective function of \eqref{o1} to $\epsilon_r \gamma + \epsilon_a \gamma_a + \epsilon_p \mathrm{tr}(P)$, with ${\epsilon_p = 0.1}$.

% To this end, we set $A_q=\text{diag}(t_1,t_2)$, and $A_h=\text{diag}(t_3,t_4)$ where $-1 < t_{i} < 1, i\in \{1,2,3,4\}$. In other words $A_q$ and $A_h$ are matrices whose diagonal elements are allowed to take values between $-1$ and $+1$, to ensure stability of the filters. Then a grid search (with grid size $n_s$=0.3) on the matrices is employed as described in Algorithm \ref{algo2}. The grid search is initialized with ${\epsilon_r=1},{\epsilon_a=50}$ and ${\epsilon_p=0.1}$. Here $\epsilon_p$ represents the weight on the trace of the matrix $P$ in the objective function.  In other words, we modify the objective function in \eqref{o1} as $\epsilon_r\gamma+\epsilon_a\gamma_a+\epsilon_p\text{tr}(P)$, for the numerical stability of the algorithm. 

\subsection{Result of Algorithm \ref{algo2}}
The optimal value of the matrices from the grid search are $A_q^* = -0.05I_2$ and $A_h^*=-0.65 I_2$. 
The corresponding value of $\mathcal{L}$ is $111.03$. The value of $D_q$ and $D_h$ were $0.1479$ and $0.1482$ respectively. The simulation is performed using Matlab 2021a with \textit{Yalmip} \citep{lofberg2004yalmip} and \textit{SDPT3v4.0} solver \citep{toh2012implementation}.
%
In the remainder, we compare the results obtained by repeated computation of Algorithm~\ref{algo2} compared to defining constant and random parameters.
Consider 
an adversary injecting the signals shown in Fig.~\ref{fig:no:switch},
\begin{equation}\label{eq:step}
    \varphi_u[k] = \begin{cases}
        150 & \text{if}\;k\;\text{mod}\;2=0 \\
        0 & \text{otherwise}
    \end{cases}
   \end{equation}
into the actuators, and $\varphi_y$ following \eqref{eq:atk:cov}.
% Next, we compare our results to the scenario where the parameters are not switched,  and where they are updated randomly. 
    
\emph{Comparison with no parameter switching:}
% The adversary constructs an attack signal $\varphi_y$ through \eqref{eq:atk:cov} which is shown in Fig.~\ref{fig:no:switch}. 
The performance of the attack is shown in Fig.~\ref{fig:no:switch}, for the cases without switching and when switching happens at the attack onset with the optimal filter parameters.
Without switching $\theta$, although the performance is strongly degraded, the attack remains stealthy. Instead, if the mWM parameters are changed, it is detected after $15 \mbox{s}$.
% We can see that the performance degradation is high without the switching, and there is no attack detection since the adversary exactly knows the parameters. This also unveils the importance of detecting covert attacks. On the other hand, when the parameters of the watermarks switch the attack is successfully detected after $15\mbox{s}$. 

\begin{figure}
    \centering
    \includegraphics[width=7cm]{Resubmission_ajg.eps}
    \caption{\ajg{(Top) The attack signal $\varphi_u$ in \eqref{eq:step} and its equivalent $\varphi_y$ from \eqref{eq:atk:cov}; (Middle) $\Vert y_r \Vert_{\ell_2,[0,k]}^2$, compared to $\epsilon_r$; (Bottom) $\Vert y_J \Vert_{\ell_2,[0,k]}^2$ before and after the mWM parameters are updated.}}
    \label{fig:no:switch}
\end{figure}



\emph{Comparison with random parameter switching:}
In this scenario, we suppose the mWM parameters are updated $5$ times, by running Algorithm~\ref{algo2}, and compared against $5$ random updates of $A_\sigma$ -- though their structure remains diagonal.
% Next, Algorithm~\ref{algo2} is run, i.e., the filter parameters are updated, $5$ times. 
The results, shown in terms of values of $\mathcal L$ for both cases, are shown in Fig.~\ref{fig:switch}.
% The results are shown in Fig.~\ref{fig:switch} where we plot the values of $\mathcal{L}$.
% For comparison, instead of selecting the optimal matrices $A_\sigma$ from an exhaustive search, we choose the matrices randomly. That is, the diagonal elements of $A_h$ and $A_q$ are chosen random and the parameters are discarded if they yield an unstable inverse. 
% The corresponding value of  $\mathcal{L}$ is also depicted in Fig. \ref{fig:switch}. 
Here, the parameters of $D_h$ and $D_q$ are the same as used for selecting the optimal parameters. Since the parameters are not chosen optimally, the value of $\mathcal{L}$, the performance loss, is higher. 

\begin{figure}
    \centering
    \includegraphics[width=6.5cm]{Switch_new.eps}
    \caption{The values of $\mathcal{L}$ corresponding to the optimal and random values of the watermarking parameters.}
    \label{fig:switch}
\end{figure}

\emph{Time complexity:}
To conclude, let us discuss thecomplexity of Algorithm~\ref{algo2}. All mWM parameters are fixed, apart from $A_\sigma$, which is a diagonal matrix of dimension $n_\sigma$, and, for each diagonal element of $A_\sigma$, $n_s$ points of the interval $(-1,1)$ are searched.
Given \eqref{eq:WM:def}, only $A_h$ and $A_q$ must be defined, while $A_w, A_g$ are defined algebraically; thus, define $n_\varsigma = n_q + n_h$.
The complexity of the algorithm grows both in $n_\varsigma$ and in $n_s$. Specifically: for $n_s = n_s^*$, the complexity is $\mathcal O(n_s^{*x})$; for $n_\varsigma = n_\varsigma^*$, the complexity is $\mathcal O(x^{n_\varsigma^*})$. Thus, the complexity is exponential in the choice of $n_\varsigma$ and polynomial in $n_s$.
%
We highlight that the average time of solution can be improved upon in two major ways. 
The first is via parallelization, as all SDPs can be solved independently; this provides a speed-up which depends on the number of compute nodes used to solve the problem.
The second method relies on reducing the number of SDPs to be solved, by removing those values of $A_h, A_q$ which do not lead to stable inverses, as defined by \eqref{eq:sys:inv}.

For the results presented here, a computer with an Intel Core i7-6500U CPU with 2 cores and 8GB RAM was used. The algorithm was run both with and without parallelization (parallelization was achieved by using Matlab's \texttt{parfor} command). Without parallelization, the algorithm took $384.25 \mathrm{s}$ to provide a result, whilst with parallelization this was $261.65 \mathrm{s}$, a  $31.9 \%$ speedup.

% \subsection{Time complexity}
% The computation complexity of a design algorithm is important, and in this subsection we briefly comment on it. In particular, we comment on the expected time taken to perform the grid search in Algorithm \ref{algo2}. As adopted in this example, let us consider the design algorithm where all the mWM parameters are fixed except for $A_{\sigma}$, which is a diagonal matrix. Let $n_s$ denote the size of the grid search, $n_{\sigma}$ denote the size of the watermarking filter, $\mathbb{E}[\tau]$ denote the mean time taken by a solver to compute the optimal value of \eqref{o1} for any given value of the filter parameters. The time $\mathbb{E}[\tau]$ depends on the SDP solver and the hardware. Then, the mean time taken to perform the grid search, denoted by $\mathbb{E}[T]$ is bounded by 
% \begin{equation}\label{eq:bound}
%     \mathbb{E}[T] \leq \mathbb{E}[\tau] \left(\ceil[\Bigg]{\frac{2}{n_s}} \right)^{n_{\sigma}}
% \end{equation}
% where $\ceil{x}$ denotes the smallest integer greater than $x$. Thus, the time complexity increases quadratically in the grid size, and exponentially in the size of the filter state. The bound \eqref{eq:bound} also holds when the matrix $A_{g}$ and $A_h$ are traingular where the diagonal elements are the decision variables and the other elements are fixed. 

% In general, the bound in \eqref{eq:bound} is loose, as some filter parameters might yield an unstable inverse, and the grid search does not need to solve an SDP in that case (see step 5 in Algorithm \ref{algo2}). Thus, the bound \eqref{eq:bound} can be largely improved by pre-processing, where the set of all matrices which yield an unstable inverse is removed. Additionally, since the SDPs can be solved independently, the grid search is parallelizable, which reduces the time complexity even further. 

\section{Conclusion and future works }\label{sec:Con}
An optimal design technique for the design of the parameters of switching multiplicative watermarking filters is presented. 
The problem is formalized by supposing the closed-loop system is subject to a covert attack with matching parameters. 
We propose an optimal control problem based on a formulation of the attack energy constrained output-to-output gain. 
We show through a numerical example that this design improves detectability by increasing the energy of the residual output before and after a switching event. 
Future works includes developing algorithms for optimal design and optimal switching times ensuring that mWM does not destabilize the closed-loop system under switching with mismatched parameters, and studying non-linear systems. 
%\bibliographystyle{plain}
\bibliography{autosam_abbrv}
%\printbibliography
% \appendix

\end{document}