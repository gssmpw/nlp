% autosam.tex
% Annotated sample file for the preparation of LaTeX files
% for the final versions of papers submitted to or accepted for 
% publication in AUTOMATICA.

% See also the Information for Authors.

% Make sure that the zip file that you send contains all the 
% files, including the files for the figures and the bib file.

% Output produced with the elsart style file does not imitate the
% AUTOMATICA style. The style file is generic for all Elsevier
% journals and the output is laid out for easy copy editing. The
% final document is produced from the source file in the
% AUTOMATICA style at Elsevier.

% You may use the style file autart.cls to obtain a two-column 
% document (see below) that more or less imitates the printed 
% Automatica style. This may helpful to improve the formatting 
% of the equations, tables and figures, and also serves to check 
% whether the paper satisfies the length requirements.

% Please note: Authors must not create their own macros.

% For further information regarding the preparation of LaTeX files 
% for Elsevier, please refer to the "Full Instructions to Authors" 
% from Elsevier's anonymous ftp server on ftp.elsevier.nl in the
% directory pub/styles, or from the internet (CTAN sites) on
% ftp.shsu.edu, ftp.dante.de and ftp.tex.ac.uk in the directory
% tex-archive/macros/latex/contrib/supported/elsevier.


%\documentclass{elsart}               % The use of LaTeX2e is preferred.

\documentclass[twocolumn]{autart}    % Enable this line and disable the 
                                     % preceding line to obtain a two-column 
                                     % document whose style resembles the
                                     % printed Automatica style.


\usepackage{graphicx}          % Include this line if your 
                               % document contains figures,
%\usepackage[dvips]{epsfig}    % or this line, depending on which
                               % you prefer.       
% \usepackage{amssymb}
% \usepackage{amsmath}
% \usepackage{enumitem}
% \usepackage{comment}
% \usepackage{mathtools}
% \usepackage{cuted}
% \usepackage{textcomp}
% \mathtoolsset{showonlyrefs}


\usepackage{cite}
\usepackage{amsmath,amssymb,amsfonts}
\usepackage{algorithmic}
\usepackage{graphicx}
\usepackage{cuted}
\usepackage{flushend}
\usepackage{mathtools}
\usepackage{array}
\usepackage{subcaption}
\mathtoolsset{showonlyrefs}
\usepackage{multirow}
\usepackage{multicol}
\usepackage{xcolor}
\usepackage{braket}
\usepackage{floatrow}
\usepackage{kantlipsum}
\usepackage{enumitem}
\usepackage{hyphenat}
\usepackage{enumitem}
\usepackage{wrapfig}

% \usepackage{algorithm}
% \usepackage{algpseudocode}
\floatsetup[table]{capposition=top}


\newtheorem{theorem}{Theorem}[section]
\newtheorem{definition}{Definition}[section]
\newtheorem{assumption}{Assumption}[section]
\newtheorem{corollary}{Corollary}[theorem]
\newtheorem{lemma}[theorem]{Lemma}
\newtheorem{remark}{Remark}
\newtheorem{problem}{Problem}
\DeclarePairedDelimiter{\ceil}{\lceil}{\rceil}

%%%%% Define symbols for estimates - we may want to change them, and this could make it easier
\newcommand{\est}[2]{\tilde{#1}_{[#2]}}
\newcommand{\err}[1]{\epsilon_{[#1]}}
\newcommand{\res}[1]{r_{[#1]}}
\newcommand{\thr}[2]{\bar{#1}_{[#2]}}

\newcommand{\E}[1]{\mathbf{E}\left[{#1}\right]}
\newcommand{\Evar}[1]{\mathbf{E}\left[{#1}{#1}^\top\right]}


% \definecolor{ajgCol}{rgb}{0.858, 0.188, 0.478}
\definecolor{ajgCol}{rgb}{0,0,0}
\newcommand{\ajg}[1]{{\color{ajgCol}#1}}

\definecolor{scaCol}{rgb}{0,0,0}
\newcommand{\sca}[1]{{\color{scaCol}#1}}

\newcommand{\PP}{\mathcal{P}}
\newcommand{\CC}{\mathcal{C}}
\newcommand{\WW}{\mathcal{W}}
\newcommand{\QQ}{\mathcal{Q}}
\newcommand{\HH}{\mathcal{H}}
\newcommand{\GG}{\mathcal{G}}

\hyphenation{con-strained}
\usepackage[round]{natbib}        % required for bibliography
\bibliographystyle{abbrvnat} % Choose a bibliography style that supports author-year
% \renewcommand{\cite}{\citep}
\allowdisplaybreaks


\begin{document}

\begin{frontmatter}
%\runtitle{Insert a suggested running title}  % Running title for regular 
                                              % papers but only if the title  
                                              % is over 5 words. Running title 
                                              % is not shown in output.

\title{Switching Multiplicative Watermark Design \\ Against Covert Attacks\thanksref{footnoteinfo}} % Title, preferably not more 
                                                % than 10 words.

\thanks[footnoteinfo]{This work has been partially supported by the Research Council of Norway through the project AIMWind (grant ID 312486), by the Swedish Research Council under the grant 2018-04396, and by the Swedish Foundation for Strategic Research. The material in this paper was partially presented at the 60th IEEE Conference on Decision and Control, Austin, Texas, 2021. Corresponding author Sribalaji. C. Anand. 
%Tel. +4618-471 7003.
}

\author[Paestum]{Alexander J. Gallo\thanksref{footnoteinfo2}}\ead{alexanderjulian.gallo@polimi.it},    % Add the 
\author[Rome]{Sribalaji C. Anand\thanksref{footnoteinfo2}}\ead{srca@kth.se},               % e-mail address 
\author[Baiae]{Andre M. H. Teixeira}\ead{andre.teixeira@it.uu.se},  % (ead) as shown
\author[Pompeii]{Riccardo M. G. Ferrari}\ead{r.ferrari@tudelft.nl}

\address[Paestum]{Department of Electronics, Information and Bioengineering, Politecnico di Milano, Milano, Italy.}  % Please supply                                              
\address[Rome]{School of Electrical Engineering and Computer Science and Digital Futures, KTH Royal Institute of Technology, Sweden}             % full addresses
\address[Baiae]{Department of Information Technology, Uppsala University, PO Box 337, SE-75105, Uppsala, Sweden.}        % here.

\address[Pompeii]{Delft Center for Systems and Control, Mechanical Engineering, TU Delft, Delft, Netherlands.}

\thanks[footnoteinfo2]{These authors contributed equally.}
          
\begin{keyword}                           % Five to ten keywords,  
Network security, Networked control systems, Fault detection and isolation 
%System security, Quadratic performance indices, Fault detection, $H_{\infty}$ control, Optimization.               % chosen from the IFAC 
\end{keyword}                             % keyword list or with the 
                                          % help of the Automatica 
                                          % keyword wizard


\begin{abstract}                          % Abstract of not more than 200 words.
\textit{Active techniques} have been introduced to give better detectability performance for cyber-attack diagnosis in cyber-physical systems (CPS). In this paper, switching multiplicative watermarking is considered, %where time-varying filters are defined to alter the dynamics of information transmitted over a communication channel between a plant and a controller in a CPS. 
% In this context, the objective of this paper is to 
whereby we propose an optimal design strategy to define switching filter parameters. 
Optimality is evaluated exploiting the so-called output-to-output gain of the closed loop system, including some supposed attack dynamics. 
A worst-case scenario of a matched covert attack is assumed, presuming that an attacker with full knowledge of the closed-loop system injects a stealthy attack of bounded energy. 
Our algorithm, given watermark filter parameters at some time instant, provides optimal next-step parameters.
Analysis of the algorithm is given, demonstrating its features, and demonstrating that through initialization of certain parameters outside of the algorithm, the parameters of the multiplicative watermarking can be randomized.
Simulation shows how, by adopting our method for parameter design, the attacker's impact on performance diminishes.
\end{abstract}

\end{frontmatter}

\section{Introduction}
\section{Introduction}

Large language models (LLMs) have achieved remarkable success in automated math problem solving, particularly through code-generation capabilities integrated with proof assistants~\citep{lean,isabelle,POT,autoformalization,MATH}. Although LLMs excel at generating solution steps and correct answers in algebra and calculus~\citep{math_solving}, their unimodal nature limits performance in plane geometry, where solution depends on both diagram and text~\citep{math_solving}. 

Specialized vision-language models (VLMs) have accordingly been developed for plane geometry problem solving (PGPS)~\citep{geoqa,unigeo,intergps,pgps,GOLD,LANS,geox}. Yet, it remains unclear whether these models genuinely leverage diagrams or rely almost exclusively on textual features. This ambiguity arises because existing PGPS datasets typically embed sufficient geometric details within problem statements, potentially making the vision encoder unnecessary~\citep{GOLD}. \cref{fig:pgps_examples} illustrates example questions from GeoQA and PGPS9K, where solutions can be derived without referencing the diagrams.

\begin{figure}
    \centering
    \begin{subfigure}[t]{.49\linewidth}
        \centering
        \includegraphics[width=\linewidth]{latex/figures/images/geoqa_example.pdf}
        \caption{GeoQA}
        \label{fig:geoqa_example}
    \end{subfigure}
    \begin{subfigure}[t]{.48\linewidth}
        \centering
        \includegraphics[width=\linewidth]{latex/figures/images/pgps_example.pdf}
        \caption{PGPS9K}
        \label{fig:pgps9k_example}
    \end{subfigure}
    \caption{
    Examples of diagram-caption pairs and their solution steps written in formal languages from GeoQA and PGPS9k datasets. In the problem description, the visual geometric premises and numerical variables are highlighted in green and red, respectively. A significant difference in the style of the diagram and formal language can be observable. %, along with the differences in formal languages supported by the corresponding datasets.
    \label{fig:pgps_examples}
    }
\end{figure}



We propose a new benchmark created via a synthetic data engine, which systematically evaluates the ability of VLM vision encoders to recognize geometric premises. Our empirical findings reveal that previously suggested self-supervised learning (SSL) approaches, e.g., vector quantized variataional auto-encoder (VQ-VAE)~\citep{unimath} and masked auto-encoder (MAE)~\citep{scagps,geox}, and widely adopted encoders, e.g., OpenCLIP~\citep{clip} and DinoV2~\citep{dinov2}, struggle to detect geometric features such as perpendicularity and degrees. 

To this end, we propose \geoclip{}, a model pre-trained on a large corpus of synthetic diagram–caption pairs. By varying diagram styles (e.g., color, font size, resolution, line width), \geoclip{} learns robust geometric representations and outperforms prior SSL-based methods on our benchmark. Building on \geoclip{}, we introduce a few-shot domain adaptation technique that efficiently transfers the recognition ability to real-world diagrams. We further combine this domain-adapted GeoCLIP with an LLM, forming a domain-agnostic VLM for solving PGPS tasks in MathVerse~\citep{mathverse}. 
%To accommodate diverse diagram styles and solution formats, we unify the solution program languages across multiple PGPS datasets, ensuring comprehensive evaluation. 

In our experiments on MathVerse~\citep{mathverse}, which encompasses diverse plane geometry tasks and diagram styles, our VLM with a domain-adapted \geoclip{} consistently outperforms both task-specific PGPS models and generalist VLMs. 
% In particular, it achieves higher accuracy on tasks requiring geometric-feature recognition, even when critical numerical measurements are moved from text to diagrams. 
Ablation studies confirm the effectiveness of our domain adaptation strategy, showing improvements in optical character recognition (OCR)-based tasks and robust diagram embeddings across different styles. 
% By unifying the solution program languages of existing datasets and incorporating OCR capability, we enable a single VLM, named \geovlm{}, to handle a broad class of plane geometry problems.

% Contributions
We summarize the contributions as follows:
We propose a novel benchmark for systematically assessing how well vision encoders recognize geometric premises in plane geometry diagrams~(\cref{sec:visual_feature}); We introduce \geoclip{}, a vision encoder capable of accurately detecting visual geometric premises~(\cref{sec:geoclip}), and a few-shot domain adaptation technique that efficiently transfers this capability across different diagram styles (\cref{sec:domain_adaptation});
We show that our VLM, incorporating domain-adapted GeoCLIP, surpasses existing specialized PGPS VLMs and generalist VLMs on the MathVerse benchmark~(\cref{sec:experiments}) and effectively interprets diverse diagram styles~(\cref{sec:abl}).

\iffalse
\begin{itemize}
    \item We propose a novel benchmark for systematically assessing how well vision encoders recognize geometric premises, e.g., perpendicularity and angle measures, in plane geometry diagrams.
	\item We introduce \geoclip{}, a vision encoder capable of accurately detecting visual geometric premises, and a few-shot domain adaptation technique that efficiently transfers this capability across different diagram styles.
	\item We show that our final VLM, incorporating GeoCLIP-DA, effectively interprets diverse diagram styles and achieves state-of-the-art performance on the MathVerse benchmark, surpassing existing specialized PGPS models and generalist VLM models.
\end{itemize}
\fi

\iffalse

Large language models (LLMs) have made significant strides in automated math word problem solving. In particular, their code-generation capabilities combined with proof assistants~\citep{lean,isabelle} help minimize computational errors~\citep{POT}, improve solution precision~\citep{autoformalization}, and offer rigorous feedback and evaluation~\citep{MATH}. Although LLMs excel in generating solution steps and correct answers for algebra and calculus~\citep{math_solving}, their uni-modal nature limits performance in domains like plane geometry, where both diagrams and text are vital.

Plane geometry problem solving (PGPS) tasks typically include diagrams and textual descriptions, requiring solvers to interpret premises from both sources. To facilitate automated solutions for these problems, several studies have introduced formal languages tailored for plane geometry to represent solution steps as a program with training datasets composed of diagrams, textual descriptions, and solution programs~\citep{geoqa,unigeo,intergps,pgps}. Building on these datasets, a number of PGPS specialized vision-language models (VLMs) have been developed so far~\citep{GOLD, LANS, geox}.

Most existing VLMs, however, fail to use diagrams when solving geometry problems. Well-known PGPS datasets such as GeoQA~\citep{geoqa}, UniGeo~\citep{unigeo}, and PGPS9K~\citep{pgps}, can be solved without accessing diagrams, as their problem descriptions often contain all geometric information. \cref{fig:pgps_examples} shows an example from GeoQA and PGPS9K datasets, where one can deduce the solution steps without knowing the diagrams. 
As a result, models trained on these datasets rely almost exclusively on textual information, leaving the vision encoder under-utilized~\citep{GOLD}. 
Consequently, the VLMs trained on these datasets cannot solve the plane geometry problem when necessary geometric properties or relations are excluded from the problem statement.

Some studies seek to enhance the recognition of geometric premises from a diagram by directly predicting the premises from the diagram~\citep{GOLD, intergps} or as an auxiliary task for vision encoders~\citep{geoqa,geoqa-plus}. However, these approaches remain highly domain-specific because the labels for training are difficult to obtain, thus limiting generalization across different domains. While self-supervised learning (SSL) methods that depend exclusively on geometric diagrams, e.g., vector quantized variational auto-encoder (VQ-VAE)~\citep{unimath} and masked auto-encoder (MAE)~\citep{scagps,geox}, have also been explored, the effectiveness of the SSL approaches on recognizing geometric features has not been thoroughly investigated.

We introduce a benchmark constructed with a synthetic data engine to evaluate the effectiveness of SSL approaches in recognizing geometric premises from diagrams. Our empirical results with the proposed benchmark show that the vision encoders trained with SSL methods fail to capture visual \geofeat{}s such as perpendicularity between two lines and angle measure.
Furthermore, we find that the pre-trained vision encoders often used in general-purpose VLMs, e.g., OpenCLIP~\citep{clip} and DinoV2~\citep{dinov2}, fail to recognize geometric premises from diagrams.

To improve the vision encoder for PGPS, we propose \geoclip{}, a model trained with a massive amount of diagram-caption pairs.
Since the amount of diagram-caption pairs in existing benchmarks is often limited, we develop a plane diagram generator that can randomly sample plane geometry problems with the help of existing proof assistant~\citep{alphageometry}.
To make \geoclip{} robust against different styles, we vary the visual properties of diagrams, such as color, font size, resolution, and line width.
We show that \geoclip{} performs better than the other SSL approaches and commonly used vision encoders on the newly proposed benchmark.

Another major challenge in PGPS is developing a domain-agnostic VLM capable of handling multiple PGPS benchmarks. As shown in \cref{fig:pgps_examples}, the main difficulties arise from variations in diagram styles. 
To address the issue, we propose a few-shot domain adaptation technique for \geoclip{} which transfers its visual \geofeat{} perception from the synthetic diagrams to the real-world diagrams efficiently. 

We study the efficacy of the domain adapted \geoclip{} on PGPS when equipped with the language model. To be specific, we compare the VLM with the previous PGPS models on MathVerse~\citep{mathverse}, which is designed to evaluate both the PGPS and visual \geofeat{} perception performance on various domains.
While previous PGPS models are inapplicable to certain types of MathVerse problems, we modify the prediction target and unify the solution program languages of the existing PGPS training data to make our VLM applicable to all types of MathVerse problems.
Results on MathVerse demonstrate that our VLM more effectively integrates diagrammatic information and remains robust under conditions of various diagram styles.

\begin{itemize}
    \item We propose a benchmark to measure the visual \geofeat{} recognition performance of different vision encoders.
    % \item \sh{We introduce geometric CLIP (\geoclip{} and train the VLM equipped with \geoclip{} to predict both solution steps and the numerical measurements of the problem.}
    \item We introduce \geoclip{}, a vision encoder which can accurately recognize visual \geofeat{}s and a few-shot domain adaptation technique which can transfer such ability to different domains efficiently. 
    % \item \sh{We develop our final PGPS model, \geovlm{}, by adapting \geoclip{} to different domains and training with unified languages of solution program data.}
    % We develop a domain-agnostic VLM, namely \geovlm{}, by applying a simple yet effective domain adaptation method to \geoclip{} and training on the refined training data.
    \item We demonstrate our VLM equipped with GeoCLIP-DA effectively interprets diverse diagram styles, achieving superior performance on MathVerse compared to the existing PGPS models.
\end{itemize}

\fi 


\section{Problem description}\label{sec:PF}


We consider the Cyber-Physical System (CPS) in Fig.~\ref{fig:sys}. This includes plant $\PP$, controller and anomaly detector $\CC$, mWM filters $\WW,\QQ,\GG,\HH$, and the malicious agent $\mathcal{A}$. The mWM filters are defined pairwise, namely $\{\QQ,\WW\}$ and $\{\GG,\HH\}$ are referred to as, respectively, the output and input \textit{mWM filter pairs}. 
% In the following, we provide the description of the CPS, and define the problem.

\begin{figure}
    \centering
    \includegraphics[width=6.5cm]{NCS.eps}
    \caption{Block diagram of the closed-loop CPS including the plant $\PP$, controller $\CC$ and watermarking filters $\{\WW,\QQ,\GG,\HH\}$. The information transmitted between $\PP$ and $\CC$ is altered by the adversary $\mathcal A$. The dashed lines represent the network affected by the adversary.}
    \label{fig:sys}
\end{figure}

\subsection{Plant and controller}
Consider an LTI discrete-time (DT) plant modeled by: 
% whose dynamics are described as:
	\begin{equation}\label{eq:sys}
	    \PP: \left\{ \begin{aligned}
	        x_p[k+1] &= A_p x_p[k] + B_p u_h[k]\\
	        y_p[k] &= C_p x_p[k]\\
            y_J[k] &= C_J x_p[k] + D_J u_h[k]
	    \end{aligned}\right.
	\end{equation}
	where $x_p \in \mathbb R^n$ is the plant's state, $u_h \in \mathbb R^m$ its input, $y_p \in \mathbb R^p$ its measured output, and all the system's matrices are of the appropriate dimension. Furthermore, suppose a (possibly unmeasured) \textit{performance output} $y_J \in \mathbb{R}^{p_J}$ is defined, such that the performance of the system, evaluated over the interval $[k-N+1,k]$, for some $N \in \mathbb{N}$ \citep{zhou1996robust}, is given by:
	\begin{align}
	    J(x_p,u_h) &= %\| C_J x_p + D_J u_h \|^2_{\ell_2,[k-N+1,k]}= 
        \|y_J\|^2_{\ell_2,[k-N+1,k]}.
	\end{align}
	% Note that $y_J$ can be viewed as a \textit{virtual} output, i.e., it may not be measured directly. Next, we establish the following assumption.
    \begin{assumption}
    The tuples $(A_p,B_p)$ and $(C_p,A_p)$ are respectively, controllable and observable pairs.
		$\hfill\triangleleft$
	\end{assumption}
 \begin{assumption}\label{ass:stable}
     The plant $\mathcal{P}$ is stable and $x_p[0]=0$. $\hfill\triangleleft$
 \end{assumption}

Assumption~\ref{ass:stable}, necessary for the OOG to be meaningful \citep{teixeira2015secure}, does not reduce generality, as stability can be ensured by a local (non-networked) controller \citep{hu2007stability,lin2023secondary}, whilst $x_p[0] = 0$ can be considered because of linearity.

% \begin{remark}
%     Assumption~\ref{ass:stable} does not reduce generality, and is required for the output-to-output gain to be meaningful \citep{teixeira2015strategic}. The first statement can be achieved by introducing a two-level controller: a local feedback controller for stabilization, while a networked controller improves performance, changes setpoints, etc. \citep{hu2007stability,lin2023secondary}. The second statement follows from superposition.
%     $\hfill\triangleleft$
% \end{remark}

The plant is regulated by an observer-based dynamic controller $\CC$, described by:
\begin{equation}\label{eq:cntrl}
    \mathcal C: \left\{
	\begin{aligned}
		\hat{x}_p[k+1] &= A_p \hat{x}_p[k] + B_p u_c[k] + Ly_r[k]\\
		u_c[k] &= K\hat{x}_p[k]\\
		\hat{y}_p[k] &= C_p \hat{x}_p[k]\\
		y_r[k] &= y_q[k] - \hat{y}_p[k]
	\end{aligned}\right.
\end{equation}
where $\hat{x}_p \in \mathbb{R}^n, \hat{y}_p \in \mathbb{R}^p$ are the state and measurement estimates, $u_c \in \mathbb{R}^m$ the control input. The matrices $K$ and $L$ are the controller and observer gains respectively. Finally, the term $y_r$ in \eqref{eq:cntrl} is the residual output, used to detect the presence of an attack: given a threshold $\epsilon_r$, an attack is detected if the inequality $\|y_r\|_{\ell_2,[0,N_r]}^2 \leq \epsilon_r$ is falsified for any $N_r \in \mathbb{N}_+$. Note that in \eqref{eq:sys}-\eqref{eq:cntrl} $y_q$ and $u_h$, the outputs of $\QQ$ and $\HH$ (to be defined), are used as the input to the controller and the plant, respectively. 
% These are the output signals of watermark removers $\QQ$ and $\HH$, to be defined. 

\subsection{Multiplicative watermarking filters}
Consider mWM filters defined as follows
\begin{equation}\label{eq:sys:WM}
		\Sigma: \begin{cases}
			x_\sigma[k+1] = A_\sigma(\theta_\sigma[k]) x_\sigma [k] + B_\sigma(\theta_\sigma[k]) \nu_\sigma [k]\\
			\gamma_\sigma [k] = C_\sigma(\theta_\sigma[k]) x_\sigma [k] + D_\sigma(\theta_\sigma[k]) \nu_\sigma [k]
		\end{cases},
	\end{equation}
	with $\Sigma \in \{\GG,\HH,\WW,\QQ\}$, $\sigma \in \{g,h,w,q\}$, \ajg{where $g,h,w,q$ refer to variables pertaining to $\mathcal G, \mathcal H, \mathcal W, \mathcal Q$, respectively}\footnote{In the sequel whenever referring to the parameters of any one of the mWM filters, the subscript $\sigma$ is used. Conversely, if referring to all parameters, $\theta$ is used.}, $x_\sigma \in \mathbb{R}^{n_\sigma}$ the state of $\Sigma$, $\nu_\sigma \in \mathbb{R}^{m_\sigma}$ its input, $\gamma_\sigma \in \mathbb{R}^{p_\sigma}$ the output, and $\theta_{\sigma}[k]$ is a vector of parameters.

\begin{definition}[mWM filter parameters]
    The parameter $\theta_\sigma[k]$ is taken to be the concatenation of the vectorized form of all matrices $A_\sigma(\cdot), B_\sigma(\cdot), C_\sigma(\cdot), D_\sigma(\cdot)$.
    $\hfill\triangleleft$
\end{definition}

The parameter $\theta_{\sigma}$ is defined to be piecewise constant:
$$\theta_\sigma[k] = \bar{\theta}_\sigma[k_i], \forall k \in \{k_i, k_i+1, \dots, k_{i+1}-1\}$$
where $k_i, i = 0,1, \dots \in \mathbb{N}_+,$ are switching instants. In the following, with some abuse of notation, the time dependencies are dropped, with $\theta_\sigma$ and $\theta_\sigma^+$ used to define the parameters before and after a switching instant, 
% with $\theta_\sigma$ written instead of $\theta_\sigma[k_i]$, and $\theta_\sigma^+$ to define the filter parameters after a switching instant, 
i.e., $\theta_\sigma = \theta_\sigma[k_i]$, $\theta_\sigma^+ = \theta_\sigma[k_{i+1}]$.
% \input{mWM_gen_remov.tex}

% It is important to note that, because of this characterization and because of the non-uniqueness of state-space representation of transfer functions, the parametrization of the transfer functions of $\Sigma$ is not unique. Two parameters $\theta_{\sigma,1}$ and $\theta_{\sigma,2}$ are said to be \textit{equivalent} if they lead to the same transfer function. In the remainder of the paper, it is supposed that the selection of a specific parameter amongst all those equivalent can be defined by imposing some \textit{structure} onto $\theta_\sigma$. In the numerical results presented in Section~\ref{sec:NE}, for example, all matrices are defined to be diagonal.

% Defining the mWM parameters as time-varying has been shown to be critical against certain classes of attacks~\citep{ferrari2020switching}.
Furthermore, all filters are taken to be square systems, i.e., $m_\sigma = p_\sigma, \forall \sigma \in \{g,h,w,q\}$, and define $\nu_g \triangleq u_c, \nu_h \triangleq \tilde{u}_g, \nu_w \triangleq y_p, \nu_q \triangleq \tilde{y}_w, \gamma_g \triangleq u_g, \gamma_h \triangleq u_h, \gamma_w \triangleq y_w, \gamma_q \triangleq y_q.$
 %    \begin{equation}\label{eq:WM:input}
 %        \begin{array}{cccc}
 %            \nu_g \triangleq u_c, & \nu_h \triangleq \tilde{u}_g, & \nu_w \triangleq y_p, & \nu_q \triangleq \tilde{y}_w, \\
 %            \gamma_g \triangleq u_g, & \gamma_h \triangleq u_h, & \gamma_w \triangleq y_w, & \gamma_q \triangleq y_q. 
 %        \end{array}
	% \end{equation}
    Here, a \textit{tilde} is used to highlight that $\tilde u_g, \tilde y_w$ are received through the insecure communication network and as such may be affected by attacks. 
%
% This choice is made as a consequence of the fact that switching the parameters of the watermarking scheme is critical to enable the detection of attacks that would otherwise be stealthy, such as covert attacks with matched parameters \cite{ferrari2020switching}. 
%
%%%%%%% DONOT REMOVE THE FOLLOWING REMARK
\begin{remark}
    The objective of this paper is to \textit{optimally} design the successive parameters of the mWM filters $\theta_\sigma^+$, given their value $\theta_\sigma$. 
    It remains out of the scope of the paper to address other aspects of the switching mechanisms, such as determining the switching time, or defining the jump functions for the states. 
    Interested readers are referred to \citep{ferrari2020switching}.
    %for an example on its definition, and an analysis of the impact on stability.
    $\hfill \triangleleft$
\end{remark}

% The following defines what properties must be satisfied at any instant $k \in \mathbb{N}_+$ by two filters to be a valid mWM pair.
\begin{definition}[Watermarking pair]\label{def:WM}
Two systems $(\mathcal W,\mathcal Q)$ \eqref{eq:sys:WM}, are a \textit{watermarking pair} if:
\begin{enumerate}[label=\alph*.]
    \item \label{def:WM:inv} $\mathcal{W}$ and $\mathcal{Q}$ are stable and invertible, i.e., exists a positive definite matrix $Z_\sigma \succ 0, \sigma \in \{w,q\}$ such that %the following Lyapunov stability conditions hold 
    \begin{equation}\label{eq:WM:stab}
        A_\sigma^\top Z_\sigma A_\sigma - Z_\sigma \prec 0;
    \end{equation}
    \item \label{def:WM:eqParam} if $\theta_w[k] = \theta_q[k]$, $y_q[k] = y_p[k]$, i.e., 
    \begin{equation}\label{eq:WM:def}
	   \QQ \triangleq \WW^{-1}\,. \qquad \triangleleft
    \end{equation}
\end{enumerate}
\end{definition}
\begin{remark}\label{rem:stabSw}
    If $Z_\sigma$ in \eqref{eq:WM:stab} is the same for all $\theta_\sigma[k], k \in \mathbb N,\sigma\in\{g,h,w,q\}$, the mWM filters, on their own, are stable under arbitrary switching, as they all share a common Lyapunov function.
    $\hfill \triangleleft$
\end{remark}

\begin{definition}[{\cite[Lemma 3.15]{zhou1996robust}}]\label{def:inv:ss}
		Define the DT transfer function resulting from the system defined by the tuple $(A,B,C,D)$ as $			G(z) = \left[\begin{array}{c|c}
				A &B\\
				\hline
				C &D
			\end{array}
			\right],$
		and suppose that $D^{-1}$ exists. Then
		\begin{equation}\label{eq:sys:inv}
			G^{-1}(z) = \left[\begin{array}{c|c}
				A - BD^{-1}C    &BD^{-1}\\
				\hline
				-D^{-1}C        &D^{-1}
			\end{array}
			\right]
		\end{equation}
		is the inverse transfer function of $G(z)$.
		$\hfill\triangleleft$
	\end{definition}
% Definition~\ref{def:inv:ss} holds true for any equivalent state space transformation of the matrices defined in \eqref{eq:sys:inv}. Next we establish the following assumption. 
\begin{assumption}\label{ass:sync}
    The mWM parameters are matched, i.e., $\theta_w[k] = \theta_q[k]$ and $\theta_g[k] = \theta_h[k]\; \forall\;k \in \mathbb{N}$.
    $\hfill \triangleleft$
\end{assumption}
% Note that Assumption~\ref{ass:sync} is equivalent to saying that all mWM parameters' switching times $k_i, i \in \mathbb N$ are synchronized.

% From Definition~\ref{def:inv:ss}, under Assumption~\ref{ass:sync}, when no attack is present on the communication network between $\PP$ and $\CC$ (i.e., $\tilde{u}_g=u_g$ and $\tilde{y}_w = y_w$), so long as $x_w[0] = x_q[0]$ and $x_g[0] = x_h[0]$, $y_q[k] = y_p[k]$ and $u_h[k] = u_c[k]$ hold for all $k \in \mathbb{N}$, i.e., the mWM filters do not influence the closed-loop dynamics \citep{ferrari2020switching}.
% the following hold:
% \begin{align}\label{eq:sys:WM:inv}
% 	&\begin{array}{ccc}
% 		\mathcal{Q}(z)\mathcal{W}(z) = I_{p}\,,&  \mathcal{H}(z)\mathcal{G}(z) = I_{m}\,, 
%         &\forall z \in \mathbb C
% 	\end{array}\\
% 	&\begin{array}{ccc}
% 		y_q[k] = y_p[k]  \,, & u_h[k] = u_c[k]\,, &\forall k \in \mathbb{N}_+,
% 	\end{array}\label{eq:sys:WM:inv:2}
% \end{align}
% so long as $x_w[0] = x_q[0], x_g[0] = x_h[0]$. 
% In other words, when no attack is present, and if the mWM parameters remain synchronized, the mWM filters do not effect the closed-loop dynamics \citep{ferrari2020switching}. 

% \begin{figure*}[t]
%     \centering
%     {\footnotesize
%     \begin{align}
%         \left[
%         \begin{array}{c|c}
%         {A} & {B}\\
%         \hline
%         \\[\dimexpr-\normalbaselineskip+1pt] \bar{C}_r & 0\\
%         \hline
%         \\[\dimexpr-\normalbaselineskip+1pt] \bar{C}_J & \bar{D}_J
%         \end{array}
%         \right] &= 
%         \left[ 
%         \begin{array}{c|c}
%         \begin{matrix}
%             A_p & B_pC_h & B_pD_hC_g & B_pD_hD_gK & 0 & 0 & 0\\
%             0 & A_h & B_hC_g & B_hD_gK & 0 & 0 & 0\\
%             0 & 0 & A_g & B_gK & 0 & 0 & 0\\
%             LD_qD_wC_p & 0 &0 & A_p+B_pK-LC_p & LC_q & LD_qC_w & -LD_qC_a\\
%             B_qD_wC_p & 0 & 0 & 0 & A_q & B_qC_w & -B_qC_a\\
%             B_wC_p & 0 & 0 & 0 & 0 & A_w & 0\\
%             0 & 0 & 0 & 0 & 0 & 0 & A_a\\
%             \hline
%             D_qD_wC_p & 0 & 0 & -C_p & C_q & D_qC_w & -D_qC_a\\
%             \hline
%             C_J & D_JC_h & D_JD_hC_g & D_JD_hD_gK & 0 & 0 & 0
%         \end{matrix}     & \begin{matrix}
%         B_pD_h \\ B_h \\ 0 \\ 0 \\ 0 \\ 0 \\ B_a \\ \hline 0 \\ \hline D_JD_h
%         \end{matrix}
%         \end{array}
%         \right]\label{matrix_strip}
%     \end{align}
%     \hrule}
% \end{figure*}

\subsection{Attack model}\label{ch:probFor:atk}
Consider the malicious agent $\mathcal A$ located in the CPS as in Fig.~\ref{fig:sys}, capable of tampering with data transmitted between $\PP$ and $\CC$. 
% \ajg{The main motivation for introducing the mWM filters on the communication channels between the plant and the controller is the possibility of a malicious agent $\mathcal A$, located in the CPS such as the one in Fig.~\ref{fig:sys}. 
% In this context, $\mathcal A$ is thought to be capable of tampering with the transmitted data between $\PP$ and $\CC$.}
Without loss of generality, the injected attacks are modeled as additive signals:
 \begin{equation}\label{eq:atk}
     \tilde{u}_g[k] \triangleq u_g[k] + \varphi_u[k],\;\;\; \tilde{y}_w[k] \triangleq y_w[k] + \varphi_y[k],
 \end{equation}
where $\varphi_u[k]$ and $\varphi_y[k]$ are actuator and sensor attack signals designed by the adversary $\mathcal A$. To properly define our design algorithm in Section~\ref{sec:main}, an explicit strategy for the attack signals $\varphi_u$ and $\varphi_y$ must be defined by the defender. %, and are therefore interpreted as a design choice.
In this paper, we focus on covert attacks \citep{smith2015covert}, which remain undetected for passive diagnosis scheme.

% In particular, we consider an adversary that adopts a worst-case attack policy: covert attack with matched parameters. %In other words, we consider an adversary which has knowledge of the plant (common assumption in security literature), and the mWM parameters (which can be obtained through system identification techniques, for instance). Such an attack policy is adopted to mitigate the worst-case attack scenarios.
% Specifically, let $\theta_\sigma^a[k]$ be the parameters of the mWM filter known by the attacker\footnote{\ajg{Here, and throughout the paper, a super- or subscript $a$ is used to indicate that a variable pertains to $\mathcal A$.}}. Then, the following assumption is required.

The covert attack strategy, under Assumption~\ref{ass:param} and~\ref{ass:atkEng}, is as follows: the malicious agent $\mathcal A$ chooses $\varphi_u[k] \in \ell_{2e}$ freely, while $\varphi_y[k]$ satisfies:
\begin{equation}\label{eq:atk:cov}
	\mathcal A:
\left\{
\begin{aligned}
    x_a[k+1] &= A_a(\theta^a) x_a[k] + B_a(\theta^a) \varphi_u[k]\\
		y_a[k] &= C_a(\theta^a) x_a[k] + D_a(\theta^a) \varphi_u[k]\\
		\varphi_y[k] &= - y_a[k]
\end{aligned}
\right.
\end{equation}
where $x_a \triangleq [x_{h,a}^\top\;x_{p,a}^\top\;x_{w,a}^\top]^\top$ is the attacker's state, and its dynamics are the same as the cascade of $\HH, \PP, \WW$, parametrized\footnote{\ajg{Here, and throughout the paper, a super- or subscript $a$ is used to indicate that a variable pertains to $\mathcal A$.}} by $\theta^a_\sigma$.

\begin{assumption}\label{ass:param}
For all $k \in [k_{i+1},k_{i+2}], i \in \mathbb N_+$, the attacker parameters $\theta_\sigma^a[k]=\theta_\sigma[k_i]$, $\sigma \in \{h,w\}. \hfill\triangleleft$
\end{assumption}

\begin{assumption}\label{ass:atkEng}
    The attack energy is bounded and finite, i.e.,: $\Vert \varphi_u\Vert_{\ell_2}^2 \leq \epsilon_a$, with $\epsilon_a$ known to $\mathcal C$. $\hfill \triangleleft$
\end{assumption}

\begin{remark}
    Assumption~\ref{ass:atkEng} is introduced as it allows for guarantees that the algorithm proposed in Section~\ref{sec:main} always returns a feasible solution (see Theorem~\ref{thm:well:posed}).
    In general, %Assumption~\ref{ass:atkEng} is limiting: indeed, 
    while it may be that the adversary has limited energy \citep{djouadi2015finite}, it is a strong assumption that the bound $\epsilon_a$ is known to the defender.
    %
    Nonetheless, the attack energy bound $\epsilon_a$ may be seen as a design variable that, together with the chosen attack model \eqref{eq:atk:cov}, facilitates the definition of a systematic design of mWM filters by the defender.
    %
    % However, given the perspective that the attack policy \eqref{eq:atk:cov} is a design choice geared toward the definition of a systematic design of mWM filters, rather than an actual attack against the system, $\epsilon_a$ itself is a design variable, and thus its value is available to the defender.
    %
    Further remarks regarding the consequences of Assumption~\ref{ass:atkEng} not holding are postponed to Remark~\ref{rem:atkEng2}, following the formal definition of the attack-energy-constrained output-to-output gain in Definition~\ref{def:o2o}. 
    $\hfill \triangleleft$
\end{remark}

% In general, the adversary has limited energy \citep{djouadi2015finite}. Thus, the adversary injects an attack for a finite amount of time (say $T_a$). Although $T_a$ is unknown, we enforce that the adversary stops the attack eventually by a adding that the attack energy should be bounded. 

% \begin{remark}
%     In \eqref{p1}, $\epsilon_r$ and $\epsilon_a$ play in a critical role.
%     Firstly, the metric is always bounded, as is demonstrated in Theorem~\ref{thm:well:posed} and thus is well-posed for a design algorithm. 
%     %
%     Furthermore, it has explicit relations to both the $H_\infty$ metric (for increasing values of $\epsilon_r$) and to the original OOG (for increasing values of $\epsilon_a$ \citep[Prop.1]{anand2023risk}.
%     %
%     Finally, let us comment on the constraint on the attack energy, and argue that the choice of $\epsilon_a$ as a design parameter, to be chosen by the defender, rather than an assumpiton on the attack strategy is warranted. To do this, we consider \eqref{p1} (which is equivalent to \eqref{o1}, under Lemma~\ref{lem:sig_2_mat}) under increasing values of $\epsilon_a$, as well as the OOG as defined in \cite{teixeira2015strategic}.
%     If there are no zero dynamics in the system, the OOG is finite, and thus there is some value $\bar\epsilon_a$ such that, for all $\epsilon_a > \bar\epsilon_a$, the value of $\|y_r\|_{\ell_2}^2 = \epsilon_r$, and $\|y_J\|_{\ell_2}^2$ remains constant. Thus, defining $\epsilon_a \geq \bar \epsilon_a$, the optimal value $\theta^{+*}$ is not influenced by the constraint on the attack energy.
%     On the other hand, if there are zero dynamics that the attacker can exploit, the OOG is infinite, and $\|y_J\|_{\ell_2}^2$ grows unbounded as $\epsilon_a \rightarrow \infty$.
%     Thus, while $\theta^{+*}$, the solution to \eqref{o1}, is only optimal for $\|\varphi_u\| \leq \epsilon_a$, defining $\theta^*$ ensures the effect of $\varphi_u$ on $y_J$ is in some sense minimal if this is not the case.
%     % if the constraint is violated, defining $\theta^*$ ensures that the effect of an attack with unbounded energy on the performance output is in some sense minimal.
%     $\hfill\triangleleft$
% \end{remark}

% The AEC-OOG summarizes the adversary's objective of obtaining the maximum impact on the performance output, while remaining undetected. Compared to the output-to-output gain (OOG) proposed in \cite{teixeira2015strategic}, the attack signal energies are considered to always be bounded by $\epsilon_a$ -- a property exploited in Theorem~\ref{thm:well:posed}. This bound is treated as a \textit{design variable} in the hands of the defender, not as a constraint on the class of malicious signals that can be injected on the system. Further comments on this are given in Remark~\ref{rem:atkE}, in the following.

% \begin{align} 
% &\left[
% \begin{array}{c|c}
% {A}_a(\theta^a) & {B}_a(\theta^a)\\
% \hline
% \\[\dimexpr-\normalbaselineskip+2pt] {C}_a(\theta^a) & {D}_a(\theta^a)
% \end{array}
% \right] \triangleq\\
% &\left[ 
% \begin{array}{c|c}
% \begin{matrix}
% A_{h}(\theta_h^a) & 0 & 0 \\
% B_pC_h(\theta_h^a) & A_p & 0\\
% 0 & B_w(\theta_w^a)C_p & A_w(\theta_w^a)\\
% \hline 
% 0 & \quad D_w(\theta_w^a)C_p & C_w(\theta_w^a)
% \end{matrix}     & \begin{matrix}
% B_h(\theta_h^a) \\ B_pD_h(\theta_h^a) \\ 0 \\ \hline 0
% \end{matrix}
% \end{array}
% \right].
% \end{align} 
%In general, covert attacks with matched parameters remain undetectable to any diagnosis scheme by construction, while still having an impact on the performance of the system, because of the design of $\varphi_u$. \ajg{Thus, we consider the the worst-case covert attack policy, and provide a design algorithm to mitigate such attacks.}

\subsection{Problem formulation}\label{sec:PF:probFor}
The objective of this paper is to propose a design strategy capable of optimally designing the mWM filter parameters $\theta^+$, supposing a covert attack is present within the CPS. To formulate a metric to be used to define optimality, the closed-loop CPS dynamics must be defined. Under the attack strategy \eqref{eq:atk:cov}, the closed-loop system with the attack $\varphi_u$ as input and the performance and detection output as system outputs can be rewritten as
\begin{equation}\label{eq:S_cl}
		{\mathcal{S}}:\left\{
\begin{aligned}
{x}[k+1] &= A x[k] + B\varphi_u[k]\\
y_J[k] &= \bar{C}_J x[k] + \bar{D}_J \varphi_u[k]\\
y_r[k] &= \bar{C}_r x[k]
\end{aligned} \right.
	\end{equation}
where $x = \begin{bmatrix}x_p^\top, &x_h^\top, &x_g^\top, &x_c^\top, &x_q^\top, &x_w^\top, &x_a^\top\end{bmatrix}^\top$ is the closed-loop system state, while $y_r$ and $y_J$ remain the residual and performance outputs. All signals in \eqref{eq:S_cl} are also a function of the parameters $\theta^+$, but this dependence is dropped for clarity.
The definition of the matrices in \eqref{eq:S_cl} follow from \eqref{eq:sys}-\eqref{eq:sys:WM} and \eqref{eq:atk:cov}.
% All matrices in \eqref{eq:S_cl} are provided in \eqref{matrix_strip}.

% For the closed loop CPS dynamics in \eqref{eq:S_cl}, 
The defender aims to quantify (and later minimize) the maximum performance loss caused by a stealthy and bounded-energy adversary on \eqref{eq:S_cl}. 
This is done by exploiting the attack-energy-constrained output-to-output gain 
(AEC-OOG) \citep{anand2023risk}. %\footnote{The AEC-OOG is a restriction of the previously defined output-to-output gain (OOG) \citep{teixeira2015strategic}, by imposing that all attack signal energies are bounded.}
\begin{definition}[AEC-OOG]\label{def:o2o}
	The AEC-OOG
    % attack\hyp{}energy\hyp{}constrained output-to-output gain 
    of $\mathcal S$ in \eqref{eq:S_cl} is the value of the following optimization problem:
		\begin{equation}\label{eq:o2o}
			\begin{aligned}
				\sup_{\varphi_u\in\ell_{2e}} &\quad \|y_J\|_{\ell_2}^2 \\
				\text{s.t.}& \quad  \|y_r\|_{\ell_2}^2 \leq \epsilon_r,\; \|\varphi_u\|_{\ell_2}^2 \leq \epsilon_a,\;x[0] = 0.
			\end{aligned}
		\end{equation}
	where $\epsilon_a$ is the energy bound of the attack signal, $\epsilon_r$ is the detection threshold, and the value of \eqref{eq:o2o} denotes the performance loss caused by a stealthy adversary.	$\hfill\triangleleft$
	\end{definition}
% The AEC-OOG summarizes the adversary's objective of obtaining the maximum impact on the performance output, while remaining undetected. Compared to the output-to-output gain (OOG) proposed in \cite{teixeira2015strategic}, the attack signal energies are considered to always be bounded by $\epsilon_a$ -- a property exploited in Theorem~\ref{thm:well:posed}. This bound is treated as a \textit{design variable} in the hands of the defender, not as a constraint on the class of malicious signals that can be injected on the system. Further comments on this are given in Remark~\ref{rem:atkE}, in the following. 
%, to ensure our design algorithm is always feasible. %Further discussion regarding the AEC-OOG can be found in \cite{anand2023risk}.
% The AEC-OOG summarizes the following case: the goal of the adversary is to maximize the performance signal energy (as opposed to classical $\mathcal{H}_{\infty}$ control) while remaining undetected. The latter objective translates to constraining the detection output to remain under the predefined threshold value $\epsilon_r$. 
% \ajg{In this definition, we take the attack input to be bounded energy, where $\epsilon_a$ acts as this bound. The inclusion of this constraint in \eqref{eq:o2o} ensures that the AEC-OOG is always bounded, a property that is exploited in Theorem~\ref{thm:well:posed} to ensure that $\theta_\sigma^+$ can always be defined. 
% }
% Using the definition above, the main problem studied in this paper can be formalized.

\begin{problem}\label{problem_main}
Given $\theta_{\sigma}$ at some switching time $k_i, i \in \mathbb{N}_+$, find the optimal set of mWM filter parameters after a switching event $\theta_{\sigma}^+$, such that the AEC-OOG of the system $\mathcal{S}$ in \eqref{eq:S_cl} is minimized. $\hfill \triangleleft$
\end{problem}

\begin{remark}
    Because of its dependence on the AEC-OOG, the solution of Problem~\ref{problem_main} at time $k_i$ relies explicitly on the attack parameters $\theta^a[k_i]$. Given the malicious agent's strategy outlined in Section~\ref{ch:probFor:atk}, and Assumption~\ref{ass:param}, $\theta_\sigma^a[k_i] = \theta_\sigma^+[k_{i-1}]$ is known to $\CC$, without any additional knowledge required. %Thus, no additional knowledge is required by the defense mechanism in $\CC$ to define the closed-loop matrices in \eqref{eq:S_cl}.
    $\hfill \triangleleft$
\end{remark}

\begin{remark}\label{rem:atkEng2}
    We are now ready to formally treat the violation of Assumption~\ref{ass:atkEng}.
    To do this, let us first remark on some properties of the AEC-OOG, which follow from using finite bounds $\epsilon_r$ and $\epsilon_a$.
    Firstly, as demonstrated in Theorem~\ref{thm:well:posed}, the metric is always bounded, making it well suited for a design algorithm.
    Furthermore, it is explicitly related to both the $H_\infty$ metric and the original OOG proposed in \cite{teixeira2015strategic}, for increasing values of $\epsilon_r$ and $\epsilon_a$, respectively \citep[Prop.1]{anand2023risk}.
    Finally, we can comment on the constraint on the attack energy.
    Consider the value of \eqref{eq:o2o} under increasing values of $\epsilon_a$, as well the OOG as defined in \cite{teixeira2015strategic}.
    If the OOG is finite, there is some value $\bar\epsilon_a$ such that the AEC-OOG is the same as the OOG for all $\epsilon_a \geq \bar\epsilon_a$.
    If there are exploitable zero dynamcis, and the OOG is unbounded, $\|y_J\|_{\ell_2}^2$ grows unbounded as $\epsilon_a \rightarrow \infty$. Thus, while $\theta_\sigma^+$, the solution to Problem~\ref{problem_main}, is only optimal for covert attacks satisfying $\|\varphi_u\|_{\ell_2}^2 \leq \epsilon_a$, it ensures that the effect of $\varphi_u$ on $y_J$ is in some sense minimal if the attack energy constraint is violated.
    $\hfill\triangleleft$
\end{remark}

% The AEC-OOG summarizes the adversary's objective of obtaining the maximum impact on the performance output, while remaining undetected. Compared to the output-to-output gain (OOG) proposed in \cite{teixeira2015strategic}, the attack signal energies are considered to always be bounded by $\epsilon_a$ -- a property exploited in Theorem~\ref{thm:well:posed}. This bound is treated as a \textit{design variable} in the hands of the defender, not as a constraint on the class of malicious signals that can be injected on the system. Further comments on this are given in Remark~\ref{rem:atkE}, in the following.


% \begin{remark}
% \ajg{The AEC-OOG defined in \eqref{eq:o2o} can be related to other metrics in the literature namely the $H_{\infty}$ metric and the OOG. Let $\gamma_{\infty}$ and  $\gamma_{\text{OOG}}$ denote the $H_{\infty}$ gain, $\gamma_{OOG}$ of the closed loop system \eqref{eq:S_cl}, and $\gamma(\epsilon_a,\epsilon_r)$ denote the value of AEC-OOG for any given value of $\epsilon_a$ and $\epsilon_r$. Then it holds that $\lim_{\epsilon_a \to \infty} \gamma(\epsilon_a,\epsilon_r) = \gamma_{\text{OOG}}$ and $\lim_{\epsilon_r \to \infty} \gamma(\epsilon_a,\epsilon_r) = \gamma_{\infty}$} $\hfill \triangleleft$
% \end{remark}

%
\section{Optimal design of filters}\label{sec:main}
\section{The \search\ Search Algorithm}
\label{sec:search}

%In traditional ML, structure changes and step (operator) changes are performed before model training, \ie, fixed to the training process, and weights are updated with SGD, because weights are continous, differentiable values, and there are significantly more weights than structure and operator changes. In workflow autotuning, all three types of cogs can be chosen with a unified search-based approach, because all of them are non-differentiable configurations and the number of cogs in different types are all small.
%Thus, \sysname\ only needs to navigate the search space of combination of cogs as the search space to produce its workflow optimization results.

%We propose, \textit{\textbf{\search}}, an adaptive hierarchical search algorithm that autotunes gen-AI workflows based on observed end-to-end workflow results. In each search iteration, \search\ selects a combination of cogs to apply to the workflow and executes the resulting workflow with user-provided training inputs. \search\ evaluates the final generation quality using the user-specified evaluator and measures the execution time and cost for each training input. These results are aggregated and serve as BO observations and pruning criteria.
%the optimizer can condition on and propose better configurations in later trials. The optimizer will also be informed about the violation of any user-specified metric thresholds. More details of this mechanism can be found in Appendix ~\ref{appdx:TPE}.

With our insights in Section~\ref{sec:theory}, we believe that search methods based on Bayesian Optimizer (BO) can work for all types of cogs in gen-AI workflow autotuning because of BO's efficiency in searching discrete search space.
A key challenge in designing a BO-based search is the limited search budgets that need to be used to search a high-dimensional cog space. 
For example, for 4 cogs each with 4 options and a workflow of 3 LLM steps, the search space is $4^{12}$. Suppose each search uses GPT-4o and has 1000 output tokens, the entire space needs around \$168K to go through. A user search budget of \$100 can cover only 0.06\% of the search space. A traditional BO approach cannot find good results with such small budgets.
%The entire search space grows exponentially with the number of cogs and the number of steps in a workflow. Moreover, different cogs and different combinations of cogs can have varying impacts on different workflows. 
%Without prior knowledge, it is difficult to determine the amount of budget to give to each cog.

To confront this challenge, we propose \textit{\textbf{\search}}, an adaptive hierarchical search algorithm that efficiently assigns search budget across cogs based on budget size and observed workflow evaluation results, as defined in Algorithms~\ref{alg:main} and \ref{alg:outer} and described below.
%autotunes gen-AI workflows based on observed end-to-end workflow results.
%\search\ includes a search layer partitioning method, a search budget initial assignment method, an evaluation-guided budget re-allocation mechanism, and a convergence-based early-exiting strategy. We discuss them in details below.

%\zijian{\search\ allows users to specify the optimization budget allowed in terms of the maximum number of search iterations. Based on the relationship between the complexity of the search space and the available budget, we will separate all tunable parameters into different layers each optimized by independent Bayesian optimization routines. Then we will decide the maximum budget each layer can get with a bottom-up partition strategy. Besides search space and resource partition, we also employ a novel allocation algorithm that integrates successive halving~\cite{successivehalving} and a convergence-based early exiting strategy to facilitate efficient usage of assigned budget.}


% The outermost layer searches and selects structures for a workflow; the middle layer searches and selects step options under the workflow structure selected in the outermost layer; the innermost layer searches and selects weights with the given workflow structure and steps. 

\begin{algorithm}[h]
    \caption{\search\ Algorithm}
    \label{alg:main}
      \small
\begin{algorithmic}[1]
\STATE \textbf{Global Value:} $R = \emptyset$ \COMMENT{Global result set}
%\STATE \textbf{Global Value:} $F = \emptyset$ \COMMENT{Global observation set}

%Reduct factor $\eta > 1$, explore width $W$
\STATE \textbf{Input:} User-specified Total Budget $TB$
\STATE \textbf{Input:} Cog set $C = \{c_{11},c_{12},...\}, \{c_{21},c_{22},...\}, \{c_{31},c_{32},...\}$

    \STATE
%\FOR{$i = 1,2,3$}
    %\COMMENT{$\alpha$ is a configurable value default to 1.1}
%\ENDFOR
%\STATE
%    \STATE \{$B_1,B_2,B_3$\} = LayerPartition($C$) \COMMENT{Calculate ideal layer budget}
    %\STATE \textbf{Glob}.budgets = budgets
%    \STATE opt\_layers = init\_opt\_routines() \COMMENT{A list of optimize routine each layer will use for search}
%\STATE
%    \FOR{$i \in L, \dots, 1$}
%        \IF{$i == L$}
 %           \STATE opt\_layers[L] = InnerLayerOpt
  %      \ELSE
   %         \STATE opt\_layers[i] = OuterLayerOpt
            %\STATE opt\_layers[i].next\_layer\_budgets = B[i+1]
            %\STATE opt\_layers[i].next\_layer\_routine = opt\_layers[i+1]
    %    \ENDIF
    %\ENDFOR
%\STATE opt\_layers[1].invoke($\emptyset$, B[1])
\STATE $U = 0$ \COMMENT{Used budget so far, initialize to 0}

\STATE \COMMENT{Perform search with 1 to 3 layers until budget runs out}
\FOR{$L = 1,2,3$} 
        \IF{$L=1$}
            \STATE $C_1 = C_1 \cup C_2 \cup C_3$ \COMMENT{Merge all cogs into a single layer}
        \ENDIF
        \IF{$L==2$}
            \STATE $C_1 = C_1 \cup C_2$ \COMMENT{Merge step and weight cogs}
            \STATE $C_2 = C_3$ \COMMENT{Architecture cog becomes the second layer}
        \ENDIF
        \STATE
    \FOR{$i = 1,..,L$}
    \STATE $NC_i = |C_i|$ \COMMENT{Total number of cogs in layer $L$} 
%    NO_i &= \sum_{L} \{\text{number of possible options in cog } c_{ij}\} \\
    \STATE $S_i = NC_i^\alpha$ \COMMENT{Estimated expected search size in layer $i$}
    \ENDFOR
    \STATE $E_L = \prod\limits_{i=1}^{L}S_i$ \COMMENT{Expected total search size in the current round}
    \STATE $E = TB - U > E_L$ ? $E_L$ : $(TB - U)$ \COMMENT{Consider insufficient budget} 
    \IF{$L==3$ and $(TB - U)$ > $E_L$}
         \STATE $E = TB - U$ \COMMENT{Spend all remaining budget if at 3 layer}
    \ENDIF
    %\STATE$TL = |N|$ \COMMENT{number of layers}
    \FOR{$i = 1,..,L$}
        \STATE $B_i =  \lfloor S_i \times \sqrt[L]{\frac{E}{E_L}}\rfloor$
        %$B$ = BudgetAssign($N$, $TL$, $TB$)
        \COMMENT{Assign budget proportionally to $S_i$}
    \ENDFOR
    \STATE
\STATE \texttt{LayerSearch} ($\emptyset$, $B$, $L$, $B_L$) \COMMENT{Hierarchical search from layer $L$}
\STATE
\STATE $U = U + E$
\IF{$U \geq TB$}
\STATE break \COMMENT{Stop search when using up all user budget}
\ENDIF
\ENDFOR
%\STATE
%\STATE $O$ = \texttt{SelectBestConfigs} ($R$)
%\IF{$L == 1$}
%    \STATE InnerLayerOpt($\emptyset$, B[1])
%\ELSE
%    \STATE OuterLayerOpt($\emptyset$, B[1], 1)
%\ENDIF
\STATE
\STATE \textbf{Output:} $O$ = \texttt{SelectBestConfigs} ($R$) \COMMENT{Return best optimizations}
\end{algorithmic}
\end{algorithm}

\subsection{Hierarchical Layer and Budget Partition}
\label{sec:ssp}

%We motivate \search's adaptive hierarchical search 
A non-hierarchical search has all cog options in a single-layer search space for an optimizer like BO to search, an approach taken by prior workflow optimizers~\cite{dspy-2-2024,gptswarm}.
With small budgets, a single-layer hierarchy allows BO-like search to spend the budget on dimensions that could potentially generate some improvements.
%While given enough budget, the single-layer space can be extensively searched to find global optimal, with little budget, 
However, a major issue with a single-layer search space is that a search algorithm like BO can be stuck at a local optimum even when budgets increase.
% (unless the budget is close to covering a very large space across dimensions).
To mitigate this issue, our idea is to perform a hierarchical search that works by choosing configurations in the outermost layer first, then under each chosen configuration, choosing the next layer's configurations until the innermost layer. 
With such a hierarchy, a search algorithm could force each layer to sample some values. Given enough budget, each dimension will receive some sampling points, allowing better coverage in the entire search space. However, with high dimensionality (\ie, many types of cogs) and insufficient budget, a hierarchical search may not be able to perform enough local search to find any good optimizations.

To support different user-specified budgets and to get the best of both approaches, we propose an adaptive hierarchical search approach, as shown in Algorithm~\ref{alg:main}.
\search\ starts the search by combining all cogs into one layer ($L=1$, line 9 in Algorithm~\ref{alg:main}) and estimating the expected search budget of this single layer to be the total number of cogs to the power of $\alpha$ (lines 16-19, by default $\alpha = 1.1$). This budget is then passed to the \texttt{LayerSearch} function (Algorithm~\ref{alg:outer}) to perform the actual cog search. When the user-defined budget is no larger than this estimated budget, we expect the single-layer, non-hierarchical search to work better than hierarchical search.
%as the budget for this single layer.

If the user-defined budget is larger, \search\ continues the search with two layers ($L=2$), combining step and weight cogs into the inner layer and architecture cogs as the outer layer (lines 11-14).
\search\ estimates the total search budget for this round as the product of the number of cogs in each of the two layers to the power of $\alpha$ (lines 16-20). It then distributes the estimated search budget between the two layers proportionally to each layer's complexity (lines 22-24) and calls the upper layer's \texttt{LayerSearch} function. Afterward, if there is still budget left, \search\ performs a last round of search using three layers and the remaining budget in a similar way as described above but with three separate layers (architecture as the outermost, step as the middle, and weight cogs as the innermost layer). Two or three layers work better for larger user-defined budgets, as they allow for a larger coverage of the high-dimensional search space.

Finally, \search\ combines all the search results to select the best configurations based on user-defined metrics (line 34).

%\search\ organizes cogs by having architecture cogs in the outer-most search layer, step cogs in the middle layer, and weight cogs in the inner-most layer (line 4 in Algorithm~\ref{alg:main}).
%This is because step cogs' input and output format are dependent on the workflow structure, and the effectiveness of weights (\eg, prompting) are dependent on steps (\eg, LLM model). 

% increases the number of layers until hitting the user-specified total search budget, $TB$

%Thus, the first step of \search\ is to determine the number of layers in its hierarchy and what cogs to include in a layer.
%Intuitively, structure cogs should be placed in the outer-most search layer to be determined first before exploring other cogs. This is because other cogs change node and edge values, and it is easier for 
%However, instead of a fixed number of layers in the hierarchy, we adapt the cog layering according to user-specified total search budgets, $TB$, and the complexity of each layer, using Algorithm~\ref{alg:main}.

% the following \texttt{LayerPartition} method.
%We begin by modeling the relationship between the expected number of evaluations and the number of cogs as well as the number of options in each layer:

%We first consider the identity of each cog in the search space. All structure-cogs will be placed in the outer-most search layer exclusively, which is similar to non-differentiable NAS in traditional ML training. This layer will fix the workflow graph and pass it to the following layer, allowing a stabilized search space for faster convergence.

%Since step-cogs will not create a changing search space, the partition of step-cogs and weight-cogs is conditioned on the search space complexity and the given total budget. Separating step-cogs out can benefit from a more flexible budget allocation strategy and broader exploration for local search at weight-cogs but performs poorly when the given budget is more constrained, in that case, we will optimize them jointly in the same layer.


%\small
%\begin{align*}
%    C &= \{c_{11},c_{12},...\}, \{c_{21},c_{22},...\}, \{c_{31},c_{32},...\} \\
%    NC_i &= \text{total number of cogs in layer i} \\
%    NO_i &= \sum_{j} \{\text{number of possible options in cog } c_{ij}\} \\
%    N_i &= max(NC_i^\alpha,NO_i) \\
%    N_i &= \sum_{j} \{\text{number of possible options in } C_{ij}\} \\
%    N_i &= max(|C_i|^\alpha, N_i) \\
%    B_j &= \prod\limits_{i=1}^{j}N_i, j \in \{1,2,3\}
%\end{align*}

%\normalsize
%where $L$ represents the total number of layers and can be 1, 2, or 3. 
%$C$ represents the entire cog search space, with each row $c_{i*}$ being one of the three types of cogs and lower layers having lower-numbered rows (\eg, $c_{1*}$ being weight cogs). $NC_i$ is the number of cogs in layer $i$, and $NO_i$ is the total number of options across all cogs in layer $i$. $N_i$ is our estimation of the complexity of layer $i$ based on $NC_i$ and $NO_i$ ($\alpha$ is a configurable weight to control the importance between $NC_i$ and $NO_i$; by default $\alpha = 1.1$). 
%$\alpha$ stands for a control parameter, setting the intensity of this scaling behavior w.r.t the number of cogs, we found that $\alpha = 1.2$ is empirically sufficient and efficient for optimizing real workloads. 
%$B_j$ is the expected total number of workflow evaluations for all the lower $j$ layers.
%After calculating $B_1$, $B_2$, and $B_3$, we compare the total budget $TB$ with them.
%When $TB \geq B_3$, we set the total number of layers, $TL$, to 3. When $B_2 \leq TB < B_3$, we set the total number of layers to 2 and merge the step and weight cogs into one layer. When $TB < B_1$, we put all cogs in one layer.
%We only create a separate layer for step-cogs when the given budget $TB$ is greater or equal to the total expected budget for three layers.

%\subsection{Seach Budget Partition}
%\label{sec:sbp}
%After determining cog layers, we distribute the total budget, $TB$, across the layers proportionally to each layer's expected budget $N_i$: , which is the \texttt{BudgetAssign} function.
%We follow a bottom-up partition strategy, where lower layers will try to greedily take the expected budget. This stems from two simple heuristics: (1) feedback to the upper layer is more accurate when the succeeding layer is trained with enough iterations, and (2) the effectiveness of a structure change depends on the setting of individual steps in the workflow (\eg, majority voting is more powerful when each LLM-agent is embedded with diverse few-shot examples or reasoning styles). In cases where the given resource exceeds the total expected budget, 
%We assign $TB$ across layers proportionally to their expected budget $N_i$. 
%The budget assigned at each layer $B_i$ given the total available number of evaluations $TB$ is obtained as follows:

%\small
%\begin{align}
%B_i &=  \lfloor N_i \times \sqrt[L]{\frac{TB}{B^*}}\rfloor
%    B_L &= \begin{cases}
%        min(N_L, TB) & TB < B^* \\
%        \lfloor N_L \times \sqrt[L]{\frac{TB}{B^*}}\rfloor & TB \geq B^*
%    \end{cases}
%    \\
%    B_i &= \begin{cases}
%        min(N_i, \lfloor\frac{TB}{\prod_{j=i+1}^L B_j}\rfloor) & TB < B^* \\
%        \lfloor N_i \times \sqrt[L]{\frac{TB}{B^*}}\rfloor & TB \geq B^*
%    \end{cases}
%\end{align}

%\normalsize


\subsection{Recursive Layer-Wise Search Algorithm}
%The calculation above pre-assigns cogs to layers and search budgets to each layer. 
We now introduce how \search\ performs the actual search in a recursive manner until the inner-most layer is searched, as presented in Algorithm~\ref{alg:outer} \texttt{LayerSearch}. 
Our overall goal is to ensure strong cog option coverage within each layer while quickly directing budgets to more promising cog options based on evaluation results.
%So far, we have determined the optimization layer structure and the maximum allowed search iteration each layer will get. Next, we introduce how the budget is consumed in each layer. The inner-most layer, where weight-cogs, and potentially step-cogs, reside, follows the conventional Bayesian optimization process, exhausting all budgets unless an early stop signal is sent. This signal will be triggered when the current optimizer witnesses $p$ consecutive iterations without any improvements above the threshold. All optimization layers use early stopping to avoid budget waste.
%Algorithm~\ref{alg:inner} describes the search happening at the inner-most (bottom) layer, and 
Specifically, every layer's search is under a chosen set of cog configurations from its upper layers ($C_{chosen}$) and is given a budget $b$. 
In the inner-most layer (lines 7-20), \search\ samples $b$ configurations and evaluates the workflow for each of them together with the configurations from all upper layers ($C_{chosen}$). The evaluation results are added to the feedback set $F$ as the return of this layer.

\begin{algorithm}[h]
  %\algsetup{linenosize=\tiny}
  \small
    \caption{\texttt{LayerSearch} Function}
    \label{alg:outer}
\begin{algorithmic}[1]
%\STATE \textbf{Global Config:} Reduct factor $\eta > 1$, explore width $W$
\STATE \textbf{Global Value:} $R$ \COMMENT{Global result set}
%\STATE \textbf{Global Value:} $F$ \COMMENT{Global observation set}
\STATE \textbf{Input:} $C_{chosen}$: configs chosen in upper layers
\STATE \textbf{Input:} $B$: Array storing assigned budgets to different layers
\STATE \textbf{Input:} $curr\_layer$: this layer's level
\STATE \textbf{Input:} $curr\_b$: this layer's assigned budget
%\STATE
%\FUNCTION{LayerSearch\hspace{0.4em}($C_{chosen}$, $B$, $curr\_layer$, $curr\_b$)}

    \STATE
    \STATE \COMMENT{Search for inner-most layer}
    \IF{curr\_layer == 1}
        \STATE $F = \emptyset$ \COMMENT{Init this layer's feedback set to empty}
        %\STATE $F^{\prime} = match(C_{chosen}, F)$ \COMMENT{Local feedback set}
        \FOR{$k = 0, \dots, curr\_b$}
            \STATE $\lambda$ = \texttt{TPESample} (1) \COMMENT{Sample one configuration using TPE}
            \STATE $f = $ \texttt{EvaluateWorkflow} ($C_{chosen} \cup \lambda$)
            \STATE $R = R \cup \{C_{chosen} \cup \lambda\}$ \COMMENT{Add configuration to global $R$}
            \IF{\texttt{EarlyStop} (f)}
            \STATE break
            \ENDIF
            \STATE $F = F \cup \{f\}$ \COMMENT{Add evaluate result to feedback $F$}
        \ENDFOR
        %\STATE $F = F \cup F^{\prime}$
        \STATE \textbf{Return} $F$
    \ENDIF
    \STATE
    \STATE \COMMENT{Search for non-inner-most layer}
    %\STATE $K = \lfloor \frac{b}{W} \rfloor$, 
    \STATE $b\_used = 0$, $TF = \emptyset$ \COMMENT{Init this layer's used budget and feedback set}
    \STATE $R = \lceil\frac{curr\_b}{\eta}\rceil$, $S = \lfloor\frac{curr\_b}{R}\rfloor$ \COMMENT{Set $R$ and $S$ based on $curr\_b$}
    \STATE
    \WHILE{$b\_{used}$ $\leq$ $curr\_b$}
        \STATE \COMMENT{Sample $W$ configs at a time until running out of $curr\_b$}
        \STATE $n = (curr\_b - b_{used})$ > $W$ ? $W$ : $(curr\_b - b_{used})$
        %\IF{$b - b_{used} < W$}
        %    \STATE $n = b_l - b_{used}$
        %\ELSE
         %   \STATE $n=W$
        %\ENDIF
        \STATE $b\_used$ += $n$
        %\STATE $n = \text{min}(W,\ b_l - kW)$ \COMMENT{Propose $W$ configs and meet $b_l$ constraint}
        \STATE $\Theta = $ \texttt{TPESample} ($n$) \COMMENT{Sample a chunk of $n$ configs in the layer} 
        %\STATE $F^{\prime} = match(C_{chosen}, F)$ \COMMENT{Per-chunk feedback set}
        \STATE $F = \emptyset$ \COMMENT{Init this layer's feedback set to empty}
        \STATE
        \FOR{$s = 0, 1, \dots, S$}
            \STATE $r_s = R\cdot \eta^s$
            \FOR{$\theta \in \Theta$}
                %\IF{$curr\_layer < max\_layer$}
                    \STATE $f =$ \texttt{LayerSearch} ($C_{chosen} \cup \{\theta\}$, $B$, curr\_layer$-1$, $r_s$)
                %\ELSE
                %    \STATE $f =$ InnerOpt($\gamma \cup \{\theta\}$, $r_s$)
                %\STATE $f$ = $opt\_layers[current\_layer+1](\gamma \cup \{\theta\}, r_s)$ \{Optimize the current config at the next layer with $r_s$ budget \}
                %\ENDIF
                \STATE $F = F \cup f$ \COMMENT{Add evaluate result to feedback}
                \IF{\texttt{EarlyStop} ($f$)}
                    \STATE $\Theta = \Theta - \{\theta\}$ \COMMENT{Skip converged configs}
                \ENDIF
            \ENDFOR
            \STATE $\Theta$ = Select top $\lfloor \frac{|\Theta|}{\eta}\rfloor$ configs from $F$ based on user-specified metrics
        \ENDFOR
        \STATE
        \IF{\texttt{EarlyStop} ($F$)}
            \STATE break \COMMENT{Skip remaining search if results converged}
        \ENDIF
        \STATE $TF = TF \cup F$
    \ENDWHILE
    %\STATE $F = F \cup TF$
        \STATE \textbf{Return} $TF$

%\ENDFUNCTION

%\STATE \textbf{Output:} Best metrics in all trials
\end{algorithmic}
\end{algorithm}

% consumption\_nextlayer\_bucket = WSR

% for s in 0, 1,...S do
%     w = W*\eta^{s}
%     r = R*\eta^{-s}

% total budget at next layer = b_l / W * WSR = b_l * SR

% b_l * SR <= b_l * B_l+1

% S = B_{l+1} / R



For a non-inner-most layer, \search\ samples a chunk ($W$) of points at a time using the TPE BO algorithm~\cite{bergstra2011tpe} until all this layer's pre-assigned budget is exhausted (lines 27-30). Within a chunk, \search\ uses a successive-halving-like approach to iteratively direct the search budget to more promising configurations within the chunk (the dynamically changing set, $\Theta$). In each iteration, \search\ calls the next-level search function for each sampled configuration in $\Theta$ with a budget of $r_s$ and adds the evaluation observations from lower layers to the feedback set $F$ for later TPE sampling to use (lines 35-37).
In the first iteration ($s=0$), $r_s$ is set to $R\cdot \eta^0=R$ (line 34). After the inner layers use this budget to search, \search\ filters out configurations with lower performance and only keeps the top $\lfloor \frac{|\Theta|}{\eta}\rfloor$ configurations as the new $\Theta$ to explore in the next iteration (line 42). In each next iteration, \search\ increases $r_s$ by $\eta$ times (line 34), essentially giving more search budget to the better configurations from the previous iteration.

The successive halving method effectively distributes the search budget to more promising configurations, while the chunk-based sampling approach allows for evaluation feedback to accumulate quickly so that later rounds of TPE can get more feedback (compared to no chunking and sampling all $b$ configurations at the same time). To further improve the search efficiency, we adopt an {\em early stop} approach where we stop a chunk or a layer's search when we find its latest few searches do not improve workflow results by more than a threshold, indicating convergence (lines 14,38,45).

%algorithm takes as input other cog settings from previous layers and the assigned budget at the current layer. It tiles the search loop into fixed-size blocks (line 4), each runs the SuccessiveHalving (SH) subroutine in the inner loop (line 7-15). In each SH iteration, only top-$1/\eta$-quantile configurations in $\Theta$ will continue in the next round with $\eta$ times larger budget consumption. As a result, exponentially more trials will be performed by more promising configurations. 

%On average, \textit{Outer-layer search} will create $K$ brackets, each granting approximately $WRS$ budget to the next layer. $R$ represents the smallest amount of resource allocated to any configurations in $\Theta$. 

% layer - 1: budget = 4
% K * W <= b\_current layer
% layer -1: itear 0: propose 2

%     SH:
%     2 config -> R
%     1 config -> 2R

%     iteration 1: propose 2 = W
%     SH:
%     2 config -> R
%     1 config -> 2R

% W configs; each has R resource

% W / eta configs; each has R * eta resource

% R -> least resource one config can get = B2 - smth
% R + R*eta + ... + R*eta\^s -> most promising = B2 + smth


% $L2$ is the middle layer where structure-cogs and step-cogs may be placed exclusively. We employ hyperband for its robustness in exploration and exploitation trace-off. If this layer exists, it will instruct $L1$ the number of search iterations to run in each invocation. Specifically, in each iteration at line 4, \sysname will propose $n$ configurations and run SuccessiveHalving (SH) subroutine (line 8-15). SH will optimize each proposal and use the search results from $L1$ to rank their performance. Each time only the top-performing $n \cdot \eta^{-i}$ can continue in the next round with a larger budget. With this strategy, exponentially more search budgets are allocated to more promising configs at $L2$.

% \input{algo-l2-search}

% $L3$ is the outer-most layer for structure-cogs only when $L2$ is created. For this layer, we employ plain SH without hyperband because of its predictable convergence behavior. This is mainly due to two factors: (1) structure change to the workflow is more significant thus different configurations are more likely to deviate after training with the following layers. (2) with the search space partition strategy in Sec ~\ref{sec:ssp}, we can assume the available budget at each layer is substantial when $L3$ exists. Given this prior knowledge, we can avoid grid searching control parameter $n$ as in the hyperband but adopt a more aggressive allocation scheme to bias towards better proposals and moderate search wastes.



%\subsubsection{Runtime Budget Adaptation}
%Using static estimation of the expected budget for each layer is not enough, we also adjust the assignment during the optimization based on real convergence behavior. Specifically, for layer $i$, we record the number of configurations evaluated in each optimize routine. We set the convergence indicator $C_{ij}$ of $j^{th}$ routine with this number if the search early exits before reaching the budget limit, otherwise 2\x of its assigned resource. Then we update $E_i$ with $\frac{\sum_{j}^M C_{ij}}{M}$. \sysname\ will update the budget partition according to Sec~\ref{sec:sbp} for any newly spawned optimizer routines. Besides controlling the proportion of budgets across layers, a smaller/larger $B_{l+1}$ will also guide the SH in Alg~\ref{alg:outer} to shrink/extend the budget $R$ for differentiating config performance.


\section{\sysname\ Design}
\label{sec:cognify}

We build \sysname, an extensible gen-AI workflow autotuning platform based on the \search\ algorithm. The input to \sysname\ is the user-written gen-AI workflow (we currently support LangChain \cite{langchain-repo}, DSPy \cite{khattab2024dspy}, and our own programming model), a user-provided workflow training set, a user-chosen evaluator, and a user-specified total search budget. \sysname\ currently supports three autotuning objectives: generation quality (defined by the user evaluator), total workflow execution cost, and total workflow execution latency. Users can choose one or more of these objectives and set thresholds for them or the remaining metrics (\eg, optimize cost and latency while ensuring quality to be at least 5\% better than the original workflow). 
\sysname\ uses the \search\ algorithm to search through the cog space.
When given multiple optimization objectives, \sysname\ maintains a sorted optimization queue for each objective and performs its pruning and final result selection from all the sorted queues (possibly with different weighted numbers).
To speed up the search process, we employ parallel execution, where a user-configurable number of optimizers, each taking a chunk of search load, work together in parallel. %Below, we introduce each type of cogs in more details.
\sysname\ returns multiple autotuned workflow versions based on user-specified objectives.
\sysname\ also allows users to continue the auto-tuning from a previous optimization result with more budgets so that users can gradually increase their search budget without prior knowledge of what budget is sufficient.
Appendix~\ref{sec:apdx-example} shows an example of \sysname-tuned workflow outputs. 
\sysname\ currently supports six cogs in three categories, as discussed below. 

%In \sysname, we call every workflow optimization technique a {\em cog}, including structure-changing cogs like task decomposition, step-changing cogs like model selection, and weight-changing cogs like adding few-shot examples to prompts. 
%\sysname\ places structure-changing cogs in the outermost layer, step cogs in the middle layer, and weight cogs in the innermost layer, because \fixme{TODO}.


\subsection{Architecture Cogs}
\label{sec:structure-cog}
%Changing the structure of a workflow can potentially improve its generation quality (\eg, by using multiple steps to attempt at a task in parallel or in chain) or reduce its execution cost and latency (\eg, by merging or removing steps).
\sysname\ currently supports two architecture cogs: task decomposition and task ensemble.
Task decomposition~\cite{khot2023decomposed} breaks a workflow step into multiple sub-steps and can potentially improve generation quality and lower execution costs, as decomposed tasks are easier to solve even with a small (cheaper) model.
There are numerous ways to perform task decomposition in a workflow. 
%, as all LM steps can potentially be decomposed and into different numbers of sub-steps in different ways. Throwing all options to the Bayesian Optimizer would drastically increase the search space for \sysname. 
To reduce the search space, we propose several ways to narrow down task decomposition options. Even though we present these techniques in the setting of task decomposition, they generalize to many other structure-changing tuning techniques.

%First, we narrow down a selected set of steps in a workflow to decompose. 
Intuitively, complex tasks are the hardest to solve and worth decomposition the most. We use a combination of LLM-as-a-judge \cite{vicuna_share_gpt} and static graph (program) analysis to identify complex steps. We instruct an LLM to give a rating of the complexity of each step in a workflow. We then analyze the relationship between steps in a workflow and find the number of out-edges of each step (\ie, the number of subsequent steps getting this step's output). More out-edges imply that a step is likely performing more tasks at the same time and is thus more complex. We multiply the LLM-produced rating and the number of out-edges for each step and pick the modules with scores above a learnable threshold as the target for task decomposition. We then instruct an LLM to propose a decomposition (\ie, generate the submodules and their prompts) for each candidate step. %We provide the LLM with few-shot examples for what proposed modules for a separate task could look like. We also add a refinement step that validates whether the proposition decomposition maintains the semantics of the original trajectory. Once candidate decompositions are generated, those are used for the entirety of the optimization.

{
\begin{figure*}[t!]
\begin{center}
\centerline{\includegraphics[width=0.95\textwidth]{Figures/big_grid.pdf}}
\vspace{-0.1in}
\mycap{Generation Quality vs Cost/Latency.}{Dashed lines show the Pareto frontier (upper left is better). Cost shown as model API dollar cost for every 1000 requests. Cognify selects models
from GPT-4o-mini and Llama-8B. DSPy and Trace do not support model selection and are given GPT-4o-mini for all steps. Trace results for Text-2-SQL and FinRobot have 0 quality and are not included.} 
\Description{Eight graphs with different shapes representing baselines compared to points on a Pareto frontier.}
\label{fig-biggrid}
\end{center}
\end{figure*}
}


The second structure-changing cog that \sysname\ supports is task ensembling. This cog spawns multiple parallel steps (or samplers) for a single step in the original workflow, as well as an aggregator step that returns the best output (or combination of outputs). By introducing parallel steps, \sysname\ can optimize these independently with step and weight cogs. This provides the aggregator with a diverse set of outputs to choose from. 
%The aggregator is prompted with the role of the samplers, as well as the inputs to each. It also receives a criteria by which it should make a decision. We choose to prompt it with a qualitative description of how it should resolve discrepancies between outputs. 


\subsection{Step Cogs}
We currently support two step-changing cogs: model selection for language-model (LM) steps and code rewriting for code steps.

For model selection, to reduce its search space, we identify ``important'' LM steps---steps that most critically impact the final workflow output to reduce the set \search\ performs TPE sampling on. Our approach is to test each step in isolation by freezing other steps with the cheapest model and trying different models on the step under testing. 
We then calculate the difference between the model yielding the best and worst workflow results as the importance of the step under testing. %For each model, we get the workflow output quality score using sampled user-supplied inputs and user-specific evaluator. We then calculate the difference between the highest and lowest scores as this module's importance. 
After testing all the steps, we choose the steps with the highest K\% importance as the ones for TPE to sample from.
%, where K is determined based on user-chosen stop criteria. We then initialize the Bayesian optimization to start with the state where important modules use the largest model and all other modules use the cheapest model. We set the TPE optimization bandwidth of each module to be the inverse of importance, \ie, the more important a module is the more TPE spends on optimizing.

The second step cog \sysname\ supports is code rewriting, where it automatically changes code steps to use better implementation. To rewrite a code step, \sysname\ finds the $k$ worst- and best-performing training data points and feeds their corresponding input and output pairs of this code step to an LLM. We let the LLM propose $n$ new candidate code pieces for the step at a time.
%in parallel to generate a set of $n$ candidates.
In subsequent trials, the optimizer dynamically updates the candidate set using feedback from the evaluator.


\subsection{Weight Cogs}
\sysname\ currently supports two weight-changing cogs: reasoning and few-shot examples.
First, \sysname\ supports adding reasoning capability to the user's original prompt, with two options: zero-shot Chain-of-Thought \cite{wei2022chain} (\ie, ``think step-by-step...'') and dynamic planning \cite{huang2022language} (\ie, ``break down the task into simpler sub-tasks...''). These prompts are appended to the user's prompt. In the case where the original module relies on structured output, we support a reason-then-format option that injects reasoning text into the prompt while maintaining the original output schema.

Second, \sysname\ supports dynamically adding few-shot examples to a prompt. At the end of each iteration, we choose the top-$k$-performing examples for an LM step in the training data and use their corresponding input-output pairs of the LM step as the few-shot examples to be appended to the original prompt to the LM step for later iterations' TPE sampling. As such, the set of few-shot examples is constantly evolving during the optimization process based on the workflow's evaluation results. 
%Few-shot examples are available to all modules, even intermediate steps in the workflow. We use the full trajectory of each request to generate examples for the intermediate steps. Furthermore, we automatically filter out examples that do not meet a user-specified threshold. 



%
\section{Numerical example}\label{sec:NE}
% In this section, the effectiveness of the proposed algorithm is described through a numerical example. 
\subsection{Plant description}
Consider a power generating system \citep[Sec.4]{park2019stealthy} %which can be modeled as as shown in Fig. \ref{turbine} and 
modeled by the dynamics:
\begin{align}
\label{power_AB} \begin{bmatrix}
\dot{\eta}_1\\ \dot{\eta}_2 \\ \dot{\eta}_3
\end{bmatrix} &= 
\begin{bmatrix}
\frac{-1}{T_{lm}} & \frac{K_{lm}}{T_{lm}} & \frac{-2K_{lm}}{T_{lm}}\\
0 & \frac{-2}{T_h} & \frac{6}{T_h}\\
\frac{-1}{T_g R} & 0 & \frac{-1}{T_g}
\end{bmatrix}
\underbrace{\begin{bmatrix}
{\eta}_1\\ {\eta}_2 \\ {\eta}_3
\end{bmatrix}}_{\eta}
+ \begin{bmatrix}
0\\ 0 \\ \frac{1}{T_g}
\end{bmatrix}
{u}\\
\label{power_C} y_p &= \underbrace{ \begin{bmatrix}
1 & 0 & 0 
\end{bmatrix}}_{C_p}\eta,\;\;
y_J = \underbrace{
\begin{bmatrix}
0 & 1 & 0
\end{bmatrix}}_{C_J}\eta.
\end{align}
Here, $\eta \triangleq [df; dp + 2 dx; dx]$, where $df$ is the frequency deviation in \mbox{Hz}, $dp$ is the change in the generator output per unit (\mbox{p.u.}), and $dx$ is the change in the valve position \mbox{p.u.}. The parameters of the plant are listed in Table \ref{param}. 
% The constants $T_{lm}, T_h$, and $T_g$ represent the time constants of load and machine, hydro turbine, and governor, respectively, and $R \,\mathrm{(Hz/p.u.)}$ is the speed regulation due to the governor action. The constant $K_{lm}$ represents the steady-state gain of the load and machine. 
The Discrete-Time system matrices $(A_p,B_p,C_p,D_p)$ are obtained by discretizing the plant \eqref{power_AB}-\eqref{power_C} using zero-order hold with a sampling time $T_s=0.1\mbox{s}$. 

\begin{table}
\centering
\begin{tabular}{||c | c || c | c|| c | c ||} 
 \hline
 $K_{lm}$ & 1 & $T_{lm}$ & 6 &  $T_g$ & 0.2 \\
 \hline
 $T_{h}$ & $4$ & $T_s$ & 0.1 & $R$ & 0.05\\
 \hline
\end{tabular}
\caption{System Parameters}
\label{param}
\end{table}


The plant is stabilized locally with a static output feedback controller with constant gain $D_c=19$. The gains in \eqref{eq:cntrl} are obtained by minimizing a quadratic cost, using the MATLAB command \emph{dlqr}, resulting in:
\begin{align}
    K&=\begin{bmatrix}
        0.1986  & -0.0913  & -0.1143
    \end{bmatrix}\\
    L &= \begin{bmatrix}
       0.2735 &  -0.0509 & -0.2035
    \end{bmatrix}^\top.
\end{align}
% The controller (detector) gain is obtained by minimizing a quadratic cost, which can be done in MATLAB using the \emph{dlqr} command. 

% %%% You can delete the following figure to save space. 
% \begin{figure}
%     \centering
%     \includegraphics[width=8.4cm]{Turbine.eps}
%     \caption{\ajg{Pictorial representation of power generating system with a hydro turbine under covert attack. The plant consists of the power generating system (governer, hydro-turbine,machine) and a stabilizing controller. The plant output is transmitted over the network to a controller for process monitoring and tracking command. The solid/dashed/red lines represent physical/cyber and covert attack components respectively.}}
%     \label{turbine}
% \end{figure}
\subsection{Initializing the mWM design algorithm}
We consider a mWM filter of state dimension $n_{\sigma}=2$. The mWM filter parameters are initialized as $A_q = 0.2I_2$, $B_q=0.7e_{2 \times 1}$, $C_q = 0.1 e_{1 \times 2}$, $B_h=0.2e_{2 \times 1}$, $C_h=0.05 e_{1 \times 2}$, $A_h=0.3I_2$, $D_q=0.15$, $D_h=0.1$ where $e_{a \times b}$ represents a unit matrix of size $a \times b$. 
The other mWM matrices are derived such that they satisfy \eqref{eq:WM:def}. 
All unspecified matrices are zero. Following Assumption~\ref{ass:param}, it is assumed that the filter parameters $\theta$ are known by the adversary. 
% Thus, the aim is to find the next-step mWM parameters $\theta^+$ minimizing the AEC-OOG. 
To ensure randomization, as mentioned in Theorem~\ref{thm:nonRepeat}, the parameters $D_h$ and $D_q$ are initialized in Algorithm~\ref{algo2} as random numbers within the range $[0.1,0.15]$. We fix the parameters of all the mWM filter parameters at their initial value except for the matrix $A_{\sigma}, \sigma \in \{q,w,h,g\}$, i.e., our aim is to find a diagonal $A_{\sigma}$ that minimizes the value of the AEC-OOG.

As discussed in Section~\ref{ch:design:algo}, $A_q$ and $A_h$ are matrices whose diagonal elements take values in $(-1,1)$.
The exhaustive search is performed with a grid size of $n_s = 0.3$, and the search is initialized with ${\epsilon_r = 1}, {\epsilon_a = 50}$. Furthermore, for numerical stability, we modify the objective function of \eqref{o1} to $\epsilon_r \gamma + \epsilon_a \gamma_a + \epsilon_p \mathrm{tr}(P)$, with ${\epsilon_p = 0.1}$.

% To this end, we set $A_q=\text{diag}(t_1,t_2)$, and $A_h=\text{diag}(t_3,t_4)$ where $-1 < t_{i} < 1, i\in \{1,2,3,4\}$. In other words $A_q$ and $A_h$ are matrices whose diagonal elements are allowed to take values between $-1$ and $+1$, to ensure stability of the filters. Then a grid search (with grid size $n_s$=0.3) on the matrices is employed as described in Algorithm \ref{algo2}. The grid search is initialized with ${\epsilon_r=1},{\epsilon_a=50}$ and ${\epsilon_p=0.1}$. Here $\epsilon_p$ represents the weight on the trace of the matrix $P$ in the objective function.  In other words, we modify the objective function in \eqref{o1} as $\epsilon_r\gamma+\epsilon_a\gamma_a+\epsilon_p\text{tr}(P)$, for the numerical stability of the algorithm. 

\subsection{Result of Algorithm \ref{algo2}}
The optimal value of the matrices from the grid search are $A_q^* = -0.05I_2$ and $A_h^*=-0.65 I_2$. 
The corresponding value of $\mathcal{L}$ is $111.03$. The value of $D_q$ and $D_h$ were $0.1479$ and $0.1482$ respectively. The simulation is performed using Matlab 2021a with \textit{Yalmip} \citep{lofberg2004yalmip} and \textit{SDPT3v4.0} solver \citep{toh2012implementation}.
%
In the remainder, we compare the results obtained by repeated computation of Algorithm~\ref{algo2} compared to defining constant and random parameters.
Consider 
an adversary injecting the signals shown in Fig.~\ref{fig:no:switch},
\begin{equation}\label{eq:step}
    \varphi_u[k] = \begin{cases}
        150 & \text{if}\;k\;\text{mod}\;2=0 \\
        0 & \text{otherwise}
    \end{cases}
   \end{equation}
into the actuators, and $\varphi_y$ following \eqref{eq:atk:cov}.
% Next, we compare our results to the scenario where the parameters are not switched,  and where they are updated randomly. 
    
\emph{Comparison with no parameter switching:}
% The adversary constructs an attack signal $\varphi_y$ through \eqref{eq:atk:cov} which is shown in Fig.~\ref{fig:no:switch}. 
The performance of the attack is shown in Fig.~\ref{fig:no:switch}, for the cases without switching and when switching happens at the attack onset with the optimal filter parameters.
Without switching $\theta$, although the performance is strongly degraded, the attack remains stealthy. Instead, if the mWM parameters are changed, it is detected after $15 \mbox{s}$.
% We can see that the performance degradation is high without the switching, and there is no attack detection since the adversary exactly knows the parameters. This also unveils the importance of detecting covert attacks. On the other hand, when the parameters of the watermarks switch the attack is successfully detected after $15\mbox{s}$. 

\begin{figure}
    \centering
    \includegraphics[width=7cm]{Resubmission_ajg.eps}
    \caption{\ajg{(Top) The attack signal $\varphi_u$ in \eqref{eq:step} and its equivalent $\varphi_y$ from \eqref{eq:atk:cov}; (Middle) $\Vert y_r \Vert_{\ell_2,[0,k]}^2$, compared to $\epsilon_r$; (Bottom) $\Vert y_J \Vert_{\ell_2,[0,k]}^2$ before and after the mWM parameters are updated.}}
    \label{fig:no:switch}
\end{figure}



\emph{Comparison with random parameter switching:}
In this scenario, we suppose the mWM parameters are updated $5$ times, by running Algorithm~\ref{algo2}, and compared against $5$ random updates of $A_\sigma$ -- though their structure remains diagonal.
% Next, Algorithm~\ref{algo2} is run, i.e., the filter parameters are updated, $5$ times. 
The results, shown in terms of values of $\mathcal L$ for both cases, are shown in Fig.~\ref{fig:switch}.
% The results are shown in Fig.~\ref{fig:switch} where we plot the values of $\mathcal{L}$.
% For comparison, instead of selecting the optimal matrices $A_\sigma$ from an exhaustive search, we choose the matrices randomly. That is, the diagonal elements of $A_h$ and $A_q$ are chosen random and the parameters are discarded if they yield an unstable inverse. 
% The corresponding value of  $\mathcal{L}$ is also depicted in Fig. \ref{fig:switch}. 
Here, the parameters of $D_h$ and $D_q$ are the same as used for selecting the optimal parameters. Since the parameters are not chosen optimally, the value of $\mathcal{L}$, the performance loss, is higher. 

\begin{figure}
    \centering
    \includegraphics[width=6.5cm]{Switch_new.eps}
    \caption{The values of $\mathcal{L}$ corresponding to the optimal and random values of the watermarking parameters.}
    \label{fig:switch}
\end{figure}

\emph{Time complexity:}
To conclude, let us discuss thecomplexity of Algorithm~\ref{algo2}. All mWM parameters are fixed, apart from $A_\sigma$, which is a diagonal matrix of dimension $n_\sigma$, and, for each diagonal element of $A_\sigma$, $n_s$ points of the interval $(-1,1)$ are searched.
Given \eqref{eq:WM:def}, only $A_h$ and $A_q$ must be defined, while $A_w, A_g$ are defined algebraically; thus, define $n_\varsigma = n_q + n_h$.
The complexity of the algorithm grows both in $n_\varsigma$ and in $n_s$. Specifically: for $n_s = n_s^*$, the complexity is $\mathcal O(n_s^{*x})$; for $n_\varsigma = n_\varsigma^*$, the complexity is $\mathcal O(x^{n_\varsigma^*})$. Thus, the complexity is exponential in the choice of $n_\varsigma$ and polynomial in $n_s$.
%
We highlight that the average time of solution can be improved upon in two major ways. 
The first is via parallelization, as all SDPs can be solved independently; this provides a speed-up which depends on the number of compute nodes used to solve the problem.
The second method relies on reducing the number of SDPs to be solved, by removing those values of $A_h, A_q$ which do not lead to stable inverses, as defined by \eqref{eq:sys:inv}.

For the results presented here, a computer with an Intel Core i7-6500U CPU with 2 cores and 8GB RAM was used. The algorithm was run both with and without parallelization (parallelization was achieved by using Matlab's \texttt{parfor} command). Without parallelization, the algorithm took $384.25 \mathrm{s}$ to provide a result, whilst with parallelization this was $261.65 \mathrm{s}$, a  $31.9 \%$ speedup.

% \subsection{Time complexity}
% The computation complexity of a design algorithm is important, and in this subsection we briefly comment on it. In particular, we comment on the expected time taken to perform the grid search in Algorithm \ref{algo2}. As adopted in this example, let us consider the design algorithm where all the mWM parameters are fixed except for $A_{\sigma}$, which is a diagonal matrix. Let $n_s$ denote the size of the grid search, $n_{\sigma}$ denote the size of the watermarking filter, $\mathbb{E}[\tau]$ denote the mean time taken by a solver to compute the optimal value of \eqref{o1} for any given value of the filter parameters. The time $\mathbb{E}[\tau]$ depends on the SDP solver and the hardware. Then, the mean time taken to perform the grid search, denoted by $\mathbb{E}[T]$ is bounded by 
% \begin{equation}\label{eq:bound}
%     \mathbb{E}[T] \leq \mathbb{E}[\tau] \left(\ceil[\Bigg]{\frac{2}{n_s}} \right)^{n_{\sigma}}
% \end{equation}
% where $\ceil{x}$ denotes the smallest integer greater than $x$. Thus, the time complexity increases quadratically in the grid size, and exponentially in the size of the filter state. The bound \eqref{eq:bound} also holds when the matrix $A_{g}$ and $A_h$ are traingular where the diagonal elements are the decision variables and the other elements are fixed. 

% In general, the bound in \eqref{eq:bound} is loose, as some filter parameters might yield an unstable inverse, and the grid search does not need to solve an SDP in that case (see step 5 in Algorithm \ref{algo2}). Thus, the bound \eqref{eq:bound} can be largely improved by pre-processing, where the set of all matrices which yield an unstable inverse is removed. Additionally, since the SDPs can be solved independently, the grid search is parallelizable, which reduces the time complexity even further. 

\section{Conclusion and future works }\label{sec:Con}
An optimal design technique for the design of the parameters of switching multiplicative watermarking filters is presented. 
The problem is formalized by supposing the closed-loop system is subject to a covert attack with matching parameters. 
We propose an optimal control problem based on a formulation of the attack energy constrained output-to-output gain. 
We show through a numerical example that this design improves detectability by increasing the energy of the residual output before and after a switching event. 
Future works includes developing algorithms for optimal design and optimal switching times ensuring that mWM does not destabilize the closed-loop system under switching with mismatched parameters, and studying non-linear systems. 
%\bibliographystyle{plain}
\bibliography{autosam_abbrv}
%\printbibliography
% \appendix

\end{document}