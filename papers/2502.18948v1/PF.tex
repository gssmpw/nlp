

We consider the Cyber-Physical System (CPS) in Fig.~\ref{fig:sys}. This includes plant $\PP$, controller and anomaly detector $\CC$, mWM filters $\WW,\QQ,\GG,\HH$, and the malicious agent $\mathcal{A}$. The mWM filters are defined pairwise, namely $\{\QQ,\WW\}$ and $\{\GG,\HH\}$ are referred to as, respectively, the output and input \textit{mWM filter pairs}. 
% In the following, we provide the description of the CPS, and define the problem.

\begin{figure}
    \centering
    \includegraphics[width=6.5cm]{NCS.eps}
    \caption{Block diagram of the closed-loop CPS including the plant $\PP$, controller $\CC$ and watermarking filters $\{\WW,\QQ,\GG,\HH\}$. The information transmitted between $\PP$ and $\CC$ is altered by the adversary $\mathcal A$. The dashed lines represent the network affected by the adversary.}
    \label{fig:sys}
\end{figure}

\subsection{Plant and controller}
Consider an LTI discrete-time (DT) plant modeled by: 
% whose dynamics are described as:
	\begin{equation}\label{eq:sys}
	    \PP: \left\{ \begin{aligned}
	        x_p[k+1] &= A_p x_p[k] + B_p u_h[k]\\
	        y_p[k] &= C_p x_p[k]\\
            y_J[k] &= C_J x_p[k] + D_J u_h[k]
	    \end{aligned}\right.
	\end{equation}
	where $x_p \in \mathbb R^n$ is the plant's state, $u_h \in \mathbb R^m$ its input, $y_p \in \mathbb R^p$ its measured output, and all the system's matrices are of the appropriate dimension. Furthermore, suppose a (possibly unmeasured) \textit{performance output} $y_J \in \mathbb{R}^{p_J}$ is defined, such that the performance of the system, evaluated over the interval $[k-N+1,k]$, for some $N \in \mathbb{N}$ \citep{zhou1996robust}, is given by:
	\begin{align}
	    J(x_p,u_h) &= %\| C_J x_p + D_J u_h \|^2_{\ell_2,[k-N+1,k]}= 
        \|y_J\|^2_{\ell_2,[k-N+1,k]}.
	\end{align}
	% Note that $y_J$ can be viewed as a \textit{virtual} output, i.e., it may not be measured directly. Next, we establish the following assumption.
    \begin{assumption}
    The tuples $(A_p,B_p)$ and $(C_p,A_p)$ are respectively, controllable and observable pairs.
		$\hfill\triangleleft$
	\end{assumption}
 \begin{assumption}\label{ass:stable}
     The plant $\mathcal{P}$ is stable and $x_p[0]=0$. $\hfill\triangleleft$
 \end{assumption}

Assumption~\ref{ass:stable}, necessary for the OOG to be meaningful \citep{teixeira2015secure}, does not reduce generality, as stability can be ensured by a local (non-networked) controller \citep{hu2007stability,lin2023secondary}, whilst $x_p[0] = 0$ can be considered because of linearity.

% \begin{remark}
%     Assumption~\ref{ass:stable} does not reduce generality, and is required for the output-to-output gain to be meaningful \citep{teixeira2015strategic}. The first statement can be achieved by introducing a two-level controller: a local feedback controller for stabilization, while a networked controller improves performance, changes setpoints, etc. \citep{hu2007stability,lin2023secondary}. The second statement follows from superposition.
%     $\hfill\triangleleft$
% \end{remark}

The plant is regulated by an observer-based dynamic controller $\CC$, described by:
\begin{equation}\label{eq:cntrl}
    \mathcal C: \left\{
	\begin{aligned}
		\hat{x}_p[k+1] &= A_p \hat{x}_p[k] + B_p u_c[k] + Ly_r[k]\\
		u_c[k] &= K\hat{x}_p[k]\\
		\hat{y}_p[k] &= C_p \hat{x}_p[k]\\
		y_r[k] &= y_q[k] - \hat{y}_p[k]
	\end{aligned}\right.
\end{equation}
where $\hat{x}_p \in \mathbb{R}^n, \hat{y}_p \in \mathbb{R}^p$ are the state and measurement estimates, $u_c \in \mathbb{R}^m$ the control input. The matrices $K$ and $L$ are the controller and observer gains respectively. Finally, the term $y_r$ in \eqref{eq:cntrl} is the residual output, used to detect the presence of an attack: given a threshold $\epsilon_r$, an attack is detected if the inequality $\|y_r\|_{\ell_2,[0,N_r]}^2 \leq \epsilon_r$ is falsified for any $N_r \in \mathbb{N}_+$. Note that in \eqref{eq:sys}-\eqref{eq:cntrl} $y_q$ and $u_h$, the outputs of $\QQ$ and $\HH$ (to be defined), are used as the input to the controller and the plant, respectively. 
% These are the output signals of watermark removers $\QQ$ and $\HH$, to be defined. 

\subsection{Multiplicative watermarking filters}
Consider mWM filters defined as follows
\begin{equation}\label{eq:sys:WM}
		\Sigma: \begin{cases}
			x_\sigma[k+1] = A_\sigma(\theta_\sigma[k]) x_\sigma [k] + B_\sigma(\theta_\sigma[k]) \nu_\sigma [k]\\
			\gamma_\sigma [k] = C_\sigma(\theta_\sigma[k]) x_\sigma [k] + D_\sigma(\theta_\sigma[k]) \nu_\sigma [k]
		\end{cases},
	\end{equation}
	with $\Sigma \in \{\GG,\HH,\WW,\QQ\}$, $\sigma \in \{g,h,w,q\}$, \ajg{where $g,h,w,q$ refer to variables pertaining to $\mathcal G, \mathcal H, \mathcal W, \mathcal Q$, respectively}\footnote{In the sequel whenever referring to the parameters of any one of the mWM filters, the subscript $\sigma$ is used. Conversely, if referring to all parameters, $\theta$ is used.}, $x_\sigma \in \mathbb{R}^{n_\sigma}$ the state of $\Sigma$, $\nu_\sigma \in \mathbb{R}^{m_\sigma}$ its input, $\gamma_\sigma \in \mathbb{R}^{p_\sigma}$ the output, and $\theta_{\sigma}[k]$ is a vector of parameters.

\begin{definition}[mWM filter parameters]
    The parameter $\theta_\sigma[k]$ is taken to be the concatenation of the vectorized form of all matrices $A_\sigma(\cdot), B_\sigma(\cdot), C_\sigma(\cdot), D_\sigma(\cdot)$.
    $\hfill\triangleleft$
\end{definition}

The parameter $\theta_{\sigma}$ is defined to be piecewise constant:
$$\theta_\sigma[k] = \bar{\theta}_\sigma[k_i], \forall k \in \{k_i, k_i+1, \dots, k_{i+1}-1\}$$
where $k_i, i = 0,1, \dots \in \mathbb{N}_+,$ are switching instants. In the following, with some abuse of notation, the time dependencies are dropped, with $\theta_\sigma$ and $\theta_\sigma^+$ used to define the parameters before and after a switching instant, 
% with $\theta_\sigma$ written instead of $\theta_\sigma[k_i]$, and $\theta_\sigma^+$ to define the filter parameters after a switching instant, 
i.e., $\theta_\sigma = \theta_\sigma[k_i]$, $\theta_\sigma^+ = \theta_\sigma[k_{i+1}]$.
% \input{mWM_gen_remov.tex}

% It is important to note that, because of this characterization and because of the non-uniqueness of state-space representation of transfer functions, the parametrization of the transfer functions of $\Sigma$ is not unique. Two parameters $\theta_{\sigma,1}$ and $\theta_{\sigma,2}$ are said to be \textit{equivalent} if they lead to the same transfer function. In the remainder of the paper, it is supposed that the selection of a specific parameter amongst all those equivalent can be defined by imposing some \textit{structure} onto $\theta_\sigma$. In the numerical results presented in Section~\ref{sec:NE}, for example, all matrices are defined to be diagonal.

% Defining the mWM parameters as time-varying has been shown to be critical against certain classes of attacks~\citep{ferrari2020switching}.
Furthermore, all filters are taken to be square systems, i.e., $m_\sigma = p_\sigma, \forall \sigma \in \{g,h,w,q\}$, and define $\nu_g \triangleq u_c, \nu_h \triangleq \tilde{u}_g, \nu_w \triangleq y_p, \nu_q \triangleq \tilde{y}_w, \gamma_g \triangleq u_g, \gamma_h \triangleq u_h, \gamma_w \triangleq y_w, \gamma_q \triangleq y_q.$
 %    \begin{equation}\label{eq:WM:input}
 %        \begin{array}{cccc}
 %            \nu_g \triangleq u_c, & \nu_h \triangleq \tilde{u}_g, & \nu_w \triangleq y_p, & \nu_q \triangleq \tilde{y}_w, \\
 %            \gamma_g \triangleq u_g, & \gamma_h \triangleq u_h, & \gamma_w \triangleq y_w, & \gamma_q \triangleq y_q. 
 %        \end{array}
	% \end{equation}
    Here, a \textit{tilde} is used to highlight that $\tilde u_g, \tilde y_w$ are received through the insecure communication network and as such may be affected by attacks. 
%
% This choice is made as a consequence of the fact that switching the parameters of the watermarking scheme is critical to enable the detection of attacks that would otherwise be stealthy, such as covert attacks with matched parameters \cite{ferrari2020switching}. 
%
%%%%%%% DONOT REMOVE THE FOLLOWING REMARK
\begin{remark}
    The objective of this paper is to \textit{optimally} design the successive parameters of the mWM filters $\theta_\sigma^+$, given their value $\theta_\sigma$. 
    It remains out of the scope of the paper to address other aspects of the switching mechanisms, such as determining the switching time, or defining the jump functions for the states. 
    Interested readers are referred to \citep{ferrari2020switching}.
    %for an example on its definition, and an analysis of the impact on stability.
    $\hfill \triangleleft$
\end{remark}

% The following defines what properties must be satisfied at any instant $k \in \mathbb{N}_+$ by two filters to be a valid mWM pair.
\begin{definition}[Watermarking pair]\label{def:WM}
Two systems $(\mathcal W,\mathcal Q)$ \eqref{eq:sys:WM}, are a \textit{watermarking pair} if:
\begin{enumerate}[label=\alph*.]
    \item \label{def:WM:inv} $\mathcal{W}$ and $\mathcal{Q}$ are stable and invertible, i.e., exists a positive definite matrix $Z_\sigma \succ 0, \sigma \in \{w,q\}$ such that %the following Lyapunov stability conditions hold 
    \begin{equation}\label{eq:WM:stab}
        A_\sigma^\top Z_\sigma A_\sigma - Z_\sigma \prec 0;
    \end{equation}
    \item \label{def:WM:eqParam} if $\theta_w[k] = \theta_q[k]$, $y_q[k] = y_p[k]$, i.e., 
    \begin{equation}\label{eq:WM:def}
	   \QQ \triangleq \WW^{-1}\,. \qquad \triangleleft
    \end{equation}
\end{enumerate}
\end{definition}
\begin{remark}\label{rem:stabSw}
    If $Z_\sigma$ in \eqref{eq:WM:stab} is the same for all $\theta_\sigma[k], k \in \mathbb N,\sigma\in\{g,h,w,q\}$, the mWM filters, on their own, are stable under arbitrary switching, as they all share a common Lyapunov function.
    $\hfill \triangleleft$
\end{remark}

\begin{definition}[{\cite[Lemma 3.15]{zhou1996robust}}]\label{def:inv:ss}
		Define the DT transfer function resulting from the system defined by the tuple $(A,B,C,D)$ as $			G(z) = \left[\begin{array}{c|c}
				A &B\\
				\hline
				C &D
			\end{array}
			\right],$
		and suppose that $D^{-1}$ exists. Then
		\begin{equation}\label{eq:sys:inv}
			G^{-1}(z) = \left[\begin{array}{c|c}
				A - BD^{-1}C    &BD^{-1}\\
				\hline
				-D^{-1}C        &D^{-1}
			\end{array}
			\right]
		\end{equation}
		is the inverse transfer function of $G(z)$.
		$\hfill\triangleleft$
	\end{definition}
% Definition~\ref{def:inv:ss} holds true for any equivalent state space transformation of the matrices defined in \eqref{eq:sys:inv}. Next we establish the following assumption. 
\begin{assumption}\label{ass:sync}
    The mWM parameters are matched, i.e., $\theta_w[k] = \theta_q[k]$ and $\theta_g[k] = \theta_h[k]\; \forall\;k \in \mathbb{N}$.
    $\hfill \triangleleft$
\end{assumption}
% Note that Assumption~\ref{ass:sync} is equivalent to saying that all mWM parameters' switching times $k_i, i \in \mathbb N$ are synchronized.

% From Definition~\ref{def:inv:ss}, under Assumption~\ref{ass:sync}, when no attack is present on the communication network between $\PP$ and $\CC$ (i.e., $\tilde{u}_g=u_g$ and $\tilde{y}_w = y_w$), so long as $x_w[0] = x_q[0]$ and $x_g[0] = x_h[0]$, $y_q[k] = y_p[k]$ and $u_h[k] = u_c[k]$ hold for all $k \in \mathbb{N}$, i.e., the mWM filters do not influence the closed-loop dynamics \citep{ferrari2020switching}.
% the following hold:
% \begin{align}\label{eq:sys:WM:inv}
% 	&\begin{array}{ccc}
% 		\mathcal{Q}(z)\mathcal{W}(z) = I_{p}\,,&  \mathcal{H}(z)\mathcal{G}(z) = I_{m}\,, 
%         &\forall z \in \mathbb C
% 	\end{array}\\
% 	&\begin{array}{ccc}
% 		y_q[k] = y_p[k]  \,, & u_h[k] = u_c[k]\,, &\forall k \in \mathbb{N}_+,
% 	\end{array}\label{eq:sys:WM:inv:2}
% \end{align}
% so long as $x_w[0] = x_q[0], x_g[0] = x_h[0]$. 
% In other words, when no attack is present, and if the mWM parameters remain synchronized, the mWM filters do not effect the closed-loop dynamics \citep{ferrari2020switching}. 

% \begin{figure*}[t]
%     \centering
%     {\footnotesize
%     \begin{align}
%         \left[
%         \begin{array}{c|c}
%         {A} & {B}\\
%         \hline
%         \\[\dimexpr-\normalbaselineskip+1pt] \bar{C}_r & 0\\
%         \hline
%         \\[\dimexpr-\normalbaselineskip+1pt] \bar{C}_J & \bar{D}_J
%         \end{array}
%         \right] &= 
%         \left[ 
%         \begin{array}{c|c}
%         \begin{matrix}
%             A_p & B_pC_h & B_pD_hC_g & B_pD_hD_gK & 0 & 0 & 0\\
%             0 & A_h & B_hC_g & B_hD_gK & 0 & 0 & 0\\
%             0 & 0 & A_g & B_gK & 0 & 0 & 0\\
%             LD_qD_wC_p & 0 &0 & A_p+B_pK-LC_p & LC_q & LD_qC_w & -LD_qC_a\\
%             B_qD_wC_p & 0 & 0 & 0 & A_q & B_qC_w & -B_qC_a\\
%             B_wC_p & 0 & 0 & 0 & 0 & A_w & 0\\
%             0 & 0 & 0 & 0 & 0 & 0 & A_a\\
%             \hline
%             D_qD_wC_p & 0 & 0 & -C_p & C_q & D_qC_w & -D_qC_a\\
%             \hline
%             C_J & D_JC_h & D_JD_hC_g & D_JD_hD_gK & 0 & 0 & 0
%         \end{matrix}     & \begin{matrix}
%         B_pD_h \\ B_h \\ 0 \\ 0 \\ 0 \\ 0 \\ B_a \\ \hline 0 \\ \hline D_JD_h
%         \end{matrix}
%         \end{array}
%         \right]\label{matrix_strip}
%     \end{align}
%     \hrule}
% \end{figure*}

\subsection{Attack model}\label{ch:probFor:atk}
Consider the malicious agent $\mathcal A$ located in the CPS as in Fig.~\ref{fig:sys}, capable of tampering with data transmitted between $\PP$ and $\CC$. 
% \ajg{The main motivation for introducing the mWM filters on the communication channels between the plant and the controller is the possibility of a malicious agent $\mathcal A$, located in the CPS such as the one in Fig.~\ref{fig:sys}. 
% In this context, $\mathcal A$ is thought to be capable of tampering with the transmitted data between $\PP$ and $\CC$.}
Without loss of generality, the injected attacks are modeled as additive signals:
 \begin{equation}\label{eq:atk}
     \tilde{u}_g[k] \triangleq u_g[k] + \varphi_u[k],\;\;\; \tilde{y}_w[k] \triangleq y_w[k] + \varphi_y[k],
 \end{equation}
where $\varphi_u[k]$ and $\varphi_y[k]$ are actuator and sensor attack signals designed by the adversary $\mathcal A$. To properly define our design algorithm in Section~\ref{sec:main}, an explicit strategy for the attack signals $\varphi_u$ and $\varphi_y$ must be defined by the defender. %, and are therefore interpreted as a design choice.
In this paper, we focus on covert attacks \citep{smith2015covert}, which remain undetected for passive diagnosis scheme.

% In particular, we consider an adversary that adopts a worst-case attack policy: covert attack with matched parameters. %In other words, we consider an adversary which has knowledge of the plant (common assumption in security literature), and the mWM parameters (which can be obtained through system identification techniques, for instance). Such an attack policy is adopted to mitigate the worst-case attack scenarios.
% Specifically, let $\theta_\sigma^a[k]$ be the parameters of the mWM filter known by the attacker\footnote{\ajg{Here, and throughout the paper, a super- or subscript $a$ is used to indicate that a variable pertains to $\mathcal A$.}}. Then, the following assumption is required.

The covert attack strategy, under Assumption~\ref{ass:param} and~\ref{ass:atkEng}, is as follows: the malicious agent $\mathcal A$ chooses $\varphi_u[k] \in \ell_{2e}$ freely, while $\varphi_y[k]$ satisfies:
\begin{equation}\label{eq:atk:cov}
	\mathcal A:
\left\{
\begin{aligned}
    x_a[k+1] &= A_a(\theta^a) x_a[k] + B_a(\theta^a) \varphi_u[k]\\
		y_a[k] &= C_a(\theta^a) x_a[k] + D_a(\theta^a) \varphi_u[k]\\
		\varphi_y[k] &= - y_a[k]
\end{aligned}
\right.
\end{equation}
where $x_a \triangleq [x_{h,a}^\top\;x_{p,a}^\top\;x_{w,a}^\top]^\top$ is the attacker's state, and its dynamics are the same as the cascade of $\HH, \PP, \WW$, parametrized\footnote{\ajg{Here, and throughout the paper, a super- or subscript $a$ is used to indicate that a variable pertains to $\mathcal A$.}} by $\theta^a_\sigma$.

\begin{assumption}\label{ass:param}
For all $k \in [k_{i+1},k_{i+2}], i \in \mathbb N_+$, the attacker parameters $\theta_\sigma^a[k]=\theta_\sigma[k_i]$, $\sigma \in \{h,w\}. \hfill\triangleleft$
\end{assumption}

\begin{assumption}\label{ass:atkEng}
    The attack energy is bounded and finite, i.e.,: $\Vert \varphi_u\Vert_{\ell_2}^2 \leq \epsilon_a$, with $\epsilon_a$ known to $\mathcal C$. $\hfill \triangleleft$
\end{assumption}

\begin{remark}
    Assumption~\ref{ass:atkEng} is introduced as it allows for guarantees that the algorithm proposed in Section~\ref{sec:main} always returns a feasible solution (see Theorem~\ref{thm:well:posed}).
    In general, %Assumption~\ref{ass:atkEng} is limiting: indeed, 
    while it may be that the adversary has limited energy \citep{djouadi2015finite}, it is a strong assumption that the bound $\epsilon_a$ is known to the defender.
    %
    Nonetheless, the attack energy bound $\epsilon_a$ may be seen as a design variable that, together with the chosen attack model \eqref{eq:atk:cov}, facilitates the definition of a systematic design of mWM filters by the defender.
    %
    % However, given the perspective that the attack policy \eqref{eq:atk:cov} is a design choice geared toward the definition of a systematic design of mWM filters, rather than an actual attack against the system, $\epsilon_a$ itself is a design variable, and thus its value is available to the defender.
    %
    Further remarks regarding the consequences of Assumption~\ref{ass:atkEng} not holding are postponed to Remark~\ref{rem:atkEng2}, following the formal definition of the attack-energy-constrained output-to-output gain in Definition~\ref{def:o2o}. 
    $\hfill \triangleleft$
\end{remark}

% In general, the adversary has limited energy \citep{djouadi2015finite}. Thus, the adversary injects an attack for a finite amount of time (say $T_a$). Although $T_a$ is unknown, we enforce that the adversary stops the attack eventually by a adding that the attack energy should be bounded. 

% \begin{remark}
%     In \eqref{p1}, $\epsilon_r$ and $\epsilon_a$ play in a critical role.
%     Firstly, the metric is always bounded, as is demonstrated in Theorem~\ref{thm:well:posed} and thus is well-posed for a design algorithm. 
%     %
%     Furthermore, it has explicit relations to both the $H_\infty$ metric (for increasing values of $\epsilon_r$) and to the original OOG (for increasing values of $\epsilon_a$ \citep[Prop.1]{anand2023risk}.
%     %
%     Finally, let us comment on the constraint on the attack energy, and argue that the choice of $\epsilon_a$ as a design parameter, to be chosen by the defender, rather than an assumpiton on the attack strategy is warranted. To do this, we consider \eqref{p1} (which is equivalent to \eqref{o1}, under Lemma~\ref{lem:sig_2_mat}) under increasing values of $\epsilon_a$, as well as the OOG as defined in \cite{teixeira2015strategic}.
%     If there are no zero dynamics in the system, the OOG is finite, and thus there is some value $\bar\epsilon_a$ such that, for all $\epsilon_a > \bar\epsilon_a$, the value of $\|y_r\|_{\ell_2}^2 = \epsilon_r$, and $\|y_J\|_{\ell_2}^2$ remains constant. Thus, defining $\epsilon_a \geq \bar \epsilon_a$, the optimal value $\theta^{+*}$ is not influenced by the constraint on the attack energy.
%     On the other hand, if there are zero dynamics that the attacker can exploit, the OOG is infinite, and $\|y_J\|_{\ell_2}^2$ grows unbounded as $\epsilon_a \rightarrow \infty$.
%     Thus, while $\theta^{+*}$, the solution to \eqref{o1}, is only optimal for $\|\varphi_u\| \leq \epsilon_a$, defining $\theta^*$ ensures the effect of $\varphi_u$ on $y_J$ is in some sense minimal if this is not the case.
%     % if the constraint is violated, defining $\theta^*$ ensures that the effect of an attack with unbounded energy on the performance output is in some sense minimal.
%     $\hfill\triangleleft$
% \end{remark}

% The AEC-OOG summarizes the adversary's objective of obtaining the maximum impact on the performance output, while remaining undetected. Compared to the output-to-output gain (OOG) proposed in \cite{teixeira2015strategic}, the attack signal energies are considered to always be bounded by $\epsilon_a$ -- a property exploited in Theorem~\ref{thm:well:posed}. This bound is treated as a \textit{design variable} in the hands of the defender, not as a constraint on the class of malicious signals that can be injected on the system. Further comments on this are given in Remark~\ref{rem:atkE}, in the following.

% \begin{align} 
% &\left[
% \begin{array}{c|c}
% {A}_a(\theta^a) & {B}_a(\theta^a)\\
% \hline
% \\[\dimexpr-\normalbaselineskip+2pt] {C}_a(\theta^a) & {D}_a(\theta^a)
% \end{array}
% \right] \triangleq\\
% &\left[ 
% \begin{array}{c|c}
% \begin{matrix}
% A_{h}(\theta_h^a) & 0 & 0 \\
% B_pC_h(\theta_h^a) & A_p & 0\\
% 0 & B_w(\theta_w^a)C_p & A_w(\theta_w^a)\\
% \hline 
% 0 & \quad D_w(\theta_w^a)C_p & C_w(\theta_w^a)
% \end{matrix}     & \begin{matrix}
% B_h(\theta_h^a) \\ B_pD_h(\theta_h^a) \\ 0 \\ \hline 0
% \end{matrix}
% \end{array}
% \right].
% \end{align} 
%In general, covert attacks with matched parameters remain undetectable to any diagnosis scheme by construction, while still having an impact on the performance of the system, because of the design of $\varphi_u$. \ajg{Thus, we consider the the worst-case covert attack policy, and provide a design algorithm to mitigate such attacks.}

\subsection{Problem formulation}\label{sec:PF:probFor}
The objective of this paper is to propose a design strategy capable of optimally designing the mWM filter parameters $\theta^+$, supposing a covert attack is present within the CPS. To formulate a metric to be used to define optimality, the closed-loop CPS dynamics must be defined. Under the attack strategy \eqref{eq:atk:cov}, the closed-loop system with the attack $\varphi_u$ as input and the performance and detection output as system outputs can be rewritten as
\begin{equation}\label{eq:S_cl}
		{\mathcal{S}}:\left\{
\begin{aligned}
{x}[k+1] &= A x[k] + B\varphi_u[k]\\
y_J[k] &= \bar{C}_J x[k] + \bar{D}_J \varphi_u[k]\\
y_r[k] &= \bar{C}_r x[k]
\end{aligned} \right.
	\end{equation}
where $x = \begin{bmatrix}x_p^\top, &x_h^\top, &x_g^\top, &x_c^\top, &x_q^\top, &x_w^\top, &x_a^\top\end{bmatrix}^\top$ is the closed-loop system state, while $y_r$ and $y_J$ remain the residual and performance outputs. All signals in \eqref{eq:S_cl} are also a function of the parameters $\theta^+$, but this dependence is dropped for clarity.
The definition of the matrices in \eqref{eq:S_cl} follow from \eqref{eq:sys}-\eqref{eq:sys:WM} and \eqref{eq:atk:cov}.
% All matrices in \eqref{eq:S_cl} are provided in \eqref{matrix_strip}.

% For the closed loop CPS dynamics in \eqref{eq:S_cl}, 
The defender aims to quantify (and later minimize) the maximum performance loss caused by a stealthy and bounded-energy adversary on \eqref{eq:S_cl}. 
This is done by exploiting the attack-energy-constrained output-to-output gain 
(AEC-OOG) \citep{anand2023risk}. %\footnote{The AEC-OOG is a restriction of the previously defined output-to-output gain (OOG) \citep{teixeira2015strategic}, by imposing that all attack signal energies are bounded.}
\begin{definition}[AEC-OOG]\label{def:o2o}
	The AEC-OOG
    % attack\hyp{}energy\hyp{}constrained output-to-output gain 
    of $\mathcal S$ in \eqref{eq:S_cl} is the value of the following optimization problem:
		\begin{equation}\label{eq:o2o}
			\begin{aligned}
				\sup_{\varphi_u\in\ell_{2e}} &\quad \|y_J\|_{\ell_2}^2 \\
				\text{s.t.}& \quad  \|y_r\|_{\ell_2}^2 \leq \epsilon_r,\; \|\varphi_u\|_{\ell_2}^2 \leq \epsilon_a,\;x[0] = 0.
			\end{aligned}
		\end{equation}
	where $\epsilon_a$ is the energy bound of the attack signal, $\epsilon_r$ is the detection threshold, and the value of \eqref{eq:o2o} denotes the performance loss caused by a stealthy adversary.	$\hfill\triangleleft$
	\end{definition}
% The AEC-OOG summarizes the adversary's objective of obtaining the maximum impact on the performance output, while remaining undetected. Compared to the output-to-output gain (OOG) proposed in \cite{teixeira2015strategic}, the attack signal energies are considered to always be bounded by $\epsilon_a$ -- a property exploited in Theorem~\ref{thm:well:posed}. This bound is treated as a \textit{design variable} in the hands of the defender, not as a constraint on the class of malicious signals that can be injected on the system. Further comments on this are given in Remark~\ref{rem:atkE}, in the following. 
%, to ensure our design algorithm is always feasible. %Further discussion regarding the AEC-OOG can be found in \cite{anand2023risk}.
% The AEC-OOG summarizes the following case: the goal of the adversary is to maximize the performance signal energy (as opposed to classical $\mathcal{H}_{\infty}$ control) while remaining undetected. The latter objective translates to constraining the detection output to remain under the predefined threshold value $\epsilon_r$. 
% \ajg{In this definition, we take the attack input to be bounded energy, where $\epsilon_a$ acts as this bound. The inclusion of this constraint in \eqref{eq:o2o} ensures that the AEC-OOG is always bounded, a property that is exploited in Theorem~\ref{thm:well:posed} to ensure that $\theta_\sigma^+$ can always be defined. 
% }
% Using the definition above, the main problem studied in this paper can be formalized.

\begin{problem}\label{problem_main}
Given $\theta_{\sigma}$ at some switching time $k_i, i \in \mathbb{N}_+$, find the optimal set of mWM filter parameters after a switching event $\theta_{\sigma}^+$, such that the AEC-OOG of the system $\mathcal{S}$ in \eqref{eq:S_cl} is minimized. $\hfill \triangleleft$
\end{problem}

\begin{remark}
    Because of its dependence on the AEC-OOG, the solution of Problem~\ref{problem_main} at time $k_i$ relies explicitly on the attack parameters $\theta^a[k_i]$. Given the malicious agent's strategy outlined in Section~\ref{ch:probFor:atk}, and Assumption~\ref{ass:param}, $\theta_\sigma^a[k_i] = \theta_\sigma^+[k_{i-1}]$ is known to $\CC$, without any additional knowledge required. %Thus, no additional knowledge is required by the defense mechanism in $\CC$ to define the closed-loop matrices in \eqref{eq:S_cl}.
    $\hfill \triangleleft$
\end{remark}

\begin{remark}\label{rem:atkEng2}
    We are now ready to formally treat the violation of Assumption~\ref{ass:atkEng}.
    To do this, let us first remark on some properties of the AEC-OOG, which follow from using finite bounds $\epsilon_r$ and $\epsilon_a$.
    Firstly, as demonstrated in Theorem~\ref{thm:well:posed}, the metric is always bounded, making it well suited for a design algorithm.
    Furthermore, it is explicitly related to both the $H_\infty$ metric and the original OOG proposed in \cite{teixeira2015strategic}, for increasing values of $\epsilon_r$ and $\epsilon_a$, respectively \citep[Prop.1]{anand2023risk}.
    Finally, we can comment on the constraint on the attack energy.
    Consider the value of \eqref{eq:o2o} under increasing values of $\epsilon_a$, as well the OOG as defined in \cite{teixeira2015strategic}.
    If the OOG is finite, there is some value $\bar\epsilon_a$ such that the AEC-OOG is the same as the OOG for all $\epsilon_a \geq \bar\epsilon_a$.
    If there are exploitable zero dynamcis, and the OOG is unbounded, $\|y_J\|_{\ell_2}^2$ grows unbounded as $\epsilon_a \rightarrow \infty$. Thus, while $\theta_\sigma^+$, the solution to Problem~\ref{problem_main}, is only optimal for covert attacks satisfying $\|\varphi_u\|_{\ell_2}^2 \leq \epsilon_a$, it ensures that the effect of $\varphi_u$ on $y_J$ is in some sense minimal if the attack energy constraint is violated.
    $\hfill\triangleleft$
\end{remark}

% The AEC-OOG summarizes the adversary's objective of obtaining the maximum impact on the performance output, while remaining undetected. Compared to the output-to-output gain (OOG) proposed in \cite{teixeira2015strategic}, the attack signal energies are considered to always be bounded by $\epsilon_a$ -- a property exploited in Theorem~\ref{thm:well:posed}. This bound is treated as a \textit{design variable} in the hands of the defender, not as a constraint on the class of malicious signals that can be injected on the system. Further comments on this are given in Remark~\ref{rem:atkE}, in the following.


% \begin{remark}
% \ajg{The AEC-OOG defined in \eqref{eq:o2o} can be related to other metrics in the literature namely the $H_{\infty}$ metric and the OOG. Let $\gamma_{\infty}$ and  $\gamma_{\text{OOG}}$ denote the $H_{\infty}$ gain, $\gamma_{OOG}$ of the closed loop system \eqref{eq:S_cl}, and $\gamma(\epsilon_a,\epsilon_r)$ denote the value of AEC-OOG for any given value of $\epsilon_a$ and $\epsilon_r$. Then it holds that $\lim_{\epsilon_a \to \infty} \gamma(\epsilon_a,\epsilon_r) = \gamma_{\text{OOG}}$ and $\lim_{\epsilon_r \to \infty} \gamma(\epsilon_a,\epsilon_r) = \gamma_{\infty}$} $\hfill \triangleleft$
% \end{remark}
