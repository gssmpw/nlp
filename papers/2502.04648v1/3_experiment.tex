\section{Experimental Work\label{sec:experiments}}

\subsection{Experimental Setup}

To validate our proposed approach, we implemented a proof-of-concept tool using Python 3.11. 
The tool processes RTL source files, such as those from opencores.org and GitHub. 
In this version, we target Verilog and SystemVerilog files. 
The dataset we use for evaluation is described in~\autoref{tab:IP-information}.
Overall, the asset search space (i.e., the number of signals an engineer would need to consider in manual analysis) comprises 9875 signals. 
Our tool requires the RTL source file directory and Top Module directory as inputs and operates through five stages as described in~\autoref{sec:proposed}. 
The tool's effectiveness is demonstrated by its application to various IP families, producing a detailed list of potential primary assets for each IP in each family. 
We evaluated the accuracy of asset detection by comparing the tool's output against our manually identified asset list. Our list consists of assets defined by the designers in documentation and comments and manually selected assets by a panel of people with experience in Verilog design and security analysis.
In this experiment, we did not consider ``Clock" and ``Reset" signals.

\begin{table}[t!]
\centering
\caption{Characteristics of the IP families used in evaluation}
\label{tab:IP-information}
\begin{tabular}{@{}cccc@{}}
\toprule
IP Family            & No. of IPs & No. of Files & No. of Lines \\ \midrule
Crypto               & 22            & 190          & 61934        \\
Interface-GPIO       & 12            & 16           & 5809         \\
Interface-Peripheral & 13            & 34           & 5118         \\ \bottomrule
\end{tabular}
\end{table}



\subsection{Results}

\begin{figure}[t]
\centering
\subfloat[Crypto]{\label{fig:Crypto}\includegraphics[width=0.30\columnwidth]{fig/crypto_crop_cm.pdf}}\hfill
\subfloat[Interface-GPIO]{\label{fig:GPIO}\includegraphics[width=0.30\columnwidth]{fig/gpio_crop_cm.pdf}}\hfill
\subfloat[Interface-Peripheral]{\label{fig:Peripheral}\includegraphics[width=0.30\columnwidth]{fig/peripheral_crop_cm.pdf}}
\caption{The confusion matrix for three IP families.\label{fig:confusion_matrix}
}
\end{figure}

Our tool's Accuracy and F1 score (out of 1) for the crypto family was 0.9902 and  0.8693, respectively. In GPIO, it was 0.9784 and 0.8400, respectively, and for peripherals, it was 0.9019 and 0.6771, respectively. 
Our results show that our proposed approach can achieve a low false positive rate and can significantly reduce the set of potential assets that need to be considered by a designer compared with a purely manual approach, such as that suggested by the IEEE White Paper~\cite{ieee_p3164_working_group_asset_2024}.
Notably, when we tried the GPIO example from~\cite{ieee_p3164_working_group_asset_2024}, our tool successfully identified the corresponding assets. 
We also evaluated our tool's identified assets against our list of manually-identified assets. 
The results are presented in~\autoref{fig:confusion_matrix} as confusion matrices that show the true positives (TP), true negatives (TN), false positives (FP), and false negatives (FN), where the ``positive'' class is that the signal is an asset. 

When the tool identified false positives and showed inaccuracies, we found these were often due to several reasons, including atypical signal names, incorrect spelling, improper spacing, and atypical abbreviated signals. 
Using external packages for nets and port declaration in RTL files and multi-level instantiations in large IP blocks also reduced accuracy.

The results highlight the potential of our proposed approach as a starting point for automated asset identification which can be used to support downstream activities such as automatic CWE scanning and creating properties for formal verification.

\subsection{Discussion \label{subsection:discussion}}
Asset detection poses challenges due to its potentially subjective nature and time-consuming reliance on manual effort. 
Hence, we were motivated to initiate the development of detection tools as a starting point for ongoing refinement and enhancement of accuracy.
Our results are centered on open-source projects, which might limit its out-of-the-box generalizability to other projects. 
However, our proposed approach remains applicable to 
source code for commercial chips and processors. 
Companies can adopt similar methodologies as discussed in our paper and tailor the developed tool to suit their particular requirements. 

As we observed, most IP designers use ``sensible'' names, often based on organizational styles like in OpenTitan~\cite{opentitan_open_nodate}. 
To use our approach, designers can explore their own styles to help create an initial set of keywords for potential assets and extend our approach to other IP families. 

Our proposed approach can be adapted to different types and sizes of IP families, although the initial adaptation requires some security expertise. 
To adapt the proposed approach for a different IP family or project, practitioners should use the initial manual process we presented to identify their own keywords or unique behaviors. Afterward, they can apply the automated approach to other related design projects. 

To prepare the keyword list, one can extract the input and output port names and net names from different projects of an IP family. 
Our results on open-source designs show that, despite different designers, it appears that patterns in names and behaviors exist, possibly due to design guidelines and conventions. 
Our approach of identifying ``partial keywords'' (as described in~\autoref{ssec:IED}) can help capture potential asset candidates.  
For a new project or design style, important signals need to be sorted into different categories of signal types (e.g., control signals, configuration signals, as in~\autoref{ssec:BP}).
Each category is linked to confidentiality, availability, integrity, or undermined behavior properties, as in~\cite{ieee_p3164_working_group_asset_2024}, so an identified asset's potential relevance to these properties can be reported. 
Using the sets of partial keyword groups and signal categories, all classification rules can be formulated for an IP class using the following strategy as follows. 

If a known signal, e.g., \texttt{test\_signal}, with one or more security objectives, according to to~\cite{ieee_p3164_working_group_asset_2024}, belongs to a partial keyword group, for instance, $PKG_{x}$, and has a \texttt{Control\_Signals[]} pattern, any signal that belongs to the same partial keyword group, $PKG_{x}$ and \texttt{Control\_Signals[]} pattern list, can be identified as a \textbf{Potential Asset}. A Signal, $s$, belongs to the known keyword group, $PKG_{x}$, and has \texttt{Control\_Signals[]} properties but not an Input signal, maybe a net or variable signal. In that case, to find out the potential \textbf{Primary Asset}, we need to identify the blocking/non-blocking assignment or continuous assignment where the signal, $s$, is assigned with a single-bit input port on the left-hand side. The Refinement Stage described in \autoref{sssec:RS} completes this process. 
