\section{Introduction}

The development of secure \acp{SoC} requires diligent consideration of security concern through a security development lifecycle (SDL)~\cite{Khattri_HSDL_2012}. 
To address security issues, design and verification teams should identify potential \textit{assets} in a system to make decisions regarding their protection and to help inform verification of any security features. 
Ideally, designers should be proactive in identifying and mitigating potential security threats, ultimately leading to more secure and trustworthy hardware systems.

Given that the detection and mitigation of potential security threats in the pre-silicon stage is ideal (because early changes are considerably easy and more cost-effective~\cite{Ahmad_2022}), designers require more tools that can help with security analysis in earlier stages of the design flow. 
This analysis starts by identifying primary security assets that need robust protection against security threats and attacks~\cite{mishra_hardware_2017}, currently a predominantly\textbf{ manual and subjective} process that relies on some level of security expertise~\cite{ieee_p3164_working_group_asset_2024}. Prior work has suggested that while some ``primary assets are often seen as `self-evident''' others are ``not \ldots{} immediately obvious''~\cite{Ayalasomayajula_Automatic_2024}. 
This initial step is crucial as it sets the direction for subsequent security measures and analyses.
Moreover, the evolving hardware security landscape has led to initiatives such as Accellera's \ac{SA-EDI}~\cite{accellera} and the related IEEE P3164 effort to establish a standard for \acp{IP} security collateral; these start with \textit{identifying assets}. 

Thus, in this work, we investigate whether it is possible to \textit{automate} initial potential \textbf{primary} asset identification. 
\textbf{This is in contrast to prior work that focuses on secondary asset identification} only~\cite{farzana_saif_2021,Ayalasomayajula_Automatic_2024}. 
Our proposed approach is based on identifying patterns in designs, and we explore a proof-of-concept by studying a series of open-source designs of several IP family types. 
Designers can adapt the proposed process to identify assets in their own projects. 
By reducing the manual workload faced by designers, this approach aims to streamline the early stages of security analysis. 
Our contributions are as follows:
\begin{itemize}
    \item Analysis of open source hardware designs for guidance on patterns related to assets in \ac{RTL} source code written in Verilog/SystemVerilog.
    \item A proposed approach for automating the primary asset identification that uses insights from our analyses of open-source hardware.
    \item Experimental evaluation of the proposed approach showing a True Positive Rate of 82.18\% compared to manually labeled asset lists.
\end{itemize}
