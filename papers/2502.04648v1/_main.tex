\documentclass[10pt, conference, letterpaper]{IEEEtran}
\pdfoutput=1 
\IEEEoverridecommandlockouts
% The preceding line is only needed to identify funding in the first footnote. If that is unneeded, please comment it out.
\usepackage{cite}%[nocompress]{cite}
\usepackage{amsmath,amssymb,amsfonts}
\usepackage{comment}
\usepackage{algorithmic}
\usepackage{booktabs}
\usepackage{graphicx}
\usepackage{textcomp}
\usepackage{xcolor}
\usepackage[pscoord]{eso-pic}
\usepackage{balance}
\usepackage{fancyvrb}

\usepackage{algorithm2e}
\RestyleAlgo{ruled}
\SetKwComment{Comment}{/* }{ */}


\def\BibTeX{{\rm B\kern-.05em{\sc i\kern-.025em b}\kern-.08em
    T\kern-.1667em\lower.7ex\hbox{E}\kern-.125emX}}
    
\newcommand{\placetextbox}[3]{% \placetextbox{<horizontal pos>}{<vertical pos>}{<stuff>}
  \setbox0=\hbox{#3}% Put <stuff> in a box
  \AddToShipoutPictureFG*{% Add <stuff> to current page foreground
    \put(\LenToUnit{#1\paperwidth},\LenToUnit{#2\paperheight}){\vtop{{\null}\makebox[0pt][c]{#3}}}%
  }%
}%
\def\BibTeX{{\rm B\kern-.05em{\sc i\kern-.025em b}\kern-.08em
    T\kern-.1667em\lower.7ex\hbox{E}\kern-.125emX}}
\def\BibTeX{{\rm B\kern-.05em{\sc i\kern-.025em b}\kern-.08em
    T\kern-.1667em\lower.7ex\hbox{E}\kern-.125emX}}

\placetextbox{0.5}{0.96}{\texttt{\textcolor{red}{\textbf{Accepted for presentation at ISQED'25 - DOI/link to IEEE Xplore will be updated}}}}

% \usepackage{minted}
% \usepackage[finalizecache,cachedir=.]{minted} 

\usepackage[frozencache,cachedir=.]{minted} 

\usepackage{listings}

\usepackage{pifont}
\usepackage{framed}
\usepackage{soul}
\usepackage{booktabs}  
\usepackage{multirow}
\usepackage{acronym}
\acrodef{IP}[IP]{intellectual property block}
\acrodef{SoC}[SoC]{System-on-Chip}
\acrodef{IC}[IC]{integrated circuit}
\acrodef{eFPGA}[eFPGA]{embedded field programmable gate array}
\acrodef{RTL}[RTL]{register-transfer level}
\acrodef{CPS}{Cyber-Physical System}
\acrodef{IoT}{Internet of Things}
\acrodef{CAD}{Computer-Aided Design}
\acrodef{EDA}{Electronic Design Automation}
\acrodef{HPC}{High-Performance Computing}
\acrodef{DL}{deep learning}
\acrodef{ML}{machine learning}
\acrodef{NLP}{natural language processing}
\acrodef{IC}{Integrated Circuit}
\acrodef{CWE}[CWE]{Common Weakness Enumeration}
\acrodef{CVE}[CVE]{Common Vulnerabilities and Exposures}
\acrodef{LLM}[LLM]{large language model}
\acrodef{NMT}[NMT]{neural machine translation}
\acrodef{IP}[IP]{hardware intellectual property block}
\acrodef{HDL}[HDL]{hardware description language}
\acrodef{RTL}[RTL]{register-transfer level}
\acrodef{SDL}[SDL]{security development lifecycle}
\acrodef{FSM}[FSM]{finite state machine}
\acrodef{AST}[AST]{abstract syntax tree}
\acrodef{SoC}[SoC]{system-on-chip}
\acrodef{SA-EDI}{Security Annotation for Electronic Design Integration standard}
\usepackage{array}
\newcolumntype{L}[1]{>{\raggedright\let\newline\\\arraybackslash\hspace{0pt}}m{#1}}
\newcolumntype{C}[1]{>{\centering\let\newline\\\arraybackslash\hspace{0pt}}m{#1}}
\newcolumntype{R}[1]{>{\raggedleft\let\newline\\\arraybackslash\hspace{0pt}}m{#1}}
\usepackage{url}
\usepackage[caption=false,font=footnotesize,labelformat=simple]{subfig} 
\renewcommand\thesubfigure{(\alph{subfigure})}
\newcommand{\todoblock}[1]{\noindent\definecolor{shadecolor}{RGB}{252, 217, 217}\colorbox{shadecolor}{\parbox{\columnwidth}{{#1}}}}
\newcommand{\todoblockblue}[1]{\noindent\definecolor{shadecolor}{RGB}{214, 254, 255}\colorbox{shadecolor}{\parbox{\columnwidth}{{#1}}}}
\newcommand{\todoblockgreen}[1]{\noindent\definecolor{shadecolor}{RGB}{175, 255, 191}\colorbox{shadecolor}{\parbox{\columnwidth}{{#1}}}}
\newcommand{\findcite}[1]{\sethlcolor{yellow}\hl{[#1]}}
\usepackage[bookmarks=false,hidelinks]{hyperref}
\def\sectionautorefname{Section}
\def\subsectionautorefname{Section}
\def\subsubsectionautorefname{Section}
\def\algorithmautorefname{Algorithm}
\def\figureautorefname{Fig.}
\def\subfigureautorefname{Fig.}
\def\tableautorefname{Table}
\newcommand{\hlr}[1]{{\color{red}{\hl{#1}}}}
\newcommand{\bt}[1]{{\color{blue}{#1}}}
\newcommand{\rt}[1]{{\color{red}{#1}}}
\newcommand{\pt}[1]{{\color{purple}{#1}}}
\newcommand{\cmark}{\ding{51}}%
\newcommand{\xmark}{\ding{55}}%
\newcommand{\sol}{MoP~}
\newcommand{\ts}[1]{\textsuperscript{#1}}
    
\begin{document}
\bstctlcite{IEEEexample:BSTcontrol}



\title{%
    Toward Automated Potential Primary Asset \\Identification in Verilog Designs%
}

\author{%

    \IEEEauthorblockN{Subroto Kumer Deb Nath and Benjamin Tan}
    \IEEEauthorblockA{\textit{Department of Electrical and Software Engineering} \\
    \textit{Schulich School of Engineering} \\
    \textit{University of Calgary}\\
    \{subroto.nath,benjamin.tan1\}@ucalgary.ca}
    \thanks{
    
    This research is supported in part by Alberta Innovates, and the Natural Sciences and Engineering Research Council of Canada (NSERC) [RGPIN-2022-03027]. Cette recherche a été financée en partie par le Conseil de recherches en sciences naturelles et en génie du Canada (CRSNG).
    This research work is partly supported by a gift from Intel Corporation. This work does not in any way constitute an Intel endorsement of a product/supplier. 

    \rt{\textcopyright 2025 IEEE. Personal use of this material is permitted. Permission from IEEE must be obtained for all other uses, in any current or future media, including reprinting/republishing this material for advertising or promotional purposes, creating new collective works, for resale or redistribution to servers or lists, or reuse of any copyrighted component of this work in other works
    }}
}


\IEEEtitleabstractindextext{
\begin{abstract}
With greater design complexity, the challenge to anticipate and mitigate security issues provides more responsibility for the designer. 
As hardware provides the foundation of a secure system, we need tools and techniques that support engineers to improve trust and help them address security concerns. 
Knowing the security assets in a design is fundamental to downstream security analyses, such as threat modeling, weakness identification, and verification. 
This paper proposes an automated approach for the initial identification of potential security assets in a Verilog design.
Taking inspiration from manual asset identification methodologies, we analyze open-source hardware designs in three IP families and identify patterns and commonalities likely to indicate structural assets. 
Through iterative refinement, we provide a potential set of primary security assets and thus help to reduce the manual search space. 
\end{abstract}
\begin{IEEEkeywords}
    Asset Identification, Verilog, Automation, Hardware Security
\end{IEEEkeywords}
}

\maketitle
\IEEEdisplaynontitleabstractindextext



\section{Introduction}
% 
Motion planning is a key ingredient in autonomous robotic systems, whose aim is computing collision-free trajectories for a robot operating in environments cluttered with obstacles~\cite{lavalle2006planning}. 
Over the years, various approaches have been developed for tackling the problem, including potential fields~\cite{luo2024potential}, geometric methods~\cite{halperin2017algorithmic}, and optimization-based approaches~\cite{SchulmanDHLABPPGA14,MalyutaEtAl2022,MarcucciEA23}. %, and sampling-based planners~\cite{}. 
In this work, we focus on sampling-based planners (SBPs), which aim to capture the structure of the robot's free space through graph approximations that result from configuration sampling (typically in a random fashion) and connecting nearby samples. 
SBPs have enjoyed popularity in recent years due to their relative scalability, in terms of the number of robot degrees of freedom (DoFs), and the ease of their implementation~\cite{OrtheyCK24}. 

\begin{figure*}[h!]
  \centering
  \subfloat[$\X_{\dZ_2}^{\delta,\epsilon}$ sample set.]{
    \includegraphics[width=0.27\textwidth, trim={2.2cm 1.7cm 0.9cm 1.0cm},clip]{Images/ZN_2D.png}
    %\label{fig:2d_lattices:z}
    }
  \hfil
  \subfloat[$\X_{D_2^*}^{\delta,\epsilon}$ sample set.]{
    \includegraphics[width=0.27\textwidth, trim={2.2cm 1.8cm 0.9cm 1.0cm},clip]{Images/DN_2D.png}
    %\label{fig:2d_lattices:d}
    }
  \hfil
  \subfloat[$\X_{A_2^*}^{\delta,\epsilon}$  sample set.]{
    \includegraphics[width=0.27\textwidth, trim={2.4cm 1.7cm 0.9cm 1.0cm},clip]{Images/AN_2D.png}
    %\label{fig:2d_lattices:a}
    }
  \caption{Sample sets within a fixed disc in $\dR^2$, derived from the lattices $\dZ^2, D_2^*$ and $A^*_2$, which yield \decomp guarantees for the same values of $\delta$ and $\eps$. The set $\X_{\dZ_2}^{\delta,\epsilon}$ can be viewed as a tessellation of space using cubes. The set $\X_{D_2^*}^{\delta,\epsilon}$ is obtained by placing a (rescaled) standard grid, and then placing another point in the middle of each cube. The set $\X_{A_2^*}^{\delta,\epsilon}$ can be viewed as a rescaled hexagonal grid as each point is surrounded by a hexagon whose vertices are points in the set. Note that the density of $\X_{\dZ^2}^{\delta,\eps}$ and $\X_{D^*_2}^{\delta,\eps}$ is the same, and higher than the density of $\X_{A^*_2}^{\delta,\eps}$.}
  \label{fig:2d_lattices}
\end{figure*}

Another key benefit is the ability of SBPs to escape local minima (unlike potential fields) and global solution guarantees (in contrast, optimization-based approaches~\cite{SchulmanDHLABPPGA14}, which typically provide only local guarantees). Earlier work on the theoretical foundations of SBPs has focused on deriving probabilistic completeness (PC) guarantees for methods such as PRM~\cite{kavraki1996probabilistic} or RRT~\cite{LaVKuf01,KunzS14,Kleinbort.Solovey.ea.19}. PC implies that the probability of a given planner finding a solution (if one exists) converges to one as the number of samples tends to infinity. The work of~\citet{karaman2011sampling} initiated studying the quality of the solution returned by SBPs. Specifically, they introduced the planners PRM* and RRT*, and proved that the solution length of those planners converges to the optimum as the number of samples tends to infinity---a property called asymptotic optimality (AO). Subsequent work has introduced even more powerful AO planners for geometric~\cite{JSCP15,GammellBS20} and dynamical~\cite{HauserZ16,LiETAL16} systems.

Unfortunately, the practical relevance of the aforementioned theoretical findings remains limited due to the lack of meaningful finite-time implications. Specifically, when a solution is obtained using a finite number of samples, it is unclear to what extent its quality can be improved with additional computation time. Moreover, in cases where no solution is returned, it is uncertain whether a solution does not exist or if the algorithm simply failed to find one. Developing finite-time bounds through randomized sampling continues to be a significant challenge~\cite{DobsonMB15,shaw2024towards}.

Deterministic sampling methods such as grid sampling or Halton sequences~\cite{lavalle2006planning}, where samples are generated according to a geometric principle, can improve the performance of SBPs in practice and simplify the algorithm analysis. Specifically, some deterministic sampling procedures have a significantly lower dispersion than uniform random sampling, which implies that the former requires fewer samples to cover the search space to a desired resolution~\cite{janson2018deterministic}. 
Recently, Tsao et al.~\cite{tsao2020sample} have leveraged deterministic sampling to disrupt the asymptotic analysis paradigm by introducing a significantly stronger notion than AO, called \decomps, that yields finite-time guarantees for PRM-based algorithms such as PRM*~\cite{karaman2011sampling}, FMT*~\cite{JSCP15}, BIT*~\cite{GammellBS20}, and GLS~\cite{MandalikaCSS19}. Informally, a \emph{finite} sample set is \decomp for a given approximation factor $\eps>0$ and clearance parameter $\delta>0$, if the corresponding planner returns a solution whose length is at most $(1+\eps)$ times the length of the shortest $\delta$-clear solution. If no solution is found using a \decomp sample set then no solution of clearance $\delta$ exists. 

The work of~\citet{tsao2020sample} derived a relation between \decomps and geometric space coverage to obtain lower bounds on the number of samples necessary to achieve \decomps, as well as upper bounds accompanied with explicit (deterministic) sampling distributions. A follow-up work by~\citet{dayan2023near} has introduced an even more compact \decomp sample distribution that is more efficient than the one proposed in~\cite{tsao2020sample} or rectangular grid sampling. In particular, the staggered grid~\cite{dayan2023near} consists of two shifted and rescaled copies of the rectangular grid (see Figure~\ref{fig:2d_lattices} and Figure~\ref{fig:3d_lattices}). 

However, the work~\cite{dayan2023near} still leaves a significant gap between the lower bound in~\cite{tsao2020sample} and the upper bound obtained with the staggered grid. In practice, this gap limits the applicability of the \decomps theory to relatively low dimensions (up to dimension 6) due to the large number of samples currently needed to satisfy this property, which can lead to excessive running times. 

\vspace{5pt}
\noindent \textbf{Contribution.} In this work, we develop a theoretical framework for obtaining highly-efficient \decomp sample sets by leveraging the foundational theory of lattices\footnote{Lattices are point sets exhibiting a regular geometric structure, which are obtained by transforming the integer lattice $\dZ^d$. For instance, the aforementioned rectangular grid and the staggered grid can be viewed as lattices.}~\cite{conway2013sphere}, which has been instrumental in diverse areas from number theory~\cite{siegel_geometry_numbers}, coding theory~\cite{ebeling2013lattices}, and crystallography~\cite{sands1994introduction}. Specifically, we show that lattices can be transformed to obtain \decomp sample sets (Theorem~\ref{thm:decomp_lattices}) and develop tight theoretical bounds on their size (Theorem~\ref{thm:general_sample_complexity}), which allows to compare between different sample sets qualitatively. 
Using this machinery, we not only refine and generalize previous results on the staggered grid~\cite{dayan2023near} but also introduce a new highly efficient \decomp sample set that is based on the $\AN$ lattice, which is famous for its minimalist coverage properties~\cite{conway2013sphere}. We also initiate the study of a new property, which estimates the computational cost resulting from using a given sample set in a more informative manner than sample complexity. In particular, the property called collision-check complexity captures the amount of collision checks, which is typically a computational bottleneck.

From a practical perspective, when solving motion-planning problems using lattice-based sample sets, we show that our $\AN$-based sample sets can result in at least order-of-magnitude improvement in terms of running time over staggered-grid samples and two orders of magnitude improvements over rectangular grids. Moreover, $\AN$-based sample sets are vastly superior in practice to the widely-used uniform random sampling, which is evident in improved running times, success rates, and solution quality.

\vspace{5pt}
\noindent \textbf{Organization.} In Section~\ref{sec:preliminaries} we review basic definitions on motion planning and \decomps, and formally define our objectives. In Section~\ref{sec:lattices}, we develop a general tool for transforming lattices into \decomp sample sets. We obtain sample-complexity bounds for lattice-based sample sets in Section~\ref{sec:sample_complexity}, and generalize those bounds to collision-check complexity in Section~\ref{sec:collision_complexity}. We evaluate the practical implications of our theory in Section~\ref{sec:experiments}, and conclude with a discussion of limitations and future directions in Section~\ref{sec:future}.


% \itai{Added the intro. used some from the thesis-proposal, added different stuff at the end}
%  the field of autonomous robots, the problem of getting a robot “from point A to point B” can be divided into three general stages: estimating the robot's position, planning the robot's path and controlling the robot. We use estimation methods (like the Kalman Filters) to understand where we are in the world, we use planning methods to figure out how to reach the goal, and we use control methods (like PID) to follow the planned path during execution.


% Focusing on the planning part of the problem, instead of using the \emph{workspace} of the robot it is convenient to use a representation of it called a \emph{configuration space}---a parameterization of the robot’s position in space, which turns the set of points defined as a \emph{robot} to a single-point robot. A quick example would be thinking of a polygon in the workspace as three parameters: $(x,y)\in \mathbb{R}^2$ for its location, and $\theta\in[0,2\pi]$ for its rotation, which means the configuration space is $\mathbb{R}^2\times S^1$. Furthermore, we use the term \emph{free space} in both contexts to describe the area of the space with no obstacles. 


% Even though using configuration spaces is much more convenient in terms of the robot being a single point, it quickly becomes apparent that even simple configuration spaces of dimensions $d\geq 3$ can be challenging to properly describe (due to the need to describe the obstacles in the new space, among other reasons). Thus, instead of explicitly representing the whole configuration space, methods were developed to sample the space: sampling-based approaches aim to approximate the space via a graph structure that is induced by sampled configurations. This can drastically reduce the computational effort of path planning.


% One of the most widely used sampling-based algorithms is the \emph{probabilistic roadmap method} (PRM)~\cite{kavraki1996probabilistic}. This approach generates (typically random) samples across the space, and connects nearby samples while checking for collisions with obstacles, which gives rise to a graph data structure---a path between two nodes in the graph yields a collision-free path for the robot connecting between the two configurations corresponding to the end-point nodes. PRM has the theoretical guarantee to return a path with a probability tending to 1 if enough samples are generated~\cite{laddgeneralizing}. 


% Another well known sampling-based planner is the \emph{rapidly-exploring random trees} (RRT)~\cite{lavalle1998rapidly}: It randomly expands towards nearby samples in space, creating in the process a “tree” structure that eventually finds a path to the goal~\cite{kleinbort2018probabilistic}. Later, a notion of \emph{asymptotically optimal} (AO) algorithms was introduced: with infinite samples, the algorithm can converge to an \textbf{optimal} path. Both PRM and RRT  were expanded to AO versions (PRM*, RRT*) in a paper by Karaman et al.~\cite{karaman2011sampling}. RRT itself had seen many expansions, including dRRT/dRRT* to apply to multiple robots~\cite{solovey2015finding, dobson2017scalable}.

% Approximately optimal methods were demonstrated using deterministic sample sets, achieving good results in finite time, in Dayan et al.'s paper~\cite{dayan2023near}, demonstrating superior results over random sets---although those improvements diminish as the desired approximation factor of the optimal path lowers.

% Still, all these methods have a main limitation: the number of points required to guarantee finding a path rises exponentially with the robot's degrees of freedom~\cite{tsao2020sample}.


% Dayan et al.~\cite{dayan2023near}, using a "staggered grid" structure (recognized in this paper as the $\DN$ set), gave guarantess for an approximately-optimal solution in finite time, which outperformed random sets in certain situations. For this, they introduced the concept of a \decomp set, a set that generates such approximate solutions. Still, as we seek better and better approximations for the optimal path, the staggered grid in the paper falls off against random sets. The question that stands, then, is what other sample sets can be used to provide better results?


% In this paper, we would like to utilize \decomp sets and investigate a specific series of deterministic sample sets, using lattices---a generalization of the regular grid structure using a general set of mutually-independent base vectors (not necessarily the usual $(0,\dots,1,\dots,0)$ vectors). We first familiarize the reader with three different lattices \Lattices we intend on investigating, and then move on to using Dayan et al.'s~\cite{dayan2023near} definition of a \decomp set to define lattice sample sets as such sets. This definition tells us that these sets can give us a good approximation for the optimal solution at a finite time. 


% After that, we use our new lattice sample sets to investigate the upper bounds on the number of sample points, and on the sum of edge length in a typical PRM vertices-connecting $r$-Ball---something we use as a measure point to the algorithm's complexity, as it is known that collision checks along the edges are the bottleneck in today's PRM algorithms.


% We will end up demonstrating, theoretically and practically, that one lattice, $\AN$, stands out as performing much better than the regular grid often used in many MP algorithms.
\section{Background}\label{sec:background}
\shortsection{Representational Alignment.}
Representational alignment studies the extent to which internal representations of machine learning models correspond to human cognitive processes. 
Early studies found that deep neural networks (DNNs) trained on large-scale image datasets develop hierarchical feature representations similar to those observed in the primate ventral stream, particularly in high-level visual areas like the inferior temporal (IT) cortex \cite{yamins_hierarchical_2013, schrimpf_brain-score_2018}. This led to efforts to quantify the alignment between artificial and biological vision, using techniques such as Representational Similarity Analysis (RSA) \cite{kriegeskorte_representational_2008} and Centered Kernel Alignment (CKA) \cite{kornblith_similarity_2019}. Current research in the area primarily focuses on measuring, bridging, and increasing both neural and behavioral alignment. 



To improve alignment, researchers have proposed strategies that incorporate cognitive constraints or psychological priors into model architectures~\cite{dapello_simulating_2020}. Supervised fine-tuning with human-annotated datasets~\cite{dosovitskiy_image_2021} ensures that learned representations align more closely with human-understandable features. Furthermore, novel techniques~\cite{muttenthaler_improving_2023,li_learning_2019,cheng_rtify_2024} have been developed to encourage similarity between model activations and human neural responses as recorded through fMRI and EEG experiments. In this study, we use a comprehensive set of neural, behavioral, and engineering alignment metrics to quantify representational alignment. 
  


\shortsection{Adversarial Examples.}
% Why do people care about adversarial examples? 
Although machine learning models have shown strong capabilities in achieving high accuracy across various tasks~\cite{liu_convnet_2022, dosovitskiy_image_2021, krizhevsky_imagenet_2017, he_deep_2016}, they remain vulnerable to adversarial examples~\cite{croce_reliable_2020,madry_towards_2019, carlini_towards_2017,goodfellow_explaining_2015, sheatsley_space_2023}. Adversarial examples are specially crafted inputs that contain perturbations which are imperceptible to humans, yet significantly decrease model accuracy. In computer vision systems, there have been many studies on developing attack algorithms, such as FGSM~\cite{goodfellow_explaining_2015}, PGD~\cite{madry_towards_2019}, and AutoAttack~\cite{croce_reliable_2020}. These methods  aim to maximize model's loss subject to constraints of perturbations defined by certain $\ell_p$-norms as follows:


\begin{center}
    $x_{adv} = \argmax_{\left \| \delta \right \|_{p}\leq\epsilon} L(x + \delta, y)$
\end{center}

where $x$ and $y$ represent the original image and its predicted label, respectively, $\delta$ is the perturbation to solve for, and $L$ is the model's loss function. The perturbation constraint $\epsilon$ is measured through an $\ell_p$-norm---most commonly $\ell_\infty$. While many works have historically evaluated the robustness of their model through the PGD attack~\cite{madry_towards_2019}, it has been shown that ``robust'' models can often suffer from gradient masking, causing gradient-based attacks like PGD to fail~\cite{athalye_obfuscated_2018}, and leading to a sense of overestimated robustness. To overcome this, multiple attacks, including both white- and black-box attacks should be used~\cite{carlini_evaluating_2019}. Thus, the  AutoAttack ensemble~\cite{croce_reliable_2020} has become the de-facto standard for evaluating robustness.




\section{Proposed Method}

We used a heatmap-based approach to detect particle points, which is the most commonly employed technique in human pose estimation~\cite{Newell2026} and facial keypoint detection~\cite{Bulat2017}.
Since this competition involves 3D images rather than 2D images, we utilized two types of U-Net~\cite{Ronneberger2015} models (yu4u's model and tattaka's model) that take 3D voxels as input and output 3D heatmaps.

\subsection{Validation Strategy}
Designing an appropriate validation strategy is crucial for selecting the model architecture and tuning hyperparameters.  
We adopted a 7-fold cross-validation (CV), where each of the seven training samples was used for validation.
However, since the number of training data was significantly smaller compared to the test data, we observed that the CV scores were not well correlated with the leaderboard scores.

We used the CV scores only to confirm that the metric produced was somewhat reasonable and to select model checkpoints. For models with potential for improvement, we submitted them and relied on the public leaderboard to decide which methods to adopt or discard.


\subsection{Creating the Ground Truth Heatmap}
We generate the ground truth heatmap necessary for model training. This involves converting the ground truth particle coordinates into the pixel coordinate system and creating a mask using a Gaussian function, where the particle center is set to 1.0 and $\sigma$ is 6 pixels for yu4u's model. For tattaka's model, different $\sigma$ values were used for different particles based on their sizes.

We argue that an offset of 1.0 should be added when converting particle coordinates into the pixel coordinate system. While the discussion~\cite{David2025} suggests adding 0.5, we demonstrate that 1.0 is the correct value~\cite{yu4u2025}. The main difference is that the previous discussion assumes the particle center is at the top-left of a pixel, whereas we argue that the circle should be drawn from the pixel center on average.


\subsection{yu4u's Model}
We adopted a 2.5D U-Net~\cite{Kumar2024}, which utilizes a 2D image-based model as the backbone.
The outputs from each stage of this backbone are pooled along the depth direction, enabling hierarchical feature extraction in the depth (Z) dimension as well.
This idea was inspired by the excellent notebook~\cite{hengck232025}.
An interesting observation is that replacing this pooling operation with strided 3D convolutions degrades performance.
This would be because the pooling method effectively aggregates depth features while preserving the original 2D backbone’s feature maps as much as possible.
Similar to many other Kaggle competitions dealing with 3D data, a U-Net utilizing a 2D backbone pretrained with ImageNet outperformed a straightforward U-Net with a 3D backbone~\cite{Cicek2016} in our preliminary experiments.

We also applied 3D convolutions between the encoder and decoder to further extract depth-wise features, inspired by the 3rd place solution of the contrails competition~\cite{knshnb2025}.


Initially, we used a plain 3D U-Net decoder, but processing high-resolution feature maps required significant memory and computation. To address this, we adopted a model that outputs the final heatmap using pixel shuffle from a feature map with a stride of 4. Pixel shuffle~\cite{Shi2016}, also known as \texttt{depth\_to\_space} in TensorFlow, is an operation that redistributes information from the channel dimension to the spatial dimensions. Compared to deconvolution, it offers advantages in computational efficiency and reducing artifacts.
In the other upsampling parts of the decoder, an upsampling layer and 3D convolution blocks are used.


For the final submission, we adopted a ConvNeXt Nano~\cite{Sanghyun2023} model as the backbone.
The overall structure of yu4u's model is shown in Fig.~\ref{fig:yu4u_model}


\begin{figure}[tb]
    \centering
    \includegraphics[width=0.5\textwidth]{fig/yu4u_model.pdf}
    \caption{The architecture of yu4u's model.}
    \label{fig:yu4u_model}
\end{figure}




\begin{table}[tb]
    \centering
    \footnotesize
    \begin{tabular}{ll}
        \toprule
        Training Epochs & 64 \\
        Learning Rate & $10^{-3}$ \\
        Optimizer & AdamW \\
        Weight Decay & 0 \\
        Warmup Epochs & 4 \\
        LR Scheduling Strategy & Cosine Decay \\
        Batch Size & 32 \\
        EMA Decay & 0.999 \\
        \bottomrule
    \end{tabular}
    \caption{Hyperparameters used for training yu4u's model.}
    \label{tab:hyperparams1}
\end{table}


\subsubsection{Loss Function for yu4u's Model}
Since the number of particles within the volume is small, there is a significant class imbalance between positive and negative samples during training.
To address this imbalance problem, we utilized the extended MSE loss function for yu4u's model, where the heatmap values were used as weights.
In practice, since areas without particles would have a weight of zero, a fixed value $\alpha = 0.1$ was added to the heatmap values to be used as weights:
\begin{equation}
    \mathcal{L}_\text{yu4u}(p, y) = \text{mean} (\text{MSE}(p, y) \cdot (y + \alpha)),
\end{equation}
where $p$ is a predicted heatmap and $y$ is the corresponding ground-truth heatmap.
The hyperparameters used for training yu4u's model are shown in Table~\ref{tab:hyperparams1}.


\subsection{tattaka's Model}
This model is a lightweight 2.5D U-Net with ResNetRS50~\cite{Bello2021} as the backbone.
The input to the model is a volume of size $32 \times 128 \times 128$, and it outputs a 3D heatmap of the same size. Within the backbone, the depth is progressively reduced by half using average pooling for the first two stages. After that, average pooling with kernel=3, stride=1, padding=1 is used to maintain the depth while facilitating information exchange along the depth dimension.

In the decoder, the three lowest-resolution feature maps are fed into joint pyramid upsampling~\cite{wu2019fastfcn} to generate a feature map that contains information at multiple resolutions. The feature map is then progressively upsampled using upsampling blocks until they reach the same size as the input volume.
The upsampling block consists of a 3D conv, an seSC~\cite{Roy2018} attention block, and an upsampling layer.
The overall structure of this 2.5D U-Net is shown in Fig.~\ref{fig:tattaka_model}.


\begin{figure}[tb]
    \centering
    \includegraphics[width=0.5\textwidth]{fig/tattaka_model.pdf}
    \caption{The architecture of tattaka's model.}
    \label{fig:tattaka_model}
\end{figure}



\subsubsection{Loss Function for tattaka's Model}
When training this model, it is also necessary to address the imbalance issue in the heatmap, just as in the training of yu4u's model.
Here, we tackle this issue by using a slightly different loss function. Specifically, we compute the MSE loss separately for the positive and negative regions, then take the sum of their weighted means as the final loss function:
\begin{equation}
\begin{aligned}
    \mathcal{L}_\text{pos}(p, y) &= \frac{\sum (\text{MSE}(p, y) \cdot y)}{\sum y + \epsilon}, \\
    \mathcal{L}_\text{neg}(p, y) &= \frac{\sum (\text{MSE}(p, y) \cdot (1 - y))}{\sum (1 - y) + \epsilon}, \\
    \mathcal{L}_\text{tattaka}(p, y) &= \mathcal{L}_\text{pos}(p, y) + \mathcal{L}_\text{neg}(p, y),
\end{aligned}
\end{equation}
where $\epsilon$ is a small constant added to prevent division by zero.
This loss function not only resolves the imbalance issue but also accelerates convergence during training.
The hyperparameters used for training tattaka's model are shown in Table~\ref{tab:hyperparams2}.


\begin{table}[tb]
    \centering
    \footnotesize
    \begin{tabular}{ll}
        \toprule
        Training Epochs & 25 \\
        Learning Rate & $10^{-3}$ \\
        Optimizer & AdamW \\
        Weight Decay & $10^{-2}$ \\
        Warmup Epochs & 5 \\
        LR Scheduling Strategy & Cosine Decay \\
        Batch Size & 32 \\
        EMA Decay & 0.999 \\
        \bottomrule
    \end{tabular}
    \caption{Hyperparameters used for training tattaka's model.}
    \label{tab:hyperparams2}
\end{table}


\subsection{Inference Procedure}
Finally, we used four yu4u's models and three tattaka's models in the final submission.
To complete prediction within the time limit, we optimized our models by converting them to TensorRT format for faster inference. The conversion process was based on the notebook~\cite{Lion2025}.  
Additionally, we selected a Kaggle Notebook instance with dual T4 GPUs and leveraged multiprocessing to parallelize inference.

The input size in the XY dimensions for both models is 128×128, while the target inference size is 630×630. Therefore, inference is performed by sliding overlapping windows as shown in Fig.~\ref{fig:window}.
First, the input is padded to 656×656, and inference is conducted by moving a 128×128 window with a stride of 48. In this case, the number of inference windows becomes 12×12.
In the z-direction, yu4u's model, which has an input depth of 16, moves with a stride of 8, while tattaka's model, which has an input depth of 32, moves with a stride of 16 during inference.
The results of these models are all aggregated by taking the average.

Since the values near the edges of the window in the inference results have lower confidence, we apply weighting to reduce their impact on the final result.
This is achieved by creating a weight matrix of the same size as the inference result, where the center value is 1 and the weights decrease linearly to 0 at the edges of the window.


\begin{figure}[tb]
    \centering
    \includegraphics[width=0.25\textwidth]{fig/window.pdf}
    \caption{Sliding overlapping windows used in inference.}
    \label{fig:window}
\end{figure}



\subsection{Post Processing}
For the final heatmap, we first detect local maxima using Non-Maximum Suppression (NMS), which is efficiently implemented via max pooling with a kernel size of 7. The detected points are then filtered using different thresholds for each particle type.

Since the detected points are in the pixel coordinate system, we need to convert them to the particle coordinate system.
This is done as follows.
\begin{enumerate}[label=\arabic*.]
  \item \textbf{Centering}: Add 0.5 to the pixel coordinates to shift from the pixel’s top-left to its center.
  \item \textbf{Offset Correction}: Subtract the 1.0 offset that was added during heatmap generation.
  \item \textbf{Scaling}: Multiply by 10.012 to convert the adjusted pixel coordinates to the particle coordinate system.
\end{enumerate}


\section{Results}
Table~\ref{tab:results} shows the competition results of our approach, including scores on the public leaderboard and private leaderboard.
Our final submission is an ensemble of four yu4u's models with different folds and three tattaka's models.

Table~\ref{tab:results} also presents two versions of our submission: one where the number of windows in the XY direction was reduced from $12\times12$ to $8\times8$, and another where no weighting was applied based on the location within the window during inference.  
As the results indicate, using more windows --- i.e., increasing the overlap between windows --- and applying location-based weighting within windows during inference are both crucial for improving the score.


\begin{table}[tb]
    \centering
    \small
    \begin{tabular}{lcc}
        \toprule
        Method & Private LB & Public LB \\
        \midrule
        final submission & 0.783 & 0.788 \\
        $8 \times 8$ window & 0.779 & 0.784 \\
        w/o weight & 0.776 & 0.780 \\
        \bottomrule
    \end{tabular}
    \caption{Final submission performance with ensemble models on the competition leaderboards.}
    \label{tab:results}
\end{table}


\subsection{Things That Does Not Work}
Below, we present representative approaches that our team attempted but did not perform well.

\begin{itemize}
  \item \textbf{Two-stage model}: We built a model that refines the scores by cropping regions around the points detected using a heatmap approach and then applying a classification model to those cropped regions. Although it worked well in terms of CV scores, it did not improve LB performance. This may be due to the high difficulty of appropriately adjusting the threshold for the first stage and the second stage in the case of a two-stage model.
  \item \textbf{Detection model}: We built a CenterNet-like~\cite{Duan2019} object detection model (more precisely, a point detection model), but it achieved a significantly lower CV score compared to heatmap-based methods.
\end{itemize}



\section{Conclusion}
This paper introduced the detailed approach of the yu4u \& tattaka team in tackling the CZII - CryoET Object Identification competition.
Our solution adopted a heatmap-based keypoint detection approach, utilizing an ensemble of two different types of 2.5D U-Net models with depth reduction.
Despite its highly unified and simple architecture, our method achieved 4th place, demonstrating its effectiveness. We hope that our approach will contribute to the advancement of machine learning-based recognition of cryoET tomograms.

\section{Experiments}
\label{sec:exp}

We conduct extensive experiments to address the following research questions:
1) How effectively can our retrieval mechanism enhance LLM-based TabICL in leveraging large-scale datasets?
2) How does LLM-based TabICL perform in comparison to numeric-based TabICL models and classic tabular models that are well-tuned on a case-by-case basis?
3) What are the unique strengths, current limitations, and potential future directions for LLM-based TabICL?


% \begin{figure*}[t]
% \vskip 0.2in
% \begin{center}
% \centerline{\includegraphics[width=\linewidth]{main_figures/main_scaling_pool_and_context.pdf}}
% \caption{
% We investigate the effects of increasing the number of training instances ($|D_{\text{train}}^{T'}|$) and the number of in-context instances per test example ($N^C$) on the TabICL performance of Phi3-GTL models.
% In each subplot, we compare the scaling effects of two Phi3-GTL models with different retrieval policies: one that randomly selects in-context instances, denoted as "Random," and the other employing our default \texttt{TabRAG} module, denoted as "RAG".
% We use violin plots to visualize the performance distribution across multiple held-out datasets. Additionally, dashed lines are used to emphasize that the median prediction error of our approach follows a power-law relationship with the number of training instances.
% }
% \label{fig:scaling_pool_ctx}
% \end{center}
% \vskip -0.2in
% \end{figure*}


\subsection{Experimental Setups}
\label{sec:exp_setup}

\paragraph{LLM Post-Training}
We use real-world tabular datasets to post-train a base LLM using generative tabular learning (GTL) objective as did in~\citep{wen2024GTL}.
However, unlike their approach, we adopt Phi-3~\citep{abdin2024phi3} as the base LLM, extending the effective context length from 4K to 128K and aligning it with our default retrieval policy.
Details of this post-training process are provided in Appendix~\ref{app:method_align_rag_llm}. For brevity and clearness, we denote our post-trained model as Phi3-GTL and refer to our approach as RAG+Phi3-GTL throughout the remainder of this paper.

\paragraph{Held-out Datasets}
We compile a comprehensive benchmark from the literature~\citep{gorishniy2021revisit_tab_dnn,grinsztajn2022tree_gt_tab_nn,gorishniy2024TabR,wen2024GTL}, ensuring diverse datasets that may favor different learning paradigms.
To avoid data leakage, we carefully examine and exclude any datasets used during the training of the Phi3-GTL model.
This process results in 29 classification datasets and 40 regression datasets for held-out evaluation, covering a wide range of domains, feature dimensions, types, and distributions.
Details of the data construction process are included in Appendix~\ref{app:data_constr}.

\begin{figure*}[t]
\vskip 0.2in
\begin{center}
\centerline{\includegraphics[width=0.98\linewidth]{main_figures/main_overall_comp_raw_metric.pdf}}
\caption{
An overall performance comparison of all models. In the left subplot, we use violin plots to show the AUROC scores of different models across 29 classification tasks, while the right subplot displays the NMAE scores for 40 regression tasks. Models are sorted by their median metric score across the held-out datasets, with dashed lines indicating these median scores in each subplot. Our approach, RAG+Phi3-GTL, is prefixed with a marker (*), for quick identification.
}
\label{fig:overall_comp}
\end{center}
\vskip -0.2in
\end{figure*}

\begin{figure*}[t]
\vskip 0.2in
\begin{center}
\centerline{\includegraphics[width=0.90\linewidth]{main_figures/main_ensemble_norm_metric.pdf}}
\caption{
    Ensemble performance comparisons of RAG+Phi3-GTL, TabPFN-v2, LightGBM, and CatBoost are presented, where normalized AUROC or NMAE scores (min-max normalized across methods for each dataset) are plotted to highlight their relative strengths across multiple datasets, while omitting absolute metric differences.
}
\label{fig:ensemble_res}
\end{center}
\vskip -0.2in
\end{figure*}


\paragraph{Baselines}
We include Phi3-GTL and RAG+KNN as two ablated variants of RAG+Phi3-GTL: the former uses randomly selected in-context instances, while the latter employs the same default retrieval policy but relies on the K-Nearest Neighbors (KNN) algorithm~\citep{fix1951knn,cover1967knn_cls} for prediction.
We compare against TabPFN-v1~\citep{hollmann2023TabPFN}, which supports only classification tasks, and TabPFN-v2~\citep{hollmann2025TabPFNv2}, the state-of-the-art TabICL model utilizing numeric representations.
In addition, our baselines include other representative tabular models such as XGBoost~\citep{chen2016XGBoost}, LightGBM~\citep{ke2017LightGBM}, CatBoost~\citep{prokhorenkova2018catboost}, MLP, FTT~\citep{gorishniy2021revisit_tab_dnn}, and TabR~\citep{gorishniy2024TabR}, all of which are extensively tuned via hyperparameter search for each dataset.
We also include several ``RAG + X'' baselines, where "X" represents models trained and inferred on the selected in-context instances using the same retrieval policy as RAG+Phi3-GTL. These include Logistic Regression (LR), TabPFN-v1, TabPFN-v2, and XGBoost.
These baselines are designed to highlight the TabICL capability of LLMs given limited in-context instances.

\paragraph{Metrics}
For classification tasks, we use the Area Under the Receiver Operating Characteristic curve (AUROC) as the primary evaluation metric. For regression tasks, we employ the Mean Absolute Error normalized by the label mean (NMAE).
Additionally, to compare the relative performance across a group of methods, we utilize group-wise min-max normalized AUROC and NMAE metrics.


\begin{figure*}[t]
\vskip 0.2in
\begin{center}
\centerline{\includegraphics[width=\linewidth]{main_figures/main_per_dataset_comp.pdf}}
\caption{
    Per-dataset performance comparisons between RAG+Phi3-GTL and the two most competitive baselines, TabPFN-v2 and CatBoost, are presented, with dataset IDs sorted by performance gaps. Dashed lines and annotations are used to indicate the proportion of datasets where RAG+Phi3-GTL outperforms these baselines and where it significantly lags behind.
}
\label{fig:per_dataset_comp}
\end{center}
\vskip -0.2in
\end{figure*}


\begin{figure*}[t]
\vskip 0.2in
\begin{center}
% \centerline{\includegraphics[width=0.9\linewidth]{main_figures/main_decision_boundary.pdf}}
\centerline{\includegraphics[width=1.0\linewidth]{main_figures/main_decision_boundary.pdf}}
\caption{
Decision boundary comparisons of various models, where each row corresponds to a specific set of training instances generated from a given data distribution. The first column visualizes these training instances, while the subsequent columns illustrate the decision boundaries of different models. The top two rows represent the same data distribution but with varying numbers of training instances, whereas the bottom two rows depict a different data distribution.
}
\label{fig:decision_boundary}
\end{center}
\vskip -0.2in
\end{figure*}


\subsection{Scaling with Available Training Instances}
\label{sec:exp_scaling_pool_ctx}

Figure~\ref{fig:scaling_pool_ctx} illustrates the performance variations as the size of the training data and the number of in-context instances increase for our approach under two retrieval policies: Random and RAG (our default retrieval policy).
It is evident that the RAG policy enables Phi3-GTL to effectively leverage larger training datasets, while the Random policy lacks this capability. Specifically, with the RAG policy, the median prediction error demonstrates a power-law relationship with the number of training instances, expressed as $L(D) = (D_c / D)^{\alpha}$. For classification tasks, $L=1-\text{AUROC}$, $D_c \sim 6.05e^{-5}$, and $\alpha \sim 0.102$, whereas for regression tasks, $L=\text{NMAE}$, $D_c \sim 8.05e^{-8}$, and $\alpha \sim 0.053$. This finding highlights a favorable statistical learning characteristic: given a distinguishable feature space and sufficient training instances, the expected prediction error approaches zero.

Moreover, the RAG policy reduces the number of in-context instances required for accurate predictions. As shown in the right two subplots of Figure~\ref{fig:scaling_pool_ctx}, the model using the Random policy benefits significantly from an increased number of in-context instances. In contrast, with the RAG policy, performance often saturates as the number of adaptive in-context instances increases. This indicates that, for most datasets, tens of training instances are sufficient to form a supportive context for inferring the label of a test instance.


\subsection{Overall Comparison}
\label{sec:exp_overall_comp}


Figure~\ref{fig:overall_comp} presents an overall comparison of all models by illustrating the error distributions across held-out datasets. TabPFN-v2 emerges as the most competitive baseline in terms of the median prediction error. However, its wider error bars indicate sub-optimal performance in certain cases. In contrast, well-tuned tree-based models such as LightGBM and CatBoost, as well as neural models like FTT and TabR, demonstrate more robust performance with narrower error distributions.

When comparing our approach, RAG+Phi3-GTL, with these baselines, we observe significant improvements over its ablated variants, Phi3-GTL and RAG+KNN, underscoring the importance of both retrieval and TabICL components. Furthermore, RAG+Phi3-GTL is among the top-performing of ``RAG + X'' baselines, even surpassing RAG+TabPFN-v2 in terms of median prediction performance across held-out datasets.
This highlights the potential of TabICL based on text representations, which can uncover novel and highly effective ICL algorithms by operating in a text space.
Besides, our approach achieves zero NMAE for a specific integer regression task without requiring explicit programming, whereas all numeric models, by default, produce float outputs.
Lastly, RAG+Phi3-GTL still lags behind well-tuned baseline models and TabPFN-v2 in overall performance.


\subsection{Ensemble Results}
\label{sec:exp_ensemble_res}

We further explore the potential of RAG+Phi3-GTL by investigating its contribution to ensemble diversity, as shown in Figure~\ref{fig:ensemble_res}.
When comparing RAG+Phi3-GTL with TabPFN-v2, LightGBM, and CatBoost, we observe that although RAG+Phi3-GTL underperforms the top-performing baselines overall, it exhibits unique strengths in certain scenarios.
Moreover, a comparison of Ensemble-All with TabPFN+LightGBM and TabPFN+CatBoost reveals that these unique strengths translate into ensemble diversity, enhancing the robustness of overall ensemble performance.
These findings highlight the potential of leveraging language as an alternative interface for tabular data learning, complementing existing tabular learning algorithms.

\subsection{Per-dataset Comparisons}
\label{sec:exp_per_ds_comp}

These findings further motivate us to conduct per-dataset comparisons to identify datasets where RAG+Phi3-GTL excels and those where it still underperforms.
Figure~\ref{fig:per_dataset_comp} summarizes these results.
We observe that on approximately 17\%-20\% of datasets, RAG+Phi3-GTL outperforms the state-of-the-art TabICL model, TabPFN-v2, as well as the classic tree-based model, CatBoost, which has been carefully tuned for each dataset.
Furthermore, on over 80\% of datasets, the performance of RAG+Phi3-GTL falls within a small gap of these two competitive baselines.
These results indicate that RAG+Phi3-GTL is already a strong prediction model for most tabular datasets, suggesting that we could not only engage with a well-prepared LLM conversationally but also leverage it to understand tabular data, provide accurate predictions, and offer potential explanations.


\paragraph{Analysis of Failure Cases}
Per-dataset comparison results also prompt an investigation into why the performance of RAG+Phi3-GTL falls short in certain cases. Case studies detailed in Appendix~\ref{app:case_study} provide insights into these failure scenarios.
In summary, most failure cases are attributed to the limitations of the default retrieval policy, which struggles to extract effective in-context instances due to specific data characteristics (e.g., datasets R-25, R-33, and R-27). In these cases, we find that slight adjustments to the retrieval strategy—such as applying alternative numerical normalization methods for feature similarity calculations or leveraging prior knowledge to define instance similarities—can lead to significant performance improvements for RAG+Phi3-GTL.
These findings suggest that a non-parametric, default retrieval mechanism may be insufficient in certain scenarios. Practitioners could potentially achieve better performance by employing "retrieval engineering" (as discussed in Section~\ref{sec:method}).
Additionally, some failure cases, such as dataset C-17, reveal limitations in Phi3-GTL’s ability to perform effective TabICL on specific data distributions, where TabPFN-v2 significantly outperforms. We hypothesize that this gap arises from the limited coverage of data patterns during Phi3-GTL’s post-training phase, which utilized approximately 300 real-world datasets from~\citeauthor{wen2024GTL}. In contrast, TabPFN-v2 likely benefits from pre-training on a much broader family of synthesized datasets.


\subsection{Decision Boundary Analysis}
\label{sec:exp_dec_bound}

Figure~\ref{fig:decision_boundary} compares the decision boundaries of various models across four groups of synthetic instances.
We observe that RAG+Phi3-GTL produces a distinctive, non-smooth decision boundary, which is entirely different from models relying on numeric representations. This boundary reflects case-by-case generalization from known training instances to unseen regions, leaving more uncertain areas when data samples are sparse.

In terms of shape, the decision boundary of RAG+Phi3-GTL bears some resemblance to that of Nearest Neighbors. However, LLM-based TabICL generalizes far beyond a simple rule-based average of neighboring training instances. We hypothesize that this unique behavior arises from the text-based representation of tabular data and the ICL capability of LLMs.
These findings also highlight opportunities for further improving LLM-based TabICL. Specifically, to encourage smoother decision boundaries when sufficient training data is available, one approach could involve generating large-scale synthetic data and fine-tuning LLMs to emulate such behaviors.



% \subsection{Case Analysis}
% \label{sec:exp_case_study}

% Winning Case analysis against TabPFN v2:
% regression-cat-medium-0-analcatdata_supreme

% Failure Case analysis:
% lack of learning certain feature distributions
% fail to select the best in-context examples
% \subsection{Experiment settints}

% \textbf{Data preparation.} For evaluating the performance of tree-based models versus large language models (LLMs) and deep learning approaches on tabular data tasks, we prepare datasets drawn from three research branches—GTL~\cite{wen2024GTL}, TabR~\cite{gorishniy2024TabR}, and Tree~\cite{grinsztajn2022tree_gt_tab_nn}, encompassing 32 regression and 23 classification tasks. This diverse dataset collection enables robust and transparent comparisons across model classes. For GTL, we filtered 50 test datasets from an initial pool of 350, focusing on feature validation to avoid data leakage and ensure accurate classification of features. Only datasets meeting size and integrity criteria were retained, ensuring robust evaluations across diverse tabular tasks. In TabR, we reused GTL’s pretrained checkpoints, eliminating redundancy by de-duplicating datasets from GTL’s collection. Large datasets were downsampled to 100K samples to ensure computational efficiency without sacrificing representativeness, as detailed in Appendix~\ref{}. For Tree, datasets with numerical and categorical features were curated with a focus on eliminating overlap with GTL and TabR datasets. Multiple versions or splits were consolidated to a single representative version, ensuring a fair and unbiased comparison across model classes, and preserving dataset diversity.

% \textbf{Baselines.}  We provide below a list of our used baselines. For most baselines, a hyperparameter search space is carefully prepared by the literature and we just follow this space setting and perform hyperparameter optimization on this then ensemble the results to boost the performance.

% \begin{itemize}
%     \item \textbf{LR} (Linear/Logistic Regression): A simple, linear baseline for tabular tasks that struggles with nonlinearity, limiting its performance on more complex datasets.
%     \item \textbf{KNN} (K-Nearest-Neighbor): A non-parametric model that makes predictions based on nearby samples. We use the same context samples as retrieved by TabRAG, providing a simple baseline for context ensemble
%     \item \textbf{GBDT} (XGBoost~\cite{chen2016XGBoost}, LightGBM~\cite{ke2017LightGBM}, CatBoost~\cite{prokhorenkova2018catboost}): Gradient boosting frameworks that build sequential decision tree models to optimize performance, particularly excelling in large, imbalanced, or missing data. Extensive hyperparameter tuning is applied for optimal performance.
%     \item \textbf{MLP} (Multi-layer Perceptron): A basic neural network used to evaluate the effectiveness of deep learning on tabular data. We applied a range of hyperparameters and ensemble strategies to maximize its performance.
%     \item \textbf{SAINT}~\cite{somepalli2021SAINT}: A novel architecture that uses self-attention and contrastive learning to capture feature relationships within rows of tabular data. It enhances performance by focusing on important features and pre-training on large tabular datasets.
%     \item \textbf{FT-Transformer}~\cite{gorishniy2021revisit_tab_dnn}: Adapts transformer architecture for tabular data by tokenizing both categorical and numerical features. It uses learned embeddings and positional encodings to process features, enabling the transformer to handle tabular structures effectively.
%     \item \textbf{TabPFN}~\cite{hollmann2022TabPFN}: A probabilistic model that uses in-context learning, pretrained on a wide variety of small tabular datasets to learn common feature interactions, allowing predictions without explicit retraining. It struggles with large datasets and regression tasks.
%     \item \textbf{TabR}~\cite{gorishniy2024TabR}: A retrieval-based foundation model that encodes query datasets and searches for similar datasets in latent space, enabling it to generalize to new tasks by leveraging historical data. Its predictions benefit from learned embeddings and in-context retrieval.
% \end{itemize}

\section{Conclusion}

%In this paper, w
We propose a new PEFT method called DiffoRA, which enables efficient and adaptive LLM fine-tuning based on LoRA. 
Instead of adjusting every interior rank, 
%of the decomposition matrices 
%of all modules, 
we argue that adopting LoRA module-wisely is sufficient. 
To achieve this, we construct a DAM to select the modules that are most suitable and essential to fine-tune. We theoretically analyze how the DAM impacts the convergence rate and generalization capability.
%of the pre-trained model. 
Furthermore, we adopt continuous relaxation and discretization to establish DAM.
%for each task. 
To alleviate the issue of discretization discrepancy, we utilize the weight-sharing strategy for optimization. 
%We fully implement our method and t
The experimental results demonstrate that our DiffoRA works consistently better than the baselines across all benchmarks. 

\IEEEtriggercmd{\balance}
\IEEEtriggeratref{12}

\bibliographystyle{IEEEtran}
\bibliography{IEEEabrv,trefs}

\end{document}
