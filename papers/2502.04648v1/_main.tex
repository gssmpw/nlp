\documentclass[10pt, conference, letterpaper]{IEEEtran}
\pdfoutput=1 
\IEEEoverridecommandlockouts
% The preceding line is only needed to identify funding in the first footnote. If that is unneeded, please comment it out.
\usepackage{cite}%[nocompress]{cite}
\usepackage{amsmath,amssymb,amsfonts}
\usepackage{comment}
\usepackage{algorithmic}
\usepackage{booktabs}
\usepackage{graphicx}
\usepackage{textcomp}
\usepackage{xcolor}
\usepackage[pscoord]{eso-pic}
\usepackage{balance}
\usepackage{fancyvrb}

\usepackage{algorithm2e}
\RestyleAlgo{ruled}
\SetKwComment{Comment}{/* }{ */}


\def\BibTeX{{\rm B\kern-.05em{\sc i\kern-.025em b}\kern-.08em
    T\kern-.1667em\lower.7ex\hbox{E}\kern-.125emX}}
    
\newcommand{\placetextbox}[3]{% \placetextbox{<horizontal pos>}{<vertical pos>}{<stuff>}
  \setbox0=\hbox{#3}% Put <stuff> in a box
  \AddToShipoutPictureFG*{% Add <stuff> to current page foreground
    \put(\LenToUnit{#1\paperwidth},\LenToUnit{#2\paperheight}){\vtop{{\null}\makebox[0pt][c]{#3}}}%
  }%
}%
\def\BibTeX{{\rm B\kern-.05em{\sc i\kern-.025em b}\kern-.08em
    T\kern-.1667em\lower.7ex\hbox{E}\kern-.125emX}}
\def\BibTeX{{\rm B\kern-.05em{\sc i\kern-.025em b}\kern-.08em
    T\kern-.1667em\lower.7ex\hbox{E}\kern-.125emX}}

\placetextbox{0.5}{0.96}{\texttt{\textcolor{red}{\textbf{Accepted for presentation at ISQED'25 - DOI/link to IEEE Xplore will be updated}}}}

% \usepackage{minted}
% \usepackage[finalizecache,cachedir=.]{minted} 

\usepackage[frozencache,cachedir=.]{minted} 

\usepackage{listings}

\usepackage{pifont}
\usepackage{framed}
\usepackage{soul}
\usepackage{booktabs}  
\usepackage{multirow}
\usepackage{acronym}
\acrodef{IP}[IP]{intellectual property block}
\acrodef{SoC}[SoC]{System-on-Chip}
\acrodef{IC}[IC]{integrated circuit}
\acrodef{eFPGA}[eFPGA]{embedded field programmable gate array}
\acrodef{RTL}[RTL]{register-transfer level}
\acrodef{CPS}{Cyber-Physical System}
\acrodef{IoT}{Internet of Things}
\acrodef{CAD}{Computer-Aided Design}
\acrodef{EDA}{Electronic Design Automation}
\acrodef{HPC}{High-Performance Computing}
\acrodef{DL}{deep learning}
\acrodef{ML}{machine learning}
\acrodef{NLP}{natural language processing}
\acrodef{IC}{Integrated Circuit}
\acrodef{CWE}[CWE]{Common Weakness Enumeration}
\acrodef{CVE}[CVE]{Common Vulnerabilities and Exposures}
\acrodef{LLM}[LLM]{large language model}
\acrodef{NMT}[NMT]{neural machine translation}
\acrodef{IP}[IP]{hardware intellectual property block}
\acrodef{HDL}[HDL]{hardware description language}
\acrodef{RTL}[RTL]{register-transfer level}
\acrodef{SDL}[SDL]{security development lifecycle}
\acrodef{FSM}[FSM]{finite state machine}
\acrodef{AST}[AST]{abstract syntax tree}
\acrodef{SoC}[SoC]{system-on-chip}
\acrodef{SA-EDI}{Security Annotation for Electronic Design Integration standard}
\usepackage{array}
\newcolumntype{L}[1]{>{\raggedright\let\newline\\\arraybackslash\hspace{0pt}}m{#1}}
\newcolumntype{C}[1]{>{\centering\let\newline\\\arraybackslash\hspace{0pt}}m{#1}}
\newcolumntype{R}[1]{>{\raggedleft\let\newline\\\arraybackslash\hspace{0pt}}m{#1}}
\usepackage{url}
\usepackage[caption=false,font=footnotesize,labelformat=simple]{subfig} 
\renewcommand\thesubfigure{(\alph{subfigure})}
\newcommand{\todoblock}[1]{\noindent\definecolor{shadecolor}{RGB}{252, 217, 217}\colorbox{shadecolor}{\parbox{\columnwidth}{{#1}}}}
\newcommand{\todoblockblue}[1]{\noindent\definecolor{shadecolor}{RGB}{214, 254, 255}\colorbox{shadecolor}{\parbox{\columnwidth}{{#1}}}}
\newcommand{\todoblockgreen}[1]{\noindent\definecolor{shadecolor}{RGB}{175, 255, 191}\colorbox{shadecolor}{\parbox{\columnwidth}{{#1}}}}
\newcommand{\findcite}[1]{\sethlcolor{yellow}\hl{[#1]}}
\usepackage[bookmarks=false,hidelinks]{hyperref}
\def\sectionautorefname{Section}
\def\subsectionautorefname{Section}
\def\subsubsectionautorefname{Section}
\def\algorithmautorefname{Algorithm}
\def\figureautorefname{Fig.}
\def\subfigureautorefname{Fig.}
\def\tableautorefname{Table}
\newcommand{\hlr}[1]{{\color{red}{\hl{#1}}}}
\newcommand{\bt}[1]{{\color{blue}{#1}}}
\newcommand{\rt}[1]{{\color{red}{#1}}}
\newcommand{\pt}[1]{{\color{purple}{#1}}}
\newcommand{\cmark}{\ding{51}}%
\newcommand{\xmark}{\ding{55}}%
\newcommand{\sol}{MoP~}
\newcommand{\ts}[1]{\textsuperscript{#1}}
    
\begin{document}
\bstctlcite{IEEEexample:BSTcontrol}



\title{%
    Toward Automated Potential Primary Asset \\Identification in Verilog Designs%
}

\author{%

    \IEEEauthorblockN{Subroto Kumer Deb Nath and Benjamin Tan}
    \IEEEauthorblockA{\textit{Department of Electrical and Software Engineering} \\
    \textit{Schulich School of Engineering} \\
    \textit{University of Calgary}\\
    \{subroto.nath,benjamin.tan1\}@ucalgary.ca}
    \thanks{
    
    This research is supported in part by Alberta Innovates, and the Natural Sciences and Engineering Research Council of Canada (NSERC) [RGPIN-2022-03027]. Cette recherche a été financée en partie par le Conseil de recherches en sciences naturelles et en génie du Canada (CRSNG).
    This research work is partly supported by a gift from Intel Corporation. This work does not in any way constitute an Intel endorsement of a product/supplier. 

    \rt{\textcopyright 2025 IEEE. Personal use of this material is permitted. Permission from IEEE must be obtained for all other uses, in any current or future media, including reprinting/republishing this material for advertising or promotional purposes, creating new collective works, for resale or redistribution to servers or lists, or reuse of any copyrighted component of this work in other works
    }}
}


\IEEEtitleabstractindextext{
\begin{abstract}
With greater design complexity, the challenge to anticipate and mitigate security issues provides more responsibility for the designer. 
As hardware provides the foundation of a secure system, we need tools and techniques that support engineers to improve trust and help them address security concerns. 
Knowing the security assets in a design is fundamental to downstream security analyses, such as threat modeling, weakness identification, and verification. 
This paper proposes an automated approach for the initial identification of potential security assets in a Verilog design.
Taking inspiration from manual asset identification methodologies, we analyze open-source hardware designs in three IP families and identify patterns and commonalities likely to indicate structural assets. 
Through iterative refinement, we provide a potential set of primary security assets and thus help to reduce the manual search space. 
\end{abstract}
\begin{IEEEkeywords}
    Asset Identification, Verilog, Automation, Hardware Security
\end{IEEEkeywords}
}

\maketitle
\IEEEdisplaynontitleabstractindextext



\begin{figure}
    \centering
    \begin{tikzpicture}[font=\footnotesize]
        \node (img) {\includegraphics[width=0.7\columnwidth]{figpaper/nfe_vs_fvd_vs_ep_ffs_teaser.pdf}};
            \node[anchor=north west, xshift=25pt, yshift=-5pt] at (img.north west) {
                \begin{tabular}{ll}
                \scriptsize
                    \textcolor[HTML]{A0A0A0}{\rule{6pt}{6pt}} &Rolling Diffusion \cite{ruhe2024rollingdiffusionmodels} \\
                    \textcolor[HTML]{e9cbc4}{\rule{6pt}{6pt}} &Diffusion Forcing \cite{chen2024diffusionforcing} \\
                    \textcolor[HTML]{F4A700}{\rule{6pt}{6pt}} &MaskFlow (\textit{Ours})
                \end{tabular}
            };
    \end{tikzpicture}
    \vspace{-7pt}
    \caption{\textbf{Our method (MaskFlow) improves video quality compared to baselines while simultaneously requiring fewer function evaluations (NFE)} when generating videos $2\times$, $5\times$, and $10\times$ longer than the training window.
}
    \label{fig:teaser}
    \vspace{-10pt}
\end{figure}

\section{Introduction}

Due to the high computational demands of both training and sampling processes, long video generation remains a challenging task in computer vision. Many recent state-of-the-art video generation approaches train on fixed sequence lengths \cite{blattmann2023stable,blattmann2023align_videoldm,ho2022video} and thus struggle to scale to longer sampling horizons. Many use cases not only require long video generation, but also require the ability to generate videos with varying length. A common way to address this is by adopting an autoregressive diffusion approach similar to LLMs \cite{gao2024vid}, where videos are generated frame by frame. This has other downsides, since it requires traversing the entire denoising chain for every frame individually, which is computationally expensive. Since autoregressive models condition the generative process recursively on previously generated frames, error accumulation, specifically when rolling out to videos longer than the training videos, is another challenge.
\par
Several recent works \cite{ruhe2024rollingdiffusionmodels, chen2024diffusionforcing} have attempted to unify the flexibility of autoregressive generation approaches with the advantages of full sequence generation. These approaches are built on the intuition that the data corruption process in diffusion models can serve as an intermediary for injecting temporal inductive bias. Progressively increasing noise schedules \cite{xie2024progressive,ruhe2024rollingdiffusionmodels} are an example of a sampling schedule enabled by this paradigm. These works impose monotonically increasing noise schedules w.r.t. frame position in the window during training, limiting their flexibility in interpolating between fully autoregressive, frame-by-frame generation and full-sequence generation. This is alleviated in \cite{chen2024diffusionforcing}, where independent, uniformly sampled noise levels are applied to frames during training, and the diffusion model is trained to denoise arbitrary sequences of noisy frames. All of these works use continuous representations.
\par
We transfer this idea to a discrete token space for two main reasons: First, it allows us to use a masking-based data corruption process, which enables confidence-based heuristic sampling that drastically speeds up the generative process. This becomes especially relevant when considering frame-by-frame autoregressive generation. Second, it allows us to use discrete flow matching dynamics, which provide a more flexible design space and the ability to further increase our sampling speed. Specifically, we adopt a \emph{frame-level masking} scheme in training (versus a \emph{constant-level masking} baseline, see Figure~\ref{fig:training}), which allows us to condition on an arbitrary number of previously generated frames while still being consistent with the training task. This makes our method inherently versatile, allowing us to generate videos using both full-sequence and autoregressive frame-by-frame generation, and use different sampling modes. We show that confidence-based masked generative model (MGM) style sampling is uniquely suited to this setting, generating high-quality results with a low number of function evaluations (NFE), and does not degrade quality compared to diffusion-like flow matching (FM)-style sampling that uses larger NFE. 
Combining frame-level masking during training with MGM-style sampling enables highly efficient long-horizon rollouts of our video generation models beyond $10 \times$ training frame lengths without degradation. We also demonstrate that this sampling method can be applied in a timestep-\emph{independent} setting that omits explicit timestep conditioning, even when models were trained in a timestep-dependent manner, which further underlines the flexibility of our approach. In summary, our contributions are the following:

\begin{itemize}
    \item To the best of our knowledge, we are the first to unify the paradigms of discrete representations in video flow matching with rolling out generative models to generate arbitrary-length videos. 
    \item We introduce MaskFlow, a frame-level masking approach that supports highly flexible sampling methods in a single unified model architecture.
    \item We demonstrate that MaskFlow with MGM-style sampling generates long videos faster while simultaneously preserving high visual quality (as shown in Figure~\ref{fig:teaser}).
    \item Additionally, we demonstrate an additional increase in quality when using full autoregressive generation or partial context guidance combined with MaskFlow for very long sampling horizons.
    \item We show that we can apply MaskFlow to both timestep-dependent and timestep-independent model backbones without re-training.
\end{itemize}

\begin{figure}
    \centering
    \includegraphics[width=0.75\linewidth]{figpaper/training.pdf}
    \caption{\textbf{MaskFlow Training:} For each video, Baseline training applies a single masking ratios to all frames, whereas our method samples masking ratios independently for each frame.}
    \vspace{-10pt}
    \label{fig:training}
\end{figure}

















\section{Background and Related Work\label{sec:background}}

\begin{figure*}[t]
\centering
\subfloat[Crypto]{\label{fig:Crypto-Graph}\includegraphics[width=0.33\textwidth]{fig/kw_crypto_rank-crop.pdf}}\hfill
\subfloat[Interface-GPIO]{\label{fig:GPIO-Graph}\includegraphics[width=0.33\textwidth]{fig/kw_gpio-crop.pdf}}\hfill
\subfloat[Interface-Peripheral]{\label{fig:Peripheral-Graph}\includegraphics[width=0.33\textwidth]{fig/kw_periph-crop.pdf}}
\caption{The number of occurrences for our identified partial keywords for three IP families.\label{fig:explanations_of_general_image}
}
\end{figure*}


An asset is any physical or logical component of \textit{value} and is essential for the proper functioning or security of the system\cite{holdings2009arm}.
The elements in an \ac{IP} that process, control, and store important values and interact with other \acp{IP} in an \ac{SoC} and communicate with the external peripherals are considered as \textit{primary} assets. 
Secondary assets are mostly internal design components of an \ac{IP} that help to propagate and handle the primary assets throughout the entire \ac{IP}.
It is helpful to know the assets in the system for several purposes, such as formulating security properties or identifying potential attack points (e.g., as in Accellera's \ac{SA-EDI} standard~\cite{accellera}). 
Recently, an IEEE P3164 white paper proposed a Conceptual and Structural Asset (CSA) methodology to help manually identify primary assets~\cite{ieee_p3164_working_group_asset_2024}, especially considering security objectives of \textbf{confidentiality}, \textbf{integrity}, and \textbf{availability} (the ``CIA triad'')~\cite{noauthor_what_nodate}, as well as the risk for ``undermined expected behavior'' in normal operation. 
Additionally, initiatives like MITRE's \acp{CWE}~\cite{mitreCommonWeakness} offer various examples of hardware weakness for identifying and mitigating vulnerabilities in hardware systems. 
This standard and methodologies, along with \ac{CWE} examples, give us valuable insights into current security challenges and how to potentially avoid them. 

Even so, as observed by the authors in~\cite{10140100}, engineers face challenges in identifying assets in the initial stages of hardware verification. 
Engineers must assess asset weaknesses, including proper initialization, information flow, and access controls. This process demands deep knowledge of security assets~\cite{ray_system--chip_2018}. However, there is a lack of tools to aid and automate this task, making it even more challenging for engineers.
Prior work attempted to perform generalized security analysis of \ac{HDL} code~\cite{Ahmad_2022} using syntactical patterns but found that more information from a designer (such as assets) was needed to address false positives. 
The authors of~\cite{polian_introduction_2017} described two situations where an element or the \ac{IP} itself as a whole can be an asset.
The authors of~\cite{meza_security_2023} demonstrated the security properties verification method with the help of security assets, where identifying hardware assets is the most crucial (manual) part. 
The authors of~\cite{farzana_saif_2021, Ayalasomayajula_Automatic_2024} proposed an automatic \textit{secondary} asset detection algorithm, that assumes that \textit{manual} \textbf{primary} security asset identification was performed. 
To the best of our knowledge, no prior work currently exists for \textbf{automated \underline{primary} asset identification} in Verilog source code, so our work complements prior works by offering an automated approach to identify potential primary assets. 

\section{Proposed Approach\label{sec:proposed}}

\subsection{Insights from Existing Designs\label{ssec:IED}}
We first discuss some insights from analyzing some existing open-source designs. 
We investigated 22 open-source cryptography IPs, 13 Peripheral Interface IPs, and 12 GPIO IPs (a subset of the family types that are defined in~\cite{accellera}) to gain insights into their design characteristics and commonalities, with a particular focus on 
recurring textual features that can be leveraged to develop an automated potential primary asset detection tool. 
These designs were collected from several sources\footnote{%
To support the community's efforts in this work, we make the details of our identified asset list and the associated IPs available here: \url{https://github.com/CalgaryISH/Asset_Dataset_using_PKG}
}, including GitHub, OpenCores, and OpenTitan, some of which have been used in prior work (e.g.,~\cite{Ahmad_2022}).
We manually examined design documentation and source code to get an intuitive sense of commonalities across IPs in each family. 

We applied the manual CSA method~\cite{ieee_p3164_working_group_asset_2024} to identify potential security assets. We also observed that most open-source designs involved ``sensible'' names for elements in the design (e.g., signals, sub-modules), and that frequently occurring signal/variable names in a given IP family provide indicators of ``important'' elements that point to assets. 

From this manual analysis of different \ac{IP} families and their source code, we curated a list of ``partial keywords'' for each \ac{IP} family that was associated with security-relevant Inputs, Outputs, and Reg/Wire/Logic(Net) signals, and we excluded ``typical'' names or tokens that were usually irrelevant to security, to increase the robustness of ``important RTL elements'' detection in the matching stage of our tool. 
This extends the similar concept of ``keyword-matching'' described in prior work~\cite{Ahmad_2022}.

For example, ``en'' represents the ``Enable'' partial keyword group ($PKG_{enable}$), which can be usable to detect signals named ``write\_en'', ``write\_enable'', ``wen'', ``gpio\_oen'', ``gpio\_out\_ena'' etc.
We sometimes added ``partial keywords'' that were especially meaningful for a given IP family even though they were not the most frequently appearing based on our manual code analysis. 
For instance, we added a partial keyword ``text'' in the list for Crypto to detect signals like ``plain\_text'', ``text\_in'', and ``text\_out'', which are important but not very common. 
\autoref{fig:explanations_of_general_image} illustrates the frequency with which the elements of our list of ``partial keywords'' appear throughout all files in an IP class. 
We also considered the following factors in the curation of partial keywords.

\subsubsection{Location and Frequency}
The location of the signals is crucial to understanding their nature. Potential important signal names often appear in logical/conditional expressions, assignments in sequential blocks, and combinational blocks. 
The frequency depends on the file size, but frequent signal names should be considered for further processing. These can be potential candidates if they frequently appear in these aforementioned locations.


\subsubsection{Keywords and Tokens of the Language}
Every language has reserved keywords, tokens, and constant names, which are special identifiers reserved to define the language constructs (e.g., \textit{wire}). We ignored everything that completely matched with these special ``words''.


\subsubsection{Types and Signal Width} We considered Input, Output, and internal Net-type signals in this study. Also, for signal width, we considered three categories: 1-bit, 2-bit to 8-bit, and larger signals up to 256-bit. A partial keyword group with a similar signal type(Input, Output, Net) and width is more desirable as it reduces the rule formulation complexities for asset classification.


\subsubsection{Overlap in Names} Due to naming conventions and functionalities of the designs, we observed some partial and full overlap in the signal names for different projects in the same IP class. As we discussed before, we used this overlapping portion of the name to create a partial keyword group like ($PKG_{enable}$). However, signal type and width should be taken into consideration as well.


\subsubsection{Commonalities in Roles} In some cases, we could not find any similarities in signal names. In that case, we created a partial keyword group according to the signal type, width, and role in the design. In this case, we used our design knowledge and ideas from the behavioral patterns. A partial keyword group for the status signal can be a good example. We have mentioned the most common status signals in \autoref{sssec:status}.


\subsection{Behavioral Patterns\label{ssec:BP}}

As simple name matching is insufficient to identify potential assets, we also manually examined how the potential candidate signals identified ``behave'' and ``function'' inside a design (e.g., if they often appear in \texttt{if-else}, \texttt{always} or \texttt{assign} statements, what are the ``width'' of the signals, in which operations they are associated to) to classify a set of signals' behavior patterns. 
Our approach does not solely depend on the keywords. 
Keywords help to identify potential candidate signals for the behavioral pattern detection stage. The behavioral pattern, including the attributes, functionalities, and location of a signal, indicates the overall structure, which is crucial to detecting structural assets~\cite{ieee_p3164_working_group_asset_2024}.

We observed four common signal behavioral patterns in the open-source hardware designs. 
Based on our manual analysis, attributes of the signals, and how they function in a design, we developed an algorithm denoted by \autoref{alg:one} to detect all four types of signals. 
\autoref{fig:code-example} provides an example of Verilog code that we use next as a running example to explain behavioral patterns. 
Each type of signal behavior can be described as relevant to confidentiality, availability, and/or integrity security properties. 

\subsubsection{Control Signals} are typically single-bit wide input signals or nets/variables (assigned or instantiated with input port in the module) that appear inside the conditional expression of \texttt{if-else} blocks, \texttt{case} blocks, and \texttt{ternary operation} statements. 
They are responsible for enabling/disabling functionality, controlling data flows and value assignment, and are often used to clear or load information from or into a memory register. Control signals are responsible for managing and controlling one or more blocking or non-blocking assignment statements inside \texttt{if-else} block, \texttt{case} block, and \texttt{ternary operation}. 
A control signal of a module can be connected with a status signal from a different module through instantiation when an interdependent sequential process occurs between two modules.
The \textit{Availability} security objective is commonly relevant to the Control signals. 
In the example (\autoref{fig:code-example}), \texttt{load} is a 1-bit input signal and controls the 128-bit data loading to \texttt{data\_in\_reg} register on line number 15. If the \texttt{load} becomes 0, no data will be loaded to the \texttt{data\_in\_reg} register. 


\subsubsection{Configuration Signals} are typically 2-bit to a few bits wide (in most cases, up to 8-bit) input signals or nets/variables (assigned or instantiated with input port in the module) that appear mostly in \texttt{case} expression, \texttt{ternary operation}, and conditional expression in multi-statements containing \texttt{if-else} blocks. 
These signals configure the operational flow of a module, data read and write direction, data splitting and loading into multiple memory registers or clear from the memory registers, select multiplexer's outputs, and control state transitions. 
\textit{Availability} and \textit{Integrity} security objectives are usually associated with the Configuration signals.
In the example, the 2-bit \texttt{bank\_selector} is a Configuration signal. According to code line 18, \texttt{bank\_selector} works as a Multiplexer output selector in the \texttt{case} expression for loading the split data into four memory banks.


\begin{figure}
\centering
\begin{minipage}{0.9\columnwidth}
\begin{minted}[breaklines=true,fontsize=\scriptsize, linenos=true]{verilog}
module data_splitter (
    input clk,
    input load,
    input [1:0] bank_selector,
    input [127:0] data,
    output reg [31:0] bank0,
    output reg [31:0] bank1,
    output reg [31:0] bank2,
    output reg [31:0] bank3,
    output reg done
);
  reg [127:0] data_in_reg;
  reg done0, done1, done2, done3;
  always @(posedge clk) begin
    if (load) data_in_reg <= data;
  end
  always @(data_in_reg or bank_selector) begin
    case (bank_selector)
      2'b00: begin
        bank0 <= data_in_reg[31:0];
        done0 <= 1'b1;
      end
      2'b01: begin
        bank1 <= data_in_reg[63:32];
        done1 <= 1'b1;
      end
      2'b10: begin
        bank2 <= data_in_reg[95:64];
        done2 <= 1'b1;
      end
      2'b11: begin
        bank3 <= data_in_reg[127:96];
        done3 <= 1'b1;
      end
      default: begin
      end
    endcase
  end
  always @(posedge clk) begin
    if (done0 && done1 && done2 && done3) 
        done <= 1'b1;
    else done <= 1'b0;
  end
endmodule
\end{minted}
\end{minipage}

\caption{A simple 128-bit to four banks of 32-bit data splitter code example.}
\label{fig:code-example}

\end{figure}

\begin{algorithm}[t!]
\footnotesize

\caption{Algorithm to detect different behavioral patterns in signals}\label{alg:one}

\KwData{RTL Code Written in Verilog or SystemVerilog, input\_ports[\ ], output\_ports[\ ], net\_variables[\ ];}
\KwResult{Control\_Signals[\ ], Configuration\_Signals[\ ], Status\_Signals[\ ], Data\_Signals[\ ];}
\vspace{1mm}
\For{$line\leftarrow 1$ \KwTo max\_line\_number of RTL\_Code} {

\If{$line\ contains\ a\ conditional\ expression$} {
\tcc{$x$ is the signal name inside the conditional expression}
\If{Width of $x ==\ 1bit$}{

\If{$x \in input\_ports[\ ]\ ||\ net\_variables[\ ] $} {append $x$ to Control\_Signals[\ ];}




}
\ElseIf{Width of $x \geq 2bit\ \&\&$ The conditional block contains multiple statements }{

\If{$x \in input\_ports[\ ]\ ||\ net\_variables[\ ] $} {append $x$ to Configuration\_Signals[\ ];}



}

}

\ElseIf{$line\ contains\ a\ blocking\ or\ non-blocking\ assignment$}{
\tcc{$l$ and $r$ are the signal names that appear on the left-hand side and right-hand side of an assignment statement, respectively}
\If{Width of $l ==\ 1bit\ \&\&\ l \in output\_ports[\ ] $}{ 
{append $l$ to Status\_Signals[\ ];}
}
\ElseIf{Width of $l \geq 2bit\ \&\&\ l \in output\_ports[\ ]$}{append $l$ to Data\_Signals[\ ];}
\ElseIf{Width of $r \geq 2bit\ \&\&\ r \in input\_ports[\ ]$}{append $r$ to Data\_Signals[\ ];}

}
\Else{do nothing;} 
}

\end{algorithm}



\subsubsection{Status Signals\label{sssec:status}} are typically single bit-width output signals or nets/variables (assigned or instantiated with output port in the module) that appear on the left-hand side of blocking/non-blocking assignments, statements under conditional code segments (e.g., \texttt{if-else} blocks, \texttt{case} blocks, and \texttt{ternary operation}). 
Usually, a status signal lets other modules know the status of a process or operation from the module to which it belongs. 
Common Status signals include \texttt{finish}, \texttt{done}, \texttt{ready}, \texttt{success}, \texttt{alert}, and \texttt{error}. 
If a status signal is connected with a control signal of another module, the \textit{Availability} objective is relevant; otherwise, \textit{Integrity} security objective is commonly relevant to the Status signals.
In the example, \texttt{done} is a Status signal. From code lines 40 to 42, the \texttt{if-else} block assigns the value of \texttt{done} signal based on \texttt{done0}, \texttt{done1}, \texttt{done2}, and \texttt{done3} signals. 

\subsubsection{Data Signals} can be inputs or outputs with a multi-bit width for assigning storage addresses, memory registers, information for processing, and processed information in a module.
Data signals propagate data and are processed in multiple modules in an \ac{IP}.
There is another kind of Data signal that does not get changed or processed during an operation but is involved in security-critical operations like encryption and decryption. ``Seed'' and ``Key'' are typical examples of this Data signal. 
Data signals appear in both blocking and non-blocking assignment-type statements in a module.
Data signals containing critical information are directly linked to \textit{Confidentiality}.
In our example, the 128-bit \texttt{data} and 32-bit \texttt{bank0}, \texttt{bank1}, \texttt{bank2}, and \texttt{bank3} are responsible for storing and carrying information; these all Data-type behavioral pattern.
The \texttt{data} may contain sensitive information like \textit{key} for a key-expansion operation in an encryption or decryption IP, or it may contain plain user information that needs to be protected equally. 
In line 15 of the \autoref{fig:code-example}, \texttt{data} appears on the right-hand side of the non-blocking assignment.
Conversely, all four banks appear on the left-hand side of the non-blocking assignment on lines 20, 24, 28, and 32, respectively. 
Such information, like which side of the assignment the signal appears, can be useful for automated asset identification. 

\begin{figure*}[t]
    \centering
    \includegraphics[width=\linewidth]{fig/overview-fig-crop.pdf}
    \caption{Overall view of our proposed automatic potential asset detection algorithm.}
    \label{fig:overview-figure}
\end{figure*}


\subsection{Automation}
In this section, we build on previously discussed observations and explain our proposed five-stage process for automated potential primary asset identification, as illustrated in \autoref{fig:overview-figure}. 

\subsubsection{Extraction Stage}
This stage uses RTL source files in Verilog or SystemVerilog. The tool iterates through all the available files in an \ac{IP}'s directory, extracting all the Input and Output Ports and Wire/Reg/Logic Nets. The tool stores these extracted ``RTL Elements'' in lists for the next stage. 

\subsubsection{Matching Stage}
In this stage, the tool finds partial matches between the RTL Elements and the IP-family-specific partial keyword lists to detect important signals that have the most potential to be assets. 
This stage outputs the subset of elements that are likely to be important from a security-perspective. 
Further pruning in the next stages narrows the potential candidate asset list. 
After this stage, the tool prepares a list of important\_rtl\_elements[ ] (reduced version of rtl\_elements[ ]) for the next stage.

\subsubsection{Width and Behavioral Pattern Detection Stage}
The matching step can sometimes result in 80\% of the RTL elements matching due to naming and multiple matches with partial keywords. 
For example, a signal \texttt{key\_rounding\_enable} in a Cryptographic \ac{IP} has three partial matches with keywords \texttt{en}, \texttt{round}, and \texttt{key}. 
The tool will detect \texttt{key\_rounding\_enable} for three partial keywords as a candidate for Important RTL Elements. 
However, not all of these should be considered structural assets (as explained in the IEEE P3164 white paper~\cite{ieee_p3164_working_group_asset_2024}). 
For this particular example, if the signal's width is single-bit and it is used to control any functionality or operation in the design (i.e., it has a ``control signal'' behavioral pattern), it can be a potential asset. 
Again, if the signal does not have the control pattern, just the partial keyword matching, it might not be a potential asset.
Therefore, we add another step to detect the signal width and behavioral patterns, as we previously described in \autoref{ssec:BP}. This will help the tool create lists of control\_signal[ ], config\_signal[ ], status\_signal[ ], and data\_signal[ ]. 

\subsubsection{IP-Specific Asset Classification and Filtering Stage}
Based on our manual analysis of the open-source designs, we designed a set of IP-family-specific classification rules, which narrow down the potential asset list. For example, for the crypto family, we consider a signal with the partial keyword \texttt{key} as the encryption/decryption key if it is a vectored signal (in most cases, the width starts from 64-bit) and has a ``Data Signal'' related behavioral pattern and the stored value needed for any encrypt/decrypt operation. 
In contrast, in the GPIO IP family, the partial keyword \texttt{data} helps to identify \textit{rdata}, \textit{wdata}, and just \textit{port\_data} or \textit{pad\_data} sometimes. The width for the GPIO port varies from 8-bit to the maximum bit at which the system operates (for instance, 32-bit for an ARM Cortex-M0 Microcontroller).
These all have \texttt{data} related behavioral patterns and store important values that are read/written to a register. 
In both cases, \texttt{key} and \texttt{data} can store potentially secret information that must be protected. Still, the detection and classification methods differ based on partial keyword group, width, signal types, and signal behavioral patterns. 
The filtered list is the set of candidate potential assets. 

\subsubsection{Refinement Stage\label{sssec:RS}}
Here, the tool identifies the ``root'' (original sources) of a candidate asset from the Input and Output ports of the TOP module. Any candidate potential asset that is related to the I/O Ports of the TOP module(for a complete IP) or in an important module (where the TOP module is not present) can be considered as a Potential Primary Asset.
At the beginning of this stage, the tool checks a candidate potential asset for its type (Input, Output, Net), behavioral patterns, and module information. Several situations can occur, as follows. 

\paragraph*{Case 1} If the candidate belongs to the Input or Output port of the TOP module, the tool appends it to the potential primary asset list.

\paragraph*{Case 2} If it does not belong to the TOP module, the tool finds the candidate's connection with the Input and output ports of the TOP module by traversing through the instantiations throughout the IP. Then, the tool appends the connected port from the TOP module to the potential primary asset list. If the tool does not find any interconnection between the candidate and the I/O ports of the TOP module, (1) the tool ignores the candidate (maybe a Secondary asset) when the candidate-containing module is instantiated inside the TOP module directly or indirectly through another submodule, or (2) the tool considers the candidate a primary asset when the candidate-containing module is not instantiated inside the TOP module directly or indirectly through another submodule.

\paragraph*{Case 3} If the candidate is a Net-type signal, then the tool identifies the assignments and connections through the instantiations related to the module to which the candidate belongs. The tool detects the candidate's connection(if any) with I/O ports. If the tool can detect any connection between the candidates and I/O ports throughout the IP, the tool repeats \textit{Case 1} and \textit{Case 2}.

After completing the process for each candidate, the tool removes duplicate potential primary assets that can be added multiple times due to the relationship with multiple candidates from multiple modules.
After this stage, the tool lists the final potential primary assets for an \ac{IP} as output.




\section{Experiments}
In this section, we conduct a systematic evaluation of state-of-the-art models on \dataset. We first detail the experiment setup in Section~\ref{sec:exp_steup}. Then in Section~\ref{sec:exp_quantitative}, we report the quantitative results and provide valuable insights derived from our analysis.

\subsection{Experiment Setup}
\label{sec:exp_steup}
\paragraph{Evaluation Models.} 
We select top-performing LMMs for comprehensive CoT evaluation. We test earlier models such as LLaVA-OneVision (7B, 72B)~\cite{li2024llava-ov}, Qwen2-VL (7B, 72B)~\cite{Qwen2-VL}, MiniCPM-V-2.6~\cite{yao2024minicpm}, and InternVL2.5 (8B)~\cite{chen2024expanding}, which are not trained for the reasoning capability. We also include GPT-4o~\cite{openai2024gpt4o} as a strong baseline model.
Besides, we test recent models targeting reasoning, including LLaVA-CoT (11B)~\cite{xu2024llavacot}, Mulberry (8B)~\cite{yao2024mulberry}, InternVL2.5-MPO (8B, 78B)~\cite{wang2024mpo}.
Finally, we evaluate LMMs with reflection capabilities, including both closed-source models like Kimi k1.5~\cite{team2025kimi} and open-source implementations such as QVQ-72B~\cite{qvq-72b-preview} and Virgo-72B~\cite{du2025virgo}.

Note that we sample 150 questions from \dataset to evaluate Kimi k1.5, due to the access limitations. The sample comprises 115 reasoning and 35 perception questions. 

\begin{table*}[!t]
\centering
\caption{\textbf{Evaluation Results of Three Aspects of CoT in Each Category in \dataset.} Best performance is marked in \colorbox{backred!60}{red}.  $*$ denotes unreliable results due to the refusal to answer directly.}
\vspace{-3pt}
\renewcommand\tabcolsep{2.0pt}
\renewcommand\arraystretch{1.25}
\resizebox{1.0\linewidth}{!}{
\begin{tabular}{l|ccc|ccc|ccc|cc|cc|cc}
\toprule
\multirow{2}*{\makecell*[l]{\large Model}} & \multicolumn{3}{c|}{\makecell*[c]{General Scenes}} & \multicolumn{3}{c|}{Space-Time} & \multicolumn{3}{c|}{OCR} & \multicolumn{2}{c|}{Math} & \multicolumn{2}{c|}{Science} & \multicolumn{2}{c}{Logic} \\
& Quality & Robustness & Efficiency & Quality & Robustness & Efficiency & Quality & Robustness & Efficiency & Quality & Efficiency & Quality & Efficiency & Quality & Efficiency \\
\midrule
Mulberry & 33.9 & \colorbox{backred!60}{4.3} & 76.0 & 18.2 & 1.0 & 38.4 & 26.7 & \colorbox{backred!60}{6.6} & 26.4 & 29.1 & 87.9 & 29.1 & 91.9 & 13.9 & \colorbox{backred!60}{99.1} \\
LLaVA-OV-7B & 41.8 & -6.2 & 81.8 & 23.8 & -6.7 & 24.8 & 44.1 & -0.2 & 42.7 & 27.4 & 97.3 & 28.5 & 95.1 & 12.2 & 98.0 \\
LLaVA-CoT & 38.2 & -2.2 & 89.9 & 33.6 & 2.8 & 68.9 & 37.4 & 0.0 & 77.8 & 35.3 & 91.0 & 36.4 & 93.4 & 14.9 & 97.1 \\
LLaVA-OV-72B & 41.8 & -2.3 & \colorbox{backred!60}{98.9} & 29.0 & -0.9 & 43.6 & 40.8 & -1.7 & 84.2 & 38.4 & 98.7 & 35.4 & 95.7 & 18.4 & 82.3 \\
MiniCPM-V-2.6 & 47.1 & 3.2 & 87.7 & 49.3 & -14.4 & 71.1 & 63.7 & -4.9 & 62.0 & 32.9 & 95.2 & 29.5 & 90.4 & 16.9 & 93.7 \\
InternVL2.5-8B & 43.8 & -6.4 & 87.1 & 50.7 & -8.9 & \colorbox{backred!60}{99.1} & 44.7 & -4.1 & \colorbox{backred!60}{98.9} & 40.9 & 98.0 & 40.8 & 97.1 & 19.5 & 96.8 \\
Qwen2-VL-7B & 46.7 & -3.4 & 79.3 & 51.7 & -11.8 & 73.0 & 65.9 & 0.9 & 86.2 & 34.0 & 97.9 & 34.6 & 95.0 & 18.4 & 76.7 \\
InternVL2.5-8B-MPO & 47.2 & 2.9 & 94.3 & 51.8 & -0.2 & 74.6 & 59.6 & -1.0 & 81.5 & 37.4 & 93.4 & 39.0 & 95.6 & 20.9 & 79.9 \\
InternVL2.5-78B-MPO & 47.9 & 0.0 & 89.3 & 55.5 & -2.3 & 91.9 & 72.2 & 2.2 & 73.1 & 50.6 & 95.1 & 48.5 & 97.7 & 24.2 & 87.2 \\
Qwen2-VL-72B & 51.9 & -2.9 & 88.9 & 59.7 & -5.3 & 86.7 & 77.6 & 2.5 & 81.7 & 49.6 & 97.8 & 53.6 & \colorbox{backred!60}{99.0} & 40.0 & 88.0 \\
Virgo-72B & 60.5 & 0.5 & 91.0 & 59.6 & -3.8 & 86.0 & 79.9 & -1.0 & 82.1 & 59.6 & 90.3 & 55.5 & 98.7 & 39.6 & 88.2 \\
QVQ-72B & \colorbox{backred!60}{62.6} & -1.5 & 86.9 & 58.2 & -2.5 & 57.7 & 76.9 & -1.4 & 52.6 & \colorbox{backred!60}{61.4} & 92.7 & 57.7 & 95.9 & \colorbox{backred!60}{44.6} & 94.9 \\
GPT4o & 62.3 & -1.7 & 96.2 & \colorbox{backred!60}{66.3} & \colorbox{backred!60}{5.5} & 64.7 & \colorbox{backred!60}{83.3} & -1.0 & 82.1 & 60.8 & \colorbox{backred!60}{98.8} & \colorbox{backred!60}{64.1} & 97.4 & 27.2 & 92.0 \\
\bottomrule
\end{tabular}
}
\label{table:category_result}
% \vspace{-0.3cm}
\end{table*}


\paragraph{Implementation Details.}
We define the CoT prompt as: \textit{Please generate a step-by-step answer, include all your intermediate reasoning process, and provide the final answer at the end.} and the direct prompt as: \textit{Please directly provide the final answer without any other output.}
We only calculate recall of image observation and logical inference on questions where key inference conclusion or image observation exists.
We employ GPT-4o mini for the direct evaluation and GPT-4o for all other criteria. For hyperparameters, we follow the settings in VLMEvalKit~\cite{duan2024vlmevalkit}. 

\subsection{Quantitative Results}
\label{sec:exp_quantitative}
We conduct extensive experiments on various LMMs with our proposed CoT evaluation suite. 
The main results are presented in Table~\ref{table:main_result} and Table~\ref{table:category_result}. We begin by analyzing the overall performance and then highlight key findings.
\paragraph{Overall Results.}
In Table~\ref{table:main_result}, we present
the overall performance of three CoT evaluation perspectives with specific metrics. 
To provide a comprehensive understanding, we report precision, recall, and relevance for both logical inference and image caption steps. For robustness, we provide the direct evaluation result on the perception and reasoning tasks, with either CoT or direct prompt. We employ the average value of the stability and efficacy as the final robustness metric. Notably, we define the reflection quality as 100 on models incapable of reflection.

For CoT quality, Kimi k1.5 achieves the highest F1 score. Open-source models with larger sizes consistently demonstrate better performance, highlighting the scalability of LMMs. Notably, Qwen2-VL-72B outperforms all other open-source models without reflection, even surpassing InternVL2.5-78B-MPO, which is specifically enhanced for reasoning. Analysis reveals that GPT-4o achieves superior performance across all recall metrics, while Kimi k1.5 demonstrates the highest scores in precision evaluations.
For CoT robustness, Mulberry obtains the highest average score. However, when we look into its output, we find it still generates lengthy rationales despite receiving a direct prompt. Even worse, the direct prompt seems to be an out-of-distribution input for Mulberry, 
frequently leading to nonsensical outputs. Further analysis of other models’ predictions reveals that LLaVA-CoT, Virgo, QVQ, and Kimi k1.5 similarly neglect the direct prompt, instead generating extended rationales before answering. Consequently, their robustness scores may be misleading. Once again, GPT-4o achieves the highest robustness score. Among open-source models, only InternVL2.5-MPO, in both its 8B and 78B variants, attains a positive robustness score.
Finally, for CoT efficiency, InternVL2.5-8B obtains the maximum relevance of 98.4\%, suggesting its consistent focus on questions.

Now, we summarize our key observations as follows:
\paragraph{\textit{Models with reflection largely benefit CoT quality.}}
As shown in Table~\ref{table:main_result}, the F1 scores of the two models with reflection capability most closely approach GPT-4o. After specifically fine-tuning for the reasoning capabilities from Qwen2-VL-72B, QVQ surpasses its base model by 5.8\%. Notably, although QVQ generates longer CoT sequences than Qwen2-VL-72B, QVQ's precision still exceeds Qwen2-VL-72B by 2.9\%, indicating superior accuracy in each reasoning step. Kimi k1.5 also surpasses the previous state-of-the-art model GPT-4o, obtaining the highest CoT quality.


\paragraph{\textit{Long CoT does not necessarily cover key steps.}} 
Despite high precision in long CoT models, the informativeness of each step is not guaranteed. We observe that the recall trend among GPT-4o, QVQ, and Virgo does not align with their CoT Rea. performance (i.e., their final answer accuracy on the reasoning tasks under the CoT prompt). Specifically, while both Virgo and QVQ outperform GPT-4o in direct evaluation, they lag behind in recall. This suggests that long CoT models sometimes reach correct answers while skipping intermediate steps, which contradicts the principle of stepwise reasoning and warrants further investigation.

\paragraph{\textit{CoT impairs perception task performance in most models.}}% 比较stability
Surprisingly, most models exhibit negative stability scores, indicating that CoT interferes with perception tasks. The most significant degradation occurs in InternVL2.5-8B, where performance drops by 6.8\%. This reveals inconsistency and potential overthinking in current models, presenting a significant barrier to adopting CoT as the default answering strategy. Among models that provide direct answers, only LLaVA-OV-72B and InternVL2.5-8B-MPO achieve a modest positive score of 0.3\%.

\paragraph{\textit{More parameters enable models to grasp reasoning better.}} 
We find that models with larger parameter counts tend to achieve higher efficacy scores. This pattern is evident across LLaVA-OV, InternVL2.5-MPO, and Qwen2-VL. For instance, while Qwen2-VL-7B shows a 4.8\% decrease in performance when applying CoT to reasoning tasks, its larger counterpart, Qwen2-VL-72B, demonstrates a 2.4\% improvement. This discrepancy suggests that models with more parameters could better grasp the reasoning ability under the same training paradigm. 


\paragraph{\textit{Long CoT models may be more susceptible to distraction.}} 
Long CoT models may demonstrate lower relevance scores compared to other models. They frequently generate content unrelated to solving the given question, corresponding to their relatively low recall scores compared to direct evaluation, like QVQ. Although a few models with short CoT, like Mulberry and LLaVA-OV-7B, also obtain a low relevance rate, we find that it is because these models may keep repeating words when dealing with specific type of questions, resulting in irrelevant judgment. The fine-grained metric reveals that models tend to lose focus when describing images, often producing exhaustive captions regardless of their relevance to the question. From Table~\ref{table:category_result}, we find that this phenomenon prevails in general scenes, space-time, and OCR tasks. This behavior can significantly slow inference by generating substantial irrelevant content. Teaching long CoT models to focus on question-critical elements represents a promising direction for future research.


\paragraph{\textit{Reflection often fails to help.}} 
While reflection is a key feature of long CoT models for answer verification, both QVQ and Virgo achieve reflection quality scores of only about 60\%, indicating that approximately 40\% of reflection attempts fail to contribute meaningfully to answer accuracy. Even for the closed-source model Kimi k1.5, over 25\% reflection steps are also invalid. This substantial failure rate compromises efficiency by potentially introducing unnecessary or distracting steps before reaching correct solutions. Future research should explore methods to reduce these ineffective reflections to improve both efficiency and quality.

\begin{figure}[t]
\begin{center}
\vspace{0.2cm}
\centerline{\includegraphics[width=0.8\columnwidth]{fig/ref_error_pie.pdf}}
\caption{\textbf{Distribution of Reflection Error Types.} We identify four types of error: ineffective reflection, incompleteness, repetition, and interference.}
\label{fig:ref_error_distribution}
\end{center}
\vspace{-0.6cm}
\end{figure}

\subsection{Error Analysis}
\label{sec:exp_analysis}
In this section, we analyze error patterns in the LMM reflection process. An effective reflection should either correct previous mistakes or validate correct conclusions through new insights. We examined 200 model predictions from QVQ and identified four distinct error types that hinder productive reflection. These patterns are illustrated in Fig.~\ref{fig:ref_error_example} and their distribution is shown in Fig.~\ref{fig:ref_error_distribution}.

The four major error types are:

\begin{itemize}
    \item \textbf{Ineffective Reflection.} The model arrives at an incorrect conclusion and, upon reflecting, continues to make incorrect adjustments. This is the most common error type and is also witnessed most frequently.
    \item \textbf{Incompleteness.} The model proposes new analytical approaches but does not execute them, only stopping at the initial thought. The reflection slows down the inference process without bringing any gain.
    \item \textbf{Repetition.} The model restates previous content or methods without introducing new insights, leading to inefficient reasoning.
    \item \textbf{Interference.} The model initially reaches a correct conclusion but, through reflection, introduces errors.
\end{itemize}

Understanding and mitigating these errors is crucial for improving the reliability of LMM reflection mechanisms. The analysis provides the opportunity to focus on solving specific error types to enhance the overall reflection quality.


\section{Conclusion}
We introduce a novel approach, \algo, to reduce human feedback requirements in preference-based reinforcement learning by leveraging vision-language models. While VLMs encode rich world knowledge, their direct application as reward models is hindered by alignment issues and noisy predictions. To address this, we develop a synergistic framework where limited human feedback is used to adapt VLMs, improving their reliability in preference labeling. Further, we incorporate a selective sampling strategy to mitigate noise and prioritize informative human annotations.

Our experiments demonstrate that this method significantly improves feedback efficiency, achieving comparable or superior task performance with up to 50\% fewer human annotations. Moreover, we show that an adapted VLM can generalize across similar tasks, further reducing the need for new human feedback by 75\%. These results highlight the potential of integrating VLMs into preference-based RL, offering a scalable solution to reducing human supervision while maintaining high task success rates. 

\section*{Impact Statement}
This work advances embodied AI by significantly reducing the human feedback required for training agents. This reduction is particularly valuable in robotic applications where obtaining human demonstrations and feedback is challenging or impractical, such as assistive robotic arms for individuals with mobility impairments. By minimizing the feedback requirements, our approach enables users to more efficiently customize and teach new skills to robotic agents based on their specific needs and preferences. The broader impact of this work extends to healthcare, assistive technology, and human-robot interaction. One possible risk is that the bias from human feedback can propagate to the VLM and subsequently to the policy. This can be mitigated by personalization of agents in case of household application or standardization of feedback for industrial applications. 

\IEEEtriggercmd{\balance}
\IEEEtriggeratref{12}

\bibliographystyle{IEEEtran}
\bibliography{IEEEabrv,trefs}

\end{document}
