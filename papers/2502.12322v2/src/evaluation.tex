
\section{Evaluation}
\label{sec:evaluation}


This section describes how we have evaluated our framework against three popular online video games. We first describe the criteria used to select the three games.  We then analyse the performance of the games when the cheats are activated and when they are deactivated. Finally, we discuss the impact, if any, our cheats have on the game's performance and show how they have remained undetected by the anti-cheat systems of the evaluated games.



The three games used for evaluating the VIC framework are TF2, BlackSquad and Fortnite. The games were selected based on the following criteria, summarised on table \ref{table:summaryGameSelectionCriteria}:  
\begin{itemize}
    \item They are first or third person free-to-play action games with an active player base. TF2 and BlackSquad have an average of 95,346 active players and 726 respectively, in the last three months at the time of writing this thesis\footnote{Data taken from \url{https://steamdb.info}, a popular measurement site for the Steam gaming platform}. Fortnite reports 253,998,243 average active players based on data from activeplayer.io \cite{fortniteStats}.
    \item They employ one or more anti-cheat systems. TF2 uses \emph{VAC} \cite{VAC}, Fortnite uses \emph{Byfron} \cite{byfron}, \emph{Easy Anti-Cheat} \cite{EAC} and \emph{BattleEye} \cite{battleEye} and BlackSquad uses BattleEye. VAC stands for Valve anti-cheat and it is a user level anti-cheat, first introduced in 2002 as part of the Steam platform for Counter Strike. Easy Anti-Cheat is a kernel level anti-cheat, it first started as a third party anti-cheat although it was later on acquired by Epic games in 2018 and currently protects a large number of games. BattleEye is a third party kernel level anti-cheat initially released in 2004. Byfron is a user level anti-tamper released in 2022 which makes it the newest of the four. It does not only try to detect hooks on game functions but it introduces obfuscation on the game binary and memory protection during gameplay.
    
    \item They are designed using different game engines, Source Engine, Unreal Engine 3 and Unreal Engine 5 respectively. Game engines provide important building blocks for creating a game, by testing the cheat methodology using different game engines, a better understanding of the performance and efficiency of the proposed attack can be achieved.
    \item They come from different eras but are still being played and frequently updated (at the time of writing this thesis). TF2 was first published on 10th October 2007 (latest update September 2022). BlackSquad and Fortnite represent more recent games as they were published 28th and 21st of July 2017 respectively (latest updates September 2022).
    \item The VM setup used could meet their hardware requirements for smooth gameplay. Smooth play for each game can have a different frame rate as games are designed in different ways based on the engine they use, their genre and finally graphics and network optimisations put in place. One example of genre design difference which impacts performance is the need to support fewer players than other games, TF2 supports up to 24 players but Fortnite supports 100 because it is a battle royal game. What is considered smooth play is a frame rate over 15 fps.
\end{itemize} 

\begin{table}[h]
\centering
\begin{tabular}{lllr}
\toprule
& Team Fortress 2 &  BlackSquad & Fortnite \\
\toprule
Genre &FPS & FPS &TPS \\
Average Player base &95,346 & 726  & 53,998,243 \\
User-level Anti-cheats    & VAC  &N/A &Byfron \\
Kernel-level Anti-cheats & N/A  & BattleEye & BattleEye \& EAC \\
Game Engine &Source Engine  &  Unreal Engine 3 & Unreal Engine 5 \\
Number of players in session  &24 &12 &100 \\           
Release year     &2007 & 2017 & 2017 \\    
\bottomrule
\end{tabular}
\caption{Summary of game selection criteria.}
\label{table:summaryGameSelectionCriteria}
\end{table}


\subsection{Setup specifications}\label{specs}

The experimenters were executed on an Intel i7 7th generation CPU 2.9GHz with 2 cores 2 threads accompanied by Intel's HD Iris 620 GPU and 16GB RAM (8GB allocated to the host and 8GB allocated to the guest). A standard optical USB mouse was used as mouse for the guest machine. The system was running Windows 10 Enterprise OS build 15063, QEMU 2.11.932.11.93 patched for introspection support and KVM Hypervisor version. Team Fortress 2 along with BlackSquad and Fortnite were running the latest version as of 27th September 2022 during the performance testing.

\subsection{Attack Validation} 
\label{attack-validation}
 
 Anti-cheats do not always react upon a cheat detection, they ban in waves. This technique helps the game developers to ban clients of popular cheat makers, as it gives time for the cheats to get shared via communities and create a big negative impact on the cheat developer's reputation \cite{ricochetWaves}. All the cheats designed for the purposes of this paper were tested on private servers when there was an option with the anti-cheat settings activated. In the case where private servers were not an option, the cheats were tested on public matches. Because of the unethical impact the cheats could have on other players' game sessions, the game was abandoned before our plays could affect the outcome of the game. For the purposes of testing the trigger-bots, they were mainly tested in private matches and the cross-hair target was off center when the match was public in order to miss the enemy avatars and avoid detection via server-side behavioural analysis (see \S\ref{sec:discussion} for a more detailed discussion on how to mitigate against this cheat detection techniques). At the time of writing, the cheats had at least an hour of play time each and a number of weeks have passed since testing the cheats live. After combining the ban status of all the accounts used during testing, with the feedback from the vendors through the responsible disclosure process \S\ref{subsec:ResponsibleDisclosure}, non of them have been banned which verifies the stealthiness of our methodology.



\subsection{Performance evaluation}
Frames per second (FPS) are used very often by PC players as a way to measure the performance of their hardware against the game and vice versa. It is considered the de facto way to test how smoothly the game runs on specific hardware. The higher the frame rate the smoother the scene transitions, thus better reaction times and overall better game experience. Having better reaction times also gives a competitive advantage over other players. This highlights the importance of FPS for both professional and hobbyist gamers. Figure \ref{fig:games-fps} demonstrate the impact our cheats have on the games' performance.

The FPS measurements were collected using FRAPS \cite{fraps} a Windows program for counting FPS, supporting Direct X and OpenGL, used also in the past in other works \cite{fraps::xueffects,fraps::abeysundaraperformance}. We carried out our analysis by monitoring the FPS count of three whole game sessions for each game repeated for every cheat scenario. The duration of a game session was selected by using the default game session time of each game, 600 seconds for Team Fortress 2 and 900 seconds for BlackSquad. In the case of Fortnite, 600 seconds was used as it was the best trade off between using the cheats for enough time to prove they are stable and being able to stay alive without killing people thus tampering with the session's outcome.

We start by measuring the FPS count without running any cheats to establish a benchmark. Then, the same steps were repeated for every scenario a) cheat-radar, b) wall-hack, c) trigger-bot with periodic reads and d) trigger-bot using memory events (only for TF2\footnote{The authors could not identify a similar address in the other two games responsible for reflecting the object under crosshair. This can occur because of different game design decisions.}). Each cheat scenario was monitored three times. From those three sessions we constructed an average scenario session by averaging the FPS count for each second. As a final step the average benchmark session was subtracted from the cheating sessions. The mean and quartiles of the difference of every cheat experiment are depicted in Figure \ref{fig:games-fps}. The y axis reflects the FPS difference with positive numbers representing how many frames the cheat was slower than the normal game operation and negative numbers how much faster. 

\begin{figure*}
    \centering
    \includegraphics[scale=0.4]{pics/games-performance-graph.pdf}
    \caption{Mean value of FPS difference and quartiles of game sessions under different cheat scenarios.}
    \label{fig:games-fps}
\end{figure*}

The performance of TF2 was measured at 60fps without any cheats activated. Cheat radar had an identical average performance with the benchmark session at 60fps followed by Wall-hack and Trigger-Bot with an average performance of 57fps. The lowest performance was reported by the page guard trigger-bot at 5fps.

The event implementation of the trigger-bot creates a performance reduction as it limits the FPS average output to 5 fps. This happens because the monitoring page frame triggers a few hundreds of events per second irrelevant to the address we are monitoring. There exist a correlation, on the examined game, between the number of memory events and the total number of players in the game session. When the number of players gets increased the number of memory events also increases. This creates a performance impact on the VM as it needs to process these events, even-though they are irrelevant. A way to overcome this limitation is by using a Sub-page write protection (SPP) event, supported by Intel 10th generation or later CPU's as described in \S28.3.4.2 of the Intel Developer's Manual \cite{IntelSPP}. This provides the ability to the user to monitor a sub range of addresses in a page frame thus eliminating the interception on page execute events irrelevant to the cheat. The use of different engines and the implementation differences did not provide us the opportunity to test the PE-Trigger-Bot on BlackSquad and Fortnite as the authors could not find a memory address reflecting the aim state of the local avatar.


Figure \ref{fig:games-fps} shows the performance comparison of VIC on BlackSquad. The game runs with an average frame rate of 60fps on our system. It is important to mention because of the use of the integrated graphics the game was set up to run on low graphics performance setting with a maximum frame rate of 60fps. The impact of the cheats on the game's performance is negligible. Our results report the lowest average frame rate to occur while running the Wall-hack cheat at 50fps only 10 frames below the average frame rate performance when having no cheats enabled. A number of outliers can be spotted across all cheat categories the lowest being at 25fps while running the trigger-bot. The reason for this is due to the graphics card's power. Passing a GPU to the guest VM will diminish these slight drops in performance and increase the average frame rate of the game. We did not test a GPU pass through setup because of hardware limitations of the equipment used for the experiments (e.g. the experiments were carried on a laptop). All cheats apart from trigger-bot with page guard events have negligible performance impact on the games tested. 




The third and final game we tested was Fortnite, because of the game's nature and requirements it was clocked at an average of 25fps on our system. Despite the relatively low fps, because of in game network optimizations it was running smoothly. Figure \ref{fig:games-fps} shows the performance comparison of our cheats on Fortnite. In contrast to Team Fortress 2, Fortnite and BlackSquad both employee Unreal Engine 5 and Unreal Engine 3 respectively. 

Our experiments show that the cheat radar introduced via \frameworkName had a negligible effect in the three games we tested (average of 3fps decrease). The wall-hacks and trigger-bots have an average performance loss of 10 and 8 fps respectively. We attribute this to two main facts. First, these games run very similar game engines, which probably means that external factors like a higher CPU load may also affect the game engines on a similar way. Second, both the wall-hack and trigger-bots were designed on a similar way, requiring more complex operations to calculate the overlay and the time to pull the trigger respectively compared to the cheat radar implementation. 

