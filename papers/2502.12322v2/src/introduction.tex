% ++++++++++++++++++++++++++++++++
% ++++++++++ Introduction ++++++++
% ++++++++++++++++++++++++++++++++
\section{Introduction}

Video game cheats are designed to give cheaters an unfair advantage over legitimate players in online games. 

To construct these, cheat developers employ a number of strategies. They manipulate the game's memory by reading and writing state variables which under normal circumstances are unavailable to them. They can also employ trampoline functions to redirect the game process execution and execute malicious code inside the game's memory \cite{FengStealthMeasurements}. Other techniques include monitoring network packets for information, and more recently computer vision was used to detect enemy avatars by reading the screen using a cheat known as PixelBot \cite{pixelbot}. 

Despite the use of sophisticated anti-cheat systems, cheat developers are constantly developing bypasses and stealthy cheating methods, resulting in a cat and mouse game where every new anti-cheat method is quickly bypassed by a new cheat. This has resulted in several layers of protection being added to modern video games to protect them from cheaters. Currently, the plethora of anti-cheat systems have extended their reach to kernel space as a response to defend against cheats which employee kernel drivers to access the game's process memory, as demonstrated by Joel Noguera \cite{unveilingUndergroundWorldAntiCheats}. This move might seem as the next natural step to take, although it comes with certain drawbacks: it can affect the operating system's performance, stability and security if not implemented correctly. One example of such failure is the Digital Rights Management System (DRM) that Sony BMG included within their CDs from 2005. This DRM was installing a rootkit on the user's OS in order to modify the operating system's behavior to stop CD copying  \cite{halderman2006SonyRootkit}.

% Mention here \frameworkName, 
This paper demonstrates a new vector of attack to enable video game cheats: Virtual machine Introspection Cheats (VIC). We show how direct memory and interrupt access to a guest OS can allow cheaters to bypass user-level and kernel-level anti-cheats. This is of particular relevance now that we have reached a point where broadband speeds and virtualization technologies are enabling a new way of accessing graphically advanced video games via streaming (as proven by services such as Xbox Cloud Gaming from Microsoft or GForce Now from Nvidia). We demonstrate our method with an implementation on three popular online games with millions of players worldwide: Team Fortress 2, BlackSquad and Fortnite. To the best of our knowledge, this is the first work that demonstrates the use of virtual machine introspection to execute video game cheats. In particular our contributions are:
\begin{itemize}
    \item We introduce a new method of cheating using a hypervisor with introspection enabled. We propose two approaches to achieve effective cheats by using a polling mechanism on the guest memory and by using page guard exceptions to assess the efficiency and applicability of this design paradigm.
    \item We use our method to implement three different types of cheats for three different games. With this, we demonstrate the effectiveness of our method to bypass both a user-level and a kernel level anti-cheats in current commercial video games.
    \item We discuss how our work could be used to help anti-cheat analysts and discuss the drawbacks involved when trying to defend against this type of cheats and mitigation's proposed in literature. 
    (\S\ref{sec:discussion}).
   
\end{itemize}

The rest of the paper is organized as follows.
We present the methodology used by \frameworkName in \S\ref{sec:vmiCheat}. 
We evaluate \frameworkName against three popular games and present our results in  \S\ref{sec:evaluation}. We discuss the implications and the limitations of our attack in \S\ref{sec:discussion} and, finally, summarize our conclusions in \S\ref{sec:conclusions}.
