\section{Related Work}
\label{section:relatedwork}

Decan et al. \cite{decan2019empirical} in their study on the empirical analysis of package dependency networks across seven packaging ecosystems (Cargo for Rust, CPAN for Perl, CRAN for R, npm for JavaScript, NuGet for the .NET platform, Packagist for PHP, and RubyGems for Ruby) identified common challenges in these ecosystems. They accessed the growth, changeability, reusability, and fragility of these ecosystems, revealing trends of network expansion, the centrality of a small number of packages in driving updates, and the prevalence of fragile packages with numerous transitive dependencies. Similarly to our approach, their study performed a dynamic analysis using the Gini index as a key metric to explore inequality and concentration within these networks. However, while their study spans a diverse range of ecosystems, differing in size, age, and policies, our present research is more focused, dealing solely with Maven's ecosystem.
Other researchers often use popularity metrics to sample datasets or investigate software properties. While some studies have described the popularity of software components in terms of social characteristics, others have described them in terms of technical aspects~\cite{Zerouali2014}. For instance, a study of GitHub developers conducted by Lee et al. \cite{lee2013github} demonstrated that very well-known developers, who are often referred to as "rock stars", have a greater influence on the projects their followers contribute to. In comparison to our study which explores the network evolution and growth in the Maven ecosystem, they conducted a dynamic analysis of how the actions and interactions of developers evolve.
Borges et al. \cite{borges2018s} conducted a survey involving 400 Stack Overflow users. The results from their poll showed that the users viewed GitHub metrics such as stars, forks, and watchers as highly valuable indicators of how popular a project is. Furthermore, the majority of the comments from OSS developers questioned by Bogart et al. \cite{bogart2016break} on why they chose the right dependencies for their software projects fell into groups pertaining, reputation and popularity in the community. 

Other researchers like Sajnani et al. \cite{sajnani2014popularity} who measured the popularity of 2,406 Maven components by analyzing how often they were used in 55,191 open-source Java projects, have also argued that usage of software components can be used as a measure of their popularity. Their interpretations work under the assumption that if a component is widely (re)used, then it is generally regarded as good.

The shortcomings of using social attributes of software components as a popularity metric are exacerbated by the fact that they cannot provide a complete picture of real usage, as they can be easily influenced by individual's preferences or trend \cite{papamichail2019measuring}. Research conducted in the past by Kitchenham et al. \cite{kitchenham1988evaluation}, Fenton et al. \cite{fenton1999critique}, and Vasa et al. \cite{vasa2007inevitable} has demonstrated how widely skewed software metrics are in general, making accurate interpretation with conventional descriptive statistical analysis challenging.