\section{CONCLUSION AND FUTURE WORK}

In this work, we addressed the limitations of existing RecLLMs in complex scenarios by introducing a novel Deliberative Recommendation task, which emphasizes explicit reasoning before predicting user feedback. 
To achieve Deliberative Recommendation, we proposed the Reason4Rec framework, which enables multi-step reasoning via three collaborative experts with three core reasoning capabilities: Preference Distillation, Preference Matching, and Feedback Prediction. 
We aligned the reasoning process with users' true preference by using verbalized user feedback, \ie reviews. 
Through extensive experiments on three real-world datasets, Reason4Rec demonstrated superior performance in both prediction accuracy and reasoning quality, highlighting the significance of slow thinking in recommendation tasks.



Despite its promising results, Reason4Rec is an initial attempt on Deliberative Recommendation, leaving many future directions. 
First, we only adopt user reviews as verbalized user feedback. Exploring multi-turn and diverse forms of verbalized user feedback, \eg conversations, may further enhance the reasoning process. 
Second, although Reason4Rec achieves comparable or even lower inference costs than Rec-SAVER and EXP3RT, there is still significant room to improve its efficiency. Future work could explore designing LLM acceleration algorithms tailored to Deliberative Recommendation. 
Third, it is promising to design an interactive learning paradigm between Reason4Rec and users. 