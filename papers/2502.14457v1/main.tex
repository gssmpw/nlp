%%%%%%%%%%%%%%%%%%%%%%%%%%%%%%%%%%%%%%%%%%%%%%%%%%%%%%%%%%%%%%%%%%%%%%%%%%%%%%%%
%2345678901234567890123456789012345678901234567890123456789012345678901234567890
%        1         2         3         4         5         6         7         8

\documentclass[letterpaper, 10 pt, conference]{ieeeconf}  
% Comment this line out if you need a4paper

%\documentclass[a4paper, 10pt, conference]{ieeeconf}      % Use this line for a4 paper

\let\labelindent\relax


\IEEEoverridecommandlockouts                              % This command is only needed if 
                                                          % you want to use the \thanks command

\overrideIEEEmargins                                      % Needed to meet printer requirements.

%In case you encounter the following error:
%Error 1010 The PDF file may be corrupt (unable to open PDF file) OR
%Error 1000 An error occurred while parsing a contents stream. Unable to analyze the PDF file.
%This is a known problem with pdfLaTeX conversion filter. The file cannot be opened with acrobat reader
%Please use one of the alternatives below to circumvent this error by uncommenting one or the other
%\pdfobjcompresslevel=0
%\pdfminorversion=4

% See the \addtolength command later in the file to balance the column lengths
% on the last page of the document

% The following packages can be found on http:\\www.ctan.org
\usepackage{graphics} % for pdf, bitmapped graphics files
\usepackage{epsfig} % for postscript graphics files
\usepackage{mathptmx} % assumes new font selection scheme installed
\usepackage{times} % assumes new font selection scheme installed
\usepackage{amsmath} % assumes amsmath package installed
\usepackage{amssymb}  % assumes amsmath package installed
\usepackage{amstext} % assumes amsmath package installed
\usepackage{dsfont}

\title{\LARGE \bf
Watch Less, Feel More: Sim-to-Real RL for Generalizable Articulated Object Manipulation via Motion Adaptation and Impedance Control
}


\author{Tan-Dzung Do$^{1,2}$, Nandiraju Gireesh$^{1,2}$, Jilong Wang$^2$, and He Wang$^{1,2,\dagger}$% <-this % stops a space
\thanks{$^{\dagger}$corresponding to \href{mailto:hewang@pku.edu.cn}{hewang@pku.edu.cn}}%
\thanks{$^{1}$CFCS, School of Computer Science, Peking University \quad
$^{2}$Galbot}%
}


% Math package
\usepackage{mathtools}
\usepackage{amsmath}
%\usepackage{amsthm}
\let\Bbbk\relax
\usepackage{amssymb}
\usepackage{amsfonts}
\usepackage{amsopn}
\usepackage{graphicx} % Necessary to use \scalebox
\usepackage{textcomp}
\usepackage{xfrac}
\usepackage{bbm}
% Figures
\usepackage{overpic}
\usepackage{subfig}
\usepackage{wrapfig}
\usepackage[ruled,vlined,linesnumbered]{algorithm2e}
\SetAlFnt{\small}
\SetAlCapFnt{\small}
\SetAlCapNameFnt{\small}
\SetAlCapHSkip{0pt}

\usepackage{multirow}

\usepackage{booktabs}
\usepackage{footnote}
\usepackage{paralist}
\usepackage{enumitem}
\usepackage{autobreak}
\usepackage{hyperref}
\usepackage{cleveref}
\usepackage{cite}
\usepackage{siunitx}


%\usepackage[T1]{fontenc}
%\usepackage{lmodern}

%%
%% color comments are made possilble below
%%
\usepackage{color}
\definecolor{green}{rgb}{0, 0.4, 0}
\definecolor{orange}{rgb}{0.8, 0.6, 0.2}
\definecolor{red}{rgb}{1.0, 0.0, 0.0}
\definecolor{teal}{rgb}{0.0, 0.4, 0.4}
\definecolor{purple}{rgb}{0.65,0,0.65}
\definecolor{saffron}{rgb}{0.95,0.75,0.2}
\definecolor{turquoise}{rgb}{0.0,0.5,0.5}
\definecolor{brown}{rgb}{0.5, 0.16, 0.16}


% Overpic + myFigurename
\usepackage{overpic}
\usepackage{currfile} % \currfiledir
\newcommand{\vertical}[1]{\rotatebox{90}{#1}}
\newcommand{\myfigurename}{\put(-3,0){\vertical{\todo{\currfiledir}}}}
\renewcommand{\myfigurename}{}
\newlength\savedwidth
\newcommand\whline[1]{\noalign{\global\savedwidth\arrayrulewidth
		\global\arrayrulewidth #1} %
	\hline
	\noalign{\global\arrayrulewidth\savedwidth}}

\newcommand{\rev}[1]{{\color{blue}#1}}

\newcommand{\fix}[1]{{\color{blue}#1}}
\newcommand{\tofix}[1]{{\color{red}[To fix: #1]}}
%\newcommand{\supl}[1]{{\color{black}\emph{#1}}}
\newcommand{\supl}[1]{{\color{black}#1}}

\newcommand{\kxc}[1]{{\color{red}[KX: #1]}}
\newcommand{\kx}[1]{{\color{red}[#1]}}

\definecolor{lightgray}{rgb}{0.6, 0.6, 0.6}
\newenvironment{draft}{\color{red}}{\color{black}}
\newcommand{\todo}[1]{{\textcolor{red}{#1}}}
\newcommand{\TODO}[1]{{\textcolor{red}{[TODO: #1]}}}
\newcommand{\xunyong}[1]{{\textcolor{blue}{[jiazhao: #1]}}}
\newcommand{\gireesh}[1]{{\textcolor{teal}{[gireesh: #1]}}}

\newcommand{\addcite}[1]{{\textcolor{red}{[cite]}}}


\newcommand{\remove}[1]{{\overline{blue}{#1}}}



\definecolor{revisedcolor}{RGB}{100,0,200}
\newcommand{\revised}[1]{{\textcolor{revisedcolor}{#1}}}

\newcommand{\st}[1]{{\color{red}\sout{#1}}}
\newcommand{\od}[1]{{\color{red}#1}}
\newcommand{\odcmt}[1]{{\color{red}COMMENT: #1}}
\newcommand{\stod}[2]{\st{#1}{\color{red}{#2}}}

\newcommand{\changed}[1]{{\color{cyan}#1}}


%%
\newcommand{\Fig}[1]{Figure~\ref{fig:#1}}
\newcommand{\Eq}[1]{Eq.~(\ref{eq:#1})}
\newcommand{\Eqs}[2]{Eq.~(\ref{eq:#1}--\ref{eq:#2})}
\newcommand{\eq}[1]{(\ref{eq:#1})}
\newcommand{\Sec}[1]{Section~\ref{sec:#1}}
\newcommand{\Algo}[1]{Algorithm~\ref{algo:#1}}
\newcommand{\AlgoL}[1]{Line~\ref{algol:#1}}
\newcommand{\Tab}[1]{Table~\ref{tab:#1}}

% Corrections style
\usepackage[normalem]{ulem}
\newcommand{\scratch}[1]{{\color{red}\sout{#1}}}

\newcommand{\textapprx}{\raisebox{0.1ex}{\texttildelow}}
\newcommand{\pcite}{\protect\cite}

% \newcommand{\num}{{\color{blue}[X]~}}
\newcommand{\perc}{{\color{blue}[X\%]~}}


%%
%% text layout (COMMENT IN CAMERA READY)
%%
% Insert whitespace at the end of a paragraph
\setlength{\parskip}{.2\baselineskip}%
% Don't intent by default on new paragraph
%\setlength{\parindent}{0pt}%
% Ugly new siggraph paragraphs
%\renewcommand{\paragraph}[1]{\textbf{#1}}

%\newcommand{\etal}{et al.}
\newcommand{\hidecomment}[1]{}
%\newcommand{\ie}{i.e.~\/}
%\newcommand{\argmin}{\arg\!\min}
\newcommand{\R}{\mathbb{R}}
\newcommand{\cM}{\mathcal{M}}
\newcommand{\cN}{\mathcal{N}}
\newcommand{\cF}{\mathcal{F}}
\newcommand{\cS}{\mathcal{S}}
\newcommand{\cP}{\mathcal{P}}
\newcommand{\Q}{\mathcal{Q}}
\newcommand{\cI}{\mathcal{I}}
\newcommand{\cD}{\mathcal{D}}
\newcommand{\cL}{\mathcal{L}}
\newcommand{\cB}{\mathcal{B}}
\newcommand{\cC}{\mathcal{C}}
\newcommand{\cG}{\mathcal{G}}
\newcommand{\cV}{\mathcal{V}}
\newcommand{\cA}{\mathcal{A}}

\newcommand{\bw}{\mathbf{w}}
\newcommand{\bR}{\mathbf{R}}
\newcommand{\bn}{\mathbf{n}}
\newcommand{\br}{\mathbf{r}}
\newcommand{\bo}{\mathbf{o}}
\newcommand{\bx}{\mathbf{x}}
\newcommand{\bc}{\mathbf{c}}
\newcommand{\be}{\mathbf{e}}
\newcommand{\bs}{\mathbf{s}}
\newcommand{\bv}{\mathbf{v}}
\newcommand{\bq}{\mathbf{q}}
\newcommand{\bp}{\mathbf{p}}
\newcommand{\bt}{\mathbf{t}}
\newcommand{\bz}{\mathbf{z}}
\newcommand{\vn}{\vec{n}}
\newcommand{\bepsilon}{{\bf \epsilon}}
\newcommand{\bT}{\mathbf{T}}
\newcommand{\vc}{\vec{c}}
\newcommand{\bM}{\mathbf{M}}
\newcommand{\tv}{\tilde{v}}
\newcommand{\tdC}{\tilde{C}}
\newcommand{\tdD}{\tilde{D}}
\newcommand{\tW}{\text{W}}
\newcommand{\te}{\text{e}}
\newcommand{\bzero}{\mathbf{0}}
\newcommand{\bone}{\mathbf{1}}
\newcommand{\tm}{\text{m}}
\newcommand{\tQ}{\tilde{Q}}
\newcommand{\tp}{\text{p}}
\newcommand{\too}{\text{o}}



\DeclareMathOperator*{\argmin}{arg\,min}
\DeclareMathOperator*{\argmax}{arg\,max}
\DeclareMathOperator{\sign}{sign}




\newcommand{\ps}{\mathcal{P}}
\newcommand{\nb}{\mathcal{N}}
\newcommand{\gt}{\mathcal{T}_\text{G}}
\newcommand{\gtx}{\mathcal{T}_\text{G}^x}
\newcommand{\gty}{\mathcal{T}_\text{G}^y}
\newcommand{\gtz}{\mathcal{T}_\text{G}^z}
\newcommand{\lt}{\mathcal{T}_\text{L}}
\newcommand{\xmin}{x_\text{min}}
\newcommand{\xmax}{x_\text{max}}
\newcommand{\ymin}{y_\text{min}}
\newcommand{\ymax}{y_\text{max}}
\newcommand{\zmin}{z_\text{min}}
\newcommand{\zmax}{z_\text{max}}

\newcommand{\best}[1]{{$\mathbf{\color{blue} #1}$}}
\newcommand{\sbest}[1]{{$\mathbf{\color{green} #1}$}}

\usepackage{xspace}
% Scene acronyms
\newcommand{\DSICL}{\texttt{ICL-NUIM}\xspace}
\newcommand{\DSTUM}{\texttt{TUM RGB-D}\xspace}
\newcommand{\DSETH}{\texttt{ETH3D}\xspace}
\newcommand{\DSETHCS}{\texttt{ETH3D-CS}\xspace}
\newcommand{\DSFCM}{\texttt{FastCaMo}\xspace}
\newcommand{\DSFCMS}{\texttt{FastCaMo-Synth}\xspace}
\newcommand{\DSFCMR}{\texttt{FastCaMo-Real}\xspace}
\newcommand{\DSFCML}{\texttt{FastCaMo-Large}\xspace}
\newcommand{\DSREPL}{\texttt{Replica}\xspace}
\newcommand{\DSSN}{\texttt{ScanNet}\xspace}

% Other algorithms acronyms
% Acronyms for algorithms 

\usepackage[table]{xcolor}
\begin{document}

% \let\labelindent\relax
\begin{figure*}[!t]
\centering 
\includegraphics[width=1\linewidth]{figure/teaser.png}
\caption{\projectname empowers users to easily define personal actions that they want to receive just-in-time intervention (JITI) from a smartwatch. The user journey is as follows: (1) Users determine one or more custom target actions. (2) They follow the instructions on the smartwatch to collect a small set of samples with the accelerometer sensor. (3) \projectname{} applies multiple data augmentation and data synthesis techniques to expand the training dataset, (4) \projectname{} adapts a pre-trained model through fine-tuning and personal customization. (5) \projectname{} leverages the custom model to provide a JITI system for real-time action recognition and intervention delivery.}
\label{fig:teaser}
\Description{}
\end{figure*}



\maketitle
\thispagestyle{empty}
\pagestyle{empty}

%%%%%%%%%%%%%%%%%%%%%%%%%%%%%%%%%%%%%%%%%%%%%%%%%%%%%%%%%%%%%%%%%%%%%%%%%%%%%%%%


\begin{abstract}  
Test time scaling is currently one of the most active research areas that shows promise after training time scaling has reached its limits.
Deep-thinking (DT) models are a class of recurrent models that can perform easy-to-hard generalization by assigning more compute to harder test samples.
However, due to their inability to determine the complexity of a test sample, DT models have to use a large amount of computation for both easy and hard test samples.
Excessive test time computation is wasteful and can cause the ``overthinking'' problem where more test time computation leads to worse results.
In this paper, we introduce a test time training method for determining the optimal amount of computation needed for each sample during test time.
We also propose Conv-LiGRU, a novel recurrent architecture for efficient and robust visual reasoning. 
Extensive experiments demonstrate that Conv-LiGRU is more stable than DT, effectively mitigates the ``overthinking'' phenomenon, and achieves superior accuracy.
\end{abstract}  

\section{Introduction}


\begin{figure}[t]
\centering
\includegraphics[width=0.6\columnwidth]{figures/evaluation_desiderata_V5.pdf}
\vspace{-0.5cm}
\caption{\systemName is a platform for conducting realistic evaluations of code LLMs, collecting human preferences of coding models with real users, real tasks, and in realistic environments, aimed at addressing the limitations of existing evaluations.
}
\label{fig:motivation}
\end{figure}

\begin{figure*}[t]
\centering
\includegraphics[width=\textwidth]{figures/system_design_v2.png}
\caption{We introduce \systemName, a VSCode extension to collect human preferences of code directly in a developer's IDE. \systemName enables developers to use code completions from various models. The system comprises a) the interface in the user's IDE which presents paired completions to users (left), b) a sampling strategy that picks model pairs to reduce latency (right, top), and c) a prompting scheme that allows diverse LLMs to perform code completions with high fidelity.
Users can select between the top completion (green box) using \texttt{tab} or the bottom completion (blue box) using \texttt{shift+tab}.}
\label{fig:overview}
\end{figure*}

As model capabilities improve, large language models (LLMs) are increasingly integrated into user environments and workflows.
For example, software developers code with AI in integrated developer environments (IDEs)~\citep{peng2023impact}, doctors rely on notes generated through ambient listening~\citep{oberst2024science}, and lawyers consider case evidence identified by electronic discovery systems~\citep{yang2024beyond}.
Increasing deployment of models in productivity tools demands evaluation that more closely reflects real-world circumstances~\citep{hutchinson2022evaluation, saxon2024benchmarks, kapoor2024ai}.
While newer benchmarks and live platforms incorporate human feedback to capture real-world usage, they almost exclusively focus on evaluating LLMs in chat conversations~\citep{zheng2023judging,dubois2023alpacafarm,chiang2024chatbot, kirk2024the}.
Model evaluation must move beyond chat-based interactions and into specialized user environments.



 

In this work, we focus on evaluating LLM-based coding assistants. 
Despite the popularity of these tools---millions of developers use Github Copilot~\citep{Copilot}---existing
evaluations of the coding capabilities of new models exhibit multiple limitations (Figure~\ref{fig:motivation}, bottom).
Traditional ML benchmarks evaluate LLM capabilities by measuring how well a model can complete static, interview-style coding tasks~\citep{chen2021evaluating,austin2021program,jain2024livecodebench, white2024livebench} and lack \emph{real users}. 
User studies recruit real users to evaluate the effectiveness of LLMs as coding assistants, but are often limited to simple programming tasks as opposed to \emph{real tasks}~\citep{vaithilingam2022expectation,ross2023programmer, mozannar2024realhumaneval}.
Recent efforts to collect human feedback such as Chatbot Arena~\citep{chiang2024chatbot} are still removed from a \emph{realistic environment}, resulting in users and data that deviate from typical software development processes.
We introduce \systemName to address these limitations (Figure~\ref{fig:motivation}, top), and we describe our three main contributions below.


\textbf{We deploy \systemName in-the-wild to collect human preferences on code.} 
\systemName is a Visual Studio Code extension, collecting preferences directly in a developer's IDE within their actual workflow (Figure~\ref{fig:overview}).
\systemName provides developers with code completions, akin to the type of support provided by Github Copilot~\citep{Copilot}. 
Over the past 3 months, \systemName has served over~\completions suggestions from 10 state-of-the-art LLMs, 
gathering \sampleCount~votes from \userCount~users.
To collect user preferences,
\systemName presents a novel interface that shows users paired code completions from two different LLMs, which are determined based on a sampling strategy that aims to 
mitigate latency while preserving coverage across model comparisons.
Additionally, we devise a prompting scheme that allows a diverse set of models to perform code completions with high fidelity.
See Section~\ref{sec:system} and Section~\ref{sec:deployment} for details about system design and deployment respectively.



\textbf{We construct a leaderboard of user preferences and find notable differences from existing static benchmarks and human preference leaderboards.}
In general, we observe that smaller models seem to overperform in static benchmarks compared to our leaderboard, while performance among larger models is mixed (Section~\ref{sec:leaderboard_calculation}).
We attribute these differences to the fact that \systemName is exposed to users and tasks that differ drastically from code evaluations in the past. 
Our data spans 103 programming languages and 24 natural languages as well as a variety of real-world applications and code structures, while static benchmarks tend to focus on a specific programming and natural language and task (e.g. coding competition problems).
Additionally, while all of \systemName interactions contain code contexts and the majority involve infilling tasks, a much smaller fraction of Chatbot Arena's coding tasks contain code context, with infilling tasks appearing even more rarely. 
We analyze our data in depth in Section~\ref{subsec:comparison}.



\textbf{We derive new insights into user preferences of code by analyzing \systemName's diverse and distinct data distribution.}
We compare user preferences across different stratifications of input data (e.g., common versus rare languages) and observe which affect observed preferences most (Section~\ref{sec:analysis}).
For example, while user preferences stay relatively consistent across various programming languages, they differ drastically between different task categories (e.g. frontend/backend versus algorithm design).
We also observe variations in user preference due to different features related to code structure 
(e.g., context length and completion patterns).
We open-source \systemName and release a curated subset of code contexts.
Altogether, our results highlight the necessity of model evaluation in realistic and domain-specific settings.






\section{Related Work}
\label{sec:related}
FL has emerged as a crucial learning scheme for distributed training that aims to preserve user data privacy. 
However, research has uncovered various privacy vulnerabilities in FL, particularly in the form of MIA and LDIA.
This section discusses relevant works that highlight these threats in both FL and FD settings, and contextualize our research within this landscape.

\BfPara{MIA and LDIA} 
Shokri \etal~\cite{shokri2017membership} pioneered MIA research by demonstrating how model output confidence scores could reveal training data membership. 
Nasr \etal~\cite{nasr2019comprehensive} extended this to FL, showing how both passive and active adversaries could exploit gradients and model updates.
LDIA represents another significant privacy threat in FL.
Gu \etal~\cite{gu2023ldia} introduced LDIA as a new attack vector where adversaries infer label distributions from model updates. Wainakh \etal~\cite{wainakh2021user} further explored user-level label leakage through gradient-based attacks in FL.
Recent works have exposed the vulnerability of FD to inference attacks.
Yang \etal~\cite{yang2022fd} proposed FD-Leaks for performing MIA in FD settings through logit analysis. Liu \etal~\cite{liu2023mia} and Wang \etal~\cite{wang2024graddiff} enhanced MIA using shadow models via respective approaches MIA-FedDL and GradDiff, though their assumptions were limited to homogeneous environments.

\iffalse
The study of MIA in ML models was pioneered by Shokri \etal~\cite{shokri2017membership}.
They demonstrated how adversaries could exploit model output confidence scores to infer whether a specific data point was used in training.
This seminal work laid the foundation for subsequent research into privacy vulnerabilities in various ML paradigms, including FL.
Building on this, Nasr \etal~\cite{nasr2019comprehensive} extended the concept to FL environments, introducing white-box MIAs.
Their privacy analysis demonstrated how both passive and active adversaries could exploit gradients and model updates to infer private information.
\BfPara{LDIA in FL} 
LDIA represents another significant privacy threat in FL.
Gu \etal~\cite{gu2023ldia} introduced LDIA as a new attack vector, where adversaries seek to infer the distribution of labels in clients' training data by analyzing model updates.
This work showed that even when individual data points are protected, the overall data distribution might still be exposed, leading to privacy concerns.
Wainakh \etal~\cite{wainakh2021user} further explored user-level label leakage by performing gradient-based attacks in FL.

\BfPara{LDIA and MIA in FD} 
% FD has been proposed as a lightweight alternative to traditional FL.
% It reduces communication overhead by exchanging logits or softmax values instead of model parameters.
Yang \etal~\cite{yang2022fd} proposed FD-Leaks, an attack designed to perform MIA in FD settings.
By analyzing logits, adversaries can infer membership information, potentially revealing sensitive training data.
Similarly, Liu \etal~\cite{liu2023mia} presented MIA-FedDL, which enhances MIA by using shadow models to infer membership information with higher accuracy in FD settings.
Despite their contributions, they made assumptions that are infeasible in heterogeneous environments.
\fi
% Despite their contributions, these works make several assumptions that may limit the feasibility of the proposed attacks in practice.
% For instance, both FD-Leaks and MIA-FedDL assume that the adversaries have access to all logits or can easily build shadow models that mimic the behavior of target models.
% Such assumptions are often unrealistic in heterogeneous environments.

\BfPara{Defenses and Countermeasures}
DPSGD~\cite{abadi2016deep} can be employed during the training phase to mitigate against privacy attacks to the client model. 
Additionally, specialized MIA defense methods such as SELENA~\cite{tang2022mitigating}, HAMP~\cite{chen2023overconfidence} and DMP\cite{shejwalkar2021membership} can be integrated into the training process.
Several studies have proposed enhanced FD frameworks with improved privacy protection mechanisms to reduce client privacy leakage.
%Several studies have proposed defenses to mitigate MIA and LDIA.
%Wang \etal~\cite{wang2024graddiff} proposed GradDiff, a gradient-based defense mechanism that employs differential comparison to detect and mitigate MIA in FD settings.
Zhu \etal~\cite{zhu2021data} investigated data-free knowledge distillation for heterogeneous federated learning.
They presented an approach that reduces the need for public datasets.
Chen \etal~\cite{chen2023best} proposed FedHKD, where clients share hyper-knowledge based on data representations from local datasets for federated distillation without requiring public datasets or models.

% Our work builds upon these foundations, specifically investigating the privacy vulnerabilities in PDA-FD frameworks. 
%Unlike previous studies that focused on traditional FL or specific attack scenarios in FD, our work aim to provide a comprehensive analysis of LDIA and MIA across multiple PDA-FD frameworks.



\section{Method}\label{sec:method}
\begin{figure}
    \centering
    \includegraphics[width=0.85\textwidth]{imgs/heatmap_acc.pdf}
    \caption{\textbf{Visualization of the proposed periodic Bayesian flow with mean parameter $\mu$ and accumulated accuracy parameter $c$ which corresponds to the entropy/uncertainty}. For $x = 0.3, \beta(1) = 1000$ and $\alpha_i$ defined in \cref{appd:bfn_cir}, this figure plots three colored stochastic parameter trajectories for receiver mean parameter $m$ and accumulated accuracy parameter $c$, superimposed on a log-scale heatmap of the Bayesian flow distribution $p_F(m|x,\senderacc)$ and $p_F(c|x,\senderacc)$. Note the \emph{non-monotonicity} and \emph{non-additive} property of $c$ which could inform the network the entropy of the mean parameter $m$ as a condition and the \emph{periodicity} of $m$. %\jj{Shrink the figures to save space}\hanlin{Do we need to make this figure one-column?}
    }
    \label{fig:vmbf_vis}
    \vskip -0.1in
\end{figure}
% \begin{wrapfigure}{r}{0.5\textwidth}
%     \centering
%     \includegraphics[width=0.49\textwidth]{imgs/heatmap_acc.pdf}
%     \caption{\textbf{Visualization of hyper-torus Bayesian flow based on von Mises Distribution}. For $x = 0.3, \beta(1) = 1000$ and $\alpha_i$ defined in \cref{appd:bfn_cir}, this figure plots three colored stochastic parameter trajectories for receiver mean parameter $m$ and accumulated accuracy parameter $c$, superimposed on a log-scale heatmap of the Bayesian flow distribution $p_F(m|x,\senderacc)$ and $p_F(c|x,\senderacc)$. Note the \emph{non-monotonicity} and \emph{non-additive} property of $c$. \jj{Shrink the figures to save space}}
%     \label{fig:vmbf_vis}
%     \vspace{-30pt}
% \end{wrapfigure}


In this section, we explain the detailed design of CrysBFN tackling theoretical and practical challenges. First, we describe how to derive our new formulation of Bayesian Flow Networks over hyper-torus $\mathbb{T}^{D}$ from scratch. Next, we illustrate the two key differences between \modelname and the original form of BFN: $1)$ a meticulously designed novel base distribution with different Bayesian update rules; and $2)$ different properties over the accuracy scheduling resulted from the periodicity and the new Bayesian update rules. Then, we present in detail the overall framework of \modelname over each manifold of the crystal space (\textit{i.e.} fractional coordinates, lattice vectors, atom types) respecting \textit{periodic E(3) invariance}. 

% In this section, we first demonstrate how to build Bayesian flow on hyper-torus $\mathbb{T}^{D}$ by overcoming theoretical and practical problems to provide a low-noise parameter-space approach to fractional atom coordinate generation. Next, we present how \modelname models each manifold of crystal space respecting \textit{periodic E(3) invariance}. 

\subsection{Periodic Bayesian Flow on Hyper-torus \texorpdfstring{$\mathbb{T}^{D}$}{}} 
For generative modeling of fractional coordinates in crystal, we first construct a periodic Bayesian flow on \texorpdfstring{$\mathbb{T}^{D}$}{} by designing every component of the totally new Bayesian update process which we demonstrate to be distinct from the original Bayesian flow (please see \cref{fig:non_add}). 
 %:) 
 
 The fractional atom coordinate system \citep{jiao2023crystal} inherently distributes over a hyper-torus support $\mathbb{T}^{3\times N}$. Hence, the normal distribution support on $\R$ used in the original \citep{bfn} is not suitable for this scenario. 
% The key problem of generative modeling for crystal is the periodicity of Cartesian atom coordinates $\vX$ requiring:
% \begin{equation}\label{eq:periodcity}
% p(\vA,\vL,\vX)=p(\vA,\vL,\vX+\vec{LK}),\text{where}~\vec{K}=\vec{k}\vec{1}_{1\times N},\forall\vec{k}\in\mathbb{Z}^{3\times1}
% \end{equation}
% However, there does not exist such a distribution supporting on $\R$ to model such property because the integration of such distribution over $\R$ will not be finite and equal to 1. Therefore, the normal distribution used in \citet{bfn} can not meet this condition.

To tackle this problem, the circular distribution~\citep{mardia2009directional} over the finite interval $[-\pi,\pi)$ is a natural choice as the base distribution for deriving the BFN on $\mathbb{T}^D$. 
% one natural choice is to 
% we would like to consider the circular distribution over the finite interval as the base 
% we find that circular distributions \citep{mardia2009directional} defined on a finite interval with lengths of $2\pi$ can be used as the instantiation of input distribution for the BFN on $\mathbb{T}^D$.
Specifically, circular distributions enjoy desirable periodic properties: $1)$ the integration over any interval length of $2\pi$ equals 1; $2)$ the probability distribution function is periodic with period $2\pi$.  Sharing the same intrinsic with fractional coordinates, such periodic property of circular distribution makes it suitable for the instantiation of BFN's input distribution, in parameterizing the belief towards ground truth $\x$ on $\mathbb{T}^D$. 
% \yuxuan{this is very complicated from my perspective.} \hanlin{But this property is exactly beautiful and perfectly fit into the BFN.}

\textbf{von Mises Distribution and its Bayesian Update} We choose von Mises distribution \citep{mardia2009directional} from various circular distributions as the form of input distribution, based on the appealing conjugacy property required in the derivation of the BFN framework.
% to leverage the Bayesian conjugacy property of von Mises distribution which is required by the BFN framework. 
That is, the posterior of a von Mises distribution parameterized likelihood is still in the family of von Mises distributions. The probability density function of von Mises distribution with mean direction parameter $m$ and concentration parameter $c$ (describing the entropy/uncertainty of $m$) is defined as: 
\begin{equation}
f(x|m,c)=vM(x|m,c)=\frac{\exp(c\cos(x-m))}{2\pi I_0(c)}
\end{equation}
where $I_0(c)$ is zeroth order modified Bessel function of the first kind as the normalizing constant. Given the last univariate belief parameterized by von Mises distribution with parameter $\theta_{i-1}=\{m_{i-1},\ c_{i-1}\}$ and the sample $y$ from sender distribution with unknown data sample $x$ and known accuracy $\alpha$ describing the entropy/uncertainty of $y$,  Bayesian update for the receiver is deducted as:
\begin{equation}
 h(\{m_{i-1},c_{i-1}\},y,\alpha)=\{m_i,c_i \}, \text{where}
\end{equation}
\begin{equation}\label{eq:h_m}
m_i=\text{atan2}(\alpha\sin y+c_{i-1}\sin m_{i-1}, {\alpha\cos y+c_{i-1}\cos m_{i-1}})
\end{equation}
\begin{equation}\label{eq:h_c}
c_i =\sqrt{\alpha^2+c_{i-1}^2+2\alpha c_{i-1}\cos(y-m_{i-1})}
\end{equation}
The proof of the above equations can be found in \cref{apdx:bayesian_update_function}. The atan2 function refers to  2-argument arctangent. Independently conducting  Bayesian update for each dimension, we can obtain the Bayesian update distribution by marginalizing $\y$:
\begin{equation}
p_U(\vtheta'|\vtheta,\bold{x};\alpha)=\mathbb{E}_{p_S(\bold{y}|\bold{x};\alpha)}\delta(\vtheta'-h(\vtheta,\bold{y},\alpha))=\mathbb{E}_{vM(\bold{y}|\bold{x},\alpha)}\delta(\vtheta'-h(\vtheta,\bold{y},\alpha))
\end{equation} 
\begin{figure}
    \centering
    \vskip -0.15in
    \includegraphics[width=0.95\linewidth]{imgs/non_add.pdf}
    \caption{An intuitive illustration of non-additive accuracy Bayesian update on the torus. The lengths of arrows represent the uncertainty/entropy of the belief (\emph{e.g.}~$1/\sigma^2$ for Gaussian and $c$ for von Mises). The directions of the arrows represent the believed location (\emph{e.g.}~ $\mu$ for Gaussian and $m$ for von Mises).}
    \label{fig:non_add}
    \vskip -0.15in
\end{figure}
\textbf{Non-additive Accuracy} 
The additive accuracy is a nice property held with the Gaussian-formed sender distribution of the original BFN expressed as:
\begin{align}
\label{eq:standard_id}
    \update(\parsn{}'' \mid \parsn{}, \x; \alpha_a+\alpha_b) = \E_{\update(\parsn{}' \mid \parsn{}, \x; \alpha_a)} \update(\parsn{}'' \mid \parsn{}', \x; \alpha_b)
\end{align}
Such property is mainly derived based on the standard identity of Gaussian variable:
\begin{equation}
X \sim \mathcal{N}\left(\mu_X, \sigma_X^2\right), Y \sim \mathcal{N}\left(\mu_Y, \sigma_Y^2\right) \Longrightarrow X+Y \sim \mathcal{N}\left(\mu_X+\mu_Y, \sigma_X^2+\sigma_Y^2\right)
\end{equation}
The additive accuracy property makes it feasible to derive the Bayesian flow distribution $
p_F(\boldsymbol{\theta} \mid \mathbf{x} ; i)=p_U\left(\boldsymbol{\theta} \mid \boldsymbol{\theta}_0, \mathbf{x}, \sum_{k=1}^{i} \alpha_i \right)
$ for the simulation-free training of \cref{eq:loss_n}.
It should be noted that the standard identity in \cref{eq:standard_id} does not hold in the von Mises distribution. Hence there exists an important difference between the original Bayesian flow defined on Euclidean space and the Bayesian flow of circular data on $\mathbb{T}^D$ based on von Mises distribution. With prior $\btheta = \{\bold{0},\bold{0}\}$, we could formally represent the non-additive accuracy issue as:
% The additive accuracy property implies the fact that the "confidence" for the data sample after observing a series of the noisy samples with accuracy ${\alpha_1, \cdots, \alpha_i}$ could be  as the accuracy sum  which could be  
% Here we 
% Here we emphasize the specific property of BFN based on von Mises distribution.
% Note that 
% \begin{equation}
% \update(\parsn'' \mid \parsn, \x; \alpha_a+\alpha_b) \ne \E_{\update(\parsn' \mid \parsn, \x; \alpha_a)} \update(\parsn'' \mid \parsn', \x; \alpha_b)
% \end{equation}
% \oyyw{please check whether the below equation is better}
% \yuxuan{I fill somehow confusing on what is the update distribution with $\alpha$. }
% \begin{equation}
% \update(\parsn{}'' \mid \parsn{}, \x; \alpha_a+\alpha_b) \ne \E_{\update(\parsn{}' \mid \parsn{}, \x; \alpha_a)} \update(\parsn{}'' \mid \parsn{}', \x; \alpha_b)
% \end{equation}
% We give an intuitive visualization of such difference in \cref{fig:non_add}. The untenability of this property can materialize by considering the following case: with prior $\btheta = \{\bold{0},\bold{0}\}$, check the two-step Bayesian update distribution with $\alpha_a,\alpha_b$ and one-step Bayesian update with $\alpha=\alpha_a+\alpha_b$:
\begin{align}
\label{eq:nonadd}
     &\update(c'' \mid \parsn, \x; \alpha_a+\alpha_b)  = \delta(c-\alpha_a-\alpha_b)
     \ne  \mathbb{E}_{p_U(\parsn' \mid \parsn, \x; \alpha_a)}\update(c'' \mid \parsn', \x; \alpha_b) \nonumber \\&= \mathbb{E}_{vM(\bold{y}_b|\bold{x},\alpha_a)}\mathbb{E}_{vM(\bold{y}_a|\bold{x},\alpha_b)}\delta(c-||[\alpha_a \cos\y_a+\alpha_b\cos \y_b,\alpha_a \sin\y_a+\alpha_b\sin \y_b]^T||_2)
\end{align}
A more intuitive visualization could be found in \cref{fig:non_add}. This fundamental difference between periodic Bayesian flow and that of \citet{bfn} presents both theoretical and practical challenges, which we will explain and address in the following contents.

% This makes constructing Bayesian flow based on von Mises distribution intrinsically different from previous Bayesian flows (\citet{bfn}).

% Thus, we must reformulate the framework of Bayesian flow networks  accordingly. % and do necessary reformulations of BFN. 

% \yuxuan{overall I feel this part is complicated by using the language of update distribution. I would like to suggest simply use bayesian update, to provide intuitive explantion.}\hanlin{See the illustration in \cref{fig:non_add}}

% That introduces a cascade of problems, and we investigate the following issues: $(1)$ Accuracies between sender and receiver are not synchronized and need to be differentiated. $(2)$ There is no tractable Bayesian flow distribution for a one-step sample conditioned on a given time step $i$, and naively simulating the Bayesian flow results in computational overhead. $(3)$ It is difficult to control the entropy of the Bayesian flow. $(4)$ Accuracy is no longer a function of $t$ and becomes a distribution conditioned on $t$, which can be different across dimensions.
%\jj{Edited till here}

\textbf{Entropy Conditioning} As a common practice in generative models~\citep{ddpm,flowmatching,bfn}, timestep $t$ is widely used to distinguish among generation states by feeding the timestep information into the networks. However, this paper shows that for periodic Bayesian flow, the accumulated accuracy $\vc_i$ is more effective than time-based conditioning by informing the network about the entropy and certainty of the states $\parsnt{i}$. This stems from the intrinsic non-additive accuracy which makes the receiver's accumulated accuracy $c$ not bijective function of $t$, but a distribution conditioned on accumulated accuracies $\vc_i$ instead. Therefore, the entropy parameter $\vc$ is taken logarithm and fed into the network to describe the entropy of the input corrupted structure. We verify this consideration in \cref{sec:exp_ablation}. 
% \yuxuan{implement variant. traditionally, the timestep is widely used to distinguish the different states by putting the timestep embedding into the networks. citation of FM, diffusion, BFN. However, we find that conditioned on time in periodic flow could not provide extra benefits. To further boost the performance, we introduce a simple yet effective modification term entropy conditional. This is based on that the accumulated accuracy which represents the current uncertainty or entropy could be a better indicator to distinguish different states. + Describe how you do this. }



\textbf{Reformulations of BFN}. Recall the original update function with Gaussian sender distribution, after receiving noisy samples $\y_1,\y_2,\dots,\y_i$ with accuracies $\senderacc$, the accumulated accuracies of the receiver side could be analytically obtained by the additive property and it is consistent with the sender side.
% Since observing sample $\y$ with $\alpha_i$ can not result in exact accuracy increment $\alpha_i$ for receiver, the accuracies between sender and receiver are not synchronized which need to be differentiated. 
However, as previously mentioned, this does not apply to periodic Bayesian flow, and some of the notations in original BFN~\citep{bfn} need to be adjusted accordingly. We maintain the notations of sender side's one-step accuracy $\alpha$ and added accuracy $\beta$, and alter the notation of receiver's accuracy parameter as $c$, which is needed to be simulated by cascade of Bayesian updates. We emphasize that the receiver's accumulated accuracy $c$ is no longer a function of $t$ (differently from the Gaussian case), and it becomes a distribution conditioned on received accuracies $\senderacc$ from the sender. Therefore, we represent the Bayesian flow distribution of von Mises distribution as $p_F(\btheta|\x;\alpha_1,\alpha_2,\dots,\alpha_i)$. And the original simulation-free training with Bayesian flow distribution is no longer applicable in this scenario.
% Different from previous BFNs where the accumulated accuracy $\rho$ is not explicitly modeled, the accumulated accuracy parameter $c$ (visualized in \cref{fig:vmbf_vis}) needs to be explicitly modeled by feeding it to the network to avoid information loss.
% the randomaccuracy parameter $c$ (visualized in \cref{fig:vmbf_vis}) implies that there exists information in $c$ from the sender just like $m$, meaning that $c$ also should be fed into the network to avoid information loss. 
% We ablate this consideration in  \cref{sec:exp_ablation}. 

\textbf{Fast Sampling from Equivalent Bayesian Flow Distribution} Based on the above reformulations, the Bayesian flow distribution of von Mises distribution is reframed as: 
\begin{equation}\label{eq:flow_frac}
p_F(\btheta_i|\x;\alpha_1,\alpha_2,\dots,\alpha_i)=\E_{\update(\parsnt{1} \mid \parsnt{0}, \x ; \alphat{1})}\dots\E_{\update(\parsn_{i-1} \mid \parsnt{i-2}, \x; \alphat{i-1})} \update(\parsnt{i} | \parsnt{i-1},\x;\alphat{i} )
\end{equation}
Naively sampling from \cref{eq:flow_frac} requires slow auto-regressive iterated simulation, making training unaffordable. Noticing the mathematical properties of \cref{eq:h_m,eq:h_c}, we  transform \cref{eq:flow_frac} to the equivalent form:
\begin{equation}\label{eq:cirflow_equiv}
p_F(\vec{m}_i|\x;\alpha_1,\alpha_2,\dots,\alpha_i)=\E_{vM(\y_1|\x,\alpha_1)\dots vM(\y_i|\x,\alpha_i)} \delta(\vec{m}_i-\text{atan2}(\sum_{j=1}^i \alpha_j \cos \y_j,\sum_{j=1}^i \alpha_j \sin \y_j))
\end{equation}
\begin{equation}\label{eq:cirflow_equiv2}
p_F(\vec{c}_i|\x;\alpha_1,\alpha_2,\dots,\alpha_i)=\E_{vM(\y_1|\x,\alpha_1)\dots vM(\y_i|\x,\alpha_i)}  \delta(\vec{c}_i-||[\sum_{j=1}^i \alpha_j \cos \y_j,\sum_{j=1}^i \alpha_j \sin \y_j]^T||_2)
\end{equation}
which bypasses the computation of intermediate variables and allows pure tensor operations, with negligible computational overhead.
\begin{restatable}{proposition}{cirflowequiv}
The probability density function of Bayesian flow distribution defined by \cref{eq:cirflow_equiv,eq:cirflow_equiv2} is equivalent to the original definition in \cref{eq:flow_frac}. 
\end{restatable}
\textbf{Numerical Determination of Linear Entropy Sender Accuracy Schedule} ~Original BFN designs the accuracy schedule $\beta(t)$ to make the entropy of input distribution linearly decrease. As for crystal generation task, to ensure information coherence between modalities, we choose a sender accuracy schedule $\senderacc$ that makes the receiver's belief entropy $H(t_i)=H(p_I(\cdot|\vtheta_i))=H(p_I(\cdot|\vc_i))$ linearly decrease \emph{w.r.t.} time $t_i$, given the initial and final accuracy parameter $c(0)$ and $c(1)$. Due to the intractability of \cref{eq:vm_entropy}, we first use numerical binary search in $[0,c(1)]$ to determine the receiver's $c(t_i)$ for $i=1,\dots, n$ by solving the equation $H(c(t_i))=(1-t_i)H(c(0))+tH(c(1))$. Next, with $c(t_i)$, we conduct numerical binary search for each $\alpha_i$ in $[0,c(1)]$ by solving the equations $\E_{y\sim vM(x,\alpha_i)}[\sqrt{\alpha_i^2+c_{i-1}^2+2\alpha_i c_{i-1}\cos(y-m_{i-1})}]=c(t_i)$ from $i=1$ to $i=n$ for arbitrarily selected $x\in[-\pi,\pi)$.

After tackling all those issues, we have now arrived at a new BFN architecture for effectively modeling crystals. Such BFN can also be adapted to other type of data located in hyper-torus $\mathbb{T}^{D}$.

\subsection{Equivariant Bayesian Flow for Crystal}
With the above Bayesian flow designed for generative modeling of fractional coordinate $\vF$, we are able to build equivariant Bayesian flow for each modality of crystal. In this section, we first give an overview of the general training and sampling algorithm of \modelname (visualized in \cref{fig:framework}). Then, we describe the details of the Bayesian flow of every modality. The training and sampling algorithm can be found in \cref{alg:train} and \cref{alg:sampling}.

\textbf{Overview} Operating in the parameter space $\bthetaM=\{\bthetaA,\bthetaL,\bthetaF\}$, \modelname generates high-fidelity crystals through a joint BFN sampling process on the parameter of  atom type $\bthetaA$, lattice parameter $\vec{\theta}^L=\{\bmuL,\brhoL\}$, and the parameter of fractional coordinate matrix $\bthetaF=\{\bmF,\bcF\}$. We index the $n$-steps of the generation process in a discrete manner $i$, and denote the corresponding continuous notation $t_i=i/n$ from prior parameter $\thetaM_0$ to a considerably low variance parameter $\thetaM_n$ (\emph{i.e.} large $\vrho^L,\bmF$, and centered $\bthetaA$).

At training time, \modelname samples time $i\sim U\{1,n\}$ and $\bthetaM_{i-1}$ from the Bayesian flow distribution of each modality, serving as the input to the network. The network $\net$ outputs $\net(\parsnt{i-1}^\mathcal{M},t_{i-1})=\net(\parsnt{i-1}^A,\parsnt{i-1}^F,\parsnt{i-1}^L,t_{i-1})$ and conducts gradient descents on loss function \cref{eq:loss_n} for each modality. After proper training, the sender distribution $p_S$ can be approximated by the receiver distribution $p_R$. 

At inference time, from predefined $\thetaM_0$, we conduct transitions from $\thetaM_{i-1}$ to $\thetaM_{i}$ by: $(1)$ sampling $\y_i\sim p_R(\bold{y}|\thetaM_{i-1};t_i,\alpha_i)$ according to network prediction $\predM{i-1}$; and $(2)$ performing Bayesian update $h(\thetaM_{i-1},\y^\calM_{i-1},\alpha_i)$ for each dimension. 

% Alternatively, we complete this transition using the flow-back technique by sampling 
% $\thetaM_{i}$ from Bayesian flow distribution $\flow(\btheta^M_{i}|\predM{i-1};t_{i-1})$. 

% The training objective of $\net$ is to minimize the KL divergence between sender distribution and receiver distribution for every modality as defined in \cref{eq:loss_n} which is equivalent to optimizing the negative variational lower bound $\calL^{VLB}$ as discussed in \cref{sec:preliminaries}. 

%In the following part, we will present the Bayesian flow of each modality in detail.

\textbf{Bayesian Flow of Fractional Coordinate $\vF$}~The distribution of the prior parameter $\bthetaF_0$ is defined as:
\begin{equation}\label{eq:prior_frac}
    p(\bthetaF_0) \defeq \{vM(\vm_0^F|\vec{0}_{3\times N},\vec{0}_{3\times N}),\delta(\vc_0^F-\vec{0}_{3\times N})\} = \{U(\vec{0},\vec{1}),\delta(\vc_0^F-\vec{0}_{3\times N})\}
\end{equation}
Note that this prior distribution of $\vm_0^F$ is uniform over $[\vec{0},\vec{1})$, ensuring the periodic translation invariance property in \cref{De:pi}. The training objective is minimizing the KL divergence between sender and receiver distribution (deduction can be found in \cref{appd:cir_loss}): 
%\oyyw{replace $\vF$ with $\x$?} \hanlin{notations follow Preliminary?}
\begin{align}\label{loss_frac}
\calL_F = n \E_{i \sim \ui{n}, \flow(\parsn{}^F \mid \vF ; \senderacc)} \alpha_i\frac{I_1(\alpha_i)}{I_0(\alpha_i)}(1-\cos(\vF-\predF{i-1}))
\end{align}
where $I_0(x)$ and $I_1(x)$ are the zeroth and the first order of modified Bessel functions. The transition from $\bthetaF_{i-1}$ to $\bthetaF_{i}$ is the Bayesian update distribution based on network prediction:
\begin{equation}\label{eq:transi_frac}
    p(\btheta^F_{i}|\parsnt{i-1}^\calM)=\mathbb{E}_{vM(\bold{y}|\predF{i-1},\alpha_i)}\delta(\btheta^F_{i}-h(\btheta^F_{i-1},\bold{y},\alpha_i))
\end{equation}
\begin{restatable}{proposition}{fracinv}
With $\net_{F}$ as a periodic translation equivariant function namely $\net_F(\parsnt{}^A,w(\parsnt{}^F+\vt),\parsnt{}^L,t)=w(\net_F(\parsnt{}^A,\parsnt{}^F,\parsnt{}^L,t)+\vt), \forall\vt\in\R^3$, the marginal distribution of $p(\vF_n)$ defined by \cref{eq:prior_frac,eq:transi_frac} is periodic translation invariant. 
\end{restatable}
\textbf{Bayesian Flow of Lattice Parameter \texorpdfstring{$\boldsymbol{L}$}{}}   
Noting the lattice parameter $\bm{L}$ located in Euclidean space, we set prior as the parameter of a isotropic multivariate normal distribution $\btheta^L_0\defeq\{\vmu_0^L,\vrho_0^L\}=\{\bm{0}_{3\times3},\bm{1}_{3\times3}\}$
% \begin{equation}\label{eq:lattice_prior}
% \btheta^L_0\defeq\{\vmu_0^L,\vrho_0^L\}=\{\bm{0}_{3\times3},\bm{1}_{3\times3}\}
% \end{equation}
such that the prior distribution of the Markov process on $\vmu^L$ is the Dirac distribution $\delta(\vec{\mu_0}-\vec{0})$ and $\delta(\vec{\rho_0}-\vec{1})$, 
% \begin{equation}
%     p_I^L(\boldsymbol{L}|\btheta_0^L)=\mathcal{N}(\bm{L}|\bm{0},\bm{I})
% \end{equation}
which ensures O(3)-invariance of prior distribution of $\vL$. By Eq. 77 from \citet{bfn}, the Bayesian flow distribution of the lattice parameter $\bm{L}$ is: 
\begin{align}% =p_U(\bmuL|\btheta_0^L,\bm{L},\beta(t))
p_F^L(\bmuL|\bm{L};t) &=\mathcal{N}(\bmuL|\gamma(t)\bm{L},\gamma(t)(1-\gamma(t))\bm{I}) 
\end{align}
where $\gamma(t) = 1 - \sigma_1^{2t}$ and $\sigma_1$ is the predefined hyper-parameter controlling the variance of input distribution at $t=1$ under linear entropy accuracy schedule. The variance parameter $\vrho$ does not need to be modeled and fed to the network, since it is deterministic given the accuracy schedule. After sampling $\bmuL_i$ from $p_F^L$, the training objective is defined as minimizing KL divergence between sender and receiver distribution (based on Eq. 96 in \citet{bfn}):
\begin{align}
\mathcal{L}_{L} = \frac{n}{2}\left(1-\sigma_1^{2/n}\right)\E_{i \sim \ui{n}}\E_{\flow(\bmuL_{i-1} |\vL ; t_{i-1})}  \frac{\left\|\vL -\predL{i-1}\right\|^2}{\sigma_1^{2i/n}},\label{eq:lattice_loss}
\end{align}
where the prediction term $\predL{i-1}$ is the lattice parameter part of network output. After training, the generation process is defined as the Bayesian update distribution given network prediction:
\begin{equation}\label{eq:lattice_sampling}
    p(\bmuL_{i}|\parsnt{i-1}^\calM)=\update^L(\bmuL_{i}|\predL{i-1},\bmuL_{i-1};t_{i-1})
\end{equation}
    

% The final prediction of the lattice parameter is given by $\bmuL_n = \predL{n-1}$.
% \begin{equation}\label{eq:final_lattice}
%     \bmuL_n = \predL{n-1}
% \end{equation}

\begin{restatable}{proposition}{latticeinv}\label{prop:latticeinv}
With $\net_{L}$ as  O(3)-equivariant function namely $\net_L(\parsnt{}^A,\parsnt{}^F,\vQ\parsnt{}^L,t)=\vQ\net_L(\parsnt{}^A,\parsnt{}^F,\parsnt{}^L,t),\forall\vQ^T\vQ=\vI$, the marginal distribution of $p(\bmuL_n)$ defined by \cref{eq:lattice_sampling} is O(3)-invariant. 
\end{restatable}


\textbf{Bayesian Flow of Atom Types \texorpdfstring{$\boldsymbol{A}$}{}} 
Given that atom types are discrete random variables located in a simplex $\calS^K$, the prior parameter of $\boldsymbol{A}$ is the discrete uniform distribution over the vocabulary $\parsnt{0}^A \defeq \frac{1}{K}\vec{1}_{1\times N}$. 
% \begin{align}\label{eq:disc_input_prior}
% \parsnt{0}^A \defeq \frac{1}{K}\vec{1}_{1\times N}
% \end{align}
% \begin{align}
%     (\oh{j}{K})_k \defeq \delta_{j k}, \text{where }\oh{j}{K}\in \R^{K},\oh{\vA}{KD} \defeq \left(\oh{a_1}{K},\dots,\oh{a_N}{K}\right) \in \R^{K\times N}
% \end{align}
With the notation of the projection from the class index $j$ to the length $K$ one-hot vector $ (\oh{j}{K})_k \defeq \delta_{j k}, \text{where }\oh{j}{K}\in \R^{K},\oh{\vA}{KD} \defeq \left(\oh{a_1}{K},\dots,\oh{a_N}{K}\right) \in \R^{K\times N}$, the Bayesian flow distribution of atom types $\vA$ is derived in \citet{bfn}:
\begin{align}
\flow^{A}(\parsn^A \mid \vA; t) &= \E_{\N{\y \mid \beta^A(t)\left(K \oh{\vA}{K\times N} - \vec{1}_{K\times N}\right)}{\beta^A(t) K \vec{I}_{K\times N \times N}}} \delta\left(\parsn^A - \frac{e^{\y}\parsnt{0}^A}{\sum_{k=1}^K e^{\y_k}(\parsnt{0})_{k}^A}\right).
\end{align}
where $\beta^A(t)$ is the predefined accuracy schedule for atom types. Sampling $\btheta_i^A$ from $p_F^A$ as the training signal, the training objective is the $n$-step discrete-time loss for discrete variable \citep{bfn}: 
% \oyyw{can we simplify the next equation? Such as remove $K \times N, K \times N \times N$}
% \begin{align}
% &\calL_A = n\E_{i \sim U\{1,n\},\flow^A(\parsn^A \mid \vA ; t_{i-1}),\N{\y \mid \alphat{i}\left(K \oh{\vA}{KD} - \vec{1}_{K\times N}\right)}{\alphat{i} K \vec{I}_{K\times N \times N}}} \ln \N{\y \mid \alphat{i}\left(K \oh{\vA}{K\times N} - \vec{1}_{K\times N}\right)}{\alphat{i} K \vec{I}_{K\times N \times N}}\nonumber\\
% &\qquad\qquad\qquad-\sum_{d=1}^N \ln \left(\sum_{k=1}^K \out^{(d)}(k \mid \parsn^A; t_{i-1}) \N{\ydd{d} \mid \alphat{i}\left(K\oh{k}{K}- \vec{1}_{K\times N}\right)}{\alphat{i} K \vec{I}_{K\times N \times N}}\right)\label{discdisc_t_loss_exp}
% \end{align}
\begin{align}
&\calL_A = n\E_{i \sim U\{1,n\},\flow^A(\parsn^A \mid \vA ; t_{i-1}),\N{\y \mid \alphat{i}\left(K \oh{\vA}{KD} - \vec{1}\right)}{\alphat{i} K \vec{I}}} \ln \N{\y \mid \alphat{i}\left(K \oh{\vA}{K\times N} - \vec{1}\right)}{\alphat{i} K \vec{I}}\nonumber\\
&\qquad\qquad\qquad-\sum_{d=1}^N \ln \left(\sum_{k=1}^K \out^{(d)}(k \mid \parsn^A; t_{i-1}) \N{\ydd{d} \mid \alphat{i}\left(K\oh{k}{K}- \vec{1}\right)}{\alphat{i} K \vec{I}}\right)\label{discdisc_t_loss_exp}
\end{align}
where $\vec{I}\in \R^{K\times N \times N}$ and $\vec{1}\in\R^{K\times D}$. When sampling, the transition from $\bthetaA_{i-1}$ to $\bthetaA_{i}$ is derived as:
\begin{equation}
    p(\btheta^A_{i}|\parsnt{i-1}^\calM)=\update^A(\btheta^A_{i}|\btheta^A_{i-1},\predA{i-1};t_{i-1})
\end{equation}

The detailed training and sampling algorithm could be found in \cref{alg:train} and \cref{alg:sampling}.





\section{Experiments}
\label{sec:exp}
Following the settings in Section \ref{sec:existing}, we evaluate \textit{NovelSum}'s correlation with the fine-tuned model performance across 53 IT datasets and compare it with previous diversity metrics. Additionally, we conduct a correlation analysis using Qwen-2.5-7B \cite{yang2024qwen2} as the backbone model, alongside previous LLaMA-3-8B experiments, to further demonstrate the metric's effectiveness across different scenarios. Qwen is used for both instruction tuning and deriving semantic embeddings. Due to resource constraints, we run each strategy on Qwen for two rounds, resulting in 25 datasets. 

\subsection{Main Results}

\begin{table*}[!t]
    \centering
    \resizebox{\linewidth}{!}{
    \begin{tabular}{lcccccccccc}
    \toprule
    \multirow{3}*{\textbf{Diversity Metrics}} & \multicolumn{10}{c}{\textbf{Data Selection Strategies}} \\
    \cmidrule(lr){2-11}
    & \multirow{2}*{\textbf{K-means}} & \multirow{2}*{\vtop{\hbox{\textbf{K-Center}}\vspace{1mm}\hbox{\textbf{-Greedy}}}}  & \multirow{2}*{\textbf{QDIT}} & \multirow{2}*{\vtop{\hbox{\textbf{Repr}}\vspace{1mm}\hbox{\textbf{Filter}}}} & \multicolumn{5}{c}{\textbf{Random}} & \multirow{2}{*}{\textbf{Duplicate}} \\ 
    \cmidrule(lr){6-10}
    & & & & & \textbf{$\mathcal{X}^{all}$} & ShareGPT & WizardLM & Alpaca & Dolly &  \\
    \midrule
    \rowcolor{gray!15} \multicolumn{11}{c}{\textit{LLaMA-3-8B}} \\
    Facility Loc. $_{\times10^5}$ & \cellcolor{BLUE!40} 2.99 & \cellcolor{ORANGE!10} 2.73 & \cellcolor{BLUE!40} 2.99 & \cellcolor{BLUE!20} 2.86 & \cellcolor{BLUE!40} 2.99 & \cellcolor{BLUE!0} 2.83 & \cellcolor{BLUE!30} 2.88 & \cellcolor{BLUE!0} 2.83 & \cellcolor{ORANGE!20} 2.59 & \cellcolor{ORANGE!30} 2.52 \\    
    DistSum$_{cosine}$  & \cellcolor{BLUE!30} 0.648 & \cellcolor{BLUE!60} 0.746 & \cellcolor{BLUE!0} 0.629 & \cellcolor{BLUE!50} 0.703 & \cellcolor{BLUE!10} 0.634 & \cellcolor{BLUE!40} 0.656 & \cellcolor{ORANGE!30} 0.578 & \cellcolor{ORANGE!10} 0.605 & \cellcolor{ORANGE!20} 0.603 & \cellcolor{BLUE!10} 0.634 \\
    Vendi Score $_{\times10^7}$ & \cellcolor{BLUE!30} 1.70 & \cellcolor{BLUE!60} 2.53 & \cellcolor{BLUE!10} 1.59 & \cellcolor{BLUE!50} 2.23 & \cellcolor{BLUE!20} 1.61 & \cellcolor{BLUE!30} 1.70 & \cellcolor{ORANGE!10} 1.44 & \cellcolor{ORANGE!20} 1.32 & \cellcolor{ORANGE!10} 1.44 & \cellcolor{ORANGE!30} 0.05 \\
    \textbf{NovelSum (Ours)} & \cellcolor{BLUE!60} 0.693 & \cellcolor{BLUE!50} 0.687 & \cellcolor{BLUE!30} 0.673 & \cellcolor{BLUE!20} 0.671 & \cellcolor{BLUE!40} 0.675 & \cellcolor{BLUE!10} 0.628 & \cellcolor{BLUE!0} 0.591 & \cellcolor{ORANGE!10} 0.572 & \cellcolor{ORANGE!20} 0.50 & \cellcolor{ORANGE!30} 0.461 \\
    \midrule    
    \textbf{Model Performance} & \cellcolor{BLUE!60}1.32 & \cellcolor{BLUE!50}1.31 & \cellcolor{BLUE!40}1.25 & \cellcolor{BLUE!30}1.05 & \cellcolor{BLUE!20}1.20 & \cellcolor{BLUE!10}0.83 & \cellcolor{BLUE!0}0.72 & \cellcolor{ORANGE!10}0.07 & \cellcolor{ORANGE!20}-0.14 & \cellcolor{ORANGE!30}-1.35 \\
    \midrule
    \midrule
    \rowcolor{gray!15} \multicolumn{11}{c}{\textit{Qwen-2.5-7B}} \\
    Facility Loc. $_{\times10^5}$ & \cellcolor{BLUE!40} 3.54 & \cellcolor{ORANGE!30} 3.42 & \cellcolor{BLUE!40} 3.54 & \cellcolor{ORANGE!20} 3.46 & \cellcolor{BLUE!40} 3.54 & \cellcolor{BLUE!30} 3.51 & \cellcolor{BLUE!10} 3.50 & \cellcolor{BLUE!10} 3.50 & \cellcolor{ORANGE!20} 3.46 & \cellcolor{BLUE!0} 3.48 \\ 
    DistSum$_{cosine}$ & \cellcolor{BLUE!30} 0.260 & \cellcolor{BLUE!60} 0.440 & \cellcolor{BLUE!0} 0.223 & \cellcolor{BLUE!50} 0.421 & \cellcolor{BLUE!10} 0.230 & \cellcolor{BLUE!40} 0.285 & \cellcolor{ORANGE!20} 0.211 & \cellcolor{ORANGE!30} 0.189 & \cellcolor{ORANGE!10} 0.221 & \cellcolor{BLUE!20} 0.243 \\
    Vendi Score $_{\times10^6}$ & \cellcolor{ORANGE!10} 1.60 & \cellcolor{BLUE!40} 3.09 & \cellcolor{BLUE!10} 2.60 & \cellcolor{BLUE!60} 7.15 & \cellcolor{ORANGE!20} 1.41 & \cellcolor{BLUE!50} 3.36 & \cellcolor{BLUE!20} 2.65 & \cellcolor{BLUE!0} 1.89 & \cellcolor{BLUE!30} 3.04 & \cellcolor{ORANGE!30} 0.20 \\
    \textbf{NovelSum (Ours)}  & \cellcolor{BLUE!40} 0.440 & \cellcolor{BLUE!60} 0.505 & \cellcolor{BLUE!20} 0.403 & \cellcolor{BLUE!50} 0.495 & \cellcolor{BLUE!30} 0.408 & \cellcolor{BLUE!10} 0.392 & \cellcolor{BLUE!0} 0.349 & \cellcolor{ORANGE!10} 0.336 & \cellcolor{ORANGE!20} 0.320 & \cellcolor{ORANGE!30} 0.309 \\
    \midrule
    \textbf{Model Performance} & \cellcolor{BLUE!30} 1.06 & \cellcolor{BLUE!60} 1.45 & \cellcolor{BLUE!40} 1.23 & \cellcolor{BLUE!50} 1.35 & \cellcolor{BLUE!20} 0.87 & \cellcolor{BLUE!10} 0.07 & \cellcolor{BLUE!0} -0.08 & \cellcolor{ORANGE!10} -0.38 & \cellcolor{ORANGE!30} -0.49 & \cellcolor{ORANGE!20} -0.43 \\
    \bottomrule
    \end{tabular}
    }
    \caption{Measuring the diversity of datasets selected by different strategies using \textit{NovelSum} and baseline metrics. Fine-tuned model performances (Eq. \ref{eq:perf}), based on MT-bench and AlpacaEval, are also included for cross reference. Darker \colorbox{BLUE!60}{blue} shades indicate higher values for each metric, while darker \colorbox{ORANGE!30}{orange} shades indicate lower values. While data selection strategies vary in performance on LLaMA-3-8B and Qwen-2.5-7B, \textit{NovelSum} consistently shows a stronger correlation with model performance than other metrics. More results are provided in Appendix \ref{app:results}.}
    \label{tbl:main}
    \vspace{-4mm}
\end{table*}


\begin{table}[t!]
\centering
\resizebox{\linewidth}{!}{
\begin{tabular}{lcccc}
\toprule
\multirow{2}*{\textbf{Diversity Metrics}} & \multicolumn{3}{c}{\textbf{LLaMA}} & \textbf{Qwen}\\
\cmidrule(lr){2-4} \cmidrule(lr){5-5} 
& \textbf{Pearson} & \textbf{Spearman} & \textbf{Avg.} & \textbf{Avg.} \\
\midrule
TTR & -0.38 & -0.16 & -0.27 & -0.30 \\
vocd-D & -0.43 & -0.17 & -0.30 & -0.31 \\
\midrule
Facility Loc. & 0.86 & 0.69 & 0.77 & 0.08 \\
Entropy & 0.93 & 0.80 & 0.86 & 0.63 \\
\midrule
LDD & 0.61 & 0.75 & 0.68 & 0.60 \\
KNN Distance & 0.59 & 0.80 & 0.70 & 0.67 \\
DistSum$_{cosine}$ & 0.85 & 0.67 & 0.76 & 0.51 \\
Vendi Score & 0.70 & 0.85 & 0.78 & 0.60 \\
DistSum$_{L2}$ & 0.86 & 0.76 & 0.81 & 0.51 \\
Cluster Inertia & 0.81 & 0.85 & 0.83 & 0.76 \\
Radius & 0.87 & 0.81 & 0.84 & 0.48 \\
\midrule
NovelSum & \textbf{0.98} & \textbf{0.95} & \textbf{0.97} & \textbf{0.90} \\
\bottomrule
\end{tabular}
}
\caption{Correlations between different metrics and model performance on LLaMA-3-8B and Qwen-2.5-7B.  “Avg.” denotes the average correlation (Eq. \ref{eq:cor}).}
\label{tbl:correlations}
\vspace{-2mm}
\end{table}

\paragraph{\textit{NovelSum} consistently achieves state-of-the-art correlation with model performance across various data selection strategies, backbone LLMs, and correlation measures.}
Table \ref{tbl:main} presents diversity measurement results on datasets constructed by mainstream data selection methods (based on $\mathcal{X}^{all}$), random selection from various sources, and duplicated samples (with only $m=100$ unique samples). 
Results from multiple runs are averaged for each strategy.
Although these strategies yield varying performance rankings across base models, \textit{NovelSum} consistently tracks changes in IT performance by accurately measuring dataset diversity. For instance, K-means achieves the best performance on LLaMA with the highest NovelSum score, while K-Center-Greedy excels on Qwen, also correlating with the highest NovelSum. Table \ref{tbl:correlations} shows the correlation coefficients between various metrics and model performance for both LLaMA and Qwen experiments, where \textit{NovelSum} achieves state-of-the-art correlation across different models and measures.

\paragraph{\textit{NovelSum} can provide valuable guidance for data engineering practices.}
As a reliable indicator of data diversity, \textit{NovelSum} can assess diversity at both the dataset and sample levels, directly guiding data selection and construction decisions. For example, Table \ref{tbl:main} shows that the combined data source $\mathcal{X}^{all}$ is a better choice for sampling diverse IT data than other sources. Moreover, \textit{NovelSum} can offer insights through comparative analyses, such as: (1) ShareGPT, which collects data from real internet users, exhibits greater diversity than Dolly, which relies on company employees, suggesting that IT samples from diverse sources enhance dataset diversity \cite{wang2024diversity-logD}; (2) In LLaMA experiments, random selection can outperform some mainstream strategies, aligning with prior work \cite{xia2024rethinking,diddee2024chasing}, highlighting gaps in current data selection methods for optimizing diversity.



\subsection{Ablation Study}


\textit{NovelSum} involves several flexible hyperparameters and variations. In our main experiments, \textit{NovelSum} uses cosine distance to compute $d(x_i, x_j)$ in Eq. \ref{eq:dad}. We set $\alpha = 1$, $\beta = 0.5$, and $K = 10$ nearest neighbors in Eq. \ref{eq:pws} and \ref{eq:dad}. Here, we conduct an ablation study to investigate the impact of these settings based on LLaMA-3-8B.

\begin{table}[ht!]
\centering
\resizebox{\linewidth}{!}{
\begin{tabular}{lccc}
\toprule
\textbf{Variants} & \textbf{Pearson} & \textbf{Spearman} & \textbf{Avg.} \\
\midrule
NovelSum & 0.98 & 0.96 & 0.97 \\
\midrule
\hspace{0.10cm} - Use $L2$ distance & 0.97 & 0.83 & 0.90\textsubscript{↓ 0.08} \\
\hspace{0.10cm} - $K=20$ & 0.98 & 0.96 & 0.97\textsubscript{↓ 0.00} \\
\hspace{0.10cm} - $\alpha=0$ (w/o proximity) & 0.79 & 0.31 & 0.55\textsubscript{↓ 0.42} \\
\hspace{0.10cm} - $\alpha=2$ & 0.73 & 0.88 & 0.81\textsubscript{↓ 0.16} \\
\hspace{0.10cm} - $\beta=0$ (w/o density) & 0.92 & 0.89 & 0.91\textsubscript{↓ 0.07} \\
\hspace{0.10cm} - $\beta=1$ & 0.90 & 0.62 & 0.76\textsubscript{↓ 0.21} \\
\bottomrule
\end{tabular}
}
\caption{Ablation Study for \textit{NovelSum}.}
\label{tbl:ablation}
\vspace{-2mm}
\end{table}

In Table \ref{tbl:ablation}, $\alpha=0$ removes the proximity weights, and $\beta=0$ eliminates the density multiplier. We observe that both $\alpha=0$ and $\beta=0$ significantly weaken the correlation, validating the benefits of the proximity-weighted sum and density-aware distance. Additionally, improper values for $\alpha$ and $\beta$ greatly reduce the metric's reliability, highlighting that \textit{NovelSum} strikes a delicate balance between distances and distribution. Replacing cosine distance with Euclidean distance and using more neighbors for density approximation have minimal impact, particularly on Pearson's correlation, demonstrating \textit{NovelSum}'s robustness to different distance measures.








\section{CONCLUSIONS}
In this work, we introduce a novel RL framework equipped with variable impedance control for end-to-end articulated object manipulation, which adaptively learns the object movement from observation and action history instead of naively executing a trajectory predicted before any robot-object contact. We demonstrate great sim-to-real transfer capability on diverse test objects in the real world and achieve 80\% and 84\% success rate for OpenDoor+ and OpenDrawer+ tasks, respectively, outperforming all existing works. Along with quantitative results, our policy can generate smooth and dexterous motion thanks to our well-designed training settings and reward functions. We hope our work can suggest an alternative way to leverage vision information, as well as other potential modalities (e.g. tactile grasp signal), to better bridge the sim-to-real gap for future RL-based manipulation works.

\section{Acknowledgment}
We thank all reviewers for their insights and suggestions.
%We thank Jiazhao Zhang and Danshi Li for proofreading the paper. We thank Zhongbin Wang for the robot setup, Chaoyi Xu for the object assembly, and Huy Ha for the visualization code \cite{ha2024umionlegs}.

%\addtolength{\textheight}{-12cm}   % This command serves to balance the column lengths
                                  % on the last page of the document manually. It shortens
                                  % the textheight of the last page by a suitable amount.
                                  % This command does not take effect until the next page
                                  % so it should come on the page before the last. Make
                                  % sure that you do not shorten the textheight too much.

%%%%%%%%%%%%%%%%%%%%%%%%%%%%%%%%%%%%%%%%%%%%%%%%%%%%%%%%%%%%%%%%%%%%%%%%%%%%%%%%



%%%%%%%%%%%%%%%%%%%%%%%%%%%%%%%%%%%%%%%%%%%%%%%%%%%%%%%%%%%%%%%%%%%%%%%%%%%%%%%%



%%%%%%%%%%%%%%%%%%%%%%%%%%%%%%%%%%%%%%%%%%%%%%%%%%%%%%%%%%%%%%%%%%%%%%%%%%%%%%%%
% \section*{APPENDIX}

% Appendixes should appear before the acknowledgment.

% \section*{ACKNOWLEDGMENT}

% The preferred spelling of the word ÒacknowledgmentÓ in America is without an ÒeÓ after the ÒgÓ. Avoid the stilted expression, ÒOne of us (R. B. G.) thanks . . .Ó  Instead, try ÒR. B. G. thanksÓ. Put sponsor acknowledgments in the unnumbered footnote on the first page.



%%%%%%%%%%%%%%%%%%%%%%%%%%%%%%%%%%%%%%%%%%%%%%%%%%%%%%%%%%%%%%%%%%%%%%%%%%%%%%%%

% References are important to the reader; therefore, each citation must be complete and correct. If at all possible, references should be commonly available publications.


\newpage
\bibliographystyle{IEEEtran}
% \balance
\bibliography{IEEEabrv,reference}

% \end{thebibliography}


\end{document}
