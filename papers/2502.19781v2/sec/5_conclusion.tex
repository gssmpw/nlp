\section{Discussion and Conclusion}

Self-supervised learning has been a key enabler of the rapid recent progress in computer vision and natural language processing. 
Representations from models like SimCLR~\cite{chen2020simple}, and CLIP~\cite{radford2021learning} have facilitated the creation of many powerful vision and language models. Recent works like SatCLIP~\cite{klemmer2023satclip}, GeoCLIP~\cite{vivanco2024geoclip}, and CSP~\cite{mai2023csp} showed that the same self-supervised techniques can be used to learn powerful representations for geographic locations, which are useful for a broad range of tasks.
%

In this paper, we introduced a simple, yet effective strategy for improving such geographic embeddings, moving past previous assumptions of multiview redundancy. We proposed a retrieval-augmented strategy for estimating visual features from an auxiliary database of visual embeddings. Our approach results in significant improvements for a variety of tasks as compared to purely parametric embedding strategies with only a modest increase in storage requirements.
Our method is efficient, robust, and multi-scale. We hope our insights and results substantiate our recommendation of using RANGE as a general-purpose location encoder for geospatial tasks. 

\section{Acknowledgments}
This research used the TGI RAILs advanced compute and data resource which is supported by the National Science Foundation (award OAC-2232860) and the Taylor Geospatial Institute.

