\section{Discussion}
\label{sec:discussion}

% Overview of Triad and it's significance

In this study, we constructed a large-scale 3D MRI dataset, known as TriadMR-131K, which consists of 131,170 volumes from 19,721 patients across 36 clinical datasets.
This extensive collection includes a diverse collection of 3D MRI data from three organs, including the brain, breast, and prostate. It includes modalities such as T1-w, T2-w, FLAIR, DWI-MRI, fMRI, and DCE-MRI.
Additionally, we extract imaging descriptions from the metadata of each 3D volume. These descriptions detail the imaging modality and associated device parameters.
Using this dataset, we develop Triad, a vision foundation model tailored for 3D MRI.
Triad employs widely used autoencoder architecture to learn robust representations and incorporates organ-independent imaging descriptions to constrain the semantic distribution of the visual modality.
We evaluate Triad on three tasks, including organ and tumor segmentation, organ and cancer classification, and medical image registration. These tasks are assessed across two data modalities---within-domain and out-of-domain--- across 25 downstream datasets.
By initializing models with Triad's pre-trained weights, nnUNet-Triad improves segmentation performance by 2.51\% over nnUNet-Scratch across 17 datasets. Swin-B-Triad achieves a 3.97\% improvement over Swin-B-Scratch in classification tasks across five datasets. SwinUNETR-Triad improves by 4.00\% compared to SwinUNETR-Scratch in registration tasks across two datasets.
Triad outperforms baseline models across all downstream tasks and exceeds existing state-of-the-art models in most cases.
Overall, Triad's seamless adaptability across downstream tasks highlights its potential as a versatile and efficient tool for diverse clinical applications.
This adaptability paves the way for enhanced diagnostic accuracy in 3D MRI.

Our findings highlight several key observations regarding the effectiveness of pretraining:

\begin{itemize}

\item Most vision foundation models for medical image analysis have been pre-trained on large-scale 3D CT datasets~\citep{tang2022self,ye2024continual,zhuang2024mim,jiang2023anatomical}, highlighting the prevalence of CT-based self-supervised learning in this domain.
However, our experimental results indicate that pre-training on modality-specific upstream data is more effective. This is demonstrated by the superior performance of Triad, which was pre-trained on MRI data, in 3D MRI-based segmentation, classification, and registration tasks.
This finding underscores a critical gap in current research, namely the absence of large-scale 3D MRI pre-training datasets for foundation models.
By assembling a diverse collection of 3D MRI data, our work addresses this gap and establishes a more appropriate pre-training paradigm for 3D MRI-based medical imaging tasks.
Furthermore, although this study employs an autoencoder (AE) architecture that is not the most advanced in self-supervised learning, our comparison with VoCo-SSL, a state-of-the-art distilled model, suggests that even a relatively simple AE-based framework can outperform more complex models under certain conditions.
This finding highlights an important insight: data quality and task-specific alignment are more critical than model complexity, suggesting that the future of medical AI should emphasize comprehensive and representative pre-training datasets rather than focusing solely on architectural advancements.

\item The experimental results of Triad pre-training highlight the dual impact of model parameter scale and task alignment.
In the context of 3D organ and tumor segmentation, pre-trained encoders consistently outperform their randomly initialized counterparts across various architectures.
However, increasing the scale of the model parameters does not consistently lead to improved performance, as evidenced by 11 out of 15 experiments showing no marked gain.
This suggests that while larger models theoretically have greater representational capacity, the limited size of downstream medical imaging datasets may lead to overfitting, offsetting the benefits of pre-training.
Furthermore, Triad, which is pre-trained on MRI data, surpasses the CT-based VoCo-SSL model in MRI-related tasks, underscoring the critical role of modality alignment in transfer learning for medical imaging.
Notably, Triad demonstrates superior segmentation performance in fine-grained tumor regions, such as the enhancing tumor component in BraTS21. This suggests that modality alignment not only affects overall segmentation accuracy but also improves the precise delineation of clinically relevant tumor subregions.

\item In out-of-domain tasks, Triad exhibits strong performance in MRI-based applications but lags behind VoCo-SSL in CT-based segmentation.
Although Triad does not surpass VoCo-SSL in CT tasks, it significantly outperforms models trained from scratch and achieves notable improvements in certain tumor segmentation tasks, such as liver cancer.
These results suggest that the benefits of pre-training are more pronounced for lesion recognition than for anatomical organ segmentation.
Additionally, in classification tasks, Swin-B consistently outperforms 3D UNet, suggesting that Transformer-based architectures may be more suitable for medical image classification.
However, in highly imbalanced datasets, Triad-pretrained models struggle with minority class discrimination, underscoring the limitations of pre-training alone in mitigating data imbalance.
These findings suggest that future work should incorporate advanced sampling strategies or hard example mining techniques to further improve model generalization across diverse medical imaging tasks.

\item In registration tasks using TransMorph, loading pre-trained weights from Triad and VoCo-SSL resulted in only marginal improvements in within-domain registration tasks, such as IXI and OASIS. However, in some cases, including the out-of-domain ACDC dataset, performance slightly declined compared to random initialization.
These results suggest that partially loading encoder weights, as in TransMorph, may introduce inconsistencies that impede optimal feature learning during fine-tuning.
In contrast, SwinUNETR, which employs a fully pre-trained encoder, achieved more substantial improvements when initialized with Triad weights.
This effect was particularly evident in the OASIS dataset, where the DSC increased markedly by 6.90\% compared to random initialization.
These findings indicate that leveraging a fully pre-trained encoder enables more effective feature transfer, thereby improving registration accuracy.
Furthermore, the superior performance of Triad over other pretraining approaches underscores the advantages of task-specific pretraining strategies tailored for 3D MRI data.

\end{itemize}


Despite the strengths of our study, there are several limitations in both our model and methodology.
First, the text data used for pre-training in Triad-131K primarily focuses on device parameters and imaging modalities rather than image content. This restricts the extension of the single-modality vision foundation model into a vision-language foundation model.
Second, the computational demands of pre-training such a large-scale 3D MRI dataset prevent us from employing ensemble methods and fixed-step validation, both of which could further enhance the performance of downstream tasks.
We retain model parameters only at iteration 20,000.
Third, fine-tuning across a broad range of downstream tasks is highly resource-intensive in terms of manpower, computational cost, and time.
Following VoCo’s approach, we report results solely on fold 0 instead of conducting multi-fold cross-validation, which may introduce variability due to data distribution effects.
Fourth, additional exploration is required to optimize Triad and enhance its performance on downstream tasks.
For instance, in the registration task, we initialize TransMorph with Triad’s pre-trained weights and subsequently fine-tune the model.
The training process involves balancing multiple loss functions; however, we use default parameters without exploring a broader optimization space.
Future research should explore more effective parameter configurations and optimization strategies.


Future research can focus on the following aspects. First, improving data quality.
Despite having implemented various preprocessing pipelines, the presence of low-quality cases remains a challenge.
Numerous studies have emphasized the crucial role of data quality in pre-training~\citep{yang2024freemask,kirillov2023segment}.
In future work, we plan to design robust automated screening pipelines to screen each volume and enhance data quality.
Second, assigning structured electronic health reports to each 3D MRI volume in the Triad-131K dataset to facilitate pre-training for vision-language foundation models.
Additionally, expanding downstream tasks to include assistive report generation, visual question answering, and cross-modal medical image retrieval.
Lastly, the current pre-training strategy is not limited to MRI scans.
Moving forward, we plan to integrate Triad-131K with the largest available CT, PET, X-ray, and ultrasound datasets to pre-train foundation models that generalize effectively across a broad range of clinical tasks, rather than being confined to specific imaging modalities.

