\section{Related work}
\label{sec:relatedwork}

Crystal structure databases such as the Materials Project ____ or the Inorganic Crystal Structure Database (ICSD) ____ contain large amounts of potential candidates for solid-state electrolytes. For example, ____ screened more than 12,000 Li-containing crystals for Li-ion SSEs using multiple criteria, thereby identifying 317 candidates, among which 21 crystals that showed promise as SSEs were selected from an ML-guided model. The ionic conductivity of these 21 structures was computed theoretically.

____ annotated 318 compounds by calculating ion migration energy barriers ($E_b$), a less accurate but computationally lighter property that relates to ionic conductivity. Bayesian optimization was employed to screen candidate compounds with low $E_b$.

____ compiled a database of over 90,000 crystal structures, including more than 7,000 structures with preliminary ion-transport data obtained through geometric analysis, and 12,000 activation energy values ($E_b$) calculated using the bond valence site energy method. Additionally, they manually extracted 75 CIF files from literature data. They employed empirical and geometrical methods to estimate the minimum energy paths of these structures and obtain $E_b$, but they did not predict $\sigma$.

In the last two years, four experimentally-derived datasets curated from a large collection of SSE literature were published. 

The Liverpool Ionics (LiIon) Dataset ____ reports 820 entries containing chemical composition, structural family, and ionic conductivity at different temperatures (from 5 to 873$^\circ C$) measured by alternating current impedance spectroscopy, among which 465 entries were at room temperature.

____ gathered a dataset of 1346 entries with compositions, space group, and corresponding $\sigma_{\text{RT}}$, with a subset of 344 compounds whose structures are manually matched with an ICSD ID. 

Finally, ____ used text mining to extract more than 4000 ionic conductivity measurements from 1457 papers. Each ionic conductivity measurement is associated with a composition and about 350 are also associated with a ``structure type''. Measurement temperature is not specified and compositions are not always fully described.

A recent study by ____ introduced the Dynamic Database of Solid-State Electrolyte (DDSE) to facilitate the exploration of structure-performance relationships and accelerate the discovery of high-performance solid-state electrolytes (SSEs). The database contains performance data for 2448 materials (at time of writing), including ionic conductivity obtained from experimental reports, across a broad temperature range (132.40–1261.60 K). By employing data mining and machine learning, the DDSE provides tools to evaluate and predict ionic conductivity for new materials.

These recent reports greatly increased the amount of readily available experimental ionic conductivity data. However, they contain limited structural information: the databases by ____ and ____ contain only a qualitative structure description for some materials, the LiIon dataset only includes the structural family and the dataset by ____ is limited to space group information. Although the full crystallographic information of the 344 compounds of the Laskowski dataset for which the ICSD ID is provided could be retrieved, the proprietary ICSD is not available to most researchers in the ML community. Table~\ref{tab:datasets_comparison} summarizes the differences in terms of available features across the databases discussed above. 

The lack of precise structural information labeled with ionic conductivity makes it difficult (1) to compare experimental values with theoretical predictions which require full crystal descriptions and (2) to train machine learning models to accurately predict ionic conductivity.

%Natural language processing (NLP) or text mining are expected to extract data much quickly from the extensive amount of scientific literature.  However, acquiring the high-quality data of SSEs have not been achievable yet ____, because the ionic conductivity is typically reported not as a single number but multiple values at different temperatures, and inconsistently presented (i.e., body text, tables, or figures) with no standardized units.

\begin{figure}
\vskip 0.2in
\begin{center}
\centerline{\includegraphics[width=\linewidth]{partial_occ.pdf}}
\caption{Examples of solid state electrolyte materials with partial occupancies.}
\label{fig:particaloccupancy}
\end{center}
\vskip -0.2in
\end{figure}
%
\begin{figure*}
\vskip 0.2in
\begin{center}
\centerline{\includegraphics[width=\linewidth]{gathered.pdf}}
\caption{\label{fig:stats} a) Distributions of ionic-conductivity values for the training and testing sets along with proportions of crystal families and space groups. Only space groups that represent more than 1\% of the sets are labeled. b) Venn diagram showing shared entries between our dataset and others c) Proportion of entries that contain each element in the periodic table. Elements that are not present in the dataset are shaded. Generated with pymatviz ____.}
\end{center}
\vskip -0.2in
\end{figure*}
%