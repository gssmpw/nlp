\begin{abstract}

A science is defined by a set of encyclopedic knowledge related to facts or phenomena following rules or evidenced by experimentally-driven observations. Computer Science and in particular computer networks is a relatively new scientific domain maturing over years and adopting the best practices inherited from more fundamental disciplines. The design of past, present and future networking components and architectures have been assisted, among other methods, by experimentally-driven research and in particular by the deployment of test platforms, usually named as testbeds. However, often experimentally-driven networking research used scattered methodologies, based on ad-hoc, small-sized testbeds, producing hardly repeatable results. We believe that computer networks needs to adopt a more structured methodology, supported by appropriate instruments, to produce credible experimental results supporting radical and incremental innovations. This paper reports lessons learned from the design and operation of test platforms for the scientific community dealing with digital infrastructures. We introduce the SLICES initiative as the outcome of several years of evolution of the concept of a networking test platform transformed into a scientific instrument. We address the challenges, requirements and opportunities that our community is facing to manage the full research-life cycle necessary to support a scientific methodology. 

\end{abstract}