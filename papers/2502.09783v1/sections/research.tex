
%#############################################################################################    
\section{The research Life Cycle}
\label{section:research}
%#############################################################################################

Experimentally-driven research should be grounded on a solid methodology that is understood and implemented by other disciplines. This is somehow the ambition of the European EOSC initiative. As a consequence, SLICES does not target only the deployment of the instrument/facility but as importantly, addresses the full research life-cycle, including open data, data management and reproducibility.

Researchers and research stakeholders nowadays require that research data is made available for other researchers to examine, experiment and develop further. Additionally, preserving the data in conjunction with how conclusions from the data were drawn, accelerates the discovery process, enable easier reproducibility of the results and thus supports evidence. It is then necessary to develop policies and procedures for regulating the management and publication of research data in order to make them interoperable and widely available.

In Europe, it is recommended to conform with the European Open Science~\cite{euos} and Open Access policy \cite{euoa}, Open Research Data Pilot \cite{eu_data_pilot} and FAIR \cite{fair} principles in producing and managing research data. This requires defining appropriate metadata (including compatible experiment description) on the data produced by or integrated into the infrastructure with the objective to ensure eventually data accessibility, trustworthiness, reusability and interoperability with data produced by similar infrastructures/experiments for enabling complex experiments and multi-domain research. 
Alignments with the relevant recommendations such as the ones published by EOSC FAIRsFAIR \cite{fairsfair} project, GO FAIR initiative \cite{gofair} and RDA for FAIR data management \cite{rda2020fair}, and general European Open Access to research publications and Open Research Data Pilot policies, are of utmost importance.

The FAIR (Findable, Accessible, Interoperable, and Reusable) \cite{fair-principles} Data Principles were developed to be used as guidelines for data producers and publishers, with regards to data management and stewardship. One important aspect that differentiates FAIR from any other related initiatives is that they move beyond the traditional data and they place specific emphasis on automatic computation, thus considering both human-driven and machine-driven data activities. Since their publication, FAIR principles became widely accepted and used.
To this end, SLICES fully endorses and adopts the FAIR principles, acting as a catalyst to enable and foster the data-driven science and scientific data-sharing in this area.


Understanding the data collected and processed within SLICES becomes essential to understand data usage from the target user groups. This should allow to develop an appropriate information model that represents the data collected from the SLICES testbeds, experimental equipment and applications. We consider that the datasets generated by the usage of the SLICES hardware and software infrastructure can be roughly organized into five main categories:

\begin{itemize}
    \item[-] \textbf{Observational Data:}  collected using methods such as surveys (e.g. online questionnaires) or recording of measurements (e.g. through sensors). The data include mostly data related to signal or performance measurements, and network or service log data that allow for experiment evaluation and reproducibility. 
    \item[-] \textbf{Experimental Data: } where researchers introduce an intervention and study the effects of certain variables, trying to determine their impact.
    \item[-] \textbf{Simulation Data: } is generated by using computer models that simulate the operation of a real-world process or system. These may use observational data.
    \item[-] \textbf{Derived Data: } involves the analysis (e.g. cleaning, transformation, summarization, predictive modeling) of existing data, often coming from different datasets (e.g. the results of two experiments), to create a new dataset for a specific purpose. 
    \item[-] \textbf{Metadata:  } concerns data that provides descriptors about all categories of data mentioned above. This information is essential in making the discovery of data easier and ensuring their interoperability.
\end{itemize}

SLICES, as an open platform, promotes interoperability, thus non-proprietary, unencrypted, uncompressed, and commonly used by the research community formats should be adopted. In addition, SLICES end users should have the ability to decide on a suitable license and attach it to their data. 

Our preliminary estimations for SLICES include up to 5,000 users and their data, accounting for up to 50GB per user on the individual nodes and up to 1TB on the cloud. This provides us with a preliminary estimation of 0.25PB-1PB of data storage for all datacenters residing on SLICES nodes, and 5PB for the cloud-based datacenter.

As a consequence, SLICES will setup a data management framework to support the efficient and effective operation of the SLICES infrastructure. To accomplish this, the data management framework sets its own design goals, which are summarized below.
%and presented in \Cref{fig:Data-managment-framework}.

\begin{itemize}
    \item[-] \textbf{Data Governance: } A systemic and effective Data Governance structure to support the data management operations through a hierarchical structure with appropriate roles (e.g. Data Manager, Data Protection Officer and Metadata administrator), implement all related policies and processes, and adopt standards and leading practices.
    \item[-] \textbf{Data Architecture: } An agile Data Architecture that can perform efficiently to fulfill the SLICES infrastructure requirements, scales gracefully to accommodate for increased workloads, is flexible to integrate new processes and technologies, and is open to interact with other systems and infrastructures. 
    \item[-] \textbf{Data Quality:} Appropriate data transformation mechanisms to ensure Data Quality across multiple dimensions (e.g. accuracy, completeness, integrity), in order to improve data utility (e.g. further processing, analysis).
    \item[-] \textbf{Metadata:} Appropriate metadata management mechanisms to facilitate collaboration between users by providing the means to share their data and also support FAIR data.  
    \item[-] \textbf{Interoperability:  } Facilitate seamless interaction with other systems and infrastructures.
    \item[-] \textbf{Analytics:} Deployment of statistical, machine learning and artificial intelligence techniques to draw valuable insights from data and appropriate visualisation techniques to interpret them.
    \item[-] \textbf{Data Security:  } Mechanisms to protect data from unauthorized access and protect its integrity.
    \item[-] \textbf{Privacy: } Strict controls to manage the sharing of data, both internally and externally.
    
\end{itemize} 


%FIGURE: Data Managment Framework
%\begin{figure}[h]
%   \centering
%   \includegraphics{././figures/png/Data-managment-framework.png}
%   \caption{Data Management Framework}
%	\label{fig:Data-managment-framework}
%\end{figure}



