%#############################################################################################	
\section{Design foundation principles}
\label{section:Design foundation principles}
%#############################################################################################

The future generation of digital infrastructures is designed to be programmable, extendable and scalable. Key enabling technologies are now available to empower this important transformation. An important concept is network disaggregation whereby networking software is separated from the switching and/or routing hardware and broken down into functional components that can be more efficiently operated. Software Defined Networking (SDN) assumes programmable network devices in which the forwarding plane is decoupled from the control plane. In addition, the control plane is logically centralized in a software-based controller (“network brain”), while the data plane is composed of network devices (“network arms”) that forward packets. Network Function Virtualization (NFV) will deliver the promises of a software framework to the network, creating the need for an efficient and effective orchestration of the network resources. 

The primary technologies and solutions to address the requirements for the SLICES facility can be classified as described in the next subsections.

\subsection{Software Defined Network and Network Function Virtualization}

Following the major evolution of telecommunications networks with the adoption of the internet technology and the emergence of cellular networks, we are now facing a paradigm shift in the way that digital infrastructures are designed and operated. Indeed, recent advances in networking such as SDN and NFV~\cite{sdn-nfv-review} are changing the way network operators deploy and manage Internet services. SDN and NFV, together or separately, bring to network operators new opportunities for reducing costs, enhancing network flexibility and scalability, and shortening the time-to-market of new applications and services. On the one hand, SDN introduces a logically centralized controller with a global view of the network state. On the other hand, NFV allows to fully decouple network functions from proprietary appliances and to run them as software applications on general–purpose machines. It is a scalable approach as it gives the operators the ability to scale their network architecture across multiple servers to adapt quickly to the changing needs of their customers. 

\subsection{Network Slicing}
Another disruptive concept that should help in realizing the vision is network slicing, which allows a single physical network to be segmented into multiple isolated logical networks of varying sizes and structures tailored to different types of services and customers~\cite{2018_cst_slicing}. It is a multi-tenant virtualization technique in which the various network functionalities are extracted from the hardware and/or software components and then offered in the form of slices to the different users of the infrastructure (tenants). Basically, each slice includes a number of dedicated physical resources and network functions, which are isolated from other slices and provide specific functionalities including RAN and core network. Network slicing aims to offer operators the possibility of creating, in real-time and on-demand, various levels of services for different enterprise verticals, enabling them to customize their operations. In particular, it allows service differentiation with different QoS levels, reliability and security. Network slicing requires a continuous reconciliation of customer-centric service level agreements (SLAs) with infrastructure-level network performance capabilities. However, one of the main issues to solve is how to meet the requirements of different network services programmed from a single physical infrastructure. Autonomic (AI-empowered) and self-optimised management is needed to dynamically create, scale down or up, and reconfigure according to application demands~\cite{2018_jsac_nfv,2022_6g_slicing}. It is important to remark that the slicing concept is one of the key innovations in 5G since Release 15, supported by NFV and SDN techniques. Gradually, this concept is evolving from supporting primarily core-network resource provisioning, to more advanced (and forward-looking) paradigms, whereby also edge and far-edge devices' resources can be virtualised and provided as dynamic, on-demand components of the network infrastructure~\cite{6953022,dressler22}. However, an end-to-end network slice composed of sub-slices that belong to different technological
domains (RAN, core, edge/cloud), requires hierarchical and distributed management solutions to cope with the heterogeneity of the orchestration systems of different technological domains~\cite{2018_access_slicing, 2022_access_zero-touch}. 

\subsection{Network disaggregation}
Legacy aggregated networking devices have been developed and commercialized by vendors for decades. The term aggregation refers here to the vertical integration of software and specialized hardware components, bundled into a proprietary networking device. Network device disaggregation is the ability to source switching hardware and network operating systems separately. The term white box switches refer to switches built on commodity hardware that run different possible Network Operating Systems (NOS). This approach is putting pressure on the legacy aggregated networking vendors, but requires talented developers to build and grow the solution. This concept has been extended to the radio access network: RAN disaggregation~\cite{bonati2020open,o-ran-disaggregation} was specified by 3GPP~\cite{functional-splits} and detailed by the Open Networking Foundation (ONF)~\cite{wypior2022open} as an important step allowing for dynamic creation and lifecycle management of use-case optimized network slices. The idea here is to split the RAN protocol stack so that the individual components can be developed independently by different vendors. This horizontal disaggregation also enables distributed deployment of RAN functions in the network.

\subsection{Distributed Platform}
All the aforementioned techniques SDN/NFV, network slicing and disaggregation can be combined in a distributed platform to test advanced networking scenarios in realistic large-scale environments. This could be done by leveraging virtualized computing and networking resources in a flexible way to provide support for solutions based on the use-case, geography and experimenter choice. In such a distributed platform, the functions of the RAN nodes (the base stations) may be deployed as a "Central Unit", centralizing the packet processing functions and executed as Virtual Network Functions (VNFs) on commodity hardware in edge cloud locations. One or more "distributed units" performing the baseband processing functions as VNFs on commodity hardware with possible hardware acceleration and several "radio units" running the radio functions with specialized hardware on antenna sites. In a more general setting, different functions can be deployed on different sites in the network in order to realize the required flexibility and assess performance of the different split options.

\subsection{Control and User-plane Separation}
Another vertical disaggregation consists in the separation of Control and User Planes (CUPS)~\cite{cups}. In fact, with the densification of the next generation radio access networks, and the availability of different spectrum bands, it is more and more difficult to optimally allocate radio resources, perform handovers, manage interfaces, and balance load between cells. It is therefore necessary to adopt centralized control of the access network in order to increase system performance. This approach can be realized by decoupling the intelligence from the underlying hardware in all parts of the network.


%Duplicate figure ???
%\begin{figure*}
%	\include{figures/tex/ESFRI_life_cycle}
%	\caption{ESFRI lifecycle approach.}
%	\label{fig:ESFRI_live_cycle}
%\end{figure*}

\subsection{Research Data Management}
SLICES wants to fully endorse and adopt the Open Science and FAIR principles, acting as a catalyst to enable and foster cutting edge research, data-driven science and scientific data-sharing. Several design considerations should be taken into consideration, including: 
(i) \emph{easy and open access} to scientific data to facilitate further knowledge discovery and research transparency ensuring the longevity of the data and access to the wider research and innovation community.
(ii) \emph{scalable architecture} to efficiently leverage a large number of storage resources to  support efficient data storage and compute, including highly parallelized data workflows to support experiments;
(iii) \emph{privacy preservation} methods for ensuring end-to-end security and privacy in compliance with relevant legal frameworks; and 
(iv) \emph{data quality assurance} methods to ensure data quality across multiple dimensions, such as accuracy, completeness and integrity, in order to improve data utility.
To address this, SLICES requires to carefully design and develop efficient and scalable data management, analysis and reporting mechanisms, supported by appropriate metadata profiles to cater for access and reuse of FAIR data and services. These tools need to capture and report the entire data lineage/provenance across the data management lifecycle, while also providing the systematic means for secure and trustworthy interoperability of data and services ensuring the authenticity and immutability of the shared data. 



 