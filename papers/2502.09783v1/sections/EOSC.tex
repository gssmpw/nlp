%#############################################################################################	
\section{Introduction to EOSC}
\label{section:EOSC}
%#############################################################################################

% 1.5.1.	Integration with EOSC infrastructure and services and international testbeds


Since SLICES aims to provide a pan-European experimental research platform by jointly utilizing the geographically dispersed computing, storage and networking RIs, it is highly important that the different RIs interacting in the experimental workflow are interoperable with each other. Similarly, existing research needs to be accessible and directly pluggable to SLICES services and sites. For example, considering a MEC use case, compute, storage and networking resources from different RIs would be used. In such a scenario, it is necessary that resource description, availability, execution and data exchange are smooth. This can only be assured if a common interoperability framework is adopted across the SLICES ecosystem so that different subsystems have a common understanding of resources, data/metadata and are on the same page with respect to the licensing, copyright and privacy requirements.  The SLICES infrastructure will be designed to ensure compatibility and integration with EOSC and existing ESFRI infrastructures, and be ready to offer advanced ICT infrastructure services to other RIs and projects, with the special focus on the FAIR data management and exchange. 

EOSC~\cite{eosc} has established itself as an important pillar in the implementation of Open science concept by accelerating the adoption of the FAIR data practices among researchers in the European Union. Integration of SLICES into European Research Infrastructure via EOSC will facilitate data sharing and reuse among SLICES partners and the larger European researchers’ community. This naturally aligns SLICES to the concepts of Open science. Interoperability-focused integration of SLICES with EOSC will make it easier for SLICES users to reap the benefits of many services and tools pertaining to diverse scientific domains that are being developed around the EOSC ecosystem. 

Therefore, it is of utmost importance to design the integration framework of SLICES with EOSC in such a way that the data exchange between SLICES and EOSC is interoperable for scientific workflow management for data storage, processing and reuse. To this end, the recommendations of the EOSC interoperability framework  are considered in great detail for the design of the  SLICES interoperability framework.

Interoperability is an essential feature of EOSC ecosystem as a federation of services and data exchange is unthinkable without interoperability among different EOSC constituents. The meaningful exchange and consumption of digital objects is necessary to generate value from EOSC, which can only be realized if different components of the EOSC ecosystem (software/machines and humans) have a common understanding of how to interpret and exchange them, what are the legal restrictions, and what processes are involved in distribution, consumption and production of them. To facilitate this, EOSC interoperability framework (EOSC-IF)~\cite{eosc-if} is defined as a generic framework for all the entities involved in the development and deployment of EOSC.

To achieve this, a dedicated interface, coined SLICES-Interoperability Framework (SLICES-IF) shall be developed. The interface will be built upon the foundations led by the European Interoperability Reference Architecture (EIRA)~\cite{eira}, where interoperability is classified at four layers, namely: (i) technical, (ii) semantic, (iii) organizational; and (iv) legal. Although the target audience for EIRA (governance and administration) was very different from the SLICES stakeholders, core principles and objectives are similar. Additionally, the different components (in particular technical and semantic) of SLICES-IF would be chosen in a such way that SLICES is fully interoperable with EOSC for uninterrupted data exchange pertaining to use of EOSC services and research data by SLICES as well as to enable the publications of SLICES infrastructure, services and data through EOSC portal. More details about the SLICES-IF interface to EOSC and external RIs is provided in SLICES-Design Study Deliverable D4.2~\cite{slices-ds-d4.2}.

%Figure: SLICES interconnection with European e-Infrastructures and digital infrastructures
\begin{figure*}[t]
    \centering
    \includegraphics{././figures/png/SLICES_interface_to_EOSC.png}
	\caption{SLICES interface to EOSC}
	\label{fig:SLICES-interface-to-EOSC}
\end{figure*}
