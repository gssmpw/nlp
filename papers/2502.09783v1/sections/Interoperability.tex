%#############################################################################################    
\section{Interoperability with EOSC and External Systems}
\label{section:Interoperability}
%#############################################################################################

Since SLICES aims to provide a pan-European experimental research platform by jointly utilizing the geographically dispersed computing, storage and networking RIs, it is highly important that the different RIs interacting in the experimental workflow are interoperable with each other. Similarly, existing research needs to be accessible and directly pluggable to SLICES services and sites. For example, considering a MEC use case, compute, storage and networking resources from different RIs can be used. In such a scenario, it is necessary that resource description, availability, execution and data exchange are smooth. This can only be assured if a common interoperability framework is adopted across the SLICES ecosystem so that different subsystems have a common understanding of resources and data/metadata are on the same page with respect to the licensing, copyright and privacy requirements.
The SLICES infrastructure is designed to ensure compatibility and integration with EOSC and existing ESFRI infrastructures, and be ready to offer advanced ICT infrastructure services to other RIs and projects, with the special focus on the FAIR data management and exchange. 

\begin{figure*}[t]
    \centering
    \includegraphics{././figures/png/SLICES_interface_to_EOSC.png}
	\caption{SLICES interconnection with European e-Infrastructures and digital infrastructures}
	\label{fig:SLICES interface to EOSC}
\end{figure*}



%\subsection{Integration with EOSC infrastructure and services and international testbeds}

EOSC~\cite{eosc} has established itself as an important pillar in the implementation of Open science concept by accelerating the adoption of the FAIR data practices among researchers in the European Union. Integration of SLICES into European Research Infrastructure via EOSC will facilitate data sharing and reuse among SLICES partners and the larger European researchers’ community. Interoperability-focused integration of SLICES with EOSC will make it easier for SLICES users to reap the benefits of many services and tools pertaining to diverse scientific domains that are being developed around the EOSC ecosystem. 

Therefore, it is of utmost importance to design the integration framework of SLICES with EOSC in such a way that the data exchange between SLICES and EOSC is interoperable for scientific workflow management for data storage, processing and reuse. To this end, the recommendations of the EOSC interoperability framework  are considered in great detail for the design of the SLICES interoperability framework.

Interoperability is an essential feature of EOSC ecosystem as a federation of services and data exchange is unthinkable without interoperability among different EOSC constituents. The meaningful exchange and consumption of digital objects is necessary to generate value from EOSC, which can only be realized if different components of the EOSC ecosystem (software/machines and humans) have a common understanding of how to interpret and exchange them, what are the legal restrictions, and what processes are involved in their distribution, consumption and production. To facilitate this, EOSC interoperability framework (EOSC-IF)~\cite{eosc-if} is defined as a generic framework for all the entities involved in the development and deployment of EOSC.

To achieve this, a dedicated interface, coined SLICES-Interoperability Framework (SLICES-IF) shall be developed. The interface will be built upon the foundations led by the European Interoperability Reference Architecture (EIRA)~\cite{eira}, where interoperability is classified at four layers, namely: (i) technical, (ii) semantic, (iii) organizational; and (iv) legal. Although the target audience for EIRA (governance and administration) was very different from the SLICES stakeholders, core principles and objectives are similar. Additionally, the different components (in particular technical and semantic) of SLICES-IF would be chosen in such a way that SLICES is fully interoperable with EOSC for uninterrupted data exchange pertaining to use of EOSC services and research data by SLICES as well as to enable the publications of SLICES infrastructure, services and data through EOSC portal. More details about the SLICES-IF interface to EOSC and external RIs is provided in SLICES-Design Study Deliverable D4.2~\cite{slices-ds-d4.2}.

%Figure: SLICES interconnection with European e-Infrastructures and digital infrastructures



%The following paragraphs have been mentioned before in the previous section

%EOSC interoperability framework is a set of not-so-specific guidelines to ensure smooth integration of infrastructure services and seamless exchange of research data across the EOSC ecosystem. EOSC-IF is derived from the European Interoperability framework which defines interoperability of an information technology system by four key elements, i.e.., technical, semantic, organization and legal interoperability.
%The FAIR principles, federated resource and user management and legal compliances pertaining to privacy, licensing and governance by European Commission are at the core of the EOSC-IF framework.

%The interface will be built upon the foundations led by the European Interoperability Reference Architecture (EIRA), where interoperability is classified at four layers, namely: (i) technical, (ii) semantic, (iii) organizational; and (iv) legal. Although the target audience for EIRA (governance and administration) was very different from that compared to SLICES, core principles and objectives remain the same. Additionally, the different components (in particular technical and semantic) of SLICES-IF would be chosen in a such way that SLICES is fully interoperable with EOSC for uninterrupted data exchange pertaining to use of EOSC services and research data by SLICES as well as to enable the publications of SLICES infrastructure, services and data through EOSC portal. 


%\subsection{EOSC FAIR Digital Object and PID framework}
%\subsubsection{FAIR digital framework in EOSC}

%FAIR Digital Object (FDO) is the core building block of EOSC-IF. Here, Digital Object refers to the kind of objects that allow binding all critical information about any entity. In EOSC, a digital object can be research data, software, scientific workflows, hardware designs, protocols, provenance logs, publications, presentations, etc., as well as all their metadata (for the complete object and for its constituents). \cref{fig:EOSC-FAIR-digital-object-illustration} shows a schematic of FDO. An FDO should conform to all the four layers of interoperability introduced earlier in this document by following the FAIR guidelines.

%FIGURE: EOSC FAIR Digital Object illustration
%\begin{figure}[h]
    %\centering
    %\includegraphics{././figures/png/EOSC-FAIR-digital-object-illustration.png}
	%\caption{EOSC FAIR Digital Object illustration}
	%\label{fig:EOSC-FAIR-digital-object-illustration}
%\end{figure}

%\textbf{FDO (FAIR Digital Object) and PID (Persistent Identifier)} are key components of the EOSC architecture and supporting federated data infrastructure that ensures consistent implementation of the FAIR data principles. FAIR principles are realized through FAIR Digital object, and PID services and infrastructure provide facilities to access and manipulate FDO.
%According to FDO Forum, the FAIR Digital Object concept brings together FAIR guiding principles and Digital Object that supports interoperability across existing and evolving data regimes/frameworks using mechanisms of the structured machine-readable identifiers and principles of persistent binding for digital object of different types. It is important to mention that activities currently coordinated by FDO Forum are including essential results and current activities at the RDA (Research Data Alliance) on Data Types and Data Types Registries, Data Factories, others, and are aligned with EOSC technical development what is reflected in the EOSC Architecture and EOSC Interoperability Framework.
%The FDO Forum provides the following definition of the FDO \cite{fdo-forum} : 
%“A FAIR digital object is a unit composed of data and/or metadata regulated by structures or schemes, and with an assigned globally unique and persistent identifier (PID), which is findable, accessible, interoperable and reusable both by humans and computers for the reliable interpretation and processing of the data represented by the object.” 
%The following main requirements to FDO services and infrastructure defined in the FDO Framework:
%\begin{itemize}
    %\item[-]General requirements include machine actionability, technology independence, persistent binding, abstraction and structured hierarchical encapsulation, compliance with standards and community policies (as specified in the FDO general requirements G3-G9);
   % \item[-]FDO is identified by PID; there are possible multiple PID frameworks defined by PDI scheme, namespaces, ontologies or controlled vocabularies (FDOF1); 
    %\item[-]A PID resolves to a structured record (PID record) with attributes that are semantically defined within a (data) type ontology (which may be defined for different application or science domains) (FDOF2);
    %\item[-]The structured PID record includes at least a reference to the location(s) where the FDO content and the type can be accessed, and also metadata describing FDO can be retrieved (FDOF3);  
    %\item[-]Each FDO is accessed via API by specifying PID and additionally point. API must support basic operation with FDO: Create, Read, Update, Delete (referred to as CRUD), however subject to access control and policy (FDOF5, FDOF6);
   % \item[-]Metadata used to describe FDO properties should use standard semantics and registered schemes to allow machine readability and actionability (FDOF8-FDOF10).
    
%\end{itemize} 

%\subsection{SLICES FAIR Digital Object}
%SLICES wants to fully endorse and adopt the Open Science and FAIR principles, acting as a catalyst to enable and foster cutting edge research, data-driven science and scientific data-sharing. SLICES, as a research and innovation eco-system, defines appropriate metadata profiles to cater for access and reuse of FAIR data and services. However, due to the problems rising from the multi/cross/inter-disciplinary nature of research SLICES is aiming to support, opting for a union of several established, domain-specific metadata schemas and vocabularies is impractical and would decrease the efficiency and effectiveness of resource discovery, access and (re)use. 

SLICES aims to allow its users (and interoperating platforms) to uniformly find, and access any object, such as data, services and software. To accomplish this, SLICES defines a hierarchical metadata structure, where each digital object is first described using a select set of common metadata attributes and then according to its type, the description is extended with a set of type-specific attributes.
The relevant information can be then accessed using SLICES authentication and authorization mechanisms. 

%\begin{figure}[h]
    %\centering
    %\includegraphics{././figures/png/SLICES Hierarchical Metadata Structure.png}
	%\caption{SLICES hierarchical Metadata Structure}
	%\label{fig:slices-metadata}
%\end{figure}

%In particular, Level 1 includes metadata related to the SLICES core FAIR Digital Object. It includes basic information, such as identification, description and its resource type. Management information is also included such as version and metadata profile used. Using its type (e.g., data, service), Level 2 provides type-specific information. For example, a dataset may have start and end date, a facility may have an address. Finally, level 3, provides further domain-specific information that may be required by a specific community. For example, specific media details can be additionally provided using the PREMIS metadata format.

%as illustrated in Figure~\ref{fig:slices-metada}