%#############################################################################################	
\section{The ESFRI framework, a scientific instrument}
\label{section:The ESFRI framework}
%#############################################################################################

% Introduce ESFRI

ESFRI, the European Strategy Forum on Research Infrastructures, established in 2002, brings together national governments, the scientific community, and the European Commission, to support a coherent and strategy-led approach to policy making on Research Infrastructures (RIs) in Europe. The ESFRI Roadmap~\footnote{ESFRI Roadmap 2021, https://www.esfri.eu/esfri-roadmap-2021} contains the best European science facilities based on a thorough evaluation and selection procedure. The Strategy Report on Research Infrastructures 2021 includes the Roadmap 2021 and the ESFRI vision of the evolution of Research Infrastructures in Europe, addressing the mandates of the European Council, and identifying strategy goals. Since 2002, within the framework of ESFRI and the ESFRI Roadmap process, national governments have worked in close partnership with the European Commission and the scientific community to catalyse the establishment of over 50 European Research Infrastructures, mobilising investments of approximately €20 billion across the EU.

ESFRI applies a lifecycle approach to the development and implementation of RIs . The lifecycle concept describes the different milestones in the development, implementation and operation of a Research Infrastructure over time, specifying minimal key requirements that must be met to enter each stage. Application of this concept allows for a coherent assessment of the scientific and organisational maturity of Research Infrastructures across all fields of science.

%Figure: ESFRI Lifecycle.
%\begin{figure*}[h]
%    \centering
%    \includegraphics[width=0.7\textwidth]{././figures/png/ESFRI_lifecycle.png}
%	\caption{ESFRI lifecycle approach.}
%	\label{fig:ESFRI_life_cycle}
%\end{figure*}

The concept of a new RI typically emerges bottom-up from the scientific communities clustering around well identified scientific needs and goals. For RIs to remain relevant throughout the entire RI lifecycle, scientific excellence is the \emph{conditio sine qua non}, which becomes, together with adequate human resources, crucial when it comes to long-term persistence in the operational phase. Effective governance and sustainable long-term funding (public and private) are other key elements for ensuring long-term sustainability of RIs at every stage in their lifecycle. 

We observe that until 2018, ESFRI was organized in 5 WGs related to energy, environment, health and food, physical sciences and engineering, social and cultural innovation. We had to wait until 2018 to applaud the creation of a working group dealing with Data, Computing and Digital Research Infrastructures that clearly differentiates digital \emph{research} infrastructures (i.e., addressing the needs of the \emph{digital research} communities) from \emph{production} research e-infrastructures defined as ICT support to other sciences.

SLICES is a distributed research infrastructure. It is organized with a central node and a set of distributed nodes. The central node hosts the governance and resources needed to steer and run the facility. As a European facility, the distributed nodes are hosted by different member states, covering different types of needs. At the time of writing, we have 15 countries supporting the effort~\footnote{Slices, https://slices-ri.eu/community/}. Industry is supporting but not directly involved in contributing to a common good.

%Figure: ESFRI Partnership.
%\begin{figure*}[h]
%   \centering
%   \includegraphics[width=\textwidth]{././figures/png/ESFRI-partnership.png}%	\caption{ESFRI Partnership}
%	\label{fig:ESFRI-Partnership}
%\end{figure}