%#############################################################################################    
 \section{Full research life-cycle example}
 \label{section:Example}
% %#############################################################################################
% The research process is iterative by nature and allows scientific model improvement by using continuous research cycle that typically includes the following basic stages:
% \begin{itemize}
%  \item[-]Define the research questions. 
%  \item[-]Design experiment representing initial model of research object or phenomena. 
%  \item[-]Collect Data.
%  \item[-]Analyse Data.
%  \item[-]Identify Patterns.
%  \item[-]Hypothesise Explanation.
%  \item[-]Test Hypothesis.
%  \item[-]Refine model and start new experiment cycle.
%  \end{itemize}
% The traditional research process may be concluded with the scientific publication and archiving of collected data. Data driven and data powered/driven research paradigm allows research data re-use and combining them with other linked data sets to reveal new relations between initially not linked processes and phenomena. 


In order to realize the vision of FAIR research, supporting the full research lifecycle, lets consider a simple example borrowed from another field of research and illustrated by the Reliance project~\footnote{Reliance, https://www.reliance-project.eu/}. Reliance delivers a suite of innovative and interconnected services that extend EOSC’s capabilities to support the management of the research lifecycle within Earth Science Communities and Copernicus Users. Consider core services provided to the research community, that could be data, software publications, others. These core services are named after research objects that are for use by the experimenters and share by the experimenters. As an illustration, assume that you are doing some research related to the Copernicus air quality. You can go to the OpenAIRE~\footnote{OpenAIRE, https://explore.openaire.eu/} explorer and search for Copernicus quality. And you will find all the associated resources as described in \Cref{fig:OpenAIRE-EXPLORE}. 
%Figure: SLICES interface to EOSC
\begin{figure*}[t]
    \centering
    %\includegraphics[width=0.8\textwidth]{././figures/png/OpenAIRE-EXPLORE.png}
    \includegraphics[width=0.9\textwidth]{././figures/png/openaire.png}
	\caption{OpenAIRE Explore}
	\label{fig:OpenAIRE-EXPLORE}
\end{figure*}
You are looking for a software, because someone has produced a software taking research data as input and producing a map of the air quality in a given region as an output. You find the software and with the software comes a set of additional metadata. So for instance, it could be a Jupyter Notebook as in \Cref{fig:EGI-Notebook}.
%Figure: SLICES interface to EOSC
\begin{figure*}[t]
    \centering
    \includegraphics[width=0.7\textwidth]{././figures/png/EGI_Notebook.png}
	\caption{EGI Notebook}
	\label{fig:EGI-Notebook}
\end{figure*}
You now have access to the software that will execute exactly what has produced this research data. What you are willing to do, at first, is to reproduce the results. On the other hand, you would like to take your own data, use the same process and produce your own new results. The last step is that you go to the service, which is named Rohub ~\footnote{Rohub, https://reliance.rohub.org/}. And then you bundle your different resources, like the Jupyter notebook that you have used, the data that you have exploited, and the output that you have produced. You can now publish this research outcome as your own contribution made available to the community, defined as PM 10 in \Cref{fig:Jupyter-notebook}. 
%Figure: SLICES interface to EOSC
\begin{figure*}[t]
    \centering
    \includegraphics[width=\columnwidth]{././figures/png/Jupyter_notebook.png}
	\caption{Jupyter notebook }
	\label{fig:Jupyter-notebook}
\end{figure*}
This full research-life cycle is really important, otherwise, the result that you produce cannot be published, because it simply cannot be reproduced. There exists data initiatives in our field, like the ACM Artifact Review level \cite{acm-artifact}, it is nice and ambitious. But it does not yet fully align with the best practices in other fields of research.

