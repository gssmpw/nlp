
%#############################################################################################
\section{Conclusion}
\label{section:conclusion}
%#############################################################################################
It is a best practice in fundamental sciences to think about thought experiments that will validate the scientific assumptions. It is indeed a challenging endeavor to design a test platform to support networking and distributed research. Up to now, networking test-beds have tried to capture a variety of demands. However, very little has been done to cover the entire research and data lineage life-cycle. SLICES is the outcome of an effort to align the methodology to build such a platform in order to satisfy the key requirements of a scientific instrument. In Europe, ESFRI provides such a framework where most of the large research infrastructures are incubated, deployed and operated. 
The paper describes our continuous work aiming at designing the SLICES end-to-end reference architecture. It emphasized the analysis of the current demand from relevant ICT stakeholders, and the foundational principles on which it will be grounded. These principles, alongside with the current trends in resource management (resource programmability, network virtualization, resource disaggregation) have resulted in the wide adoption of several Management and Orchestration (MANO) frameworks for deploying experiments and applications over distributed infrastructures. The paper also discusses major open-source software that is used by the research community for networking at large, experiments that provides opportunities for SLICES. The integration and interoperability with EOSC infrastructure are also presented and the research life-cyle is illustrated. This work is based on the long experience of the participating members in managing and operating test platforms infrastructures to best serve our research community.