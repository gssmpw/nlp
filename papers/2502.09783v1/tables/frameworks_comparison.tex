%\setcounter{table}{2}
\begin{table*}[t]
\centering
\begin{tabular}{|>{\columncolor{red!40!white!70}} p{1.3in}|p{1in} |p{4.2in} |}
    \hline
    \rowcolor{red!40!white!70}
      \textbf{Existing Tools}   &  
      \textbf{Proposed Solution(s)}  & 
      \textbf{Benefits} \\\hline
      
      \textbf{Control and Management Framework (OMF)} \cite{omf}   &
      NFV-based orchestration solution & 
      Current tools provide metal as a service access to the testbed resources, or in some cases virtualized access by interacting with the respective VIM interface of a testbed. On top, the experiments can be orchestrated by using a publish/subscribe scheme for the communication between a centralized controller and the actual resources. Adopting an NFV-based solution will allow the orchestration of experiments as Virtual Network Functions (virtualized access) or Physical Network Functions (Metal as a Service access), through the adoption of industry-grade tools. These shall allow higher utilization of the testbed resources, increasing the user capacity of each testbed, more secure end-to-end experiments, end-to-end network configuration and experiment reliability.  \\\hline
      
      \textbf{SDN programmability}    &
      SDN Assist &  
      Current tools aim in providing a programmable interface for users that shall use their own controller for managing the flows in the network. In some cases, isolation of flows between different users on a switch is possible, through the adoption of tools like FlowVisor . Moving to an NFV-based orchestration solution supporting features like SDN Assist  enables the programming of flows for an experiment during the instantiation time. Based on an end-to-end programmable SDN plane (based on Open-vSwitch or hardware OpenFlow/P4 switches) programmability extends to the entire datapath used for the experiments, isolating users and providing multi-tenancy over the infrastructure.\\\hline
      
      \textbf{Wireless programmability}    &
      Open-RAN (ORAN) & 
      Current tools for programming the wireless components rely on specific interfaces dedicated to specific equipment for the RAN. As such interfaces become standardized, through efforts like O-RAN alliance, adopting such APIs can increase the supported equipment, and open-up more programmability for the RAN. As such tools use standardized interfaces, they integrate with several NFV based orchestration solutions, allowing a truly end-to-end experiment configuration and instantiation. \\\hline
      
      \textbf{Edge and Core Configuration}    &
      NFV-based orchestration solution & 
      Current tools provide Virtualized access to the core and cloud network configuration, or in some cases metal-as-a-service access. Switching to the same NFV based orchestration solution as the rest of the nodes will enable the seamless network configuration, and move the edge/core cloud configuration to supporting a different number of settings (such as cloud-native 5G network configuration).  \\\hline
      
\end{tabular}
\caption{Comparison of different proposed frameworks vs existing ones for the SLICES architecture}
\label{tab:frameworks-comparison}
\end{table*}