%!TEX root=main.tex

\begin{figure}
    \centering
    \includegraphics[width=0.8\textwidth]{figures/docetl_cropped1.png}
    \caption{A potential interface for the data system. The user provides a task to be performed on a set of documents, here a collection of legal documents.}
    \label{fig:docetl}
\end{figure}

\begin{figure*}[t!]
\centering
\begin{minipage}{0.55\textwidth}
    \centering
    \includegraphics[width=\textwidth]{figures/proactive.pdf}
    \vspace{-15pt}
    \caption{A Proactive Data System (red = LLM-powered)}
    \label{fig:proactive}
\end{minipage}
\begin{minipage}{0.4\textwidth}
    \centering
    \includegraphics[width=0.7\textwidth]{figures/reactive.pdf}
    \vspace{-10pt}
    \caption{A Reactive Data System}
    \label{fig:reactive}
\end{minipage}
\vspace{-10pt}
\end{figure*}

\section{Typical User Workflows with Proactive Data Systems}

A proactive data system
understands, at a deeper semantic level, 
user intent,
the specific operations it performs,
and the data it performs operations on.
Typical user workflows with 
such a system are similar to
traditional database systems,
where the user
first provides (or ingests)
the data, potentially
alongside additional descriptive
information about its content or schema,
and then proceeds
to execute queries on this data.
We describe this workflow in more detail next,
while acknowledging that
a range of design choices may all
be appropriate,
depending on the use cases. 

\topic{Data Definition} 
The user first ingests
or registers their data (e.g.,
a PDF document collection of incident reports, police officer employment CSVs,
as well as audio/video of incidents 
in the police misconduct case),
along with metadata if any.
Unlike traditional database systems,
this metadata is optional,
and the system
is free to proactively understand the 
structure and semantics of the data.
That said, any 
user-provided specification or description
can help improve
accuracy and better specialize the
system for specific use-cases
of interest.
In Example~\ref{ex:police}, 
such a specification can be similar to 
the first paragraph of Example~\ref{ex:police} as plain text, 
or could include definitions about technical terms or
additional background providing domain knowledge,
e.g., defining police misconduct. 
We note that such information is often needed to allow 
users to query the data meaningfully 
even in relational databases
when using text-to-SQL~\cite{li2024can}. 
We also envision that
in certain use-cases, the users
may proactively identify the entities
of interest for downstream querying,
even if they don't register
the attributes they may care about in the future.
For example, in the police misconduct setting,
the users may want to indicate
that they intend to analyze information
about incidents, police officers,
and agencies.
Armed with all of this information, 
the system
can proactively add indexes,
reorganize the data, 
and extract
certain fields,
among other such offline actions,
as we will describe in Section~\ref{sec:data}.

\topic{Task Specification}
Users are then free
to issue queries or tasks on the data,
which can be specified
in natural language,
or a combination of natural language prompts
associated with data processing operators,
as shown in Figure~\ref{fig:docetl}
for a map operation on a collection of contracts,
with a prompt
extracting a number of legal clauses,
within DocWrangler, our IDE for DocETL~\cite{shankar2024docetl}.
If the user instead chose to preregister
a schema during the data definition
stage, they are free
to extend it with additional
LLM-populated attributes
and issue SQL queries
based on these lazily populated attributes,
as we do in ZenDB~\cite{lin2024towards}.

\topic{Task Execution} 
The system then performs the task on the underlying data.
Rather than performing the task
as is on predefined data monoliths,
a proactive data system
will try to understand
the task and the data,
performing reformulations
of this task by decomposing 
the corresponding operators,
as well as the data,
all in an effort to maximize
accuracy and minimize cost.
For example, the system can decompose 
the user-provided task 
into smaller easier-to-do operations, 
and can leverage the semantics of the document(s) 
to intelligently focus the LLM's attention on 
relevant portions. 


\smallskip
\noindent In Section~\ref{sec:op} 
we discuss various ways for a proactive 
data system to understand 
user-defined operations and reformulate them, 
and we extend the discussion in Section~\ref{sec:data} 
on how the system can similarly understand the data 
and perform data transformations to improve accuracy. 
In Section~\ref{sec:intent}, we discuss how 
the data system can ensure the task 
was performed as the user intended.   
When it comes to unstructured data, we
focus our attention mostly on documents
for concreteness,
though our general approach may be applicable
to a variety of formats.

\smallskip
\noindent
To further differentiate a proactive data system 
from a reactive one, Figs.~\ref{fig:proactive} 
and \ref{fig:reactive} provide a high-level overview 
of different components in these systems. 
While a reactive data system considers tasks and data as is, 
and performs operations on the components as instructed, 
a proactive data system leverages LLMs to understand 
user intent, 
the operations it performs, as well as 
the data to ensure maximal accuracy at minimum cost.   



% A proactive data system understands the user's intent, the specific operations it performs and the data it performs the operations on. The user interactions with the system is similar to the traditional database systems, where the user first provides the data, potentially with additional descriptions about its content,  and then proceeds to execute tasks on the data. We describe this workflow in more details.

% \textbf{Data Definition}. The user first inputs their data, potentially with any additional specifications describing their data. This step is akin to traditional data management systems where the user needs first define a schema before inserting their data. In a proactive data system, the user does not need to provide a specification of the data ---the system itself can proactively understand the data---, but any additional description can help improve accuracy and better specialize the system for the specific use-cases of the user. Such specifications can be in terms of the content of the data. For example, in Example~\ref{ex:police}, such specification can be similar to providing a description similar to what is stated in Example~\ref{ex:police} as plain text), or providing definitions about technical terms in the data that require domain knowledge (for examplem defining what police misconduct means). We note that such information is often needed to allow users to query the data meaningfully even in relational database \cite{li2024can}. Beside data description, the user can provide the data, in typical data formats, e.g., CSV or json files, or as a relational database.  

% \textbf{Task Definition}. The user can then move on to performing tasks on the data. This can be done using a natural language interface, such as the one shown in Fig.~\ref{fig:docetl}, where the user provides a prompt to be executed on the dataset. The user can alternatively define a desired schema and perform SQL queries on it (e.g., as done in \cite{lin2024towards}), but differing information extraction from the data based on the schema to the data system.  

% \textbf{Execution}. The system then performs the task on the data. A proactive data system, rather than performing the task as is on predefined data monoliths, tries to understand the task and the data, performing reformulations of the task and the data to maximize accuracy and minimize cost. For example, the system can decompose the user-provided task into easier-to-do subtasks, and can utilize the semantics of the document to intelligently route the task to relevant portions. In Sec.~\ref{sec:op} we discuss various means for a proactive data system to understand user-defined operations and reformulate them, and we extend the discussion in Sec.~\ref{sec:data} on how the system can similarly understand the data and perform data transformations to improve accuracy. Moreover, in Sec.~\ref{sec:intent}, we discuss how the data system can ensure the task was performed as the user intended.   

% To further differentiate a proactive data system from a reactive data system, Figs.~\ref{fig:proactive} provides a high-level over view of different components in the systems. While a reactive data system considers tasks and data as is, and performs operations on the components as instructed, a reactive data system understand user intents, the operations it performs as well as the data to ensure maximal accuracy at minimum const.   





%In such a system, the user provides a task as the input in natural language, describing a query (e.g., find police reports describing instances of police misconduct) or, more broadly, a data transformation (e.g., summarize police reports for a certain case). The system then tries to understand the specific user intent behind the task, and translates the given task into specific operations (potentially LLM operations) to be performed on the data. The system unboxes the given operations and the data, deciding how to parse the operations and the data into digestible units best suited to maximizing accuracy and minimizing cost. This can mean transformations on the data to identify and extract relationships between semantic units, or to create new data representations such as summaries, indexes and tables; and transformation on the operations to rewrite and decompose them into new operations that can be better executed using LLMs.  

%The three components of the system are then (1) understanding user intent and translating it into specific operations to be executed on the data (2) understanding operations and reformulating them to maximize accuracy and (3) understanding the data and performing data transformations to aid query answering. The remainder of this paper discusses operation understanding and reformulation in Sec.~\ref{sec:op}, data understanding and transformations in Sec.~\ref{sec:data} and understanding user intent in Sec.~\ref{sec:intent}.




\if 0
\agp{This needs to be more like a high-level architecture
section, or can be folded into introduction}

\sep{Probably neeed more discussion of the application domain of the system here}
A proactive data system understands the task it's given, the dataset it's performing and the intent of the user. Such a system semantically parses the tasks and the data it's provided, and maps them to internal representations it deems best suited for performing the tasks. Such mappings can include rewrites and decomposition of the task or the data. It furthermore interacts with the user to ensure correctness of the result, based on which it can update internal representations it has chosen. 


\sep{WE LIKELY WANT TO RENAME/resturcture SECTIONS}
\fi



\if 0
\sep{Suggestion Structure:}
Each subsection is centered around tools we have to improve performance, either by understanding the task, the data or the user better

2. 1. Proactive Task Understanding 
2.1.1. Breaking down complex tasks (creating multiple tasks from a task)
2.1.2. Semantic Parsing (creating a semantic representation of the task to better map to the data)
2.1.3. Task Reformulation (rewriting a new task that the LLM performs better, can be prompt optimization)
2.1.4. Other Potential Directions

2. 2. Proactive Data Understanding 
2.2.1. Breaking down complex data (decomposing into sections, chunking)
2.2.2. Semantic Parsing (creating a semantic representation of the data to better map to the tasks)
2.2.3. Data Reformulation (rewriting a data into a new form, e.g., templatized extraction)
2.2.4. Other Potential Directions

2. 3. Proactive User Understanding 
2. 3. 1. Different interfaces to understand user 
2. 3. 2. How to utilize user feedback within the system (i.e., how it changes the above two sections) 
\fi


%!TEX root=../main.tex


\section{Proactive Operation Understanding}\label{sec:op}
In reactive data systems, the onus
is on the user to author queries involving
the ``right'' LLM-powered operators,
with the system then determining how to execute
these in conjunction with other relational operators.
However, even in cases where users
are able to specify clear, unambiguous operations,
the granularity at which they specify them may
not be optimal for execution.
Fundamentally, this stems from a lack
of understanding of what LLMs can do well
versus what they can't---something most users
are not aware of.
In this section, we discuss various 
approaches to 
operation reformulation 
that can improve accuracy 
while maintaining or reducing computational costs.
We first describe new operators that
we may introduce, and then methods for 
assessing and improving cost and accuracy when leveraging
new or existing operators.

\subsection{How and Where to Introduce New Operators}
We now describe ways to reformulate existing LLM-powered
operations into different ones. 

\topic{Decomposition into simpler LLM operations}
In Example~\ref{ex:police}, we described
how sometimes journalists want to extract
dozens of fields from a given document
(e.g., police officer names, descriptions
of misconduct, locations, use of firearms, among others).
The journalist may specify the entire list of fields
along with instructions within a prompt.
However, executing this as is in a single LLM call
may lead to poor accuracies
as LLMs often struggle to identify multiple concepts simultaneously~\cite{shankar2024docetl}.
A proactive system can decompose such an operation
into separate focused ones that
each extract one type of information, improving
accuracy.
An LLM can be asked rewrite a prompt that
says ``extract fields f$_1$, ..., f$_n$ from the following ...''
into ``extract field f$_i$ from the following ...''.
In prior work, we have identified
several new decomposition-based rewrite rules~\cite{shankar2024docetl}
that can lead to higher accuracies,
when coupled with LLMs being used to instantiate
the rewrites themselves.
Recent work on Text-2-SQL also leverages 
similar ideas~\cite{pourreza2024chase}: 
rather than attempting translation 
in ``one-shot'' (i.e., single LLM call), 
which often fails due to schema mismatches 
or poorly-named schemas~\cite{luoma2025snails}, 
it is beneficial to 
parse the operation into smaller units, e.g.,
given ``find all employees who joined before 2020'', 
first identify the core concepts: employee records 
and hire dates,
and treat each separately.

\topic{Leveraging non-LLM components}
In certain cases, operations
that are assigned to an LLM may be better done
through other means, e.g., through SQL.
For example, finding the average settlement amount 
for misconduct cases can be decomposed 
into an LLM operation to 
identify relevant cases and their settlement amounts, 
followed by a SQL aggregation to compute the average.
We have employed similar techniques where
we separate operations that require real-world reasoning (suited for LLMs) from mathematical computations (better handled by traditional database engines or calculators) in
our work on ZenDB~\cite{lin2024towards}.
Determining how to do this automatically
is challenging. 

\topic{Leveraging reasoning or data feedback for reformulation}
As described avove, a proactive data system
must reformulate (e.g., decompose or rework) operations to 
execute them better. 
However, finding good reformulations is difficult. 
Current approaches to discovering good reformulations 
or decompositions
are quite naive. Systems like DocETL~\cite{shankar2024docetl} 
simply prompt LLMs to suggest rewrites, 
either taking the first 
suggestion or selecting from multiple candidates. One option is 
to use a powerful ``reasoning'' model like OpenAI's o1 model to 
rewrite the task, but this still fundamentally relies on one-shot 
prompting, and is unaware of how the rewrite will actually perform on the data. We need approaches that can learn what 
characteristics of the data make tasks challenging, what types of 
LLM errors occur in different contexts, and use this knowledge to 
guide reformulation---perhaps even in an agentic fashion.

\topic{Calibrating LLM outputs}
Independent of which decomposition or reformulation is used,
when LLMs are independently being applied
to a set of items (documents, tuples),
the outputs can often be
inconsistent and non-calibrated.
For example, if we ask LLMs to rate
the severity of every incident
in our document collection,
it often gives all of them the same score,
or worse, gives them scores that
are only losely correlated with the severity.
To remedy this lack of consistency,
we can take various actions.
We can leverage the LLMs themselves to pick 
representative examples that indicate
the full range of the categories of interest, provided
as few-shot examples.
Or they can rework the prompt to describe in more detail
the criteria used for evaluation---to ensure consistency.
Finally, they can also
restrict the space of possible outputs
(e.g., when LLMs are asked to extract state information
from a collection of US addresses, they may extract CA in some
cases and California in others).
While this doesn't change the semantic meaning of the operation, it can significantly improve LLM accuracy by providing better context and guidance, as has
also been explored in work on prompt optimization~\cite{khattab2023dspy}.

\subsection{Assessing and Improving Performance with Reformulation}
Next, we describe ways to 
assess the benefits of reformulations,
and reduce cost while preserving accuracy. 

\topic{Leveraging LLMs to assess benefits}
One question that naturally emerges when considering
decomposing operations into smaller units
is how to assess the benefits of such decompositions.
While it is a-priori hard to tell whether
a decomposition will help, we can run
both the non-decomposed and decomposed
variants on a sample. 
LLMs are much better at evaluating outputs
than generating them, so LLMs can be used
to tell which version performs better.
For example, 
one can employ a ``generate-fix \& rewrite-verify'' pattern: 
generate an initial operation formulation, 
verify its correctness (e.g., through automated checks or LLM 
verification), and if verification fails, attempt alternative 
formulations~\cite{chung2025long}. 
This pattern, 
which we also use in DocETL~\cite{shankar2024docetl}, 
allows a system to systematically explore the space of possible 
formulations until finding one 
that passes verification, effectively optimizing for accuracy 
through trial and refinement.
However, doing evaluation in a cost-effective manner remains
a challenge. 





\topic{Deferring to cheaper LLMs for the ``easy bits''}
One concern with decomposition is that
it may increase the cost of the overall pipeline,
since one LLM call per document may now become
multiple. 
One way to defray the cost is to couple decomposition
with cost optimization:
for the simpler newly decomposed operations,
we can alternatively use cheaper and smaller
models to handle them, and only use the more expensive model 
for the most complicated operations.
For example, when we're trying to extract many fields
from a police record document, we can use a cheaper
model for extraction of locations and dates, while using
a more expensive model for harder tasks such as
determining the type of misconduct incident.
While the idea of cheaper proxy models isn't new~\cite{kang2017noscope, patel2024lotus},
here, since the space of decompositions is infinite,
and for each decomposition (or sequences thereof), 
we could use different
models and different confidence thresholds, each with 
different cost-accuracy tradeoffs, the problem
becomes a lot more compliacated.
Additionally, unlike previous settings which focused primarily on 
tasks with well-defined accuracy metrics, we now must provide 
guarantees for open-ended generative operations---where quality 
is harder to quantify.

 
\topic{Expensive predicate ordering, but with synthesized predicates}
For operations that involve subselecting documents
based on certain criteria (all expressed
together in one prompt), we can leverage existing
related work on expensive predicate ordering~\cite{hellerstein1993predicate, raman1999online};
however, in our context, we can introduce an arbitrary
number of new dependent predicates (that are potentially easier
to check and therefore cheaper).
For example, instead of using an expensive model
to examine each police record document to extract
medical impacts to the victims, if any,
we can consider cheaper filters
that are easier to check, for example, 
if the document contains any medical information at all.
This check could potentially be done by a cheaper
model and rule out a substantial fraction of the documents.
Similarly, when decomposing a complex filter like ``find 
incidents involving both use of force and drug use'' into two 
filters, one for ``use of force'' and one for ``drug use,'' the 
system can evaluate the more selective filter first to minimize expensive LLM calls.









%\topic{Coalescing into more complex operations}




















%!TEX root=../main.tex

%\subsection{Extracting Document Structures}
\section{Proactive Data Understanding}\label{sec:data}
Proactive data systems take initiative to truly
understand the data, rather than simply treating
it as inputs to opaque UDF (here LLM) calls.
It can leverage
the provided data descriptions,
as well as actual content,
to create representations
that are useful for downstream data
processing tasks.
The system can understand 
each document on its own (Section~\ref{subsec:structure}), 
understand relationships between documents
or portions thereof (Section~\ref{subsec:cross-doc}), 
or preprocess documents based on anticipated future tasks (Section~\ref{subsec:taskaware}).


\subsection{Identifying Semantic Structure 
within a Document}
\label{subsec:structure}

%Unlike tables in relational databases, where schema and rows are well-defined, real-world unstructured documents present significant challenges for analytics due to the free-form nature of natural language and complex visual formatting. 

Although documents may appear unstructured, 
they often are semantically structured.
This structure may
be implicit in the text, e.g.,
content in adjoining portions of the text is often related. 
They can also be explicit, e.g., 
tables or figures embedded within a PDF document. 
We discuss how to identify,
extract, and leverage hidden structure 
from unstructured documents. 

\begin{figure}[tb]
    \centering
    \vspace{-20pt}
    \includegraphics[width=0.7\linewidth]{figures/semantic_structure.pdf}
    \vspace{-10pt}
    \caption{Semantic Hierarchy in a Civic Agenda Document (a) the Document itself (b) the Corresponding
    ``Table of Contents'' (c) the Corresponding Semantic Hierarchy.
    }
    \label{fig:civic}
    \vspace{-10pt}
\end{figure}

\topic{Leveraging implicit hierarchical structure} 
Portions of documents are often semantically related. 
A section or subsection within a document often
contains information that is semantically related,
while other parts are less related or unrelated. 
For example, the medical examiner report within
a broader police record PDF contains
most of the medically relevant information
about an incident, while the eyewitness report
contains most of the relevant information from eyewitnesses.
Identifying these subdivisions within a document
and routing a query to the subdivision at the
``right'' granularity can lead to higher accuracy
than both RAG or providing the entire document to an LLM~\cite{lin2024towards}.
This structure is best represented
as a semantic hierarchy.
There are various ways to construct
such a hierarchy,
including leveraging formatting information
that distinguishes headers from other portions, 
or using an LLM to identify which phrases
may be headers as we do in ZenDB~\cite{lin2024towards}---see 
Figure~\ref{fig:civic} for an example.
Another approach is to construct this semantic hierarchy on
content alone, where summaries of related chunks are merged
and recursively summarized~\cite{sarthi2024raptor}.  
Nonetheless, building semantic structures 
that are useful for downstream tasks remains a challenge, 
as different views of the document may be useful for different tasks,
 where even simple information such as location can have different connotations. 
 For instance, when organizing police activities in a specific case based on location 
 they occurred in, a user might be interested in geographical location 
 of activities (i.e., at a specific address) 
 while another user might be interested in types of locations (e.g., if the police activity was outdoor or inside). 
 The system needs to consider various possible semantics of the data when identifying the semantic structure. 

\topic{Leveraging explicit structure} 
Unstructured documents often contain structured
portions, such as embedded
tables and key-value pairs.
Treating them as plain text for data processing is ineffective
and error-prone.
For example, if we're not careful in preserving visual information, 
a missing value in a key-value pair
could lead to the next key being misinterpreted
as the corresponding value.
Moreover, depending on the approach used
to query such tables,
we may lose visual information (used to show table structure and group columns and rows),
and be unable to effectively process
numerical information. 
A proactive data system
therefore will identify and extract such structured portions
and represent them in a structured format, for example, as tabular data or key-value pairs in Figure~\ref{fig:force},
while preserving their context within the document (e.g., their location and semantic relationships to the rest of the document).
Our recent tool, TWIX~\cite{lin2025twix}, proposes an efficient approach 
for automatically extracting structured portions from documents, using a combination of visual
and LLM-based inference, while preserving this context for the extracted
information. 
However, many challenges remain, such as accurately representing the semantic 
relationship between structured and unstructured document portions, 
e.g., to understand which queries should be answered 
based on the structured and unstructured portions,
and how much background context is necessary to make sense of the structured portions.






\subsection{Identifying Cross-Document Relationships}\label{subsec:cross-doc}
There are multiple reasons
to perform cross-document organization.

\topic{Identifying documents that may be queried together}
Beyond understanding structure within a single document, 
it is important to understand relationships across documents,
since these related documents may often be queried together. 
In Example~\ref{ex:police}, the dataset, a single incident
can span several PDF documents, often without such information being linked to each other. 
The data system needs to proactively identify relationships 
between such documents to organize the data prior to querying. 
This, for instance, can be done by clustering the documents. 
However, clustering is challenging, since the system needs to understand
the documents to be able to cluster them properly.
Simply embedding the documents, and clustering the embeddings 
does not work since the documents can vary considerably in length.
Another approach is to leverage LLMs to check if two documents
correspond to the same incident, but this is expensive,
especially when there are $O(n^2)$ comparisons.
We may be able to leverage LLMs to identify cheaper proxies
or blocking rules (e.g., two documents
may not be related unless the date ranges overlap)
for this organization. 
In some settings, folder organization provides cues for identifying
cross-document relationships (e.g., documents that are very ``far apart''
from a folder structure standpoint may be unlikely to be related).



\begin{figure}[t]
    \centering
    \vspace{-20pt}
    \includegraphics[width=0.7\linewidth]{figures/structured_view.pdf}
    \vspace{-10pt}
    \caption{Tables and Key-Value Pairs in Use of Force Records. 
    }
    \vspace{-10pt}

    \label{fig:force}
\end{figure}

\topic{Identifying shared templates across documents}
A separate concern is to combine semantic hierarchy construction
with cross-document relationships, so that we are able
to identify shared ``templates'' across documents.
These templates can both help scale up extraction across
documents, but also help identify documents whose structure
differs considerably.
For example, journalists may want to identify incidents
where there is an internal affairs report within
a broader police record document,
since these are ones where there is a corresponding
disciplinary action. 

\subsection{Task-Aware Data Pre-Processing}\label{subsec:taskaware}
The system can attempt to proactively find and organize portions
of the data that will be useful to improve performance on a reasonable subset of data processing tasks downstream. 
Given that document collections 
can span in the millions, 
it can be expensive to do extensive processing of the data upfront, for instance,
by populating a materialized view with all the attributes a user can ever hope to query;
it can also be  time-consuming to leave all the data processing to when the user issues a task. 
As such the system needs to decide how much preprocessing is beneficial upfront, and what to perform at query time. 
To strike a balance between the two extremes, 
one option is for the system to identify and/or extract data units 
that it deems to be useful in the future for a wide variety of queries. 
This can be done by understanding the semantics of the data. 
For instance, in Example~\ref{ex:police}, the system can decide that sections 
that describe police incidents at a high level (e.g., the internal affairs report)
are typically useful for future task processing as they provide a comprehensive summary of most
relevant aspects. 
The system can keep pointers to such sections as lightweight indexes, 
but leave more specific data processing to when the user issues queries. 
Similarly, the system can do schema identification in advance 
to find what type of data is represented in the documents, 
and use the identified schema to answer queries. 
The system can decide whether to extract information 
upfront to populate the schema, 
or keep pointers to where the information 
can be found at query time. 
For instance, the system may choose to retain pointers
to all portions that mention police officers in the document
so as to accelerate analysis of those aspects downstream,
without going all the way to populating a materialized
view with officer attributes (since these can vary 
depending on user need).



% %,  semantic hierarchy present in the documents in the . Documents often contain hidden structures that can be leveraged for more effective data analytics. 

% \if 0
% Many documents within collections are created using templates, which are common across various domains, including civic agenda reports, scientific papers, employee job descriptions, and notices of violations. Such a template provides semantic hierarchy, which can be modeled as a tree~\cite{lin2024towards}, where each node represents a text portion (e.g., a section or subsection), and edges denote inclusion relationships (e.g., Section 2.1 is a subsection of Section 2).  

% The font features (e.g., font name and size) are often consistently identified within the same semantic hierarchies (e.g., header names share the same font features).  We can maintain summaries of node text spans along with hierarchical inclusions (e.g., ancestor nodes of the current node) to provide a more structured and logical data representation for analytics, akin to a table of contents. Query processing using this semantic structure resembles how users navigate a table of contents to locate relevant sections for their questions.  

% In the police misconduct example, suppose a journalist wants to determine the dates of misconduct records where force was used. The documents are structured with different types of misconduct as section headers (e.g., Corruption or Excessive Force), followed by detailed case descriptions in each section.  

% A semantic hierarchy helps locate cases under ``Excessive Force'' without relying on heuristic physical chunking to identify relevant text portions.  ~\cite{lin2024towards} presents an efficient method for uncovering and constructing semantic hierarchies across documents, demonstrating that it improves query accuracy while maintaining low cost and latency.  
% \fi



% %Given a query task, a proactive database system should analyze, decompose, abstract, or reformulate complex data to better suit downstream analytics, rather than treating the entire data object, such as a document, as a single unit. Taking unstructured documents as example,  Below, we present several use cases involving documents that range from highly structured to completely unstructured, with loosely structured cases in between. We also introduce several ideas for processing complex data based on their hidden structures.  

% %Below, we present two use cases to illustrate the structures within documents and their importance in improving downstream query analytics.  





% %Analyzing an entire complex document is not an effective approach. A proactive system should decompose complex data to extract the most relevant information for answering downstream queries. However, achieving this in a principled manner remains a challenging task. We first introduce a type of document with semantic structures that can be leveraged to break down complex documents into semantic portions of varying granularity. We demonstrate that this approach effectively supports complex queries on such documents.  




% % \noindent{\em \bf Use case 1: Civic Project Agenda Report Analysis.} Journalists at Big Local News at Stanford have collected large volumes of civic meeting agenda PDF reports from various U.S. counties, as shown in Figure~\ref{fig:civic}-a, and seek to analyze these reports.  One such query could be to count the number of construction projects of a certain type, across meetings, e.g., ``What is the number of Capital Improvement projects that started after 2022?''. 

% % To achieve this, one could use Large Language Models (LLMs). However, even advanced LLMs like GPT-4 struggle with queries on such reports, especially when involving aggregations and multiple filters over long documents. This limitation is expected, as LLMs are not well-suited for handling large contexts or complex data processing tasks. 
% % Another strategy, Retrieval-Augmented Generation (RAG), segments documents upfront at a certain granularity and then, during querying, identifies $k$ segments most relevant to the query (e.g., via embedding distance), incorporating only these segments into prompts to reduce cost.  
% % However, RAG struggles to identify the appropriate segments, even for simple queries. Suppose we want to identify capital improvement projects. RAG retrieves segments that most closely match ``capital improvement projects'' within the document, such as the red box in Figure~\ref{fig:civic}-a, but fails to capture over 20 additional projects on subsequent pages, such as the ``PCH Median Improvement Project'' (B2 in Figure~\ref{fig:civic}-b), which belongs to ``Capital Improvement Projects'' (A1).  
% % Overall, both the vanilla LLM approach and RAG are unsuitable: they have low accuracy, while the LLM approach additionally incurs high costs.  
 

% % \noindent{\bf Semantic Structure Helps. } 
% % The reason RAG did not perform well above is that the text segments provided to the LLM did not leverage the semantic structure underlying the document. Instead, if we are aware of this structure, we can identify the capital improvement projects (A1 in Figure~\ref{fig:civic}-b) by checking all its subportions (e.g., B1, B2), where each corresponds to a project description, and provide this information to the LLM for interpretation.  
% % By doing so, we {\em provide all pertinent information to the LLM, unlike RAG, while avoiding overwhelming it with excessive context}.  A semantic hierarchical structure provides a logical and principled way to decompose a complex document into smaller text portions of varying granularity. It offers valuable properties for retrieving relevant information for queries. For example, a child node (e.g., B1) in such a tree naturally inherits the properties of its ancestor node (e.g., B1 is a Capital Improvement Project because it is a child of A1). Query processing using this semantic structure resembles how users navigate a table of contents to locate relevant sections for their questions.  

% %To effectively construct such a structure for documents, a key observation is that while unstructured documents vary considerably in format, many documents within collections are created using templates, known as \textit{templatized documents}~\cite{lin2024towards}.   Templatized documents are common across various domains, including civic agenda reports, scientific papers, employee job descriptions, and notices of violations.  ~\cite{lin2024towards} presents an efficient method for constructing semantic hierarchical structures across documents and demonstrates that leveraging such structures can achieve up to 31$\times$ cost savings compared to LLM-based baselines while maintaining or improving accuracy. Additionally, it outperforms RAG-based baselines by up to 61\% in precision and 80\% in recall, with only a marginally higher cost.  

% %This use case demonstrates that by effectively analyzing the underlying semantic structures in a document, it becomes possible to break it down into smaller portions, reducing costs and improving accuracy for LLMs in downstream tasks.  

% \subsection{Identifying Cross-Document Relationships}\label{subsec:cross-doc}
% Beyond understanding information within a single document, it is important to understand the relationship across documents. In Example~\ref{ex:police}, the dataset, even considering a single incident, consists of various documents such as testimonials, police reports, medical reports, etc, often without such information being linked to each other. The data system needs to proactively identify relationships between the documents to be able to organize the data. This, for instance, can be done by clustering the documents. Clustering such complex documents is challenging since the system needs to understand the documents. For example, to check if two police reports refer to the same case, the system needs to understand the events described in two documents to see if they correspond to the same case.  To do so, the system may need to understand what type of information can be used to relate two documents and how to find them. Beyond accuracy considerations, the system also needs to be cost-aware. Performing $O(n^2)$ LLM operation to understand the relationship between $n$ documents can be prohibitively expensive and the system needs to design methods to be able to achieve the same results cheaply, e.g., by using cheaper heuristics such as pre-extracting relevant information or using data embeddings to filter out non-related documents.  

% \subsection{Task-Aware Data Processing}\label{subsec:taskaware}
% The system can attempt to proactively find parts of the data that will be useful to improve processing tasks down the road. Given that the data sets can contain millions of documents, it can be expensive to do extensive processing of the data upfront, or time-consuming to leave all the data processing when the user issues a task. As such the system needs to decide how much preprocessing it needs to upfront, and what to perform at query time. To strike a balance between the two extremes, the system can decide to identify and/or extract data units that it deems to be useful in the future. This can be done by understanding the semantics of the data. For instance, in Example~\ref{ex:police}, the system can decide that sections that describe police incidents at a high level (e.g., in a police officer's report) are useful for future task processing, given that journalists will likely need to understand incidents to be able to write articles about the police misconduct in them. The system can keep pointers to such sections, but leave more specific data processing after user issues queries. Similarly, the system can do schema identification in advance to find what type of data is represented in the text, and use the identified schema to answer queries. The system can decide whether to extract information upfront to populate the schema, or simply keep pointers to where the information can be find and do extraction at query time. 

% %\textbf{Task-aware Preprocessing}
% %\textbf{Schema Identification}



% %\subsection{Extracting Existing Structures}


% %The context and content of structured portions help guide user queries to identify the most relevant information in the document, while their structured representation ensures high accuracy.  For example, many police use-of-force records are presented in tables within documents, while metadata, such as case dates and involved police names, are often represented as key-value pairs.  A proactive database system should adapt document structures to match the granularity required by user queries and, if necessary, reformulate the data representation to improve execution accuracy.  



 


% %\subsection{Automating Indexing for Multi-Modal Structures}
% \if 0
% \subsection{Mixing Data Representations}
% \sep{I don't think, we need this section. Here's things we might want to say but I don't think they are particularly interesting. I'm leaving this as is but can com back to it if we want to add things}
% \begin{itemize}
%     \item Embedding based indexes
%     \item Generating summaries
%     \item Other data transformations, e.g., take table and add a new column using LLM
% \end{itemize}

% The same document may exhibit different structures. Beyond documents, other data modalities in the unstructured world may also follow certain structures. Selecting the right structure or a combination of multiple structures to represent data for answering downstream queries is challenging. For example, suppose a journalist wants to find the dates of police misconduct records involving the use of force in a specific city. The documents are structured with different types of misconduct as section headers (e.g., ``Corruption'' or ``Excessive Force''), followed by detailed case descriptions presented as tables within each section. One can first use the semantic hierarchy as an index to locate the ``Excessive Force'' section and then reformulate the data by extracting tabular information for each case within this section to identify the dates of incidents in the specified city.  This requires a proactive database system to automatically analyze the available structures in the document collection, determine the optimal structure as an index, and sequence it appropriately to locate the most relevant text portions while representing the data in the most suitable way for answering downstream queries.  


% \fi





% %\yiming{TODO}











% %ZenDB stuff, maybe twix?

% % \subsubsection{Data Decomposition} 
% % \label{subsec:decomposition}
% %Docetl stuff?


% % \subsection{Challenges and Future Directions}


% % \noindent{\bf Data Analytics in a Mix-Structure and No-Structure World. }
% % Beyond the two document structures presented in Section~\ref{subsec:structure}—semantic hierarchical structures and visual form-like structures—real-world datasets can be much more diverse, featuring a mix of structures or, in some cases, no discernible structure at all. For a proactive database system, analyzing and predicting meaningful properties within documents to enhance downstream query analytics in diverse real-world scenarios remains a significant challenge. While data decomposition techniques discussed in Section~\ref{subsec:decomposition} offer a complementary way to refine relevant data for queries, providing a logical and principled interpretation to the data to achieve highly accurate query results with low cost and latency remains an open problem. A proactive database system should enable agents to analyze underlying document structures, if present, based on query intent and document type, to identify the optimal data representations that best fit the data and query workload. 





% %\noindent{Data Analytics Across Diverse Documents.}


% % \noindent{\bf Data Analytics in a Multi-Modal World. }
% % So far, our analysis has focused on documents. However, the unstructured world extends beyond text to include other data modalities such as video, audio, and images. The relationships between different pieces of information can span multiple modalities, making it challenging to define the right data models for supporting queries across them. As demonstrated earlier, it is unclear whether the relational model remains universally dominant, and the answer may well be negative.  




%!TEX root=../main.tex


\section{Proactive User Understanding }\label{sec:intent}

Even with best-effort operation reformulation 
and data understanding capabilities, 
a fundamental challenge remains: the gap between what users specify 
and what they actually need. 
This challenge manifests in multiple ways---users may provide ambiguous specifications, 
fail to articulate implicit assumptions, 
or simply not know how to express their requirements fully~\cite{papicchio2024evaluating, shankar2024validates}. 
To bridge this gap between the user's intent and the operations performed, 
a proactive data system needs to be internally 
aware of this gap when executing the task (Section~\ref{sec:internal}), 
leverage user feedback to bridge the gap (Section~\ref{sec:feedback}) 
and provide mechanisms for users to externally validate query results (Section~\ref{sec:verification}). 

\subsection{Imprecision-Aware Processing}\label{sec:internal}

Unlike reactive data systems
that only execute the query as stated, proactive
data systems can leverage LLMs to help truly
identify user intent,
despite the user-provided tasks representing
an imprecise or incomplete specification thereof. 
The system should therefore internally consider 
multiple possible user intents when processing tasks, 
potentially providing different answers that correspond 
to these different interpretations. This can be done to varying degrees. 
In the simplest form, the system can consider multiple interpretations 
of the statements provided by the user. 
In Example~\ref{ex:police}, the system can consider 
various spellings of the same name, 
either in the input provided by the user or derived from the data. 
Such attempts are similar to 
possible world notions in the database community~\cite{suciu2022probabilistic}, 
where the database can consider various possibilities for ``fuzzy'' data and queries.

A proactive system can take 
more aggressive steps to understand user intent. 
For example, the system can consider adding new predicates 
that the user may be interested in---e.g., 
if a user has previously asked several questions about police misconduct 
in a specific city but submits a new query without specifying location, the system might prioritize results from that city. 
This intent discovery can be data-driven---the system might determine that records from certain cities are more relevant 
or interesting 
and prioritize them in the output. 
Moreover, if the output for an operation is too large, the system can 
selectively display what it determines to be the most relevant answers 
or provide appropriate summaries or sample outputs.

The system can also anticipate user questions, for example, ``why was a certain record not provided in the answer'', 
and proactively relevant records that, 
while not strictly matching the query, might be of interest to the user. 
Additionally, the system can modify queries by dropping or relaxing certain predicates---e.g., if the user has specified a predicate that leads to empty results. 
Or, the system might expand query scope, for example, geographically, to include potentially interesting results 
(such as when a specific type of police misconduct, while not present in the queried city, occurred in neighboring jurisdictions).


\subsection{Leveraging User Feedback}\label{sec:feedback}
A proactive system can leverage feedback from users to clarify intent in a lightweight manner.
This feedback serves two purposes: improving accuracy for the current task, and enhancing the system's understanding of user intent for future queries.


To improve the accuracy on the current task, the system can decide to ask follow-up questions~\cite{li2024llms}. This might include asking for clarification about task goals, gathering additional specifications, or presenting example results for users to indicate which best match their needs. The important challenge is balancing the need for clarity with minimizing user burden: for example, when processing police misconduct documents, rather than asking multiple detailed questions, the system might show representative document types and let users select which are most relevant---then apply this learning broadly across the document collection. Similarly, when encountering potential name variations in Example~\ref{ex:police}, the system might ask a single question about handling typos that can inform its overall matching strategy, rather than asking the user to confirm every typo correction. While LLMs offer promising capabilities for generating targeted feedback requests, automatically determining what feedback to request and when remains an open research challenge. Prior work
on predictive interaction is highly relevant here~\cite{2015-predictive-interaction}.


User feedback can also be leveraged to improve system performance on {\em future} tasks. For example, if the user provides feedback that a certain document is relevant or not relevant to a task, the system can then update its data and task processing mechanism to take that feedback into account. This can, for instance, change operation rewrite rules used internally for processing (Section~\ref{sec:op}), modify data representation~\cite{zeighami2024nudge} or change how semantic structure is extracted from the data (Section~\ref{sec:data}). While this approach shares similarities with query-by-example systems that learn from user-provided examples, e.g., \cite{fariha2018squid}, it extends the concept more broadly—--allowing the system to refine its understanding of user intent across a diverse range of tasks and feedback types.

\subsection{Verifying Execution}\label{sec:verification}

A proactive system must provide users with the means to verify that their tasks were executed correctly. Verification is particularly important when the system makes autonomous decisions---for instance, when correcting potential typos in names, the system should clearly show which corrections were made to allow users to catch any incorrect modifications. 
If, during processing, the system encountered anomalous documents, it's best to indicate them as such to the users so that they don't pollute the rest of the analysis.

While the system can provide comprehensive execution traces, including details of LLM operations performed and data sources accessed~\cite{tan2007provenance}, presenting this information in a user-friendly way remains challenging. Simply showing raw execution traces or complete datasets is overwhelming and impractical, as users cannot reasonably review large amounts of data to verify correctness. An interesting open challenge
is to determine a small subset or explanation that conveys the same information as the entire provenance; we can always verify such explanations using an LLM. 


% % We have discussed how the data system can attempt to proactively understand the data and the operations to improve the correctness of the system by breaking down complex operations, rewriting them, etc. However, even if such reformulations are done to the best of the system's ability, the effort is in vain if the user's specification is ambiguous, or it does not actually match their true intent \cite{papicchio2024evaluating, shankar2024validates}. As Example~\ref{ex:police} shows, even in very simple tasks such as looking up records by name, the system needs to proactively ensure that the operations it performs match the user intent. 

% \if 0
% %\sep{not sure if sematic gap is the phraing we want to go with, but keeping it as that rn}
% An important characteristic of performing natural language operations, either when the data is unstructured or when task specification is unstructured  is the semantic gap between the user specification and what the system perceives the user specification to be. This semantic gap can exist even in very simple query scenarios:

% \begin{example}\label{ex:typo}
%      On either a structured or unstructured data source containing names of people, consider the simple query of a user asking for records of ``Adam Smiht'', where ``Smiht'' is a misspelling of ``Smith''. The system needs to understand user intent (which name they are asking for). This may need to be done with respect to data (e.g., if neither ``Smith'' or `'Smiht'' exist, the answer is the same for both queries). The system cannot ensure correctness by always correcting what it perceives as typos, there may be deviations from common name spelling that are not typos. 
% \end{example}

% The example shows that even in simple query-answering scenarios, the correct answer cannot be ensured without taking the user into account. In a proactive data system, the data system has the agency to ensure the operations are performed correctly. This means the system needs to (1) provide an interface where it can collect additional feedback and allow the user to verify answers and (2) have internal mechanisms to take such feedback into account. 
% \fi



% %\subsubsection{Interface}
% %We envision an interface where the system is aware of imprecision in the unstructured tasks and the data definition, and provides tools for the users to reach their correct answer while keeping the imprecision in mind. Here, we discuss various frameworks that allow us to achieve this.

% %\textbf{Possible Worlds}. 
% \subsection{Imprecision-Aware Processing}\label{sec:internal}

% \shreya{Seems like the overall takeaway here is; LLMs allow us to be more aggressive in determining the user's intent.}

% User-provided tasks often represent imprecise or incomplete specifications of their true intent. The system should therefore internally consider multiple possible user intents when processing tasks, potentially providing different answers that correspond to these different interpretations. This can be done to varying degrees. In the simplest form, the system can consider multiple interpretations of the statements provided by the user. In Example~\ref{ex:police}, the system can consider various spellings of the same name, either in the input provided by the user or in the data. Such attempts are similar to the possible world's notion in the database community \cite{suciu2022probabilistic}, where the database can consider various possibilities for a data point and the query.

% % The task provided by the user can be an imprecise representation of their intent. The system needs to internally consider possible user intents when performing the task and find answers that are likely to match the user intent, potentially providing multiple different answers to users for their different possible intents. 

% % This can be done to varying degrees. In the simplest form, the system can consider multiple interpretations of the statements provided by the user. In Example~\ref{ex:police}, the system can consider various spellings of the same name, either in the input provided by the user or in the data. Such attempts are similar to the possible world's notion in the database community \cite{suciu2022probabilistic}, where the database can consider various possibilities for a data point and the query.

% Beyond simple matching of different interpretations, a proactive system can take more aggressive steps to understand user intent. For example, the system can consider adding new predicates a statement that the user may be interested in---e.g., if a user has previously asked several questions about police misconduct in a specific city but submits a new query without specifying location, the system might prioritize results from that city. This intent discovery can be data-driven---the system might determine that records from certain cities are more relevant and prioritize them in the output. Moreover, if the output for an operation is too large, the system can selectively display what it determines to be the most relevant answers or provide appropriate summaries.



% % Such intent discovery can be done data dependently, for example, the system can decide based on the data that the records for a certain city are more interesting, and put the results for that city at the top of the output. Moreover, if the output for a task is too large the system can decide to only show answers that the system thinks the user is interested, or summarize some of the answers. 

% The system can anticipate user questions, for example, ``why a certain record was not provided in the answer'', and proactively relevant records that, while not strictly matching the query, might be of interest to the user. Additionally, the system can modify queries by dropping or relaxing certain predicates---particularly if the user has incorrectly specified the predicate. \shreya{The prior sentence needs an example; it is weird for it to not have an example while the following sentence has an example.} Or, the system might expand query scope, for example geographically, to include potentially interesting results (such as when a specific type of police misconduct, while not present in the queried city, occurred in neighboring jurisdictions).


% % The system can also drop or relax certain predicates. The user may have wrongly specified a predicate and dropping such a predicate may lead to answers that match the user intents better. The system can similarly expand the scope of a query, e.g., geographically, to include results that the user may find interesting (e.g., a specific form of police misconduct may not have happened in a particular city but did happen in a neighboring city).

% %has been proposed to deal with uncertainty in query answer. The notion allows the system to consider various interpretations of the data and query, and show to the user the answers to all \textit{possible} interpretations. However, in the case of unstructured data and queries, having such an interface comes with its own challenges. Although LLMs can be used to decide what is a ``possible answer'', the notion of what is a possible interpretation is not well-defined, and on its own may require further feedback from the user. Moreover, such an approach can lead to large output sets that are hard to interpret for the user, as the user may need to understand why an answer was possible. 

% %In addition to decomposing complex operations, task rewrites can address ambiguity in user specifications. When tasks are specified in natural language, they often contain ambiguity that impacts result quality. Rather than executing these tasks directly, proactive systems can rewrite them to explicitly introduce specificity. For instance, a natural language query about ``find police misconduct cases'' might be rewritten into a series of more specific operations that check for use of force incidents, procedural violations, and subsequent investigations. The challenge of ambiguity is particularly evident in natural language to SQL translation. 


% %Existing work \cite{madden2024databases, liu2024declarative} suggests using a declarative interface, inspired by the declarative interface of relational databases, to perform data processing tasks. The goal of such a declarative interface is to allow the user to specify their task, while the system is in charge of how to perform the task. However, Example~\ref{ex:typo} shows how difficult such a separation can be even in a simple query answering scenario, where ``how'' to perform a task requires detailed instructions about the task from the user. We believe interfaces that allow the system to understand user intent, and allow the user to verify system operations are crucial for successful data processing. The database system cannot assume that the user input corresponds to well-defined logical operations, but needs to take the agency to ensure the correct interpretation of user inputs. 

% \subsection{Leveraging User Feedback}\label{sec:feedback}

% \shreya{Seems like the takeaway here is; ask for low-effort feedback and apply this across tasks, not just for the current task.}

% A proactive system can leverage feedback from users to clarify intent. This feedback serves two purposes: improving accuracy for the current task, and enhancing the system's understanding of user intent for future queries.

% % The user can additionally interact with the user to obtain clarifications of the user intent.  User feedback can be used to improve the accuracy on the current task and to better understand user intents in general for future tasks. 

% To improve the accuracy on the current task, the system can decide to ask follow-up questions~\cite{li2024llms}. This might include asking for clarification about task goals, gathering additional specifications, or presenting example results for users to indicate which best match their needs. The important challenge is balancing the need for clarity with minimizing user burden: for example, when processing police misconduct documents, rather than asking multiple detailed questions, the system might show representative document types and let users select which are most relevant---then apply this learning broadly across the document collection. Similarly, when encountering potential name variations in Example~\ref{ex:police}, the system might ask a single question about handling typos that can inform its overall matching strategy, rather than asking the user to confirm every typo correction. While LLMs offer promising capabilities for generating targeted feedback requests, automatically determining what feedback to request and when remains an open research challenge. \shreya{We need better connection to PBE.}


% % All such feedback should attempt to minimize the need for user feedback, and the system needs to decide to maximally use the feedback. For instance, by showing different types of documents to the user, the system may decide a certain type of document is useful for finding misconduct, applying user feedback broadly to new documents to be processed. %how to generalize feedback to n. example results to show the user to choose from, among many other intent discovery works. Using LLMs, the space of potential interactions with the user is large, and the database system needs to decide what the best way to elicit feedback from the user to get to the correct answer is. 
% % In Example~\ref{ex:police} where there can be a type in the names of police officers, the system may succinctly ask the user if typos should be corrected. Automatically finding how to generate such follow-up questions using LLMs based on the data and the task is challenging and requires further research. 

% %\textbf{Taking User Feedback into Account}. More importantly, the system needs to be able to take user feedback into account. We envision two modes, where the system is either getting feedback to interactively improve the accuracy on the current task, or getting feedback to improve the accuracy on future task.
% %In the first setting, the system can take feedback from the user and internally rewrite queries. It can ask the user follow up questions and rewrite the data. \sep{Probably just connect this back to the two previous sections}

% User feedback can also be leveraged to improve system performance on {\em future} tasks. For example, if the user provides feedback that a certain document is relevant or not relevant to a task, the system can then update its data and task processing mechanism to take that feedback into account. This can, for instance, change operation rewrite rules used internally for processing (discussed in Sec.~\ref{sec:op}), modify data representation \cite{zeighami2024nudge} or change how semantic structure is extracted from the data (discussed in Sec.~\ref{sec:data}). While this approach shares similarities with query-by-example systems that learn from user-provided examples \cite{fariha2018squid}, it extends the concept more broadly—allowing the system to refine its understanding of user intent across a diverse range of tasks and feedback types.

% %An example solution is NUDGE \cite{zeighami2024nudge}, where the users provide feedback on what documents are the correct answer to a query, and the system modifies the internal data representation (in the case of NUDGE the vector embeddings of data records) to improve accuracy on such a dataset. Overall, NUGDE is a specific that shows how to fine-tune vector data representations for the user to ensure that the system is well-opteimized for the specific workload the user is interested in. Ideas from such an approach can be applicable to other data representation, for example, to fine-tune extracted summaries, modify schemas and the extracted structure, to ensure that the internal representation of the data and tasks in the system is better aligned with the user after obtaining feedback from the user.     

% \subsection{Verifying Execution}\label{sec:verification}

% A proactive system must provide users with the means to verify that their tasks were executed correctly. Verification is particularly important when the system makes autonomous decisions---for instance, when correcting potential typos in names, the system should clearly show which corrections were made to allow users to catch any incorrect modifications. 

% While the system can provide comprehensive execution traces, including details of LLM operations performed and data sources accessed \cite{tan2007provenance}, presenting this information in a user-friendly way remains challenging. Simply showing raw execution traces or complete datasets is overwhelming and impractical, as users cannot reasonably review large amounts of data to verify correctness. \shreya{Ends abruptly}


% % No matter how the system performs a task, the user needs to be able to verify that the operation was done correctly. For example, if the system asks fixes typos in name, it should be able to show the user what typos were fixed, in case the system incorrectly changed the spelling of a name. The system can in general show the user the operations it performed (e.g., what LLM calls), and what part of the data were used~\cite{tan2007provenance}. However, doing so in a way that allows easy validation by the user is difficult~\cite{shankar2024validates}. For example, although showing the entire dataset allows user to validate what information was extracted, it is not helpful since user would need to spend time reading through the document. %We envision a proactive data system that will decide what to show to the user, both informative and concise. 

% %For query-answering tasks, the system can show the user the answer provenance, so the user can verify how the answer was generated. However, how the provenance should be presented to the user is non-trivial, since the answer may have been generated from a large span of text and verification from a large text span may be difficult and time-consuming. 
% %The interface may require the system to perform additional operations while answering queries. For example, the system may need to keep track of the LLM calls it did internally to show to the user for validation, or keep track of data provenance to show to the user. It may need to consider possible world semantics, especially if multiple different answers are to be shown to the user.  



