%!TEX root=main.tex
\begin{abstract}
With the power of LLMs, we now have the ability to
query data that was previously impossible to query,
including text, images, and video.
However, despite this enormous potential,
most present-day data systems that leverage
LLMs are {\em reactive},
reflecting our community's desire to map LLMs
to known abstractions. 
Most data systems treat LLMs as an opaque black box 
that operates on user inputs
and data as is,
optimizing them much like any other 
approximate, expensive UDFs, in conjunction
with other relational operators.
Such data systems do as they are told, 
but fail to understand and leverage 
what the LLM is being asked to do (i.e.
the underlying operations, which may be error-prone),
the data the LLM is operating on (e.g., long, complex documents),
or what the user really needs. 
They don't take advantage 
of the characteristics of the operations and/or 
the data at hand, or ensure correctness of results when there 
are imprecisions and ambiguities. 
We argue that data systems instead need to be {\em proactive}:
they need to be given more agency---armed with the power of 
LLMs---to understand and rework the user inputs and the data 
and to make decisions on how the operations 
and the data should be represented and processed. 
By allowing the data system to parse, rewrite, and decompose user 
inputs and data, or to interact with the user in ways
that go beyond the standard single-shot query-result paradigm, 
the data system is able to address user needs more efficiently and
effectively. 
These new capabilities lead to a rich design space where the data system takes more initiative: they are empowered
to perform optimization based on the transformation operations, 
data characteristics, and user intent. We discuss various 
successful examples of how this framework has been and can be applied in 
real-world tasks, and present future directions for this ambitious research agenda.    
\end{abstract} 

\if 0
Traditional data management systems rely on the relational data model to provide declarative abstractions for users to store and access their data. The relational dbms, as well as extensions beyond RDBMSs such as NoSQL or Data Lakes, % declarative abstractions ensure that the inputs (user queries and data) to a database system and the corresponding outputs are mathematically well-defined, and the job of the dbms is to decide how to perform such operations. , although change the data model and the interface, have continued with the 
assume that inputs to the database system is mathematically well-defined, and dbms's role is to decide how to perform such operations. We argue that the introduction of LLMs has disrupted such declarative paradigms, where inputs to a dmbs are semantically meaningful but not mathematically well-defined. We believe existing work that attempts to keep the traditional declarative interface of the dmbs, and uses LLMs to translate inputs from non-declarative interfaces to a declarative interface (e.g., through ETL or NL2SQL) or as backbox UDF calls within the declarative interface, is inherently flawed, both limiting the application domain of databases and leading to suboptimal solution in terms of accuracy and efficiency. We argue for a shift towards post-relational data management systems, where data and query processing need to be semantically, and not just mathematically, correct, necessitating a shift towards non-declarative database systems. In such a system, correctness is defined with respect to a user and real-world knowledge, which requires utilizing interactions with user, understanding real-world semantics of the data and queries, and utilizing general reasoning and knowledge to answer queries. We discuss how such a vision is possible, both describing how the interface as well as the internals of a dbms need to change to support post-relational data and query processing. %envision access to a post-relational dmbs to be through a non-declarative, but interactive interface, where intent discovery, disambiguation and validation are esseintial components of the interface. We discuss how the internals of a database system needs to change to support such an interface.   

\begin{abstract}
Real-world data is often created unstructured. Even when the data is structured, users provide queries and data processing tasks in natural language. Many recent work attempt to extend data management systems to support such datasets and tasks, but primarily propose database systems that apply LLM operations on user inputs and data as-is. Such black-box application of LLMs on user inputs and data lead to \textit{reactive} data systems that do as they are told, but fail to understand the task, the data or the user intents. Such systems fail to take advantage of the characteristics of the task and the data in hand, or to ensure correctness of results when there are imprecisions and ambiguities in the data and the task. We argue that the data systems need to be \textit{proactive}. They need to be given more agency to understand user inputs and the data and to make decisions on how the query and the data should be represented and processed. By allowing the database system to parse, rewrite and decompose user inputs and data, or to interact with the user and take into account their additional feedback, the database system is able to processes user query and data much more accurately. This leads to a rich design space where the database is allowed to perform optimization based on user task, data characteristics and user intent. We discuss various successful examples of how this framework can be applied in real-world tasks, and present future directions to build database systems that take the initiative to optimize data processing tasks.     
\end{abstract}
\fi
