\section{The Data Management System Design Space}
\subsection{Rethinking abstactions and internals of dbms}
\begin{itemize}
    \item We consider (1) what query/data interface systems support and (2) the internal data and query representation
    \begin{itemize}
    \item The above two are the same in relational database systems, which provide a declarative intereface on the relational data, where SQL instructions are dependendent on the (logical) structure of the data, as internally represented
    \end{itemize}
    \item In the interface, database should allow for both unstructured and structured interface, where data/queries can be unstructured (e.g., RAG on text), structured (SQL on relational), or somewhere in between (e.g., zendb)
    \item Internally, a dbms can decide to translate the user input/store and represent data in a structured or unstructured way
    \begin{itemize}
        \item Structure can be used to improve accuracy/efficiency, but does not necessarily exists
    \end{itemize}
    
\end{itemize}

\subsection{Why does it matter}
\begin{itemize}
    \item Relational databases store data as a materialized structured projection of the real-world, and provide a query language to query the data based on this pre-defined structure. 
    \begin{itemize}
        \item This projection is based on a pre-defined workload for the application designed by the DBA, This is good if
        \item  There is a fixed and pre-defined workload, from which a needed structure can be deduced
        \begin{itemize}
            \item The structure repeats in the data (i.e., tables have multiple rows)
        \end{itemize}
        \item The structure is known by users of the system so that they can query it
        \begin{itemize}
            \item queries need data based on the same logical structure as defined for the data
        \end{itemize}
        
    \end{itemize}
\end{itemize}
\begin{itemize}
\item However, information needs from a real-world phenomenon is not necessarily fixed or predictable, and may not follow and obvious predefined structure. Even if structure exists, it might not be known to the user. Examples:
    \item Police records
    \item Legal documents
    \begin{itemize}
        \item Information not adhering to the predefined structure
    \end{itemize}
    \item Civic meetings agenda
    \begin{itemize}
        \item Information with a mix of structure and unstructured data (e.g., some structure repeats, but some doesn’t.) Can be fixed entities but each entity has different attributes, or each query can be asking for different attributes. 
    \end{itemize}
    \item some dataset search example
    \item AI allows querying such data, so requires rethinking what are good abstractions and how database internals should change. There are bad alternatives, discussed below. 
\end{itemize}

\subsection{Bad solutions}
\begin{itemize}
\item “LLM calls as UDFs” 
\begin{itemize}
    \item Bad abstraction, stops from performing a lot of optimizations 
\end{itemize}
\item “DMBS is relational, the rest of the data/query must just be translated (outside of the dmbs) to relational“
\begin{itemize}
    \item Can be suboptimal because sometimes unstructured representation is more accurate/efficient
    \item Can be suboptimal because of poor integration of translation from unstructured to structued
\end{itemize}
\end{itemize}