%!TEX root=main.tex

\section{Conclusion}

The database community stands at a pivotal moment where LLMs offer unprecedented capabilities for processing both structured and unstructured data. In this vision paper, we proposed {\em proactive} data systems: systems that possess agency in understanding and optimizing data processing tasks. Unlike traditional {\em reactive} systems that treat LLMs as black-box UDFs operating on monolithic inputs, proactive systems go further in leveraging LLMs to aid data processing along three axes. We presented these axes---operations, data, and user intent---and demonstrated the potential of LLMs to help in each one through our recent work on operator reformulation, document organization and analytics, and intent-aware optimization. Overall, proactive data systems can achieve both higher accuracy and lower costs than reactive systems that treat LLMs as black boxes.

% We have discussed our vision for \textit{proactive} data systems that assume more agency in data processing: they understand user intent, the operations they perform and the data they perform them on. This allows the systems to perform optimization to reformulate tasks and extract semantic information from the data that both improves accuracy and costs. Proactive data systems contrast with the traditional \textit{reactive} data systems that represent LLM tasks similar to UDFs, performing them as is on monolith data and user inputs. We discussed various solutions and potentials, inspired by our previous successful work on unstructured data processing systems, to make proactive data systems a reality.