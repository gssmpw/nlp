%%%%%%%% ICML 2025 EXAMPLE LATEX SUBMISSION FILE %%%%%%%%%%%%%%%%%

\documentclass{article}

% Recommended, but optional, packages for figures and better typesetting:
\usepackage{microtype}
\usepackage{graphicx}
\usepackage{subfigure}
\usepackage{booktabs} % for professional tables

% hyperref makes hyperlinks in the resulting PDF.
% If your build breaks (sometimes temporarily if a hyperlink spans a page)
% please comment out the following usepackage line and replace
% \usepackage{icml2025} with \usepackage[nohyperref]{icml2025} above.
\usepackage{hyperref}

% Attempt to make hyperref and algorithmic work together better:
\newcommand{\theHalgorithm}{\arabic{algorithm}}

% Use the following line for the initial blind version submitted for review:
% \usepackage{icml2025}

% If accepted, instead use the following line for the camera-ready submission:
\usepackage[accepted]{icml2025}

% For theorems and such
\usepackage{amsmath}
\usepackage{amssymb}
\usepackage{mathtools}
\usepackage{amsthm}
\usepackage{bm}


% if you use cleveref..
\usepackage[capitalize,noabbrev]{cleveref}

%%%%%%%%%%%%%%%%%%%%%%%%%%%%%%%%
% THEOREMS
%%%%%%%%%%%%%%%%%%%%%%%%%%%%%%%%
\theoremstyle{plain}
\usepackage{amsmath}
\newtheorem{theorem}{Theorem}[section]
\newtheorem{proposition}[theorem]{Proposition}
\newtheorem{lemma}[theorem]{Lemma}
\newtheorem{corollary}[theorem]{Corollary}
\theoremstyle{definition}
\newtheorem{definition}[theorem]{Definition}
\newtheorem{assumption}[theorem]{Assumption}
\theoremstyle{remark}
\newtheorem{remark}[theorem]{Remark}

% Todonotes is useful during development; simply uncomment the next line
%    and comment out the line below the next line to turn off comments
%\usepackage[disable,textsize=tiny]{todonotes}
\usepackage[textsize=tiny]{todonotes}


% The \icmltitle you define below is probably too long as a header.
% Therefore, a short form for the running title is supplied here:
\icmltitlerunning{Bellman error centering}

\begin{document}

\twocolumn[
\icmltitle{Bellman Error Centering}

% It is OKAY to include author information, even for blind
% submissions: the style file will automatically remove it for you
% unless you've provided the [accepted] option to the icml2025
% package.

% List of affiliations: The first argument should be a (short)
% identifier you will use later to specify author affiliations
% Academic affiliations should list Department, University, City, Region, Country
% Industry affiliations should list Company, City, Region, Country

% You can specify symbols, otherwise they are numbered in order.
% Ideally, you should not use this facility. Affiliations will be numbered
% in order of appearance and this is the preferred way.
%\icmlsetsymbol{equal}{*}

\begin{icmlauthorlist}
\icmlauthor{Xingguo Chen}{yyy}
\icmlauthor{Yu Gong}{yyy}
\icmlauthor{Shangdong Yang}{yyy}
\icmlauthor{Wenhao Wang}{sch}
%\icmlauthor{}{sch}
%\icmlauthor{}{sch}
\end{icmlauthorlist}

\icmlaffiliation{yyy}{Nanjing University of Posts and Telecommunications, Nanjing, China}
\icmlaffiliation{sch}{National University of Defense Technology, Hefei, China}

%\icmlcorrespondingauthor{Xingguo Chen}{chenxg@njupt.edu.cn}
\icmlcorrespondingauthor{Wenhao Wang}{wangwenhao11@nudt.edu.cn}


% You may provide any keywords that you
% find helpful for describing your paper; these are used to populate
% the "keywords" metadata in the PDF but will not be shown in the document
\icmlkeywords{Reinforcement Learning}

\vskip 0.3in
]

% this must go after the closing bracket ] following \twocolumn[ ...

% This command actually creates the footnote in the first column
% listing the affiliations and the copyright notice.
% The command takes one argument, which is text to display at the start of the footnote.
% The \icmlEqualContribution command is standard text for equal contribution.
% Remove it (just {}) if you do not need this facility.

%\printAffiliationsAndNotice{}  % leave blank if no need to mention equal contribution
\printAffiliationsAndNotice{\icmlEqualContribution} % otherwise use the standard text.

\begin{abstract}
This paper revisits the recently proposed reward centering algorithms 
including simple reward centering (SRC) and value-based reward centering (VRC), 
and points
out that SRC is indeed the reward centering, while
VRC is essentially Bellman error centering (BEC). Based on
BEC, we provide the centered fixpoint for tabular value
functions, as well as the centered TD fixpoint for linear value function
approximation. We design the on-policy CTD algorithm and the off-policy CTDC
algorithm, and prove the convergence of both algorithms. Finally, we
experimentally validate the stability of our proposed algorithms. 
Bellman error centering facilitates the extension to various
reinforcement learning algorithms.
% This study revisit the new developed 
%     This study establishes a new objective function, MSCBE, 
%     to derive the centered TD algorithm. Although the update rules of 
%     this algorithm are similar to those of the value-based reward 
%     centering presented in the paper "Reward Centering," the meanings 
%     of the updated parameters are entirely different. By leveraging 
%     the concept of Bellman error centering, this study provides a 
%     new understanding of the true essence behind reward centering. 
%     Based on Bellman error centering, another objective function, MSPCBE, is 
%     designed to derive the centering TDC algorithm. Additionally, this study 
%     offers convergence proofs for both two algorithms and validates them through experiments.
\end{abstract}

\documentclass[../main.tex]{subfiles}
\graphicspath{{../images/}}
\makeatletter
\def\input@path{{../images/}}
\makeatother
\begin{document}
\section{Introduction}
\begin{figure}
\centering
\begin{tikzpicture}
\node[inner sep=0pt] (ws) at (0, 0) {
\includegraphics[height=.4\textwidth, trim={10cm 0 10cm 0},clip]{world_space.png}};
\node[inner sep=0pt] (cs) at (6,0) {\includegraphics[height=.4\textwidth, trim={10cm 1cm 10cm 4cm},clip]{conf_space.png}};
\end{tikzpicture}
\vspace{-5pt}
\label{fig:pbrm_intro}
\caption{\textbf{Left}: Shows world space obstacles as grey spheres. Robots start and goal configuration is colored red and green, respectively. Configurations along the computed path are colored transparent blue. \textbf{Right:} Mapped world space scenario to configuration space. Obstacle region is the grey mesh. Red spheres are collision-free regions computed by the neural SCDF. The optimized shortest path in the convex corridor is the blue curve.}
\vspace{-25pt}
\end{figure}
Motion planning is the problem of finding a collision-free trajectory that connects a given start and goal configuration. The planning takes place in the configuration space of the robot. For single body robots, like mobile robots or drones, the configuration space and the world space are usually the same. This simplifies the planning, since explicit obstacle representations are available which enables geometrical tools like separating hyperplanes, smallest distance to obstacles etc., to be used when designing motion planning algorithms. For multi-body robots like manipulators, the situation is completely different. The world space obstacles are usually mapped to non-convex regions, and to make the problem even harder, the mapping is usually not known. Forming explicit representations of the obstacle region in the configuration space is usually too expensive or intractable. Despite all of this, sampling based planners are used with great success, which mainly is due to their use of implicit representations of the obstacle region. The basic idea is to construct a graph in the configuration space that covers and connects the collision-free region. From this graph, a path can be extracted that connects a given start and goal configuration. The approach is computationally expensive, since the graph is constructed with the smallest geometrical building block available, points, which represents a collision-check. Furthermore, the extracted paths from the graph are non-smooth and jagged due to the stochastic nature of the approach. This adds an additional post-processing step to the process, where the paths are shortcutted and smoothened, before the path can be used for tracking. Clearly a lot of time is invested to form this graph and produce smooth paths. Thus, if the obstacles start to move, then all of this work is done in no use, since all points that make up this graph need to be re-verified, which is simply too time consuming to be done in real time.
\\\\
In this work, we want to address the existing drawbacks of the sampling based planners. Our main contribution is an improved motion planner where each vertex in the graph covers a collision-free region in the form of a sphere instead of a point and where the edges are formed with neighboring intersecting spheres. This representation has the advantage of instead of returning piecewise linear paths, returning a sequence of overlapping spheres, i.e. a convex corridor, that connects a given start and goal configuration, illustrated in Figure \ref{fig:pbrm_intro}. This convex corridor allows us to use convex optimization to produce smooth trajectories, instead of computationally expensive post-processing methods. The representation further allows us to estimate the coverage of the collision-free space, which gives us awareness and feedback in the offline roadmap construction phase. Finally, our representation is simple to adapt to moving obstacles, simply requery for the new radii and recheck for intersections. 
\\\\
The spherical collision-free regions are formed using a signed distance function (SDF), which is a function that returns the smallest distance from an arbitrary point to the boundary of an obstacle. As the name implies, the distance is signed, thus if the point is inside the obstacle it is negative otherwise positive. If the distance is positive, a sphere with radius equal to the distance is guaranteed to cover a collision-free region. Using an SDF in motion planning is not new, but what is novel about our approach is that we express the distance in the configuration space instead of the world space and by doing so allows us to form these convex collision-free regions. We refer to the resulting SDF as a signed configuration distance function (SCDF). Computing an SCDF analytically is non-trivial, our approach is therefore to parameterize the SCDF with a deep neural network and learn the mapping by supervised learning. Our resulting neural SCDF can compute distances for different parameter values of obstacle shapes and we also show how multiple distances can be combined, thus making our approach flexible.
\section{Related work}
Motion planning algorithms can roughly be divided into three families, grid-based, sampling based and optimization based methods. Grid-based methods (GBM) discretize the planning space from which a graph is then compiled. A standard search method is A$^\star$ \citep{a_star}, which is classified as an \textit{informed} search method, since it employs a heuristic function to speed up the search. A$^\star$ guarantees to return an optimal path at the level of discretization used. GBMs usually discretize the planning space by a regular lattice and this limits the GBMs to problems with low dimensionality due to the curse of dimensionality. Thus, GBMs are usually limited to single-body robots where the degrees of freedom (DOF) are low. To overcome the inherent scaling problem with the GBMs, stochastic methods are usually used for multi-body robots. These methods are termed as sampling-based methods (SBM) and core members within this family are the rapidly-exploring random trees (RRT) \citep{rrt} and the probabilistic roadmap (PRM) \citep{prm}. RRT grows a tree from the start configuration and explores the collision-free region in a rapid way until it is able to connect to the goal region. RRT is usually improved by bi-directional planning \citep{rrt_connect}, i.e. an additional tree is grown from the goal configuration and the trees are tested for connection after any tree has been expanded. RRT is a single-query method, thus it searches for a path from scratch each time it is queried. Contrary to this, PRM is a multi-query method, which solves for multiple queries without starting from scratch. PRM does this by creating a roadmap (graph) that covers the collision-free space as an offline step. The graph is then used to solve for multiple queries. PRMs are used in cases where the environment does not change since the extra offline step is too computationally costly and needs to be re-done if the environment is changed. In our work, we address this inherent issue by using a different roadmap representation. Our vertices in the graph cover a collision-free region in the form of spheres and we form the edges by checking for intersecting spheres. If something in the environment changes, we recompute the spheres radii and recheck the intersections, without relying on collision detection. We use a trained neural network to compute the sphere radius, therefore querying for the radius can be done fast, hence our representation enables the PRM for dynamic environments.
\\\\
In the recent decades, optimization based methods (OBM) \citep{chomp, schulman, itomp, stomp} have been introduced as an alternative to SBM for multi-body robots. Like the SBM, the OBMs scale well to higher dimensional problems and produce smoother motion. It is common to use a SDF in the optimization since it is a smooth function, thus enabling gradient-based methods. However, the standard way of expressing the SDF is in world space. The distance therefore needs to be mapped to the configuration space by the forward kinematics. This mapping makes the optimization problem a non-linear program (NLP), which is computationally expensive to solve. Recently, a different approach has been proposed. In \cite{mp_gcs} motion planning is formulated as a convex optimization problem by using the graph of convex sets framework \citep{gcs}. The underlying idea is to decompose the collision-free space into intersecting convex sets from which a convex optimization problem is formulated. In cases where an explicit representation of the obstacles in the configuration space exists, like for single-body robots, creating collision-free convex regions can be done fast \citep{iris}. For multi-body robots, this is non-trivial. Existing work does this successfully \citep{iris_nlp, iris_c} by an optimization based approach, but the methods are still too time consuming to be used in the presence of moving obstacles. Our approach is instead to use deep learning to learn an SDF expressed in the configuration space. With this, we can query for shortest distances to the collision boundary, which allows us to expand spherical regions which are collision-free. Our approach is fast and therefore enables our suggested roadmap planner to be used in dynamic environments.
\\\\
Recent research has focused on learning collision detection \citep{fk_kernel_distance, diffco, graphdistnet} by predicting the signed distance between the robot links and the surrounding obstacles in the world space. The learned SDF is used in trajectory optimization but since the distance is expressed in the world space, the problem becomes an NLP and therefore takes a long time to solve. We take a novel approach and suggest to instead express the signed distance in the configuration space. This allows us to improve the PRM at the same time as it enables convex optimization for trajectory optimization, which runs faster and is more reliable than NLP solvers. In \cite{cspf} a learned signed distance function in the configuration space is proposed similar to our approach. However, their approach is restricted to point cloud representations, while we propose to represent the obstacles as parameterized geometric shapes, e.g. spheres. Furthermore, we also show how to use our learned SCDF to improve an existing roadmap planner.
\section{Problem formulation}
A robot is located in the world space, $\W \subset \R^3 $. The unique location of the robot is given by its configuration $\q \in \C$, where $\C$ is the configuration space. The set of points covered by the robots bodies at a certain configuration is expressed as $\B(\q) \subset \W$. The robot is surrounded by $\NrObst$ obstacles $\O = \bigcup_{i=1}^{\NrObst} \O_i$, where  $\O_i \subset \W$. The representation of the obstacle in the configuration space is the set $\C\O_i = \{\q \in \C \: |\: \B(\q) \cap \O_i \neq \emptyset \}$. The obstacle space is formed as $\Co = \bigcup_{i=1}^{\NrObst} \C \O_i$. The complement is referred to as the free space, $\Cf = \C \setminus \Co$. The path planning problem is a tuple, ($\Cf$, $\qStart$, $\qGoal$), where we want to connect a query pair, consisting of a start, $\qStart$, and goal configuration, $\qGoal$, with a geometric path, $\q(s): [0, 1] \mapsto \Cf$, such that $\q(0)=\qStart$ and $\q(1)=\qGoal$, or report correctly when such a path does not exist.
\end{document}


\section{Preliminaries}\label{sec:preliminaries}



%We denote by $(\Ac(x_\Ac),\Bc(x_\Bc))(z)$ a random execution of $\pi$ with private inputs $(x_\Ac,y_\Ac)$, and common input $z$.

%\Jnote{Move to DP}
% At the end of such an execution, the protocol outputs a public transcript denoted by the random variable $\trans_\pi(x_\Ac,x_\Ac,z)$ we denotes the common as $\out(\trans_\pi(x_\Ac,x_\Ac,z)$, and each party $\Pc \in \set{\Ac,\Bc}$ obtains his view denoted $\view^\Pc_\pi(x_\Ac,x_\Bc,z)$, which may also contain a ``local output'' \Jnote{Local} $\out^\Pc(x_\Ac,x_\Bc,z)$ (if the protocol specifies such an output). \Jnote{Common output, and parties output}


\subsection{Distributions and Random Variables}\label{sec:prelim:dist}
The support of a distribution $P$ over a finite set $\cS$ is defined by $\Supp(P) \eqdef \set{x\in \cS: P(x)>0}$. For a distribution or a random variable $D$, let $d\from D$ denote that $d$ was sampled according to $D$. Similarly,  for a set $\cS$, let $x \from \cS$ denote that $x$ is drawn uniformly from $\cS$, and denote by $\cU_{\cS}$ the uniform distribution over $\cS$. For a finite set $\cX$ and a distribution $C_X$ over $\cX$, we use the capital letter $X$ to denote the random variable that takes values in $\cX$ and is sampled according to $C_X$. The {\sf statistical distance} (\aka {\sf~variation distance}) of two distributions $P$ and $Q$ over a discrete domain $\cX$ is defined by $\sdist{P}{Q} \eqdef \max_{\cS\subseteq \cX} \size{P(\cS)-Q(\cS)} = \frac{1}{2} \sum_{x \in \cS}\size{P(x)-Q(x)}$. 
For a vector $x = (x_1,\ldots,x_n)$ and index $i\in [n]$, we let $x_{-i} = (x_1,\ldots,x_{i-1},x_{i+1},\ldots,x_n)$ and $x^{(i)} = (x_1,\ldots,x_{i-1}, -x_i, x_{i+1},\ldots,x_n)$, for a set $\cS \subseteq [n]$ we let $x_{\cS} = (x_i)_{i \in \cS}$ and $x_{-\cS} = (x_i)_{i \in [n]\setminus \cS}$, and for a vector $r \in \zo^n$ we let $x_r = (x_i)_{\set{i \colon r_i = 1}}$ and $x_{-r} = (x_i)_{\set{i \colon r_i = 0}}$.

%For $n \in \N$ we let $U_n$ be the uniform distribution over $\oo^n$, and let $S_n$ be the distribution induces by the sum of $n$ i.i.d.\ random variables, each is distributed according to $U_1$. Let $\cN(0,1)$ be the standard normal distribution.
%For a distribution $\cD$ and a function $f$, we define by $f(\cD)$ the distribution that is induced by the output of $f(x)$ for $x \from \cD$. 





% \begin{theorem}[\cite{McGregorMPRTV10}]\label{thm:sv-extracotr}
% 	\Enote{Remove if not needed}
% 	There is a constant $c$ to make the following holds. Let $X$ be an $\alpha$-SV source on $\{0,1\}^n$, let $Y$ be a source on $\{0,1\}^n$ with min-entropy at least $\beta n$ (independent from $X$), and let $Z=\ip{X,Y}\mbox{mod m}$ for some $m\in\mathbb{N}$. Then for every $\delta\in[0,1]$, the random variable $(Y,Z)$ is $\delta$-close to $(Y,U)$ where $U$ is uniform on $\mathbb{Z}_m$ and independent of $Y$, provided that
% 	$$
% 	n\geq c\cdot\frac{m^2}{\alpha\beta}\cdot\log(\frac{m}{\beta})\cdot\log(\frac{m}{\delta}).
% 	$$
% \end{theorem}



\Enote{I removed the definition of DP since it already appears in the intro}
\remove{
\subsection{Differential Privacy}\label{sec:prelim:DP}
We use the following standard definition of (information theoretic) differential privacy, due to \citet{DMNS06}. For notational convenience, we focus on databases over $\oo$.
\begin{definition}[Differentially private mechanisms]\label{def:mech}
	A randomized function $f\colon\oo^n\mapsto \zs$ is an {\sf $n$-size, $(\eps,\delta)$-differentially private mechanism} (denoted $(\eps,\delta)$-\DP) if for every neighboring $w,w'\in \oo^n$ and every function $g\colon \zs\mapsto \zo$, it holds that 
	$$
	\pr{g(f(w))=1}\leq e^{\eps}\cdot \pr{g(f(w'))=1} +\delta.
	$$ 	
	If $\delta=0$, we omit it from the notation.
\end{definition}
}


\subsubsection{Computational Differential Privacy}
There are several ways for defining computational differential privacy (see \cref{sec:related-works}). We use the most relaxed version due to \cite{BNO08}. For notational convenience, we focus on databases over $\oo$.
\begin{definition}[Computational differentially private mechanisms]\label{def:ComMech}
	A randomized function ensemble $f=\set{f_\pk\colon\oo^{n(\pk)}\mapsto \zs}$ is an {\sf $n$-size, $(\eps,\delta)$-computationally differentially private} (denoted $(\eps,\delta)$-$\CDP$) if for every poly-size circuit family $\set{\Ac_\pk}_{\pk\in \N}$, the following holds for every large enough $\pk$ and every neighboring $w,w'\in\oo^{n(\pk)}$:
	$$
	\pr{\Ac_\pk(f_\pk(w))=1}\leq e^{\eps(\pk)}\cdot \pr{\Ac_\pk(f_\pk(w'))=1} +\delta(\pk).
	$$ 
	If $\delta(\pk) = \negl(\pk)$, we omit it from the notation. 
\end{definition}



\subsubsection{Two-Party Differential Privacy}\label{sec:DP}
In this section we formally define distributed differential privacy mechanism (\ie protocols). %For the ease of notation, we consider protocol with no common input.

\begin{definition}\label{def:DP}%\Nnote{fix security parameter}
	A two-party protocol $\Pi=(\Ac,\Bc)$ is {\sf $(\eps,\delta)$-differentially private}, denoted $(\eps,\delta)$-$\DP$, if the following holds for every algorithm $\Dc$: let $\V^\Pc(x,y)(\pk)$ be the view of party $\Pc$ in a random execution of $\Pi(x,y)(1^\pk)$. Then for every $\pk,n \in \N$, $x\in \oo^n$ and neighboring $y,y'\in\oo^n$:
	\begin{align*}
	\pr{\Dc(V^\Ac(x,y)(\pk))=1}\le e^{\eps(\pk)}\cdot \pr{\Dc(V^\Ac (x,y')(\pk))=1}+\delta(\pk),
	\end{align*} 
	and for every $y\in \oo^n$ and neighboring $x,x'\in\oo^{n}$:
	\begin{align*}
	\pr{\Dc(V^\Bc(x,y)(\pk))=1}\le e^{\eps(\pk)}\cdot \pr{\Dc(V^\Bc (x',y)(\pk))=1}+\delta(\pk).
	\end{align*} 	
	Protocol $\Pi$ is {\sf $(\eps,\delta)$-computational differentially private}, denoted $(\eps,\delta)$-$\CDP$, if the above inequalities only hold for a non-uniform \ppt $\Dc$ and large enough $\pk$. We omit $\delta = \negl(\pk)$ from the notation. \footnote{Note that define we give for two-party differentially private protocols is a semi-honest definition, in which we ask for the security to hold when the parties interact in an honest execution of the protocol. Since we are proving a lower bound, starting from this weaker guarantee (as opposed to security against malicious players), yields a stronger result.}
\end{definition}
%We omit $\delta$ from the notation if $\delta$ is a negligible function of $n$.

%\Enote{simulation-based}
\begin{remark}[The definition for computational differential privacy we use]\label{rem:comDPChannel} 
	An alternative, stronger definition of computational differential privacy, known as simulation-based computational differential privacy, requires that the distribution of each party’s view be computationally indistinguishable from a distribution that ensures privacy in an information-theoretic sense. \cref{def:DP} is a weaker notion in comparison. Consequently, establishing a lower bound for a protocol that satisfies this weaker guarantee (as we do in this work) yields a stronger result.%Actually, our lower bound only requires the privacy to hold against \emph{uniform} external observer.
	%\Nnote{Maybe add: When only interesting in \Dp against external observer, the two definitions can be achieve using key-agreement and (single-party) \Dp mechanism. }
\end{remark}




\subsection{Useful Claims}
\remove{
In this section, we state generic lemmas and propositions that we will use later in our proofs.

The following lemma which we prove in \cref{sec:missing-proofs:distance-I}, measures the distance between two uniform stings conditioned one a random index $i$ either being fixed to $0$ or to $1$.

\def\distanceILemma{
    Let $R \la \zo^n$. For any (randomized) function $f:\{0,1\}^n\rightarrow \{0,1\}$ and $\alpha > 0$, it holds that
    \begin{align}\label{eq:f-alpha}
        \ppr{i \la [n]}{\size{\:\ex{f(R) \mid R_i = 0}-\ex{f(R) \mid R_i = 1}\:}\geq \alpha} \leq \frac{2}{n \alpha^2},
    \end{align}
    where the expectations are taken over $R$ and the randomness of $f$.
}

\begin{lemma}\label{lem:distance-I}
    \distanceILemma
\end{lemma}
}

The following two propositions state that given the output of a differentially private function, it is not possible to predict well even a random index (even if all other indexes are leaked). The first proposition handles the information-theoretic case and the second handles the computation case. Both propositions are proven in \cref{sec:missing-proofs:hard-to-guess}. 

\def\propHardToGuessInf{
    Let $f\colon \oo^n \rightarrow \cY$ be an $(\eps,\delta)$-\DP function, let $g \colon [n] \times \oo^{n-1} \times \cY \rightarrow \set{-1,1,\bot}$ be a (randomized) function, and let $X = (X_1,\ldots,X_n) \la \oo^n$. Then the following holds for every $i \in [n]$ where $X_i^* = g(i,X_{-i},f(X_1,\ldots,X_n))$:
    \begin{align*}
        \pr{X_i^* = X_i} \leq e^{\eps}\cdot \pr{X_i^* = -X_i} + \delta.
    \end{align*}
}

\begin{proposition}\label{prop:hard-to-guess-inf}
    \propHardToGuessInf
\end{proposition}


\def\propHardToGuessComp{
    Let $f = \set{f_{\pk} \colon \oo^{n(\pk)} \rightarrow \zo^{m(\pk)}}_{\pk \in \bbN}$ be an $(\eps,\delta)$-\CDP function ensemble, and let $\set{g_{\pk}}_{\pk \in \bbN}$ be a poly-size circuit family. Then, for large enough $\pk$ and $X = (X_1,\ldots,X_{n(\pk)}) \la \oo^{n(\pk)}$, the following holds for every $i \in [n(\pk)]$ where $X_i^* = g_{\pk}(i,X_{-i},f_{\pk}(X_1,\ldots,X_n))$:
    \begin{align*}
        \pr{X_i^* = X_i} \leq e^{\eps(\pk)}\cdot \pr{X_i^* = -X_i} + \delta(\pk).
    \end{align*}
}

\begin{proposition}\label{prop:hard-to-guess-comp}
    \propHardToGuessComp
\end{proposition}





\remove{
\Enote{Chao's old statement:}
\begin{lemma}\label{lem:distance-I-old}
        Let $R \la \zo^n$. 
	For any function $f:\{0,1\}^n\rightarrow \{0,1\}$ and $\alpha<0.01$, it holds that
	$$
	\Pr_{i\la[n]}\left[\: \size{\:\mathbb{E}[f(R) \mid R_i = 0]-\mathbb{E}[f(R) \mid R_i = 1]\:}\geq \alpha\right]\leq \frac{2+2\log(\frac{1}{\alpha})}{n\alpha^2}.
	$$
\end{lemma}
\begin{proof}
	Define $S_1=\{r \in \zo^n \colon f(r)=1\}$. Then for any $i\in[n]$, we have
	$$
	\begin{array}{rl}
		\size{\mathbb{E}[f(R) \mid R_i = 0]-\mathbb{E}[f(R) \mid R_i = 1]}
		&=\size{\Pr[R\in S_1|R_i=0]-\Pr[R\in S_1|R_i=1]}\\
		&=\size{\frac{\Pr[R_i=0|R\in S_1]\cdot\Pr[R\in S_1]}{\Pr[R_i=0]}-\frac{\Pr[R_i=1|R\in S_1]\cdot\Pr[R\in S_1]}{\Pr[R_i=1]}}\\
		&=\frac{2\size{S_1}}{2^n}\size{\Pr[R_i=0|R\in S_1]-\Pr[R_i=1|R\in S_1]}
	\end{array}
	$$
	When $|S_1|\leq \alpha\cdot 2^{n-1}$, we have $\size{\mathbb{E}[f(R) \mid R_i = 0]-\mathbb{E}[f(R) \mid R_i = 1]}\leq\frac{2\size{S_1}}{2^n}\leq \alpha$ for any $i\in[n]$. Hence, in the following, we assume $|S_1|> \alpha\cdot 2^{n-1}$.

	%Define $I_{bad}=\{i|\size{\Pr[R_i=0|R\in S_1]-\Pr[R_i=1|R\in S_1]}>2\alpha\}$ and $k=\size{I_{bad}}$, then for any $i\notin I_{bad}$, we have 
    %$$
    %\begin{array}{rl}
    %    2\alpha&\geq \size{\Pr[R_i=0|R\in S_1]-\Pr[R_i=1|R\in S_1]}\\
    %    &=\size{\frac{\Pr[R\in S_1|R_i=0]\cdot\Pr[R_i=0]}{\Pr[R\in S_1]}-\frac{\Pr[R\in S_1|R_i=1]\cdot\Pr[R_i=1]}{\Pr[R\in S_1]}}\\
    %    &=\size{\Pr[R\in S_1|R_i=0]-\Pr[R\in S_1|R_i=1]}\cdot\frac{1}{2\Pr[R\in S_1]}\\
    %    &\geq \size{\mathbb{E}[f(R) \mid R_i = 0]-\mathbb{E}[f(R) \mid R_i = 1]}\cdot \frac{1}{2},
    %\end{array}
    %$$ 
    %where the last inequality is because $\Pr[R\in S_1]\leq 1$. So that $\size{\mathbb{E}}[f(R) \mid R_i = 0]-\mathbb{E}[f(R) \mid R_i = 1]\leq %4\alpha$.
    Define $I_{bad}=\{i \colon \size{\Pr[R_i=0|R\in S_1]-\Pr[R_i=1|R\in S_1]} \geq 2\alpha\}$ and $k=\size{I_{bad}}$, and denote $I_{bad}=\{i_1,\dots,i_k\}$. Define $(X_{i_1}, \ldots X_{i_k}) = (R_{i_1},\dots,R_{i_k})\mid_{R \in S_1}$. 
    Consider the min-entropy
	$$
	\begin{array}{rl}
		H_{min}(X_{i_1},\dots,X_{i_k})&\leq H(X_{i_1},\dots,X_{i_k})\\
		&\leq \sum_{j=1}^k H(X_{i_j})\\
		&\leq k\cdot \left(-(\frac{1}{2}+2\alpha)\cdot\log(\frac{1}{2}+2\alpha)-(\frac{1}{2}-2\alpha)\cdot\log(\frac{1}{2}-2\alpha)\right)\\
            &=k\cdot \left(-(\frac{1}{2}+2\alpha)\cdot(\log(1+4\alpha)-1)-(\frac{1}{2}-2\alpha)\cdot(\log(1-4\alpha)-1)\right)\\
            &=k\cdot \left(1-(\frac{1}{2}+2\alpha)\cdot\log(1+4\alpha)-(\frac{1}{2}-2\alpha)\cdot\log(1-4\alpha)\right),
		
	\end{array}
	$$
	where $H_{min}(Y)$ is the minimum entropy of $Y$ and $H(Y)$ is the Shannon entropy of $Y$.\Enote{add to preliminaries.}
        The third inequality holds since by the definition of $I_{bad}$, for every $j \in [k]$ it holds that $\size{\pr{X_{i_j} = 1}-\pr{X_{i_j} = 0}} > 2\alpha$, and therefore $H(X_{i_j}) \leq H(1/2 + 2\alpha)$\Enote{define}.
	
	Therefore, there exists $b_1,\dots,b_k\in\{0,1\}$, such that 
	
	\begin{align}\label{eq:min-entropy-result}
		\Pr\left[(R_{i_1},\ldots,R_{i_k}) = (b_1,\ldots,b_k) \mid R\in S_1\right]
		&= \pr{(X_{i_1},\ldots,X_{i_k}) = (b_1,\ldots,b_k)}\\
		&= 2^{-H_{min}(X_{i_1},\dots,X_{i_k})}\nonumber\\
		&\geq 2^{k\cdot \left(-1+(\frac{1}{2}+2\alpha)\cdot\log(1+4\alpha)+(\frac{1}{2}-2\alpha)\cdot\log(1-4\alpha)\right)}.\nonumber
	\end{align}
	
	Let $S_{bad}=\{r \in \zo^n  \colon \set{(r_{i_1},\ldots,r_{i_k}) = (b_1,\ldots,b_k)} \land \set{r\in S_1}\}$.
	It holds that
	\begin{align*}
		|S_{bad}|
		&= \size{S_1} \cdot \Pr\left[(R_{i_1},\ldots,R_{i_k}) = (b_1,\ldots,b_k) \mid R\in S_1\right]\\
		&\geq \alpha\cdot 2^{n-1}\cdot2^{k\cdot \left(-1+(\frac{1}{2}+2\alpha)\cdot\log(1+4\alpha)+(\frac{1}{2}-2\alpha)\cdot\log(1-4\alpha)\right)},
	\end{align*} 
	where the inequality holds by \cref{eq:min-entropy-result} and since $\size{S_1} \geq \alpha\cdot 2^{n-1}$.
	Notice that any string in $S_{bad}$ depends on at most $n-k$ bits. It implies that $|S_{bad}|\leq 2^{n-k}$. Therefore, we have
	$$
	\begin{array}{rl}
		&2^{n-k}\geq \alpha\cdot 2^{n-1}\cdot2^{k\cdot \left(-1+(\frac{1}{2}+2\alpha)\cdot\log(1+4\alpha)+(\frac{1}{2}-2\alpha)\cdot\log(1-4\alpha)\right)} \\
		\Rightarrow& n-k \geq \log \alpha+n-1+k\cdot \left(-1+(\frac{1}{2}+2\alpha)\cdot\log(1+4\alpha)+(\frac{1}{2}-2\alpha)\cdot\log(1-4\alpha)\right)\\
		\Rightarrow& 1-\log \alpha \geq k\cdot((\frac{1}{2}+2\alpha)\cdot\log(1+4\alpha)+(\frac{1}{2}-2\alpha)\cdot\log(1-4\alpha))\\
		\Rightarrow& 1-\log \alpha \geq k\cdot(4\alpha\cdot\log(1+4\alpha)+(\frac{1}{2}-2\alpha)\cdot\log(1-16\alpha^2))\\
        \Rightarrow& 1-\log\alpha \geq k\cdot(15.9\alpha^2-8\alpha^2+32\alpha^3)=k\cdot(7.9\alpha^2+32\alpha^3)>0.5k\alpha^2\\
		\Rightarrow& k\leq \frac{2-2\log \alpha}{\alpha^2} = \frac{2+2\log (1/\alpha)}{\alpha^2},
	\end{array}
	$$
	Where the third transition holds since 
	\begin{align*}
		\lefteqn{(\frac{1}{2}+2\alpha)\cdot\log(1+4\alpha)+(\frac{1}{2}-2\alpha)\cdot\log(1-4\alpha)}\\
		&= 4\alpha\cdot\log(1+4\alpha) + (\frac{1}{2}-2\alpha)\paren{\log(1+4\alpha)+\log(1-4\alpha)}\\
		&= 4\alpha\cdot\log(1+4\alpha)+(\frac{1}{2}-2\alpha)\cdot\log(1-16\alpha^2),
	\end{align*}
	and the forth transition holds since $4\alpha\cdot\log(1+4\alpha)+(\frac{1}{2}-2\alpha)\cdot\log(1-16\alpha^2) > 15.9\alpha^2-8\alpha^2+32\alpha^3$ for $\alpha < 0.01$.
	Thus, we conclude that 
	$$
	\Pr_{i\la[n]}\left[\size{\mathbb{E}[f(R) \mid R_i=0]-\mathbb{E}[f(R) \mid R_i = 1]}\geq \alpha\right]\leq \frac{k}{n}\leq \frac{2+2\log (1/\alpha)}{n\alpha^2}.
	$$
\end{proof}
}


\subsection{Channels and Two-Party Protocols}\label{sec:protocol}

\paragraph{Channels.}A channel is simply a distribution of a pair of tuples defined as follows. 
\begin{definition}[Channels]\label{def:channel} A {\sf channel} $C_{(X,U)(Y,V)}$ of size $\isize$ over alphabet $\Sigma$ is a probability distribution over $(\Sigma^\isize \times\zo^\ast) \times(\Sigma^\isize \times\zo^\ast)$. The ensemble $C_{(X,U)(Y,V)}= \set{C_{(X_\pk,U_\pk)(Y_\pk,V_\pk)}}_{\pk\in \N}$ is an $\isize$-size channel ensemble, if for every $\pk\in \N$, $C_{(X_\pk,U_\pk)(Y_\pk,V_\pk)}$ is an $\isize(\pk)$-size channel. %We denote a channel of size one by a \emph{single-bit} channel. 
We refer to $X$ and $Y$ as the {\sf local outputs}, and to $U$ and $V$ as the {\sf views}.	
\end{definition}

We view a  channel as the experiment in which there are two parties $\Ac$ and $\Bc$.  Party $\Ac$ receives ``output'' $X$ and ``view'' $U$, and party $\Bc$ receives ``output'' $Y$ and ``view'' $V$. Unless stated otherwise, the channels we consider are over the alphabet $\Sigma = \oo$. We naturally identify channels with the distribution that characterizes their output.








\subsubsection{Two-Party Protocols}

A two-party protocol $\Pi=(\Ac,\Bc)$ is \ppt if the running time of both parties is polynomial in their input length. We let $\Pi(x,y)(z)$ or $(\Ac(x),\Bc(y))(z)$ denote a random execution of $\Pi$ on a common input $z$, and private inputs $x,y$.%We assume \wlg that a protocol has a common output (part of its transcript).\Jnote{This is not really the case we consider in this paper..}

\begin{definition}[Oracle-aided protocols]\label{def:ChannelAidedProtocol}
	In a two-party protocol $\Pi$ with oracle access to a {\sf protocol} $\Psi$, denoted $\Pi^\Psi$, the parties make use of the \textit{next-message function} of $\Psi$.\footnote{The function that on a partial view of one of the parties, returns its next message.} In a two-party protocol $\Pi$ with oracle access to a {\sf channel} $C_{Z W}$, denoted $\Pi^C$, the parties can jointly invoke $C$ for several times. In each call, an independent pair $(z,w)$ is sampled according to $C_{Z W}$, one party gets $z$, the other gets $w$.
\end{definition}


\begin{definition}[The channel of a protocol]\label{def:ChannlOfProtocol}
	For a no-input two-party protocol $\Pi= (\Ac,\Bc)$, we associate the channel $C_\Pi$, defined by $\C_\Pi= C_{(X, U),(Y, V)}$, where $X$ and $Y$ are the local outputs of $\Ac$ and $\Bc$ (respectively) and
	$U$ and $V$ are the local views of $\Ac$ and $\Bc$ (respectively).
    
	For a two-party protocol $\Pi$ that gets a security parameter $1^\pk$ as its (only, common) input, we associate the channel ensemble $ \set{C_{\Pi(1^\pk)}}_{\pk\in \N}$. 
\end{definition}

\begin{definition}[$(\alpha,\gamma)$-Accurate channel]\label{def:accurate-func}
	A channel $C = C_{(X, U),(Y, V)}$ is {\sf $(\alpha,\gamma)$-accurate for the function $f$}, if $\ppr{C}{\size{\out(V)-f(X,Y)}\leq \alpha}\ge \gamma$, where $\out(V)$ is the designated output.
    A channel ensemble $C_{(X, U),(Y, V)}= \set{C_{(X_\pk, U_\pk),(Y_\pk, V_\pk)}}_{\pk\in \N}$ is  $(\alpha,\gamma)$-accurate for  $f$ if $C_{(X_\pk, U_\pk),(Y_\pk, V_\pk)}$ is $(\alpha(\pk),\gamma(\pk))$-accurate for $f$, for every $\pk \in \N$.
\end{definition}

\subsubsection{Differentially Private Channels}\label{sec:DPChannel}
Differentially private channels are naturally defined as follows:
\begin{definition}[Differentially private channels]\label{def:DPChannel}
	An $n$-size channel $C = C_{(X, U),(Y, V)}$ with $X, Y$ over $\oo^n$ 
	is {\sf$(\eps,\delta)$-differentially private} (denoted $(\eps,\delta)$-$\DP$) if for every $x \in \Supp(X)$ there exists an $n$-size $(\eps,\delta)$-$\DP$ mechanisms $\Mc_x$ such that $(X,Y,U) \equiv (X,Y,\Mc_X(Y))$, and for every $y \in \Supp(Y)$ there exists an $n$-size $(\eps,\delta)$-$\DP$ mechanisms $\Mc_y'$ such that $(X,Y,V) \equiv (X,Y,\Mc_Y'(X))$. In addition, we say that the channel is \emph{uniform} if $X$ and $Y$ are independent random variables uniformly distributed in $\oo^n$. 
\end{definition}

\begin{definition}[Computational differentially private channels]\label{def:CDPChannel}
	An $n$-size channel ensemble $C = \set{C_{(X_\pk, U_\pk),(Y_\pk, V_\pk)}}_{\pk\in\N}$ with $X_\pk, Y_\pk$ over $\oo^n$ 
	is {\sf$(\eps,\delta)$-computationally differentially private} (denoted $(\eps,\delta)$-$\CDP$) if for every ensemble $\set{x_\pk \in \Supp(X_\pk)}_{\pk\in\N}$ there exists an $n$-size $(\eps,\delta)$-\CDP mechanisms ensemble $\set{\Mc_{x_\pk}}_{\pk\in\N}$ such that $(X_\pk,Y_\pk,U_\pk) \equiv (X_\pk,Y_\pk,\Mc_{X_\pk}(Y_\pk))$, for every $\pk\in\N$, and for every ensemble $\set{y_\pk \in \Supp(Y_\pk)}_{\pk\in\N}$ there exists an $n$-size $(\eps,\delta)$-$\CDP$ mechanisms ensemble $\set{\Mc'_{y_\pk}}_{\pk\in\N}$ such that $(X_\pk,Y_\pk,V_\pk) \equiv (X_\pk,Y_\pk,\Mc_{Y_\pk}'(X_\pk))$ for every $\pk\in \N$. In addition, we say that the channel is \emph{uniform} if $X_\pk$ and $Y_\pk$ are independent random variables uniformly distributed in $\{\pm 1\}^n$ for all $\pk\in\N$.
\end{definition}




% \begin{lemma}~\label{lem:dp-sv-source}
% 	Let $P$ be an $\varepsilon$-DP randomized protocol. Let $X$ and $Y$ be independent random variables uniformly distributed in $\{\pm 1\}^n$ and let random variable $\Pi(X,Y)$ denote the transcript of running $P(X,y)$. Then for every $\pi\in Supp(\Pi)$, the random variables corresponding to the inputs conditioned on transcript $\pi$, $X_\pi$ and $Y_\pi$, are independent $e^{-\varepsilon}$-strong SV source.
% \end{lemma}





\subsubsection{Weak Erasure Channel (\WEC)}

\begin{definition}[\WEC]\label{def:WEC}
	A channel $((O_A,V_A), (O_B,V_B))$ with $O_A \in \set{0,1}$ and $O_B \in \set{0,1,\bot}$ is a {\sf weak erasure channel}, denoted $(\alpha,p,q)$-$\WEC$, if:
	\begin{itemize}
		%\item $O_A\in \set{-1,1}$ and $O_B\in \set{-1,1,\bot}$.
		\item Random erasure: $\pr{O_B = \perp} = 1/2$.
		
		\item Agreement: $\pr{O_A\ne O_B\mid O_B\ne \bot}\le \alpha$.
		
		\item Secrecy:
		
		\begin{enumerate}
			\item For every algorithm $\Dc$ it holds that\label{WEC:item:A}
			\begin{align*}
				%\size{\pr{\Ac(O_A,V_A) = 1 \mid O_B \neq \perp} - \pr{\Ac(O_A,V_A) = 1 \mid O_B = \perp}} \le p
				\size{\pr{\Dc(V_A) = 1 \mid O_B \neq \perp} - \pr{\Dc(V_A) = 1 \mid O_B = \perp}} \le p
			\end{align*}
			(Alice doesn't know if $O_B = \perp$.)
			
			\item For every algorithm $\Dc$ it holds that\label{WEC:item:B}
			\begin{align*}
				\pr{\Dc(V_B) = O_A \mid O_B=\bot} \leq \frac{1+q}{2}.
			\end{align*}
			(i.e., if $O_B=\bot$, Bob don't know what is the value of $O_A$).
			
			%\item $SD((O_A U|O_B=\bot),(O_A U|O_B\ne \bot))\le p$ (The sender don't know if $O_B=\bot$).
			
			%\item $SD(V O_A|O_B=\bot,V(-O_A)|O_B=\bot)\le q$ (If $O_B=\bot$, Bob don't know what the value of $O_A$).
		\end{enumerate}
	\end{itemize}
   We say that a channel ensemble $C=\set{C_\pk}_{\pk\in N}$ is a {\sf computational weak erasure channel}, denoted $(\alpha,p,q)$-\CompWEC, if for every \ppt algorithm $\Dc$ and every sufficiently large $\pk\in\N$, $C_\pk$ satisfies the properties stated in the items above, where the secrecy property holds with respect to a \ppt algorithm $\Dc$. A protocol $\Lambda$ is said to be $(\alpha,p,q)$-$\CompWEC$, if the ensemble induces by the protocol (that is, $C=\set{C_{\Lambda(\pk)}}_{\pk\in\N}$) is $(\alpha,p,q)$-$\CompWEC$.  
\end{definition}



\subsubsection{Approximate Weak Erasure Channel (\AWEC)}\label{sec:AWEC}

\begin{definition}[\AWEC]\label{def:AWEC}
	A channel $C = ((O_A,V_A), (O_B,V_B))$ over $([-n,n] \times \zo^*) \times (([-n,n] \cup \bot)  \times \zo^*)$ is an {\sf approximate weak erasure channel}, denoted $(\ell,\alpha,p,q)$-\AWEC if:
	\begin{itemize}
		
		\item Random erasure: $\pr{O_B = \perp} = 1/2$.
		
		\item Accuracy: $\pr{\size{O_A - O_B} > \ell \mid O_B \ne \bot}\le \alpha$.
		
		\item Secrecy:
		
		\begin{enumerate}
			\item For every algorithm $\Dc$ it holds that\label{AWEC:item:A}
			\begin{align*}
				%\size{\pr{\Ac(O_A,V_A) = 1 \mid O_B \neq \perp} - \pr{\Ac(O_A,V_A) = 1 \mid O_B = \perp}} \le p
				\size{\pr{\Dc(V_A) = 1 \mid O_B \neq \perp} - \pr{\Dc(V_A) = 1 \mid O_B = \perp}} \le p
			\end{align*}
			(Alice doesn't know if $O_B=\bot$).
			
			\item For every algorithm $\Dc$ it holds that\label{AWEC:item:B}
			\begin{align*}
				\pr{\size{\Dc(V_B) - O_A} \leq 1000 \ell \mid O_B=\bot} \leq q.
			\end{align*}
			(i.e., if $O_B=\bot$, Bob can't estimate the value of $O_A$ with error $\leq 1000 \ell$).
		\end{enumerate}
	\end{itemize}
     We say that a channel ensemble $C=\set{C_\pk}_{\pk\in N}$ is a {\sf computational approximate weak erasure channel}, denoted $(\ell,\alpha,p,q)$-\CompAWEC, if for every \ppt algorithm $\Dc$ and every sufficiently large $\pk\in\N$, $C_\pk$ satisfies the properties stated in the items above. A protocol $\Gamma$ is said to be $(\ell,\alpha,p,q)$-$\CompAWEC$, if the ensemble induced by the protocol (that is, $C=\set{C_{\Gamma(\pk)}}_{\pk\in\N}$) is $(\ell,\alpha,p,q)$-$\CompAWEC$.  
\end{definition}

We will make use of the following lemma, which shows that for some choices of the parameters, \AWEC implies \WEC. The lemma is proven in \cref{sec:AWEC-to-WEC}.

\begin{lemma}\label{lemma:AWEC-to-WEC}
	For every $\ell> 0$, there exists a \ppt protocol $\Lambda = (\Pc_1,\Pc_2)$ such that given an oracle access to an $(\ell,\alpha,p,q)$-\AWEC $C$, the channel $\tilde{C}$ induced by $\Lambda^C$ is $(\alpha'=\alpha+0.001,\: p' = p ,\:  q' = 1/2 + 2(q+0.01))$-\WEC.
	Furthermore, the proof is constructive in a black-box manner:
	\begin{enumerate}
		\item There exists an oracle-aided \ppt algorithm $\Ec_1$ such that for every channel $C = ((\OA,\VA), (\OB,\VB))$ and algorithm $\Dc$ violating the \WEC secrecy property~\ref{WEC:item:A} of $\tilde{C}$, algorithm $\Ec_1^{\Dc}$ violates the \AWEC secrecy property~\ref{AWEC:item:A} of $C$.
		
		\item There exists an oracle-aided \ppt algorithm $\Ec_2$ such that for every channel $C = ((\OA,\VA), (\OB,\VB))$ and algorithm $\Dc$ violating the \WEC secrecy property~\ref{WEC:item:B} of $\tilde{C}$, algorithm $\Ec_2^{\Dc}$ violates the \AWEC secrecy property~\ref{AWEC:item:B} of $C$.
	\end{enumerate}
\end{lemma}

Since \cref{lemma:AWEC-to-WEC} is constructive, the following is an immediate corollary.
\begin{corollary}\label{cor:CompAWEC to CompWEC}
There exists an oracle aided \ppt protocol $\Lambda$, such that given a protocol $\Gamma$ that induces $(\ell,\alpha,p,q)$-\CompAWEC, it holds that $\Lambda^\Gamma$ is $(\alpha'=\alpha+0.001,\: p' = p ,\:  q' = 1/2 + 2(q+0.01))$-\CompWEC.  
\end{corollary}
\begin{proof}[Proof of \ref{cor:CompAWEC to CompWEC}]
Let $\Lambda$ be the \ppt algorithm guaranteed  by Lemma \ref{lemma:AWEC-to-WEC}. Given an $(\ell,\alpha,p,q)$-\CompAWEC protocol $\Gamma$, we define $\Lambda(\pk)={\Lambda^{\Gamma(\pk)}(\pk)}$. Assume towards a contradiction that $\Lambda$ is not a $(\alpha',p',q')$-\CompWEC. It follows that there exists a \ppt $\Dc$ that for infinity many $\pk\in\N$ contradicts one of the \WEC secrecy properties of channel ensemble $\set{C_{\Lambda(\pk)}}_{\pk\in\N}$. Fix $\pk\in\N$ for which this holds. By Lemma \ref{lemma:AWEC-to-WEC}, there exists a \ppt $\Ec^\Dc$ that for every such $\pk$  contradicts one of the secrecy properties of the channel $C_{\Gamma(\pk)}$. This implies that for infinity many $\pk\in\N$, $\Ec^\Dc$  contradict the secrecy of the channel ensemble $\set{C_{\Gamma(\pk)}}_{\pk\in\N}$, which is a contradiction since this would means that $\Gamma$ is not a $(\ell,\alpha,p,q)$-\CompAWEC.       
\end{proof}



\subsection{Oblivious Transfer (\OT)}

\paragraph{Secure Computation.}
We use the standard notion of securely computing a functionality, \cf  \cite{Goldreich04}.
\begin{definition}[Secure computation]\label{def:SFE}
	A two-party protocol {\sf securely computes a functionality $f$}, if it does so according to the real/ideal paradigm.   We add the term perfectly/statistically/computationally/non-uniform computationally, if the simulator's output is  perfect/statistical/computationally indistinguishable/  non-uniformly indistinguishable from  the real distribution.  The protocol have the above notions of security {\sf against semi-honest  adversaries}, if its security only  guaranteed to holds against an adversary that follows the prescribed protocol.   Finally, for the case of perfectly secure computation, we naturally apply the above notion also to the non-asymptotic case: the protocol with no security parameter perfectly  compute a functionality $f$.
	
	A two-party protocol {\sf securely computes a functionality ensemble $f$ with oracle to a channel $C$}, if it does so according to the above definition when the parties have access to a trusted party computing $C$. All the above adjectives naturally extend to this setting.
\end{definition}

\paragraph{Oblivious Transfer.}
The (one-out-of-two) oblivious transfer functionality is defined as follows.
\begin{definition}[oblivious transfer functionality $f_{\OT}$]\label{def:OTfunc}
	The oblivious transfer functionality over $\zo \times (\zs)^2$ is defined by  $f_{\OT} (i,(\sigma_0,\sigma_1)) = (\perp,\sigma_i)$.
\end{definition}
A protocol is $\ast$ secure OT,   for \\$\ast\in \set{\text{semi-honest statistically/computationally/computationally non-uniform}}$, if it  compute the $f_{\OT}$  functionality with $\ast$ security.





% \begin{definition}[Computational oblivious transfer, semi-honest model]
% A protocol $\Pi=(\Ac,\Bc)$ is a semi-honest 1-out-of-2 computational oblivious transfer (comp-OT) protocol if the following holds. Given a common input $1^{\pk}$, the parties $\Ac$ and $\Bc$ run the protocol $\Pi(1^\pk)$ (in an honest manner) and    
% $\Ac$ outputs $X=(m_1,m_2)\in \zo\times\zo$ and has a view $U$ and $\Bc$ outputs $Y=(i,\hat{m})\in\zo\times\zo$ and has a view $V$, and the following properties are satisfied:
% \begin{enumerate}
%     \item \textbf{Correctness:} 
%     $\pr{\hat{m}\neq m_i}<\negl(\pk).$ 
    
%     \item \textbf{A's Privacy:} For every \ppt $\Dc$ and every sufficiently large $\pk$:
%     $\pr{\Dc(V)=m_{i-1}}<(1+\negl(\pk))/2$
    
%     \item \textbf{B's Privacy:} For every \ppt $\Dc$ and every sufficiently large $\pk$:
%     $\pr{\Dc(U)=i}<(1+\negl(\pk))/2$  
% \end{enumerate}
% \end{definition}

We make use of the following useful results by Wullschleger on oblivious transfer amplification from weak channels.
\begin{theorem}[\cite{Wullschleger09}, from \WEC to statistically secure \OT]\label{thm:WEC TO OT IT}
    There exists an oracle aided protocol $\Pi$ such that the following holds: Given a $(\alpha,p,q)$-\WEC $C$, if $44(\alpha+p)\le 1-q$ then $\Pi^{C}(1^\pk)$ is a semi-honest statistically secure \OT.
\end{theorem}

The following computational version of \cref{thm:WEC TO OT IT} is implicit in \cite{Wullschleger09} and is based on the computational proof explicitly stated in \cite{Wul07} (see Section 6 in \cite{Wullschleger09} for discussion).   

\begin{theorem}[\cite{Wullschleger09,   Wul07}, from \CompWEC to computinally secure \OT]\label{thm:WEC TO OT Comp}
    There exists an oracle aided protocol $\Pi$ such that the following holds: Given a $(\alpha,p,q)$-\CompWEC protocol $\Lambda$, if $44(\alpha+p)\le 1-q$ then $\Pi^{\Lambda}$ is a semi-honest computational secure \OT.
\end{theorem}



% \begin{definition}[Computational 1-out-of-2 Oblivious Transfer, semi-honest model]
% A protocol $\Pi=(\Ac,\Bc)$ is a semi-honest 1-out-of-2 $(\eps,\alpha,\beta)$-oblivious transfer (OT) protocol if the following holds. 

% The parties $\Ac$ and $\Bc$ run the protocol (in an honest manner) and    
% $\Ac$ outputs $X=(m_1,m_2)\in \zo\times\zo$ and has a view $U$ and $\Bc$ outputs $Y=(i,\hat{m})\in\zo\times\zo$ and has a view $V$, and following properties are satisfied:
% \begin{enumerate}
%     \item \textbf{Correctness:} 
%     $\pr{\hat{m}\neq m_i}<\eps.$ 
    
%     \item \textbf{A's Privacy:} For every adversary $\Dc$:
%     $\pr{\Dc(V)=m_{i-1}}<(1+\alpha)/2$
    
%     \item \textbf{B's Privacy:} For every adversary $\Dc$: $\pr{\Dc(U)=i}<(1+\beta)/2$  
% \end{enumerate}
% \end{definition}
\section{Motivation}
\label{sec:motivation}



% In LLM inference, not only does weight matter, but the memory requirements of the KV Cache are also considerable.
In this section, we first demonstrate that the emerging paradigm of group quantization demands a high level of adaptivity, which current adaptive methods lack.
We then discuss how adapting these methods to group quantization could compromise their efficiency.
Given that LLMs generate KV caches during runtime, real-time quantization capability is crucial.
These challenges lead to our proposal of a mathematical adaptive numerical type (\texttt{MANT}), which we will detail later.



\begin{figure}[t]
    \centering
    \begin{minipage}[t]{0.48\columnwidth}
      \centering
      \includegraphics[width=\columnwidth]{fig/moti_group_ppl.pdf}
      \caption{LLM accuracy with different quantization granularities. We report the perplexity (PPL) metric (lower is better).}\label{fig:moti_group_ppl} 
    \end{minipage}
    \hspace{2pt}
    \begin{minipage}[t]{0.48\columnwidth}
      \centering
      \includegraphics[width=\columnwidth]{fig/motivation_adaptive_ppl.pdf}
      \caption{Accuracy loss for \texttt{INT}, \texttt{ANT}, and Ideal (clustering algorithm K-Means) adaptive methods in group quantization. }\label{fig:moti_ppl} 
    \end{minipage}
    % \vspace*{-0.3cm}
\end{figure}




\subsection{Group Quantization Accuracy Analysis}
\label{sec:acc_analysis}

In this subsection, we begin by comparing the accuracy of traditional channel-wise quantization with group-wise quantization~\cite{shao2024omniquant,zhao2023atom,liu2024kivi,sheng2023flexgen,lin2023awq,zhao2023atom}, establishing the baseline for group-wise quantization in this study.
We then delve into the use of various adaptive data types in group quantization, emphasizing the necessity for full adaptivity.



\Fig{fig:moti_group_ppl} illustrates the perplexity when quantizing the LLaMA-7B model~\cite{touvron2023llama} with various granularities using the \texttt{INT4}-based symmetric quantization.
Channel-wise quantization significantly worsens the perplexity of the examined LLM, increasing it from 5.68 to 6.85.
Conversely, group-wise quantization mitigates this loss in perplexity with a group size of 128, corresponding to an average of 4.125 bits per element (16-bit scaling factor).
Additionally, we observe that a smaller group size of 32 offers only a slight improvement in perplexity, but the scaling factor overhead increases by $4\times$.



Given this analysis, we adopt a group size of 128 as our standard configuration for the remainder of this section.
Previous research indicates that the \texttt{INT} data type is not optimal for accuracy since tensors or channels exhibit varied distributions, leading to the proposal of various adaptive data types~\cite{guo2022ant, guo2023olive, zadeh2020gobo, zadeh2022mokey}.
We evaluate their efficacy in the context of group quantization, which falls into two main categories: data-type-based and clustering-based.



\textbf{Data-type-based adaptive methods} select data types from discrete sets based on tensor data distribution.
ANT~\cite{guo2022ant} is a representative example of the data-type-based method.
ANT packages several different data types for selection, including \texttt{INT} for the uniform distribution, \texttt{PoT} (Power of Two) for the Laplace distribution, and \texttt{flint} for the Gaussian distribution.
%ANT designed \texttt{flint} for Gaussian distributions.

\textbf{Clustering-based adaptive methods} utilize clustering algorithms to generate centroids that align with the data distribution and provide considerable adaptivity. 
Mokey~\cite{zadeh2022mokey} and GOBO~\cite{zadeh2020gobo} exemplify this approach, though they focus on tensor- or channel-wise quantization. In our study, we adapt them to group quantization through per-group clustering.

%Clustering-based methods employ clustering algorithms to generate centroids that fit the data distribution, demonstrating sufficient adaptivity.
%Mokey~\cite{zadeh2022mokey} and GOBO~\cite{zadeh2020gobo} are such presentative works, but only target tensor- or channel-wise quantization.
%In our work, we modify those works to support group quantization by performing per-group clustering.
\Fig{fig:moti_ppl} compares the accuracy of the methods described above for the LLaMA-7B model under 4-bit group-wise quantization. 
The group-wise \texttt{ANT} method outperforms the \texttt{INT} type by dynamically selecting from three data types to better match the value distribution, resulting in reduced perplexity (PPL) loss. 
Moreover, per-group clustering adjusts more effectively to the value distribution of each group, establishing itself as the accuracy-optimal and ideal adaptive method. 
This approach achieves nearly lossless 4-bit quantization, equivalent to 16 centroids per group. 
However, this ideal scenario is impractical due to the significant overhead associated with storing per-group centroids, effectively rendering it a 6-bit quantization.

\begin{figure}[t] 
    \centering 
    \includegraphics[width=1.0\linewidth]{fig/intro_cdf.pdf}  
    \caption{The cumulative distribution function (CDF) of the tensor, channel, and group, respectively. The tensor data were taken from layers 8 to 23, while the 16 channel and group data were sampled from one tensor with specific strides.}\label{fig:moti_dist} 
    %  \vspace*{-0.3cm}
\end{figure}

To illustrate the group-wise diversity in data distribution, we sampled the weights of the Q and V tensors in LLaMA-7B model. 
We normalized all sampled data to their absolute maximum values, which ranged from -1 to 1. \Fig{fig:moti_dist} displays the cumulative distribution function (CDF) for the tensor, channel, and group levels, respectively. 
We observed that the diversity at the group level is significantly higher than at the tensor level. 
In simpler terms, while different tensors exhibit similar distributions, groups can have markedly different distributions. This finding underscores the necessity for full adaptivity in group quantization to fully realize its potential.
\paragraph{Takeaway 1.} The group quantization is an emerging paradigm to accelerate LLMs, and the significant group-level diversity requires a high level of adaptivity to fully unleash its potential.

\subsection{Group Quantization Efficiency Analysis}
\label{subsec:efficiency}


In this subsection, we provide a detailed efficiency analysis for the above adaptive quantization methods.
In \Tbl{intro:dtype}, we compare OliVe~\cite{guo2023olive}, ANT~\cite{guo2022ant}, GOBO~\cite{zadeh2020gobo}, and Mokey~\cite{zadeh2022mokey} with \texttt{INT} regarding the efficiency of computation, encoding, and decoding. 
In this paper, we use the term encoding (decoding) interchangeably with quantization (dequantization).
 

Data-type-based adaptive methods such as ANT~\cite{guo2022ant} and Olive~\cite{guo2023olive} achieve computational efficiency comparable to \texttt{INT}. 
Both utilize specialized decoders that decode these data types prior to computation, resulting in high decoding efficiency. 
However, as previously demonstrated, these methods suffer from limited adaptivity in the group quantization paradigm. 
A straightforward approach to enhance adaptivity is to expand their set of data types. 
However, incorporating new data types necessitates additional decoders, escalating hardware design costs. 
Additionally, compatibility issues between new and existing data types may reduce computational efficiency. 
For instance, the \texttt{NF4} data type~\cite{dettmers2023qlora} requires an FP16 MAC unit, which is incompatible with existing \texttt{ANT} data types.


\paragraph{Takeaway 2.} Enhancing the data-type-based adaptive method for group quantization is challenging and requires a careful balance for the computation and decoding efficiency.

Clustering-based adaptive methods like GOBO~\cite{zadeh2020gobo} and Mokey~\cite{zadeh2022mokey} can sufficiently adapt to various distributions at the group level. 
However, they require codebooks for quantization and dequantization, leading to high adaptivity at the expense of encoding and computational efficiency. 
For instance, a 16-entry codebook with 8 bits per entry requires 128 bits per group, creating an inevitable trade-off between adaptivity and memory overhead. GOBO~\cite{zadeh2020gobo} employs the K-means algorithm to quantize weights and requires dequantization to \texttt{FP16} using a codebook lookup table before computation, resulting in high adaptivity but low computational efficiency. 
Conversely, Mokey~\cite{zadeh2022mokey} enhances the computation of clustering-based methods by using indices for centroid values via approximate calculations, though matrix multiplication still relies on floating-point units, increasing overhead compared to integer units. 
Furthermore, Mokey creates one \texttt{golden dictionary} for all activations and weights, akin to using a single data type in quantization, thus reducing adaptivity.


\paragraph{Takeaway 3.} Deploying the clustering-based adaptive methods under group quantization is challenging owing to the low encoding and computation efficiency. 


\begin{table}[t]
    \centering
    \small
    \renewcommand{\arraystretch}{1.2}
    \caption[]{Features of DNN accelerators with adaptive and flexible data types are summarized. Here, `Effi.' stands for efficiency, `Med.' for medium, and `LUT' for lookup table.}
  
    \resizebox{1.0\columnwidth}{!}{
      \begin{tabular}{c|cc|ccc|cc|c}
        \Xhline{1.2pt}
        \multirow{2}{*}{Architecture} & \multicolumn{2}{c|}{Encode} & \multicolumn{3}{c|}{Computation} & \multicolumn{2}{c|}{Decode} & \multirow{2}{*}{Adaptivity} \\ \cline{2-8}
        & Method & Effi. & Method & Bit & Effi. & Method & Effi. \\
        \Xhline{1.2pt}
        \texttt{INT} & Round & High & INT & 4 \& 8 & High & Calculation & High & Low \\ 
        OliVe~\cite{guo2023olive} & Search & Med. & INT & 4 \& 8 & High & Decoder & High & Med. \\ 
        ANT~\cite{guo2022ant} & Search & Med. & INT & 4 \& 8 & High & Decoder & High & Med. \\ 
        Mokey~\cite{zadeh2022mokey} & Cluster & Med. & Float & 4 \& 8 & Med. & Calculation & Med. & Low \\ 
        GOBO~\cite{zadeh2020gobo} & Cluster & Low & Float & 16 & Low & LUT & Med. & High \\ 
        \hline
        \multirow{2}{*}{\proj}  & Search  & Med.  & \multirow{2}{*}{INT} & \multirow{2}{*}{4 \& 8} & \multirow{2}{*}{High} & \multirow{2}{*}{Calculation} & \multirow{2}{*}{High} & \multirow{2}{*}{High} \\ \cline{2-3}
        &  Map &  High &  &&&\\ 
        \Xhline{1.2pt}
    \end{tabular}
    }
    \vspace*{0.1cm}
    \label{intro:dtype}
    \vspace*{-0.2cm}
  \end{table}

\subsection{Support for Real-time Quantization}
\label{sec:moti_kvcache}

The above group-wise diversity presents a challenge for both weights and KV cache.
In addition, KV cache faces challenges in real-time group-wise quantization because the KV cache is generated dynamically during LLM inference.


To facilitate low-precision computation in group-wise quantization, it is necessary to quantize K and V along the inner dimension. 
This requirement stems from the support for matrix inner product operations in most GPUs and TPUs. 
During these operations, the group-wise scaling factor can be extracted from the multiply-accumulate process. 
\Fig{fig:kv_process} depicts the computation process of K and V during the decode stage. We define the dimension used for matrix inner product operations as the inner dimension. 
The inner dimensions of the K and V caches differ; the K cache requires a transpose operation, whereas the V cache does not, complicating the situation.


In the prefill stage, K and V can easily compute the scaling factor for each group. 
During the decode stage, the newly generated K vector is concatenated along the inner dimension of the K cache, enabling immediate quantization. 
However, the newly generated V vector is associated with different groups, with only one element per group produced per iteration. This process prevents the scaling factor for the entire group from being obtained in a single iteration, posing a significant challenge for the real-time quantization of the V cache.


\begin{figure}[t] 
  \centering 
  % \includegraphics[width=1.0\linewidth]{fig/dse_kv_process.pdf}  
  \includegraphics[width=0.9\linewidth]{fig/moti_kv_dimension.pdf}  
  \caption{\small Comparison of group-wise K and V cache quantization. They have different inner dimensions due to the transposition of K (key).}

  \label{fig:kv_process}
  % \vspace*{-0.4cm}
\end{figure}


Given those challenges, we propose \proj with a mathematical encoding format that can fuse with integer computation and enhance the decoding efficiency.
In addition, this encoding format provides sufficient adaptivity for group-wise quantization.
Regarding the challenge in KV cache, \proj employs a real-time quantization engine that ensures efficient encoding and decoding for KV cache.
By addressing these challenges, \proj enables efficient low-bit group-wise quantization.


\section{Experiments}
\label{sec:Experiments} 

We conduct several experiments across different problem settings to assess the efficiency of our proposed method. Detailed descriptions of the experimental settings are provided in \cref{sec:apendix_experiments}.
%We conduct experiments on optimizing PINNs for convection, wave PDEs, and a reaction ODE. 
%These equations have been studied in previous works investigating difficulties in training PINNs; we use the formulations in \citet{krishnapriyan2021characterizing, wang2022when} for our experiments. 
%The coefficient settings we use for these equations are considered challenging in the literature \cite{krishnapriyan2021characterizing, wang2022when}.
%\cref{sec:problem_setup_additional} contains additional details.

%We compare the performance of Adam, \lbfgs{}, and \al{} on training PINNs for all three classes of PDEs. 
%For Adam, we tune the learning rate by a grid search on $\{10^{-5}, 10^{-4}, 10^{-3}, 10^{-2}, 10^{-1}\}$.
%For \lbfgs, we use the default learning rate $1.0$, memory size $100$, and strong Wolfe line search.
%For \al, we tune the learning rate for Adam as before, and also vary the switch from Adam to \lbfgs{} (after 1000, 11000, 31000 iterations).
%These correspond to \al{} (1k), \al{} (11k), and \al{} (31k) in our figures.
%All three methods are run for a total of 41000 iterations.

%We use multilayer perceptrons (MLPs) with tanh activations and three hidden layers. These MLPs have widths 50, 100, 200, or 400.
%We initialize these networks with the Xavier normal initialization \cite{glorot2010understanding} and all biases equal to zero.
%Each combination of PDE, optimizer, and MLP architecture is run with 5 random seeds.

%We use 10000 residual points randomly sampled from a $255 \times 100$ grid on the interior of the problem domain. 
%We use 257 equally spaced points for the initial conditions and 101 equally spaced points for each boundary condition.

%We assess the discrepancy between the PINN solution and the ground truth using $\ell_2$ relative error (L2RE), a standard metric in the PINN literature. Let $y = (y_i)_{i = 1}^n$ be the PINN prediction and $y' = (y'_i)_{i = 1}^n$ the ground truth. Define
%\begin{align*}
%    \mathrm{L2RE} = \sqrt{\frac{\sum_{i = 1}^n (y_i - y'_i)^2}{\sum_{i = 1}^n y'^2_i}} = \sqrt{\frac{\|y - y'\|_2^2}{\|y'\|_2^2}}.
%\end{align*}
%We compute the L2RE using all points in the $255 \times 100$ grid on the interior of the problem domain, along with the 257 and 101 points used for the initial and boundary conditions.

%We develop our experiments in PyTorch 2.0.0 \cite{paszke2019pytorch} with Python 3.10.12.
%Each experiment is run on a single NVIDIA Titan V GPU using CUDA 11.8.
%The code for our experiments is available at \href{https://github.com/pratikrathore8/opt_for_pinns}{https://github.com/pratikrathore8/opt\_for\_pinns}.


\subsection{2D Allen Cahn Equation}
\begin{figure*}[t]
    \centering
    \includegraphics[scale=0.38]{figs/Burgers_operator.pdf}
    \caption{1D Burgers' Equation (Operator Learning): Steady-state solutions for different initializations $u_0$ under varying viscosity $\varepsilon$: (a) $\varepsilon = 0.5$, (b) $\varepsilon = 0.1$, (c) $\varepsilon = 0.05$. The results demonstrate that all final test solutions converge to the correct steady-state solution. (d) Illustration of the evolution of a test initialization $u_0$ following homotopy dynamics. The number of residual points is $\nres = 128$.}
    \label{fig:Burgers_result}
\end{figure*}
First, we consider the following time-dependent problem:
\begin{align}
& u_t = \varepsilon^2 \Delta u - u(u^2 - 1), \quad (x, y) \in [-1, 1] \times [-1, 1] \nonumber \\
& u(x, y, 0) = - \sin(\pi x) \sin(\pi y) \label{eq.hom_2D_AC}\\
& u(-1, y, t) = u(1, y, t) = u(x, -1, t) = u(x, 1, t) = 0. \nonumber
\end{align}
We aim to find the steady-state solution for this equation with $\varepsilon = 0.05$ and define the homotopy as:
\begin{equation}
    H(u, s, \varepsilon) = (1 - s)\left(\varepsilon(s)^2 \Delta u - u(u^2 - 1)\right) + s(u - u_0),\nonumber
\end{equation}
where $s \in [0, 1]$. Specifically, when $s = 1$, the initial condition $u_0$ is automatically satisfied, and when $s = 0$, it recovers the steady-state problem. The function $\varepsilon(s)$ is given by
\begin{equation}
\varepsilon(s) = 
\left\{\begin{array}{l}
s, \quad s \in [0.05, 1], \\
0.05, \quad s \in [0, 0.05].
\end{array}\right.\label{eq:epsilon_t}
\end{equation}

Here, $\varepsilon(s)$ varies with $s$ during the first half of the evolution. Once $\varepsilon(s)$ reaches $0.05$, it remains fixed, and only $s$ continues to evolve toward $0$. As shown in \cref{fig:2D_Allen_Cahn_Equation}, the relative $L_2$ error by homotopy dynamics is $8.78 \times 10^{-3}$, compared with the result obtained by PINN, which has a $L_2$ error of $9.56 \times 10^{-1}$. This clearly demonstrates that the homotopy dynamics-based approach significantly improves accuracy.

\subsection{High Frequency Function Approximation }
We aim to approximate the following function:
$u=    \sin(50\pi x), \quad x \in [0,1].$
The homotopy is defined as $H(u,\varepsilon) = u - \sin(\frac{1}{\varepsilon}\pi x), $
where $\varepsilon \in [\frac{1}{50},\frac{1}{15}]$.

\begin{table}[htbp!]
    \caption{Comparison of the lowest loss achieved by the classical training and homotopy dynamics for different values of $\varepsilon$ in approximating $\sin\left(\frac{1}{\varepsilon} \pi x\right)$
    }
    \vskip 0.15in
    \centering
    \tiny
    \begin{tabular}{|c|c|c|c|c|} 
    \hline 
    $ $ & $\varepsilon = 1/15$ & $\varepsilon = 1/35$ & $\varepsilon = 1/50$ \\ \hline 
    Classical Loss                & 4.91e-6     & 7.21e-2     & 3.29e-1       \\ \hline 
    Homotopy Loss $L_H$                      & 1.73e-6     & 1.91e-6     & \textbf{2.82e-5}       \\ \hline
    \end{tabular}
    % On convection, \al{} provides 14.2$\times$ and 1.97$\times$ improvement over Adam or \lbfgs{} on L2RE. 
    % On reaction, \al{} provides 1.10$\times$ and 1.99$\times$ improvement over Adam or \lbfgs{} on L2RE.
    % On wave, \al{} provides 6.32$\times$ and 6.07$\times$ improvement over Adam or \lbfgs{} on L2RE.}
    \label{tab:loss_approximate}
\end{table}

As shown in \cref{fig:high_frequency_result}, due to the F-principle \cite{xu2024overview}, training is particularly challenging when approximating high-frequency functions like $\sin(50\pi x)$. The loss decreases slowly, resulting in poor approximation performance. However, training based on homotopy dynamics significantly reduces the loss, leading to a better approximation of high-frequency functions. This demonstrates that homotopy dynamics-based training can effectively facilitate convergence when approximating high-frequency data. Additionally, we compare the loss for approximating functions with different frequencies $1/\varepsilon$ using both methods. The results, presented in \cref{tab:loss_approximate}, show that the homotopy dynamics training method consistently performs well for high-frequency functions.





\subsection{Burgers Equation}
In this example, we adopt the operator learning framework to solve for the steady-state solution of the Burgers equation, given by:
\begin{align}
& u_t+\left(\frac{u^2}{2}\right)_x - \varepsilon u_{xx}=\pi \sin (\pi x) \cos (\pi x), \quad x \in[0, 1]\nonumber\\
& u(x, 0)=u_0(x),\label{eq:1D_Burgers} \\
& u(0, t)=u(1, t)=0, \nonumber 
\end{align}
with Dirichlet boundary conditions, where $u_0 \in L_{0}^2((0, 1); \mathbb{R})$ is the initial condition and $\varepsilon \in \mathbb{R}$ is the viscosity coefficient. We aim to learn the operator mapping the initial condition to the steady-state solution, $G^{\dagger}: L_{0}^2((0, 1); \mathbb{R}) \rightarrow H_{0}^r((0, 1); \mathbb{R})$, defined by $u_0 \mapsto u_{\infty}$ for any $r > 0$. As shown in Theorem 2.2 of \cite{KREISS1986161} and Theorems 2.5 and 2.7 of \cite{hao2019convergence}, for any $\varepsilon > 0$, the steady-state solution is independent of the initial condition, with a single shock occurring at $x_s = 0.5$. Here, we use DeepONet~\cite{lu2021deeponet} as the network architecture. 
The homotopy definition, similar to ~\cref{eq.hom_2D_AC}, can be found in \cref{Ap:operator}. The results can be found in \cref{fig:Burgers_result} and \cref{tab:loss_burgers}. Experimental results show that the homotopy dynamics strategy performs well in the operator learning setting as well.


\begin{table}[htbp!]
    \caption{Comparison of loss between classical training and homotopy dynamics for different values of $\varepsilon$ in the Burgers equation, along with the MSE distance to the ground truth shock location, $x_s$.}
    \vskip 0.15in
    \centering
    \tiny
    \begin{tabular}{|c|c|c|c|c|} 
    \hline  
    $ $ & $\varepsilon = 0.5$ & $\varepsilon = 0.1$ & $\varepsilon = 0.05$ \\ \hline 
    Homotopy Loss $L_H$                &  7.55e-7     & 3.40e-7     & 7.77e-7       \\ \hline 
    L2RE                      & 1.50e-3     & 7.00e-4     & 2.52e-2       \\ \hline
        MSE Distance $x_s$                      & 1.75e-8     & 9.14e-8      & 1.2e-3      \\ \hline
    \end{tabular}
    % On convection, \al{} provides 14.2$\times$ and 1.97$\times$ improvement over Adam or \lbfgs{} on L2RE. 
    % On reaction, \al{} provides 1.10$\times$ and 1.99$\times$ improvement over Adam or \lbfgs{} on L2RE.
    % On wave, \al{} provides 6.32$\times$ and 6.07$\times$ improvement over Adam or \lbfgs{} on L2RE.}
    \label{tab:loss_burgers}
\end{table}



% \begin{itemize}
%     \item Relate the curvature in the problem to the differential operator. Use this to demonstrate why the problem is ill-conditioned
%     \item Give an argument for why using Adam + L-BFGS is better than just using L-BFGS outright. The idea is that Adam lowers the errors to the point where the rest of the optimization becomes convex \ldots
%     \item Show why we need second-order methods. We would like to prove that once we are close to the optimum, second-order methods will give condition-number free linear convergence. Specialize this to the Gauss-Newton setting, with the randomized low-rank approximation.
%     % \item Show that it is not possible to get superlinear convergence under the interpolation assumption with an overparameterized neural network. This should be true b/c the Hessian at the optimum will have rank $\min(n, d)$, and when $d > n$, this guarantees that we cannot have strong convexity.
% \end{itemize}
\section{Experiments}\label{sec_exp}
%\hp{Accelerating IM simulation~\cite{tang2015influence}}

% \begin{itemize}
%     \item 6.1. Problem setting of three COPs, including the general model and three specific CO problems 
%     \item 6.2. Experiment Setting (hyperparameters, details of training, evaluation, and test) 写在appendix里吧
%     \item 6.3. Performance analysis 这个要占半页
% \end{itemize}

%\hp{need to think of a way to compress these tables / visuals.} 

%\hp{\cancel{Baselines}; hyperparamters; \cancel{metrics}; etc.}

With theoretical guarantees on the existence and convergence of NE for ACCES games, we are also interested in how our proposed algorithm CCDO-RL works empirically. To evaluate this, we conduct experiments of CCDO-RL on three distinct ACCES game instances introduced in Section \ref{sub_exp_ins} and analyze the performance of CCDO-RL in Section \ref{sub_train_eval}. Section 6.2.1 aims to empirically demonstrate the convergence (Figures \ref{fig_exploit_20} and \ref{fig_exploit_50}) of the algorithm CCDO-RL over realistic CO problems, and show its consistency with Theorem \ref{CCDOA}. Section 6.2.2 intends to show the average reward (to seen training graphs) as well as the generalizability (to unseen test graphs) of the combinatorial player in real-world ACCES games (shown in Tables \ref{tab_aver}, and \ref{tab_gene}).

\subsection{Three Instances of ACCES Games} \label{sub_exp_ins}
% \hp{This para does not make much sense. Need to follow the framework in the Preliminaries section.}
% For combinatorial optimization problems in real-world applications, situations are more complicated and intractable due to changeable environmental or physical parameters. The form of parameter sets is very crucial because different types have different solvability and computation complexity. Forms of parameter sets mainly contain discrete sets, interval sets \cite{buchheim2018robust} like polyhedral and ellipsoid, probability distributions \cite{carlsson2018wasserstein}, and variable functions \cite{krause2008robust}.

% In reality, these parameters are often impacted by some common factors, such as conditions of weather, transportation, and individual personalities. \cite{kalimeris2019robust} proposed an assumption that real instances (e.g. demands in CVRP, coverages in CSP) 
%Considering affected or attacked COPs, the real instance $\{\theta_{i}\}$ always relied on the estimated value $\{\hat{\theta}_{i}$\} and the variation determined by independent factors $\{g_{i}\}$ and environment/physical parameters/attacker actions $\{\eta\}$. The concrete parameter influence model is stated as follows:

We consider a certain COP which is parameterized with $\{\theta_{i}\}$, where $i$ is the index of nodes (such as a target in security games) -- e.g., such parameters can be interpreted as attack probability of targets.
%coverage radius, customer's demands, or attack probability of targets. 
In real-world applications, we often need to estimate such parameters before solving the COPs. Unfortunately, the estimation $\{\hat{\theta}_{i}\}$ often bears a gap to the true value $\{\theta_{i}\}$, which derives from e.g. environment (aleatoric) uncertainty, model (epistemic) uncertainty, or an attacker trying to manipulate the defender's utility. We use a generic model to formulate this gap:
\begin{equation}\label{linrob}
    \theta_{i} = \hat{\theta}_{i} + y \cdot \tau_{i},
\end{equation}
where $y$ represents the strategy of the nature/attacker, $\tau_{i}$ is the environment factors like weather and transportation conditions, or human subjective factors like the preference of the attacker. 
Such abstraction can represent a wide range of ACCES games, such as facility location covering problems \cite{an2020battery, TIRKOLAEE2020340}, CVRP \cite{vehiclerouting.ch8,dinh2018exact, FLORIO20231081}, security patrolling (OP) \citep{xu2021robust}, and influence maximization problem \cite{kalimeris2019robust}. We describe three instances of ACCES games based on the model (\ref{linrob}).%Based on this model (\ref{linrob}), we focus on three combinatorial optimization problems with attacks or environmental/physical influence.

% \hp{Hard to follow. We should point out what are the two players, what are X, Y, u etc}

\textbf{Adversarial Covering Salesman Problem (ACSP):} In a map of cities, every city $i$ has a coverage $\theta_{i}$. A salesman finds the shortest path such that all cities are visited or covered, with $\theta_{i}$ influenced by physical factors $\tau_i$ and transportation parameters $y$ based on Eq.(\ref{linrob}). The salesman is Player 1 where $X$ consists of the feasible paths of the salesman. Nature is Player 2 with $Y$ = $[0, 1]^K \ni y, K \in \mathbb{N}$. The utility function of Player 1 $u$ is the opposite of the total traveling distance.

\textbf{Adversarial Capacitated Vehicle Routing Problem (ACVRP):} A vehicle with a constrained capacity of goods finds the shortest path under the worst case with the $i_{th}$ customer's demand $\theta_i$ changed by environmental factors $\tau_i$ and weather parameter $y$ on Eq.(\ref{linrob}). The vehicle is Player 1 where $X$ is the set of the feasible path $x$. Nature is Player 2 where $Y$ is $[0, 1]^K \ni y, K \in \mathbb{N}$. The utility function of Player 1  $u$ is the opposite of total delivery distance satisfying all the demands of customers.


\textbf{Patrolling Game (PG):} The patrolling game is described in the introduction.

For all the problem instances, we run our algorithm on two problem sizes: 20 nodes and 50 nodes. The detailed description and problem parameters of the three game instances are in Appendix \ref{app_ex_para_set}.

% Similarly, in the vehicle route problem (VRP), conditions with correlated parameters arouse broad attention from scholars \cite{vehiclerouting.ch8,dinh2018exact,FLORIO20231081}. \cite{dinh2018exact} considered the demand correlation by geographical proximity of nodes, described by some independent random variables in the fractional form. \cite{FLORIO20231081} utilized 'external factors' to stand for unknown covariates affecting all demands and presented a Bayesian model to learn correlations. Further more, about IM problems, \cite{kalimeris2019robust} combined node features and uncertain hyperparameters to fit the influence probability on each edge.

% \subsection{Training CCDO-RL}

% For all the problems, CCDO-RL adopts the REINFORCE algorithm with an attention-based encoder-decoder framework \cite{kool2018attention} (used as an inductive graph representation component) to learn a (generalizable) COP solver for one player (protagonist), and PPO \cite{schulman2017proximal} to train a policy for the other player (adversary) whose strategy space is continuous. CCDO-RL is trained with 50 epochs on a set of 10,000 graphs (with 20 or 50 nodes). The hyperparameters of CCDO-RL are specified in Appendix \ref{app_ex_para_set} (Table \ref{tab_hyper_ccdorl}). Our code is included as supplementary material for ease of reproduction. 
% % \hp{need to specify hyperparas}

\subsection{Performance of CCDO-RL}\label{sub_train_eval}

Two aspects are evaluated for the performance of CCDO-RL, i.e., i) Convergence to NE (Section \ref{sub_per_conver}) exploring whether CCDO-RL can compute the NE, and ii) Protagonist policy's average reward and generalizability (Section \ref{sub_per_rob}). Generalizability refers to the ability of RL models trained on previously seen graphs (problem instances), to perform well on a new set of unseen test graphs. The model’s usability is enhanced by generalizability, rather than focusing solely on the average reward, which is a critical motivation in the literature on RL for COPs \citep{khalil2017learning, kool2018attention}.

For all the problems, CCDO-RL adopts the REINFORCE algorithm with an attention-based encoder-decoder framework \citep{kool2018attention} (used as an inductive graph representation component) to learn a generalizable COP solver for Player 1 (protagonist), and PPO to train a policy for Player 2 (adversary) whose strategy space is continuous. CCDO-RL is trained on a set of 10,000 graphs (with 20 or 50 nodes). The hyperparameters of CCDO-RL are specified in Appendix \ref{app_ex_para_set} (Table \ref{tab_hyper_ccdorl}). Our code is included as supplementary material and will be open-sourced for ease of reproduction. 

% \textbf{Training.} For all the problems, CCDO-RL adopts the REINFORCE algorithm with attention-based encoder-decoder framework \cite{kool2018attention} (used as an inductive graph representation component) to learn a (generalizable) COP solver for one player (protagonist), and PPO \cite{schulman2017proximal} to train a policy for the other player (adversary) whose strategy space is continuous. CCDO-RL is trained with 50 epochs on a set of 10,000 graphs (with 20 or 50 nodes). 

% \hp{We should first present results about convergence as it is mostly aligned with the theory.}

\subsubsection{Convergence to NE} \label{sub_per_conver}

Exploitability is a common metric to describe the closeness to true NE by calculating the sum of performance distances between each new best response and subgame NE, i.e. $\sum_{i=1,2} U(\pi_{i,k}^{br}, \sigma_{-i,k}) - U(\sigma)$ in the general two-player game. Since our game is zero-sum, the calculation is as follows:
\begin{equation*}
   \text{Exploitability}(\sigma) = \max_{\pi_1 \in \Sigma_1} U(\pi_1, \sigma_{2}) - \min_{\pi_2 \in \Sigma_2} U(\sigma_1, \pi_2).
\end{equation*}
From Figure \ref{fig_exploit_20}, we can see that CCDO-RL can converge to approximate NE in 25 iterations or less (in the PG setting), reaching 0.05 in ACSP, 0.10 in ACVRP, and 0.03 in PG with 20 nodes. Similar results are observed in problems with 50 nodes (see Figure \ref{fig_exploit_50} in Appendix \ref{app_exp}). These results validate the effectiveness of CCDO-RL in finding the NE for various types of games.

%Similarly, the exploitability of three COPs in 50 nodes is provided in the appendix \ref{app_exp}.
\vspace{-\baselineskip}
\begin{figure}[htbp]
	\centering
    \subfigure[ACSP20]{
    \label{csp20_nashconv}
    \includegraphics[scale=0.20]{Figures/nashconv_log_csp20_sm_7.eps}
    }
    \subfigure[ACVRP20]{
    \label{cvrp20_nashconv}%文中引用该图片代号
    \includegraphics[scale=0.20]{Figures/nashconv_log_svrp20_sm_7.eps}
    }
    \subfigure[PG20]{
    \label{opsa20_nashconv}
    \includegraphics[scale=0.20]{Figures/nashconv_log_pg20_sm_7.eps}
    }
    \caption{Exploitability curve of CCDO-RL on three games of 20 nodes}
    \label{fig_exploit_20}
\end{figure}
\vspace{-\baselineskip}
\subsubsection{Average reward and Generalizability of Combinatorial player} \label{sub_per_rob}
% \subsubsection{Robustness and Generalizability of Protagonist Policy} \label{sub_per_rob}
%\hp{CCDO-RL being better in these following metrics is only kind of a by-product.}

% \textbf{Evaluation.} The learned policies are then tested on 200 graphs, where 100 of them are randomly selected from the 10,000 training graphs, and the other 100 are unseen graphs. 
% We use two metrics to evaluate the performance of different policies for the protagonist player: \textbf{Average proportional loss} $R-$ describes the policy overfitting degree \citep{lanctot2017unified}; \textbf{Reward} evaluates the performance of the protagonist with the adversary under three COPs.  
% \begin{eqnarray}
%         &R- = (\hat{D} - \hat{O}) / \hat{D}.
% \end{eqnarray}
% in which $\hat{D}$ is the mean value of the diagonals and $\hat{O}$ is the mean value of the off-diagonals in the payoff matrix provided in the Appendix \ref{app_exp}.

% Because the protagonist policy is trained against a powerful adversary under our ACCES game setting, the obtained policy is naturally robust against adversarial perturbations. This subsection sheds a bit of light on this perspective and quantifies the extent of robustness of CCDO-RL as well as the ability of RL to generalize to unseen test graphs.

\textbf{Evaluation.} The learned policies are tested on 200 graphs, with 100 being randomly selected from the 10,000 training graphs (to show the average reward), and the other 100 being unseen graphs (to test policy generalization). We evaluate the performance of the protagonist with the adversary under three COPs. For each COP, the performance is considered both on the 20-node and 50-node map.
% We use two metrics to evaluate the performance of different policies for the protagonist player: \textbf{Average proportional loss} $R-$ describes the policy overfitting degree \citep{lanctot2017unified}; \textbf{Reward} evaluates the performance of the protagonist with the adversary under three COPs.

\textbf{Baselines.} There are heuristic algorithms for each game instance (Heuristic in Table \ref{tab_aver} and \ref{tab_gene}) and a single-player RL algorithm. For ACVRP, we adopt the Tabu Search algorithm (Tabu) \citep{li2020improved} as the heuristic algorithm, which is widely applied in the routing problem. For ACSP, the common benchmark local search algorithm, LS2 \citep{golden2012generalized}, is used. For PG, we choose the greedy algorithm as the baseline. The "RL against Stoc" algorithm in Tables \ref{tab_aver} and \ref{tab_gene} is identical to the protagonist model in CCDO-RL but trained in environments with stochastic adversarial perturbations.

% \textbf{Baselines.} There are a heuristic algorithms for each game instance {\color{red} (Heuristic mentioned in the Table \ref{tab_aver} and \ref{tab_gene})} and a single-player RL algorithm. For ACVRP, we adopt the Clarke-Wright (CW) algorithm \citep{pichpibul2013heuristic} and the Tabu Search algorithm (Tabu) \citep{li2020improved} as heuristics, which are applied widely in the routing problem. For ACSP, two common benchmark local search algorithms, LS1 and LS2 \citep{golden2012generalized}, are used. For PG, we choose a local search algorithm \citep{vansteenwegen2009iterated} and the greedy algorithm as the heuristic baselines. {\color{red} The "RL  against Stoc" algorithm referred to Tables \ref{tab_aver} and \ref{tab_gene}} is identical to the protagonist model in CCDO-RL {\color{red} but trained on environments with stochastic adversarial perturbations.} 

\textbf{Average Reward.}  As illustrated in Table \ref{tab_aver}, our algorithm achieves a better average reward than baselines (10.08\% improvement on average of all settings against two baselines), regardless of CO instance or problem size, when confronting the adversary trained by CCDO-RL. In the setting of CSP-20 nodes, the average reward is improved by 46.98\% compared to the heuristic and by 7.14\% compared with the RL against Stoc. For the 50-node setting, the improvements are 45.91\% and 5.28\% respectively. Similarly, the improvements in contrast to Heuristic and RL against Stoc are as follows: 1.72\% and 3.01\%  for CVRP-20 nodes, 0.75\% and 4.46\% for CVRP-50 nodes, 4.17\% and 1.48\% for PG-20 nodes, and 10.60\% and 4.38\% for PG-50 nodes.

\textbf{Generalizability.} From Table \ref{tab_gene}, CCDO-RL continues to achieve a better average reward when facing the adversary, demonstrating that the learned RL policies generalize well to unseen graphs. Even though the non-RL baselines do have access to the graph structures and other problem information of the unseen problem instances, CCDO-RL can obtain comparable performances without re-training on the new problem instances. The improvements versus Heuristic and RL against Stoc are 46.61\% and 7.02\% for CSP-20 nodes, 42.24\% and 3.94\% for CSP-50 nodes, 1.12\% and 1.56\% for CVRP-20 nodes, 0.90\% and 5.05\% for CVRP-50 nodes, 5.35\% and 2.40\% for PG-20 nodes, and 12.17\% and 10.33\% for PG-50 nodes. Even when confronting the stochastic adversary, CCDO shows superior generalizability compared to two baselines across three COPs, with average improvements of 6.31\%, 3.42\%, and 3.95\% respectively. Detailed results are provided in Appendix \ref{app_exp} (Tables \ref{tab_csp_full_20} - \ref{tab_op_full_50}). 
% The model’s usability is enhanced by the ability to generalize rather than focusing solely on the average reward, which is a critical motivation of the RL for combinatorial optimization literature \citep{khalil2017learning, kool2018attention}.  

\begin{remark}
    In CO problems (or more broadly, operations research and economics), it is known that achieving solution quality improvements against strong baselines (e.g., the RL methods trained with a stochastic adversary) is very challenging, and the margins are usually small \citep{kool2018attention}, sometimes even less than 1\%. However, these “tiny” marginal improvements in profits keep small business owners in the real world alive. Last, the improvement depends a lot on the problem settings, and we show that sometimes the improvement can be much more significant.
\end{remark}
\vspace{-\baselineskip}
% \textbf{Performance analysis.} The robustness results of CCDO-RL for ACSP are shown in Table \ref{tab_csp}. We have the following observations: 1) On both of the 100 seen/unseen graphs, single-player RL performs better than heuristic algorithms no matter whether attacked or not. (2) When confronting the adversary trained by CCDO-RL, CCDO-RL exceeds RL by 0.25 and 0.24 on the training set, and by 0.25 and 0.18 on the test set, respectively under the 20-node and 50-node graphs. This demonstrates the robustness of CCDO-RL. 3) Compared to the performance of the training set with that of the test set, we can see that RL and CCDO-RL both maintain a certain degree of generalization. Similar results for ACVRP (Table \ref{tab_cvrp}) and SPG (Table \ref{tab_op}) are provided in Appendix \ref{app_exp}. 

\begin{table}[ht]
  \caption{Average reward against CCDO-RL's adversary (on seen graphs)}
  \vspace{\baselineskip}
  \label{tab_aver}
  \centering
  \small
  \begin{tabular}{lllllll}
    \toprule
    \multirow{2}{*}{method} & \multicolumn{2}{c}{ACSP (Mean$\pm$Std)} & \multicolumn{2}{c}{ACVRP (Mean$\pm$Std)} & \multicolumn{2}{c}{PG (Mean$\pm$Std)} \\
    \cmidrule(r){2-3} \cmidrule{4-5} \cmidrule(r){6-7}
                            & 20 nodes & 50 nodes & 20 nodes & 50 nodes & 20 nodes & 50 nodes\\
    \midrule
    Heuristic & 6.13$\pm$1.20 & 7.55$\pm$1.42 & 7.65$\pm$1.23  & 13.38$\pm$1.70 & 2.64$\pm$1.03 & 4.53$\pm$1.84   \\
    RL against Stoc    & 3.50$\pm$0.47  & 4.55$\pm$0.62  & 7.55$\pm$1.16  & 13.90$\pm$1.63 & 2.71$\pm$0.90 & 4.80$\pm$2.18   \\
    CCDO-RL   & $\pmb{3.25}$$\pm$0.42 & $\pmb{4.31}$$\pm$0.51  & $\pmb{7.42}$$\pm$1.21  & $\pmb{13.28}$$\pm$1.52 &  $\pmb{2.75}$$\pm$0.87 & $\pmb{5.01}$$\pm$1.91  \\
    \bottomrule
  \end{tabular}
\end{table}
\vspace{-\baselineskip}

\begin{table}[htp]
  \caption{Generalizability against CCDO-RL's adversary (on unseen graphs)}
  \vspace{\baselineskip}
  \label{tab_gene}
  \centering
  \small
  \begin{threeparttable}
  \begin{tabular}{lllllll}
    \toprule
    \multirow{2}{*}{method} & \multicolumn{2}{c}{ACSP (Mean$\pm$Std)} & \multicolumn{2}{c}{ACVRP (Mean$\pm$Std)} & \multicolumn{2}{c}{PG (Mean$\pm$Std)} \\
    \cmidrule(r){2-3} \cmidrule{4-5} \cmidrule(r){6-7}
                            & 20 nodes & 50 nodes & 20 nodes & 50 nodes & 20 nodes & 50 nodes\\
    \midrule
    Heuristic & 6.20$\pm$1.33 & 7.60$\pm$1.37   & 7.64$\pm$1.30  & 13.27$\pm$1.87 & 2.43$\pm$0.98 & 4.19$\pm$1.69    \\
    RL against Stoc  & 3.56$\pm$0.37  & 4.57$\pm$0.58  & 7.67$\pm$1.30  & 13.85$\pm$1.53 &  2.50$\pm$0.95 & 4.26$\pm$2.17 \\
    CCDO-RL   & $\pmb{3.31}$$\pm$0.35 & $\pmb{4.39}$$\pm$0.52  & $\pmb{7.55}$$\pm$1.28  & $\pmb{13.15}$$\pm$1.59 & $\pmb{2.56}$$\pm$0.92 & $\pmb{4.70}$$\pm$1.94\\

    \bottomrule
  \end{tabular}
  \begin{tablenotes}
      \footnotesize
      \item[1] For the average reward of ACSP and ACVRP, smaller is better while for that of PG larger is better.
  \end{tablenotes}
  \end{threeparttable}
\end{table}
\vspace{-\baselineskip}
% two heuristics and one RL
% \begin{table}[ht]
%   \caption{{\color{red} Average reward of CCDO-RL (on seen graphs). For the value of CSP and CVRP, larger is better while for that of PG smaller is better.}}
%   \label{tab_aver}
%   \centering
%   \small
%   \begin{tabular}{lllllll}
%     \toprule
%     \multirow{2}{*}{method} & \multicolumn{2}{c}{CSP (Mean$\pm$Std)} & \multicolumn{2}{c}{CVRP (Mean$\pm$Std)} & \multicolumn{2}{c}{PG (Mean$\pm$Std)} \\
%     \cmidrule(r){2-3} \cmidrule{4-5} \cmidrule(r){6-7}
%                             & 20 nodes & 50 nodes & 20 nodes & 50 nodes & 20 nodes & 50 nodes\\
%     \midrule
%     Baseline 1 & 4.52$\pm$0.71  & 5.98$\pm$0.94 & 7.64$\pm$1.56  & 13.49$\pm$2.10 & 2.71$\pm$1.10 & 1.82$\pm$1.40   \\
%     Baseline 2 & 6.13$\pm$1.20 & 7.55$\pm$1.42   & 7.65$\pm$1.23  & 13.38$\pm$1.70 & 2.64$\pm$1.03 & 1.47$\pm$0.99  \\
%     RL {\color{red}against Stoc}    & 3.50$\pm$0.47  & 4.55$\pm$0.62  & 7.55$\pm$1.16  & 13.90$\pm$1.63 & 2.71$\pm$0.90 & 1.54$\pm$1.03   \\
%     CCDO-RL   & $\pmb{3.25}$$\pm$0.42 & $\pmb{4.31}$$\pm$0.51  & $\pmb{7.42}$$\pm$1.21  & $\pmb{13.28}$$\pm$1.52 &  $\pmb{2.75}$$\pm$0.87 & $\pmb{1.87}$$\pm$1.22  \\
%     \bottomrule
%   \end{tabular}
% \end{table}


% \begin{table}[htp]
%   \caption{{\color{red}Generalizability of CCDO-RL (on unseen graphs)}}
%   \label{tab_gene}
%   \centering
%   \small
%   \begin{threeparttable}
%   \begin{tabular}{lllllll}
%     \toprule
%     \multirow{2}{*}{method} & \multicolumn{2}{c}{CSP (Mean$\pm$Std)} & \multicolumn{2}{c}{CVRP (Mean$\pm$Std)} & \multicolumn{2}{c}{PG (Mean$\pm$Std)} \\
%     \cmidrule(r){2-3} \cmidrule{4-5} \cmidrule(r){6-7}
%                             & 20 nodes & 50 nodes & 20 nodes & 50 nodes & 20 nodes & 50 nodes\\
%     \midrule
%     Baseline 1 & 4.53$\pm$0.79  & 5.95$\pm$0.96 & 7.55$\pm$1.39  & 13.35$\pm$2.04 & 2.52$\pm$1.08 & $\pmb{1.86}$$\pm$1.44  \\
%     Baseline 2 & 6.20$\pm$1.33 & 7.60$\pm$1.37   & 7.64$\pm$1.3  & 13.27$\pm$1.87 & 2.43$\pm$0.98 & 1.52$\pm$1.20    \\
%     RL {\color{red}against Stoc}  & 3.56$\pm$0.37  & 4.57$\pm$0.58  & 7.67$\pm$1.30  & 13.85$\pm$1.53 &  2.50$\pm$0.95 & 1.03$\pm$5.05 \\
%     CCDO-RL   & $\pmb{3.31}$$\pm$0.35 & $\pmb{4.39}$$\pm$0.52  & $\pmb{7.55}$$\pm$1.28  & $\pmb{13.15}$$\pm$1.59 & $\pmb{2.56}$$\pm$0.92 & 1.35$\pm$5.09\\

%     \bottomrule
%   \end{tabular}
%   \begin{tablenotes}
%       \footnotesize
%       \item[1] For the value of CSP and CVRP, larger is better while for that of PG smaller is better.
%   \end{tablenotes}
%   \end{threeparttable}
% \end{table}

\section*{Conclusion}
This paper aims to enhance our understanding of the computational complexity of computing various Shapley value variants. We found that for various ML models --- including decision trees, regression tree ensembles, weighted automata, and linear regression --- both local and global interventional and baseline SHAP can be computed in polynomial time under HMM modeled distributions. This extends popular algorithms, such as TreeSHAP, beyond their empirical distributional scope. We also establish strict complexity gaps between the various SHAP variants (baseline, interventional, and conditional) and prove the intractability of computing SHAP for tree ensembles and neural networks in simplified scenarios. Overall, we present SHAP as a versatile framework whose complexity depends on four key factors: \begin{inparaenum}[(i)] \item model type, \item SHAP variant, \item distribution modeling approach, \item and local vs. global explanations\end{inparaenum}. We believe this perspective provides deeper insight into the computational complexity of SHAP, paving the way for future work.




%We believe that our framework provides a more intricate understanding of SHAP computation complexity across different models, distributions, and variants, paving the way for further research.

Our work opens promising directions for future research. First, expanding our computational analysis to other SHAP-related metrics, such as asymmetric SHAP~\citep{frye20} and SAGE~\citep{covert2020understanding}, would be valuable. Additionally, we aim to explore more expressive distribution classes and relaxed assumptions beyond those in Section \ref{sec:tractable} while maintaining tractable SHAP computation. Finally, when exact computation is intractable (Section \ref{sec:intractable}), investigating the approximability of SHAP metrics through approximation and parameterized complexity theory~\citep{downey2012parameterized} is an important direction.

%Our work opens several promising avenues for future research on the computational properties of explainable AI methods, with a particular focus on SHAP. First, it would be interesting to broaden the computational analysis conducted in this work to include other popular SHAP-related metrics in the literature, such as asymmetric SHAP \cite{frye20} and SAGE \cite{covert2020understanding}. Also, in the future, we aim to explore more expressive distribution classes and relaxed distributional assumptions—extending beyond those examined in Section \ref{sec:tractable} —that still yield tractable SHAP computation. Finally, when exact computation proves intractable (Section \ref{sec:intractable}), it is worthwhile to theoretically investigate the question of the approximability of computing the SHAP metrics across various configurations, through the lens of approximation and parametrized complexity theory \cite{arora2009computational}.

%This paper aims to deepen our understanding of the computational complexity involved in obtaining different Shapley value variants. We found that for a variety of ML models, including decision trees, tree ensembles for regression, weighted automata, and linear regression models — computing both local and global interventional and baseline SHAP can be done in polynomial time when distributions are modeled by HMMs. This extends the distributional scope of popular algorithms like TreeSHAP, which is limited to empirical distributions. Additionally, we demonstrate a strict complexity gap between SHAP variants, showing that interventional and baseline SHAP can be strictly easier to compute than conditional SHAP. Despite these positive results, we uncovered intractability for various SHAP variants in neural networks and tree ensembles. Finally, we provided generalized complexity relations across SHAP variants. We believe that our framework offers a deeper understanding of the complexity involved in computing SHAP across various variants, models, distributions, as well as in both local and global computations, laying the groundwork for future research.

\nocite{langley00}

\section{Impact Statement}
This paper presents work whose goal is to advance the field 
of Reinforcement Learning. There are many potential societal
 consequences of our work, none which we feel must be
specifically highlighted here.
\bibliography{example_paper}
\bibliographystyle{icml2025}


%%%%%%%%%%%%%%%%%%%%%%%%%%%%%%%%%%%%%%%%%%%%%%%%%%%%%%%%%%%%%%%%%%%%%%%%%%%%%%%
%%%%%%%%%%%%%%%%%%%%%%%%%%%%%%%%%%%%%%%%%%%%%%%%%%%%%%%%%%%%%%%%%%%%%%%%%%%%%%%
% APPENDIX
%%%%%%%%%%%%%%%%%%%%%%%%%%%%%%%%%%%%%%%%%%%%%%%%%%%%%%%%%%%%%%%%%%%%%%%%%%%%%%%
%%%%%%%%%%%%%%%%%%%%%%%%%%%%%%%%%%%%%%%%%%%%%%%%%%%%%%%%%%%%%%%%%%%%%%%%%%%%%%%
\newpage
\appendix
\onecolumn
\section{Experimental Details}
% \begin{figure*}[htb]
%     \vskip 0.2in
%     \begin{center}
%     \subfigure[Boyanchian]{
%         \includegraphics[width=0.75\columnwidth, height=0.4\columnwidth]{main/pic/boyanchian.png}
%         \label{boyanchianonpolicy}
%     }
%     \subfigure[2-state version of Baird's off-policy counterexample.]{
%         \includegraphics[width=0.45\columnwidth, height=0.2\columnwidth]{main/pic/2StateExample.pdf}
%         \label{bairdexample}
%     }
%     \subfigure[7-state version of Baird's off-policy counterexample.]{
%         \input{main/pic/BairdExample.tex}
%         \label{7statebairdexample}
%     }
%         \caption{Three evaluation environments.}
%         \label{Threeevaluationenvironments}
%     \end{center}
%     \vskip -0.2in
% \end{figure*}
\subsection{Boychain}
\label{appendixboyanchian}
\textbf{Boyanchian:} BoyanChain is a classic on-policy environment 
with 13 states, each represented by a 4-dimensional feature vector. The agent 
can choose between two actions at each state: move to the next state (action 1) or skip to 
the one after (action 2). State 11 always transitions to state 12, while state 12 resets to 
state 0. Rewards are structured as follows: -2 for moving from state 11 to 12, 0 
for resetting from state 12 to 0, and -3 for all other actions. The policy is 
deterministic at state 11 (always choosing action 1) and stochastic elsewhere, 
randomly selecting between action 1 and 2. 
The discount factor $\gamma =0.9$. 
The feature matrix of Boyanchain is
defined as follows:
\begin{equation*}
    \bm{\Phi}_{Boyanchain}=\left[ 
    \begin{array}{cccc}
     1& 0& 0& 0\\
     0.75& 0.25& 0& 0\\
     0.5& 0.5& 0& 0\\
     0.25&0.75&0&0\\
     0& 1& 0& 0\\
     0& 0.75& 0.25& 0\\
    0 & 0.5& 0.5& 0\\
     0& 0.25& 0.75& 0\\
     0& 0& 1& 0\\
     0& 0& 0.75& 0.25\\
    0 & 0&0.5& 0.5\\
     0& 0& 0.25& 0.75\\
     0& 0& 0& 1
    \end{array}\right]
\end{equation*}
\begin{figure}[H]
    \begin{center}
    \label{boyanchianonpolicy}
    \includegraphics[width=0.5\columnwidth, height=0.2\columnwidth]{main/pic/boyanchian.png}
        \caption{Boychain.}
    \end{center}
\end{figure}
The $\alpha$ values for all algorithms are in the range of $\{0.0001, 0.0005, 0.001, 0.005, 0.01, 0.05, 0.1, 0.2, 0.3\}$. 
For the TDC and CTDC algorithm, the $\zeta$ values are in the range of $\{0.0005, 0.001, 0.005, 0.01, 0.05, 0.1, 0.2, 0.5\}$. 
For the CTD and CTDC algorithm, the $\beta$ values are in the range of $\{0.0005, 0.001, 0.005, 0.01, 0.05, 0.1, 0.2, 0.5\}$. 
\begin{figure}
    \vskip 0.2in
    \begin{center}
    \subfigure[TD]{
        \includegraphics[width=0.3\columnwidth, height=0.25\columnwidth]{main/pic/boyanchain/td/td.pdf}
        \label{td_boyanchain}
    }
    \subfigure[TDC]{
        \includegraphics[width=0.3\columnwidth, height=0.25\columnwidth]{main/pic/boyanchain/tdc/tdc.pdf}
        \label{tdc_boyanchain}
    }
    \subfigure[CTD]{
        \includegraphics[width=0.3\columnwidth, height=0.25\columnwidth]{main/pic/boyanchain/vmtd/vmtd.pdf}
        \label{ctd_boyanchain}
    }
    \\
    \subfigure[CTDC($\alpha=0.0001$)]{
        \includegraphics[width=0.3\columnwidth, height=0.25\columnwidth]{main/pic/boyanchain/vmtdc/alpha_0_0001.pdf}
        \label{ctdc_alpha_0001_boyanchain}
    }
    \subfigure[CTDC($\alpha=0.0005$)]{
        \includegraphics[width=0.3\columnwidth, height=0.25\columnwidth]{main/pic/boyanchain/vmtdc/alpha_0_0005.pdf}
        \label{ctdc_alpha_0005_boyanchain}
    }
    \subfigure[CTDC($\alpha=0.001$)]{
        \includegraphics[width=0.3\columnwidth, height=0.25\columnwidth]{main/pic/boyanchain/vmtdc/alpha_0_001.pdf}
        \label{ctdc_alpha_001_boyanchain}
    }
    \\
    \subfigure[CTDC($\alpha=0.005$)]{
        \includegraphics[width=0.3\columnwidth, height=0.25\columnwidth]{main/pic/boyanchain/vmtdc/alpha_0_005.pdf}
        \label{ctdc_alpha_005_boyanchain}
    }
    \subfigure[CTDC($\alpha=0.01$)]{
        \includegraphics[width=0.3\columnwidth, height=0.25\columnwidth]{main/pic/boyanchain/vmtdc/alpha_0_01.pdf}
        \label{ctdc_alpha_01_boyanchain}
    }
    \subfigure[CTDC($\alpha=0.05$)]{
        \includegraphics[width=0.3\columnwidth, height=0.25\columnwidth]{main/pic/boyanchain/vmtdc/alpha_0_05.pdf}
        \label{ctdc_alpha_05_boyanchain}
    }
    \\
    \subfigure[CTDC($\alpha=0.1$)]{
        \includegraphics[width=0.3\columnwidth, height=0.25\columnwidth]{main/pic/boyanchain/vmtdc/alpha_0_1.pdf}
        \label{ctdc_alpha_1_boyanchain}
    }
    \subfigure[CTDC($\alpha=0.2$)]{
        \includegraphics[width=0.3\columnwidth, height=0.25\columnwidth]{main/pic/boyanchain/vmtdc/alpha_0_2.pdf}
        \label{ctdc_alpha_2_boyanchain}
    }
    \subfigure[CTDC($\alpha=0.3$)]{
        \includegraphics[width=0.3\columnwidth, height=0.25\columnwidth]{main/pic/boyanchain/vmtdc/alpha_0_3.pdf}
        \label{ctdc_alpha_3_boyanchain}
    }
        \caption{Sensitivity of various algorithms to learning rates for Boyanchain.}
        \label{SensitivityBoyanchain}
    \end{center}
    \vskip -0.2in
\end{figure}
\subsection{2-state counterexample}
\label{appendix2state}
\textbf{2-state off-policy counterexample:} The ``1''$\rightarrow$``2'' problem has only two states \cite{sutton2016emphatic}. From each
state, there are two actions, left and right, which take 
the agent to the left or right state. All rewards are zero.
The feature $\bm{\Phi}=(1,2)^{\top}$ 
are assigned to the left and the right 
state. The behavior policy takes equal
probability to left or right
in both states, i.e., 
$
\textbf{P}_{1}=
\begin{bmatrix}
0.5 & 0.5 \\
0.5 & 0.5
\end{bmatrix}
$.
The target policy only selects action rights in both states, i.e., 
$
\textbf{P}_{2}=
\begin{bmatrix}
0 & 1 \\
0 & 1
\end{bmatrix}
$.
The state distribution of
the first policy is $\bm{d}_1 =(0.5,0.5)^{\top}$.
The state distribution of
the second policy is $\bm{d}_1 =(0,1)^{\top}$.
The discount factor is $\gamma=0.9$.
\begin{figure}[H]
    \begin{center}
    \label{bairdexampleoffpolicy}
    \includegraphics[width=0.3\columnwidth, height=0.1\columnwidth]{main/pic/2StateExample.pdf}
        \caption{2-state off-policy counterexample.}
    \end{center}
\end{figure}
The $\alpha$ values for all algorithms are in the range of $\{0.0001, 0.0005, 0.001, 0.005, 0.01\}$. 
For the TDC and CTDC algorithm, the $\zeta$ values are in the range of $\{0.0005, 0.001, 0.005, 0.01, 0.05\}$. 
For the CTD and CTDC algorithm, the $\beta$ values are in the range of $\{0.0005, 0.001, 0.005, 0.01, 0.05, 0.1, 0.2\}$. 
\begin{figure}
    \vskip 0.2in
    \begin{center}
    \subfigure[TD]{
        \includegraphics[width=0.3\columnwidth, height=0.25\columnwidth]{main/pic/2_state/td/td.pdf}
        \label{td_2state}
    }
    \subfigure[TDC]{
        \includegraphics[width=0.3\columnwidth, height=0.25\columnwidth]{main/pic/2_state/tdc/tdc.pdf}
        \label{tdc_2state}
    }
    \subfigure[CTD]{
        \includegraphics[width=0.3\columnwidth, height=0.25\columnwidth]{main/pic/2_state/ctd/ctd.pdf}
        \label{ctd_2state}
    }
    \\
    \subfigure[CTDC($\alpha=0.0001$)]{
        \includegraphics[width=0.3\columnwidth, height=0.25\columnwidth]{main/pic/2_state/ctdc/alpha_0_0001.pdf}
        \label{ctdc_alpha_0001_2state}
    }
    \subfigure[CTDC($\alpha=0.0005$)]{
        \includegraphics[width=0.3\columnwidth, height=0.25\columnwidth]{main/pic/2_state/ctdc/alpha_0_0005.pdf}
        \label{ctdc_alpha_0005_2state}
    }
    \subfigure[CTDC($\alpha=0.001$)]{
        \includegraphics[width=0.3\columnwidth, height=0.25\columnwidth]{main/pic/2_state/ctdc/alpha_0_001.pdf}
        \label{ctdc_alpha_001_2state}
    }
    \\
    \subfigure[CTDC($\alpha=0.005$)]{
        \includegraphics[width=0.3\columnwidth, height=0.25\columnwidth]{main/pic/2_state/ctdc/alpha_0_005.pdf}
        \label{ctdc_alpha_005_2state}
    }
    \subfigure[CTDC($\alpha=0.01$)]{
        \includegraphics[width=0.3\columnwidth, height=0.25\columnwidth]{main/pic/2_state/ctdc/alpha_0_01.pdf}
        \label{ctdc_alpha_01_2state}
    }
    \subfigure[CTDC($\alpha=0.05$)]{
        \includegraphics[width=0.3\columnwidth, height=0.25\columnwidth]{main/pic/2_state/ctdc/alpha_0_05.pdf}
        \label{ctdc_alpha_05_2staten}
    }
    \\
    \subfigure[CTDC($\alpha=0.1$)]{
        \includegraphics[width=0.3\columnwidth, height=0.25\columnwidth]{main/pic/2_state/ctdc/alpha_0_1.pdf}
        \label{ctdc_alpha_1_2state}
    }
    \subfigure[CTDC($\alpha=0.2$)]{
        \includegraphics[width=0.3\columnwidth, height=0.25\columnwidth]{main/pic/2_state/ctdc/alpha_0_2.pdf}
        \label{ctdc_alpha_2_2state}
    }
    \subfigure[CTDC($\alpha=0.3$)]{
        \includegraphics[width=0.3\columnwidth, height=0.25\columnwidth]{main/pic/2_state/ctdc/alpha_0_3.pdf}
        \label{ctdc_alpha_3_2staten}
    }
        \caption{Sensitivity of various algorithms to learning rates for 2-state counterexample.}
        \label{Sensitivity2state}
    \end{center}
    \vskip -0.2in
\end{figure}
\subsection{7-state counterexample}
\label{appendix7state}
\textbf{7-state off-policy counterexample:} This task is well known as a
counterexample, in which TD diverges \cite{baird1995residual,sutton2009fast}. As
shown in Figure \ref{7statebairdexample}, reward for each transition is zero. Thus the true values are zeros for all states and for any given policy. The behaviour policy
chooses actions represented by solid lines with a probability of $\frac{1}{7}$
and actions represented by dotted lines with a probability of $\frac{6}{7}$. The
target policy is expected to choose the solid line with more probability than $\frac{1}{7}$,
and it chooses the solid line with probability of $1$ in this paper.
 The discount factor $\gamma =0.99$. 
The feature matrix of 7-state version of Baird's off-policy counterexample is
defined as follows \ref{appendix7state} \cite{baird1995residual,sutton2009fast,maei2011gradient}:
\begin{equation*}
    \bm{\Phi}_{Counter}=\left[ 
\begin{array}{cccccccc}
1 & 2& 0& 0& 0& 0& 0& 0\\
1 & 0& 2& 0& 0& 0& 0& 0\\
1 & 0& 0& 2& 0& 0& 0& 0\\
1 & 0& 0& 0& 2& 0& 0& 0\\
1 & 0& 0& 0& 0& 2& 0& 0\\
1 & 0& 0& 0& 0& 0& 2& 0\\
2 & 0& 0& 0& 0& 0& 0& 1
\end{array}\right]
\end{equation*}
\begin{figure}[H]
    \begin{center}
        \input{main/pic/BairdExample.tex}
        \label{7statebairdexample}
        \caption{7-state off-policy counterexample.}
    \end{center}
\end{figure}
The $\alpha$ values for all algorithms are in the range of $\{0.0001, 0.0005, 0.001, 0.005, 0.01, 0.05, 0.1, 0.2, 0.3\}$. 
For the TDC and CTDC algorithm, the $\zeta$ values are in the range of $\{0.0005, 0.001, 0.005, 0.01, 0.05, 0.1, 0.2\}$. 
For the CTD and CTDC algorithm, the $\beta$ values are in the range of $\{0.0005, 0.001, 0.005, 0.01, 0.05, 0.1, 0.2\}$. 
\begin{figure}
    \vskip 0.2in
    \begin{center}
    \subfigure[TD]{
        \includegraphics[width=0.3\columnwidth, height=0.25\columnwidth]{main/pic/7_state/td/td.pdf}
        \label{td_7state}
    }
    \subfigure[TDC]{
        \includegraphics[width=0.3\columnwidth, height=0.25\columnwidth]{main/pic/7_state/tdc/tdc.pdf}
        \label{tdc_7state}
    }
    \subfigure[CTD]{
        \includegraphics[width=0.3\columnwidth, height=0.25\columnwidth]{main/pic/7_state/ctd/ctd.pdf}
        \label{ctd_7state}
    }
    \\
    \subfigure[CTDC($\alpha=0.0001$)]{
        \includegraphics[width=0.3\columnwidth, height=0.25\columnwidth]{main/pic/7_state/ctdc/alpha_0_0001.pdf}
        \label{ctdc_alpha_0001_7state}
    }
    \subfigure[CTDC($\alpha=0.0005$)]{
        \includegraphics[width=0.3\columnwidth, height=0.25\columnwidth]{main/pic/7_state/ctdc/alpha_0_0005.pdf}
        \label{ctdc_alpha_0005_7state}
    }
    \subfigure[CTDC($\alpha=0.001$)]{
        \includegraphics[width=0.3\columnwidth, height=0.25\columnwidth]{main/pic/7_state/ctdc/alpha_0_001.pdf}
        \label{ctdc_alpha_001_7state}
    }
    \\
    \subfigure[CTDC($\alpha=0.005$)]{
        \includegraphics[width=0.3\columnwidth, height=0.25\columnwidth]{main/pic/7_state/ctdc/alpha_0_005.pdf}
        \label{ctdc_alpha_005_7state}
    }
    \subfigure[CTDC($\alpha=0.01$)]{
        \includegraphics[width=0.3\columnwidth, height=0.25\columnwidth]{main/pic/7_state/ctdc/alpha_0_01.pdf}
        \label{ctdc_alpha_01_7state}
    }
        \caption{Sensitivity of various algorithms to learning rates for 7-state counterexample.}
        \label{Sensitivity7state}
    \end{center}
    \vskip -0.2in
\end{figure}


\end{document}


% This document was modified from the file originally made available by
% Pat Langley and Andrea Danyluk for ICML-2K. This version was created
% by Iain Murray in 2018, and modified by Alexandre Bouchard in
% 2019 and 2021 and by Csaba Szepesvari, Gang Niu and Sivan Sabato in 2022.
% Modified again in 2023 and 2024 by Sivan Sabato and Jonathan Scarlett.
% Previous contributors include Dan Roy, Lise Getoor and Tobias
% Scheffer, which was slightly modified from the 2010 version by
% Thorsten Joachims & Johannes Fuernkranz, slightly modified from the
% 2009 version by Kiri Wagstaff and Sam Roweis's 2008 version, which is
% slightly modified from Prasad Tadepalli's 2007 version which is a
% lightly changed version of the previous year's version by Andrew
% Moore, which was in turn edited from those of Kristian Kersting and
% Codrina Lauth. Alex Smola contributed to the algorithmic style files.
