\onecolumn
\appendix
\section{Relevant proofs}
\subsection{Proof of Theorem 2}
\label{proofth1}
\begin{proof}
\label{th1proof}   
 The proof is based on Borkar's Theorem for
 general stochastic approximation recursions with two time scales \cite{borkar1997stochastic}. 
 A new one-step linear TD solution is defined as: 
\begin{equation*}
0=\mathbb{E}[(\delta-\mathbb{E}[\delta]) \phi]=-A\theta+b.
\end{equation*}
Thus, the VMTD's solution is $\theta_{\text{VMTD}}=A^{-1}b$. First, note that recursion (\ref{theta}) can be rewritten as
\begin{equation*}
\theta_{k+1}\leftarrow \theta_k+\beta_k\xi(k),
\end{equation*}
where
\begin{equation*}
\xi(k)=\frac{\alpha_k}{\beta_k}(\delta_k-\omega_k)\phi_k
\end{equation*}
Due to the settings of step-size schedule $\alpha_k = o(\beta_k)$,
$\xi(k)\rightarrow 0$ almost surely as $k\rightarrow\infty$. 
 That is the increments in iteration (\ref{omega}) are uniformly larger than
 those in (\ref{theta}), thus (\ref{omega}) is the faster recursion.
 Along the faster time scale, iterations of (\ref{omega}) and (\ref{theta})
 are associated with the ODEs system as follows:
\begin{equation}
 \dot{\theta}(t) = 0,
\label{thetaFast}
\end{equation}
\begin{equation}
 \dot{\omega}(t)=\mathbb{E}[\delta_t|\theta(t)]-\omega(t).
\label{omegaFast}
\end{equation}
 Based on the ODE (\ref{thetaFast}), $\theta(t)\equiv \theta$ when
 viewed from the faster timescale. 
 By the Hirsch lemma \cite{hirsch1989convergent}, it follows that
$||\theta_k-\theta||\rightarrow 0$ a.s. as $k\rightarrow \infty$ for some
$\theta$ that depends on the initial condition $\theta_0$ of recursion
 (\ref{theta}).
 Thus, the ODE pair (\ref{thetaFast})-(\ref{omegaFast}) can be written as
\begin{equation}
 \dot{\omega}(t)=\mathbb{E}[\delta_t|\theta]-\omega(t).
\label{omegaFastFinal}
\end{equation}
 Consider the function $h(\omega)=\mathbb{E}[\delta|\theta]-\omega$,
 i.e., the driving vector field of the ODE (\ref{omegaFastFinal}).
 It is easy to find that the function $h$ is Lipschitz with coefficient
$-1$.
 Let $h_{\infty}(\cdot)$ be the function defined by
$h_{\infty}(\omega)=\lim_{x\rightarrow \infty}\frac{h(x\omega)}{x}$.
 Then  $h_{\infty}(\omega)= -\omega$,  is well-defined. 
 For (\ref{omegaFastFinal}), $\omega^*=\mathbb{E}[\delta|\theta]$
 is the unique globally asymptotically stable equilibrium.
 For the ODE
  \begin{equation}
 \dot{\omega}(t) = h_{\infty}(\omega(t))= -\omega(t),
 \label{omegaInfty}
 \end{equation}
 apply $\vec{V}(\omega)=(-\omega)^{\top}(-\omega)/2$ as its
 associated strict Liapunov function. Then,
 the origin of (\ref{omegaInfty}) is a globally asymptotically stable
 equilibrium. Consider now the recursion (\ref{omega}).
 Let $M_{k+1}=(\delta_k-\omega_k)
 -\mathbb{E}[(\delta_k-\omega_k)|\mathcal{F}(k)]$,
 where $\mathcal{F}(k)=\sigma(\omega_l,\theta_l,l\leq k;\phi_s,\phi_s',r_s,s<k)$, $k\geq 1$ are the sigma fields
 generated by $\omega_0,\theta_0,\omega_{l+1},\theta_{l+1},\phi_l,\phi_l'$, $0\leq l<k$.
 It is easy to verify that $M_{k+1},k\geq0$ are integrable random variables that 
 satisfy $\mathbb{E}[M_{k+1}|\mathcal{F}(k)]=0$, $\forall k\geq0$.
 Because $\phi_k$, $r_k$, and $\phi_k'$ have
 uniformly bounded second moments, it can be seen that for some constant $c_1>0$, $\forall k\geq0$,
\begin{equation*}
 \mathbb{E}[||M_{k+1}||^2|\mathcal{F}(k)]\leq
 c_1(1+||\omega_k||^2+||\theta_k||^2).
\end{equation*}
Now Assumptions (A1) and (A2) of \cite{borkar2000ode} are verified.
 Furthermore, Assumptions (TS) of \cite{borkar2000ode} are satisfied by our
 conditions on the step-size sequences $\alpha_k$, $\beta_k$. Thus,
 by Theorem 2.2 of \cite{borkar2000ode} we obtain that $||\omega_k-\omega^*||\rightarrow 0$ almost surely as $k\rightarrow \infty$.
Consider now the slower time scale recursion (\ref{theta}).
 Based on the above analysis, (\ref{theta}) can be rewritten as 
\begin{equation*}
\theta_{k+1}\leftarrow
\theta_{k}+\alpha_k(\delta_k-\mathbb{E}[\delta_k|\theta_k])\phi_k.
\end{equation*}
Let $\mathcal{G}(k)=\sigma(\theta_l,l\leq k;\phi_s,\phi_s',r_s,s<k)$, 
$k\geq 1$ be the sigma fields
 generated by $\theta_0,\theta_{l+1},\phi_l,\phi_l'$,
$0\leq l<k$.
 Let $Z_{k+1} = Y_{t}-\mathbb{E}[Y_{t}|\mathcal{G}(k)]$,
 where
\begin{equation*}
 Y_{k}=(\delta_k-\mathbb{E}[\delta_k|\theta_k])\phi_k.
\end{equation*}
 Consequently,
\begin{equation*}
\begin{array}{ccl}
 \mathbb{E}[Y_t|\mathcal{G}(k)]&=&\mathbb{E}[(\delta_k-\mathbb{E}[\delta_k|\theta_k])\phi_k|\mathcal{G}(k)]\\
&=&\mathbb{E}[\delta_k\phi_k|\theta_k]
 -\mathbb{E}[\mathbb{E}[\delta_k|\theta_k]\phi_k]\\
&=&\mathbb{E}[\delta_k\phi_k|\theta_k]
 -\mathbb{E}[\delta_k|\theta_k]\mathbb{E}[\phi_k]\\
&=&\mathrm{Cov}(\delta_k|\theta_k,\phi_k),
\end{array}
\end{equation*}
 where $\mathrm{Cov}(\cdot,\cdot)$ is a covariance operator.
Thus,
 \begin{equation*}
\begin{array}{ccl}
 Z_{k+1}&=&(\delta_k-\mathbb{E}[\delta_k|\theta_k])\phi_k-\mathrm{Cov}(\delta_k|\theta_k,\phi_k).
\end{array}
\end{equation*}
 It is easy to verify that $Z_{k+1},k\geq0$ are integrable random variables that 
 satisfy $\mathbb{E}[Z_{k+1}|\mathcal{G}(k)]=0$, $\forall k\geq0$.
 Also, because $\phi_k$, $r_k$, and $\phi_k'$ have
 uniformly bounded second moments, it can be seen that for some constant
$c_2>0$, $\forall k\geq0$,
\begin{equation*}
 \mathbb{E}[||Z_{k+1}||^2|\mathcal{G}(k)]\leq
 c_2(1+||\theta_k||^2).
\end{equation*}

 Consider now the following ODE associated with (\ref{theta}):
\begin{equation}
\begin{array}{ccl}
 \dot{\theta}(t)&=&\mathrm{Cov}(\delta|\theta(t),\phi)\\
&=&\mathrm{Cov}(r+(\gamma\phi'-\phi)^{\top}\theta(t),\phi)\\
&=&\mathrm{Cov}(r,\phi)-\mathrm{Cov}(\theta(t)^{\top}(\phi-\gamma\phi'),\phi)\\
&=&\mathrm{Cov}(r,\phi)-\theta(t)^{\top}\mathrm{Cov}(\phi-\gamma\phi',\phi)\\
&=&\mathrm{Cov}(r,\phi)-\mathrm{Cov}(\phi-\gamma\phi',\phi)^{\top}\theta(t)\\
&=&\mathrm{Cov}(r,\phi)-\mathrm{Cov}(\phi,\phi-\gamma\phi')\theta(t)\\
&=&-A\theta(t)+b.
\end{array}
\label{odetheta}
\end{equation}
 Let $\vec{h}(\theta(t))$ be the driving vector field of the ODE
 (\ref{odetheta}).
\begin{equation*}
 \vec{h}(\theta(t))=-A\theta(t)+b.
\end{equation*}
 Consider the cross-covariance matrix,
\begin{equation}
\begin{array}{ccl}
 A &=& \mathrm{Cov}(\phi,\phi-\gamma\phi')\\
  &=&\frac{\mathrm{Cov}(\phi,\phi)+\mathrm{Cov}(\phi-\gamma\phi',\phi-\gamma\phi')-\mathrm{Cov}(\gamma\phi',\gamma\phi')}{2}\\
  &=&\frac{\mathrm{Cov}(\phi,\phi)+\mathrm{Cov}(\phi-\gamma\phi',\phi-\gamma\phi')-\gamma^2\mathrm{Cov}(\phi',\phi')}{2}\\
  &=&\frac{(1-\gamma^2)\mathrm{Cov}(\phi,\phi)+\mathrm{Cov}(\phi-\gamma\phi',\phi-\gamma\phi')}{2},\\
\end{array}
\label{covariance}
\end{equation}
 where we eventually used $\mathrm{Cov}(\phi',\phi')=\mathrm{Cov}(\phi,\phi)$
\footnote{The covariance matrix $\mathrm{Cov}(\phi',\phi')$ is equal to
 the covariance matrix $\mathrm{Cov}(\phi,\phi)$ if the initial state is re-reachable or
 initialized randomly in a Markov chain for on-policy update.}.
 Note that the covariance matrix $\mathrm{Cov}(\phi,\phi)$ and
$\mathrm{Cov}(\phi-\gamma\phi',\phi-\gamma\phi')$ are semi-positive
 definite. Then, the matrix $A$ is semi-positive definite because  $A$ is
 linearly combined  by  two positive-weighted semi-positive definite matrice
 (\ref{covariance}).
 Furthermore, $A$ is nonsingular due to the assumption.
 Hence, the cross-covariance matrix $A$ is positive definite.

 Therefore,
$\theta^*=A^{-1}b$ can be seen to be the unique globally asymptotically
 stable equilibrium for ODE (\ref{odetheta}).
 Let $\vec{h}_{\infty}(\theta)=\lim_{r\rightarrow
\infty}\frac{\vec{h}(r\theta)}{r}$. Then
$\vec{h}_{\infty}(\theta)=-A\theta$ is well-defined. 
 Consider now the ODE
\begin{equation}
 \dot{\theta}(t)=-A\theta(t).
\label{odethetafinal}
\end{equation}
 The ODE (\ref{odethetafinal}) has the origin of its unique globally asymptotically stable equilibrium.
 Thus, the assumption (A1) and (A2) are verified.
    \end{proof}

\subsection{Proof of Theorem 3}
\label{proofth2}
\begin{proof}
The proof is similar to that given by \cite{sutton2009fast} for TDC, but it is based on multi-time-scale stochastic approximation.

For the VMTDC algorithm, a new one-step linear TD solution is defined as:
\begin{equation*}
    0=\mathbb{E}[({\phi} - \gamma {\phi}' - \mathbb{E}[{\phi} - \gamma {\phi}']){\phi}^\top]\mathbb{E}[{\phi} {\phi}^{\top}]^{-1}\mathbb{E}[(\delta -\mathbb{E}[\delta]){\phi}]=\textbf{A}^{\top}\textbf{C}^{-1}(-\textbf{A}{\theta}+{b}).
\end{equation*}
The matrix $\textbf{A}^{\top}\textbf{C}^{-1}\textbf{A}$ is positive definite. Thus, the  VMTD's solution is
${\theta}_{\text{VMTDC}}=\textbf{A}^{-1}{b}$.

First, note that recursion (\ref{thetavmtdc}) and (\ref{uvmtdc}) can be rewritten as, respectively, 
\begin{equation*}
 {\theta}_{k+1}\leftarrow {\theta}_k+\zeta_k {x}(k),
\end{equation*}
\begin{equation*}
 {u}_{k+1}\leftarrow {u}_k+\beta_k {y}(k),
\end{equation*}
where 
\begin{equation*}
 {x}(k)=\frac{\alpha_k}{\zeta_k}[(\delta_{k}- \omega_k) {\phi}_k - \gamma{\phi}'_{k}({\phi}^{\top}_k {u}_k)],
\end{equation*}
\begin{equation*}
 {y}(k)=\frac{\zeta_k}{\beta_k}[\delta_{k}-\omega_k - {\phi}^{\top}_k {u}_k]{\phi}_k.
\end{equation*}

Recursion (\ref{thetavmtdc}) can also be rewritten as
\begin{equation*}
 {\theta}_{k+1}\leftarrow {\theta}_k+\beta_k z(k),
\end{equation*}
where
\begin{equation*}
 z(k)=\frac{\alpha_k}{\beta_k}[(\delta_{k}- \omega_k) {\phi}_k - \gamma{\phi}'_{k}({\phi}^{\top}_k {u}_k)],
\end{equation*}

Due to the settings of the step-size schedule 
$\alpha_k = o(\zeta_k)$, $\zeta_k = o(\beta_k)$, ${x}(k)\rightarrow 0$, ${y}(k)\rightarrow 0$, $z(k)\rightarrow 0$ almost surely as $k\rightarrow 0$.
That is the increments in iteration (\ref{omegavmtdc}) are uniformly larger than
those in (\ref{uvmtdc}) and  the increments in iteration (\ref{uvmtdc}) are uniformly larger than
those in (\ref{thetavmtdc}), thus (\ref{omegavmtdc}) is the fastest recursion, (\ref{uvmtdc}) is the second fast recursion, and (\ref{thetavmtdc}) is the slower recursion.
Along the fastest time scale, iterations of (\ref{thetavmtdc}), (\ref{uvmtdc}) and (\ref{omegavmtdc})
are associated with the ODEs system as follows:
\begin{equation}
 \dot{{\theta}}(t) = 0,
    \label{thetavmtdcFastest}
\end{equation}
\begin{equation}
 \dot{{u}}(t) = 0,
    \label{uvmtdcFastest}
\end{equation}
\begin{equation}
 \dot{\omega}(t)=\mathbb{E}[\delta_t|{u}(t),{\theta}(t)]-\omega(t).
    \label{omegavmtdcFastest}
\end{equation}

Based on the ODE (\ref{thetavmtdcFastest}) and (\ref{uvmtdcFastest}), both ${\theta}(t)\equiv {\theta}$
and ${u}(t)\equiv {u}$ when viewed from the fastest timescale.
By the Hirsch lemma \cite{hirsch1989convergent}, it follows that
$||{\theta}_k-{\theta}||\rightarrow 0$ a.s. as $k\rightarrow \infty$ for some
${\theta}$ that depends on the initial condition ${\theta}_0$ of recursion
(\ref{thetavmtdc}) and $||{u}_k-{u}||\rightarrow 0$ a.s. as $k\rightarrow \infty$ for some
$u$ that depends on the initial condition $u_0$ of recursion
(\ref{uvmtdc}). Thus, the ODE pair (\ref{thetavmtdcFastest})-(ref{omegavmtdcFastest})
can be written as 
\begin{equation}
 \dot{\omega}(t)=\mathbb{E}[\delta_t|{u},{\theta}]-\omega(t).
    \label{omegavmtdcFastestFinal}
\end{equation}

Consider the function $h(\omega)=\mathbb{E}[\delta|{\theta},{u}]-\omega$,
i.e., the driving vector field of the ODE (\ref{omegavmtdcFastestFinal}).
It is easy to find that the function $h$ is Lipschitz with coefficient
$-1$.
Let $h_{\infty}(\cdot)$ be the function defined by
 $h_{\infty}(\omega)=\lim_{r\rightarrow \infty}\frac{h(r\omega)}{r}$.
 Then  $h_{\infty}(\omega)= -\omega$,  is well-defined. 
 For (\ref{omegavmtdcFastestFinal}), $\omega^*=\mathbb{E}[\delta|{\theta},{u}]$
is the unique globally asymptotically stable equilibrium.
For the ODE
\begin{equation}
 \dot{\omega}(t) = h_{\infty}(\omega(t))= -\omega(t),
 \label{omegavmtdcInfty}
\end{equation}
apply $\vec{V}(\omega)=(-\omega)^{\top}(-\omega)/2$ as its
associated strict Liapunov function. Then,
the origin of (\ref{omegavmtdcInfty}) is a globally asymptotically stable
equilibrium.

Consider now the recursion (\ref{omegavmtdc}).
Let
$M_{k+1}=(\delta_k-\omega_k)
-\mathbb{E}[(\delta_k-\omega_k)|\mathcal{F}(k)]$,
where $\mathcal{F}(k)=\sigma(\omega_l,{u}_l,{\theta}_l,l\leq k;{\phi}_s,{\phi}_s',r_s,s<k)$, 
$k\geq 1$ are the sigma fields
generated by $\omega_0,u_0,{\theta}_0,\omega_{l+1},{u}_{l+1},{\theta}_{l+1},{\phi}_l,{\phi}_l'$,
$0\leq l<k$.
It is easy to verify that $M_{k+1},k\geq0$ are integrable random variables that 
satisfy $\mathbb{E}[M_{k+1}|\mathcal{F}(k)]=0$, $\forall k\geq0$.
Because ${\phi}_k$, $r_k$, and ${\phi}_k'$ have
uniformly bounded second moments, it can be seen that for some constant
$c_1>0$, $\forall k\geq0$,
\begin{equation*}
\mathbb{E}[||M_{k+1}||^2|\mathcal{F}(k)]\leq
c_1(1+||\omega_k||^2+||{u}_k||^2+||{\theta}_k||^2).
\end{equation*}


Now Assumptions (A1) and (A2) of \cite{borkar2000ode} are verified.
Furthermore, Assumptions (TS) of \cite{borkar2000ode} is satisfied by our
conditions on the step-size sequences $\alpha_k$,$\zeta_k$, $\beta_k$. Thus,
by Theorem 2.2 of \cite{borkar2000ode} we obtain that
$||\omega_k-\omega^*||\rightarrow 0$ almost surely as $k\rightarrow \infty$.

Consider now the second time scale recursion (\ref{uvmtdc}).
Based on the above analysis, (\ref{uvmtdc}) can be rewritten as
% \begin{equation*}
%     {u}_{k+1}\leftarrow u_{k}+\zeta_{k}[\delta_{k}-\mathbb{E}[\delta_k|{u}_k,{\theta}_k] - {\phi}^{\top} (s_k) {u}_k]{\phi}(s_k).
% \end{equation*}
\begin{equation}
 \dot{{\theta}}(t) = 0,
    \label{thetavmtdcFaster}
\end{equation}
\begin{equation}
 \dot{u}(t) = \mathbb{E}[(\delta_t-\mathbb{E}[\delta_t|{u}(t),{\theta}(t)]){\phi}_t|{\theta}(t)] - \textbf{C}{u}(t).
    \label{uvmtdcFaster}
\end{equation}
The ODE (\ref{thetavmtdcFaster}) suggests that ${\theta}(t)\equiv {\theta}$ (i.e., a time-invariant parameter)
when viewed from the second fast timescale.
By the Hirsch lemma \cite{hirsch1989convergent}, it follows that
$||{\theta}_k-{\theta}||\rightarrow 0$ a.s. as $k\rightarrow \infty$ for some
${\theta}$ that depends on the initial condition ${\theta}_0$ of recursion
(\ref{thetavmtdc}). 

Consider now the recursion (\ref{uvmtdc}).
Let
$N_{k+1}=((\delta_k-\mathbb{E}[\delta_k]) - {\phi}_k {\phi}^{\top}_k {u}_k) -\mathbb{E}[((\delta_k-\mathbb{E}[\delta_k]) - {\phi}_k {\phi}^{\top}_k {u}_k)|\mathcal{I} (k)]$,
where $\mathcal{I}(k)=\sigma({u}_l,{\theta}_l,l\leq k;{\phi}_s,{\phi}_s',r_s,s<k)$, 
$k\geq 1$ are the sigma fields
generated by ${u}_0,{\theta}_0,{u}_{l+1},{\theta}_{l+1},{\phi}_l,{\phi}_l'$,
$0\leq l<k$.
It is easy to verify that $N_{k+1},k\geq0$ are integrable random variables that 
satisfy $\mathbb{E}[N_{k+1}|\mathcal{I}(k)]=0$, $\forall k\geq0$.
Because ${\phi}_k$, $r_k$, and ${\phi}_k'$ have
uniformly bounded second moments, it can be seen that for some constant
$c_2>0$, $\forall k\geq0$,
\begin{equation*}
\mathbb{E}[||N_{k+1}||^2|\mathcal{I}(k)]\leq
c_2(1+||{u}_k||^2+||{\theta}_k||^2).
\end{equation*}

Because ${\theta}(t)\equiv {\theta}$ from (\ref{thetavmtdcFaster}), the ODE pair (\ref{thetavmtdcFaster})-(\ref{uvmtdcFaster})
can be written as 
\begin{equation}
 \dot{{u}}(t) = \mathbb{E}[(\delta_t-\mathbb{E}[\delta_t|{\theta}]){\phi}_t|{\theta}] - \textbf{C}{u}(t).
    \label{uvmtdcFasterFinal}
\end{equation}
Now consider the function $h({u})=\mathbb{E}[\delta_t-\mathbb{E}[\delta_t|{\theta}]|{\theta}] -\textbf{C}{u}$, i.e., the
driving vector field of the ODE (\ref{uvmtdcFasterFinal}). For (\ref{uvmtdcFasterFinal}),
${u}^* = \textbf{C}^{-1}\mathbb{E}[(\delta-\mathbb{E}[\delta|{\theta}]){\phi}|{\theta}]$ is the unique globally asymptotically
stable equilibrium. Let $h_{\infty}({u})=-\textbf{C}{u}$.
For the ODE
\begin{equation}
 \dot{{u}}(t) = h_{\infty}({u}(t))= -\textbf{C}{u}(t),
    \label{uvmtdcInfty}
\end{equation}
the origin of (\ref{uvmtdcInfty}) is a globally asymptotically stable
equilibrium because $\textbf{C}$ is a positive definite matrix (because it is nonnegative definite and nonsingular).
Now Assumptions (A1) and (A2) of \cite{borkar2000ode} are verified.
Furthermore, Assumptions (TS) of \cite{borkar2000ode} is satisfied by our
conditions on the step-size sequences $\alpha_k$,$\zeta_k$, $\beta_k$. Thus,
by Theorem 2.2 of \cite{borkar2000ode} we obtain that
$||{u}_k-{u}^*||\rightarrow 0$ almost surely as $k\rightarrow \infty$.

Consider now the slower timescale recursion (\ref{thetavmtdc}). In the light of the above,
(\ref{thetavmtdc}) can be rewritten as 
\begin{equation}
 {\theta}_{k+1} \leftarrow {\theta}_{k} + \alpha_k (\delta_k -\mathbb{E}[\delta_k|{\theta}_k]) {\phi}_k\\
 - \alpha_k \gamma{\phi}'_{k}({\phi}^{\top}_k \textbf{C}^{-1}\mathbb{E}[(\delta_k -\mathbb{E}[\delta_k|{\theta}_k]){\phi}|{\theta}_k]).
\end{equation}
Let $\mathcal{G}(k)=\sigma({\theta}_l,l\leq k;{\phi}_s,{\phi}_s',r_s,s<k)$, 
$k\geq 1$ be the sigma fields
generated by ${\theta}_0,{\theta}_{l+1},{\phi}_l,{\phi}_l'$,
$0\leq l<k$. Let
\begin{equation*}
    \begin{array}{ccl}
 Z_{k+1}&=&((\delta_k -\mathbb{E}[\delta_k|{\theta}_k]) {\phi}_k - \gamma {\phi}'_{k}{\phi}^{\top}_k \textbf{C}^{-1}\mathbb{E}[(\delta_k -\mathbb{E}[\delta_k|{\theta}_k]){\phi}|{\theta}_k])\\ 
     & &-\mathbb{E}[((\delta_k -\mathbb{E}[\delta_k|{\theta}_k]) {\phi}_k - \gamma {\phi}'_{k}{\phi}^{\top}_k \textbf{C}^{-1}\mathbb{E}[(\delta_k -\mathbb{E}[\delta_k|{\theta}_k]){\phi}|{\theta}_k])|\mathcal{G}(k)]\\
    &=&((\delta_k -\mathbb{E}[\delta_k|{\theta}_k]) {\phi}_k - \gamma {\phi}'_{k}{\phi}^{\top}_k \textbf{C}^{-1}\mathbb{E}[(\delta_k -\mathbb{E}[\delta_k|{\theta}_k]){\phi}|{\theta}_k])\\
    & &-\mathbb{E}[(\delta_k -\mathbb{E}[\delta_k|{\theta}_k]) {\phi}_k|{\theta}_k] - \gamma\mathbb{E}[{\phi}' {\phi}^{\top}]\textbf{C}^{-1}\mathbb{E}[(\delta_k -\mathbb{E}[\delta_k|{\theta}_k]) {\phi}_k|{\theta}_k].
    \end{array}
\end{equation*}
It is easy to see that $Z_k$, $k\geq 0$ are integrable random variables and $\mathbb{E}[Z_{k+1}|\mathcal{G}(k)]=0$, $\forall k\geq0$. Further,
\begin{equation*}
\mathbb{E}[||Z_{k+1}||^2|\mathcal{G}(k)]\leq
c_3(1+||{\theta}_k||^2), k\geq 0
\end{equation*}
for some constant $c_3 \geq 0$, again because ${\phi}_k$, $r_k$, and ${\phi}_k'$ have
uniformly bounded second moments, it can be seen that for some constant.

Consider now the following ODE associated with (\ref{thetavmtdc}):
\begin{equation}
 \dot{{\theta}}(t) = (\textbf{I} - \mathbb{E}[\gamma {\phi}' {\phi}^{\top}]\textbf{C}^{-1})\mathbb{E}[(\delta -\mathbb{E}[\delta|{\theta}(t)]) {\phi}|{\theta}(t)].
    \label{thetavmtdcSlowerFinal}
\end{equation}
Let 
\begin{equation*}
\begin{array}{ccl}
 \vec{h}({\theta}(t))&=&(\textbf{I} - \mathbb{E}[\gamma {\phi}' {\phi}^{\top}]\textbf{C}^{-1})\mathbb{E}[(\delta -\mathbb{E}[\delta|{\theta}(t)]) {\phi}|{\theta}(t)]\\
    &=&(\textbf{C} - \mathbb{E}[\gamma {\phi}' {\phi}^{\top}])\textbf{C}^{-1}\mathbb{E}[(\delta -\mathbb{E}[\delta|{\theta}(t)]) {\phi}|{\theta}(t)]\\
    &=& (\mathbb{E}[{\phi} {\phi}^{\top}] - \mathbb{E}[\gamma {\phi}' {\phi}^{\top}])\textbf{C}^{-1}\mathbb{E}[(\delta -\mathbb{E}[\delta|{\theta}(t)]) {\phi}|{\theta}(t)]\\
    &=& \textbf{A}^{\top}\textbf{C}^{-1}(-\textbf{A}{\theta}(t)+{b}),
\end{array}
\end{equation*}
because $\mathbb{E}[(\delta -\mathbb{E}[\delta|{\theta}(t)]) {\phi}|{\theta}(t)]=-\textbf{A}{\theta}(t)+{b}$, where 
$\textbf{A} = \mathrm{Cov}({\phi},{\phi}-\gamma{\phi}')$, ${b}=\mathrm{Cov}(r,{\phi})$, and $\textbf{C}=\mathbb{E}[{\phi}{\phi}^{\top}]$

Therefore,
${\theta}^*=\textbf{A}^{-1}{b}$ can be seen to be the unique globally asymptotically
stable equilibrium for ODE (\ref{thetavmtdcSlowerFinal}).
Let $\vec{h}_{\infty}({\theta})=\lim_{r\rightarrow
\infty}\frac{\vec{h}(r{\theta})}{r}$. Then
$\vec{h}_{\infty}({\theta})=-\textbf{A}^{\top}\textbf{C}^{-1}\textbf{A}{\theta}$ is well-defined. 
Consider now the ODE
\begin{equation}
\dot{{\theta}}(t)=-\textbf{A}^{\top}\textbf{C}^{-1}\textbf{A}{\theta}(t).
\label{odethetavmtdcfinal}
\end{equation}

Because $\textbf{C}^{-1}$ is positive definite and $\textbf{A}$ has full rank (as it
is nonsingular by assumption), the matrix $\textbf{A}^{\top} \textbf{C}^{-1}\textbf{A}$ is also
positive definite. 
The ODE (\ref{odethetavmtdcfinal}) has the origin of its unique globally asymptotically stable equilibrium.
Thus, the assumption (A1) and (A2) are verified.

The proof is given above.
In the fastest time scale, the parameter $w$ converges to
$\mathbb{E}[\delta|{u}_k,{\theta}_k]$.
In the second fast time scale,
the parameter $u$ converges to $\textbf{C}^{-1}\mathbb{E}[(\delta-\mathbb{E}[\delta|{\theta}_k]){\phi}|{\theta}_k]$.
In the slower time scale,
the parameter ${\theta}$ converges to $\textbf{A}^{-1}{b}$.
\end{proof}

\subsection{Proof of Theorem 4}
\label{proofVMETD}
\begin{proof}
\label{th4proof}   
The proof of VMETD's convergence is also based on Borkar's Theorem   for
general stochastic approximation recursions with two time scales
\cite{borkar1997stochastic}. 

The  VMTD's solution is
${\theta}_{\text{VMETD}}=\textbf{A}_{\text{VMETD}}^{-1}{b}_{\text{VMETD}}$.
First, note that recursion (\ref{thetavmetd}) can be rewritten as
\begin{equation*}
 {\theta}_{k+1}\leftarrow {\theta}_k+\beta_k{\xi}(k),
\end{equation*}
 where
\begin{equation*}
 {\xi}(k)=\frac{\alpha_k}{\beta_k} (F_k \rho_k\delta_k - \omega_{k+1}){\phi}_k
\end{equation*}
 Due to the settings of step-size schedule $\alpha_k = o(\beta_k)$,
${\xi}(k)\rightarrow 0$ almost surely as $k\rightarrow\infty$. 
 That is the increments in iteration (\ref{omegavmetd}) are uniformly larger than
 those in (\ref{thetavmetd}), thus (\ref{omegavmetd}) is the faster recursion.
 Along the faster time scale, iterations of (\ref{thetavmetd}) and (\ref{omegavmetd})
 are associated with the ODEs system as follows:
\begin{equation}
 \dot{{\theta}}(t) = 0,
\label{vmetdthetaFast}
\end{equation}
\begin{equation}
 \dot{\omega}(t)=\mathbb{E}_{\mu}[F_t\rho_t\delta_t|{\theta}(t)]-\omega(t).
\label{vmetdomegaFast}
\end{equation}
 Based on the ODE (\ref{vmetdthetaFast}), ${\theta}(t)\equiv {\theta}$ when
 viewed from the faster timescale. 
 By the Hirsch lemma \cite{hirsch1989convergent}, it follows that
$||{\theta}_k-{\theta}||\rightarrow 0$ a.s. as $k\rightarrow \infty$ for some
${\theta}$ that depends on the initial condition ${\theta}_0$ of recursion
(\ref{thetavmetd}).
 Thus, the ODE pair (\ref{vmetdthetaFast})-(\ref{vmetdomegaFast}) can be written as
\begin{equation}
 \dot{\omega}(t)=\mathbb{E}_{\mu}[F_t\rho_t\delta_t|{\theta}]-\omega(t).
\label{vmetdomegaFastFinal}
\end{equation}
 Consider the function $h(\omega)=\mathbb{E}_{\mu}[F\rho\delta|{\theta}]-\omega$,
 i.e., the driving vector field of the ODE (\ref{vmetdomegaFastFinal}).
 It is easy to find that the function $h$ is Lipschitz with coefficient
$-1$.
 Let $h_{\infty}(\cdot)$ be the function defined by
 $h_{\infty}(\omega)=\lim_{x\rightarrow \infty}\frac{h(x\omega)}{x}$.
 Then  $h_{\infty}(\omega)= -\omega$,  is well-defined. 
 For (\ref{vmetdomegaFastFinal}), $\omega^*=\mathbb{E}_{\mu}[F\rho\delta|{\theta}]$
 is the unique globally asymptotically stable equilibrium.
 For the ODE
  \begin{equation}
 \dot{\omega}(t) = h_{\infty}(\omega(t))= -\omega(t),
 \label{vmetdomegaInfty}
 \end{equation}
 apply $\vec{V}(\omega)=(-\omega)^{\top}(-\omega)/2$ as its
 associated strict Liapunov function. Then,
 the origin of (\ref{vmetdomegaInfty}) is a globally asymptotically stable
 equilibrium.


 Consider now the recursion (\ref{omegavmetd}).
 Let
$M_{k+1}=(F_k\rho_k\delta_k-\omega_k)
 -\mathbb{E}_{\mu}[(F_k\rho_k\delta_k-\omega_k)|\mathcal{F}(k)]$,
 where $\mathcal{F}(k)=\sigma(\omega_l,{\theta}_l,l\leq k;{\phi}_s,{\phi}_s',r_s,s<k)$, 
$k\geq 1$ are the sigma fields
 generated by $\omega_0,{\theta}_0,\omega_{l+1},{\theta}_{l+1},{\phi}_l,{\phi}_l'$,
$0\leq l<k$.
 It is easy to verify that $M_{k+1},k\geq0$ are integrable random variables that 
 satisfy $\mathbb{E}[M_{k+1}|\mathcal{F}(k)]=0$, $\forall k\geq0$.
 Because ${\phi}_k$, $r_k$, and ${\phi}_k'$ have
 uniformly bounded second moments, it can be seen that for some constant
$c_1>0$, $\forall k\geq0$,
\begin{equation*}
 \mathbb{E}[||M_{k+1}||^2|\mathcal{F}(k)]\leq
 c_1(1+||\omega_k||^2+||{\theta}_k||^2).
\end{equation*}


 Now Assumptions (A1) and (A2) of \cite{borkar2000ode} are verified.
 Furthermore, Assumptions (TS) of \cite{borkar2000ode} are satisfied by our
 conditions on the step-size sequences $\alpha_k$, $\beta_k$. Thus,
 by Theorem 2.2 of \cite{borkar2000ode} we obtain that
$||\omega_k-\omega^*||\rightarrow 0$ almost surely as $k\rightarrow \infty$.

 Consider now the slower time scale recursion (\ref{thetavmetd}).
 Based on the above analysis, (\ref{thetavmetd}) can be rewritten as 

\begin{equation*}
    \begin{split}
 {\theta}_{k+1}&\leftarrow {\theta}_k+\alpha_k (F_k \rho_k\delta_k - \omega_k){\phi}_k -\alpha_k \omega_{k+1}{\phi}_k\\
&={\theta}_{k}+\alpha_k(F_k\rho_k\delta_k-\mathbb{E}_{\mu}[F_k\rho_k\delta_k|{\theta}_k]){\phi}_k\\
    &={\theta}_k+\alpha_k F_k \rho_k (R_{k+1}+\gamma {\theta}_k^{\top}{\phi}_{k+1}-{\theta}_k^{\top}{\phi}_k){\phi}_k -\alpha_k \mathbb{E}_{\mu}[F_k \rho_k \delta_k]{\phi}_k\\
    &= {\theta}_k+\alpha_k \{\underbrace{(F_k\rho_kR_{k+1}-\mathbb{E}_{\mu}[F_k\rho_k R_{k+1}]){\phi}_k}_{{b}_{\text{VMETD},k}}
 -\underbrace{(F_k\rho_k{\phi}_k({\phi}_k-\gamma{\phi}_{k+1})^{\top}-{\phi}_k\mathbb{E}_{\mu}[F_k\rho_k ({\phi}_k-\gamma{\phi}_{k+1})]^{\top})}_{\textbf{A}_{\text{VMETD},k}}{\theta}_k\}
\end{split}
\end{equation*}

 Let $\mathcal{G}(k)=\sigma({\theta}_l,l\leq k;{\phi}_s,{\phi}_s',r_s,s<k)$, 
$k\geq 1$ be the sigma fields
 generated by ${\theta}_0,{\theta}_{l+1},{\phi}_l,{\phi}_l'$,
$0\leq l<k$.
 Let 
$
 Z_{k+1} = Y_{k}-\mathbb{E}[Y_{k}|\mathcal{G}(k)],
$
 where
\begin{equation*}
 Y_{k}=(F_k\rho_k\delta_k-\mathbb{E}_{\mu}[F_k\rho_k\delta_k|{\theta}_k]){\phi}_k.
\end{equation*}
 Consequently,
\begin{equation*}
\begin{array}{ccl}
 \mathbb{E}_{\mu}[Y_k|\mathcal{G}(k)]&=&\mathbb{E}_{\mu}[(F_k\rho_k\delta_k-\mathbb{E}_{\mu}[F_k\rho_k\delta_k|{\theta}_k]){\phi}_k|\mathcal{G}(k)]\\
&=&\mathbb{E}_{\mu}[F_k\rho_k\delta_k{\phi}_k|{\theta}_k]
 -\mathbb{E}_{\mu}[\mathbb{E}_{\mu}[F_k\rho_k\delta_k|{\theta}_k]{\phi}_k]\\
&=&\mathbb{E}_{\mu}[F_k\rho_k\delta_k{\phi}_k|{\theta}_k]
 -\mathbb{E}_{\mu}[F_k\rho_k\delta_k|{\theta}_k]\mathbb{E}_{\mu}[{\phi}_k]\\
&=&\mathrm{Cov}(F_k\rho_k\delta_k|{\theta}_k,{\phi}_k),
\end{array}
\end{equation*}
 where $\mathrm{Cov}(\cdot,\cdot)$ is a covariance operator.

 Thus,
 \begin{equation*}
\begin{array}{ccl}
 Z_{k+1}&=&(F_k\rho_k\delta_k-\mathbb{E}[\delta_k|{\theta}_k]){\phi}_k-\mathrm{Cov}(F_k\rho_k\delta_k|{\theta}_k,{\phi}_k).
\end{array}
\end{equation*}
 It is easy to verify that $Z_{k+1},k\geq0$ are integrable random variables that 
 satisfy $\mathbb{E}[Z_{k+1}|\mathcal{G}(k)]=0$, $\forall k\geq0$.
 Also, because ${\phi}_k$, $r_k$, and ${\phi}_k'$ have
 uniformly bounded second moments, it can be seen that for some constant
$c_2>0$, $\forall k\geq0$,
\begin{equation*}
 \mathbb{E}[||Z_{k+1}||^2|\mathcal{G}(k)]\leq
 c_2(1+||{\theta}_k||^2).
\end{equation*}

 Consider now the following ODE associated with (\ref{thetavmetd}):
\begin{equation}
\begin{array}{ccl}
 \dot{{\theta}}(t)&=&-\textbf{A}_{\text{VMETD}}{\theta}(t)+{b}_{\text{VMETD}}.
\end{array}
\label{odethetavmetd}
\end{equation}
\begin{equation}
    \begin{split}
 \textbf{A}_{\text{VMETD}}&=\lim_{k \rightarrow \infty} \mathbb{E}[\textbf{A}_{\text{VMETD},k}]\\
&= \lim_{k \rightarrow \infty} \mathbb{E}_{\mu}[F_k \rho_k {\phi}_k ({\phi}_k - \gamma {\phi}_{k+1})^{\top}]- \lim_{k\rightarrow \infty} \mathbb{E}_{\mu}[  {\phi}_k]\mathbb{E}_{\mu}[F_k \rho_k ({\phi}_k - \gamma {\phi}_{k+1})]^{\top}\\  
% &= \lim_{k \rightarrow \infty} \mathbb{E}_{\mu}[\underbrace{{\phi}_k}_{X}\underbrace{F_k \rho_k  ({\phi}_k - \gamma {\phi}_{k+1})^{\top}}_{Y}]- \lim_{k\rightarrow \infty} \mathbb{E}_{\mu}[  {\phi}_k]\mathbb{E}_{\mu}[F_k \rho_k ({\phi}_k - \gamma {\phi}_{k+1})]^{\top}\\  
&= \lim_{k \rightarrow \infty} \mathbb{E}_{\mu}[{\phi}_kF_k \rho_k  ({\phi}_k - \gamma {\phi}_{k+1})^{\top}]- \lim_{k\rightarrow \infty} \mathbb{E}_{\mu}[  {\phi}_k]\mathbb{E}_{\mu}[F_k \rho_k ({\phi}_k - \gamma {\phi}_{k+1})]^{\top}\\ 
&= \lim_{k \rightarrow \infty} \mathbb{E}_{\mu}[{\phi}_kF_k \rho_k ({\phi}_k - \gamma {\phi}_{k+1})^{\top}]- \lim_{k \rightarrow \infty} \mathbb{E}_{\mu}[ {\phi}_k]\lim_{k \rightarrow \infty}\mathbb{E}_{\mu}[F_k \rho_k ({\phi}_k - \gamma {\phi}_{k+1})]^{\top}\\   
&=\sum_{s} f(s) {\phi}(s)({\phi}(s) - \gamma \sum_{s'}[\textbf{P}_{\pi}]_{ss'}{\phi}(s'))^{\top} - \sum_{s} d_{\mu}(s) {\phi}(s) * \sum_{s} f(s)({\phi}(s) - \gamma \sum_{s'}[\textbf{P}_{\pi}]_{ss'}{\phi}(s'))^{\top}  \\
&={{\Phi}}^{\top} \textbf{F} (\textbf{I} - \gamma \textbf{P}_{\pi}) {\Phi} - {{\Phi}}^{\top} \textbf{d}_{\mu} \textbf{f}^{\top} (\textbf{I} - \gamma \textbf{P}_{\mu}) {\Phi}  \\
&={{\Phi}}^{\top} (\textbf{F} - \textbf{d}_{\mu} \textbf{f}^{\top}) (\textbf{I} - \gamma \textbf{P}_{\pi}){{\Phi}} \\
&={{\Phi}}^{\top} (\textbf{F} (\textbf{I} - \gamma \textbf{P}_{\pi})-\textbf{d}_{\mu} \textbf{f}^{\top} (\textbf{I} - \gamma \textbf{P}_{\pi})){{\Phi}} \\
&={{\Phi}}^{\top} (\textbf{F} (\textbf{I} - \gamma \textbf{P}_{\pi})-\textbf{d}_{\mu} \textbf{d}_{\mu}^{\top} ){{\Phi}} \\
    \end{split}
\end{equation}
\begin{equation}
    \begin{split}
 {b}_{\text{VMETD}}&=\lim_{k \rightarrow \infty} \mathbb{E}[{b}_{\text{VMETD},k}]\\
&= \lim_{k \rightarrow \infty} \mathbb{E}_{\mu}[F_k\rho_kR_{k+1}{\phi}_k]- \lim_{k\rightarrow \infty} \mathbb{E}_{\mu}[{\phi}_k]\mathbb{E}_{\mu}[F_k\rho_kR_{k+1}]\\  
&= \lim_{k \rightarrow \infty} \mathbb{E}_{\mu}[{\phi}_kF_k\rho_kR_{k+1}]- \lim_{k\rightarrow \infty} \mathbb{E}_{\mu}[  {\phi}_k]\mathbb{E}_{\mu}[{\phi}_k]\mathbb{E}_{\mu}[F_k\rho_kR_{k+1}]\\ 
&= \lim_{k \rightarrow \infty} \mathbb{E}_{\mu}[{\phi}_kF_k\rho_kR_{k+1}]- \lim_{k \rightarrow \infty} \mathbb{E}_{\mu}[ {\phi}_k]\lim_{k \rightarrow \infty}\mathbb{E}_{\mu}[F_k\rho_kR_{k+1}]\\  
&=\sum_{s} f(s) {\phi}(s)r_{\pi} - \sum_{s} d_{\mu}(s) {\phi}(s) * \sum_{s} f(s)r_{\pi}  \\
&={{\Phi}}^{\top}(\textbf{F}-\textbf{d}_{\mu} \textbf{f}^{\top})\textbf{r}_{\pi} \\
    \end{split}
\end{equation}
 Let $\vec{h}({\theta}(t))$ be the driving vector field of the ODE
 (\ref{odethetavmetd}).
\begin{equation*}
 \vec{h}({\theta}(t))=-\textbf{A}_{\text{VMETD}}{\theta}(t)+{b}_{\text{VMETD}}.
\end{equation*}

 An ${\Phi}^{\top}{\text{X}}{\Phi}$ matrix of this
 form will be positive definite whenever the matrix ${\text{X}}$ is positive definite.
 Any matrix ${\text{X}}$ is positive definite if and only if
 the symmetric matrix ${\text{S}}={\text{X}}+{\text{X}}^{\top}$ is positive definite. 
 Any symmetric real matrix ${\text{S}}$ is positive definite if the absolute values of
 its diagonal entries are greater than the sum of the absolute values of the corresponding
 off-diagonal entries\cite{sutton2016emphatic}. 

\begin{equation}
    \label{rowsum}
    \begin{split}
 (\textbf{F} (\textbf{I} - \gamma \textbf{P}_{\pi})-\textbf{d}_{\mu} \textbf{d}_{\mu}^{\top} )\textbf{1}
    &=\textbf{F} (\textbf{I} - \gamma \textbf{P}_{\pi})\textbf{1}-\textbf{d}_{\mu} \textbf{d}_{\mu}^{\top} \textbf{1}\\
    &=\textbf{F}(\textbf{1}-\gamma \textbf{P}_{\pi} \textbf{1})-\textbf{d}_{\mu} \textbf{d}_{\mu}^{\top} \textbf{1}\\
    &=(1-\gamma)\textbf{F}\textbf{1}-\textbf{d}_{\mu} \textbf{d}_{\mu}^{\top} \textbf{1}\\
    &=(1-\gamma)\textbf{f}-\textbf{d}_{\mu} \textbf{d}_{\mu}^{\top} \textbf{1}\\
    &=(1-\gamma)\textbf{f}-\textbf{d}_{\mu} \\
    &=(1-\gamma)(\textbf{I}-\gamma\textbf{P}_{\pi}^{\top})^{-1}\textbf{d}_{\mu}-\textbf{d}_{\mu} \\
    &=(1-\gamma)[(\textbf{I}-\gamma\textbf{P}_{\pi}^{\top})^{-1}-\textbf{I}]\textbf{d}_{\mu} \\
    &=(1-\gamma)[\sum_{t=0}^{\infty}(\gamma\textbf{P}_{\pi}^{\top})^{t}-\textbf{I}]\textbf{d}_{\mu} \\
    &=(1-\gamma)[\sum_{t=1}^{\infty}(\gamma\textbf{P}_{\pi}^{\top})^{t}]\textbf{d}_{\mu} > 0 \\
    \end{split}
    \end{equation}
\begin{equation}
    \label{columnsum}
    \begin{split}
 \textbf{1}^{\top}(\textbf{F} (\textbf{I} - \gamma \textbf{P}_{\pi})-\textbf{d}_{\mu} \textbf{d}_{\mu}^{\top} )
    &=\textbf{1}^{\top}\textbf{F} (\textbf{I} - \gamma \textbf{P}_{\pi})-\textbf{1}^{\top}\textbf{d}_{\mu} \textbf{d}_{\mu}^{\top} \\
    &=\textbf{d}_{\mu}^{\top}-\textbf{1}^{\top}\textbf{d}_{\mu} \textbf{d}_{\mu}^{\top} \\
    &=\textbf{d}_{\mu}^{\top}- \textbf{d}_{\mu}^{\top} \\
    &=0
    \end{split}
\end{equation}
 (\ref{rowsum}) and (\ref{columnsum}) show that the matrix $\textbf{F} (\textbf{I} - \gamma \textbf{P}_{\pi})-\textbf{d}_{\mu} \textbf{d}_{\mu}^{\top}$ of
 diagonal entries are positive and its off-diagonal entries are negative. So each row sum plus the corresponding column sum is positive. 
 So $\textbf{A}_{\text{VMETD}}$ is positive definite.



 Therefore,
${\theta}^*=\textbf{A}_{\text{VMETD}}^{-1}{b}_{\text{VMETD}}$ can be seen to be the unique globally asymptotically
 stable equilibrium for ODE (\ref{odethetavmetd}).
 Let $\vec{h}_{\infty}({\theta})=\lim_{r\rightarrow
\infty}\frac{\vec{h}(r{\theta})}{r}$. Then
$\vec{h}_{\infty}({\theta})=-\textbf{A}_{\text{VMETD}}{\theta}$ is well-defined. 
 Consider now the ODE
\begin{equation}
 \dot{{\theta}}(t)=-\textbf{A}_{\text{VMETD}}{\theta}(t).
\label{odethetavmetdfinal}
\end{equation}
 The ODE (\ref{odethetavmetdfinal}) has the origin of its unique globally asymptotically stable equilibrium.
 Thus, the assumption (A1) and (A2) are verified.
    \end{proof}



\section{Experimental details}
\label{experimentaldetails}
The 2-state version of Baird's off-policy counterexample: All learning rates follow linear learning rate decay.
For TD algorithm, $\frac{\alpha_k}{\omega_k}=4$ and $\alpha_0 = 0.1$.
For TDC algorithm, $\frac{\alpha_k}{\zeta_k}=5$ and $\alpha_0 = 0.1$.
For VMTDC algorithm, $\frac{\alpha_k}{\zeta_k}=5$, $\frac{\alpha_k}{\omega_k}=4$,and $\alpha_0 = 0.1$.
For VMTD algorithm, $\frac{\alpha_k}{\omega_k}=4$ and $\alpha_0 = 0.1$.

The 2-state version of Baird's off-policy counterexample: All learning rates follow linear learning rate decay.
For TD algorithm, $\frac{\alpha_k}{\omega_k}=4$ and $\alpha_0 = 0.1$.
For TDC algorithm, $\frac{\alpha_k}{\zeta_k}=5$ and $\alpha_0 = 0.1$.For ETD algorithm, $\alpha_0 = 0.1$.
For VMTDC algorithm, $\frac{\alpha_k}{\zeta_k}=5$, $\frac{\alpha_k}{\omega_k}=4$,and $\alpha_0 = 0.1$.For VMETD algorithm, $\frac{\alpha_k}{\omega_k}=4$ and $\alpha_0 = 0.1$.
For VMTD algorithm, $\frac{\alpha_k}{\omega_k}=4$ and $\alpha_0 = 0.1$.

For all policy evaluation experiments, each experiment 
is independently run 100 times.

For the four control experiments: The learning rates for each 
algorithm in all experiments are shown in Table \ref{lrofways}.
For all control experiments, each experiment is independently run 50 times.

\begin{table*}[htb]
    \centering
    \caption{Learning rates ($lr$) of four control experiments.}
    \label{lrofways}
    \begin{tabular}{ccccc}
      \toprule
      \multicolumn{1}{c}{Algorithms ($lr$)} & Maze & Cliff walking & Mountain Car & Acrobot \\
      \midrule
      Sarsa($\alpha$) & 0.1 & 0.1 & 0.1 & 0.1 \\
      GQ($\alpha,\zeta$) & 0.1, 0.003 & 0.1, 0.004 & 0.1, 0.01 & 0.1, 0.01 \\
      EQ($\alpha$) & 0.006 & 0.005 & 0.001 & 0.0005 \\
      VMSarsa($\alpha,\beta$) & 0.1, 0.001 & 0.1, 1e-4 & 0.1, 1e-4 & 0.1, 1e-4 \\
      VMGQ($\alpha,\zeta,\beta$) & 0.1, 0.001, 0.001 & 0.1, 0.005, 1e-4 & 0.1, 5e-4, 1e-4 & 0.1, 5e-4, 1e-4 \\
      VMEQ($\alpha,\beta$) & 0.001, 0.0005 & 0.005, 0.0001 & 0.001, 0.0001 & 0.0005, 0.0001 \\
      Q-learning($\alpha$) & 0.1 & 0.1 & 0.1 & 0.1 \\
      VMQ($\alpha,\beta$) & 0.1, 0.001 & 0.1, 1e-4 & 0.1, 1e-4 & 0.1, 1e-4 \\
      \bottomrule
    \end{tabular}
  \end{table*}