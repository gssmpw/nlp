\section{Proof of Convergence of On-policy CTD}
The purpose of this section
is to establish that the  on-policy CTD algorithm converges with probability one
to the centered TD fixpoint
 under standard assumptions.
\begin{theorem}
    \label{theorem1}(Convergence of on-policy CTD).
    In the case of on-policy learning, consider the iterations (\ref{omega}) and (\ref{theta}).
    Let the step-size sequences $\alpha_k$ and $\beta_k$, $k\geq 0$ satisfy in this case $\alpha_k,\beta_k>0$, for all $k$,
    $
    \sum_{k=0}^{\infty}\alpha_k=\sum_{k=0}^{\infty}\beta_k=\infty,
    $
    $
    \sum_{k=0}^{\infty}\alpha_k^2<\infty,
    $
    $
    \sum_{k=0}^{\infty}\beta_k^2<\infty,
    $
    and  
    $
    \alpha_k = o(\beta_k).
    $
    Assume that $(\bm{\bm{\phi}}_k,r_k,\bm{\bm{\phi}}_k')$ is an i.i.d. sequence with
    uniformly bounded second moments, where $\bm{\bm{\phi}}_k$ and $\bm{\bm{\phi}}'_{k}$ are sampled from the same Markov chain.
    Let $\textbf{A} = \mathrm{Cov}(\bm{\bm{\phi}},\bm{\bm{\phi}}-\gamma\bm{\bm{\phi}}')$,
    $\bm{b}=\mathrm{Cov}(r,\bm{\bm{\phi}})$.
    Assume that matrix $\textbf{A}$ is non-singular. 
    Then the parameter vector $\bm{\bm{\theta}}_k$ converges with probability one 
    to the centered TD fixpoint $\textbf{A}^{-1}\bm{b}$ (\ref{centeredTDfixpoint}).
   \end{theorem}
\begin{proof}
\label{th1proof}   
    The proof is  based on Borkar's Theorem   for
    general stochastic approximation recursions with two time scales
    \cite{borkar1997stochastic}. 
    
    % The new TD error for the linear setting is 
    % \begin{equation*}
    % \delta_{\text{new}}=r+\gamma
    % \bm{\bm{\theta}}^{\top}\bm{\bm{\phi}}'-\bm{\bm{\theta}}^{\top}\bm{\bm{\phi}}-\mathbb{E}[\delta].
    % \end{equation*}
    The centered one-step
    linear TD solution is defined
    as: 
    \begin{equation*}
    0=\mathbb{E}[(\delta-\mathbb{E}[\delta]) \bm{\bm{\phi}}]=-\textbf{A}\bm{\bm{\theta}}+\bm{b}.
    \end{equation*}
    Thus, the on-policy CTD algorithm's solution is
    $\bm{\bm{\theta}}_{\text{CTD}}=\textbf{A}^{-1}\bm{b}$.
    
    First, note that recursion (\ref{theta}) can be rewritten as
    \begin{equation*}
    \bm{\bm{\theta}}_{k+1}\leftarrow \bm{\bm{\theta}}_k+\beta_k\xi(k),
    \end{equation*}
    where
    \begin{equation*}
    \xi(k)=\frac{\alpha_k}{\beta_k}(\delta_k-\omega_k)\bm{\bm{\phi}}_k.
    \end{equation*}
    Due to the settings of step-size schedule $\alpha_k = o(\beta_k)$,
    $\xi(k)\rightarrow 0$ almost surely as $k\rightarrow\infty$. 
    That is the increments in iteration (\ref{omega}) are uniformly larger than
    those in (\ref{theta}), thus (\ref{omega}) is the faster recursion.
    Along the faster time scale, iterations of (\ref{omega}) and (\ref{theta})
    are associated to ODEs system as follows:
    \begin{equation}
    \dot{\bm{\bm{\theta}}}(t) = 0,
    \label{thetaFast}
    \end{equation}
    \begin{equation}
    \dot{\omega}(t)=\mathbb{E}[\delta_t|\bm{\bm{\theta}}(t)]-\omega(t).
    \label{omegaFast}
    \end{equation}
    Based on the ODE (\ref{thetaFast}), $\bm{\bm{\theta}}(t)\equiv \bm{\bm{\theta}}$ when
    viewed from the faster timescale. 
    By the Hirsch lemma \cite{hirsch1989convergent}, it follows that
    $||\bm{\bm{\theta}}_k-\bm{\bm{\theta}}||\rightarrow 0$ a.s. as $k\rightarrow \infty$ for some
    $\bm{\bm{\theta}}$ that depends on the initial condition $\bm{\bm{\theta}}_0$ of recursion
    (\ref{theta}).
    Thus, the ODE pair (\ref{thetaFast})-(\ref{omegaFast}) can be written as
    \begin{equation}
    \dot{\omega}(t)=\mathbb{E}[\delta_t|\bm{\bm{\theta}}]-\omega(t).
    \label{omegaFastFinal}
    \end{equation}
    Consider the function $h(\omega)=\mathbb{E}[\delta|\bm{\bm{\theta}}]-\omega$,
    i.e., the driving vector field of the ODE (\ref{omegaFastFinal}).
    It is easy to find that the function $h$ is Lipschitz with coefficient
    $-1$.
    Let $h_{\infty}(\cdot)$ be the function defined by
     $h_{\infty}(\omega)=\lim_{x\rightarrow \infty}\frac{h(x\omega)}{x}$.
     Then  $h_{\infty}(\omega)= -\omega$,  is well-defined. 
     For (\ref{omegaFastFinal}), $\omega^*=\mathbb{E}[\delta|\bm{\bm{\theta}}]$
    is the unique globally asymptotically stable equilibrium.
     For the ODE
      \begin{equation}
     \dot{\omega}(t) = h_{\infty}(\omega(t))= -\omega(t),
     \label{omegaInfty}
     \end{equation}
     apply $\vec{V}(\omega)=(-\omega)^{\top}(-\omega)/2$ as its
    associated strict Liapunov function. Then,
    the origin of (\ref{omegaInfty}) is a globally asymptotically stable
    equilibrium.
    
    
    Consider now the recursion (\ref{omega}).
    Let
    $M_{k+1}=(\delta_k-\omega_k)
    -\mathbb{E}[(\delta_k-\omega_k)|\mathcal{F}(k)]$,
    where $\mathcal{F}(k)=\sigma(\omega_l,\bm{\bm{\theta}}_l,l\leq k;\bm{\bm{\phi}}_s,\bm{\bm{\phi}}_s',r_s,s<k)$, 
    $k\geq 1$ are the sigma fields
    generated by $\omega_0,\bm{\bm{\theta}}_0,\omega_{l+1},\bm{\bm{\theta}}_{l+1},\bm{\bm{\phi}}_l,\bm{\bm{\phi}}_l'$,
    $0\leq l<k$.
    It is easy to verify that $M_{k+1},k\geq0$ are integrable random variables that 
    satisfy $\mathbb{E}[M_{k+1}|\mathcal{F}(k)]=0$, $\forall k\geq0$.
    Because $\bm{\bm{\phi}}_k$, $r_k$, and $\bm{\bm{\phi}}_k'$   have
    uniformly bounded second moments, it can be seen that for some constant
    $c_1>0$, $\forall k\geq0$,
    \begin{equation*}
    \mathbb{E}[||M_{k+1}||^2|\mathcal{F}(k)]\leq
    c_1(1+||\omega_k||^2+||\bm{\bm{\theta}}_k||^2).
    \end{equation*}
    
    
    Now Assumptions (A1) and (A2) of \cite{borkar2000ode} are verified.
    Furthermore, Assumptions (TS) of \cite{borkar2000ode} is satisfied by our
    conditions on the step-size sequences $\alpha_k$, $\beta_k$. Thus,
    by Theorem 2.2 of \cite{borkar2000ode} we obtain that
    $||\omega_k-\omega^*||\rightarrow 0$ almost surely as $k\rightarrow \infty$.
    
    Consider now the slower time scale recursion (\ref{theta}).
    Based on the above analysis, (\ref{theta}) can be rewritten as 
    \begin{equation*}
    \bm{\bm{\theta}}_{k+1}\leftarrow
    \bm{\bm{\theta}}_{k}+\alpha_k(\delta_k-\mathbb{E}[\delta_k|\bm{\bm{\theta}}_k])\bm{\bm{\phi}}_k.
    \end{equation*}
    
    Let $\mathcal{G}(k)=\sigma(\bm{\bm{\theta}}_l,l\leq k;\bm{\bm{\phi}}_s,\bm{\bm{\phi}}_s',r_s,s<k)$, 
    $k\geq 1$ be the sigma fields
    generated by $\bm{\bm{\theta}}_0,\bm{\bm{\theta}}_{l+1},\bm{\bm{\phi}}_l,\bm{\bm{\phi}}_l'$,
    $0\leq l<k$.
    Let 
    $
    Z_{k+1} = Y_{t}-\mathbb{E}[Y_{t}|\mathcal{G}(k)],
    $
    where
    \begin{equation*}
    Y_{k}=(\delta_k-\mathbb{E}[\delta_k|\bm{\bm{\theta}}_k])\bm{\bm{\phi}}_k.
    \end{equation*}
    Consequently,
    \begin{equation*}
    \begin{array}{ccl}
    \mathbb{E}[Y_t|\mathcal{G}(k)]&=&\mathbb{E}[(\delta_k-\mathbb{E}[\delta_k|\bm{\bm{\theta}}_k])\bm{\bm{\phi}}_k|\mathcal{G}(k)]\\
    &=&\mathbb{E}[\delta_k\bm{\bm{\phi}}_k|\bm{\bm{\theta}}_k]
    -\mathbb{E}[\mathbb{E}[\delta_k|\bm{\bm{\theta}}_k]\bm{\bm{\phi}}_k]\\
    &=&\mathbb{E}[\delta_k\bm{\bm{\phi}}_k|\bm{\bm{\theta}}_k]
    -\mathbb{E}[\delta_k|\bm{\bm{\theta}}_k]\mathbb{E}[\bm{\bm{\phi}}_k]\\
    &=&\mathrm{Cov}(\delta_k|\bm{\bm{\theta}}_k,\bm{\bm{\phi}}_k),
    \end{array}
    \end{equation*}
    where $\mathrm{Cov}(\cdot,\cdot)$ is a covariance operator.
    
     Thus,
     \begin{equation*}
    \begin{array}{ccl}
    Z_{k+1}&=&(\delta_k-\mathbb{E}[\delta_k|\bm{\bm{\theta}}_k])\bm{\bm{\phi}}_k-\mathrm{Cov}(\delta_k|\bm{\bm{\theta}}_k,\bm{\bm{\phi}}_k).
    \end{array}
    \end{equation*}
    It is easy to verify that $Z_{k+1},k\geq0$ are integrable random variables that 
    satisfy $\mathbb{E}[Z_{k+1}|\mathcal{G}(k)]=0$, $\forall k\geq0$.
    Also, because $\bm{\bm{\phi}}_k$, $r_k$, and $\bm{\bm{\phi}}_k'$  have
    uniformly bounded second moments, it can be seen that for some constant
    $c_2>0$, $\forall k\geq0$,
    \begin{equation*}
    \mathbb{E}[||Z_{k+1}||^2|\mathcal{G}(k)]\leq
    c_2(1+||\bm{\bm{\theta}}_k||^2).
    \end{equation*}
    
    Consider now the following ODE associated with (\ref{theta}):
    \begin{equation}
    \begin{array}{ccl}
    \dot{\bm{\bm{\theta}}}(t)&=&\mathrm{Cov}(\delta|\bm{\bm{\theta}}(t),\bm{\bm{\phi}})\\
    &=&\mathrm{Cov}(r+(\gamma\bm{\bm{\phi}}'-\bm{\bm{\phi}})^{\top}\bm{\bm{\theta}}(t),\bm{\bm{\phi}})\\
    &=&\mathrm{Cov}(r,\bm{\bm{\phi}})-\mathrm{Cov}(\bm{\bm{\theta}}(t)^{\top}(\bm{\bm{\phi}}-\gamma\bm{\bm{\phi}}'),\bm{\bm{\phi}})\\
    &=&\mathrm{Cov}(r,\bm{\bm{\phi}})-\bm{\bm{\theta}}(t)^{\top}\mathrm{Cov}(\bm{\bm{\phi}}-\gamma\bm{\bm{\phi}}',\bm{\bm{\phi}})\\
    &=&\mathrm{Cov}(r,\bm{\bm{\phi}})-\mathrm{Cov}(\bm{\bm{\phi}}-\gamma\bm{\bm{\phi}}',\bm{\bm{\phi}})^{\top}\bm{\bm{\theta}}(t)\\
    &=&\mathrm{Cov}(r,\bm{\bm{\phi}})-\mathrm{Cov}(\bm{\bm{\phi}},\bm{\bm{\phi}}-\gamma\bm{\bm{\phi}}')\bm{\bm{\theta}}(t)\\
    &=&-\textbf{A}\bm{\bm{\theta}}(t)+\bm{b}.
    \end{array}
    \label{odetheta}
    \end{equation}
    Let $\vec{h}(\bm{\bm{\theta}}(t))$ be the driving vector field of the ODE
    (\ref{odetheta}).
    \begin{equation*}
    \vec{h}(\bm{\bm{\theta}}(t))=-\textbf{A}\bm{\bm{\theta}}(t)+\bm{b}.
    \end{equation*}
     Consider the cross-covariance matrix,
    \begin{equation}
    \begin{array}{ccl}
        \textbf{A} &=& \mathrm{Cov}(\bm{\bm{\phi}},\bm{\bm{\phi}}-\gamma\bm{\bm{\phi}}')\\
      &=&\frac{\mathrm{Cov}(\bm{\bm{\phi}},\bm{\bm{\phi}})+\mathrm{Cov}(\bm{\bm{\phi}}-\gamma\bm{\bm{\phi}}',\bm{\bm{\phi}}-\gamma\bm{\bm{\phi}}')-\mathrm{Cov}(\gamma\bm{\bm{\phi}}',\gamma\bm{\bm{\phi}}')}{2}\\
      &=&\frac{\mathrm{Cov}(\bm{\bm{\phi}},\bm{\bm{\phi}})+\mathrm{Cov}(\bm{\bm{\phi}}-\gamma\bm{\bm{\phi}}',\bm{\bm{\phi}}-\gamma\bm{\bm{\phi}}')-\gamma^2\mathrm{Cov}(\bm{\bm{\phi}}',\bm{\bm{\phi}}')}{2}\\
      &=&\frac{(1-\gamma^2)\mathrm{Cov}(\bm{\bm{\phi}},\bm{\bm{\phi}})+\mathrm{Cov}(\bm{\bm{\phi}}-\gamma\bm{\bm{\phi}}',\bm{\bm{\phi}}-\gamma\bm{\bm{\phi}}')}{2},\\
    \end{array}
    \label{covariance}
    \end{equation}
    where we eventually used $\mathrm{Cov}(\bm{\bm{\phi}}',\bm{\bm{\phi}}')=\mathrm{Cov}(\bm{\bm{\phi}},\bm{\bm{\phi}})$
    \footnote{The covariance matrix $\mathrm{Cov}(\bm{\bm{\phi}}',\bm{\bm{\phi}}')$ is equal to
    the covariance matrix $\mathrm{Cov}(\bm{\bm{\phi}},\bm{\bm{\phi}})$ if the initial state is re-reachable or
    initialized randomly in a Markov chain for on-policy update.}.
    Note that the covariance matrix $\mathrm{Cov}(\bm{\bm{\phi}},\bm{\bm{\phi}})$ and
    $\mathrm{Cov}(\bm{\bm{\phi}}-\gamma\bm{\bm{\phi}}',\bm{\bm{\phi}}-\gamma\bm{\bm{\phi}}')$ are semi-positive
    definite. Then, the matrix $\textbf{A}$ is semi-positive definite because  $\textbf{A}$ is
    linearly combined  by  two positive-weighted semi-positive definite matrice
    (\ref{covariance}).
    Furthermore, $\textbf{A}$ is nonsingular due to the assumption.
    Hence, the cross-covariance matrix $\textbf{A}$ is positive definite.
    
    Therefore,
    $\bm{\bm{\theta}}^*=\textbf{A}^{-1}\bm{b}$ can be seen to be the unique globally asymptotically
    stable equilibrium for ODE (\ref{odetheta}).
    Let $\vec{h}_{\infty}(\bm{\bm{\theta}})=\lim_{r\rightarrow
    \infty}\frac{\vec{h}(r\bm{\bm{\theta}})}{r}$. Then
    $\vec{h}_{\infty}(\bm{\bm{\theta}})=-\textbf{A}\bm{\bm{\theta}}$ is well-defined. 
    Consider now the ODE
    \begin{equation}
    \dot{\bm{\bm{\theta}}}(t)=-\textbf{A}\bm{\bm{\theta}}(t).
    \label{odethetafinal}
    \end{equation}
    The ODE (\ref{odethetafinal}) has the origin as its unique globally asymptotically stable equilibrium.
    Thus, the assumption (A1) and (A2) are verified.
\end{proof}

\section{Proof of Convergence of Off-policy CTDC}
The purpose of this section
is to establish that the  off-policy CTDC algorithm converges with probability one
to the centered TD fixpoint
 under standard assumptions.

\begin{theorem}
 \label{theorem2}(Convergence of off-policy CTDC).
 In the case of off-policy learning, consider the iterations (\ref{omegavmtdc}), (\ref{uvmtdc}) and (\ref{thetavmtdc}).
 Let the step-size sequences $\alpha_k$, $\zeta_k$ and $\beta_k$, $k\geq 0$ satisfy in this case $\alpha_k,\zeta_k,\beta_k>0$, for all $k$,
 $
 \sum_{k=0}^{\infty}\alpha_k=\sum_{k=0}^{\infty}\beta_k=\sum_{k=0}^{\infty}\zeta_k=\infty,
 $
 $
 \sum_{k=0}^{\infty}\alpha_k^2<\infty,
 $
 $
 \sum_{k=0}^{\infty}\zeta_k^2<\infty,
 $
 $
 \sum_{k=0}^{\infty}\beta_k^2<\infty,
 $
 and  
 $
 \alpha_k = o(\zeta_k),
 $
 $
 \zeta_k = o(\beta_k).
 $
 Assume that $(\bm{\bm{\phi}}_k,r_k,\bm{\bm{\phi}}_k')$ is an i.i.d. sequence with
 uniformly bounded second moments.
 Let $\textbf{A} = \mathrm{Cov}(\bm{\bm{\phi}},\bm{\bm{\phi}}-\gamma\bm{\bm{\phi}}')$,
 $\bm{b}=\mathrm{Cov}(r,\bm{\bm{\phi}})$, and $\textbf{C}=\mathbb{E}[\bm{\bm{\phi}}\bm{\bm{\phi}}^{\top}]$.
 Assume that  $\textbf{A}$ and $\textbf{C}$ are non-singular matrices. 
 Then the parameter vector $\bm{\bm{\theta}}_k$ converges with probability one 
 to the centered TD fixpoint $\textbf{A}^{-1}\bm{b}$ (\ref{centeredTDfixpoint}).
\end{theorem}
\begin{proof}
    The proof is  based on multi-time-scale stochastic approximation that is    
    similar to that given by \cite{sutton2009fast} for TDC. 
    
    For the off-policy CTDC algorithm, the centered  one-step linear TD solution is defined as:
    % \begin{equation*}
    %     0=\mathbb{E}[({\bm{\phi}} - \gamma {\bm{\phi}}' - \mathbb{E}[{\bm{\phi}} - \gamma {\bm{\phi}}']){\bm{\phi}}^\top]\mathbb{E}[{\bm{\phi}} {\bm{\phi}}^{\top}]^{-1}\mathbb{E}[(\delta -\mathbb{E}[\delta]){\bm{\phi}}]=\textbf{A}^{\top}\textbf{C}^{-1}(-\textbf{A}{\bm{\theta}}+{b}).
    % \end{equation*}
    \begin{equation*}
        \begin{array}{ccl}
            0 &=& \mathbb{E}[({\bm{\phi}} - \gamma {\bm{\phi}}' - \mathbb{E}[{\bm{\phi}} - \gamma {\bm{\phi}}']){\bm{\phi}}^\top]\mathbb{E}[{\bm{\phi}} {\bm{\phi}}^{\top}]^{-1}\\
            &&\mathbb{E}[(\delta -\mathbb{E}[\delta]){\bm{\phi}}]\\
          &=&\textbf{A}^{\top}\textbf{C}^{-1}(-\textbf{A}{\bm{\theta}}+{\bm{b}}).
         \end{array}
    \end{equation*}
    The matrix $\textbf{A}^{\top}\textbf{C}^{-1}\textbf{A}$ is positive definite. Thus,   the off-policy CTDC  algorithm's solution is
    ${\bm{\theta}}_{\text{CTDC}}=\textbf{A}^{-1}{b}$.
    
    First, note that recursion (\ref{thetavmtdc}) and (\ref{uvmtdc}) can be rewritten as, respectively, 
    \begin{equation*}
     {\bm{\theta}}_{k+1}\leftarrow {\bm{\theta}}_k+\zeta_k {x}(k),
    \end{equation*}
    \begin{equation*}
     {u}_{k+1}\leftarrow {u}_k+\beta_k {y}(k),
    \end{equation*}
    where 
    \begin{equation*}
     {x}(k)=\frac{\alpha_k}{\zeta_k}[(\delta_{k}- \omega_k) {\bm{\phi}}_k - \gamma{\bm{\phi}}'_{k}({\bm{\phi}}^{\top}_k {u}_k)],
    \end{equation*}
    \begin{equation*}
     {y}(k)=\frac{\zeta_k}{\beta_k}[\delta_{k}-\omega_k - {\bm{\phi}}^{\top}_k {u}_k]{\bm{\phi}}_k.
    \end{equation*}
    
    Recursion (\ref{thetavmtdc}) can also be rewritten as
    \begin{equation*}
     {\bm{\theta}}_{k+1}\leftarrow {\bm{\theta}}_k+\beta_k z(k),
    \end{equation*}
    where
    \begin{equation*}
     z(k)=\frac{\alpha_k}{\beta_k}[(\delta_{k}- \omega_k) {\bm{\phi}}_k - \gamma{\bm{\phi}}'_{k}({\bm{\phi}}^{\top}_k {u}_k)].
    \end{equation*}
    
    Due to the settings of the step-size schedule 
    $\alpha_k = o(\zeta_k)$, $\zeta_k = o(\beta_k)$, ${x}(k)\rightarrow 0$, ${y}(k)\rightarrow 0$, $z(k)\rightarrow 0$ almost surely as $k\rightarrow 0$.
    That is the increments in iteration (\ref{omegavmtdc}) are uniformly larger than
    those in (\ref{uvmtdc}) and  the increments in iteration (\ref{uvmtdc}) are uniformly larger than
    those in (\ref{thetavmtdc}), thus (\ref{omegavmtdc}) is the fastest recursion, (\ref{uvmtdc}) is the second fast recursion, and (\ref{thetavmtdc}) is the slower recursion.
    Along the fastest time scale, iterations of (\ref{thetavmtdc}), (\ref{uvmtdc}) and (\ref{omegavmtdc})
    are associated with the ODEs system as follows:
    \begin{equation}
     \dot{{\bm{\theta}}}(t) = 0,
        \label{thetavmtdcFastest}
    \end{equation}
    \begin{equation}
     \dot{{u}}(t) = 0,
        \label{uvmtdcFastest}
    \end{equation}
    \begin{equation}
     \dot{\omega}(t)=\mathbb{E}[\delta_t|{u}(t),{\bm{\theta}}(t)]-\omega(t).
        \label{omegavmtdcFastest}
    \end{equation}
    
    Based on the ODE (\ref{thetavmtdcFastest}) and (\ref{uvmtdcFastest}), both ${\bm{\theta}}(t)\equiv {\bm{\theta}}$
    and ${u}(t)\equiv {u}$ when viewed from the fastest timescale.
    By the Hirsch lemma \cite{hirsch1989convergent}, it follows that
    $||{\bm{\theta}}_k-{\bm{\theta}}||\rightarrow 0$ a.s. as $k\rightarrow \infty$ for some
    ${\bm{\theta}}$ that depends on the initial condition ${\bm{\theta}}_0$ of recursion
    (\ref{thetavmtdc}) and $||{u}_k-{u}||\rightarrow 0$ a.s. as $k\rightarrow \infty$ for some
    $u$ that depends on the initial condition $u_0$ of recursion
    (\ref{uvmtdc}). Thus, the ODE pair (\ref{thetavmtdcFastest})-(\ref{omegavmtdcFastest})
    can be written as 
    \begin{equation}
     \dot{\omega}(t)=\mathbb{E}[\delta_t|{u},{\bm{\theta}}]-\omega(t).
        \label{omegavmtdcFastestFinal}
    \end{equation}
    
    Consider the function $h(\omega)=\mathbb{E}[\delta|{\bm{\theta}},{u}]-\omega$,
    i.e., the driving vector field of the ODE (\ref{omegavmtdcFastestFinal}).
    It is easy to find that the function $h$ is Lipschitz with coefficient
    $-1$.
    Let $h_{\infty}(\cdot)$ be the function defined by
     $h_{\infty}(\omega)=\lim_{r\rightarrow \infty}\frac{h(r\omega)}{r}$.
     Then  $h_{\infty}(\omega)= -\omega$,  is well-defined. 
     For (\ref{omegavmtdcFastestFinal}), $\omega^*=\mathbb{E}[\delta|{\bm{\theta}},{u}]$
    is the unique globally asymptotically stable equilibrium.
    For the ODE
    \begin{equation}
     \dot{\omega}(t) = h_{\infty}(\omega(t))= -\omega(t),
     \label{omegavmtdcInfty}
    \end{equation}
    apply $\vec{V}(\omega)=(-\omega)^{\top}(-\omega)/2$ as its
    associated strict Liapunov function. Then,
    the origin of (\ref{omegavmtdcInfty}) is a globally asymptotically stable
    equilibrium.
    
    Consider now the recursion (\ref{omegavmtdc}).
    Let
    $M_{k+1}=(\delta_k-\omega_k)
    -\mathbb{E}[(\delta_k-\omega_k)|\mathcal{F}(k)]$,
    where $\mathcal{F}(k)=\sigma(\omega_l,{u}_l,{\bm{\theta}}_l,l\leq k;{\bm{\phi}}_s,{\bm{\phi}}_s',r_s,s<k)$, 
    $k\geq 1$ are the sigma fields
    generated by $\omega_0,u_0,{\bm{\theta}}_0,\omega_{l+1},{u}_{l+1},{\bm{\theta}}_{l+1},{\bm{\phi}}_l,{\bm{\phi}}_l'$,
    $0\leq l<k$.
    It is easy to verify that $M_{k+1},k\geq0$ are integrable random variables that 
    satisfy $\mathbb{E}[M_{k+1}|\mathcal{F}(k)]=0$, $\forall k\geq0$.
    Because ${\bm{\phi}}_k$, $r_k$, and ${\bm{\phi}}_k'$ have
    uniformly bounded second moments, it can be seen that for some constant
    $c_1>0$, $\forall k\geq0$,
    \begin{equation*}
    \mathbb{E}[||M_{k+1}||^2|\mathcal{F}(k)]\leq
    c_1(1+||\omega_k||^2+||{u}_k||^2+||{\bm{\theta}}_k||^2).
    \end{equation*}
    
    
    Now Assumptions (A1) and (A2) of \cite{borkar2000ode} are verified.
    Furthermore, Assumptions (TS) of \cite{borkar2000ode} is satisfied by our
    conditions on the step-size sequences $\alpha_k$,$\zeta_k$, $\beta_k$. Thus,
    by Theorem 2.2 of \cite{borkar2000ode} we obtain that
    $||\omega_k-\omega^*||\rightarrow 0$ almost surely as $k\rightarrow \infty$.
    
    Consider now the second time scale recursion (\ref{uvmtdc}).
    Based on the above analysis, (\ref{uvmtdc}) can be rewritten as
    % \begin{equation*}
    %     {u}_{k+1}\leftarrow u_{k}+\zeta_{k}[\delta_{k}-\mathbb{E}[\delta_k|{u}_k,{\bm{\theta}}_k] - {\bm{\phi}}^{\top} (s_k) {u}_k]{\bm{\phi}}(s_k).
    % \end{equation*}
    \begin{equation}
     \dot{{\bm{\theta}}}(t) = 0,
        \label{thetavmtdcFaster}
    \end{equation}
    \begin{equation}
     \dot{u}(t) = \mathbb{E}[(\delta_t-\mathbb{E}[\delta_t|{u}(t),{\bm{\theta}}(t)]){\bm{\phi}}_t|{\bm{\theta}}(t)] - \textbf{C}{u}(t).
        \label{uvmtdcFaster}
    \end{equation}
    The ODE (\ref{thetavmtdcFaster}) suggests that ${\bm{\theta}}(t)\equiv {\bm{\theta}}$ (i.e., a time-invariant parameter)
    when viewed from the second fast timescale.
    By the Hirsch lemma \cite{hirsch1989convergent}, it follows that
    $||{\bm{\theta}}_k-{\bm{\theta}}||\rightarrow 0$ a.s. as $k\rightarrow \infty$ for some
    ${\bm{\theta}}$ that depends on the initial condition ${\bm{\theta}}_0$ of recursion
    (\ref{thetavmtdc}). 
    
    Consider now the recursion (\ref{uvmtdc}).
    Let
    $N_{k+1}=((\delta_k-\mathbb{E}[\delta_k]) - {\bm{\phi}}_k {\bm{\phi}}^{\top}_k {u}_k) -\mathbb{E}[((\delta_k-\mathbb{E}[\delta_k]) - {\bm{\phi}}_k {\bm{\phi}}^{\top}_k {u}_k)|\mathcal{I} (k)]$,
    where $\mathcal{I}(k)=\sigma({u}_l,{\bm{\theta}}_l,l\leq k;{\bm{\phi}}_s,{\bm{\phi}}_s',r_s,s<k)$, 
    $k\geq 1$ are the sigma fields
    generated by ${u}_0,{\bm{\theta}}_0,{u}_{l+1},{\bm{\theta}}_{l+1},{\bm{\phi}}_l,{\bm{\phi}}_l'$,
    $0\leq l<k$.
    It is easy to verify that $N_{k+1},k\geq0$ are integrable random variables that 
    satisfy $\mathbb{E}[N_{k+1}|\mathcal{I}(k)]=0$, $\forall k\geq0$.
    Because ${\bm{\phi}}_k$, $r_k$, and ${\bm{\phi}}_k'$ have
    uniformly bounded second moments, it can be seen that for some constant
    $c_2>0$, $\forall k\geq0$,
    \begin{equation*}
    \mathbb{E}[||N_{k+1}||^2|\mathcal{I}(k)]\leq
    c_2(1+||{u}_k||^2+||{\bm{\theta}}_k||^2).
    \end{equation*}
    
    Because ${\bm{\theta}}(t)\equiv {\bm{\theta}}$ from (\ref{thetavmtdcFaster}), the ODE pair (\ref{thetavmtdcFaster})-(\ref{uvmtdcFaster})
    can be written as 
    \begin{equation}
     \dot{{u}}(t) = \mathbb{E}[(\delta_t-\mathbb{E}[\delta_t|{\bm{\theta}}]){\bm{\phi}}_t|{\bm{\theta}}] - \textbf{C}{u}(t).
        \label{uvmtdcFasterFinal}
    \end{equation}
    Now consider the function $h({u})=\mathbb{E}[\delta_t-\mathbb{E}[\delta_t|{\bm{\theta}}]|{\bm{\theta}}] -\textbf{C}{u}$, i.e., the
    driving vector field of the ODE (\ref{uvmtdcFasterFinal}). For (\ref{uvmtdcFasterFinal}),
    ${u}^* = \textbf{C}^{-1}\mathbb{E}[(\delta-\mathbb{E}[\delta|{\bm{\theta}}]){\bm{\phi}}|{\bm{\theta}}]$ is the unique globally asymptotically
    stable equilibrium. Let $h_{\infty}({u})=-\textbf{C}{u}$.
    For the ODE
    \begin{equation}
     \dot{{u}}(t) = h_{\infty}({u}(t))= -\textbf{C}{u}(t),
        \label{uvmtdcInfty}
    \end{equation}
    the origin of (\ref{uvmtdcInfty}) is a globally asymptotically stable
    equilibrium because $\textbf{C}$ is a positive definite matrix (because it is nonnegative definite and nonsingular).
    Now Assumptions (A1) and (A2) of \cite{borkar2000ode} are verified.
    Furthermore, Assumptions (TS) of \cite{borkar2000ode} is satisfied by our
    conditions on the step-size sequences $\alpha_k$,$\zeta_k$, $\beta_k$. Thus,
    by Theorem 2.2 of \cite{borkar2000ode} we obtain that
    $||{u}_k-{u}^*||\rightarrow 0$ almost surely as $k\rightarrow \infty$.
    
    Consider now the slower timescale recursion (\ref{thetavmtdc}). In the light of the above,
    (\ref{thetavmtdc}) can be rewritten as 
    % \begin{equation}
    %  {\bm{\theta}}_{k+1} \leftarrow {\bm{\theta}}_{k} + \alpha_k (\delta_k -\mathbb{E}[\delta_k|{\bm{\theta}}_k]) {\bm{\phi}}_k\\
    %  - \alpha_k \gamma{\bm{\phi}}'_{k}({\bm{\phi}}^{\top}_k \textbf{C}^{-1}\mathbb{E}[(\delta_k -\mathbb{E}[\delta_k|{\bm{\theta}}_k]){\bm{\phi}}|{\bm{\theta}}_k]).
    % \end{equation}
    \begin{equation*}
        \begin{array}{ccl}
            {\bm{\theta}}_{k+1} &\leftarrow& {\bm{\theta}}_{k} + \alpha_k (\delta_k -\mathbb{E}[\delta_k|{\bm{\theta}}_k]) {\bm{\phi}}_k\\
            &&- \alpha_k \gamma{\bm{\phi}}'_{k}({\bm{\phi}}^{\top}_k \textbf{C}^{-1}\mathbb{E}[(\delta_k -\mathbb{E}[\delta_k|{\bm{\theta}}_k]){\bm{\phi}}|{\bm{\theta}}_k]).
         \end{array}
    \end{equation*}
    Let $\mathcal{G}(k)=\sigma({\bm{\theta}}_l,l\leq k;{\bm{\phi}}_s,{\bm{\phi}}_s',r_s,s<k)$, 
    $k\geq 1$ be the sigma fields
    generated by ${\bm{\theta}}_0,{\bm{\theta}}_{l+1},{\bm{\phi}}_l,{\bm{\phi}}_l'$,
    $0\leq l<k$. Let
    \begin{equation*}
        \begin{array}{ccl}
     Z_{k+1}&=&\big((\delta_k -\mathbb{E}[\delta_k|{\bm{\theta}}_k]) {\bm{\phi}}_k \\
            &&- \gamma {\bm{\phi}}'_{k}{\bm{\phi}}^{\top}_k \textbf{C}^{-1}\mathbb{E}[(\delta_k -\mathbb{E}[\delta_k|{\bm{\theta}}_k]){\bm{\phi}}|{\bm{\theta}}_k]\big)\\ 
         & &-\big(\mathbb{E}[((\delta_k -\mathbb{E}[\delta_k|{\bm{\theta}}_k]) {\bm{\phi}}_k \\
        &&- \gamma {\bm{\phi}}'_{k}{\bm{\phi}}^{\top}_k \textbf{C}^{-1}\mathbb{E}[(\delta_k -\mathbb{E}[\delta_k|{\bm{\theta}}_k]){\bm{\phi}}|{\bm{\theta}}_k])|\mathcal{G}(k)]\big)\\
        &=&\big((\delta_k -\mathbb{E}[\delta_k|{\bm{\theta}}_k]) {\bm{\phi}}_k \\
    &&-\gamma {\bm{\phi}}'_{k}{\bm{\phi}}^{\top}_k \textbf{C}^{-1}\mathbb{E}[(\delta_k -\mathbb{E}[\delta_k|{\bm{\theta}}_k]){\bm{\phi}}|{\bm{\theta}}_k]\big)\\
        & &-\big(\mathbb{E}[(\delta_k -\mathbb{E}[\delta_k|{\bm{\theta}}_k]) {\bm{\phi}}_k|{\bm{\theta}}_k]\\
    &&- \gamma\mathbb{E}[{\bm{\phi}}' {\bm{\phi}}^{\top}]\textbf{C}^{-1}\mathbb{E}[(\delta_k -\mathbb{E}[\delta_k|{\bm{\theta}}_k]) {\bm{\phi}}_k|{\bm{\theta}}_k]\big).
        \end{array}
    \end{equation*}
    It is easy to see that $Z_k$, $k\geq 0$ are integrable random variables and $\mathbb{E}[Z_{k+1}|\mathcal{G}(k)]=0$, $\forall k\geq0$. Further,
    \begin{equation*}
    \mathbb{E}[||Z_{k+1}||^2|\mathcal{G}(k)]\leq
    c_3(1+||{\bm{\theta}}_k||^2), k\geq 0,
    \end{equation*}
    for some constant $c_3 \geq 0$, again because ${\bm{\phi}}_k$, $r_k$, and ${\bm{\phi}}_k'$ have
    uniformly bounded second moments, it can be seen that for some constant.
    
    Consider now the following ODE associated with (\ref{thetavmtdc}):
    \begin{equation}
        \label{thetavmtdcSlowerFinal}
        \begin{array}{ccl}
            \dot{{\bm{\theta}}}(t) &=& \mathbb{E}[({\bm{\phi}} - \gamma {\bm{\phi}}' - \mathbb{E}[{\bm{\phi}} - \gamma {\bm{\phi}}']){\bm{\phi}}^\top]\mathbb{E}[{\bm{\phi}} {\bm{\phi}}^{\top}]^{-1}\\
            &&\quad \quad \quad\mathbb{E}[(\delta -\mathbb{E}[\delta|{\bm{\theta}}(t)]) {\bm{\phi}}|{\bm{\theta}}(t)].
         \end{array}
    \end{equation}
    Let 
    \begin{equation*}
    \begin{array}{ccl}
     \vec{h}({\bm{\theta}}(t))&=&\mathbb{E}[({\bm{\phi}} - \gamma {\bm{\phi}}' - \mathbb{E}[{\bm{\phi}} - \gamma {\bm{\phi}}']){\bm{\phi}}^\top]\mathbb{E}[{\bm{\phi}} {\bm{\phi}}^{\top}]^{-1}\\
     &&\quad \quad \quad\mathbb{E}[(\delta -\mathbb{E}[\delta|{\bm{\theta}}(t)]) {\bm{\phi}}|{\bm{\theta}}(t)]\\
        &=& \textbf{A}^{\top}\textbf{C}^{-1}(-\textbf{A}{\bm{\theta}}(t)+{\bm{b}}),
    \end{array}
    \end{equation*}
    because $\mathbb{E}[(\delta -\mathbb{E}[\delta|{\bm{\theta}}(t)]) {\bm{\phi}}|{\bm{\theta}}(t)]=-\textbf{A}{\bm{\theta}}(t)+{b}$, where 
    $\textbf{A} = \mathrm{Cov}({\bm{\phi}},{\bm{\phi}}-\gamma{\bm{\phi}}')$, ${\bm{b}}=\mathrm{Cov}(r,{\bm{\phi}})$, and $\textbf{C}=\mathbb{E}[{\bm{\phi}}{\bm{\phi}}^{\top}]$.
    
    Therefore,
    ${\bm{\theta}}^*=\textbf{A}^{-1}{\bm{b}}$ can be seen to be the unique globally asymptotically
    stable equilibrium for ODE (\ref{thetavmtdcSlowerFinal}).
    Let $\vec{h}_{\infty}({\bm{\theta}})=\lim_{r\rightarrow
    \infty}\frac{\vec{h}(r{\bm{\theta}})}{r}$. Then
    $\vec{h}_{\infty}({\bm{\theta}})=-\textbf{A}^{\top}\textbf{C}^{-1}\textbf{A}{\bm{\theta}}$ is well-defined. 
    Consider now the ODE
    \begin{equation}
    \dot{{\bm{\theta}}}(t)=-\textbf{A}^{\top}\textbf{C}^{-1}\textbf{A}{\bm{\theta}}(t).
    \label{odethetavmtdcfinal}
    \end{equation}
    
    Because $\textbf{C}^{-1}$ is positive definite and $\textbf{A}$ has full rank (as it
    is nonsingular by assumption), the matrix $\textbf{A}^{\top} \textbf{C}^{-1}\textbf{A}$ is also
    positive definite. 
    The ODE (\ref{odethetavmtdcfinal}) has the origin of its unique globally asymptotically stable equilibrium.
    Thus, the assumption (A1) and (A2) are verified.
    
    The proof is given above.
    In the fastest time scale, the parameter $w$ converges to
    $\mathbb{E}[\delta|{u}_k,{\bm{\theta}}_k]$.
    In the second fast time scale,
    the parameter $u$ converges to $\textbf{C}^{-1}\mathbb{E}[(\delta-\mathbb{E}[\delta|{\bm{\theta}}_k]){\bm{\phi}}|{\bm{\theta}}_k]$.
    In the slower time scale,
    the parameter ${\bm{\theta}}$ converges to $\textbf{A}^{-1}{\bm{b}}$.
    \end{proof}

% \begin{theorem}
%     \label{theorem3}(Convergence of VMETD).
%     In the case of off-policy learning, consider the iterations (\ref{fvmetd}), (\ref{omegavmetd}), and (\ref{thetavmetd}) of VMETD.
%     Let the step-size sequences $\alpha_k$ and $\beta_k$, $k\geq 0$ satisfy in this case $\alpha_k,\beta_k>0$, for all $k$,
%     $
%     \sum_{k=0}^{\infty}\alpha_k=\sum_{k=0}^{\infty}\beta_k=\infty,
%     $
%     $
%     \sum_{k=0}^{\infty}\alpha_k^2<\infty,
%     $
%     $
%     \sum_{k=0}^{\infty}\beta_k^2<\infty,
%     $
%     and  
%     $
%     \alpha_k = o(\beta_k).
%     $
%     Assume that $(\bm{\bm{\bm{\bm{\phi}}}}_k,r_k,\bm{\bm{\bm{\bm{\phi}}}}_k')$ is an i.i.d. sequence with
%     uniformly bounded second moments, where $\bm{\bm{\bm{\bm{\phi}}}}_k$ and $\bm{\bm{\bm{\bm{\phi}}}}'_{k}$ are sampled from the same Markov chain.
%     Let $\textbf{A}_{\textbf{VMETD}} ={\bm{\Phi}}^{\top} (\textbf{F} (\textbf{I} - \gamma \textbf{P}_{\pi})-\bm{d}_{\mu}\bm{d}_{\mu}^{\top} ){\bm{\Phi}}$,
%     $\bm{b}_{\text{VMETD}}=\bm{\Phi}^{\top}(\textbf{F}-\bm{d}_{\mu} \bm{f}^{\top})\bm{r}_{\pi}$.
%     Assume that matrix $\textbf{A}$ is non-singular. 
%     Then the parameter vector $\bm{\bm{\theta}}_k$ converges with probability one 
%     to $\textbf{A}_{\textbf{VMETD}}^{-1}\bm{b}_{\textbf{VMETD}}$.
%    \end{theorem}
%    \begin{proof}
%     The proof of VMETD's convergence is  based on Borkar's Theorem   for
%     general stochastic approximation recursions with two time scales
%     \cite{borkar1997stochastic}. 
   
%     A sketch proof is given as follows.
%     In the fast time scale, the parameter $\omega$ converges to
%      $\mathbb{E}_{\mu}[F\rho\delta|\bm{\bm{\theta}}_k]$.
%    Recursion (\ref{thetavmetd}) is considered the slower timescale. 
%    If the key matrix 
%    $\textbf{A}_{\text{VMETD}}$ is positive definite, then 
%    $\bm{\bm{\theta}}$ converges.
   
%    An ${\bm{\Phi}}^{\top}{\textbf{X}}\bm{\Phi}$ matrix of this
%     form will be positive definite whenever the matrix ${\textbf{X}}$ is positive definite.
%     Any matrix ${\textbf{X}}$ is positive definite if and only if
%     the symmetric matrix ${\textbf{S}}={\textbf{X}}+{\textbf{X}}^{\top}$ is positive definite. 
%     Any symmetric real matrix ${\textbf{S}}$ is positive definite if the absolute values of
%     its diagonal entries are greater than the sum of the absolute values of the corresponding
%     off-diagonal entries\cite{sutton2016emphatic}. 
%    \begin{equation}
%     \label{rowsum}
%     \begin{split}
%     &(\textbf{F} (\textbf{I} - \gamma \textbf{P}_{\pi})-{\bm{d}}_{\mu} {\bm{d}}_{\mu}^{\top} )\bm{e}\\
%     &=\textbf{F} (\textbf{I} - \gamma \textbf{P}_{\pi})\bm{e}-{\bm{d}}_{\mu} {\bm{d}}_{\mu}^{\top} \bm{e}\\
%     % &=\textbf{F}(\textbfe-\gamma \textbf{P}_{\pi} \textbfe)-\textbf{d}_{\mu} \textbf{d}_{\mu}^{\top} \textbfe\\
%     &=(1-\gamma)\textbf{F}e-{\bm{d}}_{\mu} {\bm{d}}_{\mu}^{\top}\bm{e}\\
%     % &=(1-\gamma)\textbf{f}-\textbf{d}_{\mu} \textbf{d}_{\mu}^{\top} \textbfe\\
%     &=(1-\gamma){f}-{\bm{d}}_{\mu} \\
%     &=(1-\gamma)(\textbf{I}-\gamma\textbf{P}_{\pi}^{\top})^{-1}{\bm{d}}_{\mu}-{\bm{d}}_{\mu} \\
%     &=(1-\gamma)[(\textbf{I}-\gamma\textbf{P}_{\pi}^{\top})^{-1}-\textbf{I}]{\bm{d}}_{\mu} \\
%     &=(1-\gamma)[\sum_{t=0}^{\infty}(\gamma\textbf{P}_{\pi}^{\top})^{t}-\textbf{I}]{\bm{d}}_{\mu} \\
%     &=(1-\gamma)[\sum_{t=1}^{\infty}(\gamma\textbf{P}_{\pi}^{\top})^{t}]{\bm{d}}_{\mu} > 0, \\
%     \end{split}
%     \end{equation}
%    \begin{equation}
%     \label{columnsum}
%     \begin{split}
%     &\bm{e}^{\top}(\textbf{F} (\textbf{I} - \gamma \textbf{P}_{\pi})-\bm{d}_{\mu} {\bm{d}}_{\mu}^{\top} )\\
%     &=\bm{e}^{\top}\textbf{F} (\textbf{I} - \gamma \textbf{P}_{\pi})-\bm{e}^{\top}{\bm{d}}_{\mu} {\bm{d}}_{\mu}^{\top} \\
%     &={\bm{d}}_{\mu}^{\top}-\bm{e}^{\top}{\bm{d}}_{\mu} {\bm{d}}_{\mu}^{\top} \\
%     &={\bm{d}}_{\mu}^{\top}- {\bm{d}}_{\mu}^{\top} \\
%     &=0,
%     \end{split}
%    \end{equation}
%    where $\bm{e}$ is the all-ones vector.
%    (\ref{rowsum}) and (\ref{columnsum}) show that the matrix $\textbf{F} (\textbf{I} - \gamma \textbf{P}_{\pi})-\bm{d}_{\mu} \bm{d}_{\mu}^{\top}$ of
%     diagonal entries are positive and its off-diagonal entries are negative. So each row sum plus the corresponding column sum is positive. 
%    So $\textbf{A}_{\text{VMETD}}$ is positive definite.
%    \end{proof}
