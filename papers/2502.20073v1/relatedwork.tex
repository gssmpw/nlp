\section{Related Work}
\paragraph{LLM-Powered Multi-Agent System}

LLM-MAS enables agents to collaboratively engage in planning, discussing,  and decision-making. Collaboration is a pivotal capability in task-oriented LLM-MAS, as it not only enhances task completion efficiency \cite{zhang2024towards,tao2024github} but also enables the pursuit of complex goals beyond the reach of single agent \cite{park2023generative,hong2023metagpt}. Recent methods for improving collaboration can be broadly categorized into (a) Structural optimization (e.g., DyLAN’s \cite{liu2023dynamic} dynamic framework), (b) Role specialization (e.g., AutoGen’s \cite{wu2023autogen} personas and AgentVerse's \cite{chen2023agentverse} role assignments), and (c) Communication paradigm (e.g., MetaGPT's \cite{hong2023metagpt} message pool). Despite these advancements, the inherent complexity and diversity of multi-agent tasks make it difficult to compare methods directly, driving the emergence of standardized benchmarks that enable quantitative evaluations under unified conditions.


\paragraph{LLM-MAS Benchmark and Evaluation}

Benchmark testing in virtual environments is the primary method for evaluating multi-agent collaboration capability. As shown in Table \ref{tab:related}, existing studies establish diverse tasks and commonly use End-to-End (E2E) metrics to assess LLM-MAS collaboration capability, with some benchmarks offering environmental scalability. However, several limitations persist. A key issue is the lack of a formal collaboration definition in most benchmarks, leading to ambiguous assessments and inconsistent comparisons across different benchmarks. Furthermore, the absence of enforced collaboration mechanisms allows agents to achieve objectives independently (e.g., in CuisineWorld, where many tasks can be completed by a single agent), undermining the true assessment of collaboration. Finally, the predominant focus on outcome-based metrics such as E2E performance overlooks the critical role of process-driven dynamics. Approaches like \cite{song2024trial}, LTC \cite{wang2023ltc}, and EvoMAC \cite{hu2024evomac} suggest refining LLMs through process behaviors to enhance adaptation and collaboration, indicating that incorporating process-oriented metrics could offer more comprehensive insights.



\begin{figure*}[t]
  \includegraphics[width=1\linewidth]{pic/main.jpg}
  \caption {Part I presents the collaboration process, which are divided into initiating collaboration and responding to collaboration, along with a general example. Part II outlines the design of the Collab-Overcooked Benchmark, emphasizing its characteristics of resource isolation and asymmetric task knowledge, and provides an example of agent collaboration in task completion.
}
  \label{fig:benchmark}
\end{figure*}

% --------------------------------------------------------------------------------------------------------------------