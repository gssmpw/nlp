\documentclass[journal]{IEEEtran}
\usepackage{pifont}

% 数学、字体、符号
\usepackage{amsmath,amsfonts,amssymb}

% 算法包
\usepackage{algorithmic}
\usepackage{algorithm}
\usepackage{utfsym}
% 表格和数组
\usepackage{array}
\usepackage{multirow}
\usepackage{makecell}
\usepackage{booktabs}
\usepackage{tabulary}
\usepackage{cite}
% 图像和颜色
\usepackage{graphicx}
%\usepackage[table]{xcolor}
\usepackage{colortbl}
\usepackage{newunicodechar}

% 其他工具包
\usepackage{hyperref}  % 超链接
\usepackage{textcomp}  % 特殊文本符号
\usepackage{stfloats}  % 浮动图形支持
\usepackage{url}       % URL 支持
\usepackage{verbatim}  % 多行注释
\usepackage{enumitem}  % 自定义列表格式
\usepackage{tikz}      % 矢量图形绘制
\usepackage{calc}      % 数值计算支持
\usepackage{xspace}    % 空格处理
\usepackage[caption=false,font=normalsize,labelfont=sf,textfont=sf]{subfig}

% 段落和排版
\setlength{\textfloatsep}{5pt} % 插图与文字的间距
\setlength{\floatsep}{10pt}    % 图形与段落间距

% 定义命令和符号
\def\ours{\textbf{BTDNet}\xspace}
\newcommand{\rmnum}[1]{\romannumeral #1}
\newcommand{\Rmnum}[1]{\expandafter\@slowromancap\romannumeral #1@}

% 单词断行设置
\hyphenation{op-tical net-works semi-conduc-tor IEEE-Xplore}

\begin{document}

\title{Referring Remote Sensing Image Segmentation via Bidirectional Alignment Guided Joint Prediction}

\author{{Tianxiang Zhang, Zhaokun Wen, Bo Kong, Kecheng Liu, Yisi Zhang, Peixian Zhuang, Jiangyun Li}
\thanks{Zhaokun Wen, Bo Kong, Kecheng liu and Yisi zhang are with the School of Automation and Electrical Engineering, University of Science and Technology Beijing, Beijing 100083, China (e-mail: M202320772@xs.ustb.edu.cn, M202320748@xs.ustb.edu.cn, 18003506710@163.com, M202210579@xs.ustb.edu.cn).

Tianxiang Zhang, Peixian Zhuang and Jiangyun Li are with the Key Laboratory of Knowledge Automation for Industrial Processes, Ministry of Education, the School of Automation and Electrical Engineering, University of Science and Technology Beijing, Beijing 100083, China (e-mail: leejy@ustb.edu.cn, txzhang@ustb.edu.cn). Jiangyun Li is also with the Shunde Graduate School of University of Science and Technology Beijing, China. (Corresponding author: Jiangyun Li.)

This work was supported by the Natural Science Foundation of China under Grant 42201386, in part by Fundamental Research Funds for the Central Universities of USTB: FRF-TP-24-060A.}}


% \address[a]{School of Automation and Electrical Engineering, University of Science and Technology Beijing, Beijing 100083, China.}
% \address[b]{Key Laboratory of Knowledge Automation for Industrial Processes, Ministry of Education, Beijing 100083, China.}
% The paper headers
\markboth{IEEE TRANSACTIONS ON GEOSCIENCE AND REMOTE SENSING}%
{Shell \MakeLowercase{\textit{}}: A Sample Article Using IEEEtran.cls for IEEE Journals}

% \IEEEpubid{0000--0000/00\$00.00~\copyright~2021 IEEE}
% Remember, if you use this, you must call \IEEEpubidadjcol in the second
% column for its text to clear the IEEEpubid mark.

\maketitle

\begin{abstract}
Referring Remote Sensing Image Segmentation (RRSIS) is critical for ecological monitoring, urban planning, and disaster management, requiring precise segmentation of objects in remote sensing imagery guided by textual descriptions. This task is uniquely challenging due to the considerable vision-language gap, the high spatial resolution and broad coverage of remote sensing imagery with diverse categories and small targets, and the presence of clustered, unclear targets with blurred edges. To tackle these issues, we propose \ours, a novel framework designed to bridge the vision-language gap, enhance multi-scale feature interaction, and improve fine-grained object differentiation.
Specifically, \ours introduces: (1) the Bidirectional Spatial Correlation (BSC) for improved vision-language feature alignment, (2) the Target-Background TwinStream Decoder (T-BTD) for precise distinction between targets and non-targets, and (3) the Dual-Modal Object Learning Strategy (D-MOLS) for robust multimodal feature reconstruction. Extensive experiments on the benchmark datasets RefSegRS and RRSIS-D demonstrate that \ours achieves state-of-the-art performance. Specifically, \ours improves the overall IoU (oIoU) by 3.76 percentage points (80.57) and 1.44 percentage points (79.23) on the two datasets, respectively. Additionally, it outperforms previous methods in the mean IoU (mIoU) by 5.37 percentage points (67.95) and 1.84 percentage points (66.04), effectively addressing the core challenges of RRSIS with enhanced precision and robustness. Datasets and codes are available at \url{https://github.com/wzk913ysq/BAJP}.
% Referring Remote Sensing Image Segmentation (RRSIS) is a challenging task that requires accurate segmentation of specific objects in remote sensing imagery based on textual descriptions. Existing methods primarily adapt approaches from natural image segmentation while facing significant limitations in the vision-language distribution gap inherited in the remote sensing domain, especially in scenarios with ambiguous targets, scattered small objects, and unclear foreground-background boundaries. To address these challenges specialized in remote sensing imaging, we propose \ours, a novel framework that emphasizes bidirectional multimodal feature alignment and joint predictions of target and background regions. Specifically, we introduce the Bidirectional Spatial Correlation (BSC) to facilitate effective feature interaction between vision and language modalities during the feature extraction process, the Target-Background TwinStream Decoder (T-BTD) to utilize relational textual priors for distinguishing target and non-target entities, and the Dual-Modal Object Learning Strategy (D-MOLS) to mitigate the vision-language gap through enhanced multimodal feature reconstruction. Extensive experiments on two benchmark datasets, RefSegRS and RRSIS-D, demonstrate that \ours achieves state-of-the-art performance, effectively addressing the challenges of RRSIS tasks. Datasets and codes are available at \url{https://github.com/wzk913ysq/BAJP}.
% Referring Remote Sensing Image Segmentation (RRSIS) aims to segment specific objects in remote sensing imagery based on textual descriptions, posing unique challenges such as ambiguous targets, scattered small objects, and unclear foreground-background boundaries. To address these, we propose \ours, a novel framework emphasizing bidirectional multimodal feature alignment and joint target-background predictions. In particular, \ours includes introduces the Bidirectional Spatial Correlation Module (BSC) for effective vision-language feature interaction, the Target-Background TwinStream Decoder (T-BTD) to leverage relational textual priors, and the Dual-Modal Object Learning Strategy (D-MOLS) bridging the vision-language gap via multimodal feature reconstruction. Experiments on RefSegRS and RRSIS-D benchmarks demonstrate state-of-the-art performance, validating the effectiveness of \ours in RRSIS tasks. Code and datasets are available at \url{https://github.com/wzk913ysq/BAJP}.
\end{abstract}

\begin{IEEEkeywords}
Remote sensing, Referring image segmentation, Ambiguous target segmentation.
\end{IEEEkeywords}

\section{Introduction}

\IEEEPARstart{R}{efering} Remote Sensing Image Segmentation (RRSIS)\cite{yuan2024rrsis,liu2024rotated} has emerged as a critical research focus in remote sensing image analysis \cite{cheng2016survey}, aiming to segment instance targets for remote sensing (RS) images based on textual descriptions. Unlike traditional domain-specific remote sensing image segmentation \cite{kotaridis2021remote,dong2023distilling} that  are constrained by a limited number of semantic labels, the RRSIS task enables open-domain segmentation by utilizing free-form textual descriptions as semantic referring information. This approach provides numerous applications in areas such as text-guided environmental monitoring\cite{he2016deep}, land cover classification\cite{talukdar2020land}, precision agriculture\cite{weiss2020remote}, and urban planning\cite{yan2015urban}, where specific objects or regions need to be identified and segmented based on natural language descriptions. By providing more flexibility in interpreting and processing remote sensing data, RRSIS enhances the ability to extract detailed, context-specific information from complex RS images.
% 介绍PRSIS,但是却少了PRSIS的应用背景,能干什么???补充

The rapid advancements in multimodal representation learning \cite{guo2019deep} have significantly elevated the importance of RRSIS by effectively integrating visual and textual information. These multimodal methods enhance the accuracy and robustness of segmentation tasks, particularly in complex remote sensing environments. Such ability to align and fuse information across modalities is critical for addressing the challenges inherent in RRSIS, such as handling ambiguous targets and fine-grained object differentiation. As RRSIS relies on both image content and textual descriptions, these multimodal techniques are critical for improving the performance and applicability of RRSIS in real-world remote sensing applications. 
%要写清楚 多模态方法出现的重要性,以及与RRSIS的关联,否则没法引出下文

RRSIS tackles the challenge of identifying instance-level targets within the same category distinguished by various attributes, laying the foundation for precise context-aware target segmentation. However, several challenges differentiate RRSIS from natural image referring tasks. The first challenge is lying in the substantial gap between vision and language in remote sensing during the referring stage. Unlike natural images, which exhibit inherent correlations with natural language descriptions and enable dense relationships effectively leveraged in benchmark models (e.g., CLIP\cite{radford2021learning}) for visual understanding, remote sensing lacks models with strong vision-language alignment. Consequently, this gap leads to significant discrepancies between vision and text feature distributions, making cross-modal alignment in the RS domain substantially more challenging.

Due to the significant vision-language gap in remote sensing, two critical issues arise, severely affecting segmentation accuracy. First, the wide coverage of satellite/UAV-based remote sensing imagery leads to diverse object categories, with targets appearing at different scales. This variation, particularly for small objects, poses substantial challenges for precise localization and segmentation. The lack of robust vision-language alignment further weakens the model’s ability to distinguish between different categories, making fine-grained object recognition more difficult.
Second, the segmentation challenge is further exacerbated when objects of the same category are clustered together or when different objects are closely connected, making boundary delineation particularly difficult. Many ambiguous targets exhibit blurred edges, increasing the likelihood of mis-segmentation. Without a well-aligned vision-language model, distinguishing object boundaries based solely on textual descriptions remains difficult, further constraining segmentation accuracy in RRSIS.
% The second challenge arises from the diverse object categories caused by the high spatial resolution and wide coverage of remote sensing images, making it complicated for fine-grained small object localization and the understanding of ambiguous attributes. The unique characteristics of remote sensing images lead to the presence of numerous small objects and blurred edge features, making high-precision and fine-grained localization difficult in RRSIS tasks. Blurred boundaries and weak clustering further obscure distinctions between regions of interest and their surroundings.
% %写出来RRSIS存在的行业问题,方法上,背景上有啥问题

% The third challenge is mis-segmentation for one specific categories due to high intra-class similarity. Targets may either span extensive regions with similar features or consist of numerous discrete small areas. Distinguishing these attributes based on textual descriptions across instance regions of varying scales demands a finer granularity in text-guided visual perception. Addressing these unique characteristics of RS images is crucial for advancing RRSIS.
%写出来RRSIS存在的行业问题,方法上,背景上有啥问题


% Notably, remote sensing images present more significant challenges than conventional referring image segmentation in natural images. 首先是更大的vision-language gap,自然图像天生便与natural language description 联系更加紧密,这而这个关系反映被潜在运用于训练视觉理解的基准模型\ref{}中,而遥感图像却缺少对应的基准模型,使得vision信息与文本信息特征分布差异更大,更难实现跨模态的对齐。其次,是更高细粒度的目标与amBAJPuous的目标属性。 natural images typically exhibit clear distinctions between foreground and background, remote sensing images often have 相对blurred boundaries,且聚类特征更不明显。 Furthermore, 遥感图像中的目标,可能会是大片的相似特征区域,也可能是大量离散的小区域,要在不同尺度的instance 区域中依照文本描述对属性进行区分,更加需求高粒度的text-based perception and vision alignment。如何处理这遥感图像的特性是实现更加精确的RRSIS的关键
\begin{figure}[!t]
    \centering
    \includegraphics[width=1\linewidth]{intro.png}
    \vspace{-8mm}
    \caption{The illustration of the pipeline comparison between existing methods and our design for the RRSIS task.}
    \label{fig:intro}
\end{figure}
% Notably, Remote sensing images present greater challenges than natural images. Firstly, the vision-language gap is more pronounced, as remote sensing lacks strong vision-language alignment models (e.g., CLIP~\cite{radford2021learning}), resulting in significant disparities between vision and text features. Secondly, RS images contain finer-grained targets and more ambiguous attributes, with blurred boundaries, similar regions, or numerous small areas, making accurate RRSIS more challenging due to these granularity and cross-modal perception issues.空间分辨率,光谱分辨率,wide coverage,sensors different resolution,for particular cls,the 类内相似性,feature 导致了xxx问题   分割过程中wide 覆盖导致大小不同类别多尺度差异大导致……   在相同物体或不同物体相连时,对边界的控制,大量的目标边界模糊误分割

%第一个挑战在于在指称阶段视觉和语言之间存在巨大差距。bridging the gap;第二个挑战来自于遥感图像的高空间分辨率和广覆盖范围导致的物体类别多样化,使得细粒度小物体的定位和变得复杂。第三个挑战涉及特定类别之间的高类内相似性,在不同语义条件下存在定位挑战。
%different sensors and 不同平台,无人机导致广覆盖,物体类别多样化,存在不同scale目标,对不同尺寸的目标,特别是小目标处理存在较大困难;check

%两条:由于gap——>在处理类间出现什么什么问题;gap——>类内
A lot of significant work has been made in this domain, most existing RRSIS algorithms~\cite{yuan2024rrsis,liu2024rotated} are adaptations of referring image segmentation (RIS) methods originally designed for natural images. As illustrated in Fig. \ref{fig:intro} (a) and (b), these approaches predominantly follow two schemes. The first scheme adopts a straightforward no-correlation pipeline, starting with independent modal encoding followed by cross-modal decoding\cite{ye2019cross,wang2022cris}. This approach involves high-dimensional cross-modal feature fusion, where features across modalities are roughly aligned. The mask prediction is jointly achieved by integrating both textual and visual features. The second scheme employs a more interactive pipeline\cite{yuan2024rrsis, liu2024rotated}. Textual information is incorporated into the vision encoding process to guide feature extraction, aligning visual features gradually from low to high dimensions. Unlike the aforementioned no-correlation method, this scheme involves multiple rounds of cross-modal correlation during the encoding and enhancement of visual information, enabling the decoding phase to rely on image features for mask prediction. While these methods achieve specific results, they face significant limitations in addressing the vision-language distribution gap inherent in RS imagery. This limitation becomes especially pronounced in RS scenes with numerous similar regions, scattered small targets, and unclear distinctions between foreground and background. Based on the unique characteristics of remote sensing images, we conducted targeted research and design. Our primary focus is on minimizing the vision-language distribution gap in remote sensing images by enhancing image-text interaction and information interpretation. Additionally, we designed refined localization strategies for small and ambiguous targets to address the challenges posed by complex object categories and high intra-class similarity.%最后一句,本篇文章最重要的解决了什么问题????解决了刚才三条里面的哪个?重点解决?
%别人做了哪些工作,现有范式是什么样的,解决了什么问题,还存在什么问题,你要解决什么问题,跟下文联系起来


% Regardless of specific model designs tailored to the remote sensing domain, most existing RRSIS algorithms~\cite{yuan2024rrsis,liu2024rotated} are adaptations of RIS methods initially developed for natural images.
% As illustrated in Fig. \ref{fig:intro} (a) and (b), These approaches predominantly follow two schemes. The first scheme \cite{ye2019cross,wang2022cris}, adopts a straightforward no-correlation pipeline, starting with independent modal encoding followed by cross-modal decoding. This involves high-dimensional cross-modal feature fusion, where features across modalities are explicitly aligned. The mask prediction is jointly achieved by integrating both textual and visual features. The second scheme \cite{yuan2024rrsis, liu2024rotated}, on the other hand, employs a more interactive pipeline. Textual information is incorporated into the vision encoding process to guide feature extraction, aligning visual features unidirectionally from low to high dimensions. Unlike the no-correlation approach, this scheme involves multiple rounds of cross-modal correlation during the encoding and enhancement of visual information, enabling the decoding phase to rely solely on image features as the foundation for mask prediction.

To cope with the aforementioned challenges, we propose a novel method shown in Fig. \ref{fig:intro} (c) \ours, which performs \textbf{B}idirectional feature \textbf{A}lignment between vision and language. Furthermore, by leveraging auxiliary inputs masking key textual reference target information, this method enables \textbf{J}oint \textbf{P}redictions of foreground objects and background information, leading to more accurate RRSIS segmentation results. In particular, our proposed Bidirectional Spatial Correlation Module (\textbf{BSC}) is integrated into the bidirectional feature extractor to enable effective bidirectional feature interaction, bridging the gap between vision and language modalities. To handle the ambiguous characteristic of RS imagery, we designed the Target-Background TwinStream Decoder (\textbf{T-BTD}), using masked relational text as prior knowledge to differentiate between target and non-target entities. This strategy facilitates a comprehensive scene understanding of foreground objects and background regions by joint predictions. In addition, the Dual-Modal Object Learning Strategy (\textbf{D-MOLS}) is introduced addressing the vision-language gap prevalent in the remote sensing domain. By reconstructing critical textual information, D-MOLS enhances the alignment of features across modalities, thereby improving the understanding of language information in the RS scene. Extensive experiments conducted on two widely used benchmarks, RefSegRS~\cite{yuan2024rrsis} and RRSIS-D~\cite{liu2024rotated}, demonstrate that our proposed method, \ours, achieves state-of-the-art (SOTA) performance in the RRSIS task, particularly in high-precision and fine-grained segmentation. Qualitative comparisons through visualizations further highlight the robust performance of \ours across various remote sensing scenarios and for different segmentation targets.
%加具体的结果
%提出你的方法

% 遥感图像中的instance通常 share a similar attributes, 
% Notably, remote sensing images present more significant challenges than conventional referring image segmentation in natural images.in image-text interaction, handling small and multiple targets, and addressing fine-grained differences. For instance, while natural images typically exhibit clear distinctions between foreground and background, remote-sensing images often have blurred boundaries. Additionally, the prevalence of scattered small targets complicates the segmentation task, underscoring the need to research text-based perception and vision alignment specific to remote sensing. Moreover, remote sensing images present a significant image-text understanding gap, making it challenging to establish a robust interaction between modalities.
% % 总的来说,上述描述的问题都说明了RRSIS任务的挑战性。
% Regardless of certain model design for remote sensing domain, most existing RRSIS algorithms \cite{} are adapted from natural image RIS methods. 
% %As shown in Figure \ref{},previous方法主要遵循两种scheme,第一种方法,following a simple one-way pipeline是以图像特征为主体,在视觉编码过程中引入文本信息进行指导,从低维到高维进行单向的vision特征对齐,之后再基于得到的vision主导feature进行target预测. 
% %第二种方法,则进行独立的模态编码,再进行交互解码的pipline,相比与前一种,其包含更加显式的交叉模态匹配,会在不同模态的单独特征提取后,进行高维的跨模态特征融合,得到有文本与视觉联合预测的mask结果。
% While these methods achieve specific results, they struggle to overcome the substantial image-text information gap inherent in remote sensing images. This limitation becomes especially pronounced in remote sensing scenes with numerous similar regions, scattered small targets, and unclear distinctions between foreground and background. 
% facilitates bidirectional interaction between text and image information, resulting in multimodal visual-textual features. These features are processed through a dual-stream decoder for both foreground and background, enabling simultaneous prediction of targets and category-agnostic objects. Additionally, we apply entity erasure on the textual input, guiding multimodal features to reconstruct masked text features, enhancing the model’s semantic understanding. Notably, our proposed Bidirectional Feature Extractor achieves finer-granularity alignment between remote sensing images and language by enabling bidirectional feature fusion, aligning more closely with human cognition. To address the ambiguity in foreground-background boundaries typical of remote sensing images, we introduce the Target-Background TwinStream strategy. This dual-stream prediction approach distinguishes between targets and non-target entities, using relational text as prior knowledge to predict unknown objects and aligning with visual features for more accurate non-target recognition. Finally, we propose a Rephrase Enhancement Loss that improves the model’s ability to interpret complex remote sensing environments by enhancing text understanding through masked text reconstruction. Consequently, our model, \textbackslash{}ref\{ours\}, achieves state-of-the-art results on the RefSegRS and RRSIS-D datasets.
In summary, the main contributions of this work are as follows:  
\begin{enumerate}
    \item We proposed a novel framework named \ours for the RRSIS task, focusing on bidirectional multimodal feature interaction and fore-background target joint prediction. Our method obtains state-of-the-art results on two public remote sensing datasets.
    \item Our proposed Bidirectional Spatial Correlation Module and dual-modal object learning strategy effectively bridging the vision-language distribution gap in the remote sensing domain, enhancing the model's multimodal alignment capability.
    \item By introducing the Target-Background TwinStream decoder into the design, our method effectively addresses the challenges posed by discrete and ambiguous targets prevalent in remote sensing scenarios.
    % \item Extensive experiments on the RefSegRS and RRSIS-D datasets demonstrate the superiority of \ours over previous SOTA methods. 
\end{enumerate}
%  In summary, our main contributions are as follows: 
% \begin{enumerate}
%     \item \textbf{Bidirectional Alignment Pipeline}: A novel pipeline with bidirectional feature extraction, enabling refined alignment between image and text features for remote sensing.
%     \item \textbf{Dual-Stream Prediction Strategy}: A Target-Background TwinStream decoder distinguishes targets from complex backgrounds, improving segmentation accuracy.
%     \item \textbf{Enhanced Text Understanding with Rephrase Loss}: A Rephrase Enhancement Loss to strengthen language comprehension, enabling better adaptation to remote sensing contexts.
%     \item \textbf{State-of-the-Art Results}: Achieves SOTA performance on RefSegRS and RRSIS-D datasets, validating its effectiveness for remote sensing segmentation tasks.
% \end{enumerate}


\section{RELATED WORK}
\subsection{Referring Image Segmentation for Natural Images}
Referring Image Segmentation (RIS), an extension of visual grounding \cite{yang2019fast,deng2021transvg}, aims to segment specific regions in an image based on textual descriptions. This task requires not only precise alignment of text and image features but also the ability to accurately delineate object boundaries, handle fine-grained details, and manage object appearance variations across different scales and contexts.

In the early stages, RIS models primarily used convolutional and recurrent neural networks to extract features, which were then combined for joint prediction. Hu et al. first introduced the RIS task to address challenges in existing semantic segmentation tasks when dealing with complex referential texts\cite{hu2016segmentation}. Since then, Li \textit{et al}. and Nagaraja \textit{et al}. employed Convolutional Neural Networks (CNNs) and Recurrent Neural Networks (RNNs) for processing bimodal information and performing joint mask prediction\cite{li2018rrn,nagaraja2016modeling}. By incorporating a modal interaction structure, Liu \textit{et al}. improved cross-modal alignment\cite{liu2017recurrent}. Moreover, Margffoy \textit{et al}. introduced a dynamic multimodal network to integrate recursive visual and linguistic features\cite{margffoy2018dynamic}.

Follow-up studies emphasize the importance of effective cross-modal interactions, with self-attention mechanisms serving as a cornerstone for facilitating such interactions. For example, Ye \textit{et al}. proposed a cross-modal self-attention module to capture long-range dependencies, paired with a gated multi-level fusion module for feature integration\cite{ye2019cross}. Similarly, Hu \textit{et al}. introduced a bidirectional cross-modal attention module to enhance alignment between linguistic and visual features\cite{hu2020bi}. Moreover, Shi \textit{et al}. developed a keyword-aware network that leverages keywords to refine region relationships\cite{shi2018key}.

With the continuous evolution of deep learning architectures, Transformers have emerged as the leading approach for RIS, offering superior global modeling and cross-modal alignment. Transformer-based methods, such as CGAN\cite{luo2020cascade}, LAVT\cite{yang2022lavt}, RESTR\cite{kim2022restr}, M3Att\cite{liu2023multi}, MagNet\cite{chng2024mask}, have significantly advanced the field. Kim \textit{et al}. introduced the first convolution-free architecture, leveraging Transformers for long-range interactions\cite{kim2022restr}. Liu \textit{et al}. proposed multi-modal mutual attention and a mutual decoder for improved feature integration\cite{liu2023multi}. Luo \textit{et al}. designed a cascaded group attention network for iterative reasoning\cite{luo2020cascade}, while Chng \textit{et al}. introduced a text reconstruction task for fine-grained cross-modal alignment\cite{chng2024mask}.

In conclusion, while RIS research has progressed from traditional to Transformer-based architectures, achieving notable improvements in multimodal interaction and alignment, it continues to face challenges such as fine-grained object differentiation, managing occlusions, and accurately aligning spatially distributed objects with their corresponding textual descriptions. Additionally, context-dependent understanding and the inherent ambiguity of language remain significant hurdles.

\subsection{Referring Remote Sensing Image Segmentation}
%They established a new dataset named RefSegRS, which collects and annotates 4,420 triplets of remote sensing images-language-label for the RRSIS task. Designed a large-scale benchmark dataset called RRSIS-D, which includes 17,402 images- language-label triples for the RRSIS task. Additionally, they %
Identifying objects in remote sensing images typically requires specialized expertise. Referring Remote Sensing Image Segmentation (RRSIS) has gained attention as it helps non-expert users extract precise information based on textual descriptions. This task is closely related to visual grounding in remote sensing \cite{sun2022visual}, where textual cues are aligned with specific image regions. However, research in this area is still in its early stages\cite{yuan2024rrsis,liu2024rotated} with limited studies available. 

%They established a new dataset named RefSegRS, which collects and annotates 4,420 triplets of remote sensing images-language-label for the RRSIS task. Designed a large-scale benchmark dataset called RRSIS-D, which includes 17,402 images- language-label triples for the RRSIS task. Additionally, they %
Liu \textit{et al}. introduced the RRSIS task, specifically designed to segment targets in remote sensing images based on natural language expressions\cite{liu2024rotated}. This work references the LAVT model proposed by\cite{yang2022lavt} for the natural image referring segmentation task. To address the common occurrence of small and dispersed objects in remote sensing images, \cite{yuan2024rrsis} designed the cross-scale enhancement module that effectively utilizes language guidance to integrate deep and shallow visual features, enhancing the model's ability to discern small targets. Liu \textit{et al}. proposed a model named RMSIN that deal with the complex spatial scale and directional issues in remote sensing images through rotated convolutions\cite{liu2024rotated}.
\begin{figure}
    \centering
    \includegraphics[width=1\linewidth]{relate.png}
    \vspace{-5mm}
    \caption{A comparison of two scenarios. (a) shows clear targets with simple descriptions, demonstrating the superior performance of existing methods. (b) illustrates complex descriptions with ambiguous backgrounds, highlighting the limitations of current approaches. }
    \label{fig:relate}
    
\end{figure}

In general, as shown in Figure \ref{fig:relate}, existing methods for referring remote sensing image segmentation achieve good performance when coping with visually clear objects and brief descriptions. However, their effectiveness diminishes when faced with complex textual relationships, as their capabilities in textual feature extraction and visual-textual modality alignment are limited, leading to inferior results. Additionally, these models cannot effectively address potential confusion between targets and backgrounds in remote sensing images.
%Fig. 2. 展示了我们的xxx模型的全部结构,主要包括三部分组件:bidirectional feature extractor,Target-Background twinStream decoder, and the Rephrase module。双向特征提取器对于一组输入的图像文本对 i-e,会在分析特征的过程中进行层级化的多尺度两模态交互,经过Bidirectional Spatial Correlation Module处理后的多模态特征 V'和L0会以视觉分支和文本分支输入注意力模块提取双向的上下文信息以强化特征的对象和环境感知信息,以此得到具备前景提示的文本嵌入。同时,xxx会先对文本进行实体对象的mask得到Lm,并将其的描述属性部分经过文本编码器做为用于背景提示文本嵌入的先验信息,经过可学习层融合后,与前景对象文本嵌入一同参与视觉特征的查询和解码,从而得到预测目标对象及其周边环境背景的掩码。此外,我们的rephrase模块会利用视觉主导的多模态特征引导“被擦除”文本Lm的重建,促进模型的语义理解学习。接下来我们将针对每个部分的具体构建情况进行阐述。
\begin{figure*}[!t]
    \centering
    \includegraphics[width=1\linewidth]{methodgai.png}
    \vspace{-6mm}
    \caption{The overall framework of the proposed \ours Network includes the following components: \textbf{(a)} \textbf{Bidirectional Feature Extractor}, in which the visual and text encoders extract features from image and text inputs, respectively, with the Bidirectional Spatial Correlation Module enabling bidirectional information exchange at the feature level. \textbf{(b)} \textbf{Target-Background TwinStream}, implementing a text-aware dual-stream inference strategy for entity targets and category-agnostic areas during the mask prediction stage. \textbf{(c)} \textbf{Dual-Modal Object Learning}, designed to apply a reconstruction loss named \textit{Lre} after erasing text entity words, in which the reconstruction module leverages multimodal visual features to guide text reconstruction.}
    \vspace{-3mm}
    \label{fig:method}
\end{figure*}
\section{Method}
% The model proposed for the RRSIS task aims to generate precise object masks under natural language guidance across diverse scenes. For this multimodal task, it requires parallel feature extraction from visual and textual branches, cross-modal interaction for feature decoding, and final mask prediction in the predictor.

% In this section, we provide a comprehensive introduction to the proposed method. First, we designed a multi-stage bidirectional fusion feature encoding process, incorporating multi-receptive fields to enhance perception of different object types. Next, to achieve clear segmentation of multi-scale and ambiguous objects, we introduced a dual-stream decoding strategy that integrates foreground and background reasoning. Moreover, considering the challenges of complex remote sensing scenarios and the difficulty of entity localization in high-resolution imagery, we devised a reconstruction-based learning strategy that leverages textual information to enhance the model’s contextual understanding and association capabilities.

% We will briefly outline the model structure below, with detailed explanations of each module provided in subsequent subsections.

\subsection{Problem Formulation}

The primary goal of this study is to address the task of referring image segmentation in remote sensing (RS) data, where the objective is to generate precise masks for target objects based on natural language expressions. Formally, let $I \in \mathbb{R}^{C\times H \times W }$ represent an RS image, where $H, W, C$ denote the height, width, and number of channels, respectively. A corresponding textual description $E = {e_1, e_2, \dots, e_n}$ provides semantic guidance for the segmentation task, with $n$ being the number of tokens in the expression.

The task is to produce an output segmentation mask $O \in {0, 1}^{H \times W}$ such that each pixel $p \in I$ is assigned a binary label indicating whether it belongs to the region referred to by $E$. Given a dataset $\Omega = {(I_i, E_i, G_i)}_{i=1}^{Num}$, where $G_i \in {0, 1}^{H \times W}$ is the ground truth mask and $Num$ is the number of samples, the aim is to design a model $f$ that maps $(I, E)$ to $O$ by learning the alignment between textual and visual features. This study focuses on evaluating and enhancing the multi-scale and cross-modal understanding of RS images, particularly in handling the challenges posed by high-resolution data, ambiguous object attributes, and fine-grained target localization.

\subsection{Overview Architecture}
Fig. \ref{fig:method} presents the overall architecture of our framework, referred to as \ours. It comprises three main components: the Bidirectional Feature Extractor, the Target-Background TwinStream Decoder, and the Dual-Modal Object Learning Strategy. 

Given an image-text pair \((I, E)\) as input, the textual component first undergoes masking of key subject phrases, resulting in the modified text \(E_m\). The Bidirectional Feature Extractor performs hierarchical feature analysis and interaction on \((I, E)\). Within this component, the visual encoder and text encoder perform multi-stage feature sampling and analysis. At each stage, the Bidirectional Spatial Correlation module facilitates spatial-level feature interaction, enabling bidirectional information exchange. This process generates vision-oriented multimodal features \(V'\) and text-oriented multimodal features \(L'_n\). Meanwhile, the masked text \(E_m\) is encoded to produce the textual features \(L_m\).

In the Target-Background TwinStream decoder, \(V'\) and \(L'_n\) are integrated through a Multi-scale Context Integration module, resulting in multi-scale contextual attention features \(L^*\) and \(V^*\) for pixel-level mask prediction. Meanwhile, the referential feature \(L_m\) is embedded into learnable representations, yielding \(L^*_m\). The foreground and background indicators \(L^*\) and \(L^*_m\) are then combined with \(V^*\), enabling the model to predict masks for referred targets and category-agnostic objects.

Additionally, our Rephrase Module leverages vision-oriented multimodal features to assist in reconstructing the masked text \(E_m\), enhancing the semantic understanding of the model. The subsequent sections provide a detailed explanation of each component's construction.

\subsection{Bidirectional Feature Extractor}
Given an image-text pair \( I \in \mathbb{R}^{3 \times H \times W} \) and \( E \in \mathbb{R}^{D \times N} \), where \( E \) is the text embedding from a tokenizer, we first use the NLTK library~\cite{bird2006nltk} to mask key objects in the text description, replacing them with padding. This results in a modified language expression, \( E_m \in \mathbb{R}^{D \times N} \), which is then processed by the BERT encoder~\cite{devlin2018bert} to obtain category-agnostic textual features \( L_m \in \mathbb{R}^{D \times N} \). Simultaneously, the BERT encoder is partitioned into stages at layers 3, 6, 9, 12 to align with hierarchical features from the vision branch, enabling bidirectional interaction during feature extraction.

The complete image-text pair \((I, E)\) is jointly fed into the multi-stage feature encoder. At each stage, a \textit{Bidirectional Spatial Correlation Module} injects information from one modality into another. We perform an "Unfold" operation on the encoded image-text features to obtain local features with multiple receptive fields. These features reweight the object modality, enabling the extractor to focus on fine-grained local contextual information. Through progressive interaction, this module generates vision-oriented and text-oriented multimodal features: \(V' = \{ V'_i \mid V'_i \in \mathbb{R}^{C_i \times H_i \times W_i}, \, i = 1, 2, 3, 4 \} \) and \(L' = \{ L'_i \mid L'_i \in \mathbb{R}^{D \times N}, \, i = 1, 2, 3, 4 \}\) This multi-stage process enables progressively more profound cross-modal information fusion.

\subsubsection{Bidirectional structure}
%所输入的图像特征与文本特征在hierarchical module内进行对齐与双向特征融合。具体而言,输入视觉特征Vi与文本特征Li经由 Bidirectional Spatial Correlation Module得到融合文本信息的视觉特征Vipie与融合视觉信息的文本特征Lipie。Vipie与Vi相加后作后送入下一阶段视觉编码器内进行特征提取。Lipie则直接作为Li+1进行后续操作。
The feature extraction backbones for the vision and text branches adopt Swin Transformer~\cite{liu2021swin} and BERT~\cite{devlin2018bert}, respectively. These backbones are divided into multi-stage feature encoders to perform multi-scale and deep information extraction across four stages while enabling bidirectional cross-modal interactions at each stage.

The input image \( I \) and text \( E \) are encoded sequentially at each stage, generating visual features \( V_i \) and text features \( L_i \) as inputs for modality interaction. The Bidirectional Spatial Correlation module fuses \( V_i \) and \( L_i \), producing cross-modality enhanced features \( V_i' \) and \( L_i' \), thereby completing the $i$-th stage of feature extraction and interaction. These enriched features are then reintroduced into the respective encoders for subsequent stages.

For each stage \( i = 1, 2, 3, 4 \), the enhanced features are computed as:
\begin{equation}
\label{feature}
V_i', L_i' = \text{BSC}(V_i, L_i) + (V_i, L_i)
\end{equation}
These features are passed to the next stage's encoder to obtain \( V_{i+1} \) and \( L_{i+1} \), while the masked text \( E_m \) is processed by BERT to extract \( L_m \).
% Similarly, \( L_0' \) undergoes a residual connection with the original \( L_0 \) before being passed to the next stage of the BERT encoder for deeper semantic understanding. Iteratively refining visual and textual features enables multi-scale cross-modal alignment, achieving bidirectional fusion of information from text to image and image to text.
\subsubsection{Bidirectional Spatial Correlation}
%\begin{figure}[!t]
% \centering
% \includegraphics[width=2.8in]{cross_interaction_test2.png}
% \caption{The illustration of Bidirectional Spatial Correlation Module.}
% \label{fig:details}
% \end{figure}
%Bidirectional Spatial Correlation Module 的输入是分别来自不同stage的视觉特征Vi和对应stage的文本特征Li。我们通过设计蕴含丰富本模态上下文特征与跨模态信息的注意力结构来充分融合不同stage中的视觉特征与文本特征。具体而言,该部分的融合分两部分,分别是以视觉引导的特征融合与以文本引导的特征融合,且两部分相互对偶。此处以“视觉引导的特征融合”为例进行说明。首先,对于每个阶段i我们将输入的视觉特征V从空间维度进行裁剪以获得逐像素点的上下文信息v\^(h,w),as follows:
\begin{figure}[!t]
\centering
\includegraphics[width=2.8in]{BSC.png}
\vspace{-2mm}
\caption{The illustration of Bidirectional Spatial Correlation Module. The correlation of information between image-to-text and text-to-image will be performed based on spatial context.}
% \vspace{-1mm}
\label{fig:details}
\end{figure}
As illustrated in Fig. \ref{fig:details}, for each stage i, the BSC module takes the visual features \( V_i \) and the corresponding textual features \( L_i \) as input. We design a bidirectional spatial correlation mechanism that leverages rich intra-modal context and cross-modal interactions to integrate visual and textual features across stages effectively. Specifically, the fusion process comprises vision-oriented and text-oriented feature fusion components. The corresponding fusion procedure is detailed as follows.
For each stage \( i \), the input visual and textual unimodal features \( V_i \) and \( L_i \) are cropped along the spatial dimension to extract per-pixel and per-token contextual information, the specific process as follows:
\begin{equation}
\begin{cases}
v_i^{(h,w)} = \text{Unfold}(V_i, k) \in \mathbb{R}^{ M }, \\
l_i^{len} = \text{Unfold}(L_i, k) \in \mathbb{R}^{ M }.
\end{cases}
\label{eq:contextual_features}
\end{equation}
Here, \( k \) represents the size of the contextual receptive field window. In the visual and textual branches, the embedding dimensions are defined as \( k^2 D \) and \( k^2 C_i\), respectively. \( h \), \( w \) and \( len \) correspond to spatial indices for pixels and word tokens. The $\text{Unfold}(·)$ operation slides a \( k \times k \) window across the feature map, extracting local patches centered around each pixel or word token. This produces feature tensors, where \( v^{(h,w)} \) and \( l^{len} \) represent pixel-wise and token-wise contextual information, respectively, under the receptive field of size \( k \). We denote the set of local vectors as \( W_v = \{ v^{(h, w)} \}\in \mathbb{R}^{H_iW_i \times M} \) and \( W_l = \{ l^{len} \} \in \mathbb{R}^{D\times M}\).

We then perform an initial mapping at the local feature set $W_v$ and $W_l$ to calculate the fine-grained visual-textual bidirectional affinity weight matrices \( W_{\text{v2l}} \in \mathbb{R}^{H_iW_i \times D} \) and \( W_{\text{l2v}} \in \mathbb{R}^{D \times H_iW_i} \), which are weighted according to different receptive fields \( k \). The specific formula is described as follows:

%Where k代表上下文感受野窗口大小,C和D分别是视觉特征和文本特征的通道维度,h,w和len表示对应像素点和词汇token的空间索引。Unfold具体来说会按照k为边长组成二维矩形框对原特征图进行滑动裁剪,依次得到以每个像素点或是每个描述词汇为中心的特征张量局部patch,相应向量v\^(h,w)、l^len为k感受野下空间位置(h,w)和文本定位len处的pixel-wise上下文信息。
%我们将所有空间点的局部特征向量拼接成一个二维矩阵,之后通过 1*1卷积,沿通道维度对视觉特征V和文本特征L进行通道压缩,实现对嵌入维度Em的对齐,以适配Wv属于HW*Em和Wl属于L*Em。从而分别计算不同感受野加权后的细粒度视觉-文本双向亲和度权重矩阵W\_v2l、W\_l2v:
%其中, $\alpha_k$, $\beta_k$ 代表可学习的加权参数,用以反映不同感受野采样的重要性。·代表矩阵乘法。基于亲和度权重矩阵Wv2l和Wl2v,我们利用文本信息L对视觉特征进行重加权以实现将文本定位于视觉的匹配过程,同时,文本分支同理。最终连接门控单元将交叉对齐后的多模态特征通过残差连接返回到特征提取backbone中用以进行下一阶段更深层的提取process。公式如下:

\begin{equation}
\begin{cases}
W_{\text{v2l}} = \sum\limits_{k \in \{1, 3, 5\}} \left\{ \alpha_k (W_v \otimes L_i) \right\} , \\
W_{\text{l2v}} = \sum\limits_{k \in \{1, 2, 3\}} \left\{ \beta_k (W_l \otimes V_i^f) \right\} .
\end{cases}
\label{eq:W_affinity}
\end{equation}

Here, \( \alpha_k \) and \( \beta_k \) represent learnable weighting parameters that reflect the importance of sampling with different receptive fields. The symbol \( \otimes\) denotes matrix multiplication. \( V_i^f \in \mathbb{R}^{C_i \times W_iH_i} \) represents the flattened visual features \( V_i \), where the spatial dimensions \( H \) and \( W \) are unfolded into a single dimension. 

Based on the affinity weight matrices \( W_{\text{v2l}} \) and \( W_{\text{l2v}} \), we reweight the visual and textual features using the cross-modal information to match the domain differences between the two modalities. Finally, a gating unit concatenates the cross-aligned multimodal features, which are then returned to the feature extraction backbone through a residual connection for bidirectional correlation, yielding \( V_i' \) and \( L_i' \), which will be used in deeper feature extraction in the subsequent stage. The process is formulated as follows:
\begin{equation}
\begin{cases}
V_i' = \text{Gate}_v \left\{ W_{\text{v2l}} \otimes L_i \right\} \circledcirc V_i + V_i, \\
L_i' = \text{Gate}_l \left\{ W_{\text{l2v}} \otimes V_i^f \right\} \circledcirc L_i + L_i.
\end{cases}
\label{eq:cross_modal_fusion}
\end{equation}
Here, $\text{Gate}(\cdot)$ represents a gating unit composed of a $1 \times 1$ convolution followed by an InstanceNorm layer, which dynamically controls the importance of the features. The symbol $\otimes$ denotes matrix multiplication, while $\circledcirc
$ represents element-wise multiplication.
% We applied the same operation in the text-to-visual interaction branch to complement the previously described visual-to-text correlation. This process generates text features enhanced with visual localization, denoted as \(L_i'\), ensuring a consistent mechanism of cross-modal interaction.

\subsection{Target-Background TwinStream}
\begin{figure}[!t]
    \centering
    \includegraphics[width=2.65in]{decoder.png}
    \vspace{-2mm}
    \caption{The illustration of the Target-Background TwinStream Decoder, comprising the Multi-scale Context Integration for multi-scale feature enhancement and the Learnable Fg-Bg Predictor for distinguishing target and background regions. }
    % \vspace{-1mm}
    \label{fig:decoder}
\end{figure}

Fig. \ref{fig:decoder} illustrates the decoding process of the Target-Background TwinStream Decoder (T-BTD). T-BTD enhances multimodal features \( V' \) and \( L'_n \) via the Multi-scale Context Integration module, achieving cross-modal alignment to produce \( L^* \) and \( V^* \) for predicting target and category-agnostic masks. Masked semantic features \( L_m \) provide prior knowledge for peripheral information.

In the Learnable Fg-Bg Predictor, \( L_m \) is enriched through embedding, generating \( L_m^* \). \( L^* \) and \( L_m^* \) are matched with visual features to predict pixel-level foreground (\( O_{\text{fg}} \)) and background (\( O_{\text{bg}} \)) masks, with cross-entropy loss computed against ground truths.

% Fig. \ref{fig:decoder} illustrates the feature decoding process and pixel-level predictions of the Target-Background TwinStream Decoder (T-BTD). Our T-BTD takes \( V' \) and \( L'_n \) as inputs. Through the Multi-scale Context Integration module, multimodal features are enhanced, and cross-modal alignment is achieved via attention mechanisms, resulting in \(L^*\) and \(V^*\), which facilitate the prediction of masks for the referred target and category-agnostic objects. Additionally, the masked semantic features \( L_m \) are used as prior knowledge for peripheral information.

% In the Learnable Fg-Bg Predictor, \( L_m \) is first integrated with learnable parameters through the embedding module to enrich its representation, resulting in \( L_m^* \). This enhanced representation is then utilized during prediction, where \( L^* \) and \( L_m^* \) are matched with visual features to generate pixel-level predictions for the foreground (\( O_{\text{fg}} \)) and background (\( O_{\text{bg}} \)). Finally, the cross-entropy loss is computed against the respective ground truths.
\subsubsection{Multi-scale Context Integration Module}
The Multi-scale Context Integration (MCI) module captures the bidirectional interaction in this work, addressing the limitations of multi-scale and textual context associations in dual-modality features extracted during the feature extraction phase. To achieve robust semantic alignment across resolutions, we first reduce the channel dimension \( C_i \) of the input feature \( V'_i \) to obtain\( V_{\text{con}} \in \mathbb{R}^{D \times S} \): 
\begin{equation}
\label{deqn_ex1dda}
V_{\text{con}} = \text{Flat} \left( \text{Concat} \left( \left\{ \phi(V'_i) \right\}_{i=2}^4 \right) \right)
\end{equation}
Where $\phi(·)$ applies a $1 \times 1$ convolution to reduce the channel dimension of \( V'_i \), $\text{Concat}(·)$ integrates the features extracted from stages \( i=2, 3, 4 \) along the spatial dimension, leveraging multi-level semantic representations from deeper network layers, and $\text{Flat}(·)$ reshapes the spatial dimensions of the tensor ($H_i \times W_i$) into a single dimension \( S \). Here, \( S = \sum_{i=2}^4 H_i W_i \) represents the total spatial locations across stages.

In the MCI, Cross-Attention operations are performed between \( V_{\text{con}} \) and \( L'_n \), facilitating the alignment of textual descriptions with image features. This addresses the challenge of insufficient context in free-form descriptions of remote sensing images. The V-L interaction can be formulated as:
\begin{equation}
\label{MSAtFDFtn}
L^* = \text{Proj}(\text{FFN}(\text{CrossAtt}(L'_n, V_{\text{con}})))
\end{equation}
Here, \( L'_n \) serves as the query in the Cross-Attention mechanism, while \( V_{\text{con}} \) functions as both the key and value. FFN and Proj represent the feed-forward network and projection network, respectively.

Subsequently, the updated \( L^* \in \mathbb{R}^{D \times N} \), which integrates multi-scale prior information with visual localization, guides pixel-level textual matching in the visual branch. This L-V interaction process can be formulated as follows:
\begin{equation}
\label{MSAttn}
V^* = \text{Proj}(\text{MSAttn}(\text{CrossAtt}(V_{\text{con}}, L^*))).
\end{equation}
Here, \( V_{\text{con}} \) is used as the query, and \( L^* \) serves as both the key and value. Additionally, we employ multi-scale attention to facilitate multi-scale interactions. Finally, we obtain context-enhanced multi-scale visual features \( V^*\in \mathbb{R}^{D \times S} \). 
At this stage, \( V^* \) and \( L^* \) will be returned as the input of MCI for multiple interactions. Ultimately, this module is executed for \( T \) times.
\subsubsection{Learnable Fg-Bg Predictor}
After obtaining spatial-aware \( L^* \) and semantic-aware $V^*$, we reshape $V^*$ to \( V_{\text{attn}} = \{ V_{\text{attn}} \mid V^i_{\text{attn}} \in \mathbb{R}^{H_i \times W_i \times D}, i =  2, 3, 4 \} \). Then, \( V^2_{\text{attn}} \) is upsampled to \( H_1 \times W_1 \) via bilinear interpolation and added to the \( V'_1 \) feature to obtain \( V_{\text{pred}} \in \mathbb{R}^{H_1 \times W_1 \times D} \), which is used for mask prediction. To address ambiguity in remote sensing images, where unmentioned objects cause segmentation issues, we introduce a background prediction branch using masked textual features \( L_m \) as prior information. These features are passed through a learnable embedding and combined with learnable parameters as follows:
\begin{equation}
\label{deqn_ex241a}
L^*_m = \text{Concat}\left( \{\text{AP}_{r_i}(L_m)\}_{i=1}^R \right) +\Delta
\end{equation}
Here, \( \Delta \) denotes the learnable parameter matrix, and \( \text{AP}_{r_i} \) denotes the adaptive pooling operation with receptive field \( r_i \), performed \( R \) times for multi-scale fusion of peripheral background information. \( L^*_m \in \mathbb{R}^{D \times B} \) represents learnable textual prompts for category-agnostic objects, where \( B \) is the length of the fused background information. These tokens are concatenated with \( L^* \) to form a dual-branch structure for pixel-level target-background segmentation. Specifically, the \([cls]\) token is extracted from \( L^* \), and average pooling is applied to \( L^*_m \) to obtain global referring prototypes \( L_{\text{fg}} \) and \( L_{\text{bg}} \) for the target and background, respectively (\( L_{\text{fg}}, L_{\text{bg}} \in \mathbb{R}^D \)). These features are matched with \( V_{\text{pred}} \) to predict pixel-wise probabilities for the foreground and background regions, respectively, as follows:
\begin{equation}
\label{eq:fg_bg_projection}
\begin{cases}
O_{\text{fg}}^{i,j} = I_{\text{proj}}(V_{\text{pred}}) \circledcirc
 R_{\text{proj}}(L_{\text{fg}}), \\
O_{\text{bg}}^{i,j} = I_{\text{proj}}(V_{\text{pred}}) \circledcirc
 R_{\text{proj}}(L_{\text{bg}}).
\end{cases}
\end{equation}
The outputs \( O_{\text{fg}}, O_{\text{bg}} \in \mathbb{R}^{H_1 \times W_1} \), are interpolated to match the input image dimensions \( H \times W \) and compared with the corresponding ground truth to compute the cross-entropy loss:
\begin{equation}
\label{eq:cross_entropy_loss}
\begin{cases}
L_{\text{fg}} = -\sum_{i,j} G_{\text{fg}}^{i,j} \log(O_{\text{fg}}^{i,j}), \\
L_{\text{bg}} = -\sum_{i,j} G_{\text{bg}}^{i,j} \log(O_{\text{bg}}^{i,j}).
\end{cases}
\end{equation}
Where \( G_{\text{fg}}^{i,j} \) and \( G_{\text{bg}}^{i,j} \) represent the ground truth labels for the target and background classes at pixel \((i,j)\), respectively. The total cross-entropy loss \cite{zhang2018generalized} is computed as:
\begin{equation}
\label{eq:total_loss1}
L_{\text{ce}} = \lambda L_{\text{fg}} + (1-\lambda) L_{\text{bg}}
\end{equation}
Here, \( \lambda \) is the weight of the foreground-background prediction in the mask loss, involved in backpropagation during training.

\subsection{Dual-Modal Object Learning}
%我们提出了双模态目标学习策略以解决遥感图像语义信息难以捕捉的问题。首先我们输入上一阶段富含目标信息的视觉特征V*和被掩盖实体的语义tokens Lm,意在利用V*指导Lm内部对于对象描述部分的重建,具体而言VTL module操作如下:公式
%其中,公式描述。对Rephrase的指代对象描述Ln与L0作重建loss如下
%在这一阶段对L0停止梯度更新。最后与上一节的celoss结合,得到我们最终反向传播的loss为:
%最后,将总损失合并为该批次上的Lce、Lre的总和。 
We introduce a dual-modality target learning strategy to address the challenge of capturing semantic information in remote sensing images. Initially, the cross-modal features \( V^* \) and \( L^* \), enriched with target-specific information from the preceding stage, are fed along with the masked semantic tokens \( L_m \). This approach aims to guide the reconstruction of object descriptions within \( L_m \). Specifically, the operation of the Joint Reconstruction (JR) module is implemented using a transformer decoder as follows:
\begin{equation}
\label{eq:lang_transform}
L_{re} = \text{CrossAtt}\left(L_m, \text{CrossAtt}\left(V^*, L^*\right)\right)
\end{equation}
Here, \(\text{CrossAtt}( \cdot )\) represents the cross-modal attention\cite{vaswani2017attention}. In this process, we first use \( V^* \) as the query and \( L^* \) as the key and value for text information enrichment and alignment. Then, \( L_m \) is used as the query to retrieve the missing information in the text from the fused multimodal features.
For the rephrased object description \( L_{re} \) and the original \( L_0 \), we define the reconstruction loss as follows:
\begin{equation}
\label{eq:reconstruction_loss}
L_{\text{re}} = \frac{1}{N} \sum_{i=1}^N \| L_{re}^{(i)} - \bar{L}_0^{(i)} \|^2
\end{equation}
Where \( N \) is the number of tokens, \( L_{re}^{(l)} \) and \( L_0^{(l)} \) represent the \( l \)-th token in the rephrased and original descriptions, respectively. Here, \( \bar{L}_0 \) indicates that the gradient of \( L_0 \) is stopped during this stage.
Finally, combining the reconstruction loss \( L_{\text{re}} \) with the cross-entropy loss \( L_{\text{ce}} \) from the previous section, we derive the final loss $L_{\text{total}}$ for backpropagation as:
\begin{equation}
L_{\text{total}} = L_{\text{ce}} + \eta \cdot L_{\text{re}}
\label{eq:total_loss}
\end{equation}
Here, \(\eta\) is a hyperparameter that balances the importance of the reconstruction loss \( L_{\text{re}} \) and the cross-entropy loss \( L_{\text{ce}} \). The total loss for the batch is computed as the sum of \( L_{\text{ce}} \) and \( L_{\text{re}} \).



\begin{table*}[htbp]
    \caption{Comparison with state-of-the-art methods on the proposed RefSegRS dataset. R-101 and Swin-B represent ResNet-101 and base Swin Transformer models, respectively. The best result is bold.}
    \label{tab:comparison_refsegrs}
    \centering
    \resizebox{\textwidth}{!}{%
    \begin{tabular}{c|c|c|c|c|c|c|c|c|c|c|c|c|c|c|c|c}
        \toprule
        \multirow{2}{*}{Method} & \makecell{Visual} & \makecell{Text} & \multicolumn{2}{c|}{P@0.5} & \multicolumn{2}{c|}{P@0.6} & \multicolumn{2}{c|}{P@0.7} & \multicolumn{2}{c|}{P@0.8} & \multicolumn{2}{c|}{P@0.9} & \multicolumn{2}{c|}{oIoU} & \multicolumn{2}{c}{mIoU} \\
        \cline{4-17} % 只在第三列之后加水平线,从第4列到第17列
        & \makecell{Encoder} & \makecell{Encoder} & Val & Test & Val & Test & Val & Test & Val & Test & Val & Test & Val & Test & Val & Test \\
        \midrule
        BRINet \cite{hu2020bi} & R-101 & LSTM & 36.86 & 20.72 & 35.53 & 14.26 & 19.93 & 9.87 & 10.66 & 2.98 & 2.84 & 1.14 & 61.59 & 58.22 & 38.73 & 31.51 \\
        LSCM \cite{hui2020linguistic} & R-101 & LSTM & 56.82 & 31.54 & 41.24 & 20.41 & 21.85 & 9.51 & 12.11 & 5.29 & 2.51 & 0.84 & 62.82 & 61.27 & 40.59 & 35.54 \\
        CMPC \cite{huang2020referring} & R-101 & LSTM &  46.09&  32.36&  26.45&  14.14&  12.76&  6.55&  7.42&  1.76&  1.39&  0.22&  63.55&  55.39&  42.08&  40.63\\
        CMSA \cite{ye2019cross} & R-101 & None & 39.24 & 28.07& 38.44 & 20.25& 20.39 & 12.71& 11.79 & 5.61& 1.52 & 0.83& 65.84& 64.53& 43.62 & 41.47\\
        RRN \cite{li2018rrn} & R-101 & LSTM & 55.43 & 30.26 & 42.98 & 23.01 & 23.11 & 14.87 & 13.72 & 7.17 & 2.64 & 0.98 & 69.24 & 65.06 & 50.81 & 41.88 \\
        CMPC+ \cite{liu2021cross} & R-101 & LSTM &56.84&  49.19&  37.59&  28.31&  20.42&  15.31&  10.67&  8.12&  2.78&  2.55&  70.62&  66.53&  47.13&  43.65  \\
        CARIS \cite{liu2023caris} & Swin-B & BERT & 68.45 & 45.40 & 47.10 & 27.19 & 25.52 & 15.08 & 14.62 & 8.87 & 3.71 & 1.98 & 75.79 & 69.74 & 54.30 & 42.66 \\
        CRIS \cite{wang2022cris} & R-101 & CLIP & 53.13 & 35.77 & 36.19 & 24.11 & 24.36 & 14.36 & 11.83 & 6.38 & 2.55 & 1.21 & 72.14 & 65.87 & 53.74 & 43.26 \\
        RefSegformer \cite{wu2024towards} & Swin-B & BERT & 81.67 & 50.25 & 52.44 & 28.01 & 30.86 & 17.83 & 17.17 & 9.19 & 5.80 & 2.48 & 77.74 & 71.13 & 60.44 & 47.12 \\
        LAVT \cite{yang2022lavt} & Swin-B & BERT & 80.97 & 51.84 & 58.70 & 30.27 & 31.09 & 17.34 & 15.55 & 9.52 & 4.64 & 2.09 & 78.50 & 71.86 & 61.53 & 47.40 \\
        RIS-DMMI \cite{hu2023beyond} & Swin-B & BERT & 86.17 & 63.89 & 74.71 & 44.30 & 38.05 & 19.81 & 18.10 & 6.49 & 3.25 & 1.00 & 74.02 & 68.58 & 65.72 & 52.15 \\
        LGCE \cite{yuan2024rrsis} & Swin-B & BERT &  90.72&  73.75&  86.31&  61.14&  71.93&  39.46&  32.95&  16.02&  10.21&  5.45&  83.56&  76.81&  72.51&  59.96\\
        RMSIN \cite{liu2024rotated} & Swin-B & BERT &  93.97&  79.20&  89.33&  65.99&  74.25&  42.98&  29.70&  16.51&  7.89&  3.25&  82.41&  75.72&  73.84&  62.58\\
        \rowcolor{gray!20}\ours (Ours)& Swin-B & BERT &  \textbf{95.13} & \textbf{83.60}& \textbf{94.20} & \textbf{75.07}& \textbf{90.72} & \textbf{62.69}& \textbf{68.91} & \textbf{34.40}& \textbf{19.49} & \textbf{9.14}& \textbf{87.92} & \textbf{80.57}& \textbf{80.61} & \textbf{67.95}\\
        \bottomrule
    \end{tabular}%
    }
    \end{table*}

\begin{table*}[htbp]
    \caption{Comparison with state-of-the-art methods on the proposed RRSIS-D dataset. R-101 and Swin-B represent ResNet-101 and base Swin Transformer models, respectively. The best result is bold.}
    \label{tab:comparison}
    \centering
    \resizebox{\textwidth}{!}{%
    \begin{tabular}{c|c|c|c|c|c|c|c|c|c|c|c|c|c|c|c|c}
        \toprule
        \multirow{2}{*}{Method} & \makecell{Visual} & \makecell{Text} & \multicolumn{2}{c|}{Pr@0.5} & \multicolumn{2}{c|}{Pr@0.6} & \multicolumn{2}{c|}{Pr@0.7} & \multicolumn{2}{c|}{Pr@0.8} & \multicolumn{2}{c|}{Pr@0.9} & \multicolumn{2}{c|}{oIoU} & \multicolumn{2}{c}{mIoU} \\
        \cline{4-17} % 只在第三列之后加水平线,从第4列到第17列
        & \makecell{Encoder} & \makecell{Encoder} & Val & Test & Val & Test & Val & Test & Val & Test & Val & Test & Val & Test & Val & Test \\
        \midrule
       RRN \cite{li2018rrn} & R-101 & LSTM & 51.09 & 51.07 & 42.47 & 42.11 & 33.04 & 32.74 & 20.80 & 21.57 & 6.14 & 6.37 & 66.53 & 66.43 & 46.06 & 45.64 \\
        CMSA \cite{ye2019cross} & R-101 & None & 55.68 & 55.32 & 48.04 & 46.45 & 38.27 & 37.43 & 26.55 & 25.39 & 9.02 & 8.15 & 69.68 & 69.39 & 48.85 & 48.54 \\
        LSCM \cite{hui2020linguistic} & R-101 & LSTM & 57.12 & 56.02 & 48.04 & 46.25 & 37.87 & 37.70 & 26.37 & 25.28 & 7.93 & 8.27 & 69.28 & 69.05 & 50.36 & 49.92 \\
        CMPC \cite{huang2020referring} & R-101 & LSTM & 57.93 & 55.83 & 48.85 & 47.40 & 38.50 & 36.94 & 25.28 & 25.45 & 9.31 & 9.19 & 70.15 & 69.22 & 50.41 & 49.24 \\
        BRINet \cite{hu2020bi} & R-101 & LSTM & 58.79 & 56.90 & 49.54 & 48.77 & 39.65 & 39.12 & 28.21 & 27.03 & 9.19 & 8.73 & 70.73 & 69.88 & 51.14 & 49.65 \\
        CMPC+ \cite{liu2021cross} & R-101 & LSTM & 59.19 & 57.65 & 49.36 & 47.51 & 38.67 & 36.97 & 25.91 & 24.33 & 8.16 & 7.78 & 70.14 & 68.64 & 51.41 & 50.24 \\
         CRIS \cite{wang2022cris} & R-101 & CLIP & 56.44 & 54.84 & 47.87 & 46.77 & 39.77 & 38.06 & 29.31 & 28.15 & 11.84 & 11.52 & 70.08 & 70.46 & 50.75 & 49.69 \\
        RefSegformer \cite{wu2024towards} & Swin-B & BERT & 64.22 & 66.59 & 58.72 & 59.58 & 50.00 & 49.93 & 35.78 & 33.78 & 24.31 & 23.30 & 76.39 & 77.40 & 58.92 & 58.99 \\
        LGCE \cite{yuan2024rrsis} & Swin-B & BERT & 68.10 & 67.65 & 60.52 & 61.53 & 52.24 & 51.45 & 42.24 & 39.62 & 23.85 & 23.33 & 76.68 & 76.34 & 60.16 & 59.37 \\
        RIS-DMMI \cite{hu2023beyond} & Swin-B & BERT & 70.40 & 68.74 & 63.05 & 60.96 & 54.14 & 50.33 & 41.95 & 38.38 & 23.85 & 21.63 & 77.01 & 76.20 & 60.72 & 60.12 \\
        LAVT \cite{yang2022lavt} & Swin-B & BERT & 69.54 & 69.52 & 63.51 & 63.61 & 53.16 & 53.29 & 43.97 & 41.60 & 24.25 & 24.94 & 77.59 & 77.19 & 61.46 & 61.04 \\
        CARIS \cite{liu2023caris} & Swin-B & BERT & 71.61 & 71.50 & 64.66 & 63.52 & 54.14 & 52.92 & 42.76 & 40.94 & 23.79 & 23.90 & 77.48 & 77.17 & 62.88 & 62.17 \\

        RMSIN \cite{liu2024rotated} & Swin-B & BERT & 74.66 & 74.26 & 68.22 & 67.25 & 57.41 & 55.93 & 45.29 & 42.55 & 24.43 & 24.53 & 78.27 & 77.79 & 65.10 & 64.20 \\
        \rowcolor{gray!20}\ours (Ours) & Swin-B & BERT & \textbf{77.87} & \textbf{75.93} & \textbf{71.26} & \textbf{69.92} & \textbf{60.11} & \textbf{59.29} & \textbf{47.53} & \textbf{46.25} & \textbf{27.70} & \textbf{27.46} & \textbf{79.29} & \textbf{79.23} & \textbf{66.89} & \textbf{66.04} \\
        \bottomrule
    \end{tabular}%
    \vspace{-6mm}
    }
    
\end{table*}

\section{EXPERIMENTS}
\subsection{Metrics and Datasets}
To evaluate the performance of our \ours model on the RRSIS task, we employ three key metrics: precision at different IoU thresholds (Pr@0.5 to Pr@0.9), mean Intersection-over-Union (mIoU), and overall Intersection-over-Union (oIoU). Precision at different thresholds (Pr) captures the segmentation accuracy at varying levels of strictness, from Pr@0.5 to Pr@0.9, providing insights into the model's ability to handle different segmentation challenges. mIoU, on the other hand, treats both small and large objects equally, ensuring a balanced evaluation of the model's performance across all object sizes. In contrast, oIoU focuses more on large-scale targets by computing the global IoU across all pixels, reflecting the overall segmentation performance irrespective of category distribution. Together, these metrics comprehensively assess the model's segmentation capabilities. 

In this study, we utilize two benchmark datasets, RefSegRS \cite{yuan2024rrsis} and RRSIS-D \cite{liu2024rotated}, designed explicitly for Referring Remote Sensing Image Segmentation (RRSIS) tasks. These datasets offer diverse and challenging scenarios, providing a comprehensive benchmark for evaluating the performance of models in understanding spatially complex and linguistically nuanced segmentation tasks. We conduct experiments on these two datasets to validate the effectiveness of our proposed \ours on the RRSIS task.

\subsubsection{RefSegRS}
The RefSegRS dataset consists of 4,420 image-expression-mask triplets, which are divided into 2,172 samples for training, 413 for validation, and 1,817 for testing. Each image has a resolution of 512$\times$512 pixels. The dataset spans a variety of object categories commonly found in remote sensing imagery, including buildings, vehicles, vegetation, and water bodies, annotated with fine-grained segmentation masks corresponding to natural language expressions.

\subsubsection{RRSIS-D}
The RRSIS-D dataset is an enormous collection comprising 17,402 triplets, 12,181 samples for training, 1,740 for validation, and 3,481 for testing. Each image in this dataset is of high resolution, with a size of 800$\times$800 pixels. This dataset includes a broader range of spatial scales and object orientations, making it suitable for evaluating models that handle multi-scale and rotational challenges. The object categories in RRSIS-D cover urban infrastructure, agricultural landscapes, and natural scenes, offering a rich diversity for segmentation tasks.
\renewcommand\arraystretch{1.0}
\setlength{\tabcolsep}{1.6mm}
\begin{table}[ht]
\centering
\caption{The results on each category of RefSegRS Dataset. The best performance is bold.}
\label{tab:clsresults1}
\begin{tabular}{@{}c|c|c|c@{}}
\toprule
\textbf{Category}                & \textbf{RMSIN} & \textbf{LGCE} & \textbf{Ours} \\ \midrule
Road                             & 75.52 & 77.78 & \textbf{82.36} \\ 
Vehicle                          & 63.66 & 63.92 & \textbf{70.79} \\ 
Car                              & 63.02 & 61.70 & \textbf{69.19} \\ 
Van                              & 52.81 & 52.86 & \textbf{57.47} \\ 
Building                         & 82.52 & 85.01 & \textbf{86.67} \\ 
Truck                            & 51.71 & 59.49 & \textbf{76.16} \\ 
Trailer                          & \textbf{42.60} & 41.27 & 37.34 \\ 
Bus                              & 41.43 & 46.74 & \textbf{49.81} \\ 
Road Marking                     & 18.19 & \textbf{18.46} & 9.49 \\ 
Bikeway                          & 53.25 & 49.84 & \textbf{64.76} \\ 
Sidewalk                         & 52.16 & 59.20 & \textbf{67.66} \\ 
Low Vegetation                   & 39.27 & \textbf{51.59} & 39.86 \\ 
Impervious Surface               & 79.16 & 81.41 & \textbf{85.62} \\ \midrule
\textbf{Average}                 & 55.02& 51.64& \textbf{61.32}\\ \bottomrule
\end{tabular}
\vspace{0mm}
\end{table}
\renewcommand\arraystretch{1.0}
\setlength{\tabcolsep}{1.6mm}
\begin{table}[ht]
\centering
\vspace{-2mm}
\caption{The results on each category of RRSIS-D Dataset. The best performance is bold.}
\label{tab:clsresults2}
\begin{tabular}{@{}c|c|c|c@{}}
\toprule
\textbf{Category}                & \textbf{RMSIN} & \textbf{LGCE} & \textbf{Ours} \\ \midrule
Airport                         & 53.90          & 53.71& \textbf{56.88}\\
Golf field                      & 80.51          & 77.41& \textbf{82.63}\\
Expressway service area         & 70.58          & 69.12& \textbf{78.38}\\
Baseball field                  & \textbf{90.49} & 85.98& 90.00\\ 
Stadium                         & 88.57 & 86.31& \textbf{89.20}\\ 
Ground track field              & 93.63          & 91.66& \textbf{93.93}\\ 
Storage tank                    & 85.93          & \textbf{86.19}& 85.88\\ 
Basketball court                & \textbf{79.13} & 68.62& 78.51\\ 
Chimney                         & 85.81          & 79.66& \textbf{87.25}\\ 
Tennis court                    & 76.32          & 67.00& \textbf{78.09}\\ 
Overpass                        & 77.62          & 73.81& \textbf{79.29}\\ 
Train station                   & \textbf{62.31} & 59.40& 61.71\\ 
Ship                            & 81.13 & 77.08& \textbf{81.22}\\ 
Expressway toll station         & \textbf{84.41} & 78.36& 80.43\\ 
Dam                             & 61.32          & 59.09& \textbf{63.11}\\ 
Harbor                          & 40.92          & 30.13& \textbf{41.63}\\
Bridge                          & 67.56          & 63.23& \textbf{67.87}\\
Vehicle                         & 61.45          & 61.41& \textbf{73.41}\\ 
Windmill                        & 62.55          & 59.74& \textbf{62.75}\\ \midrule
\textbf{Average}                & 73.90          & 69.89& \textbf{75.38}\\ \bottomrule
\end{tabular}
% \vspace{-3mm}
\end{table}

\subsection{Experimental Settings}
This paper implements the proposed method using PyTorch. Following\cite{yuan2024rrsis,liu2024rotated}, we adopt the Swin Transformer Base\cite{liu2021swin}, pre-trained on ImageNet-22K\cite{he2022partimagenet}, as the visual backbone to ensure robust feature extraction, and utilize the BERT-base model\cite{devlin2018bert} from the HuggingFace Transformers library\cite{wolf2020transformers} as the text encoder, comprising 12 layers with an embedding size of 768. The maximum token length for descriptive text is set to 20 to accommodate concise language inputs. To enable hierarchical cross-modal interaction, the 12 BERT layers are grouped into four stages of three layers each, aligning with the hierarchical visual features extracted by the Swin Transformer and facilitating effective feature fusion.

The loss function is a combination of \( L_{\text{fg}} \), \( L_{\text{bg}} \), and \( L_{\text{re}} \), where the hyperparameters \(\lambda\) and \(\eta\), representing the weights of \( L_{\text{bg}} \) and \( L_{\text{re}} \), are set to 0.6 and 0.1, respectively. 
For the datasets RefSegRS\cite{yuan2024rrsis} and RRSIS-D\cite{liu2024rotated}, the input image size is set to 512×512. During training, we employ the AdamW optimizer, setting the initial learning rates for the encoder and other modules to 0.00001 and 0.0001. Following a "poly" policy with a fixed power of 0.9, these rates are annealed to zero. The model is trained for 50 epochs on four NVIDIA 3090 GPUs with a batch size of 8. During the inference stage, the prediction result is determined by the higher value of final logits obtained from the Fg-Bg Predictor.
\begin{figure*}
    \centering
    \includegraphics[width=1\linewidth]{visrefseg.png}
    % \vspace{-5mm}
    \caption{Qualitative comparisons between \ours and the previous SOTA methods on RefSegRS datasets.}
    % \vspace{-6mm}
    \label{fig:visrefseg}
\end{figure*}
\begin{figure*}
    \centering
    \includegraphics[width=1\linewidth]{visrrsisd.png}
    % \vspace{-5mm}
    \caption{Qualitative comparisons between \ours and the previous SOTA methods on RRSIS-D datasets.}
    % \vspace{-6mm}
    \label{fig:visrrsisd}
\end{figure*}
\begin{figure}
    \centering
    \includegraphics[width=0.9\linewidth]{heatmap.png}
    % \vspace{-1mm}
    \caption{Visualization results for feature representations of different spatial correlation scales. Size 1, 3, 5 represent the unfold sizes used for visualization.}
    % \vspace{-4mm}
    \label{fig:heatmap}
    % \vspace{-10pt} % 减少距离
\end{figure}

\subsection{Performance Comparison}
Our method outperforms existing state-of-the-art approaches in the remote sensing referring segmentation task. Tables \ref{tab:comparison_refsegrs} and \ref{tab:comparison} present experiments on the RefSegRS and RRSIS-D datasets, comparing \ours with CNN-based methods like RRN \cite{li2018rrn}, CMPC \cite{huang2020referring}, BRINet \cite{hu2020bi}, and Transformer-based methods like DMMI \cite{hu2023beyond}, LAVT \cite{yang2022lavt}, and CARIS \cite{liu2023caris}. These methods, primarily designed for natural images, perform significantly worse on remote sensing data. We also compare with LGCE \cite{yuan2024rrsis} and RMSIN \cite{liu2024rotated}, which have achieved partial state-of-the-art results on RRSIS. Our approach consistently outperforms them, achieving state-of-the-art performance, especially in high-precision and fine-grained segmentation, with notable improvements in Pr@0.8, Pr@0.9, and mIoU.

\subsubsection{Quantitative Evaluations on RefSegRS} 

The targets in the RefSegRS dataset are characterized by scattered distributions, large shape variations, and a high proportion of small objects. Overall, \ours achieves state-of-the-art (SOTA) performance across all metrics, significantly surpassing LGCE, the second-best method. These substantial improvements confirm the effectiveness of \ours, which consistently outperforms LGCE on both the validation and test sets of the RefSegRS dataset.  

Specifically, \ours achieves a 7.99\% improvement in mIoU compared to LGCE, increasing from 59.96\% to 67.95\%, highlighting its strong capability to handle multi-scale objects. This improvement stems from the dual-directional spatial association design with multi-receptive fields, enhancing the model’s precision and responsiveness in spatial dimensions for individual instances. Moreover, the significant gains in Pr@0.6 to Pr@0.9, with improvements of 75.07\%, 62.69\%, and 34.40\% respectively, demonstrate that \ours excels in scenarios requiring finer-grained segmentation, achieving 10\% to 20\% higher performance compared to previous methods. This advantage, attributed to our Target-Background TwinStream Decoder (T-BTD), enables the model to better distinguish ambiguous objects and category-agnostic targets with blurred boundaries.  

To further verify the effects of our design, we calculated the average IoU for different object categories shown in Table \ref{tab:clsresults1}. For fine-grained small objects such as cars (69.19\%), vans (57.47\%), buses (49.81\%), and trucks (76.16\%), \ours delivers significantly superior results, outperforming other methods by approximately 8\% on average. Additionally, for ambiguous objects like various types of roads (82.36\%), buildings (86.67\%), bikeways (64.76\%), and sidewalks (67.66\%), \ours achieves notably high mIoU scores. This confirms the model’s exceptional entity binding and localization capabilities when dealing with small or shape-variant objects. 

%建议不要光说xxx好,而是加一下具体的数字,超过了多少,平均值达到多少,单个类别精度达到多少!!!!!
More importantly, the average IoU for all categories with \ours is at least 6.3\% higher than other methods, achieving 61.32\% compared to 55.02\%, demonstrating its superior and comprehensive segmentation performance. However, for categories like Road Marking (9.49\%) and Low Vegetation (39.86\%), which are widely distributed and visually redundant, the model’s ability to cover extensive areas is somewhat limited. This limitation arises despite \ours's strong focus on individual instances in complex scenes and presents an opportunity for further improvement in future work.

% On the RefSegRS dataset, \ours outperforms the second-best method, LGCE, in all evaluation metrics. Specifically, \ours improves precision (Pr@0.5 to Pr@0.9) by 4.41 to 11.54 points and IoU (oIoU and mIoU) by 1.07 to 7.04 points. Notably, the precision on the test set for \ours reaches 94.20 at Pr@0.6, compared to LGCE's 86.31, marking a significant improvement. For mIoU, \ours surpasses LGCE by a significant margin of 8.10 points. These substantial improvements confirm the effectiveness of our \ours model, which consistently outperforms LGCE across all metrics on both the validation and test sets of the RefSegRS dataset.

% As shown in Table \ref{tab:clsresults1}, the average IoU for each category in the RefSegRS dataset is presented for \ours. For fine-grained small objects such as cars, vans, buses, and trucks, \ours demonstrates significantly outperforming results. Additionally, for ambiguous objects like various types of roads, buildings, bikeways, and sidewalks, \ours achieves a notably high mIoU. More importantly, the average IoU for all categories is at least 6.3 higher for \ours, indicating that our model achieves the best comprehensive segmentation performance.

\subsubsection{Quantitative Evaluations on RRSIS-D}
The RRSIS-D dataset encompasses a broader range of target categories, including urban, rural, port, and industrial environments, with higher spatial resolution and increased complexity. As shown in Table \ref{tab:comparison}, \ours achieves state-of-the-art performance across all evaluation metrics, demonstrating its adaptability and effectiveness in such diverse scenarios.

In terms of localization precision, \ours consistently outperforms RMSIN by approximately 3\% in Pr@0.5 to Pr@0.7, achieving 75.93\%, 69.92\%, and 59.29\%, indicating its robust ability to locate diverse objects even in complex and cluttered environments. This improvement highlights the contribution of our Bi-directional Spatial Correlation (BSC) module, which effectively handles the semantic redundancy caused by the high-resolution and rich object diversity of remote sensing imagery. For IoU metrics, \ours achieves gains of 1.44\% in oIoU (from 77.79\% to 79.23\%) and 1.84\% in mIoU (from 64.20\% to 66.04\%) on the test set, further underscoring its superior segmentation accuracy.

Table \ref{tab:clsresults2} further demonstrates the multi-scale generalization capabilities of \ours. It excels in segmenting a wide range of objects, including small objects such as vehicles (73.41\%), windmills (62.75\%), and ships (81.22\%), as well as large targets like stadiums (89.20\%), airports (56.88\%), and various fields (90.00\% and 93.93\%). Moreover, it outperforms other methods in distinguishing challenging targets, such as chimneys (from 85.81\% to 87.25\%), courts (from 76.32\% to 78.09\%) and overpasses (from 77.62\% to 79.29\%), which are often confused with background regions.
Additionally, the RRSIS-D dataset includes richer entity attributes and longer textual descriptions. Even under these challenging conditions, \ours maintains the highest average precision (form 73.90\% to 75.38\%), attributed to the Dual-Modal Object Learning Strategy (D-MOLS) that enhances text parsing through reconstruction learning.

In summary, \ours delivers robust and precise segmentation performance on RRSIS-D, good at handling diverse, high-resolution, and textually complex scenes, making it a highly effective solution for real-world remote sensing segmentation tasks.

% Our \ours outperforms the second-best method, RMSIN, across all evaluation metrics on the validation (Val) and test (Test) sets of the RRSIS-D dataset. Specifically, \ours improves precision (Pr@0.5 to Pr@0.9) by 1.67 to 3.36 points on the test set and, similarly, by 1.60 to 3.28 points on the validation set. Regarding IoU, \ours shows gains of 1.02 points in oIoU and 1.79 points in mIoU on the test set, and 1.04 points in oIoU and 1.88 points in mIoU on the validation set. These results demonstrate that \ours consistently delivers superior performance on both the test and validation sets, establishing it as a state-of-the-art method for referring remote sensing image segmentation. 

% Table \ref{tab:clsresults2} provides a detailed breakdown of the multi-scale generalization capabilities of \ours on the RRSIS-D dataset across different object categories. Notably, \ours demonstrates exceptional performance on smaller targets such as vehicles and ships, as well as larger targets like stadiums, airports, and various fields. Moreover, \ours excels in distinguishing challenging objects that are easily confused with the background, such as chimneys and sports fields. Compared to other state-of-the-art methods for the RRSIS task, our approach achieves a significantly higher average IoU across all object categories.
% \begin{figure*}
%     \centering
%     \includegraphics[width=1\linewidth]{vis.png}
%     \caption{Qualitative comparisons between \ours and the previous SOTA methods on RefSegRS and RRSIS-D datasets.}
%     \label{fig:visrefseg}
% \end{figure*}



\subsubsection{Qualitative Comparison}
As shown in Fig. \ref{fig:visrefseg} and \ref{fig:visrrsisd}, \ours demonstrates strong performance across various remote sensing scenarios and for different segmentation targets. Compared to previous state-of-the-art models, our \ours model exhibits precise semantic understanding and instance differentiation, particularly for small redundant objects and fine-grained targets in remote sensing images. For example, in the fourth and seventh columns, \ours can accurately identify vehicles with varying shapes and colors, effectively avoiding interference from similar categories. Furthermore, our model performs better in segmenting large objects and ambiguous targets, thanks to our \textbf{T-BTD} design, which allows the model to achieve superior localization for targets with low foreground-background distinction. 

Moreover, our model shows strong text-image alignment capabilities for objects with shallow semantic descriptions. As illustrated in Fig. \ref{fig:visrefseg}, the query \textit{'\textit{Van in the parking area}'} and its corresponding target, the lack of sufficient textual context often leads to misidentification in other models. In contrast, our model can effectively extract and utilize freely input textual information in various forms, highlighting the superiority of \ours in dual-modal analysis and alignment.

Furthermore, Fig. \ref{fig:heatmap} highlights the adaptability of our Bidirectional Spatial Correlation (BSC) module to multi-scale targets under varying receptive field scales. Large-scale targets like track fields achieve optimal feature representations at a spatial scale of 5, medium-sized targets such as bridges at a scale of 3, and small-scale targets like windmills at a scale of 1.

This demonstrates that our BSC module effectively captures hierarchical spatial correlations by leveraging multi-scale receptive fields, enabling precise alignment of image and text features across diverse target sizes. This adaptability is particularly valuable in remote sensing scenarios, where objects of varying scales and complex spatial distributions demand fine-grained segmentation and robust text-to-visual matching.

\subsection{Ablation Study}
We conducted a series of ablation experiments on the test subset of the RefSegRS dataset to verify the effectiveness of our method's core components. 
\begin{figure}
    \centering
    \includegraphics[width=1\linewidth]{fusion.png}
    \vspace{-5mm}
    \caption{The comparisons of different cross-modal fusion mechanisms}
    \label{fig:fusion}
\end{figure}
% \renewcommand\arraystretch{1.05}
% \setlength{\tabcolsep}{1.6mm}
% \begin{table}[ht]
% \centering
% \caption{Ablation studies on attention mechanisms.}
%     \begin{tabular}{@{}c|c|c|c|c|c@{}}
%     \toprule
%     \textbf{Attention} & \textbf{Pr@0.5} & \textbf{Pr@0.7} & \textbf{Pr@0.9} & \textbf{oIoU} & \textbf{mIoU} \\ \midrule
%     PWAM~\cite{yang2022lavt}     & 79.31 & 50.14 & 6.55 & 78.11& 63.97         \\
%     WPA~\cite{zhang2022coupalign}      & 80.68& 51.68& 5.94& 78.53& 64.57\\
%     IIM~\cite{liu2024rotated}      & 80.21& 43.68& 4.31& 77.89& 64.62\\
%     \rowcolor{gray!20}BSC (ours)        & \textbf{83.60}&\textbf{ 62.69}& \textbf{9.14}& \textbf{80.57}& \textbf{67.95} \\ \bottomrule
%     \end{tabular}
%     \vspace{-1mm}
%     \label{tab:attention_modules}
% \end{table}
\renewcommand\arraystretch{1.0}
\setlength{\tabcolsep}{1.8mm}
\begin{table}[htbp]
    \centering
    % \vspace{-2mm}
    \caption{Ablation study comparing BSC and uni-directional V2L module at different stages.}
    \begin{tabular}{c|c|c|c|c|cc}
        \specialrule{.1em}{.05em}{.05em} 
        \multirow{2}{*}{\makecell[c]{Interaction Module}} & \multicolumn{4}{c|}{Stage} & \multicolumn{2}{c}{Metrics} \\
        \cline{2-7}
        & 1 & 2 & 3 & 4 & mIoU & oIoU \\
        \hline
        \multirow{4}{*}{\makecell[c]{BSC}} 
          & \checkmark &  &  &  & 62.67 & 78.99 \\
          & \checkmark & \checkmark &  &  & 65.55 & 79.47 \\
          & \checkmark & \checkmark & \checkmark &  & 65.90 & 79.60 \\
          & \cellcolor{gray!20}\checkmark & \cellcolor{gray!20}\checkmark & \cellcolor{gray!20}\checkmark & \cellcolor{gray!20}\checkmark & \cellcolor{gray!20}\textbf{67.95} & \cellcolor{gray!20}\textbf{80.57} \\
        \hline % 移除了 \\ 换行符
        \multirow{1}{*}{\makecell[c]{only V2L}} 
          & \checkmark & \checkmark & \checkmark & \checkmark & 67.13 & 79.70 \\
        \specialrule{.1em}{.05em}{.05em} 
    \end{tabular}
    \vspace{0mm}
    \label{tab:interaction_modules}
\end{table}
\subsubsection{Evaluation of Bidirectional Spatial Correlation}
Fig. \ref{fig:fusion} compares different cross-modal attention interaction mechanisms employed during feature extraction. Specifically, PWAM employs a unidirectional, multi-stage vision-to-text cross-attention mechanism, WPA utilizes parallel bidirectional attention learning, and IIM explores intra-scale information to enhance cross-modal interactions. In contrast, BSC, our proposed Bidirectional Spatial Correlation mechanism, surpasses these approaches with improvements of at least 2.04\% and 3.33\% in oIoU and mIoU, respectively. This demonstrates BSC's superior adaptability to remote sensing images and its enhanced capability for feature matching, especially for multi-scale targets and precise text localization.

Table \ref{tab:interaction_modules} further validates the effectiveness of bidirectional cross-modal interaction in feature extraction. Specifically, L2V refers to retaining only text-augmented visual feature matching while removing the visual-to-text pathway in BSC. Results indicate that the full BSC design significantly enhances semantic and visual alignment across depths and scales. Ablation studies on the multi-stage encoder reveal that bidirectional interactions—particularly in deeper layers—consistently outperform unidirectional approaches. Moreover, the BSC mechanism effectively leverages spatial information to align textual descriptions with targets of varying scales, as evidenced by its substantial improvements in mIoU.
\begin{figure}
    \centering
    \includegraphics[width=0.95\linewidth]{scale.png}
    \vspace{-3mm}
    \caption{The comparisons of different multi-scale fusion mechanisms}
    \label{fig:scale}
\end{figure}
% \renewcommand\arraystretch{1.05}
% \setlength{\tabcolsep}{1.6mm}
% \begin{table}[ht]
% \centering
% \caption{Ablation studies on cross-scale fusion mechanisms.}
%     \begin{tabular}{@{}c|c|c|c|c|c@{}}
%     \toprule
%     \textbf{Attention} & \textbf{Pr@0.5} & \textbf{Pr@0.7} & \textbf{Pr@0.9} & \textbf{oIoU} & \textbf{mIoU} \\ \midrule
%     LGCE~\cite{yuan2024rrsis}& 77.00& 38.14& 3.30& 75.07& 61.00\\
%     CIM~\cite{liu2024rotated}& 81.23& 55.48& 7.21& 79.46& 65.54\\
%     \rowcolor{gray!20}MCI (ours)& \textbf{83.60}&\textbf{ 62.69}& \textbf{9.14}& \textbf{80.57}& \textbf{67.95}\\ \bottomrule
%     \end{tabular}
%     \vspace{-1mm}
%     \label{tab:scale_modules}
% \end{table}
\subsubsection{Evaluation of Background Predictor}
\renewcommand\arraystretch{1.0}
\setlength{\tabcolsep}{1.6mm}
\begin{table}[htbp]
    \centering
    \footnotesize
    \vspace{-2mm}
    \caption{Ablation Study on the Impact of Background Token Length and Prior Knowledge in T-BTD.}
    \begin{tabular}{c|c|cc}
        \specialrule{0.1em}{0.05em}{0.05em} % 添加特殊规则,表格的顶部和底部线条加粗
        \multirow{2}{*}{\makecell[c]{Number of \\ Bg Semantic Tokens}} & \multirow{2}{*}{Prior Knowledge} & \multicolumn{2}{c}{Metrics} \\
        & & mIoU & oIoU \\
        \hline
        3 & With & 66.80 & 80.30 \\
        \rowcolor{gray!20}5 & With & \textbf{67.95} & \textbf{80.57}  \\
        7 & With & 67.31 & 80.19 \\
        \hline
        5 & Without & 66.46 & 79.52 \\
        \specialrule{0.1em}{0.05em}{0.05em} % 表格的底部加粗线条
    \end{tabular}
    \vspace{0mm}
    \label{tab:bi_attention_layers}
\end{table}
Fig. \ref{fig:scale} compares our proposed Multi-scale Context Integration (MCI) module with other multi-scale fusion mechanisms. LGCE uses a self-attention mechanism for parallel alignment across resolutions, while CIM employs cross-information interaction between multi-scale features at different stages. Our MCI module outperforms the second-best method by 1.11 and 2.41 in oIoU and mIoU, respectively, demonstrating superior multi-scale contextual integration. 

Table \ref{tab:bi_attention_layers} validates the effectiveness of background prediction in enhancing information richness and incorporating prior knowledge. Ablation studies show the best dual-stream performance when the number of Bg semantic tokens is set to 5, leveraging learnable parameters for semantic fitting of unknown classes. Additionally, using masked noun phrases as text references improves background prediction, as shown in the fourth row, with a ~0.7 drop in mIoU and oIoU observed when background prior knowledge is removed.
\renewcommand\arraystretch{1.0}
\setlength{\tabcolsep}{1.6mm}
\vspace{0mm} % 减少表格上方的间距
\begin{table}[htbp]
    \centering
    \footnotesize
    \vspace{-2mm}
    \caption{Ablation study on the effect of the T-BTD and D-MOLS.}
    \begin{tabular}{c|c|c|cc}
        \specialrule{.1em}{.05em}{.05em} 
        \multirow{2}{*}{\makecell[c]{\(L_{\text{fg}}\)}} & \multirow{2}{*}{\makecell[c]{\(L_{\text{bg}}\)}} & \multirow{2}{*}{\makecell[c]{\(L_{\text{re}}\)}} & \multicolumn{2}{c}{Metrics} \\
        &  &  & mIoU & oIoU \\
        \hline
        
        \usym{2714} & \usym{2718} & \usym{2718} & 64.92 & 75.13 \\
        \usym{2714} & \usym{2714} & \usym{2718} & 66.22 & 79.80 \\
        \usym{2714} & \usym{2718} & \usym{2714} & 65.09 & 76.26 \\
        \rowcolor{gray!20}\usym{2714} & \usym{2714} & \usym{2714} & \textbf{67.95} & \textbf{80.57} \\
        \specialrule{.1em}{.05em}{.05em} 
    \end{tabular}
    % \vspace{-1mm}
    \label{tab:ablation_modules}
\end{table}

\subsubsection{Evaluation of Dual-Modal and Target-Background Learning Strategy}
Table \ref{tab:ablation_modules} validates the importance of the various losses designed in our model. When background prediction and text rephrasing are excluded from the training process, meaning their corresponding loss functions are not backpropagated, the model experiences a notable performance degradation. This decline becomes particularly evident when predictions for unknown classes are omitted, as demonstrated in rows 1, 3, and 4 of the table. Additionally, when the learning process of the semantic modality is neglected, the model's performance in terms of mIoU suffers considerably. This demonstrates that text reconstruction make the model to focus on learning and matching complex edge environments and high-fidelity segmentation, while the reconstruction process achieves a more accurate text-image alignment, which is better suited for remote sensing tasks.
 
\section{Conclusion}
In conclusion, we have presented a novel approach for the Referring Remote Sensing  Image Segmentation (RRSIS) task, named \ours. Our method effectively addresses the challenges of remote sensing imagery, particularly the vision-language gap and complex object segmentation. First, to bridge the substantial vision-language gap, we introduce the \textbf{bidirectional spatial correlation module}, enabling bidirectional multimodal feature interaction, thereby improving the alignment between visual features and textual descriptions in remote sensing tasks. To tackle the challenge of diverse object categories and precise localization, especially for small objects, we employ a \textbf{dual-modal object learning strategy}. This approach enhances fine-grained segmentation and improves the model's ability to handle objects at varying scales. Lastly, to address boundary delineation and ambiguous targets, we incorporate a \textbf{target-background twin-stream decoder}, which effectively handles discrete and blurred target boundaries by performing foreground-background joint prediction.These innovations collectively advance segmentation accuracy and robustness in the context of RRSIS. Extensive experiments on the RefSegRS and RRSIS-D datasets validate the effectiveness of our proposed modules and strategies. Our work contributes to advancing the remote sensing image segmentation field, providing a robust solution to referring segmentation in challenging environments.

\textbf{Limitations and Future Works:} Despite achieving remarkable results on the RRSIS task, our method has certain limitations. One potential issue is that our language backbone, BERT, may struggle with handling complex, domain-specific descriptions in remote sensing. With the rise of large language models (LLMs) \cite{achiam2023gpt, lai2024lisa}, integrating an LLM could help bridge the vision-language gap in remote sensing tasks. Another limitation is that while our method delivers precise segmentation, it cannot determine whether the described target area actually exists within the given image. This is due to the dataset bias, where each remote sensing image-language pair typically contains only one target. Future work should focus on developing more robust RRSIS methods to address these challenges and enable practical applications in real-world scenarios.
 
 % argument is your BibTeX string definitions and bibliography database(s)
% \section{Acknowledgment}
% This work is supported by 
\newpage
\bibliographystyle{IEEEtran}
\bibliography{references_new}

%\printbibliography
\vfill


\end{document}


