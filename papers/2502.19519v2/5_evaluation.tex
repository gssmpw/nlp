\section{Comparitive User Study}

To determine whether v2 offers an enhanced player experience, we employed a counterbalanced comparative testing approach where participants played both versions of the game and provided feedback about their experiences.

Participants were 11 males and one female between the ages of 21 and 30 years (M=26) and claimed to play video games for at least 6 hours per week. Ten participants also reported playing up to 5 hours of offline tabletop games per week.

For each version, participants were asked to perform a series of tasks representing typical scenarios achievable within the game and to explore the game as they wished. 

Game tasks included:

\begin{enumerate}
    \item Explore the in-game world and arrive at a new location.
    \item Interact with an in-game object, such as a table, in any way you deem appropriate.
    \item Engage in a conversation with a non-player character.
    \item Participate in combat with an enemy.
    \item Defeat the enemy.
    \item Replenish health by using a potion or other means available to heal your character.
\end{enumerate}


Data collection involved both quantitative and qualitative data concerning participants' experiences. After playing each version of the game, participants completed a survey with quantitative questions about their experience, including a selection from the Player Experience Inventory (PXI)~\citep{player_experience_inventory_pxi_2024-1} and questions specific to the AI GM. Upon completing both versions and surveys, we conducted contextual interview sessions to gather qualitative data, emphasizing their comparative analysis of the two versions. The specific research questions used in the interviews are detailed in Appendix ~\ref{appen:questions}.



\subsection{Quantitative User Experience Analysis}





\begin{table*}[h]
    \centering
    \begin{tabular}{|l|c|c|c|c|c|}
        \hline
        \rowcolor{gray!5}
        \textbf{Construct}              & \textbf{v1 mean} & \textbf{v2 mean} & \textbf{t-statistic} & \textbf{df} & \textbf{p-value} \\ \hline
        \rowcolor{blue!10}
        \textbf{Ease of control}        & \textbf{2.08}    & \textbf{2.81} & \textbf{-3.026} & \textbf{11}   & \textbf{0.012}    \\ \hline
        \rowcolor{blue!10}
        \textbf{Goals and rules}        & \textbf{1.35}    & \textbf{2.39} & \textbf{-2.786} & \textbf{11}  & \textbf{0.018}    \\ \hline
        \rowcolor{blue!10}
        Progress feedback               & 0.89             & 2.00       & -1.872 & 11     & 0.088             \\ \hline
        \rowcolor{blue!10}
        Meaning                         & 1.36             & 1.97     & -1.677 & 11        & 0.122             \\ \hline
        \rowcolor{blue!10}
        \textbf{Curiosity}              & \textbf{1.83}    & \textbf{2.57} &  \textbf{-2.236} & \textbf{11}   & \textbf{0.047}    \\ \hline
        \rowcolor{blue!10}
        \textbf{Mastery}                & \textbf{0.68}    & \textbf{2.33} & \textbf{-3.683} & \textbf{11}   & \textbf{0.004}    \\ \hline
        \rowcolor{blue!10}
        \textbf{Immersion}              & \textbf{1.64}    & \textbf{2.42} & \textbf{-2.420} & \textbf{11}  & \textbf{0.034}    \\ \hline
        \rowcolor{blue!10}
        Autonomy                        & 2.17             & 2.67      & -1.384 & 11       & 0.194             \\ \hline
        \rowcolor{orange!10}
        \textbf{Story interesting}      & \textbf{1.17}    & \textbf{2.33}  & \textbf{-2.755} & \textbf{11}  & \textbf{0.0187}    \\ \hline
        \rowcolor{orange!10}
        \textbf{Coherent story}         & \textbf{1.00}    & \textbf{2.25}  & \textbf{-2.322} & \textbf{11}  & \textbf{0.040}    \\ \hline
        \rowcolor{orange!10}
        Story adapted                   & 1.42             & 2.27         & -1.449 & 11    & 0.175             \\ \hline
        \rowcolor{orange!10}
        Engaging NPCs                   & 1.50             & 1.92       & -1.100 & 11      & 0.295             \\ \hline
        \rowcolor{orange!10}
        \textbf{Likely to play again}            & \textbf{1.58}             & \textbf{2.50}     &  \textbf{-2.2} & \textbf{11}       & \textbf{0.050}             \\ \hline
        \rowcolor{orange!10}
        \textbf{Satisfied with game}    & \textbf{1.08}    & \textbf{2.17}   & \textbf{-2.238} & \textbf{11} & \textbf{0.0468}   \\ \hline
    \end{tabular}
    \caption{T-Test results of post-test survey feedback. Rows with the blue color correspond to the measured Player Experience Inventory (PXI) constructs, while the orange rows represent our own survey questions related to the users' experiences. The constructs with a p-value below 0.05, and therefore show statistical significance, are marked with bold. }
    \Description{A table showing the results of a post-test survey feedback, where rows with blue colors correspond to the measured Player Experience Inventory constructs and show statistical significance.}
    \label{tab:game-feedback}
\end{table*}

Results from the questionnaires suggest that the game versions were significantly different, with v2 outperforming across various measures and preferences. 

To analyze the quantitative data, paired-sample, two-tailed t-tests were conducted for each construct derived from the PXI and custom survey questions. A significance level of 0.05 was set as the threshold for determining statistical significance in the t-test results. Table~\ref{tab:game-feedback} presents the results of the t-test for each user engagement construct. Constructs depicted in blue pertain to the PXI, while those in orange originate from our custom questions. The t-tests indicate that these mean differences are statistically significant in nine constructs. These significant differences highlight v2's superior ability to support gameplay experiences compared to v1. Moreover, they reflect significant advancements in players' sense of competence and the AI's ability to create an engaging game world.

This is crucial as it not only validates the effectiveness of our enhancements but also underscores the importance of structured prompt patterns in driving user engagement and immersion in AI-driven narrative environments.

Participants were also asked to indicate their preferred version based on interview questions. Participants could choose one version over the other or could indicate that they were indistinguishable (tie). The results of these forced-choice questions are summarized in Figure~\ref{fig:interview-scores}. Preferences for v1 are shown in orange, preferences for v2 in blue, and ties in gray. Overall, the preferences favored v2 across all measures.
% Version preference based on interview questions
\begin{figure*}[ht]
    \centering
    \begin{tikzpicture}
        \begin{axis}[
            width=1.4\columnwidth,
            height=0.6\columnwidth,
            xbar stacked,
            bar width=6pt,
            xmin=0,
            xmax=12,
            xlabel={Votes},
            xlabel style={at={(axis description cs:0.5,-0.05)}, font=\scriptsize},
            symbolic y coords={
                Response Quality,
                Flexibility of the GM,
                Complexity and Depth,
                Realism of Outcomes,
                Game Flow,
                Perceived Intelligence,
                Control and Autonomy,
                Story Engagement,
                Overall Enjoyment
            },
            ytick=data,
            y dir=reverse,
            legend style={at={(0.8,-0.10)}, anchor=north, legend columns=-1, font=\scriptsize},
            legend cell align={left},
            enlarge y limits=0.12,
            enlarge x limits=0.05,
            nodes near coords,
            every node near coord/.append style={font=\tiny, text=black},
            label style={font=\small},
            tick label style={font=\scriptsize},
            ]
            
            % Version 1 Data
            \addplot+[xbar, fill=orange, draw=black, 
            legend image post style={draw=black}] plot coordinates {
                (3,Response Quality)
                (2,Flexibility of the GM)
                (2,Complexity and Depth)
                (0,Realism of Outcomes)
                (2,Game Flow)
                (0,Perceived Intelligence)
                (1,Control and Autonomy)
                (1,Story Engagement)
                (2,Overall Enjoyment)
            };

            % Tie Data
            \addplot+[xbar, fill=gray(x11gray), draw=black, 
            legend image post style={draw=black}] plot coordinates {
                (2,Response Quality)
                (4,Flexibility of the GM)
                (3,Complexity and Depth)
                (4,Realism of Outcomes)
                (2,Game Flow)
                (2,Perceived Intelligence)
                (0,Control and Autonomy)
                (2,Story Engagement)
                (0,Overall Enjoyment)
            };
            
            % Version 2 Data
            \addplot+[xbar, fill=cyan, draw=black, 
            legend image post style={draw=black}] plot coordinates {
                (7,Response Quality)
                (6,Flexibility of the GM)
                (7,Complexity and Depth)
                (8,Realism of Outcomes)
                (8,Game Flow)
                (10,Perceived Intelligence)
                (11,Control and Autonomy)
                (9,Story Engagement)
                (10,Overall Enjoyment)
            };
            
            \legend{v1, Tie, v2}
        \end{axis}
    \end{tikzpicture}
    \caption{Version preferences based on post-test interview responses.}
    \Description{A figure showing version preferences based on post-test interview responses highlighting user opinions on various game aspects.}
    \label{fig:interview-scores}
\end{figure*}












\subsection{Qualitative User Experience Analysis}

To explore the less tangible sentiments and opinions about the two systems, we present the themes resulting from our thematic analysis~\cite{braun_one_2021} of the responses in the contextual interviews.

The four main themes that emerged from our analysis included \textit{Game master flexibility dynamics, complex realism and flow, autonomous intelligence perception, and narrative satisfaction.}


\textbf{Game Master Flexibility Dynamics:} This theme encompasses participants' experiences with the responsiveness and adaptability of the game master. In examining the response quality and flexibility of the GM, participants consistently noted significant differences between v1 and v2. A primary complaint about v1 was its poor handling of combat scenarios. Participants found that unless they meticulously described their actions, v1 often resulted in non-sensical outcomes, such as characters stumbling or self-injuring when the system failed to identify an appropriate target. Conversely, v2 was perceived as more adept at managing complex battle situations, providing more accurate and engaging feedback to player actions. Participant P1 articulated this distinction: \textit{``[v2] felt like it adapted to my choices better than [v1] because [v1] just gives me a large description of the things I did, but it felt more accurate with [v2]. Like what I would get with an actual GM, like, "I want to do this," and then he just says, "Okay! You strike that way," where [v1] might add other things that I didn't really anticipate having me doing, like, tripping. Listen, I'm just kicking a ball!''} This sentiment, where P1 recounted an experience of unintended tripping while attempting a simple action, was common among participants. Moreover, while v1’s descriptive additions were appreciated by some users who preferred a passive storytelling experience, most participants shared P1's preference for precise feedback. Flexibility-wise, v1 was described as overly permissive by many, a point P8 illustrated regarding the difference in v2: \textit{``It was like... "Okay, you can try to do those things," but these people are not just going to accept anything.''} This highlighted preference indicates that constrained flexibility can enhance engagement.

\textbf{Complex Realism and Flow:} This theme addresses how participants perceived the intricacy of game mechanisms and the believability of the outcomes. When evaluating complexity, depth, and game flow, v2 again demonstrated superiority. Participants described v1 as hindering progression and "stalemating" the game's pace, as P12 noted. Compared to v1's "Tolkienesque" environmental narratives, v2 provided more interactive and thematic content. P7 characterized v2 as ``more vibey and mysterious,'' enhancing the mystery-themed adventure compared to v1's predictable narrative path. Furthermore, v2 managed combat pacing more effectively, facilitating independent NPC interactions and coordinated attacks, aspects largely absent in v1’s experience. P8 summarized this improvement; \textit{``[...] it doesn't just say "Oh wow, [character name], you are just so cool and amazing!", but acts more like "Right, now here is a new guy, let's find out who he is" as if they are their own character and has their own motivations and doesn't just placate me.''}, acknowledging the more independent and realistic portrayal of NPCs in v2.

\textbf{Autonomous Intelligence Perception:} This theme evaluates participants' views on the system's intelligence and their sense of control and independence within the game. Participants highlighted the advanced understanding of user intent in v2. Scenarios requiring creative strategizing, like P6's backstabbing plot, where they wanted to betray their pirate crew to haul back the entire treasure for themselves, demonstrated v2's capacity to interpret and actuate complex player intentions, unlike v1's reliance on explicit instruction. Participants noted v2's ability to imbue NPCs with unique characteristics, fostering a sense of autonomy similar to a human GM. P5 remarked on the extended narrative coherence provided by v2, likening it to guidance through a complex adventure, a sentiment echoed by most test users. In relation to player agency and autonomy, P2 mentions, \textit{``I'm not sure I can explain why, but [v1] made me feel like there were guard rails, but it was more like "I am in control" in [v2]''}, whereas P5 puts it into perspective with \textit{``[v2] feels like being guided by a game master, but in [v1] I am guiding the game master''}.

\textbf{Narrative Satisfaction:} This theme captures the engagement in the storyline and general satisfaction derived from the gameplay experience. In story engagement and overall enjoyment, v2 excelled by maintaining narrative direction and preventing stagnation through consistent nudging toward goals. P12 encapsulated this sentiment, highlighting the authentic D\&D experience fostered by v2 as opposed to the effort required to direct v1's narrative. P1 appreciated the decision-making freedom offered by v2, fundamental to sustaining player investment and exploration, saying \textit{``A big thing that I, like, enjoyed---the [v2] did feel like it gave me more options to show "what do you want to do?" and "how do you wish to proceed?" which made me feel more invested because it gives me the incentive so I can choose "I want to do this or that" while [v1] was more of a lush description that I could read and I would have to figure it out myself.''}.

Overall, the qualitative feedback aligns with the quantitative data collected, which is shown in Table~\ref{tab:game-feedback}, reinforcing that v2 significantly enhances user experience in alignment with traditional TTRPG values. The insights from interviews vividly illustrate participants' preferences and the disparities in satisfaction between the two versions, substantiating the quantitative data's findings.