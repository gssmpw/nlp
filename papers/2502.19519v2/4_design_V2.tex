\section{Phase 2 - Design of the Game Master for Version 2} \label{sec:design}


%Introduction
In this section, we describe the technical changes involved in the development of the advanced version (v2) of the AI GM. Building on the first version, the front-end UI remains unchanged; however, in this version, the underlying structure and technical components were redesigned to enhance the system's interactive and narrative capabilities. Moving beyond the simple prompt engineering approach, v2 is a more sophisticated multi-agent system composed of two distinct LLM agents: the \textbf{Narrator} and the \textbf{Archivist}. The high-level overview is shown in Figure~\ref{fig:chatrpgv2flow}. These agents are designed using the ReAct framework and are tasked with different roles, each performing unique functions to collectively emulate the role of a human GM in IF games, such as D\&D. A detailed architecture diagram (see Figure~\ref{fig:component_diagram}) shows how the agents call the tools. All text used in the prompts can be found in Appendix~\ref{app_chatgpt_v2_prompts} and a publically available repository\footnote{\href{https://github.com/KarmaKamikaze/ChatRPG}{\texttt{https://github.com/KarmaKamikaze/ChatRPG}}}. By leveraging the ReAct framework, this system enables more effective decision-making through self-reasoning, allowing it to generate well-considered responses and perform informed actions using integrated toolchains. This paradigm significantly enhances the upgradability and extensibility of the system, addressing the limitations of v1 and paving the way for a more immersive and flexible RPG experience.


%IA I changed the width to make it readable, otherwise the font is too small 
\begin{figure*}[ht!]
  \centering
  \includegraphics[width=.9\linewidth]{0_pictures/ChatRPGv2flow.png}
  \caption{Game interaction flow diagram showing how, in v2, user input is handled by the Narrator and Archivist agents to make tool calls and prompts to the LLM and present updates to the UI.}
  \Description{A figure showing a flow diagram illustrating how multiple agents interact in the ChatRPG v2 system. The diagram shows how the conversational user interface, world state, LLM, and two agents interact with each other.}
  \label{fig:chatrpgv2flow}
\end{figure*}

\begin{figure*}[ht]
    \centering
    \includegraphics[width=0.6\linewidth]{0_pictures/ChatRPGv2.png}
    \Description{Component diagram.}
    \caption{This figure illustrates the architecture of ChatRPG v2, which integrates user input, AI reasoning, and a dynamic world state. The system starts with the Front-End's Text-Based Interface, where players input their actions. These inputs are processed by the Back-End's Game Input Handler and passed to the Narrator Agent, which uses the ReAct framework to generate a narrative through decision-making. Reasoning and resolutions are handled via the OpenAI API, and Tools are employed when specific actions are required. The Archivist Agent ensures changes are recorded in the Campaign World State, which is stored persistently in the Database using Entity Framework. The closed loop allows for continuous gameplay driven by player input and AI responses.}
    \Description{A figure showing the architecture of ChatRPG v2. It takes the user input and generates a narrative response using the Narrator. Then the Archivist is used to update the game state. Upon receiving the narrative response, players can input their next action.}
    \label{fig:component_diagram}
\end{figure*}



\subsection{Responsibilities and Roles of the Agents}
In a live game session, a human GM manages several complex tasks: they oversee the dynamic progression of the storyline, maintain awareness of the characters and environmental elements within the game, and spontaneously generate new scenarios and NPCs as needed. Additionally, they track the impact of players' actions on the overarching narrative and environment, which can lead to the emergence of new objectives that must be remembered for future reference. These diverse tasks are mirrored within our system through the division of labor between the Narrator and Archivist agents. The seamless interaction between these agents is crucial to creating a cohesive and immersive gameplay experience. Each agent fulfills a unique function, contributing to the overall emulation of a human GM. While the \textbf{Narrator} is prominently engaged with the player, crafting rich narratives and thoughtfully responding to actions, the \textbf{Archivist} operates discretely in the background, ensuring the continuity and consistency of the game's world and its narrative elements.

\subsection{The Narrator}
The \textbf{Narrator} serves as the system's primary storyteller, focusing on delivering immersive and engaging narratives. It processes user input to create dynamic and contextually relevant responses, ensuring that the gameplay experience remains vibrant and compelling. The Narrator is programmed to emulate the thought processes of a skilled human game master by effectively reasoning about the game's fictional world. It is responsible for crafting the outcomes of player actions and maintaining the narrative continuity, aligning with the storytelling norms expected by players. It has the capability to call the following JSON-based tools:

\textbf{\textit{WoundCharacter}} - is a mechanism that inflicts injury on a character from dangerous actions or unnoticed attacks, with the severity level specified as low, medium, high, or extraordinary.

\textbf{\textit{HealCharacter}} - is a mechanism that restores a character's health through healing actions—such as spells, potions, or rest—with the healing magnitude specified as low, medium, high, or extraordinary.

\textbf{\textit{Battle}} - is a mechanism that simulates combat between two characters by specifying attack order, hit chances, and damage severity for each exchange.

These tools are employed when a character's health points in the game are subject to modification. Figure~\ref{fig:react_narrator_example} illustrates a trajectory produced by the Narrator agent, demonstrating its ability to utilize tools to generate an immersive narrative in response to player input. The trajectory's input consists of the player's input, prefaced by a description of prior events in the game, referred to as \textit{Memory}. This description may be a summary of the story or the complete sequence of player inputs and narrative responses, ensuring that the Narrator remains aware of previous developments leading up to the new player input. For brevity, the content of the \textit{Memory} is omitted from the example.

In the initial Thought substep, the Narrator evaluates whether to employ a tool based on the player's input. Given that the player is engaging in an attack, the agent decides to invoke the Battle tool in the first Action substep. It supplies the Battle tool with a JSON object containing the necessary information for proper execution. The Observation substep records the outcome of the Battle tool. Upon reviewing this Observation, the Narrator resolves not to use an additional tool in the second Thought substep, as the player's input merely indicates that a battle should ensue. Finally, the Narrator generates a narrative response for the player as its concluding response, based on the results of the Observation substep.

\begin{figure}[h!]
    \centering
    \begin{tcolorbox}[colframe=black!35!white, colback=black!5!white, coltitle=black, title=\textbf{Narrator Trajectory Example}, halign title=flush center, width=0.49\textwidth]
        \textbf{Input:} \{Memory\} + I swing my sword towards the guard's sword-wielding arm in hopes of disarming him.\\
        
        \textbf{Thought:} Do I need to use a tool? Yes. \\
        \textbf{Action:} \texttt{Battle}[\{\\
        \hspace*{1em}"participant1": \{\\
        \hspace*{2em}"name": "Ivan",\\
        \hspace*{2em}"description": "A wielder of earth, wind, and fire."\\
        \hspace*{1em}\},\\
        \hspace*{1em}"participant2": \{\\
        \hspace*{2em}"name": "Castle Guard",\\
        \hspace*{2em}"description": "A vigilant guard of the kingdom."\\
        \hspace*{1em}\},\\
        \hspace*{1em}"participant1HitChance": "Medium",\\
        \hspace*{1em}"participant2HitChance": "Low",\\
        \hspace*{1em}"participant1DamageSeverity": "High",\\
        \hspace*{1em}"participant2DamageSeverity": "Medium"\\
        \}] \\
        \textbf{Observation:} Ivan described as "A wielder of earth, wind, and fire" fights Castle Guard described as "A vigilant guard of the kingdom".\\
        Ivan deals 12 damage to Castle Guard. They have 28 health points out of 40 remaining. Castle Guard misses their attack on Ivan. Ivan and Castle Guard's battle has been resolved and this pair can not be used for the battle tool again. \\
    
        \textbf{Thought:} Do I need to use a tool? No. \\
        \textbf{Final Answer:} Your sword strikes the guard's shoulder and he winces in pain but is still able to hold on to his sword. The guard, despite his combat training, is thrown off balance and therefore misses his retaliatory attack. He realizes his mistake and assumes his combat stance, more focused than ever. 
    \end{tcolorbox}
    \caption{An example of a ReAct reasoning trajectory produced by the Narrator agent.}
    \Description{A textual representation of a ReAct trajectory produced by the Narrator agent.}
    \label{fig:react_narrator_example}
\end{figure}


\subsection{The Archivist}
The \textbf{Archivist} functions as the memory core of the AI-driven GM, meticulously logging and recalling crucial information about the game environment and character interactions. Operating behind the scenes, akin to a GM working behind the screen, the Archivist is tasked with two primary functions: preserving narrative continuity and enhancing player engagement with the game's memory system. The Archivist analyzes the Narrator's outputs to detect changes or introductions of entities, such as new locations, characters, or events. This analysis allows for the efficient updating and maintenance of the game's state, ensuring that all narrative strands remain coherent and interconnected. For example, if a new location is revealed or an existing character undergoes significant change, the Archivist logs these changes, enabling the game to dynamically reflect the evolving storyline. Beyond tracking, the Archivist provides an interactive memory interface for players. This interface allows players to access memory fragments about previously explored environments and met characters, replicating the experience of players querying a human GM about past encounters or observations. This feature not only aids players in recalling crucial details but also enriches their immersion by allowing them to 'rediscover' the game world as if through their characters' memories.
To accomplish these tasks, it utilizes the following JSON-based tools:

\textbf{\textit{UpdateCharacter}} - a mechanism that creates or modifies a character's profile by updating their name, description, type, and health state in the campaign.

\textbf{\textit{UpdateEnvironment}} - a mechanism that creates or updates an environment's description, attributes, and player presence within the campaign.

These tools receive a JSON object that describes the entity to be created or updated, which they use to modify the game state. Figure~\ref{fig:react_archivist_example} illustrates a trajectory produced by the Archivist agent, showcasing its ability to utilize tools to maintain the game state based on the player's input and the narrative response generated by the Narrator. The Archivist's input also includes \textit{Memory} of the game, which is more extensive in this case as it contains JSON descriptions of all characters and environments in the game state. This comprehensive \textit{Memory} is essential for enabling the Archivist to decide whether to create or update characters or environments. For brevity, the content of the \textit{Memory} is omitted from the example.

In the initial Thought substep, the Archivist assesses whether to employ a tool based on the player's input and the narrative response. As the player enters a new environment, the agent decides to invoke the UpdateEnvironment tool during the first Action substep, supplying it with a JSON object containing details about this new environment. The UpdateEnvironment tool executes by adding the newly defined environment to the database, as recorded in the Observation substep. Upon reviewing this Observation, the Archivist determines not to use an additional tool in the second Thought substep, as the input does not suggest the need to create or update characters or other environments. Finally, the Archivist generates a summary of the environment it has created based on the Observation substep as its concluding response.

\begin{figure}[h!]
    \centering
    \begin{tcolorbox}[colframe=black!35!white, colback=black!5!white, coltitle=black, title=\textbf{Archivist Trajectory Example}, halign title=flush center, width=0.49\textwidth]
        \textbf{Input:} \{Memory\} + Player input: I sneak towards the encampment's barracks and attempt to enter sneakily through the door. Narrator: As you sneak around like a scoundrel, you hear rustling from the barracks. As you attempt to open the door, you find that it is locked.\\
        
        \textbf{Thought:} Do I need to use a tool? Yes. \\
        \textbf{Action:} \texttt{UpdateEnvironment}[\{\\
        \hspace*{1em}"name": "Encampment Barracks",\\
        \hspace*{1em}"description": "A wooden makeshift shelter for the encampment's soldiers. The door is locked.",\\
        \hspace*{1em}"isPlayerHere": true\\
        \}] \\
        \textbf{Observation:} A new environment Encampment Barracks has been created with the following description: A wooden makeshift shelter for the encampment's soldiers. The door is locked.\\
    
        \textbf{Thought:} Do I need to use a tool? No. \\
        \textbf{Final Answer:} I have created the environment Encampment Barracks. No other new details about characters or environments are mentioned in the narrative, so I am finished. 
    \end{tcolorbox}
    \caption{An example of a ReAct reasoning trajectory produced by the Archivist agent.}
    \Description{A textual representation of a ReAct trajectory produced by the Archivist agent.}
    \label{fig:react_archivist_example}
\end{figure}


%Tool descriptions
\subsection{Agent Tools}
To assist agents in selecting the appropriate tools, we provide them with detailed descriptions of each tool in the form of short text descriptions (see Appendix~\ref{app_tool_descriptions} and publically available repository\footnote{\href{https://github.com/KarmaKamikaze/ChatRPG}{\texttt{https://github.com/KarmaKamikaze/ChatRPG}}}). Some of these descriptions utilize few-shot prompting to aid the agents in selecting the appropriate tool based on example usages. The number of examples and their depth depends on the complexity and variety of inputs to the tool. For example the Battle tool's description has one brief and one extensive example, whereas both of the examples provided for the HealCharacter tool are brief. Generally, 2-3 examples are used to capture the different facets of the tool.
 
 Figure~\ref{fig:healing-tool} exemplifies the Narrator's reasoning process when utilizing the HealCharacter Tool. The reasoning process for tool selection by agents involves three essential components: \textit{Tool Usage Instructions}, \textit{Example Usages}, and \textit{Player Action}/\textit{Narrative Response}.

\begin{enumerate}
    \item \textbf{Tool Usage Instructions:} This component delineates when a tool should be employed and how to use it. For the HealCharacter tool, this section advises using the tool when a character performs a healing action, limited to once per character.
    \item \textbf{Example Usages:} This component provides example scenarios that warrant the use of the tool. For the HealCharacter tool, examples include scenarios such as drinking a potion or resting.
    \item \textbf{Player Action/Narrative Response:} This component represents the input trajectory, where it is a \textit{Player Action} for the Narrator and a \textit{Narrative Response} for the Archivist---the agents reason using these three components to determine if a tool should be invoked. If affirmative, the reasoning process results in defining the \texttt{JSON Input} for the tool. For HealCharacter, the agent would identify the \textit{Player Action} as a healing action, generating a JSON input that includes the player's input and a magnitude property to define healing intensity.
\end{enumerate}

The Narrator relies on the description of the HealCharacter tool to ascertain the \textit{Tool Usage Instructions} and \textit{Example Usages}, as well as the JSON structure of its input.
 
\begin{figure}[h!]
    \centering
    \begin{tikzpicture}
        % Healing action circle
        \draw[fill=blue!20, draw=blue, thick] (0, -1.65) circle (1.0cm);
        \node[text width=1.5cm, align=center] at (0, -1.65) {\footnotesize\textbf{Player Action}};

        % Arrow pointing down
        \draw[->, thick] (0, -2.7) -- (0, -3.6);
        \node[anchor=west, align=center] at (0.2, -3.1) {\small \textbf{Reasoning result:}\\ \small Healing action};

        % JSON Input Box below
        \node[anchor=north] at (0, -3.5) {
 %           \begin{adjustbox}{width=0.3\columnwidth}
            \begin{tcolorbox}[colback=gray!5, colframe=black, title=JSON Input, width=0.8\columnwidth, fontupper=\small]
\begin{verbatim}
{
  "input": "The player's input",
  "magnitude": "low, medium, 
               high, extraordinary"
}
\end{verbatim}
            \end{tcolorbox}
%            \end{adjustbox}
        };

        % Instructions box above
        \node[anchor=north] at (0, 3.9) {
 %           \begin{adjustbox}{width=0.3\columnwidth}
            \begin{tcolorbox}[colback=yellow!10, colframe=black, title=Tool Usage Instructions, width=0.8\columnwidth, fontupper=\small]
            Use this tool when a character performs a healing action. Input must be JSON format. 
            Use only once per character.
            \end{tcolorbox}
%            \end{adjustbox}
        };

        % Example actions box below JSON
        \node[anchor=north] at (0, 1.6) {
 %           \begin{adjustbox}{width=0.3\columnwidth}
            \begin{tcolorbox}[colback=green!5, colframe=black, title=Example Usages
            , width=0.8\columnwidth, fontupper=\small]
            A character drinks a potion, uses a magical item, or rests in a healing zone.
            \end{tcolorbox}
 %           \end{adjustbox}
        };
    \end{tikzpicture}
    \caption{Illustration of the Narrator's reasoning process of using the HealCharacter tool.}
    \Description{An illustration of the Narrator's reasoning process using the HealCharacter tool. Specifically how it is used to determine if a character performs an action that should heal them.}
    \label{fig:healing-tool}
\end{figure}

As agents utilize tool descriptions to guide their reasoning process, all descriptions follow a standardized pattern to ensure consistent behavior:

\begin{enumerate}
    \item \textbf{When to Use:} This step ensures the tool is invoked only under appropriate circumstances, forming the basis of the \textit{Tool Usage Instructions} component of the reasoning process.
    \item \textbf{Examples of Usage:} This step provides sample scenarios for tool application, directly corresponding to the \textit{Example Usages} component. It enables the Narrator agent to associate specific actions with the respective tool.
    \item \textbf{Input Format:} This step defines the input format, which is crucial for accurate parsing by the tools. All input formats are specified in JSON to support consistent interpretation.
    \item \textbf{Number of Uses in Each Trajectory:} This step determines the permissible frequency of tool use within each ReAct trajectory, which corresponds to a single message from the player. It is vital for ensuring agents adhere to the intended usage limits, incorporated into the \textit{Tool Usage Instructions} component.
\end{enumerate}

By adhering to this structured format, the agents are equipped to make informed, consistent decisions, ensuring the execution of tools within the AI GM framework.


%Benefits of v2 approach compared to v1
%\subsection{Design Innovations}

%ChatRPG v2 introduces a dual-agent system—Narrator and Archivist—that leverages the ReAct framework and memory to simulate human-like GM decision-making. The Narrator dynamically adapts the storyline to player interactions, while the Archivist ensures narrative continuity by managing key game elements. This modular design not only enables richer, emergent gameplay and more natural NPC interactions but also simplifies feature integration. Unlike v1’s monolithic approach, v2 supports seamless upgrades by adding new tools without disrupting existing functionality.