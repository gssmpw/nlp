\section{Introduction} \label{sec:introduction}

%Intro
%IA:moved to background

%IA: moved some parts that were here to the beginning and end of the designing V1 section


%WE need more meat here since I moved the content to the background, where I believe it fits better

In a society that is socially accelerated~\cite{rosa2019resonance}, where making deeper connections and friendships is becoming ever more challenging, designers are increasingly aiming to design for resonating experiences~\cite{piHearts_2020, curran2023social, Tan02012025}, reducing loneliness and social isolation~\cite{Baecker_et_al_2014, probst2024_et_al}, and to foster wellbeing and human flourishing~\cite{aslan2023compliment,calvo2014positive}. While loneliness and isolation are often associated with older adults~\cite{Stuart_et_al_2022}, a lesson from the pandemic is that social isolation is an important matter for all ages. Previous research has outlined various psychological benefits of conversational interactions
~\cite{clark1993stress} such as coping with stress. There is mounting evidence that role-playing games can be beneficial by engaging players in conversational interactions, even those moderated by a computer whether they involve multiple or single players~\cite{arenas_therapeutic_2022}.

Large language models (LLM), which are, in essence, conversational agents, are increasingly explored in the context of games~\cite{penny_survey_CUI2024}; a recent survey of LLM use in games is provided by Gallotta et al.~\cite{gallotta2024large}, which covers LLMs assisting human game masters and, in sporadic cases, take the role of a Game Master (GM)~\cite{hua2020playing, You_et_al_2024,Triyason2023}, which is necessary for solo role-playing. Some open-source tools already exist, like, for instance, TavernAI / SillyTavern \cite{sillytavern}, which can enhance LLM-based roleplaying with user-friendly interfaces. Similarly, there is a range of "LoRA" ("Low-Rank Adaptations" small adapter-like layers that colour the use of a specific model for a specific type of roleplaying) using KoboldCPP \cite{koboldcpp} available with a role-playing bent. This for us shows a rich potential in using LLMs for solo-roleplaying experiences.

To further explore this rich potential area for LLMs, we report on our design and research journey, creating a GM for solo role-playing to help fill times of otherwise loneliness with engaging and joyful interactive fiction. Our research can be divided into three distinct phases. 
In phase one, we embarked on exploiting LLMs as an emerging technology. 
In this first phase, we set the technical and functional basics to realize and test a first interactive GM with advanced language skills and explored if and how well we can realize a first GM version with prompt engineering techniques alone. 
The resulting GM v1 was validated in a pilot study with users (N=8) already demonstrating high usability and abilities that resulted in user satisfaction and willingness of users to replay the game. The insights from phase one were then used in phase two for system improvements and the design of an advanced GM (v2) with which we aimed to go beyond what are the limits of simple prompt engineering by using a narrator and an archivist agent (see Figure \ref{fig:teaser}) to ultimately enhance the robustness of player experiences and deliver highly immersive and engaging narratives. 
In the third phase, we evaluated both versions in detail with users (N=12) and compared them during longer playing sessions. While previous work has explored LLMs as GMs based on prompt engineering (e.g., ~\cite{You_et_al_2024,Triyason2023}), applying further techniques to increase machine intelligence is yet underexplored. We aim to help close this gap by identifying and reporting on the benefits and limitations of using multiple agents and an increasingly agentic approach to GM designs. 

Overall, we found that our agentic v2 GM was preferred by the participants, achieving higher ratings in perceived intelligence, flow, and immersion, among other measures of engagement.  
We acknowledge that designing and evaluating GMs is a cumbersome technical task. To foster replicability and tool support for fellow researchers, we make the source code of both GM versions available. Especially the system improvements in v2 ease and support researchers in creating and customizing their own GMs\footnote{\href{https://github.com/KarmaKamikaze/ChatRPG}{\texttt{https://github.com/KarmaKamikaze/ChatRPG}}}.


%Contributions


%To summarize, we make the following contributions:
%\begin{itemize}
%    \item  We develop a multi-agent system functioning as an AI GM utilizing the ReAct paradigm to ensure upgradability and extensibility.
%    \item We make our implementation details publicly available through our GitHub\footnote{\href{https://github.com/KarmaKamikaze/ChatRPG}{\texttt{https://github.com/KarmaKamikaze/ChatRPG}}}.
%    \item We conduct a comparative study between ChatRPG v1 and v2, demonstrating that v2 closely matches the performance of v1 while offering superior user engagement.
%    \item We set the stage for further research into how advanced features such as multimedia content and enhanced storytelling capabilities can improve user experience in text-based RPGs.
%\end{itemize}



%The rest of this paper is structured as follows: the bacground seciton presents relevant theory on LLMs and advanced prompting techniques and reviews related work in the fields of IF, ReAct, and AI GMs.  

%Section~\ref{sec:design} details the design of the improved AI GM, while Section~\ref{sec:experiments} presents the results of our bifurcated comparative study. Section~\ref{sec:discussion} describes limitations of our study and future research avenues.  Finally, Section~\ref{sec:conclusion} concludes on the improved GM for the IF system.

