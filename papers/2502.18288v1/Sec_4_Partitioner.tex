%forked from the original Partitioner

\section{Partitioner} \label{sec:partitioner}

In this section, we present the partitioner in detail, which divides each large \zk circuit into smaller sub-circuits, which can be executed serially and independently while promising the correctness of results. 

%The primary goal is to provide flexible control over memory usage during the proof generation  phase, while maintaining the correctness of the circuit logic and ensuring that the overall execution time remains nearly the same.


\subsection{Problem Formalization}
% We start by formally defining the concept of \textit{Serializable Circuit Partitioning}.



A \zk circuit $\mathcal{F}$ can be defined as,
$\mathcal{F}(I_{p}, I_{s}) = (O_{p}, O_{s})$. Here, $I_{p}$ and $I_{s}$ represent the public and secret inputs, respectively. Similarly, $O_{p}$ and  $O_{s}$ are the public and secret outputs. The secret inputs $I_{s}$ are used internally by the prover, while the secret outputs $O_{s}$ should remain hidden and are not revealed to the verifier.
Apart from the inputs/outputs, another important component in a circuit is \emph{constraints}.  Constraints are sets of equations or inequalities that verify the correctness of the circuit.  They describe the relationship between inputs and outputs. In \cref{fig:ex_zk_circuit},  the nodes  $x_1 \sim x_4$ are inputs, and $y$ is the public output. Each arithmetic constraint corresponds to an output, where private output like $w_1 \sim w_8$ is hidden as part of the intermediate computation, while the final public output $y$ is revealed at the end of the process.



The goal of the partitioner is to divide the original circuit $\mathcal{F}$ into subcircuits  $\{F_1, \ldots, F_k\}$, such that each sub-circuit can be executed sequentially. To this end, \system extracts a \textit{Constraint Dependency Graph (CDG)} from the circuit and performs the partitioning onto it. 

% 原本带有Input的定义
% \begin{definition}[Constraint Dependency Graph]\label{def:cdg}

% A \textit{Constraint Dependency Graph (CDG)} is a directed acyclic graph  $G = (V, E)$. Here, each vertex in $V$ represents a constraint or input of the original circuit.  For each edge $(u, v) \in E$, the output of constraint $u$ is forwarded to the input of $v$. 
% \end{definition}
\begin{definition}[Constraint Dependency Graph]\label{def:cdg}
A \textit{Constraint Dependency Graph (CDG)} is a directed acyclic graph  $G = (V, E)$. Each vertex in $V$ represents a constraint of the original circuit.  For each edge $(u, v) \in E$, the output of constraint $u$ is forwarded to the input of $v$. 
\end{definition}


A CDG captures the dependencies between all constraints. The left-hand side of \cref{fig:circuit_partition} depicts the CDG for the circuit in \ref{fig:ex_zk_circuit}. Based on CDG, the circuit partitioning problem is transformed into a graph partitioning problem. For a CDG $G$, a partition  is $\mathcal{P}=\{V_1, V_2, \ldots, V_k\}$, such that $G.V=\cup_{1\le i \le k}V_i$ and $V_i \cap V_j = \emptyset, \forall i\ne j$. The nodes in $V_i$ correspond to the constraints in each subcircuit $F_i$, and the edges between nodes in $V_i$ remain the connection relationships between constraints\footnote{In this section if there is no ambiguity, we use a subcircuit and partition interchangeably. }.
Besides, we require the partition to satisfy two conditions: \emph{serializable}, which promises the sequential and independent execution of each subcircuit; \emph{balanced}, which ensures the load of each subcircuit,  in terms of computational resources to pay,  is bounded and approximately equal. 


\paragraph{Serializable partition} If a partition $\mathcal{P}$ is serializable, it means each $V_i$ only depends on the set $V_j$ proceeding itself, i.e., $j<i$. Specifically, $V_i$ depends on $V_j$ when there exists edge $(v, u) \in G.E$, such that $u \in V_i$ and $v \in V_j$. Given a serializable partition, \system can execute all sub-circuits (or partition) sequentially because when $V_i$ is going to be executed, all the sub-circuits depended on have been finished. 


\paragraph{Balanced partition} The load $L(V_i)$ of a sub-circuit $V_i$ is defined as: 
\begin{equation}\label{equ:circut_load}
L(V_i) = |V_i| + \sum_{j < i} \left| \{  (u, v) \in E \mid u \in V_j, v \in V_i\} \right|     
\end{equation}

Commonly, the load of each partition is proportional to the resources used to execute it. 
Therefore, a balanced partition aims to minimize the maximum load across partitions: 
$\min \max_{1\leq i \leq k}(L(V_i))$. 

%balanced里面,减少shared variables的一个目的是内存,另一个更主要的目的是减少验证代价,因为shared variables是否在两个子电路上匹配需要验证方额外验证
In \cref{equ:circut_load}, two parts contribute to the subcircuit load $L(V_i)$: \romannumeral1) number of nodes in $V_i$ and \romannumeral2) number of cross-partition edges between $V_i$ and other partitions. The number of nodes in a sub-circuit directly impacts memory usage during proof generation. On the other hand, more cross-partition edges indicate more \emph{shared variables}, intermediate results, should be shared between constraints, increasing memory usage during the witness generation phase. Moreover, these shared variables also need to be verified, consuming additional computation. 
Therefore, we also take it into account of the load. When dealing with large-scale circuits, we should limit the sub-circuit size and shared variable number, making the memory consumption for subcircuits can fit the real equipped resources. 


\begin{figure}[t]
    \centering
    \includegraphics[width=0.48\textwidth]{figures/zk_partition.pdf}
    \caption{An example of serializable circuit partitioning and the subcircuit construction. }
    \label{fig:circuit_partition}
\end{figure}

\subsection{Topological Sort-base Greedy Partition}

The circuit partition problem defined yet is a highly complex problem. To address this problem, we propose a greedy partition algorithm based on the topological sort.  We guarantee the serializability of partition by topological sorting and achieve the balance via two greedy strategies. 

Topological sorting performs a linear ordering on a DAG, which motivates us to design a serializable partition. Given a topological sorting $ v_1, v_2, \dots, v_N$ for the graph $G$, the serializable partition for $V_i$ is defined as: 
$$V_i = \left\{ v_t \mid (i-1) \cdot \frac{N}{k} < t \leq i \cdot \frac{N}{k} \right\}$$
Apparently, a node $v_i$ must proceed $v_j$ in the ordering if an edge $(v_i, v_j) \in G.E$ exists. Then, nodes in $V_i$ only depend on the nodes in proceeding partitions $V_j$ ($j\le i$) according to this strategy, equivalently promising the serializability of $\mathcal{P}$. 


The above approach only solves one requirement of partitioning, and the remaining focus is on achieving balance. However,  the challenge lies in finding a topological sorting that minimizes the number of shared variables across partitions. Instead, we propose a greedy algorithm to solve this problem. Before delving into the details, we have to introduce two concepts first.

\begin{definition}[Depth]
The depth $d(v)$ of a node $v \in G.V$ is the length of the longest path from any node without in-edges to $v$.
\end{definition}

\begin{definition}[Out-Degree]
The out-degree $g(v)$ of a node $v \in V$ is the number of edges directed outward from  $v$, indicating how many other constraints depend on $v$'s output.
\end{definition}
In \cref{fig:circuit_partition}, the depth and out-degree of node numbered with 7 are 3 and 1, respectively. During the topological sorting, when visiting a node, we follow two heuristic strategies to explore its children nodes:
\begin{itemize}[leftmargin=*]
    \item Nodes with smaller depth are prioritized as they depend on fewer constraints.  
    This allows shallower nodes to be processed earlier and the deeper dependencies are grouped together in later partitions.

    \item Nodes with smaller out-degrees are prioritized because they affect fewer other constraints. By this approach, the impact of the current partition on future partitions is minimized, reducing shared variables between them.
\end{itemize}



\system integrates a DFS-based topological sorting to order nodes in original $G$. In particular, DFS ensures that a node is processed only after its dependencies have been fully resolved. 
%It more effectively captures the dependencies between nodes, reducing the number of shared public variables between partitions.
When traversing from a node,  we explore its neighbored nodes with priorities according to the aforementioned two strategies to achieve a balanced partitioning.

In \cref{fig:circuit_partition}, the numbers attached to nodes indicate their order generated by the topological sorting with proposed greedy strategies. Then, the CDG is partitioned into three parts. Actually, we still have to recover the subcircuits with respect to partition. The scheduler does this, which is detailed in the next section.   

\begin{algorithm}[t]
\caption{Topological Sort-based Greedy Partition }
\label{alg:partition}

\require{\emph{Set of constraints $C$, partition number $k$}}

\ensure{\emph{Serializable partition $\mathcal{P} = \{V_1, V_2, \dots, V_k\}$}}\\

$G^{-1}\gets {\rm Rev\_CDG}(C)$ \tcp{Construct Rev-CDG} \label{stmt:rev_cdg_construct}
$L_{root}\gets$ nodes in $G^{-1}$ without incoming edges \label{stmt:root}\\
Initialize partition set $\mathcal{P} \gets \emptyset$, $V_p\gets \emptyset$\\
Initialize $s \gets \left\lceil {|V|}/{k} \right\rceil$ \label{stmt:partition_size}\\

\For{{\rm each root node $r \in L_{root}$} \label{stmt:root_partition_start}} {
    {Rec\_Partition}{$(r, \mathcal{P}, V_p, G^{-1}, s)$} \label{stmt:root_partition_end}
}
    \Return{$\mathcal{P}$} \label{stmt:return_partition}\\
\vspace{5pt}
\function{\emph{Rev\_CDG($C$)}} {
    Initialize a Rev-CDG $G^{-1} \gets (V=\emptyset, E^{-1}=\emptyset)$ \label{stmt:rev_cdg_init}\\
    \For{\rm each constraint $c \in C$}{
        $V\gets V\cup \{v_c\}$ \label{stmt:add_constraint}
    }
    \For{\rm each pair of constraints  $(c_1, c_2) \in C$ \label{stmt:add_edge_start}} {
        \If{\rm output of \( c_1 \) is an input to \( c_2 \)} {
            $E^{-1}\gets E^{-1}\cup \{(v_{c_2}, v_{c_1})\}$\\
            \tcc{Update depth and out-degree}
            $d(v_{c_2}) \gets max\left\{d(v_{c_1}) + 1, d(v_{c_2})\right\}$\\
            $g(v_{c_1})\gets g(v_{c_1})+1$ \label{stmt:add_edge_end}
        }
        
    }
    \Return{$G^{-1}$}
}

\vspace{5pt}
\function{\emph{Rec\_Partition($v,\mathcal{P}, V_p  G^{-1}, s$)} \label{stmt:rec_partition_start}} {    
    \If{ {\rm $v$ has been visited}} {
        \Return
    }
     Mark $v$ as visited\\
     \tcc{Visit nodes according to their priorities}
    \For{ {\rm $u \in  Sorted\_Children(v, G^{-1})$} \label{stmt:explore_children_start}} {
        {Rec\_Partition}{($u, \mathcal{P}, V_p, G^{-1}, s$)} \label{stmt:explore_children_end}
    }
    $V_p\gets V_p\cup \{v\}$ \tcp{Add to current partition}
    \If{$|V_p| = s$ \label{stmt:add_new_partition_start}} {
        $\mathcal{P}\gets \mathcal{P}\cup \{V_p\}$ \tcp{Add new partition} \label{stmt:add_to_partition}
        $V_p\gets \emptyset$ \label{stmt:add_new_partition_end}\label{stmt:rec_partition_end}
    }
}

\end{algorithm}



\subsection{Algorithm}
% 逻辑: 要找到一个拓扑排序-> DFS-based得到拓扑排序-> DFS(后序遍历)对等位节点的遍历顺序和结果直接绑定-> 基于depth和out-degree选择先遍历谁   
In this subsection, we present the details of our partition algorithm. Our algorithm performs the topological sorting against a \emph{Reversed Constraint Dependency Graph}, defined in \cref{def:rev_CDG}, instead of the original CDG. 
\begin{definition}[Rev-CDG] \label{def:rev_CDG}
Given a CDG \( G = (V, E) \), its corresponding Reversed Constraint Dependency Graph \( G^{-1} = (V, E^{-1}) \) satisfied $\forall (u, v) \in E$, \( (v, u) \in E^{-1} \), and \( |E| = |E^{-1}| \). 
\label{def:rcdg}
\end{definition}
\noindent This choice stems from the need to ensure that a node $v$ should be assigned to a partition until the assignment is done for all nodes it depends on, which are the source nodes for incoming edges to $v$. However, only outgoing edges are available in the original CDG. By reversing the graph $G$, these nodes become the children of $v$ in the $G^{-1}$. This reversal enables us to conduct a post-order traversal to order nodes topologically, where a node is added to the list once all its children have been visited. 







% 下面就是RCDG的构建过程和Partition算法
\Cref{alg:partition} presents the details of the topological sort-based greedy partition algorithm. It takes a set of constraints $C$ derived from the circuit directly and the expected number $k$ of partitions as inputs. First, it constructs the reversed CDG $G^{-1}$ by invoking function Rev\_CDG$()$ (Line \ref{stmt:rev_cdg_construct}). The Rev\_CDG$()$ initializes an empty Rev-CDG $G^{-1}$ and each constraint in $C$ correspond a node in $G^{-1}$ (Lines \ref{stmt:rev_cdg_init}-\ref{stmt:add_constraint}). Then, each dependency between two constraints is recognized as an edge and represents the data flow. Meanwhile, it also updates the depth and out-degree of the node accordingly (Lines \ref{stmt:add_edge_start}-\ref{stmt:add_edge_end}). 

Next, the nodes in $G^{-1}$ without incoming edges are termed as the root set $L_{root}$, which are the source nodes to perform sorting (Line \ref{stmt:root}). The upper bound $s$ of partition size is set to $\left\lceil {|V|}/{k} \right\rceil$ (Line \ref{stmt:partition_size}). After, it goes into the process of partitioning  (Lines \ref{stmt:root_partition_start}-\ref{stmt:root_partition_end}).  Specifically, it traverses from a node in $L_{root}$ and explores a connected component recursively by function Rec\_Partition $C$ (Lines \ref{stmt:rec_partition_start}-\ref{stmt:rec_partition_end}) to add nodes into the current partition. When visiting children of $v$ recursively, the algorithm accesses each child in the order of their priorities based on the depth and out-degree, as discussed in the heuristic strategies (Lines \ref{stmt:explore_children_start}-\ref{stmt:explore_children_end}). Then node $v$ is assigned to the current partition $V_p$ once all its descendants, the nodes it depends on, are processed (Line \ref{stmt:add_to_partition}). If the size of $V_p$ reaches the threshold $s$, $V_p$ is inserted into $\mathcal{P}$ as a independent partition (Lines \ref{stmt:add_new_partition_start}-\ref{stmt:add_new_partition_end}). Finally, the partition $\mathcal{P}$ is returned. 

This traversal ensures that nodes are assigned to partitions in a way that respects the dependencies captured by the Rec-CDG while prioritizing child nodes based on depth and out-degree reduces shared variables between partitions.








