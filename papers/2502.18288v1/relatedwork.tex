\section{Related Work}
\label{sec:related_work}
\paragraph{\zk protocol and implementation}
A variety of \zk protocols\cite{groth2016size, gabizon2019plonk, chen2023hyperplonk, setty2020spartan, chiesa2020marlin, wahby2018doubly, bunz2018bulletproofs} have been developed to enhance proof efficiency, focusing on reducing proof size and verification time. These innovations have significantly advanced the practicality of zero-knowledge proofs in real-world applications. To further support the practical use of zero-knowledge systems, several frameworks\cite{gnark, libsnark, circom} have been developed, providing tools for both circuit design and proof generation. These frameworks offer high-level APIs for constructing circuits, while handling the underlying cryptographic operations, such as elliptic curve computations and multi-scalar multiplications, ensuring efficient proof generation. Our implementation is based on the Gnark library, a framework that streamlines the development of \zk circuits and the proof generation process.


\paragraph{Optimizations in Proof Computation}  
Many prior works have demonstrated that FFT/NTT computations, an important part of proof computation,  can be grouped and executed in parallel \cite{chen2017big, dai2016cuhe, goey2021accelerating, kim2020accelerating}. This significantly improves computational efficiency and resource utilization. To reduce the complexity of MSM operations in proof computation, approaches such as Pippenger’s \cite{pippenger1976evaluation} and Straus’ \cite{straus1964addition} algorithms have been leveraged, while other works have decomposed MSM into smaller, parallelizable tasks \cite{zhang2021pipezk, ma2023gzkp}. 
With these optimizations, witness generation, in turn, becomes the new bottleneck in the overall process.  Furthermore, Hardware acceleration, especially using GPUs, has also been widely explored \cite{chen2017big, dai2016cuhe, goey2021accelerating, kim2020accelerating, ma2023gzkp, zhang2021pipezk, ji2024accelerating, govindaraju2008high}. These optimizations form a crucial part of improving proof computation performance, and can further be integrated into our framework to enhance the overall system's efficiency.

\paragraph{Memory Reduction}  
Earlier efforts \cite{bootle2022gemini} reduced memory usage by modifying specific protocols, but such approaches were often limited to certain ZKP systems and incurred slower proof generation. Recently, VOLE-based interactive ZKP protocols \cite{baum2021mac, dittmer2020line, weng2021wolverine} have reduced memory requirements, though at the cost of increased bandwidth usage. SPLIT \cite{qi2023split} introduced a novel method for reducing memory bottlenecks by partitioning circuits into smaller subcircuits for sequential execution. However, SPLIT relies on manual circuit partitioning, which limits its scalability and generalization. Instead, our work proposes an automatic partitioning approach that fits more complex circuits. 


\paragraph{Distributed \zk}  
Another line of research has focused on distributed ZKP protocols \cite{wu2018dizk, liu2024pianist, xie2022zkbridge, sang2023automating}, which distribute the workload across multiple machines to alleviate memory pressure and enable large-scale proof generation. However, these protocols face limitations, including inter-machine communication overhead and the complexity of coordinating multiple nodes. 
%In scenarios where distributed resources are unavailable or costly, such approaches become impractical. 
We consider distributed \zk as a valuable extension of our work in the future if our optimization for a single machine is still under demand. 


%, while our design achieves better efficiency and resource utilization for each individual machine.