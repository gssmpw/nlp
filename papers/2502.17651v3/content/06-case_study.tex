We perform a case study to better understand \model{}. Figure~\ref{fig:case_study} illustrates an example.  

In Round 1, two specialized critique agents analyze the generated chart. The visual critique agent detects inconsistencies in axis scaling and missing annotations, while the code critique agent identifies the corresponding code-level issues (e.g. incorrect tick intervals and absent annotations ). Based on these critiques, the revision agent modifies the chart by adjusting the Y-axis scale and adding the missing annotation. These corrections result in a significant improvement, reaching perfect text score, though color score remains unchanged.

In Round 2, the critique agents further refine the chart. The visual critique agent highlights inaccuracies in the color assignments of distributions, noting that the generated chart does not precisely match the reference chart’s colors. The code critique agent pinpoints the exact color discrepancies in the code and provides specific RGB values for correction. The revision agent incorporates these insights, adjusting the color specifications in the code. This final revision achieves perfect alignment with the reference chart, with 100\% f1 score across all evaluation metrics.

This case study demonstrates the effectiveness of \model’s multi-agent collaborative refinement process. By decomposing the task into distinct stages, \model{} can iteratively enhance the generated output. The separation of visual and code critiques ensures that both perceptual and implementation-level issues are systematically identified and addressed. 

