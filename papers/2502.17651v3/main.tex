% This must be in the first 5 lines to tell arXiv to use pdfLaTeX, which is strongly recommended.
\pdfoutput=1
% In particular, the hyperref package requires pdfLaTeX in order to break URLs across lines.

\documentclass[11pt]{article}

% Remove the "review" option to generate the final version.
\usepackage[]{ACL2023}

% Standard package includes
\usepackage{times}
\usepackage{latexsym}
\usepackage{graphicx}
% For proper rendering and hyphenation of words containing Latin characters (including in bib files)
\usepackage[T1]{fontenc}
% For Vietnamese characters
% \usepackage[T5]{fontenc}
% See https://www.latex-project.org/help/documentation/encguide.pdf for other character sets

% This assumes your files are encoded as UTF8
\usepackage[utf8]{inputenc}

% This is not strictly necessary, and may be commented out.
% However, it will improve the layout of the manuscript,
% and will typically save some space.
\usepackage{microtype}

% This is also not strictly necessary, and may be commented out.
% However, it will improve the aesthetics of text in
% the typewriter font.
\usepackage{inconsolata}

% If the title and author information does not fit in the area allocated, uncomment the following
%
%\setlength\titlebox{<dim>}
%
% and set <dim> to something 5cm or larger.

% manually added packages
\usepackage{inconsolata}
\usepackage{graphicx}
\usepackage{booktabs}
\usepackage{multirow}
\usepackage{amsmath}
\usepackage{xcolor}
\usepackage{pdfpages}
\usepackage{afterpage}
\usepackage{stfloats}
\usepackage{xcolor}
\usepackage{amsmath}
\usepackage{xspace}
\usepackage{cleveref}
\usepackage[normalem]{ulem}
\useunder{\uline}{\ul}{}

\usepackage{amsmath}
\usepackage{algorithm}
\usepackage{algpseudocode}

\usepackage{listings} 

\usepackage{todonotes}
\newcommand{\SideNote}[2]{\todo[color=#1,size=\small]{#2}}
\newcommand{\kw}[1]{\SideNote{blue!40}{#1 --kw}}
\newcommand{\violet}[1]{\SideNote{purple!40}{#1 --violet}}


\newcommand{\model}{\textbf{\textcolor{darkgray}{\textsc{METAL}}}}
\newcommand{\gpt}{\textsc{GPT-4o}\xspace}
\newcommand{\llama}{\textsc{LLaMA 3.2-11b}\xspace}


\title{\model{}: A Multi-Agent Framework for Chart Generation \\ with Test-Time Scaling}

% \author{First Author \\
%   Affiliation / Address line 1 \\
%   Affiliation / Address line 2 \\
%   Affiliation / Address line 3 \\
%   \texttt{email@domain} \\\And
%   Second Author \\
%   Affiliation / Address line 1 \\
%   Affiliation / Address line 2 \\
%   Affiliation / Address line 3 \\
%   \texttt{email@domain} \\}

\author{
Bingxuan Li{$^{\dagger}$} \ \ \ Yiwei Wang{$^{\dagger\mathsection}$} \ \ \  Jiuxiang Gu{$^{\ddagger}$} \ \ \  Kai-Wei Chang{$^{\dagger}$} \ \ \  Nanyun Peng{$^{\dagger}$}\\
$^\dagger$  University of California, Los Angeles \quad 
$^\mathsection$ University of California, Merced  \quad 
$^\ddagger$ Adobe Research\\
\texttt{bingxuan@ucla.edu} \\\\
\href{https://metal-chart-generation.github.io}{\textcolor{magenta}{\texttt{https://metal-chart-generation.github.io}}}
}


\begin{document}

\maketitle
\begin{abstract}


% Chart generation aims to generate code to produce the charts that satisfy the desired visual properties, e.g., texts, layout, color, type. 
% It is an important task across diverse professional fields, playing crucial roles in financial analysis, research presentation, education, and healthcare.
% In this work, we empower visual language models (VLMs) to function as intelligent AI \textit{agents} that assist users with limited coding expertise to create high-quality charts.
% \kw{This sentence is weird. Should be "we build AI agents empowered by VLMs to assist ..."}
% We propose \model{} (\textbf{M}ulti-ag\textbf{E}n\textbf{T} fr\textbf{A}mework with vision \textbf{L}anguage models for chart generation), a multi-agent framework that decomposes the complex, multi-modal reasoning process into an iterative collaboration among specialized agents. 
% Our approach improves more than 5.2\% the existing best results in the ChartMIMIC \cite{shi2024chartmimic} dataset, demonstrating that our modality-specific iterative feedback \kw{Need to first define what is modality-specific iterative feedback } can improve chart generation accuracy. 
% Empirical results highlight two key findings in the experiments: (1) There is a near-linear relationship \kw{but the curve will saturate eventually with high budget? I think we need some conditions here. } between the logarithm computational budget and \model{}'s performance, suggesting promising test-time scaling in a multi-agent system. (2) Disentangled self-critique across different modalities significantly enhances the multimodal reasoning capabilities of VLMs.



%%%%%% new version %%%%%%

Chart generation aims to generate code to produce charts satisfying the desired visual properties, e.g., texts, layout, color, and type. It has great potential to empower the automatic professional report generation in financial analysis, research presentation, education, and healthcare.
In this work, we build a vision-language model (VLM) based \textit{multi-agent} framework for effective automatic chart generation.
Generating high-quality charts requires both strong visual design skills and precise coding capabilities that embed the desired visual properties into code.
Such a complex multi-modal reasoning process is difficult for direct prompting of VLMs.
To resolve these challenges, we propose \model{} (\textbf{M}ulti-ag\textbf{E}n\textbf{T} fr\textbf{A}mework with vision \textbf{L}anguage models for chart generation), a multi-agent framework that decomposes the task of chart generation into the iterative collaboration among specialized agents. 
\model{} achieves a 5.2\% improvement in the F1 score over the current best result in the chart generation task.
Additionally, \model{} improves chart generation performance by 11.33\% over Direct Prompting with \llama.
Furthermore, the \model{} framework exhibits the phenomenon of test-time scaling: its performance increases monotonically as the logarithm of computational budget grows from $2^9$ to $2^{13}$ tokens.

% \violet{is "logarithmic computational budget" a well-established concept? If not, I'll avoid using it. Also, what does logarithmic computational budget = 512 mean?}

% \violet{this final sentence is not very important. can probably drop from the abstract.} In addition, we find that separating different modalities during the critique process of \model{} boosts the self-correction capability of VLMs in the multimodal context.



\end{abstract}

\section{Introduction}

% humans are sensitive to the way information is presented.

% introduce framing as the way we address framing. say something about political views and how information is represented.

% in this paper we explore if models show similar sensitivity.

% why is it important/interesting.



% thought - it would be interesting to test it on real world data, but it would be hard to test humans because they come already biased about real world stuff, so we tested artificial.


% LLMs have recently been shown to mimic cognitive biases, typically associated with human behavior~\citep{ malberg2024comprehensive, itzhak-etal-2024-instructed}. This resemblance has significant implications for how we perceive these models and what we can expect from them in real-world interactions and decisionmaking~\citep{eigner2024determinants, echterhoff-etal-2024-cognitive}.

The \textit{framing effect} is a well-known cognitive phenomenon, where different presentations of the same underlying facts affect human perception towards them~\citep{tversky1981framing}.
For example, presenting an economic policy as only creating 50,000 new jobs, versus also reporting that it would cost 2B USD, can dramatically shift public opinion~\cite{sniderman2004structure}. 
%%%%%%%% 图1:  %%%%%%%%%%%%%%%%
\begin{figure}[t]
    \centering
    \includegraphics[width=\columnwidth]{Figs/01.pdf}
    \caption{Performance comparison (Top-1 Acc (\%)) under various open-vocabulary evaluation settings where the video learners except for CLIP are tuned on Kinetics-400~\cite{k400} with frozen text encoders. The satisfying in-context generalizability on UCF101~\cite{UCF101} (a) can be severely affected by static bias when evaluating on out-of-context SCUBA-UCF101~\cite{li2023mitigating} (b) by replacing the video background with other images.}
    \label{fig:teaser}
\end{figure}


Previous research has shown that LLMs exhibit various cognitive biases, including the framing effect~\cite{lore2024strategic,shaikh2024cbeval,malberg2024comprehensive,echterhoff-etal-2024-cognitive}. However, these either rely on synthetic datasets or evaluate LLMs on different data from what humans were tested on. In addition, comparisons between models and humans typically treat human performance as a baseline rather than comparing patterns in human behavior. 
% \gabis{looks good! what do we mean by ``most studies'' or ``rarely'' can we remove those? or we want to say that we don't know of previous work doing both at the same time?}\gili{yeah the main point is that some work has done each separated, but not all of it together. how about now?}

In this work, we evaluate LLMs on real-world data. Rather than measuring model performance in terms of accuracy, we analyze how closely their responses align with human annotations. Furthermore, while previous studies have examined the effect of framing on decision making, we extend this analysis to sentiment analysis, as sentiment perception plays a key explanatory role in decision-making \cite{lerner2015emotion}. 
%Based on this, we argue that examining sentiment shifts in response to reframing can provide deeper insights into the framing effect. \gabis{I don't understand this last claim. Maybe remove and just say we extend to sentiment analysis?}

% Understanding how LLMs respond to framing is crucial, as they are increasingly integrated into real-world applications~\citep{gan2024application, hurlin2024fairness}.
% In some applications, e.g., in virtual companions, framing can be harnessed to produce human-like behavior leading to better engagement.
% In contrast, in other applications, such as financial or legal advice, mitigating the effect of framing can lead to less biased decisions.
% In both cases, a better understanding of the framing effect on LLMs can help develop strategies to mitigate its negative impacts,
% while utilizing its positive aspects. \gabis{$\leftarrow$ reading this again, maybe this isn't the right place for this paragraph. Consider putting in the conclusion? I think that after we said that people have worked on it, we don't necessarily need this here and will shorten the long intro}


% If framing can influence their outputs, this could have significant societal effects,
% from spreading biases in automated decision-making~\citep{ghasemaghaei2024understanding} to reducing public trust in AI-generated content~\citep{afroogh2024trust}. 
% However, framing is not inherently negative -- understanding how it affects LLM outputs can offer valuable insights into both human and machine cognition.
% By systematically investigating the framing effect,


%It is therefore crucial to systematically investigate the framing effect, to better understand and mitigate its impact. \gabis{This paragraph is important - I think that right now it's saying that we don't want models to be influenced by framing (since we want to mitigate its impact, right?) When we talked I think we had a more nuanced position?}




To better understand the framing effect in LLMs in comparison to human behavior,
we introduce the \name{} dataset (Section~\ref{sec:data}), comprising 1,000 statements, constructed through a three-step process, as shown in Figure~\ref{fig:fig1}.
First, we collect a set of real-world statements that express a clear negative or positive sentiment (e.g., ``I won the highest prize'').
%as exemplified in Figure~\ref{fig:fig1} -- ``I won the highest prize'' positive base statement. (2) next,
Second, we \emph{reframe} the text by adding a prefix or suffix with an opposite sentiment (e.g., ``I won the highest prize, \emph{although I lost all my friends on the way}'').
Finally, we collect human annotations by asking different participants
if they consider the reframed statement to be overall positive or negative.
% \gabist{This allows us to quantify the extent of \textit{sentiment shifts}, which is defined as labeling the sentiment aligning with the opposite framing, rather then the base sentiment -- e.g., voting ``negative'' for the statement ``I won the highest prize, although I lost all my friends on the way'', as it aligns with the opposite framing sentiment.}
We choose to annotate Amazon reviews, where sentiment is more robust, compared to e.g., the news domain which introduces confounding variables such as prior political leaning~\cite{druckman2004political}.


%While the implications of framing on sensitive and controversial topics like politics or economics are highly relevant to real-world applications, testing these subjects in a controlled setting is challenging. Such topics can introduce confounding variables, as annotators might rely on their personal beliefs or emotions rather than focusing solely on the framing, particularly when the content is emotionally charged~\cite{druckman2004political}. To balance real-world relevance with experimental reliability, we chose to focus on statements derived from Amazon reviews. These are naturally occurring, sentiment-rich texts that are less likely to trigger strong preexisting biases or emotional reactions. For instance, a review like ``The book was engaging'' can be framed negatively without invoking specific cultural or political associations. 

 In Section~\ref{sec:results}, we evaluate eight state-of-the-art LLMs
 % including \gpt{}~\cite{openai2024gpt4osystemcard}, \llama{}~\cite{dubey2024llama}, \mistral{}~\cite{jiang2023mistral}, \mixtral{}~\cite{mistral2023mixtral}, and \gemma{}~\cite{team2024gemma}, 
on the \name{} dataset and compare them against human annotations. We find  that LLMs are influenced by framing, somewhat similar to human behavior. All models show a \emph{strong} correlation ($r>0.57$) with human behavior.
%All models show a correlation with human responses of more than $0.55$ in Pearson's $r$ \gabis{@Gili check how people report this?}.
Moreover, we find that both humans and LLMs are more influenced by positive reframing rather than negative reframing. We also find that larger models tend to be more correlated with human behavior. Interestingly, \gpt{} shows the lowest correlation with human behavior. This raises questions about how architectural or training differences might influence susceptibility to framing. 
%\gabis{this last finding about \gpt{} stands in opposition to the start of the statement, right? Even though it's probably one of the largest models, it doesn't correlate with humans? If so, better to state this explicitly}

This work contributes to understanding the parallels between LLM and human cognition, offering insights into how cognitive mechanisms such as the framing effect emerge in LLMs.\footnote{\name{} data available at \url{https://huggingface.co/datasets/gililior/WildFrame}\\Code: ~\url{https://github.com/SLAB-NLP/WildFrame-Eval}}

%\gabist{It also raises fundamental philosophical and practical questions -- should LLMs aim to emulate human-like behavior, even when such behavior is susceptible to harmful cognitive biases? or should they strive to deviate from human tendencies to avoid reproducing these pitfalls?}\gabis{$\leftarrow$ also following Itay's comment, maybe this is better in the dicsussion, since we don't address these questions in the paper.} %\gabis{This last statement brings the nuance back, so I think it contradicts the previous parapgraph where we talked about ``mitigating'' the effect of framing. Also, I think it would be nice to discuss this a bit more in depth, maybe in the discussion section.}







\section{Related Works}
\vspace{-0.02in}
We discuss three lines of related work: chart-to-code generation, multi-agent framework, and test-time scaling research.
\vspace{-0.01in}
\subsection{Chart Generation with VLMs}

Chart generation, or chart-to-code generation,  is an emerging task aimed at automatically translating visual representations of charts into corresponding visualization code \cite{shi2024chartmimic, wu2024plot2code}. This task is inherently challenging as it requires both visual understanding and precise code synthesis, often demanding complex reasoning over visual elements.

Recent advances in Vision-Language Models (VLMs) have expanded the capabilities of language models in tackling complex multimodal problem-solving tasks, such as visually-grounded code generation.
% \kw{It's a bit unclear to me what do you mean by multimodal reasoning and how it is related to code generation}
Leading proprietary models, such as GPT-4V~\cite{GPT4V}, Gemini~\cite{Gemini}, and Claude-3~\cite{Claude}, have demonstrated impressive capabilities in understanding complex visual patterns.
% \kw{what do you mean by structured outputs here?} 
The open-source community has contributed models like LLaVA~\cite{xu2024llava-uhd, li2024llavanext-strong}, Qwen-VL~\cite{Qwen-VL}, and DeepSeek-VL~\cite{lu2024deepseekvl}, which provide researchers with greater flexibility for specific applications like chart generation.

Despite these advancements, current VLMs often struggle with accurately interpreting chart structures and faithfully reproducing visualization code. 

\begin{figure*}[ht]
    \centering
    \includegraphics[width=1\linewidth]{figs/method.pdf}
    \caption{Overview of \model{}: A multi-agents system that consists of four specialized agents working in an iterative pipeline: (1) Generation Agent creates initial Python code to reproduce the reference chart, (2) Visual Critique Agent identifies visual discrepancies between the generated and reference charts, (3) Code Critique Agent analyzes the code and provides specific improvement guidelines, and (4) Revision Agent modifies the code based on the critiques. The process iterates until either reaching the verification score or maximum attempts limit.} %Each agent has a clearly defined objective and operates on specific inputs and outputs, enabling systematic improvement of the generated visualization.}
    \vspace{-0.1in}
    \label{fig:method}
\end{figure*}


\subsection{Multi-Agents Framework}

Many researchers have suggested a paradigm shift from single monolithic models to compound systems comprising multiple specialized components~\cite{compound-ai-blog, du2024compositional}. One prominent example is the multi-agent framework.

LLMs-driven multi-agent framework  has been widely explored in various domains, including narrative generation~\cite{huot2024agents}, financial trading~\cite{xiao2024tradingagents}, and cooperative problem-solving~\cite{du2023improving}.

Our work investigates the application of multi-agent framework to the visually-grounded code generation task.



\subsection{Test-Time Scaling} 
% \kw{I feel this subsection is a bit irrelevant }

Inference strategies have been a long-studied topic in the field of language processing. Traditional approaches include greedy decoding \cite{teller2000speech}, beam search \cite{graves2012sequence}, and Best-of-N.

Recent research has explored test-time scaling law for language model inference. For example, \citet{wu2024inference} empirically demonstrated that optimizing test-time compute allocation can significantly enhance problem-solving performance, while \citet{zhang2024scaling} and \citet{snell2024scaling} highlighted that dynamic adjustments in sample allocation can maximize efficiency under compute constraints. Although these studies collectively underscore the promise of test-time scaling for enhancing reasoning performance of LLMs, its existence in other contexts, such as different model types and application to cross-modal generation,  remains under-explored. 


\section{Method}
\label{sec:method}
\section{Synthesizing Attribution Data}

\begin{figure*}[ht]
    \centering
    \includegraphics[width=\textwidth]{img/pipeline.drawio.pdf}
    \caption{\textbf{Top:} The \synatt baseline method for synthetic attribution data generation. Given context and question-answer pairs, we prompt an LLM to identify supporting sentences, which are then used to train a smaller attribution model. However, this discriminative approach may yield noisy training data as LLMs are less suited for classification tasks (see \S\ref{sec:experiments-zero-shot}). \textbf{Bottom:} The \synqa data generation pipeline leverages LLMs' generative strengths through four steps: (1) collection of Wikipedia articles as source data; (2) extraction of context attributions by creating chains of sentences that form hops between articles; (3) generation of QA pairs by prompting an LLM with only these context attribution sentences; (4) compilation of the final training samples, each containing the generated QA pair, its context attributions, and the original articles enriched with related distractors.}
    % \caption{\textbf{Top:} The \synatt baseline. Intuitively, we can prompt an LLM for context-attribution by providing the context and question-answer pairs. Then, we train a smaller model on the obtained synthetic data. However, LLMs are less suitable for discriminative (i.e., classification) tasks, and may yield noisy training data (see \S\ref{sec:experiments-zero-shot}). \textbf{Bottom:} The \synqa data generation pipeline consists of four main steps: (1) collection of Wikipedia articles as the source data; (2) extracting the context attributions by creating chains of sentences that form hops between articles; (3) generation of QA pairs by prompting an LLM with only the context attribution sentences; (4) we obtain the resulting \synqa training sample containing three components: the generated QA pair, the context attributions, and the original articles supplemented with related distractor articles.}
    \label{fig:method}
\end{figure*}

Context attribution identifies which parts of a reference text support a given question-answer pair~\cite{rashkin2023measuring}. Formally, given a question $q$, its answer $a$, and a context text $c$ consisting of sentences ${s_1, ..., s_n}$, the task is to identify the subset of sentences $S \subseteq c$ that fully support the answer $a$ to question $q$. To train efficient attribution models without requiring expensive human annotations, we explore synthetic data generation approaches using LLMs.
% Context attribution poses the following question~\cite{rashkin2023measuring}: given a generated text $t_g$ and a context text $t_c$, is $t_g$ attributable to $t_c$? To train models to perform well on this task, we explore how to best generate synthetic attribution data using LLMs. We implement two methods: a discriminative and generative method. 
We implement two methods for synthetic data generation. Our baseline method (\synatt) is discriminative: given existing question-answer pairs and their context, an LLM identifies supporting sentences, which are then used to train a smaller attribution model. Our proposed method (\synqa) takes a generative approach: given selected context sentences, an LLM generates question-answer pairs that are fully supported by these sentences. This approach better leverages LLMs' natural strengths in text generation while ensuring clear attribution paths in the synthetic training data.

%The first method is relatively straightforward and termed \synatt. A simple way to generate synthetic data for context attribution is to ask an LLM to pick out the sentences that support a given question-answer pair. 

% \subsection{Discriminative and Generative Synthetic Data Generation}

% The first method (\synatt) is relatively straightforward: ask the LLM to pick relevant sentences from a provided context that support a given question-answer pair. However, this \textit{discriminative} approach of performing sentence classification overlooks the fact that LLMs excel at \textit{generating} text. Therefore, we design a second data generation method (\synqa) that is generative and thus capitalizes on the strength of LLMs. It involves the following pipeline steps (see also Fig.~\ref{fig:method}): context collection, question-answering generation and distractor mining, which increases the difficulty of the task, thus reflecting more realistic scenarios.

%\textbf{Attribution Synthesis.} The most straightforward approach to generating synthetic data for context attribution is discriminative: prompting an LLM to identify relevant sentences from context documents given a question-answer pair. While intuitive, this approach underutilizes LLMs' capabilities, as they excel at generative rather than discriminative tasks. LLMs are fundamentally designed to generate coherent text following instructions rather than perform binary classification of sentences. In our experiments (\S\ref{sec:experiments}) we dub this method as \synatt.

\subsection{\synqa: Generative Synthetic Data Generation Method}

\synqa consists of three parts: context selection, QA generation, and distractors mining (for an illustration of the method, see Figure~\ref{fig:method}). In what follows, we describe each part in detail.

\textbf{Context Collection.} We use Wikipedia as our data source, as each article consists of sentences about a coherent and connected topic, with two collection strategies. In the first, we select individual Wikipedia articles for dialogue-centric generation and use their sentences as context. In the second, for multi-hop reasoning, we identify sentences containing Wikipedia links and follow these links to create ``hops'' between articles, limiting to a maximum of two paths to maintain semantic coherence, while enabling more complex reasoning patterns (for more details, see Appendix~\ref{app:synthetic_data}).
% \textbf{Context Collection.}  The first step is to select a dataset where each data point is a set of sentences about a coherent and connected topic. These sentences will serve as the context in which we want to find relevant attributions later. We use Wikipedia as the data source
%To better leverage LLMs' generative capabilities, we propose \synqa, a novel and simple approach for synthesizing context attribution data (see Fig.~\ref{fig:method}). 
%We first collect Wikipedia articles that are not present in our testing datasets\footnote{We detect potential data leakage by representing each Wikipedia article as a MinHash signature. Then, for each training Wikipedia article, we retrieve candidates from the testing datasets via Locality Sensitivity Hashing and compute their Jaccard similarity \cite{dasgupta2011fast}. Pairs exceeding a tunable threshold (empirically set to 0.8) are flagged as potential leaks.}.
%For each article, 
% we implement two distinct collection strategies that differ in difficulty. First, we select individual Wikipedia articles and randomly select multiple sentences within each article. Second, we start from a randomly selected sentence containing at least one Wikipedia link
%\footnote{These are human annotated in the Wikipedia articles, or alternatively, can be obtained from entity linking methods \cite{de-cao-etal-2022-multilingual}.} 
% and follow the links to other articles, creating ``hops'' between related content. We limit the chain to a maximum of two hops (connecting up to three articles) to maintain semantic coherence while enabling the more difficult multi-hop reasoning scenarios (for more details, see Appendix~\ref{app:synthetic_data}). 
%In the second strategy, we select individual Wikipedia articles and randomly select multiple sentences within each article that can serve as evidence for generated questions.

\textbf{Question-Answer Generation.} Given the set of contexts, an LLM can now generate question-answer pairs. For single articles, we prompt the model to generate multiple question-answer pairs, each grounded in specific sentences. This creates a set of dialogue-centric samples where questions build upon the previous context. For linked articles, we prompt the model to generate questions that necessitate connecting information across the articles, encouraging multi-hop reasoning.
%\footnote{Note that multi-hop reasoning is not guranteed here; rather, the LLM has the ability to decide whether the question-answer pair involves multiple hops of reasoning. See App. for details.}. 
This yields multi-hop samples requiring integration of information across documents, as well as samples that mimic a dialogue about a specific topic given the context. We provide the full prompts used for generation in Appendix \ref{app:prompts}.

\textbf{Distractors Mining.} To make the attribution task more realistic, we augment each sample with distractor articles. With E5 \cite{wang2022text}, we embed each Wikipedia article in our collection. For each article in the training sample, we randomly select up to three distractors with the highest semantic similarity to the source articles. These distractors share thematic elements with the source articles, but lack information to answer the questions.%do not contain the information necessary to answer the generated questions.

\subsection{Advantages of \synqa}
The \synqa approach has three key advantages:
%over discriminative data generation:
% (1) it leverages LLMs' natural strength in generative tasks; (2) produces diverse multi-hop reasoning scenarios; and (3) creates coherent question-answer pairs with clear attribution paths.
(1) it leverages LLMs' strength in generation rather than classification; (2) creates diverse training samples requiring both dialogue understanding and multi-hop reasoning; and (3) ensures generated questions have clear attribution paths since they are derived from specific context sentences.
By generating both entity-centric and dialogue-centric samples, \synqa produces training data that reflects the variety of real-world QA scenarios, helping models develop robust attribution capabilities, which our experiments demonstrate to generalize across different contexts and domains.
% We formalize the problem of Context Attribution QA as follows: Given a pre-defined context $T_c=\lbrace s_1, s_2, \ldots , s_n \rbrace$---where $s_i$ is a sentence---and an answer text $t_a$ generated by an LLM, the context attribution model should provide a vector $a=(a_1, \ldots , a_n)$, where each element $a_i$ has the following possible values:
% \[
% a_i =
% \begin{cases}
%     1, & \text{if } s_i \text{ supports the generated answer } t_a\\
%     0,  & \text{otherwise} 
% \end{cases}
% \]
% In our setup, we should have at least one entry $a_i = 1$.
% \begin{itemize}
%     \item The simplest way to generate synthetic data for context-attribution is in a discriminative manner: we prompt an LLM to provide the sentence level context attributions given the context documents, question and answer. We deem this generation as discriminative as the model effectively classifies the sentences that are most relevant to the question-answer pair.
%     \item The issue with this approach is that LLM are not best suitable for discriminative tasks, but rather generative. That is, an LLM is better at generating text by following instructions, than classifing sentences/etc.
%     \item To leverage what LLMs are good for, we create a simple context attribution data generation approach where we perform the following: (1) We find wikipedia articles (which are not contained in the testing datasets)\footnote{Describe the approach for dealing with data leakage}; (2) We select a random sentence in a wikipedia article, and find the links to other wikipedia articles (the hops). We select that sentence, and hop to the other Wikipedia article (given by the link). (3) We perform the hop step for maximum of 2 times (i.e., we connect at most 3 articles, and 1 at least). We end up with 3 Wikipedia articles which constitute the hops.
%     \item We provide Llama70B with either 1 wikipedia article or the hops and ask the model to generate a multi-hop question-answer pair which ideally connects all connected articles, or as many as it can; alternatively, if we provide the model with only 1 wikipedia article, we ask the model to select as many sentences as possible in the article, and for each, generate a question-answer pair (we provide the full prompts we use in Appendix).
%     \item The output of the model is a set of question-answer pairs (or a single one), that is grounded in the evidence provided by the sentence(s). We dub the entire approach as \synqa.
%     \item In summary, we develop two settings to generate synthetic data for context attribution in question answering: one is entity-centric and yield data which might be multi-hop; and the other is dialog-centric where subsequent questions build on top of previous ones.
%     \item Finally, to all context + question + answer + context-attribution samples we add distractors: we obtain embeddings using E5 of each wikipedia page, and for each sample we select up to 3 distractors which we add to the data sample. These distractors are similar are document with similar context as the one from which the context-attributions are.
% \end{itemize}



\section{Experiments}
\label{sec:experiments}
\section{Experimental Study}\label{sec:experiments}
We conduct a comprehensive evaluation across multiple aspects: zero-shot performance, comparison with training on gold attribution data, and generalization to dialogue settings.
% . Our experiments span both in-\textit{isolation} question-answering datasets and in-\textit{dialogue} scenarios,
With our experiments, we shed light on the performance and practical utility of our approach.

% \textcolor{red}{TODO: Should we introduce the two settings separately: in-isolation QA and in-dialogue QA?}

% \textcolor{red}{TODO: We need an introduction sentence for this section.}

% \textcolor{red}{TODO: We mention isolated context attribution in some parts of the paper, while it is not clear how it differs from dialog-based context attribution.}

\subsection{Experimental Setting}
We evaluate model performance using precision (P), recall (R), and F1 score. For each sentence in the LLM's output, the context-attribution models identify the set of context sentences that support that output sentence. Precision measures the proportion of predicted attributions that are correct, while recall measures the proportion of ground truth attributions that are successfully identified.
%F1 is the harmonic mean of precision and recall.

For a fair and comprehensive evaluation, we train all models with a single pass over the training data unless stated otherwise, referring to this setup as \textbf{1P} when needed. For a more controlled comparison, some experiments limit the number of training samples each model encounters. Since the synthetic dataset contains approximately 1.0M samples, we allow models to \textit{observe} an equivalent number of samples from the gold training set, ensuring comparable exposure to models trained on data from \synqa. We refer to this setting as \textbf{1M} when necessary. For all models, we fine-tune only the LoRA parameters (alpha=64, rank=32) using a fixed learning rate of 1e-5 and a weight decay of 1e-3. 

\textbf{In-domain datasets:} We use \squadcolor{SQuAD} \cite{Rajpurkar2016SQuAD1Q} and \hotpotcolor{HotpotQA} \cite{Yang2018HotpotQAAD} as our primary in-domain benchmarks.\footnote{For some experiments (e.g., in Table~\ref{table:zero-shot-models}), these datasets are also \textit{out-of-domain} w.r.t. data generated by \synqa.} SQuAD provides clear sentence-level evidence for answering questions, serving as a strong baseline for direct attribution. HotpotQA introduces multi-hop reasoning, requiring models to link information across multiple sentences (sometimes from different articles) to identify the correct evidence chain. Additionally, HotpotQA includes distractor documents—closely related yet incorrect sources—posing a more challenging but realistic setting for evaluating attribution performance.

%\textcolor{red}{TODO (Kiril): Explain what you do to OR-QUAC, you combine the background with the context?}
\textbf{Out-of-domain datasets:} To assess generalization beyond the training distribution, we evaluate models on \quaccolor{QuAC} \cite{Choi2018QuACQA}, \coqacolor{CoQA} \cite{Reddy2018CoQAAC}, \orquaccolor{OR-QuAC} \cite{qu2020open}, and \doqacolor{DoQA} \cite{campos-etal-2020-doqa}. %\footnote{We consider these datasets as \textit{out-of-domain}, as none of the models we train are exposed to the training data of these datasets.}. 
These datasets present conversational QA scenarios that differ from SQuAD and HotpotQA. Specifically, QuAC and CoQA introduce multi-turn dialogue structures with coreferences, challenging models to track context across multiple turns. This conversational nature creates a methodological challenge: while these datasets are valuable for evaluating dialogue-based attribution, their reliance on conversation history makes direct comparison with models trained on single-turn QA datasets impossible.

To enable comprehensive evaluation across dialogue QA and single-turn QA, we create two versions of each dataset:
\begin{inparaenum}[(i)]
    \item a rephrased version using Llama 70B \cite{Dubey2024TheL3} that converts questions into standalone format for fair comparison with models trained on single-turn context attribution (suffixed by ``-ST''), and
    \item the original version for assessing dialogue-based attribution.
\end{inparaenum}

% To adapt these datasets for isolated context attribution (e.g., such as SQuAD and HotpotQA, where the question-answer pair is standalone),
% \footnote{We refer to isolated context attribution the scenario where the question-answer pair are standalone: i.e., do not contain coreferences.},
% we rephrase question-answer pairs (using Llama 70B), so that coreferencing is unnecessary. However, in dialogue-based settings, we evaluate models on the original, unmodified versions of these datasets.

DoQA extends this challenge further by incorporating domain-specific dialogues (cooking, travel and movies)%\footnote{The domains covered in DoQA are: cooking, travel and movies \cite{campos-etal-2020-doqa}.}
, thus testing the models' adaptability to specialized contexts. OR-QuAC includes %open-retrieval dialogue settings, assesses models' ability to attribute context in less structured environments, adding another layer of complexity to generalization evaluation. \textcolor{red}{TODO (Kiril): Check the papers for these datasets in case something is overlooked here.}
context-independent rewrites of the dialogue questions, such that they can be posed in isolation of prior context (i.e., single-turn QA). This enables us to test the models on their capabilities in both single-turn QA and dialogue QA settings.

\subsection{Methods}
We compare our method (\synqa) against several baselines, including sentence-encoder-based models, zero-shot instruction-tuned LLMs, and models trained on synthetic and gold context-attribution data. Specifically, we experiment with the following methods:

\paragraph{Sentence-Encoders:} We embed each sentence in the context along with the question-answer pair, and select attribution sentences based on cosine similarity with a fixed threshold, tuned on a small validation set.

\paragraph{Zero-shot (L)LMs:} We evaluate various instruction-tuned (L)LMs in a zero-shot manner, as such models have been shown to perform well across a range of NLP tasks \cite{shu2023exploitability,zhang2023instruction}. During inference, we provide an instruction template describing the task to the LLM (see Appendix~\ref{app:prompts} for details).
%as such models have been shown to perform well across a range of NLP tasks.

\paragraph{Ensembles of LLMs:} We aggregate the predictions of multiple LLMs through majority voting, selecting attribution sentences that receive consensus from at least 50\% of the ensemble. In our experiments, we use Llama8B \cite{Dubey2024TheL3}, Mistral7B, and Mistral-Nemo12B \cite{Jiang2023Mistral7} as the ensemble constituents.


\paragraph{Models trained on in-domain gold data:} Fine-tuning on gold-labeled attribution data provides an upper bound on in-domain performance, helping us assess how well synthetic training data generalizes.

\paragraph{\synatt:} \synatt generates synthetic training data by prompting multiple LLMs to perform context attribution in a discriminative manner, aggregating their outputs via majority voting, and training a smaller model on the resulting dataset. To make it a stronger baseline against \synqa, we give the training data of SQuAD and HotpotQA (the context, questions, and answers) to the LLMs and ask them to perform context attribution (note that we do not use the gold attribution). Finally, we train a model on the generated synthetic data.

\paragraph{\synqa:} We train models using synthetic data generated by our proposed method \synqa. Note that even though we train models using \synqa attribution data, we ensure they are not exposed to \textit{any} parts of the evaluation data.\footnote{We identify data leakage by representing each Wikipedia article as a MinHash signature. Then, for each training Wikipedia article, we retrieve candidates from the testing datasets via Locality Sensitivity Hashing and compute their Jaccard similarity \cite{dasgupta2011fast}. We flag as potential leaks pairs exceeding a threshold empirically set to 0.8.}

\subsection{Results and Discussion}
Evaluating our context attribution models requires a multifaceted approach, as performance is influenced by both the quality of training data and the model’s ability to generalize beyond in-domain distributions. Therefore, we design our experiments to address five core questions:
\begin{inparaenum}[(i)]
    \item How well do zero-shot LLMs perform on context-attribution QA tasks (\S\ref{sec:experiments-zero-shot})?
    \item Can models trained on synthetic data generated by \synqa exceed the performance of models trained on gold context-attribution data (\S\ref{sec:experiments-gold})?
    \item To what extent do models generalize to dialogue settings where in-domain training data is unavailable (\S\ref{sec:experiments-dialog})?
    \item How well do models scale in terms of synthetic data quantity generated by \synqa (\S\ref{sec:scalling-trends})?
    \item How do improved context attributions impact the end users' speed and ability to verify questions answering outputs (\S\ref{sec:user-study})?
\end{inparaenum}
% (i) How well do zero-shot LLMs perform context-attribution? (ii) Can synthetic attribution data serve as a viable alternative to gold supervision, particularly in out-of-domain settings? (iii) How do scaling trends affect generalization performance across diverse datasets?

%By systematically comparing models trained on synthetic data to both zero-shot and gold-supervised baselines, we aim to uncover the trade-offs between scalability, performance, and generalization. 
%Collectively, our findings provide a deeper understanding of how synthetic data can be leveraged for context attribution, potentially mitigating the reliance on costly human-annotated datasets.

\subsubsection{Comparison to Zero-Shot Models}\label{sec:experiments-zero-shot}

% Zero-shot v.s. SynQA-trained Models

\begin{table*}[t]
\centering
\resizebox{1.0\textwidth}{!}{
\begin{tabular}{lccccccccccccccc} \toprule
\multirow{2}{*}{Model} & \multirow{2}{*}{Training data} & \multicolumn{3}{c}{\squadcolor{Squad}} & \multicolumn{3}{c}{\hotpotcolor{Hotpot}} & \multicolumn{3}{c}{\quaccolor{Quac-ST}} & \multicolumn{3}{c}{\coqacolor{CoQA-ST}} \\ \cmidrule(lr){3-5} \cmidrule(lr){6-8} \cmidrule(lr){9-11} \cmidrule(lr){12-14}
& & P & R & F1 & P & R & F1 & P & R & F1 & P & R & F1 \\ \midrule
\textbf{\textit{Baselines}} \\
Random & -- & 19.8 & 15.4 & 17.3 & 4.8 & 15.2 & 7.3 & 5.2 & 15.1 & 7.7 & 7.3 & 15.1 & 9.9 \\
E5 | 561M & Zero-shot & 38.1 & 76.5 & 50.9 & 12.4 & 41.4 & 19.1 & 65.0 & 73.8 & 69.1 & 61.1 & 15.2 & 24.4 \\
HF-SmolLM2 | 365M & Zero-shot & 28.1 & 46.4 & 35.0 & 5.1 & 7.3 & 6.0 & 10.6 & 22.6 & 14.4 & 10.6 & 21.5 & 14.2 \\
Llama | 1B & Zero-shot & 37.5 & 62.0 & 46.7 & 5.3 & 28.1 & 8.9 & 8.8 & 65.4 & 15.4 & 11.9 & 52.8 & 19.4 \\
Mistral | 7B & Zero-shot & 71.5 & 94.4 & 81.4 & 42.9 & 42.7 & 42.8 & 63.2 & 88.6 & 73.8 & 59.0 & 72.2 & 64.9 \\
Llama | 8B & Zero-shot & 71.9 & 96.9 & 82.6 & 49.2 & 52.9 & 51.0 & 64.1 & 92.1 & 75.6 & 55.7 & 76.4 & 64.4 \\
Mistral-NeMo | 12B & Zero-shot & 89.5 & 94.5 & 91.8 & 46.4 & 47.3 & 46.8 & 81.8 & 85.3 & 83.5 & 79.0 & 67.2 & 72.6 \\
Ensemble | 27B & Zero-shot & 83.1 & 96.3 & 89.2 & 48.1 & 59.6 & 53.2 & 74.8 & 90.3 & 81.8 & 69.5 & 73.6 & 71.5 \\
Llama | 70B & Zero-shot & 95.3 & 95.6 & 95.5 & 87.6 & 37.5 & 52.5 & 89.7 & 87.8 & 88.7 & \textbf{87.5} & \textbf{73.3} & \textbf{79.8} \\
\midrule
\textbf{\textit{Baselines}} \\
%Llama | 1B & \squadcolor{SQuAD} \& \hotpotcolor{HotpotQA}; \synatt (1P) & 89.8 & 96.5 & 93.0 & 50.6 & 58.6 & 54.3 & 64.9 & 91.5 & 75.9 & 53.1 & 75.5 & 62.3 \\
%Llama | 1B & \squadcolor{SQuAD} \& \hotpotcolor{HotpotQA}; \synatt (1M) & 84.3 & \textbf{96.9} & 90.2 & 54.4 & 58.0 & 56.1 & 63.4 & 92.4 & 75.2 & 52.5 & 77.5 & 62.6 \\ \midrule
Llama | 1B & \synatt (1P) & 89.8 & 96.5 & 93.0 & 50.6 & 58.6 & 54.3 & 64.9 & 91.5 & 75.9 & 53.1 & 75.5 & 62.3 \\
Llama | 1B & \synatt (1M) & 84.3 & \textbf{96.9} & 90.2 & 54.4 & 58.0 & 56.1 & 63.4 & 92.4 & 75.2 & 52.5 & 77.5 & 62.6 \\ \midrule
\textbf{\textit{Ours}} \\
Llama | 1B & \syntheticcolor{\synqa} & \textbf{96.0} & 96.2 & \textbf{96.1} & \textbf{89.6} & \textbf{69.4} & \textbf{78.2} & \textbf{93.3} & \textbf{89.1} & \textbf{91.1} & \underline{82.3} & 68.5 & \underline{74.8} \\
\bottomrule
\end{tabular}
}
\caption{Comparison of zero-shot models and those trained with synthetic data. Larger zero-shot LMs excel, but our \synqa model outperforms all but one for one dataset while being smaller. \textbf{Bold} denotes best method, \underline{underline} if our method is second best. 1P: models trained with a single pass over the training data. 1M: models trained with 1M samples to match the size of the \synqa data.}
\label{table:zero-shot-models}
\end{table*}

In Table~\ref{table:zero-shot-models}, we present the performance of zero-shot models, and models trained without gold context-attribution data. 
%\footnote{Note that the \synatt baseline models are trained using question-answer pairs from SQuaAD and HotpotQA, however, the context-attribution annotations are obtained using an ensemble of LLMs.}. 
State-of-the-art sentence-encoder models (e.g., E5) perform relatively poorly, consistent with prior findings \cite{CohenWang2024ContextCiteAM}. In contrast, LLMs exhibit strong performance, with improvements correlating with model size. Ensembling multiple zero-shot LLMs further enhances performance, leveraging complementary strengths across models, but making the attribution more expensive. We also tested models trained with the discriminative method \synatt. These models significantly outperform their non-fine-tuned counterparts of the same size. However, as postulated, our generative approach \synqa outperforms \synatt significantly in all but one case. Additionally, \synqa surpasses zero-shot LLMs that are orders of magnitude larger, showing that we can train a model that is both more accurate and efficient.

\subsubsection{Comparison to Models Trained on Gold Attribution Data}\label{sec:experiments-gold}

\begin{table*}[t]
\centering
\resizebox{1.0\textwidth}{!}{
\begin{tabular}{lccccccccccccccc} \toprule
\multirow{3}{*}{Model} & \multirow{3}{*}{Training data} 

& \multicolumn{6}{c}{\textbf{In-Domain}} 
& \multicolumn{6}{c}{\textbf{Out-of-Domain}} \\ \cmidrule(lr){3-8} \cmidrule(lr){9-14}

& & \multicolumn{3}{c}{\squadcolor{SQuAD}} & \multicolumn{3}{c}{\hotpotcolor{HotpotQA}} 
& \multicolumn{3}{c}{\quaccolor{QuAC-ST}} & \multicolumn{3}{c}{\coqacolor{CoQA-ST}} \\ \cmidrule(lr){3-5} \cmidrule(lr){6-8} \cmidrule(lr){9-11} \cmidrule(lr){12-14}

& & P & R & F1 & P & R & F1 & P & R & F1 & P & R & F1 \\ \midrule
\textbf{\textit{Baselines}} \\
Llama | 1B & Zero-shot & 37.5 & 62.0 & 46.7 & 5.3 & 28.1 & 8.9 & 8.8 & 65.4 & 15.4 & 11.9 & 52.8 & 19.4 \\
Llama | 1B & \squadcolor{SQuAD} (1P) & 98.4 & 98.4 & 98.4 & 48.7 & 20.0 & 28.4 & 92.6 & 85.8 & 89.0 & 79.9 & 64.3 & 71.2 \\
Llama | 1B & \hotpotcolor{HotpotQA} (1P) & 41.3 & 87.3 & 56.0 & 87.5 & 79.9 & 83.5 & 45.2 & 89.9 & 60.1 & 41.0 & 70.9 & 52.0 \\
Llama | 1B & \squadcolor{SQuAD} \& \hotpotcolor{HotpotQA} (1P) & 98.3 & 98.3 & 98.3 & \textbf{89.7} & 78.9 & 84.0 & 90.4 & 90.0 & 90.2 & 83.1 & 68.0 & 74.8 \\
Llama | 1B & \squadcolor{SQuAD} \& \hotpotcolor{HotpotQA} (1M) & \textbf{98.3} & \textbf{98.4} & \textbf{98.3} & 87.0 & \textbf{85.2} & \textbf{86.1} & 84.0 & 89.2 & 86.6 & 79.2 & 66.4 & 72.2 \\ \midrule
\textbf{\textit{Ours}} \\
Llama | 1B & \syntheticcolor{\synqa} & 96.0 & 96.2 & 96.1 & \underline{89.6} & 69.4 & 78.2 & \underline{93.3} & 89.1 & \underline{91.1} & 82.3 & 68.5 & \underline{74.8} \\
Llama | 1B & \syntheticcolor{\synqa} \& \squadcolor{SQuAD} \& \hotpotcolor{HotpotQA} & \underline{98.2} & \underline{98.3} & \underline{98.2} & 89.3 & \underline{82.4} & \underline{85.8} & \textbf{94.5} & \textbf{92.7} & \textbf{93.6} & \textbf{85.5} & \textbf{71.0} & \textbf{77.6} \\
\bottomrule
\end{tabular}
}
\caption{Comparison of models fine-tuned on synthetic vs.~gold in-domain data. Our \synqa approach generalizes better while remaining competitive in-domain. \textbf{Bold} denotes best method, \underline{underline} our method when second best. 1P: models trained with a single pass over the training data. 1M: models trained with 1M samples to match the size of the \synqa data.}
\label{table:fine-tuned-models}
\end{table*}

In Table~\ref{table:fine-tuned-models}, we compare models trained on synthetic and gold in-domain context-attribution datasets. As expected, fine-tuning on in-domain gold datasets (SQuAD and HotpotQA) yields highly specialized models that perform well on in-domain data.
% The performance on the out-of-domain datasets is comparable to Llama 70B, the best zero-shot LLM.
% In contrast, \synqa models outperform Llama 70B on out-of-domain datasets while also achieving near identical scores on the in-domain datasets.
However, models trained on data obtained by \synqa exhibit competitive performance on in-domain tasks and consistently surpass in-domain-trained models on out-of-domain datasets. 
This strong out-of-domain generalization is crucial for practical deployments, where models must handle diverse, previously unseen contexts that often differ substantially from their training data.

\subsubsection{Comparison to Zero-Shot and Fine-Tuned Models in Dialogue Contexts}\label{sec:experiments-dialog}
% \begin{table}[t]
% \centering
% \resizebox{1.0\columnwidth}{!}{
% \begin{tabular}{lccccccc} 
% \toprule

% \multirow{2}{*}{Model} & \multirow{2}{*}{Training data} & \multicolumn{3}{c}{\quaccolor{QuAC}} & \multicolumn{3}{c}{\coqacolor{CoQA}} \\ 
% \cmidrule(lr){3-5} \cmidrule(lr){6-8}

%  &  & P & R & F1 & P & R & F1 \\ 
% \midrule
% \textbf{\textit{Baselines}} \\
% Llama | 1B & Zero-shot & 20.9 & 47.9 & 29.1 & 35.6 & 40.2 & 37.8 \\
% Mistral | 7B & Zero-shot & 64.9 & 83.9 & 73.2 & 54.4 & 64.9 & 59.2 \\
% Llama | 8B & Zero-shot & 81.4 & 89.0 & 85.0 & 77.8 & 72.1 & 74.8 \\
% Mistral NeMo | 12B & Zero-shot & 84.8 & 85.4 & 85.1 & 81.7 & 68.4 & 74.5 \\
% \midrule
% Llama | 1B & \squadcolor{SQuAD} \& \hotpotcolor{HotpotQA} (1P) & 72.9 & 68.0 & 70.3 & 79.3 & 64.4 & 71.0 \\
% Llama | 1B & \squadcolor{SQuAD} \& \hotpotcolor{HotpotQA} (1M) & 56.0 & 49.0 & 52.3 & 63.0 & 51.2 & 56.5 \\ 
% \midrule
% \textbf{\textit{Ours}} \\
% Llama | 1B & \syntheticcolor{\synqa} & \textbf{91.3} & \underline{91.4} & \underline{91.3} & \underline{81.7} & \underline{71.4} & \underline{76.2} \\
% Llama | 1B & \syntheticcolor{\synqa} \& \squadcolor{SQuAD} \& \hotpotcolor{HotpotQA} & \underline{91.1} & \textbf{92.3} & \textbf{91.7} & \textbf{82.3} & \textbf{73.2} & \textbf{77.5} \\
% \bottomrule
% \end{tabular}
% }
% \caption{Context attribution on QuAC and CoQA (dialog data); both datasets are out-of-domain. Despite the size advantage of zero-shot LLMs, our \synqa models outperform fine-tuned and larger zero-shot models. \textbf{Bold} denotes best method, \underline{underline} our method when second best.}
% \label{table:dialog-datasets}
% \end{table}

\begin{table*}[t]
\centering
\resizebox{1.0\textwidth}{!}{
\begin{tabular}{lcccccccccccccc} 
\toprule

\multirow{3}{*}{Model} & \multirow{3}{*}{Training data} 

& \multicolumn{12}{c}{\textbf{Out-of-Domain}} \\ \cmidrule(lr){3-14}

& & \multicolumn{3}{c}{\quaccolor{QuAC}} & \multicolumn{3}{c}{\coqacolor{CoQA}} 
& \multicolumn{3}{c}{\orquaccolor{OR-QuAC}} & \multicolumn{3}{c}{\doqacolor{DoQA}} \\ 

\cmidrule(lr){3-5} \cmidrule(lr){6-8} \cmidrule(lr){9-11} \cmidrule(lr){12-14}

 &  & P & R & F1 & P & R & F1 & P & R & F1 & P & R & F1 \\ 
\midrule
\textbf{\textit{Baselines}} \\
Llama | 1B & Zero-shot & 30.8 & 45.5 & 36.8 & 39.4 & 37.9 & 38.6 & 33.0 & 46.6 & 38.6 & 12.2 & 22.6 & 15.9 \\
Mistral | 7B & Zero-shot & 76.6 & 81.8 & 79.1 & 67.6 & 61.3 & 64.3 & 82.5 & 85.1 & 83.8 & 74.9 & 77.9 & 76.4 \\
Llama | 8B & Zero-shot & 84.7 & 88.8 & 86.7 & 79.3 & 72.0 & 75.5 & 88.0 & 91.3 & 89.6 & 77.9 & 91.4 & 84.1 \\
Mistral-NeMo | 12B & Zero-shot & 85.7 & 85.4 & 85.5 & 81.9 & 68.4 & 74.5 & 88.9 & 88.8 & 88.8 & 86.0 & 84.2 & 85.1 \\
Llama | 70B & Zero-shot & 88.5 & 87.7 & 88.1 & \textbf{88.3} & \textbf{74.9} & \textbf{81.1} & 81.7 & 86.3 & 83.9 & 85.2 & 82.0 & 83.5 \\
\midrule
\textbf{\textit{Baselines}} \\
Llama | 1B & \squadcolor{SQuAD} \& \hotpotcolor{HotpotQA} (1P) & 71.3 & 66.8 & 69.0 & 79.0 & 64.2 & 70.8 & 61.6 & 57.5 & 59.5 & 67.4 & 57.8 & 62.2 \\
Llama | 1B & \squadcolor{SQuAD} \& \hotpotcolor{HotpotQA} (1M) & 52.6 & 49.3 & 50.9 & 61.2 & 50.2 & 55.2 & 48.5 & 44.6 & 46.5 & 53.2 & 49.1 & 51.1 \\ \midrule
\textbf{\textit{Ours}} \\
Llama | 1B & \syntheticcolor{\synqa} & \textbf{91.3} & \underline{91.4} & \underline{91.3} & 81.7 & 71.4 & 76.2 & \textbf{92.6} & \underline{95.3} & \textbf{94.0} & \textbf{86.3} & \underline{94.5} & \textbf{90.2} \\
Llama | 1B & \syntheticcolor{\synqa} \& \squadcolor{SQuAD} \& \hotpotcolor{HotpotQA} & \underline{91.1} & \textbf{92.2} & \textbf{91.7} & \underline{82.3} & \underline{73.2} & \underline{77.5} & \underline{90.3} & \textbf{96.4} & \underline{93.2} & \underline{85.1} & \textbf{96.0} & \textbf{90.2} \\
\bottomrule
\end{tabular}
}
\caption{Context attribution on QuAC, CoQA, OR-Quac, and DoQA (dialogue data); all datasets are out-of-domain. Despite the size advantage of zero-shot LLMs, our \synqa models outperform fine-tuned and larger zero-shot models. \textbf{Bold} denotes best method, \underline{underline} our method when second best. 1P: models trained with a single pass over the training data. 1M: models trained with 1M samples to match the size of the \synqa data.}
\label{table:dialog-datasets}
\end{table*}

We evaluate dialogue context attribution, for which we do not use any gold in-domain training data (Tab.~\ref{table:dialog-datasets}).
% exists.
Here, models must handle follow-up questions that rely on previous turns, often involving coreferences and other dialogue-specific complexities. As expected, zero-shot LLMs exhibit a strong size-performance correlation, with larger models consistently outperforming smaller ones—even those fine-tuned on single-turn question-answer attribution (trained on gold SQuAD and HotpotQA data). However, fine-tuning smaller models with our synthetic data generation strategy leads to superior performance, surpassing both their fine-tuned counterparts and much larger zero-shot LMs. This demonstrates the effectiveness of \synqa in enhancing context attribution in a dialogue setting and without requiring in-domain supervision.

\subsubsection{Scaling Trends and Generalization Performance}\label{sec:scalling-trends}

\begin{figure*}[t]
    \centering
    \begin{subfigure}{0.48\linewidth}
        \centering
        \includegraphics[width=\linewidth]{img/size_performance.pdf}
        \caption{Model performance vs.~size.}
        \label{fig:size_performance}
    \end{subfigure}
    \hfill
    \begin{subfigure}{0.48\linewidth}
        \centering
        \includegraphics[width=\linewidth]{img/data_quantity.pdf}
        \caption{F1 score vs.~training data size.}
        \label{fig:data_quantity}
    \end{subfigure}
    \caption{Comparison of model performance and scalability. (a) Larger zero-shot models achieve good F1 scores, but our method \synqa (based on Llama 1B) outperforms them while being orders of magnitude smaller. (b) Performance improves consistently with more \synqa training data, highlighting its scalability.}
    \label{fig:combined}
\end{figure*}

Fig.~\ref{fig:size_performance} shows F1 scores averaged across datasets, with model size on the x-axis and performance on the y-axis. Models trained on \synqa-generated data significantly outperform their baseline zero-shot counterparts, while also achieving superior performance compared to zero-shot LLMs that are orders of magnitude larger. This shows our method is highly data-efficient, enabling small models to close the gap with much larger counterparts.



In Figure~\ref{fig:data_quantity}, we analyze model performance as the quantity of synthetic training data increases, reporting F1 scores separately for in-domain and out-of-domain datasets. As we scale data quantity, performance improves consistently across datasets for isolated context attribution. This trend highlights the scalability of our approach, indicating that further gains can be achieved by increasing synthetic data availability.
%Notably, despite the lack of direct supervision on in-domain datasets, more data results in improved performance.%, reinforcing the robustness of our method.

\subsubsection{User Study: \synqa increases efficiency and accuracy assessment}\label{sec:user-study}
We conducted a user study to evaluate the efficiency and accuracy of verifying the correctness of LLM-generated answers using context attribution. Our hypothesis is that higher-quality context attributions, visualized to guide users, facilitate faster and more accurate verification of LLM outputs. Specifically, in each trial, we presented users with a question, a generated answer, and relevant context, along with
% context
attributions visualized as highlights. Their task was to leverage these attributions to judge if the answer was correct w.r.t.~a provided context. See Figure~\ref{fig:user_interface} in Appendix~\ref{app:user_study}.
% for an example. %Appendix~\ref{app:user_study} for an example.

The study compares three scenarios:
\begin{inparaenum}[(i)]
\item 
\textbf{No Alignment:} a baseline condition without context attributions, requiring users to manually read and verify the answer against the entire context;
\item 
\textbf{Llama 1B (Zero-shot):} context attributions generated by the Llama 1B model were visualized;
\item 
\textbf{\synqa}: context attributions generated by our approach were visualized.
\end{inparaenum}

We employed a within-subjects experimental design for our human evaluation (with 12 participants), ensuring that the same participants evaluate all the aforementioned alignment scenarios, thus requiring fewer participants for reliable results \cite{greenwald:1976}. However, this can be susceptible to learning effects where participants perform better in later scenarios, because they learned the task from previous examples. To mitigate this, we counterbalanced the scenario order using a Latin Square design \cite{belz:2010,bradley:1958}, where each alignment scenario appears in each position an equal number of times across all participants. Finally, we randomized the example order within each scenario per participant. For each example, we measured: \textbf{verification time} (seconds from display to judgment submission) and \textbf{verification accuracy} (binary correct/incorrect judgment).

\begin{figure}[hb!]
    \centering
    \includegraphics[width=1.0\linewidth]{img/user_study_big_font.png}
    \caption{Relationship between Evaluation Time (seconds) and Accuracy (\%) for three answer verification settings:  \emph{Llama 1B (Zero-shot)}, \emph{No Alignment} and \synqa. \synqa demonstrates the lowest evaluation time and highest accuracy, indicating its superior performance in facilitating efficient and accurate answer verification.}
    \label{fig:user_study}
\end{figure}

\noindent \textbf{Results.} We observed a clear trend in verification performance across the different attribution settings, with \synqa demonstrating superior effectiveness (Fig.~\ref{fig:user_interface}). \synqa has the lowest average verification time per example (\textbf{148.6} seconds), significantly faster than \emph{No Alignment} (171.8 seconds) and attributions from \emph{Llama 1B} (163.4 seconds). Concurrently, in terms of verification accuracy, \synqa achieved the highest average accuracy (\textbf{86.4\%}). While \emph{No Alignment} (84.1\%) and \emph{Llama 1B (77.3\%)} also yielded reasonable accuracy, attributions from \synqa are clearly of higher quality helping users be more accurate.


% \begin{table*}[t]
% \centering
% \resizebox{1.0\textwidth}{!}{
% \begin{tabular}{lccccccccccccccc} \toprule
% \multirow{2}{*}{Model} & \multirow{2}{*}{Training data} & \multicolumn{3}{c}{\squadcolor{Squad}} & \multicolumn{3}{c}{\hotpotcolor{HotPot QA}} & \multicolumn{3}{c}{\quaccolor{Quac}} & \multicolumn{3}{c}{\coqacolor{CoQA}} \\ \cmidrule(lr){3-5} \cmidrule(lr){6-8} \cmidrule(lr){9-11} \cmidrule(lr){12-14}
% & & P & R & F1 & P & R & F1 & P & R & F1 & P & R & F1 \\ \midrule
% Random & -- & 19.8 & 15.4 & 17.3 & 4.8 & 15.2 & 7.3 & 5.2 & 15.1 & 7.7 & 7.3 & 15.1 & 9.9 \\
% E5 | 561M & Zero-shot & 38.1 & 76.5 & 50.9 & 12.4 & 41.4 & 19.1 & 65.0 & 73.8 & 69.1 & 61.1 & 15.2 & 24.4 \\
% HF-SmolLM2 | 135M & Zero-shot & X & X & X & X & X & X & X & X & X & X & X & X \\
% HF-SmolLM2 | 365M & Zero-shot & 28.1 & 46.4 & 35.0 & 5.1 & 7.3 & 6.0 & 10.6 & 22.6 & 14.4 & 10.6 & 21.5 & 14.2 \\
% Llama | 1B & Zero-shot & 37.5 & 62.0 & 46.7 & 5.3 & 28.1 & 8.9 & 8.8 & 65.4 & 15.4 & 11.9 & 52.8 & 19.4 \\ %\midrule
% Mistral | 7B & Zero-shot & 71.5 & 94.4 & 81.4 & 42.9 & 42.7 & 42.8 & 63.2 & 88.6 & 73.8 & 59.0 & 72.2 & 64.9 \\
% Llama | 8B & Zero-shot & 71.9 & 96.9 & 82.6 & 49.2 & 52.9 & 51.0 & 64.1 & 92.1 & 75.6 & 55.7 & 76.4 & 64.4 \\
% Mistral NeMo | 12B & Zero-shot & 89.5 & 94.5 & 91.8 & 46.4 & 47.3 & 46.8 & 81.8 & 85.3 & 83.5 & 79.0 & 67.2 & 72.6 \\
% Ensemble | 27B & Zero-shot & 83.1 & 96.3 & 89.2 & 48.1 & 59.6 & 53.2 & 74.8 & 90.3 & 81.8 & 69.5 & 73.6 & 71.5 \\
% Llama | 70B & Zero-shot & 95.3 & 95.6 & 95.5 & 87.6 & 37.5 & 52.5 & 89.7 & 87.8 & 88.7 & 87.5 & 73.3 & 79.8 \\
% \midrule
% Llama | 1B & \squadcolor{SQ} \& \hotpotcolor{HP}; Disc. synthetic (1 pass) & 89.8 & 96.5 & 93.0 & 50.6 & 58.6 & 54.3 & 64.9 & 91.5 & 75.9 & 53.1 & 75.5 & 62.3 \\
% Llama | 1B & \squadcolor{SQ} \& \hotpotcolor{HP}; Disc. synthetic (1.0M) & 84.3 & 96.9 & 90.2 & 54.4 & 58.0 & 56.1 & 63.4 & 92.4 & 75.2 & 52.5 & 77.5 & 62.6 \\ \midrule
% Llama | 1B & \synqa (130K no dist.) & 87.1 & 88.2 & 87.6 & 67.9 & 44.8 & 54.0 & 85.3 & 82.5 & 83.9 & 72.7 & 63.8 & 68.0 \\
% Llama | 1B & \syntheticcolor{\synqa (130K)} & 89.9 & 90.7 & 90.3 & 87.9 & 63.5 & 73.7 & 88.7 & 85.4 & 87.0 & 77.2 & 65.5 & 70.9 \\
% Llama | 1B & \syntheticcolor{\synqa (550K)} & 93.6 & 94.5 & 94.0 & 88.9 & 67.3 & 76.6 & 89.8 & 87.9 & 88.9 & 77.1 & 67.1 & 71.8 \\
% Llama | 1B & \syntheticcolor{\synqa (700K)} & 95.1 & 95.5 & 95.3 & 87.8 & 69.6 & 77.6 & 93.6 & 89.1 & 91.3 & 82.0 & 68.9 & 74.9 \\
% Llama | 1B & \syntheticcolor{\synqa (1.0M)} & 96.0 & 96.2 & 96.1 & 89.6 & 69.4 & 78.2 & 93.3 & 89.1 & 91.1 & 82.3 & 68.5 & 74.8 \\
% New & \syntheticcolor{\synqa (1.0M)} & 95.7 & 96.9 & 96.3 & 89.6 & 66.4 & 76.3 & 91.8 & 91.5 & 91.6 & 80.8 & 71.3 & 75.8 \\
% Llama | 1B & \syntheticcolor{\synqa (1.0M with dialog data)} & -- & -- & -- & -- & -- & -- & -- & -- & -- & -- & -- & -- & \\
% \midrule
% HF-SmolLM2 | 135M & \synqa (690K) & X & X & X & X & X & X & X & X & X & X & X & X \\
% HF-SmolLM2 | 365M & \synqa (700K) & 73.9 & 74.2 & 74.1 & 85.2 & 68.7 & 76.0 & 79.6 & 76.5 & 78.0 & 68.8 & 59.8 & 64.0 \\ \midrule
% Llama | 1B & \squadcolor{SQ}; Gold (1 pass) & 98.4 & 98.4 & 98.4 & 48.7 & 20.0 & 28.4 & 92.6 & 85.8 & 89.0 & 79.9 & 64.3 & 71.2 \\
% Llama | 1B & HQ; Gold (1 pass) & 41.3 & 87.3 & 56.0 & 87.5 & 79.9 & 83.5 & 45.2 & 89.9 & 60.1 & 41.0 & 70.9 & 52.0 \\
% Llama | 1B & \squadcolor{SQ} \& \hotpotcolor{HQ}; Gold (1 pass) & 98.3 & 98.3 & 98.3 & 89.7 & 78.9 & 84.0 & 90.4 & 90.0 & 90.2 & 83.1 & 68.0 & 74.8 \\
% Llama | 1B & \squadcolor{SQ} \& \hotpotcolor{HQ}; Gold (1.0M) & 98.3 & 98.4 & 98.3 & 87.0 & 85.2 & 86.1 & 84.0 & 89.2 & 86.6 & 79.2 & 66.4 & 72.2 \\
% Llama | 1B & \syntheticcolor{\synqa (1.0M)} \& \squadcolor{SQ} \& \hotpotcolor{HQ}; Gold (1 pass) & 98.3 & 98.3 & 98.3 & 86.4 & 84.1 & 85.2 & 96.6 & 90.2 & 93.3 & 86.0 & 69.4 & 76.8 \\
% Llama | 1B & \syntheticcolor{\synqa (1.0M)} \& \squadcolor{SQ} \& \hotpotcolor{HQ}; Gold (1 pass) & 98.2 & 98.3 & 98.2 & 89.3 & 82.4 & 85.8 & 94.5 & 92.7 & 93.6 & 85.5 & 71.0 & 77.6 \\
% \midrule
% Llama | 1B & \synqa (1.0M) \& \squadcolor{SQ} \& \hotpotcolor{HQ} \& \quaccolor{Q} \& \coqacolor{CQ}; Gold (1 pass) & -- & -- & -- & -- & -- & -- & -- & -- & -- & -- & -- & -- & \\ \midrule
% HF-Smol | 365M & \syntheticcolor{\synqa (1.0M)} \& \squadcolor{SQ} \& \hotpotcolor{HQ}; Gold (1 pass) & 98.1 & 98.2 & 98.2 & 83.4 & 83.2 & 83.3 & 80.0 & 90.5 & 84.9 & 71.7 & 67.4 & 69.5 \\
% HF-Smol | 365M & \syntheticcolor{\synqa (1.0M)} \& \squadcolor{SQ} \& \hotpotcolor{HQ} \& \quaccolor{Q} \& \coqacolor{CQ}; Gold (1 pass) & 98.0 & 98.1 & 98.1 & 86.8 & 81.7 & 84.2 & 96.6 & 92.9 & 94.7 & 85.6 & 77.0 & 81.1 \\
% HF-Smol | 135M & \syntheticcolor{\synqa (1.0M)} \& \squadcolor{SQ} \& \hotpotcolor{HQ} \& \quaccolor{Q} \& \coqacolor{CQ}; Gold (1 pass) & 98.0 & 98.0 & 98.0 & 77.8 & 76.3 & 77.0 & 95.0 & 91.6 & 93.3 & 81.2 & 71.8 & 76.2 \\
% \bottomrule
% Llama | 1B & All; Gold & 1B   & 96.8 & 96.8 & 96.8 & 88.1 & 83.8 & 85.9 & 94.7 & 89.3 & 91.9 & 88.7 & 76.8 & 82.3 \\
% HF-SmolLM2 & All; Gold & 360M & 98.3 & 98.4 & 98.3 & 85.1 & 78.2 & 81.5 & 96.6 & 92.4 & 94.5 & 88.3 & 74.6 & 80.8 \\ \bottomrule
% \end{tabular}
% }
% \caption{Corroborative context-attribution of LMs on Squad QA, HotPot QA, Quac, and CoQA.}
% \label{table:all-datasets}
% \end{table*}

\section{Analysis and Discussion}
\label{sec:discussion}
\section{Discussion}\label{sec:disc}

While diffusion models can generate highly realistic images, most of the images they produce still contain visible artifacts. In particular, we find that only 17\% of diffusion model-generated images are misclassified as real at rates consistent with random guessing. Notably, this misclassification rate increases to 43\% when the viewing duration is restricted to 1 second. By curating a dataset of 599 images and conducting a large scale digital experiment, we can begin to answer fundamental questions about what drives the appearance of photorealism in diffusion model-generated images. 

First, we find that images with greater scene complexity tend to introduce more opportunities for artifacts to appear, making it easier for participants to detect AI-generated images. Our results reveal that participants were less accurate at identifying AI-generated portraits compared to more complex scenes, such as those involving multiple people in candid settings.  Based on qualitative analysis of the images, we identify three main reasons for this difference. First, portraits often feature a single person against a blurred background, which can obscure details and provide fewer cues compared to full-body or group images. Second, portraits typically involve fewer and simpler poses, focusing only on the face and torso, leaving fewer opportunities for errors or inconsistencies to be apparent. Third, the prevalence of edited and retouched portraits in real-world photography complicates the distinction between real and AI-generated portraits, addressing the question of how subject type and context (e.g., unknown people vs. public figures) influence the perceived authenticity of an image. In contrast, more complex images, like full-body or group shots, involve a greater number of elements, increasing the likelihood of noticeable errors or inconsistencies. Similar to our results on AI-generated images, we find that real images with lower scene complexity are also harder to identify as real. 

Second, we identify five high-level categories of artifacts and implausibilities and find that the easiest images to identify as diffusion model generated are the ones with anatomical implausibilities, such as unrealistic body proportions and stylistic artifacts like overly glossy or waxy features.

Third, by randomizing display time, we identify the relationship between how long an individual looks at an image and their accuracy at distinguishing between real and AI-generated images. Specifically, we find that participants' accuracy at identifying an AI-generated image upon a quick glance of 1 second is 72\% and increases by 5 percentage points with just an additional 4 seconds of viewing time and 10 percentage points when unconstrained by time. Given the nature of rapid scrolling on social media and how much time people have to see advertisements as they pass by billboards on a highway, these results reveal the importance of attentive viewing of images before making judgments about an image's veracity. 


Fourth, we find that human curation had a notable negative impact on participants' accuracy compared to uncurated images generated by the same prompts as the human-curated AI-generated images. In particular, the images curated by our research team were harder to identify as AI-generated than 84\% of the uncurated images generated using the same prompts as the curated images. This finding reveals the limitation of state-of-the-art diffusion models in producing images of consistent quality. It also suggests that human curation is a bottleneck to generating fake images at scale. The process of generating high-quality AI images is inherently iterative---users refine prompts and select outputs until they achieve their desired result. This fundamental aspect of AI image generation is evident across all applications, from advertising and marketing to education and beyond. While concerns exist about fake images being used to mislead or impersonate, many use cases exist for business and educational applications~\cite{vartiainen2023using, hartmann2023power, gvirtz2023text}. The critical role of human curation in this iterative process further emphasizes how the photorealism of images produced by diffusion models depends not only on the capabilities of the diffusion model but also on the quality of human curation, choice of prompts, and context of the scene. Given the importance of these factors beyond the generative AI model, these results reveal the importance of considering these factors in research examining human perception of AI-generated images. Without considering these elements, it is possible to produce biased findings showing AI-generated images are more or less realistic than they really appear in real-world settings. 

The taxonomy offers a practical framework on which to build AI literacy tools for the general public. We synthesized information from diverse sources such as social media posts, scientific literature, and our online behavioral study with 50,444 participants to systematically categorize artifacts in AI-generated images. Through this process, we identify five key categories: anatomical implausibilities, which involve unlikely artifacts in individual body parts or inconsistent proportions, particularly in images with multiple people;  stylistic artifacts, referring to overly glossy, waxy, or picturesque qualities of specific elements of an image; functional implausibilities, arising from a lack of understanding of real-world mechanics and leading to objects or details that appear impossible or nonsensical; violations of physics, which include inconsistencies in shadows, reflections, and perspective that defy physical logic; and sociocultural implausibilities, focusing on scenarios that violate social norms, cultural context, or historical accuracy. Our taxonomy builds upon the Borji 2023 taxonomy \cite{borji2023qualitative} and focuses on images that appear more realistic at first glance, which is useful for comparing and contrasting real photographs with diffusion model generated images for revealing the nuances of the artifacts and implausibilities~\cite{kamali2024distinguish}. Moreover, this taxonomy offers a shared language by which practitioners and researchers can communicate about artifacts commonly seen in AI-generated images and exposes the persistent challenges that can help people identify AI-generated images. 

\subsection{Future Work and Limitations}

In addition to aiding in identifying AI-generated content, the taxonomy offers insights into the open problems for producing realistic AI-generated images. Future work may explore integrating such taxonomies into model evaluation frameworks to provide iterative feedback during the development of generative models. As models advance to address the weaknesses presented in this taxonomy, new and more subtle artifacts may emerge, requiring future updates to this taxonomy. This dynamic interplay between detection and generation capabilities demonstrates why we need to maintain robust human detection abilities even as models evolve. We acknowledge the potential dual use of these insights to create more deceptive synthetic media, and we believe that transparent documentation of artifacts does more good than harm by offering detection strategies and an opportunity to develop general awareness in the public.

Large-scale digital experiments with participants who participate based on their own interests come with certain limitations. First, we did not collect demographic data from participants. Participants were not recruited for this experiment; instead, participants found the experiment organically and participated. Given the organic nature of the participation, we prioritized maximizing engagement, which involves questions unrelated to distinguishing AI-generated and real images like demographic questions. While this approach enabled substantial data collection, it limits analysis by excluding factors like age, gender, and cultural background that may influence detection. 

Second, we provided feedback on the correct answer after each participant made an observation, which has the potential to introduce learning effects. Future research could address these open questions by collecting demographic data to design more inclusive AI literacy tools and evaluating how performance changes with and without feedback. 

This research focused on images generated by state-of-the-art generative models available in 2024, and the findings are inherently tied to the state of diffusion models and generative AI technologies as of 2024. In the future, models are likely to change, and the somewhat visible errors that emerge will also likely change. Past state-of-the-art GAN models such as StyleGAN2~\cite{karras2020analyzingimprovingimagequality} and BigGAN~\cite{brock2018biggan}, often produced more noticeable artifacts in facial features, color balance, and overall photorealism, making their outputs more easily distinguishable. Nonetheless, the current taxonomy on diffusion models points out elements like anatomical implausibilities and stylistic artifacts that can be mapped to the facial feature and color balance cues. These recurring issues offer evidence of the taxonomy’s robustness to differences across model generations, but future studies should explore how the taxonomy may need to adapt to these changes, which may involve adding or removing categories or may involve further identifying nuances within these categories. As an example of how this taxonomy may be applied to AI-generated video, Figure~\ref{fig:sora} presents an example of an anatomical implausibility that we never saw in diffusion model-generated images because it involves a temporal inconsistency. Future research on the realism of AI-generated audio and video may also consider following the three-step process involved in building this taxonomy for images generated by diffusion models. Based on first surveying AI literacy resources, academic literature, and social media, second generating media with state-of-the-art models, and third collecting empirical data on the human ability to distinguish AI-generated media from authentically recorded media, researchers can build empirical insights towards characterizing realism and categorizing the artifacts in AI-generated media.  
 
The empirical insights on the photorealism of AI-generated images and the resulting taxonomy designed to help people better navigate real and synthetic images online lead to a practical research question: How can AI literacy interventions improve people's ability to distinguish real photographs and AI-generated images? Future research may address this question via randomized experiments comparing a control group with no intervention to a treatment group that receives training based on the taxonomy presented in this paper. Likewise, future research may explore this with just-in-time interventions to direct people's attention to the cues identified in the taxonomy.

\section{Case Study}
\label{sec:case_study}
We perform a case study to better understand \model{}. Figure~\ref{fig:case_study} illustrates an example.  

In Round 1, two specialized critique agents analyze the generated chart. The visual critique agent detects inconsistencies in axis scaling and missing annotations, while the code critique agent identifies the corresponding code-level issues (e.g. incorrect tick intervals and absent annotations ). Based on these critiques, the revision agent modifies the chart by adjusting the Y-axis scale and adding the missing annotation. These corrections result in a significant improvement, reaching perfect text score, though color score remains unchanged.

In Round 2, the critique agents further refine the chart. The visual critique agent highlights inaccuracies in the color assignments of distributions, noting that the generated chart does not precisely match the reference chart’s colors. The code critique agent pinpoints the exact color discrepancies in the code and provides specific RGB values for correction. The revision agent incorporates these insights, adjusting the color specifications in the code. This final revision achieves perfect alignment with the reference chart, with 100\% f1 score across all evaluation metrics.

This case study demonstrates the effectiveness of \model’s multi-agent collaborative refinement process. By decomposing the task into distinct stages, \model{} can iteratively enhance the generated output. The separation of visual and code critiques ensures that both perceptual and implementation-level issues are systematically identified and addressed. 



\section{Conclusion}
\section{Conclusion}\label{sec:conclusion}
This work introduces a novel approach to TOT query elicitation, leveraging LLMs and human participants to move beyond the limitations of CQA-based datasets. Through system rank correlation and linguistic similarity validation, we demonstrate that LLM- and human-elicited queries can effectively support the simulated evaluation of TOT retrieval systems. Our findings highlight the potential for expanding TOT retrieval research into underrepresented domains while ensuring scalability and reproducibility. The released datasets and source code provide a foundation for future research, enabling further advancements in TOT retrieval evaluation and system development.

\section*{Limitation}
Our work is not without limitations. First, our \model{} is based on VLMs, which require extensive prompt engineering. Although we selected the best-performing prompts available, it is possible that even more effective prompts could further enhance our results. Second, automatic evaluations have inherent imperfections and may not capture all details in the chart perfectly. We adopted the evaluation metric from previous work to ensure fairness. Third, \model{} has higher costs than direct prompting. Future work could explore how to optimize these costs.

\bibliography{custom}
\bibliographystyle{acl_natbib}

\clearpage
\section*{Appendix}
\label{sec:appendix}
\appendix
\subsection{Lloyd-Max Algorithm}
\label{subsec:Lloyd-Max}
For a given quantization bitwidth $B$ and an operand $\bm{X}$, the Lloyd-Max algorithm finds $2^B$ quantization levels $\{\hat{x}_i\}_{i=1}^{2^B}$ such that quantizing $\bm{X}$ by rounding each scalar in $\bm{X}$ to the nearest quantization level minimizes the quantization MSE. 

The algorithm starts with an initial guess of quantization levels and then iteratively computes quantization thresholds $\{\tau_i\}_{i=1}^{2^B-1}$ and updates quantization levels $\{\hat{x}_i\}_{i=1}^{2^B}$. Specifically, at iteration $n$, thresholds are set to the midpoints of the previous iteration's levels:
\begin{align*}
    \tau_i^{(n)}=\frac{\hat{x}_i^{(n-1)}+\hat{x}_{i+1}^{(n-1)}}2 \text{ for } i=1\ldots 2^B-1
\end{align*}
Subsequently, the quantization levels are re-computed as conditional means of the data regions defined by the new thresholds:
\begin{align*}
    \hat{x}_i^{(n)}=\mathbb{E}\left[ \bm{X} \big| \bm{X}\in [\tau_{i-1}^{(n)},\tau_i^{(n)}] \right] \text{ for } i=1\ldots 2^B
\end{align*}
where to satisfy boundary conditions we have $\tau_0=-\infty$ and $\tau_{2^B}=\infty$. The algorithm iterates the above steps until convergence.

Figure \ref{fig:lm_quant} compares the quantization levels of a $7$-bit floating point (E3M3) quantizer (left) to a $7$-bit Lloyd-Max quantizer (right) when quantizing a layer of weights from the GPT3-126M model at a per-tensor granularity. As shown, the Lloyd-Max quantizer achieves substantially lower quantization MSE. Further, Table \ref{tab:FP7_vs_LM7} shows the superior perplexity achieved by Lloyd-Max quantizers for bitwidths of $7$, $6$ and $5$. The difference between the quantizers is clear at 5 bits, where per-tensor FP quantization incurs a drastic and unacceptable increase in perplexity, while Lloyd-Max quantization incurs a much smaller increase. Nevertheless, we note that even the optimal Lloyd-Max quantizer incurs a notable ($\sim 1.5$) increase in perplexity due to the coarse granularity of quantization. 

\begin{figure}[h]
  \centering
  \includegraphics[width=0.7\linewidth]{sections/figures/LM7_FP7.pdf}
  \caption{\small Quantization levels and the corresponding quantization MSE of Floating Point (left) vs Lloyd-Max (right) Quantizers for a layer of weights in the GPT3-126M model.}
  \label{fig:lm_quant}
\end{figure}

\begin{table}[h]\scriptsize
\begin{center}
\caption{\label{tab:FP7_vs_LM7} \small Comparing perplexity (lower is better) achieved by floating point quantizers and Lloyd-Max quantizers on a GPT3-126M model for the Wikitext-103 dataset.}
\begin{tabular}{c|cc|c}
\hline
 \multirow{2}{*}{\textbf{Bitwidth}} & \multicolumn{2}{|c|}{\textbf{Floating-Point Quantizer}} & \textbf{Lloyd-Max Quantizer} \\
 & Best Format & Wikitext-103 Perplexity & Wikitext-103 Perplexity \\
\hline
7 & E3M3 & 18.32 & 18.27 \\
6 & E3M2 & 19.07 & 18.51 \\
5 & E4M0 & 43.89 & 19.71 \\
\hline
\end{tabular}
\end{center}
\end{table}

\subsection{Proof of Local Optimality of LO-BCQ}
\label{subsec:lobcq_opt_proof}
For a given block $\bm{b}_j$, the quantization MSE during LO-BCQ can be empirically evaluated as $\frac{1}{L_b}\lVert \bm{b}_j- \bm{\hat{b}}_j\rVert^2_2$ where $\bm{\hat{b}}_j$ is computed from equation (\ref{eq:clustered_quantization_definition}) as $C_{f(\bm{b}_j)}(\bm{b}_j)$. Further, for a given block cluster $\mathcal{B}_i$, we compute the quantization MSE as $\frac{1}{|\mathcal{B}_{i}|}\sum_{\bm{b} \in \mathcal{B}_{i}} \frac{1}{L_b}\lVert \bm{b}- C_i^{(n)}(\bm{b})\rVert^2_2$. Therefore, at the end of iteration $n$, we evaluate the overall quantization MSE $J^{(n)}$ for a given operand $\bm{X}$ composed of $N_c$ block clusters as:
\begin{align*}
    \label{eq:mse_iter_n}
    J^{(n)} = \frac{1}{N_c} \sum_{i=1}^{N_c} \frac{1}{|\mathcal{B}_{i}^{(n)}|}\sum_{\bm{v} \in \mathcal{B}_{i}^{(n)}} \frac{1}{L_b}\lVert \bm{b}- B_i^{(n)}(\bm{b})\rVert^2_2
\end{align*}

At the end of iteration $n$, the codebooks are updated from $\mathcal{C}^{(n-1)}$ to $\mathcal{C}^{(n)}$. However, the mapping of a given vector $\bm{b}_j$ to quantizers $\mathcal{C}^{(n)}$ remains as  $f^{(n)}(\bm{b}_j)$. At the next iteration, during the vector clustering step, $f^{(n+1)}(\bm{b}_j)$ finds new mapping of $\bm{b}_j$ to updated codebooks $\mathcal{C}^{(n)}$ such that the quantization MSE over the candidate codebooks is minimized. Therefore, we obtain the following result for $\bm{b}_j$:
\begin{align*}
\frac{1}{L_b}\lVert \bm{b}_j - C_{f^{(n+1)}(\bm{b}_j)}^{(n)}(\bm{b}_j)\rVert^2_2 \le \frac{1}{L_b}\lVert \bm{b}_j - C_{f^{(n)}(\bm{b}_j)}^{(n)}(\bm{b}_j)\rVert^2_2
\end{align*}

That is, quantizing $\bm{b}_j$ at the end of the block clustering step of iteration $n+1$ results in lower quantization MSE compared to quantizing at the end of iteration $n$. Since this is true for all $\bm{b} \in \bm{X}$, we assert the following:
\begin{equation}
\begin{split}
\label{eq:mse_ineq_1}
    \tilde{J}^{(n+1)} &= \frac{1}{N_c} \sum_{i=1}^{N_c} \frac{1}{|\mathcal{B}_{i}^{(n+1)}|}\sum_{\bm{b} \in \mathcal{B}_{i}^{(n+1)}} \frac{1}{L_b}\lVert \bm{b} - C_i^{(n)}(b)\rVert^2_2 \le J^{(n)}
\end{split}
\end{equation}
where $\tilde{J}^{(n+1)}$ is the the quantization MSE after the vector clustering step at iteration $n+1$.

Next, during the codebook update step (\ref{eq:quantizers_update}) at iteration $n+1$, the per-cluster codebooks $\mathcal{C}^{(n)}$ are updated to $\mathcal{C}^{(n+1)}$ by invoking the Lloyd-Max algorithm \citep{Lloyd}. We know that for any given value distribution, the Lloyd-Max algorithm minimizes the quantization MSE. Therefore, for a given vector cluster $\mathcal{B}_i$ we obtain the following result:

\begin{equation}
    \frac{1}{|\mathcal{B}_{i}^{(n+1)}|}\sum_{\bm{b} \in \mathcal{B}_{i}^{(n+1)}} \frac{1}{L_b}\lVert \bm{b}- C_i^{(n+1)}(\bm{b})\rVert^2_2 \le \frac{1}{|\mathcal{B}_{i}^{(n+1)}|}\sum_{\bm{b} \in \mathcal{B}_{i}^{(n+1)}} \frac{1}{L_b}\lVert \bm{b}- C_i^{(n)}(\bm{b})\rVert^2_2
\end{equation}

The above equation states that quantizing the given block cluster $\mathcal{B}_i$ after updating the associated codebook from $C_i^{(n)}$ to $C_i^{(n+1)}$ results in lower quantization MSE. Since this is true for all the block clusters, we derive the following result: 
\begin{equation}
\begin{split}
\label{eq:mse_ineq_2}
     J^{(n+1)} &= \frac{1}{N_c} \sum_{i=1}^{N_c} \frac{1}{|\mathcal{B}_{i}^{(n+1)}|}\sum_{\bm{b} \in \mathcal{B}_{i}^{(n+1)}} \frac{1}{L_b}\lVert \bm{b}- C_i^{(n+1)}(\bm{b})\rVert^2_2  \le \tilde{J}^{(n+1)}   
\end{split}
\end{equation}

Following (\ref{eq:mse_ineq_1}) and (\ref{eq:mse_ineq_2}), we find that the quantization MSE is non-increasing for each iteration, that is, $J^{(1)} \ge J^{(2)} \ge J^{(3)} \ge \ldots \ge J^{(M)}$ where $M$ is the maximum number of iterations. 
%Therefore, we can say that if the algorithm converges, then it must be that it has converged to a local minimum. 
\hfill $\blacksquare$


\begin{figure}
    \begin{center}
    \includegraphics[width=0.5\textwidth]{sections//figures/mse_vs_iter.pdf}
    \end{center}
    \caption{\small NMSE vs iterations during LO-BCQ compared to other block quantization proposals}
    \label{fig:nmse_vs_iter}
\end{figure}

Figure \ref{fig:nmse_vs_iter} shows the empirical convergence of LO-BCQ across several block lengths and number of codebooks. Also, the MSE achieved by LO-BCQ is compared to baselines such as MXFP and VSQ. As shown, LO-BCQ converges to a lower MSE than the baselines. Further, we achieve better convergence for larger number of codebooks ($N_c$) and for a smaller block length ($L_b$), both of which increase the bitwidth of BCQ (see Eq \ref{eq:bitwidth_bcq}).


\subsection{Additional Accuracy Results}
%Table \ref{tab:lobcq_config} lists the various LOBCQ configurations and their corresponding bitwidths.
\begin{table}
\setlength{\tabcolsep}{4.75pt}
\begin{center}
\caption{\label{tab:lobcq_config} Various LO-BCQ configurations and their bitwidths.}
\begin{tabular}{|c||c|c|c|c||c|c||c|} 
\hline
 & \multicolumn{4}{|c||}{$L_b=8$} & \multicolumn{2}{|c||}{$L_b=4$} & $L_b=2$ \\
 \hline
 \backslashbox{$L_A$\kern-1em}{\kern-1em$N_c$} & 2 & 4 & 8 & 16 & 2 & 4 & 2 \\
 \hline
 64 & 4.25 & 4.375 & 4.5 & 4.625 & 4.375 & 4.625 & 4.625\\
 \hline
 32 & 4.375 & 4.5 & 4.625& 4.75 & 4.5 & 4.75 & 4.75 \\
 \hline
 16 & 4.625 & 4.75& 4.875 & 5 & 4.75 & 5 & 5 \\
 \hline
\end{tabular}
\end{center}
\end{table}

%\subsection{Perplexity achieved by various LO-BCQ configurations on Wikitext-103 dataset}

\begin{table} \centering
\begin{tabular}{|c||c|c|c|c||c|c||c|} 
\hline
 $L_b \rightarrow$& \multicolumn{4}{c||}{8} & \multicolumn{2}{c||}{4} & 2\\
 \hline
 \backslashbox{$L_A$\kern-1em}{\kern-1em$N_c$} & 2 & 4 & 8 & 16 & 2 & 4 & 2  \\
 %$N_c \rightarrow$ & 2 & 4 & 8 & 16 & 2 & 4 & 2 \\
 \hline
 \hline
 \multicolumn{8}{c}{GPT3-1.3B (FP32 PPL = 9.98)} \\ 
 \hline
 \hline
 64 & 10.40 & 10.23 & 10.17 & 10.15 &  10.28 & 10.18 & 10.19 \\
 \hline
 32 & 10.25 & 10.20 & 10.15 & 10.12 &  10.23 & 10.17 & 10.17 \\
 \hline
 16 & 10.22 & 10.16 & 10.10 & 10.09 &  10.21 & 10.14 & 10.16 \\
 \hline
  \hline
 \multicolumn{8}{c}{GPT3-8B (FP32 PPL = 7.38)} \\ 
 \hline
 \hline
 64 & 7.61 & 7.52 & 7.48 &  7.47 &  7.55 &  7.49 & 7.50 \\
 \hline
 32 & 7.52 & 7.50 & 7.46 &  7.45 &  7.52 &  7.48 & 7.48  \\
 \hline
 16 & 7.51 & 7.48 & 7.44 &  7.44 &  7.51 &  7.49 & 7.47  \\
 \hline
\end{tabular}
\caption{\label{tab:ppl_gpt3_abalation} Wikitext-103 perplexity across GPT3-1.3B and 8B models.}
\end{table}

\begin{table} \centering
\begin{tabular}{|c||c|c|c|c||} 
\hline
 $L_b \rightarrow$& \multicolumn{4}{c||}{8}\\
 \hline
 \backslashbox{$L_A$\kern-1em}{\kern-1em$N_c$} & 2 & 4 & 8 & 16 \\
 %$N_c \rightarrow$ & 2 & 4 & 8 & 16 & 2 & 4 & 2 \\
 \hline
 \hline
 \multicolumn{5}{|c|}{Llama2-7B (FP32 PPL = 5.06)} \\ 
 \hline
 \hline
 64 & 5.31 & 5.26 & 5.19 & 5.18  \\
 \hline
 32 & 5.23 & 5.25 & 5.18 & 5.15  \\
 \hline
 16 & 5.23 & 5.19 & 5.16 & 5.14  \\
 \hline
 \multicolumn{5}{|c|}{Nemotron4-15B (FP32 PPL = 5.87)} \\ 
 \hline
 \hline
 64  & 6.3 & 6.20 & 6.13 & 6.08  \\
 \hline
 32  & 6.24 & 6.12 & 6.07 & 6.03  \\
 \hline
 16  & 6.12 & 6.14 & 6.04 & 6.02  \\
 \hline
 \multicolumn{5}{|c|}{Nemotron4-340B (FP32 PPL = 3.48)} \\ 
 \hline
 \hline
 64 & 3.67 & 3.62 & 3.60 & 3.59 \\
 \hline
 32 & 3.63 & 3.61 & 3.59 & 3.56 \\
 \hline
 16 & 3.61 & 3.58 & 3.57 & 3.55 \\
 \hline
\end{tabular}
\caption{\label{tab:ppl_llama7B_nemo15B} Wikitext-103 perplexity compared to FP32 baseline in Llama2-7B and Nemotron4-15B, 340B models}
\end{table}

%\subsection{Perplexity achieved by various LO-BCQ configurations on MMLU dataset}


\begin{table} \centering
\begin{tabular}{|c||c|c|c|c||c|c|c|c|} 
\hline
 $L_b \rightarrow$& \multicolumn{4}{c||}{8} & \multicolumn{4}{c||}{8}\\
 \hline
 \backslashbox{$L_A$\kern-1em}{\kern-1em$N_c$} & 2 & 4 & 8 & 16 & 2 & 4 & 8 & 16  \\
 %$N_c \rightarrow$ & 2 & 4 & 8 & 16 & 2 & 4 & 2 \\
 \hline
 \hline
 \multicolumn{5}{|c|}{Llama2-7B (FP32 Accuracy = 45.8\%)} & \multicolumn{4}{|c|}{Llama2-70B (FP32 Accuracy = 69.12\%)} \\ 
 \hline
 \hline
 64 & 43.9 & 43.4 & 43.9 & 44.9 & 68.07 & 68.27 & 68.17 & 68.75 \\
 \hline
 32 & 44.5 & 43.8 & 44.9 & 44.5 & 68.37 & 68.51 & 68.35 & 68.27  \\
 \hline
 16 & 43.9 & 42.7 & 44.9 & 45 & 68.12 & 68.77 & 68.31 & 68.59  \\
 \hline
 \hline
 \multicolumn{5}{|c|}{GPT3-22B (FP32 Accuracy = 38.75\%)} & \multicolumn{4}{|c|}{Nemotron4-15B (FP32 Accuracy = 64.3\%)} \\ 
 \hline
 \hline
 64 & 36.71 & 38.85 & 38.13 & 38.92 & 63.17 & 62.36 & 63.72 & 64.09 \\
 \hline
 32 & 37.95 & 38.69 & 39.45 & 38.34 & 64.05 & 62.30 & 63.8 & 64.33  \\
 \hline
 16 & 38.88 & 38.80 & 38.31 & 38.92 & 63.22 & 63.51 & 63.93 & 64.43  \\
 \hline
\end{tabular}
\caption{\label{tab:mmlu_abalation} Accuracy on MMLU dataset across GPT3-22B, Llama2-7B, 70B and Nemotron4-15B models.}
\end{table}


%\subsection{Perplexity achieved by various LO-BCQ configurations on LM evaluation harness}

\begin{table} \centering
\begin{tabular}{|c||c|c|c|c||c|c|c|c|} 
\hline
 $L_b \rightarrow$& \multicolumn{4}{c||}{8} & \multicolumn{4}{c||}{8}\\
 \hline
 \backslashbox{$L_A$\kern-1em}{\kern-1em$N_c$} & 2 & 4 & 8 & 16 & 2 & 4 & 8 & 16  \\
 %$N_c \rightarrow$ & 2 & 4 & 8 & 16 & 2 & 4 & 2 \\
 \hline
 \hline
 \multicolumn{5}{|c|}{Race (FP32 Accuracy = 37.51\%)} & \multicolumn{4}{|c|}{Boolq (FP32 Accuracy = 64.62\%)} \\ 
 \hline
 \hline
 64 & 36.94 & 37.13 & 36.27 & 37.13 & 63.73 & 62.26 & 63.49 & 63.36 \\
 \hline
 32 & 37.03 & 36.36 & 36.08 & 37.03 & 62.54 & 63.51 & 63.49 & 63.55  \\
 \hline
 16 & 37.03 & 37.03 & 36.46 & 37.03 & 61.1 & 63.79 & 63.58 & 63.33  \\
 \hline
 \hline
 \multicolumn{5}{|c|}{Winogrande (FP32 Accuracy = 58.01\%)} & \multicolumn{4}{|c|}{Piqa (FP32 Accuracy = 74.21\%)} \\ 
 \hline
 \hline
 64 & 58.17 & 57.22 & 57.85 & 58.33 & 73.01 & 73.07 & 73.07 & 72.80 \\
 \hline
 32 & 59.12 & 58.09 & 57.85 & 58.41 & 73.01 & 73.94 & 72.74 & 73.18  \\
 \hline
 16 & 57.93 & 58.88 & 57.93 & 58.56 & 73.94 & 72.80 & 73.01 & 73.94  \\
 \hline
\end{tabular}
\caption{\label{tab:mmlu_abalation} Accuracy on LM evaluation harness tasks on GPT3-1.3B model.}
\end{table}

\begin{table} \centering
\begin{tabular}{|c||c|c|c|c||c|c|c|c|} 
\hline
 $L_b \rightarrow$& \multicolumn{4}{c||}{8} & \multicolumn{4}{c||}{8}\\
 \hline
 \backslashbox{$L_A$\kern-1em}{\kern-1em$N_c$} & 2 & 4 & 8 & 16 & 2 & 4 & 8 & 16  \\
 %$N_c \rightarrow$ & 2 & 4 & 8 & 16 & 2 & 4 & 2 \\
 \hline
 \hline
 \multicolumn{5}{|c|}{Race (FP32 Accuracy = 41.34\%)} & \multicolumn{4}{|c|}{Boolq (FP32 Accuracy = 68.32\%)} \\ 
 \hline
 \hline
 64 & 40.48 & 40.10 & 39.43 & 39.90 & 69.20 & 68.41 & 69.45 & 68.56 \\
 \hline
 32 & 39.52 & 39.52 & 40.77 & 39.62 & 68.32 & 67.43 & 68.17 & 69.30  \\
 \hline
 16 & 39.81 & 39.71 & 39.90 & 40.38 & 68.10 & 66.33 & 69.51 & 69.42  \\
 \hline
 \hline
 \multicolumn{5}{|c|}{Winogrande (FP32 Accuracy = 67.88\%)} & \multicolumn{4}{|c|}{Piqa (FP32 Accuracy = 78.78\%)} \\ 
 \hline
 \hline
 64 & 66.85 & 66.61 & 67.72 & 67.88 & 77.31 & 77.42 & 77.75 & 77.64 \\
 \hline
 32 & 67.25 & 67.72 & 67.72 & 67.00 & 77.31 & 77.04 & 77.80 & 77.37  \\
 \hline
 16 & 68.11 & 68.90 & 67.88 & 67.48 & 77.37 & 78.13 & 78.13 & 77.69  \\
 \hline
\end{tabular}
\caption{\label{tab:mmlu_abalation} Accuracy on LM evaluation harness tasks on GPT3-8B model.}
\end{table}

\begin{table} \centering
\begin{tabular}{|c||c|c|c|c||c|c|c|c|} 
\hline
 $L_b \rightarrow$& \multicolumn{4}{c||}{8} & \multicolumn{4}{c||}{8}\\
 \hline
 \backslashbox{$L_A$\kern-1em}{\kern-1em$N_c$} & 2 & 4 & 8 & 16 & 2 & 4 & 8 & 16  \\
 %$N_c \rightarrow$ & 2 & 4 & 8 & 16 & 2 & 4 & 2 \\
 \hline
 \hline
 \multicolumn{5}{|c|}{Race (FP32 Accuracy = 40.67\%)} & \multicolumn{4}{|c|}{Boolq (FP32 Accuracy = 76.54\%)} \\ 
 \hline
 \hline
 64 & 40.48 & 40.10 & 39.43 & 39.90 & 75.41 & 75.11 & 77.09 & 75.66 \\
 \hline
 32 & 39.52 & 39.52 & 40.77 & 39.62 & 76.02 & 76.02 & 75.96 & 75.35  \\
 \hline
 16 & 39.81 & 39.71 & 39.90 & 40.38 & 75.05 & 73.82 & 75.72 & 76.09  \\
 \hline
 \hline
 \multicolumn{5}{|c|}{Winogrande (FP32 Accuracy = 70.64\%)} & \multicolumn{4}{|c|}{Piqa (FP32 Accuracy = 79.16\%)} \\ 
 \hline
 \hline
 64 & 69.14 & 70.17 & 70.17 & 70.56 & 78.24 & 79.00 & 78.62 & 78.73 \\
 \hline
 32 & 70.96 & 69.69 & 71.27 & 69.30 & 78.56 & 79.49 & 79.16 & 78.89  \\
 \hline
 16 & 71.03 & 69.53 & 69.69 & 70.40 & 78.13 & 79.16 & 79.00 & 79.00  \\
 \hline
\end{tabular}
\caption{\label{tab:mmlu_abalation} Accuracy on LM evaluation harness tasks on GPT3-22B model.}
\end{table}

\begin{table} \centering
\begin{tabular}{|c||c|c|c|c||c|c|c|c|} 
\hline
 $L_b \rightarrow$& \multicolumn{4}{c||}{8} & \multicolumn{4}{c||}{8}\\
 \hline
 \backslashbox{$L_A$\kern-1em}{\kern-1em$N_c$} & 2 & 4 & 8 & 16 & 2 & 4 & 8 & 16  \\
 %$N_c \rightarrow$ & 2 & 4 & 8 & 16 & 2 & 4 & 2 \\
 \hline
 \hline
 \multicolumn{5}{|c|}{Race (FP32 Accuracy = 44.4\%)} & \multicolumn{4}{|c|}{Boolq (FP32 Accuracy = 79.29\%)} \\ 
 \hline
 \hline
 64 & 42.49 & 42.51 & 42.58 & 43.45 & 77.58 & 77.37 & 77.43 & 78.1 \\
 \hline
 32 & 43.35 & 42.49 & 43.64 & 43.73 & 77.86 & 75.32 & 77.28 & 77.86  \\
 \hline
 16 & 44.21 & 44.21 & 43.64 & 42.97 & 78.65 & 77 & 76.94 & 77.98  \\
 \hline
 \hline
 \multicolumn{5}{|c|}{Winogrande (FP32 Accuracy = 69.38\%)} & \multicolumn{4}{|c|}{Piqa (FP32 Accuracy = 78.07\%)} \\ 
 \hline
 \hline
 64 & 68.9 & 68.43 & 69.77 & 68.19 & 77.09 & 76.82 & 77.09 & 77.86 \\
 \hline
 32 & 69.38 & 68.51 & 68.82 & 68.90 & 78.07 & 76.71 & 78.07 & 77.86  \\
 \hline
 16 & 69.53 & 67.09 & 69.38 & 68.90 & 77.37 & 77.8 & 77.91 & 77.69  \\
 \hline
\end{tabular}
\caption{\label{tab:mmlu_abalation} Accuracy on LM evaluation harness tasks on Llama2-7B model.}
\end{table}

\begin{table} \centering
\begin{tabular}{|c||c|c|c|c||c|c|c|c|} 
\hline
 $L_b \rightarrow$& \multicolumn{4}{c||}{8} & \multicolumn{4}{c||}{8}\\
 \hline
 \backslashbox{$L_A$\kern-1em}{\kern-1em$N_c$} & 2 & 4 & 8 & 16 & 2 & 4 & 8 & 16  \\
 %$N_c \rightarrow$ & 2 & 4 & 8 & 16 & 2 & 4 & 2 \\
 \hline
 \hline
 \multicolumn{5}{|c|}{Race (FP32 Accuracy = 48.8\%)} & \multicolumn{4}{|c|}{Boolq (FP32 Accuracy = 85.23\%)} \\ 
 \hline
 \hline
 64 & 49.00 & 49.00 & 49.28 & 48.71 & 82.82 & 84.28 & 84.03 & 84.25 \\
 \hline
 32 & 49.57 & 48.52 & 48.33 & 49.28 & 83.85 & 84.46 & 84.31 & 84.93  \\
 \hline
 16 & 49.85 & 49.09 & 49.28 & 48.99 & 85.11 & 84.46 & 84.61 & 83.94  \\
 \hline
 \hline
 \multicolumn{5}{|c|}{Winogrande (FP32 Accuracy = 79.95\%)} & \multicolumn{4}{|c|}{Piqa (FP32 Accuracy = 81.56\%)} \\ 
 \hline
 \hline
 64 & 78.77 & 78.45 & 78.37 & 79.16 & 81.45 & 80.69 & 81.45 & 81.5 \\
 \hline
 32 & 78.45 & 79.01 & 78.69 & 80.66 & 81.56 & 80.58 & 81.18 & 81.34  \\
 \hline
 16 & 79.95 & 79.56 & 79.79 & 79.72 & 81.28 & 81.66 & 81.28 & 80.96  \\
 \hline
\end{tabular}
\caption{\label{tab:mmlu_abalation} Accuracy on LM evaluation harness tasks on Llama2-70B model.}
\end{table}

%\section{MSE Studies}
%\textcolor{red}{TODO}


\subsection{Number Formats and Quantization Method}
\label{subsec:numFormats_quantMethod}
\subsubsection{Integer Format}
An $n$-bit signed integer (INT) is typically represented with a 2s-complement format \citep{yao2022zeroquant,xiao2023smoothquant,dai2021vsq}, where the most significant bit denotes the sign.

\subsubsection{Floating Point Format}
An $n$-bit signed floating point (FP) number $x$ comprises of a 1-bit sign ($x_{\mathrm{sign}}$), $B_m$-bit mantissa ($x_{\mathrm{mant}}$) and $B_e$-bit exponent ($x_{\mathrm{exp}}$) such that $B_m+B_e=n-1$. The associated constant exponent bias ($E_{\mathrm{bias}}$) is computed as $(2^{{B_e}-1}-1)$. We denote this format as $E_{B_e}M_{B_m}$.  

\subsubsection{Quantization Scheme}
\label{subsec:quant_method}
A quantization scheme dictates how a given unquantized tensor is converted to its quantized representation. We consider FP formats for the purpose of illustration. Given an unquantized tensor $\bm{X}$ and an FP format $E_{B_e}M_{B_m}$, we first, we compute the quantization scale factor $s_X$ that maps the maximum absolute value of $\bm{X}$ to the maximum quantization level of the $E_{B_e}M_{B_m}$ format as follows:
\begin{align}
\label{eq:sf}
    s_X = \frac{\mathrm{max}(|\bm{X}|)}{\mathrm{max}(E_{B_e}M_{B_m})}
\end{align}
In the above equation, $|\cdot|$ denotes the absolute value function.

Next, we scale $\bm{X}$ by $s_X$ and quantize it to $\hat{\bm{X}}$ by rounding it to the nearest quantization level of $E_{B_e}M_{B_m}$ as:

\begin{align}
\label{eq:tensor_quant}
    \hat{\bm{X}} = \text{round-to-nearest}\left(\frac{\bm{X}}{s_X}, E_{B_e}M_{B_m}\right)
\end{align}

We perform dynamic max-scaled quantization \citep{wu2020integer}, where the scale factor $s$ for activations is dynamically computed during runtime.

\subsection{Vector Scaled Quantization}
\begin{wrapfigure}{r}{0.35\linewidth}
  \centering
  \includegraphics[width=\linewidth]{sections/figures/vsquant.jpg}
  \caption{\small Vectorwise decomposition for per-vector scaled quantization (VSQ \citep{dai2021vsq}).}
  \label{fig:vsquant}
\end{wrapfigure}
During VSQ \citep{dai2021vsq}, the operand tensors are decomposed into 1D vectors in a hardware friendly manner as shown in Figure \ref{fig:vsquant}. Since the decomposed tensors are used as operands in matrix multiplications during inference, it is beneficial to perform this decomposition along the reduction dimension of the multiplication. The vectorwise quantization is performed similar to tensorwise quantization described in Equations \ref{eq:sf} and \ref{eq:tensor_quant}, where a scale factor $s_v$ is required for each vector $\bm{v}$ that maps the maximum absolute value of that vector to the maximum quantization level. While smaller vector lengths can lead to larger accuracy gains, the associated memory and computational overheads due to the per-vector scale factors increases. To alleviate these overheads, VSQ \citep{dai2021vsq} proposed a second level quantization of the per-vector scale factors to unsigned integers, while MX \citep{rouhani2023shared} quantizes them to integer powers of 2 (denoted as $2^{INT}$).

\subsubsection{MX Format}
The MX format proposed in \citep{rouhani2023microscaling} introduces the concept of sub-block shifting. For every two scalar elements of $b$-bits each, there is a shared exponent bit. The value of this exponent bit is determined through an empirical analysis that targets minimizing quantization MSE. We note that the FP format $E_{1}M_{b}$ is strictly better than MX from an accuracy perspective since it allocates a dedicated exponent bit to each scalar as opposed to sharing it across two scalars. Therefore, we conservatively bound the accuracy of a $b+2$-bit signed MX format with that of a $E_{1}M_{b}$ format in our comparisons. For instance, we use E1M2 format as a proxy for MX4.

\begin{figure}
    \centering
    \includegraphics[width=1\linewidth]{sections//figures/BlockFormats.pdf}
    \caption{\small Comparing LO-BCQ to MX format.}
    \label{fig:block_formats}
\end{figure}

Figure \ref{fig:block_formats} compares our $4$-bit LO-BCQ block format to MX \citep{rouhani2023microscaling}. As shown, both LO-BCQ and MX decompose a given operand tensor into block arrays and each block array into blocks. Similar to MX, we find that per-block quantization ($L_b < L_A$) leads to better accuracy due to increased flexibility. While MX achieves this through per-block $1$-bit micro-scales, we associate a dedicated codebook to each block through a per-block codebook selector. Further, MX quantizes the per-block array scale-factor to E8M0 format without per-tensor scaling. In contrast during LO-BCQ, we find that per-tensor scaling combined with quantization of per-block array scale-factor to E4M3 format results in superior inference accuracy across models. 






\end{document}
