


Let $\mathbb{R}^d$ denote the data space where the data samples $x_t\in \mathbb{R}^d$. Here the subscript $t\in[0,1]$ denotes inference time such that $p_1(x_1)$ is a target distribution we want to generate, and $p_0(x_0)$ is a base distribution that is easy to sample from. 
Flow-based generative models \citep{lipman_flow_2023,lipman_flow_2024} define a vector field
$v_t(x_t): [0,1] \times \mathbb{R}^d \to \mathbb{R}^d$ that generates a 
probability path $p_t(x_t):[0,1]\times\mathbb{R}^d\to \mathbb{R}_{>0}$  \emph{connecting} the tractable base distribution $p_0(x_0)$ and the target distribution $p_1(x_1)$.
By first sampling $x_0$ from $p_0(x_0)$ and then solving the ordinary differential equation (ODE) 
$\frac{d}{dt}x_t = v_t(x_t)$, one can generate clean samples $x_1\coloneqq x_t|_{t=1}$ that follow the target distribution $p_1(x_1)$.

An efficient way to learn the vector field $v_t(\cdot)$ by a model $v_\theta(\cdot, t)$ is to use flow matching 
\citep{lipman_flow_2023,lipman_flow_2024,tong_improving_2024}.
It works by first finding a conditional vector field $v_{t|z}(x_t|z)$ that generates
a \textit{conditional} probability path $p_t(x_t|z)$, where $z$ denotes sample pairs $(x_0,x_1)$
\footnote{The notation $z$ can also represent $x_1$ alone \citep{lipman_flow_2024}. Our analysis is in the general setting where $z=(x_0,x_1)$, but for ease of interpretation, one may consider $z$ as simply $x_1$.}.
The pairs (couplings) follow the probability distribution of $p(z)=\coupling(x_0,x_1)$\footnote{We use $\pi$ to denote the probability density of data couplings.}. 
We use $(x_0,x_1)$ and $z$ interchangeably throughout the paper.


It has been proved that the \textit{marginal} vector field
\vspace{-5pt}
\begin{equation}\nonumber
\vspace{-5pt}
    v_t(x_t) \coloneqq \int v_{t|z}(x_t|z) p(z|x_t) dz,
\end{equation}
where $p(z|x_t) = \frac{p_t(x_t|z)p(z)}{p(x_t)}$,
will generate the \textit{marginal} probability path $p_t(x_t) = \int p_t(x_t|z) p(z) dz$ \citep{lipman_flow_2023}. 
Thus, one only needs to fit the \textit{marginal} vector field
$v_t(x_t) = \int v_{t|z}(x_t|z) p_t(z|x_t) dz$
using a model $v_\theta (x_t,t)$.

It is intuitive to construct the loss 
\begin{equation}\nonumber
    \mathbb{E}_{t\sim\mathcal{U}(0,1),x_t\sim p(x_t)}
    \big[\|v_\theta (x_t,t) - \underbrace{v_t(x_t)}_\text{intractable}\|^2_2\big],
    \vspace{-5pt}
\end{equation}
which is, unfortunately, intractable. 
Therefore, an equivalent conditional flow matching loss has been proposed
\citep{lipman_flow_2023,tong_improving_2024}:
\vspace{-2.5pt}
\begin{equation}\nonumber
    \mathbb{E}_{t\sim\mathcal{U}(0,1),z\sim p(z),x_t \sim p(x_t|z)} 
    \left[ \left\| v_\theta (x_t,t) - v_{t|z}(x_t|z) \right\|^2_2 \right],
\end{equation}
which is tractable and can be used to train $v_\theta$.
