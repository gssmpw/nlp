







Flow matching has emerged as a prominent class of generative models. It features the ability to use a vector field to transform samples from a source distribution into samples following a target distribution, thus realizing generative modeling \citep{lipman_flow_2023}. The probability distribution the samples follow during the flow is called the probability path. By designing the probability path in a large design space, flow matching has shown improved generative modeling fidelity as well as higher sampling efficiency in a variety of generative modeling tasks including image generation \citep{lipman_flow_2023}, decision-making \citep{zheng_guided_2023}, audio generation, and molecular structure design \citep{gat_discrete_2024,chen_flow_2024,ben-hamu_d-flow_2024}. Flow matching substantially extends diffusion models \citep{ho_denoising_2020,song_score-based_2021}. Most diffusion models leverage the score matching process \citep{song_generative_2019,song_sliced_2020,song_score-based_2021}, inherently limiting them to using the Gaussian distribution as the source distribution to construct a special probability path. Meanwhile, flow matching can learn the mapping between any source distribution and target distributions \citep{lipman_flow_2023,lipman_flow_2024,chen_flow_2024,gat_discrete_2024}.

Guiding flow matching models refers to steering the generated samples toward desired properties, thus sampling from a distribution weighted with some objective function \citep{lu_contrastive_nodate} or conditioned on class labels \citep{song_score-based_2021}. It is vital in many generative modeling applications \citep{song_loss-guided_2023,zheng_guided_2023}, but in contrary to well-studied guidance in diffusion models \citep{song_loss-guided_2023,chung_diffusion_2024,dhariwal_diffusion_2021,song_pseudoinverse-guided_2022,zheng_ensemble_2024,lu_contrastive_nodate,dou_diffusion_2023,trippe_diffusion_2023}, the guidance of flow matching remains less investigated. Most existing guidance methods only apply to a subset of flow matching that assumes the source distribution to be Gaussian and the probability path to have a certain simple form \citep{lipman_flow_2024,zheng_guided_2023,pokle_training-free_2024,anonymous2025energyweighted,anonymous2025flow}. In these cases, it is allowed to simplify the guidance of flow matching to be essentially the same as diffusion model guidance, but flow matching's power of generating more flexible probability paths than diffusion models \citep{tong_improving_2024,chen_flow_2024,gat_discrete_2024} is restricted. 
There have been other controlled generation methods for flow matching, with a notable stream following the paradigm of optimizing some objective functions via differentiating through the sampling process \citep{ben-hamu_d-flow_2024,liu_flowgrad_2023,anonymous2025training}. However, their goal differs from our guidance of weighting the generated distribution.
Therefore, the guidance for flow-matching models remains unrevealed in the rest of the ample design space. 

To fill this gap, in this work, we start from a similar assumption as diffusion guidance and propose a general framework of flow matching guidance. From the perspective of this framework, we propose \emph{Monte Carlo-based training-free asymptotically exact guidance} for flow matching. We also propose different training losses for exact \emph{training-based guidance}, one of which covers existing losses as special cases \citep{lu_contrastive_nodate,anonymous2025energyweighted}. For \emph{approximate guidance methods}, we can theoretically derive from our framework many famous guidance methods that have appeared in the literature, including DPS \cite{chung_diffusion_2024}, $\Pi$GDM \cite{song_pseudoinverse-guided_2022}, LGD \citep{song_loss-guided_2023}, as well as their flow-matching extensions that are theoretically justified for general flow matching models. We demonstrate the effectiveness of our proposed method in both synthetic datasets and decision-making (offline RL) benchmarks. Furthermore, more extensive experiments are conducted on image inverse problems to provide an empirical comparison of different types of guidance methods for a guideline of choosing different methods.


We summarize our contributions as follows:
\textbf{(1)} We propose a theoretically justified unified framework to construct guidance for general flow matching, \emph{i.e.,} with arbitrary source distribution, coupling, and conditional paths.
\textbf{(2)} The framework inspires us to propose a family of new guidance methods, including Monte Carlo sampling-based asymptotically exact guidance and training-based exact guidance for flow matching.
\textbf{(3)} The framework can exactly cover multiple classical guidance methods in flow matching and diffusion models. Contrary to previous derivations relying on the flow to have a Gaussian source distribution, our derivation provides theoretical justification of these methods for general flow matching.
\textbf{(4)} Empirical comparisons between guidance methods are conducted in different tasks, providing insights into choosing appropriate guidance methods for different generative modeling tasks.
