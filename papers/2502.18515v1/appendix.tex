\section{Agent Interaction Examples}
\label{sec:appendix_examples}

This appendix provides examples of the initial interactions between the agents in the smart contract repair pipeline. Specifically, it showcases the Auditor's response to an input smart contract and the Architect's subsequent plan based on the Auditor's report.


\begin{boxE}
\subsection{Example Scenario: Unchecked Return Vulnerability}

For this example, the following vulnerable Move code snippet was provided as input to the Auditor:

\begin{boxB}
\textbf{Vulnerability Analysis}

The issue in Listing 1 stems from the caller function at Line 3 ignoring the first return value (bool) from the vector\:\:index\_of function. This leads to an ambiguous situation when the returned \_index is 0.  It's impossible to determine if the element e was found at index 0 or not found at all. This ambiguity can lead to incorrect program logic and potentially exploitable conditions depending on how the index is subsequently used.

\textbf{Solution}

To fix this vulnerability, we must always check all return values of a function and handle each return case appropriately. In this specific example, we need to capture the boolean return value and use it to distinguish between the two scenarios: element found or not found.
\end{boxB}


\begin{boxL}
module MoveScan::unchecked\_return {
    use 0x1::vector;

    public fun example<T>(v: \&vector<T>, e: \&T) {
        let (found, index) = vector::index\_of(v, e);

        if (found) {
            // Element found at index 'index'
            // ... handle the case where the element is found ...
        } else {
            // Element not found
            // ... handle the case where the element is not found ...
        }
    }
}

module 0x1::vector {
    public fun index\_of<T>(v: \&vector<T>, e: \&T): (bool, u64) {...}
    // Returns (true, index) if the element exists in the vector;
    // otherwise, returns (false, 0).
}\end{boxL}
\end{boxE}

\begin{boxE}
    \subsection{ Infinite Loop (Inf. Loop)}
    \begin{boxB}
        \textbf{Vulnerability Analysis}

The code in Listing 2 demonstrates an infinite loop because the loop counter variable i is never incremented within the while loop.

Missing Counter Update: The while loop condition is i < ADDR\_LENGTH. However, there's no statement within the loop's body to modify the value of i.
Constant Condition: Since i is presumably initialized to a value less than ADDR\_LENGTH before the loop begins (though this initialization isn't shown in the provided snippet), the condition i < ADDR\_LENGTH will always remain true.
Consequences: As you described, this leads to:
Blocked Execution: Code following the loop will never execute.
Performance Degradation: The blockchain wastes resources executing a useless loop.
Gas Waste: The caller incurs gas fees for a failed transaction.
Solution

The solution is to correctly increment the loop counter i within each iteration to ensure the loop eventually terminates.
    \end{boxB}
    \begin{boxL}
        module Example::infinite\_loop\_fix {
    use 0x1::vector;

    const ADDR\_LENGTH: u64 = 32; // Example address length (adjust as needed)

    public fun new(bytes: vector<u8>): vector<u8> {
        let new\_bytes = vector::empty<u8>();
        let i: u64 = 0; // Initialize the loop counter

        while (i < ADDR\_LENGTH) {
            // Ensure 'i' is within bounds of 'bytes'
            if (i < vector::length(\&bytes)) {
              vector::push\_back(\&mut new\_bytes, *vector::borrow(\&bytes, i));
            };
            i = i + 1; // Increment the loop counter
        };

        new\_bytes // Return the new vector
    }
}
    \end{boxL}
\end{boxE}