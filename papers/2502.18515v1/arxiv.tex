\documentclass{article}
\usepackage[dvipsnames]{xcolor}
\usepackage{arxiv}
% The preceding line is only needed to identify funding in the first footnote. If that is unneeded, please comment it out.
\usepackage{cite}
\usepackage{amsmath,amssymb,amsfonts}
\usepackage{algorithmic}
\usepackage{graphicx}
\usepackage{textcomp}
\usepackage{tikz}
\usepackage{pgfplots}
\usepackage{pgf-pie}  
\usepackage{tcolorbox}
\usepackage{enumitem}
\usepackage{algorithm}
\usepackage{listings}
\usetikzlibrary{shapes.geometric, arrows, positioning}
\usepackage{array}
\usepackage{booktabs}
\usepackage{siunitx}
\usepackage{hyperref}
\usepackage{tabularx}
\usepackage{ragged2e}
\usepackage{makecell}
\usepackage{balance}
\tcbuselibrary{skins}

\usepackage{enumitem}
\setlist[itemize]{align=parleft,left=0pt..1em}

\usepackage{titlesec}



\titlespacing{\section}{0pt}{\parskip}{-\parskip}
\titlespacing{\subsection}{0pt}{\parskip}{-\parskip}
\titlespacing{\subsubsection}{0pt}{\parskip}{-\parskip}


\def\BibTeX{{\rm B\kern-.05em{\sc i\kern-.025em b}\kern-.08em
    T\kern-.1667em\lower.7ex\hbox{E}\kern-.125emX}}
\newcommand{\sln}{Smartify}


\pgfplotsset{compat=1.18}
\def\BibTeX{{\rm B\kern-.05em{\sc i\kern-.025em b}\kern-.08em
    T\kern-.1667em\lower.7ex\hbox{E}\kern-.125emX}}

   
\begin{document}

\title{Smartify: A Multi-Agent Framework for Automated Vulnerability Detection and Repair in Solidity and Move Smart Contracts}
\author{ \href{https://orcid.org/0000-0000-0000-0000}{\includegraphics[scale=0.06]{orcid.pdf}\hspace{1mm}Rabimba Karanjai} \\
	University Of Houston\\
	\texttt{rkaranjai@uh.edu} \\
	%% examples of more authors
	\And
	{Sam Blackshear} \\
	Mysten Labs\\
	\texttt{sam@mystenlabs.com} \\
    	\And
	{Lei Xu} \\
	Kent State University\\
	\texttt{xuleimath@gmail.com} \\
    	\And
	{Weidong Shi} \\
	University Of Houston\\
	\texttt{wshi3@Central.UH.EDU} \\
}

\maketitle

\begin{abstract}

The rapid growth of the blockchain ecosystem and the increasing value locked in smart contracts necessitate robust security measures. While languages like Solidity and Move aim to improve smart contract security, vulnerabilities persist. This paper presents \sln{}, a novel multi-agent framework leveraging Large Language Models (LLMs) to automatically detect and repair vulnerabilities in Solidity and Move smart contracts.  Unlike traditional methods that rely solely on vast pre-training datasets, Smartify employs a team of specialized agents working on different specially fine tuned LLMs to analyze code based on underlying programming concepts and language-specific security principles.  We evaluated \sln{} on a dataset for Solidity and a curated dataset for Move, demonstrating its effectiveness in fixing a wide range of vulnerabilities. Our results show that \sln{} (Gemma2+codegemma) achieves state-of-the-art performance, surpassing existing LLMs and even enhancing the capabilities of general-purpose models, such as Llama 3.1. Notably, Smartify can incorporate language-specific knowledge, such as the nuances of Move, without requiring massive language-specific pre-training datasets. This work offers a detailed analysis of various LLMs' performance on smart contract repair, highlighting the strengths of our multi-agent approach and providing a blueprint for developing more secure and reliable decentralized applications in the growing blockchain landscape. We also provide a detailed recipe for extending this to other similar use cases.
\end{abstract}

\begin{keywords}
Smart Contracts, Vulnerability Detection, Code Repair, Large Language Models, Blockchain Security, Move, Solidity
\end{keywords}

\section{Introduction}

Smart contracts, self-executing agreements with terms directly written into code, have emerged as a cornerstone of blockchain technology \cite{nath2014web,ray2023web3}. Their ability to automate transactions and eliminate intermediaries has led to widespread adoption in various sectors, including finance, supply chain management, and healthcare \cite{zheng2018blockchain,karanjai2021conditional,kaleem2021event}. However, the increasing complexity of smart contracts has given rise to a growing concern: security vulnerabilities\cite{vacca2021systematic}. These vulnerabilities, often stemming from coding errors or design flaws, can be exploited by malicious actors, leading to significant financial losses and damage to the reputation of blockchain projects. 

The financial implications of smart contract vulnerabilities are substantial. Reports indicate that cumulative losses from attacks against Ethereum smart contracts alone have exceeded USD 3.1 billion as of 2023~\cite{li2023smart}. In the DeFi space, an estimated \$9.04 billion has been stolen due to vulnerabilities~\cite{wronka2023financial}. Notable incidents like the DAO hack of 2016, resulting in a \$55 million loss~\cite{popper2016hacking}, and the Poly Network hack in 2021, where over \$600 million was stolen~\cite{polyhack}, underscore the critical need for robust security measures. 

Traditional security auditing methods, while essential, often face limitations in terms of accuracy and scalability. This has spurred the exploration of automated techniques for vulnerability detection \cite{10.1145/3238147.3238177,wang2020contractward}and repair, with Large Language Models (LLMs) emerging as a promising solution \cite{joshi2023repair}. LLMs, trained on vast datasets of code, can learn to understand and generate code that adheres to specific programming paradigms and best practices. However, most of the the tools available for smart contracts are very language-specific, mostly relying on Solidity as the language of choice, as well as often sometimes requiring compiled bytecode for scanning for other languages\cite{song2024empirical}.

Apart from Solidity~\cite{dannen2017solidity}, Move~\cite{blackshear2019move} has gained significant traction lately due to its strong focus on security. Its cutting-edge features,
including a custom data type for secure operations and robust access controls via Move modules, and unique memory safety features~\cite{blackshear2022move} have been
particularly noteworthy. Moreover, the Move Prover, a native security framework, provides an additional layer of
protection \cite{dill2022fast}. Notably, several prominent blockchain platforms, such as Starcoin~\cite{starcoin}, Aptos~\cite{devaptos}, and Sui~\cite{blackshear2024sui}, have already adopted Move.

However, despite its promising architecture, the real-world security performance of Move modules remains largely
untested. Unlike Solidity-based smart contracts, which have been extensively studied through empirical research
and surveys, there is a scarcity of research focused specifically on Move modules. Although some methodologies
have been proposed for identifying defects in Move modules or conducting formal verification \cite{keilty2022model,park2024securing}, and empirical analysis\cite{song2024empirical}, a significant knowledge gap persists. Specifically, large-scale investigations into the frequency of defects
in real-world Move modules and identifying potential vulnerabilities and repairing them are lacking, highlighting the need for further research in this area.

This paper proposes a novel framework for detecting and repairing vulnerabilities in smart contracts, focusing on the Solidity and Move languages from a programming language perspective. Our hypothesis relies on understanding the code and preventing known bad practices and unsafe code from being written before even compilation to prevent vulnerability. Our approach leverages the power of a multi-LLM agent system, combining the strengths of explanation and repair models. Our framework, \sln{}, leverages a multi-agent LLM framework to understand, critique, and repair code based on previously learned vulnerabilities as well as propose patches to repair them. 
%
By integrating an LLM specialized in code explanation with another focused on code repair, we aim to improve the accuracy and efficiency of the vulnerability remediation process.

We try to answer the following research questions in this paper, related to software engineering using AI agents and in the landscape of complex smart contract reasoning.
\begin{itemize}
    \item \textbf{RQ1:} Do the present state-of-the-art LLMs can explain a Smart Contract code correctly?
    \item \textbf{RQ2:} Can they detect and explain bad coding practices or specific mistakes leading to bugs or vulnerabilities in a smart contract code?
    \item \textbf{RQ3:} Can we encode programming language-specific knowledge to train the LLMs to understand unsafe and buggy codes in detail enough to repair them?
    \item \textbf{RQ4:} Does the proposed post-training framework be generalized to a larger set of pre-trained LLMs?
\end{itemize}

The key contributions of this paper are as follows:
\begin{enumerate}
    \item We introduce \sln{}, a multi-agent LLM code detection and repair framework that can analyze and repair codes based on coding concepts instead of just using the vast amount of codes for pre-training.
    \item We propose a method that can encompass programming language-specific paradigms for smart contracts, both for established language like Solidity and low resource language like Move, without the need for significantly large pertaining dataset. 
    \item We give a detailed recipe for how this can be scaled for other languages and give a comprehensive evaluation of \sln{}'s efficacy for other pre trained LLMs.
    \item We introduce, implement, and evaluate our framework on generalized pre trained LLMs to show the efficacy of our framework. We evaluate the performance of our framework and various LLMs on a diverse set of vulnerabilities in Solidity and Move smart contracts.
    \item We provide a detailed analysis of the results, identifying the strengths and weaknesses of different approaches and highlighting the challenges in automated code repair.
\end{enumerate}

\section{Related Work}

% The growing importance of smart contracts in the blockchain ecosystem has spurred significant research into their security and the application of advanced techniques for vulnerability detection and repair. 
This section reviews related work in smart contract vulnerabilities, security auditing tools, traditional code repair techniques, and the emerging use of Large Language Models (LLMs) for code repair, particularly in the context of Solidity and Move.

\subsection{Smart Contract Vulnerabilities}

Smart contracts, while offering automation and trustless execution, are prone to security vulnerabilities due to their complex code, immutable nature, and the decentralized environment they operate in~\cite{sharma2023mixed,de2024vulnerability,2025arXiv250104600B}. Exploiting these vulnerabilities can lead to severe financial losses, service disruptions, and loss of trust in decentralized applications~\cite{q50t-pw43-24}. Common vulnerabilities include:

\textbf{Reentrancy:} This occurs when a malicious contract calls back into the original contract before the first function invocation completes\cite{so2023smartfix,tang2023deep}. This can disrupt control flow, allowing attackers to repeatedly execute a vulnerable function, potentially draining funds or manipulating the contract's state~\cite{tang2023deep,deng2023smart}. 

% Mitigation strategies include employing the checks-effects-interactions pattern and using mutual exclusion locks (mutexes)\cite{tang2023deep,deng2023smart}.

\textbf{Integer Overflow/Underflow:} These vulnerabilities arise when arithmetic operations result in values exceeding the maximum or falling below the minimum representable value for the integer type. Before Solidity 0.8.0, these errors wrapped around silently, leading to unexpected behavior. 
% Mitigation involves using Solidity 0.8.0 or later, which has built-in checks, or employing SafeMath libraries.

\textbf{Access Control Issues:} Insufficient or improperly implemented access control can allow unauthorized users to interact with sensitive functions or data. 

% Common mistakes include failing to restrict access to administrative functions. Robust access control mechanisms using modifiers and thorough testing are crucial mitigations.

\textbf{Front-Running:} This exploits the transparency of pending transactions. Attackers observe a pending transaction, craft a transaction with a higher gas price, and get it included in the next block first, gaining an unfair advantage.
% .
% Mitigation includes commit-reveal schemes, time-delayed execution, or using decentralized exchanges (DEXs) with built-in protection.

% \textbf{Denial of Service (DoS):} DoS attacks aim to make a contract unavailable by consuming all available gas or causing transactions to continually fail, preventing legitimate users from interacting with the contract. 

% Mitigation involves implementing gas limits on functions and avoiding loops over unbounded data structures.

\textbf{Oracle Manipulation:} Smart contracts often rely on external data sources (oracles). Attackers can compromise oracle integrity, manipulating data fed to the contract. Using multiple independent oracles and decentralized oracle networks can mitigate this risk.

These vulnerabilities underscore the importance of rigorous security analysis and testing during smart contract development and deployment.

\subsection{Smart Contract Security Auditing}

Various tools and techniques have been developed for detecting vulnerabilities in smart contracts:

\textbf{Static Analysis Tools:} Tools like Mythril~\cite{muellerfile} and Slither~\cite{feist2019slither} analyze contract source code to identify potential vulnerabilities. They perform symbolic execution and taint analysis to detect patterns associated with common vulnerabilities.

\textbf{Dynamic Analysis Tools:} Tools like Manticore~\cite{mossberg2019manticore}and Echidna~\cite{grieco2020echidna} execute contracts with various inputs to uncover runtime errors. They use fuzzing and symbolic execution techniques to explore different execution paths and identify potential issues.

\textbf{Formal Verification:} This approach uses mathematical techniques to rigorously prove the correctness of a contract's code against a formal specification. Tools like KEVM~\cite{hildenbrandt2018kevm} and CertiK's DeepSEA have been developed for formal verification of smart contracts~\cite{zhong2020move}.

While these tools are valuable, they often have limitations in accuracy, scalability, and the ability to handle the complexities of real-world smart contracts.

\subsection{LLMs for Code Repair}

LLMs, trained on vast datasets of code, have shown impressive capabilities in code repair tasks~\cite{chen2021evaluating}. They can learn to understand and generate code that adheres to specific programming paradigms and best practices. However, applying LLMs to smart contract code repair presents unique challenges due to the specific syntax, semantics, and security considerations of languages like Solidity and Move.

Our proposed framework, \textbf{Smartify}, addresses these challenges by combining the strengths of specialized LLMs within a multi-agent architecture. It leverages language-specific fine-tuning, safety classifiers, and Retrieval-Augmented Generation (RAG) to enhance the accuracy and security of generated code repairs. 
% Furthermore, \sln{} incorporates the SolMover~\cite{kara} tool to facilitate cross-language translation between Solidity and Move, expanding its applicability within the blockchain ecosystem.

In the following sections, we detail the architecture of \sln{}, describe the experimental setup, present the evaluation results, and discuss the implications of our findings for the future of smart contract security.


\section{NumericBench Generation}
In this section, we present our created  NumericBench, which is specifically designed to evaluate fundamental numerical capabilities of LLMs. 
NumericBench consists of diverse datasets and tasks, 
enabling a systematic and comprehensive evaluation.
We discuss the datasets included in NumericBench, the key abilities it evaluates, and the methodology for benchmark generation.

\begin{table*}[t]
	\caption{NumericBench statistics. R: contextual retrieval, C: comparison, S: summary, L: logical reasoning. The token count is calculated based on tiktoken, which is the tokenizer used by Llama3~\cite{grattafiori2024llama3herdmodels}. The sentences used for token calculation include both the context and the question.}
	\centering
	\renewcommand{\arraystretch}{1.15} % 设置行间距为默认的 1.15 倍
	\setlength{\tabcolsep}{1.5pt} % 将列间距设置为 1pt
\resizebox{\textwidth}{!}{
	\begin{tabular}{c|c|c|c|c}
		\toprule
		\textbf{Data} & \textbf{Format} & \textbf{Questions} & \textbf{\# Instance} & \textbf{Avg Token} \\ \midrule
		
		\multirow{3}{*}{} 
		& \multirow{3}{*}{} 
		& \begin{tabular}[c]{@{}c@{}}R: What is the index of the first occurrence\\ of the number -3095 in the list?\end{tabular} 
		& 1000 & 3704.23 \\ \cline{3-5}
		
		\textbf{\begin{tabular}[c]{@{}c@{}}Number\\ List\end{tabular}}
		& $[69, -1, 6.1, \ldots, 5.7]$
		& \begin{tabular}[c]{@{}c@{}}C: Which index holds the smallest number\\
			 in the list between the indices 20 and 80?\end{tabular} 
		& 1000 & 3685.57  \\ \cline{3-5}
		
		& & \begin{tabular}[c]{@{}c@{}}S: What is the average of the index of\\
			 top 30 largest numbers in the list?\end{tabular} 
		& 1000 & 3654.78 \\ \midrule
		
		\multirow{3}{*}{} 
		& \multirow{3}{*}{
		\begin{tabular}[c]{@{}c@{}}
			\{date: 2024-06-19,\\
			close\_price: 9.79, \\
			open\_price: 9.4, \\
			\ldots \\
			PE\_ratio: 4.5416\}
		\end{tabular}
		} 
		& \begin{tabular}[c]{@{}c@{}}
			R: On which date did the close price\\
			 of stock firstly reach 61.76 yuan?
		\end{tabular}
		& 1000 & 27585.35 \\ \cline{3-5}
		
		\textbf{Stock}
		& 
		& \begin{tabular}[c]{@{}c@{}}
			C: Among the top-45 trading value days, which\\
			 date did the stock have the lowest close price?
		\end{tabular}
		 & 1000 & 27595.40 \\ \cline{3-5}
		
		& & \begin{tabular}[c]{@{}c@{}} 
			S: How many days had the close price higher than\\
			 the open price from 2024-07-31 to 2024-12-13?
		\end{tabular}	
		& 1000 & 27561.29 \\ \midrule
		
		\multirow{3}{*}{} 
		& \multirow{3}{*}{
		\begin{tabular}[c]{@{}c@{}}
			\{date: 2024-07-21,\\
			pressure\_msl: 999.96,\\
			dew\_point\_2m: 26.25,\\
			\ldots \\
			cloud\_cover: 61.5\}
		\end{tabular}
		} 
		& \begin{tabular}[c]{@{}c@{}} 
			R: On which date did the dew point temperature\\
			 at two meters firstly drop below 9.15°C?
		\end{tabular}
		& 1000 & 27359.26 \\ \cline{3-5}
		
		\textbf{Weather}
		& & \begin{tabular}[c]{@{}c@{}} 
			C: On which date did the MSL pressure reach its\\
			highest value when the cloud cover was below 9\%?
		\end{tabular}
		& 1000 & 27368.19 \\ \cline{3-5}
		
		& & \begin{tabular}[c]{@{}c@{}} 
			S: What was the average temperature at two meters\\
			when the relative humidity exceeded 78.56\%?
		\end{tabular}
		& 1000 & 27331.21 \\ \midrule
		
		\textbf{Sequence} 
		& $[0.34, 3, 6, \ldots, 111]$ 
		& L: What is the next number in the sequence? & 500 & 677.57 \\ \midrule
		
		\textbf{\begin{tabular}[c]{@{}c@{}}Arithmetic \\Operation\end{tabular}} 
		& \begin{tabular}[c]{@{}c@{}} 
		$a: 6.755,
		b: -1.225$
		\end{tabular}
		& \begin{tabular}[c]{@{}c@{}} 
		 $Q_{oper}$: What is the result of $a + b$?\\
		 $Q_{context}$: What is the result of $a $ plus $b$?
		 
		\end{tabular}
		& 12000 & 112.00 \\ \midrule
		
		\textbf{\begin{tabular}[c]{@{}c@{}}Mixed-number-string\\ Sequence\end{tabular}} 
		& \begin{tabular}[c]{@{}c@{}} 
		$effV2\ldots x98o7Lo$
		\end{tabular}
		& \begin{tabular}[c]{@{}c@{}} 
		How many numbers are there in the string? Note\\
		that a sequence like 'a243b' counts as a single number.
		\end{tabular}
		& 2000 & 196.53 \\ \bottomrule

	\end{tabular}
}
	\label{tab:data_stat}
	
\end{table*}

 

\subsection{Numeric Dataset Collection}
NumericBench offers a diverse collection of numerical datasets and questions designed to reflect real-world scenarios. 
This variety ensures that LLMs are thoroughly tested on their fundamental  abilities on numerical data.

\noindent\textbf{Number List Dataset.}
The synthetic number list dataset consists of simple collections of numerical values (integer and floats) 
presented as ordered or unordered lists.
Numbers in lists are one of the simplest and most fundamental data representations encountered in real-world scenarios.
Despite their simplicity, retrieving, indexing,  comparison, and summary on numbers can verify the fundamental numerical ability of LLMs. 
This dataset serves as a fundamental dataset of how well LLMs understand numerical values as discrete entities.



\noindent\textbf{Stock Dataset.}
The time-series  stock dataset is crawled from Eastmoney website~\cite{eastmoney}, 
which has eighteen attributes, such as stock close prices, open price,  trading volumes, and price-earnings ratio, over time.
Stock  data reflects dynamic, real-world numerical reasoning challenges that involve trend analysis, comparison, and decision-making under uncertainty,  representing real-world financial workflows.
 




\noindent\textbf{Weather Dataset.}
The weather dataset is crawled from Open-Meteo python API~\citep{openmeteo}, which includes data related to weather metrics, such as temperature, precipitation, humidity, and wind speed. 
The data is presented across various longitude and latitude.
 
 




\noindent\textbf{Numeric Sequence  Dataset.}
The synthetic numeric sequence dataset comprises sequences of numbers generated by arithmetic or geometric progression, complex patterns, or noisy inputs. 
Tasks require identifying patterns, predicting the next number, or reasoning about relationships between numbers.
Numerical sequences test the logical reasoning capabilities of LLMs, requiring pattern recognition and multi-step reasoning. This dataset introduces structured challenges that are common in mathematics and algorithmic reasoning.


 
\noindent\textbf{Arithmetic Operation Dataset.}
The dataset comprises 12,000 pairs of simple numbers, each undergoing addition, subtraction, multiplication, and division operations. Each pair of numbers, $a$ and $b$, consists of $k$-digit integers with three decimal places, where $k \in \{1, 2, \cdots, 6\}$. 
For each value of $k$, there are 2,000 pairs, evenly distributed across the four basic operations (i.e, $+, -,  *, /$), with 500 pairs per operation. 
This dataset is to evaluate the fundamental mathematical operation capabilities of LLMs, simulating the majority of mathematical calculation requirements in real-world scenarios.

\noindent\textbf{Mixed-number-string Sequence Dataset.}
The dataset consists of alphanumeric strings of varying lengths $\{50, 100, 150, 200\}$, each containing a randomized mix of letters and digits. For each string length, 500 samples are generated, resulting in a total of 2,000 samples. Each sample includes a query asking for the count of contiguous numeric sequences within the string, where sequences like "a243b" count as a single number. This dataset is designed to assess the ability of LLMs to identify and count numeric sequences.
 







\subsection{Fundamental Numerical Ability}
NumericBench is designed to comprehensively evaluate six fundamental numerical reasoning abilities of LLMs, which is 
%These three fundamental abilities are 
essential for solving real-world numeric-related tasks.
%such as numeric data summary and financial price analysis.


\noindent\textbf{Contextual Retrieval Ability.}
Contextual retrieval ability evaluates how well LLMs can locate, extract, and identify specific numerical values or their positions within structured or unstructured data. 
This includes tasks like finding a specific number in a list, retrieving values , and indexing numbers based on their order.
For example, as shown in Table~\ref{tab:data_stat}, it evaluates LLMs on tasks such as retrieving stock prices and identifying key values within numerical lists or domain-specific data (e.g., stock market and weather-related information).
This ability is fundamental to numerical reasoning because it forms the foundation for higher-order tasks, such as comparison, aggregation, and logical reasoning. 
 
 



\noindent\textbf{Comparison Ability.}
Comparison ability evaluates how well LLMs can compare numerical values to determine relationships such as greater than, less than, or equal to, and identify trends or differences in datasets. 
Comparison is vital for logical reasoning and decision-making, as many real-world tasks depend on accurate numerical evaluation. 
For instance,  as shown in Table~\ref{tab:data_stat},   comparing prices is essential in stock  for assessing performance, while weather forecasting requires analysis of temperature or precipitation trends over time. 
 



\noindent\textbf{Summary Ability.}
Summary ability assesses the LLM’s capacity to aggregate numerical data into concise insights, such as calculating totals, averages, or other statistical metrics. 
Summarization is critical for condensing large datasets into actionable information, enabling decision-making based on aggregated insights rather than raw data. 
This ability is indispensable in domains like electricity usage analysis, where summarizing hourly or daily consumption helps forecast bills, in business reporting for aggregating sales and revenue data to evaluate performance, 
and in healthcare analytics to monitor trends in patient metrics over time.



\noindent\textbf{Logic Reasoning Ability.}
Logical Reasoning Ability measures the LLM’s ability to perform multi-step operations involving numerical data, 
such as recognizing patterns, inferring rules, and applying arithmetic or geometric reasoning to solve complex problems. Logical reasoning extends beyond simple numerical tasks and reflects the LLM’s capacity for deeper, structured thinking. 
This ability is crucial for algorithm design, where solving problems involving numeric sequences or patterns is essential, in scientific research for identifying relationships and correlations in data.

\noindent\textbf{Arithmetic Operation Ability.}
It reflects the LLM's capacity to perform mathematical calculations accurately. Such ability is essential for tasks involving numerical computations, such as  automated machine learning through LLMs.





\noindent\textbf{Number Recognition  Ability.}
This measures the LLM's proficiency in identifying and interpreting numerical information within a given context. It represents a fundamental requirement for handling numeric-based tasks effectively.




\subsection{NumericBench Generation}
We use the number list, stock, and weather datasets to evaluate the contextual retrieval, comparison, and summary abilities of LLMs. 
Specifically, for each ability and each dataset, we prepare a set of questions designed to assess the corresponding target ability.
As shown in Table~\ref{appx:number_question}, Table~\ref{appx:stock_question}, and Table~\ref{appx:weather_question} in Appendix, there are nine question sets in total, covering three abilities across three datasets. 
When evaluating a specific ability (e.g., contextual retrieval) on a specific dataset (e.g., stock data), we randomly select one question from the corresponding question set for each data instance (e.g., a stock instance) 
and manually label the answer. This approach enables us to generate question-answer pairs for each ability on the number list, stock, and weather datasets.

For arithmetic operations and number counting in the strings dataset, the question format is straightforward, as illustrated in Table~\ref{tab:data_stat}. These questions are designed to evaluate the basic arithmetic operation and number recognition abilities of LLMs.



\section{\sln{} System Architecture and Workflow}

\sln{} operates through a five-agent system designed for automated smart contract vulnerability detection and repair. The system functions as shown in Figure~\ref{fig:archi}.

The Smartify system operates in a five-phase process to automatically repair smart contract code.  Firstly, in the Input \& Initial Audit phase, the smart contract code, written in either Solidity or Move, is fed into the system. The Auditor, an LLM based on Gemma2 9B, analyzes the code to detect potential vulnerabilities and produces a report detailing its findings.  Secondly, during Repair Planning, the Architect receives this vulnerability report and formulates a high-level repair plan that outlines the necessary code modifications to address the identified issues. Thirdly, in Code Generation \& Refinement, an LLM called CodeGemma which has been fine-tuned for code generation, and is equipped with Retrieval-Augmented Generation (RAG) capabilities, takes the lead. It utilizes separate Move RAG and Solidity RAG components to provide language-specific context. The Code Generator, part of CodeGemma, uses the repair plan to generate the modified code, selecting the appropriate RAG based on the input language and having the capability to perform Solidity to Move translation when necessary. Subsequently, a Self Refinement process is initiated, and the Refiner component iteratively improves the generated code's quality, readability, and efficiency.  Fourthly, in the Validation phase, the Validator (which is the same agent as the Auditor) performs a final security audit on the refined code to ensure that all identified vulnerabilities have been resolved. Finally, the system outputs the repaired smart contract code.

% \begin{algorithm}[H]
% \caption{Smartify: Automated Smart Contract Repair}
% \label{alg:smartify}
% \begin{algorithmic}[1]
% \STATE \textbf{Input:} Smart contract code $C$ (Solidity or Move)
% \STATE \textbf{Output:} Repaired smart contract code $C'$

% \STATE \textbf{Procedure} Smartify($C$)
%     \STATE // \textbf{Phase 1: Initial Audit}
%     \STATE $V \gets $ Auditor($C$) \COMMENT{Auditor: LLM1 (Gemma2 9B)} 
%     \STATE \COMMENT{$V$: List of detected vulnerabilities}
    
%     \STATE // \textbf{Phase 2: Repair Planning}
%     \STATE $P \gets $ Architect($V$) \COMMENT{Architect}
%     \STATE \COMMENT{$P$: High-level repair plan}

%     \STATE // \textbf{Phase 3: Code Generation and Refinement}
%     \STATE $L \gets $ DetermineLanguage($C$) \COMMENT{Determine if $C$ is Solidity or Move}
%     \IF{$L$ = Solidity}
%       \STATE $R \gets $ SolidityRAG()
%     \ELSE
%       \STATE $R \gets $ MoveRAG()
%     \ENDIF
    
%     \STATE $C_{gen} \gets $ CodeGenerator($P, R$) \COMMENT{CodeGenerator: LLM2 (FT CodeGemma), utilizes RAG}
%     \STATE $C' \gets $ Refiner($C_{gen}$) \COMMENT{Refiner: Iterative code improvement}
    
%     \STATE // \textbf{Phase 4: Validation}
%     \STATE $V' \gets $ Validator($C'$) \COMMENT{Validator: same agent as Auditor}
%     \IF{$V' = \emptyset$} \COMMENT{No vulnerabilities detected}
%         \STATE \textbf{return} $C'$
%     \ELSE
%         \STATE \textbf{goto} Step 4 \COMMENT{Repeat the process if new vulnerabilities are found}
%     \ENDIF
% \end{algorithmic}
% \end{algorithm}

The process may iterate back to step 3 or 4 if the Validator identifies any issues. Each step plays a vital role in ensuring the accurate and secure repair of smart contract code. The workflow is designed to be efficient and effective, leveraging the strengths of each agent to achieve the desired outcome.

\subsection{Agent Prompting Strategy}

The agents within \sln{} are driven by carefully crafted prompts that guide their actions and ensure consistent performance. We employ a standardized prompt template, adapted from established practices in LLM-based agent systems. The template is structured as follows:

\begin{tcolorbox}[
  colback=gray!10, % Light gray background
  colframe=gray!50, % Medium gray border
  title=Prompt Template,
  fonttitle=\bfseries,
  boxrule=0.75mm, % Thicker border for emphasis
  rounded corners, % Smooth corners for aesthetics
  left=1mm, % Small padding on the left
  right=1mm, % Small padding on the right
  top=1mm, % Small padding on the top
  bottom=1mm, % Small padding on the bottom
  label=box:prompttemplate
]
\textbf{Role:} You are a \texttt{[role]} specializing in \texttt{[Solidity/Move]} smart contracts.

\textbf{Task:} \texttt{[task]}

\textbf{Instruction:} Based on the provided Context, please follow these steps: \texttt{[numbered steps]}

\textbf{Context:} 
\end{tcolorbox}

This template is broken down into the following components.
% \begin{itemize}
%     \item \textbf{Role:} Specifies the agent's role (e.g., Auditor, Architect, Code Generator, Refiner, Validator).
%     \item \textbf{Task:} Describes the specific task the agent is expected to perform (e.g., "Identify vulnerabilities," "Generate repair plan," "Generate code").
%     \item \textbf{Instruction:} Provides a detailed, step-by-step guide on how to accomplish the task. This section leverages chain-of-thought reasoning to guide the agent's actions and decision-making process.
%     \item \textbf{Context:} Contains all the necessary information for the agent to perform its task. This may include the input code, audit reports, architectural plans, code examples from the RAG datastore, and the conversation history between agents.
% \end{itemize}

Each agent in our framework is defined by four key components: the Role, which designates the agent’s specific function (such as Auditor, Architect, or Code Generator); the Task, which outlines the agent’s specific objectives; the Instruction, which provides detailed step-by-step guidance using chain-of-thought reasoning; and the Context, which encompasses all necessary information including input code, audit reports, architectural plans, RAG datastore examples, and inter-agent conversation history.

Table \ref{tab:agent-prompts} shows how this template is adapted for each agent.

\begin{table}[ht]
\centering
\caption{Agent Prompts for Smart Contract Repair.}
\label{tab:agent-prompts}
\renewcommand{\arraystretch}{1.2} % Adjust row height for readability
\setlength{\tabcolsep}{4pt} % Adjust column spacing
\begin{tabularx}{\linewidth}{|p{1.5cm}|X|X|p{2.5cm}|}
\hline
\textbf{Role} & \textbf{Task} & \textbf{Instruction} & \textbf{Context} \\ \hline
Auditor & Identify vulnerabilities and unsafe patterns in Solidity/Move code. & Analyze the code for security vulnerabilities and generate a detailed report. & Input smart contract code (Solidity/Move). \\ \hline
Architect & Create a high-level plan to address vulnerabilities identified by the Auditor. & Review the Auditor's report and develop a plan outlining necessary modifications. & Auditor's report. \\ \hline
Code Generator & Generate Repaired Solidity/Move code based on the Architect's plan and RAG examples. & Consult the Architect's plan, retrieve examples from the RAG datastore, and generate repaired code. & Architect's plan, Solidity/Move code examples from RAG. \\ \hline
Refiner & Iteratively refine the generated code to improve quality and efficiency. & Review the generated code, identify areas for improvement, and refine accordingly. & Generated code, previous iteration code (if any). \\ \hline
Validator & Perform a final security check on the repaired code. & Analyze the repaired code for vulnerabilities, verify issue resolution, and ensure no new vulnerabilities. & Repaired smart contract code. \\ \hline
\end{tabularx}
\end{table}

\subsection{Hardware and Model Fine-tuning}

The development and deployment of \sln{} leveraged a heterogeneous compute environment, utilizing both high-performance GPUs for computationally intensive tasks and a more resource-efficient setup for inference.

\subsubsection{Fine-tuning Setup}

% \begin{itemize}
%     \item \textbf{Hardware:} Fine-tuning the Auditor agent leveraged a cluster of \textbf{four NVIDIA A100 GPUs} to handle the computational demands of learning complex patterns in Solidity and Move code.
%     \item \textbf{Model:} The Auditor is based on the \textbf{Gemma 9B model}, chosen for its strong performance on code-related tasks and adaptability to fine-tuning, particularly in following instructions. It was fine-tuned on a comprehensive dataset of Solidity and Move code, vulnerability examples, best practices, and documentation. The dataset was further augmented with outputs from earlier pipeline stages to enhance the Auditor's safety issue detection capabilities.
%     \item \textbf{Training Recipe:} A supervised learning paradigm was employed. The model was trained to predict correct outputs (e.g., vulnerability reports, safe code patterns) from inputs (e.g., Solidity/Move code, vulnerability descriptions).
%     \begin{itemize}
%         \item \textbf{Data Preprocessing:} Tokenization, normalization, and input-output pair creation ensured data consistency and quality.
%         \item \textbf{Hyperparameter Optimization:} Learning rate, batch size, and training epochs were optimized via grid search and manual tuning. The learning rate was set to 1e-5, batch size to 8 (due to memory constraints), and training ran for 5 epochs (as validation loss plateaued).
%         \item \textbf{Regularization:} Dropout and weight decay were used to prevent overfitting and improve generalization.
%         \item \textbf{Evaluation Metrics:} Accuracy, precision, recall, and F1-score, computed on a held-out validation set, monitored model performance during training.
%     \end{itemize}
% \end{itemize}

\begin{itemize}[leftmargin=*]
    \item \textbf{Hardware:} Fine-tuning leveraged a cluster of \textbf{four NVIDIA A100 GPUs} for computationally demanding pattern learning in Solidity and Move code.
    \item \textbf{Model:} Based on the \textbf{Gemma 9B model}, selected for strong code-related task performance and fine-tuning adaptability, particularly in instruction following. Fine-tuned on a dataset of Solidity and Move code, vulnerability examples, best practices, and documentation, augmented with outputs from earlier pipeline stages to enhance safety issue detection.
    \item \textbf{Training Recipe:} Supervised learning paradigm. Trained to predict correct outputs (e.g., vulnerability reports, safe code patterns) from inputs (e.g., Solidity/Move code, vulnerability descriptions).
    \begin{itemize}
        \item \textbf{Data Preprocessing:} Tokenization, normalization, and input-output pair creation ensured data consistency and quality.
        \item \textbf{Hyperparameter Optimization:} Learning rate (1e-5), batch size (8, due to memory constraints), and training epochs (5, as validation loss plateaued) optimized via grid search and manual tuning.
        \item \textbf{Regularization:} Dropout and weight decay used to prevent overfitting and improve generalization.
        \item \textbf{Evaluation Metrics:} Accuracy, precision, recall, and F1-score on a held-out validation set monitored model performance.
    \end{itemize}
\end{itemize}

\subsubsection{Inference Setup}

\begin{itemize}[leftmargin=*]
    \item \textbf{Hardware:} Inference was performed on a single \textbf{NVIDIA RTX 4090 GPU}, balancing performance and cost-effectiveness for real-time code repair.
    \item \textbf{Models:}
    \begin{itemize}
        \item \textbf{Code Generator and Refiner:} These agents utilize a fine-tuned \textbf{CodeGemma} model, initially pre-trained on a limited Move corpus and further instruction-tuned to follow Architect-generated "recipe" patterns. Fine-tuning on Architect outputs ensured it understood these instructions, and pre-training on a limited Move corpus ensured basic syntax understanding.
        \item \textbf{Comparison Model:} A stock \textbf{Llama 3.1} model was used in some experiments for comparative analysis, helping assess the gains from fine-tuning and instruction tuning.
    \end{itemize}
\end{itemize}

\subsubsection{Key Considerations}

\begin{itemize}[leftmargin=*]
    \item A balance between performance requirements, resource availability, and cost considerations drove the choice of hardware and models.
    \item The fine-tuning process for the Auditor was particularly resource-intensive due to the complexity of the task and the size of the model.
    \item The use of a smaller, more efficient GPU for inference makes the system more accessible for practical deployment.
    \item The comparison with a stock Llama 3 model provides valuable insights into the effectiveness of our fine-tuning and instruction-tuning strategies.
\end{itemize}

This heterogeneous setup, combining high-performance GPUs for training and a more efficient GPU for inference, allows \sln{} to effectively address the computational demands of both model development and deployment. The detailed description of the fine-tuning process provides transparency and allows for replication of our results.
\section{Experimental Results and Discussion}

We run our experiments as defined in our Section \ref{evaluationmethod}. We report the results as well as the empirical performance of our models. Through that we will try to answer our Research Questions one by one in this section.

Along with \sln{} we have ran the benchmark for the following models.

% granite-code\:8b-instruct, codegemma\:7b-instruct, deepseek\-coder\-v2, starcoder2, codegeex4, codestral, deepseek-coder\:33b, codellama\:13b, codeqwen\:7b\-chat\-v1.5\-q8\_0, qwen2.5\-coder, gemma2, gemma2:27b, llama3.2, opencoder:8b-instruct-fp16, llama3.3.
\begin{table}[ht]
    \scriptsize % Reduce font size for the table
    \centering
    \setlength{\tabcolsep}{2pt} % Adjust column spacing
    \renewcommand{\arraystretch}{1.2} % Adjust row spacing
    \caption{Comparison of Code and Non-Code Models.}
    \label{tab:model-comparison}
    \begin{tabular}{@{}p{2.5cm}p{1.5cm}p{1.5cm}p{1cm}@{}}
        \toprule
        \textbf{Model Name} & \textbf{Parameters} & \textbf{Quantization} & \textbf{Code Model} \\ 
        \midrule
        granite-code     & 8B    & FP16 & Yes \\
        codegemma        & 7B    & FP16 & Yes \\
        deepseek-coder-v2            & N/A   & N/A  & Yes \\
        starcoder2                   & 15B   & FP16 & Yes \\
        codegeex4                    & 13B   & N/A  & Yes \\
        codestral                    & 7B    & FP16 & Yes \\
        deepseek-coder           & 33B   & N/A  & Yes \\
        codellama~\cite{roziere2023code}                & 13B   & N/A  & Yes \\
        codeqwen  & 7B    & Q8\_0 & Yes \\
        qwen2.5-coder                & 2.5B  & N/A  & Yes \\
        gemma2                       & N/A   & N/A  & Yes \\
        gemma2:27b                   & 27B   & FP16 & Yes \\
        llama3.2                     & 3.2B  & FP16 & No  \\
        opencoder   & 8B    & FP16 & Yes \\
        llama3.3                     & 3.3B  & FP16 & No  \\
        \bottomrule
    \end{tabular}

\end{table}

The models were chosen according to the top 8 models at Hugging Face Big Code Leaderboard~\cite{huggingfaceCodeModels} at the time of this work, and also adding general-purpose models, which are supposed to be better at reasoning.

\subsection{Solidity}

This section presents the evaluation results of various code generation models on the task of repairing vulnerabilities in Solidity smart contracts, specifically focusing on the "Not So Smart Contracts" dataset from the Trail of Bits GitHub repository. This dataset is a collection of intentionally vulnerable Solidity contracts, designed to test the ability of automated tools to detect and repair common security flaws. It contains a diverse set of vulnerabilities, including reentrancy, integer overflow/underflow, access control issues, and timestamp dependence, among others. The dataset has been publicly available for a significant period, raising the possibility that some or all of its contents might be present in the pre-training data of the evaluated models. We analyze the performance of these models based on two key metrics: the number of vulnerabilities fixed and the average inference time, as summarized in Table \ref{tab:model_performance} and Figure \ref{fig:solidityrepair}. We also introduce our framework, \sln{}, and demonstrate its effectiveness in enhancing model performance.

\begin{figure}
    \centering
    \includegraphics[width=1\linewidth]{img/solidity_result.png}
    \caption{Code Repair: Solidity.}
    \label{fig:solidityrepair}
\end{figure}

\begin{table}[ht]
\centering
\caption{Performance of Code Generation Models on Vulnerability Repair.}
\label{tab:model_performance}
\begin{tabular}{|l|c|c|}
\hline
\textbf{Model Name} & \textbf{Vuln. Fixed} & \textbf{Avg. Time (s)} \\
\hline
CodeGeex-4      & 11 & 95.50 \\
\hline
CodeGemma       & \textbf{16} & 96.50 \\
\hline
CodeLlama       & 3  & 243.93 \\
\hline
CodeQuen        & 5  & 141.05 \\
\hline
CodeStral       & 13 & 295.23 \\
\hline
DeepSeekCoder-33b & 1  & 411.75 \\
\hline
DeepSeek-V2     & 9  & \textbf{19.42} \\
\hline
Gemma2-9b       & 13 & 108.30 \\
\hline
Gemma2-27b      & 14 & 304.27 \\
\hline
Granite-Code    & 14 & 90.37 \\
\hline
LLaMA3.2        & 1  & 37.09 \\
\hline
LLaMA3.3-70b    & 13 & 741.10 \\
\hline
OpenCoder*      & 1* & 94* \\
\hline
Qwen-2.5-Code   & 13 & 79.72 \\
\hline
StarCoder2      & 0* & 89.10 \\
\hline
Smartify (Gemma2+CodeGemma) & \textbf{16} & 112.30 \\
\hline
Smartify (Gemma2+LLaMA3.1)  & 14 & 267.80 \\
\hline
\end{tabular}
\end{table}


% The results reveal significant performance disparities among the evaluated models. \textbf{CodeGemma} emerges as a top performer, successfully fixing 16 vulnerabilities with a relatively low average inference time of 96.5 seconds. This suggests that CodeGemma possesses a strong ability to understand and rectify code vulnerabilities while maintaining reasonable efficiency. Our proposed framework, \textbf{Smartify (Gemma2+codegemma)}, achieves comparable performance, also fixing 16 vulnerabilities, albeit with a slightly higher average inference time of 112.3 seconds, likely due to its iterative multi-agent process. \textbf{Gemma2 9b} and \textbf{Gemma2 27b} also demonstrate strong capabilities, fixing 13 and 14 vulnerabilities, respectively. However, the larger Gemma2 27b model exhibits a significantly higher inference time (304.27 seconds) compared to the 9b variant (108.3 seconds), highlighting the trade-off between model size and efficiency. \textbf{Granite-code} performs well, fixing 14 vulnerabilities with an inference time of 90.37 seconds.

The results reveal significant performance disparities among the evaluated models. Among the pre-trained models for Solidity \textbf{CodeGemma} surprisingly emerges as a top performer, successfully fixing 16 vulnerabilities with a relatively low average inference time of 96.5 seconds. This suggests that CodeGemma possesses a strong ability to understand and rectify code vulnerabilities while maintaining reasonable efficiency. However since most of these Solidty smart contracts were part of open githubs repositories, there can be a strong possibility fo these already being part of the pertaining data. Our proposed framework, \textbf{Smartify (Gemma2+CodeGemma)}, achieves comparable performance, also fixing 16 vulnerabilities, albeit with a slightly higher average inference time of 112.3 seconds. This increased time is likely due to its iterative multi-agent process, which enables Smartify to leverage the complementary strengths of Gemma2 and CodeGemma, resulting in robust and reliable fixes. 

While \sln{} here doesn't immediately show any benefits over codegemma here, we can notice that the same \sln{} framework when applied to llama3.1 without any fine-tuning (unlike the \sln{} with codegemma) still gives considerable performance boost over vanilla.

% \textbf{Gemma2 9b} and \textbf{Gemma2 27b} also demonstrate strong capabilities, fixing 13 and 14 vulnerabilities, respectively. However, the larger Gemma2 27b model exhibits a significantly higher inference time (304.27 seconds) compared to the 9b variant (108.3 seconds), highlighting the trade-off between model size and efficiency. \textbf{Granite-code} performs well, fixing 14 vulnerabilities with an inference time of 90.37 seconds, showcasing its competitive performance. 

\begin{figure}
    \centering
    \includegraphics[width=1\linewidth]{move_result.png}
    \caption{Move Code Repair.}
    \label{fig:enter-label}
\end{figure}

Conversely, models like \textbf{codellama}, \textbf{codequen}, \textbf{deepseekcoder 33b}, and \textbf{llama3.2} show limited effectiveness, fixing only a small number of vulnerabilities. The poor performance of these models could be attributed to several factors, such as insufficient exposure to Solidity code during pre-training or fine-tuning, or architectures ill-suited for vulnerability repair, which requires a deep understanding of both code syntax and security principles. The exceptionally poor performance of models like \textbf{starcoder2} (marked with an asterisk *), along with incomplete data for \textbf{opencoder}, suggests potential issues with their training data or a fundamental mismatch between their capabilities and the task's demands.  These models might have been trained on an older version of Solidity or different smart contract security practices than those in the Not-So-Smart-Contracts dataset. Moreover, they might prioritize other aspects of code generation, such as code completion, over security-specific tasks like vulnerability repair.

The public availability of the "Not So Smart Contracts" dataset raises the question of data contamination. Many evaluated models, especially those trained on large, public code corpora, might have encountered this dataset during pre-training, potentially inflating their performance. However, since \textbf{CodeGemma} and \textbf{Smartify (Gemma2+codegemma)} were specifically fine-tuned for this task, the issue of data contamination is likely less significant.

% In Solidity code repair, the balance between vulnerabilities fixed and inference time is crucial. \textbf{Smartify (Gemma2+codegemma)} emerges as a strong contender, fixing the most vulnerabilities with a reasonable inference time. \textbf{DeepseekV2} has a remarkably low inference time of 19.42 seconds while fixing 9 vulnerabilities, making it the fastest but significantly less accurate. This might make it suitable where speed is paramount.

% The results highlight the effectiveness of our \textbf{Smartify} framework. Combining \textbf{Gemma2} fine tuned  with \textbf{CodeGemma}, Smartify achieves performance comparable to the best individual model (CodeGemma) which should be expected.However, even with a non-finetuned model like \textbf{Llama 3.1}, Smartify significantly improves performance, fixing 14 vulnerabilities compared to Llama 3.2's single fix. Making it comparable with the much bigger llama3.3 in performance which is significantly slower due to its size and computation complexity. This suggests Smartify's multi-agent architecture and iterative refinement process can enhance even general-purpose language models for code repair. The iterative process, involving an Auditor, Architect, Code Generator, Refiner, and Validator, likely contributes to improved performance by refining the generated code. The best model choice depends on the specific application requirements and the acceptable balance between speed and accuracy. For real-world on-device applications, \textbf{Smartify (Gemma2+codegemma)} is likely the most useful, considering its high accuracy and relatively fast inference time.

\subsection{Move Code Repair}\label{sec:move_code_repair}

\begin{table}[t]
\centering
\caption{Move Vulnerability Repair (Time in seconds).}
\label{tab:vulnerability_repair}
\resizebox{\columnwidth}{!}{%
\begin{tabular}{l|c|c|c|c|c|c|c|c|c}
\toprule
Model & \rotatebox{90}{UR} & \rotatebox{90}{IL} & \rotatebox{90}{UB} & \rotatebox{90}{UC} & \rotatebox{90}{UPF} & \rotatebox{90}{UTC} & \rotatebox{90}{Ov} & \rotatebox{90}{PL} & \rotatebox{90}{Time} \\
\midrule
codegeex4 & 6 & 0 & 0 & 6 & 1 & 1 & 1 & 0 & 96 \\
codegemma & 6 & 0 & 10 & 7 & 1 & 1 & 0 & 0 & 97 \\
codellama & 15 & 0 & 1 & 17 & 2 & 2 & 2 & 1 & 244 \\
CodeQwen & 10 & 0 & 1 & 10 & 1 & 1 & 1 & 1 & 141 \\
codestral & 19 & 0 & 1 & 21 & 3 & 3 & 2 & 1 & 295 \\
deepseekcoder 33b & 26 & 0 & 2 & 30 & 4 & 4 & 3 & 1 & 412 \\
deepseekV2 & 2 & 0 & 0 & 2 & 0 & 1 & 0 & 0 & 19 \\
gemma2 9b & 7 & 0 & 1 & 8 & 1 & 1 & 0 & 10 & 108 \\
gemma2 27b & 21 & 0 & 2 & 23 & 3 & 3 & 21 & 11 & 304 \\
granite-code & 6 & 0 & 0 & 6 & 1 & 1 & 1 & 0 & 90 \\
llama3.2 & 3 & 0 & 0 & 3 & 1 & 1 & 0 & 0 & 37 \\
llama3.3 70b & 34 & 0 & 21 & 39 & 5 & 5 & 14 & 31 & 741 \\
opencoder & 7 & 0 & 1 & 8 & 1 & 1 & 0 & 0 & 94* \\
qwen 2.5 code & 5 & 0 & 0 & 6 & 1 & 1 & 1 & 0 & 80 \\
starcoder2 & 6 & 0 & 1 & 7 & 1 & 1 & 0 & 0 & 89 \\
Smartify (Gemma2+codegemma) & 293 & 2 & 16 & 189 & 41 & 48 & 51 & 10 & 112 \\
Smartify (Gemma2+llama3.1) & 97 & 0 & 1 & 90 & 13 & 34 & 12 & 10 & 268 \\
Move Prover~\cite{dill2022fast} & - & 2 & - & - & - & - & 47 & 15 & - \\
MoveLint & - & - & - & - & 19 & 30 & 0 & 0 & -\\
MoveScan~\cite{song2024empirical} & 406 & 2 & 28 & 404 & 52 & 62 & 60 & 15 & -\\
\bottomrule
\end{tabular}
}

\vspace{0.5em} % Add vertical space before the abbreviations

\textbf{Abbreviations:} \textbf{UR}: Unchecked Return; \textbf{IL}: Infinite Loop; \textbf{UB}: Unnecessary Boolean; \textbf{UC}: Unused Constant; \textbf{UPF}: Unused Private Function; \textbf{UTC}: Unnecessary Type Conversion; \textbf{Ov}: Overflow; \textbf{PL}: Precision Loss.
\end{table}

This section analyzes the efficacy of various models in repairing vulnerabilities within Move smart contracts, as detailed in Table \ref{tab:vulnerability_repair}. The evaluation encompasses eight distinct vulnerability categories: Unchecked Return (UR), Infinite Loop (IL), Unnecessary Boolean (UB), Unused Constant (UC), Unused Private Function (UPF), Unnecessary Type Conversion (UTC), Overflow (Ov), and Precision Loss (PL) following the works of Song et al~\cite{song2024empirical}. The metrics presented in the table represent the number of successfully repaired instances for each vulnerability type, with higher values indicating superior performance. The inference time, measured in seconds, is also provided for each model.

The results demonstrate a significant variance in performance across the evaluated models. Notably, the larger language models, such as \textbf{deepseekcoder 33b} and \textbf{llama3.3 70b}, exhibit a relatively higher number of successful repairs across multiple categories, albeit with a corresponding increase in inference time. Conversely, smaller models like \textbf{deepseekV2} and \textbf{llama3.2} demonstrate limited repair capabilities.  The specialized tools for Move code, namely \textbf{Move Prover}, \textbf{MoveLint}, and \textbf{MoveScan}, were employed as a benchmark for comparison. It is crucial to note that these tools are designed for vulnerability \textbf{detection} rather than repair. \textbf{MoveScan}, in particular, identified a substantial number of instances across all categories, highlighting its effectiveness as a static analysis tool. \textbf{Move Prover} demonstrated proficiency in detecting Overflow and Precision Loss vulnerabilities, while \textbf{MoveLint} focused on Unused Private Functions and Unnecessary Type Conversions.

The Smartify models, which leverage a combination of \textbf{Gemma2} with either \textbf{codegemma} or \textbf{llama3.1}, present an interesting case. Smartify (\textbf{Gemma2+codegemma}) and Smartify (\textbf{Gemma2+llama3.1}) outperform several individual models in multiple categories. This is likely because the specialized models are fine-tuned on the Move-specific dataset. For instance, Smartify (\textbf{Gemma2+codegemma}) achieves the highest number of repairs for the Unchecked Return, Infinite Loop, Unused Boolean, Unused Constant, Unused Private Function, Unnecessary Type Conversion, and Overflow categories, showcasing a substantial improvement over individual models in these areas. However, it is worth mentioning that they also have limitations compared to individual models for certain categories like Precision Loss.

% The results underscore the effectiveness of the \textbf{Smartify} framework in automated Move code vulnerability detection and repair, demonstrating the potential of model combination to enhance performance. It showcases the effectiveness of the proposed architecture. 
This answers our first two research questions.

\begin{tcolorbox}[
  colback=green!15, % Light green background
  colframe=green!40, % Medium green border
  title=RQ1 \& RQ2 - Code Understanding and Vuln. Detection,
  coltitle=black, % Set title color to black
  fonttitle=\bfseries,
  boxrule=0.75mm, % Thicker border for emphasis
  rounded corners, % Smooth corners for aesthetics
  left=1mm, % Small padding on the left
  right=1mm, % Small padding on the right
  top=1mm, % Small padding on the top
  bottom=1mm % Small padding on the bottom
]
\textbf{Yes.} Our empirical analysis with \sln{}, especially with using a fine-tuned code-gemma and also using a vanilla pre-trained llama3.1, has shown us the effectiveness of the framework's ability to understand code. And to capture bad practices leading to vulnerability. Especially for a low-resource code like move. Without significant fine-tuning (in the case of llama3.1).
\end{tcolorbox}

Notably, \textbf{Smartify (Gemma2+codegemma)}, combining fine-tuned \textbf{Gemma2} with \textbf{CodeGemma}, achieves performance on par with the best individual model, \textbf{CodeGemma}, which is expected due to one of the models being fine-tuned. This highlights the advantages of strategically combining specialized models, answering our next research question.

\begin{tcolorbox}[
  colback=green!15, % Light gray background
  colframe=green!40, % Medium gray border
  title=RQ3 - Code Repair,
  fonttitle=\bfseries,
    coltitle=black, % Set title color to black
  boxrule=0.75mm, % Thicker border for emphasis
  rounded corners, % Smooth corners for aesthetics
  left=1mm, % Small padding on the left
  right=1mm, % Small padding on the right
  top=1mm, % Small padding on the top
  bottom=1mm % Small padding on the bottom
]
Both for solidity and move, we were able to compare the efficacy of our framework with prior works and can see \sln{} outperforms all of the existing code models, even very specialized code models trained on move (opencoder~\cite{huang2024opencoder}) in generating repair codes for detected vulnerabilities. 
\end{tcolorbox}

Furthermore, Smartify's efficacy extends even when integrating a non-finetuned model like \textbf{Llama 3.1}. Smartify significantly outperforms \textbf{Llama 3.2} by fixing 14 vulnerabilities compared to Llama 3.2's single fix, making its performance comparable with the much larger and computationally intensive \textbf{Llama 3.3 70b}. This demonstrates that Smartify's architecture can enhance even general-purpose language models for code repair, offering a balance between speed and accuracy. Answering our last query:

\begin{tcolorbox}[
  colback=green!15, % Light gray background
  colframe=green!40, % Medium gray border
  title=RQ4 - Generalization,
  fonttitle=\bfseries,
    coltitle=black, % Set title color to black
  boxrule=0.75mm, % Thicker border for emphasis
  rounded corners, % Smooth corners for aesthetics
  left=1mm, % Small padding on the left
  right=1mm, % Small padding on the right
  top=1mm, % Small padding on the top
  bottom=1mm % Small padding on the bottom
]
Our implementation of \sln{} with both fine-tuned code-gemma and llama3.1 as the second agent gave us the opportunity to run our experiments on both sets of LLMs. And the results show that \sln{} is able to significantly boost performance even on non-finetuned models compared to a single model.
\end{tcolorbox}

Comparative analysis reveals trade-offs between model scale and performance in automated code repair. Larger models, such as \textbf{deepseekcoder 33b} and \textbf{Llama 3.3 70b}, exhibit broader repair capabilities but incur higher computational costs and inference times. Conversely, the \textbf{Gemma2 27b} model demonstrates notable proficiency in addressing Overflow vulnerabilities, albeit with limitations in handling Unnecessary Boolean and Unused Constant compared to \textbf{Llama 3.3 70b}. While \textbf{Llama 3.3 70b} outperforms Smartify in overall repair capability, its significantly slower inference speed poses a challenge for practical deployment. Therefore, for real-world, on-device applications, \textbf{Smartify (Gemma2+codegemma)} presents a compelling solution with its balance of strong accuracy and rapid inference.

\begin{center}
\fcolorbox{blue!20}{white!90!blue}{%  Box with specified colors
    \parbox{\dimexpr\linewidth-2\fboxsep-2\fboxrule}{ % Adjust width for box borders
\textbf{Insight:} Specialized code models like Starcoder~\cite{lozhkov2024starcoder}, Opencoder~\cite{huang2024opencoder} and deepseekcoder~\cite{guo2024deepseek} doesn't necessarily work well even if it's a coding specific task. While codemodels like codegemma~\cite{team2024codegemma} and codellama~\cite{roziere2023code} are much better at understanding instructions and working on code. This helped \sln{} for its understanding and fine-tuning for code repairability.
    }
}
\end{center}

Specialized static analysis tools for Move, including \textbf{Move Prover}, \textbf{MoveLint}, and \textbf{MoveScan}, work as baselines of detecting move vulnerabilities with which we compare our \sln{} and other LLMs. These findings underscore the need for targeted model improvements. The Smartify framework directly addresses these deficiencies, offering enhanced vulnerability repair effectiveness. 

This research also opens up future research directions of the use of this framework for context-aware test case generation.
\section{Conclusion and Discussion}
\label{sec:conclusion}

We introduce a novel framework for augmenting \emph{any} lossy compressor to preserve the contour tree of a volumetric dataset while maintaining a user-specified global error bound. 
To do this, our framework first imposes topology-informed upper and lower bounds on each data point. 
It then progressively tightens those bounds until the contour tree is preserved. 
We also introduce a novel encoding scheme that efficiently stores individual points with variable precision and maintains these upper and lower bounds. 
When our framework is used to augment state-of-the-art lossy compressors, it is shown to preserve the contour trees of various scientific datasets.
Our augmented compressors also achieve higher compression ratios and reconstruction quality than those obtained by existing topology-preserving compressors in comparable or faster time.
Our framework will benefit from any advancement with lossy compression since it can be used to augment increasingly effective lossy compressors to achieve better topology-preserving compression. 

Our framework is not without limitations. The compression times are longer than the base compressors. This difference gets worse as the topological complexity of the data increases.
However, in some use-cases, topological preservation is preferable to run time.
Regardless, our framework would benefit from more efficient or parallel implementations for the contour/merge tree computation and the encoding scheme. 


\balance
\bibliographystyle{IEEEtran}
\bibliography{ref}
% \clearpage
% \appendix  
% \subsection{Lloyd-Max Algorithm}
\label{subsec:Lloyd-Max}
For a given quantization bitwidth $B$ and an operand $\bm{X}$, the Lloyd-Max algorithm finds $2^B$ quantization levels $\{\hat{x}_i\}_{i=1}^{2^B}$ such that quantizing $\bm{X}$ by rounding each scalar in $\bm{X}$ to the nearest quantization level minimizes the quantization MSE. 

The algorithm starts with an initial guess of quantization levels and then iteratively computes quantization thresholds $\{\tau_i\}_{i=1}^{2^B-1}$ and updates quantization levels $\{\hat{x}_i\}_{i=1}^{2^B}$. Specifically, at iteration $n$, thresholds are set to the midpoints of the previous iteration's levels:
\begin{align*}
    \tau_i^{(n)}=\frac{\hat{x}_i^{(n-1)}+\hat{x}_{i+1}^{(n-1)}}2 \text{ for } i=1\ldots 2^B-1
\end{align*}
Subsequently, the quantization levels are re-computed as conditional means of the data regions defined by the new thresholds:
\begin{align*}
    \hat{x}_i^{(n)}=\mathbb{E}\left[ \bm{X} \big| \bm{X}\in [\tau_{i-1}^{(n)},\tau_i^{(n)}] \right] \text{ for } i=1\ldots 2^B
\end{align*}
where to satisfy boundary conditions we have $\tau_0=-\infty$ and $\tau_{2^B}=\infty$. The algorithm iterates the above steps until convergence.

Figure \ref{fig:lm_quant} compares the quantization levels of a $7$-bit floating point (E3M3) quantizer (left) to a $7$-bit Lloyd-Max quantizer (right) when quantizing a layer of weights from the GPT3-126M model at a per-tensor granularity. As shown, the Lloyd-Max quantizer achieves substantially lower quantization MSE. Further, Table \ref{tab:FP7_vs_LM7} shows the superior perplexity achieved by Lloyd-Max quantizers for bitwidths of $7$, $6$ and $5$. The difference between the quantizers is clear at 5 bits, where per-tensor FP quantization incurs a drastic and unacceptable increase in perplexity, while Lloyd-Max quantization incurs a much smaller increase. Nevertheless, we note that even the optimal Lloyd-Max quantizer incurs a notable ($\sim 1.5$) increase in perplexity due to the coarse granularity of quantization. 

\begin{figure}[h]
  \centering
  \includegraphics[width=0.7\linewidth]{sections/figures/LM7_FP7.pdf}
  \caption{\small Quantization levels and the corresponding quantization MSE of Floating Point (left) vs Lloyd-Max (right) Quantizers for a layer of weights in the GPT3-126M model.}
  \label{fig:lm_quant}
\end{figure}

\begin{table}[h]\scriptsize
\begin{center}
\caption{\label{tab:FP7_vs_LM7} \small Comparing perplexity (lower is better) achieved by floating point quantizers and Lloyd-Max quantizers on a GPT3-126M model for the Wikitext-103 dataset.}
\begin{tabular}{c|cc|c}
\hline
 \multirow{2}{*}{\textbf{Bitwidth}} & \multicolumn{2}{|c|}{\textbf{Floating-Point Quantizer}} & \textbf{Lloyd-Max Quantizer} \\
 & Best Format & Wikitext-103 Perplexity & Wikitext-103 Perplexity \\
\hline
7 & E3M3 & 18.32 & 18.27 \\
6 & E3M2 & 19.07 & 18.51 \\
5 & E4M0 & 43.89 & 19.71 \\
\hline
\end{tabular}
\end{center}
\end{table}

\subsection{Proof of Local Optimality of LO-BCQ}
\label{subsec:lobcq_opt_proof}
For a given block $\bm{b}_j$, the quantization MSE during LO-BCQ can be empirically evaluated as $\frac{1}{L_b}\lVert \bm{b}_j- \bm{\hat{b}}_j\rVert^2_2$ where $\bm{\hat{b}}_j$ is computed from equation (\ref{eq:clustered_quantization_definition}) as $C_{f(\bm{b}_j)}(\bm{b}_j)$. Further, for a given block cluster $\mathcal{B}_i$, we compute the quantization MSE as $\frac{1}{|\mathcal{B}_{i}|}\sum_{\bm{b} \in \mathcal{B}_{i}} \frac{1}{L_b}\lVert \bm{b}- C_i^{(n)}(\bm{b})\rVert^2_2$. Therefore, at the end of iteration $n$, we evaluate the overall quantization MSE $J^{(n)}$ for a given operand $\bm{X}$ composed of $N_c$ block clusters as:
\begin{align*}
    \label{eq:mse_iter_n}
    J^{(n)} = \frac{1}{N_c} \sum_{i=1}^{N_c} \frac{1}{|\mathcal{B}_{i}^{(n)}|}\sum_{\bm{v} \in \mathcal{B}_{i}^{(n)}} \frac{1}{L_b}\lVert \bm{b}- B_i^{(n)}(\bm{b})\rVert^2_2
\end{align*}

At the end of iteration $n$, the codebooks are updated from $\mathcal{C}^{(n-1)}$ to $\mathcal{C}^{(n)}$. However, the mapping of a given vector $\bm{b}_j$ to quantizers $\mathcal{C}^{(n)}$ remains as  $f^{(n)}(\bm{b}_j)$. At the next iteration, during the vector clustering step, $f^{(n+1)}(\bm{b}_j)$ finds new mapping of $\bm{b}_j$ to updated codebooks $\mathcal{C}^{(n)}$ such that the quantization MSE over the candidate codebooks is minimized. Therefore, we obtain the following result for $\bm{b}_j$:
\begin{align*}
\frac{1}{L_b}\lVert \bm{b}_j - C_{f^{(n+1)}(\bm{b}_j)}^{(n)}(\bm{b}_j)\rVert^2_2 \le \frac{1}{L_b}\lVert \bm{b}_j - C_{f^{(n)}(\bm{b}_j)}^{(n)}(\bm{b}_j)\rVert^2_2
\end{align*}

That is, quantizing $\bm{b}_j$ at the end of the block clustering step of iteration $n+1$ results in lower quantization MSE compared to quantizing at the end of iteration $n$. Since this is true for all $\bm{b} \in \bm{X}$, we assert the following:
\begin{equation}
\begin{split}
\label{eq:mse_ineq_1}
    \tilde{J}^{(n+1)} &= \frac{1}{N_c} \sum_{i=1}^{N_c} \frac{1}{|\mathcal{B}_{i}^{(n+1)}|}\sum_{\bm{b} \in \mathcal{B}_{i}^{(n+1)}} \frac{1}{L_b}\lVert \bm{b} - C_i^{(n)}(b)\rVert^2_2 \le J^{(n)}
\end{split}
\end{equation}
where $\tilde{J}^{(n+1)}$ is the the quantization MSE after the vector clustering step at iteration $n+1$.

Next, during the codebook update step (\ref{eq:quantizers_update}) at iteration $n+1$, the per-cluster codebooks $\mathcal{C}^{(n)}$ are updated to $\mathcal{C}^{(n+1)}$ by invoking the Lloyd-Max algorithm \citep{Lloyd}. We know that for any given value distribution, the Lloyd-Max algorithm minimizes the quantization MSE. Therefore, for a given vector cluster $\mathcal{B}_i$ we obtain the following result:

\begin{equation}
    \frac{1}{|\mathcal{B}_{i}^{(n+1)}|}\sum_{\bm{b} \in \mathcal{B}_{i}^{(n+1)}} \frac{1}{L_b}\lVert \bm{b}- C_i^{(n+1)}(\bm{b})\rVert^2_2 \le \frac{1}{|\mathcal{B}_{i}^{(n+1)}|}\sum_{\bm{b} \in \mathcal{B}_{i}^{(n+1)}} \frac{1}{L_b}\lVert \bm{b}- C_i^{(n)}(\bm{b})\rVert^2_2
\end{equation}

The above equation states that quantizing the given block cluster $\mathcal{B}_i$ after updating the associated codebook from $C_i^{(n)}$ to $C_i^{(n+1)}$ results in lower quantization MSE. Since this is true for all the block clusters, we derive the following result: 
\begin{equation}
\begin{split}
\label{eq:mse_ineq_2}
     J^{(n+1)} &= \frac{1}{N_c} \sum_{i=1}^{N_c} \frac{1}{|\mathcal{B}_{i}^{(n+1)}|}\sum_{\bm{b} \in \mathcal{B}_{i}^{(n+1)}} \frac{1}{L_b}\lVert \bm{b}- C_i^{(n+1)}(\bm{b})\rVert^2_2  \le \tilde{J}^{(n+1)}   
\end{split}
\end{equation}

Following (\ref{eq:mse_ineq_1}) and (\ref{eq:mse_ineq_2}), we find that the quantization MSE is non-increasing for each iteration, that is, $J^{(1)} \ge J^{(2)} \ge J^{(3)} \ge \ldots \ge J^{(M)}$ where $M$ is the maximum number of iterations. 
%Therefore, we can say that if the algorithm converges, then it must be that it has converged to a local minimum. 
\hfill $\blacksquare$


\begin{figure}
    \begin{center}
    \includegraphics[width=0.5\textwidth]{sections//figures/mse_vs_iter.pdf}
    \end{center}
    \caption{\small NMSE vs iterations during LO-BCQ compared to other block quantization proposals}
    \label{fig:nmse_vs_iter}
\end{figure}

Figure \ref{fig:nmse_vs_iter} shows the empirical convergence of LO-BCQ across several block lengths and number of codebooks. Also, the MSE achieved by LO-BCQ is compared to baselines such as MXFP and VSQ. As shown, LO-BCQ converges to a lower MSE than the baselines. Further, we achieve better convergence for larger number of codebooks ($N_c$) and for a smaller block length ($L_b$), both of which increase the bitwidth of BCQ (see Eq \ref{eq:bitwidth_bcq}).


\subsection{Additional Accuracy Results}
%Table \ref{tab:lobcq_config} lists the various LOBCQ configurations and their corresponding bitwidths.
\begin{table}
\setlength{\tabcolsep}{4.75pt}
\begin{center}
\caption{\label{tab:lobcq_config} Various LO-BCQ configurations and their bitwidths.}
\begin{tabular}{|c||c|c|c|c||c|c||c|} 
\hline
 & \multicolumn{4}{|c||}{$L_b=8$} & \multicolumn{2}{|c||}{$L_b=4$} & $L_b=2$ \\
 \hline
 \backslashbox{$L_A$\kern-1em}{\kern-1em$N_c$} & 2 & 4 & 8 & 16 & 2 & 4 & 2 \\
 \hline
 64 & 4.25 & 4.375 & 4.5 & 4.625 & 4.375 & 4.625 & 4.625\\
 \hline
 32 & 4.375 & 4.5 & 4.625& 4.75 & 4.5 & 4.75 & 4.75 \\
 \hline
 16 & 4.625 & 4.75& 4.875 & 5 & 4.75 & 5 & 5 \\
 \hline
\end{tabular}
\end{center}
\end{table}

%\subsection{Perplexity achieved by various LO-BCQ configurations on Wikitext-103 dataset}

\begin{table} \centering
\begin{tabular}{|c||c|c|c|c||c|c||c|} 
\hline
 $L_b \rightarrow$& \multicolumn{4}{c||}{8} & \multicolumn{2}{c||}{4} & 2\\
 \hline
 \backslashbox{$L_A$\kern-1em}{\kern-1em$N_c$} & 2 & 4 & 8 & 16 & 2 & 4 & 2  \\
 %$N_c \rightarrow$ & 2 & 4 & 8 & 16 & 2 & 4 & 2 \\
 \hline
 \hline
 \multicolumn{8}{c}{GPT3-1.3B (FP32 PPL = 9.98)} \\ 
 \hline
 \hline
 64 & 10.40 & 10.23 & 10.17 & 10.15 &  10.28 & 10.18 & 10.19 \\
 \hline
 32 & 10.25 & 10.20 & 10.15 & 10.12 &  10.23 & 10.17 & 10.17 \\
 \hline
 16 & 10.22 & 10.16 & 10.10 & 10.09 &  10.21 & 10.14 & 10.16 \\
 \hline
  \hline
 \multicolumn{8}{c}{GPT3-8B (FP32 PPL = 7.38)} \\ 
 \hline
 \hline
 64 & 7.61 & 7.52 & 7.48 &  7.47 &  7.55 &  7.49 & 7.50 \\
 \hline
 32 & 7.52 & 7.50 & 7.46 &  7.45 &  7.52 &  7.48 & 7.48  \\
 \hline
 16 & 7.51 & 7.48 & 7.44 &  7.44 &  7.51 &  7.49 & 7.47  \\
 \hline
\end{tabular}
\caption{\label{tab:ppl_gpt3_abalation} Wikitext-103 perplexity across GPT3-1.3B and 8B models.}
\end{table}

\begin{table} \centering
\begin{tabular}{|c||c|c|c|c||} 
\hline
 $L_b \rightarrow$& \multicolumn{4}{c||}{8}\\
 \hline
 \backslashbox{$L_A$\kern-1em}{\kern-1em$N_c$} & 2 & 4 & 8 & 16 \\
 %$N_c \rightarrow$ & 2 & 4 & 8 & 16 & 2 & 4 & 2 \\
 \hline
 \hline
 \multicolumn{5}{|c|}{Llama2-7B (FP32 PPL = 5.06)} \\ 
 \hline
 \hline
 64 & 5.31 & 5.26 & 5.19 & 5.18  \\
 \hline
 32 & 5.23 & 5.25 & 5.18 & 5.15  \\
 \hline
 16 & 5.23 & 5.19 & 5.16 & 5.14  \\
 \hline
 \multicolumn{5}{|c|}{Nemotron4-15B (FP32 PPL = 5.87)} \\ 
 \hline
 \hline
 64  & 6.3 & 6.20 & 6.13 & 6.08  \\
 \hline
 32  & 6.24 & 6.12 & 6.07 & 6.03  \\
 \hline
 16  & 6.12 & 6.14 & 6.04 & 6.02  \\
 \hline
 \multicolumn{5}{|c|}{Nemotron4-340B (FP32 PPL = 3.48)} \\ 
 \hline
 \hline
 64 & 3.67 & 3.62 & 3.60 & 3.59 \\
 \hline
 32 & 3.63 & 3.61 & 3.59 & 3.56 \\
 \hline
 16 & 3.61 & 3.58 & 3.57 & 3.55 \\
 \hline
\end{tabular}
\caption{\label{tab:ppl_llama7B_nemo15B} Wikitext-103 perplexity compared to FP32 baseline in Llama2-7B and Nemotron4-15B, 340B models}
\end{table}

%\subsection{Perplexity achieved by various LO-BCQ configurations on MMLU dataset}


\begin{table} \centering
\begin{tabular}{|c||c|c|c|c||c|c|c|c|} 
\hline
 $L_b \rightarrow$& \multicolumn{4}{c||}{8} & \multicolumn{4}{c||}{8}\\
 \hline
 \backslashbox{$L_A$\kern-1em}{\kern-1em$N_c$} & 2 & 4 & 8 & 16 & 2 & 4 & 8 & 16  \\
 %$N_c \rightarrow$ & 2 & 4 & 8 & 16 & 2 & 4 & 2 \\
 \hline
 \hline
 \multicolumn{5}{|c|}{Llama2-7B (FP32 Accuracy = 45.8\%)} & \multicolumn{4}{|c|}{Llama2-70B (FP32 Accuracy = 69.12\%)} \\ 
 \hline
 \hline
 64 & 43.9 & 43.4 & 43.9 & 44.9 & 68.07 & 68.27 & 68.17 & 68.75 \\
 \hline
 32 & 44.5 & 43.8 & 44.9 & 44.5 & 68.37 & 68.51 & 68.35 & 68.27  \\
 \hline
 16 & 43.9 & 42.7 & 44.9 & 45 & 68.12 & 68.77 & 68.31 & 68.59  \\
 \hline
 \hline
 \multicolumn{5}{|c|}{GPT3-22B (FP32 Accuracy = 38.75\%)} & \multicolumn{4}{|c|}{Nemotron4-15B (FP32 Accuracy = 64.3\%)} \\ 
 \hline
 \hline
 64 & 36.71 & 38.85 & 38.13 & 38.92 & 63.17 & 62.36 & 63.72 & 64.09 \\
 \hline
 32 & 37.95 & 38.69 & 39.45 & 38.34 & 64.05 & 62.30 & 63.8 & 64.33  \\
 \hline
 16 & 38.88 & 38.80 & 38.31 & 38.92 & 63.22 & 63.51 & 63.93 & 64.43  \\
 \hline
\end{tabular}
\caption{\label{tab:mmlu_abalation} Accuracy on MMLU dataset across GPT3-22B, Llama2-7B, 70B and Nemotron4-15B models.}
\end{table}


%\subsection{Perplexity achieved by various LO-BCQ configurations on LM evaluation harness}

\begin{table} \centering
\begin{tabular}{|c||c|c|c|c||c|c|c|c|} 
\hline
 $L_b \rightarrow$& \multicolumn{4}{c||}{8} & \multicolumn{4}{c||}{8}\\
 \hline
 \backslashbox{$L_A$\kern-1em}{\kern-1em$N_c$} & 2 & 4 & 8 & 16 & 2 & 4 & 8 & 16  \\
 %$N_c \rightarrow$ & 2 & 4 & 8 & 16 & 2 & 4 & 2 \\
 \hline
 \hline
 \multicolumn{5}{|c|}{Race (FP32 Accuracy = 37.51\%)} & \multicolumn{4}{|c|}{Boolq (FP32 Accuracy = 64.62\%)} \\ 
 \hline
 \hline
 64 & 36.94 & 37.13 & 36.27 & 37.13 & 63.73 & 62.26 & 63.49 & 63.36 \\
 \hline
 32 & 37.03 & 36.36 & 36.08 & 37.03 & 62.54 & 63.51 & 63.49 & 63.55  \\
 \hline
 16 & 37.03 & 37.03 & 36.46 & 37.03 & 61.1 & 63.79 & 63.58 & 63.33  \\
 \hline
 \hline
 \multicolumn{5}{|c|}{Winogrande (FP32 Accuracy = 58.01\%)} & \multicolumn{4}{|c|}{Piqa (FP32 Accuracy = 74.21\%)} \\ 
 \hline
 \hline
 64 & 58.17 & 57.22 & 57.85 & 58.33 & 73.01 & 73.07 & 73.07 & 72.80 \\
 \hline
 32 & 59.12 & 58.09 & 57.85 & 58.41 & 73.01 & 73.94 & 72.74 & 73.18  \\
 \hline
 16 & 57.93 & 58.88 & 57.93 & 58.56 & 73.94 & 72.80 & 73.01 & 73.94  \\
 \hline
\end{tabular}
\caption{\label{tab:mmlu_abalation} Accuracy on LM evaluation harness tasks on GPT3-1.3B model.}
\end{table}

\begin{table} \centering
\begin{tabular}{|c||c|c|c|c||c|c|c|c|} 
\hline
 $L_b \rightarrow$& \multicolumn{4}{c||}{8} & \multicolumn{4}{c||}{8}\\
 \hline
 \backslashbox{$L_A$\kern-1em}{\kern-1em$N_c$} & 2 & 4 & 8 & 16 & 2 & 4 & 8 & 16  \\
 %$N_c \rightarrow$ & 2 & 4 & 8 & 16 & 2 & 4 & 2 \\
 \hline
 \hline
 \multicolumn{5}{|c|}{Race (FP32 Accuracy = 41.34\%)} & \multicolumn{4}{|c|}{Boolq (FP32 Accuracy = 68.32\%)} \\ 
 \hline
 \hline
 64 & 40.48 & 40.10 & 39.43 & 39.90 & 69.20 & 68.41 & 69.45 & 68.56 \\
 \hline
 32 & 39.52 & 39.52 & 40.77 & 39.62 & 68.32 & 67.43 & 68.17 & 69.30  \\
 \hline
 16 & 39.81 & 39.71 & 39.90 & 40.38 & 68.10 & 66.33 & 69.51 & 69.42  \\
 \hline
 \hline
 \multicolumn{5}{|c|}{Winogrande (FP32 Accuracy = 67.88\%)} & \multicolumn{4}{|c|}{Piqa (FP32 Accuracy = 78.78\%)} \\ 
 \hline
 \hline
 64 & 66.85 & 66.61 & 67.72 & 67.88 & 77.31 & 77.42 & 77.75 & 77.64 \\
 \hline
 32 & 67.25 & 67.72 & 67.72 & 67.00 & 77.31 & 77.04 & 77.80 & 77.37  \\
 \hline
 16 & 68.11 & 68.90 & 67.88 & 67.48 & 77.37 & 78.13 & 78.13 & 77.69  \\
 \hline
\end{tabular}
\caption{\label{tab:mmlu_abalation} Accuracy on LM evaluation harness tasks on GPT3-8B model.}
\end{table}

\begin{table} \centering
\begin{tabular}{|c||c|c|c|c||c|c|c|c|} 
\hline
 $L_b \rightarrow$& \multicolumn{4}{c||}{8} & \multicolumn{4}{c||}{8}\\
 \hline
 \backslashbox{$L_A$\kern-1em}{\kern-1em$N_c$} & 2 & 4 & 8 & 16 & 2 & 4 & 8 & 16  \\
 %$N_c \rightarrow$ & 2 & 4 & 8 & 16 & 2 & 4 & 2 \\
 \hline
 \hline
 \multicolumn{5}{|c|}{Race (FP32 Accuracy = 40.67\%)} & \multicolumn{4}{|c|}{Boolq (FP32 Accuracy = 76.54\%)} \\ 
 \hline
 \hline
 64 & 40.48 & 40.10 & 39.43 & 39.90 & 75.41 & 75.11 & 77.09 & 75.66 \\
 \hline
 32 & 39.52 & 39.52 & 40.77 & 39.62 & 76.02 & 76.02 & 75.96 & 75.35  \\
 \hline
 16 & 39.81 & 39.71 & 39.90 & 40.38 & 75.05 & 73.82 & 75.72 & 76.09  \\
 \hline
 \hline
 \multicolumn{5}{|c|}{Winogrande (FP32 Accuracy = 70.64\%)} & \multicolumn{4}{|c|}{Piqa (FP32 Accuracy = 79.16\%)} \\ 
 \hline
 \hline
 64 & 69.14 & 70.17 & 70.17 & 70.56 & 78.24 & 79.00 & 78.62 & 78.73 \\
 \hline
 32 & 70.96 & 69.69 & 71.27 & 69.30 & 78.56 & 79.49 & 79.16 & 78.89  \\
 \hline
 16 & 71.03 & 69.53 & 69.69 & 70.40 & 78.13 & 79.16 & 79.00 & 79.00  \\
 \hline
\end{tabular}
\caption{\label{tab:mmlu_abalation} Accuracy on LM evaluation harness tasks on GPT3-22B model.}
\end{table}

\begin{table} \centering
\begin{tabular}{|c||c|c|c|c||c|c|c|c|} 
\hline
 $L_b \rightarrow$& \multicolumn{4}{c||}{8} & \multicolumn{4}{c||}{8}\\
 \hline
 \backslashbox{$L_A$\kern-1em}{\kern-1em$N_c$} & 2 & 4 & 8 & 16 & 2 & 4 & 8 & 16  \\
 %$N_c \rightarrow$ & 2 & 4 & 8 & 16 & 2 & 4 & 2 \\
 \hline
 \hline
 \multicolumn{5}{|c|}{Race (FP32 Accuracy = 44.4\%)} & \multicolumn{4}{|c|}{Boolq (FP32 Accuracy = 79.29\%)} \\ 
 \hline
 \hline
 64 & 42.49 & 42.51 & 42.58 & 43.45 & 77.58 & 77.37 & 77.43 & 78.1 \\
 \hline
 32 & 43.35 & 42.49 & 43.64 & 43.73 & 77.86 & 75.32 & 77.28 & 77.86  \\
 \hline
 16 & 44.21 & 44.21 & 43.64 & 42.97 & 78.65 & 77 & 76.94 & 77.98  \\
 \hline
 \hline
 \multicolumn{5}{|c|}{Winogrande (FP32 Accuracy = 69.38\%)} & \multicolumn{4}{|c|}{Piqa (FP32 Accuracy = 78.07\%)} \\ 
 \hline
 \hline
 64 & 68.9 & 68.43 & 69.77 & 68.19 & 77.09 & 76.82 & 77.09 & 77.86 \\
 \hline
 32 & 69.38 & 68.51 & 68.82 & 68.90 & 78.07 & 76.71 & 78.07 & 77.86  \\
 \hline
 16 & 69.53 & 67.09 & 69.38 & 68.90 & 77.37 & 77.8 & 77.91 & 77.69  \\
 \hline
\end{tabular}
\caption{\label{tab:mmlu_abalation} Accuracy on LM evaluation harness tasks on Llama2-7B model.}
\end{table}

\begin{table} \centering
\begin{tabular}{|c||c|c|c|c||c|c|c|c|} 
\hline
 $L_b \rightarrow$& \multicolumn{4}{c||}{8} & \multicolumn{4}{c||}{8}\\
 \hline
 \backslashbox{$L_A$\kern-1em}{\kern-1em$N_c$} & 2 & 4 & 8 & 16 & 2 & 4 & 8 & 16  \\
 %$N_c \rightarrow$ & 2 & 4 & 8 & 16 & 2 & 4 & 2 \\
 \hline
 \hline
 \multicolumn{5}{|c|}{Race (FP32 Accuracy = 48.8\%)} & \multicolumn{4}{|c|}{Boolq (FP32 Accuracy = 85.23\%)} \\ 
 \hline
 \hline
 64 & 49.00 & 49.00 & 49.28 & 48.71 & 82.82 & 84.28 & 84.03 & 84.25 \\
 \hline
 32 & 49.57 & 48.52 & 48.33 & 49.28 & 83.85 & 84.46 & 84.31 & 84.93  \\
 \hline
 16 & 49.85 & 49.09 & 49.28 & 48.99 & 85.11 & 84.46 & 84.61 & 83.94  \\
 \hline
 \hline
 \multicolumn{5}{|c|}{Winogrande (FP32 Accuracy = 79.95\%)} & \multicolumn{4}{|c|}{Piqa (FP32 Accuracy = 81.56\%)} \\ 
 \hline
 \hline
 64 & 78.77 & 78.45 & 78.37 & 79.16 & 81.45 & 80.69 & 81.45 & 81.5 \\
 \hline
 32 & 78.45 & 79.01 & 78.69 & 80.66 & 81.56 & 80.58 & 81.18 & 81.34  \\
 \hline
 16 & 79.95 & 79.56 & 79.79 & 79.72 & 81.28 & 81.66 & 81.28 & 80.96  \\
 \hline
\end{tabular}
\caption{\label{tab:mmlu_abalation} Accuracy on LM evaluation harness tasks on Llama2-70B model.}
\end{table}

%\section{MSE Studies}
%\textcolor{red}{TODO}


\subsection{Number Formats and Quantization Method}
\label{subsec:numFormats_quantMethod}
\subsubsection{Integer Format}
An $n$-bit signed integer (INT) is typically represented with a 2s-complement format \citep{yao2022zeroquant,xiao2023smoothquant,dai2021vsq}, where the most significant bit denotes the sign.

\subsubsection{Floating Point Format}
An $n$-bit signed floating point (FP) number $x$ comprises of a 1-bit sign ($x_{\mathrm{sign}}$), $B_m$-bit mantissa ($x_{\mathrm{mant}}$) and $B_e$-bit exponent ($x_{\mathrm{exp}}$) such that $B_m+B_e=n-1$. The associated constant exponent bias ($E_{\mathrm{bias}}$) is computed as $(2^{{B_e}-1}-1)$. We denote this format as $E_{B_e}M_{B_m}$.  

\subsubsection{Quantization Scheme}
\label{subsec:quant_method}
A quantization scheme dictates how a given unquantized tensor is converted to its quantized representation. We consider FP formats for the purpose of illustration. Given an unquantized tensor $\bm{X}$ and an FP format $E_{B_e}M_{B_m}$, we first, we compute the quantization scale factor $s_X$ that maps the maximum absolute value of $\bm{X}$ to the maximum quantization level of the $E_{B_e}M_{B_m}$ format as follows:
\begin{align}
\label{eq:sf}
    s_X = \frac{\mathrm{max}(|\bm{X}|)}{\mathrm{max}(E_{B_e}M_{B_m})}
\end{align}
In the above equation, $|\cdot|$ denotes the absolute value function.

Next, we scale $\bm{X}$ by $s_X$ and quantize it to $\hat{\bm{X}}$ by rounding it to the nearest quantization level of $E_{B_e}M_{B_m}$ as:

\begin{align}
\label{eq:tensor_quant}
    \hat{\bm{X}} = \text{round-to-nearest}\left(\frac{\bm{X}}{s_X}, E_{B_e}M_{B_m}\right)
\end{align}

We perform dynamic max-scaled quantization \citep{wu2020integer}, where the scale factor $s$ for activations is dynamically computed during runtime.

\subsection{Vector Scaled Quantization}
\begin{wrapfigure}{r}{0.35\linewidth}
  \centering
  \includegraphics[width=\linewidth]{sections/figures/vsquant.jpg}
  \caption{\small Vectorwise decomposition for per-vector scaled quantization (VSQ \citep{dai2021vsq}).}
  \label{fig:vsquant}
\end{wrapfigure}
During VSQ \citep{dai2021vsq}, the operand tensors are decomposed into 1D vectors in a hardware friendly manner as shown in Figure \ref{fig:vsquant}. Since the decomposed tensors are used as operands in matrix multiplications during inference, it is beneficial to perform this decomposition along the reduction dimension of the multiplication. The vectorwise quantization is performed similar to tensorwise quantization described in Equations \ref{eq:sf} and \ref{eq:tensor_quant}, where a scale factor $s_v$ is required for each vector $\bm{v}$ that maps the maximum absolute value of that vector to the maximum quantization level. While smaller vector lengths can lead to larger accuracy gains, the associated memory and computational overheads due to the per-vector scale factors increases. To alleviate these overheads, VSQ \citep{dai2021vsq} proposed a second level quantization of the per-vector scale factors to unsigned integers, while MX \citep{rouhani2023shared} quantizes them to integer powers of 2 (denoted as $2^{INT}$).

\subsubsection{MX Format}
The MX format proposed in \citep{rouhani2023microscaling} introduces the concept of sub-block shifting. For every two scalar elements of $b$-bits each, there is a shared exponent bit. The value of this exponent bit is determined through an empirical analysis that targets minimizing quantization MSE. We note that the FP format $E_{1}M_{b}$ is strictly better than MX from an accuracy perspective since it allocates a dedicated exponent bit to each scalar as opposed to sharing it across two scalars. Therefore, we conservatively bound the accuracy of a $b+2$-bit signed MX format with that of a $E_{1}M_{b}$ format in our comparisons. For instance, we use E1M2 format as a proxy for MX4.

\begin{figure}
    \centering
    \includegraphics[width=1\linewidth]{sections//figures/BlockFormats.pdf}
    \caption{\small Comparing LO-BCQ to MX format.}
    \label{fig:block_formats}
\end{figure}

Figure \ref{fig:block_formats} compares our $4$-bit LO-BCQ block format to MX \citep{rouhani2023microscaling}. As shown, both LO-BCQ and MX decompose a given operand tensor into block arrays and each block array into blocks. Similar to MX, we find that per-block quantization ($L_b < L_A$) leads to better accuracy due to increased flexibility. While MX achieves this through per-block $1$-bit micro-scales, we associate a dedicated codebook to each block through a per-block codebook selector. Further, MX quantizes the per-block array scale-factor to E8M0 format without per-tensor scaling. In contrast during LO-BCQ, we find that per-tensor scaling combined with quantization of per-block array scale-factor to E4M3 format results in superior inference accuracy across models. 

\end{document}
