\begin{thebibliography}{99}

\bibitem{WIRED} C. Anderson,
The end of theory: the data deluge makes the scientific method obsolete. 
Wired; 16. https://www .wired.com/2008/06/pb-theory/ (2008)

\bibitem{CoD} R. Bellman,
Dynamic Programming, Princeton University Press, Princeton, NJ, (1957)

\bibitem{NATAL} F. Bouchut, Y. Jobic , R. Natalini, R. Occelli, V. Pavan,
Second-order entropy satisfying BGK-FVS schemes for
incompressible Navier-Stokes equations,
SMAI Journal of Computational Mathematics, Vol. 4, 1-56 (2018)

\bibitem{META} G. Bussi, A. Laio,
Using metadynamics to explore complex free-energy landscapes
Nature Reviews Physics 2 (4), 200-212 498 (2020)

\bibitem{CP} R. Car, M. Parrinello,
Unified Approach for Molecular Dynamics and Density-Functional Theory,
Phys. Rev. Lett. 55, 2471 (1985)

\bibitem{NELSON} T Chotibut, DR Nelson,
Population genetics with fluctuating population sizes,
Journal of Statistical Physics 167 (3), 777-791, 2017

\bibitem{PVC} PV Coveney, ER Dougherty, RR Highfield, 
Big data need big theory too, 
Philosophical Transactions of the Royal Society A: 
Mathematical, Physical and Engineering Sciences; 374:20160153 (2016)

\bibitem{PEDRO} P. Domingos,  
The master algorithm: How the quest for the ultimate learning machine 
will remake our world. New York: Basic Books (2015)

\bibitem{FRENKEL} D. Frenkel, B. Smit,
Understanding Molecular Simulation,
Elsevier, New York, (2001)

\bibitem{DEEP} J Jumper, R Evans, A Pritzel, T Green, M Figurnov, O Ronneberger, 
K Tunyasuvunakool, R Bates, A Žídek, A Potapenko, 
A Bridgland, C Meyer, S A. A. Kohl, AJ. Ballard, 
A Cowie, B Romera-Paredes, S Nikolov, R Jain, 
J Adler, T Back, S Petersen, D Reiman, 
E Clancy, M Zielinski, …Demis Hassabis,
Highly accurate protein structure prediction with AlphaFold,
Nature, 596,583–589 (2021)

\bibitem{ML} Y. LeCun, J. Bengio and G. Hilton,
Deep Learning, Nature 521 (7553), 436-444 (2015)

\bibitem{CNN} Y LeCun, Y Bengio,
Convolutional networks for images, speech, and time series,
The handbook of brain theory and neural networks 3361 (10), 1995


\bibitem{RASIN} I. Rasin, S. Succi and W. Miller,
Phase-field lattice kinetic scheme for the numerical simulation of dendritic growth,
Physical Review E—Statistical, Nonlinear, and Soft Matter Physics,
72, 6, 066705 (2005)

\bibitem{BACK} D.E. Rumelhart, G. Hinton and R.J. Williams,
Learning representations by back-propagating errors,
Nature, 323 (9) 533, (1986)

\bibitem{OUP01} S. Succi,
The Lattice Boltzmann Equation: Theory and Applications 
Oxford: Oxford University Press (2001)

%\bibitem{OUP18} S. Succi,
%The Lattice Boltzmann Equation for Complex States of Flowing Matter,
%Oxford: Oxford University Press (2018)

\bibitem{SC} S.Succi, PV Coveney, 
Big data: the end of the scientific method? 
Philosophical Transactions of the Royal 
Society A; 377:201801 (2019)

\bibitem{CHAT} S. Succi,
Chatbots and Zero Sales Resistence, 
Frontiers in Physics 12, 1484701 (2024)

\bibitem{GOOGLE} A. Vaswani, N. Shazeer, N. Parmar, J. Uszkoreit, 
L. Jones, A. N. Gomez, L. Kaiser, I. Polosukhin,
Attention is all you need,
Advances in Neural Information Processing Systems, 30.
Curran Associates, Inc. arXiv:1706.03762 (2017).

\bibitem{E1} Weinan E, 
A proposal on machine learning via dynamical systems. Comm. Math. Stat., 5(1), 1-11, 2017.

\bibitem{E2} Qianxiao Li Weinan E, 
Machine Learning and Dynamical Systems, November 01, 2021
https://www.siam.org/publications/siam-news/articles/machine-learning-and-dynamical-systems/

\bibitem{E3} Weinan, E; Ma, C; (...); Wojtowytsch, S,
Towards a Mathematical Understanding of Neural Network-Based Machine Learning: What We Know and What We Don't
CSIAM TRANSACTIONS ON APPLIED MATHEMATICS, 1 (4) , pp.561-615, 2020

\bibitem{CLU} Hui Yin, Amir Aryani, Stephen Petrie, Aishwarya Nambissan, Aland Astudillo, Shengyuan Cao,
A Rapid Review of Clustering Algorithms,
cs.arXiv:2401.07389

%\bibitem{OUP22} S. Succi,
%Sailing the ocean of complexity: lessons
%from the physics-biology interface, 
%Oxford: Oxford University Press (2022)
%
%\bibitem{NAT} S. Succi,
%Of naturalness and complexity, 
%The European Physical Journal Plus 134 (3), 97 (2019)
%
%\bibitem{ESS} M Briscolini, P Santangelo, S Succi, R Benzi,
%Extended self-similarity in the numerical simulation of 
%three-dimensional homogeneous flows,
%Physical Review E 50 (3), R1745 (1994)


%\bibitem{LONGO} CS Calude, G. Longo,
%The Deluge of Spurious Correlations in Big Data,
%Foundatons of Science,22, 3, 595–612 (2017)

%\bibitem{PVCNature}
%https://www.nature.com/articles/d41586-024-01626-z

%\bibitem{PVC2024} Coveney PV, Highfield RR, 
%Artificial Intelligence Must Be Made More Scientific,
%J. of Chem. Information and Modelling, 
%https://doi.org./10.121/acs.jcim/4c01091 (2024)

%\bibitem{CAI1} V Del Tatto, G Fortunato, D Bueti, A Laio,
%Robust inference of causality in high-dimensional dynamical 
%processes from the Information Imbalance of distance ranks,
%Proceedings of the National Academy of Sciences 121 (19), e2317256121 (2024)

%\bibitem{CAI2} G. Gendron, M. Witbrock, G. Dobbie, 
%A survey of methods, challenges and perspectives in causality. 
%arXiv [Preprint] (2023). https://doi.org/10.48550/arXiv.2302.00293

%\bibitem{CAI3} S. Goldt, M. Mezard,F. Krzakala,  L. Zdeborová,
%Modeling the Influence of Data Structure on Learning 
%in Neural Networks: The Hidden Manifold Model,
%Phys. Rev X 10, 041044 (2020)

%\bibitem{LEW} CS Lewis,
%The Abolition of Man, Harper-Collins, 1944

%\bibitem{ZSR} G Riva, BK Wiederhold, S Succi,
%Zero sales resistance: The dark side of big data and artificial intelligence,
%Cyberpsychology, Behavior, and Social Networking, 25 (3), 169-173 (2022)

%\bibitem{WEA} Cate O'Neil, 
%Weapons of math destruction, 
%How Big Data Increases Inequality and Threatens Democracy,
%Penguin Books, (2017)

%\bibitem{VU} Coveney PV, Highfield RR, 
%Virtual You, Princeton University Press, (2023)

%\bibitem{BEN} P. Benanti,
%https://link.springer.com/article/10.1007/s44163-023-00056-6

%\bibitem{KEA} M. Kearns and A. Roth, 
%The ethical algorithm, the science of socially aware algorithm design,
%Oxford U.P., (2024)

%\bibitem{RMT4ML} Romain Couillet, Zhenyu Liao
%Random Matrix Methods for Machine Learning,
%Cambridge U.P., (2024)

\end{thebibliography}

