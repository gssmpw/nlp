\section{Conclusion and Future Work}
\label{sec:conclusion}

This paper introduced the Emirates Multi-Task (EMT) dataset, which includes tracking, trajectory prediction, and intention prediction datasets. Each dataset is accompanied by evaluation code and results from conducted experiments. For tracking, the fine-tuned detector significantly improves performance, reinforcing the importance of using region-specific data for effective adaptation of deep learning algorithms in autonomous vehicle applications. This highlights the necessity of fine-tuning models on local data to achieve optimal performance. For trajectory prediction, we evaluate both sequential and interaction-aware models, providing training pipelines for sequential and frame-based learning. The reported benchmark results serve as a reference for future research. For intention prediction, two evaluation settings are provided: cross-validation and train/test splits. We report cross-validation results as a benchmark, while the train/test setting is intended for assessing model generalization and performance on previously unseen intention classes. The EMT dataset is designed to facilitate the evaluation of models on the unique characteristics of the Gulf region, offering a valuable resource for advancing research in autonomous driving.


The evaluated algorithms operate solely on relative distances computed from absolute positions extracted from images. We leave it to the community to experiment with and design models that integrate visual cues for tracking and prediction. Additionally, the EMT dataset includes numerous highway scenarios that reflect regional traffic conditions. Balancing the dataset to address underrepresented patterns and improving model generalizability, considering the diversity gap between training and testing datasets, is left for researchers to explore.

The current dataset represents the first version collected from the region using a frontal camera. The primary direction for future work is to integrate Sim2Real scene generation to expand the dataset by including underrepresented scenarios. Additionally, we plan to collect a multimodal dataset incorporating LiDAR, camera data, and localization information. This dataset will address underrepresented scenarios identified during the analysis of the EMT dataset, ensuring a more comprehensive resource for the safe deployment of autonomous vehicles in the Gulf region. The dataset will also be sufficiently large to support large model training. For evaluation, future work will involve cross-evaluation by training models on multiple existing datasets and evaluating them on regional data. This process will incrementally include fine-tuning samples to assess generalization and model performance on rare scenarios.