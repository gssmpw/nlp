\section{Introduction}


\IEEEPARstart{A}{s} autonomous driving technology advances, the ability of data-driven models to generalize across diverse road environments and conditions is essential for safe operation, but remains a significant challenge. To achieve robust generalization, it is critical to train models on datasets that capture a wide range of traffic scenes and characteristics. Current autonomous driving datasets provide extensive coverage of regions like the USA \cite{nuscenes2019, Houston2020OneTA, Argoverse, Argoverse2, TrustButVerify}, Europe \cite{Geiger2013IJRR, Liao2022PAMI}, and parts of Asia, including China and Singapore \cite{10.1609/aaai.v33i01.33016120, nuscenes2019}. However, the Arab Gulf region, with its unique driving conditions, remains underrepresented. To address this gap, we introduce the Emirates Multi-Task (EMT) dataset, collected in the United Arab Emirates (UAE) to capture the region's distinct traffic conditions. This region offers diverse driving challenges due to its range of road layouts, including expansive highways, urban areas, and complex city junctions. Additionally, driving behavior in the UAE reflects a blend of modern regulations and traditional practices. Our dataset was gathered using dash cameras mounted on two vehicles driving across the UAE’s largest cities and on intercity routes, covering over 150 kilometers. The annotated dataset supports multiple benchmarks, including tracking, trajectory prediction, and intention prediction, aimed at advancing models robustness in complex driving environments.

\begin{figure}[t!]
\centering
\includegraphics[width=0.49\textwidth]{figures/dataset_sample_vert.pdf}
\caption{Samples from EMT dataset capturing highway scenarios in day and night time, clear and rainy weather.}
\label{fig:sample}
\end{figure}


The \textbf{tracking benchmark dataset} is designed to evaluate the ability of algorithms to accurately identify and maintain consistent object tracking over time in a complex driving environment. Similar to current state-of-the-art (SOTA) tracking benchmarks \cite{Geiger2012AreWR, nuscenes2019, 9709630}, it focuses on the motion of vehicles, pedestrians, cyclists, and motorbikes, captured from a frontal camera perspective. The benchmark is designed to test tracking models under varying levels of traffic congestion and frequent lane changes. The dataset contains 8,806 unique tracking IDs, including 8,076 vehicles, 568 pedestrians, 158 motorbikes and 14 cyclists, and with a mean tracking duration of 6.5 seconds.

Accurate trajectory prediction is a cornerstone of safe autonomous driving, enabling vehicles to navigate complex interactions and adjust their planning accordingly. The designed \textbf{trajectory prediction benchmark dataset} challenges forecasting models to predict trajectories in heterogeneous traffic and to generalize effectively across scenarios involving interacting agents, forecasts in large intersections and roundabouts, and multi-agent dynamics. The dataset contains 4,821 unique agents, including 4,347 vehicles and 386 pedestrians. Details on the size of the dataset, splits into past trajectories and prediction horizons, as well as the shifting window approach, are provided in Section \ref{sec:dataset}.

The \textbf{intention prediction benchmark} is designed to evaluate the ability of autonomous systems to anticipate the future actions of traffic participants. This benchmark enables systems to infer the likely movements of nearby agents based on their current trajectories, positions, and the surrounding context. The dataset includes detailed annotations for vehicle maneuver intentions and pedestrian behaviors. For vehicles, the labeled maneuvers comprise high-level intended actions that define their future trajectories, including: turn left/right, keep lane, merge left/right, brake, stop and reverse. For pedestrians, the dataset provides labels for four intention classes: waiting to cross, crossing, walking (on pavements or sidewalks), and stopping (e.g., waiting at bus stops). These annotations aim to provide a comprehensive understanding of pedestrian and vehicle behaviors in diverse scenarios. In total the dataset contains 4,921 sequences of agents trajectories with intentions. 

Each task-specific dataset is complemented by evaluated models and conducted experiments. For multi-agent tracking, we evaluate Kalman filter-based SOTA trackers within a pipeline that includes three configurations: off-shelf detector, a fine-tuned detector, and ground-truth detections. This setup allows us to systematically assess the trackers' abilities to handle occlusions and detector errors. For trajectory prediction, we selected three deep-learning architectures, focusing on their capacity to capture temporal dependencies and interaction dynamics in high-density scenarios. Lastly, for the intention prediction task, we conducted evaluation to assess LSTM-based models' ability to predict future intentions based on past trajectories.


The structure of the paper is as follows. Section \ref{sec:rel_work} reviews related work, focusing on existing datasets and models for multi-agent tracking, intention prediction, and trajectory prediction. Section \ref{sec:dataset} outlines the data collection methodology, explains the annotation process, and summarizes the characteristics of the collected dataset. Section \ref{sec:eval_protocol} outlines the evaluation protocol, lists the utilized evaluated models, and presents the results of the conducted experiments for all three datasets. Finally, Section \ref{sec:conclusion} concludes the paper with a summary of findings and directions for future research. 
