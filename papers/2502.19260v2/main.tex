%% bare_jrnl.tex
%% V1.4b
%% 2015/08/26
%% by Michael Shell
%% see http://www.michaelshell.org/
%% for current contact information.

%%
%% This is a skeleton file demonstrating the use of IEEEtran.cls
%% (requires IEEEtran.cls version 1.8b or later) with an IEEE
%% journal paper.
%%
%% Support sites:
%% http://www.michaelshell.org/tex/ieeetran/
%% http://www.ctan.org/pkg/ieeetran
%% and
%% http://www.ieee.org/

%%*************************************************************************
%% Legal Notice:
%% This code is offered as-is without any warranty either expressed or
%% implied; without even the implied warranty of MERCHANTABILITY or
%% FITNESS FOR A PARTICULAR PURPOSE! 
%% User assumes all risk.
%% In no event shall the IEEE or any contributor to this code be liable for
%% any damages or losses, including, but not limited to, incidental,
%% consequential, or any other damages, resulting from the use or misuse
%% of any information contained here.
%%
%% All comments are the opinions of their respective authors and are not
%% necessarily endorsed by the IEEE.
%%
%% This work is distributed under the LaTeX Project Public License (LPPL)
%% ( http://www.latex-project.org/ ) version 1.3, and may be freely used,
%% distributed and modified. A copy of the LPPL, version 1.3, is included
%% in the base LaTeX documentation of all distributions of LaTeX released
%% 2003/12/01 or later.
%% Retain all contribution notices and credits.
%% ** Modified files should be clearly indicated as such, including  **
%% ** renaming them and changing author support contact information. **
%%*************************************************************************


% *** Authors should verify (and, if needed, correct) their LaTeX system  ***
% *** with the testflow diagnostic prior to trusting their LaTeX platform ***
% *** with production work. The IEEE's font choices and paper sizes can   ***
% *** trigger bugs that do not appear when using other class files.       ***                          ***
% The testflow support page is at:
% http://www.michaelshell.org/tex/testflow/



\documentclass[journal]{IEEEtran}

\usepackage{xcolor}
\usepackage{graphicx}
\usepackage{float}
\usepackage{amsmath}
\usepackage{xltabular}
\usepackage{booktabs}
\usepackage{pdflscape}
\usepackage{adjustbox}
\usepackage{multirow}
\usepackage{comment} 
\usepackage{makecell}
\usepackage{array}
\usepackage{booktabs}
\usepackage{hyperref}
\usepackage{booktabs} % For formal tables
\usepackage{longtable} % For tables that span multiple pages
\usepackage[numbers,sort&compress]{natbib}
\usepackage{listings}
\lstset{
    basicstyle=\ttfamily\scriptsize, % Use very small monospace font
    breaklines=true,           % Allow line breaking
    keywordstyle=\color{blue}, % Style for keywords
    stringstyle=\color{red}    % Style for strings
}
\renewcommand{\arraystretch}{1.5}
%
% If IEEEtran.cls has not been installed into the LaTeX system files,
% manually specify the path to it like:
% \documentclass[journal]{../sty/IEEEtran}





% Some very useful LaTeX packages include:
% (uncomment the ones you want to load)


% *** MISC UTILITY PACKAGES ***
%
%\usepackage{ifpdf}
% Heiko Oberdiek's ifpdf.sty is very useful if you need conditional
% compilation based on whether the output is pdf or dvi.
% usage:
% \ifpdf
%   % pdf code
% \else
%   % dvi code
% \fi
% The latest version of ifpdf.sty can be obtained from:
% http://www.ctan.org/pkg/ifpdf
% Also, note that IEEEtran.cls V1.7 and later provides a builtin
% \ifCLASSINFOpdf conditional that works the same way.
% When switching from latex to pdflatex and vice-versa, the compiler may
% have to be run twice to clear warning/error messages.






% *** CITATION PACKAGES ***
%
%\usepackage{cite}
% cite.sty was written by Donald Arseneau
% V1.6 and later of IEEEtran pre-defines the format of the cite.sty package
% \cite{} output to follow that of the IEEE. Loading the cite package will
% result in citation numbers being automatically sorted and properly
% "compressed/ranged". e.g., [1], [9], [2], [7], [5], [6] without using
% cite.sty will become [1], [2], [5]--[7], [9] using cite.sty. cite.sty's
% \cite will automatically add leading space, if needed. Use cite.sty's
% noadjust option (cite.sty V3.8 and later) if you want to turn this off
% such as if a citation ever needs to be enclosed in parenthesis.
% cite.sty is already installed on most LaTeX systems. Be sure and use
% version 5.0 (2009-03-20) and later if using hyperref.sty.
% The latest version can be obtained at:
% http://www.ctan.org/pkg/cite
% The documentation is contained in the cite.sty file itself.






% *** GRAPHICS RELATED PACKAGES ***
%
\ifCLASSINFOpdf
  % \usepackage[pdftex]{graphicx}
  % declare the path(s) where your graphic files are
  % \graphicspath{{../pdf/}{../jpeg/}}
  % and their extensions so you won't have to specify these with
  % every instance of \includegraphics
  % \DeclareGraphicsExtensions{.pdf,.jpeg,.png}
\else
  % or other class option (dvipsone, dvipdf, if not using dvips). graphicx
  % will default to the driver specified in the system graphics.cfg if no
  % driver is specified.
  % \usepackage[dvips]{graphicx}
  % declare the path(s) where your graphic files are
  % \graphicspath{{../eps/}}
  % and their extensions so you won't have to specify these with
  % every instance of \includegraphics
  % \DeclareGraphicsExtensions{.eps}
\fi
% graphicx was written by David Carlisle and Sebastian Rahtz. It is
% required if you want graphics, photos, etc. graphicx.sty is already
% installed on most LaTeX systems. The latest version and documentation
% can be obtained at: 
% http://www.ctan.org/pkg/graphicx
% Another good source of documentation is "Using Imported Graphics in
% LaTeX2e" by Keith Reckdahl which can be found at:
% http://www.ctan.org/pkg/epslatex
%
% latex, and pdflatex in dvi mode, support graphics in encapsulated
% postscript (.eps) format. pdflatex in pdf mode supports graphics
% in .pdf, .jpeg, .png and .mps (metapost) formats. Users should ensure
% that all non-photo figures use a vector format (.eps, .pdf, .mps) and
% not a bitmapped formats (.jpeg, .png). The IEEE frowns on bitmapped formats
% which can result in "jaggedy"/blurry rendering of lines and letters as
% well as large increases in file sizes.
%
% You can find documentation about the pdfTeX application at:
% http://www.tug.org/applications/pdftex





% *** MATH PACKAGES ***
%
%\usepackage{amsmath}
% A popular package from the American Mathematical Society that provides
% many useful and powerful commands for dealing with mathematics.
%
% Note that the amsmath package sets \interdisplaylinepenalty to 10000
% thus preventing page breaks from occurring within multiline equations. Use:
%\interdisplaylinepenalty=2500
% after loading amsmath to restore such page breaks as IEEEtran.cls normally
% does. amsmath.sty is already installed on most LaTeX systems. The latest
% version and documentation can be obtained at:
% http://www.ctan.org/pkg/amsmath





% *** SPECIALIZED LIST PACKAGES ***
%
%\usepackage{algorithmic}
% algorithmic.sty was written by Peter Williams and Rogerio Brito.
% This package provides an algorithmic environment fo describing algorithms.
% You can use the algorithmic environment in-text or within a figure
% environment to provide for a floating algorithm. Do NOT use the algorithm
% floating environment provided by algorithm.sty (by the same authors) or
% algorithm2e.sty (by Christophe Fiorio) as the IEEE does not use dedicated
% algorithm float types and packages that provide these will not provide
% correct IEEE style captions. The latest version and documentation of
% algorithmic.sty can be obtained at:
% http://www.ctan.org/pkg/algorithms
% Also of interest may be the (relatively newer and more customizable)
% algorithmicx.sty package by Szasz Janos:
% http://www.ctan.org/pkg/algorithmicx




% *** ALIGNMENT PACKAGES ***
%
%\usepackage{array}
% Frank Mittelbach's and David Carlisle's array.sty patches and improves
% the standard LaTeX2e array and tabular environments to provide better
% appearance and additional user controls. As the default LaTeX2e table
% generation code is lacking to the point of almost being broken with
% respect to the quality of the end results, all users are strongly
% advised to use an enhanced (at the very least that provided by array.sty)
% set of table tools. array.sty is already installed on most systems. The
% latest version and documentation can be obtained at:
% http://www.ctan.org/pkg/array


% IEEEtran contains the IEEEeqnarray family of commands that can be used to
% generate multiline equations as well as matrices, tables, etc., of high
% quality.




% *** SUBFIGURE PACKAGES ***
%\ifCLASSOPTIONcompsoc
%  \usepackage[caption=false,font=normalsize,labelfont=sf,textfont=sf]{subfig}
%\else
%  \usepackage[caption=false,font=footnotesize]{subfig}
%\fi
% subfig.sty, written by Steven Douglas Cochran, is the modern replacement
% for subfigure.sty, the latter of which is no longer maintained and is
% incompatible with some LaTeX packages including fixltx2e. However,
% subfig.sty requires and automatically loads Axel Sommerfeldt's caption.sty
% which will override IEEEtran.cls' handling of captions and this will result
% in non-IEEE style figure/table captions. To prevent this problem, be sure
% and invoke subfig.sty's "caption=false" package option (available since
% subfig.sty version 1.3, 2005/06/28) as this is will preserve IEEEtran.cls
% handling of captions.
% Note that the Computer Society format requires a larger sans serif font
% than the serif footnote size font used in traditional IEEE formatting
% and thus the need to invoke different subfig.sty package options depending
% on whether compsoc mode has been enabled.
%
% The latest version and documentation of subfig.sty can be obtained at:
% http://www.ctan.org/pkg/subfig




% *** FLOAT PACKAGES ***
%
%\usepackage{fixltx2e}
% fixltx2e, the successor to the earlier fix2col.sty, was written by
% Frank Mittelbach and David Carlisle. This package corrects a few problems
% in the LaTeX2e kernel, the most notable of which is that in current
% LaTeX2e releases, the ordering of single and double column floats is not
% guaranteed to be preserved. Thus, an unpatched LaTeX2e can allow a
% single column figure to be placed prior to an earlier double column
% figure.
% Be aware that LaTeX2e kernels dated 2015 and later have fixltx2e.sty's
% corrections already built into the system in which case a warning will
% be issued if an attempt is made to load fixltx2e.sty as it is no longer
% needed.
% The latest version and documentation can be found at:
% http://www.ctan.org/pkg/fixltx2e


%\usepackage{stfloats}
% stfloats.sty was written by Sigitas Tolusis. This package gives LaTeX2e
% the ability to do double column floats at the bottom of the page as well
% as the top. (e.g., "\begin{figure*}[!b]" is not normally possible in
% LaTeX2e). It also provides a command:
%\fnbelowfloat 
% to enable the placement of footnotes below bottom floats (the standard
% LaTeX2e kernel puts them above bottom floats). This is an invasive package
% which rewrites many portions of the LaTeX2e float routines. It may not work
% with other packages that modify the LaTeX2e float routines. The latest
% version and documentation can be obtained at:
% http://www.ctan.org/pkg/stfloats
% Do not use the stfloats baselinefloat ability as the IEEE does not allow
% \baselineskip to stretch. Authors submitting work to the IEEE should note
% that the IEEE rarely uses double column equations and that authors should try
% to avoid such use. Do not be tempted to use the cuted.sty or midfloat.sty
% packages (also by Sigitas Tolusis) as the IEEE does not format its papers in
% such ways.
% Do not attempt to use stfloats with fixltx2e as they are incompatible.
% Instead, use Morten Hogholm'a dblfloatfix which combines the features
% of both fixltx2e and stfloats:
%
% \usepackage{dblfloatfix}
% The latest version can be found at:
% http://www.ctan.org/pkg/dblfloatfix




%\ifCLASSOPTIONcaptionsoff
%  \usepackage[nomarkers]{endfloat}
% \let\MYoriglatexcaption\caption
% \renewcommand{\caption}[2][\relax]{\MYoriglatexcaption[#2]{#2}}
%\fi
% endfloat.sty was written by James Darrell McCauley, Jeff Goldberg and 
% Axel Sommerfeldt. This package may be useful when used in conjunction with 
% IEEEtran.cls'  captionsoff option. Some IEEE journals/societies require that
% submissions have lists of figures/tables at the end of the paper and that
% figures/tables without any captions are placed on a page by themselves at
% the end of the document. If needed, the draftcls IEEEtran class option or
% \CLASSINPUTbaselinestretch interface can be used to increase the line
% spacing as well. Be sure and use the nomarkers option of endfloat to
% prevent endfloat from "marking" where the figures would have been placed
% in the text. The two hack lines of code above are a slight modification of
% that suggested by in the endfloat docs (section 8.4.1) to ensure that
% the full captions always appear in the list of figures/tables - even if
% the user used the short optional argument of \caption[]{}.
% IEEE papers do not typically make use of \caption[]'s optional argument,
% so this should not be an issue. A similar trick can be used to disable
% captions of packages such as subfig.sty that lack options to turn off
% the subcaptions:
% For subfig.sty:
% \let\MYorigsubfloat\subfloat
% \renewcommand{\subfloat}[2][\relax]{\MYorigsubfloat[]{#2}}
% However, the above trick will not work if both optional arguments of
% the \subfloat command are used. Furthermore, there needs to be a
% description of each subfigure *somewhere* and endfloat does not add
% subfigure captions to its list of figures. Thus, the best approach is to
% avoid the use of subfigure captions (many IEEE journals avoid them anyway)
% and instead reference/explain all the subfigures within the main caption.
% The latest version of endfloat.sty and its documentation can obtained at:
% http://www.ctan.org/pkg/endfloat
%
% The IEEEtran \ifCLASSOPTIONcaptionsoff conditional can also be used
% later in the document, say, to conditionally put the References on a 
% page by themselves.




% *** PDF, URL AND HYPERLINK PACKAGES ***
%
%\usepackage{url}
% url.sty was written by Donald Arseneau. It provides better support for
% handling and breaking URLs. url.sty is already installed on most LaTeX
% systems. The latest version and documentation can be obtained at:
% http://www.ctan.org/pkg/url
% Basically, \url{my_url_here}.




% *** Do not adjust lengths that control margins, column widths, etc. ***
% *** Do not use packages that alter fonts (such as pslatex).         ***
% There should be no need to do such things with IEEEtran.cls V1.6 and later.
% (Unless specifically asked to do so by the journal or conference you plan
% to submit to, of course. )


% correct bad hyphenation here
\hyphenation{op-tical net-works semi-conduc-tor}


\begin{document}
%
% paper title
% Titles are generally capitalized except for words such as a, an, and, as,
% at, but, by, for, in, nor, of, on, or, the, to and up, which are usually
% not capitalized unless they are the first or last word of the title.
% Linebreaks \\ can be used within to get better formatting as desired.
% Do not put math or special symbols in the title.
\title{EMT: A Visual Multi-Task Benchmark Dataset for Autonomous Driving in the Arab Gulf Region}
%
%
% author names and IEEE memberships
% note positions of commas and nonbreaking spaces ( ~ ) LaTeX will not break
% a structure at a ~ so this keeps an author's name from being broken across
% two lines.
% use \thanks{} to gain access to the first footnote area
% a separate \thanks must be used for each paragraph as LaTeX2e's \thanks
% was not built to handle multiple paragraphs
%


\author{Nadya Abdel Madjid$^{1,4*}$, Murad Mebrahtu$^{1,4*}$, Abdelmoamen Nasser$^{1,4}$,  Bilal Hassan$^{2}$, Naoufel Werghi$^{1}$, \\ Jorge Dias$^{3,4}$, and Majid Khonji$^{1,4}$% 
\thanks{This work was supported by Khalifa University of Science and Technology under Award No. RIG-2023-117. Corresponding authors: nadya.abdelmadjid@gmail.com, murad.mebrahtu@ku.ac.ae, majid.khonji@ku.ac.ae}% 
\thanks{$^{*}$ Equal contribution}
\thanks{$^{1}$ Computer Science, Khalifa University, Abu Dhabi, UAE}%
\thanks{$^{2}$ Computer Science, New York University, Abu Dhabi, UAE}%
\thanks{$^{3}$ Computer and Information Engineering, Khalifa University, Abu Dhabi, UAE}%
\thanks{$^{4}$ KUCARS-KU Center for Autonomous Robotic Systems, Khalifa University, Abu Dhabi, UAE}%
}


% note the % following the last \IEEEmembership and also \thanks - 
% these prevent an unwanted space from occurring between the last author name
% and the end of the author line. i.e., if you had this:
% 
% \author{....lastname \thanks{...} \thanks{...} }
%                     ^------------^------------^----Do not want these spaces!
%
% a space would be appended to the last name and could cause every name on that
% line to be shifted left slightly. This is one of those "LaTeX things". For
% instance, "\textbf{A} \textbf{B}" will typeset as "A B" not "AB". To get
% "AB" then you have to do: "\textbf{A}\textbf{B}"
% \thanks is no different in this regard, so shield the last } of each \thanks
% that ends a line with a % and do not let a space in before the next \thanks.
% Spaces after \IEEEmembership other than the last one are OK (and needed) as
% you are supposed to have spaces between the names. For what it is worth,
% this is a minor point as most people would not even notice if the said evil
% space somehow managed to creep in.



% The paper headers
\markboth{Journal of \LaTeX\ Class Files,~Vol.~14, No.~8, August~2015}%
{Shell \MakeLowercase{\textit{et al.}}: Bare Demo of IEEEtran.cls for IEEE Journals}
% The only time the second header will appear is for the odd numbered pages
% after the title page when using the twoside option.
% 
% *** Note that you probably will NOT want to include the author's ***
% *** name in the headers of peer review papers.                   ***
% You can use \ifCLASSOPTIONpeerreview for conditional compilation here if
% you desire.




% If you want to put a publisher's ID mark on the page you can do it like
% this:
%\IEEEpubid{0000--0000/00\$00.00~\copyright~2015 IEEE}
% Remember, if you use this you must call \IEEEpubidadjcol in the second
% column for its text to clear the IEEEpubid mark.



% use for special paper notices
%\IEEEspecialpapernotice{(Invited Paper)}




% make the title area
\maketitle

\begin{abstract}
This paper introduces the Emirates Multi-Task (EMT) dataset - the first publicly available dataset for autonomous driving collected in the Arab Gulf region. The EMT dataset captures the unique road topology, high traffic congestion, and distinctive characteristics of the Gulf region, including variations in pedestrian clothing and weather conditions. It contains over 30,000 frames from a dash-camera perspective, along with 570,000 annotated bounding boxes, covering approximately 150 kilometers of driving routes. The EMT dataset supports three primary tasks: tracking, trajectory forecasting and intention prediction. Each benchmark dataset is complemented with corresponding evaluations: (1) multi-agent tracking experiments, focusing on multi-class scenarios and occlusion handling; (2) trajectory forecasting evaluation using deep sequential and interaction-aware models; and (3) intention benchmark experiments conducted for predicting agents' intentions from observed trajectories. The dataset is publicly available at \href{http://avlab.io/emt-dataset}{avlab.io/emt-dataset}, and pre-processing scripts along with evaluation models can be accessed at \href{http://github.com/AV-Lab/emt-dataset}{github.com/AV-Lab/emt-dataset}.


\end{abstract}


% Note that keywords are not normally used for peerreview papers.
\begin{IEEEkeywords}
Autonomous Driving, Intention Prediction, Tracking, Trajectory Prediction, Trajectory Forecasting
\end{IEEEkeywords}

\IEEEpeerreviewmaketitle

\section{Introduction}
Backdoor attacks pose a concealed yet profound security risk to machine learning (ML) models, for which the adversaries can inject a stealth backdoor into the model during training, enabling them to illicitly control the model's output upon encountering predefined inputs. These attacks can even occur without the knowledge of developers or end-users, thereby undermining the trust in ML systems. As ML becomes more deeply embedded in critical sectors like finance, healthcare, and autonomous driving \citep{he2016deep, liu2020computing, tournier2019mrtrix3, adjabi2020past}, the potential damage from backdoor attacks grows, underscoring the emergency for developing robust defense mechanisms against backdoor attacks.

To address the threat of backdoor attacks, researchers have developed a variety of strategies \cite{liu2018fine,wu2021adversarial,wang2019neural,zeng2022adversarial,zhu2023neural,Zhu_2023_ICCV, wei2024shared,wei2024d3}, aimed at purifying backdoors within victim models. These methods are designed to integrate with current deployment workflows seamlessly and have demonstrated significant success in mitigating the effects of backdoor triggers \cite{wubackdoorbench, wu2023defenses, wu2024backdoorbench,dunnett2024countering}.  However, most state-of-the-art (SOTA) backdoor purification methods operate under the assumption that a small clean dataset, often referred to as \textbf{auxiliary dataset}, is available for purification. Such an assumption poses practical challenges, especially in scenarios where data is scarce. To tackle this challenge, efforts have been made to reduce the size of the required auxiliary dataset~\cite{chai2022oneshot,li2023reconstructive, Zhu_2023_ICCV} and even explore dataset-free purification techniques~\cite{zheng2022data,hong2023revisiting,lin2024fusing}. Although these approaches offer some improvements, recent evaluations \cite{dunnett2024countering, wu2024backdoorbench} continue to highlight the importance of sufficient auxiliary data for achieving robust defenses against backdoor attacks.

While significant progress has been made in reducing the size of auxiliary datasets, an equally critical yet underexplored question remains: \emph{how does the nature of the auxiliary dataset affect purification effectiveness?} In  real-world  applications, auxiliary datasets can vary widely, encompassing in-distribution data, synthetic data, or external data from different sources. Understanding how each type of auxiliary dataset influences the purification effectiveness is vital for selecting or constructing the most suitable auxiliary dataset and the corresponding technique. For instance, when multiple datasets are available, understanding how different datasets contribute to purification can guide defenders in selecting or crafting the most appropriate dataset. Conversely, when only limited auxiliary data is accessible, knowing which purification technique works best under those constraints is critical. Therefore, there is an urgent need for a thorough investigation into the impact of auxiliary datasets on purification effectiveness to guide defenders in  enhancing the security of ML systems. 

In this paper, we systematically investigate the critical role of auxiliary datasets in backdoor purification, aiming to bridge the gap between idealized and practical purification scenarios.  Specifically, we first construct a diverse set of auxiliary datasets to emulate real-world conditions, as summarized in Table~\ref{overall}. These datasets include in-distribution data, synthetic data, and external data from other sources. Through an evaluation of SOTA backdoor purification methods across these datasets, we uncover several critical insights: \textbf{1)} In-distribution datasets, particularly those carefully filtered from the original training data of the victim model, effectively preserve the model’s utility for its intended tasks but may fall short in eliminating backdoors. \textbf{2)} Incorporating OOD datasets can help the model forget backdoors but also bring the risk of forgetting critical learned knowledge, significantly degrading its overall performance. Building on these findings, we propose Guided Input Calibration (GIC), a novel technique that enhances backdoor purification by adaptively transforming auxiliary data to better align with the victim model’s learned representations. By leveraging the victim model itself to guide this transformation, GIC optimizes the purification process, striking a balance between preserving model utility and mitigating backdoor threats. Extensive experiments demonstrate that GIC significantly improves the effectiveness of backdoor purification across diverse auxiliary datasets, providing a practical and robust defense solution.

Our main contributions are threefold:
\textbf{1) Impact analysis of auxiliary datasets:} We take the \textbf{first step}  in systematically investigating how different types of auxiliary datasets influence backdoor purification effectiveness. Our findings provide novel insights and serve as a foundation for future research on optimizing dataset selection and construction for enhanced backdoor defense.
%
\textbf{2) Compilation and evaluation of diverse auxiliary datasets:}  We have compiled and rigorously evaluated a diverse set of auxiliary datasets using SOTA purification methods, making our datasets and code publicly available to facilitate and support future research on practical backdoor defense strategies.
%
\textbf{3) Introduction of GIC:} We introduce GIC, the \textbf{first} dedicated solution designed to align auxiliary datasets with the model’s learned representations, significantly enhancing backdoor mitigation across various dataset types. Our approach sets a new benchmark for practical and effective backdoor defense.



\section{Related Work}
\label{sec:relwork}

\begin{figure*}[ht]
    \centering
    \includegraphics[width=1.0\linewidth]{figures/assets/overview3.pdf}
    \vspace{-2em}
    \caption{\textbf{Model Architecture.} \ourmodel utilizes solely the input image to generate the 3D output ($\mathbf{O}$). It bootstraps a dense camera prediction ($\mathbf{C}$) from the Camera Module, injecting prior knowledge on scene scale into the Depth Module via a cross-attention layer per resolution, with 4 layers in total. The camera representation corresponds to azimuth and elevation angles. The geometric invariance loss ($\mathcal{L}_{\mathrm{con}}$) enforces consistency between geometric camera-aware output tensors from different geometric augmentations ($\mathcal{T}_1$, $\mathcal{T}_2$). The depth output ($\mathbf{Z}_{\log}$) \blue{is obtained through an FPN-based decoder that gradually upsamples the feature maps and injects multi-resolution information}. The final output is the concatenation of the camera and depth tensors ($\mathbf{C} || \mathbf{Z}_{\log}$), creating two independent optimization spaces for $\mathcal{L}_{\lambda MSE}$. \blue{The depth output is supervised with the proposed Edge-guided Normalized L1-loss $\mathcal{L}_{EG-SSI}$. In addition, \ourmodel computes a prediction uncertainty ($\mathbf{\Sigma}$) which is supervised with an L1-loss on the error in log space between predicted and ground-truth depth.}}
    \label{fig:results:overview}
    \vspace{-1em}
\end{figure*}


\PAR{Metric and Scale-Agnostic Depth Estimation.}
It is crucial to distinguish Monocular Metric Depth Estimation (MMDE) from scale-agnostic, namely up-to-a-scale, monocular depth estimation.
MMDE SotA approaches typically confine training and testing to the same domain.
However, challenges arise, such as overfitting to the training scenario leading to considerable performance drops in the presence of minor domain gaps,
% potential covariate shift issues, 
often overlooked in benchmarks like NYU-Depthv2~\cite{silberman2012nyu} (NYU) and KITTI~\cite{Geiger2012kitti}.
On the other hand, scale-agnostic depth methods, pioneered by MiDaS~\cite{ranftl2020midas}, OmniData~\cite{eftekhar2021omnidata}, and LeReS~\cite{yin2021leres}, show robust generalization by training on extensive datasets.
\blue{The paradigm has been elevated to another level by repurposing depth-conditioned generative methods for RGB to RGB-conditioned depth generative methods~\cite{ke2024marigold} or large-scale semi-supervised pre-training as in the DepthAnything series~\cite{yang2024da1, yang2024da2}.}
The limitation of all these methods lies in the absence of a metric output, hindering practical usage in downstream applications.

\PAR{Monocular Metric Depth Estimation.}
The introduction of end-to-end trainable neural networks in MMDE, pioneered by \cite{Eigen2014}, marked a significant milestone, also introducing the optimization process through the Scale-Invariant log loss ($\mathrm{SI}_{\log}$).
Subsequent developments witnessed the emergence of advanced networks, ranging from convolution-based architectures~\cite{Fu2018Dorn, Laina2016, Liu2015, Patil2022p3depth} to transformer-based approaches~\cite{Yang2021, Bhat2020adabins, Yuan2022newcrf, piccinelli2023idisc}.
Despite impressive achievements on established benchmarks, MMDE models face challenges in zero-shot scenarios, revealing the need for robust generalization against appearance and geometry domain shifts.

\PAR{General Monocular Metric Depth Estimation.}
Recent efforts focus on developing MMDE models~\cite{bhat2023zoedepth, guizilini2023zerodepth, yin2023metric3d} for general depth prediction across diverse domains.
These models often leverage camera awareness, either by directly incorporating external camera parameters into computations~\cite{facil2019camconvs, guizilini2023zerodepth} or by normalizing the shape or output depth based on intrinsic properties, as seen in~\cite{Lee2019bts, Lopez2020mapillary, yin2023metric3d, hu2024metric3dv2}.
\blue{A new paradigm recently emerged~\cite{piccinelli2024unidepth, bochkovskii2024depthpro}, where the goal is to directly estimate the 3D scene from the input image \emph{without any} additional information other than the RGB input.
Our approach fits in the latter new paradigm, namely universal MMDE: we do not require any additional prior information at test time, such as access to camera information.}


\section{Dataset Details}
\label{appendixdata}
The full list of songs we selected are detailed in Tables \ref{tab:en_list}, \ref{tab:zh_list}, \ref{tab:fr_list}, \ref{tab:ko_list}. The lyrics of these
songs will not be publicly released but will be available upon request for research purposes only, ensuring their appropriate use. The songs are selected from leaderboards on music platforms such as Apple Music, with release years ranging from 1930 to 2012. We manually ensure that all of the songs are copyright protected. U.S., China, France, and Korea are all signatories to the Berne Convention \cite{berne_convention}, which is an international treaty that ensures that works created in one member country are automatically protected by copyright in all other member countries without the need for formal registration \cite{ricketson2022international}. This means that a Chinese song is automatically protected by U.S. copyright law as soon as it is created and fixed in a tangible medium (e.g., recorded or written down).


\section{Experiments}
\label{sec:experiments}



\begin{figure*}[t]
    \centering
    \includegraphics[width=1\linewidth]{images/Environments.pdf} 
    % \vspace{-20pt}
    \captionsetup{
    width=\textwidth,
    font=Smallfont,
    labelfont=Smallfont,
    textfont=Smallfont
    }
    \captionsetup{
    width=\textwidth,
    font=Smallfont,
    labelfont=Smallfont,
    textfont=Smallfont
    }
    \caption{Four different real-world experiment environments.}
    \label{fig:environments}
    % \vspace{-6pt}
\end{figure*}

\begin{figure*}[t]
    \centering
    \captionsetup{
    width=\textwidth,
    font=Smallfont,
    labelfont=Smallfont,
    textfont=Smallfont
    }
    % Top-left subfigure
    \begin{subfigure}[b]{0.45\textwidth}
        \centering
        \includegraphics[width=\textwidth]{images/fig_office.pdf}
        \caption{Office}
        \label{fig:subfig1}
    \end{subfigure}
    \hspace{0.02\textwidth}
    % Top-right subfigure
    \begin{subfigure}[b]{0.45\textwidth}
        \centering
        \includegraphics[width=\textwidth]{images/fig_apt.pdf}
        \caption{Apartment}
        \label{fig:subfig2}
    \end{subfigure}

    \vskip\baselineskip

    % Bottom-left subfigure
    \begin{subfigure}[b]{0.45\textwidth}
        \centering
        \includegraphics[width=\textwidth]{images/fig_outdoor.pdf}
        \caption{Outdoor}
        \label{fig:subfig3}
    \end{subfigure}
    \hspace{0.02\textwidth}
    % Bottom-right subfigure
    \begin{subfigure}[b]{0.45\textwidth}
        \centering
        \includegraphics[width=\textwidth]{images/fig_hallway.pdf}
        \caption{Hallway}
        \label{fig:subfig4}
    \end{subfigure}

    \caption{Top-down view of the trajectories comparison on the value maps with the detection results across the four different environments.}
    \label{fig:value_map}
\end{figure*}


\begin{table*}[ht]
\captionsetup{
    width=\textwidth,
    font=Smallfont,
    labelfont=Smallfont,
    textfont=Smallfont
    }
\caption{Vision-language navigation performance in 4 unseen environments (SR and SPL).}
\label{SOTAResults}
\centering
\begin{tabular}{lcccc|cccc}
\toprule
\multirow{2}{*}{\textbf{Method}} & \multicolumn{4}{c}{\textbf{SR (\%)}} & \multicolumn{4}{c}{\textbf{SPL}} \\
\cmidrule(lr){2-5} \cmidrule(lr){6-9}
 & Hallway & Office & Apartment & Outdoor & Hallway & Office & Apartment & Outdoor \\
\midrule
\textbf{Frontier Exploration}  
  & 40.0 & 41.7 & 55.6 & 33.3  
  & 0.239 & 0.317 & 0.363 & 0.189 \\

\textbf{VLFM} \cite{yokoyama2024vlfm}                 
  & 53.3 & 75.0 & 66.7 & 44.4  
  & 0.366 & 0.556 & 0.412 & 0.308 \\

\textbf{VL-Nav w/o IBTP}      
  & 66.7 & 83.3 & \underline{70.2} & \underline{55.6}  
  & 0.593 & 0.738 & 0.615 & \underline{0.573} \\

\textbf{VL-Nav w/o curiosity}      
  & \underline{73.3} & \underline{86.3} & 66.7 & \underline{55.6}  
  & \underline{0.612} & \underline{0.743} & \underline{0.631} & 0.498 \\

\textbf{VL-Nav}               
  & \textbf{86.7} & \textbf{91.7} & \textbf{88.9} & \textbf{77.8}  
  & \textbf{0.672} & \textbf{0.812} & \textbf{0.733} & \textbf{0.637} \\

\bottomrule
\end{tabular}
\end{table*}








\subsection{Experimental Setting}
\label{sec:experimental_setting}

We evaluate our approach in real-robot experiments against five methods: (1) classical frontier-based exploration, (2) VLFM \cite{yokoyama2024vlfm}, (3) VLNav without instance-based target points, (4) VLNav without curiosity terms, and (5) the full VLNav configuration. Because the original VLFM relies on BLIP-2 \cite{li2023blip}, which is too computationally heavy for real-time edge deployment, we use the YOLO-World \cite{cheng2024yolo} model instead to generate per-observation similarity scores for VLFM. Each method is tested under the same conditions to ensure a fair comparison of performance.

\paragraph{Environments:}
We consider four distinct environments (shown in \fref{fig:environments}), each with a specific combination of semantic complexity (\textit{High}, \textit{Medium}, or \textit{Low}) and size (\textit{Big}, \textit{Mid}, or \textit{Small}). Concretely, we use a Hallway (\textit{Medium \& Big}), an Office (\textit{High \& Mid}), an Outdoor area (\textit{Low \& Big}), and an Apartment (\textit{High \& Small}). In each environment, we evaluate five methods using three language prompts, yielding a diverse range of spatial layouts and semantic challenges. This setup provides a rigorous assessment of each method’s adaptability.

\paragraph{Language-Described Instance:}
We define nine distinct, uncommon human-described instances to serve as target objects or persons during navigation. Examples include phrases such as “tall white trash bin,” “there seems to be a man in white,” “find a man in gray,” “there seems to be a black chair,” “tall white board,” and “there seems to be a fold chair.” The variety in these descriptions ensures that the robot must rely on vision-language understanding to accurately locate these targets.

\noindent\textbf{Robots and Sensor Setup:} 
All experiments are conducted using a four-wheel Rover equipped with a Livox Mid-360 LiDAR. The LiDAR is tilted by approximately 23 degrees to the front to achieve a $\pm 30$ degrees vertical FOV coverage closely aligned with the forward camera’s view. An Intel RealSense D455 RGB-D camera, tilted upward by 7 degrees to detect taller objects, provides visual observation, though its depth data are not used for positioning or mapping. LiDAR measurements are a primary source of mapping and localization due to their higher accuracy. The whole VL-Nav system runs on an NVIDIA Jetson Orin NX on-board computer.




\subsection{Main Results}
\label{sec:main_results}

We validate the proposed VL-Nav system in real-robot experiments across four distinct environments (\textit{Hallway}, \textit{Office}, \textit{Apartment}, and \textit{Outdoor}), each featuring different semantic levels and sizes. Building on the motivation articulated in~\sref{sec:intro}, we focus on evaluating VL-Nav’s ability to (1) interpret fine-grained vision-language features and conduct robust VLN, (2) explore efficiently in unfamiliar spaces across various environments, and (3) run in real-time on resource-constrained platforms. \fref{fig:value_map} presents a top-down comparison of trajectories and detection results on the value map.

\begin{figure*}[t]
    \centering
    \includegraphics[width=1\linewidth]{images/result_plot.pdf} 
    % \vspace{-20pt}
    \captionsetup{
    width=\textwidth,
    font=Smallfont,
    labelfont=Smallfont,
    textfont=Smallfont
    }
    \caption{Plots of performance in different environments sizes and semantic comlexities.}
    \label{fig:results}
    % \vspace{-6pt}
\end{figure*}

\paragraph{Overall Performance:}
As reported in Table~\ref{SOTAResults}, our full \textbf{VL-Nav} consistently obtains the highest Success Rate (SR) and Success weighted by Path Length (SPL) across all four environments. In particular, VL-Nav outperforms classical exploration by a large margin, confirming the advantage of integrating CVL spatial reasoning with partial frontier-based search rather than relying solely on geometric exploration.

\paragraph{Effect of Instance-Based Target Points (IBTP):}
We note a marked improvement when enabling IBTP: the variant without IBTP lags behind, particularly in complex domains like the \textit{Apartment} and \textit{Office}. As discussed in \sref{sec:method}, IBTP allows VL-Nav to pursue and verify tentative detections with confidence above a threshold, mirroring human search behavior. This pragmatic mechanism prevents ignoring possible matches to the target description and reduces overall travel distance to confirm or discard candidate objects.

\paragraph{Curiosity Contributions:}
The \emph{curiosity Score} is also significant to VL-Nav’s performance. It merges two key components:
\begin{itemize}
    \item \textbf{Distance Weighting}: Preventing easily select very far way goals to reduce travel time and energy consumption which is extremely important for the efficiency (metrics SPL) in the large-size environments.
    \item \textbf{Unknown-Area Weighting}: Rewards navigation toward regions that yield more information.
\end{itemize}
Our ablations reveal that removing the distance-scoring element (\textit{VL-Nav w/o curiosity}) degrades both SR and SPL, particularly in the more cluttered environments. Meanwhile, dropping the instance-based target points (IBTP) similarly lowers performance, reflecting how each piece of CVL addresses a complementary aspect of semantic navigation.

\paragraph{Comparison to VLFM:}
Although the VLFM approach \cite{yokoyama2024vlfm} harnesses vision-language similarity value, it lacks the pixel-wise vision-language features, instance-based target points verification mechanism, and CVL-based spatial reasoning. Consequently, VL-Nav surpasses VLFM in both SR and SPL by effectively combining the pixel-wise vision language features and the curiosity cues via the CVL spatial reasoning. These gains are especially pronounced in semantic complex (\textit{Apartment}) and open-area (\textit{Outdoor}) environments, underscoring how our CVL spatial reasoning enhance vision-language navigation in complex settings and scenarios.



\paragraph{Summary of Findings:}
In conclusion, the experimental results confirm that VL-Nav delivers superior vision-language navigation across diverse, unseen real-world environments. By fusing frontier-based target points detection, instance-based target points, and the CVL spatial reasoning for goal selection, VL-Nav balances semantic awareness and exploration efficiency. The system’s robust performance, even in large or cluttered domains, highlights its potential as a practical solution for zero-shot vision-language navigation on low-power robots.

\section{Conclusion}
We introduce a novel approach, \algo, to reduce human feedback requirements in preference-based reinforcement learning by leveraging vision-language models. While VLMs encode rich world knowledge, their direct application as reward models is hindered by alignment issues and noisy predictions. To address this, we develop a synergistic framework where limited human feedback is used to adapt VLMs, improving their reliability in preference labeling. Further, we incorporate a selective sampling strategy to mitigate noise and prioritize informative human annotations.

Our experiments demonstrate that this method significantly improves feedback efficiency, achieving comparable or superior task performance with up to 50\% fewer human annotations. Moreover, we show that an adapted VLM can generalize across similar tasks, further reducing the need for new human feedback by 75\%. These results highlight the potential of integrating VLMs into preference-based RL, offering a scalable solution to reducing human supervision while maintaining high task success rates. 

\section*{Impact Statement}
This work advances embodied AI by significantly reducing the human feedback required for training agents. This reduction is particularly valuable in robotic applications where obtaining human demonstrations and feedback is challenging or impractical, such as assistive robotic arms for individuals with mobility impairments. By minimizing the feedback requirements, our approach enables users to more efficiently customize and teach new skills to robotic agents based on their specific needs and preferences. The broader impact of this work extends to healthcare, assistive technology, and human-robot interaction. One possible risk is that the bias from human feedback can propagate to the VLM and subsequently to the policy. This can be mitigated by personalization of agents in case of household application or standardization of feedback for industrial applications. 

\bibliographystyle{ieeetr}
\bibliography{refs.bib}

%\newpage
%\onecolumn

%\appendices
%\section{Annotation Labels}

%\input{tables/agents_table}
%\input{tables/action_table}

%\newpage 
%\section{Sample Frames and Annotations}
%\begin{figure*}[h!]
%\centering
%\includegraphics[width=0.9\textwidth]{figures/appendix_1.pdf}
%\caption{Sample annotated frames (Day)}
%\label{fig:samples_day}
%\end{figure*}
%\begin{figure*}[h!]
%\centering
%\includegraphics[width=0.9\textwidth]{figures/appendix_2.pdf}
%\caption{Sample annotated frames (Night)}
%\label{fig:samples_night}
%\end{figure*}








% you can choose not to have a title for an appendix
% if you want by leaving the argument blank
%\section{}
%Appendix two text goes here.


% use section* for acknowledgment
%\section*{Acknowledgment}


%The authors would like to thank...


% Can use something like this to put references on a page
% by themselves when using endfloat and the captionsoff option.
\ifCLASSOPTIONcaptionsoff
  \newpage
\fi



% trigger a \newpage just before the given reference
% number - used to balance the columns on the last page
% adjust value as needed - may need to be readjusted if
% the document is modified later
%\IEEEtriggeratref{8}
% The "triggered" command can be changed if desired:
%\IEEEtriggercmd{\enlargethispage{-5in}}

% references section

% can use a bibliography generated by BibTeX as a .bbl file
% BibTeX documentation can be easily obtained at:
% http://mirror.ctan.org/biblio/bibtex/contrib/doc/
% The IEEEtran BibTeX style support page is at:
% http://www.michaelshell.org/tex/ieeetran/bibtex/
%\bibliographystyle{IEEEtran}
% argument is your BibTeX string definitions and bibliography database(s)
%\bibliography{IEEEabrv,../bib/paper}
%
% <OR> manually copy in the resultant .bbl file
% set second argument of \begin to the number of references
% (used to reserve space for the reference number labels box)



% biography section
% 
% If you have an EPS/PDF photo (graphicx package needed) extra braces are
% needed around the contents of the optional argument to biography to prevent
% the LaTeX parser from getting confused when it sees the complicated
% \includegraphics command within an optional argument. (You could create
% your own custom macro containing the \includegraphics command to make things
% simpler here.)
%\begin{IEEEbiography}[{\includegraphics[width=1in,height=1.25in,clip,keepaspectratio]{mshell}}]{Michael Shell}
% or if you just want to reserve a space for a photo:

%\begin{IEEEbiography}{Michael Shell}
%Biography text here.
%\end{IEEEbiography}

% if you will not have a photo at all:
%\begin{IEEEbiographynophoto}{John Doe}
%Biography text here.
%\end{IEEEbiographynophoto}

% insert where needed to balance the two columns on the last page with
% biographies
%\newpage

%\begin{IEEEbiographynophoto}{Jane Doe}
%Biography text here.
%\end{IEEEbiographynophoto}

% You can push biographies down or up by placing
% a \vfill before or after them. The appropriate
% use of \vfill depends on what kind of text is
% on the last page and whether or not the columns
% are being equalized.

%\vfill

% Can be used to pull up biographies so that the bottom of the last one
% is flush with the other column.
%\enlargethispage{-5in}



% that's all folks
\end{document}


