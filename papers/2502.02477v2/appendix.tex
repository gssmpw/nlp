\newpage
\appendix
\section{Appendix}
\label{sec:appendix}

\subsection{\textsf{FM} algorithm.}\label{sec:appendix:FM}

In Algorithm~\ref{alg:fm_org} we give a high-level description of the \textsf{FM} algorithm. 
% is the function of number of vertices $n$, edges in graph, $m$, and $\delta$
% These $\delta$-clique are complete bipartite subgraphs of the original graph, where each clique contains a specific number of edges, and the vertices are divided into two partitions, called $U$ and $W$. The algorithm relies on a binary tree data structure to select vertices and extracting a $\delta$-clique. 
The \textsf{FM} algorithm first constructs \emph{neighborhood trees} for each vertex~$u_i$ in~$U$, which are
labeled binary trees of depth~$r$, whose nodes are labeled by a bit string $\omega$, where $0 \leq |\omega| \leq r$. 
The root node of a neighborhood tree contains all the vertices in~$W$ and is labeled by the empty string, $\epsilon$. 
The bit string $\omega$ of a node is obtained by starting at the root and following the path in the tree up to the node and concatenating a 0 or a 1 to $\omega$ each time the path visits the left, or the right child, respectively.
The nodes at level~$i$ in the tree correspond to a partition of~$W$ into sets of size~$|W|/2^i$. Thus, each leaf node corresponds to a vertex in~$W$. Each node in the neighborhood tree of $u_i$ stores $d_{i, \omega}$, the number of neighbors of $u_i$ in the set $W_{\omega}$ of vertices of~$W$ that are associated with the node with label $\omega$, i.e., 
$d_{i, \omega} = |N(u_i) \cap W_{\omega}|$, where $N(u_i)$ is the set of the neighbors of $u_i$.
%Each leaf node~$w_i$ has either degree~1 or~0, depending on whether it has an edge with the corresponding vertex in~$U$. 
%The binary tree is built such that each node stores the path from the root node to its location using a binary string $\omega$, where "0" denotes a left child and "1" denotes a right child. The length of $\omega$ represents the depth of the node from the root node. 
In Figure~\ref{fig:neighborhoodTree}, we show the neighborhood tree of vertex $u_2$ for the bipartite graph with 
$|U| = |W| = n = 8$ given in Figure~\ref{fig:given_graph(fm)}. Vertex~$w_3$ is a leaf and would have $\omega =  010$ in all neighborhood trees of the graph. \looseness=-1
%The binary tree structure also includes the degree for each node, where the degree of a parent node is the sum of the degrees of its child nodes.


\begin{algorithm}[!b]
\caption{{\small \textsf{FM} Algorithm: High-level description}}
{
\small
\begin{algorithmic}[1]
\INPUT {$G(U,W,E)$: Bipartite graph;} 
%\State{Initialize: no.\ of vertices $n \gets |U|$, and no.\ of remaining edges $\hat{m} \gets |E|$.}
\State{Initialize: clique index, $q \gets 0$; no.\ of vertices $n \gets |U|$; and no.\ of remaining edges $\hat{m} \gets |E|$.}
\State{Build neighborhood trees for vertices $u_i \in U$.} 
%such that every internal node of the binary tree stores bit strings $\omega$, where $0 \leq|\omega| \leq |U|$ and every leaf stores $w_{i} \in W$. The labeling of the nodes encodes the path to it from the root in the natural fashion( i.e., 0 and 1 for left and right nodes respectively). The root node of each neighborhood trees stores the degree of the vertex $u_i \in U$. }
\While{$\hat{m} \geq n^{2-\delta}$}
    \State{$\hat{k} \gets \bigg\lfloor{\frac{\delta \log n}{\log(2n^2/\hat{m})} }\bigg\rfloor$}
    % \State{Select $\hat{k}$ vertices based on the $c_0$ and $c_1$ calculations as shown in}
    % \Statex{\hspace{4mm}Lemma 2.1 to obtain set $K_q$. Update the neighborhood trees such}
    % \Statex{\hspace{4mm}that  the degree of each node along the path of the selected vertex is}
    % \Statex{\hspace{4mm}lowered by 1 for each neighborhood tree, resulting in lowering the}
    % \Statex{\hspace{4mm}degree of the neighborhood tree by  number of selected vertices}
    % \Statex{\hspace{4mm}presented in the neighborhood tree. }

% The search procedure will construct an ordered set $\mathbf{K}=$ $\left(x_1, \ldots, x_k\right) \subseteq V$ in stages, determining $x_i$ in stage $i$. At stage $t$ of the search, the first $t-1$ elements of $\mathbf{K}$ will have been determined to be some $y_1, \ldots, y_{, \ldots,} \in V$. The remaining elements must now belong to the set $V_t=V-$ $\left\{y_1, \ldots, y_1,\right\}$. Let $U_t=\left\{u \in U \mid y_1, \ldots, y_{1,1} \in \Gamma(u)\right\}$, we are omitting vertices from $U$ which are not adjacent to each of the first $t-1$ choices of the elements in $\mathbf{K}$. Denote by $: B_t$ the subgraph of $: B$ induced by the vertex sets $U_1$ and $V_1$. The search procedure will work with the new graph $B_B$, and will update the neighborhood trees for each $u_i \in U_t$ to correspond to the structure of $B_B$.
   
    % \State \parbox[t]{\dimexpr\linewidth-\algorithmicindent}{Select $\hat{k}$ vertices based on $c_0$ and $c_1$ 
    % (calculated as in Equation~\ref{eq:c01}) %calculations as shown in Lemma 2.1 
    % to obtain set $K_q$. Update the neighborhood trees such that $d_{i, \omega}$ of the nodes along the path from root to the selected vertex is decreased by~1 for each neighborhood tree corresponding to $U_{t}$. 
    % }
    
  %   \State \parbox[t]{\dimexpr\linewidth-\algorithmicindent}{
  %  Iteratively select $\hat{k}$ vertices based on $c_0$ and $c_1$ (calculated as in Equation 1) to form the ordered set $K_{q} = \{ y_1, y_2, \ldots, y_{t} \}   \subseteq  W$ and $t \leq \hat{k}$. 
  %  At each stage $t$, obtain set $U_{t+1}=\left\{u \in U_t \mid y_1, \ldots, y_{t} \in N(u)\right\}$ which omits vertices from $U$ that are not adjacent to $K_{q}$ and update the corresponding neighborhood trees such that $d_{i,\omega}$ of the nodes along the path from root to the selected vertex is decreased by 1 for each neighborhood tree corresponding to $U_t$.
  % }

     \State \parbox[t]{\dimexpr\linewidth-\algorithmicindent}{
   Starting with $t=1$ and $U_t = U$, iterate  while $t \leq \hat{k}$ to select $\hat{k}$ vertices based on $c_0$ and $c_1$ (calculated as in Equation \ref{eq:c01}) to form the ordered set $K_{q} = \{ y_1, y_2, \ldots, y_{t} \}   \subseteq  W$. 
   At each stage $t$, obtain set $U_{t+1}=\left\{u \in U_t \mid y_1, \ldots, y_{t} \in N(u)\right\}$ which omits vertices from $U$ that are not adjacent to $K_{q}$. And update the corresponding neighborhood trees such that $d_{i,\omega}$ of the nodes along the path from root to the selected vertex is decreased by 1 for each neighborhood tree corresponding to $U_t$.
  }
% \vspace{1mm}
   % \State \parbox[t]{\dimexpr\linewidth-\algorithmicindent}{ Then the search procedure will work with the induced bipartite subgraph $B_t$ with bipartition $U_t$ and $W_{t}$ 
   % and update the neighborhood trees for each $u_i \in U_t$ to correspond to the structure of $B_t$.   
   % }
   % \vspace{1mm}
   \State \parbox[t]{\dimexpr\linewidth-\algorithmicindent}{
   Form the $\delta$-clique $C_q = (U_q, K_q)$, where, $U_q  = U_{t}$.
   % Find $U_q$, the set of common neighbours of the selected vertex set $K_q$ and extract the $\delta$-clique $C_q = (U_q, K_q)$.
   }
    \State{Update $G$ by removing the edges associated with clique~$C_q$ and update $\hat{m} \gets |E|$.}
    \State{$q \gets q + 1$}
\EndWhile
\State{For each clique in $\mathcal{C} = \{C_1, \ldots, C_q\}$  add edges from all the vertices in the left partition $U_q$ to an additional vertex~$z_q$, and from~$z_q$ to all the vertices in the right partition $K_q$. This would compress the graph by replacing $|U_q| \times |K_q|$ edges with $|U_q| + |K_q|$ edges.}
\State{The remaining edges in~$G$ are trivial cliques and are added in the compressed graph~$G^*$.}
\State{\textbf{Output:} $G^*$, the compressed graph of~$G$.}
\end{algorithmic}
\label{alg:fm_org}
}
\end{algorithm}
%\vspace{-0.5cm}



The algorithm computes~$\hat{k}$ and then, in Line~5, it performs~$\hat{k}$ iterations
(indexed by~$t$) to select vertices for the right partition of clique~$C_q$. In each iteration it selects vertices at leaf nodes of the neighborhood trees by following a path based on $d_{i, \omega}$ of child nodes and the number of distinct ordered subsets of the neighbourhood tree at each level. In order to select a path at each level, the algorithm calculates $c_{0}$ and $c_{1}$ as follows: 
\begin{equation}
c_{j} = C\left(U_t, \mathscr{C}_{t, \omega \cdot j}\right)=\sum_{i:{u_{i} \in U_{t}}} d_{i, \omega \cdot j} \cdot\left(d_{i}-1\right)^{[{k}-t]},  \, j= 0,1.  \label{eq:c01}
\end{equation}
$c_{j}$ counts the number of ordered sets in $\mathscr{C}_{t, \omega}$ whose elements are adjacent to~$u_{i}$, where $\mathscr{C}_{t, \omega}$ denotes the collection of all ordered sets $K$ of $W$ of size~$k$ in iteration~$t$, with~$t \leq k$. In each iteration, a vertex 
in~$W$ is added to the right partition of the clique. 
%Here, $d_{i}^{[k]} = d_{i}(d_{i}-1)(d_{i}-2) \cdots(d_{i}-k+1)$, is the number of distinct ordered sets of 
%size~$k$. Thus, 
If $|N(u_{i})|=d_{i}$, then the number of distinct ordered subsets of $N(u_{i})$ of size~$k$ is~$d_{i}^{[k]}$, where  $d_{i}^{[k]} = d_{i}(d_{i}-1)(d_{i}-2) \cdots(d_{i}-k+1)$. Based on $c_{j}$'s, the \textsf{FM} algorithm chooses either the left node (if $c_{0} \geq c_{1}$) 
or the right node (if $c_{0} < c_{1}$). When the algorithm reaches a leaf node, it selects the corresponding vertex in~$W$, updates the set of common neighbors ($U_{t}$), and decreases $d_{i, \omega}$ of each node along the path to the selected vertex by~1, in  neighborhood trees corresponding to $U_{t}$.
%updates $d_{i, \omega}$ of the nodes along the path by subtracting 1 in all neighborhood tree. This update reduces the $d_{i, \omega}$  of each node along the path to the selected vertex, resulting in the degree of the root node being decreased by the number of vertices having an edge in the neighborhood tree. 
Thus, it guarantees the selection of a new vertex in the following iterations. The algorithm continues this process until it extracts~$\hat{k}$ unique vertices to form the set~$K_q$, the right partition of the $q$-th $\delta$-clique~$C_q$. 



After forming set $K_q$, the \textsf{FM} algorithm proceeds to identify the common neighbors of the vertices that are part of set~$K_q$ in order to construct~$U_q$, which is the left partition of the $q$-th $\delta$-clique~$C_q$. 
The pair~$(K_q, U_q)$ forms a $\delta$-clique~$C_q$. Subsequently, the algorithm updates~$\hat{m}$, the number of remaining edges in the graph after extracting each~$\delta$-clique. This update is done by subtracting the product of the sizes of $K_q$ and $U_q$ from $\hat{m}$, i.e., $|U_q| \times |K_q|$. The algorithm then updates~$\hat{k}$ considering the updated value of~$\hat{m}$ and repeats the procedure to extract more $\delta$-cliques until no more $\delta$-cliques can be found (determined by the condition of the while loop). The edges that are not removed in this process form trivial cliques.

The \textsf{FM} algorithm compresses the graph by adding a new vertex set~$Z$ to the bipartite graph, thus converting it into a tripartite graph (Figure~\ref{fig:ch-matching-bip:bm2(fm)}), where each vertex in~$Z$ corresponds to one clique extracted by the algorithm. Each vertex of the left and right partitions of $\delta$-clique $C_q$ is then connected via an edge to a new vertex $z_q \in Z$, thus forming a tripartite graph.  This decreases the number of edges from $|U_q| \times |K_q|$ to $|U_q| + |K_q|$, thus compressing the graph. The compressed graph obtained by the algorithm preserves the path information of the original graph. The running time of the \textsf{FM} algorithm is~$O(mn^{\delta} \log^2 n)$. The \textsf{FM} algorithm can be \emph{extended to the case of non-bipartite graphs}, where it computes a compression of a graph in time $O(mn^{\delta} \log^2 n)$, as shown by Feder and Motwani~\cite{federMotwani}.


\subsection{Example: \textsf{CPGC} Execution.}\label{sec:appendix:example}
% \noindent {\bf Illustrative Example}. 
%In this section 
We now show how~\textsf{CPGC} works on a bipartite graph~$G$ with partitions~$U$ and~$W$, $n = |U| = |W| = 8$, and $|E| = 54$, shown in Figure~\ref{fig:given_graph(a)}. 
\textsf{CPGC} (Algorithm \ref{alg:CPGC}), first initializes $q = 0, n = |W| = |U| = 8 , \hat{m} = |E| = 54, \mathcal{C} = \emptyset$, and $d_w = [0]_n $. It then calculates the degree of each vertex $d_{w_j}$ in Lines 2-4, as shown in Table~\ref{table:1}. It calculates~$\hat{k}(n,m,\delta) = 2$ in Line~5. Thus, the condition for the while loop in Line~6 is met and \textsf{CPGC} calls \textsf{CSA} in Line~7. With the given inputs, \textsf{CSA} sorts~$d_w$ in non-increasing order. Let $\{w_4, w_2, w_3, w_5, w_6, w_1, w_7, w_8\}$ be the non-increasing order with corresponding degrees given as $d_{w_{\pi(j)}}$ in Table~\ref{table:1}. \looseness=-1 

\begin{table}[h!]
\centering
{\scriptsize
 \begin{tabular}{c c c c c c c c c c} 
 \hline
 \textsf{CPGC} step & $j$ & 1 & 2 & 3 & 4 & 5 & 6 & 7 & 8\\ % [0.5ex] 
  \hline
 % $d_{u_i} = |N(u_i)|$ & 6 & 6 & 7 & 8 & 7 & 5 & 7 & 8 \\
 Initialization & $d_{w_j}$ & 6 &  7 & 7 & 8 & 7 & 7 & 6 & 6\\ %[2ex] 
 Sorted $d_{w_j}$ & $d_{w_{\pi(j)}} $ & 8 &  7 & 7 & 7 & 7 & 6 & 6 & 6\\ %[1ex] 
After $1^{st}$iteration &  $d_{w_j}$ & 6 &  0 & 0 & 1 & 0 & 7 & 6 & 6\\ %[2ex] 
 \hline
\end{tabular}
\vspace*{0.15cm}
\caption{Degrees of vertices in partition $W$ at different steps of \textsf{CPGC}.}
\label{table:1}
}
\vspace*{-0.3cm}
\end{table}





%\tikzstyle{vertex}=[circle,fill=black!25,minimum size=20pt,inner sep=0pt]
%\tikzstyle{vertex}=[circle, draw, inner sep=0pt, minimum size=6pt]
\tikzstyle{vertex}=[circle,draw,minimum size=12pt,inner sep=0pt]
\tikzstyle{edge} = [draw,thick,-]
%\tikzstyle{weight} = [font=\small]
\usetikzlibrary{decorations.markings}
%\usetikzlibrary{arrows}
\usetikzlibrary{shapes,decorations,arrows,calc,arrows.meta,fit,positioning}
\tikzset{edge/.style = {->,> = latex'}}


% \newpage
\begin{figure*}[!t]
\centerline{
\subfloat[\centering]{

        \begin{tikzpicture}[scale=0.6, auto, swap]
	\tikzset{edge/.style = {->,> = latex'}}
    		\foreach \pos/\name in {{(0,7)/u_1}, {(0,6)/u_2}, {(0,5)/u_3}, {(0,4)/u_4}, {(0,3)/u_5}, {(0,2)/u_6}, {(0,1)/u_7}, {(0,0)/u_8},
      {(2.6,7)/w_1}, {(2.6,6)/w_2}, {(2.6,5)/w_3}, {(2.6,4)/w_4}, {(2.6,3)/w_5}, {(2.6,2)/w_6}, {(2.6,1)/w_7}, {(2.6,0)/w_8}}
        	\node[vertex] (\name) at \pos {$\name$};
        \draw (u_1)--(w_1);
        \draw (u_1)--(w_2);
        \draw (u_1)--(w_3);
        \draw (u_1)--(w_4);
        \draw (u_1)--(w_5);
        \draw (u_1)--(w_6);

        \draw (u_2)--(w_2);
        \draw (u_2)--(w_3);
        \draw (u_2)--(w_4);
        \draw (u_2)--(w_5);
        \draw (u_2)--(w_6);
        \draw (u_2)--(w_7);

        \draw (u_3)--(w_2);
        \draw (u_3)--(w_3);
        \draw (u_3)--(w_4);
        \draw (u_3)--(w_5);
        \draw (u_3)--(w_6);
        \draw (u_3)--(w_7);
        \draw (u_3)--(w_8);

        \draw (u_4)--(w_1);
        \draw (u_4)--(w_2);
        \draw (u_4)--(w_3);
        \draw (u_4)--(w_4);
        \draw (u_4)--(w_5);
        \draw (u_4)--(w_6);
        \draw (u_4)--(w_7);
        \draw (u_4)--(w_8);

        \draw (u_5)--(w_8);
        \draw (u_5)--(w_1);
        \draw (u_5)--(w_2);
        \draw (u_5)--(w_3);
        \draw (u_5)--(w_4);
        \draw (u_5)--(w_5);
        \draw (u_5)--(w_6);

        \draw (u_6)--(w_4);
        \draw (u_6)--(w_7);
        \draw (u_6)--(w_8);
        \draw (u_6)--(w_1);
        \draw (u_6)--(w_2);

        \draw (u_7)--(w_3);
        \draw (u_7)--(w_4);
        \draw (u_7)--(w_5);
        \draw (u_7)--(w_6);
        \draw (u_7)--(w_7);
        \draw (u_7)--(w_8);
        \draw (u_7)--(w_1);

        \draw (u_8)--(w_1);
        \draw (u_8)--(w_2);
        \draw (u_8)--(w_3);
        \draw (u_8)--(w_4);
        \draw (u_8)--(w_5);
        \draw (u_8)--(w_6);
        \draw (u_8)--(w_7);
        \draw (u_8)--(w_8);            		
 	\end{tikzpicture}
 	\label{fig:given_graph(a)}
}
% \hspace*{0.25cm}
\subfloat[\centering]{
	\begin{tikzpicture}[scale=0.6, auto, swap]
	\tikzset{edge/.style = {->,> = latex'}}
    		% draw the vertices
    		\foreach \pos/\name in {{(0,7)/u_1}, {(0,6)/u_2}, {(0,5)/u_3}, {(0,4)/u_4}, {(0,3)/u_5}, {(0,2)/u_6}, {(0,1)/u_7}, {(0,0)/u_8},
      {(2.6,7)/w_1}, {(2.6,6)/w_2}, {(2.6,5)/w_3}, {(2.6,4)/w_4}, {(2.6,3)/w_5}, {(2.6,2)/w_6}, {(2.6,1)/w_7}, {(2.6,0)/w_8}}
        	\node[vertex] (\name) at \pos {$\name$};

        \draw (u_1)--(w_1);
        \draw [blue,line width=0.5mm] (u_1)--(w_2);
        \draw (u_1)--(w_3);
        \draw [blue,line width=0.5mm] (u_1)--(w_4);
        \draw (u_1)--(w_5);
        \draw (u_1)--(w_6);

        \draw [blue,line width=0.5mm] (u_2)--(w_2);
        \draw (u_2)--(w_3);
        \draw [blue,line width=0.5mm] (u_2)--(w_4);
        \draw (u_2)--(w_5);
        \draw (u_2)--(w_6);
        \draw (u_2)--(w_7);

        \draw [blue,line width=0.5mm](u_3)--(w_2);
        \draw (u_3)--(w_3);
        \draw [blue,line width=0.5mm](u_3)--(w_4);
        \draw (u_3)--(w_5);
        \draw (u_3)--(w_6);
        \draw (u_3)--(w_7);
        \draw (u_3)--(w_8);

        \draw (u_4)--(w_1);
        \draw [blue,line width=0.5mm](u_4)--(w_2);
        \draw (u_4)--(w_3);
        \draw [blue,line width=0.5mm](u_4)--(w_4);
        \draw (u_4)--(w_5);
        \draw (u_4)--(w_6);
        \draw (u_4)--(w_7);
        \draw (u_4)--(w_8);

        \draw (u_5)--(w_8);
        \draw (u_5)--(w_1);
        \draw [blue,line width=0.5mm](u_5)--(w_2);
        \draw (u_5)--(w_3);
        \draw [blue,line width=0.5mm](u_5)--(w_4);
        \draw (u_5)--(w_5);
        \draw (u_5)--(w_6);

        \draw [blue,line width=0.5mm](u_6)--(w_4);
        \draw (u_6)--(w_7);
        \draw (u_6)--(w_8);
        \draw (u_6)--(w_1);
        \draw [blue,line width=0.5mm](u_6)--(w_2);

        \draw (u_7)--(w_3);
        \draw (u_7)--(w_4);
        \draw (u_7)--(w_5);
        \draw (u_7)--(w_6);
        \draw (u_7)--(w_7);
        \draw (u_7)--(w_8);
        \draw (u_7)--(w_1);

        \draw (u_8)--(w_1);
        \draw [blue,line width=0.5mm](u_8)--(w_2);
        \draw (u_8)--(w_3);
        \draw [blue,line width=0.5mm](u_8)--(w_4);
        \draw (u_8)--(w_5);
        \draw (u_8)--(w_6);
        \draw (u_8)--(w_7);
        \draw (u_8)--(w_8);
  		
 	\end{tikzpicture}
 	\label{fig:ch-matching-bip:bm2(b)}
}
% \hspace*{0.25cm}
\subfloat[\centering]{
	\begin{tikzpicture}[scale=0.6, auto, swap]
	\tikzset{edge/.style = {->,> = latex'}}
    		% draw the vertices
    		\foreach \pos/\name in {{(0,7)/u_1}, {(0,6)/u_2}, {(0,5)/u_3}, {(0,4)/u_4}, {(0,3)/u_5}, {(0,2)/u_6}, {(0,1)/u_7}, {(0,0)/u_8},
      {(2.6,7)/w_1}, {(2.6,6)/w_2}, {(2.6,5)/w_3}, {(2.6,4)/w_4}, {(2.6,3)/w_5}, {(2.6,2)/w_6}, {(2.6,1)/w_7}, {(2.6,0)/w_8}}
        	\node[vertex] (\name) at \pos {$\name$};
        \draw (u_1)--(w_1);
        \draw [red,line width=0.5mm](u_1)--(w_3);
        \draw [red,line width=0.5mm](u_1)--(w_5);
        \draw (u_1)--(w_6);

        \draw [red,line width=0.5mm](u_2)--(w_3);
        \draw [red,line width=0.5mm](u_2)--(w_5);
        \draw (u_2)--(w_6);
        \draw (u_2)--(w_7);

        \draw [red,line width=0.5mm](u_3)--(w_3);
        \draw [red,line width=0.5mm](u_3)--(w_5);
        \draw (u_3)--(w_6);
        \draw (u_3)--(w_7);
        \draw (u_3)--(w_8);

        \draw (u_4)--(w_1);
        \draw [red,line width=0.5mm](u_4)--(w_3);
        \draw [red,line width=0.5mm](u_4)--(w_5);
        \draw (u_4)--(w_6);
        \draw (u_4)--(w_7);
        \draw (u_4)--(w_8);

        \draw (u_5)--(w_8);
        \draw (u_5)--(w_1);
        \draw [red,line width=0.5mm](u_5)--(w_3);
        \draw [red,line width=0.5mm](u_5)--(w_5);
        \draw (u_5)--(w_6);

        \draw (u_6)--(w_7);
        \draw (u_6)--(w_8);
        \draw (u_6)--(w_1);

        \draw [red,line width=0.5mm](u_7)--(w_3);
        \draw (u_7)--(w_4);
        \draw [red,line width=0.5mm](u_7)--(w_5);
        \draw (u_7)--(w_6);
        \draw (u_7)--(w_7);
        \draw (u_7)--(w_8);
        \draw (u_7)--(w_1);

        \draw (u_8)--(w_1);
        \draw [red,line width=0.5mm](u_8)--(w_3);
        \draw [red,line width=0.5mm](u_8)--(w_5);
        \draw (u_8)--(w_6);
        \draw (u_8)--(w_7);
        \draw (u_8)--(w_8);
			
    		
 	\end{tikzpicture}
 	\label{fig:ch-matching-bip:bm2(c)}
}
% \hspace*{0.4cm}
% \subfloat[]{
    
% 	\begin{tikzpicture}[scale=0.8, auto, swap]
% 	\tikzset{edge/.style = {->,> = latex'}}
%     		% draw the vertices
%     		\foreach \pos/\name in {{(0,7)/u_1}, {(0,6)/u_2}, {(0,5)/u_3}, {(0,4)/u_4}, {(0,3)/u_5}, {(0,2)/u_6}, {(0,1)/u_7}, {(0,0)/u_8},
%       {(2,7)/w_1}, {(2,6)/w_2}, {(2,5)/w_3}, {(2,4)/w_4}, {(2,3)/w_5}, {(2,2)/w_6}, {(2,1)/w_7}, {(2,0)/w_8}}
%         	\node[vertex] (\name) at \pos {$\name$};
        	
%         	%\foreach \pos/\name in {{(0,3)/b}, {(2,4)/f}, {(2,2)/h}}
%         	%\node[vertex,fill=gray] (\name) at \pos {$\name$};
        	

% 		%\begin{scope}[every edge/.style={draw=black,thick,->,> = latex',line width=1.}]
	 
		
% 		% \draw [blue,line width=0.8mm] (a)--(f);
%         \draw (u_1)--(w_1);
%         \draw [green,line width=0.4mm](u_1)--(w_6);

%         \draw [green,line width=0.4mm](u_2)--(w_6);
%         \draw (u_2)--(w_7);

%         \draw [green,line width=0.4mm](u_3)--(w_6);
%         \draw (u_3)--(w_7);
%         \draw (u_3)--(w_8);

%         \draw (u_4)--(w_1);
%         \draw [green,line width=0.4mm](u_4)--(w_6);
%         \draw (u_4)--(w_7);
%         \draw (u_4)--(w_8);

%         \draw (u_5)--(w_8);
%         \draw (u_5)--(w_1);
%         \draw [green,line width=0.4mm](u_5)--(w_6);

%         \draw (u_6)--(w_7);
%         \draw (u_6)--(w_8);
%         \draw (u_6)--(w_1);

%         \draw (u_7)--(w_4);
%         \draw [green,line width=0.4mm](u_7)--(w_6);
%         \draw (u_7)--(w_7);
%         \draw (u_7)--(w_8);
%         \draw (u_7)--(w_1);

%         \draw (u_8)--(w_1);
%         \draw [green,line width=0.4mm](u_8)--(w_6);
%         \draw (u_8)--(w_7);
%         \draw (u_8)--(w_8);

%  	\end{tikzpicture}
%  	\label{fig:ch-matching-bip:bm2}
% }
% \hspace*{0.25cm}

\subfloat[\centering]{
\centering
	\begin{tikzpicture}[scale=0.6, auto, swap]
	\tikzset{edge/.style = {->,> = latex'}}
    		% draw the vertices
    		\foreach \pos/\name in {{(0,7)/u_1}, {(0,6)/u_2}, {(0,5)/u_3}, {(0,4)/u_4}, {(0,3)/u_5}, {(0,2)/u_6}, {(0,1)/u_7}, {(0,0)/u_8},
      {(2.6,7)/w_1}, {(2.6,6)/w_2}, {(2.6,5)/w_3}, {(2.6,4)/w_4}, {(2.6,3)/w_5}, {(2.6,2)/w_6}, {(2.6,1)/w_7}, {(2.6,0)/w_8}}
        	\node[vertex] (\name) at \pos {$\name$};
        	
        	%\foreach \pos/\name in {{(0,3)/b}, {(2,4)/f}, {(2,2)/h}}
        	%\node[vertex,fill=gray] (\name) at \pos {$\name$};
        	

		%\begin{scope}[every edge/.style={draw=black,thick,->,> = latex',line width=1.}]
	 
		
		% \draw [blue,line width=0.8mm] (a)--(f);
        \draw (u_1)--(w_1);
        \draw (u_1)--(w_6);

        \draw (u_2)--(w_6);
        \draw (u_2)--(w_7);

        \draw (u_3)--(w_7);
        \draw (u_3)--(w_6);
        \draw (u_3)--(w_8);

        \draw (u_4)--(w_1);
        \draw (u_4)--(w_7);
        \draw (u_4)--(w_6);
        \draw (u_4)--(w_8);

        \draw (u_5)--(w_8);
        \draw (u_5)--(w_1);
        \draw (u_5)--(w_6);

        \draw (u_6)--(w_7);
        \draw (u_6)--(w_8);
        \draw (u_6)--(w_1);
        

        \draw (u_7)--(w_4);
        \draw (u_7)--(w_7);
        \draw (u_7)--(w_8);
        \draw (u_7)--(w_1);
        \draw (u_7)--(w_6);

        \draw (u_8)--(w_1);
        \draw (u_8)--(w_7);
        \draw (u_8)--(w_6);
        \draw (u_8)--(w_8);

 	\end{tikzpicture}
 	\label{fig:ch-matching-bip:bm2(d)}
}

% \hspace*{0.25cm}

 \subfloat[\centering]{
	\begin{tikzpicture}[scale=0.6, auto, swap]
	\tikzset{edge/.style = {->,> = latex'}}
    		% draw the vertices
    		\foreach \pos/\name in {{(0,7)/u_1}, {(0,6)/u_2}, {(0,5)/u_3}, {(0,4)/u_4}, {(0,3)/u_5}, {(0,2)/u_6}, {(0,1)/u_7}, {(0,0)/u_8}, 
      % {(2,5)/c_1}, {(2,4)/c_2}, {(2,3)/c_3},
      {(3,7)/w_1}, {(3,6)/w_2}, {(3,5)/w_3}, {(3,4)/w_4}, {(3,3)/w_5}, {(3,2)/w_6}, {(3,1)/w_7}, {(3,0)/w_8}}
        	\node[vertex] (\name) at \pos {$\name$};
        	


		%\begin{scope}[every edge/.style={draw=black,thick,->,> = latex',line width=1.}]
	 
		
		% \draw [blue,line width=0.8mm] (a)--(f);
        \draw (u_1)--(w_1);

        \draw   (u_2)--(w_7);

        \draw  (u_3)--(w_7);
        \draw  (u_3)--(w_8);

        \draw   (u_4)--(w_1);
        \draw  (u_4)--(w_7);
        \draw  (u_4)--(w_8);

        \draw  (u_5)--(w_8);
        \draw  (u_5)--(w_1);

        \draw  (u_6)--(w_7);
        \draw  (u_6)--(w_8);
        \draw  (u_6)--(w_1);

        \draw  (u_7)--(w_4);
        \draw  (u_7)--(w_7);
        \draw  (u_7)--(w_8);
        \draw  (u_7)--(w_1);

        \draw  (u_8)--(w_1);
        \draw  (u_8)--(w_7);
        \draw  (u_8)--(w_8);
		



         
		%\end{scope}
        % \draw [blue,line width=0.5mm] (u_1)--(w_4);

        \draw (u_2)--(w_6);
        % \draw [blue,line width=0.5mm] (u_2)--(w_4);

        \draw (u_3)--(w_6);
        % \draw [blue,line width=0.5mm](u_3)--(w_4);

        \draw (u_4)--(w_6);
        % \draw [blue,line width=0.5mm](u_4)--(w_4);/

        \draw (u_5)--(w_6);
        % \draw [blue,line width=0.5mm](u_5)--(w_4);

        \draw (u_7)--(w_6);
        % \draw [blue,line width=0.5mm](u_6)--(w_2);


        \draw (u_8)--(w_6);
        % \draw [blue,line width=0.5mm](u_8)--(w_4);
        
        % \draw [green,line width=0.5mm](c_3)--(w_6);

        \draw (u_1)--(w_6);
        % \draw [blue,line width=0.5mm] (u_1)--(w_4);

        % \draw [blue,line width=0.5mm] (u_2)--(w_4);

            \foreach \pos/\name in {{(1.5,4.5)/z_1}, {(1.5,3)/z_2}}
                \node[vertex,fill=gray] (\name) at \pos {$\name$};
        \draw [red,line width=0.5mm](u_1)--(z_2);
        \draw [red,line width=0.5mm](u_2)--(z_2);
        \draw [red,line width=0.5mm](u_3)--(z_2);
        % \draw [blue,line width=0.5mm](u_3)--(w_4);

        \draw [red,line width=0.5mm](u_4)--(z_2);
        % \draw [blue,line width=0.5mm](u_4)--(w_4);/

        \draw [red,line width=0.5mm](u_5)--(z_2);
        % \draw [blue,line width=0.5mm](u_5)--(w_4);

        \draw [red,line width=0.5mm](u_7)--(z_2);
        % \draw [blue,line width=0.5mm](u_6)--(w_2);


        \draw [red,line width=0.5mm](u_8)--(z_2);
        % \draw [blue,line width=0.5mm](u_8)--(w_4);
        
        \draw [red,line width=0.5mm](z_2)--(w_3);
        \draw [red,line width=0.5mm](z_2)--(w_5);

                \draw [blue,line width=0.5mm] (u_1)--(z_1);
        % \draw [blue,line width=0.5mm] (u_1)--(w_4);

        \draw [blue,line width=0.5mm] (u_2)--(z_1);
        % \draw [blue] (u_2)--(w_4);

        \draw [blue,line width=0.5mm](u_3)--(z_1);
        % \draw [blue](u_3)--(w_4);

        \draw [blue,line width=0.5mm](u_4)--(z_1);
        % \draw [blue](u_4)--(w_4);/

        \draw [blue,line width=0.5mm](u_5)--(z_1);
        % \draw [blue](u_5)--(w_4);

        \draw [blue,line width=0.5mm](u_6)--(z_1);
        % \draw [blue](u_6)--(w_2);


        \draw [blue,line width=0.5mm](u_8)--(z_1);
        % \draw [blue,line width=0.5mm](u_8)--(w_4);
        
        \draw [blue,line width=0.5mm](z_1)--(w_2);
        \draw [blue,line width=0.5mm](z_1)--(w_4);
	

    		
 	\end{tikzpicture}
 	\label{fig:ch-matching-bip:bm2(e)}}

}
\vspace*{-0.2cm}
\caption{\small Clique partitioning: (a) given bipartite graph $G(U,W)$; (b) extracted clique $C_1$; (c) extracted clique $C_2$; (d) graph with trivial cliques; and (e) compressed tripartite graph $G(U,Z,W)$}
\label{fig:ch-mathing-bip:bm(caption)}
\end{figure*}


In Line 5, \textsf{CSA} forms set $\mathcal{K}$, where $\mathcal{K} = \{ w_4, w_2, w_3, w_5, w_6 \}$ has all 
vertices whose degrees are greater than or equal to $d_{w_{\pi(\hat{k})}}=7$. This results in $\gamma = 2$ 
in Line~7, therefore \textsf{CSA} extracts two cliques in Lines 8-13.
Since~$q$ is initialized to~0 in \textsf{CPGC}, the for loop in Line~8 iterates two times for~$c=1$ and~$2$. In Line~9, \textsf{CSA} forms $K_1 = \{w_4, w_2\}$, and in Line 10, it forms the left partition $U_{K_{1}}$ of bipartite clique~$\mathcal{C}_1$, by selecting the common neighbors of the vertices in $K_{1}$. For example, for set $K_1 = \{w_4, w_2\}$, $U_{K_1} = U - \{u_7\}$ as $w_4$ and $w_2$ have $\{u_1, u_2, u_3, u_4, u_5, u_6, u_8\}$ as common neighbors. In Line~12, \textsf{CSA} updates the adjacency matrix~$A$ by removing the edges of bipartite clique~$\mathcal{C}_1$ from the original bipartite graph~$G$ and finally in Line~13, it updates all~$d_w$ by subtracting $|U_{K_1}| = 7$ from both $d_{w_{4}}$ and $d_{w_{2}}$, thus $d_{w_{4}}=1$ and $d_{w_{2}}=0$.
Similarly, when $c=2$, $K_2 = \{w_3, w_5\}$ and $U_{K_2} = U - \{u_6\}$. \textsf{CSA} updates the adjacency matrix $A$ and $d_w$ such that $d_{w_{3}}=d_{w_{5}}=0$. 
Therefore, in the first execution, \textsf{CSA}, forms two bipartite cliques and removes a total of $(7 \times 2) + (7 \times 2)  = 28$ edges from~$G$. The updated degrees of the vertices are shown in Table~\ref{table:1}. 
It then forms the set of bipartite cliques extracted in the current execution in Line~14, $\mathscr{C} =\{C_{1}, C_{2}\}$. The two cliques are shown in Figures~\ref{fig:ch-matching-bip:bm2(b)} and~\ref{fig:ch-matching-bip:bm2(c)}.
\looseness=-1

\textsf{CSA} returns $(\mathscr{C}, \hat{A}, \hat{d}_w)$ and then \textsf{CPGC} updates the set $\mathscr{C} =\{C_{1}, C_{2}\}$, $d_w$, the adjacency matrix~$A$, and the number of edges $\hat{m}= 54-28 =26$, in Lines~\mbox{8-10}. In Line 12, \textsf{CPGC} updates $q=2$ and finally, in Line 13 it updates~$\hat{k}$ according to the new value of~$\hat{m}$ in Line 11 which results in $\hat{k} =1$. Since $\hat{k} =1$, it means \textsf{CPGC} extracts only trivial bipartite cliques (shown in Figure~\ref{fig:ch-matching-bip:bm2(d)}) which do not contribute to graph compression. Thus, it does not meet the condition in the while loop (Line 6) and \textsf{CPGC} terminates.

\textsf{CPGC} then compresses the graph by adding two vertices, $z_1$, and  $z_2$, corresponding to the two cliques $C_1$ and $C_2$, and adds the corresponding edges to form the tripartite graph. The edges in the given graph $G(U,W,E)$, that are not part of any $\delta$-clique are connected directly as shown in Figure~\ref{fig:ch-matching-bip:bm2(e)}. 



\subsection{Proofs\\ }\label{sec:appendix:proofs}


\hspace*{-0.2cm}\noindent{\bf Proof of Theorem 4.1.}
\begin{proof} Let's assume that \textsf{CPGC} extracts only one $\delta$-clique $C_q$ from the given graph~$G$.  In  the $\delta$-clique~$C_q$, the right partition~$K_q\subseteq W$ is formed by selecting~$\hat{k}$ vertices from~$W$ and the left partition~$U_q \subseteq U$ is formed with the common neighbors of~$K_q$, forming a complete bipartite graph with~$E_q$ edges. The compressed graph~$G^*$ is formed with the same left and right partitions~$U$ and~$W$ of the graph~$G$, and a third partition~$Z$ which is a set of additional vertices associated with each of the $\delta$-cliques that \textsf{CPGC} extracts.  Our main concern is with the edges $(u_i,w_j) \in E^*$ in the compressed graph~$G^*$. $E^*$ contains the edges that replace the $\delta$-clique~$C_q$ by adding the additional vertex~$z_q \in Z$, and the set of edges $\hat{E} = E \setminus E_q$ which were not part of the $\delta$-clique~$C_q$. 
It can be easily seen that~$\hat{E} \subset E$, that is, the edges in~$\hat{E}$ are edges in~$G$, and the remaining edges $E^* \setminus \hat{E}$ in $E^*$ are the edges that replace~$C_q$ in~$G^*$. Each edge $(u_i,w_j) \in E_q$, is replaced by two edges, $(u_i,z_q)$ and $(z_q,w_j)$, where $u_i \in U$, $w_j \in W$, and $z_q \in Z$. Therefore, for each edge $(u_i,w_j) \in E_q$, there exists a path from~$u_i$ to~$w_j$ composed of two edges, $(u_i,z_q)$ and $(z_q,w_j)$, that passes through the additional vertex~$z_q \in Z$, thus preserving the path information. This holds true for all the $\delta$-cliques extracted by \textsf{CPGC}.
\end{proof}

\noindent{\bf Proof of Lemma 4.1.}
\begin{proof}
    In the $\delta$-clique based graph compression performed by \textsf{CPGC}, the compressed graph $G^*(U,W,Z,E^*)$ is obtained from~$G$ by replacing $|E_q| = |U_q \times K_q|$ edges in a $\delta$-clique~$C_q$ with $|E^*| = |U_q| + |K_q|$ edges in $G^*$ and adding an additional vertex $z_q \in Z$ for each extracted $\delta$-clique~$C_q$. 
    The size of the right partition of~$C_q$ is $|K_q|={k} = k(n,m, \delta)=\Big\lfloor{\frac{\delta \log n}{\log(2n^2/m)} }\Big\rfloor$. If $m < 2n^{2-\frac{\delta}{2}}$ then the size of the right partition~$|K_q|$ is $k = 1$. Therefore, the number of edges in the $\delta$-clique is $|E_q| = |U_q| \cdot 1 = |U_q|$. Those edges are replaced by $ |E^*|  = |U_q| + \hat{k} = |U_q| + 1$ edges in~$G^*$. Thus, replacing the $\delta$-clique in~$G$ with $m < 2n^{2-\frac{\delta}{2}}$, the number of edges in $G^*$ would actually increase by 1, which does not lead to a compression of~$G$. 
\end{proof}


\subsection{Extension to non-bipartite graphs.}\label{sec:appendix:non-bip}

%\noindent {\textbf{Extension to non-bipartite graphs.}} 
\textsf{CPGC} can be extended to compress non-bipartite graphs using a similar technique to that described in~\cite{federMotwani}. Consider a directed graph $H(V, E)$ with $n$ vertices and~$m$ edges. Initially, we transform $H$ into a bipartite graph $G\left(L, R, E^{\prime}\right)$, where each vertex $v \in V$ is duplicated into $v_L \in L$ and $v_R \in R$. For each directed edge $(u, v)$ in $E$, we create a directed edge $\left(u_L, v_R\right)$ in~$E^{\prime}$, with the direction from~$L$ to~$R$. Furthermore, for each vertex $v \in V$ we add a directed edge from~$v_R$ to~$v_L$. The proposed compression algorithm is then applied to the bipartite graph~$G$, which includes only the edges from~$L$ to~$R$. Once the compressed graph is computed we add to it the~$n$ directed edges from~$u_R$ to~$u_L$, corresponding to each vertex $u\in V$. Since the path information of the original graph is preserved by this transformation, it allows different graph algorithms to work on the compressed graph as well. Undirected general graphs can be transformed first into directed graphs by simply replacing each undirected edge $(u,v)$ in the original graph by two directed edges $(u,v)$ and $(v,u)$, and then applying the technique presented above to transform the directed graph into a bipartite graph. 
%The time complexity and the compression ratio of the proposed algorithms are the same as in the case of the bipartite graphs. 
The time complexity and the compression ratio of \textsf{CPGC} are the same as in the case of the bipartite graphs. A similar approach for converting a general graph to a bipartite graph and then compressing the obtained bipartite graph was employed in~\cite{scalable_compression, compressing_bisection}. 
%~\cite{scalable_compression, compressing_bisection} use similar approach for converting a general graph to a bipartite graph and then compressing the obtained bipartite graph.  \looseness=-1


% \subsection{Modification to Dinitz algorithm}\label{sec:appendix:Dinitz_extension} 
% To adapt the Dinitz algorithm for the compressed graphs $G^*(U,W,Z,E^*)$, modifications are needed in on blocking flows on each path. Instead of finding paths from a source to a sink, we need to find paths from one partition to another, considering additional disjoint sets of vertices. This involves extending the algorithm to efficiently handle the tripartite structure, ensuring that paths traverse vertices from different partitions while respecting edge capacities. Additionally, appropriate data structures and optimizations must be employed to maintain efficiency in identifying and augmenting flows within the tripartite graph.
% Specifically, we introduced a condition for the additional vertices in the graph. This condition ensures that the flow from each additional vertex is less than or equal to the number of vertices connected to their right (i.e., $\hat{k}$ the size of right partition of the corresponding $\delta$-clique).


% \tikzstyle{vertex}=[circle,draw,minimum size=15pt,inner sep=0pt]
% \tikzstyle{edge} = [draw,thick,-]
% %\tikzstyle{weight} = [font=\small]
% \usetikzlibrary{decorations.markings}
% %\usetikzlibrary{arrows}
% \usetikzlibrary{shapes,decorations,arrows,calc,arrows.meta,fit,positioning}
% \tikzset{edge/.style = {->,> = latex'}}


% \begin{figure}
%     \centering
% 	\begin{tikzpicture}[scale=1, auto, swap]
% 	\tikzset{edge/.style = {->,> = latex'}}
%     		% draw the vertices
%     		\foreach \pos/\name in {{(0,4)/u_1}, {(0,3)/u_2}, {(0,2)/u_3}, {(0,1)/u_4},  
%       {(3,4)/w_1}, {(3,3)/w_2}, {(3,2)/w_3}, {(3,1)/w_4}}
%         	\node[vertex] (\name) at \pos {$\name$};

%         \draw (u_1)--(w_3) node[at={(4, 2)}, font=\fontsize{5}{10}\selectfont] {$(0/1)$};
        
%         \foreach \pos/\name in {{(1.5,2.5)/z_1}}
%                 \node[vertex] (\name) at \pos {$\name$};
            
%         \foreach \pos/\name in {{( -1.5,2.5)/s}, {(5,2.5)/t}}
%                 \node[vertex,fill=gray] (\name) at \pos {$\name$};
%         \draw (u_1)--(z_1) node[at={(-0.6, 3.7)}, font=\fontsize{5}{10}\selectfont] {$(0/1)$};
%         \draw (u_2)--(z_1) node[at={(0.5, 2.6)}, font=\fontsize{5}{10}\selectfont] {$(0/1)$}; 
%         \draw (u_3)--(z_1) node[at={(0.7, 2)}, font=\fontsize{5}{10}\selectfont] {$(0/1)$};
%         \draw (u_4)--(z_1) node[at={(0.67, 1.2)}, font=\fontsize{5}{10}\selectfont] {$(0/1)$};
%         \draw (z_1)--(w_2) node[at={(1.5, 3.3)}, font=\fontsize{5}{10}\selectfont] {$(0/1)$};
%         \draw (z_1)--(w_4) node[at={(0.5, 3.2)}, font=\fontsize{5}{10}\selectfont] {$(0/1)$};
%         \draw (z_1)--(w_1) node[at={(-0.6, 1.9)}, font=\fontsize{5}{10}\selectfont] {$(0/1)$};
%         \draw  (u_1)--(s) node[at={(-0.65, 1.1)}, font=\fontsize{5}{10}\selectfont] {$(0/1)$};
%         \draw  (u_2)--(s) node[at={(-0.6, 2.6)}, font=\fontsize{5}{10}\selectfont] {$(0/1)$};
%         \draw  (u_3)--(s) node[at={(2.3, 1.2)}, font=\fontsize{5}{10}\selectfont] {$(0/1)$};
%         \draw  (u_4)--(s) node[at={(4, 2.5)}, font=\fontsize{5}{10}\selectfont] {$(0/1)$};
%         \draw  (w_1)--(t) node[at={(2.34, 2.95)}, font=\fontsize{5}{10}\selectfont] {$(0/1)$};
%         \draw  (w_2)--(t) node[at={(2.3, 3.8)}, font=\fontsize{5}{10}\selectfont] {$(0/1)$};
%         \draw  (w_3)--(t) node[at={(4, 3.5)}, font=\fontsize{5}{10}\selectfont] {$(0/1)$};
%         \draw  (w_4)--(t) node[at={(4, 1.5)}, font=\fontsize{5}{10}\selectfont] {$(0/1)$};

%  	\end{tikzpicture}
%  	\label{fig:dinitz_implementation_1}

% \vspace*{-0.2cm}
% \caption{\small Implementation of Dinitz algorithm on compressed graph $G^*(U,W,Z,E^*)$.}
% \label{fig:dinitz_implementation}
% \end{figure}
