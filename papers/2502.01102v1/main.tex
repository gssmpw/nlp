\documentclass[lettersize,journal]{IEEEtran}
\usepackage{amsmath,amsfonts}
\usepackage{algorithmic}
\usepackage{array}
\usepackage[normalem]{ulem}
% \usepackage[caption=false,font=normalsize,labelfont=sf,textfont=sf]{subfig}
\usepackage{textcomp}
\usepackage{stfloats}
\usepackage{url}
% \usepackage{hyperref}
\usepackage{verbatim}
\usepackage{amsmath,bm}
\usepackage{mathrsfs}
\usepackage{amssymb}
\usepackage{booktabs}
\usepackage{svg}
\usepackage{siunitx}
% \usepackage{algorithm}
% \usepackage{algpseudocode}
\usepackage[ruled,lined]{algorithm2e}
\usepackage{makecell}
\usepackage{multirow}
\usepackage{arydshln}
\usepackage{stackengine}
\usepackage{graphicx}
\usepackage{xspace}
\usepackage[export]{adjustbox}
\usepackage{subcaption}
\usepackage{titling}

\newbox{\bigpicturebox}

% Add a period to the end of an abbreviation unless there's one
% already, then \xspace.
\makeatletter
\DeclareRobustCommand\onedot{\futurelet\@let@token\@onedot}
\def\@onedot{\ifx\@let@token.\else.\null\fi\xspace}

\def\eg{\emph{e.g}\onedot} \def\Eg{\emph{E.g}\onedot}
\def\ie{\emph{i.e}\onedot} \def\Ie{\emph{I.e}\onedot}
\def\cf{\emph{cf}\onedot} \def\Cf{\emph{Cf}\onedot}
\def\etc{\emph{etc}\onedot} \def\vs{\emph{vs}\onedot}
\def\wrt{w.r.t\onedot} \def\dof{d.o.f\onedot}
\def\iid{i.i.d\onedot} \def\wolog{w.l.o.g\onedot}
\def\etal{\emph{et al}\onedot}
\makeatother


% Support for easy cross-referencing
\usepackage[pagebackref,breaklinks]{hyperref}
\usepackage[capitalize]{cleveref}
\crefname{section}{Sec.}{Secs.}
\crefname{section}{Section}{Sections}
% \crefname{table}{Table}{Tables}
% \crefname{table}{Tab.}{Tabs.}
\crefname{algorithm}{Alg.}{Algs.}

% \algnewcommand{\LineComment}[1]{\State \(\#\) #1}
% \algnewcommand{\LineComment}[1]{\State \(\triangleright\) #1}
\usepackage[switch]{lineno}

\newcommand{\insertwithsubimagenew}[3][200 120 190 180]
{\stackinset{l}{-0.2cm}{t}{-0.2cm}
  {\scalebox{0.8}{\frame{\includegraphics[trim=#1,clip,width=1.3cm]{#3}}}}
  {\includegraphics[width=#2]{#3}}}

\hyphenation{op-tical net-works semi-conduc-tor IEEE-Xplore}
\def\BibTeX{{\rm B\kern-.05em{\sc i\kern-.025em b}\kern-.08em
    T\kern-.1667em\lower.7ex\hbox{E}\kern-.125emX}}
\usepackage{balance}

\makeatletter
\def\endthebibliography{%
	\def\@noitemerr{\@latex@warning{Empty `thebibliography' environment}}%
	\endlist
}
\makeatother


\begin{document}
\title{Towards Robust and Generalizable Lensless Imaging with Modular Learned Reconstruction}
\author{Eric Bezzam, Yohann Perron, and Martin Vetterli}
\date{}



\markboth{}
{Bezzam
\MakeLowercase{\textit{(et~al.)}: 
Towards Robust and Generalizable Lensless Imaging with Modular Learned Reconstruction}}


\maketitle

% The abstract must be between 150-250 words.

\begin{abstract}
Lensless cameras disregard the conventional design that imaging should mimic the human eye. This is done by replacing the lens with a thin mask, and moving image formation to the digital post-processing. State-of-the-art lensless imaging techniques use learned approaches that combine physical modeling and neural networks. However, these approaches make simplifying modeling assumptions for ease of calibration and computation. Moreover, the generalizability of learned approaches to lensless measurements of new masks has not been studied. To this end, we utilize a modular learned reconstruction in which a key component is a pre-processor prior to image recovery. We theoretically demonstrate the pre-processor's necessity for standard image recovery techniques (Wiener filtering and iterative algorithms), and through extensive experiments show its effectiveness for multiple lensless imaging approaches and across datasets of different mask types (amplitude and phase). We also perform the first generalization benchmark across mask types to evaluate how well reconstructions trained with one system generalize to others. Our modular reconstruction enables us to use pre-trained components and transfer learning on new systems to cut down weeks of tedious measurements and training. As part of our work, we open-source four datasets, and software for measuring datasets and for training our modular reconstruction.
\end{abstract}

\begin{IEEEkeywords}
Lensless imaging, modularity, robustness, generalizability, programmable mask, transfer learning.
\end{IEEEkeywords}


\section{Introduction}\label{sec:intro}

In computational finance, Monte Carlo simulations are used extensively to estimate the expected value of financial payoffs based on the solution of stochastic differential equations (SDEs) which model the evolution of stock prices, interest rates, exchange rates and other quantities \cite{glasserman04}.  Monte Carlo methods are very general and flexible, but for high accuracy it requires generating a large number of costly SDE path approximations, which has motivated research into a number of variance reduction or, equivalently, cost reduction techniques. One such method is
Multilevel Monte Carlo (MLMC), which was proposed in \cite{GILES2008} and was adapted for various applications that are summarised in \cite{Giles_overview17} and successfully combined with other methods such as quasi-Monte Carlo methods. The main idea of MLMC is to approximate the payoff using different time stepping resolutions when numerically solving the underlying SDE and to generate an optimal number of samples on each level, such that the overall computational cost is minimised subject to the desired bound on the variance. %, such that the total computational cost is minimised. 
The computational savings come from the fact that most samples are computed on the coarser levels and hence are less expensive while only a few samples from the finest levels are required \cite{GILES2008}.


Among the directions in which the computational cost 
of MLMC methods could further be reduced, an important avenue is the use of lower precision calculations, especially for the first Monte Carlo levels where the targeted accuracy is relatively low. 
 An overview of the research on mixed precision for the standard Monte Carlo (MC) framework is provided in \cite{ChowMixedPrecisionStandardMC} but only a few references study the potential of low precision computation in the MLMC framework \cite{Rounding_error_oliver}. To the best of our knowledge, the only MLMC framework with customised precision in the literature is \cite{brugger2014mixed}, but they use a uniform precision for all operations on each Monte Carlo level instead of optimising 
 the precision of each intermediary variable to reduce as much as possible the cost of path generation.
 
An important motivation for an MLMC framework with variable precision would be performing the low precision computations on reconfigurable hardware devices such as Field Programmable Gate Arrays (FPGAs). FPGAs contain customizable logic blocks and connectors that make it easy to adapt the digital circuit architecture for a specific application, leading to a highly parallel and optimised implementation. Therefore they are successfully exploited in applications that require high speed and have high computational workload, such as signal processing \cite{woods2008fpga}, and real time applications like high frequency trading \cite{HFT1,HFT2}. That is why a number of previous works in hardware architecture design implemented the MLMC algorithm to price financial options using FPGAs as accelerators, which resulted in improved speed and power efficiency compared to full CPU architectures \cite{Schryver2013AMM}. The paper \cite{lindsey2016domain} also proposed 
a Domain Specific Language to automate the configuration of FPGAs for this specific application. However, only \cite{brugger2014mixed} proposed a heuristic to reduce the precision in calculations.

In addition, all aforementioned works considered that the random number generation (RNG) is performed in single or double precision. Yet in most cases an important portion of the workload in the overall MLMC simulation comes from the RNG and in \cite{brugger2014mixed} this limited the total computational savings.
To reduce the cost of MLMC simulations in particular those based on the Geometric Brownian Motion (GBM), \cite{approximateICDF_Oliver, NestedOliver} have proposed to use approximate random numbers that are generated by applying an approximation of the inverse CDF to uniform random numbers. In \cite{NestedOliver}, the authors proposed a way to integrate these lower precision random variables into a \textit{nested} MLMC framework and completed a numerical analysis to bound the resulting error at each MC level by a product of the time step and the error in the random number approximation. The same authors show in \cite{approximateICDF_Oliver} that using approximate random variables reduces the cost of path generation by a factor 7.


In this paper we propose a nested MLMC framework that combines the use of approximate random normal variables and lower precision calculations to reduce the computational cost of MLMC even further than \cite{brugger2014mixed,NestedOliver}. We illustrate the efficiency of our framework in Matlab, after making several assumptions on the cost of operations and size of the errors that we carefully justify. We focus on the case of GBM and use the approximate RNG methods presented in \cite{approximateICDF_Oliver} as well as a new slightly modified method that combines CDF inversion and the central limit theorem. To choose the precision of the variables in the low precision path generation, we introduce a novel method to optimise the bit-widths. This optimisation is performed before the main path generation loop is executed and is based on a linear model of the payoff error  
due to rounding when computing in low precision. The error model relies on algorithmic differentiation in a similar manner to \cite{unifying-bwoptim,bitwidth-AD,ADAPT}. The bit-width optimisation procedure can be performed off-line, so this stage can be excluded from the on-line time complexity of our framework. The user specified desired accuracy is then enforced by calculating on-line the number of samples that need to be generated.

In terms of hardware design, we suggest implementing the low precision path generation on FPGAs and the full-precision ones on a CPU or GPU. 
The FPGA offers enough flexibility to define a separate bit-width for every variable in the low precision path generation, and can be reconfigured periodically to update the bit-widths when the market parameters have changed considerably. 


The paper is organized as follows : \Cref{sec:MLMC} introduces MLMC and nested MLMC to make clear the estimator that is implemented in our framework. Then in \Cref{sec:RNG} we detail the methods that could be used to obtain approximate random normally distributed numbers very cheaply for the low precision path generation. In \Cref{sec:error_model} and \Cref{sec:costModel} we propose an error model and a cost model (resp.) that we then use to formulate the optimisation problem that is solved to obtain the optimal bit-widths of fixed point variables in \Cref{sec:optimisation}. Finally we summarise our results and future directions in \Cref{sec:conclusion}.


  
\section{Related Work}

\paragraph{LLMs for Agent tasks.}

Our research is related to deploying large language models (LLMs) as agents for decision-making tasks in interactive environments~\citep{liu2023agentbench,zhou2023webarena,shridhar2020alfred,toyama2021androidenv}. Earlier works, such as~\citep{yao2023webshopscalablerealworldweb}, fine-tuned models like BERT~\citep{devlin2019bertpretrainingdeepbidirectional} for decision-making in simplified environments, such as online shopping or mobile phone manipulation. With the advent of large language models~\citep{brown2020languagemodelsfewshotlearners,openai2024gpt4technicalreport}, it became feasible to perform decision-making tasks through zero-shot or few-shot in-context learning. To better assess the capabilities of LLMs as agents, several models have been developed~\citep{deng2024mind2web,xiong2024watch,hong2023cogagent,yan2023gpt}. Most approaches~\citep{zheng2024seeact,deng2024mind2web} provide the agent with observation and action history, and the language model predicts the next action via in-context learning. Additionally, some methods~\citep{zhang2023building,li2023camel,song2024trial} attempt to distill trajectories from state-of-the-art language models to train more effective policy models. In contrast, our paper introduces a novel framework that automatically learns a reward model from LLM agent navigation, using it to guide the agents in making more effective plans.

\textbf{LLM Planning.} Our paper is also related to planning with large language models. Early researchers~\citep{brown2020languagemodelsfewshotlearners} often prompted large language models to directly perform agent tasks. Later, \citet{yao2022react} proposed ReAct, which combined LLMs for action prediction with chain-of-thought prompting~\citep{wei2022chain}. Several other works~\citep{yao2023treethoughtsdeliberateproblem,hao2023reasoning,zhao2023large,qiao2024agentplanningworldknowledge} have focused on enhancing multi-step reasoning capabilities by integrating LLMs with tree search methods. Our model differs from these previous studies in several significant ways. First, rather than solely focusing on text generation tasks, our pipeline addresses multi-step action planning tasks in interactive environments, where we must consider not only historical input but also multimodal feedback from the environment. Additionally, our pipeline involves automatic learning of the reward model from the environment without relying on human-annotated data, whereas previous works rely on prompting-based frameworks that require large commercial LLMs like GPT-4~\citep{openai2024gpt4technicalreport} to learn action prediction. Furthermore, \Model supports a variety of planning algorithms beyond tree search.

\textbf{Learning from AI Feedback.} In contrast to prior work on LLM planning, our approach also draws on recent advances in learning from AI feedback~\citep{bai2022constitutional,lee2023rlaif,yuan2024self,sharma2024critical,pan2024autonomous,koh2024tree}. These studies initially prompt state-of-the-art large language models to generate text responses that adhere to predefined principles and then potentially fine-tune the LLMs with reinforcement learning. Like previous studies, we also prompt large language models to generate synthetic data. However, unlike them, we focus not on fine-tuning a better generative model but on developing a classification model that evaluates how well action trajectories fulfill the intended instructions. This approach is simpler, requires no reliance on state-of-the-art LLMs, and is more efficient. We also demonstrate that our learned reward model can integrate with various LLMs and planning algorithms, consistently improving their performance.

\textbf{Inference-Time Scaling.} ~\citet{snell2024scaling} validates the efficacy of inference-time scaling for language models. Based on inference-time scaling, various methods have been proposed, such as random sampling~\citep{wang2022self} and tree-search methods~\citep{hao2023reasoning, zhang2024accessing, guan2025rstar}. Concurrently, several works have also leveraged inference-time scaling to improve the performance of agentic tasks. ~\citet{koh2024tree} adopts a training-free approach, employing MCTS to enhance policy model performance during inference and prompting the LLM to return the reward. ~\citet{gu2024your} introduces a novel speculative reasoning approach to bypass irreversible actions by leveraging LLMs or VLMs. It also employs tree search to improve performance and prompts an LLM to output rewards. ~\citet{yu2024exact} proposes Reflective-MCTS to perform tree search and fine-tune the GPT model, leading to improvements in ~\citet{koh2024visualwebarena}. ~\citet{putta2024agent} also utilizes MCTS to enhance performance on web-based tasks such as ~\citet{yao2023webshopscalablerealworldweb} and real-world booking environments. ~\cite{lin2025qlass} utilizes the stepwise reward to give effective intermediate guidance across different agentic tasks. Our work differs from previous efforts in two key aspects: (1) Broader Application Domain. Unlike prior studies that primarily focus on tasks from a single domain, our method demonstrates strong generalizability across web agents, mathematical reasoning, and scientific discovery domains, further proving its effectiveness. (2) Flexible and Effective Reward Modeling. Instead of simply prompting an LLM as a reward model, we finetune a small scale VLM~\citep{lin2023vila} to evaluate input trajectories. %Our reward scores range continuously between 0 and 1, in contrast to existing methods that rely on discrete scoring (e.g., 0 and 1, or 0, 0.5, and 1) through direct LLM prompting.

% Concurrently, several works have also leveraged inference-time scaling to improve the performance of agentic tasks. ~\citet{pan2024autonomous} demonstrates that LLMs and VLMs, such as the GPT series, can function as evaluators or reward models to provide guidance for fine-tuning or reflection, thereby enhancing digital agents. This lays the groundwork for subsequent studies that directly prompt LLMs as reward models. ~\citet{koh2024tree} adopts a training-free approach, employing MCTS to enhance policy model performance during inference. However, it is limited to web environments~\citep{koh2024visualwebarena}. Moreover, its value function relies on prompting an LLM, which is less effective than our proposed method. We validate our approach through ablation studies, demonstrating that our fine-tuned reward model is more effective. ~\citet{gu2024your} introduces a novel speculative reasoning approach to bypass irreversible actions, such as purchasing a product, by leveraging LLMs or VLMs. It also employs tree search to improve performance, but it remains restricted to the web domain~\citep{koh2024visualwebarena, deng2024mind2web}. Additionally, it lacks reward modeling and instead prompts an LLM to output rewards. ~\citet{yu2024exact} proposes Reflective-MCTS to perform tree search and fine-tune the GPT model, leading to improvements in ~\citep{koh2024visualwebarena}. However, this work focuses solely on a single web agent task, and its reward modeling is derived from multi-agent debate, differing from our more effective and efficient reward modeling approach. ~\citet{putta2024agent} also utilizes MCTS to enhance performance, but it is limited to web-based tasks such as ~\citep{yao2023webshopscalablerealworldweb} and real-world booking environments.
\section{\ours: Reproducible Feedback Environments}
\label{sec:convcodeworld}
% \begin{table*}[t!]
% \begin{wraptable}{r}{0.44\textwidth}
\begin{wraptable}{r}{0.42\textwidth}
%     \caption{Feedback categorization in conversational code generation. %Given a state $s$ that an LLM generates a code snippet, $r(s)$ is a reward from a reward function $r(\cdot)$ and $\pi(a|s)$ is an action from a policy $\pi(\cdot): S \rightarrow A$ that maps a state space $S$ to an action space $A$. 
% Our contribution introduces a control knob for execution feedback and natural language feedback, enabling a broader range of realistic scenarios. $\triangle^1$ signifies that only syntax errors can be specified by $f_c$. $\triangle^2$ signifies that only the errors that are located where the provided test cases cover ($f_e$) or by novice-expertise ($f_v$) can be specified. $\triangle^3$ signifies that the guidance for refinement by novice-expertise user might be incorrect. } 
\caption{By providing diverse feedback types, with different coverage levels in  
execution and natural language feedback, ours encompasses a broader range of realistic scenarios. 
$\triangle$ indicates partial coverage (\S\ref{convcodeworld:categorization}) with specific limitations: $^1$Syntax errors only, $^2$Limited by test coverage or feedback provider, and $^3$Potential misguidance due to limited expertise.}
    \centering
    \scriptsize
    % \small
    \begin{tabular}{c|ccc}
        \thickhline
        \multirow{2}{*}{\textbf{Feedback}} &  \textbf{Fault}  & \textbf{Guidance for}  \\
         &  \textbf{Localization}  & \textbf{Refinement}  \\\hline
        $f_c$ & \phantom{0}$\triangle^1$  & \xmark \\
        $f_e$ &  \phantom{0}$\triangle^2$  & \xmark \\ 
        $f_e^*$ & \cmark & \xmark \\
        $f_v$ & \phantom{0}$\triangle^2$ & \phantom{0}$\triangle^3$  \\
        $f_v^*$ & \cmark & \cmark \\
        \thickhline
    \end{tabular}
    \normalsize
    \label{tab:categorization}
% \end{table*}
\end{wraptable}


% \begin{table*}[t!]
%     \caption{Combination of feedback and possible scenarios.} 
%     \centering
%     \small
%     \begin{tabular}{lccc}
%         \thickhline
%         \textbf{Notations} \\\hline
%         $o_t^m = \{x, a_{t-1}^m, c_{t-1}, e_{t-1}, a_{t-1}^u\}$ \\
%         % $b_t^m(s_t) = Pr(s_t|b_0,a_0^m,o_1^m,a_1^m, ..., a_{t-1}^m,o_{t-1}^m) \approx Pr(s_t|a_{t-1}^m,o_{t-1}^m)$ \\
%         $\pi^m: o_t^m \rightarrow \Delta(A^m)$\\
%         \hdashline
%         $o_t^u = \{x, a_t^m, c_t, e_t\}$ \\
%         % $b_t^u(s_t) = Pr(s_t|b_0,a_0^u,o_1^u,a_1^u, ..., a_{t-1}^u,o_{t-1}^u) \approx Pr(s_t|a_{t-1}^u,o_{t-1}^u)$ \\
%         $\pi^u: o_t^u \rightarrow \Delta(A^u)$\\\hline
%         Execution feedback w/ full TC or expert user feedback: $b_t^m(s_t) = 1$ (now it's a Dec-MDP) \\
%         Expert users: $\pi^u(o_t^u) = a_t^{u*}$, reduces the action space of the model $A^m$ into $\hat{A}^{m} \subseteq A^m$\\
        

%         % $o_t^m = \{x, y_{t-1}, f_{t-1}^c, f_{t-1}^e, f_{t-1}^u\}$ \\
%         % $a_t^m = y_{t}$ \\ %\hdashline
%         % $b_t^m(s_t) = Pr(s_t|b_0,a_0^m,o_1^m,a_1^m, ..., a_{t-1}^m,o_{t-1}^m) \approx Pr(s_t|a_{t-1}^m,o_{t-1}^m)$ \\
%         % $\pi^m: o_t^m \rightarrow \Delta(A^m)$\\
%         % \hdashline
%         % $o_t^u = \{x, c_t, f_t^c, f_t^e\}$ \\
%         % $a_t^u = f_t^u$ \\
%         % % $b_t^u(s_t) = Pr(s_t|b_0,a_0^u,o_1^u,a_1^u, ..., a_{t-1}^u,o_{t-1}^u) \approx Pr(s_t|a_{t-1}^u,o_{t-1}^u)$ \\
%         % $\pi^u: o_t^u \rightarrow \Delta(A^u)$\\\hline
%         % Execution feedback w/ full TC or expert user feedback: $b_t^m(s_t) = b_t^*(s_t)$ (an optimal belief state) \\
%         % Expert users: $\pi^u(o_t^u) = \pi^{u*}(o_t^u) = f_t^{u*}$\\        
        
        
%         \thickhline
%     \end{tabular}
%     \normalsize
%     \label{tab:thinking}
% \end{table*}

In real-world settings of interactive code generation, the types and combinations of feedback can vary significantly due to factors such as the availability of feedback from code execution (e.g., error messages, output) and the expertise of the feedback provider. These variations, particularly the provider's expertise, can strongly influence the quality of the verbal feedback when it is offered.

% In real-world settings of interactive code generation, the types and combinations of feedback available can vary significantly due to diverse constraints such as the availability of execution and verbal feedback, and the expertise of the feedback provider---which affects the quality of the verbal feedback when it is provided. 

To effectively evaluate LLMs under these varying conditions, we propose \ours, a novel and reproducible environment designed to simulate a wide range of interactive code generation scenarios. 
Two key features of \ours are as follows: (a) \textbf{Encompassing Diverse Real-World Scenarios:} \ours covers nine distinct feedback combinations that occur in practical settings; (b) \textbf{Ensure the Reproducibility of Evaluation:} \ours provides a consistent and repeatable framework for assessing the performance of LLMs. 


% In the following sections, we will categorize the types of feedback (\S\ref{convcodeworld:categorization}), present the feedback combinations offered by \ours (\S\ref{convcodeworld:combination}), and explain how to ensure reproducible evaluation (\S\ref{convcodeworld:reproducibility}). 

% % \subsection{Motivation}
% The goal of \ours is to evaluate the performance of a code generation model while interacting with user in diverse real-world scenarios. 
% Two challenges to achieve this goal are: 
% \begin{itemize}
%     \item Cover diverse real-world scenarios
%     \item Ensure reproducibility
%     \item 
% \end{itemize}





\begin{comment}    
\subsection{Conversational Code Generation as a Partially Observable Markov Decision Process}
The goal of \ours is to evaluate the performance of a code generation model while interacting with user in diverse real-world scenarios. 
%We model this task as a 
Inspired by dialogue generation and evaluation~\cite{young2013pomdp}, where
partially observable markov decision process (POMDP), 
%as it effectively captures the 
%to capture uncertainty and 
was widely adopted to capture
sequential decision-making of utterances
%inherent in conversational code generation, where the agent must make decisions
based on incomplete information, 
we formalize conversational code generation process.


\sw{functionality?o and f?}
Formally, we define a POMDP as a tuple $\mathcal{M}= \langle \mathbb{S}, \mathbb{A}, T, \Omega, O, R \rangle$: 
$\mathbb{S}$ is the state space, and each state $s_t \in \mathbb{S}$ represents the functionality of the code at turn $t$.\footnote{This aligns with the partial observability aspect because the functionality is not directly measurable--it can only be inferred through observations like code execution with full test coverage or code reviews by expert programmers.} 
$\mathbb{A}$ is the action space, and each action $a_t \in \mathbb{A}$ represents the code snippet generated or refined by the agent at turn $t$. 
$T: \mathbb{S}\times\mathbb{A} \rightarrow \Delta(\mathbb{S})$ is a state transition probability function where $\Delta(\cdot)$ is a probability distribution of the set $(\cdot)$. 
$\Omega$ is the set of possible observations where $o_t \in \Omega$ is a set of feedback received from the environment at turn $t$. 
$O:\mathbb{S} \times \mathbb{A} \rightarrow \Delta(\Omega)$ is an observation function. 
$R: \mathbb{S}\times\mathbb{A} \rightarrow \mathbb{R}$ is an immediate reward function returning a real number. % signifying the proba. %, 1 if the code satisfies the desired functionality. 
% $h$ is the maximum number of turns (\textit{horizon}). 

% \begin{itemize}
%     \item $\mathbb{S}$ is the state space, and each state $s_t \in \mathbb{S}$ represents the functionality of the code at turn $t$. 
%     \item $\mathbb{A}$ is the action space, and each action $a_t \in \mathbb{A}$ represents the code snippet generated or refined by the agent at turn $t$. 
%     \item $T: \mathbb{S}\times\mathbb{A} \rightarrow \Delta(\mathbb{S})$ is a state transition probability function where $\Delta(\cdot)$ is a probability distribution of the set $(\cdot)$. 
%     \item $\Omega$ is the set of observations where $o_t \in \Omega$ is a set of feedback received from the environment at turn $t$. 
%     \item $O:\mathbb{S} \times \mathbb{A} \rightarrow \Delta(\Omega)$ is an observation function. 
%     \item $R: \mathbb{S}\times\mathbb{A} \rightarrow \mathbb{R}$ is an immediate reward function returning a real number. %, 1 if the code satisfies the desired functionality. 
%     \item $\gamma$ is a discount factor. 
% \end{itemize}



The conversational code generation task is conducted as follows. For each turn $t$, the agent receives an observation $o_{t} \in \Omega$ from the environment with the observation probability $O(s_t, a_t)$ representing $Pr(o_t|s_t,a_t)$, where $s_t \in \mathbb{S}$ and $a_t \in \mathbb{A}$. 
Then, the agent conducts an action $a_{t+1} \in \mathbb{A}$ from $o_{t}$ by a policy $\pi$, generating code from the current observation. 
The transition function $T$ then updates $s_t$ by the probability distribution $T(s_t,a_{t+1})$ representing probability $Pr(s_{t+1}|s_t,a_{t+1})$, where $s_{t+1} \in \mathbb{S}$. The iteration terminates in two conditions: (1) When the turn $t$ reaches to the horizon $h$; (2) When the agent reaches the goal state, i.e., the code's functionality matches the desired specification provided in the problem description. 
\end{comment}

\subsection{Feedback Categorization}
\label{convcodeworld:categorization}

To accurately simulate real-world feedback in interactive code generation, we focus on two critical components:
% In conversational code generation, the observations $\Omega$ correspond to different types of feedback the agent receives. 
% Each feedback encompasses two key types of information:
(a) \textbf{Fault Localization:} Identifying the specific parts of the code where issues or errors occur; (b) \textbf{Guidance for Refinement:} Offering suggestions or instructions on how to correct the identified issues.
% \begin{itemize}
%     \item \textbf{Fault Localization:} Identifying the specific parts of the code where issues or errors occur.
%     \item \textbf{Guidance for Refinement:} Offering suggestions or instructions on how to correct the identified issues.
% \end{itemize}

As means of obtaining such information,  we consider three different types of feedback: compilation feedback, execution feedback, and verbal feedback. 
\paragraph{Compilation Feedback ($f_c$)}
Originated from the compiler, this feedback identifies syntax and type-checking errors but cannot localize logical or runtime errors. 
As a result, Table~\ref{tab:categorization} marks this with $\triangle$  for partial fault localization. 
Additionally, compilation errors do not offer refinement guidance.
% This feedback Originated from the compiler and localizes syntax errors and type-checking issues. 
% However, compiler cannot localize where
% logical or runtime errors,
% because of which
% Table~\ref{tab:categorization}
% marks $\triangle$
% to denote a partial Fault Localization.
% Also compilation errors
% do not provide refinement guidance.

%While compiler can localize where syntax errors occur, it
%fails to provide such information on  
%of $f_c$ to be partial, and as 
%error messages do not
% for correction, 
%is not provided.
% Compilation feedback uses a compiler to check for syntax errors. This feedback is cost-effective as it does not require execution conditions or test cases; however, it fails to assess the code's alignment with user intent in terms of functionality. %Due to its cost-efficiency, we can expect that 

\paragraph{Execution Feedback}
Derived from code execution, this feedback includes runtime errors and test run results. Full or partial fault localization is provided, depending on test coverage (TC): 
(a) \textbf{Full TC ($f_e^*$):} Inspired by Test-Driven Development (TDD;~\citealp{beck2022test}), complete test cases allow precise fault localization, identifying where and under what conditions the code fails. This provides details on the failure's location and triggering inputs;
% When complete test cases are available, precise fault localization identifies exactly where and under what conditions the code fails. This includes detailed information on both the location of the failure and its triggering inputs.
(b) \textbf{Partial TC ($f_e$):} In more realistic settings with partial test coverage, fault localization is limited to tested code lines, potentially leaving faults in untested sections undetected. This type of feedback simulates incomplete real-world test suites, where only a subset of possible execution paths is covered. 
% In more realistic settings with partial test coverage, fault localization is limited to areas of the code that are tested, potentially leaving faults in untested sections undetected or ambiguous. This type of feedback simulates incomplete real-world test suites, where only a subset of possible execution paths is covered. 
% \begin{itemize}
%     \item \textbf{Full TC:} Fault localization can be precise, identifying exactly where and under what conditions the code fails. 
%     \item \textbf{Partial TC:} Fault localization is limited to the areas covered by the test cases, potentially missing faults in untested portions of the code.
% \end{itemize}
%Throughout the test coverage, execution feedback lacks guidance for 
Refinement guidance is not provided in either full or partial test coverage executions.


% Throughout test coverage, execution feedback lacks guidance for refinement, there is no explicit corrective suggestions. % as may be implicit in the nature of the failures but 
% often lacks explicit corrective suggestions.
% It involves running the code to identify runtime errors or to check if the output matches expected results based on given inputs. 
% To utilize execution feedback, we need an (secured) environment that all the dependencies regarding the code are prepared~\citep{codex}, and test cases of input and expected output pairs. 
% We further divide this feedback based on test coverage (TC) such as branch coverage:
% \begin{itemize}
%     \item \textbf{Partial TC:} Limited number of test cases is available, reducing the burden of preparing them. 
%     \item \textbf{Full TC:} Maximize coverage to improve feedback quality. 
% \end{itemize}

\paragraph{Verbal Feedback} Verbal feedback in our benchmark is generated by LLMs simulating human feedback, ranging from novice to expert levels. This feedback could emulate responses from humans, such as experts guiding LLMs to generate code, or novices without coding expertise. Both fault localization and refinement guidance are provided verbally, but the extent and accuracy of this feedback depend on the simulated provider: (a) \textbf{Novice-Level ($f_v$):} At this level, the LLM simulates novice feedback, which tends to rely heavily on other feedback types (e.g., compilation or execution feedback) and often restates observed errors without deeper understanding. Refinement guidance may be incorrect or absent, due to the novice's limited expertise simulated by the LLM's potential hallucinations. (b) \textbf{Expert-Level ($f_v^*$):} 
Expert feedback reflects scenarios where expert programmers use LLMs to automate simpler tasks, allowing them to concentrate on more complex coding challenges. 
This feedback is simulated by the LLM to provide detailed fault localization and code refinement guidance. 
It generates the feedback an expert programmer might give, focusing on resolving issues with a deep understanding of programming concepts and the expected functionality. %This feedback is generally precise and actionable, offering clear steps to correct errors.


% In contrast, expert feedback is simulated by the LLM to provide fault localization independently, drawing on a deeper understanding of programming concepts and desired functionality. Refinement guidance is generally accurate, offering clear, actionable steps to resolve identified issues. This feedback simulates scenarios where expert programmers might use LLMs to automate repetitive tasks and focus on more complex coding challenges.


\subsubsection{Verbal Feedback Generation}
We generate $f_v$ and $f_v^*$ by GPT-4o with in-context learning~\citep{dong2022survey}. We chose GPT-4o as we found it to be best at following instructions and minimizing risks such as ground truth code leakage, as discussed in Appendix~\ref{appendix:gt_code_leakage}. 
% Specifically, we conduct in-context learning~\citep{dong2022survey} for each $f_v$ and $f_v^*$:\footnote{In-context examples are provided in Appendix~\ref{appendix:icl_examples}.}
% To ensure reproducibility and enable extensive comparisons, we obtain verbal feedback through systematic procedures inspired by~\citet{wang2024mint} and~\citet{yao2024tau}, utilizing GPT-4o via in-context learning~\citep{dong2022survey}.\footnote{See Appendix~\ref{appendix:simulator_comparison} for comparative analysis of verbal feedback across different LLMs.}
% This approach avoids the challenges associated with inconsistent feedback quality and the impracticality of sourcing real-time human input:
\begin{itemize}[left=4pt]
    \item \textbf{Generation of $f_v$:} Novice-level verbal feedback is constructed by verbalizing outputs from compilation and/or execution feedback, possibly supplemented with language model predictions. %Fault localization relies on restating compiler or runtime errors, and guidance for refinement may be incorrect or vague.\footnote{An in-context example for novice-level verbal feedback generation is provided in Appendix~\ref{appendix:novice_icl}.}
    \item \textbf{Generation of $f_v^*$:} Expert-level verbal feedback is produced by showing the agent's code with the correct reference code~\citep{wang2024mint}, enabling a comparison and subsequent detailed feedback on required modifications. We perform extensive analysis to ensure no ground truth code is leaked during $f_v^*$ generation (see Appendix~\ref{appendix:gt_code_leakage} for analysis on this).
\end{itemize}
See Appendices~\ref{appendix:simulator_comparison} for comparative analysis of verbal feedback using different LLMs, ~\ref{appendix:icl_examples} for the in-context examples, and ~\ref{appendix:case_study} for a generated example of $f_v^*$.
{For the detailed prompting methods for the $f_v$ and $f_v^*$ construction, please refer to our codebase.\footnote{\footnotesize\href{https://huggingface.co/spaces/ConvCodeWorld/ConvCodeWorld}{ \texttt{https://huggingface.co/spaces/ConvCodeWorld/ConvCodeWorld}}}
% ~\footnote{\footnotesize\href{https://github.com/sfc-gh-hhan/convcodeworld}{ \texttt{https://github.com/sfc-gh-hhan/convcodeworld}}}
% \subsection{Ensuring Reproducible Evaluation}
\paragraph{Reproducibility and Cost-Efficiency Compared to Human Annotation}
\label{convcodeworld:reproducibility}
Manual annotation of verbal feedback is costly and lacks reproducibility. Instead, we use GPT-4o, as supported by prior studies demonstrating the effectiveness of LLM-generated feedback in benchmarks~\citep{wang2024mint,yao2024tau}. This approach improves reproducibility by using a consistent feedback provider and reduces annotation costs to about 1.5\% (Appendix~\ref{appendix:efficiency}) of those for human annotators. 
%\footnote{In the worst-case scenario, CodeLlama-7B-Instruct, which requested the most verbal feedback due to its low performance, incurred a total cost of \$215 (26.4M input tokens and 5.5M output tokens) for 15,905 turns using GPT-4o-2024-05-13 pricing (\$5/1M input tokens and \$15/1M output tokens). By comparison, assuming human annotation takes 96 seconds per turn~\citep{wang2024mint} and the average U.S. private non-farmer worker's hourly wage is \$35.04 according to \cite{uswage}, the human annotation cost would be approximately \$14,792. } 

% One challenge in ensuring reproducibility and enabling extensive comparisons is obtaining consistent verbal feedback for each turn of LLM-generated code. 
% Manual annotation is costly and prone to inconsistency, as feedback style can vary between annotators.
% To address this, we use GPT-4o for verbal feedback generation, leveraging previous studies that have demonstrated the efficacy of LLM-generated feedback in evaluation benchmarks~\citep{wang2024mint,yao2024tau}. Using GPT-4o reduces annotation costs to approximately 1.5\% of the cost of human annotators.\footnote{In the worst-case scenario, CodeLlama-7B-Instruct, which requested the most verbal feedback due to its low performance, incurred a total cost of \$215 (26.4M input tokens and 5.5M output tokens) for 15,905 turns using GPT-4o-2024-05-13 pricing (\$5/1M input tokens and \$15/1M output tokens). By comparison, assuming human annotation takes 96 seconds per turn~\citep{wang2024mint} and the average U.S. private non-farmer worker's hourly wage is \$35.04 according to \cite{uswage}, the human annotation cost would be approximately \$14,792. } 
% Additionally, by standardizing the feedback provider to a single model, we ensure consistent feedback across all evaluated code generation models.


% \begin{itemize}
%     \item \textbf{Novice-Level:} At this level, fault localization relies heavily on other feedback types, such as compilation or execution feedback. Novice feedback may restate observed errors without deeper insight. Guidance for refinement may be incorrect or absent due to limited expertise.\footnote{Feedback generated by LLMs, including self-corrections that might contain hallucinations, falls into this category.}
%     % At this expertise level, fault localization heavily relies on other forms of feedback, such as compilation errors or execution results. Novice-level feedback may simply echo these errors without deeper insight. Guidance for refinement may be incorrect or entirely absent, as the provider may lack the necessary knowledge to suggest effective corrections. This type of feedback often includes general expressions of dissatisfaction or confusion and may contain inaccuracies or lack specificity.\footnote{Feedback generated by LLMs, including self-corrections that might contain hallucinations, falls into this category.}
%     \item \textbf{Expert-Level:} Experts provide fault localization independently of other feedback, based on their understanding of programming concepts and desired functionality. Guidance for refinement is mostly correct and offers actionable suggestions for improvement on how to fix the identified problems.\footnote{This setting is practical as expert programmers may leverage LLMs to reduce repetitive tasks and focus on more critical coding.} 
%     % At the expert level, fault localization is provided independently of other feedback types. Experts can pinpoint issues within the code based on their understanding of programming concepts and the desired functionality. Guidance for refinement is mostly correct, offering actionable suggestions on how to fix the identified problems. This feedback provides high-level directions for necessary corrections without revealing the exact solution, ensuring that the refinement process remains a learning opportunity. 
%     % \item Expert users, who are proficient in generating code themselves, may leverage LLMs to reduce repetitive tasks and focus on more critical coding. Unlike novice users, experts not only can determine whether the generated code is correct or identify errors, but they can also provide correct guidance on how to rectify these issues. However, as providing excessively specific feedback, e.g., ground truth itself in one extreme,  it is more practical to offer high-level directions for necessary corrections. 
% \end{itemize}




% \subsection{Observations}
% To ensure reproducibility and enable extensive comparisons, we obtain verbal feedback through systematic procedures inspired by~\citet{wang2024mint} and~\citet{yao2024tau}, utilizing GPT-4o via in-context learning~\citep{dong2022survey}.\footnote{See Appendix~\ref{appendix:simulator_comparison} for comparative analysis of verbal feedback across different LLMs.}
% This approach avoids the challenges associated with inconsistent feedback quality and the impracticality of sourcing real-time human input:
% \begin{itemize}
%     \item \textbf{Novice-Level NL Feedback Generation:} This feedback is constructed by verbalizing outputs from compilation and/or execution feedback, possibly supplemented with language model predictions. Fault localization relies on restating compiler or runtime errors, and guidance for refinement may be incorrect or vague.\footnote{An in-context example for novice-level feedback generation is provided in Appendix~\ref{appendix:novice_icl}.}
%     \item \textbf{Expert-Level Verbal Feedback Generation:} This feedback is produced by showing the agent's code with the correct reference code~\citep{wang2024mint}, enabling a comparison and subsequent detailed feedback on required modifications.\footnote{An in-context example for expert-level feedback generation is provided in Appendix~\ref{appendix:expert_icl}.}
% \end{itemize}

% The human-in-the-loop setup lacks reproducibility for extensive comparisons because it involves providing immediate and consistent-quality human feedback after each code generation instance by an LLM. Moreover, this approach is cost-prohibitive.

% Instead, inspired by~\citet{wang2024mint} and~\citet{yao2024tau}, we build a reproducible environment by simulating user feedback with GPT-4o via in-context learning~\citep{dong2022survey}.\footnote{See Appendix~\ref{appendix:simulator_comparison} for comparative analysis of user feedback simulation across different LLMs. }

% \paragraph{Novice Feedback Simulation}
% We simulate novice users by verbalizing compilation and/or execution feedback and optionally including LLM-predicted information, which, like real novice feedback, can sometimes be incorrect.\footnote{The in-context example for novice feedback simulation is provided in Appendix~\ref{appendix:novice_icl}. }

% \paragraph{Expert Feedback Simulation}
% To simulate expert users, who have a clear understanding of the intended final code structure, our simulator LLM takes both the code generated by another LLM and the correct code~\citep{wang2024mint}, enabling a comparison and subsequent detailed feedback on required modifications.\footnote{The in-context example for expert feedback simulation is provided in Appendix~\ref{appendix:expert_icl}. }

% \subsection{Verbal Reinforcement Learning}


% In our POMDP framework, the agent faces partial observability because it cannot directly observe the true functionality state $s_t$ of the code at turn $t$. Instead, it must rely on observations--feedback from the environment--to update its belief state $b_t \in \mathbb{S}$, thereby the next action $a_{t+1}$ is chosen by the belief state $b_t$, i.e., $a_{t+1} = \pi(b_t)$. 



% Depending on real-world constraints, different scenarios offer varying feedback combinations, leading to different levels of the lower bound of belief accuracy:




% $\mathbb{S}$ is a state space where $s_t \in \mathbb{S}$ represents the functionality of code at turn $t$. The goal state is when the functionality of the code matches the desired specification, explained in the given problem description $x$. Formulating state as the functionality of code fits, as the functionality is not directly measurable but only through observations like code execution with full test coverage or code review by expert-level programmers, fitting the partial observability aspect.



% is not directly observable, 
% represents the underlying goal the agent is trying to achieve. Not directly observable, fitting the partial observability aspect.

% \begin{itemize}
%     \item State ($\mathbb{S}$): The true functionality of the desired code, encompassing all requirements and constraints. The rationale is that it represents the underlying goal the agent is trying to achieve. Not directly observable, fitting the partial observability aspect.
%     \item Action ($\mathbb{A}$): Code snippets generated or refined by the agent.
%     \item Observations ($\Omega$): Feedback received after submitting code, including: 
% \end{itemize}

% $\mathbb{S}$ is a state space, 
% and $\mathbb{A}$ is an action space.  
% We define a state $s \in \mathbb{S}$ as the functionality of the generated code, and $s$ is the goal state if the code has the desired functionality that $u$ requested. $a_m \in \mathbb{A}_m$ is $m$'s action of code generation or refinement. $a_u \in \mathbb{A}_u$ is an action that the user $u$ provides feedback of the current version of code. 
% $T: \mathbb{S}\times\mathbb{A} \rightarrow \Delta(\mathbb{S})$ is a state transition probability function where $\Delta(\cdot)$ is a probability distribution of the set $(\cdot)$. 
% $\Omega=\Omega_m \times \Omega_u$ is a joint set of each agent's observations $\Omega_m$ and $\Omega_u$.  
% $O:\mathbb{S} \times \mathbb{A} \rightarrow \Delta(\Omega)$ is an observation function. 
% $R: \mathbb{S}\times\mathbb{A} \rightarrow \mathbb{R}$ is an immediate reward function returning a real number. 
% $h$ is the maximum number of turns (\textit{horizon}), and $b_0 \in \Delta(\mathbb{S})$ is a distribution of initial state at turn $t=0$. 

\begin{comment}
Given a generated code and corresponding feedback through interaction, this task satisfies following properties:
\begin{itemize}
    \item \textbf{Markov Decision Process:} To generate the next version of the code, $m$ only needs to consider the current version of the code and the corresponding feedback. %The functionality (and whether it matches to the desired functionality from $u$) of the currently generated code by $m$ is solely decided by the previous version of code, feedback, and the refinement action. 
    \item \textbf{Partially Observable:} $m$ might not be able to specify the functionality of the currently generated code (and thus whether it matches to the desired functionality from $u$) due to limited or noisy information in feedback, e.g., partial test coverage in execution feedback, incorrect feedback by novice user.   %Feedback corresponding to the generated code may not fully explain the status of the code, e.g., the correctness, how far the code is from the correct code, etc. 
    % \item \textbf{Decentralized:} Heterogeneous agents--$m$ and $u$--are cooperative to return code with the desired functionality. 
\end{itemize}
Therefore, we can formulate this conversational code generation task as a decentralized partially observable markov decision process (Dec-POMDP;~\cite{oliehoek2016concise}). 

Formally, we define a Dec-POMDP by a tuple $\mathcal{M}= \langle \mathbb{D}, \mathbb{S}, \mathbb{A}, T, \Omega, O, R, h, b_0 \rangle$. 
$\mathbb{D}=\{m,u\}$ is a set of agents consists of a code generation model $m$ and a human user $u$.\footnote{For simplicity, we assume a single code generation model and user, but this can be extended to multiple models and users.} 
$\mathbb{S}$ is a state space, and $\mathbb{A}=\mathbb{A}_m \times \mathbb{A}_u$ is a joint action space of each agent's action space $\mathbb{A}_m$ and $\mathbb{A}_u$. 
We define a state $s \in \mathbb{S}$ as the functionality of the generated code, and $s$ is the goal state if the code has the desired functionality that $u$ requested. $a_m \in \mathbb{A}_m$ is $m$'s action of code generation or refinement. $a_u \in \mathbb{A}_u$ is an action that the user $u$ provides feedback of the current version of code. 
$T: \mathbb{S}\times\mathbb{A} \rightarrow \Delta(\mathbb{S})$ is a state transition probability function where $\Delta(\cdot)$ is a probability distribution of the set $(\cdot)$. 
$\Omega=\Omega_m \times \Omega_u$ is a joint set of each agent's observations $\Omega_m$ and $\Omega_u$.  
$O:\mathbb{S} \times \mathbb{A} \rightarrow \Delta(\Omega)$ is an observation function. 
$R: \mathbb{S}\times\mathbb{A} \rightarrow \mathbb{R}$ is an immediate reward function returning a real number. 
$h$ is the maximum number of turns (\textit{horizon}), and $b_0 \in \Delta(\mathbb{S})$ is a distribution of initial state at turn $t=0$. 

The conversational code generation task is conducted as follows. For each turn $t$, the code generation model $m$ receives an observation $o_{t,m} \in \Omega_m$ from the state $s_t \in \mathbb{S}$ with the observation probability $O(s_t, a_t')$ representing $Pr(o_t|s_t,a_t')$, where $o_t = (o_{t,m}, o_{t,u})\in\Omega$ and $a_t' \in \mathbb{A}$. Then, $m$ conducts an action $a_{t+1,m} \in \mathbb{A}_m$ from $o_{t,m}$ by a policy $\pi_m: \Omega_m \rightarrow \mathbb{A}_m$, generating code from the current observation. 
The transition function $T$ then updates $s_t$ by the probability distribution $T(s_t,a_{t+1})$ representing probability $Pr(s_t'|s_t,a_{t+1})$, where $s_t' \in \mathbb{S}$ is a temporal state and $a_{t+1} = (a_{t+1,m},\phi) \in \mathbb{A}$ is a joint action of the model's action $a_{t,m}$ and the user's action $\phi$, signifying no user action is taken yet. 
Then, the user $u$ receives an observation $o_{t,u}' \in \Omega_u$ with the observation probability $O(s_t', a_{t+1})$ representing $Pr(o_t'|a_{t+1}, s_t')$, where $o_t'=(o_{t,m}', o_{t,u}')\in \Omega$. Next, $u$ takes an action $a_{t+1,u} \in \mathbb{A}_u$ from $o_{t,u}'$ by a policy $\pi_u:\Omega_u \rightarrow \mathbb{A}_u$, providing user feedback to $m$. Finally, the transition function $T$ updates $s_t'$ into $s_{t+1}$ by $T(s_t', a_{t+1}')$ representing $Pr(s_{t+1}|s_t', a_{t+1}')$, where $a_{t+1}'=(\phi,a_{t+1,u}') \in \mathbb{A}$. 
% The initial condition 

%$o_t,m= \langle x, a_{t,m} \rangle$

As $m$ cannot directly fetch the state from the environment, it has a \textit{belief}
\begin{align}
\label{eq:belief}
b_m(s_t)\delequal Pr(s_t|o_{t,m}, a_{t-1,m}, o_{t-1,m},..., o_{1,m}, a_{0,m}),
\end{align}
for each $s_t \in \mathbb{S}$ to estimate the current state. 
Though the accuracy of $b_m$ varies to which $m$ we employ, the upper bound of this accuracy is limited by the degree of information loss in observations, which is determined by the feedback available from each scenario's configuration. 
We define $o_{t,m}$ the observation of $m$ at turn $t$ as a tuple:
\begin{align}
\label{eq:o_m}
o_{t,m}\delequal \langle x, \hat{y}_t, F_t \rangle,
\end{align}



How the belief state is likely to be the current state? 

The degree of observability is diverse according to the available feedback. 

\end{comment}




% We categorize three different types of feedback in conversational code generation scenario: Compilation Feedback, Execution Feedback, and User Feedback.

\begin{comment}    
In conversational code generation, we categorize the types of feedback that can be provided to LLMs.
First, we examine the possible signals that feedback can comprise.

\paragraph{Feedback Signals}
\sw{maybe we unify the category pillars? hard to match these three with abstract}
The feedback signals can be divided into three categories:
\begin{itemize}
    \item \textbf{User Intent Satisfaction:} Assessing whether the generated code aligns with user intentions. %assessing whether the generated code aligns with the user's intentions.
    \item \textbf{Error Notification:} Identifying errors within the code. % alerting errors in the code when present.
    \item \textbf{Refinement Guidance:} Proposing guidance to rectify identified issues.
\end{itemize}


Based on the defined feedback signal categories, Table~\ref{tab:categorization} categorizes the feedback types for conversational code generation. 
\paragraph{Compilation Feedback}
Compilation feedback uses a compiler to check for syntax errors. This feedback is cost-effective as it does not require execution conditions or test cases; however, it fails to assess the code's alignment with user intent in terms of functionality.
% This feedback provides whether the current code predictions is: 1) Syntactically correct (knows the target programming language’s grammar) and 2) Compilable (no undefined variables/functions/classes, type mismatch for type-sensitive languages, etc.)
% Pros: Cheap \& Easy to obtain (easily scalable to large scale)
% Cons: Weak Feedback (do not tell whether the code is functionally correct)

\paragraph{Execution Feedback}
It involves running the code to identify runtime errors or to check if the output matches expected results based on given inputs. 
% It provides whether the execution of the code prediction is as expected.
% Two subtypes of errors: 
% 1) Runtime errors: Errors only produced when the code runs. Abnormal termination such as DivisionByZero, IndexError, etc.
% 2) Logical errors: While the code runs and terminates normally, the execution result is different from the expected one. 
To utilize execution feedback, we need an (secured) environment that all the dependencies regarding the code are prepared~\citep{codex}, and test cases of inputs and expected outputs. 
% We further break down this feedback based on the coverage of test cases used, distinguishing between Public TCs settings, which offer limited test case coverage with relaxed burden of preparing test cases, and Full TCs settings, which aim to maximize coverage to enhance feedback quality.
% \sw{TCs can be rigorously annotated with coverage in mind-- We repurpose human annotated TCs from BigCodeBench,  5.6 test cases with an average branch coverage of 99\%. However, in more realistic settings, some subset of such TCs will be annotated. To cover both scenarios, we consider high/low coverage scenarios} hj: I paraphrased this in Sec 4.1
We further divide this feedback based on test coverage from low, which user provides limited number of test cases to reduce the burden of preparing them, to high, which aims to maximize coverage to improve feedback quality.  %\yh{is there a quantitative way to differentiate full TCs vs public TCs.  Also does public TCs have anything to do with "public"?  If not, maybe it should have a different name.}
% Pros: Rich Feedback. With sufficient test cases, this feedback can give the functional correctness of the code prediction. 
% Cons: For execution, it requires: 1) Sufficient amount of test cases, and 2) (Secured) Environment that all the dependencies regarding the code are prepared. 

% Regarding user feedback, we divide based on their level of expertise into two groups. 
% We divide user feedback into two groups based on the level of expertise. 
We divide user feedback into two groups based on expertise level, recognizing that the correctness of guidance may vary for lower expertise levels. 
\paragraph{Novice User Feedback}
Novice users, who are not familiar with programming, can build feedback on tool feedback, such as
%after compiling or 
compilation and execution of the model-generated code. They report either their satisfaction with the outcome or, if dissatisfied, identify where the issues arose, such as syntax errors or stack trace results. 
Consequently, feedback from novice users often resembles %the information provided in compilation and execution feedback, although it generally offers similar content. Optionally, this feedback 
but may contain refinement directions, which might include erroneous information or lack specific corrective suggestions.\footnote{Feedback from LLMs, including self-corrections, is categorized here, as such feedback may contain hallucinations.} 


\paragraph{Expert User Feedback}
Expert users, who are proficient in generating code
themselves, may leverage
%often utilize LLMs to generate code, aiming 
LLMs to reduce repetitive tasks and focus on more critical coding. Unlike novice users, experts not only can determine whether the generated code is correct or identify errors, but they can also provide correct guidance on how to rectify these issues. However, as providing excessively specific feedback, e.g., ground truth itself in one extreme,  
%could be as involved as refining the code themselves; thus, 
it is more practical to offer high-level directions for necessary corrections. 

\subsection{User Feedback Simulation as Reproducible Environments}
The human-in-the-loop setup lacks reproducibility for extensive comparisons because it involves providing immediate and consistent-quality human feedback after each code generation instance by an LLM. Moreover, this approach is cost-prohibitive.

Instead, inspired by~\citet{wang2024mint} and~\citet{yao2024tau}, we build a reproducible environment by simulating user feedback with GPT-4o via in-context learning~\citep{dong2022survey}.\footnote{See Appendix~\ref{appendix:simulator_comparison} for comparative analysis of user feedback simulation across different LLMs. }

\paragraph{Novice Feedback Simulation}
We simulate novice users by verbalizing compilation and/or execution feedback and optionally including LLM-predicted information, which, like real novice feedback, can sometimes be incorrect.\footnote{The in-context example for novice feedback simulation is provided in Appendix~\ref{appendix:novice_icl}. }

\paragraph{Expert Feedback Simulation}
To simulate expert users, who have a clear understanding of the intended final code structure,  
our simulator LLM takes both the code generated by another LLM and the correct code~\citep{wang2024mint}, enabling a comparison and subsequent detailed feedback on required modifications.\footnote{The in-context example for expert feedback simulation is provided in Appendix~\ref{appendix:expert_icl}. }
\end{comment}

\subsection{Feedback Combinations}
\label{convcodeworld:combination}
In each of our turns, we simulate different real-world interactive code generation scenarios by combining representative feedback combinations. We represent feedback settings by taking a Cartesian product across compilation feedback settings, execution feedback settings, and verbal feedback settings. In particular, we formalize a feedback combination $\Omega$ as a tuple of feedback expressed by regular expression notation:
\begin{align}
    \label{eq:observation}
    \Omega = 
    \langle f_c, [\phi | f_e | f_e^*], [\phi | f_v | f_v^*] \rangle.
    % \Omega = \{\langle x,y,z \rangle | f_c^t \in \{f_c\}, f_e^t \in \{\phi, f_e, f_e^*\}, f_v^t \in \{\phi, f_v, f_v^*\} \}\} 
\end{align} 
The choices of feedback settings is simply dictated by these observations: (a) Compilation feedback $f_c$ is always present since it is cheap and universally available; (b) Execution feedback varies among being unavailable ($\phi$), available with partial test coverage ($f_e$), or with full test coverage ($f_e^*$); (c) Verbal feedback can be also unavailable ($\phi$), available with novice-level ($f_v$), or with expert-level ($f_v^*$). By combining these options---1 for compilation feedback, 3 for execution feedback, and 3 for verbal feedback---we obtain 9 distinct feedback combinations.

% (1) $f_c$ is always available due to the easy accessibility of compilers; 
% (2) Execution feedback varies between being unavailable ($\phi$), available with partial test coverage ($f_e$), or with full test coverage ($f_e^*$);
% (3) Verbal feedback can be also unavailable ($\phi$), available with novice-level ($f_v$), or with expert-level ($f_v^*$). 

% We interpret these diverse feedback combinations as observability in the framework of partially observable Markov decision processes (POMDPs). In this setting, observability refers to the information available to the LLM at each turn, which influences its decision-making process.

% where we formally redefine each feedback symbol ($f_c$, $f_e$, $f_e^*$, $f_v$, and $f_v^*$) as the set of each feedback type. 
Each feedback combination $\Omega$ reflects a unique real-world scenario, allowing us to comprehensively evaluate LLMs under diverse conditions as listed in Table~\ref{tab:partial_observability}.

% Given a problem description $x$, the target code generation model $\mathcal{M}$ iteratively generates the next version of code from feedback:
% \begin{align}
%     \label{eq:convcodeworld}
%     \mathcal{C}_{t+1}^{\mathcal{M}} = \mathcal{M}(x;\mathcal{C}_{t}^{\mathcal{M}};\Omega_t), 
% \end{align}
% where code $\mathcal{C}_t^{\mathcal{M}}$ is generated code by $\mathcal{M}$ at turn $t$ and $\Omega_t$ is the tuple of feedback at turn $t$. 

Now it is easy to formalize the interactive code generation in \ours: For each turn $t$, the target code generation model $\mathcal{M}$ iteratively generates the next version of code $\mathcal{C}_{t+1}^{\mathcal{M}}$ from the problem description $x$, the generated code $\mathcal{C}_{t}^{\mathcal{M}}$, and the corresponding tuple of feedback $\Omega_t$:
\begin{align}
    \label{eq:convcodeworld}
    \mathcal{C}_{t+1}^{\mathcal{M}} = \mathcal{M}(x;\mathcal{C}_{t}^{\mathcal{M}};\Omega_t).
\end{align}
% where $x$ is the problem description.








\section{\oursstatic: A Static Benchmark for Efficient Evaluation}
\label{sec:convcodebench}
While \ours provides a comprehensive live benchmark for evaluating LLMs in interactive code generation scenarios, it requires access to an LLM for verbal feedback generation. Although this approach is more efficient and reproducible than using human annotators, it still introduces additional overhead, cost, and potential reproducibility issues, especially when using closed API models like GPT-4o. To address these challenges, we propose \oursstatic, a static benchmark designed to complement \ours. 

\oursstatic leverages feedback logs generated by a fixed reference model interacting with GPT-4o. The benchmark presents pre-generated conversations---including the code produced by the reference model and the corresponding feedback, such as verbal feedback by GPT-4o---and tasks the target code model with refining the code. We revise Equation~\ref{eq:convcodeworld} to formalize \oursstatic as follows.
For each turn $t$, the target code generation model $\mathcal{M}$ is provided generated code $\mathcal{C}_{t}^{\widebar{\mathcal{M}}}$ from a reference model $\widebar{\mathcal{M}}$, and the corresponding tuple of feedback $\widebar{\Omega}_t$ provided to outputs generated by $\widebar{\mathcal{M}}$. Given the model and feedback corresponding to a reference model, the target model $\mathcal{M}$ generates the next version of code $\mathcal{C}_{t+1}^{\mathcal{M}}$:
\begin{align}
    \label{eq:convcodebench}
    \mathcal{C}_{t+1}^{\mathcal{M}} = \mathcal{M}(x;\mathcal{C}_{t}^{\widebar{\mathcal{M}}};\widebar{\Omega}_t). 
\end{align}


This approach offers several advantages:
\begin{itemize}[left=4pt]
    \item \textbf{Elimination of Dependency on  External LLMs or APIs for Verbal Feedback Generation:} By using static feedback logs, \oursstatic reduces costs and latency associated with real-time LLM interactions.
    % Elimination of reliance on separate LLMs or APIs for user feedback simulation, reducing costs and latency.
    \item \textbf{Parallel Processing of Inference Calls:} The static nature of the benchmark allows for batched evaluation requests across all turns, enabling faster turnaround times.
    % Parallel processing of inference calls for all turns, enabling batched evaluation requests and faster turnaround times.
    \item \textbf{Enhanced Reproducibility:} Utilizing fixed logs ensures consistent evaluations, further increasing reproducibility.
    % Increased reproducibility due to the use of static logs.
\end{itemize}



% where $x$ is . 

% \begin{table}[t]
% \begin{wraptable}{r}{0.52\textwidth}
\begin{wraptable}{h!}{0.46\textwidth}
\centering
\caption{Performance of three different LLMs at turn 0 (i.e. the initial code generation without feedback) and at turn 10 on \ours where $\Omega = \langle f_c, \phi, f_v^* \rangle$. }
\scriptsize
% \small
\begin{tabular}{lcc}\thickhline
\multirow{2}{*}{\textbf{Model}} & \multicolumn{2}{c}{\textbf{Pass@$1$}} \\
& \textbf{Turn 0} & \textbf{Turn 10} \\ \hline
CodeLlama-7B-Instruct & 21.8 & 55.2 \\ 
DeepSeek-Coder-6.7B-Instruct & 35.2 & 83.1 \\ 
GPT-4-0613 & 46.0 & 92.5 \\\thickhline
\end{tabular}
\normalsize
\label{tab:model_select_for_statis}
% \end{table}
\end{wraptable}

One key concern when using \oursstatic is the bias introduced by pre-generated interaction logs prompting the question: \textit{Can we ensure high correlation between static and live benchmarks by an appropriate choice of reference model?} 

We hypothesize that using logs from a weaker model, where the generated code still requires refinement even after multiple turns, allows for better differentiation among models based on their ability to improve unsolved code.

Based on this rationale, we used CodeLlama-7B-Instruct as a reference model, as it is worse than many other models at both turns 0 and 10 (see Table~\ref{tab:model_select_for_statis}). %\footnote{One may consider a weaker model than CodeLlama-7B-Instruct, such as DeepSeek-Coder-}
We find that creating \oursstatic with this model yields a very strong correlations with live settings. When comparing models on \ours and \oursstatic, we obtained Spearman's rank correlations between 0.82 and 0.99. We find that using CodeLlama-7B-Instruct as the base model outperforms both DeepSeek-Coder-6.7B-Instruct (a stronger code model) and GPT-4 (one of the state-of-the-arts) as reference models (\S\ref{sec:static_results}). 

In summary, we find that \oursstatic is a great way of comparing code models within the framework of \ours despite relying on logs from a reference model because of strong rank correlations across the two setups.

\begin{comment}
Key points to address
1. \ours is a live benchmark. Needs access to an LLM for simulating user feedback. Therefore has an extra overhead (either running a separate model or calling a closed API) and cost. Close apis additionally have reproducibility issues.
2. idea: can we use a feedback logs with a fixed code model (reference model) and gpt4-o, and then use these static logs. In particular, we can take frozen conversation between reference model and gpt4-o until turn i (for all values i), and then ask our target code model to refine the outcome and check if we our target model can get it right. This will have 3 benefits: we do not need to rely on a separate LLM/Api for user feedback simulation which will drive costs and latency down; furthermore, we can make inference calls for all turns in parallel (unlike the live setting where LLM calls need to happen in sequence), which allows batching eval requests for a faster turn around time.
3. Of course, the risk here is that we are introducing a bias against the live setting where the code and feedback corresponds to the target model. Therefore the key design constraint here is: can we choose a reference model in such a way such that performance of code models on the static benchmark correlates strongly with the performance on the live benchmarks. If we choose a relatively weak model, we risk biasing the target model with very noisy interaciton logs. If we choose a model with strong turn 1 performance, we risk not exploring the space of all turns. We need a \textit{goldilocks model} -- a model whose turn 1 performance is not very high but that also manages to climb up with more feedback so we know that the reference model output and corresponding user feedback is not noisy. Indeed we find that deepseek-coder 6.7B-instruct fits this criteria -- it's turn 1 performance on bigcodebench is 35\%, but it climbs up to 82\% by turn 10 (matching gpt4-o). We hypothesize that using Deepseek Coder 6.7B as reference model will create a static benchmark that will correlate strongly with live performance. And indeed we find in section HOJAE link section here, that this way of creating \oursstatic leads to strong correlation with live benchmarks (Spearman rank correlation between 0.78 and 0.99) outperforming both codellama-34B (weaker code model) and gpt4-o (strong code model) as reference models.

To address the cost of environment API calls from \ours,
 we introduce \oursstatic, a  static benchmark not requiring expensive environments, yet offering closely correlated evaluation results with \ours.
 
 \oursstatic~replaces environments with uses logs from \ours, generated by a reference LLM, along with simulated user feedback, based on which we can evaluate the target LLMs, tasked to refine reference LLM generation.
 
The advantage of this approach is having previous interactions logged, such that it is efficient, and reproducible without re-geerating user feedbacks.
Desirably, reference LLMs should correlate sufficiently with target LLMs to test, for which we hypothesized 
with reference models with high generation accuracy, diverse accuracy, and low accuracy. 

Among hypotheses test, we found DeepSeek-Coder-6.7B-Instruct as the reference LLM \sw{HOJAEelaborate this decision} and found a high Spearman's rank correlation (0.79 to 0.99) between the rankings in \oursstatic and \ourslive.
\end{comment}

\section{Experiments}
\label{sec:experiments}

\begin{figure*}[t]
\vspace{-6mm}
    \centering
    \includegraphics[width=0.8\linewidth]{figs/compare.pdf}
    \vspace{-4mm}
    \caption{\textbf{Qualitative comparison} with the baseline for generating a sequence of novel view images.  
    The results demonstrate that our method synthesizes more consistent multi-view images compared to our baseline model (Zero123). In addition, compared to SyncDreamer, our method visually maintains better similarity to the conditioned image and appears more natural.}
    \label{fig:sota_compare}
\vspace{-5mm}
\end{figure*}

\subsection{Experimental Setups}
\textbf{Dataset.}
Following previous work~\cite{zero123, SyncDreamer}, we evaluate our work on the Google Scanned Object (GSO)~\cite{GSO} dataset to verify the zero-shot novel view image synthesis capability. 
We also provide results for additional datasets in the Supplementary Material.
Specifically, we randomly select 30 objects from the GSO dataset with various object categories. 
Unlike recent approaches~\cite{mvdream, SyncDreamer} that aim to enhance the consistency of novel view synthesis models by generating multiple fixed-view images, our method can generate images from any camera pose and any number of views. Therefore, we conduct experiments under different camera pose settings to validate our approach:
specifically, 
1) \textit{16-views with free camera pose}: for each object, we circularly render 16 views with the elevation angles ranging in $[-10\degree, 40\degree]$ and the azimuth angles are evenly distributed in $[0\degree, 360\degree]$. 
2) \textit{16-views with fixed camera pose}: We maintain a constant elevation angle of $30\degree$ and uniformly sample azimuth angles (same as SyncDreamer~\cite{SyncDreamer}).
3) \textit{32-views with free camera pose}: Similar to the first setting, but we sample 32 views.
It's important to note that our method does not require additional training or fine-tuning on any datasets.

\noindent\textbf{Metrics.}
To validate the effectiveness of our method, we mainly evaluate it based on three criteria:
1) \textit{Quality Score}. We evaluate the image quality of synthesized multi-view images by measuring their similarity with ground truth images. Following prior research~\cite{zero123, sparsefusion}, we report the similarity between the synthesized images and the ground truth images with standard metrics: PSNR, SSIM~\cite{ssim}, and LPIPS~\cite{lpips}.
2) \textit{Multi-view Consistency Score}. As the primary goal of our work is to improve the consistency of generated images, we also employ the 3D consistency score~\cite{3dim} to verify the consistency among the synthesized images. Specifically, we train an Instant-NGP~\cite{instant_ngp} with the input image and part of the synthesized novel view images of our model and evaluate the similarity between the remaining synthesized images and the rendered images of Instant-NGP. For the synthesized multi-view images of each object, we allocate $3/4$ for training and reserve the remaining $1/4$ for validation.
Intuitively, if the consistency of synthesized images is improved, the NeRF-like model will train a better object representation, and the re-rendered images will agree more with the validation images.
3) \textit{Input Consistency Score}. To assess the faithfulness of synthesized images in preserving the identity of the input condition image, we introduce the input consistency score. This score calculates the similarity of each synthesized image with the input condition image, utilizing the LPIPS metric.

In addition, we use synthesized multi-view images to train a neural 3D reconstruction model (NeuS~\cite{neus}) and report commonly used Chamfer Distances (CD) and Volume IoUs between the trained 3D model and the ground truth.

\noindent\textbf{Baselines.}
Given that our main goal is to improve the consistency of the trained baseline model without further fine-tuning, we mainly compare our approach with the used baseline model Zero123~\cite{zero123}. Additionally, we compare our method to the SOTA approaches such as PGD~\cite{tseng2023consistent} and SyncDreamer~\cite{SyncDreamer} using the same Zero123 base model.

\noindent\textbf{Implementation Details.}
We use the official checkpoint provided by Zero123~\cite{zero123}, which is trained on objaverse~\cite{objaverse} for 165,000 steps. We inject our epipolar attention layer after step $T=4$ and layer $L=10$ by default. We find that feature fusion weight $\alpha=0.5$, and the number of context views $M=2$ work better.

\begin{table}[t]
\centering
\caption{Comparison of multi-view consistency, image quality, and input consistency of synthesized multi-view images at the 16-view setting with free camera pose.}
\label{tab:view16_free_compare}
\vspace{-2mm}
\scalebox{0.6}{
\begin{tabular}{c ccc ccc c}
\toprule
              & \multicolumn{3}{c}{Multi-view Consistency} & \multicolumn{3}{c}{Quality Score} & \multicolumn{1}{c}{Input Consis.} \\
              \cmidrule(lr){2-4} \cmidrule(lr){5-7} \cmidrule(lr){8-8}
              & PSNR$\uparrow$  & SSIM$\uparrow$ & LPIPS$\downarrow$ 
              & PSNR$\uparrow$  & SSIM$\uparrow$ & LPIPS$\downarrow$ 
              & LPIPS$\downarrow$ 
              \\ \midrule

Zero123
& 15.225        & 0.645       & 0.408
& 14.255        & 0.747       &	0.208
& 0.303         
\\
SyncDreamer
& 14.830        & 0.626       & 0.434
& 12.650        & 0.713       &	0.254
& 0.317         
\\
Ours 
& \best{18.300}	& \best{0.734}	& \best{0.355}
& \best{14.947}	& \best{0.763}	& \best{0.191}
& \best{0.282}
\\

\bottomrule
\end{tabular}
}
\end{table}

\begin{table}[t]
\vspace{-1mm}
\centering
\caption{Comparison of multi-view consistency, image quality, and input consistency at the 16-view setting with fixed camera pose as SyncDreamer~\cite{SyncDreamer}.}
\label{tab:view16_fxied_compare}
\vspace{-3mm}
\scalebox{0.6}{
\begin{tabular}{c ccc ccc c}
\toprule
              & \multicolumn{3}{c}{Multi-view Consistency} & \multicolumn{3}{c}{Quality Score} & \multicolumn{1}{c}{Input Consis.} \\
              \cmidrule(lr){2-4} \cmidrule(lr){5-7} \cmidrule(lr){8-8}
              & PSNR$\uparrow$  & SSIM$\uparrow$ & LPIPS$\downarrow$ 
              & PSNR$\uparrow$  & SSIM$\uparrow$ & LPIPS$\downarrow$ 
              & LPIPS$\downarrow$ 
              \\ \midrule

Zero123
& 16.556        & 0.682       & 0.378
& 14.592        & 0.750       &	0.207
& 0.305         
\\
SyncDreamer
& \best{22.424}        & \best{0.812}       & \best{0.268}
& 15.269        & 0.749       &	0.196
& 0.300         
\\
Ours 
& 21.151	& 0.780	& 0.302
& \best{15.293}	& \best{0.764}	& \best{0.184}
& \best{0.287}
\\

\bottomrule
\end{tabular}
}
\vspace{-4mm}
\end{table}


\subsection{Comparison With Baseline Models}
The quantitative comparison on three settings are shown in Tab.~\ref{tab:view16_free_compare}, Tab.~\ref{tab:view16_fxied_compare}, and Tab.~\ref{tab:view32_free_compare}. The qualitative comparison is shown in Fig.~\ref{fig:sota_compare}.

\begin{table}[t]
\centering
\caption{Comparison of multi-view consistency and image quality scores of synthesized multi-view images at the 32-view setting with free camera pose.}
\vspace{-3mm}
\label{tab:view32_free_compare}
\scalebox{0.7}{
\begin{tabular}{c ccc ccc}
\toprule
              & \multicolumn{3}{c}{Multi-view Consistency} & \multicolumn{3}{c}{Quality Score} \\
              \cmidrule(lr){2-4} \cmidrule(lr){5-7}
              & PSNR$\uparrow$  & SSIM$\uparrow$ & LPIPS$\downarrow$ 
              & PSNR$\uparrow$  & SSIM$\uparrow$ & LPIPS$\downarrow$ 
              \\ \midrule

Zero123
& 16.515        & 0.694       & 0.378
& 15.142        & 0.733       &	0.211
\\
PGD~\cite{tseng2023consistent}
& 18.481        & 0.720       & 0.343
& 15.281        & 0.739       &	0.205
\\
Ours 
& \best{20.655}	& \best{0.792}	& \best{0.305}
& \best{15.268}	& \best{0.742}	& \best{0.203}
\\

\bottomrule
\end{tabular}
}
\vspace{-3mm}
\end{table}

\begin{table*}
  [t]
  \centering
  \resizebox{\textwidth}{!}{%
  \begin{tabular}{cccccccccccc}
    \toprule \multicolumn{2}{c}{Components}                                                             & \multicolumn{5}{c}{Re-executability Rate (\%)} & \multicolumn{5}{c}{Readability (\#)} \\
    \cmidrule(lr){1-2} \cmidrule(lr){3-7} \cmidrule(lr){8-12}        \hspace{8pt}\labelemoji\hspace{8pt}                                                                & \hspace{8pt}\toolemoji\hspace{8pt}                                      & O0                                 & O1             & O2             & O3             & AVG            & O0             & O1             & O2             & O3             & AVG            \\
    \hline
    \rowcolor[rgb]{0.93,0.93,0.93}\multicolumn{12}{c}{\textbf{Initialize with LLM4Decompile-End-6.7B~\citep{llm4decompile}}}   \\
    \xmark                                                                                              & \xmark                                    & 69.51                              & 46.95          & 50.61          & 46.34          & 53.35          & 3.98 & 3.41 & 3.44 & 3.38 & 3.55 \\
    \cmark                                                                                              & \xmark                                    & 75.61                              & 50.61          & 50.00          & 50.00          & 56.55          & 4.01 & 3.44 & 3.39 & \textbf{3.49} & 3.58 \\
    \xmark                                                                                              & \cmark                                    & 83.54                     & \textbf{56.10}          & 51.22          & 50.61 & 60.37 & 4.05 & 3.51 & 3.51 & 3.42 & 3.62 \\
    \cmark                                                                                              & \cmark                                    & \textbf{85.37}                            & \textbf{56.10}                     & \textbf{51.83} & \textbf{52.43}          & \textbf{61.43} & \textbf{4.13} & \textbf{3.60} & \textbf{3.54} & \textbf{3.49} & \textbf{3.69} \\

    \rowcolor[rgb]{0.93,0.93,0.93}\multicolumn{12}{c}{\textbf{Initialize with Deepseek-Coder-6.7B-base~\citep{deepseekcoder}}} \\
    \xmark                                                                                              & \xmark                                    & 59.15                              & 35.98          & 39.02          & 37.80          & 42.99          & 3.71 & 3.05 & 3.16 & 3.05 & 3.24 \\
    \cmark                                                                                              & \xmark                                    & 66.46                              & 41.46          & 38.41          & 36.59          & 45.73          & 3.76 & 3.17 & \textbf{3.21} & 3.08 & 3.31 \\
    \xmark                                                                                              & \cmark                                    & 70.73                              & 39.63          & 39.02          & 40.24          & 47.41          & 3.90 & 3.17 & 3.08 & 3.11 & 3.31 \\
    \cmark                                                                                              & \cmark                                    & \textbf{79.88}                     & \textbf{45.73} & \textbf{43.90} & \textbf{42.68} & \textbf{53.05} & \textbf{3.96} & \textbf{3.21} & 3.18 & \textbf{3.19} & \textbf{3.38} \\
    \bottomrule
  \end{tabular}%
  }
  \caption{The ablation study of different methods across four optimization levels
  (O0, O1, O2, O3), as well as their average scores (AVG). The results in bold represent the optimal performance. The ~\labelemoji~ and ~\toolemoji~ means Relabedling and Function Call. \textbf{Bold} denotes the best performance.}
  \label{tab:ablation}
\end{table*}



\begin{figure*}[ht]
    \centering
    \begin{minipage}{0.65\textwidth}
        \centering
        \includegraphics[width=0.95\linewidth]{figs/ablation.pdf}
        \vspace{-2mm}
        \captionof{figure}{Qualitative Comparison for different design choices. Our method, employing multi-view epipolar attention, demonstrates the best consistency.}
        \label{fig:ablation}
    \end{minipage}\hfill
    \begin{minipage}{0.33\textwidth}
        \centering
        \includegraphics[width=0.8\linewidth]{figs/neus_ver.pdf}
        \vspace{-3mm}
        \caption{Our method shows better direct 3D reconstruction~\cite{neus}.}
        \label{fig:neus}
    \end{minipage}
    \vspace{-5mm}
\end{figure*}

\noindent\textbf{Multi-view Consistency.}
Tab.~\ref{tab:view16_fxied_compare} presents the 3D consistency scores compared to our baseline model (Zero123) and SyncDreamer. The results indicate a significant improvement across all three metrics achieved by our method when compared with Zero123.
While our method exhibits a marginally lower numerical consistency score compared to SyncDreamer, it enables the synthesis of images with arbitrary camera poses.	
This capability is illustrated in Tab.~\ref{tab:view16_free_compare}, where our method consistently enhances consistency with changes in camera pose settings, whereas SyncDreamer fails to do so and exhibits inferior results compared to Zero123.
Furthermore, our method facilitates the synthesis of multi-view images with any number of camera views. This versatility is demonstrated in Tab.~\ref{tab:view32_free_compare}, where our method continues to achieve significant improvements in consistency scores, while SyncDreamer is unable to operate under such conditions.	

Meanwhile, Fig.~\ref{fig:sota_compare} provides a qualitative comparison with the baseline. While both our method and SyncDreamer enhance consistency, our method visually preserves better similarity to the input image, including color and texture details. The input consistency score further corroborates this.

\noindent\textbf{Image Quality.}
While our primary goal centers around enhancing the consistency of synthesized multi-view images, we also evaluate the image quality by comparing the similarity with the ground truth images. The results shown in Tab.~\ref{tab:view16_free_compare}, Tab.~\ref{tab:view16_fxied_compare}, and Tab.~\ref{tab:view32_free_compare} indicate that our method also enhances the image quality under different settings besides improving the consistency.
Moreover, our method shows better image quality compared with SyncDreamer even in the 16-view setting with fixed camera pose.

\noindent\textbf{Input Consistency.}
Input consistency terms whether the results align with the input image.
Fig.~\ref{fig:sota_compare} illustrates that both our method and SyncDreamer enhance multi-view consistency. However, the color and texture details of SyncDreamer's results diverge from the input image and appear visually unnatural.
This discrepancy is evident in the input consistency score presented in Tab.~\ref{tab:view16_fxied_compare}, indicating lower similarity with the condition image in the SyncDreamer results.	

\subsection{Ablation Study}
The overall quantitative results are shown in Tab.~\ref{tab:ablation}, and the qualitative comparisons are shown in Fig.~\ref{fig:ablation}.

\noindent \textbf{Full Attention \vs Epipolar Attention.}
The results presented in Tab.\ref{tab:ablation} and Fig.\ref{fig:ablation} demonstrate that our epipolar attention mechanism can synthesize more consistent multi-view images compared with full attention. Furthermore, our epipolar attention achieves a greater performance improvement compared to full attention when using multiple reference images. This could be attributed to the fact that our epipolar attention more effectively localizes target information, as depicted in Fig.~\ref{fig:full_attn_compare}, thereby reducing noise from the reference images. In the multi-view setting, where multiple reference images are utilized, this noise reduction becomes particularly crucial.
Moreover, it is noteworthy that the epipolar attention mechanism consumes less GPU memory compared to our baseline, as discussed in Sec.~\ref{sec:attn_analysis}.

\noindent \textbf{Attending Single-View \vs Multi-View.}
Applying the epipolar attention significantly improves the consistency between the input and target views. However, the consistency between different views in the unobserved regions of the input view is not well preserved.
After implementing our epipolar attention in the multi-view setting, the consistency across the generated multi-view images is further improved. The last row in Tab.~\ref{tab:ablation} shows that after applying our multi-view epipolar attention, the consistency score is further improved compared with the single-view setting. Besides, the qualitative result in Fig.~\ref{fig:ablation} also shows better consistency among different target views.



\begin{table}[t]
\centering
\vspace{-1mm}
\caption{Comparison of 3D reconstruction results. Our method significantly improves the reconstruction quality.}
\vspace{-3mm}
\label{tab:neus}
\scalebox{0.7}{
\begin{tabular}{c cc}
\toprule
              &  Chamfer Dist.$\downarrow$  & Volume IoU$\uparrow$
\\ \midrule

            Zero123         & 0.017         & 0.819    \\
            SyncDreamer     & \best{0.013}         & \best{0.847}    \\
            Ours            & 0.014	& 0.842 \\

\bottomrule
\end{tabular}
}
\vspace{-5mm}
\end{table}


\vspace{-2mm}
\subsection{Downstream Application}
\vspace{-2mm}
To demonstrate the effectiveness of our method, we also applied it to the downstream 3D reconstruction task. Specifically, we trained the NeuS model~\cite{neus} directly using images synthesized by our method, Zero123, and SyncDreamer, respectively.
The quantitative results in Tab.~\ref{tab:neus} show that the consistent multi-view images synthesized by our method can significantly improve the 3D reconstruction quality.
Additionally, our method exhibits similar performance to SyncDreamer which requires time-consuming re-training.
The qualitative results in Fig.~\ref{fig:neus} show that it is challenging to train the NeuS model directly due to the lack of consistency in the images generated by Zero123. In contrast, our method generates more consistent multi-view images and, therefore, better reconstructs the geometry and texture details.
We show improvements on other downstream applications such as image-to-3D in the Supplementary Material.


\section{Concluding Remarks}
In this paper, we proposed a novel approach utilizing multimodal LLMs to generate gesture-aware speech recognition transcripts for patients with language disorders. Our framework integrates verbal speech and iconic gestures, enabling the generation of enriched transcripts that capture the latent meaning conveyed through both modalities. Through extensive experimentation, we demonstrated that the proposed method effectively contextualizes incomplete or disfluent speech by incorporating gesture information, leading to more accurate and meaningful representations of the speaker's intent. These findings highlight the potential of our approach to significantly contribute to the field of speech and language therapy, offering innovative tools that can enhance the quality of life for individuals with language disorders by facilitating better communication and assessment methods.

\subsection{Ethical Statement} 
Our dataset was obtained from AphasiaBank with the approval of the Institutional Review Board (IRB) and adheres to the data sharing guidelines set by TalkBank\footnote{https://talkbank.org/share/ethics.html}. This includes complying with the Ground Rules for all TalkBank databases, which are based on the American Psychological Association Code of Ethics~\cite{american2002ethical}.

\subsection{Limitation \& Future Work} 
%This study represents a preliminary investigation into using multimodal LLMs to generate gesture-aware speech recognition transcripts. 
While the results are promising, we recognize several limitations and outline our plans to extend this work further.

One primary limitation is the absence of a definitive ground truth for quantitative evaluation. Since our model generates transcripts by synthesizing speech and gesture data from scratch, traditional benchmarks, such as comparisons with standard speech recognition outputs, are insufficient. Moreover, existing original transcripts lack gesture annotations, making direct comparisons challenging. In future work, we aim to address this gap by collaborating with certified pathologists to conduct qualitative assessments, such as A-B preference tests, to evaluate the effectiveness of gesture-enriched transcripts in accurately conveying the speaker's intentions.

To support quantitative evaluations, we plan to develop novel metrics that assess transcript quality, including grammar accuracy, semantic consistency, and the integration of multimodal information. Such metrics will provide a more objective basis for assessing our model's performance and facilitate comparisons with other multimodal and unimodal approaches.

Another limitation of this study is its focus on structured gestures from a specific task, the Peanut Butter Sandwich Task. While this task offers a controlled context for testing our approach, it does not encompass the diversity of gestures and communication patterns seen in everyday scenarios. As part of our future work, we plan to expand the scope of our model to include tasks such as the Cinderella Story Recall Task~\cite{bird1996cinderella}, which involves unstructured and complex narrative gestures. This expansion will allow us to evaluate the adaptability and robustness of our model in handling varied linguistic and gestural contexts.

In summary, while this study establishes a strong foundation for gesture-aware speech recognition, we aim to refine and extend our methods through collaborative qualitative evaluations, the development of robust quantitative metrics, and broader task applications. These efforts will ensure that our approach continues to evolve, ultimately contributing to more effective communication tools and interventions for individuals with language disorders.





\section*{Acknowledgment}

The authors would like to thank Jonathan Dong for discussions and helpful feedback.

\bibliographystyle{IEEEtran}
\bibliography{main}

\documentclass[journal,compsoc]{IEEEtran}
\usepackage{epsfig}
\usepackage{graphicx}
\usepackage{amsmath}
\usepackage{amssymb}
\usepackage{algorithm, algorithmic}

\usepackage{diagbox}
\usepackage{float}
\usepackage{afterpage}
\usepackage{bm}
\usepackage{subfig}

%\usepackage{tabu}
\usepackage{multirow}
\usepackage{color}
\usepackage{tablefootnote}
\usepackage{adjustbox}
\usepackage{wrapfig}

\usepackage{hyperref}       % hyperlinks
\usepackage{url}            % simple URL typesetting
\usepackage{booktabs}       % professional-quality tables
\usepackage{amsfonts}       % blackboard math symbols
\usepackage{nicefrac}       % compact symbols for 1/2, etc.
\usepackage{microtype}      % microtypography
\usepackage{times}
\usepackage{epsfig}
%\usepackage{tabu}
%\usepackage{overpic}
\usepackage{bbding}
\usepackage{etoolbox}
\usepackage{paralist}
\usepackage{ulem}
\usepackage{tikz}

\usepackage{makecell}

\usepackage{xcolor,colortbl}

% \usepackage[pagebackref=true,breaklinks=true,colorlinks,bookmarks=false]{hyperref}


\newcolumntype{Y}{p{0.5cm}<{\centering}}
\newcommand{\mc}[2]{\multicolumn{#1}{c}{#2}}
\definecolor{Gray}{gray}{0.5}
\definecolor{LightCyan}{rgb}{0.88,1,1}

\newcolumntype{a}{>{\columncolor{Gray}}c}
\newcolumntype{b}{>{\columncolor{white}}c}



\DeclareMathOperator*{\cat}{Cat}


\def\H{\operatorname{H}}
\def\I{\operatorname{I}}
\def\KL{\operatorname{KL}}


\def\etal{\textit{et al}.}
\def\ie{\textit{i.e.}}
\def\eg{\textit{e.g.}}
\def\etc{\textit{etc}}
\def\wrt{\textit{w.r.t. }}

\def\bz{\textcolor{blue}}
\def\xc{\textcolor{red}}
\newcommand{\tb}[1]{\textbf{#1}}
\newcommand{\bc}[1]{\textcolor[RGB]{192,0,0}{#1}}
\newcommand{\rc}[1]{\textcolor{blue}{#1}}
% \newcommand{\rb}[1]{\textcolor{teal}{#1}}
% \newcommand{\bb}[1]{\textcolor{blue}{#1}}
\newcommand{\bb}[1]{\textcolor[RGB]{192,0,0}{\textbf{#1}}}
\newcommand{\rb}[1]{\textcolor{blue}{\textbf{#1}}}
\newcommand{\todo}[1]{{\color{blue}{[TODO: #1]}}}

% \newcommand{\rev}[1]{\textcolor{red}{#1}}
\newcommand{\rev}[1]{{#1}}
\renewcommand{\thefootnote}{\fnsymbol{footnote}}


\normalem
\begin{document}

\title{Rotation-Adaptive Point Cloud Domain Generalization via Intricate Orientation Learning \\
—— Supplementary Material —— }

\author{{Bangzhen~Liu,~Chenxi~Zheng,~Xuemiao~Xu,~Cheng Xu,~Huaidong~Zhang, \\ and~Shengfeng~He,~\IEEEmembership{Senior Member,~IEEE}}

\thanks{ Bangzhen Liu,~Chenxi~Zheng, and~Xuemiao~Xu are with the School of Computer Science and Engineering, South China University of Technology, Guangzhou, China. E-mail: liubz.scut@gmail.com,~cszcx@mail.scut.edu.cn, and~xuemx@scut.edu.cn.}
\thanks{ Cheng Xu is with the Centre for Smart Health, The Hong Kong Polytechnic University, Hong Kong. E-mail: cschengxu@gmail.com}
\thanks{ Huaidong Zhang is with the School of Future Technology, South China University of Technology, Guangzhou, China. E-mail: huaidongz@scut.edu.cn.}
\thanks{ Shengfeng He is with the School of Computing and Information Systems, Singapore Management University, Singapore. E-mail: shengfenghe@smu.edu.sg.}
}

\markboth{IEEE Transactions on Pattern Analysis and Machine Intelligence}%
{Shell \MakeLowercase{\textit{Liu et al.}}: Rotation-Adaptive Point Cloud Domain Generalization via Intricate Orientation Learning}


\maketitle

\IEEEdisplaynontitleabstractindextext

\IEEEpeerreviewmaketitle


\section{More Experimental Results} \label{sec1}

It is worth noting that the three sub-datasets used in PointDA are all category-wise imbalanced, as shown in Table~\ref{table:dataset}, which indicates that the micro-average precision score (\textit{Acc.}) reported by previous studies is inappropriate to assess the generalizability of cross-domain classification. In the main paper, we instead report the results of PointDA in the form of the macro-average precision score (\textit{Avg.}) for a more convincing evaluation. We also report the extra evaluations in the form of \textit{Acc.} in Table~\ref{tab:pointda10_acc} for reference. Our method still outperforms all the competitors in the average metric over the six cross-domain tasks.

\begin{table}[h]
    \caption{{Number of samples for each category in PointDA~\cite{qin2019pointdan}.}}
    \vspace{-2ex}
    \label{table:dataset}
    \scriptsize
    % \begin{center}
    \setlength{\tabcolsep}{0.05cm}{
      \resizebox{0.48\textwidth}{!}{
        \begin{tabular}{c|c|c|c|c|c|c|c|c|c|c|c}
          \hline  & Tub & Bed & Shelf & Case & Chair & Lamp & Monit. & Plant & Sofa & Table & Total\\
          \hline 
          ModelNet & 106 & 515 & 572 & 200 & 889 & 124 & 465 & 240 & 680 & 392 & 4183 \\
          ShapeNet & 599 & 167 & 310 & 1076 & 4612 & 1620 & 762 & 158 & 2198 & 5876 & 17378 \\
          ScanNet & 98 & 329 & 464 & 650 & 2578 & 161 & 210 & 88 & 495 & 1037 & 6110 \\
          \hline
          \end{tabular}
      }
     }
    % \end{center}
    \vspace{-2mm}
  \end{table}
  

\noindent\textbf{Evaluation on Aligned Dataset.} {We additionally implement our method under the traditional aligned data scenario, where the rotation only happens on the z-axis. In this case, we adapt our intricate orientation mining approach to specifically identify the most intricate orientations along the z-axis. 
The comparisons with state-of-the-art 3DDG methods are shown in Table~\ref{tab:align}, where the results of competitors are directly borrowed from their papers. Our method surpasses the baselines on all six tasks, demonstrating its effectiveness. The proposed orientation-aware contrastive training enables the model to gain a more comprehensive understanding of point clouds from various challenging perspectives, thereby enhancing the generalizability of the learned features. We notice that our method is slightly inferior on M$\to$S* and S$\to$S*. Since the orientational shift is our major concern, we do not have a special design for capturing geometric information under self-occlusions. However, in this case, our method still outperforms the two 3DDG methods on three out of the six tasks, while achieving the best average accuracy. Furthermore, the experimental results also reveal the presence of rotational shifts in the aligned data scenes, demonstrating the potential of our method for solving this problem.}

\begin{table}[h] % table for OSDA setting on Office31
    \caption{Comparison of the \textit{Acc.} ($\%$) under the 3D domain generalization setting. The best records are marked in \textbf{bold}.}
    \label{tab:align} 
    \vspace{-3ex}
    \small
    \begin{center}
    \setlength{\tabcolsep}{0.1cm}{
    \resizebox{0.48\textwidth}{!}{
    \begin{tabular}{c|c|c|c|c|c|c|c}
    \hline

    Methods
    &{M$\to$S} & {M$\to$S*} & {S$\to$M} & {S$\to$S*} & {S*$\to$M} & {S*$\to$S} & {Avg}\\

    \hline
    Supervised                      &{93.9} &{78.4} &{96.2} &{78.4} &{96.2} &{93.9} &{89.5}\\
    w/o Adapt                       &{83.3} &{43.8} &{75.5} &{42.5} &{63.8} &{64.2} &{62.2}\\
    \hline
    {Metasets~\cite{huang2021metasets}} &\tb{86.0} &{52.3} &{67.3} &{42.1} &{69.8} &{69.5} &{64.5}\\
    {PDG~\cite{wei2022learning}}        &{85.6} &\tb{57.9} &{73.1} &{50.0} &{70.3} &{66.3} &{67.2}\\
    % \hline
    {Ours}                              &{83.8} &{46.0} &\tb{83.2} &{45.5} &\tb{76.4} &\tb{70.3} &\tb{67.5}\\
    \hline

    \end{tabular}

    }
    }
    \end{center}
    \vspace{-3mm}
\end{table}


\begin{table*}[h] % table for OSDA setting on Office31
    \caption{Comparison of the micro-average precision score \textit{Acc.}~($\%$) under the orientation-aware 3D domain generalization setting. The top 2 records are marked in \bc{red} and \rc{blue}.}
    \label{tab:pointda10_acc} 
    \vspace{-3ex}
    \small
    \begin{center}
    \setlength{\tabcolsep}{0.35cm}{
    \resizebox{1\textwidth}{!}{
    \begin{tabular}{c|c|c|c|c|c|c|c|c}
    \hline

    {Methods}
    &Type &{M$\to$S} & {M$\to$S*} & {S$\to$M} & {S$\to$S*} & {S*$\to$M} & {S*$\to$S} & {AVG} \\

    
    % \cline{2-19}
    % \cline{14-19}  
    % \cmidrule $\pm$ r{2-13}
    % \cmidrule $\pm$ r{14-19}
    \hline
    Supervised                            &\multirow{2}{*}{-}            &{86.6 $\pm$ 6.3}    &{69.6 $\pm$ 3.2}    &{88.4 $\pm$ 14.7}   &{69.6 $\pm$ 3.2}    &{88.4 $\pm$ 14.7}  &{86.6 $\pm$ 6.3}    &{81.5}\\
    w/o Adapt                             &          &{57.9 $\pm$ 15.1}   &{28.7 $\pm$ 5.1}    &{54.1 $\pm$ 8.6}   &{28.8 $\pm$ 4.7}    &{43.0 $\pm$ 4.8}  &{42.3 $\pm$ 6.0}    &{42.5}\\
    \hline   
    VN~\cite{Deng_2021_ICCV}              &\multirow{2}{*}{RE}             &{70.5 $\pm$ 0.0} &{30.6 $\pm$ 0.0} &{66.7 $\pm$ 0.0} &{32.0 $\pm$ 0.0} &{39.4 $\pm$ 0.0} &{44.8 $\pm$ 0.0}   &{47.3} \\  
    SVN~\cite{su2022svnet}               &            &{66.8 $\pm$ 0.6} &{32.3 $\pm$ 0.4}   &{62.0 $\pm$ 0.5} &{30.0 $\pm$ 0.6}    &{38.0 $\pm$ 0.9} &{42.2 $\pm$ 1.1}   &{45.7} \\ 
    EOMP~\cite{luo2022equivariant}        &              &{61.4 $\pm$ 0.8} &{28.1 $\pm$ 0.3}    &{60.5 $\pm$ 0.6} &{37.0 $\pm$ 0.7}  &{27.9 $\pm$ 0.8} &{37.2 $\pm$ 0.9}    &{42.0} \\
    
    \hline
    SPRIN~\cite{you2021prin}              &\multirow{5}{*}{RI}             &{68.2 $\pm$ 0.4} &{30.1 $\pm$ 0.6}   &{71.8 $\pm$ 0.6} &{30.4 $\pm$ 0.6}    &{46.8 $\pm$ 0.6} &{49.3 $\pm$ 0.5}  &{49.4}\\
    RIPCA~\cite{li2021closer}              &            &{70.3 $\pm$ 1.2} &{33.0 $\pm$ 0.7}   &{70.4 $\pm$ 0.9} &{39.1 $\pm$ 1.3}    &\rc{49.9 $\pm$ 1.6} &{50.6 $\pm$ 2.2}  &\rc{52.2}\\
    RIConv++~\cite{zhang2022riconv}        &              &{28.8 $\pm$ 0.6} &{14.2 $\pm$ 0.5}   &{55.1 $\pm$ 0.7} &{38.9 $\pm$ 0.5}    &{34.8 $\pm$ 0.7} &{47.3 $\pm$ 0.5}  &{36.5}\\
    PaRI~\cite{chen2022devil}              &             &{36.1 $\pm$ 0.0} &{29.3 $\pm$ 0.3}   &{51.8 $\pm$ 0.8} &\bc{44.8 $\pm$ 0.4}    &{43.3 $\pm$ 0.9} &{49.4 $\pm$ 0.1} &{42.5} \\
    LocoTrans~\cite{chen2024local}              &                  &\bc{76.7 $\pm$ 0.0}   &{34.5 $\pm$ 0.3}    &\rc{74.3 $\pm$ 0.4}  &\rc{43.6 $\pm$ 0.2}  &{41.6 $\pm$ 0.6} &{41.5 $\pm$ 0.0}    &{52.0}\\
    \hline
    PointDAN~\cite{qin2019pointdan}       &\multirow{5}{*}{DA}              &{59.8 $\pm$ 15.1}   &{29.5 $\pm$ 4.0}    &{55.2 $\pm$ 6.9}   &{24.0 $\pm$ 4.8}    &{38.0 $\pm$ 4.8}  &{47.4 $\pm$ 6.1}    &{42.3}\\
    DefRec~\cite{achituve2021self}        &              &{57.2 $\pm$ 13.3}   &{33.1 $\pm$ 4.6}    &{54.4 $\pm$ 8.0}    &{33.1 $\pm$ 4.4}   &{38.8 $\pm$ 6.5}  &{48.2 $\pm$ 5.7}    &{44.1}\\
    GAST~\cite{zou2021geometry}           &              &{27.7 $\pm$ 4.2}   &{7.0 $\pm$ 0.6}     &{40.8 $\pm$ 2.6}   &{5.8 $\pm$ 0.8}     &{30.7 $\pm$ 1.5}  &{50.7 $\pm$ 3.6}     &{27.1}\\
    MLSP~\cite{liang2022point}            &              &{66.5 $\pm$ 15.5}   &{32.8 $\pm$ 4.3}    &{59.7 $\pm$ 5.1}   &{30.0 $\pm$ 6.4}    &{46.3 $\pm$ 5.0}  &{52.2 $\pm$ 5.8}     &{47.9}\\
    SDDA~\cite{cardace2023self}           &              &{65.0 $\pm$ 14.5}  &\rc{37.8 $\pm$ 3.4}     &{61.4 $\pm$ 5.4}    &{40.1 $\pm$ 4.1}   &{40.7 $\pm$ 6.3}  &\rc{53.3 $\pm$ 6.4}     &{49.7}\\
    PCFEA~\cite{wang2024progressive}     &                   &{62.0 $\pm$ 13.6}   &{9.3 $\pm$ 0.2}   &{42.7 $\pm$ 8.7}   &{43.1 $\pm$ 4.0}   &{47.1 $\pm$ 4.0}   &\bc{54.0 $\pm$ 4.6}    &{43.0}    \\
    \hline
    {Metasets~\cite{huang2021metasets}}   &\multirow{3}{*}{DG}               &{53.9 $\pm$ 1.4} &\bc{40.3 $\pm$ 0.9}   &{32.2 $\pm$ 12.3}  &{33.5 $\pm$ 1.7} &{24.5 $\pm$ 4.6}    &{39.8 $\pm$ 10.0} &{37.4}\\
    {PDG~\cite{wei2022learning}}          &             &{25.4 $\pm$ 29.5}   &{21.2 $\pm$ 18.0}   &{38.4 $\pm$ 18.5}   &{8.1 $\pm$ 3.2}   &{30.3 $\pm$ 4.8}   &{29.7 $\pm$ 12.0}    &{25.5}\\
    % \hline
    {Ours}                                &              &\rb{70.8 $\pm$ 2.0}   &{37.2 $\pm$ 1.2}    &\bb{80.7 $\pm$ 0.6}   &{34.0 $\pm$ 1.0}     &\bb{50.0 $\pm$ 2.5}   &{47.1 $\pm$ 3.2}     &\bb{53.3}  \\
    \hline

    \end{tabular}

    }
    }
    \end{center}
    \vspace{-3mm}
\end{table*}


\noindent\textbf{Analysis of Hyper-parameter Sensitivity.} 
We evaluate the effects of varying $\lambda_{oc}$ and $\lambda_{ms}$, by changing the value while keeping the other frozen as 0.1. As Fig.~\ref{fig:ablation}(a) and Fig.~\ref{fig:ablation}(b) show, $\lambda_{oc}$ is insensitive across a large range, while larger $\lambda_{ms}$ may slightly decrease the performance of our model. According to the variation of performance curves, we choose $\lambda_{oc}=0.01$ and $\lambda_{ms}=0.01$ as the model setting in our main paper.
\begin{figure}[h]
    % \flushleft
    \centering
    \subfloat[Ablation of $\lambda_{oc}$]{%[b]{0.45\textwidth}
        \label{fig:plot_lambda_oc}
        \includegraphics[width=0.22\textwidth]{./resources/supp/ablation_cons_weights.pdf}
    }
    % \hspace{2mm}
    \subfloat[Ablation of $\lambda_{ms}$]{%[b]{0.45\textwidth}
        \label{fig:plot_lambda_ms}
        \includegraphics[width=0.22\textwidth]{./resources/supp/ablation_reg_weights.pdf}
    }
    % \vspace{-2mm}
    \caption{The curves of performance \wrt varying $\lambda_{oc}$ and $\lambda_{ms}$.}
    \label{fig:ablation}
    % \vspace{-5.5mm}
\end{figure} 


\noindent\textbf{Analysis of Training Stability.} 
We plot the curves of the proposed orientation consistency loss and the marginal separation loss over the training stage to demonstrate the convergence of our intricate orientational learning. As Fig.~\ref{fig:plot}(a) and Fig.~\ref{fig:plot}(b) show, all the losses gradually decrease and converge to a convincing degree. The blue curves are the orientation consistency loss, which periodically bursts every 20 epochs. This is due to the update of the intricate orientation set, which gradually adapts the model to all the intricate orientations. At the end of the training stage, the amplification tends to be stable, indicating the consistency of the object towards various rotations.


\begin{figure}[h]
    % \flushleft
    \centering
    \subfloat[M$\to$S]{%[b]{0.45\textwidth}
        \label{fig:m2s_loss}
        \includegraphics[width=0.23\textwidth]{./resources/supp/m2s_loss.pdf}
    }
    % \hspace{2mm}
    \subfloat[M$\to$S*]{%[b]{0.45\textwidth}
        \label{fig:m2ss_loss}
        \includegraphics[width=0.23\textwidth]{./resources/supp/m2ss_loss.pdf}
    }
    \vspace{-2ex}
    \caption{The training curves (\ie, $L_{cls}$, $L_{oc}$, and $L_{ms}$) on M$\to$S (a) and M$\to$S* (b).}
    \label{fig:plot}
    \vspace{-3mm}
\end{figure} 



\noindent\textbf{Analysis of Time Complexity.} {We report the computational costs of training/testing one batch of data in milliseconds for different compared methods in Table~\ref{tab:complexity}. The results are obtained by accumulating the running times within a single training/testing epoch and calculating the mean value w.r.t. one batch.} Due to the process of diversifying the intricate orientation set, our method introduces extra computational costs in the training phase. Nonetheless, our method yields the best performance among these methods while achieving the second-best inferencing speed, which is more efficient than the other RE and RI methods that require extra time-consuming modules for practical applications. 


\begin{table}[h]
    \caption{Time statistics (ms) of training/testing on one batch of data.}
    \label{tab:complexity}
    \vspace{-3ex}
    \small
    \begin{center}
    \setlength{\tabcolsep}{0.3cm}{
    \resizebox{0.4\textwidth}{!}{
      \begin{tabular}{c|c|c|c|c} 
        \hline 
        Methods & Type & Avg. & $T_{train}$& $T_{test}$ \\
        \hline
        VN~\cite{Deng_2021_ICCV} & RE & 41.3 & 808 & 27.9 \\
        SPRIN~\cite{you2021prin} & RI & 43.9 & 1551 & 370.2 \\
        RIPCA~\cite{li2021closer} & RI & 46.6 & 717 & 23.3 \\
        MLSP~\cite{liang2022point} & DA & 43.2 & 825 & 36.5 \\
        SDDA~\cite{cardace2023self} & DA & 43.1 & \bf{567} & \bf{12.0} \\
        \hline
        Ours & DG & \bf{49.6} & 2114 & 14.5 \\
        \hline
        \end{tabular}
    }
    }
    \end{center}
    \vspace{-3mm}
  \end{table}
  


\section{Extra Visualizations and Analysis} \label{sec2}

\noindent\textbf{The Learned Intricate Augmented Samples.} {In Fig.~\ref{fig:intricat_angle}, we select several point clouds and provide visualizations of how their intricate orientations evolve during training. We trained the model on ModelNet and optimized the intricate set on the testing set every 20 epochs. Each row of the point cloud sequence shows the current pose of the given point cloud augmented by its corresponding intricate orientation at that specific epoch. 
Beneath each sequence, we also visualize the distribution of predicted probabilities and the consistency of prediction over different testing orientations. 
Specifically, for each point cloud, we obtain the predicted probabilities of its 64 testing variants $P = {\{P_a|P_a = \left[p^1_a, ..., p^C_a\right]\}}^A_{a=1}$, where $A=64$ is the number of testing orientation series and $C=10$ is the number of categories. 
The visualized probabilities' distribution $P_m$ is calculated by averaging the predictions over the 64 testing rotation series, such that $P_m = \left[\frac{1}{A}\sum_{j=1}^{A}p^1_j, ..., \frac{1}{A}\sum_{j=1}^{A}p^C_j\right]$. 
To evaluate the predicted consistency, we adopt the entropy as the metric and calculate the consistency $Ent_m$ over the 64 predicted probabilities by 
\begin{equation*}
  Ent_m = \left[\frac{1}{A}\sum_{j=1}^{A}p^1_j log p^1_j, ..., \frac{1}{A}\sum_{j=1}^{A}p^C_j log p^C_j\right].
\end{equation*}
As the number of training epochs increases, both the confidence and output consistency of the model are enhanced. For samples located near the decision boundaries, such as row 6 and row 9, learning with intricate orientation mining could significantly alleviate the ambiguity of learned features, thereby producing a more robust and generalizable classifier for downstream tasks.
}




\begin{figure*}[h]
    % \flushleft
    \centering
    \subfloat[Metasets]{%[b]{0.45\textwidth}
        \label{fig:cm1_m2s}
        % \centering
        \includegraphics[width=0.24\textwidth]{./resources/supp/conf_mat_modelnet2shapenet_Metaset.pdf}
        % \vspace{10mm}
    }
    \subfloat[PDG]{%[b]{0.45\textwidth}
        \label{fig:cm2_m2s}
        % \centering
        \includegraphics[width=0.24\textwidth]{./resources/supp/conf_mat_modelnet2shapenet_PDG.pdf}
        % \vspace{10mm}
    }
    \subfloat[Ours]{%[b]{0.45\textwidth}
        \label{fig:cm3_m2s}
        % \centering
        \includegraphics[width=0.24\textwidth]{./resources/supp/conf_mat_modelnet2shapenet.pdf}
        % \vspace{10mm}
    }
    \vspace{-2ex}
    \caption{The confusion matrices of Metaset, PDG, and our method on M$\to$S. Zoom in for details.}
    \vspace{-3ex}
    \label{fig:visualization_m2s}
\end{figure*} 
% \vspace{-4mm}
\begin{figure*}[h]
    % \flushleft
    \centering
    \subfloat[Metasets]{%[b]{0.45\textwidth}
        \label{fig:cm1_s2m}
        % \centering
        \includegraphics[width=0.24\textwidth]{./resources/supp/conf_mat_shapenet2modelnet_Metaset.pdf}
        % \vspace{10mm}
    }
    \subfloat[PDG]{%[b]{0.45\textwidth}
        \label{fig:cm2_s2m}
        % \centering
        \includegraphics[width=0.24\textwidth]{./resources/supp/conf_mat_shapenet2modelnet_PDG.pdf}
        % \vspace{10mm}
    }
    \subfloat[Ours]{%[b]{0.45\textwidth}
        \label{fig:cm3_s2m}
        % \centering
        \includegraphics[width=0.24\textwidth]{./resources/supp/conf_mat_shapenet2modelnet.pdf}
        % \vspace{10mm}
    }
    \vspace{-2ex}
    \caption{The confusion matrices of Metaset, PDG, and our method on S$\to$M. Zoom in for details.}
    \label{fig:visualization_s2m}
\end{figure*} 


\begin{figure*}[h]
    % \flushleft
    \centering
    \subfloat{\label{fig:0}
        \begin{minipage}[b]{1.0\textwidth}\centering
            \includegraphics[width=0.95\textwidth]{./resources/supp/underline.pdf} 
            \\
            \includegraphics[width=0.9\textwidth]{./resources/supp/all_title.pdf} 
            \\
            \includegraphics[width=0.9\textwidth]{./resources/supp/all_cat_0_cropped.pdf} 
            \\
            \vspace{-3mm}
            \includegraphics[width=1.0\textwidth]{./resources/supp/all_predictions_cat_0_cropped.pdf}
        \end{minipage}
    }\vspace{-3mm}

    \subfloat{\label{fig:1}
        \begin{minipage}[b]{1.0\textwidth}\centering
            \includegraphics[width=0.9\textwidth]{./resources/supp/all_cat_1_cropped.pdf} 
            \\
            \vspace{-3mm}
            \includegraphics[width=1.0\textwidth]{./resources/supp/all_predictions_cat_1_cropped.pdf}
        \end{minipage}
    }\vspace{-3mm}

    \subfloat{\label{fig:2}
        \begin{minipage}[b]{1.0\textwidth}\centering
            \includegraphics[width=0.9\textwidth]{./resources/supp/all_cat_2_cropped.pdf} 
            \\
            \vspace{-3mm}
            \includegraphics[width=1.0\textwidth]{./resources/supp/all_predictions_cat_2_cropped.pdf}
        \end{minipage}
    }\vspace{-3mm}

    \subfloat{\label{fig:4}
        \begin{minipage}[b]{1.0\textwidth}\centering
            \includegraphics[width=0.9\textwidth]{./resources/supp/all_cat_3_cropped.pdf} 
            \\
            \vspace{-3mm}
            \includegraphics[width=1.0\textwidth]{./resources/supp/all_predictions_cat_3_cropped.pdf}
        \end{minipage}
    }\vspace{-3mm}

    \subfloat{\label{fig:6}
        \begin{minipage}[b]{1.0\textwidth}\centering
            \includegraphics[width=0.9\textwidth]{./resources/supp/all_cat_4_cropped.pdf} 
            \\
            \vspace{-3mm}
            \includegraphics[width=1.0\textwidth]{./resources/supp/all_predictions_cat_4_cropped.pdf}
        \end{minipage}
    }\vspace{-3mm}

    \subfloat{\label{fig:7}
        \begin{minipage}[b]{1.0\textwidth}\centering
            \includegraphics[width=0.9\textwidth]{./resources/supp/all_cat_5_cropped.pdf} 
            \\
            \vspace{-3mm}
            \includegraphics[width=1.0\textwidth]{./resources/supp/all_predictions_cat_5_cropped.pdf}
        \end{minipage}
    }\vspace{-3mm}

    \subfloat{\label{fig:8}
        \begin{minipage}[b]{1.0\textwidth}\centering
            \includegraphics[width=0.9\textwidth]{./resources/supp/all_cat_6_cropped.pdf} 
            \\
            \vspace{-3mm}
            \includegraphics[width=1.0\textwidth]{./resources/supp/all_predictions_cat_6_cropped.pdf}
        \end{minipage}
    }

    \caption{Visualization of the learned intricate orientation series on ModelNet (M). Each row of point cloud sequence records the transformation of the point cloud's poses after augmention by its corresponding intricate rotational angle during the training procedure. The interval of recording is 20 epoch. The statistic measurements at every records, including the predicted probability and rotational consistency, are presented beneath each point cloud sequence. The category which the current point cloud belongs to is marked in \bc{red}.}
    \label{fig:intricat_angle}
    % \vspace{-6mm}
\end{figure*}



\noindent\textbf{Confusion Matrices.} {we provide the evaluation results for Metasets~\cite{huang2021metasets}, PDG~\cite{wei2022learning}, and our method in the form of confusion matrix on the target domain. ShapeNet is a dataset whose samples are highly imbalanced across different categories, while ModelNet is much more balanced. The confusion matrices of the three approaches are shown in  Fig.~\ref{fig:visualization_m2s} and Fig.~\ref{fig:visualization_s2m}. Compared with the other two 3D domain generalization methods, our method has more compact confusion matrices under the orientation shift. For M$\to$S, both our method and Metaset are separated relatively well while PDG has much inaccurate classification on class "Plant". The inner reason is that the part-based feature utilized by PDG may encounter confusing local expressions, such as the plane of the table and the bottom of a potted plant. For S$\to$M, our method achieves more balanced and concise results. We observe that the sample of class "monitor" is much easier to misclassify into "bed" due to the similar plane structure of their surface. Similar trends happen for the categories "table" and "cabinet", which have less discriminative features in the view of shape. }

{In summary, the single shape cannot serve as a discriminative representation in some cases. This is the limitation of shape representation under the orientation shift since there are a lot of objects whose shapes are similar but belong to different categories. In this case, extra visual (\eg, texture or color), linguistic information, or spatial cues are important to provide complement representation, which may benefit the problem of cross-domain generalization under orientation shift. We will plan to investigate the function of these features in our future work.  }

\section{Gradient of the rotation parameters} \label{sec3}
In this section, we provide detailed calculations about the optimizable parameters $\Theta$ concerning a given model $F$. Considering the objective of optimizing $\Theta$ within a standard classification task, we have the following objective:
\begin{equation}
  \hat{\Theta} = \mathop{\arg\max_{\Theta}}L(w_{opt}, \hat{P}, y), 
\end{equation}
where $w_{opt}$ is the freeze parameter of $F$, $(\hat{P}, y)$ are the augmented point cloud and label:
\begin{equation}
  \begin{split}
    ~\hat{P}&=f(\hat{\Theta}, P)  \\
    &=R_{\theta_{x}}\cdot R_{\theta_{y}}\cdot R_{\theta_{z}}\cdot P.
  \end{split}
\end{equation}
According to the chain rules, the gradient of $\hat{\Theta}$ is calculated by:
\begin{equation}
  \begin{split}
  \frac{\partial L}{\partial \hat{\Theta}} &= \frac{\partial L}{\partial \hat{P}} \frac{\partial \hat{P}}{\partial \hat{\Theta}} \\
  &=\frac{\partial L}{\partial \hat{P}}
  \left(
  \frac{\partial R_{\theta_{x}}}{\partial \theta_{x}}
  R_{\theta_{y}}
  R_{\theta_{z}} \quad
  R_{\theta_{x}}
  \frac{\partial R_{\theta_{y}}}{\partial \theta_{y}}
  R_{\theta_{z}} \quad
  R_{\theta_{x}}
  R_{\theta_{y}}
  \frac{\partial R_{\theta_{z}}}{\partial \theta_{z}}
  \right)P,
  \end{split}
\end{equation}
where 
\begin{equation}
  \begin{split}
R_{\theta_{x}} = 
\begin{pmatrix}
  1 & 0 & 0 \\
  0 & \cos\theta_{x} & -\sin\theta_{x} \\
  0 & \sin\theta_{x} & \cos\theta_{x} 
\end{pmatrix},
\\
R_{\theta_{y}} = 
\begin{pmatrix}
  \cos\theta_{y} & 0 & \sin\theta_{y} \\
  0 & 1 & 0 \\
  -\sin\theta_{y} & 0 & \cos\theta_{y}
\end{pmatrix},
\\
R_{\theta_{z}} = 
\begin{pmatrix}
  \cos\theta_{z} & -\sin\theta_{z} & 0  \\
  \sin\theta_{z} & \cos\theta_{z} & 0 \\
  0 & 0 & 0
\end{pmatrix},
\end{split}
\end{equation}
and 
\begin{equation}
    \begin{split}
\frac{\partial R_{\theta_{x}}}{\partial \theta_{x}} = 
\begin{pmatrix}
  0 & 0 & 0 \\
  0 & -\sin\theta_{x} & -\cos\theta_{x} \\
  0 & \cos\theta_{x} & -\sin\theta_{x} 
\end{pmatrix}, \\
\frac{\partial R_{\theta_{y}}}{\partial \theta_{y}} = 
\begin{pmatrix}
  -\sin\theta_{y} & 0 & \cos\theta_{y} \\
  0 & 0 & 0 \\
  -\cos\theta_{y} & 0 & -\sin\theta_{y}
\end{pmatrix}, \\
\frac{\partial R_{\theta_{z}}}{\partial \theta_{z}} = 
\begin{pmatrix}
  -\sin\theta_{z} & -\cos\theta_{z} & 0  \\
  \cos\theta_{z} & -\sin\theta_{z} & 0 \\
  0 & 0 & 1
\end{pmatrix}.
\end{split}
\end{equation}



\section{Theoretical Analysis for Rotation-Adaptive Point Cloud Domain Generalization} \label{sec4}

In this section, we provide theoretical proof demonstrating how orientational consistency functions to bridge the domain gap, analyzed from the perspective of mutual information reduction.

Let $X\!=\!(U, V)$ represent a 3D point cloud, where $U$ corresponds to orientation-dependent variables and $V$ to orientation-independent variables. In our case, we assume that the ranges of $U$ and $V$ remain consistent across domains.
For $X_s\!\sim\!p_\mathrm{src}(x)$, where $p_\mathrm{src}(x)$ denotes the source domain data distribution, the marginal distributions \wrt $U_s$ and $V_s$ are expressed by:
\begin{equation}
 p_\mathrm{src}(u)=\int p_\mathrm{src}(x) \mathrm{d}v, \quad p_\mathrm{src}(v)=\int p_\mathrm{src}(x) \mathrm{d}u.
\end{equation}
Considering the data distribution $X_a\!\sim\!p_\mathrm{aug}(x)$ after augmentation, where each sample is assumed to be uniformly sampled \wrt orientations, the marginal distributions \wrt $U_a$ and $V_a$ are given by:
\begin{equation}
 p_\mathrm{aug}(u)=\mathcal{U}(\mathcal{D}_{U_a}), \quad p_\mathrm{aug}(v)=p_\mathrm{src}(v),
\end{equation}
where $\mathcal{U}(\cdot)$ denotes a uniform distribution over the measurable domain $\mathcal{D}_{U_a}$ of ${U_a}$. For simplicity, the subscript of $U_a$ in $\mathcal{D}_{U_a}$ is omitted without causing ambiguity in the subsequent analysis. In this work, we adopt the proposed orientation-aware contrastive learning framework to approximately achieve this, where ${U_a}$ is represented by Euler angles and $\mathcal{D}_{U}:=[-\pi, \pi)^3$. 



Based on the definition of joint entropy, the entropy of $p_\mathrm{src}(x)$ can be expressed in terms of its marginal entropies \wrt $U_s$ and $V_s$, along with an additional term presenting the mutual information between these two components:
\begin{equation}
\begin{aligned}
    \H(X_s) 
 =& \H(U_s)+\H(V_s)-\I(U_s;V_s) \\
 =& \mathbb{E}_{u\sim p_\mathrm{src}(u)}[-\log p_\mathrm{src}(u)] + \mathbb{E}_{v\sim p_\mathrm{src}(v)}[-\log p_\mathrm{src}(v)] \\
    &- \mathbb{E}_{x\sim p_\mathrm{src}(x)}\log \frac{p_\mathrm{src}(x)}{p_\mathrm{src}(u)p_\mathrm{src}(v)},
\end{aligned}
\end{equation}
where $\I(U_s;V_s)$ represents the mutual information between $U_s$ and $V_s$ in $p_\mathrm{src}(x)$. 
Since $p_\mathrm{aug}(u)$ follows a uniform distribution and $U_a$ and $V_a$ of $p_\mathrm{aug}(x)$ are independent under this setting, the entropy of $p_\mathrm{aug}(x)$ is given by $\I_\mathrm{aug}(U_a;V_a)\!=\!0$, and the entropy of $p_\mathrm{aug}(u)$ corresponds to the measure of $\mathcal{D}_{U}$, denoted as $m(\mathcal{D}_{U})$. 
Thus, the entropy of $p_\mathrm{aug}(x)$ can be simplified as follows:
\begin{equation}
\begin{aligned}
    \H(X_a) &= \H(U_a)+\H(V_a) \\
    &= \log m(\mathcal{D}_{U}) + \mathbb{E}_{v\sim p_\mathrm{aug}(v)}[-\log p_\mathrm{aug}(v)],
\end{aligned}
\end{equation}
where $m(\mathcal{D}_{U})\!=\!(2\pi)^3$ in our case.

We use the KL divergence to quantify the distributional shift between the source and the target distribution.
For any $X_t\!\sim\!p_\mathrm{tgt}(x)$, where $p_\mathrm{tgt}(x)$ represents the target domain distribution, the KL divergence between $p_\mathrm{tgt}(x)$ and $p_\mathrm{src}(x)$ (or $p_\mathrm{aug}(x)$) can be computed once the cross-entropy between them is known. 
However, directly calculating the cross-entropy between $p_\mathrm{tgt}(x)$ and $p_\mathrm{src}(x)$ (or $p_\mathrm{aug}(x)$) is intractable, and it is often treated as an optimization objective to minimize. Notably, the cross-entropy between $p_\mathrm{tgt}(x)$ and $p_\mathrm{src}(x)$ (or $p_\mathrm{aug}(x)$) shares the same upper bound, as the samples $X_s$, $X_a$, and $X_t$ all share the same dimensionality:
\begin{equation}
    \sup_{p_\mathrm{src}}{\H(p_\mathrm{tgt}, p_\mathrm{src})} = \sup_{p_\mathrm{aug}}{\H(p_\mathrm{tgt}, p_\mathrm{aug})} = \log ({m(\mathcal{D}_U) \times m(\mathcal{D}_V)}).
\end{equation}
Here, $\mathcal{D}_V$ is the measurable domain of $V_s$, $V_a$, and $V_t$.
It is straightforward to prove that $\H_\mathrm{aug}(X_a) > \H_\mathrm{src}(X_s)$, as the mutual information is non-negative and entropy reaches its upper bound when the distribution is uniform.
Therefore, the relation between the upper bound of the KL divergence from $p_\mathrm{tgt}(x)$ to $p_\mathrm{src}(x)$ and from $p_\mathrm{tgt}(x)$ to $p_\mathrm{aug}(x)$ can be expressed as:
\begin{equation}
\begin{aligned}
    \sup_{p_\mathrm{src}}{\KL(p_\mathrm{tgt}||p_\mathrm{src})} &= \sup_{p_\mathrm{src}}{\H(p_\mathrm{tgt};p_\mathrm{src})} - \sup_{p_\mathrm{src}}{\H(X_\mathrm{s})} \\
    &> \sup_{p_\mathrm{aug}}{\H(p_\mathrm{tgt};p_\mathrm{aug})} - \sup_{p_\mathrm{aug}}{\H(X_\mathrm{a})} \\
    &= \sup_{p_\mathrm{aug}}{\KL(p_\mathrm{tgt}||p_\mathrm{aug})}. \label{eq:ieq}
\end{aligned}
\end{equation}
As revealed in Eq.~\ref{eq:ieq}, the upper bound of $\KL(p_\mathrm{tgt}||p_\mathrm{aug})$ is consistently lower than $\KL(p_\mathrm{tgt}||p_\mathrm{src})$, demonstrating the effectiveness of orientation invariance in reducing the domain shift under the disturbance of varying rotations. Consequently, the final upper bound of $\KL(p_\mathrm{tgt}||p_\mathrm{aug})$ is formally given as follows:
\begin{equation}
    \sup_{p_\mathrm{aug}}{\KL(p_\mathrm{tgt}||p_\mathrm{aug})} = \log m(\mathcal{D}_V) - \mathbb{E}_{v\sim p_\mathrm{aug}(v)}[-\log p(v)].
\end{equation}


\section{Limitation and Future Work} 
Although our method shows commendable advantages in handling cross-domain orientational shifts, it faces challenges with other complex types of domain shifts, such as heavy occlusions. This is because our framework does not offer an explicit design for tackling these domain shifts. Addressing this limitation, possibly through constructing a more powerful and versatile feature space resilient to multiple domain shifts via self-supervised pre-training, is a goal for future work.

{\small
\bibliographystyle{ieee_fullname}
\bibliography{egbib}
}

\end{document} 


\end{document}


