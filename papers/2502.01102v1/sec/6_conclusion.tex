\section{Conclusion}

\noindent In this work, we address the robustness and generalizability of lensless imaging with a modular reconstruction approach, comprised of a pre-processor, camera inversion, a post-processor, and PSF correction (see \cref{fig:pipeline}).
We theoretically show the need for each component due to inevitable model mismatch in lensless imaging systems,
and experimentally demonstrate the benefit of the pre-processor across multiple imaging systems, various reconstruction approaches, and different SNRs.
We also perform the first generalizability study across lensless imaging systems, 
and demonstrate techniques to improve generalizability.
To this end, our modular reconstruction approach allows learnings from one system to transfer to a new one,
in particular the pre-processor component.
This has very practical implications as it can remove the need of collecting a lensed-lensless dataset for a new system,
which is time-consuming and/or may not be possible depending on the application.
Our investigation prioritizes \textit{accessibility} and \textit{reproducibility}.
We open-source datasets collected with inexpensive components: a Raspberry Pi sensor, double-sided tape as a phase mask, and an LCD for our programmable-mask based system -- \textit{DigiCam}.
We also release our measurement software, and reconstruction implementations for the baselines and our modular approach.
Our methods demonstrate improved performance on our low-cost systems and more expensive ones (\textit{DiffuserCam}~\cite{Monakhova:19}).
As future work, we will explore applications of generalizable lensless imaging that would benefit from models that do not need to be retrained with measurements of new PSFs.
For further performance improvements, while our study used a convolutional forward model and the DRUNet architecture~\cite{zhang2021plug} for modular components,
a non-LSI forward model~\cite{Yanny:22,cai2024phocolens} and
different architectures, \eg transformers~\cite{Zamir2021Restormer,Pan:22} or diffusion models~\cite{cai2024phocolens}, can be explored within our modular framework.


