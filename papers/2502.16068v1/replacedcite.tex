\section{Related Work}
\nosection{Multi-Modal Recommendation}
%
Multi-Modal recommendation is set to involve multi-modal information (e.g., the item image and text descriptions) for user-item modelling ____.
%
Conventional work ____ directly fused the visual/style contents with their ID embeddings with Bayesian personalized ranking ____ for collaborative filtering.
%
Nowadays, with the fast development of graph neural network ____, more researchers started to utilize the message passing mechanism among knowledge aggregation across different modalities ____.
%
For instance, \textbf{LATTICE} ____ first constructs an pair-wise item-item relation graph within each modality and then aggregates them to form a global item-item graph.
%
Recently some work ____ also adopt contrastive learning strategy ____ with attention mechanism to further boost the model performance.
%
Latest work \textbf{LGM3Rec} ____ has even adopted Gumbel-Softmax with hypergraph relationships to enhance the model's performance by capturing group-wise high-order semantic information.
%
However, current multi-modal recommendation models fail to extract or cluster item with similar characteristics in the group-wise perspectives, which is vital for user-item modelling.  






\nosection{Cross-Domain Recommendation}
%
Cross Domain Recommendation (CDR) models are set to tackle the data sparsity problem by leveraging useful knowledge across domains ____.
%
Traditional CDR models mainly rely on the overlapped users to realize the knowledge transfer ____.
%
For instance, \textbf{CoNet} ____ and \textbf{ACDN} ____ adopted the cross-connection unit with attention mechanism in the deep neural network for snitching the message from different domains.
%
However these approaches cannot handle the general case when few users are overlapped ____.
%
Recently some work ____ start to investigate that scenario with domain adaptation strategies including adversarial training ____ or distribution co-clustering ____.
%
Meanwhile, only a few papers focus on addressing the multi-modal cross-domain recommendation problem, as multi-modal information amplifies domain discrepancies, making the task even more challenging.
%
Latest, \textbf{MOTKD} ____ is the first to attempt solving the MMCDR problem using an optimal transport approach ____.
%
Nonetheless, the above methods typically separate domain adaptation for overlapped and non-overlapped users, neglecting the matching guidance provided by the overlapped users.
%
Therefore, they may lead to negative transfer, resulting in suboptimal solutions that hinder model performance.


 

\begin{figure*}
\centering
\includegraphics[width=0.95\linewidth]{pic/wwwuuu_6.pdf}
% \vspace{-0.7cm} 
\caption{The model framework of proposed \modelname~for solving MMCDR problem.}
% \xenia{indicate group, A is used for hypergraph, scalar $\alpha$}}
% \vspace{-0.5cm} 
\label{fig:sig}
\end{figure*}