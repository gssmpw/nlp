%%%%%%%% ICML 2025 EXAMPLE LATEX SUBMISSION FILE %%%%%%%%%%%%%%%%%
\PassOptionsToPackage{prologue,dvipsnames}{xcolor}
\documentclass[dvipsnames]{article}

% Recommended, but optional, packages for figures and better typesetting:
% \usepackage{microtype}
% \usepackage{graphicx}
% \usepackage{subfigure}
% \usepackage{booktabs} % for professional tables

% hyperref makes hyperlinks in the resulting PDF.
% If your build breaks (sometimes temporarily if a hyperlink spans a page)
% please comment out the following usepackage line and replace
% \usepackage{icml2025} with \usepackage[nohyperref]{icml2025} above.
\usepackage{hyperref}


% Attempt to make hyperref and algorithmic work together better:
\newcommand{\theHalgorithm}{\arabic{algorithm}}

% Use the following line for the initial blind version submitted for review:
% \usepackage{icml2025}

% If accepted, instead use the following line for the camera-ready submission:
\usepackage[accepted]{icml2025}

% For theorems and such
\usepackage{amsmath}
\usepackage{amssymb}
\usepackage{mathtools}
\usepackage{amsthm}

% if you use cleveref..
\usepackage[capitalize,noabbrev]{cleveref}


%------- My packages and commands
% basic
%\usepackage{color,xcolor}
\usepackage{color}
\usepackage{epsfig}
\usepackage{graphicx}
\usepackage{algorithm,algorithmic}
% \usepackage{algpseudocode}
%\usepackage{ulem}

% figure and table
\usepackage{adjustbox}
\usepackage{array}
\usepackage{booktabs}
\usepackage{colortbl}
\usepackage{float,wrapfig}
\usepackage{framed}
\usepackage{hhline}
\usepackage{multirow}
% \usepackage{subcaption} % issues a warning with CVPR/ICCV format
% \usepackage[font=small]{caption}
\usepackage[percent]{overpic}
%\usepackage{tikz} % conflict with ECCV format

% font and character
\usepackage{amsmath,amsfonts,amssymb}
% \let\proof\relax      % for ECCV llncs class
% \let\endproof\relax   % for ECCV llncs class
\usepackage{amsthm} 
\usepackage{bm}
\usepackage{nicefrac}
\usepackage{microtype}
\usepackage{contour}
\usepackage{courier}
%\usepackage{palatino}
%\usepackage{times}

% layout
\usepackage{changepage}
\usepackage{extramarks}
\usepackage{fancyhdr}
\usepackage{lastpage}
\usepackage{setspace}
\usepackage{soul}
\usepackage{xspace}
\usepackage{cuted}
\usepackage{fancybox}
\usepackage{afterpage}
%\usepackage{enumitem} % conflict with IEEE format
%\usepackage{titlesec} % conflict with ECCV format

% ref
% commenting these two out for this submission so it looks the same as RSS example
% \usepackage[breaklinks=true,colorlinks,backref=True]{hyperref}
% \hypersetup{colorlinks,linkcolor={black},citecolor={MSBlue},urlcolor={magenta}}
\usepackage{url}
\usepackage{quoting}
\usepackage{epigraph}

% misc
\usepackage{enumerate}
\usepackage{paralist,tabularx}
\usepackage{comment}
\usepackage{pdfpages}
% \usepackage[draft]{todonotes} % conflict with CVPR/ICCV/ECCV format



% \usepackage{todonotes}
% \usepackage{caption}
% \usepackage{subcaption}

\usepackage{pifont}% http://ctan.org/pkg/pifont

% extra symbols
\usepackage{MnSymbol}


%%%%%%%%%%%---SETME-----%%%%%%%%%%%%%
%replace @@ with the submission number submission site.
\newcommand{\thiswork}{INF$^2$\xspace}
%%%%%%%%%%%%%%%%%%%%%%%%%%%%%%%%%%%%


%\newcommand{\rev}[1]{{\color{olivegreen}#1}}
\newcommand{\rev}[1]{{#1}}


\newcommand{\JL}[1]{{\color{cyan}[\textbf{\sc JLee}: \textit{#1}]}}
\newcommand{\JW}[1]{{\color{orange}[\textbf{\sc JJung}: \textit{#1}]}}
\newcommand{\JY}[1]{{\color{blue(ncs)}[\textbf{\sc JSong}: \textit{#1}]}}
\newcommand{\HS}[1]{{\color{magenta}[\textbf{\sc HJang}: \textit{#1}]}}
\newcommand{\CS}[1]{{\color{navy}[\textbf{\sc CShin}: \textit{#1}]}}
\newcommand{\SN}[1]{{\color{olive}[\textbf{\sc SNoh}: \textit{#1}]}}

%\def\final{}   % uncomment this for the submission version
\ifdefined\final
\renewcommand{\JL}[1]{}
\renewcommand{\JW}[1]{}
\renewcommand{\JY}[1]{}
\renewcommand{\HS}[1]{}
\renewcommand{\CS}[1]{}
\renewcommand{\SN}[1]{}
\fi

%%% Notion for baseline approaches %%% 
\newcommand{\baseline}{offloading-based batched inference\xspace}
\newcommand{\Baseline}{Offloading-based batched inference\xspace}


\newcommand{\ans}{attention-near storage\xspace}
\newcommand{\Ans}{Attention-near storage\xspace}
\newcommand{\ANS}{Attention-Near Storage\xspace}

\newcommand{\wb}{delayed KV cache writeback\xspace}
\newcommand{\Wb}{Delayed KV cache writeback\xspace}
\newcommand{\WB}{Delayed KV Cache Writeback\xspace}

\newcommand{\xcache}{X-cache\xspace}
\newcommand{\XCACHE}{X-Cache\xspace}


%%% Notions for our methods %%%
\newcommand{\schemea}{\textbf{Expanding supported maximum sequence length with optimized performance}\xspace}
\newcommand{\Schemea}{\textbf{Expanding supported maximum sequence length with optimized performance}\xspace}

\newcommand{\schemeb}{\textbf{Optimizing the storage device performance}\xspace}
\newcommand{\Schemeb}{\textbf{Optimizing the storage device performance}\xspace}

\newcommand{\schemec}{\textbf{Orthogonally supporting Compression Techniques}\xspace}
\newcommand{\Schemec}{\textbf{Orthogonally supporting Compression Techniques}\xspace}



% Circular numbers
\usepackage{tikz}
\newcommand*\circled[1]{\tikz[baseline=(char.base)]{
            \node[shape=circle,draw,inner sep=0.4pt] (char) {#1};}}

\newcommand*\bcircled[1]{\tikz[baseline=(char.base)]{
            \node[shape=circle,draw,inner sep=0.4pt, fill=black, text=white] (char) {#1};}}
%--------------------------------

%%%%%%%%%%%%%%%%%%%%%%%%%%%%%%%%
% THEOREMS
%%%%%%%%%%%%%%%%%%%%%%%%%%%%%%%%
\theoremstyle{plain}
\newtheorem{theorem}{Theorem}[section]
\newtheorem{proposition}[theorem]{Proposition}
\newtheorem{lemma}[theorem]{Lemma}
\newtheorem{corollary}[theorem]{Corollary}
\theoremstyle{definition}
\newtheorem{definition}[theorem]{Definition}
\newtheorem{assumption}[theorem]{Assumption}
\theoremstyle{remark}
\newtheorem{remark}[theorem]{Remark}

% Todonotes is useful during development; simply uncomment the next line
%    and comment out the line below the next line to turn off comments
%\usepackage[disable,textsize=tiny]{todonotes}
\usepackage[textsize=tiny]{todonotes}


% The \icmltitle you define below is probably too long as a header.
% Therefore, a short form for the running title is supplied here:
\icmltitlerunning{Submission and Formatting Instructions for ICML 2025}

\begin{document}

\twocolumn[
\icmltitle{A Schema-Guided Reason-while-Retrieve framework for  \\ Reasoning on Scene Graphs with Large-Language-Models (LLMs)}

% It is OKAY to include author information, even for blind
% submissions: the style file will automatically remove it for you
% unless you've provided the [accepted] option to the icml2025
% package.

% List of affiliations: The first argument should be a (short)
% identifier you will use later to specify author affiliations
% Academic affiliations should list Department, University, City, Region, Country
% Industry affiliations should list Company, City, Region, Country

% You can specify symbols, otherwise they are numbered in order.
% Ideally, you should not use this facility. Affiliations will be numbered
% in order of appearance and this is the preferred way.
\icmlsetsymbol{equal}{*}

\begin{icmlauthorlist}
\icmlauthor{Yiye Chen}{gt}
\icmlauthor{Harpreet Sawhney}{ms}
\icmlauthor{Nicholas Gyd\'{e}}{ms}
\icmlauthor{Yanan Jian}{ms}
\icmlauthor{Jack Saunders}{ub}
\icmlauthor{Patricio Vela}{gt}
\icmlauthor{Ben Lundell}{ms}
\end{icmlauthorlist}

\icmlaffiliation{gt}{Georgia Institute of Technology}
\icmlaffiliation{ms}{Microsoft}
\icmlaffiliation{ub}{University of Bath}

\icmlcorrespondingauthor{Yiye Chen}{yychen2019@gatech.edu}
% \icmlcorrespondingauthor{Firstname2 Lastname2}{first2.last2@www.uk}

% You may provide any keywords that you
% find helpful for describing your paper; these are used to populate
% the "keywords" metadata in the PDF but will not be shown in the document
\icmlkeywords{Large Language Models, Spatial Reasoning, Scene Graphs}

\vskip 0.3in
]

% this must go after the closing bracket ] following \twocolumn[ ...

% This command actually creates the footnote in the first column
% listing the affiliations and the copyright notice.
% The command takes one argument, which is text to display at the start of the footnote.
% The \icmlEqualContribution command is standard text for equal contribution.
% Remove it (just {}) if you do not need this facility.

\printAffiliationsAndNotice{}  % leave blank if no need to mention equal contribution
% \printAffiliationsAndNotice{\icmlEqualContribution} % otherwise use the standard text.

\begin{abstract}
% Situated LLM reasoning is important, and use scene graph is one way.
    % Grounding the reasoning and planning capabilities of Large Language Models (LLMs) in specific environments remains a significant challenge. 
    % Recent advancements have demonstrated the effectiveness of representing environments as scene graphs, which offer a flexible and structured way to encode diverse semantic and spatial information.
    Scene graphs have emerged as a structured and serializable environment representation for grounded spatial reasoning with Large Language Models (LLMs).
% Existing method has drawback
    % However, existing approaches that naïvely prompt LLMs with serialized graph representations often suffer from hallucinations when processing large graphs and fail to produce graph-grounded reasoning steps for complex spatial problems.
% What we propose
    % In this work, we propose \textbf{\RwR}, an iterative scene graph reasoning framework that addresses these limitations through \textit{scene graph schema} prompting.
    % In this work, we propose \textbf{\RwR}, an iterative scene graph reasoning framework based on scene graph \textit{schema} prompting and the code-writing capacity of LLMs.
    In this work, we propose \textbf{\RwR}, a \textbf{S}chema-\textbf{G}uided \textbf{R}etrieve-\textbf{w}hile-\textbf{R}eason framework for reasoning and planning with scene graphs.
    Our approach employs two cooperative, code-writing LLM agents: a (1) \textit{Reasoner} for task planning and information queries generation, and a (2) \textit{Retriever} for extracting corresponding graph information following the queries.
    Two agents collaborate iteratively, enabling sequential reasoning and adaptive attention to graph information. 
    Unlike prior works, both agents are prompted only with the \textit{scene graph schema} rather than the full graph data, which reduces the hallucination by limiting input tokens, and drives the Reasoner to generate reasoning trace abstractly.
    % , ensuring close alignment between the reasoning and retrieval processes.
    % facilitates focused attention on task-relevant graph information and enables sequential reasoning on the graph essential for complex tasks.
    Following the trace, the Retriever programmatically query the scene graph data based on the schema understanding, allowing dynamic and global attention on the graph that enhances alignment between reasoning and retrieval. 
    % Additionally, the code-writing design allows tool-using to solve problems beyond the capacity of LLMs, which further enhance its reasoning ability facing complex tasks.
% Summarize results
    Through experiments in multiple simulation environments, we show that our framework surpasses existing LLM-based approaches in numerical Q\&A and planning tasks, and can benefit from task-level few-shot examples, even in the absence of agent-level demonstrations.
% code
    Project code will be released. %upon acceptance.
\end{abstract}

\section{Introduction}

Node classification is a fundamental task in graph analysis, with a wide range of applications such as item tagging \cite{Mao2020ItemTF}, user profiling \cite{Yan2021RelationawareHG}, and financial fraud detection \cite{Zhang2022eFraudComAE}. Developing effective algorithms for node classification is crucial, as they can significantly impact commercial success. For instance, US banks lost 6 billion USD to fraudsters in 2016. Therefore, even a marginal improvement in fraud detection accuracy could result in substantial financial savings.

Given its practical importance, node classification has been a long-standing research focus in both academia and industry. The earliest attempts to address this task adopted techniques such as Laplacian regularization \cite{belkin2006manifold}, graph embeddings \cite{yang2016revisiting}, and label propagation \cite{zhu2003semi}. Over the past decade, GNN-based methods have been developed and have quickly become prominent due to their superior performance, as demonstrated by works such as \citet{kipf2017GCN}, \citet{velickovic2018GAT}, and \citet{hamilton2017SAGE}. Additionally, the incorporation of encoded textual information has been shown to further complement GNNs' node features, enhancing their effectiveness \cite{jin2023patton, zhao2022GLEM}.

Inspired by the recent success of LLMs, there has been a surge of interest in leveraging LLMs for node classification \cite{li2023survey}. LLMs, pre-trained on extensive text corpora, possess context-aware knowledge and superior semantic comprehension, overcoming the limitations of the non-contextualized shallow embeddings used by traditional GNNs. Typically, supervised methods fall into three categories: Encoder, Reasoner, and Predictor. In the Encoder paradigm, LLMs employ their vast parameters to encode nodes' textual information, producing more expressive features that surpass shallow embeddings \cite{Zhu2024ENGINE}. The Reasoner approach utilizes LLMs' reasoning capabilities to enhance node attributes and the task descriptions with a more detailed text \cite{chen2024exploring, he2023TAPE}. This generated text augments the nodes' original information, thereby enriching their attributes. Lastly, the Predictor role involves LLMs integrating graph context through graph encoders, enabling direct text-based predictions  \cite{chen23llaga,tang2023graphgpt,chai2023graphllm,Huang2024GraphAdapter}. For zero-shot learning with LLMs, methods can be categorized into two types: Direct Inference and Graph Foundation Models (GFMs). Direct Inference involves guiding LLMs to directly perform classification tasks via crafted prompts \cite{Huang2023CanLE}. In contrast, GFMs entail pre-training on extensive graph corpora before applying the model to target graphs, thereby equipping the model with specialized graph intelligence \cite{li2024zerog}. An illustration of these methods is shown in Figure \ref{fig:llm_role}. 

Despite tremendous efforts and promising results, the design principles for LLM-based node classification algorithms remain elusive. Given the significant training and inference costs associated with LLMs, practitioners may opt to deploy these algorithms only when they provide substantial performance enhancements compared to costs. This study, therefore, seeks to identify \textbf{(1) the most suitable settings for each algorithm category, and (2) the scenarios where LLMs surpass traditional LMs such as BERT}. While recent work like GLBench \cite{Li2024GLBench} has evaluated various methods using consistent data splits in semi-supervised and zero-shot settings, differences in backbone architectures and implementation codebases still hinder fair comparisons and rigorous conclusions. To address these limitations, we introduce a new benchmark that further standardizes backbones and codebases. Additionally, we extend GLBench by incorporating three new E-Commerce datasets relevant to practical applications and expanding the evaluation settings. Specifically, we assess the impact of supervision signals (e.g., supervised, semi-supervised), different language model backbones (e.g., RoBERTa, Mistral, LLaMA, GPT-4o), and various prompt types (e.g., CoT, ToT, ReAct). These enhancements enable a more detailed and reliable analysis of LLM-based node classification methods. In summary, our contributions to the field of LLMs for graph analysis are as follows:


% A fair comparison necessitates a benchmark that evaluates all methods using consistent data splitting ratios, learning paradigms, backbone architectures, and implementation codebases. A very recent work, GLBench~\cite{Li2024GLBench}, tested various methods on several datasets in a semi-supervised/zero-shot setting, maintaining the same data splits. However, differences in the underlying backbones and implementation codebases still pose challenges for a fair comparison and drawing rigorous conclusions of the above questions. This paper introduces a benchmark that further standardizes the backbones and implementation codebases. Moreover, we expand upon GLBench by providing additional datasets and evaluation settings. Specifically, we include three new datasets from the E-Commerce sector, which are more relevant for practical commercial applications. We also assess the influence of supervision signals (e.g., supervised or semi-supervised), various language model backbones (e.g., RoBERTa, Mistral, GPT-4o), and prompts (e.g., CoT, ToT, and ReAct). These datasets and settings enable a detailed analysis of the aforementioned questions. 



% However, existing works lack the necessary standardization for such comparisons. An algorithm that performs exceptionally well in its original paper might underperform when used as a baseline in subsequent studies. This discrepancy often arises from variations in data splitting, learning paradigms, backbone architectures, and implementation codebases.  The backbone architecture and implementations are adopted from the original papers, which 

% To address this issue, this paper introduces a testbed for LLM-based node classification algorithms and conducts extensive experiments to derive insights and guidelines. 

\begin{itemize}
    \item \textbf{A Testbed:} We release LLMNodeBed, a PyG-based testbed designed to facilitate reproducible and rigorous research in LLM-based node classification algorithms. The initial release includes ten datasets, eight LLM-based algorithms, and three learning configurations. LLMNodeBed allows for easy addition of new algorithms or datasets, and a single command to run all experiments, and to automatically generate all tables included in this work.
    
    \item \textbf{Comprehensive Experiments:} By training and evaluating over 2,200 models, we analyzed how the learning paradigm, homophily, language model type and size, and prompt design impact the performance of each algorithm category.
    
    \item \textbf{Insights and Tips:} Detailed experiments were conducted to analyze each influencing factor. We identified the settings where each algorithm category performs best and the key components for achieving this performance. Our work provides intuitive explanations, practical tips, and insights about the strengths and limitations of each algorithm category.
\end{itemize}




%It has been a research focus in both academia and industry due to its wide range of applications, including item tagging \cite{Mao2020ItemTF}, user profiling \cite{Yan2021RelationawareHG}, and financial fraud detection \cite{Zhang2022eFraudComAE}. 


%Building effective algorithms for node classification is a long-standing topic as it has a direct impact on commercial success \cite{Lo2022InspectionLSG}.

%Before the popularity of LLMs, node classification is typically tackled by graph neural networks (GNNs) or language models (LMs) such as BERT \cite{Devlin2019BERTPO}. GNNs \cite{kipf2017GCN,velickovic2018GAT,hamilton2017SAGE} enhance node representations by aggregating information from neighboring nodes, thereby capturing the structural context essential for accurate classification. In contrast, LMs \cite{Wang2022e5-large, Liu2019roberta} focus on semantic representations by encoding the textual information associated with each node, transforming the node classification into a text classification task. The encoded textual information can further complement GNNs' node features \cite{jin2023patton, zhao2022GLEM}. Yifei: I think the current intro is too long, to move it to related works

%Over the past decade, we have witnessed great progress in node classification algorithms. The classical ones include Graph Neural Networks (GNNs) \cite{kipf2017GCN,velickovic2018GAT,hamilton2017SAGE} and additional language modeling to enhance the node features \cite{jin2023patton, zhao2022GLEM}. Recently, there has been a surge of interest in applying LLMs for node classification \cite{li2023survey}. In these studies, the roles performed by LLMs can be primarily 


% Despite the importance of this area, the literature of LLM-based node classification is scattered: the algorithms are evaluated under different datasets, learning paradigms, baselines, and implementation codebases. The purpose of this work is to perform rigorous comparisons among algorithms, as well as to open-source our software for anyone to replicate and extend our analysis. This manuscript investigates the question: \emph{How useful are LLMs for node classification under a fair setting?}

% To answer this question, we implement and tune eight LLM-based node classification algorithms, to compare them across ten datasets and three learning paradigms.  There are four major takeaways from our investigations: (1) \textbf{LLM-as-Encoder is effective for low-homophily graphs:} These methods outperform classic LM counterparts on low-homophily graphs, with the advantages being more obvious under limited supervision.
% (2) \textbf{LLM-as-Reasoner is the most effective when LLMs have prior knowledge of the target graph:} These methods achieve superior performance on datasets where the LLMs possess prior knowledge like academic and web link datasets, and benefit from more powerful models like GPT-4o. 
% (3) \textbf{LLM-as-Predictor methods is highly effective when labeled data is abundant}: Predictor methods require extensive supervision for model training, with their performance improving as larger LLMs adhering to scaling laws \cite{Kaplan2020ScalingLF} are utilized. Among different LLMs, Mistral-7B \cite{Jiang2023Mistral7B} consistently serves as a robust backbone. (4) \textbf{Zero-shot methods are most effective when neighbor information is injected:} Although Graph Foundation Models (GFMs) \cite{liu2023one, li2024zerog, Zhu2024GraphCLIPET} outperform open-source LLMs in zero-shot settings, they still lag behind advanced models like GPT-4o. The most effective zero-shot approaches involve injecting neighbor information to guide LLMs for direct inference.

% As a result of this paper, we release LLMNodeBed, a PyTorch-based testbed designed to facilitate reproducible and rigorous research in node classification algorithms. The initial release includes ten datasets, eight algorithms, three learning configurations, and the infrastructure to run all experiments. Our experimental framework can be easily extended to include new methods and datasets. We are committed to updating this repository with new algorithms and datasets and welcome pull requests from fellow researchers to ensure its ongoing development.


%While a myriad of algorithms exists, diverse datasets, architectures, learning configurations, and implementation codebases, rendering fair and realistic comparisons difficult and conclusions inconsistent. Inspired by standardized benchmarks in computer vision like ImageNet, this paper conducts a rigorous comparison of various LLM-based node classification methods to assess the true efficacy of LLMs. This investigation addresses the following research question:

%\textit{Under What Circumstances do LLMs Help Node Classification Task?}

%At a first step, we implement LLMNodeBed, a codebase and testbed for node classification with LLMs. It includes ten multi-domain graph datasets with varying scales and levels of homophily, supports eight representative algorithms that represent diverse LLM roles, and offers three learning configurations: semi-supervised, fully-supervised, and zero-shot. Through extensive experiments, we provide empirical insights into when LLMs contribute to node classification performance: 



% In summary, we make the following contributions: 

% \begin{enumerate}
%     \item \textbf{LLMNodeBed:} We introduce LLMNodeBed, a comprehensive and extensible testbed for evaluating LLM-based node classification algorithms. It comprises ten datasets, eight representative algorithms, and three learning scenarios, and can easily accommodate new datasets, methods, and backbones.
%     \item \textbf{Comprehensive Evaluation:} We conduct extensive empirical analysis across different datasets, algorithms, and learning settings to elucidate the efficacy of different LLM roles in node classification performance. 
%     \item \textbf{Practical Guidelines:} Based on our findings, we provide actionable guidelines for effectively applying LLMs to diverse real-world node classification tasks, enhancing their performance and applicability in various scenarios.
% \end{enumerate}

\section{Stochastic STL verification}\label{sec:main}
Our strategy to solve Problem \ref{prob: verification} is to convert the probabilistic STL satisfaction problem into a deterministic one with a tighter STL formula, as described in Section~\ref{sec: stl erosion} and Figure~\ref{fig: method}. Combining this strategy and a sharp bound on the deviation of a stochastic trajectory from its deterministic counterpart, we address Problem \ref{prob: verification} in Section~\ref{sec: probabilistic bound}.

\subsection{STL Erosion}
\label{sec: stl erosion}

\begin{figure}
\centering
\includegraphics[width =0.9\linewidth]{figures/method.pdf}
\caption{An illustration of the STL erosion method. Here we consider a simple STL specification: $\varphi:=\square_{[t_1,t_2]}\pi$, where the super level set of the predicate $\pi$ is $\CC$. This STL specification requires the state to remain inside a safe set $\CC$ from $t_1$ to $t_2$. The predicate $\pi$ is eroded by $\tilde{E}$. If the deterministic trajectory can satisfy the tightened STL formula $\tilde \varphi = \square_{[t_1,t_2]}\tilde\pi$, then the stochastic trajectory would satisfy $\varphi$ with a high probability.}
\label{fig: method}
\end{figure}

The deterministic system~\eqref{eq: deterministic dynamics} can be viewed as a noise-free version of the stochastic system~\eqref{eq: stochastic dynamics}. We refer to a deterministic trajectory of \eqref{eq: deterministic dynamics} and a stochastic trajectory of \eqref{eq: stochastic dynamics} from the same initial state and subjected to the same sequence of $d_t$ as \textit{associated} trajectories. Our strategy relies on the intuition that the trajectory of \eqref{eq: stochastic dynamics} should fluctuate around and remain close to its associated trajectory of \eqref{eq: deterministic dynamics} with high probability. 


Building on the set erosion strategy~\cite{liu2024safety}, we introduce the \textit{predicate erosion strategy}. 
For a predicate $\mu(\cdot)$, let $\CC$ be its superlevel set, \ie, $\CC=\setb{x\in \real^n:\mu(x)\geq 0}$. We shrink $\CC$ based on the amount of fluctuation to get a subset $\tilde{\CC}\subset\CC$. If for the deterministic state $x_t$, $x_t \in \tilde{\CC}$, then for the stochastic state $X_t$, $X_t\in \CC$ holds with a high probability since the fluctuation of the stochastic trajectory probably not exceed the margin $\CC \backslash \tilde{\CC}$. Therefore, we can verify whether the stochastic trajectory satisfies $\varphi$ with a high probability by verifying whether the deterministic trajectory satisfies $\varphi$ with the eroded predicates. 

To quantify the fluctuations of a stochastic trajectory around its deterministic counterpart, we formalize the concepts of stochastic fluctuation and stochastic deviation.

\begin{definition}
    Define $e_t = X_t - x_t$ as the stochastic fluctuation of the stochastic trajectory around the associated deterministic trajectory at time step $t$. Define $\|e_t\| = \| X_t-x_t\|$ as the stochastic deviation.
\end{definition}

Analogous to the reachability analysis of the system state, we introduce the probabilistic reachable set of the stochastic fluctuation. 

\begin{definition}[PRS] 
    Let $e_t$ be the stochastic fluctuation at time step $t$.
    % t \in \n_{[0, T]}$. 
    A set $E_{\theta, t}$ is called a Probabilistic Reachable Set (PRS) of $e_t$ at probability level $1-\theta$, $\theta \in (0,1)$, if for any $x_0=X_0\in \XX_0$ and any $d_s \in \DD, 0\le s\le t$, it holds that
    \begin{equation}
        \bP(e_t\in E_{\theta, t})\geq 1-\theta.
    \end{equation}
\end{definition}

Note that the PRS of $e_t$ is not unique. If $E_{\theta, t}$ is a PRS, then $E'_{\theta, t}$ is also a PRS if $E_{\theta, t}\subseteq E'_{\theta, t}$ We say $E_{\theta, t}$ is tighter than $E'_{\theta, t}$ if $E'_{\theta, t} \subseteq E_{\theta, t}$. We present a tight PRS in \ref{sec: probabilistic bound}.

For ease of analysis, we introduce the following sets:
\begin{equation} \label{eq: E_theta}
    \tilde{E}_\theta = \bigcup_{t=0}^{T} E_{\theta, t},
\end{equation}
\begin{equation} \label{eq: E_theta T}
    \boldsymbol{\tilde E}_{\theta} = \underbrace{\tilde{E}_\theta \times \tilde{E}_\theta \times \dots \times \tilde{E}_\theta}_{T \text{ times}}.
\end{equation}
At any time step \( t \), the stochastic fluctuation \( e_t \) belongs to the set \( \tilde{E}_\theta \) with probability at least \( 1 - \theta \). By union bound, the probability that \( e_t \) belongs to \( \tilde{E}_\theta \) for all time steps is
\begin{equation}
   \mathbb{P}(e_t \in \tilde{E}_\theta, \forall t \in \mathbb{N}_{[0,T]}) = \mathbb{P}(\boldsymbol{e}_{[0,T]} \in \boldsymbol{\tilde E}_{\theta}) \geq 1 - T\theta
\end{equation}

Using the PRS, we can tighten the predicates to eliminate the probabilistic term in \eqref{eq: probability constraint}.

\begin{proposition}[Predicate erosion]
    \label{prop: predicate erosion}
    Assume $E_{\theta, t}$ is a PRS of $e_t$. Let $\tilde{E}_\theta$ be as defined in \eqref{eq: E_theta}. Consider a predicate $\pi = (\mu(\cdot)\geq 0)$ with superlevel set $\CC$. Let $\tilde \pi$ be a tighter version of $\pi$, \st $x_t \models \tilde \pi \Leftrightarrow x_t \in \CC \ominus \tilde{E}_\theta$, then for any time step $t$, $t \in \n_{[0, T]}$, $x_t \models \tilde \pi \implies x_t + e_t \models \pi, \forall e_t \in \tilde{E}_\theta$.
\end{proposition}
\begin{proof}
    $x_t\models \tilde \pi \implies x_t \in \CC \ominus \tilde E_{\theta} \implies \forall e_t \in \tilde E_{\theta}, x_t + e_t \in (\CC \ominus \tilde E_{\theta, t})\oplus \tilde E_{\theta} \subseteq \CC$. Therefore $ x_t + e_t \models \pi, \forall e_t\in \tilde E_{\theta}$.
\end{proof}
Proposition~\ref{prop: predicate erosion} extends \cite{vlahakis2024probabilistic} which considers predicate tightening for affine predicates. 
In Proposition~\ref{prop: predicate erosion}, we erode $\CC$ with $\tilde{E}_\theta$, as verifying an STL formula may require evaluating a predicate at any timestep along a trajectory. Compared to the Set Erosion method for the safe set specification~\cite{liu2024safety}, the STL specification involves both spatial and temporal constraints, requiring us to shrink $\CC$ for all time steps using the union of the PRSs, as opposed to the time-varying safe set in~\cite{liu2024safety}. The following theorem shows that the stochastic STL satisfaction problem can be converted into a deterministic one by eroding every predicate in an STL formula.

\begin{theorem}[STL formula erosion]
\label{thm: STL erosion}
    Consider associated trajectory $\boldsymbol{X}_{[0,T]}$ and $\boldsymbol{x}_{[0,T]}$. Let $\tilde{E}_\theta$ be as defined in \eqref{eq: E_theta}, and $\boldsymbol{\tilde E}_{\theta}$ be as defined in \eqref{eq: E_theta T}. Given an STL formula $\varphi$ constructed with predicates $\pi_1, \pi_2, \dots, \pi_m$, and corresponding superlevel sets $\CC_1, \CC_2, \dots, \CC_m$. For every predicate $\pi_i,i \in \n_{[1,m]}$, we substitute $\pi_i$ with $\tilde \pi_i$, where the superlevel set of $\tilde \pi_i$ is $\CC_i \ominus \tilde{E}_\theta$ and keep all other operators unchanged to get $\tilde{\varphi}$, a eroded version of $\varphi$. Then $\boldsymbol{x}_{[0,T]} \models \tilde{\varphi} \implies \bP(\boldsymbol{X}_{[0,T]}\models \varphi) \geq 1-T\theta$
\end{theorem}
\begin{proof}
    By definition, it suffices to show $\boldsymbol{x}_{[0,T]} \models \tilde{\varphi} \implies \boldsymbol{x}_{[0,T]} +  \boldsymbol{e}_{[0,T]}\models \varphi, \forall \boldsymbol{e}_{[0,T]} \in \boldsymbol{\tilde E}_{\theta}$. Since the STL formulas are recursively defined, we prove by induction.

    \textit{Base Case}: By Proposition~\ref{prop: predicate erosion}, $\forall t \in \n_{[0,T]}$, $\boldsymbol{x}_{[t,T]} \models \tilde \pi_i\implies \boldsymbol{x}_{[t,T]} +  \boldsymbol{e}_{[t,T]} \models \pi_i, \forall \boldsymbol{e}_{[t,T]} \in \boldsymbol{\tilde E}_{\theta [t, T]}$. 

   \textit{Induction Hypothesis}: For any STL sub-formulas $\varphi_j$ and their eroded version $\tilde \varphi_j$, assume $\forall t \in \n_{[0,T]}$, $\boldsymbol{x}_{[t,T]} \models \tilde \varphi_j\implies \boldsymbol{x}_{[t,T]} +  \boldsymbol{e}_{[t,T]} \models \varphi_j, \forall \boldsymbol{e}_{[t,T]} \in \boldsymbol{\tilde E}_{\theta [t, T]}.$ 

   \textit{Induction Step}: We discuss logical operators and temporal operators respectively:

    \begin{enumerate}
        \item \textit{Logical operators:}  Let $\varphi' = \varphi_1 \wedge \varphi_2$ and assume $\boldsymbol{x}_{[t,T]} \models \tilde{\varphi}'$, which is equivalent to $\boldsymbol{x}_{[t,T]}\models\tilde \varphi_1 \wedge \boldsymbol{x}_{[t,T]}\models \tilde\varphi_2$. This implies $\boldsymbol{x}_{[t,T]} +  \boldsymbol{e}_{[t,T]} \models \varphi_1 \wedge \boldsymbol{x}_{[t,T]} +  \boldsymbol{e}_{[t,T]} \models \varphi_2$, which is equivalent to $\boldsymbol{x}_{[t,T]} +  \boldsymbol{e}_{[t,T]} \models \varphi'$, $\forall \boldsymbol{e}_{[t,T]} \in \boldsymbol{\tilde E}_{\theta [t, T]}$. We can similarly prove the induction step for the disjunctive operator. 
        \item \textit{Temporal Operators:} Let $\varphi' = \varphi_1\UU_{[t_1,t_2]}\varphi_2$ and assume $\boldsymbol{x}_{[t,T]} \models \tilde{\varphi}'$, which is equivalent to $\exists \tau \in \n_{[t+t_1,t+t_2]}, \boldsymbol{x}_{[\tau,T]} \models\tilde{\varphi}'_2 \wedge \forall \tau' \in \n_{[t, \tau]}, \boldsymbol{x}_{[\tau',T]} \models \tilde{\varphi}'_1$. This implies that, for the same $\tau$, $\boldsymbol{x}_{[\tau,T]} + \boldsymbol{e}_{[\tau,T]} \models {\varphi}'_2 \wedge \forall \tau' \in \n_{[t, \tau]}, \boldsymbol{x}_{[\tau',T]} + \boldsymbol{e}_{[\tau',T]} \models {\varphi}'_1$, which is equivalent to $\boldsymbol{x}_{[t,T]} + \boldsymbol{e}_{[t,T]} \models \varphi', \forall \boldsymbol{e}_{[t,T]} \in \boldsymbol{\tilde E}_{\theta [t, T]}$. We can similarly prove the induction step for other temporal operators.
        
    \end{enumerate}
    
    By induction, if $\boldsymbol{x}_{[0,T]} \models \tilde{\varphi}$, then $\boldsymbol{x}_{[0,T]} +  \boldsymbol{e}_{[0,T]}\models \varphi, \forall \boldsymbol{e}_{[0,T]} \in \boldsymbol{E}_{\theta,T}$.
\end{proof}

An illustration of the proposed method is shown in Figure~\ref{fig: method}. According to Theorem~\ref{thm: STL erosion}, instead of verifying whether a stochastic trajectory satisfies an STL formula with a certain probability, one can verify whether a corresponding deterministic trajectory satisfies a tighter STL formula. Consequently, the stochastic STL verification problem (Problem~\ref{prob: verification}) reduces to a deterministic verification problem, which can be solved by any existing methods.

\begin{theorem}[STL verification of stochastic systems]\label{thm: verification}
    Consider the stochastic system \eqref{eq: stochastic dynamics} and the associated deterministic system \eqref{eq: deterministic dynamics} with initial set $\XX_0$ and bounded input set $\DD$. Given an STL specification $\varphi$ with bounded horizon $T$, $\delta \in (0,1)$, define $\tilde{E}_\theta$ as in \eqref{eq: E_theta} with $\theta = \delta/T$ and construct $\tilde \varphi$ as in Theorem~\ref{thm: STL erosion}. If \eqref{eq: deterministic dynamics} satisfies $\tilde{\varphi}$, then the stochastic system \eqref{eq: stochastic dynamics} satisfies $\varphi$ with $1-\delta$ guarantee.
\end{theorem}

\begin{proof}
    The deterministic system satisfies $\tilde{\varphi}$ implies that, for every initial state $x_0 \in \XX_0$ and every bounded disturbance $d_t \in \DD$, it holds that $\boldsymbol{x}_{[0,T]} \models \tilde \varphi$. By Theorem~\ref{thm: STL erosion}, for every initial state $x_0 \in \XX_0$ and every bounded disturbance $d_t \in \DD$, $\bP(\boldsymbol{X}_{[0,T]}\models \varphi) \geq 1-T\theta = 1-\delta$. Therefore the stochastic system satisfies $\varphi$ with $1-\delta$ guarantee.
\end{proof}

\subsection{STL verification}
\label{sec: probabilistic bound}
To effectively apply the STL erosion strategy, a tight PRS of the stochastic fluctuation is the key. Next, we utilize recent results on bounding stochastic derivation to obtain a tight PRS~\cite{liu2024probabilistic,jafarpour2024probabilistic}. 
% To achieve less conservative verification results by using Theorem \ref{thm: STL erosion}, a tight PRS of the stochastic fluctuation is desired. In this section, we leverage our recent result on bounding stochastic derivation to achieve this tight PRS~\cite{liu2024probabilistic,jafarpour2024probabilistic}. Another widely used approach for acquiring PRS is introduced as the baseline.
% \subsubsection{Probabilistic Bound on Stochastic Deviation}
% A tight bound on stochastic deviation $\|X_t-x_t\|$ is derived in~\cite{liu2024probabilistic}, which we restate in the following proposition.
\begin{proposition}\label{prop: stochastic deviation}\cite{liu2024probabilistic}
Let $\boldsymbol{X}_{[0,\infty]}$ be the trajectory of the stochastic system \eqref{eq: stochastic dynamics} and $\boldsymbol{x}_{[0,\infty]}$ be the trajectory of the deterministic system \eqref{eq: deterministic dynamics} with the same initial state $x_0 \in \XX_0$ and the same input sequence $\boldsymbol{d}_{[0, \infty]}$. Then for any $\varepsilon \in (0,1)$, $\theta \in (0,1)$, and $t\geq 0$,
\begin{equation}\label{eq: bound r}
 \bP\Big( \|X_t-x_t\|\leq r_{\theta, t} \Big) \geq 1-\theta,
\end{equation}
 where 
$r_{\theta, t} = \sqrt{\Psi_t(\varepsilon_1n+\varepsilon_2\log(1/\theta))}$, $\psi_t=\prod_{k=0}^{t}L_k^{2}$,\quad
$\Psi_t=\psi_{t-1}\sum_{k=0}^{t-1}\sigma_{k}^2\psi_k^{-1}$, 
$\varepsilon_1=\frac{2\log(1+2/\varepsilon)}{(1-\varepsilon)^2}, \varepsilon_2=\frac{2}{(1-\varepsilon)^2}$.
\end{proposition}

The bound in \eqref{eq: bound r} holds for both the Euclidean norm and the weighted norm \( \|\cdot \|_P \)~\cite[Section V-D]{jafarpour2024probabilistic}.
% Under the weighted norm \( \|\cdot \|_P \), a tighter bound can be obtained due to the smaller Lipschitz constant
The above bound scales logarithmically with $T$ and $1/\delta$, which is tight for stochastic systems \cite{liu2024probabilistic}. By Proposition~\ref{prop: stochastic deviation}, the ball $\BB(r_{\theta, t}, 0)$ (or an ellipsoid when using the weighted norm) serves as a tight PRS of $e_t$.

With the tight PRS of $e_t$, we can combine it with any existing methods for STL verification for deterministic systems, and leverage the STL erosion strategy stated in Theorem~\ref{thm: STL erosion} to verify STL satisfaction of the stochastic system.

\begin{theorem}\label{thm: overall}
        Consider the stochastic system \eqref{eq: stochastic dynamics} and the associated deterministic system \eqref{eq: deterministic dynamics} with initial set $\XX_0$, and the bounded input set $\DD$. Given an STL specification $\varphi$ with bounded horizon $T$, $\delta \in (0,1)$, let $\BB(r_{\delta, t}, 0)$ serve as the PRS of $e_t$ in Theorem~\ref{thm: verification}, where $r_{\delta, t} = \sqrt{\Psi_t(\varepsilon_1n+\varepsilon_2\log(T/\delta))}$. If the deterministic system satisfies $\tilde{\varphi}$, then the stochastic system satisfies $\varphi$ with $1-\delta$ guarantee.
\end{theorem}

\begin{proof}
    Plug the bound stated in Proposition~\ref{prop: stochastic deviation} into Theorem~\ref{thm: verification}, then Theorem~\ref{thm: overall} follows.
\end{proof}

% \subsubsection{Probabilistic Bound by Worst-Case Analysis} 
% \label{sec: worst-case analysis}
Worst-case analysis is used to verify the safety of systems under bounded disturbances~\cite{prajna2007framework}. It can also be applied to systems with unbounded noise for probabilistic verification by treating the noise as bounded noise with a high probability. This idea is used in \cite{vlahakis2024probabilistic} to handle probabilistic STL constraints for linear stochastic systems with affine predicates. The same idea can be readily extended to nonlinear stochastic systems.
% 
By~\cite[Section IV-B]{liu2024safety},
\begin{equation}\label{eq: sd by worst}
    \|X_t-x_t\|\leq \sqrt{\psi_{t-1}}\sum_{k=0}^{t-1}\sigma_{k}\sqrt{\psi_k^{-1}(\varepsilon_1n+\varepsilon_2\log\frac{T}{\delta})}
\end{equation}
holds for all $t\in\n_{[0,T]}$ with probability at least $1-\delta$. Denote the right-hand side as $r_{\delta,t}^{\text w}$. It is shown that $r_{\delta,t}^{\text w}$ is always greater than $r_{\delta,t}$ \cite{liu2024safety}. Thus, using the worst-case bound $r_{\delta,t}^{\text w}$ in STL erosion is inherently more conservative than Theorem~\ref{thm: overall}. 


     
%%%%% figure %%%%%%%%%%%%%%%%%%%%%%%%%%%%%%%%%%%%%%%%%%%%%%%%%
\begin{figure*}[t!]
 
  % \vspace*{-0.1in}
  \centering
  \scalebox{0.8}{
    \begin{tikzpicture}
     \node[anchor=north west] at (0in,0in)
      {{\includegraphics[width=1.0\linewidth,clip=true,trim=0
      220pt 60pt 0]{figs/exp_settings.pdf}}};
    \end{tikzpicture}
  }
  \vspace*{-0.1in}
  \caption{\textbf{Experiment Settings}. (Best viewed in color) The environment and tasks for evaluation.
  \textbf{(a)} BabyAI Trv-1 task with single-side door obstacle;
  \textbf{(b)} BabyAI Trv-2 task with double-side door obstacles;
  \textbf{(c)} BabyAI Numerical Q\&A task;
  \textbf{(d)} Two VirtualHome household environments (left: VH-1; right: VH-2) and an examplar task.
  }
  \vspace*{-0.15in}
 \label{fig:expSettings}
\end{figure*}
%%%%%%%%%%%%%%%%%%%%%%%%%%%%%%%%%%%%%%%%%%%%%%%%%%%%%%%%%%%%%

\section{Experimental Settings}

We evaluate our methods on a series of numerical Q\&A (NumQ\&A) and planning tasks in the BabyAI \citep{babyai, minigrid} and VirtualHome (VH) \citep{virtualhome} environments. For each environment, we provide an unified scene graph schema consistent across epoches with potentially distinct scene graphs. Each task requires reasoning on both the spatial structure and the semantic information encoded in the graph. We use the \textbf{success rate} as our evaluation metric, where success is defined as either providing the correct answer or achieving the desired outcome with simulation, depending on the task. 
Note that all experiments in this paper are conducted in the static setting, where the tested methods generate solutions solely based on the initial scene graph without interacting with the environment that will modify the graph. We also provide preliminary results under the dynamic settings in Appendix~\ref{app:DynExp}.

We use GPT-4o for both \RwR and baselines. \RwR is implemented using AutoGen \citep{autogen}. The detailed prompts are shown in Appendix~\ref{app:PromptTemplate}. 

\paragraph{Baselines} Following NLGraph \citep{NLGraph}, we compare our approach against direct reasoning methods based on whole graph prompting, including three zero-shot approaches: \textbf{zero-shot prompting} (\textsc{zero-shot}), \textbf{Zero-Shot Chain-of-Thought} (\textsc{0-cot}) \citep{zeroShotCot}, \textbf{Least-to-Most} (\textsc{ltm}) \citep{LTM}; and three few-shot methods: \textbf{Chain-of-Thought} (\textsc{cot}) \citep{CoT}, \textbf{Build-a-Graph} (\textsc{bag}) \citep{NLGraph}, \textbf{Algorithmic Prompting} (\textsc{algorithm}) \citep{NLGraph}. In addition to the few-shot examples, \textsc{algorithm} also require a language description of the task solving method. 
We also compare against \textbf{SayPlan} \citep{sayplan}, a retrieve-then-reason baseline specifically designed for the scene graphs, and \textbf{ReAct} \citep{react}, a generic iterative reasoning and acting approach that invokes database APIs to aggregate information. We test two versions of ReAct, one with graph-traversal actions only (\textsc{ReAct}) and the other with an additional \texttt{\small traversal\_room} function in BabyAI traversal task as explained in Sec. \ref{sec:BabyAITrv} (\textsc{ReAct-Trv}).
Both SayPlan and ReAct are provided with graph APIs for retrieving graph data. Please refer to Appendix~\ref{app:BaseDetail} for more details.

\paragraph{Few-shot \RwR}
We investigate the performance of \RwR in both zero-shot and few-shot settings. For the latter, we examine two types of few-shot prompting: \textbf{\RwR+FewShot(\RwR-FS)}, which incorporates additional in-context learning examples, and \textbf{\RwR+Algorithm(\RwR-A)}: which adds both in-context examples and algorithmic prompts following \textsc{algorithm}. Notably, we do not provide any fine-grained agent-level exemplar operations as in SayPlan \cite{sayplan}, which can be impractical to collect and may constrain the reasoning flexibility of LLMs. Instead, we examine whether our framework can leverage task-level annotations to enhance its reasoning capacity.


\begin{figure}[t!]
    \centering
    \vspace*{-0pt}
    \scalebox{0.67}{
    	\begin{tikzpicture}
         \node[anchor=north west] at (0in,0in)
          {{\includegraphics[width=1.0\linewidth,clip=true,trim=0
          260pt 580pt 15pt]{figs/babyai_sg.pdf}}};
        \end{tikzpicture}
    }
    \vspace*{-15pt}
    \caption{
        \textbf{BabyAI Scene Graph Representation}. Graph nodes represent \green{items}, \red{agents}, \orange{rooms}, and \yellow{doors}. Edges indicate items or agents located inside a  room, or doors that connect rooms. Room nodes are connected to a \gray{root} node.
    }
    \label{fig:BabyAISG}
    \vspace*{-10pt}
\end{figure}

Next few subsections describe the environment and task details. The scene graph node and edge information described in the schema are shown in Appendix~\ref{app:babyAIDetail}.

\subsection{2D Grid World Numerical Q\&A}
Our first experiment is on a numerical Q\&A task in a customized 9-room 2D BabyAI \citep{babyai} environment, as shown in Figure \ref{fig:expSettings}(c).
We generate scene graph representation of the environment following the hierachical graph design from 3DSG \citep{3dsg}, illustrated in Figure \ref{fig:BabyAISG}. Specifically, the graph represents the spatial scene layout through three levels: root, rooms, and objects, with additional door nodes connecting room pairs.
Please check Appendix \ref{app:babyAIDetail} for more details.
 
Following SayPlan \citep{sayplan}, we design the following question template: \texttt{\small find the color of the \string{TARGET\_OBJECT\string} in a room next to the room with \string{NUM\_IDENTIFIER\string} \string{COLOR\_IDENTIFIER\string} \string{IDENTIFIER\_OBJECT\string}}, where contents in curley brackets are populated based on each environment instance. 
The environment and question pairs are designed to ensure that there is only one answer.

We test each method in 100 task instances. For few-shot methods, we sample two instances and manually annotate the solutions as the in-context prompt. 

\subsection{2D Grid World Traversal Planning}
\label{sec:BabyAITrv}
We also test on the traversal planning in BabyAI, where the task is to generate a sequence of node-centric actions to pick up a target item. We design three atomic actions, including (1) \texttt{\small pickup(nodeID)}: Walk to and pickup an object by the node ID; (2) \texttt{\small remove(nodeID)}: Walk to and remove an object by the node ID; (3) \texttt{\small open(nodeID)}: Walk to and open a door by the node ID. 
%We directly query \RwR and all baselines to generate the actions in the format above.

As shown in Figure \ref{fig:expSettings}(a)(b), the traversal planning task is tested in two related double-room environments, both of which require the agent to pick up the key of the correct color to unlock the door, remove any obstacle that blocks the door, open the door, and pick up the target. The difference is that the first environment, dubbed \textbf{Trv\-1}, contains only the agent-side obstacle, whereas the second environment, dubbed \textbf{Trv\-2}, contains another target-side obstacle. We generate the in-context examples \textit{only in Trv\-1} , and test if the methods can extrapolate to Trv\-2. As before, we evaluate each method in 100 times in different instance of both types of the environment. For \RwR, we provide the reasoning function \texttt{\small traversal\_room} programmed based on the $A^*$ algorithm, which identifies the item to remove in order to reach from an initial to a desired location within the same room.  As we will show, \RwR is able to leverage this external tool to compensate for the limited mathematical problem solving ability of LLMs.

\subsection{Household Task Planning}
The last evaluation is in two VirtualHome (VH) \citep{virtualhome} environments shown in Figure \ref{fig:expSettings}(d), denoted as \textbf{VH-1} and \textbf{VH-2}, respectively. We use the built-in environmental graph as the scene graph. Compared to BabyAI, VH environments have larger state space and action space, containing 115 object instances, 8 relationship types, and multiple object properties and states. Hence, VH environments are more challenging with richer information in the graphs.
% The action space is: $\mathcal{A}$ = \texttt{\{\}}. 
For each environment, we adopt the 10 household tasks from ProgPrompt \citep{progprompt}, such as \texttt{\small "put the soap in the bathroom cabinet"}, and query each method for the action sequence in the VH action format to accomplish the task. 
% As before, we task each method to directly generate the plan in the VH action format. It includes \texttt{\small [action\_name]<object\_name>(object\_id)} for one argument actions, and \texttt{\small [action\_name]<object\_name1>(object\_id1)<object\_name2>(object\_id2)} for two argument actions.
We use two of the tasks, together with the ground truth actions, as the few-shot examples, and test with the other eight. We follow CoELA \cite{coopEmbod} to specify the task as the desired states. For example, the task of above is specified as \texttt{\small soap INSIDE bathroomcabinet}. 
% To achieve the desired state, LLMs need to reason over the current state of the environment in order to identify the sequence of actions that ultimately achieve the achieve the desired outcome. A plan is considered successful if the desired states are reached after simulation. 
For more details, please refer to Appendix \ref{app:VHDetail}.



\section{Results and Analysis}



%% BabyAI in one table
% \begin{table}[t!]
\begin{table*}[t!]
    \centering
    \setlength\tabcolsep{3.pt}
     \setlength\extrarowheight{-9pt}
    \begin{tabular}{l c c c c c c c c c c c c}
        \toprule[1.5pt]
              & \multicolumn{4}{c}{Zero-Shot} & \multicolumn{6}{c}{Few-Shot}
             \\
             \cmidrule(lr){2-5}
             \cmidrule(lr){6-12}
             \\
             \textbf{Task} & ZeroShot & 0-CoT & LTM & \textbf{\RwR} & CoT & BAG & Alg & ReAct & ReAct-Trv & SayPlan & \makecell{\textbf{\RwR} \\(FS)} & \makecell{\textbf{\RwR} \\ (Alg)} 
             \\
        \midrule[1pt]
             NumQ\&A & 55\% & 48\% & 52\% & \textbf{95\%} &  53\% & 51\% & 65\% & 
             24\% & 24\% &
             35\% & 
             \textbf{94\%} & \textbf{97\%}
             \\
             Trv-1 & 20\% & 23\% & 17\% & \textbf{61\%} & 34\% & 35\% & \textbf{64\%} & 
             13\% & 62\% &
             18\% &
             \textbf{67\%} & \textbf{64\%} 
             \\
             Trv-2 & 11\% & 7\% & 6\% & \textbf{56\%} & 1\% & 1\% & 0\% & 
             0\% & \textbf{56\%} &
             0\% &
             \textbf{61\%} & \textbf{56\%} 
             \\
         \bottomrule[1.5pt]
    \end{tabular}
    \caption{\textbf{Results in BabyAI} \RwR achieves the best performance across all tasks in both zero-shot and few-shot settings, showing that \RwR (1) is effective in solving spatial tasks; (2) can harness the information from in-context examples and extrapolate better to unseen tasks. We highlight the top-1 performance under the zero-shot setting and top-2 performances, including ties, under the few-shot setting.
    }
    \label{tab:BabyAI}
    \vspace{-0.2in}
\end{table*}

\begin{table}[t!]
    \centering
    % \setlength\tabcolsep{2.5 pt}
	\begin{tabular}{l  c c  c }
        \toprule[1.5pt]
        Method & \makecell{Few-Shot \\ Examples} & VH-1 & VH-2 \\
        \hline 
         \textbf{ZeroShot}  &  & 87.5\% & 75\% \\
         \textbf{0-CoT}     &  & 87.5\% & 75\% \\
         \textbf{LTM}       &  & 87.5\% & 62.5\% \\
         \hline
         \textbf{CoT}       & \bluecheck & 87.5\% & 75\% \\
         \textbf{BAG}       & \bluecheck & 87.5\% & 62.5\% \\
         \hline
         \textbf{RwR} &  & \bf{100\%} & \bf{100\%} \\
         % \cmidrule(r){1-2} \cmidrule(lr){3-14} \cmidrule{15-16}
        \bottomrule[1.5pt]
    \end{tabular}
    \caption{
        \textbf{Results in VirtualHome}. The superior performance of \RwR shows that it is capable of grounding its plan to the environmental states.
    }\label{tab:VHResults}
    \vspace{-12pt}
\end{table}


\subsection{Experiment Results}
\paragraph{Numerical Q\&A Resutls} The results are collected in Table \ref{tab:BabyAI}. Zero-shot \RwR outperforms the best baseline by 30 percentage points (pp), even without taking advantage of the few-shot examples. Few-shot methods do not show significant advantage over zero-shot methods, as the reasoning trace for this task is simple. However, they all tend to make mistakes when addressing the substeps such as counting the item or locating the neighboring rooms. The retrieval mechanisms in both ReAct and SayPlan further degrades the performance. SayPlan cannot effectively retrieve information, due to its retrieve-then-reason framework that does not condition the retrieval on intermediate reasoning. ReAct employs the iterative reasoning, but API-calling is less effective for large scene graphs. In contrast, \RwR retrieves information that better facilitate reasoning. %attends to the graph information in the correct order by querying for it based on the reasoning process.


\paragraph{2D Traversal Results} Table \ref{tab:BabyAI} also reports the success rate in the traversal task. In the seen Trv1 environment, our method achieves 38pp and 3pp higher success rate against the best performing baselines under zero-shot and few-shot settings, respectively. While few-shot baselines perform more than 10pp better compared to zero-shot baselines, they perform even worse in the unseen settings, achieving $\leq 1\%$ success rate. This indicates that although few-shot examples is helpful in the seen tasks, LLMs do not \textit{learn} the reasoning process to extrapolate to similar unseen tasks. Rather, LLMs might only \textit{memorize} the heuristic mechanism in the solution, such as always removing the item on the left of the door. On the other hand, by not directly processing scene graphs, the Reasoner in \RwR better learns the reasoning process essential for the task, and can thus extrapolate well to similar problems. SayPlan and ReAct achieve inferior results \RwR, indicating that their heuristic or API-based retrieval methods are less suitable for complex tasks concerning global information.


\paragraph{Household Task Planning Results}
The planning success rate on the 8 tasks in the 2 VH environments are shown in Table \ref{tab:VHResults}. We observe that all baselines consistently fail to address the precondition of the planned action. For example, all of them failed to generate \texttt{\small [open] <garbagecan> (ID)} before \texttt{\small [putin] <plum> (ID) <garbagecan> (ID)}, forgetting that the state of the garbage can is \texttt{\small state:\{CLOSED\}} from the extensive graph input. On the other hand, \RwR doesn't process the entire graph. Instead, it queries for the specific object information, which helps to better determine the action parameter and examine the action preconditions. 
% For qualitatitve demonstration, please refer to Figure \ref{fig:vhQual} for an examplar task and solution by our method.

We also present qualitative results in Appendix~\ref{app:RwRTrvDemo} (BabyAI), Appendix~\ref{app:RwRVHDemo} (VirtualHome), and Appendix~\ref{app:BaselineFailures} (Baseline failure cases). We provide analysis on the compute cost with the iterative design in Appendix~\ref{app:ComputeAnly}.


\begin{table*}[t!]
    \centering
    % \setlength\tabcolsep{2.5 pt}
    \begin{tabular}{l | c c c | c c c}
        \toprule[1.5pt]
        Method & \makecell{Code-Writing \\ \& Tool-Use} & \makecell{Iterative Reason} & Two-Agent & Numerical Q\&A & Trv-1 & Trv-2\\
         \hline
         \textbf{SingleCoder} & \bluecheck &  &  & 80\% & 33\% & 25\%\\
         \textbf{\RwR-T} &  & \bluecheck & \bluecheck & 57\% & 18\% & 8\%\\
         \textbf{\RwR-S} & \bluecheck & \bluecheck & &  90\% & 46\% & 31\%\\
         \textbf{\RwR} & \bluecheck & \bluecheck & \bluecheck & \bf{95\%} & \bf{61\%} &  \textbf{56\%} \\
         % \cmidrule(r){1-2} \cmidrule(lr){3-14} \cmidrule{15-16}
        \bottomrule[1.5pt]
    \end{tabular}
    \caption{
        \textbf{Ablation in BabyAI traversal and numerical Q\&A}. 
        The best result is achieved by combining both Reason-while-Retrieve framework and the code-writing, justifying the key designs in our method.
    }\label{tab:ablate}
    \vspace{-10pt}
\end{table*}


\subsection{Ablation}
\paragraph{Setup} To further validate the designs of \RwR framework, we conduct an ablation study for the key component of our method. To this end, we compare against the following variants of \RwR:
\begin{itemize}
    \item \textbf{SingleCoder}: 
    Single-shot code writing with LLMs to address a given task, without iterative retrieval and reasoning or multiple generation. This variant verifies the benefit of \textbf{the iterative retrieval-generation}.
    % A single LLM that directly writes the entire code to address a given task. It benefits from the accurate numerical reasoning and tool-use capacity from the code-writing, but does not have the opportunity to analyze the intermediate graph information from the iterative retrieving and reasoning (dubbed \textit{Iterative RetRea} in this section). 
    The SingleCoder is prompted with the combination of the information for both the Reasoner and the Retriever in \RwR, including the environment and action space information, scene graph schema, and tool annotations. The self-debugging mechanism is also enabled.

    \item \textbf{\RwR-T}: 
    Language-only two-agent iterative method, without the schema-based code-writing or tool-using mechanism. Both agents cooperate purely in the language space, and the graph is directly prompted to the retriever as texts. This variant validates the \textbf{schema-based code-writing} design. We also use an additional LLM-based action format corrector following \citep{LLMPlanner, groundedDecoding}, as it consistently fails to generate actions in the executable format.
    % The other variant disables the code-writing ability of both the Retriever and Reasoner in \RwR. Instead, both cooperative agents rely purely on language reasoning and communication skill to solve a given task. This design evaluates the performance of the iterative retrieve and reason process without the code-writing.

    \item \textbf{\RwR-S}: 
    An one-agent version of \RwR with both code-writing and iterative mechanism. This variant verifies the benefits of the \textbf{two-agent} design against one-agent. Specifically, the one-agent \RwR-S is fed with the union of the Reasoner and Retriever prompts. At each iteration, it processes the entire reasoning and retrieval history to generate the next step:
    \begin{align}
        \anly_t, \query_t, \code_t &= LLM(\{\anly_0, \query_0, \code_0, \gG^\prime_0\}, \cdots; \gs) \\
        \gG^\prime_{t} &= \code_t(\gG)
        \end{align}
    \end{itemize}
    
    
All variants are tested in BabyAI under the zero-shot setting. 

\paragraph{Results} The ablation study results are demonstrated in Table \ref{tab:ablate}, where all variants are out-performed by the original \RwR, justifying the core designs. Specifically, 
{\em SingleCoder} is capable of solving numerical problems, but is unable to address complex planning tasks without the iterative cooperation.
{\em \RwR-T} can better break the task down with iterative task solving, but cannot consistently obtain the correct solution for each substep without code-writing. For example, queried with \texttt{\small "Find all rooms that contain 5 green balls"}, the non-code-writing Retriever might struggle with the counting problem in the language space.
{\em \RwR-S} performs both iterative reasoning and code-writing, but can be misguided by the redundant historical information when addressing complex tasks.

% \subsection{Two-agent v.s. One-Agent}
% \paragraph{Setup}
% We further demonstrate the importance of the two-agent design of our framework, by 
\section{Related Literature}
\label{sec:lit}

\paragraph{Language models for Task and Motion Planning}
% With the advance of large language or multimodal models, many earlier
Many existing efforts harness the power of large language models for decision making \cite{xi2023rise, leap, llm+p} and robotic control \cite{plan-seq-learn, zhang2023bootstrap, text2motion, CGNet, hatori2018interactively}. With rich built-in knowledge and in-context learning ability,
%trained from the large internet-scale text corpora, 
language models are used for generating task-level plans \cite{raman2022planning, gao2024physically}, action selection \cite{saycan, pivot}, processing environmental or human feedback \cite{CLAIRIFY}, training or finetuning language-conditioned policy models \cite{octo, rt-x, szot2023large}, and more. 
To factor in the environment during planning, recent studies have explored using LLMs for programmatic plan generation \cite{progprompt}, combining knowledge from external perception tools \cite{CaP, voxposer} or grounded decoding \cite{groundedDecoding}, and value function generation \cite{language2reward}. While proven effective, those methods are limited to small scale environments, and rely on expert perception models to extract task-related states from the scene representation with implicit spatial structure. In this work, we study using pretrained LLMs to process the the global representation of large environments with explicit structure.%, and generate the solution that is grounded in the environment. 

\paragraph{Graph as the Scene Representation}
The scope of the solvable task is largely determined by the state representation. Compare to sensory representation such as images or point clouds, scene graphs are compact thus scalable to large environments \cite{greve2024collaborative}, structured to represent spatial layout explicitly \cite{hydra, wu2021scenegraphfusion}, and efficient in representing diverse states of the environment \cite{3dsg}. Therefore, they have been used in various manipulation or navigation tasks \cite{3dsgNav, hierarchicalSg}. In this paper, we exploit these favorable features of the scene graph representation to ground the reasoning process of LLMs to the environment. 

\paragraph{LLMs for Reasoning on Graph}
Leveraging language models to reason with graphs is a growing area. While prior works trains to integrates graph and language knowledge \cite{instructGLM, GRID}, recent study explores serializing graph-structured data as prompts for pretrained LLMs \cite{NLGraph, talkLikeGraph}. This strategy has been successfully used in knowledge-graph-enhanced LLMs reasoning \cite{thinkOnGraph, reasonOnGraph} and scene-graph-based robotic task planning \cite{conceptgraphs}. Closest to our work, SayPlan \cite{sayplan} prompts scene graphs to LLMs and designs a Retrieve-then-Reason framework for robotic planning. However, its room-by-room retrieval heuristic is only effective in the object search task. Instead, we design the Reason-while-Retrieve framework for general spatial reasoning with scene graphs.

%We further incorporate the code-writing and tool-use ability to LLMs, so that our proposed method can effectively retrieve information based on scene graphs and address numerical tasks that fall beyond the expertise of LLMs \cite{AliceInWonderland}.
\vspace{-2mm}
\section{Conclusion}\label{sec:conclusion}
In this paper, we presented RecDreamer, a novel approach to mitigating the Multi-Face Janus problem in text-to-3D generation. Our solution introduces a rectification function to modify the prior distribution, ensuring that the resulting joint distribution achieves uniformity across poses. By expressing the modified data distribution as the product of the original density and the rectification function, we seamlessly integrate this adjustment into the score distillation algorithm. This allows us to derive a particle optimization framework for uniform score distillation. Additionally, we developed a pose classifier and implemented reliable approximations and simulations to enhance the particle optimization process. Extensive experiments on both 2D and 3D synthesis tasks demonstrate the effectiveness of our approach in addressing the Multi-Face Janus problem, resulting in more consistent geometries and textures across different views.

\textbf{Limitations.} While our method significantly reduces bias in prior distributions, further exploration of 3D modeling with multi-view priors could improve geometric and texture consistency. Extending our approach through deeper research into conditional control presents another promising avenue for addressing these challenges in future work. 

% Acknowledgements should only appear in the accepted version.
% \section*{Acknowledgements}

% \textbf{Do not} include acknowledgements in the initial version of
% the paper submitted for blind review.

% If a paper is accepted, the final camera-ready version can (and
% usually should) include acknowledgements.  Such acknowledgements
% should be placed at the end of the section, in an unnumbered section
% that does not count towards the paper page limit. Typically, this will 
% include thanks to reviewers who gave useful comments, to colleagues 
% who contributed to the ideas, and to funding agencies and corporate 
% sponsors that provided financial support.

% Impact Statement does not count towards the page limit
\section*{Impact Statement}

The paper presents a framework that enables Large Language Models to solve spatial tasks with scene graphs. 
Due to the wide usage of scene graphs as the environmental representation in both simulators and real world, the proposed framework can be potentially embedded in numerous applications.  
Such applications involve high-level task planning~\cite{sayplan, conceptgraphs}, data collection~\cite{octopus, spatialrgpt}, and building autonomous agent in the gaming or Virtual Reality (VR) environments~\cite{gameAgentSurvey}.
In summary, the framework has the potential to serve as a key module in autonomous systems, enhancing the ability to reason spatially and interact with the environments.



% In the unusual situation where you want a paper to appear in the
% references without citing it in the main text, use \nocite
% \nocite{langley00}

\bibliography{reference}
\bibliographystyle{icml2025}


%%%%%%%%%%%%%%%%%%%%%%%%%%%%%%%%%%%%%%%%%%%%%%%%%%%%%%%%%%%%%%%%%%%%%%%%%%%%%%%
%%%%%%%%%%%%%%%%%%%%%%%%%%%%%%%%%%%%%%%%%%%%%%%%%%%%%%%%%%%%%%%%%%%%%%%%%%%%%%%
% APPENDIX
%%%%%%%%%%%%%%%%%%%%%%%%%%%%%%%%%%%%%%%%%%%%%%%%%%%%%%%%%%%%%%%%%%%%%%%%%%%%%%%
%%%%%%%%%%%%%%%%%%%%%%%%%%%%%%%%%%%%%%%%%%%%%%%%%%%%%%%%%%%%%%%%%%%%%%%%%%%%%%%
\newpage
\appendix
\onecolumn
\subsection{Lloyd-Max Algorithm}
\label{subsec:Lloyd-Max}
For a given quantization bitwidth $B$ and an operand $\bm{X}$, the Lloyd-Max algorithm finds $2^B$ quantization levels $\{\hat{x}_i\}_{i=1}^{2^B}$ such that quantizing $\bm{X}$ by rounding each scalar in $\bm{X}$ to the nearest quantization level minimizes the quantization MSE. 

The algorithm starts with an initial guess of quantization levels and then iteratively computes quantization thresholds $\{\tau_i\}_{i=1}^{2^B-1}$ and updates quantization levels $\{\hat{x}_i\}_{i=1}^{2^B}$. Specifically, at iteration $n$, thresholds are set to the midpoints of the previous iteration's levels:
\begin{align*}
    \tau_i^{(n)}=\frac{\hat{x}_i^{(n-1)}+\hat{x}_{i+1}^{(n-1)}}2 \text{ for } i=1\ldots 2^B-1
\end{align*}
Subsequently, the quantization levels are re-computed as conditional means of the data regions defined by the new thresholds:
\begin{align*}
    \hat{x}_i^{(n)}=\mathbb{E}\left[ \bm{X} \big| \bm{X}\in [\tau_{i-1}^{(n)},\tau_i^{(n)}] \right] \text{ for } i=1\ldots 2^B
\end{align*}
where to satisfy boundary conditions we have $\tau_0=-\infty$ and $\tau_{2^B}=\infty$. The algorithm iterates the above steps until convergence.

Figure \ref{fig:lm_quant} compares the quantization levels of a $7$-bit floating point (E3M3) quantizer (left) to a $7$-bit Lloyd-Max quantizer (right) when quantizing a layer of weights from the GPT3-126M model at a per-tensor granularity. As shown, the Lloyd-Max quantizer achieves substantially lower quantization MSE. Further, Table \ref{tab:FP7_vs_LM7} shows the superior perplexity achieved by Lloyd-Max quantizers for bitwidths of $7$, $6$ and $5$. The difference between the quantizers is clear at 5 bits, where per-tensor FP quantization incurs a drastic and unacceptable increase in perplexity, while Lloyd-Max quantization incurs a much smaller increase. Nevertheless, we note that even the optimal Lloyd-Max quantizer incurs a notable ($\sim 1.5$) increase in perplexity due to the coarse granularity of quantization. 

\begin{figure}[h]
  \centering
  \includegraphics[width=0.7\linewidth]{sections/figures/LM7_FP7.pdf}
  \caption{\small Quantization levels and the corresponding quantization MSE of Floating Point (left) vs Lloyd-Max (right) Quantizers for a layer of weights in the GPT3-126M model.}
  \label{fig:lm_quant}
\end{figure}

\begin{table}[h]\scriptsize
\begin{center}
\caption{\label{tab:FP7_vs_LM7} \small Comparing perplexity (lower is better) achieved by floating point quantizers and Lloyd-Max quantizers on a GPT3-126M model for the Wikitext-103 dataset.}
\begin{tabular}{c|cc|c}
\hline
 \multirow{2}{*}{\textbf{Bitwidth}} & \multicolumn{2}{|c|}{\textbf{Floating-Point Quantizer}} & \textbf{Lloyd-Max Quantizer} \\
 & Best Format & Wikitext-103 Perplexity & Wikitext-103 Perplexity \\
\hline
7 & E3M3 & 18.32 & 18.27 \\
6 & E3M2 & 19.07 & 18.51 \\
5 & E4M0 & 43.89 & 19.71 \\
\hline
\end{tabular}
\end{center}
\end{table}

\subsection{Proof of Local Optimality of LO-BCQ}
\label{subsec:lobcq_opt_proof}
For a given block $\bm{b}_j$, the quantization MSE during LO-BCQ can be empirically evaluated as $\frac{1}{L_b}\lVert \bm{b}_j- \bm{\hat{b}}_j\rVert^2_2$ where $\bm{\hat{b}}_j$ is computed from equation (\ref{eq:clustered_quantization_definition}) as $C_{f(\bm{b}_j)}(\bm{b}_j)$. Further, for a given block cluster $\mathcal{B}_i$, we compute the quantization MSE as $\frac{1}{|\mathcal{B}_{i}|}\sum_{\bm{b} \in \mathcal{B}_{i}} \frac{1}{L_b}\lVert \bm{b}- C_i^{(n)}(\bm{b})\rVert^2_2$. Therefore, at the end of iteration $n$, we evaluate the overall quantization MSE $J^{(n)}$ for a given operand $\bm{X}$ composed of $N_c$ block clusters as:
\begin{align*}
    \label{eq:mse_iter_n}
    J^{(n)} = \frac{1}{N_c} \sum_{i=1}^{N_c} \frac{1}{|\mathcal{B}_{i}^{(n)}|}\sum_{\bm{v} \in \mathcal{B}_{i}^{(n)}} \frac{1}{L_b}\lVert \bm{b}- B_i^{(n)}(\bm{b})\rVert^2_2
\end{align*}

At the end of iteration $n$, the codebooks are updated from $\mathcal{C}^{(n-1)}$ to $\mathcal{C}^{(n)}$. However, the mapping of a given vector $\bm{b}_j$ to quantizers $\mathcal{C}^{(n)}$ remains as  $f^{(n)}(\bm{b}_j)$. At the next iteration, during the vector clustering step, $f^{(n+1)}(\bm{b}_j)$ finds new mapping of $\bm{b}_j$ to updated codebooks $\mathcal{C}^{(n)}$ such that the quantization MSE over the candidate codebooks is minimized. Therefore, we obtain the following result for $\bm{b}_j$:
\begin{align*}
\frac{1}{L_b}\lVert \bm{b}_j - C_{f^{(n+1)}(\bm{b}_j)}^{(n)}(\bm{b}_j)\rVert^2_2 \le \frac{1}{L_b}\lVert \bm{b}_j - C_{f^{(n)}(\bm{b}_j)}^{(n)}(\bm{b}_j)\rVert^2_2
\end{align*}

That is, quantizing $\bm{b}_j$ at the end of the block clustering step of iteration $n+1$ results in lower quantization MSE compared to quantizing at the end of iteration $n$. Since this is true for all $\bm{b} \in \bm{X}$, we assert the following:
\begin{equation}
\begin{split}
\label{eq:mse_ineq_1}
    \tilde{J}^{(n+1)} &= \frac{1}{N_c} \sum_{i=1}^{N_c} \frac{1}{|\mathcal{B}_{i}^{(n+1)}|}\sum_{\bm{b} \in \mathcal{B}_{i}^{(n+1)}} \frac{1}{L_b}\lVert \bm{b} - C_i^{(n)}(b)\rVert^2_2 \le J^{(n)}
\end{split}
\end{equation}
where $\tilde{J}^{(n+1)}$ is the the quantization MSE after the vector clustering step at iteration $n+1$.

Next, during the codebook update step (\ref{eq:quantizers_update}) at iteration $n+1$, the per-cluster codebooks $\mathcal{C}^{(n)}$ are updated to $\mathcal{C}^{(n+1)}$ by invoking the Lloyd-Max algorithm \citep{Lloyd}. We know that for any given value distribution, the Lloyd-Max algorithm minimizes the quantization MSE. Therefore, for a given vector cluster $\mathcal{B}_i$ we obtain the following result:

\begin{equation}
    \frac{1}{|\mathcal{B}_{i}^{(n+1)}|}\sum_{\bm{b} \in \mathcal{B}_{i}^{(n+1)}} \frac{1}{L_b}\lVert \bm{b}- C_i^{(n+1)}(\bm{b})\rVert^2_2 \le \frac{1}{|\mathcal{B}_{i}^{(n+1)}|}\sum_{\bm{b} \in \mathcal{B}_{i}^{(n+1)}} \frac{1}{L_b}\lVert \bm{b}- C_i^{(n)}(\bm{b})\rVert^2_2
\end{equation}

The above equation states that quantizing the given block cluster $\mathcal{B}_i$ after updating the associated codebook from $C_i^{(n)}$ to $C_i^{(n+1)}$ results in lower quantization MSE. Since this is true for all the block clusters, we derive the following result: 
\begin{equation}
\begin{split}
\label{eq:mse_ineq_2}
     J^{(n+1)} &= \frac{1}{N_c} \sum_{i=1}^{N_c} \frac{1}{|\mathcal{B}_{i}^{(n+1)}|}\sum_{\bm{b} \in \mathcal{B}_{i}^{(n+1)}} \frac{1}{L_b}\lVert \bm{b}- C_i^{(n+1)}(\bm{b})\rVert^2_2  \le \tilde{J}^{(n+1)}   
\end{split}
\end{equation}

Following (\ref{eq:mse_ineq_1}) and (\ref{eq:mse_ineq_2}), we find that the quantization MSE is non-increasing for each iteration, that is, $J^{(1)} \ge J^{(2)} \ge J^{(3)} \ge \ldots \ge J^{(M)}$ where $M$ is the maximum number of iterations. 
%Therefore, we can say that if the algorithm converges, then it must be that it has converged to a local minimum. 
\hfill $\blacksquare$


\begin{figure}
    \begin{center}
    \includegraphics[width=0.5\textwidth]{sections//figures/mse_vs_iter.pdf}
    \end{center}
    \caption{\small NMSE vs iterations during LO-BCQ compared to other block quantization proposals}
    \label{fig:nmse_vs_iter}
\end{figure}

Figure \ref{fig:nmse_vs_iter} shows the empirical convergence of LO-BCQ across several block lengths and number of codebooks. Also, the MSE achieved by LO-BCQ is compared to baselines such as MXFP and VSQ. As shown, LO-BCQ converges to a lower MSE than the baselines. Further, we achieve better convergence for larger number of codebooks ($N_c$) and for a smaller block length ($L_b$), both of which increase the bitwidth of BCQ (see Eq \ref{eq:bitwidth_bcq}).


\subsection{Additional Accuracy Results}
%Table \ref{tab:lobcq_config} lists the various LOBCQ configurations and their corresponding bitwidths.
\begin{table}
\setlength{\tabcolsep}{4.75pt}
\begin{center}
\caption{\label{tab:lobcq_config} Various LO-BCQ configurations and their bitwidths.}
\begin{tabular}{|c||c|c|c|c||c|c||c|} 
\hline
 & \multicolumn{4}{|c||}{$L_b=8$} & \multicolumn{2}{|c||}{$L_b=4$} & $L_b=2$ \\
 \hline
 \backslashbox{$L_A$\kern-1em}{\kern-1em$N_c$} & 2 & 4 & 8 & 16 & 2 & 4 & 2 \\
 \hline
 64 & 4.25 & 4.375 & 4.5 & 4.625 & 4.375 & 4.625 & 4.625\\
 \hline
 32 & 4.375 & 4.5 & 4.625& 4.75 & 4.5 & 4.75 & 4.75 \\
 \hline
 16 & 4.625 & 4.75& 4.875 & 5 & 4.75 & 5 & 5 \\
 \hline
\end{tabular}
\end{center}
\end{table}

%\subsection{Perplexity achieved by various LO-BCQ configurations on Wikitext-103 dataset}

\begin{table} \centering
\begin{tabular}{|c||c|c|c|c||c|c||c|} 
\hline
 $L_b \rightarrow$& \multicolumn{4}{c||}{8} & \multicolumn{2}{c||}{4} & 2\\
 \hline
 \backslashbox{$L_A$\kern-1em}{\kern-1em$N_c$} & 2 & 4 & 8 & 16 & 2 & 4 & 2  \\
 %$N_c \rightarrow$ & 2 & 4 & 8 & 16 & 2 & 4 & 2 \\
 \hline
 \hline
 \multicolumn{8}{c}{GPT3-1.3B (FP32 PPL = 9.98)} \\ 
 \hline
 \hline
 64 & 10.40 & 10.23 & 10.17 & 10.15 &  10.28 & 10.18 & 10.19 \\
 \hline
 32 & 10.25 & 10.20 & 10.15 & 10.12 &  10.23 & 10.17 & 10.17 \\
 \hline
 16 & 10.22 & 10.16 & 10.10 & 10.09 &  10.21 & 10.14 & 10.16 \\
 \hline
  \hline
 \multicolumn{8}{c}{GPT3-8B (FP32 PPL = 7.38)} \\ 
 \hline
 \hline
 64 & 7.61 & 7.52 & 7.48 &  7.47 &  7.55 &  7.49 & 7.50 \\
 \hline
 32 & 7.52 & 7.50 & 7.46 &  7.45 &  7.52 &  7.48 & 7.48  \\
 \hline
 16 & 7.51 & 7.48 & 7.44 &  7.44 &  7.51 &  7.49 & 7.47  \\
 \hline
\end{tabular}
\caption{\label{tab:ppl_gpt3_abalation} Wikitext-103 perplexity across GPT3-1.3B and 8B models.}
\end{table}

\begin{table} \centering
\begin{tabular}{|c||c|c|c|c||} 
\hline
 $L_b \rightarrow$& \multicolumn{4}{c||}{8}\\
 \hline
 \backslashbox{$L_A$\kern-1em}{\kern-1em$N_c$} & 2 & 4 & 8 & 16 \\
 %$N_c \rightarrow$ & 2 & 4 & 8 & 16 & 2 & 4 & 2 \\
 \hline
 \hline
 \multicolumn{5}{|c|}{Llama2-7B (FP32 PPL = 5.06)} \\ 
 \hline
 \hline
 64 & 5.31 & 5.26 & 5.19 & 5.18  \\
 \hline
 32 & 5.23 & 5.25 & 5.18 & 5.15  \\
 \hline
 16 & 5.23 & 5.19 & 5.16 & 5.14  \\
 \hline
 \multicolumn{5}{|c|}{Nemotron4-15B (FP32 PPL = 5.87)} \\ 
 \hline
 \hline
 64  & 6.3 & 6.20 & 6.13 & 6.08  \\
 \hline
 32  & 6.24 & 6.12 & 6.07 & 6.03  \\
 \hline
 16  & 6.12 & 6.14 & 6.04 & 6.02  \\
 \hline
 \multicolumn{5}{|c|}{Nemotron4-340B (FP32 PPL = 3.48)} \\ 
 \hline
 \hline
 64 & 3.67 & 3.62 & 3.60 & 3.59 \\
 \hline
 32 & 3.63 & 3.61 & 3.59 & 3.56 \\
 \hline
 16 & 3.61 & 3.58 & 3.57 & 3.55 \\
 \hline
\end{tabular}
\caption{\label{tab:ppl_llama7B_nemo15B} Wikitext-103 perplexity compared to FP32 baseline in Llama2-7B and Nemotron4-15B, 340B models}
\end{table}

%\subsection{Perplexity achieved by various LO-BCQ configurations on MMLU dataset}


\begin{table} \centering
\begin{tabular}{|c||c|c|c|c||c|c|c|c|} 
\hline
 $L_b \rightarrow$& \multicolumn{4}{c||}{8} & \multicolumn{4}{c||}{8}\\
 \hline
 \backslashbox{$L_A$\kern-1em}{\kern-1em$N_c$} & 2 & 4 & 8 & 16 & 2 & 4 & 8 & 16  \\
 %$N_c \rightarrow$ & 2 & 4 & 8 & 16 & 2 & 4 & 2 \\
 \hline
 \hline
 \multicolumn{5}{|c|}{Llama2-7B (FP32 Accuracy = 45.8\%)} & \multicolumn{4}{|c|}{Llama2-70B (FP32 Accuracy = 69.12\%)} \\ 
 \hline
 \hline
 64 & 43.9 & 43.4 & 43.9 & 44.9 & 68.07 & 68.27 & 68.17 & 68.75 \\
 \hline
 32 & 44.5 & 43.8 & 44.9 & 44.5 & 68.37 & 68.51 & 68.35 & 68.27  \\
 \hline
 16 & 43.9 & 42.7 & 44.9 & 45 & 68.12 & 68.77 & 68.31 & 68.59  \\
 \hline
 \hline
 \multicolumn{5}{|c|}{GPT3-22B (FP32 Accuracy = 38.75\%)} & \multicolumn{4}{|c|}{Nemotron4-15B (FP32 Accuracy = 64.3\%)} \\ 
 \hline
 \hline
 64 & 36.71 & 38.85 & 38.13 & 38.92 & 63.17 & 62.36 & 63.72 & 64.09 \\
 \hline
 32 & 37.95 & 38.69 & 39.45 & 38.34 & 64.05 & 62.30 & 63.8 & 64.33  \\
 \hline
 16 & 38.88 & 38.80 & 38.31 & 38.92 & 63.22 & 63.51 & 63.93 & 64.43  \\
 \hline
\end{tabular}
\caption{\label{tab:mmlu_abalation} Accuracy on MMLU dataset across GPT3-22B, Llama2-7B, 70B and Nemotron4-15B models.}
\end{table}


%\subsection{Perplexity achieved by various LO-BCQ configurations on LM evaluation harness}

\begin{table} \centering
\begin{tabular}{|c||c|c|c|c||c|c|c|c|} 
\hline
 $L_b \rightarrow$& \multicolumn{4}{c||}{8} & \multicolumn{4}{c||}{8}\\
 \hline
 \backslashbox{$L_A$\kern-1em}{\kern-1em$N_c$} & 2 & 4 & 8 & 16 & 2 & 4 & 8 & 16  \\
 %$N_c \rightarrow$ & 2 & 4 & 8 & 16 & 2 & 4 & 2 \\
 \hline
 \hline
 \multicolumn{5}{|c|}{Race (FP32 Accuracy = 37.51\%)} & \multicolumn{4}{|c|}{Boolq (FP32 Accuracy = 64.62\%)} \\ 
 \hline
 \hline
 64 & 36.94 & 37.13 & 36.27 & 37.13 & 63.73 & 62.26 & 63.49 & 63.36 \\
 \hline
 32 & 37.03 & 36.36 & 36.08 & 37.03 & 62.54 & 63.51 & 63.49 & 63.55  \\
 \hline
 16 & 37.03 & 37.03 & 36.46 & 37.03 & 61.1 & 63.79 & 63.58 & 63.33  \\
 \hline
 \hline
 \multicolumn{5}{|c|}{Winogrande (FP32 Accuracy = 58.01\%)} & \multicolumn{4}{|c|}{Piqa (FP32 Accuracy = 74.21\%)} \\ 
 \hline
 \hline
 64 & 58.17 & 57.22 & 57.85 & 58.33 & 73.01 & 73.07 & 73.07 & 72.80 \\
 \hline
 32 & 59.12 & 58.09 & 57.85 & 58.41 & 73.01 & 73.94 & 72.74 & 73.18  \\
 \hline
 16 & 57.93 & 58.88 & 57.93 & 58.56 & 73.94 & 72.80 & 73.01 & 73.94  \\
 \hline
\end{tabular}
\caption{\label{tab:mmlu_abalation} Accuracy on LM evaluation harness tasks on GPT3-1.3B model.}
\end{table}

\begin{table} \centering
\begin{tabular}{|c||c|c|c|c||c|c|c|c|} 
\hline
 $L_b \rightarrow$& \multicolumn{4}{c||}{8} & \multicolumn{4}{c||}{8}\\
 \hline
 \backslashbox{$L_A$\kern-1em}{\kern-1em$N_c$} & 2 & 4 & 8 & 16 & 2 & 4 & 8 & 16  \\
 %$N_c \rightarrow$ & 2 & 4 & 8 & 16 & 2 & 4 & 2 \\
 \hline
 \hline
 \multicolumn{5}{|c|}{Race (FP32 Accuracy = 41.34\%)} & \multicolumn{4}{|c|}{Boolq (FP32 Accuracy = 68.32\%)} \\ 
 \hline
 \hline
 64 & 40.48 & 40.10 & 39.43 & 39.90 & 69.20 & 68.41 & 69.45 & 68.56 \\
 \hline
 32 & 39.52 & 39.52 & 40.77 & 39.62 & 68.32 & 67.43 & 68.17 & 69.30  \\
 \hline
 16 & 39.81 & 39.71 & 39.90 & 40.38 & 68.10 & 66.33 & 69.51 & 69.42  \\
 \hline
 \hline
 \multicolumn{5}{|c|}{Winogrande (FP32 Accuracy = 67.88\%)} & \multicolumn{4}{|c|}{Piqa (FP32 Accuracy = 78.78\%)} \\ 
 \hline
 \hline
 64 & 66.85 & 66.61 & 67.72 & 67.88 & 77.31 & 77.42 & 77.75 & 77.64 \\
 \hline
 32 & 67.25 & 67.72 & 67.72 & 67.00 & 77.31 & 77.04 & 77.80 & 77.37  \\
 \hline
 16 & 68.11 & 68.90 & 67.88 & 67.48 & 77.37 & 78.13 & 78.13 & 77.69  \\
 \hline
\end{tabular}
\caption{\label{tab:mmlu_abalation} Accuracy on LM evaluation harness tasks on GPT3-8B model.}
\end{table}

\begin{table} \centering
\begin{tabular}{|c||c|c|c|c||c|c|c|c|} 
\hline
 $L_b \rightarrow$& \multicolumn{4}{c||}{8} & \multicolumn{4}{c||}{8}\\
 \hline
 \backslashbox{$L_A$\kern-1em}{\kern-1em$N_c$} & 2 & 4 & 8 & 16 & 2 & 4 & 8 & 16  \\
 %$N_c \rightarrow$ & 2 & 4 & 8 & 16 & 2 & 4 & 2 \\
 \hline
 \hline
 \multicolumn{5}{|c|}{Race (FP32 Accuracy = 40.67\%)} & \multicolumn{4}{|c|}{Boolq (FP32 Accuracy = 76.54\%)} \\ 
 \hline
 \hline
 64 & 40.48 & 40.10 & 39.43 & 39.90 & 75.41 & 75.11 & 77.09 & 75.66 \\
 \hline
 32 & 39.52 & 39.52 & 40.77 & 39.62 & 76.02 & 76.02 & 75.96 & 75.35  \\
 \hline
 16 & 39.81 & 39.71 & 39.90 & 40.38 & 75.05 & 73.82 & 75.72 & 76.09  \\
 \hline
 \hline
 \multicolumn{5}{|c|}{Winogrande (FP32 Accuracy = 70.64\%)} & \multicolumn{4}{|c|}{Piqa (FP32 Accuracy = 79.16\%)} \\ 
 \hline
 \hline
 64 & 69.14 & 70.17 & 70.17 & 70.56 & 78.24 & 79.00 & 78.62 & 78.73 \\
 \hline
 32 & 70.96 & 69.69 & 71.27 & 69.30 & 78.56 & 79.49 & 79.16 & 78.89  \\
 \hline
 16 & 71.03 & 69.53 & 69.69 & 70.40 & 78.13 & 79.16 & 79.00 & 79.00  \\
 \hline
\end{tabular}
\caption{\label{tab:mmlu_abalation} Accuracy on LM evaluation harness tasks on GPT3-22B model.}
\end{table}

\begin{table} \centering
\begin{tabular}{|c||c|c|c|c||c|c|c|c|} 
\hline
 $L_b \rightarrow$& \multicolumn{4}{c||}{8} & \multicolumn{4}{c||}{8}\\
 \hline
 \backslashbox{$L_A$\kern-1em}{\kern-1em$N_c$} & 2 & 4 & 8 & 16 & 2 & 4 & 8 & 16  \\
 %$N_c \rightarrow$ & 2 & 4 & 8 & 16 & 2 & 4 & 2 \\
 \hline
 \hline
 \multicolumn{5}{|c|}{Race (FP32 Accuracy = 44.4\%)} & \multicolumn{4}{|c|}{Boolq (FP32 Accuracy = 79.29\%)} \\ 
 \hline
 \hline
 64 & 42.49 & 42.51 & 42.58 & 43.45 & 77.58 & 77.37 & 77.43 & 78.1 \\
 \hline
 32 & 43.35 & 42.49 & 43.64 & 43.73 & 77.86 & 75.32 & 77.28 & 77.86  \\
 \hline
 16 & 44.21 & 44.21 & 43.64 & 42.97 & 78.65 & 77 & 76.94 & 77.98  \\
 \hline
 \hline
 \multicolumn{5}{|c|}{Winogrande (FP32 Accuracy = 69.38\%)} & \multicolumn{4}{|c|}{Piqa (FP32 Accuracy = 78.07\%)} \\ 
 \hline
 \hline
 64 & 68.9 & 68.43 & 69.77 & 68.19 & 77.09 & 76.82 & 77.09 & 77.86 \\
 \hline
 32 & 69.38 & 68.51 & 68.82 & 68.90 & 78.07 & 76.71 & 78.07 & 77.86  \\
 \hline
 16 & 69.53 & 67.09 & 69.38 & 68.90 & 77.37 & 77.8 & 77.91 & 77.69  \\
 \hline
\end{tabular}
\caption{\label{tab:mmlu_abalation} Accuracy on LM evaluation harness tasks on Llama2-7B model.}
\end{table}

\begin{table} \centering
\begin{tabular}{|c||c|c|c|c||c|c|c|c|} 
\hline
 $L_b \rightarrow$& \multicolumn{4}{c||}{8} & \multicolumn{4}{c||}{8}\\
 \hline
 \backslashbox{$L_A$\kern-1em}{\kern-1em$N_c$} & 2 & 4 & 8 & 16 & 2 & 4 & 8 & 16  \\
 %$N_c \rightarrow$ & 2 & 4 & 8 & 16 & 2 & 4 & 2 \\
 \hline
 \hline
 \multicolumn{5}{|c|}{Race (FP32 Accuracy = 48.8\%)} & \multicolumn{4}{|c|}{Boolq (FP32 Accuracy = 85.23\%)} \\ 
 \hline
 \hline
 64 & 49.00 & 49.00 & 49.28 & 48.71 & 82.82 & 84.28 & 84.03 & 84.25 \\
 \hline
 32 & 49.57 & 48.52 & 48.33 & 49.28 & 83.85 & 84.46 & 84.31 & 84.93  \\
 \hline
 16 & 49.85 & 49.09 & 49.28 & 48.99 & 85.11 & 84.46 & 84.61 & 83.94  \\
 \hline
 \hline
 \multicolumn{5}{|c|}{Winogrande (FP32 Accuracy = 79.95\%)} & \multicolumn{4}{|c|}{Piqa (FP32 Accuracy = 81.56\%)} \\ 
 \hline
 \hline
 64 & 78.77 & 78.45 & 78.37 & 79.16 & 81.45 & 80.69 & 81.45 & 81.5 \\
 \hline
 32 & 78.45 & 79.01 & 78.69 & 80.66 & 81.56 & 80.58 & 81.18 & 81.34  \\
 \hline
 16 & 79.95 & 79.56 & 79.79 & 79.72 & 81.28 & 81.66 & 81.28 & 80.96  \\
 \hline
\end{tabular}
\caption{\label{tab:mmlu_abalation} Accuracy on LM evaluation harness tasks on Llama2-70B model.}
\end{table}

%\section{MSE Studies}
%\textcolor{red}{TODO}


\subsection{Number Formats and Quantization Method}
\label{subsec:numFormats_quantMethod}
\subsubsection{Integer Format}
An $n$-bit signed integer (INT) is typically represented with a 2s-complement format \citep{yao2022zeroquant,xiao2023smoothquant,dai2021vsq}, where the most significant bit denotes the sign.

\subsubsection{Floating Point Format}
An $n$-bit signed floating point (FP) number $x$ comprises of a 1-bit sign ($x_{\mathrm{sign}}$), $B_m$-bit mantissa ($x_{\mathrm{mant}}$) and $B_e$-bit exponent ($x_{\mathrm{exp}}$) such that $B_m+B_e=n-1$. The associated constant exponent bias ($E_{\mathrm{bias}}$) is computed as $(2^{{B_e}-1}-1)$. We denote this format as $E_{B_e}M_{B_m}$.  

\subsubsection{Quantization Scheme}
\label{subsec:quant_method}
A quantization scheme dictates how a given unquantized tensor is converted to its quantized representation. We consider FP formats for the purpose of illustration. Given an unquantized tensor $\bm{X}$ and an FP format $E_{B_e}M_{B_m}$, we first, we compute the quantization scale factor $s_X$ that maps the maximum absolute value of $\bm{X}$ to the maximum quantization level of the $E_{B_e}M_{B_m}$ format as follows:
\begin{align}
\label{eq:sf}
    s_X = \frac{\mathrm{max}(|\bm{X}|)}{\mathrm{max}(E_{B_e}M_{B_m})}
\end{align}
In the above equation, $|\cdot|$ denotes the absolute value function.

Next, we scale $\bm{X}$ by $s_X$ and quantize it to $\hat{\bm{X}}$ by rounding it to the nearest quantization level of $E_{B_e}M_{B_m}$ as:

\begin{align}
\label{eq:tensor_quant}
    \hat{\bm{X}} = \text{round-to-nearest}\left(\frac{\bm{X}}{s_X}, E_{B_e}M_{B_m}\right)
\end{align}

We perform dynamic max-scaled quantization \citep{wu2020integer}, where the scale factor $s$ for activations is dynamically computed during runtime.

\subsection{Vector Scaled Quantization}
\begin{wrapfigure}{r}{0.35\linewidth}
  \centering
  \includegraphics[width=\linewidth]{sections/figures/vsquant.jpg}
  \caption{\small Vectorwise decomposition for per-vector scaled quantization (VSQ \citep{dai2021vsq}).}
  \label{fig:vsquant}
\end{wrapfigure}
During VSQ \citep{dai2021vsq}, the operand tensors are decomposed into 1D vectors in a hardware friendly manner as shown in Figure \ref{fig:vsquant}. Since the decomposed tensors are used as operands in matrix multiplications during inference, it is beneficial to perform this decomposition along the reduction dimension of the multiplication. The vectorwise quantization is performed similar to tensorwise quantization described in Equations \ref{eq:sf} and \ref{eq:tensor_quant}, where a scale factor $s_v$ is required for each vector $\bm{v}$ that maps the maximum absolute value of that vector to the maximum quantization level. While smaller vector lengths can lead to larger accuracy gains, the associated memory and computational overheads due to the per-vector scale factors increases. To alleviate these overheads, VSQ \citep{dai2021vsq} proposed a second level quantization of the per-vector scale factors to unsigned integers, while MX \citep{rouhani2023shared} quantizes them to integer powers of 2 (denoted as $2^{INT}$).

\subsubsection{MX Format}
The MX format proposed in \citep{rouhani2023microscaling} introduces the concept of sub-block shifting. For every two scalar elements of $b$-bits each, there is a shared exponent bit. The value of this exponent bit is determined through an empirical analysis that targets minimizing quantization MSE. We note that the FP format $E_{1}M_{b}$ is strictly better than MX from an accuracy perspective since it allocates a dedicated exponent bit to each scalar as opposed to sharing it across two scalars. Therefore, we conservatively bound the accuracy of a $b+2$-bit signed MX format with that of a $E_{1}M_{b}$ format in our comparisons. For instance, we use E1M2 format as a proxy for MX4.

\begin{figure}
    \centering
    \includegraphics[width=1\linewidth]{sections//figures/BlockFormats.pdf}
    \caption{\small Comparing LO-BCQ to MX format.}
    \label{fig:block_formats}
\end{figure}

Figure \ref{fig:block_formats} compares our $4$-bit LO-BCQ block format to MX \citep{rouhani2023microscaling}. As shown, both LO-BCQ and MX decompose a given operand tensor into block arrays and each block array into blocks. Similar to MX, we find that per-block quantization ($L_b < L_A$) leads to better accuracy due to increased flexibility. While MX achieves this through per-block $1$-bit micro-scales, we associate a dedicated codebook to each block through a per-block codebook selector. Further, MX quantizes the per-block array scale-factor to E8M0 format without per-tensor scaling. In contrast during LO-BCQ, we find that per-tensor scaling combined with quantization of per-block array scale-factor to E4M3 format results in superior inference accuracy across models. 


%%%%%%%%%%%%%%%%%%%%%%%%%%%%%%%%%%%%%%%%%%%%%%%%%%%%%%%%%%%%%%%%%%%%%%%%%%%%%%%
%%%%%%%%%%%%%%%%%%%%%%%%%%%%%%%%%%%%%%%%%%%%%%%%%%%%%%%%%%%%%%%%%%%%%%%%%%%%%%%


\end{document}


% This document was modified from the file originally made available by
% Pat Langley and Andrea Danyluk for ICML-2K. This version was created
% by Iain Murray in 2018, and modified by Alexandre Bouchard in
% 2019 and 2021 and by Csaba Szepesvari, Gang Niu and Sivan Sabato in 2022.
% Modified again in 2023 and 2024 by Sivan Sabato and Jonathan Scarlett.
% Previous contributors include Dan Roy, Lise Getoor and Tobias
% Scheffer, which was slightly modified from the 2010 version by
% Thorsten Joachims & Johannes Fuernkranz, slightly modified from the
% 2009 version by Kiri Wagstaff and Sam Roweis's 2008 version, which is
% slightly modified from Prasad Tadepalli's 2007 version which is a
% lightly changed version of the previous year's version by Andrew
% Moore, which was in turn edited from those of Kristian Kersting and
% Codrina Lauth. Alex Smola contributed to the algorithmic style files.
