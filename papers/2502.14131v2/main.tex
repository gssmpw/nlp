\documentclass[11pt,letterpaper,english]{article}

%%%%%%%%%%%%%%%%%%%%%%%%%%%%%%%%%%%%%%%%%%%%%%%%%%%%%%%%%%%%%%%%%%%%%%%%%%
%% Packages from both documents
%%%%%%%%%%%%%%%%%%%%%%%%%%%%%%%%%%%%%%%%%%%%%%%%%%%%%%%%%%%%%%%%%%%%%%%%%%

% Essential packages
\usepackage{amsmath}
\usepackage{amsthm}
\usepackage{amssymb}
\usepackage{graphicx}
\usepackage{hyperref}
\usepackage{natbib}
\usepackage{times}
\usepackage{setspace}
\usepackage{booktabs}
\usepackage{algorithm}
\usepackage{algpseudocode}
\usepackage{enumitem}
\usepackage{microtype}
\usepackage{tikz}
\usepackage{pgfplots}
\pgfplotsset{compat=1.17}
\usepackage{multirow}
\usepackage{soul}
\usepackage{caption}
\usepackage{subcaption}
\usepackage{float}
\usepackage{rotating}
\usepackage{array}
\usepackage{bm}
\usepackage{bbm}
\PassOptionsToPackage{table}{xcolor}
\usepackage{xcolor}
\usepackage{colortbl} 

\usepackage{pifont}
\usepackage[bottom]{footmisc}
\usepackage{wrapfig}
\usepackage[makeroom]{cancel}
\usepackage[edges]{forest}
\usepackage{makecell}
\usepackage{longtable}
\usepackage{dsfont}
\usepackage{comment}
\usepackage{blkarray}
\usepackage{stackengine}
\usepackage{accents}
\usepackage[utf8]{inputenc}
\usepackage{epstopdf}
\usepackage{xfrac}
\usepackage{fullpage}

% TikZ libraries
\usetikzlibrary{arrows}
\usetikzlibrary{positioning}

% Configure bibliography
\bibpunct[, ]{(}{)}{,}{a}{}{,}
\def\bibfont{\small}
\def\bibsep{\smallskipamount}
\def\bibhang{24pt}
\def\newblock{\ }
\def\BIBand{and}

% Line spacing
\onehalfspacing

% Math commands from first document
\newcommand{\cmark}{\ding{51}} % Checkmark
\newcommand{\xmark}{\ding{55}} % Crossmark
\newcommand{\lmark}{$\bigtriangleup$} 
\newcommand{\lbe}{\mathcal{L}_{BE}}

\newcommand*{\QED}[1][$\square$]{%
\leavevmode\unskip\penalty9999 \hbox{}\nobreak\hfill
    \quad\hbox{#1}%
}

% Math commands from second document
\newcommand{\todo}[1]{\noindent{\textcolor{red}{\{TODO:  #1\}}}}
\newcommand{\norm}[1]{\left\lVert#1\right\rVert}
\newcommand{\indep}{\perp \!\!\! \perp}
\newcommand{\argmax}{\arg\!\max}
\DeclareMathOperator*{\argmin}{arg\,min}
\def\ss#1{{\bf \color{red}#1}}
\def\thetaa{\theta_\alpha}
\def\bP{\mathbb{P}}
\def\bE{\mathbb{E}}
\def\bW{\mathbf{W}}
\def\bC{\mathbf{C}}
\def\bb{\mathbf{b}}
\def\bm{\textbf{m}}
\def\bx{\textbf{x}}
\def\bg{\textbf{g}}
\def\bd{\textbf{d}}
\def\bY{\textbf{Y}}
\def\cA{\mathcal{A}}
\def\bZ{\mathbf{Z}}
\def\bzero{\textbf{0}}
\def\tm{\text{-}}
\def\bTheta{\boldsymbol{\Theta}}
\def\thetab{\theta_\beta}
\def\btau{\boldsymbol{\tau}}
\def\sm#1{{\bf \color{red}#1}}
\def\hy#1{{\bf \color{blue}#1}}
\def\cents{\hbox{\rm\rlap/c}}
\def\aps{\textsc{\char13}}

% Custom list environments
\newcommand{\squishlist}{
   \begin{list}{$\bullet$}
    { \setlength{\itemsep}{0pt} \setlength{\parsep}{1pt}
      \setlength{\topsep}{1pt} \setlength{\partopsep}{1pt}
      \setlength{\leftmargin}{1.5em} \setlength{\labelwidth}{1em}
      \setlength{\labelsep}{0.5em} } }

\newcommand{\squishlisttwo}{
   \begin{list}{$\bullet$}
    { \setlength{\itemsep}{0pt} \setlength{\parsep}{0pt}
      \setlength{\topsep}{0pt} \setlength{\partopsep}{0pt}
      \setlength{\leftmargin}{1em} \setlength{\labelwidth}{1.5em}
      \setlength{\labelsep}{0.5em} } }

\newcommand{\squishend}{
    \end{list}  }

% Theorem environments from both documents
\newtheorem{thm}{Theorem}[section]
\newtheorem{lem}[thm]{Lemma}
\newtheorem{cor}[thm]{Corollary}
\newtheorem{prop}{Proposition}[section]
\newtheorem{asmp}{Assumption}[section]
\newtheorem{defn}{Definition}[section]
\newtheorem{fact}{Fact}[section]
\newtheorem{conj}{Conjecture}[section]
\newtheorem{rem}{Remark}[section]
\newtheorem{challenge}{Challenge}
\newtheorem{remark}{Remark}
\newtheorem{assumption}{Assumption}
\newtheorem{claim}{Claim}

% For DVI -> PNG conversion (from second document)
\DeclareGraphicsRule{.tif}{png}{.png}{`convert #1 `dirname #1`/`basename #1 .tif`.png}

% Math alphabet declaration from first document
\DeclareMathAlphabet{\mathcalligra}{T1}{calligra}{m}{n}

% Appendix setup
\usepackage{appendix}

% Title and author information
\title{Gradients can train reward models:
\\
An Empirical Risk Minimization Approach for \\Offline Inverse RL and Dynamic Discrete Choice Model}

\author{
Enoch H. Kang\\
Foster School of Business, University of Washington\\
\texttt{ehwkang@uw.edu}
\and
Hema Yoganarasimhan\\
Foster School of Business, University of Washington\\
\texttt{hemay@uw.edu}
\and
Lalit Jain\\
Foster School of Business, University of Washington\\
\texttt{lalitj@uw.edu}
}

\date{\today}

\begin{document}

\maketitle

\begin{abstract}
We study the problem of estimating Dynamic Discrete Choice (DDC) models, also known as offline Maximum Entropy-Regularized Inverse Reinforcement Learning (offline MaxEnt-IRL) in machine learning. The objective is to recover reward or $Q$ functions that govern agent behavior from offline behavior data. In this paper, we propose a globally convergent gradient-based method for solving these problems without the restrictive assumption of linearly parameterized rewards. The novelty of our approach lies in introducing the Empirical Risk Minimization (ERM) based IRL/DDC framework, which circumvents the need for explicit state transition probability estimation in the Bellman equation. Furthermore, our method is compatible with non-parametric estimation techniques such as neural networks. Therefore, the proposed method has the potential to be scaled to high-dimensional, infinite state spaces. A key theoretical insight underlying our approach is that the Bellman residual satisfies the Polyak-Łojasiewicz (PL) condition--a property that, while weaker than strong convexity, is sufficient to ensure fast global convergence guarantees. Through a series of synthetic experiments, we demonstrate that our approach consistently outperforms benchmark methods and state-of-the-art alternatives.
\end{abstract}

\textbf{Keywords:} Dynamic Discrete Choice, Offline Inverse Reinforcement Learning, Gradient-based methods, Empirical Risk Minimization, Machine learning


% Main content sections
%\section{Introduction}
\section{Introduction}


\begin{figure}[t]
\centering
\includegraphics[width=0.6\columnwidth]{figures/evaluation_desiderata_V5.pdf}
\vspace{-0.5cm}
\caption{\systemName is a platform for conducting realistic evaluations of code LLMs, collecting human preferences of coding models with real users, real tasks, and in realistic environments, aimed at addressing the limitations of existing evaluations.
}
\label{fig:motivation}
\end{figure}

\begin{figure*}[t]
\centering
\includegraphics[width=\textwidth]{figures/system_design_v2.png}
\caption{We introduce \systemName, a VSCode extension to collect human preferences of code directly in a developer's IDE. \systemName enables developers to use code completions from various models. The system comprises a) the interface in the user's IDE which presents paired completions to users (left), b) a sampling strategy that picks model pairs to reduce latency (right, top), and c) a prompting scheme that allows diverse LLMs to perform code completions with high fidelity.
Users can select between the top completion (green box) using \texttt{tab} or the bottom completion (blue box) using \texttt{shift+tab}.}
\label{fig:overview}
\end{figure*}

As model capabilities improve, large language models (LLMs) are increasingly integrated into user environments and workflows.
For example, software developers code with AI in integrated developer environments (IDEs)~\citep{peng2023impact}, doctors rely on notes generated through ambient listening~\citep{oberst2024science}, and lawyers consider case evidence identified by electronic discovery systems~\citep{yang2024beyond}.
Increasing deployment of models in productivity tools demands evaluation that more closely reflects real-world circumstances~\citep{hutchinson2022evaluation, saxon2024benchmarks, kapoor2024ai}.
While newer benchmarks and live platforms incorporate human feedback to capture real-world usage, they almost exclusively focus on evaluating LLMs in chat conversations~\citep{zheng2023judging,dubois2023alpacafarm,chiang2024chatbot, kirk2024the}.
Model evaluation must move beyond chat-based interactions and into specialized user environments.



 

In this work, we focus on evaluating LLM-based coding assistants. 
Despite the popularity of these tools---millions of developers use Github Copilot~\citep{Copilot}---existing
evaluations of the coding capabilities of new models exhibit multiple limitations (Figure~\ref{fig:motivation}, bottom).
Traditional ML benchmarks evaluate LLM capabilities by measuring how well a model can complete static, interview-style coding tasks~\citep{chen2021evaluating,austin2021program,jain2024livecodebench, white2024livebench} and lack \emph{real users}. 
User studies recruit real users to evaluate the effectiveness of LLMs as coding assistants, but are often limited to simple programming tasks as opposed to \emph{real tasks}~\citep{vaithilingam2022expectation,ross2023programmer, mozannar2024realhumaneval}.
Recent efforts to collect human feedback such as Chatbot Arena~\citep{chiang2024chatbot} are still removed from a \emph{realistic environment}, resulting in users and data that deviate from typical software development processes.
We introduce \systemName to address these limitations (Figure~\ref{fig:motivation}, top), and we describe our three main contributions below.


\textbf{We deploy \systemName in-the-wild to collect human preferences on code.} 
\systemName is a Visual Studio Code extension, collecting preferences directly in a developer's IDE within their actual workflow (Figure~\ref{fig:overview}).
\systemName provides developers with code completions, akin to the type of support provided by Github Copilot~\citep{Copilot}. 
Over the past 3 months, \systemName has served over~\completions suggestions from 10 state-of-the-art LLMs, 
gathering \sampleCount~votes from \userCount~users.
To collect user preferences,
\systemName presents a novel interface that shows users paired code completions from two different LLMs, which are determined based on a sampling strategy that aims to 
mitigate latency while preserving coverage across model comparisons.
Additionally, we devise a prompting scheme that allows a diverse set of models to perform code completions with high fidelity.
See Section~\ref{sec:system} and Section~\ref{sec:deployment} for details about system design and deployment respectively.



\textbf{We construct a leaderboard of user preferences and find notable differences from existing static benchmarks and human preference leaderboards.}
In general, we observe that smaller models seem to overperform in static benchmarks compared to our leaderboard, while performance among larger models is mixed (Section~\ref{sec:leaderboard_calculation}).
We attribute these differences to the fact that \systemName is exposed to users and tasks that differ drastically from code evaluations in the past. 
Our data spans 103 programming languages and 24 natural languages as well as a variety of real-world applications and code structures, while static benchmarks tend to focus on a specific programming and natural language and task (e.g. coding competition problems).
Additionally, while all of \systemName interactions contain code contexts and the majority involve infilling tasks, a much smaller fraction of Chatbot Arena's coding tasks contain code context, with infilling tasks appearing even more rarely. 
We analyze our data in depth in Section~\ref{subsec:comparison}.



\textbf{We derive new insights into user preferences of code by analyzing \systemName's diverse and distinct data distribution.}
We compare user preferences across different stratifications of input data (e.g., common versus rare languages) and observe which affect observed preferences most (Section~\ref{sec:analysis}).
For example, while user preferences stay relatively consistent across various programming languages, they differ drastically between different task categories (e.g. frontend/backend versus algorithm design).
We also observe variations in user preference due to different features related to code structure 
(e.g., context length and completion patterns).
We open-source \systemName and release a curated subset of code contexts.
Altogether, our results highlight the necessity of model evaluation in realistic and domain-specific settings.







\putsec{related}{Related Work}

\noindent \textbf{Efficient Radiance Field Rendering.}
%
The introduction of Neural Radiance Fields (NeRF)~\cite{mil:sri20} has
generated significant interest in efficient 3D scene representation and
rendering for radiance fields.
%
Over the past years, there has been a large amount of research aimed at
accelerating NeRFs through algorithmic or software
optimizations~\cite{mul:eva22,fri:yu22,che:fun23,sun:sun22}, and the
development of hardware
accelerators~\cite{lee:cho23,li:li23,son:wen23,mub:kan23,fen:liu24}.
%
The state-of-the-art method, 3D Gaussian splatting~\cite{ker:kop23}, has
further fueled interest in accelerating radiance field
rendering~\cite{rad:ste24,lee:lee24,nie:stu24,lee:rho24,ham:mel24} as it
employs rasterization primitives that can be rendered much faster than NeRFs.
%
However, previous research focused on software graphics rendering on
programmable cores or building dedicated hardware accelerators. In contrast,
\name{} investigates the potential of efficient radiance field rendering while
utilizing fixed-function units in graphics hardware.
%
To our knowledge, this is the first work that assesses the performance
implications of rendering Gaussian-based radiance fields on the hardware
graphics pipeline with software and hardware optimizations.

%%%%%%%%%%%%%%%%%%%%%%%%%%%%%%%%%%%%%%%%%%%%%%%%%%%%%%%%%%%%%%%%%%%%%%%%%%
\myparagraph{Enhancing Graphics Rendering Hardware.}
%
The performance advantage of executing graphics rendering on either
programmable shader cores or fixed-function units varies depending on the
rendering methods and hardware designs.
%
Previous studies have explored the performance implication of graphics hardware
design by developing simulation infrastructures for graphics
workloads~\cite{bar:gon06,gub:aam19,tin:sax23,arn:par13}.
%
Additionally, several studies have aimed to improve the performance of
special-purpose hardware such as ray tracing units in graphics
hardware~\cite{cho:now23,liu:cha21} and proposed hardware accelerators for
graphics applications~\cite{lu:hua17,ram:gri09}.
%
In contrast to these works, which primarily evaluate traditional graphics
workloads, our work focuses on improving the performance of volume rendering
workloads, such as Gaussian splatting, which require blending a huge number of
fragments per pixel.

%%%%%%%%%%%%%%%%%%%%%%%%%%%%%%%%%%%%%%%%%%%%%%%%%%%%%%%%%%%%%%%%%%%%%%%%%%
%
In the context of multi-sample anti-aliasing, prior work proposed reducing the
amount of redundant shading by merging fragments from adjacent triangles in a
mesh at the quad granularity~\cite{fat:bou10}.
%
While both our work and quad-fragment merging (QFM)~\cite{fat:bou10} aim to
reduce operations by merging quads, our proposed technique differs from QFM in
many aspects.
%
Our method aims to blend \emph{overlapping primitives} along the depth
direction and applies to quads from any primitive. In contrast, QFM merges quad
fragments from small (e.g., pixel-sized) triangles that \emph{share} an edge
(i.e., \emph{connected}, \emph{non-overlapping} triangles).
%
As such, QFM is not applicable to the scenes consisting of a number of
unconnected transparent triangles, such as those in 3D Gaussian splatting.
%
In addition, our method computes the \emph{exact} color for each pixel by
offloading blending operations from ROPs to shader units, whereas QFM
\emph{approximates} pixel colors by using the color from one triangle when
multiple triangles are merged into a single quad.


\newcommand{\tabincell}[2]{\begin{tabular}{@{}#1@{}}#2\end{tabular}}
\newcommand{\rowstyle}[1]{\gdef\currentrowstyle{#1}%
	#1\ignorespaces
}

\newcommand{\className}[1]{\textbf{\sf #1}}
\newcommand{\functionName}[1]{\textbf{\sf #1}}
\newcommand{\objectName}[1]{\textbf{\sf #1}}
\newcommand{\annotation}[1]{\textbf{#1}}
\newcommand{\todo}[1]{\textcolor{blue}{\textbf{[[TODO: #1]]}}}
\newcommand{\change}[1]{\textcolor{blue}{#1}}
\newcommand{\fetch}[1]{\textbf{\em #1}}
\newcommand{\phead}[1]{\vspace{1mm} \noindent {\bf #1}}
\newcommand{\wei}[1]{\textcolor{blue}{{\it [Wei says: #1]}}}
\newcommand{\peter}[1]{\textcolor{red}{{\it [Peter says: #1]}}}
\newcommand{\zhenhao}[1]{\textcolor{dkblue}{{\it [Zhenhao says: #1]}}}
\newcommand{\feng}[1]{\textcolor{magenta}{{\it [Feng says: #1]}}}
\newcommand{\jinqiu}[1]{\textcolor{red}{{\it [Jinqiu says: #1]}}}
\newcommand{\shouvick}[1]{\textcolor{violet(ryb)}{{\it [Shouvick says: #1]}}}
\newcommand{\pattern}[1]{\emph{#1}}
%\newcommand{\tool}{{{DectGUILag}}\xspace}
\newcommand{\tool}{{{GUIWatcher}}\xspace}


\newcommand{\guo}[1]{\textcolor{yellow}{{\it [Linqiang says: #1]}}}

\newcommand{\rqbox}[1]{\begin{tcolorbox}[left=4pt,right=4pt,top=4pt,bottom=4pt,colback=gray!5,colframe=gray!40!black,before skip=2pt,after skip=2pt]#1\end{tcolorbox}}

\section{ERM-IRL (ERM-DDC) framework}\label{sec:ERM-IRL}
\subsection{Identification via expected risk minimization}

%The goal of IRL (or DDC) is slightly different: we would like to find $\tilde{r} \in \mathcal{R} \subseteq \mathbb{R}^{\bar{\mathcal{S}} \times \mathcal{A}}$ such that
%\begin{align}
 %   \underset{\tilde{r}\in \mathcal{R}}{\operatorname{argmin}} \; \mathbb{E}_{(s,a)\sim \nu_0, \pi^*}[(\tilde{r}(s,a)-r(s,a))^2] \tag{Equation \ref{eq:rObjective}}
%\end{align}

%In Theorem \ref{thm:MagnacThesmar}, we saw that $Q^*$ and $r(s,a)$ can be uniquely identified by solving the system of equations shown in Equation \eqref{eq:HotzMillereqs} for all $s\in\mathcal{S}$ and $a\in\mathcal{A}$. 

We now propose a one-shot Empirical Risk Minimization framework (ERM-IRL/ERM-DDC) to solve the IRL problem stated in Definition \ref{def:IRLproblem}. First, we recast the IRL problem as the following \textit{expected risk} minimization problem under infinite data regime.

\begin{defn}[Expected risk minimization problem] The expected risk minimization problem is defined as the problem of finding $Q$ that minimizes  the expected risk $\mathcal{R}_{exp}(Q)$, which is defined as
\begin{align}
  &\mathcal{R}_{exp}(Q):= \mathbb{E}_{(s, a)\sim \pi^*, \nu_0}  \left[\mathcal{L}_{NLL}(Q)(s,a) + \lambda \mathbbm{1}_{a = a_s} \mathcal{L}_{BE}(Q)(s,a)\right] \! \notag \noindent
  \\
  & = \mathbb{E}_{(s, a)\sim \pi^*, \nu_0}\bigl[-\log\left(\hat{p}_{Q}(a
\mid s)\right) +  \lambda \mathbbm{1}_{a = a_s} \left( \mathcal{T}Q(s, a) - Q(s, a) \right)^2 \bigr] \label{eq:mainopt}
\end{align}
\noindent where $a_s$ is defined in Assumption \ref{ass:anchor}.
\end{defn}


\noindent\textbf{Remark.} The joint minimization of the NLL term and BE term is the key novelty in our approach. Prior work on the IRL and DDC literature \citep{hotz1993conditional, zeng2023understanding} typically minimizes the log-likelihood of the observed choice probabilities (the NLL term), given observed or estimated state transition probabilities. The standard solution is to first estimate/assume state transition probabilities, then obtain estimates of future value functions, plug them into the choice probability, and then minimize NLL term. In contrast, our recast problem avoids the estimation of state-transition probabilities and instead jointly minimizes the NLL term along with the Bellman-error term. This is particularly helpful in large-state spaces since the estimation of state-transition probabilities can be infeasible/costly in such settings. In Theorem \ref{thm:mainopt}, we show that the solution to our recast problem in Equation \eqref{eq:mainopt} identifies the reward function. 

\begin{thm}[Identification through expected risk minimization]
\label{thm:mainopt} 
\;
\\
The solution to the expected risk minimization problem (Equation \eqref{eq:mainopt}) with any $\lambda>0$
uniquely identifies $Q^\ast$ up to $s\in\bar{\mathcal{S}}
$ and $a \in \mathcal{A}$, i.e., finds $\widehat{Q}$ that satisfies $\widehat{Q}(s,a)=Q^\ast(s,a)$ for  $s\in\bar{\mathcal{S}}
$ and $a \in \mathcal{A}$. Furthermore, we can uniquely identify $r$ up to $s\in\bar{\mathcal{S}}$ and $a \in \mathcal{A}$ by $r(s, a)= \widehat{Q}(s, a) -  \beta \cdot \mathbb{E}_{s^{\prime} \sim P(s, a)} 
    \bigl[ V_{\widehat{Q}} \bigr]$.
\end{thm}
\noindent Essentially, Theorem \ref{thm:mainopt} ensures that solving Equation \eqref{eq:mainopt} gives the exact $r$ and $Q^\ast$ up to 
$\bar{\mathcal{S}}$ and thus provides the solution to the IRL problem defined in Definition \ref{def:IRLproblem}. See Appendix \ref{sec:pfOfmainOpt} for the proof. 

\subsubsection*{Comparison with Imitation Learning} 
Having established the identification guarantees for the ERM-IRL/DDC framework, it is natural to compare this formulation to the identification properties of Imitation Learning (IL). Unlike IRL, which seeks to infer the underlying reward function that explains expert behavior, IL directly aims to recover the expert policy without modeling the transition dynamics. The objective of imitation learning is often defined to as finding policy $\hat{p}$ with 
$$
\min _{\hat{p}} \mathbb{E}_{(s, a) \sim \pi^\ast, \nu_0}\left[\ell\left(\hat{p}(a \mid s), \pi^\ast(a \mid s)\right)\right], \text{$\ell$ is the cross-entropy loss}
$$
or equivalently, 
\begin{align}
\min _{\hat{p}} \mathbb{E}_{(s, a) \sim \pi^\ast , \nu_0}\left[-\log \hat{p}(a|s)\right]\label{eq:mleEqBC}    
\end{align}
Equation \eqref{eq:mleEqBC} is exactly what a typical Behavioral Cloning (BC) \citep{torabi2018behavioral} minimizes under entropy regularization, as the objective of BC is
\begin{align}
  & \!\!\underset{Q\in \mathcal{Q}}{\min }  \;\mathbb{E}_{(s, a)\sim \pi^*, \nu_0}  \left[-\log \hat{p}_Q(a|s)\right] \label{eq:BC}
\end{align}



\noindent where $\hat{p}_Q(a\mid s) = \frac{Q(s,a)}{\sum_{\tilde{a}\in\mathcal{A} }Q(s,\tilde{a})}$. Note that the solution set of Equation \eqref{eq:BC} fully contains the solution set of the ERM-IRL/DDC objective. This means that any solution to the ERM-IRL/DDC problem also minimizes the imitation learning objective, but not necessarily vice versa. Consequently, under entropy regularization, the IL objective is fundamentally easier to solve than the offline IRL/DDC problem, as it only requires minimizing the negative log-likelihood term without enforcing Bellman consistency. One of the key contributions of this paper is to formally establish and clarify this distinction: IL operates within a strictly simpler optimization landscape than the offline IRL/DDC, making it a computationally and statistically more tractable problem. This distinction further underscores the advantage of Behavioral Cloning (BC) over ERM-IRL/DDC for imitation learning (IL) tasks—since BC does not require modeling transition dynamics or solving an optimization problem involving the Bellman residual, it benefits from significantly lower computational and statistical complexity, making it a more efficient approach for IL.



\subsection{Estimation via minimax-formulated empirical risk minimization}
\label{sec:DoubleSampling}
While the idea of expected risk minimization -- minimizing Equation \eqref{eq:mainopt} -- is straightforward, empirically approximating $\mathcal{L}_{B E}(Q)(s, a) = (\mathcal{T} Q(s, a)-Q(s, a))^2$ and its gradient is quite challenging. 
As discussed in Section \ref{sec:BE&TD}, $\mathcal{T}Q$ is not available unless we know the transition function. As a result, we have to rely on an estimate of $\mathcal{T}$. A natural choice, common in TD-methods, is $\hat{\mathcal{T}} Q\left(s, a, s^{\prime}\right)=r(s, a)+\beta \cdot V_Q(s^\prime)$ which is computable given $Q$ and data $\mathcal{D}$. Thus, a natural proxy objective to minimize is:
\begin{align}
    &\mathbb{E}_{s' \sim P(s, a)} [\mathcal{L}_{\mathrm{TD}}(Q)\left(s, a, s^{\prime}\right)] :=\mathbb{E}_{s' \sim P(s, a)} [(\hat{\mathcal{T}} Q\left(s, a, s^{\prime}\right)-Q(s, a))^2] \notag
\end{align}
Temporal Difference (TD) methods typically use stochastic approximation to obtain an estimate of this proxy objective \citep{tesauro1995temporal, adusumilli2019temporal}. However, the issue with TD methods is that minimizing the proxy objective will not minimize the Bellman error in general (see Appendix \ref{sec:BiconjProofs} for details), because of the extra variance term, as shown below. 
\begin{align}
&\mathbb{E}_{s' \sim P(s, a)} 
\bigl[\mathcal{L}_{TD}(Q)(s, a, s^\prime)\bigr] = \mathcal{L}_{BE}(Q)(s, a) + \mathbb{E}_{s' \sim P(s, a)} 
\bigl[(\mathcal{T}Q(s, a) - \hat{\mathcal{T}}Q(s, a, s^\prime))^2\bigr]    \notag
\end{align}
As defined, $\hat{\mathcal{T}}$ is a one-step estimator, and the second term in the above equation does not vanish even in infinite data regimes. So, simply using the TD approach to approximate squared Bellman error provides a biased estimate. Intuitively, this problem happens because expectation and square are not exchangeable, i.e., 
$
\mathbb{E}_{ s^\prime \sim P(s, a)}\left[\delta_{Q}\left(s,a, s^\prime\right)\mid s, a\right]^2 \neq \mathbb{E}_{ s^\prime \sim P(s, a)}\left[\delta_{Q}\left(s,a, s^\prime\right)^2\mid s, a\right]
$. To remove this problematic square term, we employ an approach often referred to as the ``Bi-Conjugate Trick'' which replaces a square function by a linear function called the bi-conjugate:
\begin{align}
     \mathcal{L}_{BE}(s,a)(Q)&:=\mathbb{E}_{ s^\prime \sim P(s, a)}\left[\delta_{Q}\left(s,a, s^\prime\right)\mid s, a\right]^2\notag
     \\
     &=\max_{h\in \mathbb{R}}2\cdot\mathbb{E}_{ s^\prime \sim P(s, a)}\left[\delta_{Q}\left(s,a, s^\prime\right)\mid s, a\right]\cdot h-h^2 \notag
\end{align}

\noindent By further re-parametrizing $h$ using $\zeta = h - r + Q(s,a)$, after some algebra, we arrive at Lemma \ref{lem:OurBiconj}. (See Appendix \ref{sec:BiconjProofs} for the detailed derivation.)

\begin{lem}
\label{lem:OurBiconj}
\;\\
(a) We can express the squared Bellman error as
\begin{align}
    \mathcal{L}_{BE}(Q)(s, a)&:=(\mathcal{T} Q(s, a)-Q(s, a))^2 \notag
    \\
    &=\mathbb{E}_{s^\prime \sim P(s, a)} 
    \bigl[\mathcal{L}_{TD}(s, a, s^\prime)(Q)\bigr] - \beta^2 D(Q)(s, a) \label{eq:OurBiconj}
\end{align}
where
\begin{align}
    D(Q)(s, a):=\min _{\zeta \in \mathbb{R}} \mathbb{E}_{s^{\prime} \sim P(s, a)}\left[\left(V_{Q}\left(s^{\prime}\right)-\zeta\right)^2 \mid s, a\right]\label{eq:D(Q)}
\end{align}
(b) Define the minimizer (over all states and actions) of objective \eqref{eq:D(Q)} as
$$ \zeta^*: (s,a)\mapsto \arg\min_{\zeta\in \mathbb{R}}\mathbb{E}_{ s^\prime \sim P(s, a)}\left[\left(V^\ast(s^\prime)- \zeta\right)^2\mid s, a\right]$$
then
$r(s,a) = Q^*(s,a)-\beta \zeta^*(s,a)$.
\end{lem}
\noindent The reformulation of $\lbe$ proposed in Lemma \ref{lem:OurBiconj} enjoys the advantage of minimizing the squared TD-error ($\mathcal{L}_{TD}$) but without bias. Combining Theorem \ref{thm:mainopt} and Lemma \ref{lem:OurBiconj}, we arrive at the following Theorem \ref{thm:algoEQ}, which gives the expected risk minimization formulation of IRL we propose.

\begin{thm}
\label{thm:algoEQ} $Q^*$ is uniquely identified by expected risk minimization, i.e.,  
\begin{align}
     &\underset{Q\in \mathcal{Q}}{\min }  \;\mathcal{R}_{exp}(Q) \notag
     \\
     &=\min _{Q \in \mathcal{Q}} \mathbb{E}_{(s, a) \sim \pi^*, \nu_0}\bigl[{\mathcal{L}_{N L L}(Q)(s, a)} +\lambda \mathbbm{1}_{a=a_s} \bigl\{\mathbb{E}_{s^{\prime} \sim P(s, a)}\left[{\mathcal{L}_{T D}(Q)\left(s, a, s^{\prime}\right)}\right]-\beta^2 {D(Q)(s, a)}\bigr\} \notag
     \\
    &=\min _{Q \in \mathcal{Q}}\max_{\zeta\in \mathbb{R}^{S\times A}} \mathbb{E}_{(s, a) \sim \pi^*, \nu_0, s^{\prime} \sim P(s, a)}\bigl[\underbrace{\textcolor{blue}{-\log \left(\hat{p}_Q(a \mid s)\right)}}_{1)} + \lambda\mathbbm{1}_{a=a_s}\bigl\{\underbrace{\textcolor{red}{\bigl(\hat{\mathcal{T}} Q\left(s, a, s^{\prime}\right)-Q(s, a)\bigr)^2}}_{{2)}} \notag
    \\
    & \quad -\beta^2 \underbrace{ \bigl(\textcolor{orange}{\left(V_{Q}\left(s^{\prime}\right)-\zeta(s,a)\right)^2} }_{{3)}}\bigr\} \bigr]\label{eq:AlgoOpt}
\end{align}

Furthermore, $r(s, a)=Q^*(s, a) - \beta \zeta^*(s, a)$ where $\zeta^*$ is defined in Lemma \ref{lem:OurBiconj}.
\end{thm}

\noindent Equation \eqref{eq:AlgoOpt} in Theorem \ref{thm:algoEQ} is a mini-max problem in terms of $Q\in\mathcal{Q}$ and the introduced dual function $\zeta\in \mathbb{R}^{S\times A}$. To summarize, term 1) is the negative log-likelihood equation, term 2) is the TD error, and term 3) introduces a dual function $\zeta$.
The introduction of the dual function $\zeta$ in term 3) may seem a bit strange. 
In particular, note that $\arg\max_{\zeta \in \mathbb{R}} -\mathbb{E}_{s^{\prime} \sim P(s, a)}\left[\left(V_Q\left(s^{\prime}\right)-\zeta\right)^2 \mid s, a\right]$ is just $\zeta = \mathbb{E}_{s'\sim P(s,a)}[V\left(s^{\prime}\right)|s,a]$. 
However, we do not have access to the transition kernel and the state and action spaces may be large. 
Instead, we think of $\zeta$ as a function of states and actions, $\zeta(s,a)$ as introduced in Lemma~\ref{lem:OurBiconj}. This parametrization allows us to optimize over a class of functions containing $\zeta(s,a)$ directly. 

Given the minimax resolution for the expected risk minimization problem in Theorem \ref{thm:algoEQ} finds $Q$ under an infinite number of data, we are now ready to discuss the case when we are only given a finite dataset $\mathcal{D}$ instead. In this case, we solve the \textit{empirical risk minimization} problem.

\begin{defn}[Empirical risk minimization problem]
\label{def:ERM}
Given $N := |\mathcal{D}|$ where $\mathcal{D}$ is a finite dataset. An empirical risk minimization problem is defined as the problem of finding $Q$ that minimizes the empirical risk $\mathcal{R}_{emp}(Q;\mathcal{D})$, which is defined as
\begin{align}
     &\;\mathcal{R}_{emp}(Q;\mathcal{D}):=\max_{\zeta\in \mathbb{R}^{S\times A}}\frac{1}{N}  \sum_{(s,a,s^\prime)\in \mathcal{D}}\notag
    \\
    &\bigl[\textcolor{blue}{-\log \left(\hat{p}_Q(a \mid s)\right)} + 
   \lambda\mathbbm{1}_{a=a_s}\bigl\{\textcolor{red}{\bigl(\hat{\mathcal{T}} Q\left(s, a, s^{\prime}\right)-Q(s, a)\bigr)^2} -\beta^2  \textcolor{orange}{\left(V_{Q}\left(s^{\prime}\right)-\zeta(s,a)\right)^2} \bigr\} \bigr]\notag
   \\
   &= \frac{1}{N}\bigl[\sum_{(s,a,s^\prime)\in \mathcal{D}}\bigl(\textcolor{blue}{-\log \left(\hat{p}_Q(a \mid s)\right)}\bigr)+ 
\lambda\mathbbm{1}_{a=a_s}\notag
    \\
    &\bigl(  \sum_{(s,a,s^\prime)\in \mathcal{D}} \textcolor{red}{\bigl(\hat{\mathcal{T}} Q\left(s, a, s^{\prime}\right)-Q(s, a)\bigr)^2}  -\beta^2 \min_{\zeta\in \mathbb{R}^{S\times A}} 
   \sum_{(s,a,s^\prime)\in \mathcal{D}} \textcolor{orange}{\left(V_{Q}\left(s^{\prime}\right)-\zeta(s,a)\right)^2}\bigr) \bigr] \label{eq:EmpiricalERMIRL}
\end{align}
\end{defn}


\begin{algorithm}[ht!]
\caption{\textit{NovelSelect}}
\label{alg:novelselect}
\begin{algorithmic}[1]
\State \textbf{Input:} Data pool $\mathcal{X}^{all}$, data budget $n$
\State Initialize an empty dataset, $\mathcal{X} \gets \emptyset$
\While{$|\mathcal{X}| < n$}
    \State $x^{new} \gets \arg\max_{x \in \mathcal{X}^{all}} v(x)$
    \State $\mathcal{X} \gets \mathcal{X} \cup \{x^{new}\}$
    \State $\mathcal{X}^{all} \gets \mathcal{X}^{all} \setminus \{x^{new}\}$
\EndWhile
\State \textbf{return} $\mathcal{X}$
\end{algorithmic}
\end{algorithm}

\section{Experiments}
\label{sec:Experiments} 

We conduct several experiments across different problem settings to assess the efficiency of our proposed method. Detailed descriptions of the experimental settings are provided in \cref{sec:apendix_experiments}.
%We conduct experiments on optimizing PINNs for convection, wave PDEs, and a reaction ODE. 
%These equations have been studied in previous works investigating difficulties in training PINNs; we use the formulations in \citet{krishnapriyan2021characterizing, wang2022when} for our experiments. 
%The coefficient settings we use for these equations are considered challenging in the literature \cite{krishnapriyan2021characterizing, wang2022when}.
%\cref{sec:problem_setup_additional} contains additional details.

%We compare the performance of Adam, \lbfgs{}, and \al{} on training PINNs for all three classes of PDEs. 
%For Adam, we tune the learning rate by a grid search on $\{10^{-5}, 10^{-4}, 10^{-3}, 10^{-2}, 10^{-1}\}$.
%For \lbfgs, we use the default learning rate $1.0$, memory size $100$, and strong Wolfe line search.
%For \al, we tune the learning rate for Adam as before, and also vary the switch from Adam to \lbfgs{} (after 1000, 11000, 31000 iterations).
%These correspond to \al{} (1k), \al{} (11k), and \al{} (31k) in our figures.
%All three methods are run for a total of 41000 iterations.

%We use multilayer perceptrons (MLPs) with tanh activations and three hidden layers. These MLPs have widths 50, 100, 200, or 400.
%We initialize these networks with the Xavier normal initialization \cite{glorot2010understanding} and all biases equal to zero.
%Each combination of PDE, optimizer, and MLP architecture is run with 5 random seeds.

%We use 10000 residual points randomly sampled from a $255 \times 100$ grid on the interior of the problem domain. 
%We use 257 equally spaced points for the initial conditions and 101 equally spaced points for each boundary condition.

%We assess the discrepancy between the PINN solution and the ground truth using $\ell_2$ relative error (L2RE), a standard metric in the PINN literature. Let $y = (y_i)_{i = 1}^n$ be the PINN prediction and $y' = (y'_i)_{i = 1}^n$ the ground truth. Define
%\begin{align*}
%    \mathrm{L2RE} = \sqrt{\frac{\sum_{i = 1}^n (y_i - y'_i)^2}{\sum_{i = 1}^n y'^2_i}} = \sqrt{\frac{\|y - y'\|_2^2}{\|y'\|_2^2}}.
%\end{align*}
%We compute the L2RE using all points in the $255 \times 100$ grid on the interior of the problem domain, along with the 257 and 101 points used for the initial and boundary conditions.

%We develop our experiments in PyTorch 2.0.0 \cite{paszke2019pytorch} with Python 3.10.12.
%Each experiment is run on a single NVIDIA Titan V GPU using CUDA 11.8.
%The code for our experiments is available at \href{https://github.com/pratikrathore8/opt_for_pinns}{https://github.com/pratikrathore8/opt\_for\_pinns}.


\subsection{2D Allen Cahn Equation}
\begin{figure*}[t]
    \centering
    \includegraphics[scale=0.38]{figs/Burgers_operator.pdf}
    \caption{1D Burgers' Equation (Operator Learning): Steady-state solutions for different initializations $u_0$ under varying viscosity $\varepsilon$: (a) $\varepsilon = 0.5$, (b) $\varepsilon = 0.1$, (c) $\varepsilon = 0.05$. The results demonstrate that all final test solutions converge to the correct steady-state solution. (d) Illustration of the evolution of a test initialization $u_0$ following homotopy dynamics. The number of residual points is $\nres = 128$.}
    \label{fig:Burgers_result}
\end{figure*}
First, we consider the following time-dependent problem:
\begin{align}
& u_t = \varepsilon^2 \Delta u - u(u^2 - 1), \quad (x, y) \in [-1, 1] \times [-1, 1] \nonumber \\
& u(x, y, 0) = - \sin(\pi x) \sin(\pi y) \label{eq.hom_2D_AC}\\
& u(-1, y, t) = u(1, y, t) = u(x, -1, t) = u(x, 1, t) = 0. \nonumber
\end{align}
We aim to find the steady-state solution for this equation with $\varepsilon = 0.05$ and define the homotopy as:
\begin{equation}
    H(u, s, \varepsilon) = (1 - s)\left(\varepsilon(s)^2 \Delta u - u(u^2 - 1)\right) + s(u - u_0),\nonumber
\end{equation}
where $s \in [0, 1]$. Specifically, when $s = 1$, the initial condition $u_0$ is automatically satisfied, and when $s = 0$, it recovers the steady-state problem. The function $\varepsilon(s)$ is given by
\begin{equation}
\varepsilon(s) = 
\left\{\begin{array}{l}
s, \quad s \in [0.05, 1], \\
0.05, \quad s \in [0, 0.05].
\end{array}\right.\label{eq:epsilon_t}
\end{equation}

Here, $\varepsilon(s)$ varies with $s$ during the first half of the evolution. Once $\varepsilon(s)$ reaches $0.05$, it remains fixed, and only $s$ continues to evolve toward $0$. As shown in \cref{fig:2D_Allen_Cahn_Equation}, the relative $L_2$ error by homotopy dynamics is $8.78 \times 10^{-3}$, compared with the result obtained by PINN, which has a $L_2$ error of $9.56 \times 10^{-1}$. This clearly demonstrates that the homotopy dynamics-based approach significantly improves accuracy.

\subsection{High Frequency Function Approximation }
We aim to approximate the following function:
$u=    \sin(50\pi x), \quad x \in [0,1].$
The homotopy is defined as $H(u,\varepsilon) = u - \sin(\frac{1}{\varepsilon}\pi x), $
where $\varepsilon \in [\frac{1}{50},\frac{1}{15}]$.

\begin{table}[htbp!]
    \caption{Comparison of the lowest loss achieved by the classical training and homotopy dynamics for different values of $\varepsilon$ in approximating $\sin\left(\frac{1}{\varepsilon} \pi x\right)$
    }
    \vskip 0.15in
    \centering
    \tiny
    \begin{tabular}{|c|c|c|c|c|} 
    \hline 
    $ $ & $\varepsilon = 1/15$ & $\varepsilon = 1/35$ & $\varepsilon = 1/50$ \\ \hline 
    Classical Loss                & 4.91e-6     & 7.21e-2     & 3.29e-1       \\ \hline 
    Homotopy Loss $L_H$                      & 1.73e-6     & 1.91e-6     & \textbf{2.82e-5}       \\ \hline
    \end{tabular}
    % On convection, \al{} provides 14.2$\times$ and 1.97$\times$ improvement over Adam or \lbfgs{} on L2RE. 
    % On reaction, \al{} provides 1.10$\times$ and 1.99$\times$ improvement over Adam or \lbfgs{} on L2RE.
    % On wave, \al{} provides 6.32$\times$ and 6.07$\times$ improvement over Adam or \lbfgs{} on L2RE.}
    \label{tab:loss_approximate}
\end{table}

As shown in \cref{fig:high_frequency_result}, due to the F-principle \cite{xu2024overview}, training is particularly challenging when approximating high-frequency functions like $\sin(50\pi x)$. The loss decreases slowly, resulting in poor approximation performance. However, training based on homotopy dynamics significantly reduces the loss, leading to a better approximation of high-frequency functions. This demonstrates that homotopy dynamics-based training can effectively facilitate convergence when approximating high-frequency data. Additionally, we compare the loss for approximating functions with different frequencies $1/\varepsilon$ using both methods. The results, presented in \cref{tab:loss_approximate}, show that the homotopy dynamics training method consistently performs well for high-frequency functions.





\subsection{Burgers Equation}
In this example, we adopt the operator learning framework to solve for the steady-state solution of the Burgers equation, given by:
\begin{align}
& u_t+\left(\frac{u^2}{2}\right)_x - \varepsilon u_{xx}=\pi \sin (\pi x) \cos (\pi x), \quad x \in[0, 1]\nonumber\\
& u(x, 0)=u_0(x),\label{eq:1D_Burgers} \\
& u(0, t)=u(1, t)=0, \nonumber 
\end{align}
with Dirichlet boundary conditions, where $u_0 \in L_{0}^2((0, 1); \mathbb{R})$ is the initial condition and $\varepsilon \in \mathbb{R}$ is the viscosity coefficient. We aim to learn the operator mapping the initial condition to the steady-state solution, $G^{\dagger}: L_{0}^2((0, 1); \mathbb{R}) \rightarrow H_{0}^r((0, 1); \mathbb{R})$, defined by $u_0 \mapsto u_{\infty}$ for any $r > 0$. As shown in Theorem 2.2 of \cite{KREISS1986161} and Theorems 2.5 and 2.7 of \cite{hao2019convergence}, for any $\varepsilon > 0$, the steady-state solution is independent of the initial condition, with a single shock occurring at $x_s = 0.5$. Here, we use DeepONet~\cite{lu2021deeponet} as the network architecture. 
The homotopy definition, similar to ~\cref{eq.hom_2D_AC}, can be found in \cref{Ap:operator}. The results can be found in \cref{fig:Burgers_result} and \cref{tab:loss_burgers}. Experimental results show that the homotopy dynamics strategy performs well in the operator learning setting as well.


\begin{table}[htbp!]
    \caption{Comparison of loss between classical training and homotopy dynamics for different values of $\varepsilon$ in the Burgers equation, along with the MSE distance to the ground truth shock location, $x_s$.}
    \vskip 0.15in
    \centering
    \tiny
    \begin{tabular}{|c|c|c|c|c|} 
    \hline  
    $ $ & $\varepsilon = 0.5$ & $\varepsilon = 0.1$ & $\varepsilon = 0.05$ \\ \hline 
    Homotopy Loss $L_H$                &  7.55e-7     & 3.40e-7     & 7.77e-7       \\ \hline 
    L2RE                      & 1.50e-3     & 7.00e-4     & 2.52e-2       \\ \hline
        MSE Distance $x_s$                      & 1.75e-8     & 9.14e-8      & 1.2e-3      \\ \hline
    \end{tabular}
    % On convection, \al{} provides 14.2$\times$ and 1.97$\times$ improvement over Adam or \lbfgs{} on L2RE. 
    % On reaction, \al{} provides 1.10$\times$ and 1.99$\times$ improvement over Adam or \lbfgs{} on L2RE.
    % On wave, \al{} provides 6.32$\times$ and 6.07$\times$ improvement over Adam or \lbfgs{} on L2RE.}
    \label{tab:loss_burgers}
\end{table}



% \begin{itemize}
%     \item Relate the curvature in the problem to the differential operator. Use this to demonstrate why the problem is ill-conditioned
%     \item Give an argument for why using Adam + L-BFGS is better than just using L-BFGS outright. The idea is that Adam lowers the errors to the point where the rest of the optimization becomes convex \ldots
%     \item Show why we need second-order methods. We would like to prove that once we are close to the optimum, second-order methods will give condition-number free linear convergence. Specialize this to the Gauss-Newton setting, with the randomized low-rank approximation.
%     % \item Show that it is not possible to get superlinear convergence under the interpolation assumption with an overparameterized neural network. This should be true b/c the Hessian at the optimum will have rank $\min(n, d)$, and when $d > n$, this guarantees that we cannot have strong convexity.
% \end{itemize}
\section{Experiments}
\label{sec:experiments}
The experiments are designed to address two key research questions.
First, \textbf{RQ1} evaluates whether the average $L_2$-norm of the counterfactual perturbation vectors ($\overline{||\perturb||}$) decreases as the model overfits the data, thereby providing further empirical validation for our hypothesis.
Second, \textbf{RQ2} evaluates the ability of the proposed counterfactual regularized loss, as defined in (\ref{eq:regularized_loss2}), to mitigate overfitting when compared to existing regularization techniques.

% The experiments are designed to address three key research questions. First, \textbf{RQ1} investigates whether the mean perturbation vector norm decreases as the model overfits the data, aiming to further validate our intuition. Second, \textbf{RQ2} explores whether the mean perturbation vector norm can be effectively leveraged as a regularization term during training, offering insights into its potential role in mitigating overfitting. Finally, \textbf{RQ3} examines whether our counterfactual regularizer enables the model to achieve superior performance compared to existing regularization methods, thus highlighting its practical advantage.

\subsection{Experimental Setup}
\textbf{\textit{Datasets, Models, and Tasks.}}
The experiments are conducted on three datasets: \textit{Water Potability}~\cite{kadiwal2020waterpotability}, \textit{Phomene}~\cite{phomene}, and \textit{CIFAR-10}~\cite{krizhevsky2009learning}. For \textit{Water Potability} and \textit{Phomene}, we randomly select $80\%$ of the samples for the training set, and the remaining $20\%$ for the test set, \textit{CIFAR-10} comes already split. Furthermore, we consider the following models: Logistic Regression, Multi-Layer Perceptron (MLP) with 100 and 30 neurons on each hidden layer, and PreactResNet-18~\cite{he2016cvecvv} as a Convolutional Neural Network (CNN) architecture.
We focus on binary classification tasks and leave the extension to multiclass scenarios for future work. However, for datasets that are inherently multiclass, we transform the problem into a binary classification task by selecting two classes, aligning with our assumption.

\smallskip
\noindent\textbf{\textit{Evaluation Measures.}} To characterize the degree of overfitting, we use the test loss, as it serves as a reliable indicator of the model's generalization capability to unseen data. Additionally, we evaluate the predictive performance of each model using the test accuracy.

\smallskip
\noindent\textbf{\textit{Baselines.}} We compare CF-Reg with the following regularization techniques: L1 (``Lasso''), L2 (``Ridge''), and Dropout.

\smallskip
\noindent\textbf{\textit{Configurations.}}
For each model, we adopt specific configurations as follows.
\begin{itemize}
\item \textit{Logistic Regression:} To induce overfitting in the model, we artificially increase the dimensionality of the data beyond the number of training samples by applying a polynomial feature expansion. This approach ensures that the model has enough capacity to overfit the training data, allowing us to analyze the impact of our counterfactual regularizer. The degree of the polynomial is chosen as the smallest degree that makes the number of features greater than the number of data.
\item \textit{Neural Networks (MLP and CNN):} To take advantage of the closed-form solution for computing the optimal perturbation vector as defined in (\ref{eq:opt-delta}), we use a local linear approximation of the neural network models. Hence, given an instance $\inst_i$, we consider the (optimal) counterfactual not with respect to $\model$ but with respect to:
\begin{equation}
\label{eq:taylor}
    \model^{lin}(\inst) = \model(\inst_i) + \nabla_{\inst}\model(\inst_i)(\inst - \inst_i),
\end{equation}
where $\model^{lin}$ represents the first-order Taylor approximation of $\model$ at $\inst_i$.
Note that this step is unnecessary for Logistic Regression, as it is inherently a linear model.
\end{itemize}

\smallskip
\noindent \textbf{\textit{Implementation Details.}} We run all experiments on a machine equipped with an AMD Ryzen 9 7900 12-Core Processor and an NVIDIA GeForce RTX 4090 GPU. Our implementation is based on the PyTorch Lightning framework. We use stochastic gradient descent as the optimizer with a learning rate of $\eta = 0.001$ and no weight decay. We use a batch size of $128$. The training and test steps are conducted for $6000$ epochs on the \textit{Water Potability} and \textit{Phoneme} datasets, while for the \textit{CIFAR-10} dataset, they are performed for $200$ epochs.
Finally, the contribution $w_i^{\varepsilon}$ of each training point $\inst_i$ is uniformly set as $w_i^{\varepsilon} = 1~\forall i\in \{1,\ldots,m\}$.

The source code implementation for our experiments is available at the following GitHub repository: \url{https://anonymous.4open.science/r/COCE-80B4/README.md} 

\subsection{RQ1: Counterfactual Perturbation vs. Overfitting}
To address \textbf{RQ1}, we analyze the relationship between the test loss and the average $L_2$-norm of the counterfactual perturbation vectors ($\overline{||\perturb||}$) over training epochs.

In particular, Figure~\ref{fig:delta_loss_epochs} depicts the evolution of $\overline{||\perturb||}$ alongside the test loss for an MLP trained \textit{without} regularization on the \textit{Water Potability} dataset. 
\begin{figure}[ht]
    \centering
    \includegraphics[width=0.85\linewidth]{img/delta_loss_epochs.png}
    \caption{The average counterfactual perturbation vector $\overline{||\perturb||}$ (left $y$-axis) and the cross-entropy test loss (right $y$-axis) over training epochs ($x$-axis) for an MLP trained on the \textit{Water Potability} dataset \textit{without} regularization.}
    \label{fig:delta_loss_epochs}
\end{figure}

The plot shows a clear trend as the model starts to overfit the data (evidenced by an increase in test loss). 
Notably, $\overline{||\perturb||}$ begins to decrease, which aligns with the hypothesis that the average distance to the optimal counterfactual example gets smaller as the model's decision boundary becomes increasingly adherent to the training data.

It is worth noting that this trend is heavily influenced by the choice of the counterfactual generator model. In particular, the relationship between $\overline{||\perturb||}$ and the degree of overfitting may become even more pronounced when leveraging more accurate counterfactual generators. However, these models often come at the cost of higher computational complexity, and their exploration is left to future work.

Nonetheless, we expect that $\overline{||\perturb||}$ will eventually stabilize at a plateau, as the average $L_2$-norm of the optimal counterfactual perturbations cannot vanish to zero.

% Additionally, the choice of employing the score-based counterfactual explanation framework to generate counterfactuals was driven to promote computational efficiency.

% Future enhancements to the framework may involve adopting models capable of generating more precise counterfactuals. While such approaches may yield to performance improvements, they are likely to come at the cost of increased computational complexity.


\subsection{RQ2: Counterfactual Regularization Performance}
To answer \textbf{RQ2}, we evaluate the effectiveness of the proposed counterfactual regularization (CF-Reg) by comparing its performance against existing baselines: unregularized training loss (No-Reg), L1 regularization (L1-Reg), L2 regularization (L2-Reg), and Dropout.
Specifically, for each model and dataset combination, Table~\ref{tab:regularization_comparison} presents the mean value and standard deviation of test accuracy achieved by each method across 5 random initialization. 

The table illustrates that our regularization technique consistently delivers better results than existing methods across all evaluated scenarios, except for one case -- i.e., Logistic Regression on the \textit{Phomene} dataset. 
However, this setting exhibits an unusual pattern, as the highest model accuracy is achieved without any regularization. Even in this case, CF-Reg still surpasses other regularization baselines.

From the results above, we derive the following key insights. First, CF-Reg proves to be effective across various model types, ranging from simple linear models (Logistic Regression) to deep architectures like MLPs and CNNs, and across diverse datasets, including both tabular and image data. 
Second, CF-Reg's strong performance on the \textit{Water} dataset with Logistic Regression suggests that its benefits may be more pronounced when applied to simpler models. However, the unexpected outcome on the \textit{Phoneme} dataset calls for further investigation into this phenomenon.


\begin{table*}[h!]
    \centering
    \caption{Mean value and standard deviation of test accuracy across 5 random initializations for different model, dataset, and regularization method. The best results are highlighted in \textbf{bold}.}
    \label{tab:regularization_comparison}
    \begin{tabular}{|c|c|c|c|c|c|c|}
        \hline
        \textbf{Model} & \textbf{Dataset} & \textbf{No-Reg} & \textbf{L1-Reg} & \textbf{L2-Reg} & \textbf{Dropout} & \textbf{CF-Reg (ours)} \\ \hline
        Logistic Regression   & \textit{Water}   & $0.6595 \pm 0.0038$   & $0.6729 \pm 0.0056$   & $0.6756 \pm 0.0046$  & N/A    & $\mathbf{0.6918 \pm 0.0036}$                     \\ \hline
        MLP   & \textit{Water}   & $0.6756 \pm 0.0042$   & $0.6790 \pm 0.0058$   & $0.6790 \pm 0.0023$  & $0.6750 \pm 0.0036$    & $\mathbf{0.6802 \pm 0.0046}$                    \\ \hline
%        MLP   & \textit{Adult}   & $0.8404 \pm 0.0010$   & $\mathbf{0.8495 \pm 0.0007}$   & $0.8489 \pm 0.0014$  & $\mathbf{0.8495 \pm 0.0016}$     & $0.8449 \pm 0.0019$                    \\ \hline
        Logistic Regression   & \textit{Phomene}   & $\mathbf{0.8148 \pm 0.0020}$   & $0.8041 \pm 0.0028$   & $0.7835 \pm 0.0176$  & N/A    & $0.8098 \pm 0.0055$                     \\ \hline
        MLP   & \textit{Phomene}   & $0.8677 \pm 0.0033$   & $0.8374 \pm 0.0080$   & $0.8673 \pm 0.0045$  & $0.8672 \pm 0.0042$     & $\mathbf{0.8718 \pm 0.0040}$                    \\ \hline
        CNN   & \textit{CIFAR-10} & $0.6670 \pm 0.0233$   & $0.6229 \pm 0.0850$   & $0.7348 \pm 0.0365$   & N/A    & $\mathbf{0.7427 \pm 0.0571}$                     \\ \hline
    \end{tabular}
\end{table*}

\begin{table*}[htb!]
    \centering
    \caption{Hyperparameter configurations utilized for the generation of Table \ref{tab:regularization_comparison}. For our regularization the hyperparameters are reported as $\mathbf{\alpha/\beta}$.}
    \label{tab:performance_parameters}
    \begin{tabular}{|c|c|c|c|c|c|c|}
        \hline
        \textbf{Model} & \textbf{Dataset} & \textbf{No-Reg} & \textbf{L1-Reg} & \textbf{L2-Reg} & \textbf{Dropout} & \textbf{CF-Reg (ours)} \\ \hline
        Logistic Regression   & \textit{Water}   & N/A   & $0.0093$   & $0.6927$  & N/A    & $0.3791/1.0355$                     \\ \hline
        MLP   & \textit{Water}   & N/A   & $0.0007$   & $0.0022$  & $0.0002$    & $0.2567/1.9775$                    \\ \hline
        Logistic Regression   &
        \textit{Phomene}   & N/A   & $0.0097$   & $0.7979$  & N/A    & $0.0571/1.8516$                     \\ \hline
        MLP   & \textit{Phomene}   & N/A   & $0.0007$   & $4.24\cdot10^{-5}$  & $0.0015$    & $0.0516/2.2700$                    \\ \hline
       % MLP   & \textit{Adult}   & N/A   & $0.0018$   & $0.0018$  & $0.0601$     & $0.0764/2.2068$                    \\ \hline
        CNN   & \textit{CIFAR-10} & N/A   & $0.0050$   & $0.0864$ & N/A    & $0.3018/
        2.1502$                     \\ \hline
    \end{tabular}
\end{table*}

\begin{table*}[htb!]
    \centering
    \caption{Mean value and standard deviation of training time across 5 different runs. The reported time (in seconds) corresponds to the generation of each entry in Table \ref{tab:regularization_comparison}. Times are }
    \label{tab:times}
    \begin{tabular}{|c|c|c|c|c|c|c|}
        \hline
        \textbf{Model} & \textbf{Dataset} & \textbf{No-Reg} & \textbf{L1-Reg} & \textbf{L2-Reg} & \textbf{Dropout} & \textbf{CF-Reg (ours)} \\ \hline
        Logistic Regression   & \textit{Water}   & $222.98 \pm 1.07$   & $239.94 \pm 2.59$   & $241.60 \pm 1.88$  & N/A    & $251.50 \pm 1.93$                     \\ \hline
        MLP   & \textit{Water}   & $225.71 \pm 3.85$   & $250.13 \pm 4.44$   & $255.78 \pm 2.38$  & $237.83 \pm 3.45$    & $266.48 \pm 3.46$                    \\ \hline
        Logistic Regression   & \textit{Phomene}   & $266.39 \pm 0.82$ & $367.52 \pm 6.85$   & $361.69 \pm 4.04$  & N/A   & $310.48 \pm 0.76$                    \\ \hline
        MLP   &
        \textit{Phomene} & $335.62 \pm 1.77$   & $390.86 \pm 2.11$   & $393.96 \pm 1.95$ & $363.51 \pm 5.07$    & $403.14 \pm 1.92$                     \\ \hline
       % MLP   & \textit{Adult}   & N/A   & $0.0018$   & $0.0018$  & $0.0601$     & $0.0764/2.2068$                    \\ \hline
        CNN   & \textit{CIFAR-10} & $370.09 \pm 0.18$   & $395.71 \pm 0.55$   & $401.38 \pm 0.16$ & N/A    & $1287.8 \pm 0.26$                     \\ \hline
    \end{tabular}
\end{table*}

\subsection{Feasibility of our Method}
A crucial requirement for any regularization technique is that it should impose minimal impact on the overall training process.
In this respect, CF-Reg introduces an overhead that depends on the time required to find the optimal counterfactual example for each training instance. 
As such, the more sophisticated the counterfactual generator model probed during training the higher would be the time required. However, a more advanced counterfactual generator might provide a more effective regularization. We discuss this trade-off in more details in Section~\ref{sec:discussion}.

Table~\ref{tab:times} presents the average training time ($\pm$ standard deviation) for each model and dataset combination listed in Table~\ref{tab:regularization_comparison}.
We can observe that the higher accuracy achieved by CF-Reg using the score-based counterfactual generator comes with only minimal overhead. However, when applied to deep neural networks with many hidden layers, such as \textit{PreactResNet-18}, the forward derivative computation required for the linearization of the network introduces a more noticeable computational cost, explaining the longer training times in the table.

\subsection{Hyperparameter Sensitivity Analysis}
The proposed counterfactual regularization technique relies on two key hyperparameters: $\alpha$ and $\beta$. The former is intrinsic to the loss formulation defined in (\ref{eq:cf-train}), while the latter is closely tied to the choice of the score-based counterfactual explanation method used.

Figure~\ref{fig:test_alpha_beta} illustrates how the test accuracy of an MLP trained on the \textit{Water Potability} dataset changes for different combinations of $\alpha$ and $\beta$.

\begin{figure}[ht]
    \centering
    \includegraphics[width=0.85\linewidth]{img/test_acc_alpha_beta.png}
    \caption{The test accuracy of an MLP trained on the \textit{Water Potability} dataset, evaluated while varying the weight of our counterfactual regularizer ($\alpha$) for different values of $\beta$.}
    \label{fig:test_alpha_beta}
\end{figure}

We observe that, for a fixed $\beta$, increasing the weight of our counterfactual regularizer ($\alpha$) can slightly improve test accuracy until a sudden drop is noticed for $\alpha > 0.1$.
This behavior was expected, as the impact of our penalty, like any regularization term, can be disruptive if not properly controlled.

Moreover, this finding further demonstrates that our regularization method, CF-Reg, is inherently data-driven. Therefore, it requires specific fine-tuning based on the combination of the model and dataset at hand.
\section{Imitation Learning experiments}\label{sec:imitation}
One of the key contributions of this paper is the characterization of the relationship between imitation learning (IL) and inverse reinforcement learning (IRL)/Dynamic Discrete Choice (DDC) model, particularly through the ERM-IRL/DDC framework. Given that much of the IRL literature has historically focused on providing experimental results for IL tasks, we conduct a series of experiments to empirically validate our theoretical findings. Specifically, we aim to test our
prediction in Section \ref{sec:ERM-IRL} that \textit{behavioral cloning (BC) should outperform ERM-IRL for IL tasks}, as BC directly optimizes the negative log-likelihood objective without the additional complexity of Bellman error minimization. By comparing BC and ERM-IRL across various IL benchmark tasks, we demonstrate that BC consistently achieves better performance in terms of both computational efficiency and policy accuracy, reinforcing our claim that IL is a strictly easier problem than IRL.

\subsection{Experimental Setup}

As in \citet{garg2021iq}, we employ three OpenAI Gym environments for algorithms with discrete actions \citep{brockman2016openai}: Lunar Lander v2, Cartpole v1, and Acrobot v1. These environments are widely used in IL and RL research, providing well-defined optimal policies and performance metrics. 

\noindent \textbf{Dataset.}  
For each environment, we generate expert demonstrations using a pre-trained policy. We use publicly available expert policies\footnote{\url{https://huggingface.co/sb3/}} trained via Proximal Policy Optimization (PPO) \cite{schulman2017proximal}, as implemented in the Stable-Baselines3 library \citep{raffin2021stable}. Each expert policy is run to generate demonstration trajectories, and we vary the number of expert trajectories across experiments for training. For all experiments, we used the expert policy demonstration data from 10 episodes for testing.

\noindent \textbf{Performance Metric.}  
The primary evaluation metric is \% optimality, defined as:
\begin{align}
    \text{\% optimality of an episode} := \frac{\text{One episode's episodic reward of the algorithm}}{\text{Mean of 1,000 episodic rewards of the expert}} \times 100. \notag
\end{align}
For each baseline, we report the mean and standard deviation of 100 evaluation episodes after training. A higher \% optimality indicates that the algorithm's policy closely matches the expert. The 1000-episodic mean and standard deviation ([mean$\pm$std]) of the episodic reward of expert policy for each environment was $[232.77\pm73.77]$ for Lunar-Lander v2 (larger the better), $[-82.80\pm27.55]$ for Acrobot v1 (smaller the better), and $[500\pm 0]$ for Cartpole v1 (larger the better).

\noindent \textbf{Training Details.}  
All algorithms were trained for 5,000 epochs. Since our goal in this experiment is to show superiority of BC for IL tasks, we only include ERM-IRL and IQ-learn \cite{garg2021iq} as baselines. Specifically, we exclude baselines such as Rust \citep{rust1987optimal} and ML-IRL \citep{zeng2023understanding}, which require explicit transition probability estimation.

\subsection{Experiment results}


%\noindent \textbf{GLADIUS (Ours)}  
%The ERM-IRL framework minimizes both the negative log-likelihood (NLL) and Bellman error (BE) terms, making it computationally more complex than BC. 

%\noindent \textbf{IQ-Learn} \citep{garg2021iq}  
%A popular \cite{rafailov2024r} IRL method that minimizes an occupancy-matching objective, i.e., it does not enforce Bellman consistency. For details, refer to Section \ref{sec:ImitationID}.

%\noindent \textbf{Behavioral Cloning (BC)}  
%BC minimizes only the NLL term, making it computationally simple and sample-efficient.

Table \ref{fig:OpenAI_gym} presents the \% optimality results for Lunar Lander v2, Cartpole v1, and Acrobot v1. As predicted in our theoretical analysis, BC consistently outperforms ERM-IRL in terms of \% optimality, validating our theoretical claims.


\begin{table*}[ht]
    \centering
    \scalebox{0.85}{
    \begin{tabular}{l
            >{\centering\arraybackslash}p{1.5cm}
            >{\centering\arraybackslash}p{1.5cm}
            >{\centering\arraybackslash}p{1.5cm}
            >{\centering\arraybackslash}p{1.5cm}
            >{\centering\arraybackslash}p{1.5cm}
            >{\centering\arraybackslash}p{1.5cm}
            >{\centering\arraybackslash}p{1.5cm}
            >{\centering\arraybackslash}p{1.5cm}
            >{\centering\arraybackslash}p{1.5cm}}
\toprule
\multirow{4}{*}{\parbox{0.5cm}{Trajs}} 
& \multicolumn{3}{c}{\makecell{\ul{Lunar Lander v2 (\%)} \\ (Larger \% the better)}} 
& \multicolumn{3}{c}{\makecell{\ul{Cartpole v1 (\%)} \\ (Larger \% the better)}} 
& \multicolumn{3}{c}{\makecell{\ul{Acrobot v1 (\%)} \\ (Smaller \% the better)}} \\
\cmidrule(lr){2-4} \cmidrule(lr){5-7} \cmidrule(lr){8-10}
& {\ul{Gladius}} & IQ-learn & BC 
& {\ul{Gladius}} & IQ-learn & BC
& {\ul{Gladius}} & IQ-learn & BC \\
\midrule
1  & \textbf{107.30 }& 83.78 & 103.38   & 100.00  & 100.00  & 100.00     & 103.67  & 103.47  & \textbf{100.56}  \\
    & (10.44)  & (22.25)  & (13.78)   & (0.00)  & (0.00)  & (0.00)   & (32.78)  & (55.44)  & (26.71)  \\
\midrule
3   & \textbf{106.64}  &  102.44 & 104.46 & 100.00  & 100.00  & 100.00 & 102.19  & 101.28  & \textbf{101.25}  \\
    & (11.11)  & (20.66)   & (11.57)  & (0.00)  & (0.00)  & (0.00)  & (22.69)  & (37.51)  & (36.42)  \\
\midrule
7   &  101.10 & 104.91  & \textbf{ 105.99}  & 100.00  & 100.00  & 100.00 & 100.67 & 100.58  &\textbf{98.08} \\
    & (16.28)  & (13.98) &  (10.20) & (0.00)         & (0.00)  & (0.00)  & (22.30)      & (30.09)  &  (24.27)  \\
\midrule
10  & 104.46  & 105.13   & \textbf{ 107.01}   & 100.00  & 100.00  & 100.00  & 99.07  & 101.10 &  \textbf{97.75}\\
    & (13.65)  & (13.83)   & (10.75)  & (0.00)           & (0.00)  &  (0.00) &  (20.58)    & (30.40)  & (16.67)  \\
\midrule
15  & 106.11  &  106.51 & \textbf{107.42}  & 100.00  & 100.00  &100.00 & 96.50  & 95.34  & \textbf{95.33}  \\
    & (10.65)  & (14.10)  & (10.45)  & (0.00)         & (0.00)  &  (0.00)  & (18.53)        & (26.92)  & (15.42)  \\
\bottomrule
\multicolumn{10}{l}{\footnotesize 
Based on 100 episodes for each baseline. Each baseline was trained for 5000 epochs.}
\end{tabular}
    }
\caption{Mean and standard deviation of \% optimality of 100 episodes}
\label{fig:OpenAI_gym}   
\end{table*}

\iffalse

\subsection{Reward transfer experiments}
As we discussed in Section \ref{sec:Intro}, the main advantage of IRL over Imitation Learning (IL) is its premise that the learned rewards can be used for counterfactual simulations. That is, given the rewards we learn from IRL, we would expect that such learned rewards can be useful for RL training \cite{zeng2023understanding}. We use the Lunar Lander v2 from OpenAI Gym for the experiment. Specifically, for each number of expert trajectories we consider ([1, 3, 7, 10, 15]):
\begin{enumerate}[noitemsep]
    \item Train the reward function using GLADIUS for 5000 epochs.
    \item Use the trained reward function  for training a Proximal Policy Optimization (PPO) policy.
\end{enumerate}
As discussed in \cite{zeng2023understanding}, IQ-learn and BC performed no better than randomly generated reward for this task. Therefore, we don't include them for the comparison. 

\begin{table*}[ht]
    \centering
    \scalebox{0.9}{
    \begin{tabular}{l
            >{\centering\arraybackslash}p{1.5cm}
            >{\centering\arraybackslash}p{1.5cm}
            >{\centering\arraybackslash}p{1.5cm}
            >{\centering\arraybackslash}p{1.5cm}
            >{\centering\arraybackslash}p{1.5cm}}
\toprule
\multirow{2}{*}{\parbox{2.5cm}{\centering Lunar Lander v2}} 
& \multicolumn{5}{c}{\textbf{Number of Trajectories (Trajs)}} \\
\cmidrule(lr){2-6}
& \textbf{1} & \textbf{3} & \textbf{7} & \textbf{10} & \textbf{15} \\
\midrule
Mean  & 97.63 & 100.12 & 104.23 & 106.52 &  \\
Std   & (28.15)  & (29.12) & (24.57) &(19.88)  &  \\
\bottomrule
\multicolumn{6}{l}{\footnotesize Based on 1000 episodes for each baseline. Each baseline was trained 5000 epochs.}
\end{tabular}
    }
\caption{Mean and standard deviation of \% optimality for Lunar Lander v2 using GLADIUS}
\label{fig:LunarLander_Gladius}   
\end{table*}

\fi
\section{Conclusion}
In this work, we propose a simple yet effective approach, called SMILE, for graph few-shot learning with fewer tasks. Specifically, we introduce a novel dual-level mixup strategy, including within-task and across-task mixup, for enriching the diversity of nodes within each task and the diversity of tasks. Also, we incorporate the degree-based prior information to learn expressive node embeddings. Theoretically, we prove that SMILE effectively enhances the model's generalization performance. Empirically, we conduct extensive experiments on multiple benchmarks and the results suggest that SMILE significantly outperforms other baselines, including both in-domain and cross-domain few-shot settings.
\input

\bibliographystyle{plainnat}
\bibliography{bib}

\appendix

\section{Additional Experiments}\label{app: additional}
We evaluate our methods on large-scale models: Pythia-70M~\citep{pythia} and Board Games Models~\citep{karvonen2024measuring}. These models present significant computational challenges, as different feedback methods generate millions of constraints. This scale necessitates specialized approaches for both memory management and computational efficiency.

\paragraph{Memory-efficient constraint storage} The high dimensionality of model dictionaries makes storing complete activation indices for each feature prohibitively memory-intensive. We address this by enforcing constant sparsity constraints, limiting activations to a maximum sparsity of 3. This constraint enables efficient storage of large-dimensional arrays while preserving the essential characteristics of the features.
\paragraph{Computational optimization} To efficiently handle constraint satisfaction at scale, we reformulate the problem as a matrix regression task, as detailed in \figref{fig:gradient}. The learner maintains a low-rank decomposition of the feature matrix $\pphi$, assuming $\pphi = UU^\top$, where $U$ represents the learned dictionary. This formulation allows for efficient batch-wise optimization over the constraint set while maintaining feasible memory requirements.
% In this section, we provide experiments on large scale models: Pythia-70M~\cite{pythia} and Board Games Models~\cite{karvonen2024measuring}. For these models, the number of constraints generated for different feedback methods scale in millions, we can not explicitly store all the constraints or fit to them directly. We propose the following schemes to resolve these computational and memory issues.

% \paragraph{Memory bottleneck of storing constraints:} Since the dimension of the dictionary could be very high, strong all the indices of an activation used for feature is memory inefficient., which could be handled with constant sparsity constraints. In our experiments, we assume that sparsity is at most 3. THis allows to store numpy arrays of large dimensions. 

% \paragraph{Computational issues:} In order to solve the problem constraints satisfaction, we exploy a matrix version of regression with constraints as samples as shown in \figref{fig:gradient}. Learner maintains a decomposition $U$ of the feature matrix $\pphi$ assuming $\pphi = UU^\top$.


\paragraph{Dictionary features of Pythia-70M} We use the publicly available repository for dictionary learning via sparse autoencoders on neural network activations~\citep{marks2024dictionarylearning}. We consider the dictionaries trained for Pythia-70M~\citep{pythia} (a general-purpose LLM trained on publicly available datasets). We retrieve the corresponding autoencoders for attention output layers which have dimensions $32768 \times 512$. Note that $p(p+1)/2 \approx, 512M$.
For the experiments, we use $3$-sparsity on uniform sparse distributions. We present the plots for ChessGPT in two parts in \figref{fig: subsample} and \figref{fig: pythiasample} for different feedback methods.


\newpage

\begin{figure}[h!]
\centering
    \begin{minipage}{0.99
    \textwidth}  % Controls width of the content
    \begin{algorithm}[H]
    \small  % or \footnotesize for smaller text
    \caption{Optimization via Gradient Descent}
    \label{alg:gradient}
    \begin{enumerate}
        \item Given a dictionary $U \in \mathbb{R}^{p \times r}$, minimize the loss $\mathcal{L}(U)$:
        \begin{equation}
            \mathcal{L}(U) = \mathcal{L}_{\text{MSE}}(U) + \mathcal{L}_{\text{reg}}(U)
        \end{equation}
        where MSE loss is:
        \begin{equation}
            \mathcal{L}_{\text{MSE}}(U) = \frac{1}{|B|}\sum_{i \in B} (\|U^\top u_i\|^2 - c_i\|U^\top y\|^2)^2
        \end{equation}
        and regularization term is:
        \begin{equation}
            \mathcal{L}_{\text{reg}}(U) = \lambda\|U\|_F^2
        \end{equation}
        
        \item For each batch containing indices $i$, values $v$, and targets $c$:
            \begin{enumerate}
                \item Construct sparse vectors $u_i$ using $(i,v)$ pairs
                \item Compute projected values: $U^\top u_i$ and $U^\top y$ where $y = e_1$
                \item Calculate residuals: $r_i = \|U^\top u_i\|^2 - c_i\|U^\top y\|^2$
            \end{enumerate}
        \item Update $U$ using Adam optimizer with gradient clipping
        \item Enforce fixed entries in $U$ after each update ($U[0,0] = 1$ \text{is enforced to be 1}.)
    \end{enumerate}
    where $B$ represents the batch of samples, $\lambda=10^{-4}$ is the regularization coefficient, and $y = e_1$ is the fixed unit vector.
    \end{algorithm}
    \end{minipage}
    \caption{Gradient-based optimization procedure for learning a dictionary decomposition $U$ with fixed entries.}
    \label{fig:gradient}
\end{figure}
% First Page of the Figure



% Pythia Code starts here
%\iffalse
\begin{figure*}[!]
    \centering
    % First Row of Subfigures
    \begin{subfigure}[b]{0.4\textwidth}
        \centering
        \includegraphics[width=\textwidth]{target_sae_pythia.png}
        \label{fig:sub1}
    \end{subfigure}
    \qquad
    \begin{subfigure}[b]{0.4\textwidth}
        \centering
        \includegraphics[width=\textwidth]{learnt_PCC-0.9367,feedbacks-135316.png}
        \label{fig: pythiaeigen}
    \end{subfigure}
    \caption{Feature learning on a subsampled dictionary of dimension $4500 \times 512$ of SAE trained for Pythia-70M. \thmref{thm: constructgeneral} states that Eigendecompostion method requires 135316 constructive feedback. After few 100 iterations of gradient descent as shown in \figref{fig:gradient}, a PCC of 93\% is achieved on ground truth. For visualization, only the first 100 dimensions are used.}
    \label{fig: subsample}
\end{figure*}

\begin{figure*}[t]\ContinuedFloat
    \centering

    % First Row of Subfigures
    \begin{subfigure}[b]{0.4\textwidth}
        \centering
        \includegraphics[width=\textwidth]{target_sae_pythiafull.png}
        \label{fig:sub3}
    \end{subfigure}
    \qquad
    \begin{subfigure}[b]{0.4\textwidth}
        \centering
        \includegraphics[width=\textwidth]{learnt_PCC-0.0242,feedbacks-200000.png}
        \label{fig: chessconst}
    \end{subfigure}
    \par % Start a new row

    % Second Row of Subfigures
    \begin{subfigure}[b]{0.4\textwidth}
        \centering
        \includegraphics[width=\textwidth]{learnt_PCC-0.3815,feedbacks-2000000.png}
        \label{fig:sub5}
    \end{subfigure}
\qquad    
    \begin{subfigure}[b]{0.4\textwidth}
        \centering
        \includegraphics[width=\textwidth]{learnt_PCC-0.5759,feedbacks-5000000.png}
        \label{fig:sub6}
    \end{subfigure}
    \par % Start a new row
\begin{subfigure}[b]{0.4\textwidth}
        \centering
        \includegraphics[width=\textwidth]{learnt_PCC-0.6570,feedbacks-10000000.png}
        \label{fig:sub5}
    \end{subfigure}
\qquad  
    % Third Row of Subfigures
    \begin{subfigure}[b]{0.4\textwidth}
        \centering
        \includegraphics[width=\textwidth]{learnt_PCC-0.7716,feedbacks-20000000.png}
        \label{fig:sub7}
    \end{subfigure}
    % \qquad
    % \begin{subfigure}[b]{0.4\textwidth}
    %     \centering
    %     \includegraphics[width=\textwidth]{ICML'25/Images/learnt_PCC- 0.9741, feedbacks- 10000000.png}
    %     \label{fig: chessssample}
    % \end{subfigure}

    \caption{\textbf{Sparse sampling for Pythia-70M}: Dimension of feature matrix: $32768 \times 512$ and the rank is 215. Plots for varying feedback complexity sizes. Note that $p(p+1)/2 \approx$ 512M. We run experiments with 3-sparse activations for uniform sparse distributions. The Pearson Correlation Coefficient (PCC) to feedback size (PCC, Feedback size) improves as follows: $(200k, .0242), (2M, .38), (5M, .54),(10M, .65)$, and $(20M, .77)$.
    %Plots show the improvement in PCC with 3-sparse uniformly sampled activations with respect to the Ground Truth shown above.
    }
    \label{fig: pythiasample}
\end{figure*}
%\fi
\newpage
\section{Technical Proofs}

\subsection{Theory of TD correction using biconjugate trick}\label{sec:BiconjProofs}


\begin{proof}[Proof of Lemma \ref{lem:OurBiconj}]
      \begin{align}
     &\mathcal{L}_{BE}(s,a)(Q):=\mathbb{E}_{ s^\prime \sim P(s, a)}\left[\delta_{Q}\left(s,a, s^\prime\right)\mid s, a\right]^2\notag
     \\
     &=\max_{h\in \mathbb{R}}2\cdot\mathbb{E}_{ s^\prime \sim P(s, a)}\left[\delta_{Q}\left(s,a, s^\prime\right)\mid s, a\right]\cdot h-h^2\tag{Biconjugate}
     \\
     &=\max_{h\in \mathbb{R}}2\cdot\mathbb{E}_{ s^\prime \sim P(s, a)}\left[\hat{\mathcal{T}}Q - Q\mid s, a\right]\cdot \underbrace{h}_{=\rho-Q(s,a)}-h^2\notag
    \\
     &=\max_{\rho(s,a)\in \mathbb{R}}\mathbb{E}_{ s^\prime \sim P(s, a)}\left[2\left(\hat{\mathcal{T}}Q - Q\right)\left(\rho-Q\right)
    -\left(\rho-Q\right)^2\mid s, a\right]\notag
    \\
    &= \max_{\rho(s,a)\in \mathbb{R}}\mathbb{E}_{ s^\prime \sim P(s, a)}\left[   \left(\hat{\mathcal{T}}{Q}-Q\right)^2-\left(\hat{\mathcal{T}}{Q}- \rho\right)^2\mid s, a\right]\label{eq:SBEED} 
    \end{align}
where the unique maximum is with 
\begin{align}
    \rho^{*}(s,a) &= h^{*}(s,a)+Q(s,a) = \mathcal{T}Q(s,a)- Q(s,a)+Q(s,a)\notag
    \\    &=\mathcal{T}Q(s,a)\notag
\end{align}
and where the equality of \ref{eq:SBEED} is from
\begin{align}
&\!\!\!\!\!\!\!\!\!2\left(\hat{\mathcal{T}}Q - Q\right)\left(\rho-Q\right)
    -\left(\rho-Q\right)^2\notag
    \\
    &=2(\hat{\mathcal{T}}{Q}\rho - \hat{\mathcal{T}}{Q}Q - \cancel{Q\rho} + Q^2)  - (\rho^2 - \cancel{2Q\rho} + Q^2)\notag
    \\
    &=2\hat{\mathcal{T}}{Q}\rho - 2\hat{\mathcal{T}}{Q}Q +\cancel{2} Q^2- \rho^2 - \cancel{Q^2}\notag
    \\
    &=\hat{\mathcal{T}}{Q}^2- 2\hat{\mathcal{T}}{Q}Q+Q^2-\hat{\mathcal{T}}{Q}^2+2\hat{\mathcal{T}}{Q}\rho- \rho^2\notag
    \\
    &=\left(\hat{\mathcal{T}}{Q}-Q\right)^2-\left(\hat{\mathcal{T}}{Q}- \rho\right)^2\notag
\end{align}

Now note that
    
     \begin{align}
     &\mathcal{L}_{BE}(s,a)(Q)= \max_{\rho(s,a)\in \mathbb{R}}\mathbb{E}_{ s^\prime \sim P(s, a)}\left[   \left(\hat{\mathcal{T}}{Q}-Q\right)^2-\left(\hat{\mathcal{T}}{Q}- \rho\right)^2\mid s, a\right]] \tag{equation \ref{eq:SBEED}}
    \\
    &= \mathbb{E}_{ s^\prime \sim P(s, a)}\left[ \left(\hat{\mathcal{T}}{Q}-Q\right)^2\mid s,a \right]-\min_{\rho(s,a)\in \mathbb{R}}\mathbb{E}_{ s^\prime \sim P(s, a)}\left[\left(\hat{\mathcal{T}}{Q}- \underbrace{\rho}_{= r+\beta \zeta}\right)^2\mid s, a\right]  \notag
    \\
    &= \mathbb{E}_{ s^\prime \sim P(s, a)}\left[\mathcal{L}_{TD}(Q)(s,a,s^\prime)\right]-\beta^2 \min_{\zeta\in \mathbb{R}}\mathbb{E}_{ s^\prime \sim P(s, a)}\left[\left(\hat{V}(s^\prime)- \zeta\right)^2\mid s, a\right]\label{eq:yesmin}
    \\
    &= \mathbb{E}_{ s^\prime \sim P(s, a)}\left[\mathcal{L}_{TD}(Q)(s,a, s^\prime)\right]-\beta^2 \mathbb{E}_{ s^\prime \sim P(s, a)}\left[\left(\hat{V}(s^\prime)- \mathbb{E}_{ s^\prime \sim P(s, a)}[\hat{V}(s^\prime)\mid s,a]\right)^2\mid s, a\right]\label{eq:nomin}
    \end{align}
where the equality of equation \ref{eq:nomin} comes from the fact that the $\zeta$ that maximize equation \ref{eq:yesmin} is $\zeta^* := \mathbb{E}_{ s^\prime \sim P(s, a)}[\hat{V}(s')\mid s,a]$, because
\begin{align}
r(s,a)+\beta \cdot \zeta^{*} (s,a) &:=  \rho^{*}(s,a)\notag
     \\
     &= \mathcal{T}Q(s,a) \notag
     \\
     &:=r(s,a)+\beta \cdot \mathbb{E}_{ s^\prime \sim P(s, a)}\left[\hat{V}(s^\prime)\mid s,a\right]\notag
\end{align}
For $Q^\ast$, $\mathcal{T}Q^\ast=Q^\ast$ holds. Therefore, we get
\begin{align}
r(s,a)+\beta \cdot \zeta^{*} (s,a) &:=  \rho^{*}(s,a)\notag
     \\
     &= \mathcal{T}Q^\ast(s,a) = Q^\ast(s,a) \notag
\end{align}
\end{proof}


\subsection{Proof of Theorem \ref{thm:MagnacThesmar}}\label{sec:PfMagnac}
\begin{proof}
    Suppose that the system of equations (Equation \ref{eq:HotzMillereqs})
\begin{equation}
\left\{
\begin{array}{l}
    \dfrac{\exp({Q}\left(s,a\right))}{\sum_{a^\prime\in \mathcal{A}} \exp({Q}\left(s,a^\prime\right))} = \pi^*(\;a
    \mid s) \; \; \; \forall s\in \mathcal{S}, a\in\mathcal{A}
    \\[1em]
    r(s, a_s)+\beta \cdot \mathbb{E}_{s^{\prime} \sim P(s, a_s)}\left[\log(\sum_{a^\prime\in\mathcal{A}}\exp Q(s^\prime, a^\prime)) \mid s, a_s\right]-Q(s, a_s)=0 \;\; \; \forall s\in \mathcal{S} 
\end{array}
\right.
\notag
\end{equation} 
is satisfied for $Q\in \mathcal{Q}$, where $\mathcal{Q}$ denote the space of all $Q$ functions. Then we have the following equivalent recharacterization of the second condition $\forall s\in \mathcal{S}$,
\begin{align}
     Q(s, a_s)&=r(s, a_s)+\beta \cdot \mathbb{E}_{s^{\prime} \sim P(s, a_s)}\left[\log(\sum_{a^\prime\in\mathcal{A}}\exp Q(s^\prime, a^\prime)) \mid s, a_s\right]\;\; \; \notag
     \\
     &=  r(s, a_s)+\beta \cdot \mathbb{E}_{s^\prime \sim P(s, a_s)}\left[Q(s^\prime, a^\prime) - \log \pi^*(a^\prime \mid s^\prime) \mid s, a_s\right] \;\; \forall a^\prime\in\mathcal{A} \notag
     \\
     &=  r(s, a_s)+\beta \cdot \mathbb{E}_{s^\prime \sim P(s, a_s)}\left[Q(s^\prime, a_{s^\prime}) - \log \pi^*(a_{s^\prime} \mid s^\prime) \mid s, a_s\right]
\end{align}

We will now show the existence and uniqueness of a solution using a standard fixed point argument on a Bellman operator. Let $\mathcal{F}$ be the space of functions $f: \mathcal{S} \rightarrow \mathbb{R}$ induced by elements of $\mathcal{Q}$, where each $Q \in \mathcal{Q}$ defines an element of $\mathcal{F}$ via

$$
f_Q(s):=Q\left(s, a_s\right)
$$

and define an operator $\mathcal{T}_f: \mathcal{F} \rightarrow$ $\mathcal{F}$ that acts on functions $f_Q$ :

$$
\left(\mathcal{T}_f f_Q\right)(s):=r\left(s, a_s\right)+\beta \sum_{s^{\prime}} P\left(s^{\prime} \mid s, a_s\right)\left[f_Q\left(s^{\prime}\right)-\log \pi^*\left(a_{s^{\prime}} \mid s^{\prime}\right)\right]
$$

Then for $Q_1, Q_2 \in\mathcal{Q}$, We have
\begin{align} & \left(\mathcal{T}_f f_{Q_1}\right)(s):=r\left(s, a_s\right)+\beta \sum_{s^{\prime}} P\left(s^{\prime} \mid s, a_s\right)\left[f_{Q_1}\left(s^{\prime}\right)-\log \pi^*\left(a_{s^{\prime}} \mid s^{\prime}\right)\right] \notag
\\ & \left(\mathcal{T}_f f_{Q_2}\right)(s):=r\left(s, a_s\right)+\beta \sum_{s^{\prime}} P\left(s^{\prime} \mid s, a_s\right)\left[f_{Q_2}\left(s^{\prime}\right)-\log \pi^*\left(a_{s^{\prime}} \mid s^{\prime}\right)\right]\notag
\end{align}

Subtracting the two, we get
\begin{align}
\left|\left(\mathcal{T}_f f_{Q_1}\right)(s)-\left(\mathcal{T}_f f_{Q_2}\right)(s)\right| 
&\leq \beta \sum_{s^{\prime}} P\left(s^{\prime} \mid s, a_s\right)\left|f_{Q_1}\left(s^{\prime}\right)-f_{Q_2}\left(s^{\prime}\right)\right| \notag
\\
&\leq \beta\left\|f_{Q_1}-f_{Q_2}\right\|_{\infty} \notag
\end{align}

Taking supremum norm over $s\in\mathcal{S}$, we get
$$\left\|\mathcal{T}_f f_{Q_1}-\mathcal{T}_f f_{Q_2}\right\|_{\infty} \leq \beta\left\|f_{Q_1}-f_{Q_2}\right\|_\infty$$

This implies that $\mathcal{T}_f$ is a contraction mapping under supremum norm, with $\beta\in (0,1)$. Since $\mathcal{Q}$ is a Banach space under sup norm (Lemma \ref{lem:completeMetric}), we can apply Banach fixed point theorem to show that there exists a unique $f_Q$ that satisfies $\mathcal{T}_f(f_Q) = f_Q$, and by definition of $f_Q$ there exists a unique $Q$ that satisfies $\mathcal{T}_f(f_Q) = f_Q$, i.e., 
$$r\left(s, a_s\right)+\beta \cdot \mathbb{E}_{s^{\prime} \sim P\left(s, a_s\right)}\left[\log \left(\sum_{a^{\prime} \in \mathcal{A}} \exp Q\left(s^{\prime}, a^{\prime}\right)\right) \mid s, a_s\right]-Q\left(s, a_s\right)=0 \quad \forall s \in \mathcal{S}$$

Since $Q^\ast$ satisfies the system of equations \ref{eq:HotzMillereqs}, $Q^\ast$ is the only solution to the system of equations.

Also, since $Q^\ast = \mathcal{T}Q^\ast = r(s,a)+\beta \cdot \mathbb{E}_{s^{\prime} \sim P(s, a)}\bigl[\log(\sum_{a^\prime\in\mathcal{A}}\exp Q^\ast(s^\prime, a^\prime)) \mid s, a\bigr]$ holds, we can identify $r$ as
\begin{align}
    r(s,a) &= Q^\ast(s, a) - \beta \cdot \mathbb{E}_{s^{\prime} \sim P(s, a)}\bigl[\log(\sum_{a^\prime\in\mathcal{A}}\exp Q^\ast(s^\prime, a^\prime)) \mid s, a\bigr] \notag
\end{align}


\end{proof}
\begin{lem}\label{lem:completeMetric} Suppose that $\mathcal{Q}$ consists of bounded functions on $\mathcal{S} \times \mathcal{A}$. Then $\mathcal{Q}$ is a Banach space with the supremum norm as the induced norm.
\end{lem}
\begin{proof}
Suppose a sequence of functions $\left\{Q_n\right\}$ in $\mathcal{Q}$ is Cauchy in the supremum norm. We must show that $ Q_n\rightarrow Q^\ast$ as $n\rightarrow \infty$ for some $Q^\ast$ and $Q^\ast$ is also bounded. %because $\left|Q^*(s, a)\right|=\lim _{n \rightarrow \infty}\left|Q_n(s, a)\right| \leq M$. 
Note that $Q_n$ being Cauchy in sup norm implies that for every $(s, a)$, the sequence $\left\{Q_n(s, a)\right\}$ is Cauchy in $\mathbb{R}$. Since $\mathbb{R}$ is a complete space, every Cauchy sequence of real numbers has a limit; this allows us to define function $Q^\ast:\mathcal{S}\times \mathcal{A} \mapsto \mathbb{R}$ such that $Q^*(s, a)=\lim _{n \rightarrow \infty} Q_n(s, a)$. Then we can say that $Q_n(s, a) \rightarrow Q^*(s, a) $ for every $(s, a) \in \mathcal{S} \times \mathcal{A}$. Since each $Q_n$ is bounded, we take the limit and obtain:
$$
\sup _{s, a}\left|Q^*(s, a)\right|=\lim _{n \rightarrow \infty} \sup _{s, a}\left|Q_n(s, a)\right| \leq M
$$
which implies $Q^* \in \mathcal{Q}$.

\noindent Now what's left is to show that the supremum norm
$$
\|Q\|_{\infty}=\sup _{(s, a) \in \mathcal{S} \times \mathcal{A}}|Q(s, a)|
$$
induces the metric, i.e., 
$$
d\left(Q_1, Q_2\right):=\left\|Q_1-Q_2\right\|_{\infty}=\sup _{(s, a) \in \mathcal{S} \times \mathcal{A}}\left|Q_1(s, a)-Q_2(s, a)\right|
$$
The function $d$ satisfies the properties of a metric:

- Non-negativity: $d\left(Q_1, Q_2\right) \geq 0$ and $d\left(Q_1, Q_2\right)=0$ if and only if $Q_1=$ $Q_2$.

- Symmetry: $d\left(Q_1, Q_2\right)=d\left(Q_2, Q_1\right)$ by the absolute difference.

- Triangle inequality:
$$
d\left(Q_1, Q_3\right)=\sup _{s, a}\left|Q_1(s, a)-Q_3(s, a)\right| \leq \sup _{s, a}\left|Q_1(s, a)-Q_2(s, a)\right|+\sup _{s, a}\left|Q_2(s, a)-Q_3(s, a)\right|
$$

which shows $d\left(Q_1, Q_3\right) \leq d\left(Q_1, Q_2\right)+d\left(Q_2, Q_3\right)$.


\end{proof}

\subsection{Proof of Theorem 
\ref{thm:mainopt}}\label{sec:pfOfmainOpt}

Define $\hat{Q}$ as 
%
\begin{align}
    \hat{Q} &\in \underset{Q\in \mathcal{Q}}{\arg\min } \; \;\mathbb{E}_{(s, a)\sim \pi^*, \nu_0}  \left[-\log\left(\hat{p}_{Q}(a
\mid s)\right)\right] + \lambda\mathbb{E}_{(s, a)\sim \pi^*, \nu_0}\left[ \mathbbm{1}_{a = a_s} \mathcal{L}_{BE}(Q)(s,a)\right] \tag{Equation \ref{eq:mainopt}}
\end{align}
From Theorem \ref{thm:MagnacThesmar}, it is sufficient to show that $\hat{Q}$
satisfies the equations \ref{eq:HotzMillereqs} of Theorem \ref{thm:MagnacThesmar} for any $\lambda>0$, i.e., 
\begin{equation}
\left\{
\begin{array}{l}
    \dfrac{\exp({\hat{Q}}\left(s,a\right))}{\sum_{a^\prime\in \mathcal{A}} \exp({\hat{Q}}\left(s,a^\prime\right))} = \pi^*(a
    \mid s) \; \; \; \forall s\in \bar{\mathcal{S}}, a \in \mathcal{A}
    \\[1em]
    r(s, a_s)+\beta \cdot \mathbb{E}_{s^{\prime} \sim P(s, a_s)}\left[\log(\sum_{a^\prime\in\mathcal{A}}\exp \hat{Q}(s^\prime, a^\prime)) \mid s, a_s\right]-\hat{Q}(s, a_s)=0 \;\;\; \forall s\in \bar{\mathcal{S}}
    
\end{array}\tag{Equation \ref{eq:HotzMillereqs}}
\right. 
\end{equation}
where $\bar{\mathcal{S}}$ (the reachable states from $\nu_0$, $\pi^\ast$) was defined as:
$$
\bar{\mathcal{S}}=\left\{s \in \mathcal{S} \mid \operatorname{Pr}\left(s_t=s \mid s_0 \sim \nu_0, \pi^*\right)>0 \text { for some } t \geq 0\right\} 
$$
Now note that:
  \begin{align}
     &\left\{Q \in \mathcal{Q} \mid\hat{p}_{Q}(\;\cdot
    \mid s) = \pi^*(\;\cdot
    \mid s)\quad  \forall s\in\bar{\mathcal{S}}\quad\text{a.e.}\right\} \notag
    \\
    &=\underset{Q\in \mathcal{Q}}{\arg\max } \; \;\mathbb{E}_{(s, a)\sim \pi^*, \nu_0}  \left[\log\left(\hat{p}_{Q}(\;\cdot
    \mid s)\right)\right] \tag{$\because$ Lemma \ref{lem:minMLE}}
    \\
    &=\underset{Q\in \mathcal{Q}}{\arg\min } \; \;\mathbb{E}_{(s, a)\sim \pi^*, \nu_0}  \left[-\log\left(\hat{p}_{Q}(\;\cdot
    \mid s)\right)\right] \notag
    \end{align}
and 
    \begin{align}
     &\left\{Q \in \mathcal{Q} \mid\mathcal{L}_{BE}(Q)(s,a_s) = 0\quad  \forall s\in\bar{\mathcal{S}}\right\} \notag
    \\
    &=\underset{Q\in \mathcal{Q}}{\arg\min } \; \;\mathbb{E}_{(s, a)\sim \pi^*, \nu_0}  \left[\mathbbm{1}_{a = a_s} \mathcal{L}_{BE}(Q)(s,a)\right] \notag
    \end{align}
Therefore what we want to prove, equations \ref{eq:HotzMillereqs}, becomes the following equation \ref{eq:modifiedHotz}:

\begin{equation}
\left\{
\begin{array}{l}
    \hat{Q} \in \underset{Q\in \mathcal{Q}}{\arg\min } \; \;\mathbb{E}_{(s, a)\sim \pi^*, \nu_0}  \left[-\log\left(\hat{p}_{Q}(\;\cdot
    \mid s)\right)\right] 
    \\[1em]
     \hat{Q} \in \underset{Q\in \mathcal{Q}}{\arg\min } \; \;\mathbb{E}_{(s, a)\sim \pi^*, \nu_0}  \left[\mathbbm{1}_{a = a_s} \mathcal{L}_{BE}(Q)(s,a)\right]
    
\end{array}\label{eq:modifiedHotz}
\right. 
\end{equation}
where its solution set is nonempty by Theorem \ref{thm:MagnacThesmar}, i.e., 
$$ \underset{Q\in \mathcal{Q}}{\arg\min } \; \;\mathbb{E}_{(s, a)\sim \pi^*, \nu_0} \left[-\log\left(\hat{p}_{Q}(a
\mid s)\right)\right] \;\; \cap \;\; \underset{Q\in \mathcal{Q}}{\arg\min } \; \; \mathbb{E}_{(s, a)\sim \pi^*, \nu_0} \left[\mathbbm{1}_{a = a_s}\mathcal{L}_{BE}(\hat{Q})(s,a)\right] \;\; \neq \;\; \emptyset$$ 

Under this non-emptiness, according to Lemma \ref{lem:sharingsol}, $\hat{Q}$ satisfies equation \ref{eq:modifiedHotz}. This implies that $\hat{Q}(s,a) = Q^\ast(s,a)$ for $s\in\bar{\mathcal{S}}$ and $a\in\mathcal{A}$, as the solution to set of equations \ref{eq:HotzMillereqs} is $Q^*$. This implies that 
\begin{align}
    r(s, a)=\hat{Q}(s, a)-\beta \cdot \mathbb{E}_{s^{\prime} \sim P(s, a)}\left[\log \left(\sum_{a^{\prime} \in \mathcal{A}} \exp \hat{Q}\left(s^{\prime}, a^{\prime}\right)\right) \mid s, a\right] \notag
\end{align}
for $s\in\bar{\mathcal{S}}$ and $a\in\mathcal{A}$.
\QED


\begin{lem}\label{lem:minMLE}
    \begin{align}
           \underset{Q\in \mathcal{Q}}{\arg\max } &\; \;\mathbb{E}_{(s, a)\sim \pi^*, \nu_0}  \left[\log\left(\hat{p}_{Q}(\;\cdot
    \mid s)\right)\right] \notag
    \\
     &=\left\{Q \in \mathcal{Q} \mid\hat{p}_{Q}(\;\cdot
    \mid s) = \pi^*(\;\cdot
    \mid s)\quad  \forall s\in\bar{\mathcal{S}}\quad\text{a.e.}\right\}\notag
    \\
     &=\left\{Q \in \mathcal{Q} \mid Q(s,a_1)-Q(s,a_2)= Q^*(s,a_1)-Q^*(s,a_2) \quad \forall a_1, a_2\in\mathcal{A}, s\in\bar{\mathcal{S}}\right\} \notag
    \end{align}
\end{lem}

\begin{proof}[Proof of Lemma \ref{lem:minMLE}]
    \begin{align}
    \mathbb{E}_{(s, a)\sim \pi^*, \nu_0}  \left[\log\left(\hat{p}_{Q}(\;\cdot
    \mid s)\right)\right] & = 
\mathbb{E}_{(s,a)\sim\pi^*, \nu_0} [\log \hat{p}_{Q}(a|s) - \ln \pi^*(a|s) + \ln \pi^*(a|s)]\notag \\
    &=-\mathbb{E}_{(s,a)\sim\pi^*, \nu_0} \left[\ln \frac{\pi^*(a|s)}{\hat{p}_{Q}(a|s)} \right] + \mathbb{E}_{(s,a)\sim\pi^*, \nu_0} [\ln \pi^*(a|s)]\notag \\
    &= -\mathbb{E}_{s\sim\pi^*, \nu_0} \left[D_{KL}(\pi^*(\cdot\mid s) \| \hat{p}_{Q}(\cdot\mid s))\right] + \mathbb{E}_{(s,a)\sim\pi^*, \nu_0} [\ln \pi^*(a|s)]\notag \notag
\end{align}
Therefore,
\begin{align}
    \underset{Q\in\mathcal{Q}}{\arg\max}&\; \mathbb{E}_{(s, a)\sim \pi^*, \nu_0}  \left[\log\left(\hat{p}_{Q}(\;\cdot
    \mid s)\right)\right] =\underset{Q\in\mathcal{Q}}{\arg\min}\;\mathbb{E}_{s\sim\pi^*, \nu_0} \left[D_{KL}(\pi^*(\cdot\mid s) \| \hat{p}_{Q}(\cdot\mid s))\right]\notag
    \\
    &=\{Q\in\mathcal{Q}\mid D_{KL}(\pi^*(\cdot\mid s) \| \hat{p}_{Q}(\cdot\mid s))=0 \text{ for all }s\in\bar{\mathcal{S}}\} \tag{$\because \; Q^*\in \mathcal{Q}$ and $D_{KL}(\pi^* \| \pi^*)=0$}\notag
    \\
    &=\{Q\in\mathcal{Q}\mid \hat{p}_Q(\cdot \mid s)=\pi^*(\cdot \mid s)\; \; \text{a.e.} \text{ for all }s\in\bar{\mathcal{S}}\}\notag
    \\
    &= \{Q\in\mathcal{Q}\mid \frac{\hat{p}_Q\left(a_1 \mid s\right)}{\hat{p}_Q\left(a_2 \mid s\right)}=\frac{\pi^*\left(a_1 \mid s\right)}{\pi^*\left(a_2 \mid s\right)} \quad \forall a_1, a_2 \in \mathcal{A}, s\in\bar{\mathcal{S}}\}\notag   
    \\
    &=\left\{Q \in \mathcal{Q} \mid \exp (Q(s,a_1)-Q(s,a_2))=\exp \left(Q^*(s,a_1)-Q^*(s,a_2)\right) \quad \forall a_1, a_2\in\mathcal{A}, s\in\bar{\mathcal{S}} \right\}\notag
    \\
    &=\left\{Q \in \mathcal{Q} \mid Q(s,a_1)-Q(s,a_2)= Q^*(s,a_1)-Q^*(s,a_2) \quad \forall a_1, a_2\in\mathcal{A}, s\in\bar{\mathcal{S}}\right\} \notag
\end{align}
\end{proof}

\begin{lem}\label{lem:sharingsol}
    Let $f_1: \mathcal{X} \rightarrow \mathbb{R}$ and $f_2: \mathcal{X} \rightarrow \mathbb{R}$ be two functions defined on a common domain $\mathcal{X}$. Suppose the sets of minimizers of $f_1$ and $f_2$ intersect, i.e.,
$$
\arg \min f_1 \cap \arg \min f_2 \neq \emptyset
$$
Then, any minimizer of the sum $f_1+f_2$ is also a minimizer of both $f_1$ and $f_2$ individually. That is, if
$$
x^* \in \arg \min \left(f_1+f_2\right)
$$
then
$$
x^* \in \arg \min f_1 \; \cap \; \arg \min f_2
$$
\end{lem}
\begin{proof}
    Since \( \arg\min f_1 \cap \arg\min f_2 \neq \emptyset \), let \( x^\dagger \) be a common minimizer such that $$
x^\dagger \in \arg\min f_1 \cap \arg\min f_2$$
This implies that  
\begin{align}
    f_1(x^\dagger) &= \min_{x \in \mathcal{X}} f_1(x) =: m_1, \notag
    \\
    f_2(x^\dagger) &= \min_{x \in \mathcal{X}} f_2(x) =: m_2. \notag
\end{align}

Now, let \( x^* \) be any minimizer of \( f_1 + f_2 \), so  
\begin{align}
    x^* \in \arg\min (f_1 + f_2) &\iff f_1(x^*) + f_2(x^*) \leq f_1(x) + f_2(x), \quad \forall x \in \mathcal{X}. \notag
\end{align}
Evaluating this at \( x^\dagger \), we obtain  
\begin{align}
    f_1(x^*) + f_2(x^*) &\leq f_1(x^\dagger) + f_2(x^\dagger) \notag
    \\
    &= m_1 + m_2. \notag
\end{align}

Now, suppose for contradiction that $x^* \notin \arg\min f_1$, 
meaning  
\begin{align}
    f_1(x^*) &> m_1 \notag
\end{align}
But then
\begin{align}
    f_2(x^*) & \le m_1 + m_2 - f_1(x^*) \notag
    \\
    &< m_1 + m_2 - m_1 = m_2 \notag
\end{align}

This contradicts the fact that \( m_2 = \min f_2 \), so \( x^* \) must satisfy  
\begin{align}
    f_1(x^*) &= m_1 \notag
\end{align}

By symmetry, assuming $x^* \notin \arg\min f_2$ leads to the same contradiction, forcing  
\begin{align}
    f_2(x^*) &= m_2 \notag
\end{align}

Thus, we conclude  
\begin{align}
    x^* \in \arg\min f_1 \cap \arg\min f_2 \notag
\end{align}
\end{proof}

%%%%%% Mirror descent %%%%%%%%%%%

\iffalse

\subsection{Proof of Theorem \ref{thm:mirror}}\label{sec:Proofofmirror}

\begin{thm}[Mirror descent equivalence]\label{thm:mirror}
    Equation \ref{eq:mainopt} is equivalent to the mirror descent algorithm for minimizing Bellman error only, i.e., 
    
    \begin{align}
       \underset{Q \in \mathcal{Q}}{\arg\min} &\; \mathbb{E}_{(s, a) \sim \pi^*, \nu_0, s' \sim P(s, a)} \left[\mathbbm{1}_{a = a_s} \left( \mathcal{L}_{TD}(Q)(s,a,s') - \beta^2 D(Q)(s,a)\right) \right]\notag
    \end{align}
   with Bregman divergence associated with $F$, where $F(p)=\sum_{a \in \mathcal{A}} p(a \mid s) \log(p(a \mid s))$, which defines the negative Shannon entropy of a distribution $p(a \mid s)$.
\end{thm}
See Appendix \ref{sec:Proofofmirror} for the proof. Note that $F$ makes the Bregman divergence be $D_{KL}\left(\pi^* \| \hat{p}_Q\right):=\sum_a \pi^*(a \mid s) \log \left(\frac{\pi^*(a \mid s)}{\hat{p}_Q(a \mid s)}\right)$, the mirror map be $\phi(p)=\log(p)$, and the inverse mirror map be the softmax transformation.

\begin{align}
    \hat{Q} &\in  \underset{Q\in \mathcal{Q}}{\arg\min } \; \;\mathbb{E}_{(s, a)\sim \pi^*, \nu_0,  s^\prime \sim P(s, a)}  \left[-\log\left(\hat{p}_{Q}(a
\mid s)\right) + \lambda \mathbbm{1}_{a = a_s}\left( \mathcal{L}_{TD}(Q)(s,a,s^\prime)-\beta^2D(Q)(s,a)\right)\right]\tag{Equation \ref{eq:mainopt}}
\\
&= \underset{Q\in \mathcal{Q}}{\arg\min } \; \;\mathbb{E}_{(s, a)\sim \pi^*, \nu_0}  \left[-\log\left(\hat{p}_{Q}(a
\mid s)\right)\right] + \lambda\mathbb{E}_{(s, a)\sim \pi^*, \nu_0}\left[ \mathbbm{1}_{a = a_s} \mathcal{L}_{BE}(Q)(s,a)\right] \tag{$\because$ Lemma \ref{thm:BEbiconjWrtV}}
\\
&= \underset{Q\in \mathcal{Q}}{\arg\min } \; \; \lambda\mathbb{E}_{(s, a)\sim \pi^*, \nu_0}\left[ \mathbbm{1}_{a = a_s} \mathcal{L}_{BE}(Q)(s,a)\right] + D_{KL}(\pi^*(\cdot\mid s) \| \pi_{Q}(\cdot\mid s)) \notag
\\
&\quad\;\; \quad \text{s.t.} \quad \;\{Q\in\mathcal{Q}\mid \pi_Q = \pi^* \; \; \text{a.e.} \text{ for all }s\in\mathcal{S}\}\tag{$\because$Proof of Lemma \ref{lem:minMLE}}
\end{align}
Define $f(Q) := \lambda\mathbb{E}_{(s, a)\sim \pi^*, \nu_0}\left[ \mathbbm{1}_{a = a_s} \mathcal{L}_{BE}(Q)(s,a)\right]$, $f^\prime(Q) := \lambda\mathbb{E}_{(s, a)\sim \pi^*, \nu_0}\left[ \mathbbm{1}_{a = a_s} \mathcal{L}_{BE}(Q)(s,a)\right] + D_{KL}(\pi^*(\cdot\mid s) \| \pi_{Q}(\cdot\mid s))$ and $K:=\{Q\in\mathcal{Q}\mid \pi_Q = \pi^* \; \; \text{a.e.} \text{ for all }s\in\mathcal{S}\}$. Then the mirror descent method of minimization of $f^\prime(Q)$ subject to constraint $K$ is defined as 
\begin{align}
    Q_{t+1} \leftarrow \operatorname{\argmin}_{Q\in K}\left\{\eta\left\langle\nabla f^\prime\left(x_t\right), x\right\rangle+D_h\left(x \| x_t\right)\right\}
\end{align}

\begin{align}
&\Phi_{t+1}-\Phi_t=  \frac{1}{\eta}\left(h\left(x_t\right)-h\left(x_{t+1}\right)-\frac{1}{2}\left\langle\nabla h\left(x_t\right), x^*-x_{t+1}\right\rangle-\frac{1}{2}\left\langle\nabla h\left(x^*\right), x^*-x_{t+1}\right\rangle\right. \\
& \left.+\frac{\eta}{2}\left\langle\nabla f_t\left(x_t\right), x^*-x_{t+1}\right\rangle+\left\langle\nabla h\left(x_t\right), x^*-x_t\right\rangle\right)
\\
&= \frac{1}{\eta}\left(h\left(x_t\right)-h\left(x_{t+1}\right)-\frac{1}{2}\left\langle\nabla h\left(x_t\right), x^*-x_{t+1}\right\rangle-\frac{1}{2}\left\langle\nabla h\left(x_{t}\right), x^*-x_{t+1}\right\rangle\right. \\
& \left.+\frac{\eta}{2}\left\langle\nabla f_t\left(x_t\right), x^*-x_{t+1}\right\rangle+\left\langle\nabla h\left(x_t\right), x^*-x_t\right\rangle+ \frac{1}{2  } \left\langle\nabla h\left(x^*\right)-\nabla h\left(x_t\right), x^*-x_{t+1}\right\rangle\right) 
\\
&= \frac{1}{\eta}\left(h\left(x_t\right)-h\left(x_{t+1}\right)-\left\langle\nabla h\left(x_t\right), x_t-x_{t+1}\right\rangle+\frac{\eta}{2}\left\langle\nabla f_t\left(x_t\right), x^*-x_{t+1}\right\rangle+ \frac{1}{2  } \left\langle\nabla h\left(x^*\right)-\nabla h\left(x_t\right), x^*-x_{t+1}\right\rangle\right) 
\\
&= \frac{1}{\eta}\left(h\left(x_t\right)-h\left(x_{t+1}\right)-\left\langle\nabla h\left(x_t\right), x_t-x_{t+1}\right\rangle + \frac{\eta}{2}\left\langle\nabla f_t\left(x_t\right), x^*-x_{t}\right\rangle\right. \\
& \left.+\frac{\eta}{2}\left\langle\nabla f_t\left(x_t\right), x_t-x_{t+1}\right\rangle+ \frac{1}{2  } \left\langle\nabla h\left(x^*\right)-\nabla h\left(x_t\right), x^*-x_{t}\right\rangle+\frac{1}{2  } \left\langle\nabla h\left(x^*\right)-\nabla h\left(x_t\right), x_t-x_{t+1}\right\rangle\right) \notag
\\
&= \frac{1}{\eta}\left(h\left(x_t\right)-h\left(x_{t+1}\right)-\left\langle\nabla h\left(x_t\right), x_t-x_{t+1}\right\rangle + \frac{\eta}{2}\left\langle\nabla f_t\left(x_t\right), x^*-x_{t}\right\rangle\right. \\
& \left.+ \frac{1}{2  } \left\langle\nabla h\left(x^*\right)-\nabla h\left(x_t\right), x^*-x_{t}\right\rangle+\frac{1}{2  } \left\langle \eta \nabla f_t + \nabla h\left(x^*\right)-\nabla h\left(x_t\right), x_t-x_{t+1}\right\rangle\right) 
\\
&= \frac{1}{\eta}\left(h\left(x_t\right)-h\left(x_{t+1}\right)-\left\langle\nabla h\left(x_t\right), x_t-x_{t+1}\right\rangle + \frac{\eta}{2}\left\langle\nabla f_t\left(x_t\right), x^*-x_{t}\right\rangle\right. \\
& \left.+\frac{1}{2  } \left\langle\nabla h\left(x^*\right)-\nabla h\left(x_t\right), x^*-x_{t}\right\rangle+\left\langle \nabla h(x^*)-\nabla h(x_{t+1}), x_t-x_{t+1}\right\rangle\right) \notag 
\end{align}

\fi
%%%%%% End of Mirror descent %%%%%%%%%%%


\subsection{Proof of Lemma \ref{lem:ConvexityMLE}}
\begin{proof}[Proof of Lemma \ref{lem:ConvexityMLE}] \label{sec:NLLproperties}
Denote $Q(s, \cdot)=\left[Q\left(s, a^{\prime}\right)\right]_{a^{\prime} \in \mathcal{A}}$. Then,
\begin{align}
    \text{Convexity}& \text{ of }  \mathbb{E}_{(s, a)\sim \pi^*, \nu_0}  \left[-\log\left(\hat{p}_{Q}(\;\cdot
    \mid s)\right)\right] \text{ w.r.t. }Q\in\mathcal{Q}\notag
    \\
    &\iff \text{Concavity of } \mathbb{E}_{(s,a)\sim\pi^*, \nu_0}\left[\ln \hat{p}_{Q}\left( \cdot \mid s\right)\right] \text{ w.r.t. } Q\in\mathcal{Q} \notag
    \\
    &\;\Longleftarrow \text{Concavity of } \ln \hat{p}_{Q}\left(\cdot \mid s\right)\text{ w.r.t. } Q\in\mathcal{Q} \text{ for all } s\in\mathcal{S} \tag{$\because$ linearity of expectation}
    \\
    &\iff \text{Concavity of }  Q(s, \cdot)-log \sum_{a^{\prime}\in\mathcal{A}} \exp \left(Q\left(s, a^{\prime}\right)\right) \text{ w.r.t. } Q(s, \cdot)  \text{ for all } s\in\mathcal{S} \notag
    \\
    &\iff \text{Convexity of } log \sum_{a^{\prime}\in\mathcal{A}} \exp \left(Q\left(s, a^{\prime}\right)\right) \text{ w.r.t. } Q(s, \cdot)  \text{ for all } s\in\mathcal{S}  \notag
\end{align}
Since the function logsumexp is a known convex function, we are done.

\begin{align}
    &\text{Lipschitz} \text{ smoothness } \text{of }  \mathbb{E}_{(s, a)\sim \pi^*, \nu_0}  \left[\log\left(\hat{p}_{Q}(\;\cdot
    \mid s)\right)\right] \text{ w.r.t. }Q\in\mathcal{Q}\notag
    \\
    &\iff \text{ Lipschitz continuity of } \nabla_Q \; \mathbb{E}_{(s,a)\sim\pi^*, \nu_0}\left[\ln \hat{p}_{Q}\left( \cdot \mid s\right)\right] \text{ w.r.t. } Q\in\mathcal{Q} \notag
    \\
    &\iff \text{ Lipschitz continuity of } \mathbb{E}_{(s, a) \sim \pi^*, \nu_0}\left[\delta_{a, a^{\prime}}-\hat{p}_Q\left(a^{\prime} \mid s\right)\right]_{a^{\prime}\in\mathcal{A}} \notag
    \\
    &\iff \text{ Lipschitz continuity of } \mathbb{E}_{s \sim \pi^*, \nu_0}\left[\pi^*\left(a^{\prime}\mid s\right) -\hat{p}_Q\left(a^{\prime} \mid s\right)\right]_{a^{\prime}\in\mathcal{A}} \notag
    \\
    &\iff \exists \; c>0 \; s.t. \;  \|\mathbb{E}_{s \sim \pi^*, \nu_0}\left[\hat{p}_{Q^\prime}\left(a^{\prime}\mid s\right) -\hat{p}_Q\left(a^{\prime} \mid s\right)\right]_{a^{\prime}\in\mathcal{A}}\| \le c\|Q-Q^\prime\|_{L_2(\pi^\ast, \nu_0)} \quad \forall Q, Q^\prime \in\mathcal{Q} \notag
\end{align}
Since softmax is 1-Lipschitz continuous for each $s\in\mathcal{S}$ with respect to $\ell_2$ norm \cite{gao2017properties}, for all $s\in\mathcal{S}$ we have
$$
\left\|\hat{p}_{Q^{\prime}}(\cdot \mid s)-\hat{p}_Q(\cdot \mid s)\right\|_2 \leq \left\|Q^{\prime}(s, \cdot)-Q(s, \cdot)\right\|_2
$$
Therefore
\begin{align}
  \|\mathbb{E}_{s \sim \pi^*, \nu_0}\left[\hat{p}_{Q^\prime}\left(\cdot\mid s\right) -\hat{p}_Q\left(\cdot\mid s\right)\right]\|_2 & \leq \mathbb{E}_{s \sim \pi^*, \nu_0}\bigl[\left\|\hat{p}_{Q^{\prime}}(\cdot \mid s)-\hat{p}_Q(\cdot \mid s)\right\|_2\bigr] \tag {Norm is convex}
  \\
  &\leq \mathbb{E}_{s \sim \pi^*, \nu_0}\left[\left\|Q^{\prime}(s, \cdot)-Q(s, \cdot)\right\|_2\right] \tag{Softmax is 1-Lipschitz}
  \\
  &\leq\left(\mathbb{E}_{s \sim \pi^*, \nu_0}\left\|Q^{\prime}(s, \cdot)-Q(s, \cdot)\right\|_2^2\right)^{1 / 2} \tag{$x^{1/2}$ is concave}
  \\
  &=\left\|Q-Q^{\prime}\right\|_{L_2(\pi^\ast, \nu_0)} \notag
\end{align}


\iffalse
Lastly for Lipschitz continuity, 
$$
\log \hat{p}_Q(a \mid s)=Q(s, a)-\log \sum_{a^{\prime} \in \mathcal{A}} \exp \left(Q\left(s, a^{\prime}\right)\right)
$$
This implies
\begin{align}
    \!\!\!\!\!\!\!&\left|\log \hat{p}_{Q_1}(a \mid s)-\log \hat{p}_{Q_2}(a \mid s)\right|
    \\
    &=\left|\left[Q_1(s, a)-Q_2(s, a)\right]-\left[\log \sum_{a^{\prime}} \exp \left(Q_1\left(s, a^{\prime}\right)\right)-\log \sum_{a^{\prime}} \exp \left(Q_2\left(s, a^{\prime}\right)\right)\right]\right|
    \\
    &\le \left|\left[Q_1(s, a)-Q_2(s, a)\right]\right|+\left|\left[\log \sum_{a^{\prime}} \exp \left(Q_1\left(s, a^{\prime}\right)\right)-\log \sum_{a^{\prime}} \exp \left(Q_2\left(s, a^{\prime}\right)\right)\right]\right|
    \\
    &\le \left\|Q_1-Q_2\right\|_{\infty} + \left\|Q_1-Q_2\right\|_{\infty} = 2 \left\|Q_1-Q_2\right\|_{\infty} 
\end{align}
Therefore
$$\left|\mathbb{E}_{(s, a) \sim \pi^*, \nu_0}\left[\log \hat{p}_{Q_1}(a \mid s)\right]-\mathbb{E}_{(s, a) \sim \pi^*, \nu_0}\left[\log \hat{p}_{Q_2}(a \mid s)\right]\right| \leq 2\left\|Q_1-Q_2\right\|_{\infty}$$

\fi
\end{proof}

\subsection{Proof of Lemma \ref{lem:BELipschitz} (Properties of Bellman error)}
For showing that $\overline{\mathcal{L}_{BE}}(Q)$ is of $\mathcal{C}^2 \text{ w.r.t. 
    } Q\in\mathcal{Q}$,
\begin{align}
    &\mathcal{C}^2 \text{ of } \overline{\mathcal{L}_{BE}}(Q) \text{ w.r.t. 
    } Q\in\mathcal{Q}\notag
    \\
    & \Longleftarrow \mathcal{C}^2 \text{ of } Q(s, a)-\left[R(s, a)+\gamma \mathbb{E}_{s^{\prime} \sim P(\cdot \mid \cdot s, a)} \log \sum_{a^{\prime}} \exp \left(Q\left(s^{\prime}, a^{\prime}\right)\right)\right] \text{ w.r.t. 
    } Q\in\mathcal{Q} \text{ for }s\in\mathcal{S}\notag
    \\
    &\Longleftarrow   \mathcal{C}^2 \text{ of }  \log \sum_{a^{\prime}} \exp \left(Q\left(s, a^{\prime}\right)\right) \text{ w.r.t. 
    } Q\in\mathcal{Q} \text{ for }s\in\mathcal{S}\notag
\end{align}
As it is known that logsumexp is of $\mathcal{C}^2$ \cite{kan2023lseminkmodifiednewtonkrylovmethod}, we are done. 
\;
\\
\;
\\
For Lipschitz smoothness, 
\begin{align}
    &\text{Lipschitz} \text{ smoothness } \text{of } \overline{\mathcal{L}_{BE}}(Q) \text{ w.r.t. 
    } Q\in\mathcal{Q}\notag
    \\
    &\!\!\iff \text{ Lipschitz continuity of } \nabla_Q \; \overline{\mathcal{L}_{BE}}(Q) \text{ w.r.t. 
    } Q\in\mathcal{Q} \notag
    \\
    &\!\!\iff \text{ Lipschitz continuity of } \mathbb{E}_{(s, a) \sim \pi^*, \nu_0}\left[2 \delta_Q(s, a) \nabla_Q \delta_Q(s, a)\right] \text{ w.r.t. 
    } Q\in\mathcal{Q} \notag
\end{align}
Now note that
\begin{align}
    &\|\mathbb{E}_{s,a \sim \pi^*, \nu_0}\left[2 \delta_Q(s, a) \nabla_Q \delta_Q(s, a)-2 \delta_{Q^\prime}(s, a) \nabla_{Q^\prime} \delta_{Q^\prime}(s, a)\right]\|_2 \notag 
    \\
    & \leq \mathbb{E}_{s,a \sim \pi^*, \nu_0}\bigl[\left\|2 \delta_Q(s, a) \nabla_Q \delta_Q(s, a)-2 \delta_{Q^\prime}(s, a) \nabla_{Q^\prime} \delta_{Q^\prime}(s, a)\right\|_2\bigr] \tag {Norm is convex}
  \\
  &\leq \mathbb{E}_{s,a \sim \pi^*, \nu_0}\left[\left\|Q^{\prime}(s,a)-Q(s, a)\right\|_2\right] \tag{Lemma \ref{lem:deltaGradDeltaLipschitz}}
  \\
  &\leq\left(\mathbb{E}_{s \sim \pi^*, \nu_0}\left\|Q^{\prime}(s, a)-Q(s, a)\right\|_2^2\right)^{1 / 2} \tag{$x^{1/2}$ is concave}
  \\
  &=\left\|Q-Q^{\prime}\right\|_{L_2(\pi^\ast, \nu_0)} \notag
\end{align}
This proves $ \text{ Lipschitz continuity of } \mathbb{E}_{(s, a) \sim \pi^*, \nu_0}\left[2 \delta_Q(s, a) \nabla_Q \delta_Q(s, a)\right] \text{ w.r.t. 
    } Q\in\mathcal{Q}$. Therefore, we can conclude the Lipschitz smoothness of $\overline{\mathcal{L}_{BE}}(Q)$ w.r.t. $ Q\in\mathcal{Q}$.
    \QED

\begin{lem}[$ \delta_Q(s, a) \nabla_Q \delta_Q(s, a)$ is Lipschitz]
\label{lem:deltaGradDeltaLipschitz}
For given fixed $(s,a)$, 
$$
\left\|2 \delta_Q(s, a) \nabla_Q \delta_Q(s, a)-2 \delta_{Q^\prime}(s, a) \nabla_{Q^\prime} \delta_{Q^\prime}(s, a)\right\|_2 \le \left\|Q^{\prime}(s,a)-Q(s, a)\right\|_2 
$$
holds for any $Q, Q^\prime \in\mathcal{Q}$.
\end{lem}
\begin{proof}[Proof of Lemma \ref{lem:deltaGradDeltaLipschitz}]
    Note that 
\begin{align}
   &\left\|\delta_Q(s, a) \nabla_Q \delta_Q(s, a)-\delta_{Q^{\prime}}(s, a) \nabla_{Q^{\prime}} \delta_{Q^{\prime}}(s, a)\right\|_2   \notag
   \\
   &\le \left\|\delta_Q(s, a)\right\|_2\left\|\nabla_Q \delta_Q(s, a) - \nabla_{Q^\prime} \delta_{Q^\prime}(s, a) \right\|_2 +  \left\|\delta_Q(s, a)-\delta_{Q^\prime}(s, a)\right\|_2 \left\| \nabla_{Q^\prime}\delta_{Q^\prime}(s,a)\right\|_2 \notag
\end{align}
Now what's left is to prove that for given fixed $(s,a)$, 
\begin{enumerate}
    \item $ \left\|\delta_Q(s, a)\right\|_2$ is bounded
    \item $\left\| \nabla_{Q^\prime}\delta_{Q^\prime}(s,a)\right\|_2$ is bounded
    \item $\delta_Q(s, a)$ is Lipschitz in $Q(s,a)$
    \item $\nabla_{Q^\prime}\delta_{Q^\prime}(s,a)$ is Lipschitz in $Q(s,a)$
\end{enumerate}

\noindent (1) Boundedness of \( \delta_Q(s,a) \):
\begin{align}
    |\delta_Q(s, a)| &= \left| \mathcal{T}Q(s,a) - Q(s,a) \right| \notag
    \\
    &= \left| r(s,a) + \beta \mathbb{E}_{s' \sim P(\cdot \mid s,a)} \left[V_Q(s')\right] - Q(s,a) \right|. \notag
\end{align}
Since \(V_Q(s') = \ln \sum_{b \in \mathcal{A}} \exp(Q(s',b))\), we use the bound:
\begin{align}
    \max_{b \in \mathcal{A}} Q(s',b) \leq V_Q(s') \leq \max_{b \in \mathcal{A}} Q(s',b) + \ln |\mathcal{A}| \notag
\end{align}
Taking expectations preserves boundedness, so we conclude:
\begin{align}
    |\delta_Q(s,a)| \leq |r(s,a)| + \beta \max_{s' \in \mathcal{S}} \max_{b \in \mathcal{A}} |Q(s',b)| + \beta \ln |\mathcal{A}| + \max_{s,a} |Q(s,a)| \notag
\end{align}
This shows \( \delta_Q(s,a) \) is uniformly bounded as long as \( Q \) is bounded, which is assured by $\beta<1$. 

\noindent (2) Boundedness of \( \nabla_Q \delta_Q(s,a) \):  
The gradient is given by:
\begin{align}
    \nabla_Q \delta_Q(s,a) &= \nabla_Q \mathcal{T} Q(s,a) - e_{(s,a)} \notag
\end{align}
where
\begin{align}
    \nabla_Q \mathcal{T} Q(s,a) &= \beta \mathbb{E}_{s' \sim P(\cdot \mid s,a)} \left[ \nabla_Q V_Q(s') \right] \notag
\end{align}
Since the softmax function \( \nabla_Q V_Q(s') \) satisfies
\begin{align}
    \sum_{b \in \mathcal{A}} \mathrm{softmax}(s',b; Q) = 1, \quad 0 \leq \mathrm{softmax}(s',b;Q) \leq 1 \notag
\end{align}
we obtain:
\begin{align}
    \|\nabla_Q \mathcal{T} Q(s,a)\|_2 \leq \beta \notag
\end{align}
Thus,
\begin{align}
    \|\nabla_Q \delta_Q(s,a)\|_2 = \|\nabla_Q \mathcal{T} Q(s,a) - e_{(s,a)}\|_2 \leq \beta + 1 \notag
\end{align}
Hence, \( \nabla_Q \delta_Q(s,a) \) is bounded.

\noindent (3) Lipschitz continuity of \( \delta_Q(s,a) \):  
Consider two functions \( Q \) and \( Q' \), and their corresponding Bellman errors:
\begin{align}
    \left| \delta_Q(s, a) - \delta_{Q'}(s, a) \right| &= \left| \mathcal{T}Q(s,a) - Q(s,a) - \mathcal{T}Q'(s,a) + Q'(s,a) \right| \notag
    \\
    &= \left| \mathcal{T}Q(s,a) - \mathcal{T}Q'(s,a) - (Q(s,a) - Q'(s,a)) \right| \notag
    \\
    &\leq \left| \mathcal{T}Q(s,a) - \mathcal{T}Q'(s,a) \right| + \left| Q(s,a) - Q'(s,a) \right| \notag
\end{align}
Since \( \mathcal{T}Q(s,a) \) depends on \( Q \) only through \( V_Q(s') \), we use the Lipschitz property of log-sum-exp:
\begin{align}
    |V_Q(s') - V_{Q'}(s')| \leq \max_{b \in \mathcal{A}} |Q(s',b) - Q'(s',b)| \notag
\end{align}
Taking expectations, we get:
\begin{align}
    |\mathcal{T}Q(s,a) - \mathcal{T}Q'(s,a)| \leq \beta \max_{s',b} |Q(s',b) - Q'(s',b)| \notag
\end{align}
Therefore,
\begin{align}
    |\delta_Q(s,a) - \delta_{Q'}(s,a)| \leq (1 + \beta) \max_{s',b} |Q(s',b) - Q'(s',b)| \notag
\end{align}
This proves \( \delta_Q(s,a) \) is Lipschitz in \( Q(s,a) \) with Lipschitz constant \( 1+\beta \).

\noindent (4) Lipschitz continuity of \( \nabla_Q \delta_Q(s,a) \):  
From the expression:
\begin{align}
    \nabla_Q \delta_Q(s,a) = \nabla_Q \mathcal{T}Q(s,a) - e_{(s,a)} \notag
\end{align}
we focus on \( \nabla_Q \mathcal{T}Q(s,a) \), which satisfies:
\begin{align}
    \|\nabla_Q \mathcal{T}Q(s,a) - \nabla_Q \mathcal{T}Q'(s,a)\|_2 &= \left\|\beta \mathbb{E}_{s' \sim P(\cdot \mid s,a)} \left[\nabla_Q V_Q(s') - \nabla_Q V_{Q'}(s') \right] \right\|_2 \notag
\end{align}
Using the Lipschitz property of Softmax,
\begin{align}
    \|\nabla_Q V_Q(s') - \nabla_Q V_{Q'}(s')\|_2 \leq \|Q(s',\cdot) - Q'(s',\cdot)\|_2 \notag
\end{align}


Taking expectations, we get:
\begin{align}
    \|\nabla_Q \mathcal{T}Q(s,a) - \nabla_Q \mathcal{T}Q'(s,a)\|_2 \leq \beta \max_{s',b} |Q(s',b) - Q'(s',b)| \notag
\end{align}
Since
\begin{align}
    \|\nabla_Q \delta_Q(s,a) - \nabla_{Q'} \delta_{Q'}(s,a)\|_2 \leq \|\nabla_Q \mathcal{T}Q(s,a) - \nabla_Q \mathcal{T}Q'(s,a)\|_2 \notag
\end{align}
we conclude that \( \nabla_Q \delta_Q(s,a) \) is Lipschitz with constant at most \( \beta \).



\end{proof}

%since softmax is the derivative of logsumexp, we need to show Lipschitz continuity of $\frac{\exp Q\left(s^{\prime}, a^{\prime}\right)}{\sum_{a^{\prime}} \exp \left(Q\left(s^{\prime}, a^{\prime}\right)\right) }$ w.r.t. $Q\in\mathcal{Q}$, i.e., 



%showing that $Q(s, a)-\left[R(s, a)+\gamma \mathbb{E}_{s^{\prime} \sim P(\cdot \mid \cdot s, a)} \log \sum_{a^{\prime}} \exp \left(Q\left(s^{\prime}, a^{\prime}\right)\right)\right]$ is smooth is enough, as this immediately proves smoothness of $\overline{\mathcal{L}_{BE}}(Q)$ ($\because$ square of the smooth function is also a smooth function and linear function of smooth function is also a smooth function). Since $\log \sum_{a^{\prime}} \exp \left(Q\left(s^{\prime}, a^{\prime}\right)\right)$ is smooth and  $\mathcal{C}^2$ in $Q$ \cite{kan2023lseminkmodifiednewtonkrylovmethod}, we are done. 

\iffalse
For Lipschitz continuity, 
\begin{align}
\left|\overline{\mathcal{L}_{\mathrm{BE}}}(Q)-\overline{\mathcal{L}_{\mathrm{BE}}}\left(Q^{\prime}\right)\right| & \leq \mathbb{E}_{(s, a) \sim \pi^*, \nu_0}\left[\left|\mathcal{L}_{\mathrm{BE}}(Q)(s, a)-\mathcal{L}_{\mathrm{BE}}\left(Q^{\prime}\right)(s, a)\right|\right] \notag \\
& \leq \mathbb{E}_{(s, a) \sim \pi^*, \nu_0}\left[L\left\|Q-Q^{\prime}\right\|_{\infty}\right] \notag
\\
&=L\left\|Q-Q^{\prime}\right\|_{\infty} \notag
\end{align}
where the second inequality is from Lemma \ref{lem:BEsaLip} along with the definition of the constant $L$.

\begin{lem}\label{lem:BEsaLip} There exists a constant $L>0$ such that for $s\in \mathcal{S}$ and $a\in \mathcal{A}$, 
$$\left|\mathcal{L}_{\mathrm{BE}}(Q)(s, a)-\mathcal{L}_{\mathrm{BE}}\left(Q^{\prime}\right)(s, a)\right| \leq L\left\|Q-Q^{\prime}\right\|_{\infty}$$    
\end{lem}
\begin{proof} Define $e_Q(s, a):=\mathcal{T} Q(s, a)-Q(s, a)$. Then $\mathcal{L}_{\mathrm{BE}}(Q)(s, a)=\left[e_Q(s, a)\right]^2$. 
    \begin{align}
    \left|\mathcal{L}_{\mathrm{BE}}(Q)(s, a)-\mathcal{L}_{\mathrm{BE}}\left(Q^{\prime}\right)(s, a)\right| &= \left|\left[e_Q(s, a)\right]^2-\left[e_{Q^{\prime}}(s, a)\right]^2\right| \notag
    \\
    &=\left|e_Q(s, a)+e_{Q^{\prime}}(s, a)\right|\cdot \left|e_Q(s, a)-e_{Q^{\prime}}(s, a)\right|\notag
    \\
    &\le C \cdot (1+\beta)\left\|Q-Q^{\prime}\right\|_{\infty}\label{eq:BEtwoerrorprod}
    \end{align}
where the inequality in equation \ref{eq:BEtwoerrorprod} is from Lemma \ref{lem:smoothBellman} and 
\begin{align}
    e_Q(s, a)+e_{Q^{\prime}}(s, a)&=[\mathcal{T} Q(s, a)-Q(s, a)]+\left[\mathcal{T} Q^{\prime}(s, a)-Q^{\prime}(s, a)\right]\notag
    \\
    &=2 r(s, a)+\beta \cdot \mathbb{E}_{s^{\prime} \sim P(s, a)}\left[V_Q\left(s^{\prime}\right)+V_{Q^{\prime}}\left(s^{\prime}\right)\right]-\left[Q(s, a)+Q^{\prime}(s, a)\right]\notag
    \\
    &\leq 2|r(s, a)|+\beta \cdot \mathbb{E}_{s^{\prime}}\left[\left|V_Q\left(s^{\prime}\right)\right|+\left|V_{Q^{\prime}}\left(s^{\prime}\right)\right|\right]+\left|Q(s, a)+Q^{\prime}(s, a)\right| \notag
    \\
    &\leq 2 R_{\max }+\beta \cdot 2(M+\log |\mathcal{A}|)+2 M \notag
    \\
    &=C:=2 R_{\max }+2 M(1+\beta)+2 \beta \log |\mathcal{A}| \notag
\end{align}
\end{proof}
\;
\\
\fi
 

\subsection{Proof of Theorem \ref{thm:BEenjoyPL} (Bellman error satisfying the PL condition)}
By Lemma \ref{lem:BE(s,a)PL} (Below), $\mathcal{L}_{BE}(Q)(s,a)$ satisfies PL condition with respect to $Q$ for all $s\in\mathcal{S}$ and $a\in\mathcal{A}$. By Lemma \ref{lem:f1f2sumPL}, $\frac{1}{|\mathcal{D}|}\sum_{(s,a)\in\mathcal{D}}\mathcal{L}_{BE}(s,a)$ is also PL. Now we would like to show that $\overline{\mathcal{L}_{\mathrm{BE}}}(Q):=\mathbb{E}_{(s, a) \sim \pi^*, \nu_0}\left[\mathcal{L}_{\mathrm{BE}}(Q)(s, a)\right]$ is also PL in terms of $L^2(\pi^\ast, \nu_0)$. Since $\overline{\mathcal{L}_{\mathrm{BE}}}(Q)$ is of $\mathcal{C}^2$, by \cite{rebjock2023fast}, showing PL is equivalent to showing to Quadratic Growth (QG), i.e., there exists $c^\prime>0$ such that
$$
\mathbb{E}_{(s, a) \sim \pi^*, \nu_0}\left[\mathcal{L}_{\mathrm{BE}}(Q)(s, a)\right]-\mathbb{E}_{(s, a) \sim \pi^*, \nu_0}\left[\mathcal{L}_{\mathrm{BE}}(Q^\ast)(s, a)\right] \ge c^\prime\|Q-Q^\ast\|^2_{L^2(\pi^\ast, \nu_0)}.
$$
But note that
\begin{align}
    &\mathbb{E}_{(s, a) \sim \pi^*, \nu_0}\left[\mathcal{L}_{\mathrm{BE}}(Q)(s, a)\right]-\mathbb{E}_{(s, a) \sim \pi^*, \nu_0}\left[\mathcal{L}_{\mathrm{BE}}(Q^\ast)(s, a)\right] \notag
    \\
    &= \mathbb{E}_{(s, a) \sim \pi^*, \nu_0}\left[\mathcal{L}_{\mathrm{BE}}(Q)(s, a)-\mathcal{L}_{\mathrm{BE}}(Q^\ast)(s, a)\right] \notag
    \\
    &\ge \mathbb{E}_{(s, a) \sim \pi^*, \nu_0}\left[c(s,a)^2(Q(s,a)-Q^\ast(s,a))^2\right] \label{eq:BEisC2}
    \\
    &=c^2\|Q-Q^\ast\|^2_{L^2(\pi^\ast, \nu_0)} \notag
\end{align}
where equation \eqref{eq:BEisC2} is due to $\mathcal{L}_{BE}(Q)(s,a)$ being QG because it is smooth and therefore PL implies QG \citep{liao2024error}. ($c(s,a)>0$ is the QG constant for $(s,a)$ and $c = \inf_{(s,a)\in \mathcal{S}\times\mathcal{A}} c(s,a)$.) This finishes the proof.
\QED

\begin{lem}\label{lem:BE(s,a)PL} For any given fixed $s\in\mathcal{S}$ and $a\in\mathcal{A}$,
$\mathcal{L}_{BE}(Q)(s,a)$ satisfies PL condition with respect to $Q$ in terms of euclidean norm.
\end{lem}
\begin{proof}[Proof of Lemma \ref{lem:BE(s,a)PL}] Throughout the proof, we extend \cite{ruszczynski2024functional} to deal with soft-max Bellman equation with infinite-dimensional state space $\mathcal{S}$. Given that $|\mathcal{A}|<\infty$, 
for each $s\in\mathcal{S}$, $Q(s, \cdot)$ can be expressed as a finite-dimensional vector $\left[Q\left(s, a^{\prime}\right)\right]_{a^{\prime} \in \mathcal{A}}\in\mathbb{R}^{|\mathcal{A}|}$; For convenience in notation, we define $q:\mathcal{S}\mapsto \mathbb{R}^{|A|}$ and
$$\mathcal{G}(s): \{q(s)\in \mathbb{R}^{|\mathcal{A}|}\mid q(s) = \left[Q\left(s, a^{\prime}\right)\right]_{a^{\prime} \in \mathcal{A}} \text{ for some }Q\in\mathcal{Q}\}$$
and use $q(s)$ instead of $Q(s,\cdot)$ and $q^\ast(s)$ instead of $Q^\ast(s,\cdot)$. 
% We can rewrite
%  $r(s,a) + \beta \cdot \mathbb{E}_{s^{\prime} \sim P(s, a)}\left[\log(\sum_{a^\prime\in\mathcal{A}}\exp Q(s^\prime, a^\prime)) \mid s, a\right]- Q(s, a)$ as $\Psi(s,a,q)$, where
We define
 \begin{align}
 \Psi(s, a, q)&:= r(s,a) + \beta \cdot \mathbb{E}_{s^{\prime} \sim P(s, a)}\left[\log(\sum_{a^\prime\in\mathcal{A}}\exp q(s^\prime)_{(a^\prime)}) \mid s, a\right]- q(s)_{(a)} \notag
 \end{align}
Now with $q^\ast(\cdot) := \left[Q^\ast\left(\cdot, a\right)\right]_{a \in \mathcal{A}}$, let's define
\begin{align}
     f(s, a, q) &:= \frac{1}{2}(\Psi(s, a, q^\ast)-\Psi(s, a, q))^2 \notag
\end{align}
 Then, for $s\in\mathcal{S}$, with the choice of $q(\tau):= q^\ast + \tau(q-q^\ast)$,
\begin{align}
  f_q(s, a, q):= \partial_q f(s, a, q)&=-\Psi_q(s,a, q)(\Psi(s, a, q^\ast)-\Psi(s, a,q))\notag
    \\
    &= - \Psi_q(s, a, q) \int_0^1 \Psi_q\left(s,a, q(\tau)\right)^\top(q^\ast(s)-q(s))d\tau \tag{Theorem \ref{thm:bolte}}
    \\
    &= - \int_0^1 \Psi_q(s,a, q) \Psi_q\left(s, a, q(\tau)\right)^\top  d\tau \cdot (q^\ast(s)-q(s))\notag
\end{align}

By Lemma \ref{lem:smallestEigen},there exists $\tilde{\lambda}$ such that for all $s\in\mathcal{S}$ and $a\in\mathcal{A}$, $\Psi_q(s, a, q^\prime) \Psi_q(s, a, q^{\prime\prime})^\top \succeq \tilde{\lambda} \cdot I$ for any choice of $q^\prime(s), q^{\prime\prime}(s) \in\mathcal{G}(s)$. %we can define $\tilde{\lambda}>0$ such that it satisfies $\Psi_q(s, a, q^\prime) \Psi_q(s, a, q^{\prime\prime})^\top \succeq \tilde{\lambda} I$ for all $q^\prime(s), q^{\prime\prime}(s) \in\mathcal{G}(s)$. 
Therefore we have
\begin{align}
    \left\langle f_q(s, a, q), q(s)-q^\ast(s)\right\rangle \ge \tilde{\lambda} \|q(s)-q^\ast(s)\|_2^2. \notag
\end{align}
This implies that
\begin{align}
    &\|f_q(s, a, q)\|_2=\max _{\|z\|=1}\left\langle f_q(s, a, q), z\right\rangle \ge \left\langle f_q(s,a,q), \frac{q(s)-q^*(s)}{\left\|q(s)-q^*(s)\right\|_2}\right\rangle \notag
    \\
    &\ge  \tilde{\lambda}\|q(s)-q^\ast(s)\|_2  \ge \tilde{\lambda}\|q(s)-q^\ast(s)\|_{\infty}  \label{eq:forPL1}
\end{align}
Therefore,
\begin{align}
    &\|f_q(s, a, q)\|_2\ge \tilde{\lambda}\|q(s)-q^\ast(s)\|_{\infty}   \label{eq:forPL3}
\end{align}
(Note: Equation \ref{eq:forPL3} is a regularity condition called sub-differential error bound.) Also, from Lemma \ref{lem:smoothBellman}, 
\begin{align}
    f(s, a, q) &= \frac{1}{2}(\Psi(s, a, q^\ast)-\Psi(s, a, q))^2\notag
    \\
    &\le \frac{1}{2} (1+\beta)^2 \|q(s)-q^\ast(s)\|_{\infty}^2\label{eq:forPL2}
\end{align}
Combining equation \ref{eq:forPL2} and \ref{eq:forPL3}, we get \begin{align}
  f(s,a,q)\le \frac{1}{2} \left(\frac{1+\beta}{\tilde{\lambda}}\right)^2 \|f_q(s, a, q)\|_2^2 \quad\text{for all }s\in\mathcal{S}, a\in\mathcal{A} \notag
\end{align}

Since $\Psi\left(s, a, q^*\right)=0$, $f(s,a,q) = \mathcal{L}_{BE}(Q)(s,a)$, where $q(s)=\left[Q\left(s, a^{\prime}\right)\right]_{a^{\prime} \in \mathcal{A}}$. This finishes the proof.

\end{proof}
\begin{thm}[\cite{bolte2023subgradient}]\label{thm:bolte}
Let $f:\mathbb{R}^n \rightarrow \mathbb{R}$ be a differentiable function. If a path $q:[0, \infty) \rightarrow \mathbb{R}^n$ is a absolutely continuous path in $\mathbb{R}^n$, $f$ admits the chain rule on the path $q(t)$ as

$$
f(q(T))-f(q(0))=\int_0^T f_q(q(t))[\dot{q}(t)] d t
$$
where $\dot{q}(t)$ is the derivative of the function path $q(t)$ with respect to $t$ and $T>0$.
\end{thm}

\begin{lem}[Positive smallest eigenvalue]\label{lem:smallestEigen} Suppose that the discount factor $\beta<1$. Then for there exists $\tilde{\lambda}>0$ such that for all $s\in\mathcal{S}$ and $a\in\mathcal{A}$, $\lambda_{\min}(\Psi_q\left(s, a, q^{\prime}\right) \Psi_q\left(s, a, q^{\prime \prime}\right)^{\top})> \tilde{\lambda}$ holds for any choice of $ q^{\prime}, q^{\prime \prime} \in \mathcal{G}(s)$.
\end{lem}
\begin{proof}
First, note that we can define the policy $\pi_q(a|s) = \frac{\exp q_{(a)}}{\sum_{a^{\prime }} \exp q_{(a^\prime)}}$ for $q\in\mathbb{R}^{|\mathcal{A}|}$, where $x_{(a)}$ implies the $a$th element of vector $x$. 
\begin{align}
     \frac{\partial \Psi(s, a, q)}{\partial q_{(a^{\prime})}}&=\beta \mathbb{E}_{s^{\prime} \sim P(s, a)}\left[\pi_q\left(a^{\prime} \mid s^{\prime}\right)\right]-\delta_{a, a^{\prime}}\notag
\end{align}
That is, $\Psi_q(s, a, q)=\beta \mu_q-e_q$, where $\mu_q=\mathbb{E}_{s^{\prime} \sim P(s, a)}\left[\pi_q\left(a^{\prime} \mid s^{\prime}\right)\right]$ is a probability vector, as it's an expectation over probability distributions. Then for any choice of $q^{\prime}, q^{\prime \prime} \in \mathcal{G}(s)$, denoting $\mu_{q^\prime} = \mu^\prime$ and $\mu_{q^{\prime\prime}} = \mu^{\prime\prime}$
\begin{align}
    \lambda\left(\Psi_q\left(s, a, q^{\prime}\right) \Psi_q\left(s, a, q^{\prime \prime}\right)^{\top}\right)& = \lambda\left(\left(\beta \mu^{\prime}-e_a\right)\left(\beta \mu^{\prime \prime}-e_a\right)^{\top}\right)\notag
    \\
    &= \left(\beta \mu^{\prime}-e_a\right)^{\top}\left(\beta \mu^{\prime \prime}-e_a\right) \notag
    \\
    &=\beta^2\left(\mu^{\prime}\right)^{\top} \mu^{\prime \prime}-\beta \mu^{\prime}(a)-\beta \mu^{\prime \prime}(a)+1 \notag
    \\
    &\ge \beta^2 \mu^{\prime}(a) \mu^{\prime \prime}(a)-\beta \mu^{\prime}(a)-\beta \mu^{\prime \prime}(a)+1 \notag
    \\
    &=(1-\beta \mu^{\prime}(a))(1-\beta \mu^{\prime\prime}(a))
    \\
    &\ge (1-\beta)^2 \notag
\end{align}
Since $\beta\in(0,1)$, $\tilde{\lambda} = (1-\beta)^2$ serves as the uniform lower bound of $\lambda_{\min}(\Psi_q\left(s, a, q^{\prime}\right) \Psi_q\left(s, a, q^{\prime \prime}\right)^{\top})$ for all $s\in\mathcal{S}$ and $a\in\mathcal{A}$, for any choice of $q^{\prime}, q^{\prime \prime} \in \mathcal{G}(s)$.
\end{proof}
\begin{lem}\label{lem:smoothBellman} $|(\mathcal{T}Q-Q)(s,a)-(\mathcal{T}Q^\ast-Q^\ast)(s,a)| \le (1+\beta)\left\|Q\left(s^{\prime}, \cdot\right)-Q^*\left(s^{\prime}, \cdot\right)\right\|_{\infty}$ for all $s\in\mathcal{S}$ and $a\in\mathcal{A}$.
\end{lem}
\begin{proof}

   \begin{align}
       &|(\mathcal{T}Q-Q)(s,a)-(\mathcal{T}Q^\ast-Q^\ast)(s,a)| \notag
       \\
       &=|\beta \cdot \mathbb{E}_{s^{\prime} \sim P(s, a)}\left[\log \left(\sum_{a^{\prime} \in \mathcal{A}} \exp Q\left(s^{\prime}, a^{\prime}\right)\right)-\log \left(\sum_{a^{\prime} \in \mathcal{A}} \exp Q^\ast\left(s^{\prime}, a^{\prime}\right)\right)  \mid s, a\right] +(Q^\ast(s, a)-Q(s, a))|\notag
       \\
       &\le|\beta \cdot \mathbb{E}_{s^{\prime} \sim P(s, a)}\left[\left\|Q\left(s^{\prime}, \cdot\right)-Q^*\left(s^{\prime}, \cdot\right)\right\|_{\infty}\right] + \left|Q^*(s, a)-Q(s, a)\right| \tag{logsumexp Liptshitz in 1}
       \\
       &\le(\beta+1)\left\|Q\left(s^{\prime}, \cdot\right)-Q^*\left(s^{\prime}, \cdot\right)\right\|_{\infty}\notag
   \end{align}
\end{proof}




\subsection{Proof of Theorem \ref{thm:NLLenjoyPL} (NLL loss satisfying the PL condition)}
From Lemma \ref{lem:KLPL} and Lemma \ref{lem:f1f2sumPL}, ${L}_{NLL}(s,a)$ and $\frac{1}{|\mathcal{D}|}\sum_{(s,a)\in\mathcal{D}}\mathcal{L}_{NLL}(s,a)$ are PL. 

\noindent What remains is to show that $\mathbb{E}_{(s, a) \sim \pi^*, \nu_0}\left[-\log \left(\hat{p}_Q(a \mid s)\right)\right]$ satisfies PL. From Lemma \ref{lem:minMLE}, we know
\begin{align}
\mathbb{E}_{(s, a) \sim \pi^*, \nu_0}\left[-\log \left(\hat{p}_Q(a \mid s)\right)\right] =\mathbb{E}_{s \sim \pi^*, \nu_0}\left[D_{K L}\left(\pi^*(\cdot \mid s) \| \hat{p}_Q(\cdot \mid s)\right)\right]+\mathbb{E}_{(s, a) \sim \pi^*, \nu_0}\left[\ln \pi^*(a \mid s)\right] \notag
\end{align}
Note that the second term is not dependent on $Q$. Therefore, we will instead show that the PL condition holds for $\mathbb{E}_{s \sim \pi^*, \nu_0}\left[D_{K L}\left(\pi^*(\cdot \mid s) \| \hat{p}_Q(\cdot \mid s)\right)\right]$. Since $\mathbb{E}_{s \sim \pi^*, \nu_0}\left[D_{K L}\left(\pi^*(\cdot \mid s) \| \hat{p}_Q(\cdot \mid s)\right)\right]$ is convex, by \cite{liao2024error}, showing that  $\mathbb{E}_{s \sim \pi^*, \nu_0}\left[D_{K L}\left(\pi^*(\cdot \mid s) \| \hat{p}_Q(\cdot \mid s)\right)\right]$ is PL is equivalent to showing that $\mathbb{E}_{s \sim \pi^*, \nu_0}\left[D_{K L}\left(\pi^*(\cdot \mid s) \| \hat{p}_Q(\cdot \mid s)\right)\right]$ satisfies Quadratic Growth (QG) condition, i.e., there exists $c^\prime>0$ such that
\begin{align}
    &\mathbb{E}_{s \sim \pi^*, \nu_0}\left[D_{K L}\left(\pi^*(\cdot \mid s) \| \hat{p}_Q(\cdot \mid s)\right)\right]-\mathbb{E}_{s \sim \pi^*, \nu_0}\left[D_{K L}\left(\pi^*(\cdot \mid s) \| \hat{p}_{Q^\ast}(\cdot \mid s)\right)\right] \geq 
c^{\prime}\left\|Q-Q^*\right\|_{L^2\left(\pi^*, v_0\right)}^2 \notag
\end{align}
\noindent But note that
\begin{align}
    &\mathbb{E}_{s \sim \pi^*, \nu_0}\left[D_{K L}\left(\pi^*(\cdot \mid s) \| \hat{p}_Q(\cdot \mid s)\right)\right]-\mathbb{E}_{s \sim \pi^*, \nu_0}\left[D_{K L}\left(\pi^*(\cdot \mid s) \| \hat{p}_{Q^\ast}(\cdot \mid s)\right)\right] \notag
     \\
     &=\mathbb{E}_{s \sim \pi^*, \nu_0}\left[D_{K L}\left(\pi^*(\cdot \mid s) \| \hat{p}_Q(\cdot \mid s)\right)-D_{K L}\left(\pi^*(\cdot \mid s) \| \hat{p}_{Q^\ast}(\cdot \mid s)\right)\right] \notag
     \\
    &\ge \mathbb{E}_{(s, a) \sim \pi^*, \nu_0}\left[c(s,a)^2(Q(s,a)-Q^\ast(s,a))^2\right] \tag{Lemma \ref{lem:KLPL} and convexity}
    \\
    &=c^2\|Q-Q^\ast\|^2_{L^2(\pi^\ast, \nu_0)} \notag
\end{align}
\noindent where $c(s,a)>0$ is the QG constant for $(s,a)$ and $c = \inf_{(s,a)\in \mathcal{S}\times\mathcal{A}} c(s,a)$. Done. \QED
%By the Lemma \ref{lem:KLPL} below, $D_{K L}\left(\pi^*(\cdot \mid s) \| \hat{p}_Q(\cdot \mid s)\right)$ satisfies PL condition for each $s\in\mathcal{S}$. 


\iffalse

Now note that $\left\{Q \in \mathcal{Q} \mid D_{K L}\left(\pi^*(\cdot \mid s) \| \hat{p}_Q(\cdot \mid s)\right)=0\right.$ for all $\left.s \in \overline{\mathcal{S}}\right\}$ is nonempty if $Q^\ast \in \mathcal{Q}$, because 
\begin{align}
    &\{Q\in\mathcal{Q}\mid D_{KL}(\pi^*(\cdot\mid s) \| \hat{p}_{Q}(\cdot\mid s))=0 \text{ for all }s\in\bar{\mathcal{S}}\} \notag
    \\
    &=\{Q\in\mathcal{Q}\mid \hat{p}_Q(\cdot \mid s)=\pi^*(\cdot \mid s)\; \; \text{a.e.} \text{ for all }s\in\bar{\mathcal{S}}\}\notag
    \\
    &= \{Q\in\mathcal{Q}\mid \frac{\hat{p}_Q\left(a_1 \mid s\right)}{\hat{p}_Q\left(a_2 \mid s\right)}=\frac{\pi^*\left(a_1 \mid s\right)}{\pi^*\left(a_2 \mid s\right)} \quad \forall a_1, a_2 \in \mathcal{A}, s\in\bar{\mathcal{S}}\}\notag   
    \\
    &=\left\{Q \in \mathcal{Q} \mid \exp (Q(s,a_1)-Q(s,a_2))=\exp \left(Q^*(s,a_1)-Q^*(s,a_2)\right) \quad \forall a_1, a_2\in\mathcal{A}, s\in\bar{\mathcal{S}} \right\}\notag
    \\
    &=\left\{Q \in \mathcal{Q} \mid Q(s,a_1)-Q(s,a_2)= Q^*(s,a_1)-Q^*(s,a_2) \quad \forall a_1, a_2\in\mathcal{A}, s\in\bar{\mathcal{S}}\right\} \notag
\end{align}


Therefore, $\mathbb{E}_{s \sim \pi^*, \nu_0}\left[D_{K L}\left(\pi^*(\cdot \mid s) \| \hat{p}_Q(\cdot \mid s)\right)\right]$ also satisfies PL by Lemma \ref{lem:expDKLalsoPL}, finishing the proof.
\QED

\fi


\begin{lem}\label{lem:KLPL}
$D_{K L}\left(\pi^*(\cdot \mid s) \| \hat{p}_Q(\cdot \mid s)\right)$ satisfies the PL condition for each $s\in\mathcal{S}$. This implies that $-\log \left(\hat{p}_Q(\cdot \mid s)\right) =D_{K L}\left(\pi^*(\cdot \mid s) \| \hat{p}_Q(\cdot \mid s)\right)+\ln \pi^*(\cdot\mid s)\notag$ is also PL for each $s\in \mathcal{S}$.
\end{lem}
\begin{proof}
Note that
\begin{align}
     \nabla_{Q(s, \cdot)}D_{K L}\left(\pi^*(\cdot \mid s) \| \hat{p}_Q(\cdot \mid s)\right)&= \nabla_{Q(s, \cdot)}\left(-\sum_a \pi^*(a \mid s) \log \hat{p}_Q(a \mid s)\right) \notag
     \\
     &=-\sum_a \pi^*(a \mid s)\left(\delta_{a, a^{\prime}}-\hat{p}_Q\left(a^{\prime} \mid s\right)\right)\notag
     \\
     &=-\left[\pi^*\left(a^{\prime} \mid s\right)-\hat{p}_Q\left(a^{\prime} \mid s\right) \sum_a \pi^*(a \mid s)\right]_{a^{\prime} \in \mathcal{A}}\notag
     \\
     &=\left[\hat{p}_Q\left(a^{\prime} \mid s\right)-\pi^*\left(a^{\prime} \mid s\right)\right]_{a^{\prime} \in \mathcal{A}} \notag
\end{align}
Then, 
\begin{align}
    \|\nabla_{Q(s,)} D_{K L}\left(\pi^*(\cdot \mid s) \| \hat{p}_Q(\cdot \mid s)\right)\|^2 & = \|\left[\hat{p}_Q\left(a^{\prime} \mid s\right)-\pi^*\left(a^{\prime} \mid s\right)\right]_{a^{\prime} \in \mathcal{A}}\|_2^2\notag
    \\
    & \ge \frac{1}{|\mathcal{A}|} \|\left[\hat{p}_Q\left(a^{\prime} \mid s\right)-\pi^*\left(a^{\prime} \mid s\right)\right]_{a^{\prime} \in \mathcal{A}}\|_1^2 \notag
    \\
    & = \frac{1}{|\mathcal{A}|} \text{TV}\left(\hat{p}_Q\left(\cdot \mid s\right),\pi^*\left(\cdot \mid s\right)\right)^2 \notag
    \\
    &\ge \frac{\alpha_Q \ln 2}{|\mathcal{A}|} D_{K L}\left(\pi^*(\cdot \mid s) \| \hat{p}_Q(\cdot \mid s)\right) \notag
\end{align}
where,
\begin{itemize}
    \item TV denotes the total variation distance.
    \item The last inequality is from Lemma \ref{lem:RevPinsker}, where $\alpha_Q:=\min _{a \in A_{+}} Q(s,a)>0$ with $A_{+}=\{a\in\mathcal{A}$ : $Q(s,a)>0\}$.
\end{itemize}
\end{proof}

\begin{lem}[Reverse Pinsker's inequality]\label{lem:RevPinsker}
    \begin{align}
        D(P \| Q)&=\sum_{a \in A_{+}} P(a) \log _2 \frac{P(a)}{Q(a)} \leq \frac{1}{\ln 2} \sum_{a \in A_{+}} P(a)\left(\frac{P(a)}{Q(a)}-1\right) \notag
        \\
        &=\frac{1}{\ln 2}\sum_{a \in A_{+}} \frac{(P(a)-Q(a))^2}{Q(a)}+\sum_{a \in A_{+}}(P(a)-Q(a)) \notag
        \\
        &=\frac{1}{\ln 2} \sum_{a \in A_{+}} \frac{(P(a)-Q(a))^2}{Q(a)} \notag
        \\
        &\leq \frac{d(P, Q)^2}{\alpha_Q \cdot \ln 2} \notag
    \end{align}
\end{lem}

\begin{lem}\label{lem:expDKLalsoPL}  Suppose that given fixed $z\in \mathcal{Z}$, a smooth function $f(x,z)$ 1) either satisfies convexity in $x$ or of $\mathcal{C}^2$ in $x$ and 2) satisfies Polyak-Łojasiewicz condition in $x$ with the coefficient $\mu_z>0$, i.e., $$
\left\|\nabla_x f(x,z)\right\|_2^2 \geq 2 \mu_z\left[f(x,z)-f_z^*\right]
$$
where $f_z^*=\min _x f_z(x)$ and $\mu_z>0$. In addition, suppose that $\arg\min_x f(x,z)=\arg\min_x f(x, z^\prime)$ for all $z, z^\prime \in \mathcal{Z}$, where we define the common minimizer as $x^\ast$. Then $F(x):=\mathbb{E}_{z\sim \nu}[f(x,z)]$ satisfies Polyak-Łojasiewicz condition with respect to $x$, given that $\nu$ is a measure defined on $\mathcal{Z}$. That is, 
$$
\left\|\nabla_x F(x)\right\|_2^2 \geq 2 \mu\left[F(x)-F^*\right],
$$
where $F^*:=\min _x F(x)=\mathbb{E}_{z \sim \nu}\left[f_z^*\right]$, and $\mu=\inf _{z \in \mathcal{Z}} \mu_z>0$.
\end{lem}

\begin{proof}
    Since $f$ is smooth and satisfies PL condition with respect to $x$ for given $z\in\mathcal{Z}$, it satisfies the Quadratic Growth (QG) condition \cite{liao2024error}, i.e., for fixed $z\in\mathcal{Z}$,
    there exists $\alpha_z>0$ such that:
$$
f(x,z)-f_z^\ast \geq \alpha_z\left\|x-x^*\right\|^2 \quad \forall x\in\mathcal{X}
$$
Therefore, 
\begin{align}
    F(x)-F^*&=\mathbb{E}_z\left[f(x, z)-f_z^*\right] \notag
    \\
    &\geq \mathbb{E}_z\left[\alpha_z\left\|x-x^*\right\|^2\right] \notag
    \\
    &\geq \alpha\left\|x-x^*\right\|^2 \quad (\alpha:=\inf _z \alpha_z>0) \notag
\end{align}
This implies that $F(x)$ satisfies the QG condition in $x$. If $f$ satisfies convexity, then by \citet{liao2024error}, Quadratic growth and PL are equivalent; if $f$ is of $\mathcal{C}^2$, then by \citet{rebjock2023fast}, Quadratic Growth and PL are equivalent. Therefore, $F(x)$ satisfies PL.
\end{proof}


\subsection{Proof of Lemma \ref{lem:f1f2sumPL}}
Let $f_1(Q):=\mathcal{L}_{NLL}(Q)$ and $f_2(Q):=\mathcal{L}_{BE}(Q)$. 
 Let $M_1:=\left\{Q \in \mathcal{Q}: f_1(Q)=f_1^*\right\}, $ and $ M_2:=\left\{Q \in \mathcal{Q}: f_2(Q)=f_2^*\right\}$.
 By Theorem \ref{thm:mainopt}, the minimizer of $f_1+f_2$ is in both the minimizer of $f_1$ and the minimizer $f_2$. Therefore, by Lemma \ref{lem:linsumPL}, $f_1+f_2$ is also PL. This implies that $\mathcal{R}_{exp}(Q)$ satisfies the PL. Now, given a finite dataset $\mathcal{D} = \{(s_i, a_i, s_i')\}_{i=1}^N$, note that the empirical risk function $\mathcal{R}_{emp}(Q)$ is equivalent to the expected risk function with the transition probability being $\hat{P}(s'|s,a) = \frac{\sum_{i=1}^N \mathbbm{1}[(s_i,a_i,s_i') = (s,a,s')]}{\sum_{i=1}^N \mathbbm{1}[(s_i,a_i) = (s,a)]}$ and expert policy being $\hat{\pi}^*(a|s) = \frac{\sum_{i=1}^N \mathbbm{1}[(s_i,a_i) = (s,a)]}{\sum_{i=1}^N \mathbbm{1}[s_i = s]}$. (By Theorem \ref{thm:MagnacThesmar}, we know that minimization of this problem is well-defined.) 
 Since the expected risk in this case satisfies the PL condition and has a unique solution, and is equivalent to $\mathcal{R}_{emp}(Q)$, $\mathcal{R}_{emp}(Q)$ satisfies the PL condition and has a unique solution. 

 
 
 \QED


 
\begin{lem}\label{lem:linsumPL}
Suppose that $f_1$ and $f_2$ are both PL and Lipschitz smooth. Furthermore, the minimizer of $f_1+f_2$ is unique, where the minimizer of $f_1$ and the minimizer $f_2$ coincides. Then $f_1+\lambda f_2$ satisfies PL condition for any $\lambda>0$.
\end{lem}

\begin{proof}[Proof of Lemma \ref{lem:linsumPL}] Without loss of generality, we prove that $f := f_1+f_2$ satisfies PL condition. Recall that we say $f$ satisfies $\mu$-$PL$ condition if $2\mu(f(Q)-f\left(Q^\ast\right)) \leq \|\nabla f(Q)\|^2$.
\begin{align}
    \|\nabla f(Q)\|^2&=\left\|\nabla f_1(Q)+\nabla f_2(Q)\right\|^2 \notag
    \\
    &=\left\|\nabla f_1(Q)\right\|^2 + \left\|\nabla f_2(Q)\right\|^2 + 2 \nabla f_1(Q)^{\top} \nabla f_2(Q) \notag
    \\
    &\ge 2\mu_1(f_1(Q)-f_1(Q^\ast)) +  2\mu_2(f_2(Q)-f_2(Q^\ast))+ 2 \nabla f_1(Q)^{\top} \nabla f_2(Q) \notag
    \\
    &\ge 2\mu(f_1(Q)+f_2(Q)-f_1(Q^\ast)-f_2(Q^\ast))+ 2 \nabla f_1(Q)^{\top} \nabla f_2(Q) \notag
    \\
    &=2\mu(f(Q)-f(Q^*) + 2 \nabla f_1(Q)^{\top} \nabla f_2(Q) \notag
    \\
    &\ge 2\mu(f(Q)-f(Q^*)\tag{Lemma \ref{lem:crossPos}}
\end{align}
The last inequality is not trivial, and therefore requires Lemma \ref{lem:crossPos}.

    
\end{proof}

\begin{lem}\label{lem:crossPos}
Suppose that $f_1$ and $f_2$ satisfies PL in $Q$ and minimizer of $f_1+f_2$ is in both the minimizer of $f_1$ and the minimizer $f_2$. Then for all $Q\in\mathcal{Q}$, $\left\langle\nabla f_1(Q), \nabla f_2(Q)\right\rangle \geq 0$.
    
\end{lem}
\begin{proof}
 Let $M_1:=\left\{Q \in \mathcal{Q}: f_1(Q)=f_1^*\right\}, $ and $ M_2:=\left\{Q \in \mathcal{Q}: f_2(Q)=f_2^*\right\}$. From what is assumed, $f_1+f_2$ has a minimizer $Q^*$ that belongs to both $M_1$ and $M_2$.

    Since $f_1$ and $f_2$ are both Lipschitz smooth and satisfy PL condition, they both satisfy Quadratic Growth (QG) condition \cite{liao2024error}, i.e., 
    there exists $\alpha_1, \alpha_2>0$ such that:
$$
f_1(Q)-f_1\left(Q^*\right) \geq \alpha_1\left\|Q-Q^*\right\|^2 \quad \forall Q\in\mathcal{Q}
$$
$$
f_2(Q)-f_2\left(Q^*\right) \geq \alpha_2\left\|Q-Q^*\right\|^2 \quad \forall Q\in\mathcal{Q}
$$
Now suppose, for the purpose of contradiction, that there exists $\hat{Q}\in\mathcal{Q}$ such that $\left\langle\nabla f_1(\hat{Q}), \nabla f_2(\hat{Q})\right\rangle<0$. Consider the direction $d:=-g_1=-\nabla f_1(\hat{Q})$. Then $\nabla f_1(\hat{Q})^{\top} d=g_1^{\top}\left(-g_1\right)=-\left\|g_1\right\|^2<0$ holds. This implies that $f_1(\hat{Q}+\eta d)<f_1(\hat{Q})$. Then QG condition for $f_1$ implies that
$$
\left\|\hat{Q}+\eta d-Q^*\right\|<\left\|\hat{Q}-Q^*\right\|
$$
Now, note that $\nabla f_2(\hat{Q})^{\top} d=g_2^{\top}\left(-g_1\right)=-g_1^{\top} g_2$. Since $g_1^{\top} g_2<0$, $\nabla f_2(\hat{Q})^{\top} d>0$. Therefore, $f_2(\hat{Q}+\eta d)>f_2(\hat{Q})$ for sufficiently small $\eta>0$. That is, $f_2(\hat{Q}+\eta d)-f_2(Q^\ast)>f_2(\hat{Q})-f_2(Q^\ast)$. By QG condition, this implies that $\left\|\hat{Q}+\eta d-Q^*\right\| >\left\|\hat{Q}-Q^*\right\|$. Contradiction.

\end{proof}



\subsection{Proof of Lemma \ref{lem:linPolyNonsingular}}
We consider the function class
$$
Q_{\boldsymbol{\theta}}(s, a)=\boldsymbol{\theta}^{\top} \phi(s, a)
$$
where $\phi: \mathcal{S} \times \mathcal{A} \rightarrow \mathbb{R}^d$ is a known feature map with $\|\phi(s, a)\| \leq B$ almost surely and $\boldsymbol{\theta} \in \mathbb{R}^d$ is the parameter vector. Then for any unit vector $u \in \mathbb{R}^d$, 
$$
\left|u^{\top} \phi(s, a)\right| \leq\|u\|\|\phi(s, a)\|=B
$$
Then by using Hoeffding's Lemma, we have
$$
\mathbb{E}\left[e^{\lambda u^{\top} \phi(s, a)}\right] \leq \exp \left(\frac{\lambda^2 B^2}{2}\right)
$$
Therefore we have
$$
\mathbb{P}\left(\left|u^{\top} \phi(s, a)\right| \geq t\right) \leq 2 e^{-t^2 /\left(2 B^2\right)} \quad \forall t>0
$$
Now for the given dataset $\mathcal{D}$, define
$$M=\left[\begin{array}{c}\phi\left(s_1, a_1\right)^{\top} \\ \phi\left(s_2, a_2\right)^{\top} \\ \vdots \\ \phi\left(s_{|\mathcal{D}|}, a_{|\mathcal{D}|}\right)^{\top}\end{array}\right] \in \mathbb{R}^{|\mathcal{D}| \times d}$$
Note that $D Q_\theta=M$. Then by \citet{rudelson2009smallest}, we have

$$\mathbb{P}\left(\sigma_{\min }(D Q_\theta) \geq \sqrt{|\mathcal{D}|}-\sqrt{d}\right) \geq 1-e^{-C|\mathcal{D}|}$$
provided that the dataset size satisfies $|\mathcal{D}| \geq C d$ with $C>1$.


\subsection{Proof of Theorem \ref{thm:thetaEnjoysPL}}\label{sec:proofOfthetaEnjoysPL}
\begin{proof}[Proof of Theorem \ref{thm:thetaEnjoysPL}] 
For convenience in notation,
$$
f(Q_{\boldsymbol{\theta}}):=\overline{\mathcal{L}_{\mathrm{NLL}}}(Q_{\boldsymbol{\theta}})+\lambda \mathbbm{1}_{a=a_s}\overline{\mathcal{L}_{\mathrm{BE}}}(Q_{\boldsymbol{\theta}})
$$
and denote $Q_{\boldsymbol{\theta}} = Q (\boldsymbol{\theta})$. Set $
h(\boldsymbol{\theta}):=f(Q(\boldsymbol{\theta}))
$, where $f$ is the loss in terms of the function $Q$. Then $
h\left(\boldsymbol{\theta}^*\right)=f\left(Q\left(\boldsymbol{\theta}^*\right)\right)=f\left(Q^*\right)
$
by realizability (Assumption \ref{ass:realizability}). Then
\begin{align}
    \left\|\nabla_\theta h(\boldsymbol{\theta})\right\|_2^2&=\left\|D Q({\boldsymbol{\theta}})^{\top} \nabla_Q f(Q(\boldsymbol{\theta}))\right\|_2^2 \notag
    \\
    &\ge \sigma^2_{\min} \left(D Q(\boldsymbol{\theta})\right)\left\|\nabla_Q f\left(Q({\boldsymbol{\theta})}\right)\right\|_2^2 \tag{$\dim(\mathcal{S}), \dim(\mathcal{A})<\infty$}
    \\
    &\ge m^2 \left\|\nabla_Q f\left(Q({\boldsymbol{\theta}})\right)\right\|_2^2\tag{Assumption \ref{ass:nonSingularJac}}
    \\
    &\ge 2(m^2 c)(f(Q({\boldsymbol{\theta}}))-f(Q^\ast)) \tag{PL in terms of $Q$}
    \\
    &=2(m^2c)(h(\boldsymbol{\theta})-h(\boldsymbol{\theta}^\ast)) \notag
\end{align}
\end{proof}


\iffalse
\begin{proof}[Proof of Lemma \ref{lem:expDKLalsoPL}]
    \begin{align}
    \left\|\nabla_x F(x)\right\|^2&=\left\langle\nabla_x F(x), \nabla_x F(x)\right\rangle \notag
    \\
&=\left\langle\mathbb{E}_{z\sim \nu}\left[\nabla_x f(x, z)\right], \mathbb{E}_{z^{\prime}\sim\nu}\left[\nabla_x f\left(x, z^{\prime}\right)\right]\right\rangle \notag
    \\
    &=\mathbb{E}_{z\sim \nu, z^{\prime}\sim \nu}\left[\left\langle\nabla_x f(x, z), \nabla_x f\left(x, z^{\prime}\right)\right\rangle\right] \notag
    \\
    &=\mathbb{E}_{z \sim \nu}\left[\left\|\nabla_x f(x,z)\right\|^2\right]+\mathbb{E}_{z\neq z^{\prime} \sim \nu}\left[\left\langle\nabla_x f(x,z), \nabla_x f(x,{z^{\prime}})\right\rangle\right] \notag
    \\
    &\geq 2 \mu \mathbb{E}_{z \sim \mu}\left[f(x,z)-f_z^*\right]+\mathbb{E}_{z, z^{\prime} \sim \nu}\left[\left\langle\nabla_x f(x,z), \nabla_x f(x,z^{\prime})\right\rangle\right] \notag
    \\
    &=2 \mu\left[F(x)-F^*\right]+\mathbb{E}_{z\neq z^{\prime} \sim \nu}\left[\left\langle\nabla_x f(x,z), \nabla_x f(x,z^{\prime})(x)\right\rangle\right] \notag
    \\
    &=2 \mu\left[F(x)-F^*\right] \tag{Lemma \ref{lem:Expcross}}
\end{align}
\end{proof}

\begin{lem}\label{lem:Expcross}
    For two PL and Lipschitz smooth $f_1$ and $f_2$ with the same set of minimizers $x^\ast:=\underset{x}{\arg \min } f_1(x)=\underset{x}{\arg \min } f_2(x)$, $\left\langle\nabla_x f_1(x), \nabla_x f_2(x)\right\rangle \ge 0\;\;\forall x \in \mathcal{X}$.
\end{lem}
\begin{proof}[Proof of Lemma \ref{lem:Expcross}]
    Define $\left\|x-x^\ast\right\|:=\inf _{\tilde{x} \in x^\ast}\|x-\tilde{x}\|$. From the fact that $f_1$ and $f_2$ both satisfy PL condition and Lipschitz smoothness, $f_1$ and $f_2$ satisfy the Quadratic Growth condition, i.e., Quadratic Growth (QG) condition \cite{liao2024error}, i.e., 
    there exists $\alpha_1, \alpha_2>0$ such that:
$$
f_1(x)-f_1^\ast \geq \alpha_1\left\|x-x^\ast\right\|^2 \quad \forall x\in\mathcal{X}
$$
$$
f_2(x)-f_2^\ast \geq \alpha_2\left\|x-x^\ast\right\|^2 \quad \forall x\in\mathcal{X}
$$
Now suppose, for the purpose of contradiction, that there exists $\hat{x} \in \mathcal{X}$ such that $
\left\langle\nabla_x f_1(\hat{x}), \nabla_x f_2(\hat{x})\right\rangle<0
$. Define 
$g_1:=\nabla_x f_1(\hat{x}), \quad g_2:=\nabla_x f_2(\hat{x})$ and consider the direction $d:=-g_1$. Then $g_1^{\top} d=-\left\|g_1\right\|^2<0$ holds. For sufficiently small step size $\eta>0$, $
f_1(\hat{x}+\eta d)<f_1(\hat{x})
$ holds.
By the QG condition for $f_1$ :
$$
\left\|\hat{x}+\eta d-x^\ast\right\|<\left\|\hat{x}-x^\ast\right\| .
$$
Now consider $f_2$. Using the assumption $\left\langle g_1, g_2\right\rangle<0$, $
g_2^{\top} d=g_2^{\top}\left(-g_1\right)=-\left\langle g_1, g_2\right\rangle>0
$ holds.
Thus, $f_2(x)$ increases along $d$. For sufficiently small $\eta>0$, 
$
f_2(\hat{x}+\eta d)>f_2(\hat{x})
$ must hold. 
By the QG condition for $f_2$ :
$$\left\|\hat{x}+\eta d-x^\ast\right\|>\left\|\hat{x}-x^\ast\right\| .
$$
This is a contradiction.
\end{proof}
\fi



\subsection{Proof of Proposition \ref{prop:linConvergence} (Global optima convergence under ERM-IRL)}\label{sec:ProofLinConv}


\subsubsection*{1. Optimization error analysis}\;
\\
Define 
$f_1(Q)=\mathbb{E}_{(s, a)\sim \pi^*, \nu_0}  \left[-\log\left(\hat{p}_{Q}(\;a
    \mid s)\right)\right]$ and $f_2(Q)=\mathbb{E}_{(s, a)\sim \pi^*, \nu_0}  \left[ \mathbbm{1}_{a = a_s} \mathcal{L}_{BE}(Q)\left(s,a\right)\right]$.
    By Theorem \ref{thm:mainopt}, there is a unique minimizer $Q^*$ for $f_1+\lambda f_2$, which is the same for all $\lambda>0$. Also, $ f_1+\lambda f_2$ satisfies PL condition by Lemma \ref{lem:f1f2sumPL}.

    In equation \ref{eq:mainopt} of Theorem \ref{thm:mainopt}, we saw that $f_2(Q)$ is actually of form $\max_\zeta$ $f_2(Q, \zeta)$. This implies that minimization of $f_1+\lambda f_2$, a mini-max optimization problem that satisfies two-sided PL. (The inner maximization problem is trivially strongly convex, which implies PL). 
    
    Now denote $$
    f_\lambda(Q, \zeta) :=  f_1(Q)+\lambda f_2(Q, \zeta)$$
    $$g_\lambda(Q) := \max_\zeta f_\lambda(Q, \zeta)
    $$ 
    $$
    g^*_\lambda = \min_Q g_\lambda(Q) = \min_Q \max_\zeta f_\lambda(Q, \zeta)$$
    Note that $$g_\lambda(Q)-g^*_\lambda\ge 0$$
    $$g_\lambda(Q)-f_\lambda(Q, \zeta)\ge 0
    $$
    for any $(Q, \zeta)$. Furthermore, they are both equal to 0 if and only if $(Q, \zeta)$ is a minimax point, which is $Q^\ast$ and $\zeta^\ast$. More precisely, we have
    $$
    |f_\lambda(Q,\zeta) - g_\lambda^\ast| \le (g_\lambda(Q)-g^*_\lambda) + (g_\lambda(Q)-f_\lambda(Q, \zeta))
    $$    
    Therefore, we would like to find $Q, \zeta$ that for $\alpha>0$ $a_\lambda(Q) + \alpha b(Q, \zeta) = 0$, where 
    $$a_\lambda(Q): = g_\lambda(Q)-g^*_\lambda$$
    $$b_\lambda(Q, \zeta): = g_\lambda(Q)-f_\lambda(Q, \zeta)$$
    
    \noindent At iteration 0, algorithm starts from $\hat{Q}_0$ and $\zeta=\zeta_0$. We denote the $Q, \zeta$ value at iteration $T$ as $\hat{Q}_T$ and $\zeta_T$. Also, we define $P_T$ as
    $$P_T:= a_\lambda(\hat{Q}_T) + \alpha b_\lambda(\hat{Q}_T, \zeta_T)$$
    Set that $f_\lambda(Q, \zeta)$ satisfies $\mu_1$-PL for $Q$ and $\mu_2$-PL for $\zeta$. Let $\alpha = 1/10$, $\tau_1^T=\frac{\beta}{\gamma+T}, \tau_2^T=\frac{18 l^2 \beta}{\mu_2^2(\gamma+T)}$ for some $\beta>2 / \mu_1$, $L=l+l^2 / \mu_2$, and $ \gamma>0$ such that $\tau_1^1 \leq \min \left\{1 / L, \mu_2^2 / 18 l^2\right\}$. Then the following Theorem holds.

    \begin{thm}[Theorem 3.3, Yang et al  \cite{yang2020global}]\label{thm:Yang} Consider the setup where $\lambda>0$ is fixed. Then applying Algorithm \ref{alg:estimation1} using stochastic gradient descent (SGD), $P_T$ satisfies
    $$
    P_T \leq \frac{\nu}{\gamma+T}
    $$
    \end{thm}

Note that $a_\lambda$ satisfies the PL condition with respect to $Q$ and smoothness since subtracting a constant from PL is PL. Therefore, $a_\lambda$ satisfies Quadratic Growth (QG) condition by \cite{liao2024error}, i.e., 
$$
C \cdot \mathbb{E}_{(s, a) \sim \pi^*, \nu_0}\left[\left(\hat{Q}_T(s, a)-\hat{Q}_N(s, a)\right)^2\right] \le a_\lambda(Q)-0 \le \mathcal{O}(1/T).
$$
Since $a_\lambda \le P_T$, we can conclude that $ \mathbb{E}_{(s, a) \sim \pi^*, \nu_0}\left[\left(\hat{Q}_T(s, a)-\hat{Q}_N(s, a)\right)^2\right]$ is $\mathcal{O}(1/T)$.
 

\subsubsection*{2. Statistical error analysis.}

\noindent 
Throughout, we closely follow \citet{charles2018stability}. First note that:
\begin{itemize}[leftmargin=*]
\item $Q\in\mathcal{Q}$ is assumed to be bounded, as \citep{rust1994structural} implies that $Q^\ast$ is bounded for $\beta<1$. Therefore, by Lemma \ref{lem:JointLipschitz} (below), the Lipschitz smoothness we proved in Lemma \ref{lem:ConvexityMLE} and \ref{lem:BELipschitz} implies $L$-Lipschitzness for some $L<0$. Since composition of Lipschitz continuous functions are Lipschitz continuous, both $\frac{1}{|\mathcal{D}|} \sum_{(s, a) \in \mathcal{D}} \mathcal{L}_{N L L}(Q_{\boldsymbol{\theta}})(s, a)$ and $\frac{1}{|\mathcal{D}|} \sum_{(s, a) \in \mathcal{D}} \mathcal{L}_{B E}(Q_{\boldsymbol{\theta}})(s, a)$ are Lipschitz continuous in $\theta$. Therefore, $\mathcal{R}_{emp}$ is also $L$-Lipschitz continuous for some $L>0$.
\item As discussed in Lemma \ref{lem:f1f2sumPL} and its proof, $\mathcal{R}_{emp}$ satisfies $\mu$-PL condition for some $\mu>0$ and has a unique minimizer. 
\end{itemize}
Denote the minimizer of empirical risk function $\mathcal{R}_{emp}$ for the data set $\mathcal{D}$ as $Q^\ast_{D}$, where $|\mathcal{D}|=N$. Then by \citet{charles2018stability}, $Q^\ast_{D}$ and $Q^\ast$ satisfies
$$
\bigl|\mathbb{E}_{\mathcal{D}}\bigl[\mathcal{R}_{exp}(Q^\ast_{\mathcal{D}})-\mathcal{R}_{emp}(Q^\ast_{\mathcal{D}};\mathcal{D})\bigr]\bigr| \le \frac{2L^2}{\mu N}.
$$
$$
\bigl|\mathbb{E}_{\mathcal{D}}\bigl[\mathcal{R}_{exp}(Q^\ast)-\mathcal{R}_{emp}(Q^\ast;\mathcal{D})\bigr]\bigr| \le \frac{2L^2}{\mu N}.
$$
Since
\begin{align}
    &\mathcal{R}_{\exp }\left(Q_D^*\right)-\mathcal{R}_{\exp }\left(Q^*\right) \notag
    \\
    &=\left[\mathcal{R}_{\exp }\left(Q_D^*\right)-\mathcal{R}_{\text {emp }}\left(Q_D^* ; \mathcal{D}\right)\right]+\underbrace{\left[\mathcal{R}_{\text {emp }}\left(Q_D^* ; \mathcal{D}\right)-\mathcal{R}_{e m p}\left(Q^* ; \mathcal{D}\right)\right]}_{\leq 0}+\left[\mathcal{R}_{\text {emp }}\left(Q^* ; \mathcal{D}\right)-\mathcal{R}_{\text {exp }}\left(Q^*\right)\right] \notag
\end{align}
We have 
$$
\mathbb{E}_{\mathcal{D}}\bigl[\mathcal{R}_{\exp }\left(Q_D^*\right)-\mathcal{R}_{\exp }\left(Q^*\right)\bigr] \le \frac{4L^2}{\mu N}.
$$
Since smoothness and PL implies Quadratic growth (QG) condition \citep{liao2024error}, we have 
$$
\mathbb{E}_{\mathcal{D}}\bigl[\mathbb{E}_{(s, a) \sim \pi^*, v_0}\left[\left(Q^\ast_\mathcal{D}(s, a)-Q^\ast(s, a)\right)^2\right]\bigr] \le  C\frac{4 L^2}{\mu N}
$$

\begin{lem}
\label{lem:JointLipschitz}
  Let $f: \mathcal{Q} \to \mathbb{R}$ be a differentiable function defined on a space of bounded functions $\mathcal{Q} \subseteq L^2(\mu)$, where $\mathcal{Q}$ is assumed to be bounded in $L^2(\mu)$. That is, there exists a constant $M > 0$ such that for all $Q \in \mathcal{Q}$,
$$
\| Q \|_{L^2(\mu)} \leq M.
$$
If $f$ is differentiable in the Fréchet sense, then $f$ is Lipschitz continuous in the $L^2(\mu)$ norm. That is, there exists a constant $K > 0$ such that for all $Q_1, Q_2 \in \mathcal{Q}$,
$$
|f(Q_1) - f(Q_2)| \leq K \|Q_1 - Q_2\|_{L^2(\mu)}.
$$
\end{lem}

\begin{proof}
    Since $f$ is differentiable, we use the mean value theorem in Banach spaces (see, e.g., \citet{yosida2012functional}). Specifically, for any $Q_1, Q_2 \in \mathcal{Q}$, there exists some intermediate function $\tilde{Q}$ on the line segment between $Q_1$ and $Q_2$ such that:
$$
f(Q_1) - f(Q_2) = \langle \nabla f(\tilde{Q}), Q_1 - Q_2 \rangle_{L^2(\mu)}.
$$
Applying the Cauchy-Schwarz inequality in $L^2(\mu)$, we obtain:
$$
|f(Q_1) - f(Q_2)| = |\langle 
\nabla f(\tilde{Q}), Q_1 - Q_2 \rangle_{L^2(\mu)}|
$$
$$
\leq \|\nabla f(\tilde{Q})\|_{L^2(\mu)} \cdot \|Q_1 - Q_2\|_{L^2(\mu)}.
$$
Since $\mathcal{Q}$ is bounded in $L^2(\mu)$, there exists a constant $B > 0$ such that:
$$
\sup_{Q \in \mathcal{Q}} \|\nabla f(Q)\|_{L^2(\mu)} \leq B.
$$
Thus, we can take $K = B$, yielding the desired Lipschitz continuity bound:
$$
|f(Q_1) - f(Q_2)| \leq B \|Q_1 - Q_2\|_{L^2(\mu)}.
$$
\end{proof}

\iffalse


\begin{lem}
$$
\overline{\mathcal{L}_{\mathrm{BE}}}(Q)-\frac{1}{N}\sum_{(s,a,s)\in \mathcal{D}} \bigl(\bigl(\hat{\mathcal{T}} Q\left(s, a, s^{\prime}\right)-Q(s, a)\bigr)^2 -\min_\zeta\beta^2  \left(V_{Q}\left(s^{\prime}\right)-\zeta\right)^2\bigr)=\mathcal{O}(1/N) $$
\end{lem}
\begin{proof}
According to the proof procedure of Lemma \ref{lem:OurBiconj}, we have
    \begin{align}
    &\overline{\mathcal{L}_{\mathrm{BE}}}(Q)-\frac{1}{N}\sum_{(s,a,s)\in \mathcal{D}} \bigl(\bigl(\hat{\mathcal{T}} Q\left(s, a, s^{\prime}\right)-Q(s, a)\bigr)^2 -\min_\zeta\beta^2  \left(V_{Q}\left(s^{\prime}\right)-\zeta\right)^2\bigr)\notag
    \\
    &=\overline{\mathcal{L}_{\mathrm{BE}}}(Q)-\frac{1}{N}\sum_{(s,a,s)\in \mathcal{D}} \bigl(\bigl(\hat{\mathcal{T}} Q\left(s, a, s^{\prime}\right)-Q(s, a)\bigr)^2 -\min_\rho
    (\rho - \hat{\mathcal{T}}Q(s, a, s^\prime))^2\bigr)\notag
    \end{align}
    By adapting Lemma E.5 of \citet{touati2020sharp} for expert training data, we have $$\overline{\mathcal{L}_{\mathrm{BE}}}(Q)-\frac{1}{N}\sum_{(s,a,s)\in \mathcal{D}} \bigl(\bigl(\hat{\mathcal{T}} Q\left(s, a, s^{\prime}\right)-Q(s, a)\bigr)^2 -\min_\rho
    (\rho - \hat{\mathcal{T}}Q(s, a, s^\prime))^2\bigr)=\mathcal{O}(1/N)$$This finishes the proof.
\end{proof}


\begin{lem} 
\;
\\
With $|\mathcal{D}|=N$,
    $|\mathbb{E}_{(s, a) \sim \pi^*, \nu_0}\left[-\log \left(\hat{p}_Q(a \mid s)\right)\right]-\frac{1}{|\mathcal{D}|}\sum_{(s,a,s^\prime)\in \mathcal{D}} (-\log(\hat{p}_Q(a \mid s)))| = \mathcal{O}(1/\sqrt{N})$.
\end{lem}
\begin{proof}
Define 
$$
F(\mathcal{D})=\frac{1}{N} \sum_{\left(s, a, s^{\prime}\right) \in \mathcal{D}}\left[-\log \hat{p}_Q(a \mid s)\right]
$$
Consider replacing one data sample $\left(s_i, a_i, s_i^{\prime}\right)$ in $\mathcal{D}$ with another independent sample, resulting $\mathcal{D}^\prime$. This leads to a different estimate $\hat{Q}_N^{\prime}$ and corresponding softmax policy $\hat{p}_Q^{\prime}$. 
\end{proof}


Combining the above two lemma, we have
\begin{align}
  &\frac{1}{N}\bigl[\sum_{(s,a,s^\prime)\in \mathcal{D}}\bigl({-\log \left(\hat{p}_Q(a \mid s)\right)}\bigr)+ 
\lambda\mathbbm{1}_{a=a_s}\notag
    \\
    &\bigl(  \sum_{(s,a,s^\prime)\in \mathcal{D}} {\bigl(\hat{\mathcal{T}} Q\left(s, a, s^{\prime}\right)-Q(s, a)\bigr)^2}  -\beta^2 \min_{\zeta\in \mathbb{R}^{S\times A}} 
   \sum_{(s,a,s^\prime)\in \mathcal{D}} {\left(V_{Q}\left(s^{\prime}\right)-\zeta(s,a)\right)^2}\bigr) \bigr]\notag
   \\
   &- \notag
   \\
   & \mathbb{E}_{(s, a) \sim \pi^*, \nu_0, s^{\prime} \sim P(s, a)}\bigl[-\log \left(\hat{p}_Q(a \mid s)\right) + \lambda\mathbbm{1}_{a=a_s}\bigl\{\bigl(\hat{\mathcal{T}} Q\left(s, a, s^{\prime}\right)-Q(s, a)\bigr)^2 \notag
    \\
    & \quad -\beta^2  \min_{\zeta\in \mathbb{R}^{S\times A}}  \mathbb{E}_{(s, a) \sim \pi^*, \nu_0, s^{\prime} \sim P(s, a)} \bigl(\left(V_{Q}\left(s^{\prime}\right)-\zeta(s,a)\right)^2\bigr) \bigr\} \bigr]\notag
    &
    \\
     &= \mathcal{O}(1/\sqrt{N}). \notag
\end{align}
\noindent 
Now recall that we showed that the sum of $\overline{\mathcal{L}_{\mathrm{BE}}}(Q)$ term and $\mathbb{E}_{(s, a) \sim \pi^*, \nu_0}\left[-\log \left(\hat{p}_Q(a \mid s)\right)\right]$ term satisfy PL. Since Those two terms are also smooth, the sum satisfies the Quadratic Growth condition (\cite{liao2024error}). Therefore
\begin{align}
  &\frac{1}{N}\bigl[\sum_{(s,a,s^\prime)\in \mathcal{D}}\bigl({-\log \left(\hat{p}_Q(a \mid s)\right)}\bigr)+ 
\lambda\mathbbm{1}_{a=a_s}\notag
    \\
    &\bigl(  \sum_{(s,a,s^\prime)\in \mathcal{D}} {\bigl(\hat{\mathcal{T}} Q\left(s, a, s^{\prime}\right)-Q(s, a)\bigr)^2}  -\beta^2 \min_{\zeta\in \mathbb{R}^{S\times A}} 
   \sum_{(s,a,s^\prime)\in \mathcal{D}} {\left(V_{Q}\left(s^{\prime}\right)-\zeta(s,a)\right)^2}\bigr) \bigr]\notag
   \\
   &- \notag
   \\
   & \mathbb{E}_{(s, a) \sim \pi^*, \nu_0, s^{\prime} \sim P(s, a)}\bigl[-\log \left(\hat{p}_Q(a \mid s)\right) + \lambda\mathbbm{1}_{a=a_s}\bigl\{\bigl(\hat{\mathcal{T}} Q\left(s, a, s^{\prime}\right)-Q(s, a)\bigr)^2 \notag
    \\
    & \quad -\beta^2  \min_{\zeta\in \mathbb{R}^{S\times A}}  \mathbb{E}_{(s, a) \sim \pi^*, \nu_0, s^{\prime} \sim P(s, a)} \bigl(\left(V_{Q}\left(s^{\prime}\right)-\zeta(s,a)\right)^2\bigr) \bigr\} \bigr]\notag
    &
    \\
     &\ge  C\cdot\mathbb{E}_{(s, a) \sim \pi^*, \nu_0}\left[\left(Q(s, a)-\hat{Q}_N(s, a)\right)^2\right] \notag
\end{align}
 for some $C$. This implies that $\cdot\mathbb{E}_{(s, a) \sim \pi^*, \nu_0}\left[\left(Q(s, a)-\hat{Q}_N(s, a)\right)^2\right]=\mathcal{O}(1/\sqrt{N})$.






\subsection{Temp}

\begin{lem}[Near-strong convexity of $\overline{\mathcal{L}_{NLL}}(Q)$]\label{lem:NLLnearstrongconv} Hessian of $\overline{\mathcal{L}_{NLL}}(Q)$ has strictly positive eigenvalues except one eigenvalue of 0, of which eigenvector is $\mathbf{1}$. 

    
\end{lem}
\begin{proof}
From Lemma \ref{lem:minMLE}, we know
\begin{align}
\mathbb{E}_{(s, a) \sim \pi^*, \nu_0}\left[-\log \left(\hat{p}_Q(a \mid s)\right)\right] =\mathbb{E}_{s \sim \pi^*, \nu_0}\left[D_{K L}\left(\pi^*(\cdot \mid s) \| \hat{p}_Q(\cdot \mid s)\right)\right]+\mathbb{E}_{(s, a) \sim \pi^*, \nu_0}\left[\ln \pi^*(a \mid s)\right] \notag
\end{align}
Note that the second term is not dependent on $Q$. Therefore, we can instead prove that the first term $\mathbb{E}_{s \sim \pi^*, \nu_0}\left[D_{K L}\left(\pi^*(\cdot \mid s) \| \hat{p}_Q(\cdot \mid s)\right)\right]$ has strictly positive eigenvalues except one eigenvalue of 0, of which eigenvector is $\mathbf{1}$. First, note that
\begin{align}
     \nabla_{Q(s, \cdot)}D_{K L}\left(\pi^*(\cdot \mid s) \| \hat{p}_Q(\cdot \mid s)\right)&= \nabla_{Q(s, \cdot)}\left(-\sum_a \pi^*(a \mid s) \log \hat{p}_Q(a \mid s)\right) \notag
     \\
     &=-\sum_a \pi^*(a \mid s)\left(\delta_{a, a^{\prime}}-\hat{p}_Q\left(a^{\prime} \mid s\right)\right)\notag
     \\
     &=-\left[\pi^*\left(a^{\prime} \mid s\right)-\hat{p}_Q\left(a^{\prime} \mid s\right) \sum_a \pi^*(a \mid s)\right]_{a^{\prime} \in \mathcal{A}}\notag
     \\
     &=\left[\hat{p}_Q\left(a^{\prime} \mid s\right)-\pi^*\left(a^{\prime} \mid s\right)\right]_{a^{\prime} \in \mathcal{A}} \notag
\end{align}
Also, from $\frac{\partial \hat{p}_Q(a\mid s)}{\partial Q_{a^{\prime}}}=\hat{p}_Q(a\mid s)\left(\delta_{a, a^{\prime}}-\hat{p}_Q\left(a^{\prime}\mid s\right)\right)$, Hessian $H=\operatorname{diag}\left(\hat{p}_Q\right)-\hat{p}_Q \hat{p}_Q^{\top}$. Interestingly, this Hessian is equivalent to the covariance matrix of random vector $X \in \mathbb{R}^{|\mathcal{A}|}$, defined by $$X(a)= \begin{cases}1 & \text { if } A=a \\ 0 & \text { otherwise }\end{cases}, \text{ where } \mathbb{P}(A=a)=\hat{p}_Q(a)=\frac{\exp (Q(a))}{\sum_b \exp (Q(b))}$$
\;
\\
We will now show that this Hessian $H$ has only one eigenvector direction with eigenvalue 0, and other eigenvalues are all strictly positive. First, $H \mathbf{1}=\hat{p}_Q-\hat{p}_Q\left(\sum_{a^{\prime}} \hat{p}_Q\left(a^{\prime}\right)\right)=\hat{p}_Q-\hat{p}_Q=0$. 
Now define $V=\left\{v \in \mathbb{R}^{|\mathcal{A}|}: v^{\top} \mathbf{1}=\sum_a v(a)=0\right\}$, which is the subspace that is orthogonal to $\mathbf{1}$. Then for $v \in V$,
$$
v^{\top} H v=v^{\top} \operatorname{Cov}(X) v=\operatorname{Var}(v(A)),
$$
where $v(A)=\sum_a v(a) \delta_{A, a}$ is a scalar-valued random variable depending on the random draw $A \sim \hat{p}_Q$. We have $\operatorname{Var}(v(A))>0$, because:
\begin{itemize}
    \item If $v(a)$ are not all identical, then $v(A)$ is a non-constant random variable under $\hat{p}_Q$. Because $\hat{p}_Q$ assigns positive probability to each action, the random variable $v(A)$ takes on at least two distinct values with positive probability. Hence, its variance $\operatorname{Var}(v(A))$ is strictly greater than zero.
    \item If $v(a)$ were constant for all $a$, say $v(a)=c$, then $\sum_a v(a)=c \sum_a 1=$ $c|\mathcal{A}|$. This contradicts $v\in V$.
\end{itemize}
    
\end{proof} 

\fi
\newpage

\section{Equivalence between Dynamic Discrete choice and Entropy regularized Inverse Reinforcement learning}\label{sec:DDCIRL}


\subsection{Properties of Type 1 Extreme Value (T1EV) distribution}
Type 1 Extreme Value (T1EV), or Gumbel distribution, has a location parameter and a scale parameter. The T1EV distribution with location parameter $\nu$ and scale parameter 1 is denoted as Gumbel $(\nu, 1)$ and has its CDF, PDF, and mean as follows:
$$
\begin{gathered}
\text { CDF: } F(x ; \nu)=e^{-e^{-(x-\nu)}}
\\
\text { PDF: } f(x ; \nu)=e^{-\left((x-\nu)+e^{-(x-\nu)}\right)}
\\
\text { Mean } = \nu + \gamma
\end{gathered}
$$

Suppose that we are given a set of $N$ independent Gumbel random variables $G_i$, each with their own parameter $\nu_i$, i.e. $G_i \sim \operatorname{Gumbel}\left(\nu_i, 1\right)$.

\begin{lem}\label{lem:GumbelMax}
    Let $Z=\max G_i$. Then $Z \sim \operatorname{Gumbel}\left(\nu_Z=\log \sum_{i} e^{\nu_i}, 1\right)$.
\end{lem}
\begin{proof}
    $F_Z(x)=\prod_{i} F_{G_i}(x)=\prod_{i} e^{-e^{-\left(x-\nu_i\right)}}=e^{-\sum_{i} e^{-\left(x-\nu_i\right)}}=e^{-e^{-x} \sum_{i} e^{\nu_i}}=e^{-e^{-\left(x-\nu_Z\right)}}$
\end{proof}

\begin{cor}\label{cor:GumbelOptProb}
    $P\left(G_k>\max _{i \neq k} G_i\right)=\frac{e^{\nu_k}}{\sum_{i} e^{\nu_i}}$.
\end{cor}
\begin{proof}
\begin{align}
    P\left(G_k>\max _{i \neq k} G_i\right)&=\int_{-\infty}^{\infty} P\left(\max _{i \neq k} G_i<x\right) f_{G_{k}}(x) d x\notag
    \\&=\int_{-\infty}^{\infty} e^{-\sum_{i \neq k} e^{-\left(x-\nu_i\right)}} e^{-\left(x-\nu_k\right)} e^{-e^{-\left(x-\nu_k\right)}} d x \notag
    \\&=e^{\nu_k} \int_{-\infty}^{\infty} e^{-e^{-x} \sum_{i} e^{\nu_i}} e^{-x} d x \notag
    \\&=e^{\nu_k}\int_{\infty}^0 e^{-u S} u\left(-\frac{d u}{u}\right)  \; \; \left(\text{Let } \sum_{i} e^{\nu_i}=S, u=e^{-x}\right)\notag
    \\&=e^{\nu_k}\int_0^{\infty} e^{-u S} d u=e^{\nu_k}\left[-\frac{1}{S} e^{-u S}\right]_0^{\infty} = \frac{e^{\nu_k}}{S}\notag 
    \\&=\frac{e^{\nu_k}}{\sum_{i} e^{\nu_i}} \notag
\end{align}
\end{proof}

\begin{lem}\label{lem:ExpofLargerGumbel}
    Let $G_1\sim \text{Gumbel }(\nu_1, 1)$ and $G_2\sim \text{Gumbel }(\nu_2, 1)$. Then $\mathbb{E}\left[G_1 \mid G_1\geq G_2\right]=\gamma + \log \left( 1+ e^{\left(-(\nu_1-\nu_2)\right)} \right)$ holds. 
\end{lem}

\begin{proof}
    Let $\nu_1-\nu_2 = c$. Then $\mathbb{E}\left[G_1 \mid G_1\geq G_2\right]$ is equivalent to $\nu_1 + \frac{\int_{-\infty}^{+\infty} x F(x+c) f(x) \mathrm{d} x}{\int_{-\infty}^{+\infty} F(x+c) f(x) \mathrm{d} x}$, where the pdf $f$ and cdf $F$ are associated with $\text{Gumbel }(0, 1)$, because

    \begin{align}
        P\left(G_1 \leq x, G_1 \geq G_2\right)&=\int_{-\infty}^x F_{G_2}(t) f_{G_1}(t) d t=\int_{-\infty}^x F\left(t-\nu_2\right) f\left(t-\nu_1\right) d t\notag
        \\
        \mathbb{E}\left[G_1 \mid G_1 \geq G_2\right]&=\frac{\int_{-\infty}^{\infty} x F\left(x+c-\nu_1\right) f\left(x-\nu_1\right) d x}{\int_{-\infty}^{\infty} F\left(x+c-\nu_1\right) f\left(x-\nu_1\right) d x}\notag
        \\
        &=\frac{\int_{-\infty}^{\infty}\left(y+\nu_1\right) F(y+c) f(y) d y}{\int_{-\infty}^{\infty} F(y+c) f(y) d y} \notag
        \\
        &=\nu_1+\frac{\int_{-\infty}^{\infty} y F(y+c) f(y) d y}{\int_{-\infty}^{\infty} F(y+c) f(y) d y} \notag
    \end{align}

    
    Now note that 

    $\begin{aligned} \int_{-\infty}^{+\infty} F(x+c) f(x) \mathrm{d} x & =\int_{-\infty}^{+\infty} \exp \{-\exp [-x-c]\} \exp \{-x\} \exp \{-\exp [-x]\} \mathrm{d} x \\ & \stackrel{a=e^{-c}}{=} \int_{-\infty}^{+\infty} \exp \{-(1+a) \exp [-x]\} \exp \{-x\} \mathrm{d} x \\ & =\frac{1}{1+a}\left[\exp \left\{-(1+a) e^{-x}\right\}\right]_{-\infty}^{+\infty} \\ & =\frac{1}{1+a}\end{aligned}$
    \\
    and
    \\
    $\begin{aligned}  \int_{-\infty}^{+\infty} x F(x+c) f(x) \mathrm{d} x&=\int_{-\infty}^{+\infty} x \exp \{-(1+a) \exp [-x]\} \exp \{-x\} \mathrm{d} x \\ & \stackrel{z=e^{-x}}{=} \int_0^{+\infty} \log (z) \exp \{-(1+a) z\} \mathrm{d} z \\ & =\frac{-1}{1+a}\left[\operatorname{Ei}(-(1+a) z)-\log (z) e^{-(1+a) z}\right]_0^{\infty} \\ & =\frac{\gamma+\log (1+a)}{1+a} \\ & \end{aligned}$
    \\
    Therefore, $\mathbb{E}\left[G_1 \mid G_1\geq G_2\right]=\gamma + \nu_k+ \log \left( 1+ e^{\left(-(\nu_1-\nu_2)\right)} \right)$ holds.
\end{proof}

\begin{cor}\label{cor:GumbelMaxasProb} $ \mathbb{E}\left[G_k \mid G_k = \max G_i\right] = \gamma + \nu_k - \log \left(\frac{e^{\nu_k}}{\sum_{i} e^{\nu_i}}\right)$. 
\end{cor}
\begin{proof}

\begin{align}
    \mathbb{E}\left[G_k \mid G_k = \max G_i\right] &= \mathbb{E}\left[G_k \mid G_k \geq \max_{i\neq k} G_i\right]\notag
    \\
    &=\gamma + \nu_k + \log \left( 1+ e^{\left(-(\nu_k-\log\sum_{i\neq k} e^{\nu_i})\right)}\right)\tag{Lemma \ref{lem:ExpofLargerGumbel}}
    \\
    &=\gamma + \nu_k + \log \left( 1+ \frac{\sum_{i\neq k} e^{\nu_i}}{e^{-\nu_k}}\right) \notag
    \\
    &=\gamma + \nu_k + \log \left(\sum_{i} e^{\nu_i}/e^{\nu_k} \right)\notag
    \\
    &=\gamma + \nu_k -\log \left(e^{\nu_k} / \sum_{i} e^{\nu_i} \right)\notag
\end{align}

\end{proof}


\subsection{Properties of entropy regularization}
Suppose we have a choice out of discrete choice set $\mathcal{A} = \{x_i\}_{i=1}^{|\mathcal{A}|}$. A choice policy can be a deterministic policy such as $\operatorname{argmax}_{i \in 1, \ldots, |\mathcal{A}|} x_i$, or stochastic policy that is characterized by $\mathbf{q}\in \triangle_{\mathcal{A}}$. When we want to enforce smoothness in choice, we can regularize choice by newly defining the choice rule 
$$\arg\max _{\mathbf{q} \in \Delta_{\mathcal{A}}}\left(\langle\mathbf{q}, \mathbf{x}\rangle-\Omega(\mathbf{q})\right)$$
where $\Omega$ is a regularizing function. 

\begin{lem}\label{lem:logsumexp_Shannon}
When the regularizing function is constant $-\tau$ multiple of Shannon entropy $H(\mathbf{q})=-\sum_{i=1}^{|\mathcal{A}|} q_i \log \left(q_i\right)$, $$\max _{\mathbf{q} \in \Delta_{\mathcal{A}}}\left(\langle\mathbf{q}, \mathbf{x}\rangle-\Omega(\mathbf{q})\right)=\tau \log \left(\sum_i \exp \left(x_i / \tau\right)\right)$$ and

$$\arg\max _{\mathbf{q} \in \Delta_{\mathcal{A}}}\left(\langle\mathbf{q}, \mathbf{x}\rangle-\Omega(\mathbf{q})\right)=\frac{\exp \left(\frac{x_i}{\tau}\right)}{\sum_{j=1}^n \exp \left(\frac{x_j}{\tau}\right)}$$
\end{lem}
\begin{proof}
    In the following, I will assume $\tau>0$. Let
\begin{align}
G(\mathbf{q})&=\langle\mathbf{q}, \mathbf{x}\rangle-\Omega(\mathbf{q})\notag
\\
&=\sum_{i=1}^n q_i x_i-\tau \sum_{i=1}^n q_i \log \left(q_i\right)
\notag
\\
&=\sum_{i=1}^n q_i \left(x_i-\tau \log \left(q_i\right)\right) \notag
\end{align}


We are going to find the max by computing the gradient and setting it to 0 . We have
$$
\frac{\partial G}{\partial q_i}=x_i-\tau\left(\log \left(q_i\right)+1\right)
$$
and
$$
\frac{\partial G}{\partial q_i \partial q_j}=\left\{\begin{array}{l}
-\frac{\tau}{q_i}, \quad \text { if } i=j \\
0, \quad \text { otherwise. }
\end{array}\right.
$$

This last equation states that the Hessian matrix is negative definite (since it is diagonal and $-\frac{\tau}{q_1}<0$ ), and thus ensures that the stationary point we compute is actually the maximum. Setting the gradient to $\mathbf{0}$ yields $q_i^*=\exp \left(\frac{x_i}{\tau}-1\right)$, however the resulting $\mathbf{q}^*$ might not be a probability distribution. To ensure $\sum_{i=1}^n q_i^*=1$, we add a normalization:
$$
q_i^*=\frac{\exp \left(\frac{x_i}{\tau}-1\right)}{\sum_{j=1}^n \exp \left(\frac{x_j}{\tau}-1\right)}=\frac{\exp \left(\frac{x_i}{\tau}\right)}{\sum_{j=1}^n \exp \left(\frac{x_j}{\tau}\right)} .
$$

This new $\mathbf{q}^*$ is still a stationary point and belongs to the probability simplex, so it must be the maximum. Hence, you get
$$
\begin{aligned}
\max _{\tau H}(\mathbf{x})& =G\left(\mathbf{q}^*\right)=\sum_{i=1}^n \frac{\exp \left(\frac{x_i}{\tau}\right)}{\sum_{j=1}^n \exp \left(\frac{x_j}{\tau}\right)} x_i-\tau \sum_{i=1}^n \frac{\exp \left(\frac{x_1}{\tau}\right)}{\sum_{j=1}^n \exp \left(\frac{x_j}{\tau}\right)}\left(\frac{x_i}{\tau}-\log \sum_{i=1}^n \exp \left(\frac{x_j}{\tau}\right)\right) \\
&= \tau \log \sum_{i=1}^n \exp \left(\frac{x_j}{\tau}\right)
\end{aligned}
$$
as desired.
\end{proof}





\subsection{IRL with entropy regularization}\label{sec:IRLentropy}

\subsubsection*{Markov decision processes} 
Consider an MDP defined by the tuple $\left(\mathcal{S}, \mathcal{A}, P, \nu_0, r, \beta\right)$:
\begin{itemize}
    \item $\mathcal{S}$ and $\mathcal{A}$ denote finite state and action spaces
    \item $P \in \Delta_{\mathcal{S}}^{\mathcal{S} \times \mathcal{A}}$ is a Markovian transition kernel, and $\nu_0 \in \Delta_{\mathcal{S}}$ is the initial state distribution. 
    \item $r \in \mathbb{R}^{\mathcal{S} \times \mathcal{A}}$ is a reward function.
    \item $\beta \in(0,1)$ a discount factor
\end{itemize}
\subsubsection{Agent behaviors}

Denote the distribution of agent's initial state $s_0\in \mathcal{S}$ as $\nu_0$. Given a stationary Markov policy $\pi \in \Delta_{\mathcal{A}}^{\mathcal{S}}$, an agent starts from initial state $s_0$ and make an action $a_h\in \mathcal{A}$ at state $s_h\in \mathcal{S}$ according to $a_h\sim\pi\left(\cdot \mid s_h\right)$ at each period $h$. We use $\mathbb{P}_{\nu_0}^\pi$ to denote the distribution over the sample space $(\mathcal{S} \times \mathcal{A})^{\infty}=\left\{\left(s_0, a_0, s_1, a_1, \ldots\right): s_h \in \mathcal{S}, a_h \in \mathcal{A}, h \in \mathbb{N}\right\}$ induced by the policy $\pi$ and the initial distribution $\nu_0$. We also use $\mathbb{E}_\pi$ to denote the expectation with respect to $\mathbb{P}_{\nu_0}^\pi$. Maximum entropy inverse reinforcement learning (MaxEnt-IRL) makes the following assumption: 

\begin{asmp}[Assumption \ref{ass:IRLoptimaldecision}] Agent follows the policy $\pi^*=\operatorname{argmax}_{\pi \in \Delta_{\mathcal{A}}^{\mathcal{S}}}$
$\mathbb{E}_\pi\left[\sum_{h=0}^{\infty} \beta^h \left(r\left(s_h, a_h\right)+\lambda\mathcal{H}\left(\pi\left(\cdot \mid s_h\right)\right)\right)\right]$, where $\mathcal{H}$ denotes the Shannon entropy and $\lambda$ is the regularization parameter.
\end{asmp}
For the rest of the section, we use $\lambda=1$.
\\
\;
\\
We define the function $V$ as $V(s_{h^\prime})=\mathbb{E}_{\pi^*}\left[\sum_{h=h^\prime}^{\infty} \beta^{h} \left(r\left(s_h, a_h\right)+\mathcal{H}\left(\pi^*\left(\cdot \mid s_h\right)\right)\right)\right]$ and call it the \textit{value function}. According to Assumption \ref{ass:IRLoptimaldecision}, the value function $V$ must satisfy the Bellman equation, i.e., 
\begin{align}
V\left(s\right)&=\max _{\mathbf{q} \in \triangle_\mathcal{A}}\left\{\mathbb{E}_{a\sim\mathbf{q} }\left[r\left(s, a\right)+\beta \cdot \mathbb{E}\left[V\left(s^\prime\right)\mid s, a\right]\right]+\mathcal{H}(\mathbf{q})\right\}\notag
\\
&=\max _{\mathbf{q} \in \triangle_\mathcal{A}}\left\{\sum_{a\in \mathcal{A}} q_a\left(r\left(s, a\right)+\beta \cdot \mathbb{E}\left[V\left(s^\prime\right)\mid s, a\right]\right)+\mathcal{H}(\mathbf{q})\right\}=\max _{\mathbf{q} \in \triangle_\mathcal{A}}\left\{\sum_{a\in \mathcal{A}} q_a Q(s,a)+\mathcal{H}(\mathbf{q})\right\}\label{eq:VandmaxQ}
\\
&=\ln \left[\sum_{a\in \mathcal{A}}\exp\left(r\left(s, a\right)+\beta \cdot \mathbb{E}\left[{V}\left(s^\prime\right)\mid s, a\right]\right)\right]\label{eq:logsumexp_reg}
\\
&=\ln \left[\sum_{a\in \mathcal{A}}\exp\left(Q(s, a)\right)\right]\label{eq:IRLlogsumQ}
\end{align}
and $\mathbf{q}^*:=\arg\max_{\mathbf{q} \in \triangle_\mathcal{A}} \left\{\mathbb{E}_{a\sim\mathbf{q} }\left[r\left(s, a\right)+\beta \cdot \mathbb{E}\left[V\left(s^\prime\right)\mid s, a\right]\right]+\mathcal{H}(\mathbf{q})\right\}$ is characterized by
\\
\begin{align}
\mathbf{q}^* = [q_1^* \ldots q^*_{|\mathcal{A}|}], \text{ where }
    q^*_a= \frac{\exp \left({Q(s, a)}\right)}{\sum_{a^\prime\in \mathcal{A}} \exp \left({Q(s, a^\prime)}\right)} \text{ for } a\in \mathcal{A}  \label{eq:IRLopt}
\end{align}
where:
\begin{itemize}
    \item $Q(s, a):=r\left(s, a\right)+\beta \cdot \mathbb{E}\left[{V}\left(s^\prime\right)\mid s, a\right]$
    \item Equality in equation \ref{eq:logsumexp_reg} and equality in equation \ref{eq:IRLopt} is from Lemma \ref{lem:logsumexp_Shannon}
    
\end{itemize}
\;
\\
This implies that
\begin{align}
    \pi^*(a\mid s) = q^*_a= \frac{\exp \left({Q(s, a)}\right)}{\sum_{a^\prime\in \mathcal{A}} \exp \left({Q(s, a^\prime)}\right)} \text{ for } a\in \mathcal{A}. \notag
\end{align}
\\
In addition to the Bellman equation in terms of value function $V$, 
Bellman equation in terms of choice-specific value function $Q(s,a)$ can be derived by combining $Q(s, a):=r\left(s, a\right)+\beta \cdot \mathbb{E}\left[{V}\left(s^\prime\right)\mid s, a\right]$ and equation \ref{eq:IRLlogsumQ}:
\begin{align}
    Q(s, a)=r(s, a)+\beta \mathbb{E}_{s^\prime \sim P(s, a)}\left[\ln \left(\sum_{a^{\prime} \in \mathcal{A}} \exp \left(Q\left(s^{\prime}, a^{\prime}\right)\right)\right)\mid s, a\right] \notag
\end{align}
\;
\\
We can also derive an alternative form of choice-specific value function $Q(s,a)$ by combining $Q(s, a):=r\left(s, a\right)+\beta \cdot \mathbb{E}_{s^\prime \sim P(s, a)}\left[{V}\left(s^\prime\right)\mid s, a\right]$ and equation \ref{eq:VandmaxQ}:

\begin{align}
    Q(s, a)&=r\left(s, a\right)+\beta \cdot \mathbb{E}_{s^\prime \sim P(s, a)}\left[\max _{\mathbf{q} \in \triangle_\mathcal{A}}\left\{\sum_{a\in \mathcal{A}} q_a Q(s^\prime,a)+\mathcal{H}(\mathbf{q})\right\}\mid s, a\right]\notag
    \\
    &=r\left(s, a\right)+\beta \cdot \mathbb{E}_{s^\prime \sim P(s, a)}\left[\max _{\mathbf{q} \in \triangle_\mathcal{A}}\left\{\sum_{a\in \mathcal{A}} q_a \left(Q(s^\prime,a) - \log q_a\right)\right\}\mid s, a\right]\notag
    \\
    &=r\left(s, a\right)+\beta \cdot \mathbb{E}_{s^\prime \sim P(s, a), a^\prime \sim \pi^*(a\mid \cdot)}\left[ \left(Q(s^\prime,a^\prime) - \log \pi^*(a^\prime\mid s^\prime)\right)\mid s, a\right] \label{eqn:IRLQBellman_new}
    \\
    &=r\left(s, a\right)+\beta \cdot \mathbb{E}_{s^\prime \sim P(s, a)}\left[ \left(Q(s^\prime,a^\prime) - \log \pi^*(a^\prime\mid s^\prime)\right)\mid s, a\right]\text{ for all } a^\prime\in \mathcal{A} \notag
\end{align}
The last line comes from the fact that $Q(s^\prime,a^\prime) - \log \pi^*(a^\prime\mid s^\prime)$ is equivalent to $\log \left(\sum_{a^{\prime} \in \mathcal{A}} \exp \left(Q\left(s^{\prime}, a^{\prime}\right)\right)\right)$, which is a quantity that does not depend on the realization of specific action $a^\prime$.


\subsection{Single agent Dynamic Discrete Choice (DDC) model}\label{sec:SingleDDC}

\subsubsection*{Markov decision processes} 
Consider an MDP $\tau:=\left(\mathcal{S}, \mathcal{A}, P, \nu_0, r, G(\delta,1), \beta \right)$:
\begin{itemize}
    \item $\mathcal{S}$ and $\mathcal{A}$ denote finite state and action spaces
    \item $P \in \Delta_{\mathcal{S}}^{\mathcal{S} \times \mathcal{A}}$ is a Markovian transition kernel, and $\nu_0 \in \Delta_{\mathcal{S}}$ is the initial state distribution. 
    \item $r(s_h,a_h)+\epsilon_{ah}$ is the immediate reward (called the flow utility in the Discrete Choice Model literature) from taking action $a_h$ at state $s_h$ at time-step $h$, where:
    \begin{itemize}
        \item $r \in \mathbb{R}^{\mathcal{S} \times \mathcal{A}}$ is a deterministic reward function
        \item  
    $\epsilon_{ah}\overset{i.i.d.}{\sim}  G(\delta, 1)$ is the random part of the reward, where $G$ is Type 1 Extreme Value (T1EV) distribution (a.k.a. Gumbel distribution). The mean of $G(\delta, 1)$ is $\delta + \gamma$, where $\gamma$ is the Euler constant. 
    \item In the econometrics literature, this reward setting is considered as a result of a combination of two assumptions: conditional independence (CI) and additive separability (AS) \cite{magnac2002identifying}. 
\begin{figure}[H]
    \centering
    \includegraphics[width=0.3\linewidth]{Figures/Gumbel.png}
    \caption{Gumbel distribution $G(-\gamma, 1)$}
\end{figure}
    \end{itemize}
    \item $\beta \in(0,1)$ a discount factor
    
\end{itemize}
\;

\subsubsection{Agent behaviors} Denote the distribution of agent's initial state $s_0\in \mathcal{S}$ as $\nu_0$. Given a stationary Markov policy $\pi \in \Delta_{\mathcal{A}}^{\mathcal{S}}$, an agent starts from initial state $s_0$ and make an action $a_h\in \mathcal{A}$ at state $s_h\in \mathcal{S}$ according to $a_h\sim\pi\left(\cdot \mid s_h\right)$ at each period $h$. We use $\mathbb{P}_{\nu_0}^\pi$ to denote the distribution over the sample space $(\mathcal{S} \times \mathcal{A})^{\infty}=\left\{\left(s_0, a_0, s_1, a_1, \ldots\right): s_h \in \mathcal{S}, a_h \in \mathcal{A}, h \in \mathbb{N}\right\}$ induced by the policy $\pi$ and the initial distribution $\nu_0$. We also use $\mathbb{E}_\pi$ to denote the expectation with respect to $\mathbb{P}_{\nu_0}^\pi$. As in Inverse Reinforcement learning (IRL), a Dynamic Discrete Choice (DDC) model makes the following assumption: 

\begin{asmp}\label{ass:optimaldecision} Agent makes decision according to the policy $\operatorname{argmax}_{\pi \in \Delta_{\mathcal{A}}^{\mathcal{S}}}$
$\mathbb{E}_\pi\left[\sum_{h=0}^{\infty} \beta^h( r\left(s_h, a_h\right)+\epsilon_{ah})\right]$.
\end{asmp}

As Assumption \ref{ass:optimaldecision} specifies the agent's policy, we omit $\pi$ in the notations from now on. Define $\boldsymbol{\epsilon}_h = [\epsilon_{1h}\ldots \epsilon_{|\mathcal{A}|h}]$, where $\epsilon_{ih}\overset{i.i.d}{\sim} G(\delta, 1)$ for $i=1\ldots |\mathcal{A}|$. We define a function $V$ as
\begin{align}
    V\left(s_{h^\prime}, \boldsymbol{\epsilon_{h^\prime}}\right) = \mathbb{E}\left[\sum_{h=h^\prime}^{\infty} \beta^h( r\left(s_h, a_h\right)+\epsilon_{ah})\mid s_{h^\prime}\right] \notag
\end{align}
and call it the value function. According to Assumption \ref{ass:optimaldecision}, the value function $V$ must satisfy the Bellman equation, i.e., 
\begin{align}
V\left(s, \boldsymbol{\epsilon}\right)=\max _{a \in \mathcal{A}}\left\{r\left(s, a\right)+\epsilon_{a}+\beta \cdot \mathbb{E}_{s^\prime \sim P(s, a), \boldsymbol{\epsilon^\prime }\sim \boldsymbol{\epsilon}}\left[V\left(s^\prime, \boldsymbol{\epsilon}^\prime\right)\mid s, a\right]\right\} \label{eq:VBellman}.
\end{align}
\;
\\
Define 
\begin{align}
    \bar{V}\left(s\right) &\triangleq E_{\boldsymbol{\epsilon}}\left[V\left(s, \boldsymbol{\epsilon}\right)\right] \notag
    \\
    Q(s, a) &\triangleq r\left(s, a\right)+\beta \cdot \mathbb{E}_{s^\prime \sim P(s, a)}\left[\bar{V}\left(s^\prime\right)\mid s, a\right]\label{eq:QandexpV}
\end{align}
We call $\bar{V}$ the expected value function, and $Q(s, a)$ as the choice-specific value function. Then the Bellman equation can be written as

\;
\begin{align}
\bar{V}\left(s\right) &=\mathbb{E}_{
\boldsymbol{\epsilon}}\left[\max _{a \in \mathcal{A}}\left\{r\left(s, a\right)+\epsilon_{a}+\beta \cdot \mathbb{E}\left[\bar{V}\left(s^\prime\right)\mid s, a\right]\right\}\right]\label{eq:DP_DDC_pre} 
\\
&=\ln \left[\sum_{a\in \mathcal{A}}\exp\left(r\left(s, a\right)+\beta \cdot \mathbb{E}\left[\bar{V}\left(s^\prime\right)\mid s, a\right]\right)\right] + \delta + \gamma \tag{$\because$ Lemma \ref{lem:GumbelMax}}\label{eq:DP_DDC}
\\
&=\ln \left[\sum_{a\in \mathcal{A}}\exp\left(Q(s,a)\right)\right]  + \delta + \gamma \label{eq:logsumQ}
\end{align}
\\
\;
\\
Furthermore, Corollary \ref{cor:GumbelOptProb} characterizes that the agent's optimal policy is characterized by 
\begin{align}
    \pi^*(a \mid s) =\frac{\exp \left({Q(s, a)}\right)}{\sum_{a^\prime\in \mathcal{A}} \exp \left({Q(s, a^\prime)}\right)} \text{ for } a\in \mathcal{A} \label{eq:DDCopt}
\end{align}
\;
\\
In addition to Bellman equation in terms of value function $V$ in equation \ref{eq:VBellman}, 
Bellman equation in terms of choice-specific value function $Q$ comes from combining equation \ref{eq:QandexpV} and equation \ref{eq:logsumQ}:
\begin{align}
    Q(s, a)=r(s, a)+\beta \mathbb{E}_{s^\prime \sim P(s, a)}\left[\ln \left(\sum_{a^{\prime} \in \mathcal{A}} \exp \left(Q\left(s^{\prime}, a^{\prime}\right)\right)\right)\mid s, a\right] + \delta + \gamma
\end{align}
\;
\\
When $\delta = -\gamma$ (i.e., the Gumbel noise is mean 0), we have 
\begin{align}
    Q(s, a)=r(s, a)+\beta \mathbb{E}_{s^\prime \sim P(s, a)}\left[\ln \left(\sum_{a^{\prime} \in \mathcal{A}} \exp \left(Q\left(s^{\prime}, a^{\prime}\right)\right)\right)\mid s, a\right] \tag{ \ref{eq:QBellmanDDC}}
\end{align}
\;
\\
This Bellman equation can be also written in another form.

\begin{align}
  Q(s, a) &\triangleq r\left(s, a\right)+\beta \cdot \mathbb{E}_{s^\prime \sim P(s, a)}\left[\bar{V}\left(s^\prime\right)\mid s, a\right]\tag{Equation \ref{eq:QandexpV}}
  \\
  &= r\left(s, a\right)+\beta \cdot \mathbb{E}_{s^\prime \sim P(s, a), \boldsymbol{\epsilon}^\prime\sim \boldsymbol{\epsilon} }\left[{V}\left(s^\prime, \boldsymbol{\epsilon}^\prime\right)\mid s, a\right]\notag
  \\
  &=r\left(s, a\right)+\beta \cdot \mathbb{E}_{s^\prime \sim P(s, a),  \boldsymbol{\epsilon}^\prime\sim \boldsymbol{\epsilon}}\left[\max_{a^\prime\in \mathcal{A}} \left(Q\left(s^\prime, a^\prime\right)+\epsilon^\prime_a\right)\mid s, a\right] \label{eq:AnotherQBellman}
  \\
  &=r\left(s, a\right)+\beta \cdot \mathbb{E}_{s^\prime \sim P(s, a), a^\prime \sim \pi^*(\cdot\mid s^\prime)}\left[Q(s^\prime, a^\prime) + \delta + \gamma - \log  \pi^*(a^\prime \mid s^\prime) \mid s, a\right] \makebox[2em][l]{\quad(Corollary \ref{cor:GumbelMaxasProb})} \notag
  \\\label{DDCBellman_new}
\end{align}
where $\pi^*(s, a) = \left(\frac{Q(s, a)}{\sum_{a^{\prime}\in \mathcal{A}}Q(s, a^{\prime})}\right)$.




\subsection{Equivalence between DDC and Entropy regularized IRL}\label{sec:DDCIRLequiv}

Equation \ref{eq:logsumexp_reg}, equation \ref{eq:IRLopt} and equation \ref{eqn:IRLQBellman_new} characterizes the choice-specific value function's Bellman equation and optimal policy in entropy regularized IRL setting when regularizing coefficient is 1: 
$$
    Q(s, a)=r(s, a)+\beta \mathbb{E}_{s^\prime \sim P(s, a)}\left[\ln \left(\sum_{a^{\prime} \in \mathcal{A}} \exp \left(Q\left(s^{\prime},a^{\prime}\right)\right)\right)\mid s, a\right]
$$

$$
\pi^*(a \mid s) =\frac{\exp \left({Q(s, a)}\right)}{\sum_{a^\prime\in \mathcal{A}} \exp \left({Q(s, a^\prime)}\right)} \text{ for } a\in \mathcal{A} 
$$

$$
Q(s,a)=r\left(s, a\right)+\beta \cdot \mathbb{E}_{s^\prime \sim P(s, a), a^\prime \sim \pi^*(\cdot \mid s^\prime)}\left[Q(s^\prime, a^\prime) - \log  \pi^*(a^\prime\mid s^\prime) \mid s, a\right]
$$
\\
Equation \ref{eq:DDCopt}, equation \ref{eq:QBellmanDDC}, and equation \ref{DDCBellman_new} (when $\delta = -\gamma$) characterizes the choice-specific value function's Bellman equation and optimal policy of Dynamic Discrete Choice setting:
$$
    Q(s, a)=r(s, a)+\beta \mathbb{E}_{s^\prime \sim P(s, a)}\left[\ln \left(\sum_{a^{\prime} \in \mathcal{A}} \exp \left(Q\left(s^{\prime}, a^{\prime}\right)\right)\right)\mid s, a\right] 
$$

$$
\pi^*(a \mid s) =\frac{\exp \left({Q(s, a)}\right)}{\sum_{a^\prime\in \mathcal{A}} \exp \left({Q(s, a^\prime)}\right)} \text{ for } a\in \mathcal{A} 
$$

$$
Q(s,a) = r\left(s, a\right)+\beta \cdot \mathbb{E}_{s^\prime \sim P(s, a), a^\prime \sim \pi^*(\cdot\mid s^\prime)}\left[Q(s^\prime, a^\prime) - \log  \pi^*(a^\prime \mid s^\prime) \mid s, a\right]
$$
\\
$Q$ that satisfies \ref{eq:DDCopt} is unique \cite{rust1994structural}, and $Q-r$ forms a one-to-one relationship. Therefore, the exact equivalence between these two setups implies that the same reward function $r$ and discount factor $\beta$ will lead to the same choice-specific value function $Q$ and the same optimal policy for the two problems.


\section{IRL with occupancy matching}\label{sec:occupancy}

\cite{ho2016generative} defines another inverse reinforcement learning problem that is based on the notion of occupancy matching. Let $\nu_0$ be the initial state distribution and $d^\pi$ be the discounted state-action occupancy of $\pi$ which is defined as $ d^\pi=(1-$ $\beta) \sum_{t=0}^{\infty} \beta^t d_t^\pi$, with $d_t^\pi(s, a)=\mathbb{P}_{\pi, \nu_0}\left[s_t=s, a_t=a\right]$. Note that $Q^\pi(s, a):=\mathbb{E}_\pi\left[\sum_{t=0}^{\infty} \beta^t r(s_t,a_t) \mid s_0=s, a_0=a\right] = \sum_{t=0}^{\infty} \beta^t \mathbb{E}_{(\tilde{s},\tilde{a})\sim d_t^\pi}[r(\tilde{s},\tilde{a})\mid s_0=a,a_0=a].$ Defining the discounted state-action occupancy of the expert policy $\pi^\ast$ as $d^\ast$, \cite{ho2016generative} defines the inverse reinforcement learning problem as the following max-min problem:
\begin{align}
    \underset{r \in \mathcal{C}}{\operatorname{max}}\underset{{\pi \in \Pi}}{\min}\left(\mathbb{E}_{d^\ast}[r(s, a)]-\mathbb{E}_{d^\pi}[r(s, a)]-\mathcal{H}(\pi)-\psi(r)\right) \label{eq:occupancyObj}
\end{align}
where $\mathcal{H}$ is the Shannon entropy we used in MaxEnt-IRL formulation and $\psi$ is the regularizer imposed on the reward model $r$.  
\;
\\
\;
\\
Would occupancy matching find $Q$ that satisfies the Bellman equation? Denote the policy as $\pi^\ast$ and its corresponding discounted state-action occupancy measure as $d^\ast=(1-$ $\beta) \sum_{t=0}^{\infty} \beta^t d_t^\ast$, with $d_t^\ast(s, a)=\mathbb{P}_{\pi^\ast, \nu_0}\left[s_t=s, a_t=a\right]$. We define the expert's action-value function as $Q^\ast(s, a):=\mathbb{E}_{\pi^\ast}\left[\sum_{t=0}^{\infty} \beta^t r(s_t,a_t) \mid s_0=s, a_0=a\right]$ and the Bellman operator of $\pi^\ast$ as $\mathcal{T}^\ast$. Then we have the following Lemma \ref{cor:occp=naiveBE} showing that occupancy matching (even without regularization) may not minimize Bellman error for every state and action.
\begin{lem}[Occupancy matching is equivalent to naive weighted Bellman error sum]\label{cor:occp=naiveBE} The perfect occupancy matching given the same $(s_0, a_0)$ satisfies
    $$\mathbb{E}_{(s, a) \sim d^\ast}[r(s, a)\mid s_0, a_0]-\mathbb{E}_{(s, a) \sim d^\pi}[r(s, a)\mid s_0, a_0] = \mathbb{E}_{(s,a)\sim d^{\ast}}[(\mathcal{T}^\ast Q^\pi-Q^\pi)(s,a)\mid s_0, a_0]$$
\end{lem}
\begin{proof} 
   Note that $\mathbb{E}_{(s, a) \sim d^{\ast}}[r(s, a)\mid s_0, a_0]=\sum_{t=0}^{\infty} \beta^t \mathbb{E}_{(s,a)\sim d_t^\ast}[r(s,a)\mid s_0,a_0]=Q^\ast(s, a)$ and $\mathbb{E}_{(s, a) \sim d^{\pi}}[r(s_, a)\mid s_0, a_0]=\sum_{t=0}^{\infty} \beta^t \mathbb{E}_{(s,a)\sim d_t^\pi}[r(s,a)\mid s_0,a_0] = Q^\pi(s, a)$. Therefore
   \begin{align}
       &\mathbb{E}_{(s, a) \sim d^\ast}[r(s, a)\mid s_0,a_0]-\mathbb{E}_{(s, a) \sim d^\pi}[r(s, a)\mid s_0,a_0] =(1-\beta)Q^\ast(s_0, a_0)-(1-\beta)Q^\pi(s_0, a_0)\notag
       \\
       &=(1-\beta)\frac{1}{1-\beta} \mathbb{E}_{(s,a)\sim d^{\ast}}[(\mathcal{T}^\ast Q^\pi-Q^\pi)(s,a)\mid s_0, a_0]\tag{Lemma \ref{lem:telescoping}}
       \\
       &= \mathbb{E}_{(s,a)\sim d^{\ast}}[(\mathcal{T}^\ast Q^\pi-Q^\pi)(s,a)\mid s_0, a_0] \notag
   \end{align}
\end{proof}
\;
\\
Lemma \ref{cor:occp=naiveBE} implies that occupancy measure matching, even without reward regularization, does not necessarily imply Bellman errors being 0 for every state and action. In fact, what they minimize is the \textit{average Bellman error} \cite{jiang2017contextual,uehara2020minimax}. This implies that $r$ cannot be inferred from $Q$ using the Bellman equation after deriving $Q$ using occupancy matching. 



\begin{lem}[Bellman Error Telescoping]\label{lem:telescoping} 
Let the Bellman operator $\mathcal{T}^\pi$ is defined to map $f\in \mathbb{R}^{S\times A}$ to $\mathcal{T}^\pi f := r(s,a) + \mathbb{E}_{s^\prime\sim P(s,a), a^\prime\sim \pi(\cdot\mid s^\prime)}[f(s^\prime,a^\prime)\mid s,a]$. 
For any $\pi$, and any $f\in \mathbb{R}^{S\times A}$,
$$
Q^\pi(s_0, a_0)-f(s_0, a_0) = \frac{1}{1-\beta} \mathbb{E}_{(s,a)\sim d^\pi}[(\mathcal{T}^\pi f-f)(s,a)\mid s_0, a_0].
$$

\end{lem}

\begin{proof}
Note that the right-hand side of the statement can be expanded as
\begin{align}
    &r(s_0, a_0)+\beta \cancel{\mathbb{E}_{s^\prime\sim P(s,a), a^\prime\sim \pi(\cdot\mid s^\prime)}[f(s^\prime,a^\prime)\mid s,a]}-f(s_0,a_0)\notag
    \\
    &+\beta\mathbb{E}_{(s,a)\sim d_1^\pi}\left[r(s, a)+\beta \cancel{\mathbb{E}_{s^\prime\sim P(s,a), a^\prime\sim \pi(\cdot\mid s^\prime)}[f(s^\prime,a^\prime)\mid s,a]}-\cancel{f(s,a)}\mid s_0, a_0\right]\notag
    \\
    &+\beta^2\mathbb{E}_{(s,a)\sim d_2^\pi}\left[r(s, a)+\beta \cancel{\mathbb{E}_{s^\prime\sim P(s,a), a^\prime\sim \pi(\cdot\mid s^\prime)}[f(s^\prime,a^\prime)\mid s,a]}-\cancel{f(s,a)}\mid s_0, a_0\right]\notag
    \\&\quad\quad\quad\quad\quad\quad\quad\quad\quad\quad\quad\quad\ldots\notag
    \\
    &=Q^\pi(s_0, a_0)-f(s_0, a_0)\notag
\end{align}
which is the left-hand side of the statement.

\end{proof}

\iffalse

\begin{lem}[Equivalence between Occupancy matching and Behavioral Cloning]
The solution set to the Occupancy matching objective (equation \ref{eq:occupancyObj}) without regularization terms is equivalent to the solution set to the behavioral cloning objective (equation \ref{eq:BC}).
\end{lem}

\begin{proof}
    In proving Lemma \ref{cor:occp=naiveBE}, we saw that $\mathbb{E}_{(s, a) \sim d^\ast}[r(s, a)\mid s_0,a_0]-\mathbb{E}_{(s, a) \sim d^\pi}[r(s, a)\mid s_0,a_0] =(1-\beta)Q^\ast(s_0, a_0)-(1-\beta)Q^\pi(s_0, a_0)$. 

    Now from Lemma \ref{lem:minMLE}, we have 
      \begin{align}
           \underset{Q\in \mathcal{Q}}{\arg\max } &\; \;\mathbb{E}_{(s, a)\sim \pi^*, \nu_0}  \left[\log\left(\hat{p}_{Q}(\;\cdot
    \mid s)\right)\right] \notag
    \\
     &=\left\{Q \in \mathcal{Q} \mid\hat{p}_{Q}(\;\cdot
    \mid s) = \pi^*(\;\cdot
    \mid s)\quad  \forall s\in\bar{\mathcal{S}}\quad\text{a.e.}\right\}\notag
    \\
     &=\left\{Q \in \mathcal{Q} \mid Q(s,a_1)-Q(s,a_2)= Q^*(s,a_1)-Q^*(s,a_2) \quad \forall a_1, a_2\in\mathcal{A}, s\in\bar{\mathcal{S}}\right\} \notag
    \end{align}
This concludes that the solution set to the behavioral cloning objective is equivalent to the occupancy matching objective without the regularization term.
\end{proof}

\fi


%\section*{Acknowledgments}


\end{document}