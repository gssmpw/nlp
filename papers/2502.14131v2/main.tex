\documentclass[11pt,letterpaper,english]{article}

%%%%%%%%%%%%%%%%%%%%%%%%%%%%%%%%%%%%%%%%%%%%%%%%%%%%%%%%%%%%%%%%%%%%%%%%%%
%% Packages from both documents
%%%%%%%%%%%%%%%%%%%%%%%%%%%%%%%%%%%%%%%%%%%%%%%%%%%%%%%%%%%%%%%%%%%%%%%%%%

% Essential packages
\usepackage{amsmath}
\usepackage{amsthm}
\usepackage{amssymb}
\usepackage{graphicx}
\usepackage{hyperref}
\usepackage{natbib}
\usepackage{times}
\usepackage{setspace}
\usepackage{booktabs}
\usepackage{algorithm}
\usepackage{algpseudocode}
\usepackage{enumitem}
\usepackage{microtype}
\usepackage{tikz}
\usepackage{pgfplots}
\pgfplotsset{compat=1.17}
\usepackage{multirow}
\usepackage{soul}
\usepackage{caption}
\usepackage{subcaption}
\usepackage{float}
\usepackage{rotating}
\usepackage{array}
\usepackage{bm}
\usepackage{bbm}
\PassOptionsToPackage{table}{xcolor}
\usepackage{xcolor}
\usepackage{colortbl} 

\usepackage{pifont}
\usepackage[bottom]{footmisc}
\usepackage{wrapfig}
\usepackage[makeroom]{cancel}
\usepackage[edges]{forest}
\usepackage{makecell}
\usepackage{longtable}
\usepackage{dsfont}
\usepackage{comment}
\usepackage{blkarray}
\usepackage{stackengine}
\usepackage{accents}
\usepackage[utf8]{inputenc}
\usepackage{epstopdf}
\usepackage{xfrac}
\usepackage{fullpage}

% TikZ libraries
\usetikzlibrary{arrows}
\usetikzlibrary{positioning}

% Configure bibliography
\bibpunct[, ]{(}{)}{,}{a}{}{,}
\def\bibfont{\small}
\def\bibsep{\smallskipamount}
\def\bibhang{24pt}
\def\newblock{\ }
\def\BIBand{and}

% Line spacing
\onehalfspacing

% Math commands from first document
\newcommand{\cmark}{\ding{51}} % Checkmark
\newcommand{\xmark}{\ding{55}} % Crossmark
\newcommand{\lmark}{$\bigtriangleup$} 
\newcommand{\lbe}{\mathcal{L}_{BE}}

\newcommand*{\QED}[1][$\square$]{%
\leavevmode\unskip\penalty9999 \hbox{}\nobreak\hfill
    \quad\hbox{#1}%
}

% Math commands from second document
\newcommand{\todo}[1]{\noindent{\textcolor{red}{\{TODO:  #1\}}}}
\newcommand{\norm}[1]{\left\lVert#1\right\rVert}
\newcommand{\indep}{\perp \!\!\! \perp}
\newcommand{\argmax}{\arg\!\max}
\DeclareMathOperator*{\argmin}{arg\,min}
\def\ss#1{{\bf \color{red}#1}}
\def\thetaa{\theta_\alpha}
\def\bP{\mathbb{P}}
\def\bE{\mathbb{E}}
\def\bW{\mathbf{W}}
\def\bC{\mathbf{C}}
\def\bb{\mathbf{b}}
\def\bm{\textbf{m}}
\def\bx{\textbf{x}}
\def\bg{\textbf{g}}
\def\bd{\textbf{d}}
\def\bY{\textbf{Y}}
\def\cA{\mathcal{A}}
\def\bZ{\mathbf{Z}}
\def\bzero{\textbf{0}}
\def\tm{\text{-}}
\def\bTheta{\boldsymbol{\Theta}}
\def\thetab{\theta_\beta}
\def\btau{\boldsymbol{\tau}}
\def\sm#1{{\bf \color{red}#1}}
\def\hy#1{{\bf \color{blue}#1}}
\def\cents{\hbox{\rm\rlap/c}}
\def\aps{\textsc{\char13}}

% Custom list environments
\newcommand{\squishlist}{
   \begin{list}{$\bullet$}
    { \setlength{\itemsep}{0pt} \setlength{\parsep}{1pt}
      \setlength{\topsep}{1pt} \setlength{\partopsep}{1pt}
      \setlength{\leftmargin}{1.5em} \setlength{\labelwidth}{1em}
      \setlength{\labelsep}{0.5em} } }

\newcommand{\squishlisttwo}{
   \begin{list}{$\bullet$}
    { \setlength{\itemsep}{0pt} \setlength{\parsep}{0pt}
      \setlength{\topsep}{0pt} \setlength{\partopsep}{0pt}
      \setlength{\leftmargin}{1em} \setlength{\labelwidth}{1.5em}
      \setlength{\labelsep}{0.5em} } }

\newcommand{\squishend}{
    \end{list}  }

% Theorem environments from both documents
\newtheorem{thm}{Theorem}[section]
\newtheorem{lem}[thm]{Lemma}
\newtheorem{cor}[thm]{Corollary}
\newtheorem{prop}{Proposition}[section]
\newtheorem{asmp}{Assumption}[section]
\newtheorem{defn}{Definition}[section]
\newtheorem{fact}{Fact}[section]
\newtheorem{conj}{Conjecture}[section]
\newtheorem{rem}{Remark}[section]
\newtheorem{challenge}{Challenge}
\newtheorem{remark}{Remark}
\newtheorem{assumption}{Assumption}
\newtheorem{claim}{Claim}

% For DVI -> PNG conversion (from second document)
\DeclareGraphicsRule{.tif}{png}{.png}{`convert #1 `dirname #1`/`basename #1 .tif`.png}

% Math alphabet declaration from first document
\DeclareMathAlphabet{\mathcalligra}{T1}{calligra}{m}{n}

% Appendix setup
\usepackage{appendix}

% Title and author information
\title{Gradients can train reward models:
\\
An Empirical Risk Minimization Approach for \\Offline Inverse RL and Dynamic Discrete Choice Model}

\author{
Enoch H. Kang\\
Foster School of Business, University of Washington\\
\texttt{ehwkang@uw.edu}
\and
Hema Yoganarasimhan\\
Foster School of Business, University of Washington\\
\texttt{hemay@uw.edu}
\and
Lalit Jain\\
Foster School of Business, University of Washington\\
\texttt{lalitj@uw.edu}
}

\date{\today}

\begin{document}

\maketitle

\begin{abstract}
We study the problem of estimating Dynamic Discrete Choice (DDC) models, also known as offline Maximum Entropy-Regularized Inverse Reinforcement Learning (offline MaxEnt-IRL) in machine learning. The objective is to recover reward or $Q$ functions that govern agent behavior from offline behavior data. In this paper, we propose a globally convergent gradient-based method for solving these problems without the restrictive assumption of linearly parameterized rewards. The novelty of our approach lies in introducing the Empirical Risk Minimization (ERM) based IRL/DDC framework, which circumvents the need for explicit state transition probability estimation in the Bellman equation. Furthermore, our method is compatible with non-parametric estimation techniques such as neural networks. Therefore, the proposed method has the potential to be scaled to high-dimensional, infinite state spaces. A key theoretical insight underlying our approach is that the Bellman residual satisfies the Polyak-Łojasiewicz (PL) condition--a property that, while weaker than strong convexity, is sufficient to ensure fast global convergence guarantees. Through a series of synthetic experiments, we demonstrate that our approach consistently outperforms benchmark methods and state-of-the-art alternatives.
\end{abstract}

\textbf{Keywords:} Dynamic Discrete Choice, Offline Inverse Reinforcement Learning, Gradient-based methods, Empirical Risk Minimization, Machine learning


% Main content sections
%\section{Introduction}
\section{Introduction}
\label{sec:intro}

\begin{figure*}[tb]
    \centering
    \includegraphics[width=0.848\linewidth]{figs/circuitnn.pdf} 
    \caption{Illustration of differentiable CircuitNN. CircuitNN is designed based on differentiable NAND gates. After DAS is guided by PI and PO pairs of the truth table, CircuitNN can get the precise circuit architecture logic equivalent to the truth table.}
    \label{fig:circuitnn}
\end{figure*}

% 1. Describe the importance of logic synthesis
% 2. Existing Problems
% (a) Neural Architecture Search: Unstable, Predefined Setting, etc.
% (b) Circuit Generation: Probabilistic Model, Logic Equivalence

With the rapid advancement of technology, the scale of integrated circuits (ICs) has expanded exponentially. 
This expansion has introduced significant challenges in chip manufacturing, particularly concerning power and area metrics.
A primary objective in IC design is achieving the same circuit function with fewer transistors, thereby reducing power usage and area occupancy.

Logic synthesis~\cite{hachtel2005logicsynth}, a critical step in electronic design automation (EDA), transforms behavioral-level circuit designs into optimized gate-level circuits, ultimately yielding the final IC layout. 
The primary goal of logic synthesis is to identify the physical implementation with the fewest gates for a given circuit function. 
This task constitutes a challenging NP-hard combinatorial optimization problem. 
Current logic synthesis tools~\cite{brayton2010abc, wolf2013yosys} rely on human-designed heuristics, often leading to sub-optimal outcomes.

Differentiable architecture search (DAS) techniques~\cite{liu2018darts, chu2020darts} offer novel perspectives on addressing challenges in this problem.
Circuit functions can be represented through truth tables, which map binary inputs to their corresponding outputs. 
Truth tables provide a precise representation of input-output relationships, ensuring the design of functionally equivalent circuits.
Inspired by this, researchers~\cite{deepmind2024ai4sys, wang2024tnet} have begun exploring the application of DAS to synthesize circuits directly from truth tables.
Specifically, \citet{deepmind2024ai4sys} proposed CircuitNN, a framework that learns differentiable connection structures with logic gates, enabling the automatic generation of logic circuits from truth tables.
This approach significantly reduces the complexity of traditional circuit generation. 
Building on this, \citet{wang2024tnet} introduced T-Net, a triangle-shaped variant of CircuitNN, incorporating regularization techniques to enhance the efficiency of DAS.

Despite these advancements, several challenges remain. 
The computational complexity of DAS grows quadratically with the number of gates, posing scalability issues.
Although triangle-shaped architecture~\cite{wang2024tnet} partially mitigates this problem, redundancy persists. 
%Additionally, DAS is susceptible to converging to local optima, limiting the ability to search architectures that satisfy the given truth tables~\cite{liu2018darts}. 
%Furthermore, hyperparameters (network depth and layer width) require extensive searches, introducing complexity and prolonging the synthesis process. 
Additionally, DAS is susceptible to converging to local optima~\cite{liu2018darts} and hyperparameters (network depth and layer width) require extensive searches. 
The challenges arise from the vast search space in DAS. 
% Even with predefined settings for CircuitNN, finding a configuration that meets the truth table requires extensive trial and error during the DAS process. 
Intuitively, limiting the search space through predefined parameters (network depth, gates per layer, and connection probabilities) can significantly reduce the complexity.

Recent advances~\cite{openai2023gpt4, abramson2024alphafold3, esser2024sd3, li2024mar} in conditional generative models have demonstrated remarkable performance across language, vision, and graph generation tasks. 
Motivated by these developments, we propose a novel approach to circuit generation that generates preliminary circuit structures to guide DAS in generating refined circuits matching specified truth tables. 
Firstly, we introduce CircuitVQ, a tokenizer with a discrete codebook for circuit tokenization. 
Built upon our Circuit AutoEncoder framework~\cite{hou2022graphmae,li2023maskgae,wu2025mgvga}, CircuitVQ is trained through a circuit reconstruction task. 
Specifically, the CircuitVQ encoder encodes input circuits into discrete tokens using a learnable codebook, while the decoder reconstructs the circuit adjacency matrix based on these tokens.
Subsequently, the CircuitVQ encoder serves as a circuit tokenizer for CircuitAR pretraining, which employs a masked autoregressive modeling paradigm~\cite{chang2022maskgit, li2023mage}. 
In this process, the discrete codes function as supervision signals. 
After training, CircuitAR can generate discrete tokens progressively, which can be decoded into initial circuit structures by the decoder of the CircuitVQ. 
These prior insights can guide DAS in producing refined circuits that match the target truth tables precisely.

Our key contributions can be summarized as follows:
\begin{itemize}
\item We introduce CircuitVQ, a circuit tokenizer that facilitates graph autoregressive modeling for circuit generation, based on our Circuit AutoEncoder framework;
\item Develop CircuitAR, a model trained using masked autoregressive modeling, which generates initial circuit structures conditioned on given truth tables;
\item Propose a refinement framework that integrates differentiable architecture search to produce functionally equivalent circuits guided by target truth tables;
\item Comprehensive experiments demonstrating the scalability and capability emergence of our CircuitAR and the superior performance of the proposed circuit generation approach.
\end{itemize}

% Motivation
% (a) Diffusion (Vision, Graph), Autoregressive (Language, Vision)
% (b) Circuit Generation for Predefined Setting
% (c) Neural Architecture Search for Strict Logic Equivalence

% Contribution
% (a) Circuit Tokenizer (new transformer arch, training strategy)
% (b) CircuitAR (train and gen strategies, post-ar strategy)
% (c) Extensive Evaluation including BitD (Bit Distance) for Scalability




\section{Related Work} \label{sec:related}

% \textbf{Adversarial Attack}
\textbf{Attacks on SLAM.} 
%With the rise of machine learning, 
The robustness of computer vision systems is being actively investigated. With the emergence of adversarial images in the digital domain by adding optimized noise directly to images~\cite{szegedy2013intriguing,carlini2017towards}, researchers find that such attacks also exist physically in the real world \cite{eykholt2018robust,song2018physical,zhao2019seeing}. To fill the gap between attacks in the digital and physical worlds, recent studies have demonstrated that attacks on real-world computer vision systems are practical \cite{eykholt2018robust,li2019adversarial,man2020ghostimage,sharif2016accessorize,zhao2019seeing,zhou2018invisible}. However, attacks on traditional computer vision methods such as SLAM are relatively less explored. \cite{yoshida2022adversarial} proposes an attack against the scan matching algorithm in LiDAR-based SLAM, while most SLAMs in AR/VR devices rely on different sensors like RGB/depth cameras and IMUs. \cite{ikram2022perceptual} and \cite{chen2024adversary} mislead visual SLAM by poisoning the images with special patterns, and \cite{wang2021can} causes the camera to fail using infrared light. In our work, we demonstrate attacks on Visual-Inertial SLAM (VI-SLAM) by perturbing the IMU readings, rather than cameras, and showing its impact on XR user experience. 

\textbf{Acoustic Injection Attacks.} Among various physical attacks, acoustic injection attacks are attractive due to their low cost. Son~\etal~\cite{son2015rocking} were the first to introduce acoustic attacks on MEMS gyroscopes, demonstrating how these attacks could lead to sensor denial-of-service and result in drone crashes. WALNUT~\cite{trippel2017walnut} expanded on this by developing output biasing and control attacks that enable precise manipulation of MEMS accelerometer outputs using modulated sound waves. Wang et al.~\cite{wang2017sonic} demonstrated a sonic gun, showcasing the vulnerability of various smart devices (\eg drones and self-balancing vehicles) to acoustic attacks. Tu et al. \cite{tu2018injected} designed side-swing and switching attacks to alter the outputs of MEMS gyroscopes and accelerometers. Furthermore, Ji et al. \cite{ji2021poltergeist} fool the object detectors by applying acoustic attack to the image stabilizers commonly used in modern cameras. However, none of the existing works study the relationship between the acoustic injections and SLAM outputs on recent XR devices. 

% \zijian{Do we need one session about security in AR/VR?}
% \yicheng{TODO}
%\jiasi{cite the AIVR paper (UMass Amherst?) paper is we have not already. They add IMU perturbation but w/o SLAM, iirc} \yicheng{Cited}

\textbf{XR Security and Privacy.} 
%Security and privacy concerns in XR systems have gained significant attention. 
For single-user XR systems, researchers have demonstrated various side-channel attacks to extract sensitive information (\eg keystrokes) through video feeds~\cite{ling2019know}, head movements~\cite{nair2023unique, slocum2023going}, architectural hints~\cite{zhang2023its,shang2020arspy}, power usage~\cite{li2024dangers}, and EM side-channel leakages~\cite{al2021vr}. In multi-user XR systems, Su et al.~\cite{su2024remote} use avatar motion data to infer keystrokes in shared VR environments. Slocum et al.~\cite{slocum2024doesn} reveal vulnerabilities in the shared state frameworks of multi-user AR. Similarly, Lebeck et al.~\cite{lebeck2017securing} highlight risks like deceptive virtual objects and emphasize access control for managing shared physical and virtual spaces. Ruth et al.~\cite{ruth2019secure} further propose a secure multi-user AR framework focusing on content sharing and permissions.
Chandio et al.~\cite{chandio2024stealthy} %introduced a multi-modal spatiotemporal attack that 
simultaneously manipulated visual and inertial sensors to disrupt XR pose estimation. However, their study evaluated the attack using offline datasets and assumed the attacker's capability to manipulate IMU data streams through acoustic means, without real experiments. Ours is the first to demonstrate acoustic injection attacks on recent XR devices, like the Hololens 2, in the real world.
 


\iffalse
\begin{table*}[htbp]
\tiny
\begin{center}
\begin{tabular}{lccccccccccccc}\toprule
Model, ft setting & mem & \#param & ARC-c & ARC-e & BoolQ & HS & OBQA & PIQA & rte & SIQA & WG & Avg
%\\\cmidrule(lr){2-3}\cmidrule(lr){4-5} \cmidrule(lr){6-7} \cmidrule(lr){8-9}\cmidrule(lr){10-11} \cmidrule(lr){12-13} \cmidrule(lr){14-15} \cmidrule(lr){16-17} 
\\\cmidrule(lr){1-13}
Llama2(7B), LoRA, $r=64$ & 23.46GB & 159.9M(2.37\%) & \textbf{44.97} & 77.02 & 77.43 & \textbf{57.75} & 32.0 & \textbf{78.45} & 62.09 & \textbf{47.75} & 68.75 & 60.69\\
Llama2(7B), SPruFT, $r=128$ & \textbf{17.62GB} & 145.8M(2.16\%) & 43.60 & \textbf{77.26} & \textbf{77.77} & 57.47 & \textbf{32.6} & 78.07 & \textbf{64.98} & 46.67 & \textbf{69.30} & \textbf{60.86} \\\cmidrule(lr){2-13}
Llama2(7B), FA-LoRA, $r=64$ & 17.25GB & 92.8M(1.38\%) & 43.77 & \textbf{77.57} & 77.74 & \textbf{57.45} & 31.0 & 77.86 & \textbf{66.06} & \textbf{47.13} & 69.06 & 60.85\\
Llama2(7B), FA-SPruFT, $r=128$ & \textbf{15.21GB} & 78.6M(1.17\%) & \textbf{43.94} & 77.22 & \textbf{77.83} & 57.11 & \textbf{32.0} & \textbf{78.18} & 65.70 & 46.47 & \textbf{69.38} & \textbf{60.87}\\\midrule
Llama3(8B), LoRA, $r=64$ & 30.37GB & 167.8M(2.09\%) & \textbf{53.07} & \textbf{81.40} & \textbf{82.32} & \textbf{60.67} & 34.2 & \textbf{79.98} & 69.68 & \textbf{48.52} & \textbf{73.56} & \textbf{64.82}\\
Llama3(8B), SPruFT, $r=128$ & \textbf{24.49GB} & 159.4M(1.98\%) & 52.47 & 81.10 & 81.28 & 60.29 & \textbf{34.6} & 79.76 & \textbf{70.04} & 47.75 & 73.24 & 64.50 \\\cmidrule(lr){2-13}
Llama3(8B), FA-LoRA, $r=64$ & 24.55GB & 113.2M(1.41\%) & \textbf{52.47} & \textbf{81.36} & \textbf{82.23} & 60.17 & \textbf{35.0} & \textbf{79.76} & \textbf{70.04} & \textbf{48.31} & \textbf{73.56} & \textbf{64.77}\\
Llama3(8B), FA-SPruFT, $r=128$ & \textbf{22.41GB} & 92.3M(1.15\%) & 52.22 & 81.19 & 81.35 & \textbf{60.20} & 34.2 & 79.71 & 69.31 & 47.13 & 73.01 & 64.26 \\\bottomrule
\end{tabular}
%\vspace{-0.2cm}
\caption{Fine-tuning Llama on Alpaca dataset for 5 epochs and evaluating on 9 tasks from EleutherAI LM Harness. "mem" represents the memory usage, with further details provided in Appendix~\ref{apdx:measure}. \#param is the number of trainable parameters, where the difference of \#param between the two approaches depends on the architecture of Llama, as some layers have $d_{in} \neq d_{out}$. Note that 10 million trainable parameters only account for less than 0.15GB of memory requirement. FA indicates that we freeze attention layers, but not including MLP layers followed by attention blocks. HS, OBQA, and WG represent HellaSwag, OpenBookQA, and WinoGrande datasets. More details of datasets can be found in Appendix~\ref{apdx:data}. The ablation study for different $r$ and the comparison with other LoRA variants can be found in Appendix~\ref{apdx:ablation}. All reported results are accuracies on the corresponding tasks. \textbf{Bold} indicates the best results of two approaches on the same task.} \label{tab:llm} 
\end{center}
\end{table*}
\fi

\begin{table*}[htbp]
\tiny
\begin{center}
\begin{tabular}{lccccccccccccc}\toprule
Model, ft setting & mem & \#param & ARC-c & ARC-e & BoolQ & HS & OBQA & PIQA & rte & SIQA & WG & Avg
\\\cmidrule(lr){1-13}
Llama2(7B)\\ \cmidrule(lr){1-1} 
LoRA, $r=64$ & 23.46GB & 159.9M(2.37\%) & \textbf{44.97} & 77.02 & 77.43 & 57.75 & 32.0 & \textbf{78.45} & 62.09 & 47.75 & 68.75 & 60.69\\
VeRA, $r=64$ & 22.97GB & 1.374M(0.02\%) & 43.26 & 76.43 & 77.40 & 57.26 & 31.6 & 78.02 & 62.09 & 45.85 & 68.75 & 60.07\\
DoRA, $r=64$ & 44.85GB & 161.3M(2.39\%) & 44.71 & 77.02 & 77.55 & \textbf{57.79} & 32.4 & 78.29 & 61.73 & \textbf{47.90} & 68.98 & 60.71\\
RoSA, $r=32, d=1.2\%$ & 44.69GB & 157.7M(2.34\%) & 43.86 & \textbf{77.48} & \textbf{77.86} & 57.42 & 32.2 & 77.97 & 63.90 &  47.29 & 69.06 & 60.78\\
SPruFT, $r=128$ & \textbf{17.62GB} & 145.8M(2.16\%) & 43.60 & 77.26 & 77.77 & 57.47 & \textbf{32.6} & 78.07 & \textbf{64.98} & 46.67 & \textbf{69.30} & \textbf{60.86} %\\\cmidrule(lr){2-13}
%FA-LoRA, $r=64$ & 17.25GB & 92.8M(1.38\%) & 43.77 & \textbf{77.57} & 77.74 & \textbf{57.45} & 31.0 & 77.86 & 66.06 & \textbf{47.13} & 69.06 & 60.85\\
%FA-DoRA, $r=64$ & 30.61GB & 93.6M(1.39\%) & 43.94 & 77.44 & 77.49 & 57.44 & 31.0 & 77.86 & \textbf{66.43} & 46.98 & 69.14 & 60.86\\
%FA-RoSA, $r=32, d=1.2\%$ & 38.34GB & 98.3M(1.46\%) & \textbf{44.28} & 77.02 & 77.68 & 57.22 & 31.0 & 77.97 & 64.26 & 46.32 & 69.22 & 60.55\\
%FA-SPruFT, $r=128$ & \textbf{15.21GB} & 78.6M(1.17\%) & 43.94 & 77.22 & \textbf{77.83} & 57.11 & \textbf{32.0} & \textbf{78.18} & 65.70 & 46.47 & \textbf{69.38} & \textbf{60.87}
\\\midrule
Llama3(8B)\\ \cmidrule(lr){1-1} 
LoRA, $r=64$ & 30.37GB & 167.8M(2.09\%) & 53.07 & 81.40 & 82.32 & 60.67 & 34.2 & 79.98 & 69.68 & 48.52 & 73.56 & 64.82\\
VeRA, $r=64$ & 29.49GB & 1.391M(0.02\%) & 50.26 & 80.30 & 81.41 & 60.16 & 34.4 & 79.60 & 69.31 & 46.93 & 72.77 & 63.90\\
DoRA, $r=64$ & 51.45GB & 169.1M(2.11\%) & \textbf{53.33} & \textbf{81.57} & \textbf{82.45} & \textbf{60.71} & 34.2 & \textbf{80.09} & 69.31 & \textbf{48.67} & \textbf{73.64} & \textbf{64.88}\\
RoSA, $r=32, d=1.2\%$ & 48.40GB & 167.6M(2.09\%) & 51.28 & 81.27 & 81.80 & 60.18 & 34.4 & 79.87 & 69.31 & 47.95 & 73.16 & 64.36\\
SPruFT, $r=128$ & \textbf{24.49GB} & 159.4M(1.98\%) & 52.47 & 81.10 & 81.28 & 60.29 & \textbf{34.6} & 79.76 & \textbf{70.04} & 47.75 & 73.24 & 64.50 %\\\cmidrule(lr){2-13}
%FA-LoRA, $r=64$ & 24.55GB & 113.2M(1.41\%) & 52.47 & 81.36 & 82.23 & 60.17 & \textbf{35.0} & 79.76 & 70.04 & 48.31 & \textbf{73.56} & 64.77\\
%FA-DoRA, $r=64$ & 40.62GB & 114.3M(1.42\%) & \textbf{52.56} & \textbf{81.69} & \textbf{82.26} & \textbf{60.20} & 34.4 & \textbf{79.82} & \textbf{70.40} & \textbf{48.46} & 73.40 & \textbf{64.80}\\
%FA-RoSA, $r=32, d=1.2\%$ & 42.31GB & 124.3M(1.55\%) & 52.22 & 81.19 & 82.05 & 60.11 & 34.4 & 79.76 & 69.31 & 47.70 & 73.16 & 64.43\\
%FA-SPruFT, $r=128$ & \textbf{22.41GB} & 92.3M(1.15\%) & 52.22 & 81.19 & 81.35 & \textbf{60.20} & 34.2 & 79.71 & 69.31 & 47.13 & 73.01 & 64.26 
\\\bottomrule
\end{tabular}
%\vspace{-0.2cm}
\caption{Fine-tuning Llama on Alpaca dataset for 5 epochs and evaluating on 9 tasks from EleutherAI LM Harness. ``mem" represents the memory usage, with further details provided in Appendix~\ref{apdx:measure}. \#param is the number of trainable parameters, where the difference of \#param between the two approaches depends on the architecture of Llama, as some layers have $d_{in} \neq d_{out}$. %FA indicates that we freeze attention layers, but not including MLP layers followed by attention blocks. 
HS, OBQA, and WG represent HellaSwag, OpenBookQA, and WinoGrande datasets. %More details of datasets can be found in Appendix~\ref{apdx:data}. 
The ablation study for different $r$ can be found in Appendix~\ref{apdx:ranks}. All reported results are accuracies on the corresponding tasks. \textbf{Bold} indicates the best result on the same task. } \label{tab:llm} 
\end{center}
\end{table*}

\section{Experimental Setup}\label{sec:setup}

%(0.5 page)
%Why the chosen framework?
%Some prior approaches

%- parameter settings
%- uniform across layers vs greedy ... 
%- potential transformer-specific details

%Equations about what these methods do.. 

%(0.5 page)
%Which NN architectures are used, why?
%Number of parameters, layers, ...

%(Potential prior work on compression -- )

\subsection{Datasets} \label{subsec:dataset}
We use multiple datasets for different tasks. For image classification, we fine-tune models on the training split and evaluate it on the validation split of Tiny-ImageNet~\citep{tavanaei2020embedded}, CIFAR100~\citep{alex2009learning}, and Caltech101~\citep{li_andreeto_ranzato_perona_2022}. For text generation, we fine-tune LLMs on 256 samples from Stanford-Alpaca~\citep{alpaca} and assess zero-shot performance on nine EleutherAI LM Harness tasks~\citep{gao2021framework}. See Appendix~\ref{apdx:data} for details.

\subsection{Models and Baselines} \label{subsec:models}

We fine-tune full-precision Llama-2-7B and Llama-3-8B (float32) using our SPruFT, LoRA~\citep{hulora}, VeRA~\citep{kopiczko2024vera}, DoRA~\citep{liu2024dora}, and RoSA~\citep{nikdan2024rosa}. RoSA is chosen as the representative SFT method and is the only SFT due to the high memory demands of other SFT approaches, while full fine-tuning is excluded for the same reason. We freeze Llama’s classification layers and fine-tune only the linear layers in attention and MLP blocks.

Next, we evaluate importance metrics by fine-tuning Llamas and image models, including DeiT~\citep{touvron2021training}, ViT~\citep{dosovitskiy2020image}, ResNet101~\citep{he2016deep}, and ResNeXt101~\citep{xie2017aggregated} on CIFAR100, Caltech101, and Tiny-ImageNet. For image tasks, we set the fine-tuning ratio at 5\%, meaning the trainable parameters are a total of 5\% of the backbone plus classification layers.

\subsection{Training Details} \label{subsec:training}
Our fine-tuning framework is built on torch-pruning\footnote{Torch-pruning is not required, all their implementations are based on PyTorch.}~\citep{fang2023depgraph}, PyTorch~\citep{paszke2019pytorch}, PyTorch-Image-Models~\citep{rw2019timm}, and HuggingFace Transformers~\citep{wolf2020transformers}. Most experiments run on a single A100-80GB GPU, while DoRA and RoSA use an H100-96GB GPU. We use the Adam optimizer~\citep{KingBa15} and fine-tune all models for a fixed number of epochs without validation-based model selection.

%Structured pruning often considers parameter dependencies in importance evaluation~\citep{liu2021group, fang2023depgraph, ma2023llmpruner}. This becomes the following process in our work: first, searching for dependencies by tracing the computation graph of gradient; next, evaluating the importance of parameter groups; and finally, fine-tuning the parameters within those important groups collectively. For instance, if $\W^{a}_{\cdot j}$ and $\W^{b}_{i\cdot}$ are dependent, where $\W^{a}_{\cdot j}$ is the $j$-th column in parameter matrix (or the $j$-th input channels/features) of layer $a$ and $\W^{b}_{i\cdot}$ is the $i$-th row in parameter matrix (or the $i$-th output channels/features) of layer $b$, then $\W^{a}_{\cdot j}$ and $\W^{b}_{i\cdot}$ will be fine-tuned simultaneously while the corresponding $\M^{a}_{dep}$ for $\W^{a}_{\cdot j}$ becomes column selection matrix and $\W^a_s$ becomes $\W^a_{f,dep}\M^a_{dep}$. Consequently, fine-tuning $2.5\%$ output channels for layer $b$ will result in fine-tuning additional $2.5\%$ input channels in each dependent layer. Therefore, for the $5\%$ of desired fine-tuning ratio, the fine-tuning ratio with considering dependencies is set to $2.5\%$\footnote{In some complex models, considering dependencies results in slightly more than twice the number of trainable parameters. However, in most cases, the factor is 2.} for the approach that includes dependencies. More details for dependencies of NN can be found in Appendix~\ref{apdx:dep}. 

\textbf{Image models}: The learning rate is set to $10^{-4}$ with cosine annealing decay~\citep{loshchilov2017sgdr}, where the minimum learning rate is $10^{-9}$. All image models used in this study are pre-trained on ImageNet. 

\textbf{Llama}: For LoRA and DoRA, we set $\alpha = 16$, a dropout rate of $0.1$, and a learning rate of $10^{-4}$  with linear decay (
$0.01$ decay rate). For SPruFT, we control trainable parameters using rank instead of fine-tuning ratio for direct comparison. The learning rate is $2 \cdot 10^{-5}$ with the same decay settings. Linear decay is applied after a warmup over the first $3$\% of training steps. The maximum sequence length is $2048$, with truncation for longer inputs and padding for shorter ones.


\section{ERM-IRL (ERM-DDC) framework}\label{sec:ERM-IRL}
\subsection{Identification via expected risk minimization}

%The goal of IRL (or DDC) is slightly different: we would like to find $\tilde{r} \in \mathcal{R} \subseteq \mathbb{R}^{\bar{\mathcal{S}} \times \mathcal{A}}$ such that
%\begin{align}
 %   \underset{\tilde{r}\in \mathcal{R}}{\operatorname{argmin}} \; \mathbb{E}_{(s,a)\sim \nu_0, \pi^*}[(\tilde{r}(s,a)-r(s,a))^2] \tag{Equation \ref{eq:rObjective}}
%\end{align}

%In Theorem \ref{thm:MagnacThesmar}, we saw that $Q^*$ and $r(s,a)$ can be uniquely identified by solving the system of equations shown in Equation \eqref{eq:HotzMillereqs} for all $s\in\mathcal{S}$ and $a\in\mathcal{A}$. 

We now propose a one-shot Empirical Risk Minimization framework (ERM-IRL/ERM-DDC) to solve the IRL problem stated in Definition \ref{def:IRLproblem}. First, we recast the IRL problem as the following \textit{expected risk} minimization problem under infinite data regime.

\begin{defn}[Expected risk minimization problem] The expected risk minimization problem is defined as the problem of finding $Q$ that minimizes  the expected risk $\mathcal{R}_{exp}(Q)$, which is defined as
\begin{align}
  &\mathcal{R}_{exp}(Q):= \mathbb{E}_{(s, a)\sim \pi^*, \nu_0}  \left[\mathcal{L}_{NLL}(Q)(s,a) + \lambda \mathbbm{1}_{a = a_s} \mathcal{L}_{BE}(Q)(s,a)\right] \! \notag \noindent
  \\
  & = \mathbb{E}_{(s, a)\sim \pi^*, \nu_0}\bigl[-\log\left(\hat{p}_{Q}(a
\mid s)\right) +  \lambda \mathbbm{1}_{a = a_s} \left( \mathcal{T}Q(s, a) - Q(s, a) \right)^2 \bigr] \label{eq:mainopt}
\end{align}
\noindent where $a_s$ is defined in Assumption \ref{ass:anchor}.
\end{defn}


\noindent\textbf{Remark.} The joint minimization of the NLL term and BE term is the key novelty in our approach. Prior work on the IRL and DDC literature \citep{hotz1993conditional, zeng2023understanding} typically minimizes the log-likelihood of the observed choice probabilities (the NLL term), given observed or estimated state transition probabilities. The standard solution is to first estimate/assume state transition probabilities, then obtain estimates of future value functions, plug them into the choice probability, and then minimize NLL term. In contrast, our recast problem avoids the estimation of state-transition probabilities and instead jointly minimizes the NLL term along with the Bellman-error term. This is particularly helpful in large-state spaces since the estimation of state-transition probabilities can be infeasible/costly in such settings. In Theorem \ref{thm:mainopt}, we show that the solution to our recast problem in Equation \eqref{eq:mainopt} identifies the reward function. 

\begin{thm}[Identification through expected risk minimization]
\label{thm:mainopt} 
\;
\\
The solution to the expected risk minimization problem (Equation \eqref{eq:mainopt}) with any $\lambda>0$
uniquely identifies $Q^\ast$ up to $s\in\bar{\mathcal{S}}
$ and $a \in \mathcal{A}$, i.e., finds $\widehat{Q}$ that satisfies $\widehat{Q}(s,a)=Q^\ast(s,a)$ for  $s\in\bar{\mathcal{S}}
$ and $a \in \mathcal{A}$. Furthermore, we can uniquely identify $r$ up to $s\in\bar{\mathcal{S}}$ and $a \in \mathcal{A}$ by $r(s, a)= \widehat{Q}(s, a) -  \beta \cdot \mathbb{E}_{s^{\prime} \sim P(s, a)} 
    \bigl[ V_{\widehat{Q}} \bigr]$.
\end{thm}
\noindent Essentially, Theorem \ref{thm:mainopt} ensures that solving Equation \eqref{eq:mainopt} gives the exact $r$ and $Q^\ast$ up to 
$\bar{\mathcal{S}}$ and thus provides the solution to the IRL problem defined in Definition \ref{def:IRLproblem}. See Appendix \ref{sec:pfOfmainOpt} for the proof. 

\subsubsection*{Comparison with Imitation Learning} 
Having established the identification guarantees for the ERM-IRL/DDC framework, it is natural to compare this formulation to the identification properties of Imitation Learning (IL). Unlike IRL, which seeks to infer the underlying reward function that explains expert behavior, IL directly aims to recover the expert policy without modeling the transition dynamics. The objective of imitation learning is often defined to as finding policy $\hat{p}$ with 
$$
\min _{\hat{p}} \mathbb{E}_{(s, a) \sim \pi^\ast, \nu_0}\left[\ell\left(\hat{p}(a \mid s), \pi^\ast(a \mid s)\right)\right], \text{$\ell$ is the cross-entropy loss}
$$
or equivalently, 
\begin{align}
\min _{\hat{p}} \mathbb{E}_{(s, a) \sim \pi^\ast , \nu_0}\left[-\log \hat{p}(a|s)\right]\label{eq:mleEqBC}    
\end{align}
Equation \eqref{eq:mleEqBC} is exactly what a typical Behavioral Cloning (BC) \citep{torabi2018behavioral} minimizes under entropy regularization, as the objective of BC is
\begin{align}
  & \!\!\underset{Q\in \mathcal{Q}}{\min }  \;\mathbb{E}_{(s, a)\sim \pi^*, \nu_0}  \left[-\log \hat{p}_Q(a|s)\right] \label{eq:BC}
\end{align}



\noindent where $\hat{p}_Q(a\mid s) = \frac{Q(s,a)}{\sum_{\tilde{a}\in\mathcal{A} }Q(s,\tilde{a})}$. Note that the solution set of Equation \eqref{eq:BC} fully contains the solution set of the ERM-IRL/DDC objective. This means that any solution to the ERM-IRL/DDC problem also minimizes the imitation learning objective, but not necessarily vice versa. Consequently, under entropy regularization, the IL objective is fundamentally easier to solve than the offline IRL/DDC problem, as it only requires minimizing the negative log-likelihood term without enforcing Bellman consistency. One of the key contributions of this paper is to formally establish and clarify this distinction: IL operates within a strictly simpler optimization landscape than the offline IRL/DDC, making it a computationally and statistically more tractable problem. This distinction further underscores the advantage of Behavioral Cloning (BC) over ERM-IRL/DDC for imitation learning (IL) tasks—since BC does not require modeling transition dynamics or solving an optimization problem involving the Bellman residual, it benefits from significantly lower computational and statistical complexity, making it a more efficient approach for IL.



\subsection{Estimation via minimax-formulated empirical risk minimization}
\label{sec:DoubleSampling}
While the idea of expected risk minimization -- minimizing Equation \eqref{eq:mainopt} -- is straightforward, empirically approximating $\mathcal{L}_{B E}(Q)(s, a) = (\mathcal{T} Q(s, a)-Q(s, a))^2$ and its gradient is quite challenging. 
As discussed in Section \ref{sec:BE&TD}, $\mathcal{T}Q$ is not available unless we know the transition function. As a result, we have to rely on an estimate of $\mathcal{T}$. A natural choice, common in TD-methods, is $\hat{\mathcal{T}} Q\left(s, a, s^{\prime}\right)=r(s, a)+\beta \cdot V_Q(s^\prime)$ which is computable given $Q$ and data $\mathcal{D}$. Thus, a natural proxy objective to minimize is:
\begin{align}
    &\mathbb{E}_{s' \sim P(s, a)} [\mathcal{L}_{\mathrm{TD}}(Q)\left(s, a, s^{\prime}\right)] :=\mathbb{E}_{s' \sim P(s, a)} [(\hat{\mathcal{T}} Q\left(s, a, s^{\prime}\right)-Q(s, a))^2] \notag
\end{align}
Temporal Difference (TD) methods typically use stochastic approximation to obtain an estimate of this proxy objective \citep{tesauro1995temporal, adusumilli2019temporal}. However, the issue with TD methods is that minimizing the proxy objective will not minimize the Bellman error in general (see Appendix \ref{sec:BiconjProofs} for details), because of the extra variance term, as shown below. 
\begin{align}
&\mathbb{E}_{s' \sim P(s, a)} 
\bigl[\mathcal{L}_{TD}(Q)(s, a, s^\prime)\bigr] = \mathcal{L}_{BE}(Q)(s, a) + \mathbb{E}_{s' \sim P(s, a)} 
\bigl[(\mathcal{T}Q(s, a) - \hat{\mathcal{T}}Q(s, a, s^\prime))^2\bigr]    \notag
\end{align}
As defined, $\hat{\mathcal{T}}$ is a one-step estimator, and the second term in the above equation does not vanish even in infinite data regimes. So, simply using the TD approach to approximate squared Bellman error provides a biased estimate. Intuitively, this problem happens because expectation and square are not exchangeable, i.e., 
$
\mathbb{E}_{ s^\prime \sim P(s, a)}\left[\delta_{Q}\left(s,a, s^\prime\right)\mid s, a\right]^2 \neq \mathbb{E}_{ s^\prime \sim P(s, a)}\left[\delta_{Q}\left(s,a, s^\prime\right)^2\mid s, a\right]
$. To remove this problematic square term, we employ an approach often referred to as the ``Bi-Conjugate Trick'' which replaces a square function by a linear function called the bi-conjugate:
\begin{align}
     \mathcal{L}_{BE}(s,a)(Q)&:=\mathbb{E}_{ s^\prime \sim P(s, a)}\left[\delta_{Q}\left(s,a, s^\prime\right)\mid s, a\right]^2\notag
     \\
     &=\max_{h\in \mathbb{R}}2\cdot\mathbb{E}_{ s^\prime \sim P(s, a)}\left[\delta_{Q}\left(s,a, s^\prime\right)\mid s, a\right]\cdot h-h^2 \notag
\end{align}

\noindent By further re-parametrizing $h$ using $\zeta = h - r + Q(s,a)$, after some algebra, we arrive at Lemma \ref{lem:OurBiconj}. (See Appendix \ref{sec:BiconjProofs} for the detailed derivation.)

\begin{lem}
\label{lem:OurBiconj}
\;\\
(a) We can express the squared Bellman error as
\begin{align}
    \mathcal{L}_{BE}(Q)(s, a)&:=(\mathcal{T} Q(s, a)-Q(s, a))^2 \notag
    \\
    &=\mathbb{E}_{s^\prime \sim P(s, a)} 
    \bigl[\mathcal{L}_{TD}(s, a, s^\prime)(Q)\bigr] - \beta^2 D(Q)(s, a) \label{eq:OurBiconj}
\end{align}
where
\begin{align}
    D(Q)(s, a):=\min _{\zeta \in \mathbb{R}} \mathbb{E}_{s^{\prime} \sim P(s, a)}\left[\left(V_{Q}\left(s^{\prime}\right)-\zeta\right)^2 \mid s, a\right]\label{eq:D(Q)}
\end{align}
(b) Define the minimizer (over all states and actions) of objective \eqref{eq:D(Q)} as
$$ \zeta^*: (s,a)\mapsto \arg\min_{\zeta\in \mathbb{R}}\mathbb{E}_{ s^\prime \sim P(s, a)}\left[\left(V^\ast(s^\prime)- \zeta\right)^2\mid s, a\right]$$
then
$r(s,a) = Q^*(s,a)-\beta \zeta^*(s,a)$.
\end{lem}
\noindent The reformulation of $\lbe$ proposed in Lemma \ref{lem:OurBiconj} enjoys the advantage of minimizing the squared TD-error ($\mathcal{L}_{TD}$) but without bias. Combining Theorem \ref{thm:mainopt} and Lemma \ref{lem:OurBiconj}, we arrive at the following Theorem \ref{thm:algoEQ}, which gives the expected risk minimization formulation of IRL we propose.

\begin{thm}
\label{thm:algoEQ} $Q^*$ is uniquely identified by expected risk minimization, i.e.,  
\begin{align}
     &\underset{Q\in \mathcal{Q}}{\min }  \;\mathcal{R}_{exp}(Q) \notag
     \\
     &=\min _{Q \in \mathcal{Q}} \mathbb{E}_{(s, a) \sim \pi^*, \nu_0}\bigl[{\mathcal{L}_{N L L}(Q)(s, a)} +\lambda \mathbbm{1}_{a=a_s} \bigl\{\mathbb{E}_{s^{\prime} \sim P(s, a)}\left[{\mathcal{L}_{T D}(Q)\left(s, a, s^{\prime}\right)}\right]-\beta^2 {D(Q)(s, a)}\bigr\} \notag
     \\
    &=\min _{Q \in \mathcal{Q}}\max_{\zeta\in \mathbb{R}^{S\times A}} \mathbb{E}_{(s, a) \sim \pi^*, \nu_0, s^{\prime} \sim P(s, a)}\bigl[\underbrace{\textcolor{blue}{-\log \left(\hat{p}_Q(a \mid s)\right)}}_{1)} + \lambda\mathbbm{1}_{a=a_s}\bigl\{\underbrace{\textcolor{red}{\bigl(\hat{\mathcal{T}} Q\left(s, a, s^{\prime}\right)-Q(s, a)\bigr)^2}}_{{2)}} \notag
    \\
    & \quad -\beta^2 \underbrace{ \bigl(\textcolor{orange}{\left(V_{Q}\left(s^{\prime}\right)-\zeta(s,a)\right)^2} }_{{3)}}\bigr\} \bigr]\label{eq:AlgoOpt}
\end{align}

Furthermore, $r(s, a)=Q^*(s, a) - \beta \zeta^*(s, a)$ where $\zeta^*$ is defined in Lemma \ref{lem:OurBiconj}.
\end{thm}

\noindent Equation \eqref{eq:AlgoOpt} in Theorem \ref{thm:algoEQ} is a mini-max problem in terms of $Q\in\mathcal{Q}$ and the introduced dual function $\zeta\in \mathbb{R}^{S\times A}$. To summarize, term 1) is the negative log-likelihood equation, term 2) is the TD error, and term 3) introduces a dual function $\zeta$.
The introduction of the dual function $\zeta$ in term 3) may seem a bit strange. 
In particular, note that $\arg\max_{\zeta \in \mathbb{R}} -\mathbb{E}_{s^{\prime} \sim P(s, a)}\left[\left(V_Q\left(s^{\prime}\right)-\zeta\right)^2 \mid s, a\right]$ is just $\zeta = \mathbb{E}_{s'\sim P(s,a)}[V\left(s^{\prime}\right)|s,a]$. 
However, we do not have access to the transition kernel and the state and action spaces may be large. 
Instead, we think of $\zeta$ as a function of states and actions, $\zeta(s,a)$ as introduced in Lemma~\ref{lem:OurBiconj}. This parametrization allows us to optimize over a class of functions containing $\zeta(s,a)$ directly. 

Given the minimax resolution for the expected risk minimization problem in Theorem \ref{thm:algoEQ} finds $Q$ under an infinite number of data, we are now ready to discuss the case when we are only given a finite dataset $\mathcal{D}$ instead. In this case, we solve the \textit{empirical risk minimization} problem.

\begin{defn}[Empirical risk minimization problem]
\label{def:ERM}
Given $N := |\mathcal{D}|$ where $\mathcal{D}$ is a finite dataset. An empirical risk minimization problem is defined as the problem of finding $Q$ that minimizes the empirical risk $\mathcal{R}_{emp}(Q;\mathcal{D})$, which is defined as
\begin{align}
     &\;\mathcal{R}_{emp}(Q;\mathcal{D}):=\max_{\zeta\in \mathbb{R}^{S\times A}}\frac{1}{N}  \sum_{(s,a,s^\prime)\in \mathcal{D}}\notag
    \\
    &\bigl[\textcolor{blue}{-\log \left(\hat{p}_Q(a \mid s)\right)} + 
   \lambda\mathbbm{1}_{a=a_s}\bigl\{\textcolor{red}{\bigl(\hat{\mathcal{T}} Q\left(s, a, s^{\prime}\right)-Q(s, a)\bigr)^2} -\beta^2  \textcolor{orange}{\left(V_{Q}\left(s^{\prime}\right)-\zeta(s,a)\right)^2} \bigr\} \bigr]\notag
   \\
   &= \frac{1}{N}\bigl[\sum_{(s,a,s^\prime)\in \mathcal{D}}\bigl(\textcolor{blue}{-\log \left(\hat{p}_Q(a \mid s)\right)}\bigr)+ 
\lambda\mathbbm{1}_{a=a_s}\notag
    \\
    &\bigl(  \sum_{(s,a,s^\prime)\in \mathcal{D}} \textcolor{red}{\bigl(\hat{\mathcal{T}} Q\left(s, a, s^{\prime}\right)-Q(s, a)\bigr)^2}  -\beta^2 \min_{\zeta\in \mathbb{R}^{S\times A}} 
   \sum_{(s,a,s^\prime)\in \mathcal{D}} \textcolor{orange}{\left(V_{Q}\left(s^{\prime}\right)-\zeta(s,a)\right)^2}\bigr) \bigr] \label{eq:EmpiricalERMIRL}
\end{align}
\end{defn}


\section{The general case: Proof of \texorpdfstring{\Cref{thm:main-decomp}}{Theorem 1.6}}\label{sec:algo}

First, we show that data structure of \Cref{l:max_min_query} can be used to compute distances witnessed by shortest paths that pass through a constant-size separator.

\begin{lemma}\label{l:single_adhesion}
Fix a constant $k \in \mathbb{N}$. There exists an algorithm which as the input receives an edge-weighted graph $G$ on $n$ vertices and $m$ edges together with a partition of its vertices into three sets $A, B, C$ such that $|B| \leq k$ and there are no edges between $A$ and $C$, and as the output computes $\max_{c \in C} \dist(a, c)$ for every $a \in A$. The running time is $\Oh(m \log n + n \log^{k - 1} n)$.
\end{lemma}

\begin{proof}
Let $B = \{b_1, \ldots, b_k\}$. For any $a \in A, c \in C$, we have $\dist(a, c) = \min_{i \in [k]} \dist(a, b_i) + \dist(c, b_i)$. First, we run Dijkstra's algorithm from every vertex in $B$ to find $\dist(v, b_i)$ for every $v \in V(G)$ and $i \in [k]$. Next, we use \Cref{l:max_min_query} to construct a data structure $\mathbb{D}$ for the point set $\{(\dist(c, b_1), \dots, \dist(c, b_k))\colon c\in C\}\subseteq \mathbb{R}^k$. Now, the value $\max_{c \in C} \dist(a, c)$ for any given $a$ is equal to the answer of $\mathbb{D}$ to the query with argument $(\dist(a, b_1), \dots, \dist(a, b_k))$.
\end{proof}

After computing the distances over a constant-size separator, we will use the following observation to simplify one of the sides of the separation.

\begin{lemma}\label{l:inserting_paths}
Let $G$ be a edge-weighted connected graph and let $A, B, C$ be a partition of its vertices such that there are no edges between $A$ and $C$. For every pair of vertices $u, v \in B$, let $P_{u, v}$ be any shortest path from $u$ to $v$ with all internal vertices in $C$ (assuming such a path exists).

Let $G'$ denote a graph obtained from $G[A \cup B]$ by adding an edge from $u$ to $v$ of weight equal to the length of $P_{u, v}$, for all $u, v \in B$ for which $P_{u, v}$ exists. Then,  $$\dist_G(s, t) = \dist_{G'}(s, t)\qquad\textrm{for all }s,t\in A\cup B.$$
\end{lemma}
\begin{proof}
Let $G''$ be the graph obtained by adding new edges of $G'$ to $G$.
Fix any $s, t \in A \cup B$ and let $P$ denote the shortest path from $s$ to $t$ in $G''$ which minimizes the number of vertices from $C$ visited. Naturally, the weight of $P$ is equal $\dist_G(s, t)$. Assume that such path visits at least one vertex of $C$. Then, the path $P$ is of the form $s \xrightarrow{P_1} x \xrightarrow{P_2} y \xrightarrow{P_3} t$, where $x, y \in B$ and all the internal vertices of $P_2$ are in $C$. By the construction of $G'$, $P_2$ can be replaced with a direct edge from $x$ to $y$ of the same weight. We obtain a same weight path with a smaller number of vertices of $C$ visited, which is a contradiction. Therefore, $P$ is entirely contained in $A \cup B$, hence it exists in $G'$. This shows that $\dist_G(s, t) = \dist_{G'}(s, t)$.
\end{proof}


The next lemma encapsulates the main algorithmic content of the proof of \Cref{thm:main-decomp}. The algorithm will split the tree decomposition provided on input into smaller parts for which the eccentricities are easier to calculate. We use the following lemma to handle a single such part.
\begin{lemma}\label{l:star}
Fix constants $k, g \in \mathbb{N}, 0 < \delta < \frac{1}{54}$. Assume we are given $n \in \mathbb{N}$, an edge-weighted graph $G$ on at most $n$ vertices with a weight function $w \colon E(G) \to \mathbb{N}$, a vertex subset $A$ and a collection of non-empty vertex subsets $V_0, V_1, \dots, V_\ell$ satisfying the following conditions:
\begin{itemize}[nosep]
	\item The sum of weights of all the edges in $G$ is bounded by $\Oh(n)$.
	\item $V(G) \setminus A = V_0 \cup V_1 \cup \dots \cup V_\ell$.
	\item $|A| \leq k$.
	\item For every $i \in [\ell]$, $G[V_i \setminus V_0]$ is connected, $N_G(V_i \setminus V_0) = V_i \cap V_0$, $|V_i| = \Oh(n^\delta)$, and $|V_0 \cap V_i| \leq 4$.
	\item For all $i, j \in [\ell], i \neq j$, $V_i \setminus V_0$ and $V_j \setminus V_0$ are disjoint and non-adjacent in $G$.
	\item Every edge $uv \in E(G)$ with $u, v \not\in A$ is contained in $G[V_i]$ for some $i\in \{0,1,\ldots,\ell\}$.
	\item The graph obtained by taking $G[V_0]$ and adding a clique on $V_0 \cap V_i$ for every $i \in [\ell]$ has Euler genus bounded by $g$.
\end{itemize}
Then, we can compute the eccentricity of every vertex of $G$ in time $\Oh \left( n^{1 + \frac{150 + 54 \delta}{151}} \log^k n \right)$.
\end{lemma}

\begin{proof}
Fix $\delta' = \frac{1 + 97 \delta}{151}$; we have $\delta' - \delta = \frac{1 - 54\delta}{151} > 0$.
Let $E_i$ denote the set of edges with one endpoint in $V_i$ and the other endpoint in $V_i \setminus V_0$. For $i \in [\ell]$, we shall say that $V_i$ is {\em{heavy}} if the sum of weights of $E_i$ is larger than $n^{\delta'}$. Since the sets $E_i$ are pairwise disjoint and the total sum of weights of all the edges is bounded by $\Oh(n)$, the number of heavy subsets is bounded by $\Oh(n^{1 - \delta'})$. Without loss of generality, we may assume that $V_{\ell' + 1}, \dots, V_\ell$ are heavy and $V_1, \dots, V_{\ell'}$ are not, for some $\ell'\in \{0,\ldots,\ell\}$.


For any source vertex $s$, we can calculate distances from $s$ to every vertex of $G$  using breadth first search in time $\Oh(\sum_{e \in E(G)} w(e)) = \Oh(n)$.
In particular, for every $\ell' < i \leq \ell$, we can compute the distances from every vertex of $V_i$ to every vertex of $G$ in total time $\Oh(n^{2 - \delta' + \delta})$, because $$|V_{\ell'+1}\cup \ldots\cup V_{\ell}|\leq n^{1-\delta'}\cdot \Oh(n^\delta)=\Oh(n^{1-\delta'+
\delta}).$$
Additionally, we calculate distances $\dist_G(a, v)$ for every $a \in A, v \in V(G)$ in time $O(n)$.

For every $i \in [\ell]$ and $u,v \in V_0 \cap V_i$, there exists a shortest path $P_{i,u,v}$ from $u$ to $v$ with all internal vertices belonging to $V_i - V_0$ due to the assumption that $G[V_i - V_0]$ is connected and $N_G(V_i - V_0) = V_i \cap V_0$. Therefore, the distance from $u$ to $v$ is bounded by the sum of weights of edges in $E_i$. In particular, for $i \in [\ell']$, $\dist_G(u, v) \leq n^{\delta'}$.

We define $\widetilde{G}$ to be the graph obtained by taking $G[A \cup V_0 \cup \dots \cup V_{\ell'}]$ and applying the following operation for every $i \in \{\ell' + 1, \dots, \ell\}$:
for each pair of vertices $u, v \in A \cup (V_0 \cap V_i)$, add an edge in $\widetilde{G}$ between $u$ and $v$ with weight equal to the total weight of $P_{i,u,v}$. For a fixed $i, u$, we can find $P_{i, u, v}$ for all $v$ using breadth first search in time $\Oh(n)$. Taking a sum over all $i, u$, we get that $\tilde{G}$ can be computed in total time $\Oh(n^{2 - \delta'})$.


\begin{claim}\label{cl:wG}
The sum of the edge weights in $\widetilde{G}$ is $\Oh(n)$. Moreover, for all $u, v \in V(\widetilde{G})$, we have $\dist_{\widetilde{G}}(u, v) = \dist_{G}(u, v)$.
\end{claim}

\begin{proof}
Consider $i \in \{\ell' + 1, \dots, \ell\}$ and any $u, v \in A \cup (V_0 \cap V_i)$ for which we added an edge. Its weight is bounded by the sum of weights of edges in $E_i$. Therefore, the total weight of all edges added is at most
$$
\sum_{i \in \{\ell' + 1, \dots, \ell\}} \left( |A \cup (V_0 \cap V_i)|^2 \sum_{e \in E_i} w(e) \right) \leq (4 + k)^2 \sum_{e \in E(G)} w(e) = \Oh(n).
$$
This proves the first part of the claim.

For the second part of the claim, consider any $i \in \{\ell' + 1, \dots, \ell \}$ and observe that by our assumptions, $A \cup (V_0 \cap V_i)$ separates $(V_0 \cup \dots \cup V_{\ell'} \cup V_{i + 1} \cup \dots \cup V_\ell) \setminus V_i$ from $V_i \setminus V_0$. Hence it suffices to repeatedly apply \Cref{l:inserting_paths}.
\end{proof}

For every $u \in V(\widetilde{G})$, we have $\ecc_G(u) = \max(\ecc_{\widetilde{G}}(v), \max_{v \in V(G) \setminus V(\widetilde{G})} \dist_G(u, v))$. Note, that we already know all the distances $\dist_G(u, v)$ for $v \in V(G) \setminus V(\widetilde{G})$. Similarly, we can already compute $\ecc_G(u)$ for every $u \in V(G) \setminus V(\widetilde{G})$. Therefore, it remains to compute $\ecc_{\widetilde{G}}(v)$ for each $v \in V(\widetilde{G})$. Our goal is to show that this can be done efficiently using \Cref{l:main_ecc}.

Now, let $G'$ be the graph obtained from $\tilde{G}$ by replacing every edge $e$ non-indicent to $A$ with $w(e)\geq 2$ with a path of length $w(e)$ consisting of unit-weight edges. This operation again preserves the distances. Since the sum of edge weights in $\tilde{G}$ is of $\Oh(n)$, the total number of vertices in $G'$ is of $\Oh(n)$. For $0 \leq i \leq \ell'$, we write $V'_i$ to denote the set $V_i$ together with all the vertices added as a part of a path between two endpoints in $V_i$.
As $V_i$ is not heavy for each $i\in [\ell']$, we have
$$
|V'_i \setminus V'_0| \leq |V_i| + \sum_{e \in E_i} w(e) = \Oh(n^{\delta'})\qquad \textrm{for all }i\in [\ell'].
$$

Let $G_0$ denote the graph $G'[V'_0]$ and let $G_0^*$ denote the graph $G'- A$ with $V'_i - V'_0$ contracted to a single vertex $v_i^*$, for each $i \in [\ell']$; note that, all edges of $G_0$ and $G_0^*$ have unit weight.

\begin{claim}
	The graph $G_0^*$ is does not contain $K_{t}$ as a minor, where $t = \Oh(\sqrt{g})$.
\end{claim}

\begin{proof}
Let $\bar{G}_0$ denote the graph obtained by taking $G_0$ and adding a clique on $V_0 \cap V_i$ for every $i \in [\ell']$.
By lemma assumptions and the fact that subdividing edges does not increase the Euler genus, $\bar{G}_0$ has Euler genus at most $g$. In particular, $\bar{G}_0$ is $K_{t'}$-minor-free for some $t' = \Oh(\sqrt{g})$, because the Euler genus of $K_{t'}$ is $\Omega({t'}^2)$.

Similarly, let $\bar{G}_0^*$ be the graph obtained by taking $G_0^*$ and adding a clique on each $V_0 \cap V_i$.
Note, that $\bar{G}_0^* - \{v_1^*, \dots, v_{\ell'}^*\}$ is precisely $\bar{G}_0$. Let $t = \max(t', 6)$.
Recall that a minor model of a clique $K_t$ consists of $t$ pairwise vertex-disjoint connected subgraphs, called
branch sets, such that there is at least one edge between each pair of the branch sets.
Consider a minor model $\varphi$ of $K_{t}$ in $\bar{G}^*_0$.
Note that $\varphi$ cannot contain any singleton branch set of the form $\{v^*_i\}$, for the degree of $v^*_i$ in $\bar{G}^*_0$ is at most $4 < t - 1$. Furthermore, since $N_{\bar{G}^*_0}(v^*_i) = V_0 \cap V_i$, any branch set containing $v^*_i$ and at least one other vertex contains some $u \in V_0 \cap V_i$, and $N_{\bar{G}^*_0}(v^*_i)\subseteq N_{\bar{G}^*_0}(u)$, hence removing $v^*_i$ from this branch set preserves the model. Therefore, we can assume without loss of generality that all branch sets of $\varphi$ are disjoint from $\{v^*_1, \dots, v^*_{\ell'}\}$, hence $\varphi$ is a minor model of $K_{t}$ in $\bar{G}_0$. This is a contradiction, as $t \geq t'$ and $\bar{G}_0$ is $K_{t'}$-minor-free. Therefore, $\bar{G}_0^*$ is $K_t$-minor-free, hence $G_0^*$ also.
\end{proof}

Let $\rho' = \frac{2 - 108 \delta}{151} > 0$. The graph $G^*_0$ is a unit-weight graph and is $K_{t}$-minor-free.
Hence, by applying \Cref{t:r_division} to $G^*_0$ (with $\varepsilon = \rho'/2$)
we obtain an $n^{\rho'}$-division $\mathcal{R}_0$ in time $\Oh(n^{1 + \rho'})$.
We extend it to $G' - A$ by mapping every contracted vertex $v^*_i$ to $N_{G' - A}[V'_i - V'_0] = (V'_i - V'_0) \cup (V_0 \cap V_i)$. Formally, we put $V''_i \coloneqq N_{G' - A}[V'_i - V'_0]$ and 
$$
\mathcal{R} \coloneqq \left\{ (R_0 \cap V'_0) \cup \bigcup_{i \colon v^*_i \in R_0} V''_i \colon R_0 \in \mathcal{R}_0 \right\}.
$$

Now, we argue that $\mathcal{R}$ is a reasonable division of $G' - A$. Clearly, all sets $R \in \mathcal{R}$ are connected in $G' - A$. Pick any $R \in \mathcal{R}$ and let $R_0$ be its corresponding set in $\mathcal{R}_0$.
Every vertex $v^*_i$ is mapped to a set of size $\Oh(n^{\delta'})$, therefore
$$|R| \leq |R_0| \cdot \Oh(n^{\delta'}) = \Oh(n^{\rho' + \delta'}).$$

By our construction, for every $i \in [\ell']$, $R$ is either disjoint from $V'_i - V'_0$ or contains whole $N_{G' - A}[V'_i - V'_0]$. This means that no vertex belonging to any $V'_i - V'_0$ can be in $\partial R$, hence $\partial R \subseteq V'_0$.

Pick any $u \in \partial R \cap R_0$. Assume that $u \not\in \partial R_0$. Then every vertex of $N_{G_0^*}(u)$ must be in $R_0$, hence $N_{G - A'}(u) \subseteq R$, which is a contradiction. This means that $\partial R \cap R_0 \subseteq \partial R_0$.

Pick any $u \in \partial R - R_0$. Then, $u \in V_0 \cap V_i$ for some $i \in [\ell']$ such that $v_i^* \in R_0$. Moreover, $v_i^* \in \partial R_0$ and is adjacent to $u$ in $G_0^*$. The number of such $u$ is bounded by $4 |\partial R_0 \cap \{ v_1^*, \dots, v_{\ell'}^* \}|$.

Putting two cases together, we obtain:
$$
\sum_{R \in \mathcal{R}} |\partial R| = \sum_{R \in \mathcal{R}} \left(|\partial R \cap R_0| + |\partial R - R_0|\right) \leq \sum_{R_0 \in \mathcal{R}_0} \left(|\partial R_0| + 4 |\partial R_0 \cap \{ v_1^*, \dots, v_{\ell'}^* \}|\right) = \Oh(n^{1 - \frac{1}{2}\rho'}).
$$

It remains to show the following claim.

\begin{claim}
Pick any $R \in \mathcal{R}, s_R \in R$. The number of different distance profiles on $R$ relative to $s_R$ in $G' - A$ is of $\Oh(n^{48\rho' + 54\delta'})$.
\end{claim}
\begin{proof}
We look at every vertex $v \in V(G') \setminus A$ and consider three cases: $v \in R$, $v \in V'_0$, and $v \in V'_i \setminus (V'_0 \cup R)$ for some $i \in [\ell']$. By our construction, $R \cap V'_0$ is non-empty, hence w.l.o.g. we can assume that $s_R \in V'_0$ as whether two vertices have the same profile on $R$ is independent of the choice of the pivot vertex.

In the first case, there are at most $|R| = \Oh(n^{\rho' + \delta'})$ such vertices, hence they realise at most that many profiles.

In the second case, we want to observe that profile of any vertex $v \in V'_0$ on $R$ depends only on its profile on $R \cap V'_0$ (relative to $s_R$). Pick any $t \in R - V'_0$. Then $t \in V'_i - V'_0$ for some $i \in [\ell']$, $V_i \cap V_0 \subseteq R \cap V'_0$, and every path from $v$ to $t$ intersects $V_i \cap V_0$. In particular, distances from $v$ to vertices of $V_i \cap V_0$ determine its distance to $t$, which proves the observation.

Let $\tilde{G}_0$ denote the graph obtained by taking $G'[V'_0]$ and for every $i \in [\ell'], u, v \in V_0 \cap V_i$ adding a disjoint path from $u$ to $v$ of length $\dist(u, v)$. Let $P_i$ denote the vertex set of paths added between $V_0 \cap V_i$. For every $t \in V'_0$ we have $\dist_{G' - A}(v, t) = \dist_{\tilde{G}_0}(v, t)$, so it suffices to bound the number of profiles on $R \cap V'_0$ in $\tilde{G}_0$. By our assumptions, $\tilde{G}_0$ has Euler genus bounded by $g$ and all $P_i$ are of size $\Oh(n^{\delta'})$.

Let $R_0$ be the set of $\mathcal{R}_0$ corresponding to $R$. Let $\tilde{R}_0$ denote the set $(R \cap V'_0) \cup \bigcup_{i : v^*_i \in R_0} P_i$. Such set is connected in $\tilde{G}_0$. Moreover, similarly to $R$, its size is $\Oh(n^{\rho' + \delta'})$. Applying \Cref{thm:distprofiles}, we get that the number of distance profiles on $\tilde{R}_0$ in $\tilde{G}_0$ is $\Oh(n^{12(\rho' + \delta')})$, which also bounds the number of profiles on $R$ in $G' - A$ realised by $V'_0$.

For the third case, assume $v \in V'_i \setminus (V'_0 \cup R)$ for some $i\in [\ell']$. Every path from $v$ to any vertex of $R$ in $G' - A$ intersects $V_i \cap V_0$. Let $v_1, \dots v_p$ be the vertices of $V_i \cap V_0$, where $p \leq 4$. The profile of $v$ on $R$ is then determined by the following:
\begin{itemize}[nosep]
 \item[(a)] the profile of each $v_j$ on $R$,
 \item[(b)] $\dist_{G' - A}(v, v_j) - \dist_{G' - A}(v, v_1)$ for each $2 \leq j \leq p$, and
 \item[(c)] $\dist_{G' - A}(s_R, v_j) - \dist_{G' - A}(s_R, v_1)$ for each $2 \leq j \leq p$ where $s_R$ is some pivot vertex of $R$.
\end{itemize}
By the previous case, the number of distance profiles of each $v_j$ is $\Oh(n^{12(\rho' + \delta')})$. The distances between $v$ and $v_j$ are bounded by $|V'_i|$, hence each quantity described in (b) can take $\Oh(n^{\delta'})$ different possible values. Similarly, since $v_1$ and $v_j$ are connected via $V'_i$, $|\dist_{G' - A}(s_R, v_j) - \dist_{G' - A}(s_R, v_1)| \leq \Oh(n^{\delta'})$. The number of different possible profiles of such $v$ is therefore bounded by $\Oh(n^{48(\rho' + \delta') + 6\delta'}) = \Oh(n^{48\rho' + 54\delta'})$. This finishes the proof of the claim.
\end{proof}

Now we can apply \Cref{l:main_ecc} to graph $G'$ with apex set $A$, $X = V(\widetilde{G})$, and the following constants: $$\rho = \rho' + \delta',\qquad \gamma = 1 - \frac{1}{2}\rho',\quad \textrm{and}\quad \alpha = 48\rho' + 54 \delta'.$$ This allows us to calculate all $V(\widetilde{G})$-eccentricities in $G'$ in time
$$
\Oh \left( \left(
	n^{ 2 - \frac{1}{2} \rho' } +
	n^{ 1 + 48\rho' + 54 \delta' }
\right) \log^k n \right) =
\Oh \left( n^{1 + \frac{150 + 54 \delta}{151}} \log^k n \right).
$$
Since for each $v\in V(\widetilde{G})$ we have $\ecc_{\widetilde{G}}(v) = \max_{u \in V(\widetilde{G})} \dist_{\widetilde{G}}(v, u) = \max_{u \in V(\widetilde{G})} \dist_{G'}(v, u)$, this means that we have successfully computed all the eccentricities in $\widetilde{G}$; and as we argued, this is enough to compute all the eccentricities in $G$ as well.

Finally, the total running time of the algorithm is
$$
\Oh \left( n^{1 + \frac{150 + 54 \delta}{151}} \log^k n + n^{2 - \delta' + \delta} \right) =
\Oh \left( n^{1 + \frac{150 + 54 \delta}{151}} \log^k n \right).
$$\qedhere
\end{proof}


\begin{lemma}\label{l:star2}
Fix constants $k, g \in \mathbb{N}, 0 < \delta < \frac{1}{54}$. Assume we are given $n \in \mathbb{N}$, an edge-weighted graph $G$ on at most $n$ vertices with a weight function $w \colon E(G) \to \mathbb{N}$, a vertex subset $A$ and a collection of non-empty vertex subsets $V_0, V_1, \dots, V_\ell$ satisfying the same conditions as in \Cref{l:star} with the following differences:
\begin{itemize}
	\item we don't require $G[V_i - V_0]$ to be connected and $V_i - V_0$ to be adjacent to whole $V_i \cap V_0$;
	\item instead of $|V_0 \cap V_i| \leq 4$, we require $|V_0 \cap V_i| \leq k$.
\end{itemize}
Then, we can compute the eccentricity of every vertex of $G$ in time $\Oh \left( n^{1 + \frac{150 + 54 \delta}{151}} \log^{k + 5g} n \right)$.
\end{lemma}

\begin{proof}
We will reduce our input to one which will satisfy the conditions of \Cref{l:star}. We start by addressing the adhesions $V_0 \cap V_i$ containing too many vertices.

Let $G_0$ denote the graph $G[V_0]$ with cliques placed at $V_0 \cap V_i$ for every $i \in [\ell]$.
For every $i \in [\ell]$ we repeat the following procedure: while $|V_0 \cap V_i| > 4$,
remove arbitrary $5$ vertices from $V_0 \cap V_i$. Since $|V_0 \cap V_i| \leq k$ for each $i\in [\ell]$,
this procedure can be implemented in total time $\Oh(n)$. As a result, at the end we have $|V_0 \cap V_i| \leq 4$ for all $i \in [\ell]$. Let $M$ be the set of all the removed vertices. By our assumptions, $G_0$ has Euler genus bounded by $g$, hence it cannot contain $g + 1$ pairwise disjoint copies of $K_5$
(as the Euler genus of a graph is the sum of the Euler genera of its 2-connected components~\cite{StahlB77} and $K_5$ is not planar). Each removed quintiple of vertices induces a $K_5$ in $G_0$, hence we have $|M| \leq 5g$. We set $A' = A \cup M$ and may thus assume that $V_i$ is disjoint from $A'$ for all $0 \leq i \leq \ell$.

Now, fix $i \in [\ell]$. Let $C^i_1, \dots, C^i_{r_i}$ denote the connected components of $V_i - V_0$ in $G - A'$. We define $W^i_j := N_{G - A'}[C^i_j]$ for every $j \in [r_i]$. Clearly, all $W^i_j$ induce a connected subgraph of $G$ and satisfy $N_{G - A'}(W^i_j - V_0) = W^i_j \cap V_0$. We put $V'_0 := V_0$ and enumerate
$$
\{V'_1, V'_2, \dots V'_{\ell'}\} := \{ W^i_j \colon i \in [\ell], j \in [r_i] \}.
$$
It is easy to verify that the sets $A'$ and $V'_0, V'_1, \dots, V'_{\ell'}$ satisfy the conditions of \Cref{l:star}. We apply said lemma to calculate the eccentricity of every vertex of $G$ in the desired time.
\end{proof}



The next statement is a reformulation of \Cref{thm:main-decomp}.

\begin{theorem}
Fix constants $k, g \in \mathbb{N}$. Assume we are given a graph $G$ on $n$ vertices together with its tree decomposition $(T, \beta)$ and a set of private apices $A_t \subseteq \beta(t)$ for each node $t\in V(T)$ such that the following conditions hold:
\begin{itemize}[nosep]
 \item For every node $t \in V(T)$, we have $|A_t| \leq k$.
 \item For every edge $st \in E(T)$,  we have $|\beta(v) \cap \beta(u)|\leq k$.
 \item For every node $t \in V(T)$, graph obtained by taking $G[\beta(t)] - A_t$ and turning  $(\beta(t) \cap \beta(s))\setminus A_t$ into a clique for every edge $st \in E(T)$ has Euler genus bounded by $g$.
\end{itemize}
Then, we can compute the eccentricity of every vertex of $G$ in time $\Oh \left( n^{1 + \frac{355}{356}} \log^{k + 5g} n \right)$.
\end{theorem}

\begin{proof}
We may assume that $|V(T)|\leq n$, for every tree decomposition with no two bags comparable by inclusion has this property; and adjacent comparable bags can be merged by contracting the edge between them.

For a node $t\in V(T)$, by the {\em{weight}} of $t$ we mean the size of the corresponding bag, that is, $|\beta(t)|$. For any subset of nodes $S \subseteq V(T)$, we define $\beta(S) \coloneqq \bigcup_{t \in S} \beta(t)$ By the {\em{weight}} of $S$, we mean the total weight of the elements of $S$, that is, $\sum_{t\in S} |\beta(t)|$. 

\begin{claim}\label{cl:weight-T}
The weight of $V(T)$ is of $\Oh(n)$.
\end{claim}

\begin{proof}
The sets $\beta'(t) := \beta(t) - \bigcup_{s \in N_T(t)} \beta(s)$ are pairwise disjoint. We have
$$
\sum_{t \in V(T)} |\beta(t)| =
\sum_{t \in V(T)} |\beta'(t)| + 2 \cdot \sum_{st \in E(T)} |\beta(s) \cap \beta(t)| \leq
|V(T)| + 2k|E(T)| = \Oh(n).
$$
\end{proof}

Since every bag induces a graph of bounded Euler genus, the number of edges contained in a bag is linear in its size. In particular, this implies that the total number of edges of $G$ is also bounded by $\Oh(n)$.

We set $$\delta \coloneqq \frac{1}{356}\qquad\textrm{and}\qquad \Delta \coloneqq \frac{355}{356}.$$ Root the tree $T$ in an arbitrarily chosen node; this naturally imposes an ancestor-descendant relation in $T$ (for convenience, every node is considered its own ancestor and descendant).

We start by partitioning $T$ into connected subtrees using the following procedure.
We proceed bottom-up over $T$, processing nodes in any order so that a node is processed after all its strict descendants have been processed. Along the way, we mark some nodes and split the edges of $T$ into heavy and light. Let $t \in V(T)$ be the currently processed non-root node of $T$ and let $e \in E(T)$ be the edge connecting $t$ with its parent. If the total weight of all the unmarked nodes that are descendants of $t$ is at least $n^\delta$ (recall that this includes $t$ itself as well), then we declare $e$ heavy and mark all the descendants of $t$ that were unmarked so far. Otherwise, the edge $e$ is declared light and the procedure proceeds to further nodes of $T$.

Observe that
removing all heavy edges splits $T$ into connected subtrees, say $T'_1, \cdots T'_m$. All of the subtrees, except for possibly the subtree containing the root node, are of weight at least $n^\delta$. In particular, the number of subtrees $m$, and therefore the number of heavy edges, is  bounded by $\Oh(n^{1 - \delta})$. Moreover, in every subtree $T'_i$, removing the node closest to the root splits $T'_i$ into smaller components, each of weight less than $n^\delta$.

Fix a heavy edge $e$ and let $T^e_1$ and $T^e_2$ be the two subtrees into which $T$ splits after removing~$e$. Let $X^e_i = \beta(T^e_i)$ for $i \in \{1, 2\}$. Put $A_e = X^e_1 \setminus X^e_2$, $C_e = X^e_2 \setminus X^e_1$, and $B_e = X^e_1 \cap X^e_2$. By the properties of tree decompositions, such choice of $A_e, B_e, C_e$ satisfies the conditions of \Cref{l:single_adhesion}, hence in time $\Oh(n \log^{k - 1} n)$ we can compute $\max_{v \in X^e_2} \dist_G(u,v)$ for every $u \in X^e_1$, and $\max_{u \in X^e_1} \dist_G(u,v)$ for every $v \in X^e_2$. Computing this for every heavy edge $e$ takes total time $\Oh(n^{2 - \delta} \log^{k - 1} n)$.

Fix any subtree $T'=T'_j$. Let $e_1 = t^{e_1}_1t^{e_1}_2, e_2 = t^{e_2}_1 t^{e_2}_2, \dots, e_\ell = t^{e_\ell}_1 t^{e_\ell}_2$ denote the heavy edges incident to $T'$, where $t^{e_i}_1 \in V(T')$ and $V(T') \subseteq V(T_1^{e_i})$ for every $i \in [\ell]$.
For a vertex $v \in \beta(T')$, let
$$d_0(v) = \max_{u \in \beta(T')} \dist_G(v, u)\qquad\textrm{and}\qquad d_i(v) = \max_{u \in X_2^{e_i}}\dist_G(v,u),\quad\textrm{for } i \in [\ell].$$ We have $\ecc(v) = \max \{ d_i(v)\colon i\in \{0,1,\ldots,\ell\}\}$.The values of $d_i(v)$ are already calculated for all $i\in [\ell]$, hence it remains to compute $d_0(v)$.

For every $i \in [\ell]$ and every pair of vertices $u, v \in \beta(t^{e_i}_1) \cap \beta(t^{e_i}_2)$ we find a shortest path between $u$ and $v$ with all internal vertices inside $X^{e_i}_2$ (or determine that it doesn't exist). For a fixed $u, v$ this can be done in time $\Oh(n)$. Since in total we perform this step at most $2k^2$ times per heavy edge, it takes $\Oh(n^{2 - \delta})$ time in total. Let $P_{i, u, v}$ denote such path, assuming it exists.

Let $G'$ denote the graph obtained from $G[\beta(T')]$ by taking every $i, u, v$ for which $P_{i, u, v}$ exists and adding an edge between $u$ and $v$ of weight equal to the total weight of $P_{i, u, v}$.
The weight of every edge inserted in $\beta(t^{e_i}_1) \cap \beta(t^{e_i}_2)$ is bounded by $|X^{e_i}_2|+1$. The total weight of all edges inserted is therefore at most
$$
\sum_{i \in [\ell]} |\beta(t^{e_i}_1) \cap \beta(t^{e_i}_2)|^2 \cdot (|X^{e_i}_2|+1) \leq
k^2 \sum_{i \in [\ell]} (|X^{e_i}_2|+1) = \Oh(n),
$$
where the last equality follows from the fact that all the trees $T^{e_i}_2$ are pairwise disjoint.
By \Cref{l:inserting_paths}, we have $\dist_{G'}(u, v) = \dist_G(u, v)$ for each $u, v \in \beta(T')$. Hence, computing $d_0(v)$ for every $v \in \beta(T')$ is equivalent to computing the eccentricity of every vertex in $G'$.

If the size of $\beta(T')$ is smaller than $n^\Delta$, we compute the eccentricities naively in time $\Oh(|\beta(T')|^2)$, 
noting that $G'$ has $\Oh(|\beta(T')|)$ edges (thanks to Claim~\ref{cl:weight-T} and bounded genus assumption 
of the last bullet of the theorem statement). Otherwise, we argue that we can use the algorithm in \Cref{l:star} as follows.

Let $t$ be the node of $T'$ closest to the root. Let $s_1, \dots, s_p$ be the children of $t$ in $T$ and let $T''_i$ denote the connected component of $T' - \{t\}$ containing $s_i$. Set $V_0 = \beta(t)$ and $V_i = \beta(T''_i)$ for $i \in [p]$.

It is now easy to verify that $G'$ and sets $A, \{V_i\colon 0\leq i\leq p\}$ selected this way satisfy the assumptions of \Cref{l:star2}. This allows us to use it to compute the eccentricities in $G'$ in time
$$
\Oh \left( n^{1 + \frac{150 + 54\delta}{151}} \log^{k + 5g} n \right) =
\Oh \left( n^{1 + \frac{354}{356}} \log^{k + 5g} n \right).
$$
As we argued, from these eccentricities, we may easily compute all the eccentricities in $G$.

Now, let us analyse the total running time of the whole algorithm. We invoke \Cref{l:star} $\Oh(n^{1 - \Delta})$ times, since we apply it only to subtrees $T'_i$ of size at least $n^\Delta$. The total running time of those applications is hence
$$
\Oh \left( n^{2 + \frac{354}{356} - \Delta} \log^{k + 5g} n \right) =
\Oh \left( n^{1 + \frac{355}{356}} \log^{k + 5g} n \right).
$$
We compute the eccentricities naively for subtrees smaller than $n^\Delta$, hence the total running time of this computation is
$$
\sum_{i \in [m] \colon |\beta(T'_i)| \leq n^\Delta} |\beta(T'_i)|^2 \leq
n^\Delta \cdot \sum_{i \in m} |\beta(T'_i)| = \Oh(n^{1 + \Delta})=\Oh\left(n^{1+\frac{355}{356}}\right).
$$
The rest of computation can be done in $\Oh(n^{2 - \delta} \log^k n)$. Therefore, the whole algorithm runs in time $\Oh \left( n^{1 + \frac{355}{356}} \log^{k + 5g} n \right)$.
\end{proof}

\section{Experiments}
\label{sec:Experiments} 

We conduct several experiments across different problem settings to assess the efficiency of our proposed method. Detailed descriptions of the experimental settings are provided in \cref{sec:apendix_experiments}.
%We conduct experiments on optimizing PINNs for convection, wave PDEs, and a reaction ODE. 
%These equations have been studied in previous works investigating difficulties in training PINNs; we use the formulations in \citet{krishnapriyan2021characterizing, wang2022when} for our experiments. 
%The coefficient settings we use for these equations are considered challenging in the literature \cite{krishnapriyan2021characterizing, wang2022when}.
%\cref{sec:problem_setup_additional} contains additional details.

%We compare the performance of Adam, \lbfgs{}, and \al{} on training PINNs for all three classes of PDEs. 
%For Adam, we tune the learning rate by a grid search on $\{10^{-5}, 10^{-4}, 10^{-3}, 10^{-2}, 10^{-1}\}$.
%For \lbfgs, we use the default learning rate $1.0$, memory size $100$, and strong Wolfe line search.
%For \al, we tune the learning rate for Adam as before, and also vary the switch from Adam to \lbfgs{} (after 1000, 11000, 31000 iterations).
%These correspond to \al{} (1k), \al{} (11k), and \al{} (31k) in our figures.
%All three methods are run for a total of 41000 iterations.

%We use multilayer perceptrons (MLPs) with tanh activations and three hidden layers. These MLPs have widths 50, 100, 200, or 400.
%We initialize these networks with the Xavier normal initialization \cite{glorot2010understanding} and all biases equal to zero.
%Each combination of PDE, optimizer, and MLP architecture is run with 5 random seeds.

%We use 10000 residual points randomly sampled from a $255 \times 100$ grid on the interior of the problem domain. 
%We use 257 equally spaced points for the initial conditions and 101 equally spaced points for each boundary condition.

%We assess the discrepancy between the PINN solution and the ground truth using $\ell_2$ relative error (L2RE), a standard metric in the PINN literature. Let $y = (y_i)_{i = 1}^n$ be the PINN prediction and $y' = (y'_i)_{i = 1}^n$ the ground truth. Define
%\begin{align*}
%    \mathrm{L2RE} = \sqrt{\frac{\sum_{i = 1}^n (y_i - y'_i)^2}{\sum_{i = 1}^n y'^2_i}} = \sqrt{\frac{\|y - y'\|_2^2}{\|y'\|_2^2}}.
%\end{align*}
%We compute the L2RE using all points in the $255 \times 100$ grid on the interior of the problem domain, along with the 257 and 101 points used for the initial and boundary conditions.

%We develop our experiments in PyTorch 2.0.0 \cite{paszke2019pytorch} with Python 3.10.12.
%Each experiment is run on a single NVIDIA Titan V GPU using CUDA 11.8.
%The code for our experiments is available at \href{https://github.com/pratikrathore8/opt_for_pinns}{https://github.com/pratikrathore8/opt\_for\_pinns}.


\subsection{2D Allen Cahn Equation}
\begin{figure*}[t]
    \centering
    \includegraphics[scale=0.38]{figs/Burgers_operator.pdf}
    \caption{1D Burgers' Equation (Operator Learning): Steady-state solutions for different initializations $u_0$ under varying viscosity $\varepsilon$: (a) $\varepsilon = 0.5$, (b) $\varepsilon = 0.1$, (c) $\varepsilon = 0.05$. The results demonstrate that all final test solutions converge to the correct steady-state solution. (d) Illustration of the evolution of a test initialization $u_0$ following homotopy dynamics. The number of residual points is $\nres = 128$.}
    \label{fig:Burgers_result}
\end{figure*}
First, we consider the following time-dependent problem:
\begin{align}
& u_t = \varepsilon^2 \Delta u - u(u^2 - 1), \quad (x, y) \in [-1, 1] \times [-1, 1] \nonumber \\
& u(x, y, 0) = - \sin(\pi x) \sin(\pi y) \label{eq.hom_2D_AC}\\
& u(-1, y, t) = u(1, y, t) = u(x, -1, t) = u(x, 1, t) = 0. \nonumber
\end{align}
We aim to find the steady-state solution for this equation with $\varepsilon = 0.05$ and define the homotopy as:
\begin{equation}
    H(u, s, \varepsilon) = (1 - s)\left(\varepsilon(s)^2 \Delta u - u(u^2 - 1)\right) + s(u - u_0),\nonumber
\end{equation}
where $s \in [0, 1]$. Specifically, when $s = 1$, the initial condition $u_0$ is automatically satisfied, and when $s = 0$, it recovers the steady-state problem. The function $\varepsilon(s)$ is given by
\begin{equation}
\varepsilon(s) = 
\left\{\begin{array}{l}
s, \quad s \in [0.05, 1], \\
0.05, \quad s \in [0, 0.05].
\end{array}\right.\label{eq:epsilon_t}
\end{equation}

Here, $\varepsilon(s)$ varies with $s$ during the first half of the evolution. Once $\varepsilon(s)$ reaches $0.05$, it remains fixed, and only $s$ continues to evolve toward $0$. As shown in \cref{fig:2D_Allen_Cahn_Equation}, the relative $L_2$ error by homotopy dynamics is $8.78 \times 10^{-3}$, compared with the result obtained by PINN, which has a $L_2$ error of $9.56 \times 10^{-1}$. This clearly demonstrates that the homotopy dynamics-based approach significantly improves accuracy.

\subsection{High Frequency Function Approximation }
We aim to approximate the following function:
$u=    \sin(50\pi x), \quad x \in [0,1].$
The homotopy is defined as $H(u,\varepsilon) = u - \sin(\frac{1}{\varepsilon}\pi x), $
where $\varepsilon \in [\frac{1}{50},\frac{1}{15}]$.

\begin{table}[htbp!]
    \caption{Comparison of the lowest loss achieved by the classical training and homotopy dynamics for different values of $\varepsilon$ in approximating $\sin\left(\frac{1}{\varepsilon} \pi x\right)$
    }
    \vskip 0.15in
    \centering
    \tiny
    \begin{tabular}{|c|c|c|c|c|} 
    \hline 
    $ $ & $\varepsilon = 1/15$ & $\varepsilon = 1/35$ & $\varepsilon = 1/50$ \\ \hline 
    Classical Loss                & 4.91e-6     & 7.21e-2     & 3.29e-1       \\ \hline 
    Homotopy Loss $L_H$                      & 1.73e-6     & 1.91e-6     & \textbf{2.82e-5}       \\ \hline
    \end{tabular}
    % On convection, \al{} provides 14.2$\times$ and 1.97$\times$ improvement over Adam or \lbfgs{} on L2RE. 
    % On reaction, \al{} provides 1.10$\times$ and 1.99$\times$ improvement over Adam or \lbfgs{} on L2RE.
    % On wave, \al{} provides 6.32$\times$ and 6.07$\times$ improvement over Adam or \lbfgs{} on L2RE.}
    \label{tab:loss_approximate}
\end{table}

As shown in \cref{fig:high_frequency_result}, due to the F-principle \cite{xu2024overview}, training is particularly challenging when approximating high-frequency functions like $\sin(50\pi x)$. The loss decreases slowly, resulting in poor approximation performance. However, training based on homotopy dynamics significantly reduces the loss, leading to a better approximation of high-frequency functions. This demonstrates that homotopy dynamics-based training can effectively facilitate convergence when approximating high-frequency data. Additionally, we compare the loss for approximating functions with different frequencies $1/\varepsilon$ using both methods. The results, presented in \cref{tab:loss_approximate}, show that the homotopy dynamics training method consistently performs well for high-frequency functions.





\subsection{Burgers Equation}
In this example, we adopt the operator learning framework to solve for the steady-state solution of the Burgers equation, given by:
\begin{align}
& u_t+\left(\frac{u^2}{2}\right)_x - \varepsilon u_{xx}=\pi \sin (\pi x) \cos (\pi x), \quad x \in[0, 1]\nonumber\\
& u(x, 0)=u_0(x),\label{eq:1D_Burgers} \\
& u(0, t)=u(1, t)=0, \nonumber 
\end{align}
with Dirichlet boundary conditions, where $u_0 \in L_{0}^2((0, 1); \mathbb{R})$ is the initial condition and $\varepsilon \in \mathbb{R}$ is the viscosity coefficient. We aim to learn the operator mapping the initial condition to the steady-state solution, $G^{\dagger}: L_{0}^2((0, 1); \mathbb{R}) \rightarrow H_{0}^r((0, 1); \mathbb{R})$, defined by $u_0 \mapsto u_{\infty}$ for any $r > 0$. As shown in Theorem 2.2 of \cite{KREISS1986161} and Theorems 2.5 and 2.7 of \cite{hao2019convergence}, for any $\varepsilon > 0$, the steady-state solution is independent of the initial condition, with a single shock occurring at $x_s = 0.5$. Here, we use DeepONet~\cite{lu2021deeponet} as the network architecture. 
The homotopy definition, similar to ~\cref{eq.hom_2D_AC}, can be found in \cref{Ap:operator}. The results can be found in \cref{fig:Burgers_result} and \cref{tab:loss_burgers}. Experimental results show that the homotopy dynamics strategy performs well in the operator learning setting as well.


\begin{table}[htbp!]
    \caption{Comparison of loss between classical training and homotopy dynamics for different values of $\varepsilon$ in the Burgers equation, along with the MSE distance to the ground truth shock location, $x_s$.}
    \vskip 0.15in
    \centering
    \tiny
    \begin{tabular}{|c|c|c|c|c|} 
    \hline  
    $ $ & $\varepsilon = 0.5$ & $\varepsilon = 0.1$ & $\varepsilon = 0.05$ \\ \hline 
    Homotopy Loss $L_H$                &  7.55e-7     & 3.40e-7     & 7.77e-7       \\ \hline 
    L2RE                      & 1.50e-3     & 7.00e-4     & 2.52e-2       \\ \hline
        MSE Distance $x_s$                      & 1.75e-8     & 9.14e-8      & 1.2e-3      \\ \hline
    \end{tabular}
    % On convection, \al{} provides 14.2$\times$ and 1.97$\times$ improvement over Adam or \lbfgs{} on L2RE. 
    % On reaction, \al{} provides 1.10$\times$ and 1.99$\times$ improvement over Adam or \lbfgs{} on L2RE.
    % On wave, \al{} provides 6.32$\times$ and 6.07$\times$ improvement over Adam or \lbfgs{} on L2RE.}
    \label{tab:loss_burgers}
\end{table}



% \begin{itemize}
%     \item Relate the curvature in the problem to the differential operator. Use this to demonstrate why the problem is ill-conditioned
%     \item Give an argument for why using Adam + L-BFGS is better than just using L-BFGS outright. The idea is that Adam lowers the errors to the point where the rest of the optimization becomes convex \ldots
%     \item Show why we need second-order methods. We would like to prove that once we are close to the optimum, second-order methods will give condition-number free linear convergence. Specialize this to the Gauss-Newton setting, with the randomized low-rank approximation.
%     % \item Show that it is not possible to get superlinear convergence under the interpolation assumption with an overparameterized neural network. This should be true b/c the Hessian at the optimum will have rank $\min(n, d)$, and when $d > n$, this guarantees that we cannot have strong convexity.
% \end{itemize}
\section{Experiments: Planning outperforms Heuristics}
\label{sec:experiment}

We begin our empirical demonstrations by showcasing the effectiveness of our planning framework on both synthetic and real datasets. We focus on the simplest planning algorithm, 1-step lookaheads (Algorithm~\ref{alg:complete}), and show that even basic planning can hold great promise. 
We illustrate our framework using two uncertainty quantification modules---GPs and 
\ensembles/ \ensembleplus. 

Throughout this section, we focus on evaluating the mean squared error of 
a regression model $\model$,  and develop adaptive policies that minimize uncertainty on $g(f)$ defined in~\eqref{eqn:l2-g-f}.
When GPs provide a valid model of uncertainty, 
our experiments show that our planning framework significantly outperforms other baselines. 
We further demonstrate that our conceptual framework extends to deep learning-based uncertainty quantification methods such as  \ensembleplus while highlighting computational challenges that need to be resolved in order to scale our ideas. 
For simplicity, we assume a naive predictor, i.e., $\psi(\cdot) \equiv 0$. However, we emphasize that this problem is just as complex as if we were using a sophisticated model $\psi(.)$. The performance gap between the algorithms 
primarily depends
on the level  of uncertainty in our prior beliefs.

To evaluate the performance of our algorithm, we benchmark it against several baselines. 
%Active learning baselines use an acquisition function $\ac$ to select points that have the highest   function value: $X\opt_t \in \argmax_{X \in \xpoolj{t}} \ac({X})$ at every step $t$. These methods may also need an UQ module, which we simply use the same UQ module as in our algorithm, and it  outputs $V(X)$ that measures the the uncertainty of each point $X \in \xpoolj{t}$.
Our first set of baselines are from active learning~\citep{AggarwalKoGuHaPh14}:
\\ % \noindent\textbf{Active Learning Heuristics:} 
\textbf{(1)} 
\textsf{Uncertainty Sampling (Static):}  In this approach, we query the samples for which the model is least certain about. Specifically, we estimate the variance of the latent output $f(X)$ for each $X \in \xpool$ using the UQ module and select the top-$K$ points with the highest uncertainty. \\
\textbf{(2)} \textsf{Uncertainty Sampling (Sequential):} This is a greedy heuristic that sequentially selects the points with the highest uncertainty within a batch, while updating the posterior beliefs using pseudo labels from the current posterior state. Unlike \textsf{Uncertainty Sampling (Static)}, this method takes into account the information gained from each point within batch, and hence tries to diversify the selected points within a batch. 

 
We also compare our approach to the  \textbf{(3)} \textsf{Random Sampling}, which selects each batch uniformly at random from the pool. Additionally, we compare solving the planning problem using  \textsf{REINFORCE}-based policy gradients with   $\mathsf{Smoothed\text{-}Autodiff}$ policy gradients.\footnote{Our code repository is available at
  \url{https://github.com/namkoong-lab/adaptive-labeling}.}
%Detailed experimental setups are provided in Section \ref{sec:details-experiments}.

%We repeat all experiments with 10 random seeds.




\begin{figure}[t]
\centering
\begin{minipage}[b]{0.49\textwidth}
\centering
\includegraphics[width=\textwidth, height=5cm]{figures/original_scale/Var_of_l_2_loss.pdf}
\caption{(Synthetic data) Variance of mean squared loss evaluated through the posterior belief $\mu_t$ at each horizon $t$. This is the objective that policy gradient methods like \textsf{REINFORCE} and $\ouralgo$ optimizes. 1-step lookaheads are surprisingly effective even in long horizons.}
\label{fig:var-l2-sim}
\end{minipage}
\hfill
\begin{minipage}[b]{0.49\textwidth}
\centering \includegraphics[width=\textwidth, height=5cm]{figures/original_scale/Error_of_estimated_model_l_2_loss.pdf}
\caption{(Synthetic data) Error between MSE calculated based on collected data $\mc{D}^{0:T}$ vs. population oracle MSE over $\mc{D}_{\rm eval} \sim P_X$. Reducing uncertainty over posteriors directly leads to better OOD evaluations. 1-step lookaheads significantly outperform active learning heuristics in small horizons.}
\label{fig:mean-l2-sim}
\end{minipage}
%\caption{Simulated data for GPs}
%\label{fig:both_plots}
\end{figure}

\subsection{Planning with Gaussian processes}
\label{sec:experiment-plan-GP}
We now briefly describe the data generation process for the GP experiments,  deferring a more detailed discussion of the dataset generation to Section~\ref{sec:details-experiments}. 
We use both the synthetic data and the real data to test our methodology.
For the \emph{simulated data},  we construct a setting where the general population is distributed across \emph{51 non-overlapping clusters} while the initial labeled data $\dtrain$ just comes from one cluster. In contrast, both $\dpool \defeq (\xpool,\ypool),\deval \defeq (\xeval,\yeval)$ are generated   from all the clusters. 
We begin with a low-dimensional scenario, generating a one-dimensional regression setting using a GP. %Gaussian Process (GP).
Although the data-generating process is not known to the algorithms,  we assume that the GP hyperparameters are known to all the algorithms
to ensure fair comparisons. This can be viewed as a setting where our prior is well-specified, allowing us to isolate the effects
of different policy optimization approaches
 without any concerns about the misspecified priors. We select $10$ batches, each of size $K=5$ across $T = 10$ time horizons.

To examine the robustness of our method against the distributional assumptions made  in the simulated case, we then move to a real dataset where the correct prior is not known. We simulate selection bias from the eICU dataset~\citep{PollardJoRaCeMaBa18}, which contains real-world patient data with in-hospital mortality outcomes. 
We conduct a $k$-means clustering to generate 51 clusters and then select data from those clusters. We view this to be a credible replication of practice, as severe distribution shifts are common due to selection bias in clinical labels.  To convert the binary mortality labels into a regression setting, we train a  random forest classifier and fit a GP on predicted scores, which serves as the UQ module for all the algorithms. As before, the task is to select 10 batches, each consisting of 5 samples, across 10 time horizons.

 In Figures~\ref{fig:var-l2-sim} and~\ref{fig:mean-l2-sim}, we present results for the simulated data. 
Figure~\ref{fig:var-l2-sim} shows the variance of $\ell_2$ loss, and Figure~\ref{fig:mean-l2-sim} presents the error in the estimated $\ell_2$ loss using $\mu_t$ (relative to true $\ell_2$ loss, that is unknown to the algorithm). 
As we can see from these plots, our method one-step lookahead  gives substantial improvements  over active learning baselines and random sampling. In addition,
compared to the one-step lookahead planning approach using \textsf{REINFORCE}-based policy gradients, 
we observe that $\mathsf{Smoothed\text{-}Autodiff}$-based policy gradients provide significantly more robust performance over all horizons.

In Figures~\ref{fig:var-l2-real}~and~\ref{fig:mean-l2-real}, we observe similar findings on the eICU data. We see that planning policies (\textsf{REINFORCE} and $\mathsf{Smoothed\text{-}Autodiff}$) consistently outperform other heuristics by a large margin.  Active learning baselines perform poorly in these small-horizon batched problems and can sometimes be even worse than the random search baselines.  Overall, our results show the importance of careful planning in adaptive labeling for reliable model evaluation. 

We offer some intuition as to why one-step lookahead planning may outperform other heuristic algorithms. 
 First,  \textsf{Uncertainty sampling (Static)} while myopically selects the
 top-$K$ inputs with the highest uncertainty, it fails to consider 
the overlap in information content among the ``best” instances; see \citep{AggarwalKoGuHaPh14} for more details. 
In other words,  it might acquire points from the same region with high uncertainty while failing to induce diversity among the batch.
Although \textsf{Uncertainty Sampling (Sequential)} somewhat addresses the issue of information overlap, a significant drawback of 
this algorithm
is the disconnect between the objective we aim to optimize and the algorithm. For example, it might sample from a region with high uncertainty but very low density. 

\begin{figure}[t]
\centering
\begin{minipage}[b]{0.48\textwidth}
\centering
\includegraphics[width=\textwidth, height=5cm]{figures/original_scale/Var_of_l_2_loss_real.pdf}
\caption{(Real-world eICU data) Variance of mean squared loss evaluated through the posterior belief $\mu_t$ at each horizon $t$. Even 1-step lookaheads are extremely effective planners, and auto-differentiation-based pathwise policy gradients provide a reliable optimization algorithm based on low-variance gradient estimates.}
\label{fig:var-l2-real}
\end{minipage}
\hfill
\begin{minipage}[b]{0.48\textwidth}
\centering \includegraphics[width=\textwidth, height=5cm]{figures/original_scale/Error_of_estimated_model_l_2_loss_real.pdf}
\caption{(Real-world eICU data) Error between MSE calculated based on collected data $\mc{D}^{0:T}$ vs. population oracle MSE over $\mc{D}_{\rm eval} \sim P_X$. Reducing uncertainty over posteriors directly leads to better OOD evaluations. Our method significantly outperforms active learning-based heuristics, and random sampling.}
\label{fig:mean-l2-real}
\end{minipage}
%\caption{Real data for GPs}
\end{figure}
 
%\vspace{-1.5cm}
% \begin{wrapfigure}{r}{.32\columnwidth}
%   \vspace{-.5cm} 
%   \centering
% \includegraphics[scale=.29]{figures/Var of l2l_2 loss.pdf}
%   \vspace{-0.2cm}
%   \caption{Results of GP}
% \label{fig:var-l2-gp}
%   \vspace{-0.1cm}
% \end{wrapfigure}


% Attempts have been made  in the past to address these  drawbacks heuristically  (see \citep{AggarwalKoGuHaPh14}). We give a unified computational framework while approaching the problem in a more principled manner and solving it more optimally.




\subsection{Planning with  neural network-based uncertainty quantification methods ($\ensembleplus$)}


We now provide a proof-of-concept that shows the generalizability of our conceptual framework  to the deep learning-based UQ modules, specifically focusing on $\ensembleplus$ due to their previously observed superior performance~\citep{OsbandWenAsDwIbLuRo23}. Recall that implementing our framework with deep learning-based UQ modules  requires us to retrain the model across multiple possible random actions $\bm{a}(\theta)$ sampled from the current policy $\pi_\theta$.
This requires significant computational resources, in sharp contrast to the GPs where the posteriors are in closed form and can be readily updated and differentiated. 

Due to the computational constraints, we test $\ensembleplus$ on a toy setting to demonstrate the generalizability of our framework. We consider a setting where the general population consists of four clusters, while the initial labeled data only comes from one cluster. Again we generate data using GPs.  The task is to select a batch of 2 points in one horizon. We detail the $\ensembleplus$ architecture in Section \ref{sec:details-experiments}, and we assume prior uncertainty to be large (depends on the scaling of the prior generating functions). 
The results are summarized in the Table~\ref{tab:UQ_ensemble}.

% \begin{table}[H]
% \vspace{-10pt}
% \caption{Performance under \ensembleplus as UQ module}
%     \centering
%     \begin{tabular}{|m{3cm}|m{2.5cm}|m{2cm}|} 
%     \hline
%       Algorithm   & Variance of $\loss_2$ loss estimate & Error of $\loss_2$ loss estimate  \\ \hline Random Sampling 
%          & $1710.9 \pm 1352.1$ & $8.67\pm6.62$ 
%       \\ \hline \ouralgo & $1.30 \pm 0.68$ & $0.91\pm0.25$ \\ \hline
%     \end{tabular}
%     \label{tab:UQ_ensemble}
%     %\vspace{-10pt}
% \end{table}




\begin{table}[h]
\vspace{-10pt}
\caption{Performance under \ensembleplus as the UQ module}
\centering
\begin{tabular}{|l|l|l|}
\hline
Algorithm   & Variance of $\loss_2$ loss estimate & Error of $\loss_2$ loss estimate  \\
\hline
\textsf{Random sampling} & 7129.8 $\pm$ 1027.0 & 136.2 $\pm$ 8.28 \\ \hline
\textsf{Uncertainty sampling (Static)} & 10852 $\pm$ 0.0 & 162.156 $\pm$ 0.0 \\ \hline
\textsf{Uncertainty sampling (Sequential)} & 8585.5 $\pm$ 898.9 & 144 $\pm$ 6.93 \\ \hline
\textsf{REINFORCE} & 1697.1 $\pm$ 0.0 & 45.27 $\pm$ 0.0 \\ \hline
\ouralgo & 1697.1 $\pm$ 0.0 & 45.27 $\pm$ 0.0 \\ \hline
\end{tabular}
%\caption{Comparison of different algorithms based on variance   and   error in $\ell_2$ loss estimation with Ensemble $+$ as the UQ module. Our results demonstrate that {\ouralgo} and REINFORCE outperformthe other active learning based heuristics, confirming the benefits of our MDP formulation for the adaptive labeling problem, as also demonstrated in Section 4.\\
%\footnotesize{Experimental details: We use Gaussian Processes as our data generating process, GP parameters are the same as in Section D.3.  The task is to select a batch of 2 points along one horizon.The marginal distribution $p_X$ has 4 \textit{non-overlapping} clusters. Initial data comes from one cluster, while pool and evaluation points comes from all the clusters. We have $20$ initial labeled data points, $10$ pool points, and $252$ evaluation points.  Training procedures are similar to the one in Section D.3.} }
\label{tab:UQ_ensemble}
\end{table}



% We faced  issues in scaling up these experiments which will be our focus in the future. 





% \begin{itemize}
%     \item Posteriors should be consistent. Two dimensions: even with less training,  
%     \item the inference should be  fast enough
% \end{itemize}


% Potential research directions for uncertainty quantification

% In this section we consider a simple setting We consider a simpler setting and 


% For synthetic dataset generation, we use ...... For real datasets, we use ...... We compare our methodolgy to several baselines ()    This Section is structured as follows:
% \begin{itemize}
%     \item \textbf{GPs, square loss objective} (Section \ref{}): 
%     %the broad aim of the experiments  in this section is to isolate the performance of our methodology without any concerns for the inefficiencies induced due to a mis-specified prior or imperfect posterior inference. To accomplish this we generate synthetic datasets using GPs (detailed later). We use the well specified prior (GPs - with same hyperparameter setting) as our UQ module.   
%      As GPs provide differentaible posterior inference - any errors induced due to imperfect posterior updates are also isolated. We note that under this setting
%      \item In Section\ref{} we demonstrate why our methodology performs better than other baselines - by devising various synthetic experiments ()
%     \item  \textbf{UQ Benchmarking }(Section \ref{}): Before diving into the experiments using $\ensembleplus$ and ENNs,  we showcase our benchmarking experiments in Section \ref{}. We use real datasets We observe that ENNs perform better
%      \item \textbf{Ensemble $+$}, objective: recall, accuracy
%     \item \textbf{ENN}, objective: recall, accuracy
% \end{itemize}




% In Section {}, we test 
% \subsection{Experimental details}

% \begin{itemize}
%     \item UQ methodologies - GPs, ENNs
%     \item Objectives - Recall,  ATE
%     \item Datasets - ATE-synthetic datasets, Recall-synthetic, real datasets
%     \item Baselines - 
%     \begin{itemize}
%         \item Random sampling
%         \item Active learning - Uncertainty based sampling - In regression setting almost all of the 
%         \item Myopic greedy - Greedy Batch based sampling
%         \item Policy Gradient
%     \end{itemize}
    
% \end{itemize}

% \subsection{Experiments}
%     \begin{itemize}
%     \item GPs with square loss
%     \item Benchmarking ENN
%         \item ENNs with ATE
%         \item ENNs with Recall
%     \end{itemize}

% \subsection{Benefits over other algorithms - intuition and experiments}

%Active learning - Myopic greedy / Don't rely on the objective rather some entropy version.


%%% Local Variables:
%%% mode: latex
%%% TeX-master: "main"
%%% End:

\section{Imitation Learning experiments}\label{sec:imitation}
One of the key contributions of this paper is the characterization of the relationship between imitation learning (IL) and inverse reinforcement learning (IRL)/Dynamic Discrete Choice (DDC) model, particularly through the ERM-IRL/DDC framework. Given that much of the IRL literature has historically focused on providing experimental results for IL tasks, we conduct a series of experiments to empirically validate our theoretical findings. Specifically, we aim to test our
prediction in Section \ref{sec:ERM-IRL} that \textit{behavioral cloning (BC) should outperform ERM-IRL for IL tasks}, as BC directly optimizes the negative log-likelihood objective without the additional complexity of Bellman error minimization. By comparing BC and ERM-IRL across various IL benchmark tasks, we demonstrate that BC consistently achieves better performance in terms of both computational efficiency and policy accuracy, reinforcing our claim that IL is a strictly easier problem than IRL.

\subsection{Experimental Setup}

As in \citet{garg2021iq}, we employ three OpenAI Gym environments for algorithms with discrete actions \citep{brockman2016openai}: Lunar Lander v2, Cartpole v1, and Acrobot v1. These environments are widely used in IL and RL research, providing well-defined optimal policies and performance metrics. 

\noindent \textbf{Dataset.}  
For each environment, we generate expert demonstrations using a pre-trained policy. We use publicly available expert policies\footnote{\url{https://huggingface.co/sb3/}} trained via Proximal Policy Optimization (PPO) \cite{schulman2017proximal}, as implemented in the Stable-Baselines3 library \citep{raffin2021stable}. Each expert policy is run to generate demonstration trajectories, and we vary the number of expert trajectories across experiments for training. For all experiments, we used the expert policy demonstration data from 10 episodes for testing.

\noindent \textbf{Performance Metric.}  
The primary evaluation metric is \% optimality, defined as:
\begin{align}
    \text{\% optimality of an episode} := \frac{\text{One episode's episodic reward of the algorithm}}{\text{Mean of 1,000 episodic rewards of the expert}} \times 100. \notag
\end{align}
For each baseline, we report the mean and standard deviation of 100 evaluation episodes after training. A higher \% optimality indicates that the algorithm's policy closely matches the expert. The 1000-episodic mean and standard deviation ([mean$\pm$std]) of the episodic reward of expert policy for each environment was $[232.77\pm73.77]$ for Lunar-Lander v2 (larger the better), $[-82.80\pm27.55]$ for Acrobot v1 (smaller the better), and $[500\pm 0]$ for Cartpole v1 (larger the better).

\noindent \textbf{Training Details.}  
All algorithms were trained for 5,000 epochs. Since our goal in this experiment is to show superiority of BC for IL tasks, we only include ERM-IRL and IQ-learn \cite{garg2021iq} as baselines. Specifically, we exclude baselines such as Rust \citep{rust1987optimal} and ML-IRL \citep{zeng2023understanding}, which require explicit transition probability estimation.

\subsection{Experiment results}


%\noindent \textbf{GLADIUS (Ours)}  
%The ERM-IRL framework minimizes both the negative log-likelihood (NLL) and Bellman error (BE) terms, making it computationally more complex than BC. 

%\noindent \textbf{IQ-Learn} \citep{garg2021iq}  
%A popular \cite{rafailov2024r} IRL method that minimizes an occupancy-matching objective, i.e., it does not enforce Bellman consistency. For details, refer to Section \ref{sec:ImitationID}.

%\noindent \textbf{Behavioral Cloning (BC)}  
%BC minimizes only the NLL term, making it computationally simple and sample-efficient.

Table \ref{fig:OpenAI_gym} presents the \% optimality results for Lunar Lander v2, Cartpole v1, and Acrobot v1. As predicted in our theoretical analysis, BC consistently outperforms ERM-IRL in terms of \% optimality, validating our theoretical claims.


\begin{table*}[ht]
    \centering
    \scalebox{0.85}{
    \begin{tabular}{l
            >{\centering\arraybackslash}p{1.5cm}
            >{\centering\arraybackslash}p{1.5cm}
            >{\centering\arraybackslash}p{1.5cm}
            >{\centering\arraybackslash}p{1.5cm}
            >{\centering\arraybackslash}p{1.5cm}
            >{\centering\arraybackslash}p{1.5cm}
            >{\centering\arraybackslash}p{1.5cm}
            >{\centering\arraybackslash}p{1.5cm}
            >{\centering\arraybackslash}p{1.5cm}}
\toprule
\multirow{4}{*}{\parbox{0.5cm}{Trajs}} 
& \multicolumn{3}{c}{\makecell{\ul{Lunar Lander v2 (\%)} \\ (Larger \% the better)}} 
& \multicolumn{3}{c}{\makecell{\ul{Cartpole v1 (\%)} \\ (Larger \% the better)}} 
& \multicolumn{3}{c}{\makecell{\ul{Acrobot v1 (\%)} \\ (Smaller \% the better)}} \\
\cmidrule(lr){2-4} \cmidrule(lr){5-7} \cmidrule(lr){8-10}
& {\ul{Gladius}} & IQ-learn & BC 
& {\ul{Gladius}} & IQ-learn & BC
& {\ul{Gladius}} & IQ-learn & BC \\
\midrule
1  & \textbf{107.30 }& 83.78 & 103.38   & 100.00  & 100.00  & 100.00     & 103.67  & 103.47  & \textbf{100.56}  \\
    & (10.44)  & (22.25)  & (13.78)   & (0.00)  & (0.00)  & (0.00)   & (32.78)  & (55.44)  & (26.71)  \\
\midrule
3   & \textbf{106.64}  &  102.44 & 104.46 & 100.00  & 100.00  & 100.00 & 102.19  & 101.28  & \textbf{101.25}  \\
    & (11.11)  & (20.66)   & (11.57)  & (0.00)  & (0.00)  & (0.00)  & (22.69)  & (37.51)  & (36.42)  \\
\midrule
7   &  101.10 & 104.91  & \textbf{ 105.99}  & 100.00  & 100.00  & 100.00 & 100.67 & 100.58  &\textbf{98.08} \\
    & (16.28)  & (13.98) &  (10.20) & (0.00)         & (0.00)  & (0.00)  & (22.30)      & (30.09)  &  (24.27)  \\
\midrule
10  & 104.46  & 105.13   & \textbf{ 107.01}   & 100.00  & 100.00  & 100.00  & 99.07  & 101.10 &  \textbf{97.75}\\
    & (13.65)  & (13.83)   & (10.75)  & (0.00)           & (0.00)  &  (0.00) &  (20.58)    & (30.40)  & (16.67)  \\
\midrule
15  & 106.11  &  106.51 & \textbf{107.42}  & 100.00  & 100.00  &100.00 & 96.50  & 95.34  & \textbf{95.33}  \\
    & (10.65)  & (14.10)  & (10.45)  & (0.00)         & (0.00)  &  (0.00)  & (18.53)        & (26.92)  & (15.42)  \\
\bottomrule
\multicolumn{10}{l}{\footnotesize 
Based on 100 episodes for each baseline. Each baseline was trained for 5000 epochs.}
\end{tabular}
    }
\caption{Mean and standard deviation of \% optimality of 100 episodes}
\label{fig:OpenAI_gym}   
\end{table*}

\iffalse

\subsection{Reward transfer experiments}
As we discussed in Section \ref{sec:Intro}, the main advantage of IRL over Imitation Learning (IL) is its premise that the learned rewards can be used for counterfactual simulations. That is, given the rewards we learn from IRL, we would expect that such learned rewards can be useful for RL training \cite{zeng2023understanding}. We use the Lunar Lander v2 from OpenAI Gym for the experiment. Specifically, for each number of expert trajectories we consider ([1, 3, 7, 10, 15]):
\begin{enumerate}[noitemsep]
    \item Train the reward function using GLADIUS for 5000 epochs.
    \item Use the trained reward function  for training a Proximal Policy Optimization (PPO) policy.
\end{enumerate}
As discussed in \cite{zeng2023understanding}, IQ-learn and BC performed no better than randomly generated reward for this task. Therefore, we don't include them for the comparison. 

\begin{table*}[ht]
    \centering
    \scalebox{0.9}{
    \begin{tabular}{l
            >{\centering\arraybackslash}p{1.5cm}
            >{\centering\arraybackslash}p{1.5cm}
            >{\centering\arraybackslash}p{1.5cm}
            >{\centering\arraybackslash}p{1.5cm}
            >{\centering\arraybackslash}p{1.5cm}}
\toprule
\multirow{2}{*}{\parbox{2.5cm}{\centering Lunar Lander v2}} 
& \multicolumn{5}{c}{\textbf{Number of Trajectories (Trajs)}} \\
\cmidrule(lr){2-6}
& \textbf{1} & \textbf{3} & \textbf{7} & \textbf{10} & \textbf{15} \\
\midrule
Mean  & 97.63 & 100.12 & 104.23 & 106.52 &  \\
Std   & (28.15)  & (29.12) & (24.57) &(19.88)  &  \\
\bottomrule
\multicolumn{6}{l}{\footnotesize Based on 1000 episodes for each baseline. Each baseline was trained 5000 epochs.}
\end{tabular}
    }
\caption{Mean and standard deviation of \% optimality for Lunar Lander v2 using GLADIUS}
\label{fig:LunarLander_Gladius}   
\end{table*}

\fi
\section*{Conclusion}
This paper aims to enhance our understanding of the computational complexity of computing various Shapley value variants. We found that for various ML models --- including decision trees, regression tree ensembles, weighted automata, and linear regression --- both local and global interventional and baseline SHAP can be computed in polynomial time under HMM modeled distributions. This extends popular algorithms, such as TreeSHAP, beyond their empirical distributional scope. We also establish strict complexity gaps between the various SHAP variants (baseline, interventional, and conditional) and prove the intractability of computing SHAP for tree ensembles and neural networks in simplified scenarios. Overall, we present SHAP as a versatile framework whose complexity depends on four key factors: \begin{inparaenum}[(i)] \item model type, \item SHAP variant, \item distribution modeling approach, \item and local vs. global explanations\end{inparaenum}. We believe this perspective provides deeper insight into the computational complexity of SHAP, paving the way for future work.




%We believe that our framework provides a more intricate understanding of SHAP computation complexity across different models, distributions, and variants, paving the way for further research.

Our work opens promising directions for future research. First, expanding our computational analysis to other SHAP-related metrics, such as asymmetric SHAP~\citep{frye20} and SAGE~\citep{covert2020understanding}, would be valuable. Additionally, we aim to explore more expressive distribution classes and relaxed assumptions beyond those in Section \ref{sec:tractable} while maintaining tractable SHAP computation. Finally, when exact computation is intractable (Section \ref{sec:intractable}), investigating the approximability of SHAP metrics through approximation and parameterized complexity theory~\citep{downey2012parameterized} is an important direction.

%Our work opens several promising avenues for future research on the computational properties of explainable AI methods, with a particular focus on SHAP. First, it would be interesting to broaden the computational analysis conducted in this work to include other popular SHAP-related metrics in the literature, such as asymmetric SHAP \cite{frye20} and SAGE \cite{covert2020understanding}. Also, in the future, we aim to explore more expressive distribution classes and relaxed distributional assumptions—extending beyond those examined in Section \ref{sec:tractable} —that still yield tractable SHAP computation. Finally, when exact computation proves intractable (Section \ref{sec:intractable}), it is worthwhile to theoretically investigate the question of the approximability of computing the SHAP metrics across various configurations, through the lens of approximation and parametrized complexity theory \cite{arora2009computational}.

%This paper aims to deepen our understanding of the computational complexity involved in obtaining different Shapley value variants. We found that for a variety of ML models, including decision trees, tree ensembles for regression, weighted automata, and linear regression models — computing both local and global interventional and baseline SHAP can be done in polynomial time when distributions are modeled by HMMs. This extends the distributional scope of popular algorithms like TreeSHAP, which is limited to empirical distributions. Additionally, we demonstrate a strict complexity gap between SHAP variants, showing that interventional and baseline SHAP can be strictly easier to compute than conditional SHAP. Despite these positive results, we uncovered intractability for various SHAP variants in neural networks and tree ensembles. Finally, we provided generalized complexity relations across SHAP variants. We believe that our framework offers a deeper understanding of the complexity involved in computing SHAP across various variants, models, distributions, as well as in both local and global computations, laying the groundwork for future research.
\input

\bibliographystyle{plainnat}
\bibliography{bib}

\appendix

\section{Experiments}
\label{app:experiments}

\paragraph{Implementation details.} Our code is based on Pytorch \citep{pytorch} and Huggingface \cite{huggingface-transformers}. Experiments were ran on NVidia 80GB A100 GPUs. BLEU is computed using the SacreBLEU \citep{sacrebleu} implementation. For NLI metric we use the model available at \url{https://huggingface.co/cross-encoder/nli-deberta-v3-large}.
% \subsection{Training dynamics}
% We plot on the training dynamics for three values of alpha for E2E, FeTaQa, WebNLG on \Cref{fig:train_alpha_e2e,fig:train_alpha_fetaqa} respectively. We observe the same pattern than the one described in \Cref{sec:alpha-effect}.

% \subsection{Effect of $\alpha$ on BLEU}
% \Cref{fig:bleu_alpha} shows that the fluency grows with $\alpha$, acknowledging the effect observed during training.

\subsection{Baselines}\label{app:baselines}
For each baseline we choose the best hyper-parameters by conducting a grid-search. We initially conducted the search over ranges disclosed in original publications and refined based on our own experiments.

\subsubsection{Context-aware Decoding} \Cref{tab:cad-hyperparameter} shows the best hyperparameters for \cad method. In the original paper, it is recommended to select \( \alpha \) between 0 and 1, with 0.5 being a suitable choice.
\begin{table}[ht]
\centering

\begin{tabular}{lccccccc}
    & \textbf{ToTTo} & \textbf{FeTaQA} & \textbf{WebNLG} & \textbf{E2E} & \textbf{SAMSum} & \textbf{XSum} & \textbf{PubMed} \\
    & $\alpha$ & $\alpha$ & $\alpha$ & $\alpha$ & $\alpha$ & $\alpha$ & $\alpha$ \\
    \midrule
    \textsc{Llama-2-7B} & 0.01 & 0.05 & 0.03 & 0.03 & 0.05 & 0.40 & 0.40 \\
    \textsc{Llama-13B}   & 0.01 & 0.01 & 0.07 & 0.01 & 0.3 & 0.10 & 0.40 \\
    \textsc{Mistral-7B}  & 0.01 & 0.09 & 0.01 & 0.04 & 0.04 & 0.30 & 0.20 \\
    \midrule
\end{tabular}

\caption{Best Context-Aware Decoding (\cad) $\alpha$ hyperparameter.}
\label{tab:cad-hyperparameter}
\end{table}

\subsubsection{Pointwise Mutual Information}
\Cref{tab:pmi-hyperparameter} shows best hyperparameters for \pmi method.
\begin{table}[h!]
\centering
\resizebox{\columnwidth}{!}{%
\begin{tabular}{lccccccc}
    & \textbf{ToTTo} & \textbf{FeTaQA} & \textbf{WebNLG} & \textbf{E2E} & \textbf{SAMSum} & \textbf{XSum} & \textbf{PubMed} \\
    & $(\lambda, \tau)$ & $(\lambda, \tau)$ & $(\lambda, \tau)$ & $(\lambda, \tau)$ & $(\lambda, \tau)$ & $(\lambda, \tau)$ & $(\lambda, \tau)$ \\
    \midrule
    \textsc{Llama-2-7B} & (0.07, 3.25) & (0.07, 3.25) & (0.05, 3.5) & (0.06, 3.25) & (0.20, 3.25) & (0.15, 3.25) & (0.20, 3.25) \\
    \textsc{Llama-13B}   & (0.07, 3.25) & (0.05, 3.25) & (0.05, 3.25) & (0.05, 3.40) & (0.15, 3.25) & (0.10, 3.75) & (0.15, 3.25) \\
    \textsc{Mistral-7B}  & (0.06, 3.5)  & (0.07, 3.25) & (0.05, 3.25) & (0.09, 3.25) & (0.05, 3.25) & (0.20, 3.25) & (0.15, 3.25) \\
    \midrule
\end{tabular}

}
\caption{Best PMI Decoding (\pmi) $(\lambda, \tau)$ hyperparameters.}
\label{tab:pmi-hyperparameter}
\end{table}


\subsubsection{Critic-driven Decoding}
For the classifier, we replace the original XLM-RoBERTa-base \citep{xlm-roberta} with a stronger DebertaV3-large \citep{deberta} model allowing for much larger contexts, since the linearized data did not fit in the context-window of XLM-RoBERTa-base. In our experiments, we trained a classifier on each dataset using the method \textit{"base with full sentences"} reported to give the highest NLI score on WebNLG dataset in the original publication.
\Cref{tab:critic-hyperparameter} shows the best hyperparameters for the method.

\begin{table}[h!]
\centering
%\resizebox{\columnwidth}{!}{%
\begin{tabular}{lccccccc}
    & \textbf{ToTTo} & \textbf{FeTaQA} & \textbf{WebNLG} & \textbf{E2E} & \textbf{SAMSum} & \textbf{XSum} & \textbf{PubMed} \\
    & $\lambda$ & $\lambda$ & $\lambda$ & $\lambda$ & $\lambda$ & $\lambda$ & $\lambda$ \\
    \midrule
    \textsc{Llama-2-7B} & 0.02 & 0.03 & 0.01 & 0.01 & 0.07 & 0.25 & 0.25 \\
    \textsc{Llama-13B}   & 0.03 & 0.07 & 0.01 & 0.06 & 0.25 & 0.10 & 0.75 \\
    \textsc{Mistral-7B}  & 0.05 & 0.01 & 0.01 & 0.10 & 0.05 & 0.75 & 0.50 \\
    \midrule
\end{tabular}

%}
\caption{Best Critic-driven Decoding (\critic) $\lambda$ hyperparameter.}
\label{tab:critic-hyperparameter}
\end{table}

\subsection{Hyperparameters}

\paragraph{\scope $\alpha$.} Selected value of $\alpha$ for \scope for each dataset are presented in \Cref{tab:scope-hyperparameter}.
\begin{table}[h!]
\centering

\begin{tabular}{lccccccc}
    & \textbf{ToTTo} & \textbf{FeTaQA} & \textbf{WebNLG} & \textbf{E2E} & \textbf{SAMSum} & \textbf{XSum} & \textbf{PubMed} \\
    & $\alpha$ & $\alpha$ & $\alpha$ & $\alpha$ & $\alpha$ & $\alpha$ & $\alpha$ \\
    \midrule
    \textsc{Llama-2-7B} & 0.5 & 0.5 & 0.5 & 0.6 & 0.5 & 0.5 & 0.4 \\
    Llama-2-13B & 0.4 & 0.4 & 0.4 & 0.5 & 0.5 & 0.5 & 0.4 \\
    \textsc{Mistral-7B} & 0.5 & 0.5 & 0.5 & 0.6 & 0.6 & 0.5 & 0.4 \\
    \midrule
\end{tabular}


\caption{Best \scope value of $\alpha$ for \textsc{Llama-2-7b} and \textsc{Mistral-7b} on ToTTo, FeTaQA, WebNLG, and E2E.}
\label{tab:scope-hyperparameter}
\end{table}

\paragraph{Full \sft training.}
\begin{itemize}
    \item \textsc{Llama-2-7b.} The \sft version of \textsc{Llama-2-7b} where fine-tuned using a batch size of 16, a learning rate of $2\times 10^{-5}$, using a linear scheduler with a warm-up ratio of 0.1 on all datasets. The model is optimized with Adam optimizer.
    \item \textsc{Mistral-7b.} We used a batch size of 16, a learning rate of $2\times 10^{-6}$ using a linear scheduler with a warm-up ratio of 0.1 on all datasets. The model is optimized with Adam optimizer.
\end{itemize}

\paragraph{\scope training.}
\begin{itemize}
    \item \textbf{Training }$\bm{p_{\theta_0}}$\textbf{ on }$\bm{\dataset_1}$. For training the fine-tuned version of each model on the split $\dataset_1$, we used the exact same setting than for the full \sft training described above, except that we only performed one epoch for \textsc{Llama-2-7b} and two epochs for \textsc{Mistral-7b}.
    \item \textbf{Preference tuning.} Regarding the hyperparameter of \Cref{eq:dpo}, we set $\beta = 0.1$ for all models and datasets.
\end{itemize}

\subsection{Fine-tuning on half datasets}
\label{app:half_ft}
When fine-tuned on half the samples, we observe experimentally that the models have very close performances to the model fine-tuned on the full train set, see \Cref{tab:sft-05-d2t,tab:sft-05-summ}. The models fine-tuned on half the samples are therefore a strong initialization for the subsequent stages of the method.
\label{app:sft-05}
\begin{table}[h]
\small
\centering
\begin{tabular}{lcccccccc}
    & \multicolumn{2}{c}{\textbf{WebNLG}} & \multicolumn{2}{c}{\textbf{ToTTo}} & \multicolumn{2}{c}{\textbf{E2E}} & \multicolumn{2}{c}{\textbf{FetaQA}} \\
    \cmidrule(lr){2-3} \cmidrule(lr){4-5} \cmidrule(lr){6-7} \cmidrule(lr){8-9}
    & \textbf{NLI} & \textbf{PARENT} & \textbf{NLI} & \textbf{PARENT} & \textbf{NLI} & \textbf{PARENT}  & \textbf{NLI} & \textbf{PARENT}\\
    \midrule
    \textsc{Llama-2-7b}  \\
    \midrule
    \sft on $\dataset_1$ & 86.6 & 72.0 & 45.6 & 80.4 & 80.9 & 82.2 &36.2 & 76.6 \\
    \sft on $\dataset$ & 87.4 & 82.1 & 46.0 & 80.2 & 87.4 & 86.9 & 37.5 & 77.1 \\

    \midrule
    \textsc{Mistral-7b} \\
    \midrule
    \sft on $\dataset_1$ & 87.2 & 81.6 & 46.5 & 80.3 & 87.0 & 87.4 & 34.1 & 74.6\\
    \sft on $\dataset$ & 87.5 & 81.9 & 46.7 & 80.1 & 86.5 & 85.2 & 34.1 & 74.8 \\
    \bottomrule
\end{tabular}

\caption{Results are on the validation sets. NLI Score and PARENT for models fine-tuned on half of the training set of a data-to-text datasets. On average, the score are slightly lower compared to models trained on the full dataset.}
\label{tab:sft-05-d2t}

\end{table}

\begin{table}[h]
\small
\centering
\begin{tabular}{lcccc}
    & \multicolumn{2}{c}{\textbf{XSum}} & \multicolumn{2}{c}{\textbf{SAMSum}}  \\
    \cmidrule(lr){2-3} \cmidrule(lr){4-5} 
    & \textbf{AlignScore} & \textbf{Rouge-L} &  \textbf{AlignScore} & \textbf{Rouge-L}\\
    \midrule
    \textsc{Llama-2-7b}  \\
    \midrule
    \sft on $\dataset_1$ & 56.2 & 33.8 & 80.5 & 43.2\\
    \sft on $\dataset$ &  56.4 & 35.2 & 82.6 & 45.2 \\

    \midrule
    \textsc{Mistral-7b} \\
    \midrule
    \sft on $\dataset_1$ & 57.3 & 35.1 & 81.9 & 44.7 \\
    \sft on $\dataset$ & 57.3 & 36.2 & 82.5 & 45.2\\
    \bottomrule
\end{tabular}

\caption{Results are on the validation sets. AlignScore and Rouge-L for models fine-tuned on half of the training set of a summarization datasets. Like for data-to-text generation, on average, the score are slightly lower compared to models trained on the full dataset.}
\label{tab:sft-05-summ}

\end{table}

\subsection{Ablation by varying the dataset proportions used in the first phase of fine-tuning}
Based on the observations in \Cref{app:half_ft}, we chose to use 50\% of the data for the first phase of fine-tuning given the considered datasets and tasks. Here, we present an ablation study on ToTTo. In this study, we fine-tuned a model on 25\% (resp. 75\%) of the dataset and preference-tuned on the remaining 75\% (resp. 25\%) with noisy samples. Results on the validation set, are shown in the table below. On automatic faithfulness metrics (NLI and PARENT), all splits yield comparable results, though a bit higher with a split of 50/50.
\begin{table}[h]
\centering
\begin{tabular}{lccc}
\toprule
\textbf{First phase trained on} &\textbf{NLI} & \textbf{PARENT} \\
\midrule
25\% &  49.57 & 86.08 \\
50\% &  50.64 & 86.34 \\
75\% & 49.07 & 84.10 \\
\bottomrule
\end{tabular}
\caption{NLI and PARENT scores on the validation set of ToTTo when varying the proportion used in the first phase of fine-tuning and using the remaining split for the second phase of preference tuning.}
\label{tab:ablation_ft}
\end{table}


\subsection{Preference loss}
We chose to use DPO \citep{dpo} for its seminal work and its widespread usage.  But our self-supervised framework has no dependency with DPO and should also work with other preference tuning approaches. We tested with ORPO \citep{orpo} and observed very similar results to DPO, see \Cref{tab:orpo-results}.
\label{app:orpo}

\begin{table}[h]
    \centering
    \begin{tabular}{lcc}
        \textbf{Method} & \textbf{NLI} & \textbf{PARENT} \\
        \hline
        \sft & 46.0 & 80.2 \\
        \scope with \textsc{Dpo} loss & 49.9 & 84.2 \\
        \scope with \textsc{Orpo} loss & 49.3 & 85.9 \\
        \hline
    \end{tabular}
    \caption{Results on the validation set of ToTTo with different preference optimization losses applied to \textsc{Llama-2-7b}.}
    \label{tab:orpo-results}
\end{table}


\subsection{Ablation on the value of $\beta$ in preference-tuning stage}
\Cref{tab:beta-totto-xsum} presents faithfulness metrics as we change the value of $\beta$ in the preference-tuning phase of \scope. In the original DPO paper \citep{dpo}, authors use a value $\beta=0.1$ which we found to also work well for \scope.
\begin{table}[h!]
\small
\centering

\begin{tabular}{lcccc}
    & \multicolumn{2}{c}{\textbf{ToTTo}} & \multicolumn{2}{c}{\textbf{XSum}} \\
    \cmidrule(lr){2-3} \cmidrule(lr){4-5}
    \textbf{\(\beta\)} & \textbf{PARENT} & \textbf{NLI} & \textbf{ROUGE-L} & \textbf{AlignScore} \\
    \midrule
    0.05 & 83.54 & 48.31 & 29.51 & 65.16 \\
    0.1  & \textbf{85.39} & \textbf{49.21} & 30.66 & \textbf{65.37} \\
    1    & 81.98 & 46.24 & 33.80 & 59.30 \\
    5    & 81.04 & 45.80 & \textbf{33.84} & 57.45 \\
    \bottomrule
\end{tabular}
\caption{The effect of different \(\beta\) values on performance for ToTTo and XSum tasks.}
\label{tab:beta-totto-xsum}
\end{table}


\subsection{\scope on instruction-tuned models}
\label{app:scope-alpaca}


We intentionally focused on a task-specific setup, targeting use cases where specialized models are most applicable. However, to explore \scope's performance in a general-purpose context, we conducted additional experiments. Specifically, we fine-tuned a Llama-2-7b model on the Alpaca instruction dataset and compared it to a model fine-tuned using the \scope pipeline. Both models were evaluated on our initial tasks, including data-to-text and summarization. As shown in \Cref{tab:alpaca-totto-xsum}, \scope continues to demonstrate consistent gains in faithfulness according to our metrics. However, these improvements are smaller than those observed for domain-specific models, suggesting that \scope is particularly effective in specialized contexts.
\begin{table}[h!]
\small
\centering
\begin{tabular}{lcccc}
    & \multicolumn{2}{c}{\textbf{ToTTo}} & \multicolumn{2}{c}{\textbf{XSum}} \\
    \cmidrule(lr){2-3} \cmidrule(lr){4-5}
    & \textbf{NLI} & \textbf{PARENT} & \textbf{AlignScore} & \textbf{Rouge-L} \\
    \midrule
    \textsc{Model} \\
    \midrule
    \sft   & 35.89 & 66.97 & 84.70 & \textbf{19.46} \\
    \scope & \textbf{37.81} & \textbf{68.69} & \textbf{86.59} & 16.97 \\
    \bottomrule
    \end{tabular}
\caption{On context-intensive tasks, \scope applied to generalist instruction-tuned models improves the faithfulness of the generation.}
\label{tab:alpaca-totto-xsum}


\end{table}

To ensure that these gains in faithfulness do not compromise reasoning capabilities, we benchmarked both models on tasks from the OpenLLM Leaderboard. The results indicate similar overall performance for both models, with \scope outperforming supervised fine-tuning (SFT) on tasks such as TruthfulQA, WinoGrande, and HellaSwag-tasks that require strong context comprehension rather than general knowledge. These results, presented in the appendix, reinforce our contributions. Nonetheless, a more comprehensive exploration of \scope's advantages in broader setups is left for future work.



\begin{table}[ht]
\centering
\small
\begin{tabular}{lccccc|c}
         & \textbf{ARC} & \textbf{HellaSwag} & \textbf{MMLU} & \textbf{TruthfulQA} & \textbf{Winogrande} & \textbf{Avg} \\
\midrule
\sft   & \textbf{47.61}             & 56.50                   & \textbf{41.06}      & 30.72                            & 70.96   & 49.37                 \\
\scope & 47.27                      & \textbf{57.07}           & 39.75               & \textbf{31.95}                   & \textbf{71.98}      & \textbf{49.60}       \\
\bottomrule
\end{tabular}
\caption{Performance comparison between \sft and \scope on various tasks. Scores are percentages, and the best result for each task is highlighted in bold. Metrics: Accuracy is used for ARC, HellaSwag, MMLU, and Winogrande, while BLEU-Acc is used for TruthfulQA to evaluate faithfulness to the reference responses.}
\label{tab:alpaca_sft_vs_scope}
\end{table}


\newpage
\section{Technical Proofs}

\subsection{Theory of TD correction using biconjugate trick}\label{sec:BiconjProofs}


\begin{proof}[Proof of Lemma \ref{lem:OurBiconj}]
      \begin{align}
     &\mathcal{L}_{BE}(s,a)(Q):=\mathbb{E}_{ s^\prime \sim P(s, a)}\left[\delta_{Q}\left(s,a, s^\prime\right)\mid s, a\right]^2\notag
     \\
     &=\max_{h\in \mathbb{R}}2\cdot\mathbb{E}_{ s^\prime \sim P(s, a)}\left[\delta_{Q}\left(s,a, s^\prime\right)\mid s, a\right]\cdot h-h^2\tag{Biconjugate}
     \\
     &=\max_{h\in \mathbb{R}}2\cdot\mathbb{E}_{ s^\prime \sim P(s, a)}\left[\hat{\mathcal{T}}Q - Q\mid s, a\right]\cdot \underbrace{h}_{=\rho-Q(s,a)}-h^2\notag
    \\
     &=\max_{\rho(s,a)\in \mathbb{R}}\mathbb{E}_{ s^\prime \sim P(s, a)}\left[2\left(\hat{\mathcal{T}}Q - Q\right)\left(\rho-Q\right)
    -\left(\rho-Q\right)^2\mid s, a\right]\notag
    \\
    &= \max_{\rho(s,a)\in \mathbb{R}}\mathbb{E}_{ s^\prime \sim P(s, a)}\left[   \left(\hat{\mathcal{T}}{Q}-Q\right)^2-\left(\hat{\mathcal{T}}{Q}- \rho\right)^2\mid s, a\right]\label{eq:SBEED} 
    \end{align}
where the unique maximum is with 
\begin{align}
    \rho^{*}(s,a) &= h^{*}(s,a)+Q(s,a) = \mathcal{T}Q(s,a)- Q(s,a)+Q(s,a)\notag
    \\    &=\mathcal{T}Q(s,a)\notag
\end{align}
and where the equality of \ref{eq:SBEED} is from
\begin{align}
&\!\!\!\!\!\!\!\!\!2\left(\hat{\mathcal{T}}Q - Q\right)\left(\rho-Q\right)
    -\left(\rho-Q\right)^2\notag
    \\
    &=2(\hat{\mathcal{T}}{Q}\rho - \hat{\mathcal{T}}{Q}Q - \cancel{Q\rho} + Q^2)  - (\rho^2 - \cancel{2Q\rho} + Q^2)\notag
    \\
    &=2\hat{\mathcal{T}}{Q}\rho - 2\hat{\mathcal{T}}{Q}Q +\cancel{2} Q^2- \rho^2 - \cancel{Q^2}\notag
    \\
    &=\hat{\mathcal{T}}{Q}^2- 2\hat{\mathcal{T}}{Q}Q+Q^2-\hat{\mathcal{T}}{Q}^2+2\hat{\mathcal{T}}{Q}\rho- \rho^2\notag
    \\
    &=\left(\hat{\mathcal{T}}{Q}-Q\right)^2-\left(\hat{\mathcal{T}}{Q}- \rho\right)^2\notag
\end{align}

Now note that
    
     \begin{align}
     &\mathcal{L}_{BE}(s,a)(Q)= \max_{\rho(s,a)\in \mathbb{R}}\mathbb{E}_{ s^\prime \sim P(s, a)}\left[   \left(\hat{\mathcal{T}}{Q}-Q\right)^2-\left(\hat{\mathcal{T}}{Q}- \rho\right)^2\mid s, a\right]] \tag{equation \ref{eq:SBEED}}
    \\
    &= \mathbb{E}_{ s^\prime \sim P(s, a)}\left[ \left(\hat{\mathcal{T}}{Q}-Q\right)^2\mid s,a \right]-\min_{\rho(s,a)\in \mathbb{R}}\mathbb{E}_{ s^\prime \sim P(s, a)}\left[\left(\hat{\mathcal{T}}{Q}- \underbrace{\rho}_{= r+\beta \zeta}\right)^2\mid s, a\right]  \notag
    \\
    &= \mathbb{E}_{ s^\prime \sim P(s, a)}\left[\mathcal{L}_{TD}(Q)(s,a,s^\prime)\right]-\beta^2 \min_{\zeta\in \mathbb{R}}\mathbb{E}_{ s^\prime \sim P(s, a)}\left[\left(\hat{V}(s^\prime)- \zeta\right)^2\mid s, a\right]\label{eq:yesmin}
    \\
    &= \mathbb{E}_{ s^\prime \sim P(s, a)}\left[\mathcal{L}_{TD}(Q)(s,a, s^\prime)\right]-\beta^2 \mathbb{E}_{ s^\prime \sim P(s, a)}\left[\left(\hat{V}(s^\prime)- \mathbb{E}_{ s^\prime \sim P(s, a)}[\hat{V}(s^\prime)\mid s,a]\right)^2\mid s, a\right]\label{eq:nomin}
    \end{align}
where the equality of equation \ref{eq:nomin} comes from the fact that the $\zeta$ that maximize equation \ref{eq:yesmin} is $\zeta^* := \mathbb{E}_{ s^\prime \sim P(s, a)}[\hat{V}(s')\mid s,a]$, because
\begin{align}
r(s,a)+\beta \cdot \zeta^{*} (s,a) &:=  \rho^{*}(s,a)\notag
     \\
     &= \mathcal{T}Q(s,a) \notag
     \\
     &:=r(s,a)+\beta \cdot \mathbb{E}_{ s^\prime \sim P(s, a)}\left[\hat{V}(s^\prime)\mid s,a\right]\notag
\end{align}
For $Q^\ast$, $\mathcal{T}Q^\ast=Q^\ast$ holds. Therefore, we get
\begin{align}
r(s,a)+\beta \cdot \zeta^{*} (s,a) &:=  \rho^{*}(s,a)\notag
     \\
     &= \mathcal{T}Q^\ast(s,a) = Q^\ast(s,a) \notag
\end{align}
\end{proof}


\subsection{Proof of Theorem \ref{thm:MagnacThesmar}}\label{sec:PfMagnac}
\begin{proof}
    Suppose that the system of equations (Equation \ref{eq:HotzMillereqs})
\begin{equation}
\left\{
\begin{array}{l}
    \dfrac{\exp({Q}\left(s,a\right))}{\sum_{a^\prime\in \mathcal{A}} \exp({Q}\left(s,a^\prime\right))} = \pi^*(\;a
    \mid s) \; \; \; \forall s\in \mathcal{S}, a\in\mathcal{A}
    \\[1em]
    r(s, a_s)+\beta \cdot \mathbb{E}_{s^{\prime} \sim P(s, a_s)}\left[\log(\sum_{a^\prime\in\mathcal{A}}\exp Q(s^\prime, a^\prime)) \mid s, a_s\right]-Q(s, a_s)=0 \;\; \; \forall s\in \mathcal{S} 
\end{array}
\right.
\notag
\end{equation} 
is satisfied for $Q\in \mathcal{Q}$, where $\mathcal{Q}$ denote the space of all $Q$ functions. Then we have the following equivalent recharacterization of the second condition $\forall s\in \mathcal{S}$,
\begin{align}
     Q(s, a_s)&=r(s, a_s)+\beta \cdot \mathbb{E}_{s^{\prime} \sim P(s, a_s)}\left[\log(\sum_{a^\prime\in\mathcal{A}}\exp Q(s^\prime, a^\prime)) \mid s, a_s\right]\;\; \; \notag
     \\
     &=  r(s, a_s)+\beta \cdot \mathbb{E}_{s^\prime \sim P(s, a_s)}\left[Q(s^\prime, a^\prime) - \log \pi^*(a^\prime \mid s^\prime) \mid s, a_s\right] \;\; \forall a^\prime\in\mathcal{A} \notag
     \\
     &=  r(s, a_s)+\beta \cdot \mathbb{E}_{s^\prime \sim P(s, a_s)}\left[Q(s^\prime, a_{s^\prime}) - \log \pi^*(a_{s^\prime} \mid s^\prime) \mid s, a_s\right]
\end{align}

We will now show the existence and uniqueness of a solution using a standard fixed point argument on a Bellman operator. Let $\mathcal{F}$ be the space of functions $f: \mathcal{S} \rightarrow \mathbb{R}$ induced by elements of $\mathcal{Q}$, where each $Q \in \mathcal{Q}$ defines an element of $\mathcal{F}$ via

$$
f_Q(s):=Q\left(s, a_s\right)
$$

and define an operator $\mathcal{T}_f: \mathcal{F} \rightarrow$ $\mathcal{F}$ that acts on functions $f_Q$ :

$$
\left(\mathcal{T}_f f_Q\right)(s):=r\left(s, a_s\right)+\beta \sum_{s^{\prime}} P\left(s^{\prime} \mid s, a_s\right)\left[f_Q\left(s^{\prime}\right)-\log \pi^*\left(a_{s^{\prime}} \mid s^{\prime}\right)\right]
$$

Then for $Q_1, Q_2 \in\mathcal{Q}$, We have
\begin{align} & \left(\mathcal{T}_f f_{Q_1}\right)(s):=r\left(s, a_s\right)+\beta \sum_{s^{\prime}} P\left(s^{\prime} \mid s, a_s\right)\left[f_{Q_1}\left(s^{\prime}\right)-\log \pi^*\left(a_{s^{\prime}} \mid s^{\prime}\right)\right] \notag
\\ & \left(\mathcal{T}_f f_{Q_2}\right)(s):=r\left(s, a_s\right)+\beta \sum_{s^{\prime}} P\left(s^{\prime} \mid s, a_s\right)\left[f_{Q_2}\left(s^{\prime}\right)-\log \pi^*\left(a_{s^{\prime}} \mid s^{\prime}\right)\right]\notag
\end{align}

Subtracting the two, we get
\begin{align}
\left|\left(\mathcal{T}_f f_{Q_1}\right)(s)-\left(\mathcal{T}_f f_{Q_2}\right)(s)\right| 
&\leq \beta \sum_{s^{\prime}} P\left(s^{\prime} \mid s, a_s\right)\left|f_{Q_1}\left(s^{\prime}\right)-f_{Q_2}\left(s^{\prime}\right)\right| \notag
\\
&\leq \beta\left\|f_{Q_1}-f_{Q_2}\right\|_{\infty} \notag
\end{align}

Taking supremum norm over $s\in\mathcal{S}$, we get
$$\left\|\mathcal{T}_f f_{Q_1}-\mathcal{T}_f f_{Q_2}\right\|_{\infty} \leq \beta\left\|f_{Q_1}-f_{Q_2}\right\|_\infty$$

This implies that $\mathcal{T}_f$ is a contraction mapping under supremum norm, with $\beta\in (0,1)$. Since $\mathcal{Q}$ is a Banach space under sup norm (Lemma \ref{lem:completeMetric}), we can apply Banach fixed point theorem to show that there exists a unique $f_Q$ that satisfies $\mathcal{T}_f(f_Q) = f_Q$, and by definition of $f_Q$ there exists a unique $Q$ that satisfies $\mathcal{T}_f(f_Q) = f_Q$, i.e., 
$$r\left(s, a_s\right)+\beta \cdot \mathbb{E}_{s^{\prime} \sim P\left(s, a_s\right)}\left[\log \left(\sum_{a^{\prime} \in \mathcal{A}} \exp Q\left(s^{\prime}, a^{\prime}\right)\right) \mid s, a_s\right]-Q\left(s, a_s\right)=0 \quad \forall s \in \mathcal{S}$$

Since $Q^\ast$ satisfies the system of equations \ref{eq:HotzMillereqs}, $Q^\ast$ is the only solution to the system of equations.

Also, since $Q^\ast = \mathcal{T}Q^\ast = r(s,a)+\beta \cdot \mathbb{E}_{s^{\prime} \sim P(s, a)}\bigl[\log(\sum_{a^\prime\in\mathcal{A}}\exp Q^\ast(s^\prime, a^\prime)) \mid s, a\bigr]$ holds, we can identify $r$ as
\begin{align}
    r(s,a) &= Q^\ast(s, a) - \beta \cdot \mathbb{E}_{s^{\prime} \sim P(s, a)}\bigl[\log(\sum_{a^\prime\in\mathcal{A}}\exp Q^\ast(s^\prime, a^\prime)) \mid s, a\bigr] \notag
\end{align}


\end{proof}
\begin{lem}\label{lem:completeMetric} Suppose that $\mathcal{Q}$ consists of bounded functions on $\mathcal{S} \times \mathcal{A}$. Then $\mathcal{Q}$ is a Banach space with the supremum norm as the induced norm.
\end{lem}
\begin{proof}
Suppose a sequence of functions $\left\{Q_n\right\}$ in $\mathcal{Q}$ is Cauchy in the supremum norm. We must show that $ Q_n\rightarrow Q^\ast$ as $n\rightarrow \infty$ for some $Q^\ast$ and $Q^\ast$ is also bounded. %because $\left|Q^*(s, a)\right|=\lim _{n \rightarrow \infty}\left|Q_n(s, a)\right| \leq M$. 
Note that $Q_n$ being Cauchy in sup norm implies that for every $(s, a)$, the sequence $\left\{Q_n(s, a)\right\}$ is Cauchy in $\mathbb{R}$. Since $\mathbb{R}$ is a complete space, every Cauchy sequence of real numbers has a limit; this allows us to define function $Q^\ast:\mathcal{S}\times \mathcal{A} \mapsto \mathbb{R}$ such that $Q^*(s, a)=\lim _{n \rightarrow \infty} Q_n(s, a)$. Then we can say that $Q_n(s, a) \rightarrow Q^*(s, a) $ for every $(s, a) \in \mathcal{S} \times \mathcal{A}$. Since each $Q_n$ is bounded, we take the limit and obtain:
$$
\sup _{s, a}\left|Q^*(s, a)\right|=\lim _{n \rightarrow \infty} \sup _{s, a}\left|Q_n(s, a)\right| \leq M
$$
which implies $Q^* \in \mathcal{Q}$.

\noindent Now what's left is to show that the supremum norm
$$
\|Q\|_{\infty}=\sup _{(s, a) \in \mathcal{S} \times \mathcal{A}}|Q(s, a)|
$$
induces the metric, i.e., 
$$
d\left(Q_1, Q_2\right):=\left\|Q_1-Q_2\right\|_{\infty}=\sup _{(s, a) \in \mathcal{S} \times \mathcal{A}}\left|Q_1(s, a)-Q_2(s, a)\right|
$$
The function $d$ satisfies the properties of a metric:

- Non-negativity: $d\left(Q_1, Q_2\right) \geq 0$ and $d\left(Q_1, Q_2\right)=0$ if and only if $Q_1=$ $Q_2$.

- Symmetry: $d\left(Q_1, Q_2\right)=d\left(Q_2, Q_1\right)$ by the absolute difference.

- Triangle inequality:
$$
d\left(Q_1, Q_3\right)=\sup _{s, a}\left|Q_1(s, a)-Q_3(s, a)\right| \leq \sup _{s, a}\left|Q_1(s, a)-Q_2(s, a)\right|+\sup _{s, a}\left|Q_2(s, a)-Q_3(s, a)\right|
$$

which shows $d\left(Q_1, Q_3\right) \leq d\left(Q_1, Q_2\right)+d\left(Q_2, Q_3\right)$.


\end{proof}

\subsection{Proof of Theorem 
\ref{thm:mainopt}}\label{sec:pfOfmainOpt}

Define $\hat{Q}$ as 
%
\begin{align}
    \hat{Q} &\in \underset{Q\in \mathcal{Q}}{\arg\min } \; \;\mathbb{E}_{(s, a)\sim \pi^*, \nu_0}  \left[-\log\left(\hat{p}_{Q}(a
\mid s)\right)\right] + \lambda\mathbb{E}_{(s, a)\sim \pi^*, \nu_0}\left[ \mathbbm{1}_{a = a_s} \mathcal{L}_{BE}(Q)(s,a)\right] \tag{Equation \ref{eq:mainopt}}
\end{align}
From Theorem \ref{thm:MagnacThesmar}, it is sufficient to show that $\hat{Q}$
satisfies the equations \ref{eq:HotzMillereqs} of Theorem \ref{thm:MagnacThesmar} for any $\lambda>0$, i.e., 
\begin{equation}
\left\{
\begin{array}{l}
    \dfrac{\exp({\hat{Q}}\left(s,a\right))}{\sum_{a^\prime\in \mathcal{A}} \exp({\hat{Q}}\left(s,a^\prime\right))} = \pi^*(a
    \mid s) \; \; \; \forall s\in \bar{\mathcal{S}}, a \in \mathcal{A}
    \\[1em]
    r(s, a_s)+\beta \cdot \mathbb{E}_{s^{\prime} \sim P(s, a_s)}\left[\log(\sum_{a^\prime\in\mathcal{A}}\exp \hat{Q}(s^\prime, a^\prime)) \mid s, a_s\right]-\hat{Q}(s, a_s)=0 \;\;\; \forall s\in \bar{\mathcal{S}}
    
\end{array}\tag{Equation \ref{eq:HotzMillereqs}}
\right. 
\end{equation}
where $\bar{\mathcal{S}}$ (the reachable states from $\nu_0$, $\pi^\ast$) was defined as:
$$
\bar{\mathcal{S}}=\left\{s \in \mathcal{S} \mid \operatorname{Pr}\left(s_t=s \mid s_0 \sim \nu_0, \pi^*\right)>0 \text { for some } t \geq 0\right\} 
$$
Now note that:
  \begin{align}
     &\left\{Q \in \mathcal{Q} \mid\hat{p}_{Q}(\;\cdot
    \mid s) = \pi^*(\;\cdot
    \mid s)\quad  \forall s\in\bar{\mathcal{S}}\quad\text{a.e.}\right\} \notag
    \\
    &=\underset{Q\in \mathcal{Q}}{\arg\max } \; \;\mathbb{E}_{(s, a)\sim \pi^*, \nu_0}  \left[\log\left(\hat{p}_{Q}(\;\cdot
    \mid s)\right)\right] \tag{$\because$ Lemma \ref{lem:minMLE}}
    \\
    &=\underset{Q\in \mathcal{Q}}{\arg\min } \; \;\mathbb{E}_{(s, a)\sim \pi^*, \nu_0}  \left[-\log\left(\hat{p}_{Q}(\;\cdot
    \mid s)\right)\right] \notag
    \end{align}
and 
    \begin{align}
     &\left\{Q \in \mathcal{Q} \mid\mathcal{L}_{BE}(Q)(s,a_s) = 0\quad  \forall s\in\bar{\mathcal{S}}\right\} \notag
    \\
    &=\underset{Q\in \mathcal{Q}}{\arg\min } \; \;\mathbb{E}_{(s, a)\sim \pi^*, \nu_0}  \left[\mathbbm{1}_{a = a_s} \mathcal{L}_{BE}(Q)(s,a)\right] \notag
    \end{align}
Therefore what we want to prove, equations \ref{eq:HotzMillereqs}, becomes the following equation \ref{eq:modifiedHotz}:

\begin{equation}
\left\{
\begin{array}{l}
    \hat{Q} \in \underset{Q\in \mathcal{Q}}{\arg\min } \; \;\mathbb{E}_{(s, a)\sim \pi^*, \nu_0}  \left[-\log\left(\hat{p}_{Q}(\;\cdot
    \mid s)\right)\right] 
    \\[1em]
     \hat{Q} \in \underset{Q\in \mathcal{Q}}{\arg\min } \; \;\mathbb{E}_{(s, a)\sim \pi^*, \nu_0}  \left[\mathbbm{1}_{a = a_s} \mathcal{L}_{BE}(Q)(s,a)\right]
    
\end{array}\label{eq:modifiedHotz}
\right. 
\end{equation}
where its solution set is nonempty by Theorem \ref{thm:MagnacThesmar}, i.e., 
$$ \underset{Q\in \mathcal{Q}}{\arg\min } \; \;\mathbb{E}_{(s, a)\sim \pi^*, \nu_0} \left[-\log\left(\hat{p}_{Q}(a
\mid s)\right)\right] \;\; \cap \;\; \underset{Q\in \mathcal{Q}}{\arg\min } \; \; \mathbb{E}_{(s, a)\sim \pi^*, \nu_0} \left[\mathbbm{1}_{a = a_s}\mathcal{L}_{BE}(\hat{Q})(s,a)\right] \;\; \neq \;\; \emptyset$$ 

Under this non-emptiness, according to Lemma \ref{lem:sharingsol}, $\hat{Q}$ satisfies equation \ref{eq:modifiedHotz}. This implies that $\hat{Q}(s,a) = Q^\ast(s,a)$ for $s\in\bar{\mathcal{S}}$ and $a\in\mathcal{A}$, as the solution to set of equations \ref{eq:HotzMillereqs} is $Q^*$. This implies that 
\begin{align}
    r(s, a)=\hat{Q}(s, a)-\beta \cdot \mathbb{E}_{s^{\prime} \sim P(s, a)}\left[\log \left(\sum_{a^{\prime} \in \mathcal{A}} \exp \hat{Q}\left(s^{\prime}, a^{\prime}\right)\right) \mid s, a\right] \notag
\end{align}
for $s\in\bar{\mathcal{S}}$ and $a\in\mathcal{A}$.
\QED


\begin{lem}\label{lem:minMLE}
    \begin{align}
           \underset{Q\in \mathcal{Q}}{\arg\max } &\; \;\mathbb{E}_{(s, a)\sim \pi^*, \nu_0}  \left[\log\left(\hat{p}_{Q}(\;\cdot
    \mid s)\right)\right] \notag
    \\
     &=\left\{Q \in \mathcal{Q} \mid\hat{p}_{Q}(\;\cdot
    \mid s) = \pi^*(\;\cdot
    \mid s)\quad  \forall s\in\bar{\mathcal{S}}\quad\text{a.e.}\right\}\notag
    \\
     &=\left\{Q \in \mathcal{Q} \mid Q(s,a_1)-Q(s,a_2)= Q^*(s,a_1)-Q^*(s,a_2) \quad \forall a_1, a_2\in\mathcal{A}, s\in\bar{\mathcal{S}}\right\} \notag
    \end{align}
\end{lem}

\begin{proof}[Proof of Lemma \ref{lem:minMLE}]
    \begin{align}
    \mathbb{E}_{(s, a)\sim \pi^*, \nu_0}  \left[\log\left(\hat{p}_{Q}(\;\cdot
    \mid s)\right)\right] & = 
\mathbb{E}_{(s,a)\sim\pi^*, \nu_0} [\log \hat{p}_{Q}(a|s) - \ln \pi^*(a|s) + \ln \pi^*(a|s)]\notag \\
    &=-\mathbb{E}_{(s,a)\sim\pi^*, \nu_0} \left[\ln \frac{\pi^*(a|s)}{\hat{p}_{Q}(a|s)} \right] + \mathbb{E}_{(s,a)\sim\pi^*, \nu_0} [\ln \pi^*(a|s)]\notag \\
    &= -\mathbb{E}_{s\sim\pi^*, \nu_0} \left[D_{KL}(\pi^*(\cdot\mid s) \| \hat{p}_{Q}(\cdot\mid s))\right] + \mathbb{E}_{(s,a)\sim\pi^*, \nu_0} [\ln \pi^*(a|s)]\notag \notag
\end{align}
Therefore,
\begin{align}
    \underset{Q\in\mathcal{Q}}{\arg\max}&\; \mathbb{E}_{(s, a)\sim \pi^*, \nu_0}  \left[\log\left(\hat{p}_{Q}(\;\cdot
    \mid s)\right)\right] =\underset{Q\in\mathcal{Q}}{\arg\min}\;\mathbb{E}_{s\sim\pi^*, \nu_0} \left[D_{KL}(\pi^*(\cdot\mid s) \| \hat{p}_{Q}(\cdot\mid s))\right]\notag
    \\
    &=\{Q\in\mathcal{Q}\mid D_{KL}(\pi^*(\cdot\mid s) \| \hat{p}_{Q}(\cdot\mid s))=0 \text{ for all }s\in\bar{\mathcal{S}}\} \tag{$\because \; Q^*\in \mathcal{Q}$ and $D_{KL}(\pi^* \| \pi^*)=0$}\notag
    \\
    &=\{Q\in\mathcal{Q}\mid \hat{p}_Q(\cdot \mid s)=\pi^*(\cdot \mid s)\; \; \text{a.e.} \text{ for all }s\in\bar{\mathcal{S}}\}\notag
    \\
    &= \{Q\in\mathcal{Q}\mid \frac{\hat{p}_Q\left(a_1 \mid s\right)}{\hat{p}_Q\left(a_2 \mid s\right)}=\frac{\pi^*\left(a_1 \mid s\right)}{\pi^*\left(a_2 \mid s\right)} \quad \forall a_1, a_2 \in \mathcal{A}, s\in\bar{\mathcal{S}}\}\notag   
    \\
    &=\left\{Q \in \mathcal{Q} \mid \exp (Q(s,a_1)-Q(s,a_2))=\exp \left(Q^*(s,a_1)-Q^*(s,a_2)\right) \quad \forall a_1, a_2\in\mathcal{A}, s\in\bar{\mathcal{S}} \right\}\notag
    \\
    &=\left\{Q \in \mathcal{Q} \mid Q(s,a_1)-Q(s,a_2)= Q^*(s,a_1)-Q^*(s,a_2) \quad \forall a_1, a_2\in\mathcal{A}, s\in\bar{\mathcal{S}}\right\} \notag
\end{align}
\end{proof}

\begin{lem}\label{lem:sharingsol}
    Let $f_1: \mathcal{X} \rightarrow \mathbb{R}$ and $f_2: \mathcal{X} \rightarrow \mathbb{R}$ be two functions defined on a common domain $\mathcal{X}$. Suppose the sets of minimizers of $f_1$ and $f_2$ intersect, i.e.,
$$
\arg \min f_1 \cap \arg \min f_2 \neq \emptyset
$$
Then, any minimizer of the sum $f_1+f_2$ is also a minimizer of both $f_1$ and $f_2$ individually. That is, if
$$
x^* \in \arg \min \left(f_1+f_2\right)
$$
then
$$
x^* \in \arg \min f_1 \; \cap \; \arg \min f_2
$$
\end{lem}
\begin{proof}
    Since \( \arg\min f_1 \cap \arg\min f_2 \neq \emptyset \), let \( x^\dagger \) be a common minimizer such that $$
x^\dagger \in \arg\min f_1 \cap \arg\min f_2$$
This implies that  
\begin{align}
    f_1(x^\dagger) &= \min_{x \in \mathcal{X}} f_1(x) =: m_1, \notag
    \\
    f_2(x^\dagger) &= \min_{x \in \mathcal{X}} f_2(x) =: m_2. \notag
\end{align}

Now, let \( x^* \) be any minimizer of \( f_1 + f_2 \), so  
\begin{align}
    x^* \in \arg\min (f_1 + f_2) &\iff f_1(x^*) + f_2(x^*) \leq f_1(x) + f_2(x), \quad \forall x \in \mathcal{X}. \notag
\end{align}
Evaluating this at \( x^\dagger \), we obtain  
\begin{align}
    f_1(x^*) + f_2(x^*) &\leq f_1(x^\dagger) + f_2(x^\dagger) \notag
    \\
    &= m_1 + m_2. \notag
\end{align}

Now, suppose for contradiction that $x^* \notin \arg\min f_1$, 
meaning  
\begin{align}
    f_1(x^*) &> m_1 \notag
\end{align}
But then
\begin{align}
    f_2(x^*) & \le m_1 + m_2 - f_1(x^*) \notag
    \\
    &< m_1 + m_2 - m_1 = m_2 \notag
\end{align}

This contradicts the fact that \( m_2 = \min f_2 \), so \( x^* \) must satisfy  
\begin{align}
    f_1(x^*) &= m_1 \notag
\end{align}

By symmetry, assuming $x^* \notin \arg\min f_2$ leads to the same contradiction, forcing  
\begin{align}
    f_2(x^*) &= m_2 \notag
\end{align}

Thus, we conclude  
\begin{align}
    x^* \in \arg\min f_1 \cap \arg\min f_2 \notag
\end{align}
\end{proof}

%%%%%% Mirror descent %%%%%%%%%%%

\iffalse

\subsection{Proof of Theorem \ref{thm:mirror}}\label{sec:Proofofmirror}

\begin{thm}[Mirror descent equivalence]\label{thm:mirror}
    Equation \ref{eq:mainopt} is equivalent to the mirror descent algorithm for minimizing Bellman error only, i.e., 
    
    \begin{align}
       \underset{Q \in \mathcal{Q}}{\arg\min} &\; \mathbb{E}_{(s, a) \sim \pi^*, \nu_0, s' \sim P(s, a)} \left[\mathbbm{1}_{a = a_s} \left( \mathcal{L}_{TD}(Q)(s,a,s') - \beta^2 D(Q)(s,a)\right) \right]\notag
    \end{align}
   with Bregman divergence associated with $F$, where $F(p)=\sum_{a \in \mathcal{A}} p(a \mid s) \log(p(a \mid s))$, which defines the negative Shannon entropy of a distribution $p(a \mid s)$.
\end{thm}
See Appendix \ref{sec:Proofofmirror} for the proof. Note that $F$ makes the Bregman divergence be $D_{KL}\left(\pi^* \| \hat{p}_Q\right):=\sum_a \pi^*(a \mid s) \log \left(\frac{\pi^*(a \mid s)}{\hat{p}_Q(a \mid s)}\right)$, the mirror map be $\phi(p)=\log(p)$, and the inverse mirror map be the softmax transformation.

\begin{align}
    \hat{Q} &\in  \underset{Q\in \mathcal{Q}}{\arg\min } \; \;\mathbb{E}_{(s, a)\sim \pi^*, \nu_0,  s^\prime \sim P(s, a)}  \left[-\log\left(\hat{p}_{Q}(a
\mid s)\right) + \lambda \mathbbm{1}_{a = a_s}\left( \mathcal{L}_{TD}(Q)(s,a,s^\prime)-\beta^2D(Q)(s,a)\right)\right]\tag{Equation \ref{eq:mainopt}}
\\
&= \underset{Q\in \mathcal{Q}}{\arg\min } \; \;\mathbb{E}_{(s, a)\sim \pi^*, \nu_0}  \left[-\log\left(\hat{p}_{Q}(a
\mid s)\right)\right] + \lambda\mathbb{E}_{(s, a)\sim \pi^*, \nu_0}\left[ \mathbbm{1}_{a = a_s} \mathcal{L}_{BE}(Q)(s,a)\right] \tag{$\because$ Lemma \ref{thm:BEbiconjWrtV}}
\\
&= \underset{Q\in \mathcal{Q}}{\arg\min } \; \; \lambda\mathbb{E}_{(s, a)\sim \pi^*, \nu_0}\left[ \mathbbm{1}_{a = a_s} \mathcal{L}_{BE}(Q)(s,a)\right] + D_{KL}(\pi^*(\cdot\mid s) \| \pi_{Q}(\cdot\mid s)) \notag
\\
&\quad\;\; \quad \text{s.t.} \quad \;\{Q\in\mathcal{Q}\mid \pi_Q = \pi^* \; \; \text{a.e.} \text{ for all }s\in\mathcal{S}\}\tag{$\because$Proof of Lemma \ref{lem:minMLE}}
\end{align}
Define $f(Q) := \lambda\mathbb{E}_{(s, a)\sim \pi^*, \nu_0}\left[ \mathbbm{1}_{a = a_s} \mathcal{L}_{BE}(Q)(s,a)\right]$, $f^\prime(Q) := \lambda\mathbb{E}_{(s, a)\sim \pi^*, \nu_0}\left[ \mathbbm{1}_{a = a_s} \mathcal{L}_{BE}(Q)(s,a)\right] + D_{KL}(\pi^*(\cdot\mid s) \| \pi_{Q}(\cdot\mid s))$ and $K:=\{Q\in\mathcal{Q}\mid \pi_Q = \pi^* \; \; \text{a.e.} \text{ for all }s\in\mathcal{S}\}$. Then the mirror descent method of minimization of $f^\prime(Q)$ subject to constraint $K$ is defined as 
\begin{align}
    Q_{t+1} \leftarrow \operatorname{\argmin}_{Q\in K}\left\{\eta\left\langle\nabla f^\prime\left(x_t\right), x\right\rangle+D_h\left(x \| x_t\right)\right\}
\end{align}

\begin{align}
&\Phi_{t+1}-\Phi_t=  \frac{1}{\eta}\left(h\left(x_t\right)-h\left(x_{t+1}\right)-\frac{1}{2}\left\langle\nabla h\left(x_t\right), x^*-x_{t+1}\right\rangle-\frac{1}{2}\left\langle\nabla h\left(x^*\right), x^*-x_{t+1}\right\rangle\right. \\
& \left.+\frac{\eta}{2}\left\langle\nabla f_t\left(x_t\right), x^*-x_{t+1}\right\rangle+\left\langle\nabla h\left(x_t\right), x^*-x_t\right\rangle\right)
\\
&= \frac{1}{\eta}\left(h\left(x_t\right)-h\left(x_{t+1}\right)-\frac{1}{2}\left\langle\nabla h\left(x_t\right), x^*-x_{t+1}\right\rangle-\frac{1}{2}\left\langle\nabla h\left(x_{t}\right), x^*-x_{t+1}\right\rangle\right. \\
& \left.+\frac{\eta}{2}\left\langle\nabla f_t\left(x_t\right), x^*-x_{t+1}\right\rangle+\left\langle\nabla h\left(x_t\right), x^*-x_t\right\rangle+ \frac{1}{2  } \left\langle\nabla h\left(x^*\right)-\nabla h\left(x_t\right), x^*-x_{t+1}\right\rangle\right) 
\\
&= \frac{1}{\eta}\left(h\left(x_t\right)-h\left(x_{t+1}\right)-\left\langle\nabla h\left(x_t\right), x_t-x_{t+1}\right\rangle+\frac{\eta}{2}\left\langle\nabla f_t\left(x_t\right), x^*-x_{t+1}\right\rangle+ \frac{1}{2  } \left\langle\nabla h\left(x^*\right)-\nabla h\left(x_t\right), x^*-x_{t+1}\right\rangle\right) 
\\
&= \frac{1}{\eta}\left(h\left(x_t\right)-h\left(x_{t+1}\right)-\left\langle\nabla h\left(x_t\right), x_t-x_{t+1}\right\rangle + \frac{\eta}{2}\left\langle\nabla f_t\left(x_t\right), x^*-x_{t}\right\rangle\right. \\
& \left.+\frac{\eta}{2}\left\langle\nabla f_t\left(x_t\right), x_t-x_{t+1}\right\rangle+ \frac{1}{2  } \left\langle\nabla h\left(x^*\right)-\nabla h\left(x_t\right), x^*-x_{t}\right\rangle+\frac{1}{2  } \left\langle\nabla h\left(x^*\right)-\nabla h\left(x_t\right), x_t-x_{t+1}\right\rangle\right) \notag
\\
&= \frac{1}{\eta}\left(h\left(x_t\right)-h\left(x_{t+1}\right)-\left\langle\nabla h\left(x_t\right), x_t-x_{t+1}\right\rangle + \frac{\eta}{2}\left\langle\nabla f_t\left(x_t\right), x^*-x_{t}\right\rangle\right. \\
& \left.+ \frac{1}{2  } \left\langle\nabla h\left(x^*\right)-\nabla h\left(x_t\right), x^*-x_{t}\right\rangle+\frac{1}{2  } \left\langle \eta \nabla f_t + \nabla h\left(x^*\right)-\nabla h\left(x_t\right), x_t-x_{t+1}\right\rangle\right) 
\\
&= \frac{1}{\eta}\left(h\left(x_t\right)-h\left(x_{t+1}\right)-\left\langle\nabla h\left(x_t\right), x_t-x_{t+1}\right\rangle + \frac{\eta}{2}\left\langle\nabla f_t\left(x_t\right), x^*-x_{t}\right\rangle\right. \\
& \left.+\frac{1}{2  } \left\langle\nabla h\left(x^*\right)-\nabla h\left(x_t\right), x^*-x_{t}\right\rangle+\left\langle \nabla h(x^*)-\nabla h(x_{t+1}), x_t-x_{t+1}\right\rangle\right) \notag 
\end{align}

\fi
%%%%%% End of Mirror descent %%%%%%%%%%%


\subsection{Proof of Lemma \ref{lem:ConvexityMLE}}
\begin{proof}[Proof of Lemma \ref{lem:ConvexityMLE}] \label{sec:NLLproperties}
Denote $Q(s, \cdot)=\left[Q\left(s, a^{\prime}\right)\right]_{a^{\prime} \in \mathcal{A}}$. Then,
\begin{align}
    \text{Convexity}& \text{ of }  \mathbb{E}_{(s, a)\sim \pi^*, \nu_0}  \left[-\log\left(\hat{p}_{Q}(\;\cdot
    \mid s)\right)\right] \text{ w.r.t. }Q\in\mathcal{Q}\notag
    \\
    &\iff \text{Concavity of } \mathbb{E}_{(s,a)\sim\pi^*, \nu_0}\left[\ln \hat{p}_{Q}\left( \cdot \mid s\right)\right] \text{ w.r.t. } Q\in\mathcal{Q} \notag
    \\
    &\;\Longleftarrow \text{Concavity of } \ln \hat{p}_{Q}\left(\cdot \mid s\right)\text{ w.r.t. } Q\in\mathcal{Q} \text{ for all } s\in\mathcal{S} \tag{$\because$ linearity of expectation}
    \\
    &\iff \text{Concavity of }  Q(s, \cdot)-log \sum_{a^{\prime}\in\mathcal{A}} \exp \left(Q\left(s, a^{\prime}\right)\right) \text{ w.r.t. } Q(s, \cdot)  \text{ for all } s\in\mathcal{S} \notag
    \\
    &\iff \text{Convexity of } log \sum_{a^{\prime}\in\mathcal{A}} \exp \left(Q\left(s, a^{\prime}\right)\right) \text{ w.r.t. } Q(s, \cdot)  \text{ for all } s\in\mathcal{S}  \notag
\end{align}
Since the function logsumexp is a known convex function, we are done.

\begin{align}
    &\text{Lipschitz} \text{ smoothness } \text{of }  \mathbb{E}_{(s, a)\sim \pi^*, \nu_0}  \left[\log\left(\hat{p}_{Q}(\;\cdot
    \mid s)\right)\right] \text{ w.r.t. }Q\in\mathcal{Q}\notag
    \\
    &\iff \text{ Lipschitz continuity of } \nabla_Q \; \mathbb{E}_{(s,a)\sim\pi^*, \nu_0}\left[\ln \hat{p}_{Q}\left( \cdot \mid s\right)\right] \text{ w.r.t. } Q\in\mathcal{Q} \notag
    \\
    &\iff \text{ Lipschitz continuity of } \mathbb{E}_{(s, a) \sim \pi^*, \nu_0}\left[\delta_{a, a^{\prime}}-\hat{p}_Q\left(a^{\prime} \mid s\right)\right]_{a^{\prime}\in\mathcal{A}} \notag
    \\
    &\iff \text{ Lipschitz continuity of } \mathbb{E}_{s \sim \pi^*, \nu_0}\left[\pi^*\left(a^{\prime}\mid s\right) -\hat{p}_Q\left(a^{\prime} \mid s\right)\right]_{a^{\prime}\in\mathcal{A}} \notag
    \\
    &\iff \exists \; c>0 \; s.t. \;  \|\mathbb{E}_{s \sim \pi^*, \nu_0}\left[\hat{p}_{Q^\prime}\left(a^{\prime}\mid s\right) -\hat{p}_Q\left(a^{\prime} \mid s\right)\right]_{a^{\prime}\in\mathcal{A}}\| \le c\|Q-Q^\prime\|_{L_2(\pi^\ast, \nu_0)} \quad \forall Q, Q^\prime \in\mathcal{Q} \notag
\end{align}
Since softmax is 1-Lipschitz continuous for each $s\in\mathcal{S}$ with respect to $\ell_2$ norm \cite{gao2017properties}, for all $s\in\mathcal{S}$ we have
$$
\left\|\hat{p}_{Q^{\prime}}(\cdot \mid s)-\hat{p}_Q(\cdot \mid s)\right\|_2 \leq \left\|Q^{\prime}(s, \cdot)-Q(s, \cdot)\right\|_2
$$
Therefore
\begin{align}
  \|\mathbb{E}_{s \sim \pi^*, \nu_0}\left[\hat{p}_{Q^\prime}\left(\cdot\mid s\right) -\hat{p}_Q\left(\cdot\mid s\right)\right]\|_2 & \leq \mathbb{E}_{s \sim \pi^*, \nu_0}\bigl[\left\|\hat{p}_{Q^{\prime}}(\cdot \mid s)-\hat{p}_Q(\cdot \mid s)\right\|_2\bigr] \tag {Norm is convex}
  \\
  &\leq \mathbb{E}_{s \sim \pi^*, \nu_0}\left[\left\|Q^{\prime}(s, \cdot)-Q(s, \cdot)\right\|_2\right] \tag{Softmax is 1-Lipschitz}
  \\
  &\leq\left(\mathbb{E}_{s \sim \pi^*, \nu_0}\left\|Q^{\prime}(s, \cdot)-Q(s, \cdot)\right\|_2^2\right)^{1 / 2} \tag{$x^{1/2}$ is concave}
  \\
  &=\left\|Q-Q^{\prime}\right\|_{L_2(\pi^\ast, \nu_0)} \notag
\end{align}


\iffalse
Lastly for Lipschitz continuity, 
$$
\log \hat{p}_Q(a \mid s)=Q(s, a)-\log \sum_{a^{\prime} \in \mathcal{A}} \exp \left(Q\left(s, a^{\prime}\right)\right)
$$
This implies
\begin{align}
    \!\!\!\!\!\!\!&\left|\log \hat{p}_{Q_1}(a \mid s)-\log \hat{p}_{Q_2}(a \mid s)\right|
    \\
    &=\left|\left[Q_1(s, a)-Q_2(s, a)\right]-\left[\log \sum_{a^{\prime}} \exp \left(Q_1\left(s, a^{\prime}\right)\right)-\log \sum_{a^{\prime}} \exp \left(Q_2\left(s, a^{\prime}\right)\right)\right]\right|
    \\
    &\le \left|\left[Q_1(s, a)-Q_2(s, a)\right]\right|+\left|\left[\log \sum_{a^{\prime}} \exp \left(Q_1\left(s, a^{\prime}\right)\right)-\log \sum_{a^{\prime}} \exp \left(Q_2\left(s, a^{\prime}\right)\right)\right]\right|
    \\
    &\le \left\|Q_1-Q_2\right\|_{\infty} + \left\|Q_1-Q_2\right\|_{\infty} = 2 \left\|Q_1-Q_2\right\|_{\infty} 
\end{align}
Therefore
$$\left|\mathbb{E}_{(s, a) \sim \pi^*, \nu_0}\left[\log \hat{p}_{Q_1}(a \mid s)\right]-\mathbb{E}_{(s, a) \sim \pi^*, \nu_0}\left[\log \hat{p}_{Q_2}(a \mid s)\right]\right| \leq 2\left\|Q_1-Q_2\right\|_{\infty}$$

\fi
\end{proof}

\subsection{Proof of Lemma \ref{lem:BELipschitz} (Properties of Bellman error)}
For showing that $\overline{\mathcal{L}_{BE}}(Q)$ is of $\mathcal{C}^2 \text{ w.r.t. 
    } Q\in\mathcal{Q}$,
\begin{align}
    &\mathcal{C}^2 \text{ of } \overline{\mathcal{L}_{BE}}(Q) \text{ w.r.t. 
    } Q\in\mathcal{Q}\notag
    \\
    & \Longleftarrow \mathcal{C}^2 \text{ of } Q(s, a)-\left[R(s, a)+\gamma \mathbb{E}_{s^{\prime} \sim P(\cdot \mid \cdot s, a)} \log \sum_{a^{\prime}} \exp \left(Q\left(s^{\prime}, a^{\prime}\right)\right)\right] \text{ w.r.t. 
    } Q\in\mathcal{Q} \text{ for }s\in\mathcal{S}\notag
    \\
    &\Longleftarrow   \mathcal{C}^2 \text{ of }  \log \sum_{a^{\prime}} \exp \left(Q\left(s, a^{\prime}\right)\right) \text{ w.r.t. 
    } Q\in\mathcal{Q} \text{ for }s\in\mathcal{S}\notag
\end{align}
As it is known that logsumexp is of $\mathcal{C}^2$ \cite{kan2023lseminkmodifiednewtonkrylovmethod}, we are done. 
\;
\\
\;
\\
For Lipschitz smoothness, 
\begin{align}
    &\text{Lipschitz} \text{ smoothness } \text{of } \overline{\mathcal{L}_{BE}}(Q) \text{ w.r.t. 
    } Q\in\mathcal{Q}\notag
    \\
    &\!\!\iff \text{ Lipschitz continuity of } \nabla_Q \; \overline{\mathcal{L}_{BE}}(Q) \text{ w.r.t. 
    } Q\in\mathcal{Q} \notag
    \\
    &\!\!\iff \text{ Lipschitz continuity of } \mathbb{E}_{(s, a) \sim \pi^*, \nu_0}\left[2 \delta_Q(s, a) \nabla_Q \delta_Q(s, a)\right] \text{ w.r.t. 
    } Q\in\mathcal{Q} \notag
\end{align}
Now note that
\begin{align}
    &\|\mathbb{E}_{s,a \sim \pi^*, \nu_0}\left[2 \delta_Q(s, a) \nabla_Q \delta_Q(s, a)-2 \delta_{Q^\prime}(s, a) \nabla_{Q^\prime} \delta_{Q^\prime}(s, a)\right]\|_2 \notag 
    \\
    & \leq \mathbb{E}_{s,a \sim \pi^*, \nu_0}\bigl[\left\|2 \delta_Q(s, a) \nabla_Q \delta_Q(s, a)-2 \delta_{Q^\prime}(s, a) \nabla_{Q^\prime} \delta_{Q^\prime}(s, a)\right\|_2\bigr] \tag {Norm is convex}
  \\
  &\leq \mathbb{E}_{s,a \sim \pi^*, \nu_0}\left[\left\|Q^{\prime}(s,a)-Q(s, a)\right\|_2\right] \tag{Lemma \ref{lem:deltaGradDeltaLipschitz}}
  \\
  &\leq\left(\mathbb{E}_{s \sim \pi^*, \nu_0}\left\|Q^{\prime}(s, a)-Q(s, a)\right\|_2^2\right)^{1 / 2} \tag{$x^{1/2}$ is concave}
  \\
  &=\left\|Q-Q^{\prime}\right\|_{L_2(\pi^\ast, \nu_0)} \notag
\end{align}
This proves $ \text{ Lipschitz continuity of } \mathbb{E}_{(s, a) \sim \pi^*, \nu_0}\left[2 \delta_Q(s, a) \nabla_Q \delta_Q(s, a)\right] \text{ w.r.t. 
    } Q\in\mathcal{Q}$. Therefore, we can conclude the Lipschitz smoothness of $\overline{\mathcal{L}_{BE}}(Q)$ w.r.t. $ Q\in\mathcal{Q}$.
    \QED

\begin{lem}[$ \delta_Q(s, a) \nabla_Q \delta_Q(s, a)$ is Lipschitz]
\label{lem:deltaGradDeltaLipschitz}
For given fixed $(s,a)$, 
$$
\left\|2 \delta_Q(s, a) \nabla_Q \delta_Q(s, a)-2 \delta_{Q^\prime}(s, a) \nabla_{Q^\prime} \delta_{Q^\prime}(s, a)\right\|_2 \le \left\|Q^{\prime}(s,a)-Q(s, a)\right\|_2 
$$
holds for any $Q, Q^\prime \in\mathcal{Q}$.
\end{lem}
\begin{proof}[Proof of Lemma \ref{lem:deltaGradDeltaLipschitz}]
    Note that 
\begin{align}
   &\left\|\delta_Q(s, a) \nabla_Q \delta_Q(s, a)-\delta_{Q^{\prime}}(s, a) \nabla_{Q^{\prime}} \delta_{Q^{\prime}}(s, a)\right\|_2   \notag
   \\
   &\le \left\|\delta_Q(s, a)\right\|_2\left\|\nabla_Q \delta_Q(s, a) - \nabla_{Q^\prime} \delta_{Q^\prime}(s, a) \right\|_2 +  \left\|\delta_Q(s, a)-\delta_{Q^\prime}(s, a)\right\|_2 \left\| \nabla_{Q^\prime}\delta_{Q^\prime}(s,a)\right\|_2 \notag
\end{align}
Now what's left is to prove that for given fixed $(s,a)$, 
\begin{enumerate}
    \item $ \left\|\delta_Q(s, a)\right\|_2$ is bounded
    \item $\left\| \nabla_{Q^\prime}\delta_{Q^\prime}(s,a)\right\|_2$ is bounded
    \item $\delta_Q(s, a)$ is Lipschitz in $Q(s,a)$
    \item $\nabla_{Q^\prime}\delta_{Q^\prime}(s,a)$ is Lipschitz in $Q(s,a)$
\end{enumerate}

\noindent (1) Boundedness of \( \delta_Q(s,a) \):
\begin{align}
    |\delta_Q(s, a)| &= \left| \mathcal{T}Q(s,a) - Q(s,a) \right| \notag
    \\
    &= \left| r(s,a) + \beta \mathbb{E}_{s' \sim P(\cdot \mid s,a)} \left[V_Q(s')\right] - Q(s,a) \right|. \notag
\end{align}
Since \(V_Q(s') = \ln \sum_{b \in \mathcal{A}} \exp(Q(s',b))\), we use the bound:
\begin{align}
    \max_{b \in \mathcal{A}} Q(s',b) \leq V_Q(s') \leq \max_{b \in \mathcal{A}} Q(s',b) + \ln |\mathcal{A}| \notag
\end{align}
Taking expectations preserves boundedness, so we conclude:
\begin{align}
    |\delta_Q(s,a)| \leq |r(s,a)| + \beta \max_{s' \in \mathcal{S}} \max_{b \in \mathcal{A}} |Q(s',b)| + \beta \ln |\mathcal{A}| + \max_{s,a} |Q(s,a)| \notag
\end{align}
This shows \( \delta_Q(s,a) \) is uniformly bounded as long as \( Q \) is bounded, which is assured by $\beta<1$. 

\noindent (2) Boundedness of \( \nabla_Q \delta_Q(s,a) \):  
The gradient is given by:
\begin{align}
    \nabla_Q \delta_Q(s,a) &= \nabla_Q \mathcal{T} Q(s,a) - e_{(s,a)} \notag
\end{align}
where
\begin{align}
    \nabla_Q \mathcal{T} Q(s,a) &= \beta \mathbb{E}_{s' \sim P(\cdot \mid s,a)} \left[ \nabla_Q V_Q(s') \right] \notag
\end{align}
Since the softmax function \( \nabla_Q V_Q(s') \) satisfies
\begin{align}
    \sum_{b \in \mathcal{A}} \mathrm{softmax}(s',b; Q) = 1, \quad 0 \leq \mathrm{softmax}(s',b;Q) \leq 1 \notag
\end{align}
we obtain:
\begin{align}
    \|\nabla_Q \mathcal{T} Q(s,a)\|_2 \leq \beta \notag
\end{align}
Thus,
\begin{align}
    \|\nabla_Q \delta_Q(s,a)\|_2 = \|\nabla_Q \mathcal{T} Q(s,a) - e_{(s,a)}\|_2 \leq \beta + 1 \notag
\end{align}
Hence, \( \nabla_Q \delta_Q(s,a) \) is bounded.

\noindent (3) Lipschitz continuity of \( \delta_Q(s,a) \):  
Consider two functions \( Q \) and \( Q' \), and their corresponding Bellman errors:
\begin{align}
    \left| \delta_Q(s, a) - \delta_{Q'}(s, a) \right| &= \left| \mathcal{T}Q(s,a) - Q(s,a) - \mathcal{T}Q'(s,a) + Q'(s,a) \right| \notag
    \\
    &= \left| \mathcal{T}Q(s,a) - \mathcal{T}Q'(s,a) - (Q(s,a) - Q'(s,a)) \right| \notag
    \\
    &\leq \left| \mathcal{T}Q(s,a) - \mathcal{T}Q'(s,a) \right| + \left| Q(s,a) - Q'(s,a) \right| \notag
\end{align}
Since \( \mathcal{T}Q(s,a) \) depends on \( Q \) only through \( V_Q(s') \), we use the Lipschitz property of log-sum-exp:
\begin{align}
    |V_Q(s') - V_{Q'}(s')| \leq \max_{b \in \mathcal{A}} |Q(s',b) - Q'(s',b)| \notag
\end{align}
Taking expectations, we get:
\begin{align}
    |\mathcal{T}Q(s,a) - \mathcal{T}Q'(s,a)| \leq \beta \max_{s',b} |Q(s',b) - Q'(s',b)| \notag
\end{align}
Therefore,
\begin{align}
    |\delta_Q(s,a) - \delta_{Q'}(s,a)| \leq (1 + \beta) \max_{s',b} |Q(s',b) - Q'(s',b)| \notag
\end{align}
This proves \( \delta_Q(s,a) \) is Lipschitz in \( Q(s,a) \) with Lipschitz constant \( 1+\beta \).

\noindent (4) Lipschitz continuity of \( \nabla_Q \delta_Q(s,a) \):  
From the expression:
\begin{align}
    \nabla_Q \delta_Q(s,a) = \nabla_Q \mathcal{T}Q(s,a) - e_{(s,a)} \notag
\end{align}
we focus on \( \nabla_Q \mathcal{T}Q(s,a) \), which satisfies:
\begin{align}
    \|\nabla_Q \mathcal{T}Q(s,a) - \nabla_Q \mathcal{T}Q'(s,a)\|_2 &= \left\|\beta \mathbb{E}_{s' \sim P(\cdot \mid s,a)} \left[\nabla_Q V_Q(s') - \nabla_Q V_{Q'}(s') \right] \right\|_2 \notag
\end{align}
Using the Lipschitz property of Softmax,
\begin{align}
    \|\nabla_Q V_Q(s') - \nabla_Q V_{Q'}(s')\|_2 \leq \|Q(s',\cdot) - Q'(s',\cdot)\|_2 \notag
\end{align}


Taking expectations, we get:
\begin{align}
    \|\nabla_Q \mathcal{T}Q(s,a) - \nabla_Q \mathcal{T}Q'(s,a)\|_2 \leq \beta \max_{s',b} |Q(s',b) - Q'(s',b)| \notag
\end{align}
Since
\begin{align}
    \|\nabla_Q \delta_Q(s,a) - \nabla_{Q'} \delta_{Q'}(s,a)\|_2 \leq \|\nabla_Q \mathcal{T}Q(s,a) - \nabla_Q \mathcal{T}Q'(s,a)\|_2 \notag
\end{align}
we conclude that \( \nabla_Q \delta_Q(s,a) \) is Lipschitz with constant at most \( \beta \).



\end{proof}

%since softmax is the derivative of logsumexp, we need to show Lipschitz continuity of $\frac{\exp Q\left(s^{\prime}, a^{\prime}\right)}{\sum_{a^{\prime}} \exp \left(Q\left(s^{\prime}, a^{\prime}\right)\right) }$ w.r.t. $Q\in\mathcal{Q}$, i.e., 



%showing that $Q(s, a)-\left[R(s, a)+\gamma \mathbb{E}_{s^{\prime} \sim P(\cdot \mid \cdot s, a)} \log \sum_{a^{\prime}} \exp \left(Q\left(s^{\prime}, a^{\prime}\right)\right)\right]$ is smooth is enough, as this immediately proves smoothness of $\overline{\mathcal{L}_{BE}}(Q)$ ($\because$ square of the smooth function is also a smooth function and linear function of smooth function is also a smooth function). Since $\log \sum_{a^{\prime}} \exp \left(Q\left(s^{\prime}, a^{\prime}\right)\right)$ is smooth and  $\mathcal{C}^2$ in $Q$ \cite{kan2023lseminkmodifiednewtonkrylovmethod}, we are done. 

\iffalse
For Lipschitz continuity, 
\begin{align}
\left|\overline{\mathcal{L}_{\mathrm{BE}}}(Q)-\overline{\mathcal{L}_{\mathrm{BE}}}\left(Q^{\prime}\right)\right| & \leq \mathbb{E}_{(s, a) \sim \pi^*, \nu_0}\left[\left|\mathcal{L}_{\mathrm{BE}}(Q)(s, a)-\mathcal{L}_{\mathrm{BE}}\left(Q^{\prime}\right)(s, a)\right|\right] \notag \\
& \leq \mathbb{E}_{(s, a) \sim \pi^*, \nu_0}\left[L\left\|Q-Q^{\prime}\right\|_{\infty}\right] \notag
\\
&=L\left\|Q-Q^{\prime}\right\|_{\infty} \notag
\end{align}
where the second inequality is from Lemma \ref{lem:BEsaLip} along with the definition of the constant $L$.

\begin{lem}\label{lem:BEsaLip} There exists a constant $L>0$ such that for $s\in \mathcal{S}$ and $a\in \mathcal{A}$, 
$$\left|\mathcal{L}_{\mathrm{BE}}(Q)(s, a)-\mathcal{L}_{\mathrm{BE}}\left(Q^{\prime}\right)(s, a)\right| \leq L\left\|Q-Q^{\prime}\right\|_{\infty}$$    
\end{lem}
\begin{proof} Define $e_Q(s, a):=\mathcal{T} Q(s, a)-Q(s, a)$. Then $\mathcal{L}_{\mathrm{BE}}(Q)(s, a)=\left[e_Q(s, a)\right]^2$. 
    \begin{align}
    \left|\mathcal{L}_{\mathrm{BE}}(Q)(s, a)-\mathcal{L}_{\mathrm{BE}}\left(Q^{\prime}\right)(s, a)\right| &= \left|\left[e_Q(s, a)\right]^2-\left[e_{Q^{\prime}}(s, a)\right]^2\right| \notag
    \\
    &=\left|e_Q(s, a)+e_{Q^{\prime}}(s, a)\right|\cdot \left|e_Q(s, a)-e_{Q^{\prime}}(s, a)\right|\notag
    \\
    &\le C \cdot (1+\beta)\left\|Q-Q^{\prime}\right\|_{\infty}\label{eq:BEtwoerrorprod}
    \end{align}
where the inequality in equation \ref{eq:BEtwoerrorprod} is from Lemma \ref{lem:smoothBellman} and 
\begin{align}
    e_Q(s, a)+e_{Q^{\prime}}(s, a)&=[\mathcal{T} Q(s, a)-Q(s, a)]+\left[\mathcal{T} Q^{\prime}(s, a)-Q^{\prime}(s, a)\right]\notag
    \\
    &=2 r(s, a)+\beta \cdot \mathbb{E}_{s^{\prime} \sim P(s, a)}\left[V_Q\left(s^{\prime}\right)+V_{Q^{\prime}}\left(s^{\prime}\right)\right]-\left[Q(s, a)+Q^{\prime}(s, a)\right]\notag
    \\
    &\leq 2|r(s, a)|+\beta \cdot \mathbb{E}_{s^{\prime}}\left[\left|V_Q\left(s^{\prime}\right)\right|+\left|V_{Q^{\prime}}\left(s^{\prime}\right)\right|\right]+\left|Q(s, a)+Q^{\prime}(s, a)\right| \notag
    \\
    &\leq 2 R_{\max }+\beta \cdot 2(M+\log |\mathcal{A}|)+2 M \notag
    \\
    &=C:=2 R_{\max }+2 M(1+\beta)+2 \beta \log |\mathcal{A}| \notag
\end{align}
\end{proof}
\;
\\
\fi
 

\subsection{Proof of Theorem \ref{thm:BEenjoyPL} (Bellman error satisfying the PL condition)}
By Lemma \ref{lem:BE(s,a)PL} (Below), $\mathcal{L}_{BE}(Q)(s,a)$ satisfies PL condition with respect to $Q$ for all $s\in\mathcal{S}$ and $a\in\mathcal{A}$. By Lemma \ref{lem:f1f2sumPL}, $\frac{1}{|\mathcal{D}|}\sum_{(s,a)\in\mathcal{D}}\mathcal{L}_{BE}(s,a)$ is also PL. Now we would like to show that $\overline{\mathcal{L}_{\mathrm{BE}}}(Q):=\mathbb{E}_{(s, a) \sim \pi^*, \nu_0}\left[\mathcal{L}_{\mathrm{BE}}(Q)(s, a)\right]$ is also PL in terms of $L^2(\pi^\ast, \nu_0)$. Since $\overline{\mathcal{L}_{\mathrm{BE}}}(Q)$ is of $\mathcal{C}^2$, by \cite{rebjock2023fast}, showing PL is equivalent to showing to Quadratic Growth (QG), i.e., there exists $c^\prime>0$ such that
$$
\mathbb{E}_{(s, a) \sim \pi^*, \nu_0}\left[\mathcal{L}_{\mathrm{BE}}(Q)(s, a)\right]-\mathbb{E}_{(s, a) \sim \pi^*, \nu_0}\left[\mathcal{L}_{\mathrm{BE}}(Q^\ast)(s, a)\right] \ge c^\prime\|Q-Q^\ast\|^2_{L^2(\pi^\ast, \nu_0)}.
$$
But note that
\begin{align}
    &\mathbb{E}_{(s, a) \sim \pi^*, \nu_0}\left[\mathcal{L}_{\mathrm{BE}}(Q)(s, a)\right]-\mathbb{E}_{(s, a) \sim \pi^*, \nu_0}\left[\mathcal{L}_{\mathrm{BE}}(Q^\ast)(s, a)\right] \notag
    \\
    &= \mathbb{E}_{(s, a) \sim \pi^*, \nu_0}\left[\mathcal{L}_{\mathrm{BE}}(Q)(s, a)-\mathcal{L}_{\mathrm{BE}}(Q^\ast)(s, a)\right] \notag
    \\
    &\ge \mathbb{E}_{(s, a) \sim \pi^*, \nu_0}\left[c(s,a)^2(Q(s,a)-Q^\ast(s,a))^2\right] \label{eq:BEisC2}
    \\
    &=c^2\|Q-Q^\ast\|^2_{L^2(\pi^\ast, \nu_0)} \notag
\end{align}
where equation \eqref{eq:BEisC2} is due to $\mathcal{L}_{BE}(Q)(s,a)$ being QG because it is smooth and therefore PL implies QG \citep{liao2024error}. ($c(s,a)>0$ is the QG constant for $(s,a)$ and $c = \inf_{(s,a)\in \mathcal{S}\times\mathcal{A}} c(s,a)$.) This finishes the proof.
\QED

\begin{lem}\label{lem:BE(s,a)PL} For any given fixed $s\in\mathcal{S}$ and $a\in\mathcal{A}$,
$\mathcal{L}_{BE}(Q)(s,a)$ satisfies PL condition with respect to $Q$ in terms of euclidean norm.
\end{lem}
\begin{proof}[Proof of Lemma \ref{lem:BE(s,a)PL}] Throughout the proof, we extend \cite{ruszczynski2024functional} to deal with soft-max Bellman equation with infinite-dimensional state space $\mathcal{S}$. Given that $|\mathcal{A}|<\infty$, 
for each $s\in\mathcal{S}$, $Q(s, \cdot)$ can be expressed as a finite-dimensional vector $\left[Q\left(s, a^{\prime}\right)\right]_{a^{\prime} \in \mathcal{A}}\in\mathbb{R}^{|\mathcal{A}|}$; For convenience in notation, we define $q:\mathcal{S}\mapsto \mathbb{R}^{|A|}$ and
$$\mathcal{G}(s): \{q(s)\in \mathbb{R}^{|\mathcal{A}|}\mid q(s) = \left[Q\left(s, a^{\prime}\right)\right]_{a^{\prime} \in \mathcal{A}} \text{ for some }Q\in\mathcal{Q}\}$$
and use $q(s)$ instead of $Q(s,\cdot)$ and $q^\ast(s)$ instead of $Q^\ast(s,\cdot)$. 
% We can rewrite
%  $r(s,a) + \beta \cdot \mathbb{E}_{s^{\prime} \sim P(s, a)}\left[\log(\sum_{a^\prime\in\mathcal{A}}\exp Q(s^\prime, a^\prime)) \mid s, a\right]- Q(s, a)$ as $\Psi(s,a,q)$, where
We define
 \begin{align}
 \Psi(s, a, q)&:= r(s,a) + \beta \cdot \mathbb{E}_{s^{\prime} \sim P(s, a)}\left[\log(\sum_{a^\prime\in\mathcal{A}}\exp q(s^\prime)_{(a^\prime)}) \mid s, a\right]- q(s)_{(a)} \notag
 \end{align}
Now with $q^\ast(\cdot) := \left[Q^\ast\left(\cdot, a\right)\right]_{a \in \mathcal{A}}$, let's define
\begin{align}
     f(s, a, q) &:= \frac{1}{2}(\Psi(s, a, q^\ast)-\Psi(s, a, q))^2 \notag
\end{align}
 Then, for $s\in\mathcal{S}$, with the choice of $q(\tau):= q^\ast + \tau(q-q^\ast)$,
\begin{align}
  f_q(s, a, q):= \partial_q f(s, a, q)&=-\Psi_q(s,a, q)(\Psi(s, a, q^\ast)-\Psi(s, a,q))\notag
    \\
    &= - \Psi_q(s, a, q) \int_0^1 \Psi_q\left(s,a, q(\tau)\right)^\top(q^\ast(s)-q(s))d\tau \tag{Theorem \ref{thm:bolte}}
    \\
    &= - \int_0^1 \Psi_q(s,a, q) \Psi_q\left(s, a, q(\tau)\right)^\top  d\tau \cdot (q^\ast(s)-q(s))\notag
\end{align}

By Lemma \ref{lem:smallestEigen},there exists $\tilde{\lambda}$ such that for all $s\in\mathcal{S}$ and $a\in\mathcal{A}$, $\Psi_q(s, a, q^\prime) \Psi_q(s, a, q^{\prime\prime})^\top \succeq \tilde{\lambda} \cdot I$ for any choice of $q^\prime(s), q^{\prime\prime}(s) \in\mathcal{G}(s)$. %we can define $\tilde{\lambda}>0$ such that it satisfies $\Psi_q(s, a, q^\prime) \Psi_q(s, a, q^{\prime\prime})^\top \succeq \tilde{\lambda} I$ for all $q^\prime(s), q^{\prime\prime}(s) \in\mathcal{G}(s)$. 
Therefore we have
\begin{align}
    \left\langle f_q(s, a, q), q(s)-q^\ast(s)\right\rangle \ge \tilde{\lambda} \|q(s)-q^\ast(s)\|_2^2. \notag
\end{align}
This implies that
\begin{align}
    &\|f_q(s, a, q)\|_2=\max _{\|z\|=1}\left\langle f_q(s, a, q), z\right\rangle \ge \left\langle f_q(s,a,q), \frac{q(s)-q^*(s)}{\left\|q(s)-q^*(s)\right\|_2}\right\rangle \notag
    \\
    &\ge  \tilde{\lambda}\|q(s)-q^\ast(s)\|_2  \ge \tilde{\lambda}\|q(s)-q^\ast(s)\|_{\infty}  \label{eq:forPL1}
\end{align}
Therefore,
\begin{align}
    &\|f_q(s, a, q)\|_2\ge \tilde{\lambda}\|q(s)-q^\ast(s)\|_{\infty}   \label{eq:forPL3}
\end{align}
(Note: Equation \ref{eq:forPL3} is a regularity condition called sub-differential error bound.) Also, from Lemma \ref{lem:smoothBellman}, 
\begin{align}
    f(s, a, q) &= \frac{1}{2}(\Psi(s, a, q^\ast)-\Psi(s, a, q))^2\notag
    \\
    &\le \frac{1}{2} (1+\beta)^2 \|q(s)-q^\ast(s)\|_{\infty}^2\label{eq:forPL2}
\end{align}
Combining equation \ref{eq:forPL2} and \ref{eq:forPL3}, we get \begin{align}
  f(s,a,q)\le \frac{1}{2} \left(\frac{1+\beta}{\tilde{\lambda}}\right)^2 \|f_q(s, a, q)\|_2^2 \quad\text{for all }s\in\mathcal{S}, a\in\mathcal{A} \notag
\end{align}

Since $\Psi\left(s, a, q^*\right)=0$, $f(s,a,q) = \mathcal{L}_{BE}(Q)(s,a)$, where $q(s)=\left[Q\left(s, a^{\prime}\right)\right]_{a^{\prime} \in \mathcal{A}}$. This finishes the proof.

\end{proof}
\begin{thm}[\cite{bolte2023subgradient}]\label{thm:bolte}
Let $f:\mathbb{R}^n \rightarrow \mathbb{R}$ be a differentiable function. If a path $q:[0, \infty) \rightarrow \mathbb{R}^n$ is a absolutely continuous path in $\mathbb{R}^n$, $f$ admits the chain rule on the path $q(t)$ as

$$
f(q(T))-f(q(0))=\int_0^T f_q(q(t))[\dot{q}(t)] d t
$$
where $\dot{q}(t)$ is the derivative of the function path $q(t)$ with respect to $t$ and $T>0$.
\end{thm}

\begin{lem}[Positive smallest eigenvalue]\label{lem:smallestEigen} Suppose that the discount factor $\beta<1$. Then for there exists $\tilde{\lambda}>0$ such that for all $s\in\mathcal{S}$ and $a\in\mathcal{A}$, $\lambda_{\min}(\Psi_q\left(s, a, q^{\prime}\right) \Psi_q\left(s, a, q^{\prime \prime}\right)^{\top})> \tilde{\lambda}$ holds for any choice of $ q^{\prime}, q^{\prime \prime} \in \mathcal{G}(s)$.
\end{lem}
\begin{proof}
First, note that we can define the policy $\pi_q(a|s) = \frac{\exp q_{(a)}}{\sum_{a^{\prime }} \exp q_{(a^\prime)}}$ for $q\in\mathbb{R}^{|\mathcal{A}|}$, where $x_{(a)}$ implies the $a$th element of vector $x$. 
\begin{align}
     \frac{\partial \Psi(s, a, q)}{\partial q_{(a^{\prime})}}&=\beta \mathbb{E}_{s^{\prime} \sim P(s, a)}\left[\pi_q\left(a^{\prime} \mid s^{\prime}\right)\right]-\delta_{a, a^{\prime}}\notag
\end{align}
That is, $\Psi_q(s, a, q)=\beta \mu_q-e_q$, where $\mu_q=\mathbb{E}_{s^{\prime} \sim P(s, a)}\left[\pi_q\left(a^{\prime} \mid s^{\prime}\right)\right]$ is a probability vector, as it's an expectation over probability distributions. Then for any choice of $q^{\prime}, q^{\prime \prime} \in \mathcal{G}(s)$, denoting $\mu_{q^\prime} = \mu^\prime$ and $\mu_{q^{\prime\prime}} = \mu^{\prime\prime}$
\begin{align}
    \lambda\left(\Psi_q\left(s, a, q^{\prime}\right) \Psi_q\left(s, a, q^{\prime \prime}\right)^{\top}\right)& = \lambda\left(\left(\beta \mu^{\prime}-e_a\right)\left(\beta \mu^{\prime \prime}-e_a\right)^{\top}\right)\notag
    \\
    &= \left(\beta \mu^{\prime}-e_a\right)^{\top}\left(\beta \mu^{\prime \prime}-e_a\right) \notag
    \\
    &=\beta^2\left(\mu^{\prime}\right)^{\top} \mu^{\prime \prime}-\beta \mu^{\prime}(a)-\beta \mu^{\prime \prime}(a)+1 \notag
    \\
    &\ge \beta^2 \mu^{\prime}(a) \mu^{\prime \prime}(a)-\beta \mu^{\prime}(a)-\beta \mu^{\prime \prime}(a)+1 \notag
    \\
    &=(1-\beta \mu^{\prime}(a))(1-\beta \mu^{\prime\prime}(a))
    \\
    &\ge (1-\beta)^2 \notag
\end{align}
Since $\beta\in(0,1)$, $\tilde{\lambda} = (1-\beta)^2$ serves as the uniform lower bound of $\lambda_{\min}(\Psi_q\left(s, a, q^{\prime}\right) \Psi_q\left(s, a, q^{\prime \prime}\right)^{\top})$ for all $s\in\mathcal{S}$ and $a\in\mathcal{A}$, for any choice of $q^{\prime}, q^{\prime \prime} \in \mathcal{G}(s)$.
\end{proof}
\begin{lem}\label{lem:smoothBellman} $|(\mathcal{T}Q-Q)(s,a)-(\mathcal{T}Q^\ast-Q^\ast)(s,a)| \le (1+\beta)\left\|Q\left(s^{\prime}, \cdot\right)-Q^*\left(s^{\prime}, \cdot\right)\right\|_{\infty}$ for all $s\in\mathcal{S}$ and $a\in\mathcal{A}$.
\end{lem}
\begin{proof}

   \begin{align}
       &|(\mathcal{T}Q-Q)(s,a)-(\mathcal{T}Q^\ast-Q^\ast)(s,a)| \notag
       \\
       &=|\beta \cdot \mathbb{E}_{s^{\prime} \sim P(s, a)}\left[\log \left(\sum_{a^{\prime} \in \mathcal{A}} \exp Q\left(s^{\prime}, a^{\prime}\right)\right)-\log \left(\sum_{a^{\prime} \in \mathcal{A}} \exp Q^\ast\left(s^{\prime}, a^{\prime}\right)\right)  \mid s, a\right] +(Q^\ast(s, a)-Q(s, a))|\notag
       \\
       &\le|\beta \cdot \mathbb{E}_{s^{\prime} \sim P(s, a)}\left[\left\|Q\left(s^{\prime}, \cdot\right)-Q^*\left(s^{\prime}, \cdot\right)\right\|_{\infty}\right] + \left|Q^*(s, a)-Q(s, a)\right| \tag{logsumexp Liptshitz in 1}
       \\
       &\le(\beta+1)\left\|Q\left(s^{\prime}, \cdot\right)-Q^*\left(s^{\prime}, \cdot\right)\right\|_{\infty}\notag
   \end{align}
\end{proof}




\subsection{Proof of Theorem \ref{thm:NLLenjoyPL} (NLL loss satisfying the PL condition)}
From Lemma \ref{lem:KLPL} and Lemma \ref{lem:f1f2sumPL}, ${L}_{NLL}(s,a)$ and $\frac{1}{|\mathcal{D}|}\sum_{(s,a)\in\mathcal{D}}\mathcal{L}_{NLL}(s,a)$ are PL. 

\noindent What remains is to show that $\mathbb{E}_{(s, a) \sim \pi^*, \nu_0}\left[-\log \left(\hat{p}_Q(a \mid s)\right)\right]$ satisfies PL. From Lemma \ref{lem:minMLE}, we know
\begin{align}
\mathbb{E}_{(s, a) \sim \pi^*, \nu_0}\left[-\log \left(\hat{p}_Q(a \mid s)\right)\right] =\mathbb{E}_{s \sim \pi^*, \nu_0}\left[D_{K L}\left(\pi^*(\cdot \mid s) \| \hat{p}_Q(\cdot \mid s)\right)\right]+\mathbb{E}_{(s, a) \sim \pi^*, \nu_0}\left[\ln \pi^*(a \mid s)\right] \notag
\end{align}
Note that the second term is not dependent on $Q$. Therefore, we will instead show that the PL condition holds for $\mathbb{E}_{s \sim \pi^*, \nu_0}\left[D_{K L}\left(\pi^*(\cdot \mid s) \| \hat{p}_Q(\cdot \mid s)\right)\right]$. Since $\mathbb{E}_{s \sim \pi^*, \nu_0}\left[D_{K L}\left(\pi^*(\cdot \mid s) \| \hat{p}_Q(\cdot \mid s)\right)\right]$ is convex, by \cite{liao2024error}, showing that  $\mathbb{E}_{s \sim \pi^*, \nu_0}\left[D_{K L}\left(\pi^*(\cdot \mid s) \| \hat{p}_Q(\cdot \mid s)\right)\right]$ is PL is equivalent to showing that $\mathbb{E}_{s \sim \pi^*, \nu_0}\left[D_{K L}\left(\pi^*(\cdot \mid s) \| \hat{p}_Q(\cdot \mid s)\right)\right]$ satisfies Quadratic Growth (QG) condition, i.e., there exists $c^\prime>0$ such that
\begin{align}
    &\mathbb{E}_{s \sim \pi^*, \nu_0}\left[D_{K L}\left(\pi^*(\cdot \mid s) \| \hat{p}_Q(\cdot \mid s)\right)\right]-\mathbb{E}_{s \sim \pi^*, \nu_0}\left[D_{K L}\left(\pi^*(\cdot \mid s) \| \hat{p}_{Q^\ast}(\cdot \mid s)\right)\right] \geq 
c^{\prime}\left\|Q-Q^*\right\|_{L^2\left(\pi^*, v_0\right)}^2 \notag
\end{align}
\noindent But note that
\begin{align}
    &\mathbb{E}_{s \sim \pi^*, \nu_0}\left[D_{K L}\left(\pi^*(\cdot \mid s) \| \hat{p}_Q(\cdot \mid s)\right)\right]-\mathbb{E}_{s \sim \pi^*, \nu_0}\left[D_{K L}\left(\pi^*(\cdot \mid s) \| \hat{p}_{Q^\ast}(\cdot \mid s)\right)\right] \notag
     \\
     &=\mathbb{E}_{s \sim \pi^*, \nu_0}\left[D_{K L}\left(\pi^*(\cdot \mid s) \| \hat{p}_Q(\cdot \mid s)\right)-D_{K L}\left(\pi^*(\cdot \mid s) \| \hat{p}_{Q^\ast}(\cdot \mid s)\right)\right] \notag
     \\
    &\ge \mathbb{E}_{(s, a) \sim \pi^*, \nu_0}\left[c(s,a)^2(Q(s,a)-Q^\ast(s,a))^2\right] \tag{Lemma \ref{lem:KLPL} and convexity}
    \\
    &=c^2\|Q-Q^\ast\|^2_{L^2(\pi^\ast, \nu_0)} \notag
\end{align}
\noindent where $c(s,a)>0$ is the QG constant for $(s,a)$ and $c = \inf_{(s,a)\in \mathcal{S}\times\mathcal{A}} c(s,a)$. Done. \QED
%By the Lemma \ref{lem:KLPL} below, $D_{K L}\left(\pi^*(\cdot \mid s) \| \hat{p}_Q(\cdot \mid s)\right)$ satisfies PL condition for each $s\in\mathcal{S}$. 


\iffalse

Now note that $\left\{Q \in \mathcal{Q} \mid D_{K L}\left(\pi^*(\cdot \mid s) \| \hat{p}_Q(\cdot \mid s)\right)=0\right.$ for all $\left.s \in \overline{\mathcal{S}}\right\}$ is nonempty if $Q^\ast \in \mathcal{Q}$, because 
\begin{align}
    &\{Q\in\mathcal{Q}\mid D_{KL}(\pi^*(\cdot\mid s) \| \hat{p}_{Q}(\cdot\mid s))=0 \text{ for all }s\in\bar{\mathcal{S}}\} \notag
    \\
    &=\{Q\in\mathcal{Q}\mid \hat{p}_Q(\cdot \mid s)=\pi^*(\cdot \mid s)\; \; \text{a.e.} \text{ for all }s\in\bar{\mathcal{S}}\}\notag
    \\
    &= \{Q\in\mathcal{Q}\mid \frac{\hat{p}_Q\left(a_1 \mid s\right)}{\hat{p}_Q\left(a_2 \mid s\right)}=\frac{\pi^*\left(a_1 \mid s\right)}{\pi^*\left(a_2 \mid s\right)} \quad \forall a_1, a_2 \in \mathcal{A}, s\in\bar{\mathcal{S}}\}\notag   
    \\
    &=\left\{Q \in \mathcal{Q} \mid \exp (Q(s,a_1)-Q(s,a_2))=\exp \left(Q^*(s,a_1)-Q^*(s,a_2)\right) \quad \forall a_1, a_2\in\mathcal{A}, s\in\bar{\mathcal{S}} \right\}\notag
    \\
    &=\left\{Q \in \mathcal{Q} \mid Q(s,a_1)-Q(s,a_2)= Q^*(s,a_1)-Q^*(s,a_2) \quad \forall a_1, a_2\in\mathcal{A}, s\in\bar{\mathcal{S}}\right\} \notag
\end{align}


Therefore, $\mathbb{E}_{s \sim \pi^*, \nu_0}\left[D_{K L}\left(\pi^*(\cdot \mid s) \| \hat{p}_Q(\cdot \mid s)\right)\right]$ also satisfies PL by Lemma \ref{lem:expDKLalsoPL}, finishing the proof.
\QED

\fi


\begin{lem}\label{lem:KLPL}
$D_{K L}\left(\pi^*(\cdot \mid s) \| \hat{p}_Q(\cdot \mid s)\right)$ satisfies the PL condition for each $s\in\mathcal{S}$. This implies that $-\log \left(\hat{p}_Q(\cdot \mid s)\right) =D_{K L}\left(\pi^*(\cdot \mid s) \| \hat{p}_Q(\cdot \mid s)\right)+\ln \pi^*(\cdot\mid s)\notag$ is also PL for each $s\in \mathcal{S}$.
\end{lem}
\begin{proof}
Note that
\begin{align}
     \nabla_{Q(s, \cdot)}D_{K L}\left(\pi^*(\cdot \mid s) \| \hat{p}_Q(\cdot \mid s)\right)&= \nabla_{Q(s, \cdot)}\left(-\sum_a \pi^*(a \mid s) \log \hat{p}_Q(a \mid s)\right) \notag
     \\
     &=-\sum_a \pi^*(a \mid s)\left(\delta_{a, a^{\prime}}-\hat{p}_Q\left(a^{\prime} \mid s\right)\right)\notag
     \\
     &=-\left[\pi^*\left(a^{\prime} \mid s\right)-\hat{p}_Q\left(a^{\prime} \mid s\right) \sum_a \pi^*(a \mid s)\right]_{a^{\prime} \in \mathcal{A}}\notag
     \\
     &=\left[\hat{p}_Q\left(a^{\prime} \mid s\right)-\pi^*\left(a^{\prime} \mid s\right)\right]_{a^{\prime} \in \mathcal{A}} \notag
\end{align}
Then, 
\begin{align}
    \|\nabla_{Q(s,)} D_{K L}\left(\pi^*(\cdot \mid s) \| \hat{p}_Q(\cdot \mid s)\right)\|^2 & = \|\left[\hat{p}_Q\left(a^{\prime} \mid s\right)-\pi^*\left(a^{\prime} \mid s\right)\right]_{a^{\prime} \in \mathcal{A}}\|_2^2\notag
    \\
    & \ge \frac{1}{|\mathcal{A}|} \|\left[\hat{p}_Q\left(a^{\prime} \mid s\right)-\pi^*\left(a^{\prime} \mid s\right)\right]_{a^{\prime} \in \mathcal{A}}\|_1^2 \notag
    \\
    & = \frac{1}{|\mathcal{A}|} \text{TV}\left(\hat{p}_Q\left(\cdot \mid s\right),\pi^*\left(\cdot \mid s\right)\right)^2 \notag
    \\
    &\ge \frac{\alpha_Q \ln 2}{|\mathcal{A}|} D_{K L}\left(\pi^*(\cdot \mid s) \| \hat{p}_Q(\cdot \mid s)\right) \notag
\end{align}
where,
\begin{itemize}
    \item TV denotes the total variation distance.
    \item The last inequality is from Lemma \ref{lem:RevPinsker}, where $\alpha_Q:=\min _{a \in A_{+}} Q(s,a)>0$ with $A_{+}=\{a\in\mathcal{A}$ : $Q(s,a)>0\}$.
\end{itemize}
\end{proof}

\begin{lem}[Reverse Pinsker's inequality]\label{lem:RevPinsker}
    \begin{align}
        D(P \| Q)&=\sum_{a \in A_{+}} P(a) \log _2 \frac{P(a)}{Q(a)} \leq \frac{1}{\ln 2} \sum_{a \in A_{+}} P(a)\left(\frac{P(a)}{Q(a)}-1\right) \notag
        \\
        &=\frac{1}{\ln 2}\sum_{a \in A_{+}} \frac{(P(a)-Q(a))^2}{Q(a)}+\sum_{a \in A_{+}}(P(a)-Q(a)) \notag
        \\
        &=\frac{1}{\ln 2} \sum_{a \in A_{+}} \frac{(P(a)-Q(a))^2}{Q(a)} \notag
        \\
        &\leq \frac{d(P, Q)^2}{\alpha_Q \cdot \ln 2} \notag
    \end{align}
\end{lem}

\begin{lem}\label{lem:expDKLalsoPL}  Suppose that given fixed $z\in \mathcal{Z}$, a smooth function $f(x,z)$ 1) either satisfies convexity in $x$ or of $\mathcal{C}^2$ in $x$ and 2) satisfies Polyak-Łojasiewicz condition in $x$ with the coefficient $\mu_z>0$, i.e., $$
\left\|\nabla_x f(x,z)\right\|_2^2 \geq 2 \mu_z\left[f(x,z)-f_z^*\right]
$$
where $f_z^*=\min _x f_z(x)$ and $\mu_z>0$. In addition, suppose that $\arg\min_x f(x,z)=\arg\min_x f(x, z^\prime)$ for all $z, z^\prime \in \mathcal{Z}$, where we define the common minimizer as $x^\ast$. Then $F(x):=\mathbb{E}_{z\sim \nu}[f(x,z)]$ satisfies Polyak-Łojasiewicz condition with respect to $x$, given that $\nu$ is a measure defined on $\mathcal{Z}$. That is, 
$$
\left\|\nabla_x F(x)\right\|_2^2 \geq 2 \mu\left[F(x)-F^*\right],
$$
where $F^*:=\min _x F(x)=\mathbb{E}_{z \sim \nu}\left[f_z^*\right]$, and $\mu=\inf _{z \in \mathcal{Z}} \mu_z>0$.
\end{lem}

\begin{proof}
    Since $f$ is smooth and satisfies PL condition with respect to $x$ for given $z\in\mathcal{Z}$, it satisfies the Quadratic Growth (QG) condition \cite{liao2024error}, i.e., for fixed $z\in\mathcal{Z}$,
    there exists $\alpha_z>0$ such that:
$$
f(x,z)-f_z^\ast \geq \alpha_z\left\|x-x^*\right\|^2 \quad \forall x\in\mathcal{X}
$$
Therefore, 
\begin{align}
    F(x)-F^*&=\mathbb{E}_z\left[f(x, z)-f_z^*\right] \notag
    \\
    &\geq \mathbb{E}_z\left[\alpha_z\left\|x-x^*\right\|^2\right] \notag
    \\
    &\geq \alpha\left\|x-x^*\right\|^2 \quad (\alpha:=\inf _z \alpha_z>0) \notag
\end{align}
This implies that $F(x)$ satisfies the QG condition in $x$. If $f$ satisfies convexity, then by \citet{liao2024error}, Quadratic growth and PL are equivalent; if $f$ is of $\mathcal{C}^2$, then by \citet{rebjock2023fast}, Quadratic Growth and PL are equivalent. Therefore, $F(x)$ satisfies PL.
\end{proof}


\subsection{Proof of Lemma \ref{lem:f1f2sumPL}}
Let $f_1(Q):=\mathcal{L}_{NLL}(Q)$ and $f_2(Q):=\mathcal{L}_{BE}(Q)$. 
 Let $M_1:=\left\{Q \in \mathcal{Q}: f_1(Q)=f_1^*\right\}, $ and $ M_2:=\left\{Q \in \mathcal{Q}: f_2(Q)=f_2^*\right\}$.
 By Theorem \ref{thm:mainopt}, the minimizer of $f_1+f_2$ is in both the minimizer of $f_1$ and the minimizer $f_2$. Therefore, by Lemma \ref{lem:linsumPL}, $f_1+f_2$ is also PL. This implies that $\mathcal{R}_{exp}(Q)$ satisfies the PL. Now, given a finite dataset $\mathcal{D} = \{(s_i, a_i, s_i')\}_{i=1}^N$, note that the empirical risk function $\mathcal{R}_{emp}(Q)$ is equivalent to the expected risk function with the transition probability being $\hat{P}(s'|s,a) = \frac{\sum_{i=1}^N \mathbbm{1}[(s_i,a_i,s_i') = (s,a,s')]}{\sum_{i=1}^N \mathbbm{1}[(s_i,a_i) = (s,a)]}$ and expert policy being $\hat{\pi}^*(a|s) = \frac{\sum_{i=1}^N \mathbbm{1}[(s_i,a_i) = (s,a)]}{\sum_{i=1}^N \mathbbm{1}[s_i = s]}$. (By Theorem \ref{thm:MagnacThesmar}, we know that minimization of this problem is well-defined.) 
 Since the expected risk in this case satisfies the PL condition and has a unique solution, and is equivalent to $\mathcal{R}_{emp}(Q)$, $\mathcal{R}_{emp}(Q)$ satisfies the PL condition and has a unique solution. 

 
 
 \QED


 
\begin{lem}\label{lem:linsumPL}
Suppose that $f_1$ and $f_2$ are both PL and Lipschitz smooth. Furthermore, the minimizer of $f_1+f_2$ is unique, where the minimizer of $f_1$ and the minimizer $f_2$ coincides. Then $f_1+\lambda f_2$ satisfies PL condition for any $\lambda>0$.
\end{lem}

\begin{proof}[Proof of Lemma \ref{lem:linsumPL}] Without loss of generality, we prove that $f := f_1+f_2$ satisfies PL condition. Recall that we say $f$ satisfies $\mu$-$PL$ condition if $2\mu(f(Q)-f\left(Q^\ast\right)) \leq \|\nabla f(Q)\|^2$.
\begin{align}
    \|\nabla f(Q)\|^2&=\left\|\nabla f_1(Q)+\nabla f_2(Q)\right\|^2 \notag
    \\
    &=\left\|\nabla f_1(Q)\right\|^2 + \left\|\nabla f_2(Q)\right\|^2 + 2 \nabla f_1(Q)^{\top} \nabla f_2(Q) \notag
    \\
    &\ge 2\mu_1(f_1(Q)-f_1(Q^\ast)) +  2\mu_2(f_2(Q)-f_2(Q^\ast))+ 2 \nabla f_1(Q)^{\top} \nabla f_2(Q) \notag
    \\
    &\ge 2\mu(f_1(Q)+f_2(Q)-f_1(Q^\ast)-f_2(Q^\ast))+ 2 \nabla f_1(Q)^{\top} \nabla f_2(Q) \notag
    \\
    &=2\mu(f(Q)-f(Q^*) + 2 \nabla f_1(Q)^{\top} \nabla f_2(Q) \notag
    \\
    &\ge 2\mu(f(Q)-f(Q^*)\tag{Lemma \ref{lem:crossPos}}
\end{align}
The last inequality is not trivial, and therefore requires Lemma \ref{lem:crossPos}.

    
\end{proof}

\begin{lem}\label{lem:crossPos}
Suppose that $f_1$ and $f_2$ satisfies PL in $Q$ and minimizer of $f_1+f_2$ is in both the minimizer of $f_1$ and the minimizer $f_2$. Then for all $Q\in\mathcal{Q}$, $\left\langle\nabla f_1(Q), \nabla f_2(Q)\right\rangle \geq 0$.
    
\end{lem}
\begin{proof}
 Let $M_1:=\left\{Q \in \mathcal{Q}: f_1(Q)=f_1^*\right\}, $ and $ M_2:=\left\{Q \in \mathcal{Q}: f_2(Q)=f_2^*\right\}$. From what is assumed, $f_1+f_2$ has a minimizer $Q^*$ that belongs to both $M_1$ and $M_2$.

    Since $f_1$ and $f_2$ are both Lipschitz smooth and satisfy PL condition, they both satisfy Quadratic Growth (QG) condition \cite{liao2024error}, i.e., 
    there exists $\alpha_1, \alpha_2>0$ such that:
$$
f_1(Q)-f_1\left(Q^*\right) \geq \alpha_1\left\|Q-Q^*\right\|^2 \quad \forall Q\in\mathcal{Q}
$$
$$
f_2(Q)-f_2\left(Q^*\right) \geq \alpha_2\left\|Q-Q^*\right\|^2 \quad \forall Q\in\mathcal{Q}
$$
Now suppose, for the purpose of contradiction, that there exists $\hat{Q}\in\mathcal{Q}$ such that $\left\langle\nabla f_1(\hat{Q}), \nabla f_2(\hat{Q})\right\rangle<0$. Consider the direction $d:=-g_1=-\nabla f_1(\hat{Q})$. Then $\nabla f_1(\hat{Q})^{\top} d=g_1^{\top}\left(-g_1\right)=-\left\|g_1\right\|^2<0$ holds. This implies that $f_1(\hat{Q}+\eta d)<f_1(\hat{Q})$. Then QG condition for $f_1$ implies that
$$
\left\|\hat{Q}+\eta d-Q^*\right\|<\left\|\hat{Q}-Q^*\right\|
$$
Now, note that $\nabla f_2(\hat{Q})^{\top} d=g_2^{\top}\left(-g_1\right)=-g_1^{\top} g_2$. Since $g_1^{\top} g_2<0$, $\nabla f_2(\hat{Q})^{\top} d>0$. Therefore, $f_2(\hat{Q}+\eta d)>f_2(\hat{Q})$ for sufficiently small $\eta>0$. That is, $f_2(\hat{Q}+\eta d)-f_2(Q^\ast)>f_2(\hat{Q})-f_2(Q^\ast)$. By QG condition, this implies that $\left\|\hat{Q}+\eta d-Q^*\right\| >\left\|\hat{Q}-Q^*\right\|$. Contradiction.

\end{proof}



\subsection{Proof of Lemma \ref{lem:linPolyNonsingular}}
We consider the function class
$$
Q_{\boldsymbol{\theta}}(s, a)=\boldsymbol{\theta}^{\top} \phi(s, a)
$$
where $\phi: \mathcal{S} \times \mathcal{A} \rightarrow \mathbb{R}^d$ is a known feature map with $\|\phi(s, a)\| \leq B$ almost surely and $\boldsymbol{\theta} \in \mathbb{R}^d$ is the parameter vector. Then for any unit vector $u \in \mathbb{R}^d$, 
$$
\left|u^{\top} \phi(s, a)\right| \leq\|u\|\|\phi(s, a)\|=B
$$
Then by using Hoeffding's Lemma, we have
$$
\mathbb{E}\left[e^{\lambda u^{\top} \phi(s, a)}\right] \leq \exp \left(\frac{\lambda^2 B^2}{2}\right)
$$
Therefore we have
$$
\mathbb{P}\left(\left|u^{\top} \phi(s, a)\right| \geq t\right) \leq 2 e^{-t^2 /\left(2 B^2\right)} \quad \forall t>0
$$
Now for the given dataset $\mathcal{D}$, define
$$M=\left[\begin{array}{c}\phi\left(s_1, a_1\right)^{\top} \\ \phi\left(s_2, a_2\right)^{\top} \\ \vdots \\ \phi\left(s_{|\mathcal{D}|}, a_{|\mathcal{D}|}\right)^{\top}\end{array}\right] \in \mathbb{R}^{|\mathcal{D}| \times d}$$
Note that $D Q_\theta=M$. Then by \citet{rudelson2009smallest}, we have

$$\mathbb{P}\left(\sigma_{\min }(D Q_\theta) \geq \sqrt{|\mathcal{D}|}-\sqrt{d}\right) \geq 1-e^{-C|\mathcal{D}|}$$
provided that the dataset size satisfies $|\mathcal{D}| \geq C d$ with $C>1$.


\subsection{Proof of Theorem \ref{thm:thetaEnjoysPL}}\label{sec:proofOfthetaEnjoysPL}
\begin{proof}[Proof of Theorem \ref{thm:thetaEnjoysPL}] 
For convenience in notation,
$$
f(Q_{\boldsymbol{\theta}}):=\overline{\mathcal{L}_{\mathrm{NLL}}}(Q_{\boldsymbol{\theta}})+\lambda \mathbbm{1}_{a=a_s}\overline{\mathcal{L}_{\mathrm{BE}}}(Q_{\boldsymbol{\theta}})
$$
and denote $Q_{\boldsymbol{\theta}} = Q (\boldsymbol{\theta})$. Set $
h(\boldsymbol{\theta}):=f(Q(\boldsymbol{\theta}))
$, where $f$ is the loss in terms of the function $Q$. Then $
h\left(\boldsymbol{\theta}^*\right)=f\left(Q\left(\boldsymbol{\theta}^*\right)\right)=f\left(Q^*\right)
$
by realizability (Assumption \ref{ass:realizability}). Then
\begin{align}
    \left\|\nabla_\theta h(\boldsymbol{\theta})\right\|_2^2&=\left\|D Q({\boldsymbol{\theta}})^{\top} \nabla_Q f(Q(\boldsymbol{\theta}))\right\|_2^2 \notag
    \\
    &\ge \sigma^2_{\min} \left(D Q(\boldsymbol{\theta})\right)\left\|\nabla_Q f\left(Q({\boldsymbol{\theta})}\right)\right\|_2^2 \tag{$\dim(\mathcal{S}), \dim(\mathcal{A})<\infty$}
    \\
    &\ge m^2 \left\|\nabla_Q f\left(Q({\boldsymbol{\theta}})\right)\right\|_2^2\tag{Assumption \ref{ass:nonSingularJac}}
    \\
    &\ge 2(m^2 c)(f(Q({\boldsymbol{\theta}}))-f(Q^\ast)) \tag{PL in terms of $Q$}
    \\
    &=2(m^2c)(h(\boldsymbol{\theta})-h(\boldsymbol{\theta}^\ast)) \notag
\end{align}
\end{proof}


\iffalse
\begin{proof}[Proof of Lemma \ref{lem:expDKLalsoPL}]
    \begin{align}
    \left\|\nabla_x F(x)\right\|^2&=\left\langle\nabla_x F(x), \nabla_x F(x)\right\rangle \notag
    \\
&=\left\langle\mathbb{E}_{z\sim \nu}\left[\nabla_x f(x, z)\right], \mathbb{E}_{z^{\prime}\sim\nu}\left[\nabla_x f\left(x, z^{\prime}\right)\right]\right\rangle \notag
    \\
    &=\mathbb{E}_{z\sim \nu, z^{\prime}\sim \nu}\left[\left\langle\nabla_x f(x, z), \nabla_x f\left(x, z^{\prime}\right)\right\rangle\right] \notag
    \\
    &=\mathbb{E}_{z \sim \nu}\left[\left\|\nabla_x f(x,z)\right\|^2\right]+\mathbb{E}_{z\neq z^{\prime} \sim \nu}\left[\left\langle\nabla_x f(x,z), \nabla_x f(x,{z^{\prime}})\right\rangle\right] \notag
    \\
    &\geq 2 \mu \mathbb{E}_{z \sim \mu}\left[f(x,z)-f_z^*\right]+\mathbb{E}_{z, z^{\prime} \sim \nu}\left[\left\langle\nabla_x f(x,z), \nabla_x f(x,z^{\prime})\right\rangle\right] \notag
    \\
    &=2 \mu\left[F(x)-F^*\right]+\mathbb{E}_{z\neq z^{\prime} \sim \nu}\left[\left\langle\nabla_x f(x,z), \nabla_x f(x,z^{\prime})(x)\right\rangle\right] \notag
    \\
    &=2 \mu\left[F(x)-F^*\right] \tag{Lemma \ref{lem:Expcross}}
\end{align}
\end{proof}

\begin{lem}\label{lem:Expcross}
    For two PL and Lipschitz smooth $f_1$ and $f_2$ with the same set of minimizers $x^\ast:=\underset{x}{\arg \min } f_1(x)=\underset{x}{\arg \min } f_2(x)$, $\left\langle\nabla_x f_1(x), \nabla_x f_2(x)\right\rangle \ge 0\;\;\forall x \in \mathcal{X}$.
\end{lem}
\begin{proof}[Proof of Lemma \ref{lem:Expcross}]
    Define $\left\|x-x^\ast\right\|:=\inf _{\tilde{x} \in x^\ast}\|x-\tilde{x}\|$. From the fact that $f_1$ and $f_2$ both satisfy PL condition and Lipschitz smoothness, $f_1$ and $f_2$ satisfy the Quadratic Growth condition, i.e., Quadratic Growth (QG) condition \cite{liao2024error}, i.e., 
    there exists $\alpha_1, \alpha_2>0$ such that:
$$
f_1(x)-f_1^\ast \geq \alpha_1\left\|x-x^\ast\right\|^2 \quad \forall x\in\mathcal{X}
$$
$$
f_2(x)-f_2^\ast \geq \alpha_2\left\|x-x^\ast\right\|^2 \quad \forall x\in\mathcal{X}
$$
Now suppose, for the purpose of contradiction, that there exists $\hat{x} \in \mathcal{X}$ such that $
\left\langle\nabla_x f_1(\hat{x}), \nabla_x f_2(\hat{x})\right\rangle<0
$. Define 
$g_1:=\nabla_x f_1(\hat{x}), \quad g_2:=\nabla_x f_2(\hat{x})$ and consider the direction $d:=-g_1$. Then $g_1^{\top} d=-\left\|g_1\right\|^2<0$ holds. For sufficiently small step size $\eta>0$, $
f_1(\hat{x}+\eta d)<f_1(\hat{x})
$ holds.
By the QG condition for $f_1$ :
$$
\left\|\hat{x}+\eta d-x^\ast\right\|<\left\|\hat{x}-x^\ast\right\| .
$$
Now consider $f_2$. Using the assumption $\left\langle g_1, g_2\right\rangle<0$, $
g_2^{\top} d=g_2^{\top}\left(-g_1\right)=-\left\langle g_1, g_2\right\rangle>0
$ holds.
Thus, $f_2(x)$ increases along $d$. For sufficiently small $\eta>0$, 
$
f_2(\hat{x}+\eta d)>f_2(\hat{x})
$ must hold. 
By the QG condition for $f_2$ :
$$\left\|\hat{x}+\eta d-x^\ast\right\|>\left\|\hat{x}-x^\ast\right\| .
$$
This is a contradiction.
\end{proof}
\fi



\subsection{Proof of Proposition \ref{prop:linConvergence} (Global optima convergence under ERM-IRL)}\label{sec:ProofLinConv}


\subsubsection*{1. Optimization error analysis}\;
\\
Define 
$f_1(Q)=\mathbb{E}_{(s, a)\sim \pi^*, \nu_0}  \left[-\log\left(\hat{p}_{Q}(\;a
    \mid s)\right)\right]$ and $f_2(Q)=\mathbb{E}_{(s, a)\sim \pi^*, \nu_0}  \left[ \mathbbm{1}_{a = a_s} \mathcal{L}_{BE}(Q)\left(s,a\right)\right]$.
    By Theorem \ref{thm:mainopt}, there is a unique minimizer $Q^*$ for $f_1+\lambda f_2$, which is the same for all $\lambda>0$. Also, $ f_1+\lambda f_2$ satisfies PL condition by Lemma \ref{lem:f1f2sumPL}.

    In equation \ref{eq:mainopt} of Theorem \ref{thm:mainopt}, we saw that $f_2(Q)$ is actually of form $\max_\zeta$ $f_2(Q, \zeta)$. This implies that minimization of $f_1+\lambda f_2$, a mini-max optimization problem that satisfies two-sided PL. (The inner maximization problem is trivially strongly convex, which implies PL). 
    
    Now denote $$
    f_\lambda(Q, \zeta) :=  f_1(Q)+\lambda f_2(Q, \zeta)$$
    $$g_\lambda(Q) := \max_\zeta f_\lambda(Q, \zeta)
    $$ 
    $$
    g^*_\lambda = \min_Q g_\lambda(Q) = \min_Q \max_\zeta f_\lambda(Q, \zeta)$$
    Note that $$g_\lambda(Q)-g^*_\lambda\ge 0$$
    $$g_\lambda(Q)-f_\lambda(Q, \zeta)\ge 0
    $$
    for any $(Q, \zeta)$. Furthermore, they are both equal to 0 if and only if $(Q, \zeta)$ is a minimax point, which is $Q^\ast$ and $\zeta^\ast$. More precisely, we have
    $$
    |f_\lambda(Q,\zeta) - g_\lambda^\ast| \le (g_\lambda(Q)-g^*_\lambda) + (g_\lambda(Q)-f_\lambda(Q, \zeta))
    $$    
    Therefore, we would like to find $Q, \zeta$ that for $\alpha>0$ $a_\lambda(Q) + \alpha b(Q, \zeta) = 0$, where 
    $$a_\lambda(Q): = g_\lambda(Q)-g^*_\lambda$$
    $$b_\lambda(Q, \zeta): = g_\lambda(Q)-f_\lambda(Q, \zeta)$$
    
    \noindent At iteration 0, algorithm starts from $\hat{Q}_0$ and $\zeta=\zeta_0$. We denote the $Q, \zeta$ value at iteration $T$ as $\hat{Q}_T$ and $\zeta_T$. Also, we define $P_T$ as
    $$P_T:= a_\lambda(\hat{Q}_T) + \alpha b_\lambda(\hat{Q}_T, \zeta_T)$$
    Set that $f_\lambda(Q, \zeta)$ satisfies $\mu_1$-PL for $Q$ and $\mu_2$-PL for $\zeta$. Let $\alpha = 1/10$, $\tau_1^T=\frac{\beta}{\gamma+T}, \tau_2^T=\frac{18 l^2 \beta}{\mu_2^2(\gamma+T)}$ for some $\beta>2 / \mu_1$, $L=l+l^2 / \mu_2$, and $ \gamma>0$ such that $\tau_1^1 \leq \min \left\{1 / L, \mu_2^2 / 18 l^2\right\}$. Then the following Theorem holds.

    \begin{thm}[Theorem 3.3, Yang et al  \cite{yang2020global}]\label{thm:Yang} Consider the setup where $\lambda>0$ is fixed. Then applying Algorithm \ref{alg:estimation1} using stochastic gradient descent (SGD), $P_T$ satisfies
    $$
    P_T \leq \frac{\nu}{\gamma+T}
    $$
    \end{thm}

Note that $a_\lambda$ satisfies the PL condition with respect to $Q$ and smoothness since subtracting a constant from PL is PL. Therefore, $a_\lambda$ satisfies Quadratic Growth (QG) condition by \cite{liao2024error}, i.e., 
$$
C \cdot \mathbb{E}_{(s, a) \sim \pi^*, \nu_0}\left[\left(\hat{Q}_T(s, a)-\hat{Q}_N(s, a)\right)^2\right] \le a_\lambda(Q)-0 \le \mathcal{O}(1/T).
$$
Since $a_\lambda \le P_T$, we can conclude that $ \mathbb{E}_{(s, a) \sim \pi^*, \nu_0}\left[\left(\hat{Q}_T(s, a)-\hat{Q}_N(s, a)\right)^2\right]$ is $\mathcal{O}(1/T)$.
 

\subsubsection*{2. Statistical error analysis.}

\noindent 
Throughout, we closely follow \citet{charles2018stability}. First note that:
\begin{itemize}[leftmargin=*]
\item $Q\in\mathcal{Q}$ is assumed to be bounded, as \citep{rust1994structural} implies that $Q^\ast$ is bounded for $\beta<1$. Therefore, by Lemma \ref{lem:JointLipschitz} (below), the Lipschitz smoothness we proved in Lemma \ref{lem:ConvexityMLE} and \ref{lem:BELipschitz} implies $L$-Lipschitzness for some $L<0$. Since composition of Lipschitz continuous functions are Lipschitz continuous, both $\frac{1}{|\mathcal{D}|} \sum_{(s, a) \in \mathcal{D}} \mathcal{L}_{N L L}(Q_{\boldsymbol{\theta}})(s, a)$ and $\frac{1}{|\mathcal{D}|} \sum_{(s, a) \in \mathcal{D}} \mathcal{L}_{B E}(Q_{\boldsymbol{\theta}})(s, a)$ are Lipschitz continuous in $\theta$. Therefore, $\mathcal{R}_{emp}$ is also $L$-Lipschitz continuous for some $L>0$.
\item As discussed in Lemma \ref{lem:f1f2sumPL} and its proof, $\mathcal{R}_{emp}$ satisfies $\mu$-PL condition for some $\mu>0$ and has a unique minimizer. 
\end{itemize}
Denote the minimizer of empirical risk function $\mathcal{R}_{emp}$ for the data set $\mathcal{D}$ as $Q^\ast_{D}$, where $|\mathcal{D}|=N$. Then by \citet{charles2018stability}, $Q^\ast_{D}$ and $Q^\ast$ satisfies
$$
\bigl|\mathbb{E}_{\mathcal{D}}\bigl[\mathcal{R}_{exp}(Q^\ast_{\mathcal{D}})-\mathcal{R}_{emp}(Q^\ast_{\mathcal{D}};\mathcal{D})\bigr]\bigr| \le \frac{2L^2}{\mu N}.
$$
$$
\bigl|\mathbb{E}_{\mathcal{D}}\bigl[\mathcal{R}_{exp}(Q^\ast)-\mathcal{R}_{emp}(Q^\ast;\mathcal{D})\bigr]\bigr| \le \frac{2L^2}{\mu N}.
$$
Since
\begin{align}
    &\mathcal{R}_{\exp }\left(Q_D^*\right)-\mathcal{R}_{\exp }\left(Q^*\right) \notag
    \\
    &=\left[\mathcal{R}_{\exp }\left(Q_D^*\right)-\mathcal{R}_{\text {emp }}\left(Q_D^* ; \mathcal{D}\right)\right]+\underbrace{\left[\mathcal{R}_{\text {emp }}\left(Q_D^* ; \mathcal{D}\right)-\mathcal{R}_{e m p}\left(Q^* ; \mathcal{D}\right)\right]}_{\leq 0}+\left[\mathcal{R}_{\text {emp }}\left(Q^* ; \mathcal{D}\right)-\mathcal{R}_{\text {exp }}\left(Q^*\right)\right] \notag
\end{align}
We have 
$$
\mathbb{E}_{\mathcal{D}}\bigl[\mathcal{R}_{\exp }\left(Q_D^*\right)-\mathcal{R}_{\exp }\left(Q^*\right)\bigr] \le \frac{4L^2}{\mu N}.
$$
Since smoothness and PL implies Quadratic growth (QG) condition \citep{liao2024error}, we have 
$$
\mathbb{E}_{\mathcal{D}}\bigl[\mathbb{E}_{(s, a) \sim \pi^*, v_0}\left[\left(Q^\ast_\mathcal{D}(s, a)-Q^\ast(s, a)\right)^2\right]\bigr] \le  C\frac{4 L^2}{\mu N}
$$

\begin{lem}
\label{lem:JointLipschitz}
  Let $f: \mathcal{Q} \to \mathbb{R}$ be a differentiable function defined on a space of bounded functions $\mathcal{Q} \subseteq L^2(\mu)$, where $\mathcal{Q}$ is assumed to be bounded in $L^2(\mu)$. That is, there exists a constant $M > 0$ such that for all $Q \in \mathcal{Q}$,
$$
\| Q \|_{L^2(\mu)} \leq M.
$$
If $f$ is differentiable in the Fréchet sense, then $f$ is Lipschitz continuous in the $L^2(\mu)$ norm. That is, there exists a constant $K > 0$ such that for all $Q_1, Q_2 \in \mathcal{Q}$,
$$
|f(Q_1) - f(Q_2)| \leq K \|Q_1 - Q_2\|_{L^2(\mu)}.
$$
\end{lem}

\begin{proof}
    Since $f$ is differentiable, we use the mean value theorem in Banach spaces (see, e.g., \citet{yosida2012functional}). Specifically, for any $Q_1, Q_2 \in \mathcal{Q}$, there exists some intermediate function $\tilde{Q}$ on the line segment between $Q_1$ and $Q_2$ such that:
$$
f(Q_1) - f(Q_2) = \langle \nabla f(\tilde{Q}), Q_1 - Q_2 \rangle_{L^2(\mu)}.
$$
Applying the Cauchy-Schwarz inequality in $L^2(\mu)$, we obtain:
$$
|f(Q_1) - f(Q_2)| = |\langle 
\nabla f(\tilde{Q}), Q_1 - Q_2 \rangle_{L^2(\mu)}|
$$
$$
\leq \|\nabla f(\tilde{Q})\|_{L^2(\mu)} \cdot \|Q_1 - Q_2\|_{L^2(\mu)}.
$$
Since $\mathcal{Q}$ is bounded in $L^2(\mu)$, there exists a constant $B > 0$ such that:
$$
\sup_{Q \in \mathcal{Q}} \|\nabla f(Q)\|_{L^2(\mu)} \leq B.
$$
Thus, we can take $K = B$, yielding the desired Lipschitz continuity bound:
$$
|f(Q_1) - f(Q_2)| \leq B \|Q_1 - Q_2\|_{L^2(\mu)}.
$$
\end{proof}

\iffalse


\begin{lem}
$$
\overline{\mathcal{L}_{\mathrm{BE}}}(Q)-\frac{1}{N}\sum_{(s,a,s)\in \mathcal{D}} \bigl(\bigl(\hat{\mathcal{T}} Q\left(s, a, s^{\prime}\right)-Q(s, a)\bigr)^2 -\min_\zeta\beta^2  \left(V_{Q}\left(s^{\prime}\right)-\zeta\right)^2\bigr)=\mathcal{O}(1/N) $$
\end{lem}
\begin{proof}
According to the proof procedure of Lemma \ref{lem:OurBiconj}, we have
    \begin{align}
    &\overline{\mathcal{L}_{\mathrm{BE}}}(Q)-\frac{1}{N}\sum_{(s,a,s)\in \mathcal{D}} \bigl(\bigl(\hat{\mathcal{T}} Q\left(s, a, s^{\prime}\right)-Q(s, a)\bigr)^2 -\min_\zeta\beta^2  \left(V_{Q}\left(s^{\prime}\right)-\zeta\right)^2\bigr)\notag
    \\
    &=\overline{\mathcal{L}_{\mathrm{BE}}}(Q)-\frac{1}{N}\sum_{(s,a,s)\in \mathcal{D}} \bigl(\bigl(\hat{\mathcal{T}} Q\left(s, a, s^{\prime}\right)-Q(s, a)\bigr)^2 -\min_\rho
    (\rho - \hat{\mathcal{T}}Q(s, a, s^\prime))^2\bigr)\notag
    \end{align}
    By adapting Lemma E.5 of \citet{touati2020sharp} for expert training data, we have $$\overline{\mathcal{L}_{\mathrm{BE}}}(Q)-\frac{1}{N}\sum_{(s,a,s)\in \mathcal{D}} \bigl(\bigl(\hat{\mathcal{T}} Q\left(s, a, s^{\prime}\right)-Q(s, a)\bigr)^2 -\min_\rho
    (\rho - \hat{\mathcal{T}}Q(s, a, s^\prime))^2\bigr)=\mathcal{O}(1/N)$$This finishes the proof.
\end{proof}


\begin{lem} 
\;
\\
With $|\mathcal{D}|=N$,
    $|\mathbb{E}_{(s, a) \sim \pi^*, \nu_0}\left[-\log \left(\hat{p}_Q(a \mid s)\right)\right]-\frac{1}{|\mathcal{D}|}\sum_{(s,a,s^\prime)\in \mathcal{D}} (-\log(\hat{p}_Q(a \mid s)))| = \mathcal{O}(1/\sqrt{N})$.
\end{lem}
\begin{proof}
Define 
$$
F(\mathcal{D})=\frac{1}{N} \sum_{\left(s, a, s^{\prime}\right) \in \mathcal{D}}\left[-\log \hat{p}_Q(a \mid s)\right]
$$
Consider replacing one data sample $\left(s_i, a_i, s_i^{\prime}\right)$ in $\mathcal{D}$ with another independent sample, resulting $\mathcal{D}^\prime$. This leads to a different estimate $\hat{Q}_N^{\prime}$ and corresponding softmax policy $\hat{p}_Q^{\prime}$. 
\end{proof}


Combining the above two lemma, we have
\begin{align}
  &\frac{1}{N}\bigl[\sum_{(s,a,s^\prime)\in \mathcal{D}}\bigl({-\log \left(\hat{p}_Q(a \mid s)\right)}\bigr)+ 
\lambda\mathbbm{1}_{a=a_s}\notag
    \\
    &\bigl(  \sum_{(s,a,s^\prime)\in \mathcal{D}} {\bigl(\hat{\mathcal{T}} Q\left(s, a, s^{\prime}\right)-Q(s, a)\bigr)^2}  -\beta^2 \min_{\zeta\in \mathbb{R}^{S\times A}} 
   \sum_{(s,a,s^\prime)\in \mathcal{D}} {\left(V_{Q}\left(s^{\prime}\right)-\zeta(s,a)\right)^2}\bigr) \bigr]\notag
   \\
   &- \notag
   \\
   & \mathbb{E}_{(s, a) \sim \pi^*, \nu_0, s^{\prime} \sim P(s, a)}\bigl[-\log \left(\hat{p}_Q(a \mid s)\right) + \lambda\mathbbm{1}_{a=a_s}\bigl\{\bigl(\hat{\mathcal{T}} Q\left(s, a, s^{\prime}\right)-Q(s, a)\bigr)^2 \notag
    \\
    & \quad -\beta^2  \min_{\zeta\in \mathbb{R}^{S\times A}}  \mathbb{E}_{(s, a) \sim \pi^*, \nu_0, s^{\prime} \sim P(s, a)} \bigl(\left(V_{Q}\left(s^{\prime}\right)-\zeta(s,a)\right)^2\bigr) \bigr\} \bigr]\notag
    &
    \\
     &= \mathcal{O}(1/\sqrt{N}). \notag
\end{align}
\noindent 
Now recall that we showed that the sum of $\overline{\mathcal{L}_{\mathrm{BE}}}(Q)$ term and $\mathbb{E}_{(s, a) \sim \pi^*, \nu_0}\left[-\log \left(\hat{p}_Q(a \mid s)\right)\right]$ term satisfy PL. Since Those two terms are also smooth, the sum satisfies the Quadratic Growth condition (\cite{liao2024error}). Therefore
\begin{align}
  &\frac{1}{N}\bigl[\sum_{(s,a,s^\prime)\in \mathcal{D}}\bigl({-\log \left(\hat{p}_Q(a \mid s)\right)}\bigr)+ 
\lambda\mathbbm{1}_{a=a_s}\notag
    \\
    &\bigl(  \sum_{(s,a,s^\prime)\in \mathcal{D}} {\bigl(\hat{\mathcal{T}} Q\left(s, a, s^{\prime}\right)-Q(s, a)\bigr)^2}  -\beta^2 \min_{\zeta\in \mathbb{R}^{S\times A}} 
   \sum_{(s,a,s^\prime)\in \mathcal{D}} {\left(V_{Q}\left(s^{\prime}\right)-\zeta(s,a)\right)^2}\bigr) \bigr]\notag
   \\
   &- \notag
   \\
   & \mathbb{E}_{(s, a) \sim \pi^*, \nu_0, s^{\prime} \sim P(s, a)}\bigl[-\log \left(\hat{p}_Q(a \mid s)\right) + \lambda\mathbbm{1}_{a=a_s}\bigl\{\bigl(\hat{\mathcal{T}} Q\left(s, a, s^{\prime}\right)-Q(s, a)\bigr)^2 \notag
    \\
    & \quad -\beta^2  \min_{\zeta\in \mathbb{R}^{S\times A}}  \mathbb{E}_{(s, a) \sim \pi^*, \nu_0, s^{\prime} \sim P(s, a)} \bigl(\left(V_{Q}\left(s^{\prime}\right)-\zeta(s,a)\right)^2\bigr) \bigr\} \bigr]\notag
    &
    \\
     &\ge  C\cdot\mathbb{E}_{(s, a) \sim \pi^*, \nu_0}\left[\left(Q(s, a)-\hat{Q}_N(s, a)\right)^2\right] \notag
\end{align}
 for some $C$. This implies that $\cdot\mathbb{E}_{(s, a) \sim \pi^*, \nu_0}\left[\left(Q(s, a)-\hat{Q}_N(s, a)\right)^2\right]=\mathcal{O}(1/\sqrt{N})$.






\subsection{Temp}

\begin{lem}[Near-strong convexity of $\overline{\mathcal{L}_{NLL}}(Q)$]\label{lem:NLLnearstrongconv} Hessian of $\overline{\mathcal{L}_{NLL}}(Q)$ has strictly positive eigenvalues except one eigenvalue of 0, of which eigenvector is $\mathbf{1}$. 

    
\end{lem}
\begin{proof}
From Lemma \ref{lem:minMLE}, we know
\begin{align}
\mathbb{E}_{(s, a) \sim \pi^*, \nu_0}\left[-\log \left(\hat{p}_Q(a \mid s)\right)\right] =\mathbb{E}_{s \sim \pi^*, \nu_0}\left[D_{K L}\left(\pi^*(\cdot \mid s) \| \hat{p}_Q(\cdot \mid s)\right)\right]+\mathbb{E}_{(s, a) \sim \pi^*, \nu_0}\left[\ln \pi^*(a \mid s)\right] \notag
\end{align}
Note that the second term is not dependent on $Q$. Therefore, we can instead prove that the first term $\mathbb{E}_{s \sim \pi^*, \nu_0}\left[D_{K L}\left(\pi^*(\cdot \mid s) \| \hat{p}_Q(\cdot \mid s)\right)\right]$ has strictly positive eigenvalues except one eigenvalue of 0, of which eigenvector is $\mathbf{1}$. First, note that
\begin{align}
     \nabla_{Q(s, \cdot)}D_{K L}\left(\pi^*(\cdot \mid s) \| \hat{p}_Q(\cdot \mid s)\right)&= \nabla_{Q(s, \cdot)}\left(-\sum_a \pi^*(a \mid s) \log \hat{p}_Q(a \mid s)\right) \notag
     \\
     &=-\sum_a \pi^*(a \mid s)\left(\delta_{a, a^{\prime}}-\hat{p}_Q\left(a^{\prime} \mid s\right)\right)\notag
     \\
     &=-\left[\pi^*\left(a^{\prime} \mid s\right)-\hat{p}_Q\left(a^{\prime} \mid s\right) \sum_a \pi^*(a \mid s)\right]_{a^{\prime} \in \mathcal{A}}\notag
     \\
     &=\left[\hat{p}_Q\left(a^{\prime} \mid s\right)-\pi^*\left(a^{\prime} \mid s\right)\right]_{a^{\prime} \in \mathcal{A}} \notag
\end{align}
Also, from $\frac{\partial \hat{p}_Q(a\mid s)}{\partial Q_{a^{\prime}}}=\hat{p}_Q(a\mid s)\left(\delta_{a, a^{\prime}}-\hat{p}_Q\left(a^{\prime}\mid s\right)\right)$, Hessian $H=\operatorname{diag}\left(\hat{p}_Q\right)-\hat{p}_Q \hat{p}_Q^{\top}$. Interestingly, this Hessian is equivalent to the covariance matrix of random vector $X \in \mathbb{R}^{|\mathcal{A}|}$, defined by $$X(a)= \begin{cases}1 & \text { if } A=a \\ 0 & \text { otherwise }\end{cases}, \text{ where } \mathbb{P}(A=a)=\hat{p}_Q(a)=\frac{\exp (Q(a))}{\sum_b \exp (Q(b))}$$
\;
\\
We will now show that this Hessian $H$ has only one eigenvector direction with eigenvalue 0, and other eigenvalues are all strictly positive. First, $H \mathbf{1}=\hat{p}_Q-\hat{p}_Q\left(\sum_{a^{\prime}} \hat{p}_Q\left(a^{\prime}\right)\right)=\hat{p}_Q-\hat{p}_Q=0$. 
Now define $V=\left\{v \in \mathbb{R}^{|\mathcal{A}|}: v^{\top} \mathbf{1}=\sum_a v(a)=0\right\}$, which is the subspace that is orthogonal to $\mathbf{1}$. Then for $v \in V$,
$$
v^{\top} H v=v^{\top} \operatorname{Cov}(X) v=\operatorname{Var}(v(A)),
$$
where $v(A)=\sum_a v(a) \delta_{A, a}$ is a scalar-valued random variable depending on the random draw $A \sim \hat{p}_Q$. We have $\operatorname{Var}(v(A))>0$, because:
\begin{itemize}
    \item If $v(a)$ are not all identical, then $v(A)$ is a non-constant random variable under $\hat{p}_Q$. Because $\hat{p}_Q$ assigns positive probability to each action, the random variable $v(A)$ takes on at least two distinct values with positive probability. Hence, its variance $\operatorname{Var}(v(A))$ is strictly greater than zero.
    \item If $v(a)$ were constant for all $a$, say $v(a)=c$, then $\sum_a v(a)=c \sum_a 1=$ $c|\mathcal{A}|$. This contradicts $v\in V$.
\end{itemize}
    
\end{proof} 

\fi
\newpage

\section{Equivalence between Dynamic Discrete choice and Entropy regularized Inverse Reinforcement learning}\label{sec:DDCIRL}


\subsection{Properties of Type 1 Extreme Value (T1EV) distribution}
Type 1 Extreme Value (T1EV), or Gumbel distribution, has a location parameter and a scale parameter. The T1EV distribution with location parameter $\nu$ and scale parameter 1 is denoted as Gumbel $(\nu, 1)$ and has its CDF, PDF, and mean as follows:
$$
\begin{gathered}
\text { CDF: } F(x ; \nu)=e^{-e^{-(x-\nu)}}
\\
\text { PDF: } f(x ; \nu)=e^{-\left((x-\nu)+e^{-(x-\nu)}\right)}
\\
\text { Mean } = \nu + \gamma
\end{gathered}
$$

Suppose that we are given a set of $N$ independent Gumbel random variables $G_i$, each with their own parameter $\nu_i$, i.e. $G_i \sim \operatorname{Gumbel}\left(\nu_i, 1\right)$.

\begin{lem}\label{lem:GumbelMax}
    Let $Z=\max G_i$. Then $Z \sim \operatorname{Gumbel}\left(\nu_Z=\log \sum_{i} e^{\nu_i}, 1\right)$.
\end{lem}
\begin{proof}
    $F_Z(x)=\prod_{i} F_{G_i}(x)=\prod_{i} e^{-e^{-\left(x-\nu_i\right)}}=e^{-\sum_{i} e^{-\left(x-\nu_i\right)}}=e^{-e^{-x} \sum_{i} e^{\nu_i}}=e^{-e^{-\left(x-\nu_Z\right)}}$
\end{proof}

\begin{cor}\label{cor:GumbelOptProb}
    $P\left(G_k>\max _{i \neq k} G_i\right)=\frac{e^{\nu_k}}{\sum_{i} e^{\nu_i}}$.
\end{cor}
\begin{proof}
\begin{align}
    P\left(G_k>\max _{i \neq k} G_i\right)&=\int_{-\infty}^{\infty} P\left(\max _{i \neq k} G_i<x\right) f_{G_{k}}(x) d x\notag
    \\&=\int_{-\infty}^{\infty} e^{-\sum_{i \neq k} e^{-\left(x-\nu_i\right)}} e^{-\left(x-\nu_k\right)} e^{-e^{-\left(x-\nu_k\right)}} d x \notag
    \\&=e^{\nu_k} \int_{-\infty}^{\infty} e^{-e^{-x} \sum_{i} e^{\nu_i}} e^{-x} d x \notag
    \\&=e^{\nu_k}\int_{\infty}^0 e^{-u S} u\left(-\frac{d u}{u}\right)  \; \; \left(\text{Let } \sum_{i} e^{\nu_i}=S, u=e^{-x}\right)\notag
    \\&=e^{\nu_k}\int_0^{\infty} e^{-u S} d u=e^{\nu_k}\left[-\frac{1}{S} e^{-u S}\right]_0^{\infty} = \frac{e^{\nu_k}}{S}\notag 
    \\&=\frac{e^{\nu_k}}{\sum_{i} e^{\nu_i}} \notag
\end{align}
\end{proof}

\begin{lem}\label{lem:ExpofLargerGumbel}
    Let $G_1\sim \text{Gumbel }(\nu_1, 1)$ and $G_2\sim \text{Gumbel }(\nu_2, 1)$. Then $\mathbb{E}\left[G_1 \mid G_1\geq G_2\right]=\gamma + \log \left( 1+ e^{\left(-(\nu_1-\nu_2)\right)} \right)$ holds. 
\end{lem}

\begin{proof}
    Let $\nu_1-\nu_2 = c$. Then $\mathbb{E}\left[G_1 \mid G_1\geq G_2\right]$ is equivalent to $\nu_1 + \frac{\int_{-\infty}^{+\infty} x F(x+c) f(x) \mathrm{d} x}{\int_{-\infty}^{+\infty} F(x+c) f(x) \mathrm{d} x}$, where the pdf $f$ and cdf $F$ are associated with $\text{Gumbel }(0, 1)$, because

    \begin{align}
        P\left(G_1 \leq x, G_1 \geq G_2\right)&=\int_{-\infty}^x F_{G_2}(t) f_{G_1}(t) d t=\int_{-\infty}^x F\left(t-\nu_2\right) f\left(t-\nu_1\right) d t\notag
        \\
        \mathbb{E}\left[G_1 \mid G_1 \geq G_2\right]&=\frac{\int_{-\infty}^{\infty} x F\left(x+c-\nu_1\right) f\left(x-\nu_1\right) d x}{\int_{-\infty}^{\infty} F\left(x+c-\nu_1\right) f\left(x-\nu_1\right) d x}\notag
        \\
        &=\frac{\int_{-\infty}^{\infty}\left(y+\nu_1\right) F(y+c) f(y) d y}{\int_{-\infty}^{\infty} F(y+c) f(y) d y} \notag
        \\
        &=\nu_1+\frac{\int_{-\infty}^{\infty} y F(y+c) f(y) d y}{\int_{-\infty}^{\infty} F(y+c) f(y) d y} \notag
    \end{align}

    
    Now note that 

    $\begin{aligned} \int_{-\infty}^{+\infty} F(x+c) f(x) \mathrm{d} x & =\int_{-\infty}^{+\infty} \exp \{-\exp [-x-c]\} \exp \{-x\} \exp \{-\exp [-x]\} \mathrm{d} x \\ & \stackrel{a=e^{-c}}{=} \int_{-\infty}^{+\infty} \exp \{-(1+a) \exp [-x]\} \exp \{-x\} \mathrm{d} x \\ & =\frac{1}{1+a}\left[\exp \left\{-(1+a) e^{-x}\right\}\right]_{-\infty}^{+\infty} \\ & =\frac{1}{1+a}\end{aligned}$
    \\
    and
    \\
    $\begin{aligned}  \int_{-\infty}^{+\infty} x F(x+c) f(x) \mathrm{d} x&=\int_{-\infty}^{+\infty} x \exp \{-(1+a) \exp [-x]\} \exp \{-x\} \mathrm{d} x \\ & \stackrel{z=e^{-x}}{=} \int_0^{+\infty} \log (z) \exp \{-(1+a) z\} \mathrm{d} z \\ & =\frac{-1}{1+a}\left[\operatorname{Ei}(-(1+a) z)-\log (z) e^{-(1+a) z}\right]_0^{\infty} \\ & =\frac{\gamma+\log (1+a)}{1+a} \\ & \end{aligned}$
    \\
    Therefore, $\mathbb{E}\left[G_1 \mid G_1\geq G_2\right]=\gamma + \nu_k+ \log \left( 1+ e^{\left(-(\nu_1-\nu_2)\right)} \right)$ holds.
\end{proof}

\begin{cor}\label{cor:GumbelMaxasProb} $ \mathbb{E}\left[G_k \mid G_k = \max G_i\right] = \gamma + \nu_k - \log \left(\frac{e^{\nu_k}}{\sum_{i} e^{\nu_i}}\right)$. 
\end{cor}
\begin{proof}

\begin{align}
    \mathbb{E}\left[G_k \mid G_k = \max G_i\right] &= \mathbb{E}\left[G_k \mid G_k \geq \max_{i\neq k} G_i\right]\notag
    \\
    &=\gamma + \nu_k + \log \left( 1+ e^{\left(-(\nu_k-\log\sum_{i\neq k} e^{\nu_i})\right)}\right)\tag{Lemma \ref{lem:ExpofLargerGumbel}}
    \\
    &=\gamma + \nu_k + \log \left( 1+ \frac{\sum_{i\neq k} e^{\nu_i}}{e^{-\nu_k}}\right) \notag
    \\
    &=\gamma + \nu_k + \log \left(\sum_{i} e^{\nu_i}/e^{\nu_k} \right)\notag
    \\
    &=\gamma + \nu_k -\log \left(e^{\nu_k} / \sum_{i} e^{\nu_i} \right)\notag
\end{align}

\end{proof}


\subsection{Properties of entropy regularization}
Suppose we have a choice out of discrete choice set $\mathcal{A} = \{x_i\}_{i=1}^{|\mathcal{A}|}$. A choice policy can be a deterministic policy such as $\operatorname{argmax}_{i \in 1, \ldots, |\mathcal{A}|} x_i$, or stochastic policy that is characterized by $\mathbf{q}\in \triangle_{\mathcal{A}}$. When we want to enforce smoothness in choice, we can regularize choice by newly defining the choice rule 
$$\arg\max _{\mathbf{q} \in \Delta_{\mathcal{A}}}\left(\langle\mathbf{q}, \mathbf{x}\rangle-\Omega(\mathbf{q})\right)$$
where $\Omega$ is a regularizing function. 

\begin{lem}\label{lem:logsumexp_Shannon}
When the regularizing function is constant $-\tau$ multiple of Shannon entropy $H(\mathbf{q})=-\sum_{i=1}^{|\mathcal{A}|} q_i \log \left(q_i\right)$, $$\max _{\mathbf{q} \in \Delta_{\mathcal{A}}}\left(\langle\mathbf{q}, \mathbf{x}\rangle-\Omega(\mathbf{q})\right)=\tau \log \left(\sum_i \exp \left(x_i / \tau\right)\right)$$ and

$$\arg\max _{\mathbf{q} \in \Delta_{\mathcal{A}}}\left(\langle\mathbf{q}, \mathbf{x}\rangle-\Omega(\mathbf{q})\right)=\frac{\exp \left(\frac{x_i}{\tau}\right)}{\sum_{j=1}^n \exp \left(\frac{x_j}{\tau}\right)}$$
\end{lem}
\begin{proof}
    In the following, I will assume $\tau>0$. Let
\begin{align}
G(\mathbf{q})&=\langle\mathbf{q}, \mathbf{x}\rangle-\Omega(\mathbf{q})\notag
\\
&=\sum_{i=1}^n q_i x_i-\tau \sum_{i=1}^n q_i \log \left(q_i\right)
\notag
\\
&=\sum_{i=1}^n q_i \left(x_i-\tau \log \left(q_i\right)\right) \notag
\end{align}


We are going to find the max by computing the gradient and setting it to 0 . We have
$$
\frac{\partial G}{\partial q_i}=x_i-\tau\left(\log \left(q_i\right)+1\right)
$$
and
$$
\frac{\partial G}{\partial q_i \partial q_j}=\left\{\begin{array}{l}
-\frac{\tau}{q_i}, \quad \text { if } i=j \\
0, \quad \text { otherwise. }
\end{array}\right.
$$

This last equation states that the Hessian matrix is negative definite (since it is diagonal and $-\frac{\tau}{q_1}<0$ ), and thus ensures that the stationary point we compute is actually the maximum. Setting the gradient to $\mathbf{0}$ yields $q_i^*=\exp \left(\frac{x_i}{\tau}-1\right)$, however the resulting $\mathbf{q}^*$ might not be a probability distribution. To ensure $\sum_{i=1}^n q_i^*=1$, we add a normalization:
$$
q_i^*=\frac{\exp \left(\frac{x_i}{\tau}-1\right)}{\sum_{j=1}^n \exp \left(\frac{x_j}{\tau}-1\right)}=\frac{\exp \left(\frac{x_i}{\tau}\right)}{\sum_{j=1}^n \exp \left(\frac{x_j}{\tau}\right)} .
$$

This new $\mathbf{q}^*$ is still a stationary point and belongs to the probability simplex, so it must be the maximum. Hence, you get
$$
\begin{aligned}
\max _{\tau H}(\mathbf{x})& =G\left(\mathbf{q}^*\right)=\sum_{i=1}^n \frac{\exp \left(\frac{x_i}{\tau}\right)}{\sum_{j=1}^n \exp \left(\frac{x_j}{\tau}\right)} x_i-\tau \sum_{i=1}^n \frac{\exp \left(\frac{x_1}{\tau}\right)}{\sum_{j=1}^n \exp \left(\frac{x_j}{\tau}\right)}\left(\frac{x_i}{\tau}-\log \sum_{i=1}^n \exp \left(\frac{x_j}{\tau}\right)\right) \\
&= \tau \log \sum_{i=1}^n \exp \left(\frac{x_j}{\tau}\right)
\end{aligned}
$$
as desired.
\end{proof}





\subsection{IRL with entropy regularization}\label{sec:IRLentropy}

\subsubsection*{Markov decision processes} 
Consider an MDP defined by the tuple $\left(\mathcal{S}, \mathcal{A}, P, \nu_0, r, \beta\right)$:
\begin{itemize}
    \item $\mathcal{S}$ and $\mathcal{A}$ denote finite state and action spaces
    \item $P \in \Delta_{\mathcal{S}}^{\mathcal{S} \times \mathcal{A}}$ is a Markovian transition kernel, and $\nu_0 \in \Delta_{\mathcal{S}}$ is the initial state distribution. 
    \item $r \in \mathbb{R}^{\mathcal{S} \times \mathcal{A}}$ is a reward function.
    \item $\beta \in(0,1)$ a discount factor
\end{itemize}
\subsubsection{Agent behaviors}

Denote the distribution of agent's initial state $s_0\in \mathcal{S}$ as $\nu_0$. Given a stationary Markov policy $\pi \in \Delta_{\mathcal{A}}^{\mathcal{S}}$, an agent starts from initial state $s_0$ and make an action $a_h\in \mathcal{A}$ at state $s_h\in \mathcal{S}$ according to $a_h\sim\pi\left(\cdot \mid s_h\right)$ at each period $h$. We use $\mathbb{P}_{\nu_0}^\pi$ to denote the distribution over the sample space $(\mathcal{S} \times \mathcal{A})^{\infty}=\left\{\left(s_0, a_0, s_1, a_1, \ldots\right): s_h \in \mathcal{S}, a_h \in \mathcal{A}, h \in \mathbb{N}\right\}$ induced by the policy $\pi$ and the initial distribution $\nu_0$. We also use $\mathbb{E}_\pi$ to denote the expectation with respect to $\mathbb{P}_{\nu_0}^\pi$. Maximum entropy inverse reinforcement learning (MaxEnt-IRL) makes the following assumption: 

\begin{asmp}[Assumption \ref{ass:IRLoptimaldecision}] Agent follows the policy $\pi^*=\operatorname{argmax}_{\pi \in \Delta_{\mathcal{A}}^{\mathcal{S}}}$
$\mathbb{E}_\pi\left[\sum_{h=0}^{\infty} \beta^h \left(r\left(s_h, a_h\right)+\lambda\mathcal{H}\left(\pi\left(\cdot \mid s_h\right)\right)\right)\right]$, where $\mathcal{H}$ denotes the Shannon entropy and $\lambda$ is the regularization parameter.
\end{asmp}
For the rest of the section, we use $\lambda=1$.
\\
\;
\\
We define the function $V$ as $V(s_{h^\prime})=\mathbb{E}_{\pi^*}\left[\sum_{h=h^\prime}^{\infty} \beta^{h} \left(r\left(s_h, a_h\right)+\mathcal{H}\left(\pi^*\left(\cdot \mid s_h\right)\right)\right)\right]$ and call it the \textit{value function}. According to Assumption \ref{ass:IRLoptimaldecision}, the value function $V$ must satisfy the Bellman equation, i.e., 
\begin{align}
V\left(s\right)&=\max _{\mathbf{q} \in \triangle_\mathcal{A}}\left\{\mathbb{E}_{a\sim\mathbf{q} }\left[r\left(s, a\right)+\beta \cdot \mathbb{E}\left[V\left(s^\prime\right)\mid s, a\right]\right]+\mathcal{H}(\mathbf{q})\right\}\notag
\\
&=\max _{\mathbf{q} \in \triangle_\mathcal{A}}\left\{\sum_{a\in \mathcal{A}} q_a\left(r\left(s, a\right)+\beta \cdot \mathbb{E}\left[V\left(s^\prime\right)\mid s, a\right]\right)+\mathcal{H}(\mathbf{q})\right\}=\max _{\mathbf{q} \in \triangle_\mathcal{A}}\left\{\sum_{a\in \mathcal{A}} q_a Q(s,a)+\mathcal{H}(\mathbf{q})\right\}\label{eq:VandmaxQ}
\\
&=\ln \left[\sum_{a\in \mathcal{A}}\exp\left(r\left(s, a\right)+\beta \cdot \mathbb{E}\left[{V}\left(s^\prime\right)\mid s, a\right]\right)\right]\label{eq:logsumexp_reg}
\\
&=\ln \left[\sum_{a\in \mathcal{A}}\exp\left(Q(s, a)\right)\right]\label{eq:IRLlogsumQ}
\end{align}
and $\mathbf{q}^*:=\arg\max_{\mathbf{q} \in \triangle_\mathcal{A}} \left\{\mathbb{E}_{a\sim\mathbf{q} }\left[r\left(s, a\right)+\beta \cdot \mathbb{E}\left[V\left(s^\prime\right)\mid s, a\right]\right]+\mathcal{H}(\mathbf{q})\right\}$ is characterized by
\\
\begin{align}
\mathbf{q}^* = [q_1^* \ldots q^*_{|\mathcal{A}|}], \text{ where }
    q^*_a= \frac{\exp \left({Q(s, a)}\right)}{\sum_{a^\prime\in \mathcal{A}} \exp \left({Q(s, a^\prime)}\right)} \text{ for } a\in \mathcal{A}  \label{eq:IRLopt}
\end{align}
where:
\begin{itemize}
    \item $Q(s, a):=r\left(s, a\right)+\beta \cdot \mathbb{E}\left[{V}\left(s^\prime\right)\mid s, a\right]$
    \item Equality in equation \ref{eq:logsumexp_reg} and equality in equation \ref{eq:IRLopt} is from Lemma \ref{lem:logsumexp_Shannon}
    
\end{itemize}
\;
\\
This implies that
\begin{align}
    \pi^*(a\mid s) = q^*_a= \frac{\exp \left({Q(s, a)}\right)}{\sum_{a^\prime\in \mathcal{A}} \exp \left({Q(s, a^\prime)}\right)} \text{ for } a\in \mathcal{A}. \notag
\end{align}
\\
In addition to the Bellman equation in terms of value function $V$, 
Bellman equation in terms of choice-specific value function $Q(s,a)$ can be derived by combining $Q(s, a):=r\left(s, a\right)+\beta \cdot \mathbb{E}\left[{V}\left(s^\prime\right)\mid s, a\right]$ and equation \ref{eq:IRLlogsumQ}:
\begin{align}
    Q(s, a)=r(s, a)+\beta \mathbb{E}_{s^\prime \sim P(s, a)}\left[\ln \left(\sum_{a^{\prime} \in \mathcal{A}} \exp \left(Q\left(s^{\prime}, a^{\prime}\right)\right)\right)\mid s, a\right] \notag
\end{align}
\;
\\
We can also derive an alternative form of choice-specific value function $Q(s,a)$ by combining $Q(s, a):=r\left(s, a\right)+\beta \cdot \mathbb{E}_{s^\prime \sim P(s, a)}\left[{V}\left(s^\prime\right)\mid s, a\right]$ and equation \ref{eq:VandmaxQ}:

\begin{align}
    Q(s, a)&=r\left(s, a\right)+\beta \cdot \mathbb{E}_{s^\prime \sim P(s, a)}\left[\max _{\mathbf{q} \in \triangle_\mathcal{A}}\left\{\sum_{a\in \mathcal{A}} q_a Q(s^\prime,a)+\mathcal{H}(\mathbf{q})\right\}\mid s, a\right]\notag
    \\
    &=r\left(s, a\right)+\beta \cdot \mathbb{E}_{s^\prime \sim P(s, a)}\left[\max _{\mathbf{q} \in \triangle_\mathcal{A}}\left\{\sum_{a\in \mathcal{A}} q_a \left(Q(s^\prime,a) - \log q_a\right)\right\}\mid s, a\right]\notag
    \\
    &=r\left(s, a\right)+\beta \cdot \mathbb{E}_{s^\prime \sim P(s, a), a^\prime \sim \pi^*(a\mid \cdot)}\left[ \left(Q(s^\prime,a^\prime) - \log \pi^*(a^\prime\mid s^\prime)\right)\mid s, a\right] \label{eqn:IRLQBellman_new}
    \\
    &=r\left(s, a\right)+\beta \cdot \mathbb{E}_{s^\prime \sim P(s, a)}\left[ \left(Q(s^\prime,a^\prime) - \log \pi^*(a^\prime\mid s^\prime)\right)\mid s, a\right]\text{ for all } a^\prime\in \mathcal{A} \notag
\end{align}
The last line comes from the fact that $Q(s^\prime,a^\prime) - \log \pi^*(a^\prime\mid s^\prime)$ is equivalent to $\log \left(\sum_{a^{\prime} \in \mathcal{A}} \exp \left(Q\left(s^{\prime}, a^{\prime}\right)\right)\right)$, which is a quantity that does not depend on the realization of specific action $a^\prime$.


\subsection{Single agent Dynamic Discrete Choice (DDC) model}\label{sec:SingleDDC}

\subsubsection*{Markov decision processes} 
Consider an MDP $\tau:=\left(\mathcal{S}, \mathcal{A}, P, \nu_0, r, G(\delta,1), \beta \right)$:
\begin{itemize}
    \item $\mathcal{S}$ and $\mathcal{A}$ denote finite state and action spaces
    \item $P \in \Delta_{\mathcal{S}}^{\mathcal{S} \times \mathcal{A}}$ is a Markovian transition kernel, and $\nu_0 \in \Delta_{\mathcal{S}}$ is the initial state distribution. 
    \item $r(s_h,a_h)+\epsilon_{ah}$ is the immediate reward (called the flow utility in the Discrete Choice Model literature) from taking action $a_h$ at state $s_h$ at time-step $h$, where:
    \begin{itemize}
        \item $r \in \mathbb{R}^{\mathcal{S} \times \mathcal{A}}$ is a deterministic reward function
        \item  
    $\epsilon_{ah}\overset{i.i.d.}{\sim}  G(\delta, 1)$ is the random part of the reward, where $G$ is Type 1 Extreme Value (T1EV) distribution (a.k.a. Gumbel distribution). The mean of $G(\delta, 1)$ is $\delta + \gamma$, where $\gamma$ is the Euler constant. 
    \item In the econometrics literature, this reward setting is considered as a result of a combination of two assumptions: conditional independence (CI) and additive separability (AS) \cite{magnac2002identifying}. 
\begin{figure}[H]
    \centering
    \includegraphics[width=0.3\linewidth]{Figures/Gumbel.png}
    \caption{Gumbel distribution $G(-\gamma, 1)$}
\end{figure}
    \end{itemize}
    \item $\beta \in(0,1)$ a discount factor
    
\end{itemize}
\;

\subsubsection{Agent behaviors} Denote the distribution of agent's initial state $s_0\in \mathcal{S}$ as $\nu_0$. Given a stationary Markov policy $\pi \in \Delta_{\mathcal{A}}^{\mathcal{S}}$, an agent starts from initial state $s_0$ and make an action $a_h\in \mathcal{A}$ at state $s_h\in \mathcal{S}$ according to $a_h\sim\pi\left(\cdot \mid s_h\right)$ at each period $h$. We use $\mathbb{P}_{\nu_0}^\pi$ to denote the distribution over the sample space $(\mathcal{S} \times \mathcal{A})^{\infty}=\left\{\left(s_0, a_0, s_1, a_1, \ldots\right): s_h \in \mathcal{S}, a_h \in \mathcal{A}, h \in \mathbb{N}\right\}$ induced by the policy $\pi$ and the initial distribution $\nu_0$. We also use $\mathbb{E}_\pi$ to denote the expectation with respect to $\mathbb{P}_{\nu_0}^\pi$. As in Inverse Reinforcement learning (IRL), a Dynamic Discrete Choice (DDC) model makes the following assumption: 

\begin{asmp}\label{ass:optimaldecision} Agent makes decision according to the policy $\operatorname{argmax}_{\pi \in \Delta_{\mathcal{A}}^{\mathcal{S}}}$
$\mathbb{E}_\pi\left[\sum_{h=0}^{\infty} \beta^h( r\left(s_h, a_h\right)+\epsilon_{ah})\right]$.
\end{asmp}

As Assumption \ref{ass:optimaldecision} specifies the agent's policy, we omit $\pi$ in the notations from now on. Define $\boldsymbol{\epsilon}_h = [\epsilon_{1h}\ldots \epsilon_{|\mathcal{A}|h}]$, where $\epsilon_{ih}\overset{i.i.d}{\sim} G(\delta, 1)$ for $i=1\ldots |\mathcal{A}|$. We define a function $V$ as
\begin{align}
    V\left(s_{h^\prime}, \boldsymbol{\epsilon_{h^\prime}}\right) = \mathbb{E}\left[\sum_{h=h^\prime}^{\infty} \beta^h( r\left(s_h, a_h\right)+\epsilon_{ah})\mid s_{h^\prime}\right] \notag
\end{align}
and call it the value function. According to Assumption \ref{ass:optimaldecision}, the value function $V$ must satisfy the Bellman equation, i.e., 
\begin{align}
V\left(s, \boldsymbol{\epsilon}\right)=\max _{a \in \mathcal{A}}\left\{r\left(s, a\right)+\epsilon_{a}+\beta \cdot \mathbb{E}_{s^\prime \sim P(s, a), \boldsymbol{\epsilon^\prime }\sim \boldsymbol{\epsilon}}\left[V\left(s^\prime, \boldsymbol{\epsilon}^\prime\right)\mid s, a\right]\right\} \label{eq:VBellman}.
\end{align}
\;
\\
Define 
\begin{align}
    \bar{V}\left(s\right) &\triangleq E_{\boldsymbol{\epsilon}}\left[V\left(s, \boldsymbol{\epsilon}\right)\right] \notag
    \\
    Q(s, a) &\triangleq r\left(s, a\right)+\beta \cdot \mathbb{E}_{s^\prime \sim P(s, a)}\left[\bar{V}\left(s^\prime\right)\mid s, a\right]\label{eq:QandexpV}
\end{align}
We call $\bar{V}$ the expected value function, and $Q(s, a)$ as the choice-specific value function. Then the Bellman equation can be written as

\;
\begin{align}
\bar{V}\left(s\right) &=\mathbb{E}_{
\boldsymbol{\epsilon}}\left[\max _{a \in \mathcal{A}}\left\{r\left(s, a\right)+\epsilon_{a}+\beta \cdot \mathbb{E}\left[\bar{V}\left(s^\prime\right)\mid s, a\right]\right\}\right]\label{eq:DP_DDC_pre} 
\\
&=\ln \left[\sum_{a\in \mathcal{A}}\exp\left(r\left(s, a\right)+\beta \cdot \mathbb{E}\left[\bar{V}\left(s^\prime\right)\mid s, a\right]\right)\right] + \delta + \gamma \tag{$\because$ Lemma \ref{lem:GumbelMax}}\label{eq:DP_DDC}
\\
&=\ln \left[\sum_{a\in \mathcal{A}}\exp\left(Q(s,a)\right)\right]  + \delta + \gamma \label{eq:logsumQ}
\end{align}
\\
\;
\\
Furthermore, Corollary \ref{cor:GumbelOptProb} characterizes that the agent's optimal policy is characterized by 
\begin{align}
    \pi^*(a \mid s) =\frac{\exp \left({Q(s, a)}\right)}{\sum_{a^\prime\in \mathcal{A}} \exp \left({Q(s, a^\prime)}\right)} \text{ for } a\in \mathcal{A} \label{eq:DDCopt}
\end{align}
\;
\\
In addition to Bellman equation in terms of value function $V$ in equation \ref{eq:VBellman}, 
Bellman equation in terms of choice-specific value function $Q$ comes from combining equation \ref{eq:QandexpV} and equation \ref{eq:logsumQ}:
\begin{align}
    Q(s, a)=r(s, a)+\beta \mathbb{E}_{s^\prime \sim P(s, a)}\left[\ln \left(\sum_{a^{\prime} \in \mathcal{A}} \exp \left(Q\left(s^{\prime}, a^{\prime}\right)\right)\right)\mid s, a\right] + \delta + \gamma
\end{align}
\;
\\
When $\delta = -\gamma$ (i.e., the Gumbel noise is mean 0), we have 
\begin{align}
    Q(s, a)=r(s, a)+\beta \mathbb{E}_{s^\prime \sim P(s, a)}\left[\ln \left(\sum_{a^{\prime} \in \mathcal{A}} \exp \left(Q\left(s^{\prime}, a^{\prime}\right)\right)\right)\mid s, a\right] \tag{ \ref{eq:QBellmanDDC}}
\end{align}
\;
\\
This Bellman equation can be also written in another form.

\begin{align}
  Q(s, a) &\triangleq r\left(s, a\right)+\beta \cdot \mathbb{E}_{s^\prime \sim P(s, a)}\left[\bar{V}\left(s^\prime\right)\mid s, a\right]\tag{Equation \ref{eq:QandexpV}}
  \\
  &= r\left(s, a\right)+\beta \cdot \mathbb{E}_{s^\prime \sim P(s, a), \boldsymbol{\epsilon}^\prime\sim \boldsymbol{\epsilon} }\left[{V}\left(s^\prime, \boldsymbol{\epsilon}^\prime\right)\mid s, a\right]\notag
  \\
  &=r\left(s, a\right)+\beta \cdot \mathbb{E}_{s^\prime \sim P(s, a),  \boldsymbol{\epsilon}^\prime\sim \boldsymbol{\epsilon}}\left[\max_{a^\prime\in \mathcal{A}} \left(Q\left(s^\prime, a^\prime\right)+\epsilon^\prime_a\right)\mid s, a\right] \label{eq:AnotherQBellman}
  \\
  &=r\left(s, a\right)+\beta \cdot \mathbb{E}_{s^\prime \sim P(s, a), a^\prime \sim \pi^*(\cdot\mid s^\prime)}\left[Q(s^\prime, a^\prime) + \delta + \gamma - \log  \pi^*(a^\prime \mid s^\prime) \mid s, a\right] \makebox[2em][l]{\quad(Corollary \ref{cor:GumbelMaxasProb})} \notag
  \\\label{DDCBellman_new}
\end{align}
where $\pi^*(s, a) = \left(\frac{Q(s, a)}{\sum_{a^{\prime}\in \mathcal{A}}Q(s, a^{\prime})}\right)$.




\subsection{Equivalence between DDC and Entropy regularized IRL}\label{sec:DDCIRLequiv}

Equation \ref{eq:logsumexp_reg}, equation \ref{eq:IRLopt} and equation \ref{eqn:IRLQBellman_new} characterizes the choice-specific value function's Bellman equation and optimal policy in entropy regularized IRL setting when regularizing coefficient is 1: 
$$
    Q(s, a)=r(s, a)+\beta \mathbb{E}_{s^\prime \sim P(s, a)}\left[\ln \left(\sum_{a^{\prime} \in \mathcal{A}} \exp \left(Q\left(s^{\prime},a^{\prime}\right)\right)\right)\mid s, a\right]
$$

$$
\pi^*(a \mid s) =\frac{\exp \left({Q(s, a)}\right)}{\sum_{a^\prime\in \mathcal{A}} \exp \left({Q(s, a^\prime)}\right)} \text{ for } a\in \mathcal{A} 
$$

$$
Q(s,a)=r\left(s, a\right)+\beta \cdot \mathbb{E}_{s^\prime \sim P(s, a), a^\prime \sim \pi^*(\cdot \mid s^\prime)}\left[Q(s^\prime, a^\prime) - \log  \pi^*(a^\prime\mid s^\prime) \mid s, a\right]
$$
\\
Equation \ref{eq:DDCopt}, equation \ref{eq:QBellmanDDC}, and equation \ref{DDCBellman_new} (when $\delta = -\gamma$) characterizes the choice-specific value function's Bellman equation and optimal policy of Dynamic Discrete Choice setting:
$$
    Q(s, a)=r(s, a)+\beta \mathbb{E}_{s^\prime \sim P(s, a)}\left[\ln \left(\sum_{a^{\prime} \in \mathcal{A}} \exp \left(Q\left(s^{\prime}, a^{\prime}\right)\right)\right)\mid s, a\right] 
$$

$$
\pi^*(a \mid s) =\frac{\exp \left({Q(s, a)}\right)}{\sum_{a^\prime\in \mathcal{A}} \exp \left({Q(s, a^\prime)}\right)} \text{ for } a\in \mathcal{A} 
$$

$$
Q(s,a) = r\left(s, a\right)+\beta \cdot \mathbb{E}_{s^\prime \sim P(s, a), a^\prime \sim \pi^*(\cdot\mid s^\prime)}\left[Q(s^\prime, a^\prime) - \log  \pi^*(a^\prime \mid s^\prime) \mid s, a\right]
$$
\\
$Q$ that satisfies \ref{eq:DDCopt} is unique \cite{rust1994structural}, and $Q-r$ forms a one-to-one relationship. Therefore, the exact equivalence between these two setups implies that the same reward function $r$ and discount factor $\beta$ will lead to the same choice-specific value function $Q$ and the same optimal policy for the two problems.


\section{IRL with occupancy matching}\label{sec:occupancy}

\cite{ho2016generative} defines another inverse reinforcement learning problem that is based on the notion of occupancy matching. Let $\nu_0$ be the initial state distribution and $d^\pi$ be the discounted state-action occupancy of $\pi$ which is defined as $ d^\pi=(1-$ $\beta) \sum_{t=0}^{\infty} \beta^t d_t^\pi$, with $d_t^\pi(s, a)=\mathbb{P}_{\pi, \nu_0}\left[s_t=s, a_t=a\right]$. Note that $Q^\pi(s, a):=\mathbb{E}_\pi\left[\sum_{t=0}^{\infty} \beta^t r(s_t,a_t) \mid s_0=s, a_0=a\right] = \sum_{t=0}^{\infty} \beta^t \mathbb{E}_{(\tilde{s},\tilde{a})\sim d_t^\pi}[r(\tilde{s},\tilde{a})\mid s_0=a,a_0=a].$ Defining the discounted state-action occupancy of the expert policy $\pi^\ast$ as $d^\ast$, \cite{ho2016generative} defines the inverse reinforcement learning problem as the following max-min problem:
\begin{align}
    \underset{r \in \mathcal{C}}{\operatorname{max}}\underset{{\pi \in \Pi}}{\min}\left(\mathbb{E}_{d^\ast}[r(s, a)]-\mathbb{E}_{d^\pi}[r(s, a)]-\mathcal{H}(\pi)-\psi(r)\right) \label{eq:occupancyObj}
\end{align}
where $\mathcal{H}$ is the Shannon entropy we used in MaxEnt-IRL formulation and $\psi$ is the regularizer imposed on the reward model $r$.  
\;
\\
\;
\\
Would occupancy matching find $Q$ that satisfies the Bellman equation? Denote the policy as $\pi^\ast$ and its corresponding discounted state-action occupancy measure as $d^\ast=(1-$ $\beta) \sum_{t=0}^{\infty} \beta^t d_t^\ast$, with $d_t^\ast(s, a)=\mathbb{P}_{\pi^\ast, \nu_0}\left[s_t=s, a_t=a\right]$. We define the expert's action-value function as $Q^\ast(s, a):=\mathbb{E}_{\pi^\ast}\left[\sum_{t=0}^{\infty} \beta^t r(s_t,a_t) \mid s_0=s, a_0=a\right]$ and the Bellman operator of $\pi^\ast$ as $\mathcal{T}^\ast$. Then we have the following Lemma \ref{cor:occp=naiveBE} showing that occupancy matching (even without regularization) may not minimize Bellman error for every state and action.
\begin{lem}[Occupancy matching is equivalent to naive weighted Bellman error sum]\label{cor:occp=naiveBE} The perfect occupancy matching given the same $(s_0, a_0)$ satisfies
    $$\mathbb{E}_{(s, a) \sim d^\ast}[r(s, a)\mid s_0, a_0]-\mathbb{E}_{(s, a) \sim d^\pi}[r(s, a)\mid s_0, a_0] = \mathbb{E}_{(s,a)\sim d^{\ast}}[(\mathcal{T}^\ast Q^\pi-Q^\pi)(s,a)\mid s_0, a_0]$$
\end{lem}
\begin{proof} 
   Note that $\mathbb{E}_{(s, a) \sim d^{\ast}}[r(s, a)\mid s_0, a_0]=\sum_{t=0}^{\infty} \beta^t \mathbb{E}_{(s,a)\sim d_t^\ast}[r(s,a)\mid s_0,a_0]=Q^\ast(s, a)$ and $\mathbb{E}_{(s, a) \sim d^{\pi}}[r(s_, a)\mid s_0, a_0]=\sum_{t=0}^{\infty} \beta^t \mathbb{E}_{(s,a)\sim d_t^\pi}[r(s,a)\mid s_0,a_0] = Q^\pi(s, a)$. Therefore
   \begin{align}
       &\mathbb{E}_{(s, a) \sim d^\ast}[r(s, a)\mid s_0,a_0]-\mathbb{E}_{(s, a) \sim d^\pi}[r(s, a)\mid s_0,a_0] =(1-\beta)Q^\ast(s_0, a_0)-(1-\beta)Q^\pi(s_0, a_0)\notag
       \\
       &=(1-\beta)\frac{1}{1-\beta} \mathbb{E}_{(s,a)\sim d^{\ast}}[(\mathcal{T}^\ast Q^\pi-Q^\pi)(s,a)\mid s_0, a_0]\tag{Lemma \ref{lem:telescoping}}
       \\
       &= \mathbb{E}_{(s,a)\sim d^{\ast}}[(\mathcal{T}^\ast Q^\pi-Q^\pi)(s,a)\mid s_0, a_0] \notag
   \end{align}
\end{proof}
\;
\\
Lemma \ref{cor:occp=naiveBE} implies that occupancy measure matching, even without reward regularization, does not necessarily imply Bellman errors being 0 for every state and action. In fact, what they minimize is the \textit{average Bellman error} \cite{jiang2017contextual,uehara2020minimax}. This implies that $r$ cannot be inferred from $Q$ using the Bellman equation after deriving $Q$ using occupancy matching. 



\begin{lem}[Bellman Error Telescoping]\label{lem:telescoping} 
Let the Bellman operator $\mathcal{T}^\pi$ is defined to map $f\in \mathbb{R}^{S\times A}$ to $\mathcal{T}^\pi f := r(s,a) + \mathbb{E}_{s^\prime\sim P(s,a), a^\prime\sim \pi(\cdot\mid s^\prime)}[f(s^\prime,a^\prime)\mid s,a]$. 
For any $\pi$, and any $f\in \mathbb{R}^{S\times A}$,
$$
Q^\pi(s_0, a_0)-f(s_0, a_0) = \frac{1}{1-\beta} \mathbb{E}_{(s,a)\sim d^\pi}[(\mathcal{T}^\pi f-f)(s,a)\mid s_0, a_0].
$$

\end{lem}

\begin{proof}
Note that the right-hand side of the statement can be expanded as
\begin{align}
    &r(s_0, a_0)+\beta \cancel{\mathbb{E}_{s^\prime\sim P(s,a), a^\prime\sim \pi(\cdot\mid s^\prime)}[f(s^\prime,a^\prime)\mid s,a]}-f(s_0,a_0)\notag
    \\
    &+\beta\mathbb{E}_{(s,a)\sim d_1^\pi}\left[r(s, a)+\beta \cancel{\mathbb{E}_{s^\prime\sim P(s,a), a^\prime\sim \pi(\cdot\mid s^\prime)}[f(s^\prime,a^\prime)\mid s,a]}-\cancel{f(s,a)}\mid s_0, a_0\right]\notag
    \\
    &+\beta^2\mathbb{E}_{(s,a)\sim d_2^\pi}\left[r(s, a)+\beta \cancel{\mathbb{E}_{s^\prime\sim P(s,a), a^\prime\sim \pi(\cdot\mid s^\prime)}[f(s^\prime,a^\prime)\mid s,a]}-\cancel{f(s,a)}\mid s_0, a_0\right]\notag
    \\&\quad\quad\quad\quad\quad\quad\quad\quad\quad\quad\quad\quad\ldots\notag
    \\
    &=Q^\pi(s_0, a_0)-f(s_0, a_0)\notag
\end{align}
which is the left-hand side of the statement.

\end{proof}

\iffalse

\begin{lem}[Equivalence between Occupancy matching and Behavioral Cloning]
The solution set to the Occupancy matching objective (equation \ref{eq:occupancyObj}) without regularization terms is equivalent to the solution set to the behavioral cloning objective (equation \ref{eq:BC}).
\end{lem}

\begin{proof}
    In proving Lemma \ref{cor:occp=naiveBE}, we saw that $\mathbb{E}_{(s, a) \sim d^\ast}[r(s, a)\mid s_0,a_0]-\mathbb{E}_{(s, a) \sim d^\pi}[r(s, a)\mid s_0,a_0] =(1-\beta)Q^\ast(s_0, a_0)-(1-\beta)Q^\pi(s_0, a_0)$. 

    Now from Lemma \ref{lem:minMLE}, we have 
      \begin{align}
           \underset{Q\in \mathcal{Q}}{\arg\max } &\; \;\mathbb{E}_{(s, a)\sim \pi^*, \nu_0}  \left[\log\left(\hat{p}_{Q}(\;\cdot
    \mid s)\right)\right] \notag
    \\
     &=\left\{Q \in \mathcal{Q} \mid\hat{p}_{Q}(\;\cdot
    \mid s) = \pi^*(\;\cdot
    \mid s)\quad  \forall s\in\bar{\mathcal{S}}\quad\text{a.e.}\right\}\notag
    \\
     &=\left\{Q \in \mathcal{Q} \mid Q(s,a_1)-Q(s,a_2)= Q^*(s,a_1)-Q^*(s,a_2) \quad \forall a_1, a_2\in\mathcal{A}, s\in\bar{\mathcal{S}}\right\} \notag
    \end{align}
This concludes that the solution set to the behavioral cloning objective is equivalent to the occupancy matching objective without the regularization term.
\end{proof}

\fi


%\section*{Acknowledgments}


\end{document}