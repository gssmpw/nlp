%%%%%%%% ICML 2025 EXAMPLE LATEX SUBMISSION FILE %%%%%%%%%%%%%%%%%

\documentclass{article}

% Recommended, but optional, packages for figures and better typesetting:
\usepackage{microtype}
\usepackage{graphicx}
\usepackage{subfigure}
\usepackage{booktabs} % for professional tables

% hyperref makes hyperlinks in the resulting PDF.
% If your build breaks (sometimes temporarily if a hyperlink spans a page)
% please comment out the following usepackage line and replace
% \usepackage{icml2025} with \usepackage[nohyperref]{icml2025} above.
\usepackage{hyperref}


% Attempt to make hyperref and algorithmic work together better:
\newcommand{\theHalgorithm}{\arabic{algorithm}}

% Use the following line for the initial blind version submitted for review:
% \usepackage{icml2025}

% If accepted, instead use the following line for the camera-ready submission:
\usepackage[accepted]{icml2025}

% For theorems and such
\usepackage{amsmath}
\usepackage{amssymb}
\usepackage{mathtools}
\usepackage{amsthm}

% if you use cleveref..
\usepackage[capitalize,noabbrev]{cleveref}

%%%%%%%%%%%%%%%%%%%%%%%%%%%%%%%%
% THEOREMS
%%%%%%%%%%%%%%%%%%%%%%%%%%%%%%%%
\theoremstyle{plain}
\newtheorem{theorem}{Theorem}[section]
\newtheorem{proposition}[theorem]{Proposition}
\newtheorem{lemma}[theorem]{Lemma}
\newtheorem{corollary}[theorem]{Corollary}
\theoremstyle{definition}
\newtheorem{definition}[theorem]{Definition}
\newtheorem{assumption}[theorem]{Assumption}
\theoremstyle{remark}
\newtheorem{remark}[theorem]{Remark}

% Todonotes is useful during development; simply uncomment the next line
%    and comment out the line below the next line to turn off comments
%\usepackage[disable,textsize=tiny]{todonotes}
\usepackage[textsize=tiny]{todonotes}

%%%%%%%%%%%%%%%%%%%% my newcommands %%%%%%%%%%%%%%%%%%%%
\usepackage{transparent}
\usepackage{color}
\usepackage{xcolor}
\usepackage{wrapfig}
\usepackage{colortbl}
\usepackage[makeroom]{cancel}
\usepackage{soul}  % background text color
\usepackage{pifont}% http://ctan.org/pkg/pifont
\usepackage{rotating} % rotatebox
\usepackage{enumitem}  % margin for itemize

\definecolor{gray}{rgb}{0.46,0.46,0.46}
\definecolor{darkergreen}{RGB}{21, 152, 56}
\definecolor{darkerred}{RGB}{220, 35, 120}
\definecolor{darkerblue}{rgb}{0,0.08,0.45} % icml
\definecolor{royalblue}{RGB}{65,105,225}
\definecolor{lightblue}{RGB}{221,235,247}
\definecolor{gray94}{gray}{.94}
\definecolor{gray90}{gray}{.90}

% \newcommand{\ul}{\underline}
\newcommand{\red}[1]{\textcolor{red}{#1}}
\newcommand{\blue}[1]{\textcolor{royalblue}{#1}}
\newcommand{\green}[1]{\textcolor{darkergreen}{#1}}
\newcommand{\yellow}[1]{\textcolor{orange}{#1}}
\newcommand{\purple}[1]{\textcolor{purple}{#1}}
\newcommand{\gray}[1]{\textcolor{gray}{#1}}
\newcommand{\fn}[1]{\footnotesize{#1}}
\newcommand{\pl}[1]{\textcolor{darkerred}{#1}}  % rebuttal

\newcommand{\rbf}[1]{\red{\bf{#1}}}
\newcommand{\gbf}[1]{\green{\bf{#1}}}
\newcommand{\gfn}[1]{\green{\fn{#1}}}
\newcommand{\rbfup}[1]{\red{$^{\bf{{#1}}}$}}
\newcommand{\gbfup}[1]{\green{$^{\bf{{#1}}}$}}
\newcommand{\dgbfup}[1]{\textcolor{darkergreen}{$^{\bf{{#1}}}$}}
\newcolumntype{g}{>{\columncolor{gray94}}c} % column as gray
\newcolumntype{b}{>{\columncolor{lightblue}}c} % column as blue
\newcommand{\grow}[1]{\rowcolor{gray94}{#1}} % row as gray
\newcommand{\brow}[1]{\rowcolor{lightblue}{#1}} % row as blue
\newcommand{\gcell}[1]{\cellcolor{gray94}{#1}} % cell as gray
\newcommand{\bcell}[1]{\cellcolor{lightblue}{#1}} % cell as blue
% Markers
\newcommand{\cmark}{\ding{51}}%
\newcommand{\cmarkg}{\textcolor{gray}{\ding{51}}}%
\newcommand{\xmark}{\ding{55}}%
\newcommand{\xmarkg}{\textcolor{gray}{\ding{55}}}%
\renewcommand{\baselinestretch}{0.95}
%%%%%%%%%%%%%%%%%%%%%%%%%%%%%%%%%%%%%%%%%%%%%%%%%%%%%%%%%%


% The \icmltitle you define below is probably too long as a header.
% Therefore, a short form for the running title is supplied here:
% \icmltitlerunning{Submission and Formatting Instructions for ICML 2025}
\icmltitlerunning{Life-Code: Central Dogma Modeling with Multi-Omics Sequence Unification}

\begin{document}

\twocolumn[
\icmltitle{Life-Code: Central Dogma Modeling with Multi-Omics \\ Sequence Unification}

% It is OKAY to include author information, even for blind
% submissions: the style file will automatically remove it for you
% unless you've provided the [accepted] option to the icml2025
% package.

% List of affiliations: The first argument should be a (short)
% identifier you will use later to specify author affiliations
% Academic affiliations should list Department, University, City, Region, Country
% Industry affiliations should list Company, City, Region, Country

% You can specify symbols, otherwise they are numbered in order.
% Ideally, you should not use this facility. Affiliations will be numbered
% in order of appearance and this is the preferred way.
\icmlsetsymbol{equal}{*}

% \begin{icmlauthorlist}
% \icmlauthor{Firstname1 Lastname1}{equal,yyy}
% \icmlauthor{Firstname2 Lastname2}{equal,yyy,comp}
% \icmlauthor{Firstname3 Lastname3}{comp}
% \icmlauthor{Firstname4 Lastname4}{sch}
% \icmlauthor{Firstname5 Lastname5}{yyy}
% \icmlauthor{Firstname6 Lastname6}{sch,yyy,comp}
% \icmlauthor{Firstname7 Lastname7}{comp}
% %\icmlauthor{}{sch}
% \icmlauthor{Firstname8 Lastname8}{sch}
% \icmlauthor{Firstname8 Lastname8}{yyy,comp}
% %\icmlauthor{}{sch}
% %\icmlauthor{}{sch}
% \end{icmlauthorlist}

\begin{icmlauthorlist}
\icmlauthor{~Zicheng Liu}{equal,west,zju,biomap}
\icmlauthor{~Siyuan Li}{equal,west,zju,biomap}
\icmlauthor{~Zhiyuan Chen}{west,hku}
\icmlauthor{~Lei Xin}{west}
\icmlauthor{~Fang Wu}{west,stan}
\icmlauthor{~Chang Yu}{west}
\icmlauthor{~Qirong Yang}{biomap}
\icmlauthor{~Yucheng Guo}{biomap}
\icmlauthor{~Yujie Yang}{biomap}
\icmlauthor{~Stan Z. Li$^\dag$}{west}
\end{icmlauthorlist}

% \icmlaffiliation{yyy}{Department of XXX, University of YYY, Location, Country}
% \icmlaffiliation{comp}{Company Name, Location, Country}
% \icmlaffiliation{sch}{School of ZZZ, Institute of WWW, Location, Country}
\icmlaffiliation{west}{AI Lab, Research Center for Industries of the Future, Westlake University, Hangzhou, China}
\icmlaffiliation{zju}{Zhejiang University, Hangzhou, China}
\icmlaffiliation{biomap}{BioMap Research, Beijing, China}
\icmlaffiliation{hku}{University of Hong Kong, Hong Kong, China}
\icmlaffiliation{stan}{Stanford University, CA, USA}

% \icmlcorrespondingauthor{Firstname1 Lastname1}{first1.last1@xxx.edu}
% \icmlcorrespondingauthor{Firstname2 Lastname2}{first2.last2@www.uk}
\icmlcorrespondingauthor{Stan Z. Li}{stan.z.li@westlake.edu.cn}

% You may provide any keywords that you
% find helpful for describing your paper; these are used to populate
% the "keywords" metadata in the PDF but will not be shown in the document
\icmlkeywords{Machine Learning, ICML}

\vskip 0.3in
]

% this must go after the closing bracket ] following \twocolumn[ ...

% This command actually creates the footnote in the first column
% listing the affiliations and the copyright notice.
% The command takes one argument, which is text to display at the start of the footnote.
% The \icmlEqualContribution command is standard text for equal contribution.
% Remove it (just {}) if you do not need this facility.

%\printAffiliationsAndNotice{}  % leave blank if no need to mention equal contribution
\printAffiliationsAndNotice{\icmlEqualContribution} % otherwise use the standard text.

\begin{abstract}


The choice of representation for geographic location significantly impacts the accuracy of models for a broad range of geospatial tasks, including fine-grained species classification, population density estimation, and biome classification. Recent works like SatCLIP and GeoCLIP learn such representations by contrastively aligning geolocation with co-located images. While these methods work exceptionally well, in this paper, we posit that the current training strategies fail to fully capture the important visual features. We provide an information theoretic perspective on why the resulting embeddings from these methods discard crucial visual information that is important for many downstream tasks. To solve this problem, we propose a novel retrieval-augmented strategy called RANGE. We build our method on the intuition that the visual features of a location can be estimated by combining the visual features from multiple similar-looking locations. We evaluate our method across a wide variety of tasks. Our results show that RANGE outperforms the existing state-of-the-art models with significant margins in most tasks. We show gains of up to 13.1\% on classification tasks and 0.145 $R^2$ on regression tasks. All our code and models will be made available at: \href{https://github.com/mvrl/RANGE}{https://github.com/mvrl/RANGE}.

\end{abstract}


\section{Introduction}
Backdoor attacks pose a concealed yet profound security risk to machine learning (ML) models, for which the adversaries can inject a stealth backdoor into the model during training, enabling them to illicitly control the model's output upon encountering predefined inputs. These attacks can even occur without the knowledge of developers or end-users, thereby undermining the trust in ML systems. As ML becomes more deeply embedded in critical sectors like finance, healthcare, and autonomous driving \citep{he2016deep, liu2020computing, tournier2019mrtrix3, adjabi2020past}, the potential damage from backdoor attacks grows, underscoring the emergency for developing robust defense mechanisms against backdoor attacks.

To address the threat of backdoor attacks, researchers have developed a variety of strategies \cite{liu2018fine,wu2021adversarial,wang2019neural,zeng2022adversarial,zhu2023neural,Zhu_2023_ICCV, wei2024shared,wei2024d3}, aimed at purifying backdoors within victim models. These methods are designed to integrate with current deployment workflows seamlessly and have demonstrated significant success in mitigating the effects of backdoor triggers \cite{wubackdoorbench, wu2023defenses, wu2024backdoorbench,dunnett2024countering}.  However, most state-of-the-art (SOTA) backdoor purification methods operate under the assumption that a small clean dataset, often referred to as \textbf{auxiliary dataset}, is available for purification. Such an assumption poses practical challenges, especially in scenarios where data is scarce. To tackle this challenge, efforts have been made to reduce the size of the required auxiliary dataset~\cite{chai2022oneshot,li2023reconstructive, Zhu_2023_ICCV} and even explore dataset-free purification techniques~\cite{zheng2022data,hong2023revisiting,lin2024fusing}. Although these approaches offer some improvements, recent evaluations \cite{dunnett2024countering, wu2024backdoorbench} continue to highlight the importance of sufficient auxiliary data for achieving robust defenses against backdoor attacks.

While significant progress has been made in reducing the size of auxiliary datasets, an equally critical yet underexplored question remains: \emph{how does the nature of the auxiliary dataset affect purification effectiveness?} In  real-world  applications, auxiliary datasets can vary widely, encompassing in-distribution data, synthetic data, or external data from different sources. Understanding how each type of auxiliary dataset influences the purification effectiveness is vital for selecting or constructing the most suitable auxiliary dataset and the corresponding technique. For instance, when multiple datasets are available, understanding how different datasets contribute to purification can guide defenders in selecting or crafting the most appropriate dataset. Conversely, when only limited auxiliary data is accessible, knowing which purification technique works best under those constraints is critical. Therefore, there is an urgent need for a thorough investigation into the impact of auxiliary datasets on purification effectiveness to guide defenders in  enhancing the security of ML systems. 

In this paper, we systematically investigate the critical role of auxiliary datasets in backdoor purification, aiming to bridge the gap between idealized and practical purification scenarios.  Specifically, we first construct a diverse set of auxiliary datasets to emulate real-world conditions, as summarized in Table~\ref{overall}. These datasets include in-distribution data, synthetic data, and external data from other sources. Through an evaluation of SOTA backdoor purification methods across these datasets, we uncover several critical insights: \textbf{1)} In-distribution datasets, particularly those carefully filtered from the original training data of the victim model, effectively preserve the model’s utility for its intended tasks but may fall short in eliminating backdoors. \textbf{2)} Incorporating OOD datasets can help the model forget backdoors but also bring the risk of forgetting critical learned knowledge, significantly degrading its overall performance. Building on these findings, we propose Guided Input Calibration (GIC), a novel technique that enhances backdoor purification by adaptively transforming auxiliary data to better align with the victim model’s learned representations. By leveraging the victim model itself to guide this transformation, GIC optimizes the purification process, striking a balance between preserving model utility and mitigating backdoor threats. Extensive experiments demonstrate that GIC significantly improves the effectiveness of backdoor purification across diverse auxiliary datasets, providing a practical and robust defense solution.

Our main contributions are threefold:
\textbf{1) Impact analysis of auxiliary datasets:} We take the \textbf{first step}  in systematically investigating how different types of auxiliary datasets influence backdoor purification effectiveness. Our findings provide novel insights and serve as a foundation for future research on optimizing dataset selection and construction for enhanced backdoor defense.
%
\textbf{2) Compilation and evaluation of diverse auxiliary datasets:}  We have compiled and rigorously evaluated a diverse set of auxiliary datasets using SOTA purification methods, making our datasets and code publicly available to facilitate and support future research on practical backdoor defense strategies.
%
\textbf{3) Introduction of GIC:} We introduce GIC, the \textbf{first} dedicated solution designed to align auxiliary datasets with the model’s learned representations, significantly enhancing backdoor mitigation across various dataset types. Our approach sets a new benchmark for practical and effective backdoor defense.



% \section{Related Work}
\label{sec:relatedwork}

\subsection{Current AI Tools for Social Service}
\label{subsec:relatedtools}
% the title I feel is quite broad

Harnessing technology for social good has always been a grand challenge in social service \cite{berzin_practice_2015}. As early as the 90s, artificial neural networks and predictive models have been employed as tools for risk assessments, decision-making, and workload management in sectors like child protective services and mental health treatment \cite{fluke_artificial_1989, patterson_application_1999}. The recent rise of generative AI is poised to further advance social service practice, facilitating the automation of administrative tasks, streamlining of paperwork and documentation, optimisation of resource allocation, data analysis, and enhancing client support and interventions \cite{fernando_integration_2023, perron_generative_2023}.

Today, AI solutions are increasingly being deployed in both policy and practice \cite{goldkind_social_2021, hodgson_problematising_2022}. In clinical social work, AI has been used for risk assessments, crisis management, public health initiatives, and education and training for practitioners \cite{asakura_call_2020, gillingham2019can, jacobi_functions_2023, liedgren_use_2016, molala_social_2023, rice_piloting_2018, tambe_artificial_2018}. AI has also been employed for mental health support and therapeutic interventions, with conversational agents serving as on-demand virtual counsellors to provide clinical care and support \cite{lisetti_i_2013, reamer_artificial_2023}.
% commercial solutions include Woebot, which simulates therapeutic conversation, and Wysa, an “emotionally intelligent” AI coach, powered by evidenced-based clinical techniques \cite{reamer_artificial_2023}. 
% Non-clinical AI agents like Replika and companion robots can also provide social support and reduce loneliness amongst individuals \cite{ahmed_humanrobot_2024, chaturvedi_social_2023, pani_can_2024, ta_user_2020}.

Present research largely focuses on \textit{\textbf{AI-based decision support tools}} in social service \cite{james_algorithmic_2023, kawakami2022improving}, especially predictive risk models (PRMs) used to predict social service risks and outcomes \cite{gillingham2019can, van2017predicting}, like the Allegheny Family Screening Tool (AFST), which assesses child abuse risk using data from US public systems \cite{chouldechova_case_2018, vaithianathan2017developing}. Elsewhere, researchers have also piloted PRMs to predict social service needs for the homeless using Medicaid data\cite{erickson_automatic_2018, pourat_easy_2023}, and AI-powered algorithms to promote health interventions for at-risk populations, such as HIV testing among Californian homeless \cite{rice_piloting_2018, yadav_maximizing_2017}.

\subsection{Generative AI and Human-AI Collaboration}
\label{subsec:relatedworkhaicollaboration}
Beyond decision-making algorithms and PRMs, advancements in generative AI, such as large language models (LLMs), open new possibilities for human-AI (HAI) collaboration in social services. 
LLMs have been called "revolutionary" \cite{fui2023generative} and a "seismic shift" \cite{cooper2023examining}, offering "content support" \cite{memmert2023towards} by generating realistic and coherent responses to user inputs \cite{cascella2023evaluating}. Their vastly improved capabilities and ubiquity \cite{cooper2023examining} makes them poised to revolutionise work patterns \cite{fui2023generative}. Generative AI is already used in fields like design, writing, music, \cite{han2024teams, suh2021ai, verheijden2023collaborative, dhillon2024shaping, gero2023social} healthcare, and clinical settings \cite{zhang2023generative, yu2023leveraging, biswas2024intelligent}, with promising results. However, the social service sector has been slower in adopting AI \cite{diez2023artificial, kawakami2023training}.

% Yet, the social service sector is one that could perhaps stand to gain the most from AI technologies. As Goldkind \cite{goldkind_social_2021} writes, social service, as a "values-centred profession with a robust code of ethics" (p. 372), is uniquely placed to inform the development of thoughtful algorithmic policy and practice. 
Social service, however, stands to benefit immensely from generative AI. SSPs work in time-poor environments \cite{tiah_can_2024}, often overwhelmed with tedious administrative work \cite{meilvang_working_2023} and large amounts of paperwork and data processing \cite{singer_ai_2023, tiah_can_2024}. 
% As such, workers often work in time-poor environments and are burdened with information overload and administrative tasks \cite{tiah_can_2024, meilvang_working_2023}. 
Generative AI is well-placed to streamline and automate tasks like formatting case notes, formulating treatment plans and writing progress reports, which can free up valuable time for more meaningful work like client engagement and enhance service quality \cite{fernando_integration_2023, perron_generative_2023, tiah_can_2024, thesocialworkaimentor_ai_nodate}. 

Given the immense potential, there has been emerging research interest in HAI collaboration and teamwork in the Human-Computer Interaction and Computer Supported Cooperative Work space \cite{wang_human-human_2020}. HAI collaboration and interaction has been postulated by researchers to contribute to new forms of HAI symbiosis and augmented intelligence, where algorithmic and human agents work in tandem with one another to perform tasks better than they could accomplish alone by augmenting each other's strengths and capabilities  \cite{dave_augmented_2023, jarrahi_artificial_2018}.

However, compared to the focus on AI decision-making and PRM tools, there is scant research on generative AI and HAI collaboration in the social service sector \cite{wykman_artificial_2023}. This study therefore seeks to fill this critical gap by exploring how SSPs use and interact with a novel generative AI tool, helping to expand our understanding of the new opportunities that HAI collaboration can bring to the social service sector.

\subsection{Challenges in AI Use in Social Service}
\label{subsec:relatedworkaiuse}

% Despite the immense potential of AI systems to augment social work practice, there are multiple challenges with integrating such systems into real-life practice. 
Despite its evident benefits, multiple challenges plague the integration of AI and its vast potential into real-life social service practice.
% Numerous studies have investigated the use of PRMs to help practitioners decide on a course of action for their clients. 
When employing algorithmic decision-making systems, practitioners often experience tension in weighing AI suggestions against their own judgement \cite{kawakami2022improving, saxena2021framework}, being uncertain of how far they should rely on the machine. 
% Despite often being instructed to use the tool as part of evaluating a client, 
Workers are often reluctant to fully embrace AI assessments due to its inability to adequately account for the full context of a case \cite{kawakami2022improving, gambrill2001need}, and lack of clarity and transparency on AI systems and limitations \cite{kawakami2022improving}. Brown et al. \cite{brown2019toward} conducted workshops using hypothetical algorithmic tools 
% to understand service providers' comfort levels with using such tools in their work,
and found similar issues with mistrust and perceived unreliability. Furthermore, introducing AI tools can  create new problems of its own, causing confusion and distrust amongst workers \cite{kawakami2022improving}. Such factors are critical barriers to the acceptance and effective use of AI in the sector.

\citeauthor{meilvang_working_2023} (2023) cites the concept of \textit{boundary work}, which explores the delineation between "monotonous" administrative labour and "professional", "knowledge based" work drawing on core competencies of SSPs. While computers have long been used for bureaucratic tasks like client registration, the introduction of decision support systems like PRMs stirred debate over AI "threatening professional discretion and, as such, the profession itself" \cite{meilvang_working_2023}. Such latent concerns arguably drive the resistance to technology adoption described above. Generative AI is only set to further push this boundary, 
% these concerns are only set to grow in tandem with the vast capabilities of generative and other modern AI systems. Compared to the relatively primitive AI systems in past years, perceived as statistical algorithms \cite{brown2019toward} turning preset inputs like client age and behavioural symptoms \cite{vaithianathan2017developing} into simple numerical outputs indicating various risk scores, modern AI systems are vastly more capable: LLMs 
with its ability to formulate detailed reports and assessments that encroach upon the "core" work of SSPs.
% accept unrestricted and unstructured inputs and return a range of verbose and detailed evaluations according to the user's instructions. 
Introducing these systems exacerbate previously-raised issues such as understanding the limitations and possibilities of AI systems \cite{kawakami2022improving} and risk of overreliance on AI \cite{van2023chatgpt}, and requires a re-examination of where users fall on the algorithmic aversion-bias scale \cite{brown2019toward} and how they detect and react to algorithmic failings \cite{de2020case}. We address these critical issues through an empirical, on-the-ground study that to our knowledge is the first of its kind since the new wave of generative AI.

% W 

% Yet, to date, we have limited knowledge on the real-world impacts and implications of human-AI collaboration, and few studies have investigated practitioners’ experiences working with and using such AI systems in practice, especially within the social work context \cite{kawakami2022improving}. A small number of studies have explored practitioner perspectives on the use of AI in social work, including Kawakami et al. \cite{kawakami2022improving}, who interviewed social workers on their experiences using the AFST; Stapleton et al. \cite{stapleton_imagining_2022}, who conducted design workshops with caseworkers on the use of PRMs in child welfare; and Wassal et al. \cite{wassal_reimagining_2024}, who interviewed UK social work professionals on the use of AI. A common thread from all these studies was a general disregard for the context and users, with many practitioners criticising the failure of past AI tools arising from the lack of participation and involvement of social workers and actual users of such systems in the design and development of algorithmic systems \cite{wassal_reimagining_2024}. Similarly, in a scoping review done on decision-support algorithms in social work, Jacobi \& Christensen \cite{jacobi_functions_2023} reported that the majority of studies reveal limited bottom-up involvement and interaction between social workers, researchers and developers, and that algorithms were rarely developed with consideration of the perspective of social workers.
% so the \cite{yang_unremarkable_2019} and \cite{holten_moller_shifting_2020} are not real-world impacts? real-world means to hear practitioner's voice? I feel this is quite important but i didnt get this point in intro!

% why mentioning 'which have largely focused on existing ADS tools (e.g., AFST)'? i can see our strength is more localized, but without basic knowledge of social work i didnt get what's the 'departure' here orz
% the paragraph is great! do we need to also add one in line 20 21?

\subsection{Designing AI for Social Service through Participatory Design}
\label{subsec:relatedworkpd}
% i think it's important! but maybe not a whole subsection? but i feel the strong connection with practitioners is indeed one of our novelties and need to highlight it, also in intro maybe
% Participatory design (PD) has long been used extensively in HCI \cite{muller1993participatory}, to both design effective solutions for a specific community and gain a deep understanding of that community. Of particular interest here is the rich body of literature on PD in the field of healthcare \cite{donetto2015experience}, which in this regard shares many similarities and concerns with social work. PD has created effective health improvement apps \cite{ryu2017impact}, 

% PD offers researchers the chance to gather detailed user requirements \cite{ryu2017impact}...

Participatory design (PD) is a staple of HCI research \cite{muller1993participatory}, facilitating the design of effective solutions for a specific community while gaining a deep understanding of its stakeholders. The focus in PD of valuing the opinions and perspectives of users as experts \cite{schuler_participatory_1993} 
% In recent years, the tech and social work sectors have awakened to the importance of involving real users in designing and implementing digital technologies, developing human-centred design processes to iteratively design products or technologies through user feedback 
has gained importance in recent years \cite{storer2023reimagining}. Responding to criticisms and failures of past AI tools that have been implemented without adequate involvement and input from actual users, HCI scholars have adopted PD approaches to design predictive tools to better support human decision-making \cite{lehtiniemi_contextual_2023}.
% ; accordingly, in social service, a line of research has begun studying and designing for human-AI collaboration with real-world users (e.g. \cite{holten_moller_shifting_2020, kawakami2022improving, yang_unremarkable_2019}).
Section \ref{subsec:relatedworkaiuse} shows a clear need to better understand SSP perspectives when designing and implementing AI tools in the social sector. 
Yet, PD research in this area has been limited. \citeauthor{yang2019unremarkable} (2019), through field evaluation with clinicians, investigated reasons behind the failure of previous AI-powered decision support tools, allowing them to design a new-and-improved AI decision-support tool that was better aligned with healthcare workers’ workflows. Similarly, \citeauthor{holten_moller_shifting_2020} (2020) ran PD workshops with caseworkers, data scientists and developers in public service systems to identify the expectations and needs that different stakeholders had in using ADS tools.

% Indeed, it is as Wise \cite{wise_intelligent_1998} noted so many years ago on the rise of intelligent agents: “it is perhaps when technologies are new, when their (and our) movements, habits and attitudes seem most awkward and therefore still at the forefront of our thoughts that they are easiest to analyse” (p. 411). 
Building upon this existing body of work, we thus conduct a study to co-design an AI tool \textit{for} and \textit{with} SSPs through participatory workshops and focus group discussions. In the process, we revisit many of the issues mentioned in Section \ref{subsec:relatedworkaiuse}, but in the context of novel generative AI systems, which are fundamentally different from most historical examples of automation technologies \cite{noy2023experimental}. This valuable empirical inquiry occurs at an opportune time when varied expectations about this nascent technology abound \cite{lehtiniemi_contextual_2023}, allowing us to understand how SSPs incorporate AI into their practice, and what AI can (or cannot) do for them. In doing so, we aim to uncover new theoretical and practical insights on what AI can bring to the social service sector, and formulate design implications for developing AI technologies that SSPs find truly meaningful and useful.
% , and drive future technological innovations to transform the social service sector not just within [our country], but also on a global scale.

 % with an on-the-ground study using a real prototype system that reflects the state of AI in current society. With the presumption that AI will continue to be used in social work given the great benefits it brings, we address the pressing need to investigate these issues to ensure that any potential AI systems are designed and implemented in a responsible and effective manner.

% Building upon these works, this study therefore seeks to adopt a participatory design methodology to investigate social workers’ perspectives and attitudes on AI and human-AI collaboration in their social work practice, thus contributing to the nascent body of practitioner-centred HCI research on the use of AI in social work. Yet, in a departure from prior work, which have largely focused on existing ADS tools (e.g., AFST) and were situated in a Western context, our paper also aims to expand the scope by piloting a novel generative AI tool that was designed and developed by the researchers in partnership with a social service agency based in Singapore, with aims of generating more insights on wider use cases of AI beyond what has been previously studied.

% i may think 'While the current lacunae of research on applications of AI in social work may appear to be a limitation, it simultaneously presents an exciting opportunity for further research and exploration \cite{dey_unleashing_2023},' this point is already convincing enough, not sure if we need to quote here
% I like this end! it's a good transition to our study design, do we need to mention the localization in intro as well? like we target at singapore

% Given the increasing prominence and acceptance of AI in modern society, 

% These increased capabilities vastly exacerbate the issues already present with a simpler tool like the AFST: the boundaries and limitations of an LLM system are significantly more difficult to understand and its possible use cases are exponentially greater in scope. 

% Put this in discussion section instead?
% Kawakami et al's work "highlights the importance of studying how collaborative decision-making... impacts how people rely upon and make sense of AI models," They conclude by recommending designing tools that "support workers in understanding the boundaries of [an AI system's] capabilities", and implementing design procedures that "support open cultures for critical discussion around AI decision making". The authors outline critical challenges of implementing AI systems, elucidating factors that may hinder their effectiveness and even negatively affect operations within the organisation.


% Is this needed?:
% talk about the strengths of PD in eliciting user viewpoints and knowledge, in particular when it is a field that is novel or where a certain system has not been used or developed or tested before
\section{Study Design}
% robot: aliengo 
% We used the Unitree AlienGo quadruped robot. 
% See Appendix 1 in AlienGo Software Guide PDF
% Weight = 25kg, size (L,W,H) = (0.55, 0.35, 06) m when standing, (0.55, 0.35, 0.31) m when walking
% Handle is 0.4 m or 0.5 m. I'll need to check it to see which type it is.
We gathered input from primary stakeholders of the robot dog guide, divided into three subgroups: BVI individuals who have owned a dog guide, BVI individuals who were not dog guide owners, and sighted individuals with generally low degrees of familiarity with dog guides. While the main focus of this study was on the BVI participants, we elected to include survey responses from sighted participants given the importance of social acceptance of the robot by the general public, which could reflect upon the BVI users themselves and affect their interactions with the general population \cite{kayukawa2022perceive}. 

The need-finding processes consisted of two stages. During Stage 1, we conducted in-depth interviews with BVI participants, querying their experiences in using conventional assistive technologies and dog guides. During Stage 2, a large-scale survey was distributed to both BVI and sighted participants. 

This study was approved by the University’s Institutional Review Board (IRB), and all processes were conducted after obtaining the participants' consent.

\subsection{Stage 1: Interviews}
We recruited nine BVI participants (\textbf{Table}~\ref{tab:bvi-info}) for in-depth interviews, which lasted 45-90 minutes for current or former dog guide owners (DO) and 30-60 minutes for participants without dog guides (NDO). Group DO consisted of five participants, while Group NDO consisted of four participants.
% The interview participants were divided into two groups. Group DO (Dog guide Owner) consisted of five participants who were current or former dog guide owners and Group NDO (Non Dog guide Owner) consisted of three participants who were not dog guide owners. 
All participants were familiar with using white canes as a mobility aid. 

We recruited participants in both groups, DO and NDO, to gather data from those with substantial experience with dog guides, offering potentially more practical insights, and from those without prior experience, providing a perspective that may be less constrained and more open to novel approaches. 

We asked about the participants' overall impressions of a robot dog guide, expectations regarding its potential benefits and challenges compared to a conventional dog guide, their desired methods of giving commands and communicating with the robot dog guide, essential functionalities that the robot dog guide should offer, and their preferences for various aspects of the robot dog guide's form factors. 
For Group DO, we also included questions that asked about the participants' experiences with conventional dog guides. 

% We obtained permission to record the conversations for our records while simultaneously taking notes during the interviews. The interviews lasted 30-60 minutes for NDO participants and 45-90 minutes for DO participants. 

\subsection{Stage 2: Large-Scale Surveys} 
After gathering sufficient initial results from the interviews, we created an online survey for distributing to a larger pool of participants. The survey platform used was Qualtrics. 

\subsubsection{Survey Participants}
The survey had 100 participants divided into two primary groups. Group BVI consisted of 42 blind or visually impaired participants, and Group ST consisted of 58 sighted participants. \textbf{Table}~\ref{tab:survey-demographics} shows the demographic information of the survey participants. 

\subsubsection{Question Differentiation} 
Based on their responses to initial qualifying questions, survey participants were sorted into three subgroups: DO, NDO, and ST. Each participant was assigned one of three different versions of the survey. The surveys for BVI participants mirrored the interview categories (overall impressions, communication methods, functionalities, and form factors), but with a more quantitative approach rather than the open-ended questions used in interviews. The DO version included additional questions pertaining to their prior experience with dog guides. The ST version revolved around the participants' prior interactions with and feelings toward dog guides and dogs in general, their thoughts on a robot dog guide, and broad opinions on the aesthetic component of the robot's design. 


\section{Dataset}
\label{sec:dataset}

\subsection{Data Collection}

To analyze political discussions on Discord, we followed the methodology in \cite{singh2024Cross-Platform}, collecting messages from politically-oriented public servers in compliance with Discord's platform policies.

Using Discord's Discovery feature, we employed a web scraper to extract server invitation links, names, and descriptions, focusing on public servers accessible without participation. Invitation links were used to access data via the Discord API. To ensure relevance, we filtered servers using keywords related to the 2024 U.S. elections (e.g., Trump, Kamala, MAGA), as outlined in \cite{balasubramanian2024publicdatasettrackingsocial}. This resulted in 302 server links, further narrowed to 81 English-speaking, politics-focused servers based on their names and descriptions.

Public messages were retrieved from these servers using the Discord API, collecting metadata such as \textit{content}, \textit{user ID}, \textit{username}, \textit{timestamp}, \textit{bot flag}, \textit{mentions}, and \textit{interactions}. Through this process, we gathered \textbf{33,373,229 messages} from \textbf{82,109 users} across \textbf{81 servers}, including \textbf{1,912,750 messages} from \textbf{633 bots}. Data collection occurred between November 13th and 15th, covering messages sent from January 1st to November 12th, just after the 2024 U.S. election.

\subsection{Characterizing the Political Spectrum}
\label{sec:timeline}

A key aspect of our research is distinguishing between Republican- and Democratic-aligned Discord servers. To categorize their political alignment, we relied on server names and self-descriptions, which often include rules, community guidelines, and references to key ideologies or figures. Each server's name and description were manually reviewed based on predefined, objective criteria, focusing on explicit political themes or mentions of prominent figures. This process allowed us to classify servers into three categories, ensuring a systematic and unbiased alignment determination.

\begin{itemize}
    \item \textbf{Republican-aligned}: Servers referencing Republican and right-wing and ideologies, movements, or figures (e.g., MAGA, Conservative, Traditional, Trump).  
    \item \textbf{Democratic-aligned}: Servers mentioning Democratic and left-wing ideologies, movements, or figures (e.g., Progressive, Liberal, Socialist, Biden, Kamala).  
    \item \textbf{Unaligned}: Servers with no defined spectrum and ideologies or opened to general political debate from all orientations.
\end{itemize}

To ensure the reliability and consistency of our classification, three independent reviewers assessed the classification following the specified set of criteria. The inter-rater agreement of their classifications was evaluated using Fleiss' Kappa \cite{fleiss1971measuring}, with a resulting Kappa value of \( 0.8191 \), indicating an almost perfect agreement among the reviewers. Disagreements were resolved by adopting the majority classification, as there were no instances where a server received different classifications from all three reviewers. This process guaranteed the consistency and accuracy of the final categorization.

Through this process, we identified \textbf{7 Republican-aligned servers}, \textbf{9 Democratic-aligned servers}, and \textbf{65 unaligned servers}.

Table \ref{tab:statistics} shows the statistics of the collected data. Notably, while Democratic- and Republican-aligned servers had a comparable number of user messages, users in the latter servers were significantly more active, posting more than double the number of messages per user compared to their Democratic counterparts. 
This suggests that, in our sample, Democratic-aligned servers attract more users, but these users were less engaged in text-based discussions. Additionally, around 10\% of the messages across all server categories were posted by bots. 

\subsection{Temporal Data} 

Throughout this paper, we refer to the election candidates using the names adopted by their respective campaigns: \textit{Kamala}, \textit{Biden}, and \textit{Trump}. To examine how the content of text messages evolves based on the political alignment of servers, we divided the 2024 election year into three periods: \textbf{Biden vs Trump} (January 1 to July 21), \textbf{Kamala vs Trump} (July 21 to September 20), and the \textbf{Voting Period} (after September 20). These periods reflect key phases of the election: the early campaign dominated by Biden and Trump, the shift in dynamics with Kamala Harris replacing Joe Biden as the Democratic candidate, and the final voting stage focused on electoral outcomes and their implications. This segmentation enables an analysis of how discourse responds to pivotal electoral moments.

Figure \ref{fig:line-plot} illustrates the distribution of messages over time, highlighting trends in total messages volume and mentions of each candidate. Prior to Biden's withdrawal on July 21, mentions of Biden and Trump were relatively balanced. However, following Kamala's entry into the race, mentions of Trump surged significantly, a trend further amplified by an assassination attempt on him, solidifying his dominance in the discourse. The only instance where Trump’s mentions were exceeded occurred during the first debate, as concerns about Biden’s age and cognitive abilities temporarily shifted the focus. In the final stages of the election, mentions of all three candidates rose, with Trump’s mentions peaking as he emerged as the victor.
\section{Experimental Methodology}\label{sec:exp}
In this section, we introduce the datasets, evaluation metrics, baselines, and implementation details used in our experiments. More experimental details are shown in Appendix~\ref{app:experiment_detail}.

\textbf{Dataset.}
We utilize various datasets for training and evaluation. Data statistics are shown in Table~\ref{tab:dataset}.

\textit{Training.}
We use the publicly available E5 dataset~\cite{wang2024improving,springer2024repetition} to train both the LLM-QE and dense retrievers. We concentrate on English-based question answering tasks and collect a total of 808,740 queries. From this set, we randomly sample 100,000 queries to construct the DPO training data, while the remaining queries are used for contrastive training. During the DPO preference pair construction, we first prompt LLMs to generate expansion documents, filtering out queries where the expanded documents share low similarity with the query. This results in a final set of 30,000 queries.

\textit{Evaluation.}
We evaluate retrieval effectiveness using two retrieval benchmarks: MS MARCO \cite{bajaj2016ms} and BEIR \cite{thakur2021beir}, in both unsupervised and supervised settings.

\textbf{Evaluation Metrics.}
We use nDCG@10 as the evaluation metric. Statistical significance is tested using a permutation test with $p<0.05$.

\textbf{Baselines.} We compare our LLM-QE model with three unsupervised retrieval models and five query expansion baseline models.
% —

Three unsupervised retrieval models—BM25~\cite{robertson2009probabilistic}, CoCondenser~\cite{gao2022unsupervised}, and Contriever~\cite{izacard2021unsupervised}—are evaluated in the experiments. Among these, Contriever serves as our primary baseline retrieval model, as it is used as the backbone model to assess the query expansion performance of LLM-QE. Additionally, we compare LLM-QE with Contriever in a supervised setting using the same training dataset.

For query expansion, we benchmark against five methods: Pseudo-Relevance Feedback (PRF), Q2Q, Q2E, Q2C, and Q2D. PRF is specifically implemented following the approach in~\citet{yu2021improving}, which enhances query understanding by extracting keywords from query-related documents. The Q2Q, Q2E, Q2C, and Q2D methods~\cite{jagerman2023query,li2024can} expand the original query by prompting LLMs to generate query-related queries, keywords, chains-of-thought~\cite{wei2022chain}, and documents.


\textbf{Implementation Details.} 
For our query expansion model, we deploy the Meta-LLaMA-3-8B-Instruct~\cite{llama3modelcard} as the backbone for the query expansion generator. The batch size is set to 16, and the learning rate is set to $2e-5$. Optimization is performed using the AdamW optimizer. We employ LoRA~\cite{hu2022lora} to efficiently fine-tune the model for 2 epochs. The temperature for the construction of the DPO data varies across $\tau \in \{0.8, 0.9, 1.0, 1.1\}$, with each setting sampled eight times. For the dense retriever, we utilize Contriever~\cite{izacard2021unsupervised} as the backbone. During training, we set the batch size to 1,024 and the learning rate to $3e-5$, with the model trained for 3 epochs.

\section{Related Work}
\label{sec:relatedwork}

\subsection{Current AI Tools for Social Service}
\label{subsec:relatedtools}
% the title I feel is quite broad

Harnessing technology for social good has always been a grand challenge in social service \cite{berzin_practice_2015}. As early as the 90s, artificial neural networks and predictive models have been employed as tools for risk assessments, decision-making, and workload management in sectors like child protective services and mental health treatment \cite{fluke_artificial_1989, patterson_application_1999}. The recent rise of generative AI is poised to further advance social service practice, facilitating the automation of administrative tasks, streamlining of paperwork and documentation, optimisation of resource allocation, data analysis, and enhancing client support and interventions \cite{fernando_integration_2023, perron_generative_2023}.

Today, AI solutions are increasingly being deployed in both policy and practice \cite{goldkind_social_2021, hodgson_problematising_2022}. In clinical social work, AI has been used for risk assessments, crisis management, public health initiatives, and education and training for practitioners \cite{asakura_call_2020, gillingham2019can, jacobi_functions_2023, liedgren_use_2016, molala_social_2023, rice_piloting_2018, tambe_artificial_2018}. AI has also been employed for mental health support and therapeutic interventions, with conversational agents serving as on-demand virtual counsellors to provide clinical care and support \cite{lisetti_i_2013, reamer_artificial_2023}.
% commercial solutions include Woebot, which simulates therapeutic conversation, and Wysa, an “emotionally intelligent” AI coach, powered by evidenced-based clinical techniques \cite{reamer_artificial_2023}. 
% Non-clinical AI agents like Replika and companion robots can also provide social support and reduce loneliness amongst individuals \cite{ahmed_humanrobot_2024, chaturvedi_social_2023, pani_can_2024, ta_user_2020}.

Present research largely focuses on \textit{\textbf{AI-based decision support tools}} in social service \cite{james_algorithmic_2023, kawakami2022improving}, especially predictive risk models (PRMs) used to predict social service risks and outcomes \cite{gillingham2019can, van2017predicting}, like the Allegheny Family Screening Tool (AFST), which assesses child abuse risk using data from US public systems \cite{chouldechova_case_2018, vaithianathan2017developing}. Elsewhere, researchers have also piloted PRMs to predict social service needs for the homeless using Medicaid data\cite{erickson_automatic_2018, pourat_easy_2023}, and AI-powered algorithms to promote health interventions for at-risk populations, such as HIV testing among Californian homeless \cite{rice_piloting_2018, yadav_maximizing_2017}.

\subsection{Generative AI and Human-AI Collaboration}
\label{subsec:relatedworkhaicollaboration}
Beyond decision-making algorithms and PRMs, advancements in generative AI, such as large language models (LLMs), open new possibilities for human-AI (HAI) collaboration in social services. 
LLMs have been called "revolutionary" \cite{fui2023generative} and a "seismic shift" \cite{cooper2023examining}, offering "content support" \cite{memmert2023towards} by generating realistic and coherent responses to user inputs \cite{cascella2023evaluating}. Their vastly improved capabilities and ubiquity \cite{cooper2023examining} makes them poised to revolutionise work patterns \cite{fui2023generative}. Generative AI is already used in fields like design, writing, music, \cite{han2024teams, suh2021ai, verheijden2023collaborative, dhillon2024shaping, gero2023social} healthcare, and clinical settings \cite{zhang2023generative, yu2023leveraging, biswas2024intelligent}, with promising results. However, the social service sector has been slower in adopting AI \cite{diez2023artificial, kawakami2023training}.

% Yet, the social service sector is one that could perhaps stand to gain the most from AI technologies. As Goldkind \cite{goldkind_social_2021} writes, social service, as a "values-centred profession with a robust code of ethics" (p. 372), is uniquely placed to inform the development of thoughtful algorithmic policy and practice. 
Social service, however, stands to benefit immensely from generative AI. SSPs work in time-poor environments \cite{tiah_can_2024}, often overwhelmed with tedious administrative work \cite{meilvang_working_2023} and large amounts of paperwork and data processing \cite{singer_ai_2023, tiah_can_2024}. 
% As such, workers often work in time-poor environments and are burdened with information overload and administrative tasks \cite{tiah_can_2024, meilvang_working_2023}. 
Generative AI is well-placed to streamline and automate tasks like formatting case notes, formulating treatment plans and writing progress reports, which can free up valuable time for more meaningful work like client engagement and enhance service quality \cite{fernando_integration_2023, perron_generative_2023, tiah_can_2024, thesocialworkaimentor_ai_nodate}. 

Given the immense potential, there has been emerging research interest in HAI collaboration and teamwork in the Human-Computer Interaction and Computer Supported Cooperative Work space \cite{wang_human-human_2020}. HAI collaboration and interaction has been postulated by researchers to contribute to new forms of HAI symbiosis and augmented intelligence, where algorithmic and human agents work in tandem with one another to perform tasks better than they could accomplish alone by augmenting each other's strengths and capabilities  \cite{dave_augmented_2023, jarrahi_artificial_2018}.

However, compared to the focus on AI decision-making and PRM tools, there is scant research on generative AI and HAI collaboration in the social service sector \cite{wykman_artificial_2023}. This study therefore seeks to fill this critical gap by exploring how SSPs use and interact with a novel generative AI tool, helping to expand our understanding of the new opportunities that HAI collaboration can bring to the social service sector.

\subsection{Challenges in AI Use in Social Service}
\label{subsec:relatedworkaiuse}

% Despite the immense potential of AI systems to augment social work practice, there are multiple challenges with integrating such systems into real-life practice. 
Despite its evident benefits, multiple challenges plague the integration of AI and its vast potential into real-life social service practice.
% Numerous studies have investigated the use of PRMs to help practitioners decide on a course of action for their clients. 
When employing algorithmic decision-making systems, practitioners often experience tension in weighing AI suggestions against their own judgement \cite{kawakami2022improving, saxena2021framework}, being uncertain of how far they should rely on the machine. 
% Despite often being instructed to use the tool as part of evaluating a client, 
Workers are often reluctant to fully embrace AI assessments due to its inability to adequately account for the full context of a case \cite{kawakami2022improving, gambrill2001need}, and lack of clarity and transparency on AI systems and limitations \cite{kawakami2022improving}. Brown et al. \cite{brown2019toward} conducted workshops using hypothetical algorithmic tools 
% to understand service providers' comfort levels with using such tools in their work,
and found similar issues with mistrust and perceived unreliability. Furthermore, introducing AI tools can  create new problems of its own, causing confusion and distrust amongst workers \cite{kawakami2022improving}. Such factors are critical barriers to the acceptance and effective use of AI in the sector.

\citeauthor{meilvang_working_2023} (2023) cites the concept of \textit{boundary work}, which explores the delineation between "monotonous" administrative labour and "professional", "knowledge based" work drawing on core competencies of SSPs. While computers have long been used for bureaucratic tasks like client registration, the introduction of decision support systems like PRMs stirred debate over AI "threatening professional discretion and, as such, the profession itself" \cite{meilvang_working_2023}. Such latent concerns arguably drive the resistance to technology adoption described above. Generative AI is only set to further push this boundary, 
% these concerns are only set to grow in tandem with the vast capabilities of generative and other modern AI systems. Compared to the relatively primitive AI systems in past years, perceived as statistical algorithms \cite{brown2019toward} turning preset inputs like client age and behavioural symptoms \cite{vaithianathan2017developing} into simple numerical outputs indicating various risk scores, modern AI systems are vastly more capable: LLMs 
with its ability to formulate detailed reports and assessments that encroach upon the "core" work of SSPs.
% accept unrestricted and unstructured inputs and return a range of verbose and detailed evaluations according to the user's instructions. 
Introducing these systems exacerbate previously-raised issues such as understanding the limitations and possibilities of AI systems \cite{kawakami2022improving} and risk of overreliance on AI \cite{van2023chatgpt}, and requires a re-examination of where users fall on the algorithmic aversion-bias scale \cite{brown2019toward} and how they detect and react to algorithmic failings \cite{de2020case}. We address these critical issues through an empirical, on-the-ground study that to our knowledge is the first of its kind since the new wave of generative AI.

% W 

% Yet, to date, we have limited knowledge on the real-world impacts and implications of human-AI collaboration, and few studies have investigated practitioners’ experiences working with and using such AI systems in practice, especially within the social work context \cite{kawakami2022improving}. A small number of studies have explored practitioner perspectives on the use of AI in social work, including Kawakami et al. \cite{kawakami2022improving}, who interviewed social workers on their experiences using the AFST; Stapleton et al. \cite{stapleton_imagining_2022}, who conducted design workshops with caseworkers on the use of PRMs in child welfare; and Wassal et al. \cite{wassal_reimagining_2024}, who interviewed UK social work professionals on the use of AI. A common thread from all these studies was a general disregard for the context and users, with many practitioners criticising the failure of past AI tools arising from the lack of participation and involvement of social workers and actual users of such systems in the design and development of algorithmic systems \cite{wassal_reimagining_2024}. Similarly, in a scoping review done on decision-support algorithms in social work, Jacobi \& Christensen \cite{jacobi_functions_2023} reported that the majority of studies reveal limited bottom-up involvement and interaction between social workers, researchers and developers, and that algorithms were rarely developed with consideration of the perspective of social workers.
% so the \cite{yang_unremarkable_2019} and \cite{holten_moller_shifting_2020} are not real-world impacts? real-world means to hear practitioner's voice? I feel this is quite important but i didnt get this point in intro!

% why mentioning 'which have largely focused on existing ADS tools (e.g., AFST)'? i can see our strength is more localized, but without basic knowledge of social work i didnt get what's the 'departure' here orz
% the paragraph is great! do we need to also add one in line 20 21?

\subsection{Designing AI for Social Service through Participatory Design}
\label{subsec:relatedworkpd}
% i think it's important! but maybe not a whole subsection? but i feel the strong connection with practitioners is indeed one of our novelties and need to highlight it, also in intro maybe
% Participatory design (PD) has long been used extensively in HCI \cite{muller1993participatory}, to both design effective solutions for a specific community and gain a deep understanding of that community. Of particular interest here is the rich body of literature on PD in the field of healthcare \cite{donetto2015experience}, which in this regard shares many similarities and concerns with social work. PD has created effective health improvement apps \cite{ryu2017impact}, 

% PD offers researchers the chance to gather detailed user requirements \cite{ryu2017impact}...

Participatory design (PD) is a staple of HCI research \cite{muller1993participatory}, facilitating the design of effective solutions for a specific community while gaining a deep understanding of its stakeholders. The focus in PD of valuing the opinions and perspectives of users as experts \cite{schuler_participatory_1993} 
% In recent years, the tech and social work sectors have awakened to the importance of involving real users in designing and implementing digital technologies, developing human-centred design processes to iteratively design products or technologies through user feedback 
has gained importance in recent years \cite{storer2023reimagining}. Responding to criticisms and failures of past AI tools that have been implemented without adequate involvement and input from actual users, HCI scholars have adopted PD approaches to design predictive tools to better support human decision-making \cite{lehtiniemi_contextual_2023}.
% ; accordingly, in social service, a line of research has begun studying and designing for human-AI collaboration with real-world users (e.g. \cite{holten_moller_shifting_2020, kawakami2022improving, yang_unremarkable_2019}).
Section \ref{subsec:relatedworkaiuse} shows a clear need to better understand SSP perspectives when designing and implementing AI tools in the social sector. 
Yet, PD research in this area has been limited. \citeauthor{yang2019unremarkable} (2019), through field evaluation with clinicians, investigated reasons behind the failure of previous AI-powered decision support tools, allowing them to design a new-and-improved AI decision-support tool that was better aligned with healthcare workers’ workflows. Similarly, \citeauthor{holten_moller_shifting_2020} (2020) ran PD workshops with caseworkers, data scientists and developers in public service systems to identify the expectations and needs that different stakeholders had in using ADS tools.

% Indeed, it is as Wise \cite{wise_intelligent_1998} noted so many years ago on the rise of intelligent agents: “it is perhaps when technologies are new, when their (and our) movements, habits and attitudes seem most awkward and therefore still at the forefront of our thoughts that they are easiest to analyse” (p. 411). 
Building upon this existing body of work, we thus conduct a study to co-design an AI tool \textit{for} and \textit{with} SSPs through participatory workshops and focus group discussions. In the process, we revisit many of the issues mentioned in Section \ref{subsec:relatedworkaiuse}, but in the context of novel generative AI systems, which are fundamentally different from most historical examples of automation technologies \cite{noy2023experimental}. This valuable empirical inquiry occurs at an opportune time when varied expectations about this nascent technology abound \cite{lehtiniemi_contextual_2023}, allowing us to understand how SSPs incorporate AI into their practice, and what AI can (or cannot) do for them. In doing so, we aim to uncover new theoretical and practical insights on what AI can bring to the social service sector, and formulate design implications for developing AI technologies that SSPs find truly meaningful and useful.
% , and drive future technological innovations to transform the social service sector not just within [our country], but also on a global scale.

 % with an on-the-ground study using a real prototype system that reflects the state of AI in current society. With the presumption that AI will continue to be used in social work given the great benefits it brings, we address the pressing need to investigate these issues to ensure that any potential AI systems are designed and implemented in a responsible and effective manner.

% Building upon these works, this study therefore seeks to adopt a participatory design methodology to investigate social workers’ perspectives and attitudes on AI and human-AI collaboration in their social work practice, thus contributing to the nascent body of practitioner-centred HCI research on the use of AI in social work. Yet, in a departure from prior work, which have largely focused on existing ADS tools (e.g., AFST) and were situated in a Western context, our paper also aims to expand the scope by piloting a novel generative AI tool that was designed and developed by the researchers in partnership with a social service agency based in Singapore, with aims of generating more insights on wider use cases of AI beyond what has been previously studied.

% i may think 'While the current lacunae of research on applications of AI in social work may appear to be a limitation, it simultaneously presents an exciting opportunity for further research and exploration \cite{dey_unleashing_2023},' this point is already convincing enough, not sure if we need to quote here
% I like this end! it's a good transition to our study design, do we need to mention the localization in intro as well? like we target at singapore

% Given the increasing prominence and acceptance of AI in modern society, 

% These increased capabilities vastly exacerbate the issues already present with a simpler tool like the AFST: the boundaries and limitations of an LLM system are significantly more difficult to understand and its possible use cases are exponentially greater in scope. 

% Put this in discussion section instead?
% Kawakami et al's work "highlights the importance of studying how collaborative decision-making... impacts how people rely upon and make sense of AI models," They conclude by recommending designing tools that "support workers in understanding the boundaries of [an AI system's] capabilities", and implementing design procedures that "support open cultures for critical discussion around AI decision making". The authors outline critical challenges of implementing AI systems, elucidating factors that may hinder their effectiveness and even negatively affect operations within the organisation.


% Is this needed?:
% talk about the strengths of PD in eliciting user viewpoints and knowledge, in particular when it is a field that is novel or where a certain system has not been used or developed or tested before
\section{Conclusion}
We introduce a novel approach, \algo, to reduce human feedback requirements in preference-based reinforcement learning by leveraging vision-language models. While VLMs encode rich world knowledge, their direct application as reward models is hindered by alignment issues and noisy predictions. To address this, we develop a synergistic framework where limited human feedback is used to adapt VLMs, improving their reliability in preference labeling. Further, we incorporate a selective sampling strategy to mitigate noise and prioritize informative human annotations.

Our experiments demonstrate that this method significantly improves feedback efficiency, achieving comparable or superior task performance with up to 50\% fewer human annotations. Moreover, we show that an adapted VLM can generalize across similar tasks, further reducing the need for new human feedback by 75\%. These results highlight the potential of integrating VLMs into preference-based RL, offering a scalable solution to reducing human supervision while maintaining high task success rates. 

\section*{Impact Statement}
This work advances embodied AI by significantly reducing the human feedback required for training agents. This reduction is particularly valuable in robotic applications where obtaining human demonstrations and feedback is challenging or impractical, such as assistive robotic arms for individuals with mobility impairments. By minimizing the feedback requirements, our approach enables users to more efficiently customize and teach new skills to robotic agents based on their specific needs and preferences. The broader impact of this work extends to healthcare, assistive technology, and human-robot interaction. One possible risk is that the bias from human feedback can propagate to the VLM and subsequently to the policy. This can be mitigated by personalization of agents in case of household application or standardization of feedback for industrial applications. 

\bibliography{reference}
\bibliographystyle{icml2025}


%%%%%%%%%%%%%%%%%%%%%%%%%%%%%%%%%%%%%%%%%%%%%%%%%%%%%%%%%%%%%%%%%%%%%%%%%%%%%%%
%%%%%%%%%%%%%%%%%%%%%%%%%%%%%%%%%%%%%%%%%%%%%%%%%%%%%%%%%%%%%%%%%%%%%%%%%%%%%%%
% APPENDIX
%%%%%%%%%%%%%%%%%%%%%%%%%%%%%%%%%%%%%%%%%%%%%%%%%%%%%%%%%%%%%%%%%%%%%%%%%%%%%%%
%%%%%%%%%%%%%%%%%%%%%%%%%%%%%%%%%%%%%%%%%%%%%%%%%%%%%%%%%%%%%%%%%%%%%%%%%%%%%%%
\newpage
% \section{You \emph{can} have an appendix here.}
% \appendix
% \section{Appendix}
% You may include other additional sections here.

\renewcommand\thefigure{A\arabic{figure}}
\renewcommand\thetable{A\arabic{table}}
\setcounter{table}{0}
\setcounter{figure}{0}

\appendix

\newpage
\onecolumn

\section*{Appendix}
% The appendix is structured as follows:
\begin{itemize}[leftmargin=1.25em]
    \vspace{-0.5em}
    \item In Appendix~\ref{app:implement}, we provide implementation details of pre-training datasets, network architectures, and training schemes of pre-training and fine-tuning stages with hyper-parameter settings.
    \vspace{-0.5em}
    \item In Appendix~\ref{app:additional_res}, we provide detailed descriptions of downstream tasks of biological applications and full comparison results.
\end{itemize}


\begin{figure}[ht]
    % \vspace{-0.5em}
    \centering
    \includegraphics[width=0.65\linewidth]{figs/fig_data_collection.pdf}
    % \vspace{-2.0em}
    \caption{\textbf{Data Collection Pipeline}. We collect the reference sequences from the NCBI RefSeq database as the DNA dataset and collect the cDNA of coding sequences with the corresponding amino acids from the GenBank and UniRef50 databases to build up our DNA-AA paring datasets.
    }
    \label{fig:data_collection}
    \vspace{-1.0em}
\end{figure}


\section{Implementation Details}
\label{app:implement}

\begin{wraptable}{r}{0.5\linewidth}
% \begin{table}[h]
    \vspace{-2.25em}
    \setlength{\tabcolsep}{1.4mm}
    \centering
    \caption{
    Configuration of pre-training datasets. As for the DNA dataset, we collect DNA/RNA sequences from the NCBI RefSeq database. As for the DNA-AA pairing dataset, we collect the cDNA (coding sequences) and its corresponding Amino Acids (or using translation) from the GenBank database and also collect Amino Acids and their corresponding cDNA sequences using reverse translation from the UniRef50 database.
    }
    \vspace{1pt}
\resizebox{1.0\linewidth}{!}{
\begin{tabular}{l|ccc}
    \toprule
% 
Dateset & Data Type    & Seq Count  & Data Source \\ \hline
DNA     & DNA          & 51,257,875 & NCBI RefSeq \\
        & RNA          & 6,463,852  & NCBI RefSeq \\ \hline
        & cDNA-AA pair & 16,786,593 & GenBank     \\
DNA-AA  & AA-cDNA pair & 17,245,138 & UniRef50    \\
        & mRNA-AA pair & 3.908,074  & GenBank     \\
% 
    \bottomrule
    \end{tabular}
    }
    \label{tab:app_dataset}
    \vspace{-0.5em}
% \end{table}
\end{wraptable}

\subsection{Pre-training Dataset}
We collect two datasets for pre-training of Life-Code models, \textit{i.e.}, a pure DNA dataset and a DNA-AA pairing dataset, with the collection process shown in Figure~\ref{fig:data_collection}.
As for the DNA dataset, we collect the reference sequences (RefSeq) of multiple species to ensure generalization abilities from the database of the National Center for Biotechnology Information (NCBI) at \url{https://www.ncbi.nlm.nih.gov}, following the Multi-species Genomes\footnote{Multi-species Genomes are originally provided in \url{https://huggingface.co/datasets/InstaDeepAI/multi_species_genomes}, which is further extended by DNABERT-2 in \url{https://github.com/MAGICS-LAB/DNABERT_2}} provided by Nucleotide Transformer~\citep{NM2023NucleotideTrans} and DNABERT2~\citep{iclr2024dnabert2}.
As for the DNA-AA pairing dataset, we collect the cDNA of coding sequences (CDS) and its corresponding Amino Acids (AA) in the GenBank database at \url{https://www.ncbi.nlm.nih.gov/genbank}, which aims to model the transcription and translation processes of the central dogma. We also collect some Amino Acids in the UniRef50 database following LucaOne~\citep{he2024lucaone} and obtain their corresponding cDNA by reverse translation with online tools.
We provide detailed information for used datasets in Table~\ref{tab:app_dataset}.


\subsection{Life-Code Tokenizer}
\label{app:impl_tokenizer}
\paragraph{Vocabulary.}
There are two vocabularies used in Life-Code. The unified vocabulary only uses 4 nucleotides \{A, T/U, C, G\} of nucleic acid with 5 special tokens, including ``[U]"/``[UNK]", ``[PAD]", ``[CLS]", ``[SEP]", and “[MASK]” for unknown nucleotides, padding tokens, the class token, separator tokens, and masking tokens. Meanwhile, the Life-Code can also use the codon vocabulary (\textit{i.e.}, the 3-mer of 4 nucleotides that constructs 64 codon tokens), which could be merged into 20 amino acids of protein (20 uppercase letters excluding ``B", ``J", ``O", ``U", ``X", and ``Z"). It can only be applied when the length of an input sequence is multiples of 3, \textit{i.e.}, the cDNA of amino acids or matured mRNA (CDS). The pre-trained protein language model employs the amino acid vocabulary of ESM-2 \citep{lin2022ESM2}.

\begin{wraptable}{r}{0.45\linewidth}
% \begin{table}[ht]
    \vspace{-0.75em}
    % \setlength{\tabcolsep}{1.6mm}
    \centering
    \caption{
    Configuration of the network designs and pre-training settings for Life-Code models. GDN blocks denote the Gated DeltaNet block with linear complexity, while Attention blocks denote the self-attention block with FLASH-Attention implementation.
    }
    \vspace{1pt}
\resizebox{0.95\linewidth}{!}{
\begin{tabular}{l|cc}
    \toprule
% 
Configuration       & Tokenizer          & Encoder         \\ \hline
Embedding dim       & 384                & 1024            \\
Block number        & 1                  & 24              \\
GDN blocks          & 1                  & 22              \\
Attention blocks    & 0                  & 2               \\
Attention heads     & 0                  & 24              \\
Parameters          & 8M                 & 340M            \\ \hline
Optimizer           & \multicolumn{2}{c}{AdamW}            \\
$(\beta_1,\beta_2)$ & \multicolumn{2}{c}{$(0.9,0.98)$}     \\
Training iterations & 100,000            & 1,000,000       \\
Weight decay        & \multicolumn{2}{c}{$1\times 10^{-2}$}  \\
Base learning rate  & $2\times 10^{-4}$  & $1\times 10^{-4}$ \\
Batch size          & 512                & 256             \\
LR scheduler        & \multicolumn{2}{c}{Cosine Annealing} \\
Warmup iterations   & 5000               & 10,000          \\
Gradient clipping   & \multicolumn{2}{c}{1.0}              \\
% 
    \bottomrule
    \end{tabular}
    }
    \label{tab:app_lcode_config}
    \vspace{-0.5em}
% \end{table}
\end{wraptable}

\vspace{-0.5em}
\paragraph{Tokenizer Network.}
As shown in Figure~\ref{fig:tokenizer}, with the nucleotide vocabulary (including 4 nucleic acids and 5 special symbols), the \textit{Life-Code tokenizer} contains the following modules: (a) A linear projection from 9-dim to 384-dim implemented by \texttt{nn.Embedding}; (b) A GDN block (Gated DeltaNet) with 384-dim for global contextual modeling with linear computational complexity; (c) A 1-d Convolution with a kernel size of 3 and stride of 1, followed by an \texttt{UnFold} operation to merge every three nucleotide tokens into 768-dim codon embedding. Similarly, we design the \textit{DNA De-Tokenizer} with the symmetrical network as the Life-Code tokenizer: (a) \texttt{Fold} operation with a 1-d Convolution with a kernel size of 3 to unmerge the codon embedding to 384-dim, (b) A linear projection from 384-dim to 9-dim vocabulary to reconstruct the original DNA sequences. We also design the \textit{Amino Acid Translator} as a two-layer MLP that translates the codon embedding to the corresponding Amino Acid sequences.

\vspace{-0.5em}
\paragraph{Pre-training Settings.}
As shown in Figure~\ref{fig:tokenizer}, we pre-train the Life-Code tokenizer with the DNA de-tokenizer and Amino Acid de-tokenizer by AdamW optimizer for 100,000 iterations (randomly sampled datasets) with a basic learning rate of 2e-4 and a batch size of 512, as detailed in Table~\ref{tab:app_lcode_config}. We utilize 8 Nvidia A100-80G GPUs with a per-GPU batch size of 8 and a gradient accumulation time of 4.


\subsection{Life-Code Encoder}
\label{app:impl_encoder}
\paragraph{Encoder Architecture.}
As shown in Table~\ref{tab:app_lcode_config}, the Life-Code encoder has 24 layers in total with the embedding dim of 1024 with the following designs: (1) Mixture of GDN blocks (DeltaNet~\citep{yang2025deltanet}) and multi-head self-attention (MHSA) blocks as a hybrid model, especially every 11 GDN blocks followed by a self-attention block like MiniMax-01~\citep{MiniMax2025MiniMax01}, which could utilize the complementary properties of GDN and MHSA while maintaining efficiency. (2) The model macro design employs pre-norm~\citep{acl2019PreNorm} with RMSNorm~\citep{Zhang2019RMSNorm}, Layer Scale~\citep{iccv2021CaiT}, Rotary Position Embedding (RoPE)~\citep{Su2021RoFormer}, SwiGLU~\citep{Touvron2023LLaMA}, and FlashAttention implementations to facilitate training large-scale models stably with long sequences. (3) During pre-training, we apply the packing strategy~\citep{Warner2024ModernBERT} to build up a long sequence with several CDS, which compromises the gap between different lengths of the reference sequence and the coding sequences, as shown in Figure~\ref{fig:data_packing}.

\vspace{-0.5em}
\paragraph{Pre-training Settings.}
As shown in Figure~\ref{fig:lcode_pretraining}, we further pre-train the Life-Code tokenizer and Encoder with three tasks in Eq.~\ref{eq:loss_total} for 1M steps with the batch size of 256 and the basic learning rate of $1\times 10^{-4}$. We adopt 15\% random masking in BERT for Masked DNA Reconstruction and the 3-mer span masking for CDS-to-Amino-Acid Translation. As for Knowledge Distillation from a pre-trained Protein LM, we adopted ESM2-650M (\textit{esm2\_t33\_650M\_UR50D})~\citep{lin2022ESM2} and a protein decoder with the output dimension of 1280. During the warmup periods, the maximum sequence length is 1024, with a linear warmup of the learning rate for 10 iterations. After that, the maximum sequence length is set to 4k with the learning rate adjusted by the Cosine Annealing scheduler (decay to $1\times 10^{-6}$). We utilize 8 Nvidia A100-80G GPUs with a per-GPU batch size of 2 and a gradient accumulation time of 16.


\subsection{Supervised Fine-tuning}
\label{app:impl_SFT}
In most cases, we apply Supervised Fine-tuning (SFT) to transfer pre-trained models to downstream tasks. Following \citep{nips2024hyenadna, iclr2024dnabert2}, adding the decoder head (\textit{e.g.}, an MLP head) to a specific downstream task, the linear attention (RNN) or self-attention blocks in the pre-trained encoder models are frozen, while Low-Rank Adaptation (LoRA) strategy~\citep{Hu2021LoRA} is employed to parameter-efficiently fine-tuning the models by AdamW optimizer with a batch size of 32. For each task, if the benchmark and models have provided hyper-parameters, we follow the official settings, or we choose the best combinations of the basic learning rate \{1e-5, 5e-5, 1e-4\}, the weight decay \{0, 0.01\}, the LoRA rank \{4, 8, 16, 24, 48\}, the LoRA alpha \{8, 16, 24, 48, 96\}, and the total fine-tuning epoch \{5, 10\} on the validation set following the GUE benchmark and GenBench~\citep{liu2024genbench}. Note that the maximum input length will be determined for different tasks since the sequence lengths of downstream tasks vary widely. We report the averaged results over three runs with the optimal settings.


\begin{table*}[t]
    \centering
    \vspace{-0.5em}
    \caption{\textbf{Full Results on Genomic Benchmarks}. Top-1 accuracy (\%) averaged across three trials is reported for the latest DNA foundation models, where the best and the second best results are marked as the \textbf{bold} and \underline{underlined} types.
    }
    \vspace{1pt}
    \setlength{\tabcolsep}{0.8mm}
\resizebox{1.0\linewidth}{!}{
    \begin{tabular}{l|cccccccccb}
    \toprule
% 
Method                  & HyenaDNA & DNABERT & DNABERT2  & GENA-LM     & NT-500M     & Caduceus-16 & VQDNA & MxDNA          & ConvNova & \textbf{Life-Code} \\
\# Params (M)           & 6.6      & 86      & 117       & 113         & 498         & 7.9         & 93    & 100            & 1.7      & 350                \\ \hline
Mouse Enhancers         & 79.34    & 80.99   & 81.82     & 82.97       & \ul{85.12}  & 81.63       & 81.06 & 80.57          & 78.40    & \textbf{85.46}     \\
Human Enhancers Cohn    & 72.96    & 70.23   & 75.87     & 75.63       & \ul{76.12}  & 73.76       & 75.63 & 74.67          & 74.30    & \textbf{76.85}     \\
Human Enhancers Ensembl & 90.33    & 89.19   & 90.75     & 91.07       & 92.44       & 84.48       & 90.41 & \ul{93.13}     & 90.00    & \textbf{93.49}     \\ \hline
Coding vs Intergenomic  & 90.97    & 93.64   & 93.58     & 93.24       & \ul{95.76}  & 93.72       & 94.35 & 95.28          & 94.30    & \textbf{96.14}     \\
Human vs Worm           & 96.24    & 95.84   & 97.39     & 96.98       & 97.51       & 95.57       & 97.23 & \ul{97.64}     & 96.70    & \textbf{97.75}     \\ \hline
Human Regulatory        & 93.08    & 88.16   & 87.94     & 88.10       & 93.79       & 87.30       & 90.92 & \textbf{94.11} & 87.30    & \ul{93.93}         \\
Human OCR Ensembl       & 79.14    & 74.96   & 75.82     & 78.98       & 80.42       & \ul{81.76}  & 76.58 & 81.05          & 79.30    & \textbf{82.02}     \\
Human NonTATA Promoters & 94.45    & 87.13   & 95.24     & \ul{96.60}  & 92.95       & 88.85       & 95.37 & 96.56          & 95.30    & \textbf{96.65}     \\
% 
    \bottomrule
    \end{tabular}
    }
    \label{tab:app_genomic_benchmark}
    \vspace{-0.5em}
\end{table*}

\begin{table*}[t]
    \centering
    \vspace{-0.5em}
    \caption{\textbf{Full Results on GUE benchmark}. MCC (\%) is reported for Epigenetic Marks Prediction, Human Transcription Factor (TF) Prediction, Mouse Transcription Factor Prediction, Core Promoter Detection, Promoter Detection, Splice Site Reconstructed, and Covid Variants Classification (Virus Covid). The best and the second best results are marked as the \textbf{bold} and \underline{underlined} types.
    }
    \vspace{1pt}
    \setlength{\tabcolsep}{1.0mm}
\resizebox{1.0\linewidth}{!}{
    \begin{tabular}{l|ccccccccb}
    \toprule
% 
Method                     & HyenaDNA & DNABERT & NT-2500M-multi & DNABERT2   & Caduceus-PS & VQDNA          & MxDNA          & ConvNova       & \textbf{Life-Code} \\
\# Params (M)              & 6.6      & 86      & 2537           & 117        & 1.9         & 93             & 100            & 1.7            & 350                \\ \hline
Human TF-0                 & $-$      & 66.84   & 66.64          & \ul{71.99} & $-$         & \textbf{72.48} & $-$            & $-$            & 71.58              \\
Human TF-1                 & $-$      & 70.14   & 70.28          & \ul{76.06} & $-$         & \textbf{76.43} & $-$            & $-$            & 75.92              \\
Human TF-2                 & $-$      & 61.03   & 58.72          & 66.52      & $-$         & \ul{66.85}     & $-$            & $-$            & \textbf{70.63}     \\
Human TF-3                 & $-$      & 51.89   & 51.65          & 58.54      & $-$         & \textbf{58.92} & $-$            & $-$            & \ul{58.74}         \\
Human TF-4                 & $-$      & 70.97   & 69.43          & 77.43      & $-$         & \ul{78.10}     & $-$            & $-$            & \textbf{79.45}     \\ \hline
Mouse TF-0                 & $-$      & 44.42   & \ul{63.31}     & 56.76      & $-$         & 58.34          & $-$            & $-$            & \textbf{64.10}     \\
Mouse TF-1                 & $-$      & 78.94   & 83.76          & 84.77      & $-$         & \ul{85.81}     & $-$            & $-$            & \textbf{86.51}     \\
Mouse TF-2                 & $-$      & 71.44   & 71.52          & 79.32      & $-$         & \ul{80.39}     & $-$            & $-$            & \textbf{80.49}     \\
Mouse TF-3                 & $-$      & 44.89   & 69.44          & 66.47      & $-$         & \ul{69.72}     & $-$            & $-$            & \textbf{71.25}     \\
Mouse TF-4                 & $-$      & 42.48   & 47.07          & 52.66      & $-$         & \ul{54.73}     & $-$            & $-$            & \textbf{55.46}     \\ \hline
Core Promoter (all)        & $-$      & 68.90   & 70.33          & 69.37      & $-$         & \textbf{71.02} & $-$            & $-$            & \ul{70.69}         \\
Core Promoter (no TATA)    & $-$      & 70.47   & \textbf{71.58} & 68.04      & $-$         & 70.58          & $-$            & $-$            & \ul{71.05}         \\
Core Promoter (TATA)       & $-$      & 76.06   & 72.97          & 74.17      & $-$         & \ul{78.50}     & $-$            & $-$            & \textbf{78.78}     \\ \hline
Promoter (all)             & $-$      & 90.48   & \ul{91.01}     & 86.77      & $-$         & 90.75          & $-$            & $-$            & \textbf{91.33}     \\
Promoter (no TATA)         & $-$      & 93.05   & 94.00          & 94.27      & $-$         & \ul{94.48}     & $-$            & $-$            & \textbf{95.03}     \\
Promoter (TATA)            & $-$      & 61.56   & \textbf{79.43} & 71.59      & $-$         & 74.52          & $-$            & $-$            & \ul{78.97}         \\ \hline
Splice Reconstructed       & $-$      & 84.07   & 89.35          & 84.99      & $-$         & \ul{89.53}     & $-$            & $-$            & \textbf{89.76}     \\ \hline
H3                         & 78.14    & 73.10   & 78.77          & 78.27      & 77.90       & 79.21          & \textbf{82.14} & \ul{81.50}     & 81.28              \\
H3K14ac                    & 56.71    & 40.06   & 56.20          & 52.57      & 54.10       & 54.46          & 68.29          & \textbf{70.71} & \ul{68.41}         \\
H3K36me3                   & 59.92    & 47.25   & 61.99          & 56.88      & 60.90       & 61.75          & 65.46          & \textbf{68.31} & \ul{67.25}         \\
H3K4me1                    & 44.52    & 41.44   & 55.30          & 50.52      & 48.80       & 53.28          & 54.97          & \ul{56.60}     & \textbf{57.32}     \\
H3K4me2                    & 42.68    & 32.27   & 36.49          & 31.13      & 38.80       & 34.05          & \ul{55.30}     & \textbf{57.45} & 50.31              \\
H3K4me3                    & 50.41    & 27.81   & 40.34          & 36.27      & 44.00       & 39.10          & \ul{63.82}     & \textbf{67.15} & 53.97              \\
H3K79me3                   & 66.25    & 61.17   & 64.70          & 67.39      & 67.60       & 68.47          & \textbf{73.74} & 72.08          & \ul{72.26}         \\
H3K9ac                     & 58.50    & 51.22   & 56.01          & 55.63      & 60.40       & 56.63          & 63.15          & \textbf{68.10} & \ul{65.45}         \\
H4                         & 78.15    & 79.26   & 81.67          & 80.71      & 78.90       & \ul{81.84}     & 80.89          & 81.12          & \textbf{81.89}     \\
H4ac                       & 54.15    & 37.24   & 49.13          & 50.43      & 52.50       & 50.69          & \ul{65.14}     & \textbf{66.10} & 61.37              \\ \hline
Virus Covid Classification & $-$      & 55.50   & 73.04          & 71.02      & $-$         & \textbf{74.32} & $-$            & $-$            & \ul{73.82}         \\
% 
    \bottomrule
    \end{tabular}
    }
    \label{tab:app_gue}
    \vspace{-0.5em}
\end{table*}


\section{Downstream Task Settings and Extensive Comparison Results}
\label{app:additional_res}
% 
\subsection{DNA Tasks with Genomics Benchmark}
As proposed by \citep{BMC2023genomicbenchmark}, three groups of basic genomic tasks are collected as binary classification with top-1 accuracy in the Genomics Benchmark. As for the enhancer prediction, three datasets are provided for identifying enhancer regions in the mouse or human genome. As for the species classification, two datasets are selected for identifying sequences as either coding (exonic) or intergenic (non-coding) and classifying sequences as originating from humans or worms (C. elegans). As for the regulatory elements classification, three datasets are used for classifying sequences as regulatory regions based on Ensembl annotations, identifying open chromatin regions, or identifying non-TATA promoter regions in the human genome. We utilize the fully reproduced results of various DNA models in GenBench \citep{liu2024genbench}.


\subsection{DNA Tasks with GUE Benchmark}
As proposed by DNABERT2~\citep{iclr2024dnabert2}, the GUE benchmark contains 24 datasets of 7 practical biological genome analysis tasks for 4 different species using Matthews Correlation Coefficient (MCC) as the evaluation metric. To comprehensively evaluate the genome foundation models in modeling variable-length sequences, tasks with input lengths ranging from 70 to 1000 are selected. The following descriptions of the supported tasks are included in the GUE benchmark, where these resources are attached for illustration.

\textbf{Promoter Detection (Human).}\quad
This task identifies human proximal promoter regions essential for transcription initiation. Accurate detection aids in understanding gene regulation and disease mechanisms. The dataset includes TATA and non-TATA promoters, with sequences -249 to +50 bp around the Transition Start Site (TSS) from Eukaryotic Promoter Database (EPDnew) \citep{dreos2013epdpromoter}. Meanwhile, we construct the non-promoter class with equal-sized randomly selected sequences outside of promoter regions but with TATA motif (TATA non-promoters) or randomly substituted sequences (non-TATA, non-promoters). We also combine the TATA and non-TATA datasets to obtain a combined dataset named \textit{all}.

\textbf{Core Promoter Detection (Human).}\quad
This task is similar to the detection of the proximal promoter with a focus on predicting only the core promoter region, the central region closest to the TSS, and the start codon. A much shorter context window (center -34~+35 bp around TSS) is provided, making this a more challenging task than the prediction of the proximal promoter. 

\textbf{Transcription Factor Binding Site Prediction (Human).}\quad
This task predicts human transcription factor binding sites (TFBS), crucial for gene expression regulation. Data from 690 ENCODE ChIP-seq experiments (161 TF binding profiles in 91 cell lines) \citep{mouse2012encyclopedia} are collected via the UCSC genome browser. TFBS sequences are 101-bp regions around peaks, while non-TFBS sequences match in length and GC content. There are 5 datasets selected from a curated subset of 690, excluding trivial or overly challenging tasks.

\textbf{Splice Site Prediction (Human).}\quad
This task predicts splice donor and acceptor sites, the exact locations in the human genome where alternative splicing occurs. This prediction is crucial to understanding protein diversity and the implications of aberrant splicing in genetic disorders. The dataset \citep{BMC2021spliceator} consists of 400-bp-long sequences extracted from Ensembl GRCh38 human reference genome. As suggested by \citet{BioInfo2021dnabert}, existing models can achieve almost perfect performance on the original dataset, containing 10,000 splice donors, acceptors, and non-splice site sequences, which is overly optimistic about detecting non-canonical sites in reality. As such, we reconstruct the dataset by iteratively adding adversarial examples (unseen false positive predictions in the hold-out set) in order to make this task more challenging.  
 
\textbf{Transcription Factor Binding Site Prediction (Mouse).}\quad
This task predicts mouse transcription factor binding sites using mouse ENCODE ChIP-seq data (n=78) \citep{mouse2012ChIPseq} from the UCSC genome browser. Negative examples are created by di-nucleotide shuffling. Five datasets are randomly selected from the 78 datasets using the same process as the human TFBS prediction dataset.

\textbf{Epigenetic Marks Prediction (Yeast).}\quad
This task predicts epigenetic marks in yeast, which influence gene expression without altering DNA sequences. Precise prediction of these marks aids in elucidating the role of epigenetics in yeast. We download the 10 datasets from \url{http://www.jaist.ac.jp/~tran/nucleosome/members.htm} and randomly split each dataset into training, validation, and test sets.

\textbf{Covid Variant Prediction (Virus).}\quad
This task aims to predict the variant type of the SARS\_CoV\_2 virus based on 1000-length genome sequences. We download the genomes from the EpiCoV database \citep{khare202SARS_CoV_2} of the Global Initiative on Sharing Avian Influenza Data (GISAID). We consider 9 types of SARS\_CoV\_2 variants, including \textit{Alpha}, \textit{Beta}, \textit{Delta}, \textit{Eta}, \textit{Gamma}, \textit{Iota}, \textit{Kappa}, \textit{Lambda} and \textit{Zeta}.


\begin{table}[htb]
    \centering
    \vspace{-0.5em}
    \caption{\textbf{Full Results of mRNA Splicing Site Prediction}. With two splicing datasets proposed by SpliceAI and Spliceator, the top-1 AUC score or F1 score is reported with DNA models or RNA models, respectively. The best and the second best results are marked as the \textbf{bold} and \underline{underlined} types.
    }
    \vspace{1pt}
    % \setlength{\tabcolsep}{0.5mm}
\resizebox{0.75\linewidth}{!}{
    \begin{tabular}{l|cccccb}
    \toprule
% 
Dataset  & SpliceAI   & DNABERT2     & NT   & GENA-LM & Caduceus      & \textbf{Life-Code} \\ \hline
Donor    & 57.4       & 63.5         & 55.7 & 62.9    & \ul{64.2}     & \textbf{64.3}      \\
Acceptor & 69.1       & 70.7         & 72.2 & 73.0    & \ul{74.0}     & \textbf{74.6}      \\
Mean     & 63.2       & 67.1         & 63.9 & 67.9    & \ul{69.1}     & \textbf{70.0}      \\
\toprule
Dataset  & Spliceator & SpliceFinder & DDSP & RNA-FM  & RINALMo       & \textbf{Life-Code} \\ \hline
Human    & 90.0       & 84.5         & 90.6 & 90.7    & \ul{91.3}     & \textbf{91.5}      \\
Fish     & 91.9       & 91.8         & 93.6 & 93.7    & \textbf{97.4} & \ul{97.3}          \\
Fly      & 91.0       & 84.2         & 91.4 & 91.9    & \textbf{95.8} & \ul{94.7}          \\
% 
    \bottomrule
    \end{tabular}
    }
    \label{tab:app_splicing}
    \vspace{-0.5em}
\end{table}


\subsection{mRNA Splicing Tasks}
% \label{app:rna_splicing}
Following \citep{iclr2024dnabert2, shen2024rnafm}, we evaluate pre-mRNA Splicing Site Prediction as the RNA task, which is a crucial process in eukaryotic gene expression. During splicing, introns are removed from precursor messenger RNAs (pre-mRNAs), and exons are joined together to form mature mRNAs. This process is essential for generating functional
mRNAs that could be translated into proteins. Identifying splice sites—the donor sites at the 5' end of introns and the acceptor sites at the 3' end—is vital for accurately predicting gene structure and location.
Concretely, we regard this task as binary classification of RNA splicing site prediction specifically for acceptor sites and consider two splicing datasets in addition to the \textit{Splicefinder} dataset \citep{wang2019splicefinder} used in the GUE benchmark.

\textbf{Spliceator dataset.}\quad
This dataset~\citep{BMC2021spliceator} consists of ``confirmed" error-free splice-site sequences from a diverse set of 148 eukaryotic organisms, including humans. The gold standard dataset GS 1 is adopted, which contains an equal number of positive and negative samples, and the F1 score is used as the evaluation metric. We chose three independent test datasets containing the samples from 3 different species of humans, fish (Danio rerio), and fruit fly (Drosophila melanogaster).

\textbf{SpliceAI dataset.}\quad
This dataset~\citep{JAGANATHAN2019SpliceAI} also constructs a binary classification dataset similar to Spliceator, which utilizes the GTEx (Genotype-Tissue Expression) project for RNA sequencing data from various human tissues and the GENCODE V24lift37 canonical annotation for gene structure information. SpliceAI also references the ClinVar database to evaluate the clinical significance of predicted splicing variants, which contains information on clinically relevant variants and their associations with diseases. This dataset can be regarded as a long-range evaluation and adopts the top-1 AUC-ROC score as the metric.


\begin{table}[htb]
    \centering
    \vspace{-0.5em}
    \caption{\textbf{Full Results of Foundation Models}. As for DNA and RNA tasks, MCC (\%) is reported for Promoter Detection (all), Covid Variants Classification, and Splice Site Reconstructed. SRCC (\%) is reported for Bacterial and Human Protein Fitness Prediction with DMS. Top-1 accuracy (\%) is reported for the Central Dogma evaluation. The best result is marked as the \textbf{bold} type.
    }
    \vspace{1pt}
    % \setlength{\tabcolsep}{0.7mm}
\resizebox{0.67\linewidth}{!}{
    \begin{tabular}{l|ccgb}
    \toprule
% 
Method                     & NT-2500M & EVO-7B & LucaOne       & \textbf{Life-Code} \\ \hline
Promoter (all)             & 91.0     & 88.5   & \textbf{91.6} & 91.3               \\
Virus Covid Classification & 73.0     & 58.7   & \textbf{75.1} & 73.8               \\ \hline
Splice Reconstructed       & 89.4     & 87.5   & 89.1          & \textbf{89.8}      \\ \hline
Bacterial Protein          & 9.4      & 45.3   & \textbf{46.1} & 45.7               \\
Human Protein              & 4.7      & 11.1   & 19.6          & \textbf{22.4}      \\ \hline
Central Dogma              & $-$      & 75.5   & 84.8          & \textbf{85.6}      \\
% 
    \bottomrule
    \end{tabular}
    }
    \label{tab:app_overall}
    \vspace{-0.5em}
\end{table}


\subsection{Protein and Multi-omic Tasks}
% \label{app:protein_multiomics}
\paragraph{Zero-shot Protein Fitness Prediction.}
Following EVO~\citep{nguyen2024evo} and protein language models~\citep{lin2023esm}, we employ Deep Mutational Scanning (DMS) studies to evaluate the models' abilities for protein tasks, which introduce many mutations to a protein coding sequence and then experimentally measure the effects of these mutations (as fitness scores) on various definitions of fitness~\citep{icml2022Tranception}. EVO obtained (DMS) datasets with bacterial (prokaryote) and human (eukaryote) proteins from ProteinGYM at \url{https://proteingym.org}.
To adapt this task to nucleotide sequences, EVO proposes to use the wild-type coding sequence and nucleotide mutations reported in the original DMS studies \citep{icml2022Tranception, altae2021widespread}. For generative pre-trained models such as EVO, we rely on likelihood-based scores under the same masking scheme, assessing how well the model anticipates mutations.
The model performances of zero-shot function prediction are measured by the strength of Spearman's Rank Correlation Coefficient (SRCC) that correlates the predicted likelihoods with the experimental fitness measurements.
Full results of protein fitness prediction are shown in Table~\ref{tab:protein_dms} and Table~\ref{tab:app_overall}, in which our Life-Code achieves balancing performances with bacterial and human proteins and surpasses NT-2500M and EVO-7B.


\vspace{-0.5em}
\paragraph{ncRNA-Protein Interactions.}
Following LucaOne~\citep{he2024lucaone}, we consider the multi-omics task of ncRNA-protein interactions (ncRPI), which identifies the interaction strengths between non-coding RNAs (\textit{e.g.}, snRNAs, snoRNAs, miRNAs, and lncRNAs) and proteins. Since experimentally identifying ncRPI) is typically expensive and time-consuming, the AI-based ncRPI can be a promising task. LucaOne proposes a binary classification task involving pairs of ncRNA and Amino Acid sequences (20,824 pairs in total) with top-1 accuracy as the metric.

\vspace{-0.5em}
\paragraph{Central Dogma Evaluation.}
To evaluate the modeling of the translation rule in the central dogma, we follow LucaOne~\citep{he2024lucaone} to conduct a binary classification task with top-1 accuracy, which determines whether the DNA sequences and the given proteins are correlated.
LucaOne collects a total of 8,533 accurate DNA-protein pairs from 13 species in the NCBI RefSeq database, with each DNA sequence extending to include an additional 100 nucleotides at both the 5' and 3' contexts, preserving intron sequences within the data. LucaOne generated
double the number of negative samples by implementing substitutions, insertions, and deletions within the DNA sequences or altering amino acids in the protein sequences to ensure the resultant DNA sequences could not be accurately translated into their respective proteins.



%%%%%%%%%%%%%%%%%%%%%%%%%%%%%%%%%%%%%%%%%%%%%%%%%%%%%%%%%%%%%%%%%%%%%%%%%%%%%%%
%%%%%%%%%%%%%%%%%%%%%%%%%%%%%%%%%%%%%%%%%%%%%%%%%%%%%%%%%%%%%%%%%%%%%%%%%%%%%%%


\end{document}


% This document was modified from the file originally made available by
% Pat Langley and Andrea Danyluk for ICML-2K. This version was created
% by Iain Murray in 2018, and modified by Alexandre Bouchard in
% 2019 and 2021 and by Csaba Szepesvari, Gang Niu and Sivan Sabato in 2022.
% Modified again in 2023 and 2024 by Sivan Sabato and Jonathan Scarlett.
% Previous contributors include Dan Roy, Lise Getoor and Tobias
% Scheffer, which was slightly modified from the 2010 version by
% Thorsten Joachims & Johannes Fuernkranz, slightly modified from the
% 2009 version by Kiri Wagstaff and Sam Roweis's 2008 version, which is
% slightly modified from Prasad Tadepalli's 2007 version which is a
% lightly changed version of the previous year's version by Andrew
% Moore, which was in turn edited from those of Kristian Kersting and
% Codrina Lauth. Alex Smola contributed to the algorithmic style files.
