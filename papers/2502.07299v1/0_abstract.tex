\begin{abstract}

The interactions between DNA, RNA, and proteins are fundamental to biological processes, as illustrated by the central dogma of molecular biology.
While modern biological pre-trained models have achieved great success in analyzing these macromolecules individually, their interconnected nature remains under-explored.
In this paper, we follow the guidance of the central dogma to redesign both the data and model pipeline and offer a comprehensive framework, Life-Code, that spans different biological functions. As for data flow, we propose a unified pipeline to integrate multi-omics data by reverse-transcribing RNA and reverse-translating amino acids into nucleotide-based sequences. As for the model, we design a codon tokenizer and a hybrid long-sequence architecture to encode the interactions of both coding and non-coding regions with masked modeling pre-training.
% We utilize these encoded sequences for both genome and protein understanding tasks, offering a comprehensive framework that spans different biological functions.
To model the translation and folding process with coding sequences, Life-Code learns protein structures of the corresponding amino acids by knowledge distillation from off-the-shelf protein language models.
% to optimize performance under the resource of rich large pre-trained models,
% we apply knowledge distillation techniques on large protein models on specific coding regions on DNA sequences.
Such designs enable Life-Code to capture complex interactions within genetic sequences, providing a more comprehensive understanding of multi-omics with the central dogma.
% providing a more comprehensive understanding of the genome’s role in metabolic activities.
Extensive Experiments show that Life-Code achieves state-of-the-art performance on various tasks across three omics, highlighting its potential for advancing multi-omics analysis and interpretation.

\end{abstract}