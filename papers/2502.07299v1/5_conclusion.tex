\section{Conclusion and Limitations}
We have introduced Life-Code, a method that unifies DNA, RNA, and protein sequences by mapping the latter two back to a DNA-like representation. This design offers straightforward \emph{multi-omics} integration and maintains \emph{biological interpretability} via differential encoding of coding versus non-coding regions. To handle the long-range dependencies inherent in genomic data, we employ an efficient \emph{symmetric} convolution-based architecture. Moreover, \emph{knowledge distillation} from large protein models allows our approach to scale without excessive resource demands.

% \vspace{-1em}
\paragraph{Limitations} Experimental results demonstrate that Life-Code achieves competitive performance in protein structure prediction and phenotype analysis. However, limitations remain: refining the modeling of heterogeneous non-coding regions, incorporating \emph{post-translational modifications}, and addressing the loss of fine-grained information during knowledge distillation. Future work will also focus on integrating additional omics layers (e.g., epigenomics, metabolomics) for a deeper understanding of biological systems. Despite these challenges, we believe \textbf{Life-Code} offers a promising step toward comprehensive \emph{multi-omics} data. 

% \section{Conclusion}
% In this paper, we present a novel method that encodes RNA and protein sequences into DNA sequences using reverse transcription and reverse translation, providing a unified approach to integrate multiple sets of omics data. Our model incorporates biological interpretability through its differential encoding of coding and non-coding regions, while also addressing long-range interactions using a symmetric long convolutional architecture. Through knowledge distillation, we can reduce the resource demands of training on large protein models, making the approach scalable for practical applications. The experimental results demonstrate that our method achieves competitive performance in protein structure and phenotype prediction tasks. Despite its current limitations, this work provides an important step forward in multi-omics data integration. It lays the foundation for future research in modeling the complex relationships between DNA, RNA, and proteins. Future work will address the challenges of encoding non-coding regions, post-translational modifications, and expanding the model to incorporate additional omics data for a more holistic understanding of biological systems.

% \textbf{Limitations} While our proposed method presents a novel approach to integrate RNA and protein sequences into DNA sequences using reverse transcription and reverse translation, several limitations need to be addressed in future work. First, the current encoding scheme does not fully capture the biological complexity of non-coding regions. Although our model differentiates between coding and non-coding regions, the heterogeneity within non-coding regions, such as regulatory elements and non-coding RNAs, requires more refined modeling to improve phenotype prediction. Second, while our method leverages the codon degeneracy rules for encoding, it does not currently account for post-translational modifications or non-standard amino acids, which may limit the applicability of the model to more complex protein structures and functions. Additionally, while reducing computational overhead, the knowledge distillation process may lead to a loss of fine-grained information that could otherwise enhance the performance of our model in capturing subtle biological interactions. Finally, our model's performance on multi-omics data could be improved by integrating additional omics layers, such as epigenomics and metabolomics, to provide a more comprehensive understanding of biological systems.

\section*{Impact Statements}
This paper presents a novel approach, Life-Code, aimed at advancing the field of machine learning and bioinformatics by unifying the modeling of multi-omics data through the guidance of the Central Dogma of molecular biology. By integrating DNA, RNA, and protein data into a single framework, our work has the potential to improve the understanding of complex biological processes and drive advancements in genomics, proteomics, and systems biology.
\paragraph{Ethical Aspects}
This work has significant potential applications in fields such as personalized medicine, drug discovery, and disease diagnostics. However, we acknowledge the need to address concerns related to data privacy and the responsible use of sensitive genetic information. We have adhered to ethical guidelines in our experiments, ensuring that no private or identifiable data was used.