%%
%% This is file `sample-manuscript.tex',
%% generated with the docstrip utility.
%%
%% The original source files were:
%%
%% samples.dtx  (with options: `manuscript')
%% 
%% IMPORTANT NOTICE:
%% 
%% For the copyright see the source file.
%% 
%% Any modified versions of this file must be renamed
%% with new filenames distinct from sample-manuscript.tex.
%% 
%% For distribution of the original source see the terms
%% for copying and modification in the file samples.dtx.
%% 
%% This generated file may be distributed as long as the
%% original source files, as listed above, are part of the
%% same distribution. (The sources need not necessarily be
%% in the same archive or directory.)
%%  
%% Commands for TeXCount
%TC:macro \cite [option:text,text]
%TC:macro \citep [option:text,text]
%TC:macro \citet [option:text,text]
%TC:envir table 0 1
%TC:envir table* 0 1
%TC:envir tabular [ignore] word
%TC:envir displaymath 0 
%TC:envir math 0 word
%TC:envir comment 0 0
%%
%%
%% The first command in your LaTeX source must be the \documentclass command.
%%%% Small single column format, used for CIE, CSUR, DTRAP, JACM, JDIQ, JEA, JERIC, JETC, PACMCGIT, TAAS, TACCESS, TACO, TALG, TALLIP (formerly TALIP), TCPS, TDSCI, TEAC, TECS, TELO, THRI, TIIS, TIOT, TISSEC, TIST, TKDD, TMIS, TOCE, TOCHI, TOCL, TOCS, TOCT, TODAES, TODS, TOIS, TOIT, TOMACS, TOMM (formerly TOMCCAP), TOMPECS, TOMS, TOPC, TOPLAS, TOPS, TOS, TOSEM, TOSN, TQC, TRETS, TSAS, TSC, TSLP, TWEB.
% \documentclass[acmsmall]{acmart}

%%%% Large single column format, used for IMWUT, JOCCH, PACMPL, POMACS, TAP, PACMHCI
% \documentclass[acmlarge,screen]{acmart}

%%%% Large double column format, used for TOG
% \documentclass[acmtog, authorversion]{acmart}

%%%% Generic manuscript mode, required for submission
%%%% and peer review
% \documentclass[manuscript,review,anonymous]{acmart}
\documentclass[sigconf]{acmart}
%% Fonts used in the template cannot be substituted; margin 
%% adjustments are not allowed.
%%
%% \BibTeX command to typeset BibTeX logo in the docs
\AtBeginDocument{%
  \providecommand\BibTeX{{%
    \normalfont B\kern-0.5em{\scshape i\kern-0.25em b}\kern-0.8em\TeX}}}

%% Rights management information.  This information is sent to you
%% when you complete the rights form.  These commands have SAMPLE
%% values in them; it is your responsibility as an author to replace
%% the commands and values with those provided to you when you
%% complete the rights form.
% \setcopyright{acmcopyright}
% \copyrightyear{2024}
% \acmYear{2024}
% \acmDOI{XXXXXXX.XXXXXXX}

%% These commands are for a PROCEEDINGS abstract or paper.
% \acmConference[]{}{}{}
%
%  Uncomment \acmBooktitle if th title of the proceedings is different
%  from ``Proceedings of ...''!
%
% \acmBooktitle{} 
% \acmPrice{}
% \acmISBN{}


\copyrightyear{2025}
\acmYear{2025}
\makeatletter
\def\@ACM@copyright@check@cc{}
\makeatother
\setcopyright{cc}
\setcctype{by}
\acmConference[CHI '25]{CHI Conference on Human Factors in Computing Systems}{April 26-May 1, 2025}{Yokohama, Japan}
\acmBooktitle{CHI Conference on Human Factors in Computing Systems (CHI '25), April 26-May 1, 2025, Yokohama, Japan}\acmDOI{10.1145/3706598.3713392}
\acmISBN{979-8-4007-1394-1/25/04}


%%
%% Submission ID.
%% Use this when submitting an article to a sponsored event. You'll
%% receive a unique submission ID from the organizers
%% of the event, and this ID should be used as the parameter to this command.
%%\acmSubmissionID{123-A56-BU3}

%%
%% For managing citations, it is recommended to use bibliography
%% files in BibTeX format.
%%
%% You can then either use BibTeX with the ACM-Reference-Format style,
%% or BibLaTeX with the acmnumeric or acmauthoryear sytles, that include
%% support for advanced citation of software artefact from the
%% biblatex-software package, also separately available on CTAN.
%%
%% Look at the sample-*-biblatex.tex files for templates showcasing
%% the biblatex styles.
%%

%%
%% The majority of ACM publications use numbered citations and
%% references.  The command \citestyle{authoryear} switches to the
%% "author year" style.
%%
%% If you are preparing content for an event
%% sponsored by ACM SIGGRAPH, you must use the "author year" style of
%% citations and references.
%% Uncommenting
%% the next command will enable that style.
%%\citestyle{acmauthoryear}

\usepackage{float}
\usepackage{subcaption}
\usepackage{multirow}
\usepackage{makecell}
% \usepackage{soul}
% \usepackage{ulem}


% \newcommand{\zhipeng}[1]{\textcolor{cyan}{ZP: #1}}
% \newcommand{\yj}[1]{\textcolor{teal}{[YJ: #1]}}
% \newcommand\change[1]{{\textcolor{blue}{#1}}}
% \newcommand \change[1]{{\textcolor{black}{#1}}}
% \newcommand\delete[1]{\textcolor{red}{\sout{#1}}}
% \newcommand\delete[1]{}

%%
%% end of the preamble, start of the body of the document source.
\begin{document}


%%
%% The "title" command has an optional parameter,
%% allowing the author to define a "short title" to be used in page headers.
% \title{
% \change{Modeling the Impact of Visual Stimuli on Redirection Noticeability with Gaze Behavior in Virtual Reality}
% \delete{Modelling Effects of Visual Attention on Noticeability of Body-Avatar Offsets in Virtual Reality}
% }
\title{Modeling the Impact of Visual Stimuli on Redirection Noticeability with Gaze Behavior in Virtual Reality}

%%
%% The "author" command and its associated commands are used to define
%% the authors and their affiliations.
%% Of note is the shared affiliation of the first two authors, and the
%% "authornote" and "authornotemark" commands
%% used to denote shared contribution to the research.
\author{Zhipeng Li}
\affiliation{%
  \institution{Tsinghua Univeristy}
  \city{Beijing}
  \country{China}
}
\affiliation{%
  \institution{ETH Zürich}
  \city{Zürich}
  \country{Switzerland}
}
\email{zhipeng.li@inf.ethz.ch}

\author{Yishu Ji}
\affiliation{%
  \institution{Georgia Institute of Technology}
  \city{Atlanta}
  \state{Georgia}
  \country{USA}
}
\email{yji329@gatech.edu}

\author{Ruijia Chen}
\affiliation{%
  \institution{University of Wisconsin-Madison}
  \city{Madison}
  \state{Wisconsin}
  \country{USA}
}
\email{ruijia.chen@wisc.edu}

\author{Tianqi Liu}
\affiliation{%
  \institution{Cornell University}
  \city{Ithaca}
  \state{New York}
  \country{USA}
}
\email{tl889@cornell.edu}

\author{Yuntao Wang}
\authornote{Corresponding author}
\affiliation{%
  \institution{Key Laboratory of Pervasive Computing, Ministry of Education, Tsinghua University}
  \city{Beijing}
  \country{China}
}
\email{yuntaowang@tsinghua.edu.cn}

\author{Yuanchun Shi}
\affiliation{%
  \institution{Tsinghua University}
  \city{Beijing}
  \country{China}
}
\email{shiyc@tsinghua.edu.cn}

\author{Yukang Yan}
\affiliation{%
  \institution{University of Rochester}
  \city{Rochester}
  \state{New York}
  \country{USA}
}
\email{yanyukanglwy@gmail.com}

% \author{Ben Trovato}
% \authornote{Both authors contributed equally to this research.}
% \email{trovato@corporation.com}
% \orcid{1234-5678-9012}
% \author{G.K.M. Tobin}
% \authornotemark[1]
% \email{webmaster@marysville-ohio.com}
% \affiliation{%
%   \institution{Institute for Clarity in Documentation}
%   \streetaddress{P.O. Box 1212}
%   \city{Dublin}
%   \state{Ohio}
%   \country{USA}
%   \postcode{43017-6221}
% }

% \author{Lars Th{\o}rv{\"a}ld}
% \affiliation{%
%   \institution{The Th{\o}rv{\"a}ld Group}
%   \streetaddress{1 Th{\o}rv{\"a}ld Circle}
%   \city{Hekla}
%   \country{Iceland}}
% \email{larst@affiliation.org}

% \author{Valerie B\'eranger}
% \affiliation{%
%   \institution{Inria Paris-Rocquencourt}
%   \city{Rocquencourt}
%   \country{France}
% }

% \author{Aparna Patel}
% \affiliation{%
%  \institution{Rajiv Gandhi University}
%  \streetaddress{Rono-Hills}
%  \city{Doimukh}
%  \state{Arunachal Pradesh}
%  \country{India}}

% \author{Huifen Chan}
% \affiliation{%
%   \institution{Tsinghua University}
%   \streetaddress{30 Shuangqing Rd}
%   \city{Haidian Qu}
%   \state{Beijing Shi}
%   \country{China}}

% \author{Charles Palmer}
% \affiliation{%
%   \institution{Palmer Research Laboratories}
%   \streetaddress{8600 Datapoint Drive}
%   \city{San Antonio}
%   \state{Texas}
%   \country{USA}
%   \postcode{78229}}
% \email{cpalmer@prl.com}

% \author{John Smith}
% \affiliation{%
%   \institution{The Th{\o}rv{\"a}ld Group}
%   \streetaddress{1 Th{\o}rv{\"a}ld Circle}
%   \city{Hekla}
%   \country{Iceland}}
% \email{jsmith@affiliation.org}

% \author{Julius P. Kumquat}
% \affiliation{%
%   \institution{The Kumquat Consortium}
%   \city{New York}
%   \country{USA}}
% \email{jpkumquat@consortium.net}

%%
%% By default, the full list of authors will be used in the page
%% headers. Often, this list is too long, and will overlap
%% other information printed in the page headers. This command allows
%% the author to define a more concise list
%% of authors' names for this purpose.
\renewcommand{\shortauthors}{Li, et al.}
%%
%% The abstract is a short summary of the work to be presented in the
%% article.
\begin{abstract}

% \textbf{150}:
% Users embody virtual avatars that mirror their physical movements in Virtual Reality. We propose to measure how the user's visual attention to the avatar impacts the probability of them noticing an offset applied on the avatar’s body movement with respect to their own motion. We conduct two user studies applying a dual-task paradigm to control and alter their visual attention and record noticing probability. Results confirm that more visual attention attracted away from the avatar leads to lower the noticeability of the offset. In addition, we identified the behavioral pattern of users' gaze can serve as an effective indicator of noticeability. Based on the findings, we implement a regression model that predicts noticeability taking the user's gaze data as input. We evaluate the extendability of the model on unseen visual attractions. Results show that the model achieves an MSE of 0.012 with new participants exposed to new visual attractions. 
While users could embody virtual avatars that mirror their physical movements in Virtual Reality, these avatars' motions can be redirected to enable novel interactions.
Excessive redirection, however, could break the user's sense of embodiment due to perceptual conflicts between vision and proprioception. 
While prior work focused on avatar-related factors influencing the noticeability of redirection, we investigate how the visual stimuli in the surrounding virtual environment affect user behavior and, in turn, the noticeability of redirection.
Given the wide variety of different types of visual stimuli and their tendency to elicit varying individual reactions, 
we propose to use users' gaze behavior as an indicator of their response to the stimuli and model the noticeability of redirection.
We conducted two user studies to collect users' gaze behavior and noticeability, investigating the relationship between them and identifying the most effective gaze behavior features for predicting noticeability. 
Based on the data, we developed a regression model that takes users' gaze behavior as input and outputs the noticeability of redirection. 
We then conducted an evaluation study to test our model on unseen visual stimuli, achieving an accuracy of 0.012 MSE. 
We further implemented an adaptive redirection technique and conducted a preliminary study to evaluate its effectiveness with complex visual stimuli in two applications. 
The results indicated that participants experienced less physical demanding and a stronger sense of body ownership when using our adaptive technique, demonstrating the potential of our model to support real-world use cases.
\end{abstract}

%%
%% The code below is generated by the tool at http://dl.acm.org/ccs.cfm.
%% Please copy and paste the code instead of the example below.
%%
\begin{CCSXML}
<ccs2012>
   <concept>
       <concept_id>10003120.10003121.10003128.10011755</concept_id>
       <concept_desc>Human-centered computing~Gestural input</concept_desc>
       <concept_significance>300</concept_significance>
       </concept>
   <concept>
       <concept_id>10003120.10003121.10003126</concept_id>
       <concept_desc>Human-centered computing~HCI theory, concepts and models</concept_desc>
       <concept_significance>300</concept_significance>
       </concept>
 </ccs2012>
\end{CCSXML}

\ccsdesc[300]{Human-centered computing~Gestural input}
\ccsdesc[300]{Human-centered computing~HCI theory, concepts and models}

%%
%% Keywords. The author(s) should pick words that accurately describe
%% the work being presented. Separate the keywords with commas.
\keywords{Virtual Reality, visual attention, noticeability, embodiment}

%% A "teaser" image appears between the author and affiliation
%% information and the body of the document, and typically spans the
%% page.
\begin{teaserfigure}
  \includegraphics[width=\textwidth]{figures/teaser/teaser.pdf}
  \caption{
    In this paper, we explored the impact of visual stimuli on the noticeability of redirection in Virtual Reality. 
    We developed a computational model that takes users' gaze behavior as input and predicts the noticeability of redirection under different visual stimuli.
    Using this model, we implemented an adaptive redirection technique, demonstrated in a boxing training scenario:
    Left: When the opponent approaches and attacks, visual stimuli are intense, making the redirection unnoticable.
    Middle: As the opponent retreats, visual stimuli decrease that causes the noticeability becoming higher during the interaction.
    Right: When the model detects the change in noticeability, the system dynamically adjusts the redirection magnitude, ensuring it remains unnoticed.
  }
  \label{fig:teaser}
\end{teaserfigure}

% \received{20 February 2007}
% \received[revised]{12 March 2009}
% \received[accepted]{5 June 2009}

%%
%% This command processes the author and affiliation and title
%% information and builds the first part of the formatted document.
\maketitle
\section{Introduction}
\label{section:introduction}

% redirection is unique and important in VR
Virtual Reality (VR) systems enable users to embody virtual avatars by mirroring their physical movements and aligning their perspective with virtual avatars' in real time. 
As the head-mounted displays (HMDs) block direct visual access to the physical world, users primarily rely on visual feedback from the virtual environment and integrate it with proprioceptive cues to control the avatar’s movements and interact within the VR space.
Since human perception is heavily influenced by visual input~\cite{gibson1933adaptation}, 
VR systems have the unique capability to control users' perception of the virtual environment and avatars by manipulating the visual information presented to them.
Leveraging this, various redirection techniques have been proposed to enable novel VR interactions, 
such as redirecting users' walking paths~\cite{razzaque2005redirected, suma2012impossible, steinicke2009estimation},
modifying reaching movements~\cite{gonzalez2022model, azmandian2016haptic, cheng2017sparse, feick2021visuo},
and conveying haptic information through visual feedback to create pseudo-haptic effects~\cite{samad2019pseudo, dominjon2005influence, lecuyer2009simulating}.
Such redirection techniques enable these interactions by manipulating the alignment between users' physical movements and their virtual avatar's actions.

% % what is hand/arm redirection, motivation of study arm-offset
% \change{\yj{i don't understand the purpose of this paragraph}
% These illusion-based techniques provide users with unique experiences in virtual environments that differ from the physical world yet maintain an immersive experience. 
% A key example is hand redirection, which shifts the virtual hand’s position away from the real hand as the user moves to enhance ergonomics during interaction~\cite{feuchtner2018ownershift, wentzel2020improving} and improve interaction performance~\cite{montano2017erg, poupyrev1996go}. 
% To increase the realism of virtual movements and strengthen the user’s sense of embodiment, hand redirection techniques often incorporate a complete virtual arm or full body alongside the redirected virtual hand, using inverse kinematics~\cite{hartfill2021analysis, ponton2024stretch} or adjustments to the virtual arm's movement as well~\cite{li2022modeling, feick2024impact}.
% }

% noticeability, motivation of predicting a probability, not a classification
However, these redirection techniques are most effective when the manipulation remains undetected~\cite{gonzalez2017model, li2022modeling}. 
If the redirection becomes too large, the user may not mitigate the conflict between the visual sensory input (redirected virtual movement) and their proprioception (actual physical movement), potentially leading to a loss of embodiment with the virtual avatar and making it difficult for the user to accurately control virtual movements to complete interaction tasks~\cite{li2022modeling, wentzel2020improving, feuchtner2018ownershift}. 
While proprioception is not absolute, users only have a general sense of their physical movements and the likelihood that they notice the redirection is probabilistic. 
This probability of detecting the redirection is referred to as \textbf{noticeability}~\cite{li2022modeling, zenner2024beyond, zenner2023detectability} and is typically estimated based on the frequency with which users detect the manipulation across multiple trials.

% version B
% Prior research has explored factors influencing the noticeability of redirected motion, including the redirection's magnitude~\cite{wentzel2020improving, poupyrev1996go}, direction~\cite{li2022modeling, feuchtner2018ownershift}, and the visual characteristics of the virtual avatar~\cite{ogawa2020effect, feick2024impact}.
% While these factors focus on the avatars, the surrounding virtual environment can also influence the users' behavior and in turn affect the noticeability of redirection.
% One such prominent external influence is through the visual channel - the users' visual attention is constantly distracted by complex visual effects and events in practical VR scenarios.
% Although some prior studies have explored how to leverage user blindness caused by visual distractions to redirect users' virtual hand~\cite{zenner2023detectability}, there remains a gap in understanding how to quantify the noticeability of redirection under visual distractions.

% visual stimuli and gaze behavior
Prior research has explored factors influencing the noticeability of redirected motion, including the redirection's magnitude~\cite{wentzel2020improving, poupyrev1996go}, direction~\cite{li2022modeling, feuchtner2018ownershift}, and the visual characteristics of the virtual avatar~\cite{ogawa2020effect, feick2024impact}.
While these factors focus on the avatars, the surrounding virtual environment can also influence the users' behavior and in turn affect the noticeability of redirection.
This, however, remains underexplored.
One such prominent external influence is through the visual channel - the users' visual attention is constantly distracted by complex visual effects and events in practical VR scenarios.
We thus want to investigate how \textbf{visual stimuli in the virtual environment} affect the noticeability of redirection.
With this, we hope to complement existing works that focus on avatars by incorporating environmental visual influences to enable more accurate control over the noticeability of redirected motions in practical VR scenarios.
% However, in realistic VR applications, the virtual environment often contains complex visual effects beyond the virtual avatar itself. 
% We argue that these visual effects can \textbf{distract users’ visual attention and thus affect the noticeability of redirection offsets}, while current research has yet taken into account.
% For instance, in a VR boxing scenario, a user’s visual attention is likely focused on their opponent rather than on their virtual body, leading to a lower noticeability of redirection offsets on their virtual movements. 
% Conversely, when reaching for an object in the center of their field of view, the user’s attention is more concentrated on the virtual hand’s movement and position to ensure successful interaction, resulting in a higher noticeability of offsets.

Since each visual event is a complex choreography of many underlying factors (type of visual effect, location, duration, etc.), it is extremely difficult to quantify or parameterize visual stimuli.
Furthermore, individuals respond differently to even the same visual events.
Prior neuroscience studies revealed that factors like age, gender, and personality can influence how quickly someone reacts to visual events~\cite{gillon2024responses, gale1997human}. 
Therefore, aiming to model visual stimuli in a way that is generalizable and applicable to different stimuli and users, we propose to use users' \textbf{gaze behavior} as an indicator of how they respond to visual stimuli.
In this paper, we used various gaze behaviors, including gaze location, saccades~\cite{krejtz2018eye}, fixations~\cite{perkhofer2019using}, and the Index of Pupil Activity (IPA)~\cite{duchowski2018index}.
These behaviors indicate both where users are looking and their cognitive activity, as looking at something does not necessarily mean they are attending to it.
Our goal is to investigate how these gaze behaviors stimulated by various visual stimuli relate to the noticeability of redirection.
With this, we contribute a model that allows designers and content creators to adjust the redirection in real-time responding to dynamic visual events in VR.

To achieve this, we conducted user studies to collect users' noticeability of redirection under various visual stimuli.
To simulate realistic VR scenarios, we adopted a dual-task design in which the participants performed redirected movements while monitoring the visual stimuli.
Specifically, participants' primary task was to report if they noticed an offset between the avatar's movement and their own, while their secondary task was to monitor and report the visual stimuli.
As realistic virtual environments often contain complex visual effects, we started with simple and controlled visual stimulus to manage the influencing factors.

% first user study, confirmation study
% collect data under no visual stimuli, different basic visual stimuli
We first conducted a confirmation study (N=16) to test whether applying visual stimuli (opacity-based) actually affects their noticeability of redirection. 
The results showed that participants were significantly less likely to detect the redirection when visual stimuli was presented $(F_{(1,15)}=5.90,~p=0.03)$.
Furthermore, by analyzing the collected gaze data, results revealed a correlation between the proposed gaze behaviors and the noticeability results $(r=-0.43)$, confirming that the gaze behaviors could be leveraged to compute the noticeability.

% data collection study
We then conducted a data collection study to obtain more accurate noticeability results through repeated measurements to better model the relationship between visual stimuli-triggered gaze behaviors and noticeability of redirection.
With the collected data, we analyzed various numerical features from the gaze behaviors to identify the most effective ones. 
We tested combinations of these features to determine the most effective one for predicting noticeability under visual stimuli.
Using the selected features, our regression model achieved a mean squared error (MSE) of 0.011 through leave-one-user-out cross-validation. 
Furthermore, we developed both a binary and a three-class classification model to categorize noticeability, which achieved an accuracy of 91.74\% and 85.62\%, respectively.

% evaluation study
To evaluate the generalizability of the regression model, we conducted an evaluation study (N=24) to test whether the model could accurately predict noticeability with new visual stimuli (color- and scale-based animations).
Specifically, we evaluated whether the model's predictions aligned with participants' responses under these unseen stimuli.
The results showed that our model accurately estimated the noticeability, achieving mean squared errors (MSE) of 0.014 and 0.012 for the color- and scale-based visual stimili, respectively, compared to participants' responses.
Since the tested visual stimuli data were not included in the training, the results suggested that the extracted gaze behavior features capture a generalizable pattern and can effectively indicate the corresponding impact on the noticeability of redirection.

% application
Based on our model, we implemented an adaptive redirection technique and demonstrated it through two applications: adaptive VR action game and opportunistic rendering.
We conducted a proof-of-concept user study (N=8) to compare our adaptive redirection technique with a static redirection, evaluating the usability and benefits of our adaptive redirection technique.
The results indicated that participants experienced less physical demand and stronger sense of embodiment and agency when using the adaptive redirection technique. 
These results demonstrated the effectiveness and usability of our model.

In summary, we make the following contributions.
% 
\begin{itemize}
    \item 
    We propose to use users' gaze behavior as a medium to quantify how visual stimuli influences the noticebility of redirection. 
    Through two user studies, we confirm that visual stimuli significantly influences noticeability and identify key gaze behavior features that are closely related to this impact.
    \item 
    We build a regression model that takes the user's gaze behavioral data as input, then computes the noticeability of redirection.
    Through an evaluation study, we verify that our model can estimate the noticeability with new participants under unseen visual stimuli.
    These findings suggest that the extracted gaze behavior features effectively capture the influence of visual stimuli on noticeability and can generalize across different users and visual stimuli.
    \item 
    We develop an adaptive redirection technique based on our regression model and implement two applications with it.
    With a proof-of-concept study, we demonstrate the effectiveness and potential usability of our regression model on real-world use cases.

\end{itemize}

% \delete{
% Virtual Reality (VR) allows the user to embody a virtual avatar by mirroring their physical movements through the avatar.
% As the user's visual access to the physical world is blocked in tasks involving motion control, they heavily rely on the visual representation of the avatar's motions to guide their proprioception.
% Similar to real-world experiences, the user is able to resolve conflicts between different sensory inputs (e.g., vision and motor control) through multisensory integration, which is essential for mitigating the sensory noise that commonly arises.
% However, it also enables unique manipulations in VR, as the system can intentionally modify the avatar's movements in relation to the user's motions to achieve specific functional outcomes,
% for example, 
% % the manipulations on the avatar's movements can 
% enabling novel interaction techniques of redirected walking~\cite{razzaque2005redirected}, redirected reaching~\cite{gonzalez2022model}, and pseudo haptics~\cite{samad2019pseudo}.
% With small adjustments to the avatar's movements, the user can maintain their sense of embodiment, due to their ability to resolve the perceptual differences.
% % However, a large mismatch between the user and avatar's movements can result in the user losing their sense of embodiment, due to an inability to resolve the perceptual differences.
% }

% \delete{
% However, multisensory integration can break when the manipulation is so intense that the user is aware of the existence of the motion offset and no longer maintains the sense of embodiment.
% Prior research studied the intensity threshold of the offset applied on the avatar's hand, beyond which the embodiment will break~\cite{li2022modeling}. 
% Studies also investigated the user's sensitivity to the offsets over time~\cite{kohm2022sensitivity}.
% Based on the findings, we argue that one crucial factor that affects to what extent the user notices the offset (i.e., \textit{noticeability}) that remains under-explored is whether the user directs their visual attention towards or away from the virtual avatar.
% Related work (e.g., Mise-unseen~\cite{marwecki2019mise}) has showcased applications where adjustments in the environment can be made in an unnoticeable manner when they happen in the area out of the user's visual field.
% We hypothesize that directing the user's visual attention away from the avatar's body, while still partially keeping the avatar within the user's field-of-view, can reduce the noticeability of the offset.
% Therefore, we conduct two user studies and implement a regression model to systematically investigate this effect.
% }

% \delete{
% In the first user study (N = 16), we test whether drawing the user's visual attention away from their body impacts the possibility of them noticing an offset that we apply to their arm motion in VR.
% We adopt a dual-task design to enable the alteration of the user's visual attention and a yes/no paradigm to measure the noticeability of motion offset. 
% The primary task for the user is to perform an arm motion and report when they perceive an offset between the avatar's virtual arm and their real arm.
% In the secondary task, we randomly render a visual animation of a ball turning from transparent to red and becoming transparent again and ask them to monitor and report when it appears.
% We control the strength of the visual stimuli by changing the duration and location of the animation.
% % By changing the time duration and location of the visual animation, we control the strengths of attraction to the users.
% As a result, we found significant differences in the noticeability of the offsets $(F_{(1,15)}=5.90,~p=0.03)$ between conditions with and without visual stimuli.
% Based on further analysis, we also identified the behavioral patterns of the user's gaze (including pupil dilation, fixations, and saccades) to be correlated with the noticeability results $(r=-0.43)$ and they may potentially serve as indicators of noticeability.
% }

% \delete{
% To further investigate how visual attention influences the noticeability, we conduct a data collection study (N = 12) and build a regression model based on the data.
% The regression model is able to calculate the noticeability of the offset applied on the user's arm under various visual stimuli based on their gaze behaviors.
% Our leave-one-out cross-validation results show that the proposed method was able to achieve a mean-squared error (MSE) of 0.012 in the probability regression task.
% }

% \delete{
% To verify the feasibility and extendability of the regression model, we conduct an evaluation study where we test new visual animations based on adjustments on scale and color and invite 24 new participants to attend the study.
% Results show that the proposed method can accurately estimate the noticeability with an MSE of 0.014 and 0.012 in the conditions of the color- and scale-based visual effects.
% Since these animations were not included in the dataset that the regression model was built on, the study demonstrates that the gaze behavioral features we extracted from the data capture a generalizable pattern of the user's visual attention and can indicate the corresponding impact on the noticeability of the offset.
% }

% \delete{
% Finally, we demonstrate applications that can benefit from the noticeability prediction model, including adaptive motion offsets and opportunistic rendering, considering the user's visual attention. 
% We conclude with discussions of our work's limitations and future research directions.
% }

% \delete{
% In summary, we make the following contributions.
% }
% % 
% \begin{itemize}
%     \item 
%     \delete{
%     We quantify the effects of the user's visual attention directed away by stimuli on their noticeability of an offset applied to the avatar's arm motion with respect to the user's physical arm. 
%     Through two user studies, we identified gaze behavioral features that are indicative of the changes in noticeability.
%     }
%     \item 
%     \delete{We build a regression model that takes the user's gaze behavioral data and the offset applied to the arm motion as input, then computes the probability of the user noticing the offset.
%     Through an evaluation study, we verified that the model needs no information about the source attracting the user's visual attention and can be generalizable in different scenarios.
%     }
%     \item 
%     \delete{We demonstrate two applications that potentially benefit from the regression model, including adaptive motion offsets and opportunistic rendering.
%     }

% \end{itemize}

\begin{comment}
However, users will lose the sense of embodiment to the virtual avatars if they notice the offset between the virtual and physical movements.
To address this, researchers have been exploring the noticing threshold of offsets with various magnitudes and proposing various redirection techniques that maintain the sense of embodiment~\cite{}.

However, when users embody virtual avatars to explore virtual environments, they encounter various visual effects and content that can attract their attention~\cite{}.
During this, the user may notice an offset when he observes the virtual movement carefully while ignoring it when the virtual contents attract his attention from the movements.
Therefore, static offset thresholds are not appropriate in dynamic scenarios.

Past research has proposed dynamic mapping techniques that adapted to users' state, such as hand moving speed~\cite{frees2007prism} or ergonomically comfortable poses~\cite{montano2017erg}, but not considering the influence of virtual content.
More specifically, PRISM~\cite{frees2007prism} proposed adjusting the C/D ratio with a non-linear mapping according to users' hand moving speed, but it might not be optimal for various virtual scenarios.
While Erg-O~\cite{montano2017erg} redirected users' virtual hands according to the virtual target's relative position to reduce physical fatigue, neglecting the change of virtual environments. 

Therefore, how to design redirection techniques in various scenarios with different visual attractions remains unknown.
To address this, we investigate how visual attention affects the noticing probability of movement offsets.
Based on our experiments, we implement a computational model that automatically computes the noticing probability of offsets under certain visual attractions.
VR application designers and developers can easily leverage our model to design redirection techniques maintaining the sense of embodiment adapt to the user's visual attention.
We implement a dynamic redirection technique with our model and demonstrate that it effectively reduces the target reaching time without reducing the sense of embodiment compared to static redirection techniques.

% Need to be refined
This paper offers the following contributions.
\begin{itemize}
    \item We investigate how visual attractions affect the noticing probability of redirection offsets.
    \item We construct a computational model to predict the noticing probability of an offset with a given visual background.
    \item We implement a dynamic redirection technique adapting to the visual background. We evaluate the technique and develop three applications to demonstrate the benefits. 
\end{itemize}



First, we conducted a controlled experiment to understand how users perceived the movement offset while subjected to various distractions.
Since hand redirection is one of the most frequently used redirections in VR interactions, we focused on the dynamic arm movements and manually added angular offsets to the' elbow joint~\cite{li2022modeling, gonzalez2022model, zenner2019estimating}. 
We employed flashing spheres in the user's field of view as distractions to attract users' visual attention.
Participants were instructed to report the appearing location of the spheres while simultaneously performing the arm movements and reporting if they perceived an offset during the movement. 
(\zhipeng{Add the results of data collection. Analyze the influence of the distance between the gaze map and the offset.}
We measured the visual attraction's magnitude with the gaze distribution on it.
Results showed that stronger distractions made it harder for users to notice the offset.)
\zhipeng{Need to rewrite. Not sure to use gaze distribution or a metric obtained from the visual content.}
Secondly, we constructed a computational model to predict the noticing probability of offsets with given visual content.
We analyzed the data from the user studies to measure the influence of visual attractions on the noticing probability of offsets.
We built a statistical model to predict the offset's noticing probability with a given visual content.
Based on the model, we implement a dynamic redirection technique to adjust the redirection offset adapted to the user's current field of view.
We evaluated the technique in a target selection task compared to no hand redirection and static hand redirection.
\zhipeng{Add the results of the evaluation.}
Results showed that the dynamic hand redirection technique significantly reduced the target selection time with similar accuracy and a comparable sense of embodiment.
Finally, we implemented three applications to demonstrate the potential benefits of the visual attention adapted dynamic redirection technique.
\end{comment}

% This one modifies arm length, not redirection
% \citeauthor{mcintosh2020iteratively} proposed an adaptation method to iteratively change the virtual avatar arm's length based on the primary tasks' performance~\cite{mcintosh2020iteratively}.



% \zhipeng{TO ADD: what is redirection}
% Redirection enables novel interactions in Virtual Reality, including redirected walking, haptic redirection, and pseudo haptics by introducing an offset to users' movement.
% \zhipeng{TO ADD: extend this sentence}
% The price of this is that users' immersiveness and embodiment in VR can be compromised when they notice the offset and perceive the virtual movement not as theirs~\cite{}.
% \zhipeng{TO ADD: extend this sentence, elaborate how the virtual environment attracts users' attention}
% Meanwhile, the visual content in the virtual environment is abundant and consistently captures users' attention, making it harder to notice the offset~\cite{}.
% While previous studies explored the noticing threshold of the offsets and optimized the redirection techniques to maintain the sense of embodiment~\cite{}, the influence of visual content on the probability of perceiving offsets remains unknown.  
% Therefore, we propose to investigate how users perceive the redirection offset when they are facing various visual attractions.


% We conducted a user study to understand how users notice the shift with visual attractions.
% We used a color-changing ball to attract the user's attention while instructing users to perform different poses with their arms and observe it meanwhile.
% \zhipeng{(Which one should be the primary task? Observe the ball should be the primary one, but if the primary task is too simple, users might allocate more attention on the secondary task and this makes the secondary task primary.)}
% \zhipeng{(We need a good and reasonable dual-task design in which users care about both their pose and the visual content, at least in the evaluation study. And we need to be able to control the visual content's magnitude and saliency maybe?)}
% We controlled the shift magnitude and direction, the user's pose, the ball's size, and the color range.
% We set the ball's color-changing interval as the independent factor.
% We collect the user's response to each shift and the color-changing times.
% Based on the collected data, we constructed a statistical model to describe the influence of visual attraction on the noticing probability.
% \zhipeng{(Are we actually controlling the attention allocation? How do we measure the attracting effect? We need uniform metrics, otherwise it is also hard for others to use our knowledge.)}
% \zhipeng{(Try to use eye gaze? The eye gaze distribution in the last five seconds to decide the attention allocation? Basically constructing a model with eye gaze distribution and noticing probability. But the user's head is moving, so the eye gaze distribution is not aligned well with the current view.)}

% \zhipeng{Saliency and EMD}
% \zhipeng{Gaze is more than just a point: Rethinking visual attention
% analysis using peripheral vision-based gaze mapping}

% Evaluation study(ideal case): based on the visual content, adjusting the redirection magnitude dynamically.

% \zhipeng{(The risk is our model's effect is trivial.)}

% Applications:
% Playing Lego while watching demo videos, we can accelerate the reaching process of bricks, and forbid the redirection during the manipulation.

% Beat saber again: but not make a lot of sense? Difficult game has complicated visual effects, while allows larger shift, but do not need large shift with high difficulty



\section{Background and Related Work}\label{sec:related}

\paragraph{\textbf{Privacy of Human-Centered Systems}}
Ensuring privacy in human-centric ML-based systems presents inherent conflicts among service utility, cost, and personal and institutional privacy~\cite{sztipanovits2019science}. Without appropriate incentives for societal information sharing, we may face decision-making policies that are either overly restrictive or that compromise private information, leading to adverse selection~\cite{jin2016enabling}. Such compromises can result in privacy violations, exacerbating societal concerns regarding the impact of emerging technology trends in human-centric systems~\cite{mulligan2016privacy,fox2021exploring,goldfarb2012shifts}. Consequently, several studies have aimed to establish privacy guarantees that allow auditing and quantifying compromises to make these systems more acceptable~\cite{jagielski2020auditing, raji2020saving}. ML models in decision-making systems have also been shown to leak significant amounts of private information that requires auditing platforms~\cite{hamon2022bridging}. Various studies focused on privacy-preserving machine learning techniques targeting decision-making systems~\cite{abadi2016deep, cummings2019compatibility, taherisadr2023adaparl, taherisadr2024hilt}. Recognizing that perfect privacy is often unattainable, this paper examines privacy from an equity perspective. We investigate how to ensure a fair distribution of harm when privacy leaks occur, addressing the technical challenges alongside the ethical imperatives of equitable privacy protection.


\paragraph{\textbf{\acf{fl}}}
\ac{fl} is an approach in machine learning that enables the collaborative training of models across multiple devices or institutions without requiring data to be centralized. This decentralized setup is particularly beneficial in fields where data-sharing restrictions are enforced by privacy regulations, such as healthcare and finance. \ac{fl} allows organizations to derive insights from data distributed across various locations while adhering to legal constraints, including the General Data Protection Regulation (GDPR) \cite{BG_Survey2,BG_Survey1}.

One of the most widely adopted methods in \ac{fl} is \ac{fedavg}, which operates through iterative rounds of communication between a central server and participating clients to collaboratively train a shared model. During each communication round, the server sends the current global model to each client, which uses their locally stored data to perform optimization steps. These optimized models are subsequently sent back to the server, where they are aggregated to update the global model. The process repeats until the model converges. Known for its simplicity and effectiveness, \ac{fedavg} serves as the primary technique for coordinating model updates across distributed clients in our work. Additionally, we specifically employ horizontal federated learning, where data is distributed across entities with similar feature spaces but distinct user groups \cite{BG_HorizontalFL}.

\paragraph{\textbf{Privacy Risks in \ac{fl}}}
Privacy risks are a critical concern in \ac{fl}, as collaborative training on decentralized data can inadvertently expose sensitive information. A primary threat is the \ac{mia}, where adversaries determine whether specific data records were part of the model's training set \cite{shokri2017membership,BG_MIA}. Researchers have since demonstrated \ac{mia}'s effectiveness across various machine learning models, including \ac{fl}, showing, for example, that adversaries can infer if a specific location profile contributed to an FL model \cite{BG_MIA_1,BG_MIA_2}. However, while \ac{mia} identifies training members, it does not reveal the client that contributed the data. \ac{sia}, introduced in \cite{BG_SIA_2}, extends \ac{mia} by identifying which client owns a training record, thus posing significant security risks by exposing client-specific information in \ac{fl} settings.

The \ac{noniid} nature of data in federated learning presents additional privacy challenges, as variations in data distributions across clients heighten the risk of privacy leakage. When data distributions differ widely among clients, individual model updates become more distinguishable, potentially allowing attackers to infer sensitive information \cite{BG_NON_IID}. This distinctiveness in updates can make federated models more susceptible to inference attacks, such as \ac{mia} and \ac{sia}, as malicious actors may exploit these distributional differences to trace updates back to specific clients. This vulnerability is especially relevant in our work, as we use the \ac{har} dataset, which is inherently \ac{noniid} across clients, thus posing an increased risk for privacy leakage.




\paragraph{\textbf{Fairness in \ac{fl}}}
Fairness in \ac{fl} is crucial due to the varied data distributions among clients, which can lead to biased outcomes favoring certain groups \cite{BG_Fairness_2}. Achieving fairness involves balancing the global model's benefits across clients despite the decentralized nature of the data. Approaches include group fairness, ensuring performance equity across client groups, and performance distribution fairness, which focuses on fair accuracy distribution~\cite{selialia2024mitigating}. Additional types are selection fairness (equitable client participation), contribution fairness (rewards based on contributions), and expectation fairness (aligning performance with client expectations) \cite{BG_Fairness}. Achieving fairness in \ac{fl} across these various dimensions remains challenging due to the inherent heterogeneity of client data and environments. In response to this heterogeneity, personalization has emerged as a strategy to tailor models to individual clients, enhancing local performance~\cite{BG_Personalization,BG_Personalization_2, BG_FairnessPrivacy}.   

When considering fairness in FL, it is crucial to address the interplay with privacy. Specifically, ensuring an equitable distribution of privacy risks across clients is paramount, preventing any group from being disproportionately vulnerable to privacy leakage, particularly under attacks such as source inference attacks (SIAs).



\section{Methodology}

We utilized LLMs to tackle the ASQP task across 0-, 10-, 20-, 30-, 40-, and 50-shot settings on different datasets. The performance is compared to that achieved using a dedicated training set to fine-tune smaller pre-trained language models. Furthermore, we report performance results for the TASD task.

\subsection{Evaluation}

\subsubsection{Datasets}

\begin{table*}[h]
\centering
\resizebox{1.8\columnwidth}{!}{%
\begin{tabular}{lccccc}
\hline
\textbf{}                    & \textbf{Rest15} & \textbf{Rest16} & \textbf{FlightABSA} & \textbf{OATS Coursera} & \textbf{OATS Hotels} \\ \hline
\textbf{\# Train}             & 834             & 1,264           & 1,351               & 1,400               & 1,400                \\
\textbf{\# Test}              & 537             & 544             & 387               & 400                 & 400                  \\
\textbf{\# Dev}              & 209             & 316             & 192               & 200                 & 200                  \\ \hline
\textbf{\# Aspect Categories} & 13              & 13              & 13              & 28                  & 33                   \\
\textbf{Language} & en              & en              & en              & en                  & en                   \\
\textbf{Domain} & restaurant              & restaurant              & airline              & e-learning                  & hotel                   \\
\hline
\end{tabular}
}
\caption{Overview of all ASQP datasets considered for evaluation. The datasets cover a range of different numbers of considered aspect categories and domains. }
\label{tab:overview-datasets}
\end{table*}

Table \ref{tab:overview-datasets} presents an overview of the datasets used in this study, including Rest15 and Rest16, along with three additional datasets covering diverse domains.


\textbf{Rest15 \& Rest16:} ASQP annotations originate from \citet{zhang2021aspect} and the TASD annotations from \citet{wan2020target}. This ensured comparability with the performance scores reported in previous research.

\textbf{FlightABSA:} A novel dataset containing 1,930 sentences annotated for ASQP. Properties of the annotated dataset are provided in Appendix \ref{appendix:flightabsa}. 

\textbf{OATS Hotels \& OATS Coursera:} We utilized a subset of two corpora recently introduced by \citet{chebolu2024oats} comprising ASQP-annotated sentences from reviews on hotels and e-learning courses. A detailed description of the data preprocessing for the OATS datasets can be found in Appendix \ref{appendix:oats-dataset}.

For the TASD task, we removed the opinion terms from the quadruples in annotations from FlightABSA, OATS Coursera and OATS Hotels. Subsequently, any duplicate triplets (\textit{a}, \textit{c}, \textit{p}) that appeared twice in a sentence were discarded.

\subsubsection{Setting}

For evaluation, the test dataset was considered for all datasets. An LLM was prompted five times with different seeds (0 to 4) for each combination of ABSA task (ASQP and TASD), dataset and amount of random few-shot examples (0, 10, 20, 30, 40 or 50) taken from the training set in order to get five label predictions. For all seeds, the same few-shot examples were used; however, they were shuffled differently for each prompt execution. The average performance across all five runs is reported.

\subsubsection{Metrics}

As in previous works in the field of ABSA, we report the micro-averaged F1 score as well as precision and recall to assess the model's performance. The F1 score is the harmonic mean of precision and recall. Precision measures the proportion of correctly predicted positive instances out of all instances predicted as positive \cite[p.~67]{jurafsky2000speech}. Recall quantifies the proportion of correctly predicted positive instances out of all actual positive instances in the dataset \cite[p.~67]{jurafsky2000speech}.

%, and is computed as follows:

%\[
%F1 = 2 \times \frac{\text{Precision} \times \text{Recall}}{\text{Precision} + \text{Recall}}
%\]
Similar to \citet{zhang2021aspect}, a quad prediction was considered correct if all the predicted sentiment elements are exactly the same as the gold labels. Recognizing the potential interest in class-level performance metrics for subsequent research, we have shared the predicted labels for every evaluated setting in our GitHub repository, allowing detailed class-level analysis.

\begin{figure*}[!htbp]
    \centering
    \includegraphics[width=2.1\columnwidth]{material/prompt.pdf}
    \caption{The prompt includes both a task description and specification of the output format. The LLM is run with five different seeds and in the case of self-consistency prompting, the tuple that appears most often across the five predictions is incorporated into the final label.}

\end{figure*}
\label{figure:study-prompt}

\subsection{Large Language Models}

We employed Gemma-2-27B\footnote{google/gemma-2-27b: \url{https://ollama.com/library/gemma2:27b}} by Google, which comprises 27.2 billion parameters \citep{team2024gemma}. Ollama\footnote{ollama: \url{https://ollama.com}} was employed for inference, and the LLMs were loaded with 4-bit quantization. The model was chosen for its efficiency in terms of generated tokens per second, which is a critical factor given the extensive prompt execution requirements. Notably, our study required over 342,720 prompts to be executed, with many few-shot learning prompts encompassing over a thousand tokens. For larger models, such as Llama-3.3-70B \citet{dubey2024llama}, the required computational costs would have been hardly feasible with our resources. For comparison purposes, we also report performance for the smaller-sized LLM, Gemma-2-9B\footnote{google/gemma-2-9b: \url{https://ollama.com/library/gemma2:9b}}.

The experiments were conducted on two NVIDIA RTX A5000 GPUs equipped with 24 GB of VRAM each. The LLM's temperature parameter was set to 0.8 and generation was terminated upon encountering the closing square bracket character (\texttt{"]"}) signifying the ending of a predicted label.

\subsection{Prompt}

\subsubsection{Components}

We adopted the prompting framework introduced by \citet{gou2023mvp} with some modifications. The employed prompt is illustrated in Figure \ref{figure:study-prompt} and an example is provided in Appendix \ref{appendix:prompt-example}. The main components of the prompt include a list of explanation on all considered sentiment elements and the specification of the output format. 

Unlike \citet{gou2023mvp}, our prompt instructs the LLM to pay attention to case sensitivity when returning aspect and opinion terms. Hence, the identified phrases should appear in the predicted tuple as they do in the sentence, similar to all supervised approaches mentioned in the related work section. Therefore, in the prompt, we clearly stated that the exact phrases should appear in the predicted label. 

Since we execute each prompt with five different seeds, we also report the performance when employing the self-consistency prompting technique introduced by \citet{wang2022self}. The key idea is to select the most consistent answer from multiple prompt executions. We adapted the approach for ABSA by incorporating a tuple into the merged label if it appears in the majority of the predicted labels. As illustrated in Figure \ref{figure:study-prompt}, this corresponds to a tuple appearing in at least 3 out of 5 predicted labels.

\subsection{Output Validation}

Since LLMs such as Gemma-2-27B cannot be strictly constrained to a fixed output format, we programmatically validated the output of the LLM. For the predicted label, several criteria needed to be met for the generation to be considered valid:

\begin{itemize}
    \item \textbf{Format}: The output must be a list of one or more tuples consisting of strings (quadruples for ASQP, triplets for TASD).
    \item \textbf{Sentiment}: The sentiment must be either 'positive', 'negative' or 'neutral'.
    \item \textbf{Aspect category}: Only the categories considered for the respective dataset and thus being mentioned in the prompt should be predicted as a part of a tuple.
    \item \textbf{Aspect and opinion terms}: Both must appear in the given sentence as predicted.
\end{itemize}

If any of the specified criteria for reasoning or label validation is not met, a regeneration attempt was triggered. If the predicted label was still invalid after 10 attempts, an empty label (\texttt{[]}) was considered as the predicted label.

\subsection{Baseline Model}

We compared the previously mentioned zero- and few-shot conditions against three SOTA baseline approaches, which are, the three best-performing methods for ASQP and TASD on the Rest15 and Rest16 datasets: Paraphrase \citep{zhang2021aspect}, DLO \citep{hu2022improving} and MVP \citep{gou2023mvp}.

\begin{description}
    \item[Paraphrase \citep{zhang2021aspect}:] \textit{Paraphrase} is used to linearize sentiment quads into a natural language sequence to construct the input target pair.
    \item[DLO \citep{hu2022improving}:] \textit{Dataset-level order} is a method designed for ASQP that leverages the order-free property of quadruplets. It identifies and utilizes optimal template orders through entropy minimization and combines multiple effective templates for data augmentation.
    \item[MVP \citep{gou2023mvp}:] \textit{Multi-view-Prompting} introduces element order prompts. The language model is guided to generate multiple sentiment tuples, with a different element order each, and then selects the most reasonable tuples by a voting mechanism. This method is highly resource-intensive, as multiple input-output pairs are created for each example in the train set, each comprising different sentiment element positions.
\end{description}

For all three approaches, we conducted training using the entire dataset and performed training with only 10, 20, 30, 40, or 50 training examples equally to the ones employed for the few-shot learning conditions. Training was conducted using five different random seeds (0 to 4). Moreover, to facilitate comparisons across datasets, we trained models using 800 training examples, as this represents the largest multiple of 100 examples available for all train sets (900 training examples are not available for Rest15). The results obtained using the full training sets of Rest15 and Rest16 were extracted from the works of \citet{zhang2021aspect}, \citet{hu2022improving}, and \citet{gou2023mvp}.

For all methods, we used the hyperparameter configurations used by \citet{zhang2021aspect}, \citet{hu2022improving} and \citet{gou2023mvp}. The only exception was the 10-shot condition, where batch size was set to 8 instead of 16, as the limited number of examples (10) could not form a batch of 16 examples.



\section{Confirmation Study}
\label{section:formativestudy}

Although it seems evident that adding additional visual stimuli may distract users and influence the noticeability of redirection, we conducted a confirmation study to validate this hypothesis and assess the effectiveness of our dual-task design.

% \delete{
% To investigate whether and how visual attention will influence the motion offset's noticeability, we leveraged visual stimuli to induce gaze saccades and visual attention shifts in users while users were performing hand reaching tasks with redirected motions.
% Specifically, we applied a dual-task paradigm where 16 participants performed an arm motion and decided whether there was a significant difference between the virtual and the physical motion (\textbf{primary task}), while monitoring and reporting when a red ball gradually changed from full transparency to full opacity at randomized locations in the field of view (\textbf{secondary task}).
% We controlled the intensity of the visual stimuli by altering the duration and the location of the animations.
% With the stimuli, we employed a yes/no paradigm in psychophysics~\cite{leek2001adaptive} to record the participants' responses and estimate the noticeability by the ratio of correct responses to the number of trials referring to previous research~\cite{li2022modeling}.
% }

% \delete{
% Though recent works proved that the noticeability of virtual hand offsets could be influenced by eye gaze saccades\cite{zenner2023detectability}, we aimed to further quantify the relationship between users' gaze behavioral patterns and noticeability.
% Therefore, we leveraged visual stimuli to induce gaze saccades and other gaze behaviors in users.
% We investigated how directing the user's visual attention away from their body at different strengths influences the probability of them noticing an offset applied to the avatar's motion with respect to their real motion (i.e., noticeability).
% We adopted the Method of Constant Stimuli~\cite{simpson1988method} to obtain the detection threshold (DT) and noticing probability when presenting the stimuli, including the motion offset and visual animations.
% We controlled the intensity of the stimuli by altering the strength of motion offsets, the duration and the location of the animations.
% To present the stimuli, we applied a dual-task paradigm where 16 participants performed an guided arm motion and decided whether there was a significant difference between the virtual and the physical motion (\textbf{primary task}), while monitoring and reporting when a red ball gradually changed from full transparency to full opacity at randomized locations in the field of view (\textbf{secondary task}).
% With the stimuli, we adopted a two-alternative forced-choice (2AFC) procedure to record the participants' responses and estimate the noticeability by the ratio of correct responses to the number of trials referring to previous research~\cite{li2022modeling}.
% }

% By referring to previous research~\cite{li2022modeling}, we estimate the noticeability by the frequency of participants noticing the offsets.


%invited 16 participants to decide if they noticed the offset under visual attractions with different intensities and positions.

% The primary objective of Study 1 was to investigate the potential impact of visual attention on users' perceptual likelihood of noticing limb offsets. We introduced offsets in varying magnitudes and directions to randomized poses, and combined with diverse levels of visual focus tasks.

\subsection{Design}

%We applied a dual-task paradigm in this study consisting of a primary limb movement task and a secondary visual attraction task. 
%In the primary task, participants executed prescribed movements using their left arm with the VR headset on, while concurrently evaluating virtual movements to determine their consistency with corresponding physical actions. 
% Subsequently, participants reported the immediate appearance position as pertaining to either the left or right half.

%We employed a factorial experimental design.

We employed a factorial study design to manage both independent and control variables.

% \delete{
% To investigate the noticeability under various stimuli, we altered the strength of motion offsets and visual animations (i.e., red balls rendered at different locations in the virtual environment).
% Since we aimed to investigate if visual attention influences the noticeability of the motion offset, we list the intensity of visual stimuli as the independent variable and the noticeability as the dependent variable.
% }

% independent factors
\subsubsection{Independent variables}
In this study, we aimed to investigate whether applying visual stimuli affects noticeability. 
Therefore, our initial independent variable was the presence or absence of visual stimuli. 
To further explore the impact of various visual stimuli, we extended the independent variable to the intensity of visual stimuli, ranging from none to high.
We manipulated intensity by adjusting the duration and placement of virtual animations, following previous studies~\cite{gutwin2017peripheral, li2024predicting}. 
Through a pilot study, we identified three levels of duration: Short (0.2 sec), Medium (1 sec), and Long (2 sec).
For placement, we defined three layout configurations: Sparse (stimuli appear only in the corner areas), Median (stimuli appear in both the corner and peripheral areas), and Dense (stimuli appear throughout the entire field of view), as shown in~\autoref{figure:formalapparatus}.
In each layout, we randomly picked one candidate to animate the visual stimuli.
Additionally, we included a baseline condition with no visual stimuli.
The order of these conditions was randomized.

% \delete{
% To control the intensity of the visual stimuli, we altered the duration and placement of the red ball, since previous works showed that animations with a shorter duration or appearing in the central area are more noticeable than subtle and peripheral ones~\cite{gutwin2017peripheral, li2024predicting}.
% Through a pilot study, we chose three levels for the duration (Short: 0.2 sec, Medium: 1 sec, Long: 2 sec) that were reported to direct participants' attention to different degrees of success.
% Regarding the placement, we defined three layouts (Sparse, Median, Dense) where the stimuli appears only in the corner area, in both corner and peripheral areas, and in all areas of the field of view, respectively. 
% To analyze whether at all the visual stimuli affect the noticeability of offsets, we include a baseline condition without any visual stimuli.
% }

% control factors
\subsubsection{Control variables}
We varied the magnitude and direction of the redirection as control variables. 
Based on the results from related research~\cite{li2022modeling}, we set the redirection magnitude from 0 to 30 degrees with an interval of 5 degrees, which covers the from being unnoticeable (no redirection) to easily noticeable.
We also varied the direction of redirection, sampling both horizontal and vertical directions.
As a result, each participant completed $(3~durations~\times~3~layouts~+~1~baseline)$  $\times (7$ redirection magnitudes $\times 2$ redirection direction - 1) $= 130$ trials in total.
The order of all redirection magnitudes and directions was randomized.

% \delete{
% We tested 25 different ending poses sampled from CMU MoCap dataset~\cite{CMUMocap} to enhance the generalizability of the collected data.
% To focus on the impact of visual stimuli on the noticeability of different offsets, we fixed the magnitude and direction of the offset as control variables.
% Based on findings of related research~\cite{li2022modeling},  we set the offset magnitude from -30 to 30 degrees with a step of 5 degrees, which we believe cover the offset range from being unnoticeable (no offset) to easily noticeable.
% To fix the offset direction, we randomly picked horizontal or vertical direction for each trial, while we ensured the number of trials in each direction to be the same.
% As a result, each participant completed $(3~durations~\times~3~layouts~+~1~baseline)$  $\times 13$ offset strengths $\times 1$ offset directions $= 130$ trials in total.
% }

\subsubsection{Dependent variables}
The noticeability of redirection was recorded as the primary dependent variable in this study and was estimated with the proportion of positive responses across all trials for each condition where redirection was applied.
Additionally, we captured participants' gaze behavior data with the HMD's eye tracker.

% \delete{
% To measure the noticeability of the motion offsets under each condition, we adopted a yes/no paradigm~\cite{leek2001adaptive}.
% Participants pressed the buttons on the VR controller to report a binary response if they observed a stimulus.
% Thus, we leveraged the proportion of the correct responses to the motion offset (hit rate) as an estimation of the noticeability.
% }

\begin{figure}[t]
    \centering
    \includegraphics[width=0.9\columnwidth]{figures/formativeStudy/apparatus.png}
    \caption{The apparatus of the formative study. 
    Wearing a headset, the participant wears three motion trackers to track their arm pose and sit on a comfortable chair. 
    While the virtual avatar mirrors the arm movement of the participant, the participant observes the virtual avatar's movement from a first-person point of view and follows the semi-transparent checkpoint pose to reach the semi-transparent target pose.
    As the secondary task, a virtual animation will start with different durations and locations.
    The right figure illustrates the possible locations of the red ball, named Sparse, Median, and Dense, accordingly.}
    \label{figure:formalapparatus}
\end{figure}


% \subsection{Task}

% \delete{
% To present the stimuli consisting of motion offsets and visual animations, we adopted a dual-task paradigm in this study.
% }

% \subsubsection{Primary task}
% \delete{
% Our primary task required participants to perform a guided arm motion in VR.
% The motion task is defined by the starting arm pose and the ending arm pose. 
% We use a fixed starting arm pose of naturally resting beside the body and sample 25 arm gestures as the ending poses from CMU MoCap database~\cite{CMUMocap}.
% While the participant performed each motion task, we added an offset on the avatar's elbow joint with respect to the participant's real elbow joint.
% Similar to previous research on offset noticeability~\cite{li2022modeling}, the offset is set to be an angular rotation applied to the joint.
% The offset was applied dynamically, beginning with no offset at the initial pose (pointing to the ground) and reaching the maximum offset at the ending pose. 
% The offset during the intermediate motion was calculated based on the relative angular distance from the starting pose and was adjusted linearly throughout the movement.
% Additionally, an intermediate checkpoint pose was added to ensure that participants performed the motion in a consistent manner.
% We adjusted the strength of the offset by altering the maximum offset applied at the ending pose.
% All motion tasks were performed with the left arm.
% }

%Participants were asked to navigate their movements and reach the checkpoint poses before achieving the target pose. 
%Once they achieved the target pose, participants were asked to retrace their movement and return to the initial pose.
%Then we asked participants to decide if they detected any offset during the movement and we measured the noticeability of the offset during the movement with the proportion of times that participants were able to detect it.


%To circumvent the potential detection of the offset arising from the uniform starting pose, a gradual introduction of the angular offset was implemented, progressively incrementing it from zero to the desired target offset magnitude, while participants transitioned from the initial pose to the target posture. Subsequently, once the target pose was attained, participants were instructed to retrace their movement to revert to the initial pointing downward posture. Subsequent to these motion sequences, participants were prompted to ascertain whether they perceived any offset throughout the course of their movements.

%To simulate realistic interaction scenarios, we asked participants to perform dynamic movements from a uniform initial posture of pointing downward to control the moving trajectory.
%To avoid participants noticing the offset from the uniform initial pose, we implemented a gradually increasing process in which we increased the offset from zero to the target magnitude while participants started from the initial pose and moved their left arm to the target pose.
%To control the trajectory of participants' movements, we displayed two checkpoint poses by interpolating the initial pose and the target pose.


% \subsubsection{Secondary task}
% \delete{
% In parallel to performing the primary task, we asked participants to monitor visual stimuli that appeared at random times and locations in their field of view.
% We designed the visual stimuli as a red ball that transitions from full transparency to full opacity and resets to full transparency.
% The red ball was initially registered in a fixed position in the participant's field of view following their head movements, with a fixed depth of 0.5 m and a radius of 0.02 m.
% We instructed participants to report whether the red ball appears in the left or right side of their field of view by pressing the corresponding part of the touchpad on the hand-held controller as soon as they notice it.
% An interval randomly selected from 1-3 seconds is set between every two visual stimuli to ensure that the participant cannot predict when they arise.
% }

%Visual attraction design.

%Participants verbally reported existence and left or right.

%In the secondary task, participants observed visual attractions that appeared randomly within their field of view at unpredictable intervals (1-3 seconds) and reported as soon as they noticed. 




% The target poses were sampled from the domain of human upper limb movement and chosen randomly to serve as the control variable. 
% Independent variables encompassed the magnitude and direction of offsets, in addition to the temporal duration and spatial layout of the visual attractions. 
% The dependent variables included noticeability of offset under conditions of visual focus, measured by the proportion of times that users were able to detect the offset. Furthermore, the reaction time (RT) and accuracy of the visual focus task were measured.

% In each experimental session, one out of ten combinations entailing three levels of visual stimulus duration, three types of visual layouts, and an additional control group without the visual attraction was employed. Offsets were implemented on virtual movements, ranging from 0 to 30 degrees in increments of 5 degrees, alongside randomly assigned plus-minus directions on the x and y axes, which were uniformly distributed among participants. This led to a total of 13 target pose configurations. As such, each participant evaluated $10 visual attraction settings \times 13 target pose setting = 130 tasks$.


% \emph{Primary task:} perform given movements while observing virtual movements and decide if the virtual movement is the same as the physical movement.

% \emph{Secondary task:} observe the visual attractions which randomly appeared in the field of view and report the appearing position as soon as possible.

% \emph{Independent variables:} visual attraction duration and visual attraction layout

% \emph{Control variables:} offset magnitude, offset direction, target movements

% \emph{Metrics:} noticeability of offsets, visual attraction response accuracy and visual attraction response time


\subsection{Participants \& Apparatus}
The participants (N = 16) were recruited through an online questionnaire from a local university. 
The participants (7 females, 9 males) had an average age of 21.25 years ($SD = 1.71$). 
All were trichromats and right-handed. 
Prior to the experiment, participants self-evaluated on their familiarity with VR, reporting an average score of $3.75\ (SD=0.75) $on a 7-point Likert scale (1 - not at all familiar, 4 - neutral, 7 - very familiar).

% \subsection{Apparatus}
We implemented the experimental application in VR with a HTC Vive pro headset in Unity 2019, powered by an Intel Core i7 CPU and an NVIDIA GeForce RTX 3080 GPU. 
Throughout the experimental sessions, participants were seated and equipped with three Vive Trackers affixed to their left shoulder, elbow, and waist using nylon straps.
Based on data given by the tracker, we reconstructed the left arm movement on a virtual humanoid avatar from the Microsoft RocketBox avatar library~\cite{gonzalez2020rocketbox} with the user’s viewpoint coinciding with the avatar’s (as shown in \autoref{figure:formalapparatus}).
All gaze data was recorded with the HTC Viveo pro built-in gaze tracker.
All statistical analyses were conducted with SPSS 26.0.

% Results: Layout: F(2,147)=33.87, p<0.001. Duration: F(2,147)=43.40, p<0.001.

\subsection{Procedure}


To avoid bias from the participants knowing that we were intentionally introducing redirection, we introduced the purpose of the study as an evaluation of a motion capture and reconstruction technique and clarified the real purpose to participants after the study.
Participants were first provided with a walk-through of the platform.
Then, participants were provided with a warm-up session to ensure that they were familiar with the primary and secondary tasks.
After that, each participant completed 10 sessions ($3~durations~\times~3~layouts~+~1~baseline$) of experiments.
They took 2-minute breaks after every two sessions to reduce fatigue.
We recorded the participant's behavioral data, including the position and orientation of hand, elbow, shoulder, gaze, and pupil dilation, at a rate of 60 Hz.
The study lasted around 40 minutes and each participant was compensated with 15 US dollars.
%For the primary task, participants were asked to start from the initial pose and reach the target pose, bypassing two checkpoint poses.
%After retracing the trajectory, they were asked to report if they recognized the virtual arm movement as the same as their physical movement.
%During the primary task, visual attractions were displayed at a random time interval with designed intensity and position.
%Participants were asked to report the visual attraction's position with the touchpad on the controller as soon as they noticed it.


% Participants were first instructed to put on the VR headset and Vive trackers. Following a warm-up session, participants completed ten experimental sessions in a predetermined sequence, with a short break after every five sessions. During the experiment, participants executed left-arm movements while using the Vive controller with right hand to report their perception of offset and the position of visual attractions.

\begin{figure}[t]
    \centering
    \includegraphics[width=0.9\columnwidth]{figures/formativeStudy/noticeability.png}
    \caption{Noticeability results of the formative study in every condition.
    The error bars represent the standard errors.}
\end{figure}

\subsection{Results}
\label{section:study1results}

We first conducted Shapiro-Wilk tests on the noticeability results which showed that all 10 conditions followed a normal distribution, requiring no correction.
We then conducted Repeated-Measures ANOVA with Bonferroni-corrected post hoc T-tests on the results.
The average response time to visual stimuli was 327 ms (SD = 168 ms), indicating that participants were actively engaged in both tasks.

\begin{figure*}
    \centering
    \includegraphics[width=0.9\linewidth]{figures/formativeStudy/psychometric3.png}
    \caption{Psychometric functions of the noticeability in each condition.}
    \label{fig:psychometric}
\end{figure*}

\textit{With visual stimuli, participants noticed the redirection significantly less than without visual stimuli.}
We conducted a one-factor ANOVA between the baseline and the averaged nine other conditions.
Our statistical analysis showed that participants detected the redirection significantly $(F_{(1,15)}=5.90,~p=0.03)$ less frequently when they were exposed to the visual stimuli $(M=0.43,~SD=0.13)$ compared to none visual stimuli $(M=0.51,~SD=0.08)$.
These results confirm that the noticeability of redirection was reduced when visual stimuli were presented and further validate the design of our study.


We then evaluated whether participants' physical movements were effectively redirected under both noticed and unnoticed conditions. 
We divided all trials into two categories based on the participants' response to the redirection (\textit{noticed} or \textit{unnoticed}).
We then analyzed the lengths of participants' virtual and physical hand trajectories within these two categories.
The \textbf{physical trajectory length} refers to the ratio of the participant's physical hand movement trajectory length to the distance between the starting and ending pose.
Similarly, the \textbf{virtual trajectory length} refers to the participants' virtual hand movement trajectory length to the distance between the starting and ending pose.
In the unnoticed condition, participants' physical movement trajectories were significantly shorter than their virtual ones: \textit{Physical} $(AVG = 1.16,~SD = 0.04)$ and \textit{Virtual} $(AVG = 1.24,~SD = 0.05,~t(15) = 3.64,~p < 0.05)$.
Similarly, in the noticed condition, participants' physical movement trajectories were still significantly shorter than their virtual ones: \textit{Physical} $(AVG = 1.17,~SD = 0.04)$ and \textit{Virtual} $(AVG = 1.33,~SD = 0.04,~t(15) = 4.88,~p < 0.01)$.
These results suggest that participants' physical movements were successfully redirected,regardless of whether they noticed the redirection.

% \delete{
% \textbf{Less visual attention directed towards the avatar leads to lower noticeability of the applied offset.}
% }
% \delete{
% \textbf{Visual attention shifts away from the avatar lead to lower noticeability of the applied offset.}
% We conducted a one-factor ANOVA between the baseline and the averaged nine other conditions.
% Our statistical analysis showed that participants detected the offset significantly $(F_{(1,15)}=5.90,~p=0.03)$ less frequently when they were exposed to the visual stimuli $(M=0.43,~SD=0.13)$ compared to none visual stimuli $(M=0.51,~SD=0.08)$.
% This indicates that our visual stimuli successfully occupied participants' visual attention and made them less sensitive to the motion offset.
% We then conducted a two-factor ANOVA to explore the impact of the stimuli's duration and position on the noticeability.
% Results showed that both the stimuli's duration $(F_{(2,30)}=24.02,~p<0.001)$ and position $(F_{(2,30)}=27.83,~p<0.001)$ had a significant influence on the noticeability and there was not an interaction effect between these two factors $(F_{(4,60)}=1.70,~p=0.20)$.
% Therefore, we conducted post hoc tests on duration and layout separately.
% Post hoc results showed that the noticeability with \textit{short} duration $(M=0.35,~SD=0.10)$ was significantly lower than \textit{medium} $(M=0.46,~SD=0.07),~(t(15)=-3.34,~p<0.001)$, and \textit{long} $(M=0.50,~SD=0.07),~(t(15)=-4.39,~p<0.001)$ duration stimuli.
% The noticeability with \textit{sparse} layout $(M=0.37,~SD=0.09)$ was significantly lower than \textit{medium} $(M=0.43,~SD=0.07),~(t(15)=-3.07,~p<0.01)$, and \textit{dense} $(M=0.49,~SD=0.09),~(t(15)=-4.01,~p<0.001)$ duration stimuli.
% This indicates that both the visual stimuli' duration and position could affect the noticeability of offset significantly.
% We further plotted the psychometric functions of each condition to demonstrate how visual stimuli impact the noticeability behavior of users, which can be helpful for designers to decide the appropriate offsets to apply in different scenarios.
% To be noted, the noticeability of motion offsets under visual stimuli with the sparse layout and short duration did not achieve 100\%, due to participants' visual attention being strongly directed towards the visual stimuli.
% }

\begin{figure*}
    \centering
    \includegraphics[width=0.9\linewidth]{figures/formativeStudy/correlation_v3.png}
    \caption{The regression results between gaze distance, saccade frequency, fixation frequency, IPA, and noticeability are presented. 
    The correlation coefficients are indicated in the top right corner.}
    \label{figure:correlation}
\end{figure*}

We further analyzed the gaze behaviors (gaze location, saccades, fixation, and pupil activity) based on the recorded gaze data. 
Specifically, we computed the average gaze distance relative to the virtual hand, saccade and fixation frequencies, and Index of Pupil Activity (IPA) in each condition. 
We then calculated the correlation coefficients between these gaze behaviors and the noticeability results.
As shown in~\autoref{figure:correlation}, the results revealed significant correlations: gaze distance ($r = -0.26, p = 0.001$), saccade ($r = -0.43, p < 0.001$), fixation ($r = -0.27, p < 0.001$), and IPA ($r = -0.26, p = 0.001$). 
These results suggest that all the examined gaze behaviors exhibit a negative relationship with noticeability, with gaze saccades showing a stronger effect compared to the others. 
This may be because rapid saccadic movements often indicate high cognitive load or attentional shifts, making them more directly and negatively associated with the noticing of redirection. 
In contrast, fixation frequency does not inherently reflect cognitive load, although fixation duration might serve as a useful indicator.
Regarding gaze distance to the virtual hand, participants likely shifted their gaze between the virtual body and the stimuli, making it less consistently related to noticeability. 
For IPA, its design as a long-term estimator of cognitive load may render it less sensitive to subtle or transient changes in cognitive load caused by visual stimuli.
Overall, while each of these gaze behaviors responds to visual stimuli in distinct ways, they all show promise as predictors of noticeability.


% \delete{
% Since recent papers showed that the gaze saccade could affect the noticeability of motion offsets~\cite{zenner2023detectability}, we further analyzed other eye gaze behavior to investigate if participants more frequently switched their visual focus or had a higher cognitive load in the condition with stronger visual stimuli.
% To this end, we calculated participants' eye saccades and fixations (as implemented in \cite{pymovements}) from the gaze position data and Index of Pupillary Activity (IPA)~\cite{duchowski2018index} as the estimated cognitive load.
% By calculating the correlation coefficients between the average of these metrics and noticeability, we found that IPA $(r=-0.26)$, saccade $(r=-0.43)$ and fixation frequency $(r=-0.27)$ were correlated with the noticeability.
% This relationship suggests that participants were less likely to notice the offset when they had a high cognitive load or they had more frequent shifts in their visual focus.
% These results motivated us to further explore the relationship between users' gaze behavioral data and noticeability.
% Since there is no ground truth for visual attention, we aim to build an extendable model which takes users' gaze behavioral patterns as input and outputs the noticeability of applied offsets.
% }
% \delete{
% Our aim was to build an extendable model to predict the noticeability of the applied offsets without knowing the specific visual stimulus that leads to the identified gaze behavioral patterns.
% }


\section{Data Collection}
\label{section:datacollection}
After confirming that visual stimuli influence the noticeability of redirection, we conducted another user study using the same dual-task design to gather more data for developing a prediction model. 
This model aims to estimate noticeability based on users' gaze behavior.

% \delete{
% Next, we conducted a study to collect necessary data for constructing a model of the noticeability of arm motion offsets, taking visual attention into account.
% We built a dataset containing the participants' behavioral data and the corresponding noticeability results of arm motion offsets.
% The offsets had a constant strength of 20 degrees while participants were exposed to an opacity-based visual effect with different levels of intensities and displayed in different areas in the field of view.
% }

\subsection{Design}

To collect the noticeability results more accurately, we measured the noticeability of each redirection magnitude repeatedly for each participant, and tested on less redirection magnitude levels.
In future work, we consider it important to extend the experiments to include a wider range of redirection magnitude.s 
%It is worth noting that previous research has already explored the impact of offset strengths on noticeability~\cite{li2022modeling}.
%Additionally, since previous works had explored the influence of offset strength on noticeability, we conducted the data collection study with a fixed offset strength to predict the influence of visual attractions.
As per prior work that investigated the impact of redireciton magnitude on noticeability, we chose 20 degrees as the tested magnitude, as the reported noticeability rate was around 75\% without visual stimuli in~\cite{li2022modeling}. 
The relatively high rate allowed us to detect the impact of visual stimuli effectively.
% We verified the reported rate through a characterizing study. 
We randomly selected horizontal or vertical as the redirection direction.
%We envision that constructing a prediction model which considers the offset strength and the visual attractions simultaneously as an important future work.

We adopted the same dual-task design with a yes/no paradigm detailed in \autoref{section:methodology}.
As our formative study results showed, the visual stimuli with medium duration ($(M=0.46,~SD=0.07)$) did not yield statistically significant differences in terms of noticeability compared to the stimuli with long duration ($(M=0.50,~SD=0.07),~(t(15)=-0.39,~p>0.05)$ ).
Therefore, we excluded the medium condition and only selected the short (0.2s) and long duration (2s) to control the intensity in this study.
To further control the position of visual stimuli within participants' visual field, we displayed the stimuli at central vision (5 degrees from the central point of vision), near peripheral vision (30 degrees), and mid peripheral vision (60 degrees), illustrated in \autoref{figure:datacollectionlayout} and as in prior research~\cite{grosvenor2007primary, gutwin2017peripheral}.
Therefore, we had $2~\times~3~=~6$ conditions, named as \textbf{CS} (central layout with short duration), \textbf{CL} (central layout with long duration), \textbf{NS} (near peripheral layout with short duration), \textbf{NL} (near peripheral layout with long duration), \textbf{MS} (mid peripheral layout with short duration), and \textbf{ML} (mid peripheral layout with long duration), and we used a Latin square to counterbalance them.
Each participant completed $(2~durations~\times~3~layouts)~\times~24$ measurements $=~144$ trials in total. 

\begin{figure}[!htbp]
    \centering
    \includegraphics[width=0.9\columnwidth]{figures/dataCollection/datacollectionlayout.png}
    \caption{The possible locations of the visual stimuli in the data collection study.
    The locations are divided into three conditions: Central (5 degrees), Near Peripheral (30 degrees), and Mid Peripheral (60 degrees), based on the angular distance to the user's head direction.}
    \label{figure:datacollectionlayout}
\end{figure}


\subsection{Apparatus \& Procedure}
The apparatus and procedure were almost identical to those of our formative study (\autoref{section:formativestudy}). 
We recorded the position and orientation of hand, elbow, shoulder, gaze, and pupil dilation with a sample rate of 60 Hz.
All gaze data was recorded with the HTC Viveo pro built-in gaze tracker.
After the warm-up session, participants took 2-minute breaks after every two sessions to reduce fatigue.
The study lasted around 40 minutes and each participant was compensated with \$15 USD.

\subsection{Participants}
We recruited 12 participants (5 females, 7 males) from a local university.
The participants had an average age of 22.91 years $(SD=1.90)$. All were trichromats and right-handed. 
% All participants with an average age of 22.91 $(SD=1.90)$ were trichromats and right-handed. 
Participants' self-reported their familiarity with VR at an average of 3.17 $(SD=1.27)$ on a 7-point Likert scale from 1 (not at all familiar) to 7 (very familiar).

\subsection{Summary of data statistics}

\begin{figure}[!htbp]
    \centering
    \includegraphics[width=0.9\columnwidth]{figures/dataCollection/noticeability_collection.png}
    \caption{Noticeability results of the data collection study in each condition. The error bars represent the standard errors.}
    \label{figure:datacollectionresult}
\end{figure}

In total, we collected 1728 responses.
To estimate the noticeability, we calculated the ratio of trials in which participants reported noticing the redirection to the total number of trials for each session and participant.
As shown in \autoref{figure:datacollectionresult}, the noticeability result differed across the visual stimuli's duration and layout.
The noticeability results ranged from 16.7\% to 79.2\% with an average of 50.1\% and a standard deviation of 18.9\%.
The maximum and minimum indicate that we controlled the noticeability with the visual stimuli's duration and layout successfully.
Additionally, the high standard deviation suggest a high variability across conditions, which is beneficial for training a model to predict the influence of visual stimuli on noticeability.

To verify that participants' physical movements were effectively redirected, we analyzed participants virtual and physical trajectory lengths as defined in~\autoref{section:study1results}.
In the unnoticed condition, participants' physical movement trajectories were significantly shorter than their virtual ones: \textit{Physical} $(AVG = 1.14,~SD = 0.04)$ and \textit{Virtual} $(AVG = 1.20,~SD = 0.04,~t(11) = 3.97,~p < 0.05)$.
In the noticed condition, participants' physical movement trajectories were also significantly shorter than their virtual ones: \textit{Physical} $(AVG = 1.14,~SD = 0.04)$ and \textit{Virtual} $(AVG = 1.27,~SD = 0.05,~t(11) = 5.38,~p < 0.01)$.
These results indicated that participants' physical movements were successfully redirected.


\section{Implementation}
\label{section:implementation}

In this section, we investigated the best combination of gaze behavior features to predict the noticeability of redirection.
We described participants' gaze behaviors with pupil activity, gaze angular distance to users' hands, eye saccade, and fixation.
For each category of participants' gaze behavioral data, we systematically examine different feature combinations to identify those that most accurately characterize the participants' visual responses.
We then develop a regression model to explain the relationship between the selected gaze behavioral features and the noticeability of redirection.

% \delete{
% In this section, we describe our implementation of our prediction model with the collected data from Section~\ref{section:datacollection}, including the participants' pupil activity, gaze movement, eye saccade, and fixation.
% In each category of participants' gaze behavioral data, we enumerate the combination of features to select the features that describe the gaze behavior best.
% Then we combine the four categories and explore the best prediction performance with different prediction models.
% Meanwhile, we leveraged the collected noticeability under each visual attention condition (i.e., the proportion of correct responses in each condition) as the ground truth.
% In this way, we aimed to build a model that takes the user's gaze behavioral patterns as input and predicts the noticeability of the motion offset under different visual attention conditions.
% }

\subsection{Gaze behavioral patterns}
\label{section:featureselection}

Results in \autoref{section:formativestudy} showed that the users' gaze behavioral data was correlated to the noticeability.
We divided the gaze behavioral data into four categories:

\begin{itemize}
    \item \textit{Index of Pupillary Activity}: 
    Index of Pupillary Activity (IPA) has been used to reflect users' cognitive load by analyzing the change of users' pupil dilation~\cite{duchowski2018index, lindlbauer2019context}. 
    While visual stimuli were presented, users' cognitive load might be inadvertently affected and could therefore impact the noticeability results.
    \item \textit{Gaze angular distance to elbow and hand}:  
    We calculated the vector starting from the user's eye to the elbow and hand joint. 
    We then calculated the angular distance between this vector and the gaze vector.
    These metrics reflect whether the participant was looking at the primary task or attracted by the visual stimuli.
    We decided not to calculate the distance between the focus point and the visual stimuli, as in real-world use cases, there is no single visual stimuli but only complicated ones, which make it hard to compute this distance.
    \item \textit{Eye saccade frequency, duration and interval}:
    Eye saccade is a rapid eye movement that shifts the eye from one area to another.
    We leveraged the detection algorithm from \citeauthor{pymovements} to detect the saccade frequency and duration~\cite{pymovements}. 
    The saccade frequency and duration indicate how often and how quickly users shift their eye gaze separately.
    The saccade interval suggests the temporal distribution, which indicates whether the saccades are uniformly distributed across the session.
    \item \textit{Eye fixation frequency, duration and interval}:
    Eye fixations represent when eyes stop scanning the scene and hold the foveal vision on an object of interest. 
    We also used the frequency, duration, and interval of eye fixation to indicate how often and how long users stared at a place and the temporal distribution of fixation.
\end{itemize}

\subsection{Regression model}

To better represent the previous gaze behavioral patterns, we computed \textit{mean, standard deviation, median, maximum and minimum} of IPA and gaze angular distance and combined them with the eye saccade and fixation features.
Therefore, we had 3 behaviors(IPA, gaze angular distance to hand, gaze angular distance to elbow) $\times$ 5 features (mean, standard deviation, median, maximum and minimum) $+$ 3 saccade (saccade frequency, duration and interval) $+$ 3 fixation (fixation frequency, duration and interval) $=~21$ features in total.
% \delete{
% We used \textit{mean, standard deviation, median, maximum and minimum} of IPA and gaze distance to describe the features.
% Combining the eye saccade and fixation features, we had $3~\times~5~+3+3~=~21$ features in total.
% }
However, the search space to determine the combination of features that provides the highest predictive power for noticeability includes as many as $\sum_{i=1}^{21}\frac{21!}{i!(21-i!)}~=~2^{21}-1$ conditions, which means that a grid search is not practical.
Therefore, we adopted a similar method as \cite{maslych2023effective} to select the features.
We first selected the best combination of features within each category and then searched the combination of these categories iteratively to figure out the best combination.

In this process, we used Support Vector Regression (SVR) from scikit-learn package~\footnote{https://scikit-learn.org/stable/modules/generated/sklearn.svm.SVR.html} as the benchmark model since SVR has a stable performance on various data.
The SVR model took the selected features as input, then output a probability ranging from 0 to 1 as the predicted noticeability.
We leveraged the leave-one-user-out cross-validation in the test and the mean squared error (MSE) as the metric.

\autoref{table:featureselection} lists the best combination of features within each of four categories.
Among them, the selected combination in the \textit{gaze angular distance} achieved the best performance, while the other features also demonstrated the potential for predicting noticeability.
Therefore, we combined the features from different categories and further tested them.

\begin{table*}[htb]
  \centering
  \small
    \begin{tabular}[width=\columnwidth]{ccc}
    \toprule
    \textbf{Category} & \textbf{Best combination} & \textbf{MSE}\\
    \midrule
    IPA & mean, maximum, minimum & 0.039 (0.013) \\
    % \midrule
    Gaze angular distance & \makecell{mean(hand), std(hand), median(hand) \\ mean(elbow), std(elbow), maximum(elbow)} & 0.017(0.008) \\
    % \midrule
    Eye saccade & frequency, duration, interval & 0.027(0.009) \\
    % \midrule
    Eye fixation & frequency, duration & 0.040 (0.012) \\
    \bottomrule
    \end{tabular}
    \caption{The best feature combination of each category and the prediction performance. The prediction performance is presented as the average (standard deviation) of MSE.}
    \label{table:featureselection}
\end{table*}

We then tested the regression error of all combinations of the feature category with leave-one-user-out cross-validation.
For each feature combination, we filtered the data with it and then fitted a model with 11 participants' data and tested it on the one remaining participant's data.
After repeating this 12 times, we determined the overall regression error for one feature combination.
\autoref{figure:featureselection} illustrates the regression error of all 15 feature combinations.
The results demonstrate that combining all these four category features achieves the best performance with an MSE of 0.011.

To further understand the best feature combination across the users, we also explored the best feature set for each test user in the leave-one-user-out cross-validation process.
For each test user, we trained a model with each feature combination and selected the best one.
The results showed that for 7 out of the 12 participants, the best feature set was the combination of all four feature categories.
For 3 of the 12 participants, the best set was the combination of IPA, Gaze Angular Distance and Fixation and for the other 2 of the 12 participants, the best set contained IPA, Gaze Angular Distance and Saccade.
% \delete{
% These results show that the selected features were able to indicate the visual stimuli's influence on the noticeability for various users.
% }
The results suggest that each feature captures distinct aspects of gaze behavior that contribute to predicting noticeability. 
Although the gaze angular distance showed a lower correlation with noticeability compared to saccades in \autoref{section:formativestudy}, it performs as the most powerful feature for predicting noticeability. 
This may due to the fact that in~\autoref{section:formativestudy}, we only considered the mean distance, whereas in this study, we included additional numerical features, which could provide more informative insights than the mean alone.
As for eye gaze saccade, it also contributes significantly to the prediction, aligning with the correlation results in~\autoref{section:formativestudy}, as it indicates users' visual focus shift and cognitive activity.
While IPA and fixation also have the potential to predict noticeability, their prediction accuracy is lower compared to the other features. 
This could be because they reflect more general cognitive activity and engagement, rather than specific responses to visual stimuli.
However, combining these features allows us to capture both where users are looking at and the dynamic shifts in focus, which together indicate the noticeability of redirection.
% IPA reflects users' cognitive load, which is correlated with their level of engagement and noticeability. 
% Fixations reveal sustained attention, indicating users' intentional focus on a specific region, while saccades highlight transitions in gaze, suggesting more exploratory or spontaneous actions. 
% Together, these features capture both where users are looking at and the dynamic shifts in focus that influence noticeability.

To further investigate if the selected features could model noticeability, we analyzed the regression error for each individual data point in a per user manner. 
As shown in \autoref{figure:predictionperformance_peruser}, the outputs from our model preserved the relative order of noticeability across the six conditions in 90.3\% data points.
The fitted noticeability in various conditions mostly remained in the range of the ground truth, while most errors came from the two most similar conditions (\textbf{CS} and \textbf{NL}). 
Furthermore, \autoref{figure:predictionperformance_average} illustrates the noticeability average and standard deviation of the data collection results and our model's output.
Our model's output average approximates the participant's results while simultaneously exhibiting a lower standard deviation.
This could be due to the inherent noise introduced from estimating the noticeability using the frequency of participants who reported the noticing of redirection in the study.

\begin{figure}[t]
    \centering
    \includegraphics[width=0.9\columnwidth]{figures/implementation/category.png}
    \caption{The regression error of all combinations of the feature category. The error bars denote the standard deviations.}
    \label{figure:featureselection}
\end{figure}

% \subsection{Model selection}

% \delete{
% Subsequently, we tested different models with the selected features to determine the best prediction models.
% We acknowledge that a different set of features might be optimal for a prediction models other than SVR.
% However, we did not aim to fine tune the feature sets for each model; this is not feasible considering the large search space for feature combinations.
% Instead, we compared the performance of different models on the same set of features that we think are most indicative of the noticeability results.
% We implemented several common regression models from scikit-learn package and reported the MSE with leave-one-user-out cross-validation.
% As shown in \autoref{table:modelselection}, the Adaboost regression model achieved the best performance, while other models also achieved a comparable performance.
% Notably, we did not implement more complicated models (e.g., deep learning models) to prevent the model from overfitting to our relatively small dataset.
% In this paper, we aim to demonstrate the potential of predicting noticeability under various visual effects.
% We regard collecting more data and implementing more sophisticated models as important avenues for future work.
% }

% \begin{table}[htb]
%   \centering
%   \small
%     \begin{tabular}[width=\columnwidth]{cc}
%     \toprule
%     \textbf{Model} & \textbf{MSE}\\
%     \midrule
%     SVR & 0.012(0.008) \\
%     % \midrule
%     Linear Regressor & 0.012(0.008) \\
%     % \midrule
%     Adaboost Regressor & 0.008(0.005) \\
%     Decision Tree Regressor & 0.013 (0.007) \\
%     Random Forest Regressor & 0.018 (0.010) \\
%     % \midrule
%     \bottomrule
%     \end{tabular}
%     \caption{The regression performance of different models, presented as the average (standard deviation) of MSE.}
%     \label{table:modelselection}
% \end{table}

% \subsection{Performance analysis}

% \delete{
% To further investigate if our model and selected features could predict noticeability, we analyzed the prediction percentage error for each individual data point in a per user manner. 
% We performed the tests with the SVR model since it was robust and could represent the average performance of different models.
% As shown in \autoref{figure:predictionperformance_peruser}, the predictions from our model preserved the relative order of noticeability across the six conditions in 90.3\% data points.
% The predicted noticeability in various conditions mostly remained in the range of the ground truth, while most errors came from the two most similar conditions (\textbf{CS} and \textbf{NL}). 
% Furthermore, \autoref{figure:predictionperformance_average} illustrates the noticeability average and standard deviation of the data collection results and our model prediction.
% Our model's prediction average approximates the participant's results while simultaneously exhibiting a lower standard deviation.
% This could be due to the inherent noise introduced from estimating the noticeability using the frequency of participants who reported the offset in the study.
% }

\subsection{Classficiation model}
\label{section:classfication_model}

In our studies, noticeability was measured by the frequency with which participants detected redirection during the trials. Based on this, we developed a regression model that outputs the probability of noticing the redirection as a float value between 0 and 1. While this probability effectively indicates how likely users are to notice the redirection, a classification model providing a simple yes/no result could offer greater practical utility.
To explore this, we trained a classification model by applying various thresholds to the noticeability results and categorizing it into distinct classes. 

\begin{description}
\item[Binary model]
We applied a threshold of 0.5 to transform the collected noticeability results into binary labels: Unnoticeable $(\leq 0.5)$ and Noticeable $(> 0.5)$.
With these, we trained a Support Vector Machine (SVM) classification model with the same features selected in \autoref{section:featureselection}; this model achieved an accuracy of 0.9174 $(SD=0.1126)$ and an F1-score of 0.8968 $(SD=0.1342)$ with leave-one-user-out cross-validation on our collected dataset.
\item[Three-class model]
Then we divided the noticeability into three categories with two thresholds: Low Noticeability $(\leq 0.4)$, Medium Noticeability $(0.4 <$ noticeability $\leq 0.7)$, and High Noticeability $(>0.7)$.
With the same SVM classification model and selected feature, our re-trained model achieved an accuracy of 0.8562 $(SD=0.1240)$ and an F1-score of 0.8478 $(SD=0.1276)$.
To be noted, the prediction accuracy was affected by how we converted the noticeability value to separate labels and might increase with fine-tuned features tailored to the classification task.
This indicates that the selected features from the gaze behavioral pattern have the potential to predict the noticeability as separate categories.
\end{description}

\begin{figure*}[t]
    \centering
    \begin{subfigure}{0.47\linewidth}
        \includegraphics[width=\columnwidth]{figures/implementation/opacity_scatterplot.png}
        \caption{}
        \label{figure:predictionperformance_peruser}
    \end{subfigure}
    \centering
    \begin{subfigure}{0.47\linewidth}
        \includegraphics[width=\columnwidth]{figures/implementation/opacity_boxplot_regression.png}
        \caption{}
        \label{figure:predictionperformance_average}
    \end{subfigure}
    \caption{(a) illustrates the regression results in each condition for each user. (b) illustrates the regression results with leave-one-user-out cross-validation.}
    \label{figure:predictionperformance}
\end{figure*}



\section{User Evaluation}
\label{sec:eval}
%include metodo, setup, ipotesi, risultati e grafici per i test con le due coorti di utenti
To assess the effectiveness and usability of Cyri as a tool for phishing detection and management from a human user, a user study was conducted involving ten participants with varying levels of expertise in computer security. This section details the methodology of the user study, the setup, and discusses the findings.

\subsection{Experiment Setup}
\label{sec:setup}

The study involved 10 participants, split equally between computer security
experts (meaning having at least two years of expertise and being knowledgeable of phishing tactics and techniques) and non-experts (meaning not being knowledgeable of phishing tactics and techniques but capable of using an email account).\\
The study duration for each participant was 60 minutes, split into 15 minutes of initial explanation on what are the most important features of Cyri and how to install it. This first step was then followed by two main tasks:
\begin{itemize}
    \item Controlled Email Identification Task: Participants were put in front of a preconfigured installation of Cyri and received five emails, four safe emails, and one phishing email sent by us. They were instructed to review these emails with Cyri and identify the phishing emails among them and the motivating factors for their final decision.
    This test was used both to let the participants gain confidence with Cyri usage and interface and as a controlled experiment where to evaluate how users interpreted and used the different results and functionalities Cyri exposes in a controlled situation equals for all of them. This step lasted, on average, from 10 to 15 minutes.
    \item Exploration with Personal Emails: Participants were then tasked to use Cyri to analyze their inbox emails from one personal account, such as those in their spam folder, unopened ones, or newly received messages. This allowed them to interact with the application in a context familiar to them and to assess its usefulness beyond the controlled task of provided emails, resulting in a more personal experience capable of letting them assess the degree of support they received from Cyri. This task lasted, on average, 25 minutes.
\end{itemize}
After completing the second task, participants were asked to compile a survey
comprising several questions aimed at evaluating Cyri’s effectiveness in assisting users in identifying phishing emails, usability and intuitiveness of the application interface, impact on users' understanding of phishing tactics, the likelihood of continued use, and preference for platform availability.
In particular, the questions proposed to the participants were the following:

\begin{enumerate}
    \item Are you a computer security expert? (Yes or No)
    \item How confident are you in identifying phishing emails without assistance?  (Scale 1 to 5)
    \item How useful was Cyri in helping you identify the phishing email?  (Scale 1 to 5)
    \item Did Cyri provide information that you wouldn't have noticed on your own? (Yes or No)
    \item How would you rate the overall usability of Cyri? (Scale 1 to 5)
    \item How intuitive did you find the Cyri interface?  (Scale 1 to 5)
    \item How much do you think using Cyri would improve your understanding of phishing tactics? (Scale 1 to 5)
    \item Would you use Cyri regularly as part of your email routine?  (Yes or No)
    \item Would you prefer if Cyri was available on your mobile phone instead of your computer?  (Yes or No)
\end{enumerate}

A final free text form allows the insertion of open comments and suggestions. Overall 5 minutes were dedicated on average to this activity.

\subsection{Results}
\label{sec:userresults}

We analyzed the survey results by splitting participants into their expertise level into two groups: Figure~\ref{fig:secexp} reports results for computer security experts while Figure~\ref{fig:nonsecexp} reports them for non-expert users. This distinction allowed us to understand how Cyri is perceived by users with different levels of expertise.

\begin{figure}[htbp]
  \centering
  \includegraphics[width=0.45\textwidth]{figures/SecurityExpertsBarchart.PNG}
  \caption{Security Experts Average Scores}
  \label{fig:secexp}
\end{figure}

\begin{figure}[htbp]
  \centering
  \includegraphics[width=0.45\textwidth]{figures/NonSecurityExpertsBarchart.PNG}
  \caption{Non-Security Experts Average Scores}
  \label{fig:nonsecexp}
\end{figure}

Cyri has been declared to be highly beneficial by non-expert participants, reporting generally low confidence in their ability to identify phishing emails without Cyri's assistance (Q2). All non-expert participants affirmed that Cyri provided information they would not have noticed on their own (Q4). This suggests that Cyri effectively highlights phishing indicators that might be overlooked, adding significant value in assisting them in identifying potential threats. 
Furthermore, non-experts provided very high ratings for both the usability (Q5) and intuitiveness (Q6) of Cyri. These results indicate that they found the application user-friendly and accessible. Non-experts believed using Cyri would significantly improve their understanding of phishing tactics (Q7), underscoring the application’s educational value.

Expert participants, even if they declared an expected good capability of identification and management of the phishing email with and without Cyri support (Q2), acknowledged that Cyri can enhance analysis capabilities by providing an additional layer of information (Q3). Interestingly, all expert participants also affirmed that Cyri provided information they would not have noticed on their own (Q4). This indicates that Cyri can uncover subtle phishing indicators and offer insights that even experienced users might overlook. Experts rated the overall usability (Q5) and intuitiveness (Q6) of Cyri highly, similar to non-experts, suggesting that the application is well-designed for users across different expertise levels. Moreover, experts believed that using Cyri could further improve their understanding of phishing tactics (Q7).
All participants expressed their willingness to use Cyri regularly as part of their email routine (Q8) and showed a clear preference for having it available also on their mobile devices (Q9).
\section{Towards real-world use cases}

While the previous study results suggest that our proposed model could effectively compute the noticeability of redirection under various basic visual stimuli (transparency-, color- and scale-based), we aimed to explore how the model could be used in real-world scenarios.
To showcase the potential benefits of our model in practical use cases, we implemented \textbf{an adaptive redirection technique} and developed \textbf{two real-world applications} to demonstrate its generalizability and usability.
We also performed a proof-of-concept study to gather user feedback while interacting with the two applications and the adaptive redirection technique.

% \delete{
% Our regression model can compute the noticeability of motion offsets under various visual stimuli.
% We envision that the model can support various interaction techniques by adjusting the noticeability of motion offsets to match contextual requirements.
% In the following, we first outline how our model could potentially be utilized by content creators and then we demonstrate its applicability in two scenarios.
% Although the system has not been formally evaluated through user studies, we believe that they showcase the potential benefits that our regression model could provide.
% }

% \delete{
% Based on our noticeability regression model, we implemented two applications to demonstrate future interaction scenarios.
% In the applications, our regression model was integrated with the system to compute the noticeability using the user's eye behavioral patterns in the last 30 seconds.
% Although the system has not been formally evaluated through user studies, we believe that they showcase the potential benefits that our regression model could provide.
% }

\begin{figure}[t]
    \centering
    \begin{subfigure}{0.45\columnwidth}
        \includegraphics[width=0.9\columnwidth]{figures/applications/app1.png}
        \caption{}
        \label{figure:application_adaptive}
    \end{subfigure}
    \centering
    \begin{subfigure}{0.45\columnwidth}
        \includegraphics[width=0.9\columnwidth]{figures/applications/app2.png}
        \caption{}
        \label{figure:opportunistic}
    \end{subfigure}
    \caption{
    We developed two real-world applications to demonstrate the capabilities of our adaptive redirection technique:
    (a) Adjusting the difficulty of VR action game: 
    In this application, mid-air coins and monsters serve as visual cues for the target poses that users are asked to perform. 
    Our adaptive redirection technique enables the system to adjust the game’s difficulty without the user noticing, ensuring a balanced and engaging experience.
    (b) Opportunistic rendering for boxing training in VR : 
    Here, users are learning boxing movements by following a blue avatar. 
    Leveraging our adaptive redirection technique, the system can simulate opportunistic rendering which reduces requirements for computation resources.}
    \label{figure:application}
\end{figure}

\subsection{Adaptive motion redirection technique}
As discussed in the Introduction (\autoref{section:introduction}) and Related Work (\autoref{section:related_work}), users in real VR applications may face complicated visual effects that can impact the noticeability of redirection movements. 
This, in turn, influences the effectiveness and overall user experience of redirection techniques.
While it is impractical to predict the specific visual effects users will encounter beforehand, content creators can only predefine a static redirection intensity, which limits the effectiveness of redirection techniques. 
To address this limitation, our proposed model enables designers to dynamically adjust the redirection during usage based on the user’s gaze behavior.

Our model computes the noticeability of redirection as a float value ranging from 0 to 1. 
With this output, we implemented an adaptive redirection technique by using the Three-class model described in \autoref{section:classfication_model}.
For each class of noticeability, we predefined corresponding redirection: 25 degrees for Low Noticeability, 15 degrees for Medium Noticeability, and 5 degrees for High Noticeability. 
When the computed noticeability falls into one of these classes, the corresponding redirection is applied.
The redirection technique initializes with a 10 degree offset. 
When a change in redirection is required according to the noticeability changes, we use linear interpolation to transition the redirection gradually over a 10-second period. 
To maintain immersion, the redirection is adjusted only when the user’s arm is in motion, since if the redirection changes while the physical arm remains static, the virtual arm will be moved and lead to break of immersion and sense of embodiment.

To be noted, this adaptive redirection technique serves as a demonstration of the usability of our proposed model. 
Designers can leverage the model's probabilistic output to create their own redirection techniques tailored to specific applications.

% \subsection{\delete{Adjusting motion offsets to meet noticeability requirements}}
% \delete{
% Currently, designers typically aim to minimize the noticeability of motion offsets to preserve the user's sense of embodiment when employing redirection techniques in VR interactions. 
% However, the challenge lies in the difficulty of predicting the exact visual animations or effects that users will experience in advance, limiting designers to pre-setting motion offset thresholds.
% For instance, in the context of an action game, a designer might adopt a redirection technique to enhance the user experience. 
% The motion offsets might go unnoticed when complex visual effects and animations capture the user's visual attention. 
% However, if these visual effects diminish in intensity, the user's focus may return to their virtual body, making the same motion offset detectable.
% Therefore, we envision that our model could enable designers to dynamically adjust motion offsets by tracking users' gaze patterns and visual attention in real time. 
% This would allow for a more responsive and adaptable redirection technique, ensuring that motion offsets remain unnoticed under varying visual conditions.
% }

% \delete{
% With our model, content creators can design redirection techniques and their noticeability requirements in advance, while leaving their intensity to our model to decide. 
% During runtime, the interactive system can maintain a historical record of users' gaze behavioral data and input it to our model.
% Based on this, our model can simulate various motion offsets and compute their noticeability results.
% Then the interactive system can apply an appropriate motion offset to meet the predefined noticeability requirement.
% }


% \change{
% To further understand the prediction performance of our model, we converted the regression model to a classification model by applying different thresholds and dividing the noticeability into separate categories.
% We first converted the regression model into a binary classification model with a noticeability threshold of 0.4, which could classify the noticeability between low and high visual attention based on \autoref{figure:predictionperformance_average}.
% We trained an SVM classification model with the same features selected in \autoref{section:featureselection}; this model achieved an accuracy of 0.917 $(SD=0.112)$ and an F1-score of 0.896 $(SD=0.134)$ with leave-one-user-out cross-validation.
% Then we divided the noticeability into three categories with two thresholds: Low Noticeability $(\leq 0.33)$, Medium Noticeability $(0.33 <$ noticeability $\leq 0.66)$, and High Noticeability $(>0.66)$.
% With the same SVM classification model and selected feature, our re-trained model achieved an accuracy of 0.856 $(SD=0.124)$ and an F1-score of 0.847 $(SD=0.127)$.
% Notably, the prediction accuracy was affected by how we converted the noticeability value to separate labels and might increase with fine-tuned features tailored to the classification task.
% This indicates that the selected features from the gaze behavioral pattern have the potential to predict the noticeability as separate categories.
% }

\subsection{Real-world applications}

\subsubsection{Adjusting the difficulty of VR action game}
% \delete{
% First, we implemented an adaptive motion offset adjustment based on the status of the user's visual attention (as indicated by their gaze behavior).
% With the user's gaze behavioral data, our model is able to compute the noticeability of motion offsets.
% Thus, designers can limit the noticeability of motion offsets to a desired level by adjusting the motion offsets, based on our model.
% We demonstrate this with a VR action game}
Based on our adaptive redirection technique, we implemented a VR action game inspired by the VR game Beat Saber~\footnote{https://beatsaber.com/}.
In the game, users are asked to perform certain poses with their arms based on visual and musical guidance.
The game difficulty could be adjusted by redirecting the user's movement, for example, slightly amplifying their movements could make it easier and faster to achieve the targets.
Meanwhile, users need to focus on the targets to obtain sufficient information, and thus they paid less visual attention to their virtual body movements.
As shown in \autoref{figure:application_adaptive}, when the visual guidance for the target arm pose is highly detailed and draws significant attention from the user, 
the noticeability of redirection might be lower and 
the system can take the risk of applying large redirection for functional gains.
However, when the user interacts with a simpler interface and focuses mainly on their virtual arm, 
a low level redirection might be applied with the high noticeability prediction.
%designers can just amplify users' motion a bit and provide limited guidance when the visual effects are simple.
%Accordingly, designers can provide more attractive visual effects to provide stronger guidance by applying larger offsets to users' motion.

% Adjust the strength of the offsets according to the user's gaze behavior, as an indicator of their visual attention allocation status, to maintain the same level of noticeability.


\subsubsection{Opportunistic rendering for boxing training in VR}
We implemented a boxing training system designed to reduce rendering computation as our second application.
Accurate motion reconstruction and rendering may require high computing power~\cite{chen2021towards}.
While users may not always focus on their virtual movements, there is a chance to apply opportunistic rendering based on the user's visual attention to save computing capability and avoid being noticed by users.
As shown in \autoref{figure:opportunistic}, the user is learning boxing poses with a virtual coach in VR.
When the user is looking at the coach and observing them performing the pose, our model may output a lower level of noticeability and thus it allows the system to update the user's movement less frequently which leads to the virtual movement has a offset with the user's physical movement and save computing resources.
While the user shifts their attention back to his arm and is going to practice the boxing poses, our model can compute that the noticeability of motion offset is higher than in the previous scenario.
Therefore, the system can allocate more resources to render the user's movement, to ensure that they can perform and learn the accurate poses in VR.
To be noted, we implemented this application as a simulation of opportunistic rendering to demonstrate the potential of our model, rather than fully implementing it and measuring the computational resources it would save.

% When gaze behaviors tells the system, the user's visual focus is attracted by notifications/distractions, it can decide to lower the requirement on the sensing/rendering of the user's body motion.

\subsection{Proof-of-Concept study}

To further demonstrate and evaluate the how our model supports adaptive redirection techniques, we conducted a proof-of-concept evaluation study on two applications.

\subsubsection{Design}

We conducted a within-subject factorial study design, with the independent variable being the experimental conditions, including Adaptive Redirection (\textbf{AR}) and Static Redirection (\textbf{SR}).
In the VR action game, participants were tasked with performing poses that aligned with a moving target. 
The target’s appearance frequency progressively increased, starting at intervals of 2 seconds and accelerating to 0.5 seconds and the game lasted for 60 seconds.
In the boxing training application, participants engaged in a 60-second motion-learning task, attempting to replicate the movements demonstrated by a virtual coach.
For the \textbf{SR} condition, the redirection magnitude was fixed at 15 degrees, which is the same as the medium level magnitude used in the \textbf{AR} condition.
After completing the tasks in each condition, participants rated the tested conditions on physical demand ("\textit{The interaction was physically demanding}"), mental demand ("\textit{The interaction was mentally demanding and I had to concentrate a lot.}"), embodiment ("\textit{I felt as if the virtual body was my body}") and agency ("\textit{I felt like I could control the virtual body as if it was my own body}") with a 7-point Likert scale, using the questions from similar studies in prior work~\cite{peck2021avatar, feick2023investigating}.


\subsubsection{Apparatus \& Procedure}

We implemented the applications with a HTC Vive pro headset in Unity 2019, powered by an Intel Core i7 CPU and an NVIDIA GeForce RTX 2080 GPU. 
During the study, participants were equipped with three Vive Trackers affixed to their left shoulder, elbow, and waist using nylon straps.

After being introduced to the study, participants had a warm-up session to learn about the study tasks and get familiar with controlling the virtual movements.
Once they were comfortable with the virtual movements and tasks, they proceeded to experience one condition across both applications.
After completing the two applications under the first condition, participants provided their ratings before moving on to experience the second condition. 
The order of conditions and applications was counterbalanced.
The study lasted around 20 minutes, and each participant received a compensation of 10 US dollars for their participation.

\subsubsection{Participants}

We recruited 8 new participants (2 females, 6 males, average age of 25.63 with $SD=1.85$) from a local university.
These participants reported their familiarity with VR as an average of 3.75 $(SD=1.16)$ on a 7-point Likert-type scale from 1 (not at all familiar) to 7 (very familiar).

\subsubsection{Result}

\begin{figure}[t]
    \centering
    \includegraphics[width=0.9\linewidth]{figures/applications/preliminary_results.png}
    \caption{Proof-of-concept study results indicate that participants experienced less physical demand and a stronger sense of embodiment and agency when using the adaptive redirection technique compared to the static technique.}
    \label{figure:preliminary_result}
\end{figure}

\autoref{figure:preliminary_result} summarizes the study results.
We conducted Wilcoxon signed-rank tests to analyze the reported subjective metrics.
Participants reported lower physical demand in the adaptive redirection (\textbf{AR}) condition ($M=3.50, SD=1.00$) compared to the static redirection (\textbf{SR}) condition ($M=4.50, SD=0.50$, $W=2.00, p<0.05$). 
This is due to the larger redirection allowed in \textbf{AR} when visual stimuli were intense, reducing the need for extensive physical movement.
Despite the adaptive nature of \textbf{AR}, participants did not perceive a higher mental demand ($M=4.25, SD=0.60$) compared to \textbf{SR} ($M=4.25, SD=0.43$, $W=5.00, p>0.05$). 
This suggests that \textbf{AR} does not introduce additional cognitive effort for participants to control their virtual motion during interactions.
Participants reported a stronger sense of embodiment ($M=5.13, SD=0.60$) and agency ($M=5.25, SD=0.43$) in \textbf{AR} compared to \textbf{SR}, where embodiment ($M=4.13, SD=0.92$, $W=2.50, p<0.05$) and agency ($M=4.13, SD=0.92$, $W=3.00, p<0.05$) were rated lower. 
This can be attributed to the reduced possibility of detecting the redirection in \textbf{AR}, which enhanced participants' sense of control and immersion. 
In contrast, the frequent detection of redirection in \textbf{SR} reduced their sense of agency and embodiment.

These results suggest that our technique effectively adapts the redirection magnitude to the visual stimuli, aligning with the predicted noticeability from our computational model. 
This demonstrate the potential benefits and capabilities of the model in enhancing redirection interactions.

\section{Discussion}
% Short summary of the paper.
In this paper, we investigated the effects of visual stimuli on to what extent users notice inconsistencies in their physical movements versus avatar movements. 
We further contribute a regression model that computes the noticeability of redirection under various visual stimuli, based on users' gaze behavioral data.
With the model, we constructed two applications in realistic scenarios with different types of visual stimuli to demonstrate the potential advantages and extensions of our method.
In the following, we discuss possible extensions to our model, as well as limitations and future work.

\subsection{Redirection and visual stimuli}
While prior work~\cite{li2022modeling, feick2023investigating, feick2021visuo} explored how the properties (such as magnitude, direction, location) of redirection influenced its noticeabiltiy, we investigated the noticeability under visual stimuli in this paper.
However, we acknowledge that the redirection properties and the visual stimuli may affect the noticeability in different manners. 
The redirection properties could determine the upper and lower bounds of redirection noticeability, while the visual stimuli can only reduce the noticeability in a limited range.
For subtle redirection that are barely noticeable even when the user is focused on their body movements, adjusting visual stimuli does not significantly alter the noticeability.
Similarly, users will likely notice salient redirection even with a glance, unless the redirection is completely out of their field of view.
Therefore, in this paper, we fixed the redirection magnitude to be 20 degrees (as a control variable), for which the resulting noticeability ranged from approximately 20\% to 80\%.
This relatively large range enables us to quantify the impacts of visual stimuli extensively.
However, we believe that exploring the interaction effect of redirection properties and visual stimuli and combining their influence on noticeability could be important and interesting future work.

\subsection{Diverse visual stimuli}
In this paper, we used several abstract visual stimuli and changed their intensity in our user studies.
We acknowledge that beyond these static visual stimuli tested in this paper, there exist various complicated visual stimuli in realistic use cases.
For instance, a moving object or a wiggling notification may also affect users' gaze behavior and therefore influence the noticeability of redirection.
We consider visual stimuli appearing and staying at a static location to be a standard design paradigm in presenting notifications (e.g., highlighting app icons when new messages are received) on desktop~\cite{muller2023notification}, VR~\cite{rzayev2019notification}, and AR interfaces~\cite{lee2023effects}.
Though we validated our model on new type of abstract visual stimuli and in realistic scenarios,
we acknowledge that verifying the generalizability of our regression model on motion-based or other more complicated visual stimuli is an important future work.
We expect that our research methodology and the presented gaze behavioral patterns can also apply to the investigation of other visual stimuli.

Besides, the visual stimuli investigated in this paper primarily served as external cues for object selection or observation, rather than being directly related to users' body movements. 
In scenarios such as motion training and learning, users may observe their body movements through a mirror or from a third-person perspective, making redirected motion part of the visual stimuli. 
This raises an open question of how to decouple redirection from visual stimuli to investigate their specific influence on the noticeability of redirection. 
We acknowledge this as an important direction for future research.

Furthermore, in more realistic usage scenarios, the stimuli could be in different formats, including instant notifications, environmental events, or even the user's implicit observation of the virtual scene.
We acknowledge that in such cases, 
% besides the gaze patterns captured by the proposed method, 
different behavioral patterns or even physiological signals, such as EEG signals and heart rate variations, can also be indicative of the noticeability of redirection.
We expect that our research method can be adopted to explore further the behavioral patterns that reflect the noticeability in a more realistic setting.

% In a more practical usage scenario, it could be instant notifications, environmental events, mind wandering of oneself.
% Different attractions may elicit different behavioral patterns (?), discuss based on the evaluation results as we preliminarily test different types of attractions.


\subsection{User awareness and adaptation}
% To what extent should the user be aware of the manipulation.
% How to involve the user into the decision making process.
% Users might be able to adapt to the offset gradually, the system should be able to keep adapting the strategy accordingly.

In our user studies, we hide the true purpose from the participants by disguising it as an accuracy evaluation for a motion tracking system.
The consideration was to mitigate the potential bias of users being aware of the existence of redirection, which might nudge them to be more attuned to or hyper-aware of the redirection.
In addition, our study lasted at most 40 minutes with multiple breaks, which allowed users to regain their perception of their physical movements and prevent fatigue.
% This helped to mitigate the influence of user awareness and adaptation to the offset.
However, if a long-term redirection is applied in real-life applications, users might become desensitized to the redirection gradually.
Users may adapt to the redirection after noticing them multiple times and assume that the redirection exists consistently, which could reduce the noticeability of the redirection.
We argue that the regression model should take into account the user's awareness and their ability to adapt their interaction behavior to continue computing the noticeability accurately.


%\subsection{Attention estimation based on offset noticeability}
%We established a prediction model that takes as input the users' gaze behavioral data and predict the noticeability of an offset applied on the user's body movement.
%The rationale behind the model is that the user's gaze behavior reflect their attention to the body movements, which eventually impacts the probability that they notice the offset.
%In other words, gaze data serves as the bridging connecting the user's visual attention to the offset noticeability.
%Therefore, theoretically, it is also possible to leverage the noticeability results to infer the amount of visual attention of the user.
%This opens up new opportunities to estimate visual attention based on users' responses, which does not require tracking the user's gaze with extra sensors.
%For instance, when applying a visual attraction, our model can predict the noticeability results of users paying different levels of visual attention to it.
%If users consistently report detecting the offset during interactions, it could imply that the visual attraction has failed to attract the user's visual attention.
%This opens up new opportunities to estimate visual attention based on users' responses, which might help designers while designing interactive systems.


% Estimate the amount of visual attention the user pays to a certain task based on the behavioral patterns of their gaze.


% \subsection{VR vs. AR}
% Are we exclusive to AR? See-through might be applicable.




\subsection{Limitations and Future Work}

% arm poses
In our user studies, we treated the ending arm poses that we applied redirection on as a control variable.
We clustered 25 arm poses from the CMU MoCap dataset~\cite{CMUMocap} that are common poses in real-life activities, randomly selecting and testing one of them in each trial.
This enabled us to average the impacts of different arm poses and focus on the influence of visual stimuli on redirection noticeability.
However, we acknowledge that the selected pose set is still limited in size compared to the amount of arm poses that are possible to perform in real life.
We regard extending our study to include more arm poses and apply redirection on other body parts as important future work.

% personalized
% physiological data
%Regarding the implementation of our prediction model, we leveraged the user's gaze behavioral patterns to predict the noticeability of the applied offsets.
We implemented the regression model with the data from 12 users and evaluated it with another 24 new users.
The results showed that our model could compute the noticeability accurately with new users while they experienced novel visual stimuli that never appeared in the training set.
However, we envision that a personalized model could improve the regression performance by collecting more data from the same user and capturing their unique behavioral patterns more accurately.
In addition, as we primarily focus on modeling the relationship between the visual stimuli and the redirection noticeability, we adopted SVR in the implementation of the regression model as it is relatively stable and did not overfit.
We note that when applying the findings into real life applications, more advanced regression/classification methods (e.g., deep learning models) and more fine-tuned parameters are worth exploring to optimize the regression performance.
As past work has demonstrated the relationship between hand redirection noticeability and users' physiological data~\cite{feick2023investigating}, we will explore how to add physiological data into the regression model in the future.

% abstract and controlled visual attractions
% more realistic tasks
% long-term adaptation
We investigated the effects of visual stimuli on noticeability and implemented a regression model with highly-controlled study designs and abstract visual stimuli.
Our goal was to study whether and how visual stimuli affects noticeability by controlling the factors and showcasing the potential applications that can benefit from our model.
We regard it as important future work to investigate the effect in a field study with more realistic tasks.
We will also further generalize our contribution with a longitudinal study to consider how users adapt their interaction patterns to redirection over time.


% Regarding the user studies: 
% 1) coverage of possible arm poses by the tested set;
% 2) nature of being highly controlled and abstract;

% Regarding the implementation:
% 1) taking arm pose into consideration
% 2) taking user's adaptivity into consideration
% 3) personalization
% 4) connect to the biophysical signals (?)
% 5) arm motion -> body motion

% Generic:
% 1) field study with more realistic tasks
% 2) longitudinal study on long-term adaptation



\section{Conclusion}

In this paper, we investigated and modeled the effects of visual stimuli on the noticeability of redirection using users' gaze behaviors in VR.
We first conducted a confirmation study to verify if users' noticeability of redirection was affected by visual stimuli and whether their gaze behaviors were correlated with the noticeability results.
After confirming that visual stimuli could influence the noticeability of redirection, we conducted a data collection study with refined visual stimuli.
With the data, we built up a regression model and selected effective features to compute the noticeability of redirection based on gaze behavior data, achieving an accuracy of 0.011 MSE.
We then evaluated our model on unseen visual stimuli with 24 new users and results suggested that our prediction model could generalize to new visual stimuli.
We then implemented an adaptive redirection technique based on our model and conducted a proof-of-concept study comparing it to static redirection technique.
Results suggested that participants felt less physical demanding while kept a high sense of body ownership using the adaptive redirection technique based on our model.
We believe that our model could support more effective and immersive redirection interactions in VR.


%%
%% The acknowledgments section is defined using the "acks" environment
%% (and NOT an unnumbered section). This ensures the proper
%% identification of the section in the article metadata, and the
%% consistent spelling of the heading.
\begin{acks}
We thank Yu Jiang for her research insights.
This work is supported by the National Key Research and Development Program of China under Grant No.2024YFB2808803 and the Natural Science Foundation of China under Grant No. 62102221, 62132010, 62472244, the Tsinghua University Initiative Scientific Research Program, and the Undergraduate Education Innovation Grants, Tsinghua University.
\end{acks}

%%
%% The next two lines define the bibliography style to be used, and
%% the bibliography file.
% \normalem
\bibliographystyle{ACM-Reference-Format}
\bibliography{reference}

%%
%% If your work has an appendix, this is the place to put it.
\appendix

\label{sec:appendix_section}

\section*{Overview}
This supplementary material presents additional details and results to complement the main manuscript. In Section \ref{sec:Implementation}, we provide comprehensive implementation details, including dataset preprocessing protocols and training configurations. Section \ref{sec:Laplacian} presents an empirical analysis of the impact of different pyramid levels in our Laplacian decomposition technique and provides implementation details of the algorithm. Section \ref{sec:Qualitative} showcases qualitative results demonstrating our method's effectiveness across various datasets and real-world scenarios. We will release our complete training and inference code along with pre-trained weights to facilitate future research in this area.

\section{Implementation Details}~\label{sec:Implementation}
\subsection{Datasets and Preprocessing}
We use two publicly available color constancy benchmark datasets in our experiments: the NUS-8
dataset~\cite{cheng2014illuminant} and the Gehler dataset~\cite{4587765}. The Gehler dataset~\cite{4587765} contains 568 original images captured by two different cameras, while the NUS-8 dataset~\cite{cheng2014illuminant} contains 1736 original images captured by eight different cameras. Each image in both datasets includes a Macbeth Color Checker (MCC) chart, which serves as a reference for the ground-truth illuminant color.



% We use two publicly available color constancy benchmark datasets in our experiments: the NUS-8 Camera dataset \cite{Cheng:14} and the re-processed Color Checker dataset \cite{4587765} (termed as the Gehler dataset in this paper).
% The Gehler dataset contains 568 original images, while the NUS-8 Camera dataset contains 1736 original images captured by eight different cameras. Each image in both datasets includes a Macbeth Color Checker (MCC) chart, which serves as a reference for the ground-truth illuminant color.

Following the evaluation protocol in \cite{afifi2019sensor}, several standard metrics are reported in terms of angular error in degrees: mean, median, tri-mean of all the errors, the mean of the lowest 25\% of errors, and the mean of the highest 25\% of errors.

\subsection{Training Details}

For all experiments, we process the raw image data before applying gamma correction for sRGB space conversion following the preprocessing protocol from \cite{hu2017fc4}. Since the pre-trained VAE was trained on sRGB images, we apply a gamma correction of $\gamma = 1/2.2$ on linear RGB images before encoding to minimize the domain gap. Conversely, after VAE decoding, we apply inverse gamma correction to convert the output back to the linear domain for metric evaluation.

All experiments are trained for 20000 iterations on an NVIDIA A6000 GPU using the Adam optimizer with an initial learning rate of $5 \times 10^{-5}$ and apply exponential learning rate decay after a 150-step warm-up period. For data augmentation, we follow FC4~\cite{hu2017fc4} to rescale images by random RGB values in [0.6, 1.4], noting that we only rescale the input images since our training does not require ground truth illumination. The rescaling is performed in the raw domain, followed by gamma correction. This is implemented through a 3×3 color transformation matrix, where diagonal elements control the intensity of individual RGB channels (color strength), and off-diagonal elements determine the degree of color mixing between channels (color offdiag). For Laplacian decomposition, we use a two-level pyramid ($L = 2$) to balance the preservation of high-frequency structural details and the suppression of low-frequency color information. Additionally, we apply local transformations to masked regions only, including brightness adjustment ($[0.8, 2.0]$), saturation adjustment ($[0.8, 1.4]$), and contrast adjustment ($[0.8, 1.4]$).

\paragraph{Three-fold Cross-validation}
For cross-validation experiments on both the NUS-8 dataset~\cite{cheng2014illuminant} and the Gehler dataset~\cite{4587765}, we use a batch size of 8. During training, we apply random crop with a probability of $p_{crop} = 0.7$, where the crop size ranges from 70\% to 100\% of the original dimensions. Color augmentation is applied with a probability of $p_{color} = 0.3$.

\paragraph{Leave-one-out Evaluation}
For the leave-one-out experiments on the NUS-8 dataset~\cite{cheng2014illuminant}, we use a batch size of 8 with gradient accumulation over 2 steps (effective batch size of 16). We apply random crop with a probability of $p_{crop} = 0.75$, where the crop size ranges from 70\% to 100\% of the original image dimensions, and color augmentation with a probability of $p_{color} = 0.65$. 

For the Gehler dataset~\cite{4587765}, when training on Canon5D and evaluating on Canon1D, we use a batch size of 8, apply random crop with a probability of $p_{crop} = 0.75$ (crop size from 70\% to 100\%), and color augmentation with a probability of $p_{color} = 0.85$. Similarly, when training on Canon1D and evaluating on Canon5D, we maintain the same batch size of 8, with random crop probability of $p_{crop} = 0.7$ and crop size ranging from 50\% to 100\%, while keeping the color augmentation probability at $p_{color} = 0.85$.

\paragraph{Cross-dataset Evaluation}
When training on NUS-8~\cite{cheng2014illuminant} and testing on the Gehler dataset\cite{4587765}, we use a batch size of 8 with gradient accumulation over 2 steps (effective batch size of 16). We apply random crop with a probability of $p_{crop} = 0.75$, where the crop size ranges from 70\% to 100\% of the original dimensions, and color augmentation with a probability of $p_{color} = 0.6$. Conversely, when training on the Gehler dataset~\cite{4587765} and testing on NUS-8~\cite{cheng2014illuminant}, we use a batch size of 8 without gradient accumulation. We apply random crop with the same probability of $p_{crop} = 0.75$ and size range of 70\% to 100\%, while color augmentation is applied with a probability of $p_{color} = 1.0$.

\paragraph{SDXL Inpainting (SDEdit)}
For the SDXL inpainting model \cite{rombach2021highresolution} with LoRA fine-tuning experiments, we use a learning rate of $5 \times 10^{-5}$ and a LoRA rank of 4. In the cross-dataset experiment from the NUS-8 dataset~\cite{cheng2014illuminant} to the Gehler dataset~\cite{4587765}, we train for 20,000 iterations with batch size 4.

\subsection{Inference Settings}
\paragraph{Full Model} Following \citeauthor{garcia2024fine} \cite{garcia2024fine}, we employ DDIM scheduler with a fixed timestep $t=T$ and \textbf{trailing} strategy during inference for deterministic single-step generation. Our implementation is based on the stable-diffusion-2-inpainting model \cite{rombach2021highresolution}.
\paragraph{SDXL Inpainting (SDEdit)} 
For comparison, we also implement a version using SDXL inpainting model \cite{rombach2021highresolution} with LoRA \cite{hu2021lora} fine-tuning. During inference, we use the DDIM scheduler with 25 denoising steps and SDEdit with a noise strength of 0.6, a guidance scale of 7.5, and a LoRA scale of 1. The final illumination estimation is obtained by computing the median from an ensemble of 10 generated samples.
\section{Laplacian Decomposition}~\label{sec:Laplacian}
\begin{figure}[t]
    \centering
    \includegraphics[width=1\linewidth]{figures/lapacian.pdf}
    \caption{\textbf{Flow diagram of Laplacian decomposition.} Frequency component fusion through two-level ($1/2$ resolution) blur, downsample, and composition operations.}
    \label{fig:lapacian}
\end{figure}


\subsection{Laplacian Decomposition Visualization}
Figure~\ref{fig:lapacian} visualizes the algorithm flow of our Laplacian decomposition technique. Algorithm~\ref{alg:laplacian} outlines the detailed steps of this process, which preserves high-frequency structural details while allowing illumination-dependent color adaptation, enabling accurate scene illumination estimation.

\begin{algorithm}[h!]
\footnotesize
\SetAlgoLined
\DontPrintSemicolon
\KwIn{Input latent $z \in \mathbb{R}^{B \times C \times H \times W}$, pyramid levels $L$}
\KwOut{High-frequency components $z_h$}
\texttt{Initialize} $z_h = 0$\;
$k \leftarrow$ \texttt{3×3 Gaussian kernel}\;
\For{\texttt{each channel} $c$ \texttt{in} $C$}{
    $z_{\text{curr}} \leftarrow z[c]$ \tcp*[r]{\texttt{Current level features}}
    \For{$l = 0$ \texttt{to} $L-1$}{
        $z_{\text{blur}} \leftarrow k * z_{\text{curr}}$ \tcp*[r]{\texttt{Gaussian blur}}
        $z_{\text{high}} \leftarrow z_{\text{curr}} - z_{\text{blur}}$ \tcp*[r]{\texttt{High-freq details}}
        \eIf{$l = 0$}{
            $z_h[c] \leftarrow z_{\text{high}}$
        }{
            $z_h[c] \leftarrow z_h[c] + \texttt{Upsample}(z_{\text{high}})$
        }
        $z_{\text{curr}} \leftarrow \texttt{AvgPool}(z_{\text{blur}})$ \tcp*[r]{\texttt{Downsample}}
    }
}
\Return $z_h$
\caption{High-frequency Extraction via Laplacian Pyramid}
\label{alg:laplacian}
\end{algorithm}



\subsection{Analysis of Pyramid Level Selection}
We conduct experiments with different numbers of pyramid levels (L = 1,2,3) to analyze the effectiveness of our Laplacian decomposition. As shown in \cref{tab:pyramid_levels}, using two-level decomposition (L = 2) achieves the best performance across all metrics. Adding more levels not only increases computational complexity but also leads to performance degradation, as the additional levels introduce more low-frequency information that can adversely affect the harmonious generation of color checkers.
\section{Additional Qualitative Results}~\label{sec:Qualitative}

\subsection{Benchmark Datasets}
On the NUS-8 dataset~\cite{cheng2014illuminant} and Gehler dataset~\cite{4587765}, we utilize the original mask locations to place fixed-size neutral color checkers in our experiments. The results \cref{fig:suppl_NUS_demo} and \cref{fig:suppl_gehler_demo} demonstrate our method's ability to generate structurally coherent color checkers that naturally blend with the scene while accurately reflecting local illumination conditions, enabling effective color cast removal across diverse lighting scenarios.

\subsection{In-the-wild Images}
For in-the-wild scenes, we adopt a center-aligned placement strategy to address camera vignetting effects, which can impact color accuracy near image edges. This consistent central positioning not only mitigates lens shading issues but also demonstrates our method's flexibility in color checker placement. The results \cref{fig:suppl_inthewild_demo} validate our approach's robustness in practical photography applications, showing consistent performance in white balance correction despite the fixed central placement strategy.
\subsection{Interactive Visualization}
We provide an interactive HTML interface that visualizes results with color checkers placed at different locations within scenes. The visualization demonstrates that our method produces accurate outputs with minimal variation across different placement positions. The results show that the estimated illumination values consistently cluster near the ground truth target regardless of the checker's position, confirming our method's reliability and position-independence in illumination estimation.

\begin{figure}[t]
    \centering
    \includegraphics[width=1\linewidth]{figures/fail_case.pdf}
    \vspace{-7mm}
    \caption{\textbf{Failure cases.} Our approach struggles when there is a significant mismatch between the illumination of the original color checker and the ambient lighting in the scene.}
    \label{fig:failure}
\end{figure}
% \section{Cross-validation Results}~\label{sec:cross_validation}

% For evaluation, we follow the standard protocol of three-fold cross-validation on both the NUS-8 Camera dataset \cite{Cheng:14} and the Gehler dataset \cite{4587765}. The results are presented in \cref{table:cross_nus8} and \cref{table:cross_gehler}.
% As FC4 notes, \emph{many scenes have multiple light sources with differences up to 10 degrees, so further reducing an error already under 2 degrees may not be a strong comparison.} Instead, our method \textbf{inpaints physically plausible color checkers}—a different strategy than directly optimizing for a ground truth illumination RGB, which can yield lower single-camera performance but enables \textbf{strong cross-camera generalization}, as shown in our cross-dataset experiments.
% Following the evaluation protocol in \cite{afifi2019sensor}, different from traditional learning-based methods that use three-fold cross-validation, we evaluate our method by completely excluding the testing camera's data from the training set. During each evaluation, we train our model using images from all cameras except the target testing camera, ensuring our model has never seen examples from the test camera during training. Despite this challenging setup, our method shows competitive performance, as shown in ~\cref{table:cross_nus8} and \cref{table:cross_gehler}, particularly in the worst 25\% cases where we outperform several sensor-specific methods that were trained with the test camera's data, demonstrating the strong generalization capability of our approach across different camera sensors.


\begin{table}[t]
\centering
\small
\caption{Analysis of different pyramid levels in Laplacian composition. Results are trained on the NUS-8 dataset~\cite{cheng2014illuminant} and tested on Gehler dataset~\cite{4587765} .}
\begin{tabular}{lccccc}
\toprule
% Pyramid & \multicolumn{5}{c}{Angular Error} \\
Level & Mean & Median & Best-25\% & Worst-25\% &\\
\midrule
L = 1 & 3.53 & 3.27 & 1.48 & 6.03\\
L = 2 & \textbf{2.35} & \textbf{2.02} & \textbf{0.78} & \textbf{4.57}  \\
L = 3 & 3.16 & 2.83 & 1.25 & 5.62\\
\bottomrule
\end{tabular}
\label{tab:pyramid_levels}
\end{table}
% \begin{table}
    \centering
    \small
    \caption{\textbf{Three-fold cross-validation on NUS-8 dataset~\cite{cheng2014illuminant}.}}
    \label{tab:cross_validation_nus8}
    \vspace{-3mm}
    \resizebox{\columnwidth}{!}{
    \begin{tabular}{l|cccccc}
        \toprule
        NUS-8 dataset~\cite{cheng2014illuminant}  & Mean & Med. & Tri. & Best 25\% & Worst 25\%\\
        \midrule
        CCC \cite{barron2015convolutional} & 2.38 & 1.48 & 1.69 & 0.45 & 5.85  \\
        AlexNet-FC4 \cite{hu2017fc4} & 2.12 & 1.53 & 1.67 & 0.48 & 4.78 \\
        FFCC \cite{barron2017fast}& 1.99 & \cellcolor{red!25}1.31 & \cellcolor{orange!25}1.43 & \cellcolor{red!25}0.35 & 4.75 \\
        C$^4_{\text{SqueezeNet-FC4}}$~\cite{yu2020cascading} & \cellcolor{orange!25}1.96 & \cellcolor{orange!25}1.42 & 1.53 & 0.48 & 4.40 \\
        CLCC \cite{lo2021clcc} & \cellcolor{red!25}1.84 & \cellcolor{red!25}1.31 & \cellcolor{red!25}1.42 & \cellcolor{orange!25}0.41 & \cellcolor{red!25}4.20 \\
        Ours &2.10 & 1.52 & 1.69 & 0.56 & \cellcolor{orange!25}4.38 \\
        \bottomrule
    \end{tabular}
    }
\end{table}
% \begin{table}
%     \centering
%     \small
%     \caption{Result on NUS-8 Camera dataset, with mean angular error in degrees.}
%     \label{table:cross_nus8}
%     \resizebox{\columnwidth}{!}{
%     \begin{tabular}{l|cccccc}
%         \toprule
%         NUS-8 Camera dataset  & Mean & Med. & Tri. & Best & Worst \\
%         & & & & 25\% & 25\%\\
%         \midrule
%         CCC \cite{barron2015convolutional} & 2.38 & 1.48 & 1.69 & 0.45 & 5.85 \\
%         AlexNet-FC4 \cite{hu2017fc4} & 2.12 & 1.53 & 1.67 & 0.48 & 4.78 \\
%         FFCC \cite{barron2017fast}& 1.99 & \textbf{1.31} & 1.43 & \textbf{0.35} & 4.75 \\
%         C$^4_{\text{SqueezeNet-FC4}}$~\cite{yu2020cascading} & 1.96 & 1.42 & 1.53 & 0.48 &4.40 \\
%         CLCC \cite{lo2021clcc} & \textbf{1.84} & \textbf{1.31} & \textbf{1.42} & 0.41 & \textbf{4.2} \\
%         Ours & 2.57 & 2.36 & 2.40 & 0.89 & 4.52  \\
%         \bottomrule
%     \end{tabular}
%     }
% \end{table}
% 
\begin{table}
    \centering
    \small
    \vspace{-1mm}
    \caption{\textbf{Three-fold cross-validation on Gehler dataset~\cite{4587765}.}}
    \label{tab:cross_validation_gehler}
    \vspace{-3mm}
    \resizebox{\columnwidth}{!}{
    \begin{tabular}{l|ccccc}
        \toprule
        Gehler dataset ~\cite{4587765}  & Mean & Med. & Tri. & Best 25\% & Worst 25\%\\
        \midrule
        CCC \cite{barron2015convolutional}& 1.95 & 1.22 & 1.38 & 0.35 & 4.76  \\
        SqueezeNet-FC4 \cite{hu2017fc4} & 1.65 & 1.18 & 1.27 & 0.38 & 3.78  \\
        FFCC \cite{barron2017fast} & 1.61 & \cellcolor{red!25}{0.86} & \cellcolor{red!25}1.02 & \cellcolor{red!25}{0.23} & 4.27  \\
        C$^4_{\text{SqueezeNet-FC4}}$~\cite{yu2020cascading} & \cellcolor{red!25}{1.35} & \cellcolor{orange!25}0.88 & \cellcolor{red!25}{0.99} & 0.28 & \cellcolor{red!25}{3.21} \\
        CLCC\cite{lo2021clcc}  &  \cellcolor{orange!25}1.44 & 0.92 & 1.04 & \cellcolor{orange!25}0.27 & 3.48  \\
        Ours & 1.91 & 1.80 & 1.84 & 0.60 &  \cellcolor{orange!25}{3.46}\\
        \bottomrule
    \end{tabular}
    }
\end{table}


% \begin{table}
%     \centering
%     \small
%     \caption{Result on Gehler dataset, with mean angular error in degrees.}
%     \label{table:cross_gehler}
%     \resizebox{\columnwidth}{!}{
%     \begin{tabular}{l|ccccc}
%         \toprule
%         Gehler dataset  & Mean & Med. & Tri. & Best & Worst \\
%         & & & & 25\% & 25\% \\
%         \midrule
%         CCC \cite{barron2015convolutional}& 1.95 & 1.22 & 1.38 & 0.35 & 4.76  \\
%         SqueezeNet-FC4 \cite{hu2017fc4} & 1.65 & 1.18 & 1.27 & 0.38 & 3.78  \\
%         FFCC \cite{barron2017fast} & 1.61 & \textbf{0.86} & 1.02 & \textbf{0.23} & 4.27  \\
%         C$^4_{\text{SqueezeNet-FC4}}$~\cite{yu2020cascading} & \textbf{1.35} & 0.88 & \textbf{0.99} & 0.28 & \textbf{3.21} \\
%         CLCC\cite{lo2021clcc}  & 1.44 & 0.92 & 1.04 & 0.27 & 3.48  \\
%         Ours & 2.23 & 2.25 & 2.20 & 0.92 & 3.65 \\
%         \bottomrule
%     \end{tabular}
%     }
% \end{table}


\begin{figure*}[t]
    \centering
    \includegraphics[width=0.8\linewidth]{figures/suppl_NUS_demo.pdf}
    \caption{Qualitative results for the NUS-8 dataset~\cite{cheng2014illuminant}.}
    \label{fig:suppl_NUS_demo}
\end{figure*}
\begin{figure*}[t]
    \centering
    \includegraphics[width=0.8\linewidth]{figures/suppl_gehler_demo.pdf}
    \caption{Qualitative results for the Gehler dataset~\cite{4587765}. }
    \label{fig:suppl_gehler_demo}
\end{figure*}
\begin{figure*}[t]
    \centering
    \includegraphics[width=0.8\linewidth]{figures/suppl_inthewild_demo.pdf}
    \caption{Qualitative results for in-the-wild images with center-placed color checkers.}
    \label{fig:suppl_inthewild_demo}
\end{figure*}

\section{Limitations}
As shown in \cref{fig:failure}, our method struggles when there is a significant mismatch between the inpainted color checker and the scene's ambient lighting. This typically occurs in challenging scenarios with multiple strong light sources of different colors or complex spatially-varying illumination. While diffusion models provide strong image priors, they sometimes prioritize visual plausibility over physical accuracy, especially in extreme lighting conditions.

Our approach also shows sensitivity to dataset size, similar to personalization effects observed in DreamBooth~\cite{ruiz2022dreambooth}. For datasets with limited samples, we need to crop smaller mask regions to ensure the model can effectively learn the color checker's appearance and structure. In our experiments, we found that when the training dataset is extremely small, the model generates color checkers with unexpected appearances and distorted structures, preventing accurate color extraction for illumination estimation. This limitation suggests potential future directions for improving our method through more efficient learning strategies or additional data augmentation techniques to better handle scenarios with limited training data.

\end{document}
\endinput
%%
%% End of file `sample-authordraft.tex'.
