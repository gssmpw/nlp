%%
%% This is file `sample-manuscript.tex',
%% generated with the docstrip utility.
%%
%% The original source files were:
%%
%% samples.dtx  (with options: `manuscript')
%% 
%% IMPORTANT NOTICE:
%% 
%% For the copyright see the source file.
%% 
%% Any modified versions of this file must be renamed
%% with new filenames distinct from sample-manuscript.tex.
%% 
%% For distribution of the original source see the terms
%% for copying and modification in the file samples.dtx.
%% 
%% This generated file may be distributed as long as the
%% original source files, as listed above, are part of the
%% same distribution. (The sources need not necessarily be
%% in the same archive or directory.)
%%  
%% Commands for TeXCount
%TC:macro \cite [option:text,text]
%TC:macro \citep [option:text,text]
%TC:macro \citet [option:text,text]
%TC:envir table 0 1
%TC:envir table* 0 1
%TC:envir tabular [ignore] word
%TC:envir displaymath 0 
%TC:envir math 0 word
%TC:envir comment 0 0
%%
%%
%% The first command in your LaTeX source must be the \documentclass command.
%%%% Small single column format, used for CIE, CSUR, DTRAP, JACM, JDIQ, JEA, JERIC, JETC, PACMCGIT, TAAS, TACCESS, TACO, TALG, TALLIP (formerly TALIP), TCPS, TDSCI, TEAC, TECS, TELO, THRI, TIIS, TIOT, TISSEC, TIST, TKDD, TMIS, TOCE, TOCHI, TOCL, TOCS, TOCT, TODAES, TODS, TOIS, TOIT, TOMACS, TOMM (formerly TOMCCAP), TOMPECS, TOMS, TOPC, TOPLAS, TOPS, TOS, TOSEM, TOSN, TQC, TRETS, TSAS, TSC, TSLP, TWEB.
% \documentclass[acmsmall]{acmart}

%%%% Large single column format, used for IMWUT, JOCCH, PACMPL, POMACS, TAP, PACMHCI
% \documentclass[acmlarge,screen]{acmart}

%%%% Large double column format, used for TOG
% \documentclass[acmtog, authorversion]{acmart}

%%%% Generic manuscript mode, required for submission
%%%% and peer review
% \documentclass[manuscript,review,anonymous]{acmart}
\documentclass[sigconf]{acmart}
%% Fonts used in the template cannot be substituted; margin 
%% adjustments are not allowed.
%%
%% \BibTeX command to typeset BibTeX logo in the docs
\AtBeginDocument{%
  \providecommand\BibTeX{{%
    \normalfont B\kern-0.5em{\scshape i\kern-0.25em b}\kern-0.8em\TeX}}}

%% Rights management information.  This information is sent to you
%% when you complete the rights form.  These commands have SAMPLE
%% values in them; it is your responsibility as an author to replace
%% the commands and values with those provided to you when you
%% complete the rights form.
% \setcopyright{acmcopyright}
% \copyrightyear{2024}
% \acmYear{2024}
% \acmDOI{XXXXXXX.XXXXXXX}

%% These commands are for a PROCEEDINGS abstract or paper.
% \acmConference[]{}{}{}
%
%  Uncomment \acmBooktitle if th title of the proceedings is different
%  from ``Proceedings of ...''!
%
% \acmBooktitle{} 
% \acmPrice{}
% \acmISBN{}


\copyrightyear{2025}
\acmYear{2025}
\makeatletter
\def\@ACM@copyright@check@cc{}
\makeatother
\setcopyright{cc}
\setcctype{by}
\acmConference[CHI '25]{CHI Conference on Human Factors in Computing Systems}{April 26-May 1, 2025}{Yokohama, Japan}
\acmBooktitle{CHI Conference on Human Factors in Computing Systems (CHI '25), April 26-May 1, 2025, Yokohama, Japan}\acmDOI{10.1145/3706598.3713392}
\acmISBN{979-8-4007-1394-1/25/04}


%%
%% Submission ID.
%% Use this when submitting an article to a sponsored event. You'll
%% receive a unique submission ID from the organizers
%% of the event, and this ID should be used as the parameter to this command.
%%\acmSubmissionID{123-A56-BU3}

%%
%% For managing citations, it is recommended to use bibliography
%% files in BibTeX format.
%%
%% You can then either use BibTeX with the ACM-Reference-Format style,
%% or BibLaTeX with the acmnumeric or acmauthoryear sytles, that include
%% support for advanced citation of software artefact from the
%% biblatex-software package, also separately available on CTAN.
%%
%% Look at the sample-*-biblatex.tex files for templates showcasing
%% the biblatex styles.
%%

%%
%% The majority of ACM publications use numbered citations and
%% references.  The command \citestyle{authoryear} switches to the
%% "author year" style.
%%
%% If you are preparing content for an event
%% sponsored by ACM SIGGRAPH, you must use the "author year" style of
%% citations and references.
%% Uncommenting
%% the next command will enable that style.
%%\citestyle{acmauthoryear}

\usepackage{float}
\usepackage{subcaption}
\usepackage{multirow}
\usepackage{makecell}
% \usepackage{soul}
% \usepackage{ulem}


% \newcommand{\zhipeng}[1]{\textcolor{cyan}{ZP: #1}}
% \newcommand{\yj}[1]{\textcolor{teal}{[YJ: #1]}}
% \newcommand\change[1]{{\textcolor{blue}{#1}}}
% \newcommand \change[1]{{\textcolor{black}{#1}}}
% \newcommand\delete[1]{\textcolor{red}{\sout{#1}}}
% \newcommand\delete[1]{}

%%
%% end of the preamble, start of the body of the document source.
\begin{document}


%%
%% The "title" command has an optional parameter,
%% allowing the author to define a "short title" to be used in page headers.
% \title{
% \change{Modeling the Impact of Visual Stimuli on Redirection Noticeability with Gaze Behavior in Virtual Reality}
% \delete{Modelling Effects of Visual Attention on Noticeability of Body-Avatar Offsets in Virtual Reality}
% }
\title{Modeling the Impact of Visual Stimuli on Redirection Noticeability with Gaze Behavior in Virtual Reality}

%%
%% The "author" command and its associated commands are used to define
%% the authors and their affiliations.
%% Of note is the shared affiliation of the first two authors, and the
%% "authornote" and "authornotemark" commands
%% used to denote shared contribution to the research.
\author{Zhipeng Li}
\affiliation{%
  \institution{Tsinghua Univeristy}
  \city{Beijing}
  \country{China}
}
\affiliation{%
  \institution{ETH Zürich}
  \city{Zürich}
  \country{Switzerland}
}
\email{zhipeng.li@inf.ethz.ch}

\author{Yishu Ji}
\affiliation{%
  \institution{Georgia Institute of Technology}
  \city{Atlanta}
  \state{Georgia}
  \country{USA}
}
\email{yji329@gatech.edu}

\author{Ruijia Chen}
\affiliation{%
  \institution{University of Wisconsin-Madison}
  \city{Madison}
  \state{Wisconsin}
  \country{USA}
}
\email{ruijia.chen@wisc.edu}

\author{Tianqi Liu}
\affiliation{%
  \institution{Cornell University}
  \city{Ithaca}
  \state{New York}
  \country{USA}
}
\email{tl889@cornell.edu}

\author{Yuntao Wang}
\authornote{Corresponding author}
\affiliation{%
  \institution{Key Laboratory of Pervasive Computing, Ministry of Education, Tsinghua University}
  \city{Beijing}
  \country{China}
}
\email{yuntaowang@tsinghua.edu.cn}

\author{Yuanchun Shi}
\affiliation{%
  \institution{Tsinghua University}
  \city{Beijing}
  \country{China}
}
\email{shiyc@tsinghua.edu.cn}

\author{Yukang Yan}
\affiliation{%
  \institution{University of Rochester}
  \city{Rochester}
  \state{New York}
  \country{USA}
}
\email{yanyukanglwy@gmail.com}

% \author{Ben Trovato}
% \authornote{Both authors contributed equally to this research.}
% \email{trovato@corporation.com}
% \orcid{1234-5678-9012}
% \author{G.K.M. Tobin}
% \authornotemark[1]
% \email{webmaster@marysville-ohio.com}
% \affiliation{%
%   \institution{Institute for Clarity in Documentation}
%   \streetaddress{P.O. Box 1212}
%   \city{Dublin}
%   \state{Ohio}
%   \country{USA}
%   \postcode{43017-6221}
% }

% \author{Lars Th{\o}rv{\"a}ld}
% \affiliation{%
%   \institution{The Th{\o}rv{\"a}ld Group}
%   \streetaddress{1 Th{\o}rv{\"a}ld Circle}
%   \city{Hekla}
%   \country{Iceland}}
% \email{larst@affiliation.org}

% \author{Valerie B\'eranger}
% \affiliation{%
%   \institution{Inria Paris-Rocquencourt}
%   \city{Rocquencourt}
%   \country{France}
% }

% \author{Aparna Patel}
% \affiliation{%
%  \institution{Rajiv Gandhi University}
%  \streetaddress{Rono-Hills}
%  \city{Doimukh}
%  \state{Arunachal Pradesh}
%  \country{India}}

% \author{Huifen Chan}
% \affiliation{%
%   \institution{Tsinghua University}
%   \streetaddress{30 Shuangqing Rd}
%   \city{Haidian Qu}
%   \state{Beijing Shi}
%   \country{China}}

% \author{Charles Palmer}
% \affiliation{%
%   \institution{Palmer Research Laboratories}
%   \streetaddress{8600 Datapoint Drive}
%   \city{San Antonio}
%   \state{Texas}
%   \country{USA}
%   \postcode{78229}}
% \email{cpalmer@prl.com}

% \author{John Smith}
% \affiliation{%
%   \institution{The Th{\o}rv{\"a}ld Group}
%   \streetaddress{1 Th{\o}rv{\"a}ld Circle}
%   \city{Hekla}
%   \country{Iceland}}
% \email{jsmith@affiliation.org}

% \author{Julius P. Kumquat}
% \affiliation{%
%   \institution{The Kumquat Consortium}
%   \city{New York}
%   \country{USA}}
% \email{jpkumquat@consortium.net}

%%
%% By default, the full list of authors will be used in the page
%% headers. Often, this list is too long, and will overlap
%% other information printed in the page headers. This command allows
%% the author to define a more concise list
%% of authors' names for this purpose.
\renewcommand{\shortauthors}{Li, et al.}
%%
%% The abstract is a short summary of the work to be presented in the
%% article.
\begin{abstract}

% \textbf{150}:
% Users embody virtual avatars that mirror their physical movements in Virtual Reality. We propose to measure how the user's visual attention to the avatar impacts the probability of them noticing an offset applied on the avatar’s body movement with respect to their own motion. We conduct two user studies applying a dual-task paradigm to control and alter their visual attention and record noticing probability. Results confirm that more visual attention attracted away from the avatar leads to lower the noticeability of the offset. In addition, we identified the behavioral pattern of users' gaze can serve as an effective indicator of noticeability. Based on the findings, we implement a regression model that predicts noticeability taking the user's gaze data as input. We evaluate the extendability of the model on unseen visual attractions. Results show that the model achieves an MSE of 0.012 with new participants exposed to new visual attractions. 
While users could embody virtual avatars that mirror their physical movements in Virtual Reality, these avatars' motions can be redirected to enable novel interactions.
Excessive redirection, however, could break the user's sense of embodiment due to perceptual conflicts between vision and proprioception. 
While prior work focused on avatar-related factors influencing the noticeability of redirection, we investigate how the visual stimuli in the surrounding virtual environment affect user behavior and, in turn, the noticeability of redirection.
Given the wide variety of different types of visual stimuli and their tendency to elicit varying individual reactions, 
we propose to use users' gaze behavior as an indicator of their response to the stimuli and model the noticeability of redirection.
We conducted two user studies to collect users' gaze behavior and noticeability, investigating the relationship between them and identifying the most effective gaze behavior features for predicting noticeability. 
Based on the data, we developed a regression model that takes users' gaze behavior as input and outputs the noticeability of redirection. 
We then conducted an evaluation study to test our model on unseen visual stimuli, achieving an accuracy of 0.012 MSE. 
We further implemented an adaptive redirection technique and conducted a preliminary study to evaluate its effectiveness with complex visual stimuli in two applications. 
The results indicated that participants experienced less physical demanding and a stronger sense of body ownership when using our adaptive technique, demonstrating the potential of our model to support real-world use cases.
\end{abstract}

%%
%% The code below is generated by the tool at http://dl.acm.org/ccs.cfm.
%% Please copy and paste the code instead of the example below.
%%
\begin{CCSXML}
<ccs2012>
   <concept>
       <concept_id>10003120.10003121.10003128.10011755</concept_id>
       <concept_desc>Human-centered computing~Gestural input</concept_desc>
       <concept_significance>300</concept_significance>
       </concept>
   <concept>
       <concept_id>10003120.10003121.10003126</concept_id>
       <concept_desc>Human-centered computing~HCI theory, concepts and models</concept_desc>
       <concept_significance>300</concept_significance>
       </concept>
 </ccs2012>
\end{CCSXML}

\ccsdesc[300]{Human-centered computing~Gestural input}
\ccsdesc[300]{Human-centered computing~HCI theory, concepts and models}

%%
%% Keywords. The author(s) should pick words that accurately describe
%% the work being presented. Separate the keywords with commas.
\keywords{Virtual Reality, visual attention, noticeability, embodiment}

%% A "teaser" image appears between the author and affiliation
%% information and the body of the document, and typically spans the
%% page.
\begin{teaserfigure}
  \includegraphics[width=\textwidth]{figures/teaser/teaser.pdf}
  \caption{
    In this paper, we explored the impact of visual stimuli on the noticeability of redirection in Virtual Reality. 
    We developed a computational model that takes users' gaze behavior as input and predicts the noticeability of redirection under different visual stimuli.
    Using this model, we implemented an adaptive redirection technique, demonstrated in a boxing training scenario:
    Left: When the opponent approaches and attacks, visual stimuli are intense, making the redirection unnoticable.
    Middle: As the opponent retreats, visual stimuli decrease that causes the noticeability becoming higher during the interaction.
    Right: When the model detects the change in noticeability, the system dynamically adjusts the redirection magnitude, ensuring it remains unnoticed.
  }
  \label{fig:teaser}
\end{teaserfigure}

% \received{20 February 2007}
% \received[revised]{12 March 2009}
% \received[accepted]{5 June 2009}

%%
%% This command processes the author and affiliation and title
%% information and builds the first part of the formatted document.
\maketitle
\begin{figure}[ht]
    \centering
    \includegraphics[width=0.8\linewidth]{graphs/greater_than_naive.pdf}
    \vspace{0.5cm}
    \includegraphics[width=0.8\linewidth]{graphs/p1_bottom.png}
    \vspace{-5pt}
    \caption{\textcolor{positional}{Positional} vs.\ \textcolor{nonpositional}{non-positional} circuits. In a \textcolor{nonpositional}{non-positional} circuit, the same edges must be included at all positions. A \textcolor{positional}{positional} circuit can distinguish between the same edge at different positions. This specificity yields better trade-offs between circuit size and faithfulness. It can also increase both precision and recall.}
    \label{fig:p1}
    \vspace{-5pt}
\end{figure}

\section{Introduction}

\looseness=-1
A primary goal of interpretability research is to characterize the internal mechanisms in language models (LMs) and other NLP models. 
A core approach in this area is \textbf{circuit discovery}---identifying the minimal subgraph within the model's computation graph that performs a specific task \citep{olah2021framework,olah-mech}.
Typically, the nodes of a circuit represent model components (e.g., attention heads, neurons, or layers).
While manual circuit discovery methods can yield position-specific insights \citep{wanginterpretability,goldowskydill2023localizingmodelbehaviorpath}, \emph{automatic methods often overlook positional information}, treating components as uniformly relevant across all input token positions \citep{conmytowards,syed2023attribution}. 
For instance, if an attention head is included in a circuit, it is assumed to contribute equally to the computation for every position in the input sequence.
The assumption that circuits are position-invariant ignores the fact that different positions often require distinct computations.
By ignoring positions, current methods limit their ability to capture mechanisms that operate across positions, such as interactions between attention heads across positions.

In this study, we start by demonstrating that positional agnosticism is a significant limitation (\S\ref{sec:motivating}). Then, to address these limitations, we introduce a new approach: position-aware edge attribution patching (PEAP; \S\ref{sec:full_circ_discovery}; Figure~\ref{fig:p1}). Current approaches  assume that if an edge is in a circuit, then the same edge will be in the circuit at all positions, thus leading to low precision. It is also assumed that an edge's importance should be aggregated across positions before deciding whether it should be included in the circuit; this can lead to cancellation effects, and thus low recall. PEAP instead allows us to compute the importance of cross-positional edges, and separately evaluates edge importance at each position. We show that this leads to smaller and more accurate circuits; see Figure~\ref{fig:p1}.

Incorporating positional information into circuit discovery is straightforward when inputs have the same length and structure across examples.

However, realistic datasets are not nearly this templatic.
How, then, can we incorporate positional information into automatic circuit discovery?
To address this challenge, we propose \textbf{schemas} (\S\ref{sec:schema}). 
Schemas assign semantic labels to spans of tokens, enabling information aggregation across examples even when the spans differ in length.

For example, in the input ``The \textcolor{positional}{war} lasted from 1453 to 14\underline{\hspace{1em}},'' the span ``\textcolor{positional}{war}'' could be labeled as ``\emph{Subject}''.
This enables handling spans with varying lengths: the phrase ``\textcolor{positional}{Black Plague}'' in another example can be treated as a single positional span with the same role as ``\textcolor{positional}{war}''.
In experiments with two LMs and three tasks, we find that circuits discovered using schemas achieve a better trade-off between circuit size and faithfulness to the model's behavior than position-agnostic circuits.
Importantly, position-aware circuits offer a more precise representation of the underlying mechanisms, providing a more concise foundation for mechanistic explanations.

We also present a fully automated pipeline for schema generation and application (\S\ref{sec:schema-generation}) using large language models (LLMs). 
We evaluate the quality of the generated schemas and their utility in discovering position-aware circuits (\S\ref{sec:schema-eval}).
Notably, circuits derived using automatically generated and applied schemas achieve comparable faithfulness scores to circuits discovered with human-designed and manually applied schemas.

We summarize our contributions as follows:
\begin{itemize}[noitemsep,leftmargin=*,topsep=1pt,parsep=1pt]
    \item Introduce a position-aware circuit discovery method, which obtains better faithfulness than position-agnostic discovery.  
    \item Introduce dataset schemas,  facilitating positional circuit discovery in more naturalistic settings. 
    \item Develop an automated schema generation and application pipeline with LLMs, yielding schemas that are comparable to manually-annotated ones.
\end{itemize}

\section{Related Work}

\subsection{Penetration Depth Computation}

The computation of penetration depth often utilizes the Minkowski sum, a well-regarded algorithm documented in Dobkin et al.'s work~\cite{dobkin1993computing}.
This method shows high efficacy for convex shapes, where the simplicity of the objects allows for accurate and computationally efficient penetration depth calculations~\cite{dobkin1993computing,varadhan2004accurate,hachenberger2009exact}.
However, applying this algorithm to concave shapes significantly increases computational complexity.  
As a result, research has focused on developing methods to approximate penetration depth more efficiently for these shapes~\cite{cameron1997enhancing,bergen1999fast,lien2010simple,je2012polydepth}.  

Beyond the Minkowski sum, other methods have been explored, including techniques such as utilizing distance fields or the Hausdorff distance for penetration depth calculations~\cite{fisher2001fast,sud2006fast,SIG09HIST}.

Tang et al.\cite{SIG09HIST} devised an efficient algorithm for calculating the Hausdorff distance between two objects within a given error bound.
They also demonstrated that the proposed algorithm can accelerate penetration depth computation by focusing on the Hausdorff distance in overlapping regions of objects.
Building upon Tang et al.'s method, Zheng et al.\cite{zheng2022economic} improved performance using a BVH-based framework with a four-point strategy.
This method has achieved a performance improvement of up to 20 times compared to Tang et al.'s technique~\cite{SIG09HIST}.
\revision{A common feature of these works, known as the culling-based method, is computing bounds for the Hausdorff distance and reducing the search space.}

\revision{Although culling-based methods have demonstrated significant performance gains, they face challenges in leveraging parallel hardware.  
Updating and sharing bounds require synchronization, which is not well-suited for massively parallel processing architectures such as GPUs.}

\revision{In this work, we propose a GPU-based penetration depth algorithm that specifically accelerates two key processes using RT core technology:  
(1) detecting the overlapping volume and (2) calculating the Hausdorff distance.  
To highlight the effectiveness of our approach, we also implemented a CPU-based penetration depth algorithm based on Tang et al.~\cite{SIG09HIST} and Zheng et al.~\cite{zheng2022economic} for performance comparison.}

%In this work, we propose a GPU-based penetration depth algorithm, specifically accelerating two key processes with RT core technology:  
%first, detecting the overlapping volume; and second, calculating the Hausdorff distance.  
%To highlight our method's effectiveness, we also implemented a CPU-based penetration depth algorithm based on Tang et al.~\cite{SIG09HIST} and Zheng et al.~\cite{zheng2022economic} for performance comparison.  

%utilize a Hausdorff distance-based method for penetration depth calculation, accelerating two key processes with RT core technology: 

%A notable development in this area is the work of Tang et al., who devised algorithms for the rapid calculation of the Hausdorff distance between two objects~\cite{SIG09HIST}.
%Their approach is geared towards efficient penetration depth calculation by focusing on the Hausdorff distance in overlapping object regions.


%One of the algorithms for calculating penetration depth is the Minkowski sum.\cite{dobkin1993computing} The Minkowski sum is useful to compute penetration depth between two convex objects because they have a simple shape so the Minkowski sum can calculate accurate penetration depth with low computational complexity~\cite{dobkin1993computing,varadhan2004accurate,hachenberger2009exact}.
%However, applying the Minkowski sum in cases involving concave objects is challenging due to higher computational complexity. As a result, prior research has focused on quickly computing an approximate penetration depth in these scenarios~\cite{cameron1997enhancing,bergen1999fast,lien2010simple,je2012polydepth}.

%Instead of the Minkowski sum method, there have also been attempts to calculate the penetration depth based on the distance field or the vertices that make up the objects~\cite{fisher2001fast,sud2006fast,SIG09HIST}. Tang et al.~\cite{SIG09HIST} proposed the algorithms that compute the Hausdorff distance between two objects quickly and showed that can be computed penetration depth to fast by calculating the Hausdorff distance for the overlapping area of two objects.

%In this paper, the proposed method is based on Tang's methods~\cite{SIG09HIST}, and then partially divided into steps detecting overlapping volume step and the Hausdorff distance step. These two steps accelerated with RT core.

\subsection{Ray-Tracing Core-Based Acceleration}

\revision{Recent advancements in GPU technology have led to the integration of dedicated ray-tracing cores (RT cores), enabling hardware-accelerated ray tracing.
These cores optimize intersection checks between rays and objects, allowing for efficient ray-bounding box and ray-triangle intersection tests.
To utilize RT cores, various frameworks such as DXR, OptiX~\cite{parker2010optix}, and Vulkan have been developed.
RT cores primarily accelerate ray intersection tasks by efficiently traversing acceleration hierarchies.}

%The Ray-Tracing Core (RT-core) is NVIDIA’s specialized hardware for accelerating ray tracing.
%Integrated into RTX GPUs like the GeForce RTX series

%\revision{Notably, OptiX~\cite{parker2010optix} is an NVIDIA-supported SDK.
%The ray-tracing core primarily facilitates two tasks: building an acceleration hierarchy and executing ray intersection tasks with traversal.}
%OptiX operates by launching a CUDA kernel and invoking a ray generation ($ray_{gen}$) shader.
%Each CUDA core thread makes requests to the ray-tracing core, which then executes appropriate shaders like intersection ($IS$), miss($miss_{hit}$), closest hit($closest$), and any hit($any_{hit}$).
%Consequently, OptiX enables access to the results of ray-primitive intersection tests.

While the core purpose of ray-tracing cores is to expedite ray tracing, recent studies have explored their application beyond this traditional scope~\cite{wald2019rtx,zhu2022rtnn,thoman2022multi,nagarajan2023rt,meneses2023accelerating,morrical2023attribute}.
Wald et al.~\cite{wald2019rtx} addressed the problem of locating points within tetrahedra using ray-tracing cores.
Zhu et al.~\cite{zhu2022rtnn} introduced a K-Nearest Neighbor (K-NN) algorithm utilizing ray-tracing cores, achieving performance improvements of 2.2 to 65.0 times compared to previous GPU-based nearest neighbor search algorithms.
Thoman et al.~\cite{thoman2022multi} employed RT cores for Room Impulse Response (RIR) simulation.
Nagarajan et al.~\cite{nagarajan2023rt} implemented RT core-based DBSCAN clustering, reporting up to 4 times higher performance enhancement.
Meneses et al.~\cite{meneses2023accelerating} proposed RT core-based Range Minimum Query (RMQ) algorithms, yielding performance up to 2.3 times faster than existing RMQ methods.

\revision{
For collision detection between objects, one of the fundamental proximity queries, researchers have explored ray-tracing approaches even before the introduction of RT-core technology.
Hermann et al.\cite{hermann2008ray} proposed ray-tracing-based collision detection methods for deformable bodies.
Youngjun et al.\cite{kim2010mesh} applied Hermann's idea to medical simulation.
Lehericey et al.\cite{lehericey2015gpu} introduced GPU ray-traced collision detection algorithms for cloth simulation.
Recently, these approaches have been extended to utilize RT cores, as demonstrated by Sui et al.\cite{sui2024hardware}, who proposed discrete and continuous collision detection algorithms using ray-tracing cores.
Unlike these works, which focus on determining when and where collisions occur, our work focuses on calculating penetration depth.
}

In line with these advancements, this study uniquely applies RT-core technology to compute penetration depth, diverging from traditional ray-tracing applications and thereby contributing a novel approach to this field.

%\subsection{Collision detection with Ray-tracing}

%\YW{There have been attempts to apply the ray tracing approaches for collision detection~\cite{hermann2008ray, kim2010mesh, lehericey2015gpu}. Hermann et al~\cite{hermann2008ray} proposed ray tracing collision detection methods for deformable bodies. Youngjun et al~\cite{kim2010mesh} apply Hermann's idea for Medical simulation. Lehericey et al~\cite{lehericey2015gpu} introduced GPU ray-traced collision detection algorithms for cloth simulation.
%However, these methods proposed deformable objects, not solid- or discrete- objects, and there is no report about the result using ray tracing core yet. Therefore, our research implements the penetration depth algorithm with ray tracing methods and reports the benefit of ray tracing core.}

%\YW{Sui et al~\cite{sui2024hardware} proposed the method for discrete and continuous collision detection with ray tracing core. They generate the ray candidate as much as the edge of the source mesh and investigate the intersections to solve discrete collision detection. And also, to solve continuous collision detection, they build sphere-swept volumes with OptiX B-Spline curves using continuous trajectory points that are pre-computed and trace the ray samely. However, their implementation only considers non-penetrating collision, and because of that reason, there need for other approaches to compute penetration cases.}

%\YW{To address this issue, our approacthe has propose the methods to find penetration surface with RT core (that called RT-PPE). Not only that, our methods report the penetration depth as computing the Hausdorff distance between the penetration surface.}

%Recently, modern GPU embedded ray tracing core for hardware accelerated ray tracing.

%To access the ray tracing core, we can use DXR, OptiX~\cite{parker2010optix}, and Vulkan.
%Above all, OptiX~\cite{parker2010optix} is NVIDIA NVIDIA-supported SDK. The ray tracing core actually works about two tasks. One is a built acceleration hierarchy, and another is ray intersection task with traversal. Therefore OptiX launches one CUDA kernel and called $ray\_gen$ shader. Each CUDA core thread requests to ray tracing core, and then ray tracing core executes a suitable shader such as $IS$, $miss\_hit$, $closest$, $any\_hit$ shader.
%Finally, we can access ray-primitive intersection test results using OptiX shader.

%While the ray tracing core is designed for accelerating ray tracing, recent research tried using the ray tracing core for other purposes~\cite{wald2019rtx,zhu2022rtnn,thoman2022multi,nagarajan2023rt,meneses2023accelerating,morrical2023attribute}.
%%Beyond ray tracing
%Wald et al~\cite{wald2019rtx} solved the point in location of tetrahedron problem using ray tracing cores.
%%RTNN
%Zhu et al~\cite{zhu2022rtnn} proposed K-NN(K-Nearest Neighbor) algorithms using ray tracing cores. They achieved a performance of 2.2-65.0 times faster than prior GPU-based nearest neighbor search algorithms.
%%RIR Simulation
%Thoman et al~\cite{thoman2022multi} utilized the RT core to RIR(Room impulse response) simulation,
%%RT-DBSCAN
%Nagarajan et al~\cite{nagarajan2023rt} implemented DBSCAN clustering with RT core and achieved performance up to 4x times.
%%RTX-RMQ
%Meneses et al~\cite{meneses2023accelerating} proposed RT core-based RMQ(Range minimum query) algorithms, and they got performance up to 2.3x than state-of-the-art RMQ algorithms.

%%
%Similar to prior research, this study is distinguished by utilizing RT-core for computing penetration depth, as opposed to conventional ray tracing problems.



\section{Methodology}
 

\begin{figure*}[t]
    \centering
    \scalebox{0.9}{
    % Define a style for the special labels (for 1, 2, 3)
\begin{tikzpicture}[
    >=Stealth,     % arrow tips
    thick,         % line thickness
    node distance=2.5cm
]

%--- User (person) on the left ---
\node[
  person,
  minimum size=1.25cm,
  label=below:User
] (user) {};

%--- Rectangle in the middle: "1" inside, "operations" below ---
\node[
  draw,
  rectangle,
  minimum width=2.0cm,
  minimum height=1.2cm,
  label=below:operations,
  right=of user
] (box) {\textcircled{1}};

%--- Cylinder (DB) on the right ---
\node[
  draw,
  cylinder,
  shape border rotate=90,
  aspect=1.5,
  minimum height=1.5cm,
  minimum width=1cm,
  label=below:DB,
  right=of box
] (db) {};

%--- Bidirectional arrows with special labels ---
% Arrows between user and box labeled "3"
\draw[->] (user) to[bend left=10] node[above]{\textcircled{3}} (box);
\draw[->] (box)  to[bend left=10]  (user);

% Arrows between box and DB labeled "2"
\draw[->] (box)  to[bend left=10] node[above]{\textcircled{2}} (db);
\draw[->] (db)   to[bend left=10]  (box);

\end{tikzpicture}}
    \caption{An overview of {\sc Demotic} is shown. {\sc Demotic} takes a Verilog instance describing a combinational circuit and parse it into its corresponding probabilistic model described in PyTorch. The embedding layer converts the learnable real-value inputs into probabilities. The $\ell_2$-loss function is calculated in each training iteration and the input variables are updated using GD.}
    \label{fig1}
\end{figure*}


In this section, we describe our differentiable solver/sampler for multi-level digital circuits. While the common approach in solving CircuitSAT typically involves converting the underlying circuit into CNF and employing a SAT solver to find the satisfying solution, we take a completely different approach. Instead, we re-frame the CircuitSAT problem as a multi-output regression task, transforming it into a learning problem. Digital circuits are inherently discrete and non-differentiable. Therefore, we first need to relax the CircuitSAT problem into a continuous form while accurately capturing the structure and behavior of the circuit. To accomplish this, we leverage the probability model of digital gates, as shown in Table \ref{tab1}. This probability model is commonly used in different domains such as stochastic computing \cite{Ardakani2017SC} and dynamic power estimation of digital circuits \cite{harris2010cmos}. We use these probabilities to model each gate in the circuit. The result of such modeling is a differentiable formulation of the underlying circuit that accurately describes its functionality while preserving its spatial structure. Of course, the outcome of this model remains identical to the original circuit in its discrete form for any binary input valuations.




\begin{table}[t]
    \centering
    \caption{Probability model of logic gates.}
    \vspace{-0.25cm}
    \begin{table*}[!th]
\centering
\resizebox{\textwidth}{!}{%
\begin{tabular}{@{}llcccccccccc@{}}
\toprule
& & \multicolumn{2}{c}{\textbf{Intent Detection}} & \multicolumn{2}{c}{\textbf{Topic Mining}} & \multicolumn{2}{c}{\textbf{Domain Discovery}} & \multicolumn{1}{c}{\textbf{Type}} & \multicolumn{1}{c}{\textbf{Emotion}} & \\
\cmidrule(lr){3-4} \cmidrule(lr){5-6} \cmidrule(lr){7-8} \cmidrule(lr){9-9} \cmidrule(lr){10-10}  %\cmidrule(lr){11-11}
\textbf{Model} & \textbf{Method} & \textbf{BANKING} & \textbf{CLINC} & \textbf{Reddit} & \textbf{StackEx} & \textbf{MTOP} & \textbf{CLINC(D)} & \textbf{FewEvent} & \textbf{GoEmotion} & \textbf{AVG} \\ \midrule \midrule
GPT-4o-mini & Standard Prompting & 0.652 & 0.792 & 0.534 & 0.482 & 0.896 & 0.536 & 0.630 & 0.378 & 0.613 \\
& Self-Consistency & 0.666 & 0.802 & 0.586 & 0.494 & 0.902 & 0.530 & 0.640 & 0.382 & 0.625 \\
& TestNUC & 0.712 & 0.858 & 0.614 & 0.528 & 0.936 & 0.544 & 0.674 & 0.410 & 0.660 \\
& \cellcolor{gray!18}TestNUC\textdagger & \cellcolor{gray!18}\textbf{0.764} & \cellcolor{gray!18}\textbf{0.864} & \cellcolor{gray!18}\textbf{0.646} & \cellcolor{gray!18}\textbf{0.540} & \cellcolor{gray!18}\textbf{0.948} & \cellcolor{gray!18}\textbf{0.554} & \cellcolor{gray!18}\textbf{0.680} & \cellcolor{gray!18}\textbf{0.414} & \cellcolor{gray!18}\textbf{0.676} \\ \midrule \midrule
Llama-3.1-8B & Standard Prompting & 0.572 & 0.726 & 0.502 & 0.492 & 0.892 & 0.528 & 0.530 & 0.332 & 0.572 \\
& Self-Consistency & 0.620 & 0.774 & 0.564 & 0.526 & 0.902 & 0.518 & 0.564 & 0.340 & 0.601 \\
& TestNUC & 0.694 & 0.806 & 0.618 & 0.558 & 0.934 & 0.528 & 0.596 & 0.356 & 0.636 \\
& \cellcolor{gray!18}TestNUC\textdagger & \cellcolor{gray!18}\textbf{0.724} & \cellcolor{gray!18}\textbf{0.812} & \cellcolor{gray!18}\textbf{0.646} & \cellcolor{gray!18}\textbf{0.576} & \cellcolor{gray!18}\textbf{0.940} & \cellcolor{gray!18}\textbf{0.542} & \cellcolor{gray!18}\textbf{0.614} & \cellcolor{gray!18}\textbf{0.360} & \cellcolor{gray!18}\textbf{0.652} \\ \midrule \midrule
Claude-3-Haiku & Standard Prompting & 0.680 & 0.848 & 0.486 & 0.564 & 0.892 & 0.552 & 0.594 & 0.336 & 0.619 \\
& Self-Consistency & 0.702 & 0.870 & 0.510 & 0.578 & 0.904 & 0.564 & 0.568 & 0.350 & 0.631 \\
& TestNUC & 0.762 & 0.894 & 0.596 & 0.588 & 0.940 & 0.590 & 0.620 & 0.348 & 0.667 \\
& \cellcolor{gray!18}TestNUC\textdagger & \cellcolor{gray!18}\textbf{0.804} & \cellcolor{gray!18}\textbf{0.902} & \cellcolor{gray!18}\textbf{0.612} & \cellcolor{gray!18}\textbf{0.600} & \cellcolor{gray!18}\textbf{0.946} & \cellcolor{gray!18}\textbf{0.622} & \cellcolor{gray!18}\textbf{0.660} & \cellcolor{gray!18}\textbf{0.368} & \cellcolor{gray!18}\textbf{0.689} \\ \midrule \midrule
GPT-4o & Standard Prompting & 0.746 & 0.924 & 0.712 & 0.674 & 0.962 & 0.614 & 0.682 & 0.406 & 0.715 \\
& Self-Consistency & 0.758 & 0.922 & 0.720 & 0.688 & 0.958 & 0.624 & 0.696 & 0.426 & 0.724 \\
&TestNUC & 0.804 & 0.934 & 0.744 & \textbf{0.710} & 0.974 & 0.644 & 0.692 & 0.446 & 0.744 \\
& \cellcolor{gray!18}TestNUC\textdagger & \cellcolor{gray!18}\textbf{0.824} & \cellcolor{gray!18}\textbf{0.940} & \cellcolor{gray!18}\textbf{0.750} & \cellcolor{gray!18}\textbf{0.710} & \cellcolor{gray!18}\textbf{0.978} & \cellcolor{gray!18}\textbf{0.654} & \cellcolor{gray!18}\textbf{0.708} & \cellcolor{gray!18}\textbf{0.464} & \cellcolor{gray!18}\textbf{0.754} \\
\bottomrule
\end{tabular}%
}
\caption{Accuracy comparison with Standard Prompting and Self-Consistency across four diverse LLMs. TestNUC consistently improves the inference performance on all benchmark datasets. $\dagger$ denotes that 50 neighbors are utilized.}
\label{tab:main_compare_sc}
\end{table*}
    \label{tab1}
    \vspace{-0.25cm}
\end{table}






Given the differentiable model of the circuit obtained by replacing its discrete logic gates with their corresponding probability model, our objective now is to generate a set of inputs that satisfy a desired constraint. This constraint could pertain to any desired valuation of intermediate signals or outputs. To generate satisfying solutions to the CircuitSAT problem, we represent the input variables to the circuit as $\textbf{V} \in \mathbb{R}^{b\times n}$, where $n$ represents the number of variables and $b$ denotes the batch size. We define the matrix $\textbf{V}$ as the parameters of an embedding layer in our circuit model, which will be updated during the learning process. It is worth mentioning that the number of variables in our sampling method is significantly fewer than that of SAT samplers, remaining the same as the number of inputs in the circuit. This discrepancy arises because SAT samplers deal with the CNF of the circuit, where each gate or component introduces additional variables. The embedding layer converts the real-value input variables of the circuit into probabilities in the range from $0$ to $1$ using the sigmoid function $\sigma(\cdot)$, expressed as:
\begin{equation}
    \textbf{P} = \sigma(\textbf{V}) = \dfrac{1}{1 + e^{-\textbf{V}}},
\end{equation}
where $\textbf{P} \in [0, 1]^{b\times n}$ represents the input probabilities to the underlying circuit. The circuit functionality is then computed as:
\begin{equation}
    \textbf{Y} = \mathcal{F}(\textbf{P}),
\end{equation}
where $\mathcal{F}:[0, 1]^{b \times n} \rightarrow [0, 1]^{b \times m}$ denotes the probabilistic model of the circuit. The matrix $\textbf{Y} \in [0, 1]^{b \times m}$ denotes the $m$ outputs across $b$ data batches. The $\ell_2$-loss function $\mathcal{L}$ can be constructed by measuring the distance between $\textbf{Y}$ and the target output valuation matrix $\textbf{T} \in \{0, 1\}^{b \times m}$ as follows:
\begin{equation}
    \mathcal{L} = \sum_{b,m} \left|\left| \textbf{Y} - \textbf{T} \right|\right|^2_2.
\end{equation}
The above loss function can be minimized, and the input variables (i.e., $\textbf{V}$) can be updated using GD in an iterative manner. Upon convergence, the $b$ solutions to the CircuitSAT problem are obtained by converting the soft input values (i.e., $\textbf{V}$) into hard values (i.e., $\widetilde{\textbf{V}} \in \{0, 1\}^{b\times n}$).

Fig. \ref{fig1} illustrates the overview of {\sc Demotic}. {\sc Demotic} is equipped with a parser to covert the circuit described in either bit-blasted Verilog or Berkeley Logic Interchange Format (BLIF) into its corresponding probabilistic model. Consequently, {\sc Demotic} can describe combinational circuits and generate satisfying solutions for any arbitrary constraint on the circuit. Such a sampling paradigm can also benefit from GPU acceleration due to the parallel independent computations across the data batches, enabling a high-throughput sampling procedure. 

To better understand our methodology, let us consider a quantitative example using the module ``c$15$'' shown in Fig. \ref{fig1}. We set the output node $G19$ to $1$ as an output constraint, while the output node $G22$ can take any value of either $0$ or $1$. Therefore, the goal in this example is to find a pair of inputs such that the output node $G19$ is equal to $1$. In this example, the input nodes contributing to our output constraint are $G3$, $G6$, and $G7$. These inputs are learned iteratively using gradient descent. The remaining input nodes, $G1$, $G2$, and $G3$, will not be updated and can take any arbitrary binary values. During each training iteration, each input node is updated by computing the derivative of the loss function with respect to each input node.

To illustrate the process, we generate two samples. In the first step, we randomly assign two values to each input node as follows:
\begin{equation}
    \textbf{v}_{G3} = \begin{bmatrix}
           0.1 \\
           -0.2 
         \end{bmatrix}, \textbf{v}_{G6} = \begin{bmatrix}
           0.5 \\
           -0.4 
         \end{bmatrix}, \textbf{v}_{G7} = \begin{bmatrix}
           -0.7 \\
           -0.8 
         \end{bmatrix},
\end{equation}
where the concatenation of the above vectors forms the matrix $\textbf{V}$. Next, the input probabilities to the circuit are calculated using the sigmoid function:
\begin{equation}
    \textbf{p}_{G3} = \begin{bmatrix}
           0.5250 \\
           0.4502
         \end{bmatrix}, \textbf{p}_{G6} = \begin{bmatrix}
           0.6225 \\
           0.4013
         \end{bmatrix}, \textbf{p}_{G7} = \begin{bmatrix}
           0.3318 \\
           0.3100
         \end{bmatrix}.
\end{equation}
Using the probability model of each gate shown in Table \ref{tab1}, the probabilities of the intermediate node $G11$ and the output node $G19$ are calculated as follows:
\begin{equation}
    \textbf{p}_{G11} = \begin{bmatrix}
           0.4939 \\
           0.4902
         \end{bmatrix}, \textbf{p}_{G19} = \begin{bmatrix}
           0.1639 \\
           0.1520
         \end{bmatrix}.
\end{equation}
Given the target value of 1 for the output node $G19$, the loss is calculated as:
\begin{equation}
    \mathcal{L} = (\textbf{p}_{G19} - 1)^2 = \begin{bmatrix}
           (0.1639 - 1)^2  \\
           (0.1520 - 1)^2 
         \end{bmatrix} = \begin{bmatrix}
           0.6991  \\
           0.7192 
         \end{bmatrix}.
\end{equation}


The above computations are commonly referred to as forward computations. To update the value of the input variables, we need to calculate the derivative of the loss with respect to each input variable, which is referred to as backward computations. This involves using the derivatives of each gate (as shown in Table \ref{tab1}) and applying the chain rule. The process is derived as follows:
\begin{align}
    \dfrac{\partial \mathcal{L}}{\partial \textbf{v}_{G3}} &= \dfrac{\partial \mathcal{L}}{\partial \textbf{p}_{G19}} \dfrac{\partial \textbf{p}_{G19}}{\partial \textbf{p}_{G11}} \dfrac{\partial \textbf{p}_{G11}} {\partial \textbf{p}_{G3}}
    \dfrac{\partial \textbf{p}_{G3}} {\partial \textbf{v}_{G3}} = 2\textbf{p}_{G19} \cdot \textbf{p}_{G7} \cdot (1 - 2\textbf{p}_{G6}) \nonumber \\ 
    &\cdot \sigma(\textbf{v}_{G3})\cdot (1 - \sigma(\textbf{v}_{G3})) = \begin{bmatrix}
           0.0339  \\
           -0.0257 
         \end{bmatrix}, \nonumber 
\end{align}
\begin{align}
    \dfrac{\partial \mathcal{L}}{\partial \textbf{v}_{G6}} &= \dfrac{\partial \mathcal{L}}{\partial \textbf{p}_{G19}} \dfrac{\partial \textbf{p}_{G19}}{\partial \textbf{p}_{G11}} \dfrac{\partial \textbf{p}_{G11}} {\partial \textbf{p}_{G6}}
    \dfrac{\partial \textbf{p}_{G6}} {\partial \textbf{v}_{G6}} = 2\textbf{p}_{G19} \cdot \textbf{p}_{G7} \cdot (1 - 2\textbf{p}_{G3})  \nonumber 
 \\ 
    & \cdot \sigma(\textbf{v}_{G6}) \cdot (1 - \sigma(\textbf{v}_{G6}))  = \begin{bmatrix}
           0.0065   \\
           -0.0126 
         \end{bmatrix}, \nonumber 
\end{align}
\begin{align}
    \dfrac{\partial \mathcal{L}}{\partial \textbf{v}_{G7}} &= \dfrac{\partial \mathcal{L}}{\partial \textbf{p}_{G19}} \dfrac{\partial \textbf{p}_{G19}}{\partial \textbf{p}_{G7}} \dfrac{\partial \textbf{p}_{G7}} {\partial \textbf{v}_{G7}} = 2\textbf{p}_{G19} \cdot \textbf{p}_{G11} \nonumber 
 \\ 
    & \cdot \sigma(\textbf{v}_{G7})\cdot (1 - \sigma(\textbf{v}_{G7})) = \begin{bmatrix}
           -0.1831   \\
           -0.1778 
         \end{bmatrix},
\end{align}
where ``$\cdot$'' denotes element-wise multiplication.


At this point, each variable is updated using the gradient descent update rule. This involves subtracting the derivative of the loss, scaled by the learning rate, from the corresponding input variables. Given a learning rate of $\gamma = 10$, the new values of the input variables at the end of this iteration are obtained as follows:
\begin{align}
    \textbf{v}_{G3} &= \textbf{v}_{G3} - \gamma \dfrac{\partial \mathcal{L}}{\partial \textbf{v}_{G3}} =  \begin{bmatrix}
           -0.2389 \\
           0.0569
         \end{bmatrix}, \textbf{v}_{G6} = \begin{bmatrix}
           0.4349 \\
           -0.2741
         \end{bmatrix}, \nonumber \\ \textbf{v}_{G7} & = \begin{bmatrix}
           1.1311 \\
           0.9783
         \end{bmatrix}.
\end{align}
This process can be repeated multiple times until convergence. However, even after one iteration in this specific example, we obtain two valid and distinct solutions by rounding the input variables to their nearest discrete values after applying the sigmoid function. In this example, the two input pairs of $(v_{G3} = -0.2389, v_{G6} = 0.4349, v_{G7} = 1.1311)$ and $(v_{G3} = 0.0569, v_{G6} = -0.2741, v_{G7} = 0.9783)$ are rounded to $(\widetilde{v}_{G3} = 0, \widetilde{v}_{G6} = 1, \widetilde{v}_{G7} = 1)$ and $(\widetilde{v}_{G3} = 1, \widetilde{v}_{G6} = 0, \widetilde{v}_{G7} = 1)$, respectively. As demonstrated through this example, the forward and backward computations of the two samples are independent of each other. This allows for the parallel execution of the learning process across multiple samples (i.e., batches), enabling GPU acceleration.


% \begin{figure*}[t]
%     \centering
%      \begin{subfigure}[b]{0.6\textwidth}
%          \centering
%          \scalebox{0.9}{\begin{tikzpicture}[auto, node distance=2cm,>=latex']
    % Combinational block 1
    \node [draw, fill = dateblue!30, shape=rectangle, minimum width=2.5cm, minimum height=1.5cm, text width=2cm, align = center, line width=1.5pt] (comb1) {Combinational Circuit ($\mathcal{F}_h$)};
    % Flip-Flop block
    \node [draw, fill = datemagenta!40, shape=rectangle, right = 1.25cm of comb1, align = center, minimum height=2cm, line width=1.5pt] (ff) {Flip-Flops};
    % Combinational block 2
    \node [draw, fill = dateblue!30,shape=rectangle, minimum width=2.5cm, minimum height=1.5cm, right =1.25cm of ff, text width=2cm, align = center, yshift=0.5cm, line width=1.5pt] (comb2) {Combinational Circuit ($\mathcal{F}_o$)};
    % Input nodes
    \node [left =1cm of comb1, coordinate] (input1) {};
    % Output node
    \node [right of=comb2, coordinate] (output) {};
    % Connection arrows
    \draw [->, >=stealth, line width=1.5pt] (input1) -- node[left, xshift = -0.5cm] {$\mathbf P_{t}$} (comb1);
    \draw [->, >=stealth, line width=1.5pt] (comb1) -- node {$\mathbf H_{t}$} (ff);
    \draw [->, >=stealth, line width=1.5pt] (comb1.east) -- node {} (ff);
    \draw [->, >=stealth, line width=1.5pt] (ff) -- node {$\mathbf H_{t-1}$} ([yshift=-0.5cm]comb2.west);
    \draw [->, >=stealth, line width=1.5pt] ([xshift=0.5cm]ff.east) |-  ([yshift=-1.5cm,xshift = -0.5cm]comb1.west) |-  ([yshift=-0.25cm, xshift = -0.15cm]comb1.west) |- ([yshift=-0.25cm]comb1.west);
    \draw [->, >=stealth, line width=1.5pt] (comb2) -- node[right, xshift = 0.5cm] {$\mathbf Y_t$} (output);
    \draw [->, >=stealth, line width=1.5pt, dashed] ([yshift=0cm, xshift = -0.5cm]comb1.west) |- ([yshift=1cm, xshift = -0.75cm]comb2.west) |- ([yshift=0.5cm, xshift =0cm]comb2.west);
\end{tikzpicture}}
%          \vspace{-0.5cm}
%          \caption{}
%          \label{fig2a}
%      \end{subfigure}
%      \hfill
%      \begin{subfigure}[b]{0.39\textwidth}
%          \centering
%          \scalebox{0.9}{\begin{tikzpicture}[scale=1, transform shape];
                \node [nnlayer]                     at ( 0,     0)    (sig1)          {$\mathcal{F}_o$};
                \node [nnlayer]                     at ( 1.5,   0)    (sig2)          {$\mathcal{F}_h$};

                \node [anchor=east]                 at ( -1,  -0.5) (hiddenlast)    {$\mathbf H_{t-1}$};
                \node [anchor=west]                 at ( 3.5,     1.) (hiddennext)    {$\mathbf H_{t}$};

                \node [anchor=west]                 at ( 0,     2.25) (hiddennext1)    {$\mathbf Y_{t}$};

                \node [anchor=east]                 at ( -1, -1.5)  (input)         {$\mathbf P_{t}$};

                \draw [ultra thick, rounded corners=0.2cm] (hiddenlast) -| (sig1);
                \draw [ultra thick, rounded corners=0.2cm] (hiddenlast) -| (sig2);


                \draw [->, >=stealth, ultra thick, rounded corners=0.2cm] (sig1.north) -- ([yshift = 1.5cm]sig1.north);;

                % \draw [->, >=stealth, ultra thick, rounded corners=0.2cm] (sig2.north) |- ([xshift=2cm, yshift=0.5cm]sig2.north) |- ( 6,  -0.5);

                \draw [->, >=stealth, ultra thick, rounded corners=0.2cm] (sig2.north) |- ([xshift=1.5cm, yshift=0.5cm]sig2.north);

                \draw [ultra thick, rounded corners=0.2cm] (input) -- ++(1.2,0) |- (0.5,-0.5);

                \begin{pgfonlayer}{background}
                \draw [fill=dateblue!10, rounded corners=.5cm] (-.5, -1) rectangle (3,1.5);
                \end{pgfonlayer}


            \end{tikzpicture}}
%          \vspace{-0.5cm}
%          \caption{}
%          \label{fig2b}
%      \end{subfigure}
%     \caption{The general form of a sequential circuit is shown in (a), and the recurrent cell for sequential circuits is depicted in (b).}
%     \label{fig2}
%     \vspace{-0.5cm}
% \end{figure*}



% \section{Sequential Circuits}
% So far, we have described how combinational circuits can be modeled and analyzed using {\sc Demotic}. In contrast to combinational circuits, where outputs are determined solely by their present inputs, the output in sequential circuits depends on both the past behavior of the circuit and the present values of inputs. The temporal operations of sequential circuits are controlled by a clock signal. The contents of memory elements (i.e., flip-flops) represent the past behavior of such a circuit, which is commonly referred to as the \textit{state} of the circuit. 


% Solving CircuitSAT problems for sequential circuits presents a unique challenge as it requires finding a sequence of inputs that satisfies the target constraint over a series of clock cycles. To tackle such problems, we can leverage a novel technique inspired by recurrent neural networks (RNNs). In RNNs, backpropagation through time is utilized during the learning process, allowing for updates to the network's hidden state at each time step. Similarly, in the context of solving CircuitSAT problems, we perform forward computations to iteratively update the state values at each clock cycle. During backward computations, gradients are backpropagated through time, extending back to the initial time step (i.e., the first clock cycle), to adjust the input sequence accordingly. While this approach draws parallels to RNN training, it is tailored to the unique challenges posed by solving CircuitSAT problems for sequential circuits. 



% Fig. \ref{fig2a} shows the general structure of a sequential circuit. We use this structure to formulate the CircuitSAT problem for sequential circuits to find satisfying solutions using {\sc Demotic}. In this structure, there are two combination circuits: one to update the state of the circuit (i.e., the content values of flip-flops) and the other one to generate the output. It is worth mentioning that both of these combinational circuits take the present values of flip-flops and primary inputs at the current time step as their inputs. Let us represent the primary input variables at time step $t$ as $\textbf{V}_t \in \mathbb{R}^{b\times n}$. We encode the primary input variables at each time step as learnable parameters to an embedding layer followed by the sigmoid function to provide input probabilities at time $t$ as $\textbf{P}_t \in [0, 1]^{p\times n}$ to the combinational circuits, i.e.,
% \begin{equation}
%     \textbf{P}_t = \sigma(\textbf{V}_t).
% \end{equation}
% The present output of the circuit (i.e., $\textbf{Y}_t \in [0, 1]^{b\times m}$) is computed as: 
% \begin{equation}
%     \textbf{Y}_{t} = \mathcal{F}_o(\textbf{P}_t, \textbf{H}_{t-1}),
% \end{equation}
% where $\mathcal{F}_o$ and $\textbf{H}_t \in [0, 1]^{b \times r}$ denote the functionality of the combinational circuit generating outputs and the present values of flip-flops at each time step, respectively. The number of flip-flops in the circuit is represented by $r$. The state of the circuit for the next time step is obtained as:
% \begin{equation}
%     \textbf{H}_{t} = \mathcal{F}_h(\textbf{P}_t, \textbf{H}_{t-1}),
% \end{equation}
% where the functionality of the combinational circuit updating the values of flip-flops is denoted by $\mathcal{F}_h$. The $\ell_2$-loss function $\mathcal{L}$ can then be constructed by measuring the distance between $\textbf{Y}_t$ at the desired time step $N$ and the target output valuation matrix $\textbf{T} \in \{0, 1\}^{b \times m}$ as follows:
% \begin{equation}
%     \mathcal{L} = \sum_{b,m} \left|\left| \textbf{T} - \textbf{Y}_N \right|\right|^2_2.
% \end{equation}
% With such a formulation for sequential circuits, {\sc Demotic} can solve the CircuitSAT problem and provide $b$ solutions. The general form of the recurrent cell for sequential circuits is shown in Fig. \ref{fig2b}, which is analogous to the RNN cell.




\section{Confirmation Study}
\label{section:formativestudy}

Although it seems evident that adding additional visual stimuli may distract users and influence the noticeability of redirection, we conducted a confirmation study to validate this hypothesis and assess the effectiveness of our dual-task design.

% \delete{
% To investigate whether and how visual attention will influence the motion offset's noticeability, we leveraged visual stimuli to induce gaze saccades and visual attention shifts in users while users were performing hand reaching tasks with redirected motions.
% Specifically, we applied a dual-task paradigm where 16 participants performed an arm motion and decided whether there was a significant difference between the virtual and the physical motion (\textbf{primary task}), while monitoring and reporting when a red ball gradually changed from full transparency to full opacity at randomized locations in the field of view (\textbf{secondary task}).
% We controlled the intensity of the visual stimuli by altering the duration and the location of the animations.
% With the stimuli, we employed a yes/no paradigm in psychophysics~\cite{leek2001adaptive} to record the participants' responses and estimate the noticeability by the ratio of correct responses to the number of trials referring to previous research~\cite{li2022modeling}.
% }

% \delete{
% Though recent works proved that the noticeability of virtual hand offsets could be influenced by eye gaze saccades\cite{zenner2023detectability}, we aimed to further quantify the relationship between users' gaze behavioral patterns and noticeability.
% Therefore, we leveraged visual stimuli to induce gaze saccades and other gaze behaviors in users.
% We investigated how directing the user's visual attention away from their body at different strengths influences the probability of them noticing an offset applied to the avatar's motion with respect to their real motion (i.e., noticeability).
% We adopted the Method of Constant Stimuli~\cite{simpson1988method} to obtain the detection threshold (DT) and noticing probability when presenting the stimuli, including the motion offset and visual animations.
% We controlled the intensity of the stimuli by altering the strength of motion offsets, the duration and the location of the animations.
% To present the stimuli, we applied a dual-task paradigm where 16 participants performed an guided arm motion and decided whether there was a significant difference between the virtual and the physical motion (\textbf{primary task}), while monitoring and reporting when a red ball gradually changed from full transparency to full opacity at randomized locations in the field of view (\textbf{secondary task}).
% With the stimuli, we adopted a two-alternative forced-choice (2AFC) procedure to record the participants' responses and estimate the noticeability by the ratio of correct responses to the number of trials referring to previous research~\cite{li2022modeling}.
% }

% By referring to previous research~\cite{li2022modeling}, we estimate the noticeability by the frequency of participants noticing the offsets.


%invited 16 participants to decide if they noticed the offset under visual attractions with different intensities and positions.

% The primary objective of Study 1 was to investigate the potential impact of visual attention on users' perceptual likelihood of noticing limb offsets. We introduced offsets in varying magnitudes and directions to randomized poses, and combined with diverse levels of visual focus tasks.

\subsection{Design}

%We applied a dual-task paradigm in this study consisting of a primary limb movement task and a secondary visual attraction task. 
%In the primary task, participants executed prescribed movements using their left arm with the VR headset on, while concurrently evaluating virtual movements to determine their consistency with corresponding physical actions. 
% Subsequently, participants reported the immediate appearance position as pertaining to either the left or right half.

%We employed a factorial experimental design.

We employed a factorial study design to manage both independent and control variables.

% \delete{
% To investigate the noticeability under various stimuli, we altered the strength of motion offsets and visual animations (i.e., red balls rendered at different locations in the virtual environment).
% Since we aimed to investigate if visual attention influences the noticeability of the motion offset, we list the intensity of visual stimuli as the independent variable and the noticeability as the dependent variable.
% }

% independent factors
\subsubsection{Independent variables}
In this study, we aimed to investigate whether applying visual stimuli affects noticeability. 
Therefore, our initial independent variable was the presence or absence of visual stimuli. 
To further explore the impact of various visual stimuli, we extended the independent variable to the intensity of visual stimuli, ranging from none to high.
We manipulated intensity by adjusting the duration and placement of virtual animations, following previous studies~\cite{gutwin2017peripheral, li2024predicting}. 
Through a pilot study, we identified three levels of duration: Short (0.2 sec), Medium (1 sec), and Long (2 sec).
For placement, we defined three layout configurations: Sparse (stimuli appear only in the corner areas), Median (stimuli appear in both the corner and peripheral areas), and Dense (stimuli appear throughout the entire field of view), as shown in~\autoref{figure:formalapparatus}.
In each layout, we randomly picked one candidate to animate the visual stimuli.
Additionally, we included a baseline condition with no visual stimuli.
The order of these conditions was randomized.

% \delete{
% To control the intensity of the visual stimuli, we altered the duration and placement of the red ball, since previous works showed that animations with a shorter duration or appearing in the central area are more noticeable than subtle and peripheral ones~\cite{gutwin2017peripheral, li2024predicting}.
% Through a pilot study, we chose three levels for the duration (Short: 0.2 sec, Medium: 1 sec, Long: 2 sec) that were reported to direct participants' attention to different degrees of success.
% Regarding the placement, we defined three layouts (Sparse, Median, Dense) where the stimuli appears only in the corner area, in both corner and peripheral areas, and in all areas of the field of view, respectively. 
% To analyze whether at all the visual stimuli affect the noticeability of offsets, we include a baseline condition without any visual stimuli.
% }

% control factors
\subsubsection{Control variables}
We varied the magnitude and direction of the redirection as control variables. 
Based on the results from related research~\cite{li2022modeling}, we set the redirection magnitude from 0 to 30 degrees with an interval of 5 degrees, which covers the from being unnoticeable (no redirection) to easily noticeable.
We also varied the direction of redirection, sampling both horizontal and vertical directions.
As a result, each participant completed $(3~durations~\times~3~layouts~+~1~baseline)$  $\times (7$ redirection magnitudes $\times 2$ redirection direction - 1) $= 130$ trials in total.
The order of all redirection magnitudes and directions was randomized.

% \delete{
% We tested 25 different ending poses sampled from CMU MoCap dataset~\cite{CMUMocap} to enhance the generalizability of the collected data.
% To focus on the impact of visual stimuli on the noticeability of different offsets, we fixed the magnitude and direction of the offset as control variables.
% Based on findings of related research~\cite{li2022modeling},  we set the offset magnitude from -30 to 30 degrees with a step of 5 degrees, which we believe cover the offset range from being unnoticeable (no offset) to easily noticeable.
% To fix the offset direction, we randomly picked horizontal or vertical direction for each trial, while we ensured the number of trials in each direction to be the same.
% As a result, each participant completed $(3~durations~\times~3~layouts~+~1~baseline)$  $\times 13$ offset strengths $\times 1$ offset directions $= 130$ trials in total.
% }

\subsubsection{Dependent variables}
The noticeability of redirection was recorded as the primary dependent variable in this study and was estimated with the proportion of positive responses across all trials for each condition where redirection was applied.
Additionally, we captured participants' gaze behavior data with the HMD's eye tracker.

% \delete{
% To measure the noticeability of the motion offsets under each condition, we adopted a yes/no paradigm~\cite{leek2001adaptive}.
% Participants pressed the buttons on the VR controller to report a binary response if they observed a stimulus.
% Thus, we leveraged the proportion of the correct responses to the motion offset (hit rate) as an estimation of the noticeability.
% }

\begin{figure}[t]
    \centering
    \includegraphics[width=0.9\columnwidth]{figures/formativeStudy/apparatus.png}
    \caption{The apparatus of the formative study. 
    Wearing a headset, the participant wears three motion trackers to track their arm pose and sit on a comfortable chair. 
    While the virtual avatar mirrors the arm movement of the participant, the participant observes the virtual avatar's movement from a first-person point of view and follows the semi-transparent checkpoint pose to reach the semi-transparent target pose.
    As the secondary task, a virtual animation will start with different durations and locations.
    The right figure illustrates the possible locations of the red ball, named Sparse, Median, and Dense, accordingly.}
    \label{figure:formalapparatus}
\end{figure}


% \subsection{Task}

% \delete{
% To present the stimuli consisting of motion offsets and visual animations, we adopted a dual-task paradigm in this study.
% }

% \subsubsection{Primary task}
% \delete{
% Our primary task required participants to perform a guided arm motion in VR.
% The motion task is defined by the starting arm pose and the ending arm pose. 
% We use a fixed starting arm pose of naturally resting beside the body and sample 25 arm gestures as the ending poses from CMU MoCap database~\cite{CMUMocap}.
% While the participant performed each motion task, we added an offset on the avatar's elbow joint with respect to the participant's real elbow joint.
% Similar to previous research on offset noticeability~\cite{li2022modeling}, the offset is set to be an angular rotation applied to the joint.
% The offset was applied dynamically, beginning with no offset at the initial pose (pointing to the ground) and reaching the maximum offset at the ending pose. 
% The offset during the intermediate motion was calculated based on the relative angular distance from the starting pose and was adjusted linearly throughout the movement.
% Additionally, an intermediate checkpoint pose was added to ensure that participants performed the motion in a consistent manner.
% We adjusted the strength of the offset by altering the maximum offset applied at the ending pose.
% All motion tasks were performed with the left arm.
% }

%Participants were asked to navigate their movements and reach the checkpoint poses before achieving the target pose. 
%Once they achieved the target pose, participants were asked to retrace their movement and return to the initial pose.
%Then we asked participants to decide if they detected any offset during the movement and we measured the noticeability of the offset during the movement with the proportion of times that participants were able to detect it.


%To circumvent the potential detection of the offset arising from the uniform starting pose, a gradual introduction of the angular offset was implemented, progressively incrementing it from zero to the desired target offset magnitude, while participants transitioned from the initial pose to the target posture. Subsequently, once the target pose was attained, participants were instructed to retrace their movement to revert to the initial pointing downward posture. Subsequent to these motion sequences, participants were prompted to ascertain whether they perceived any offset throughout the course of their movements.

%To simulate realistic interaction scenarios, we asked participants to perform dynamic movements from a uniform initial posture of pointing downward to control the moving trajectory.
%To avoid participants noticing the offset from the uniform initial pose, we implemented a gradually increasing process in which we increased the offset from zero to the target magnitude while participants started from the initial pose and moved their left arm to the target pose.
%To control the trajectory of participants' movements, we displayed two checkpoint poses by interpolating the initial pose and the target pose.


% \subsubsection{Secondary task}
% \delete{
% In parallel to performing the primary task, we asked participants to monitor visual stimuli that appeared at random times and locations in their field of view.
% We designed the visual stimuli as a red ball that transitions from full transparency to full opacity and resets to full transparency.
% The red ball was initially registered in a fixed position in the participant's field of view following their head movements, with a fixed depth of 0.5 m and a radius of 0.02 m.
% We instructed participants to report whether the red ball appears in the left or right side of their field of view by pressing the corresponding part of the touchpad on the hand-held controller as soon as they notice it.
% An interval randomly selected from 1-3 seconds is set between every two visual stimuli to ensure that the participant cannot predict when they arise.
% }

%Visual attraction design.

%Participants verbally reported existence and left or right.

%In the secondary task, participants observed visual attractions that appeared randomly within their field of view at unpredictable intervals (1-3 seconds) and reported as soon as they noticed. 




% The target poses were sampled from the domain of human upper limb movement and chosen randomly to serve as the control variable. 
% Independent variables encompassed the magnitude and direction of offsets, in addition to the temporal duration and spatial layout of the visual attractions. 
% The dependent variables included noticeability of offset under conditions of visual focus, measured by the proportion of times that users were able to detect the offset. Furthermore, the reaction time (RT) and accuracy of the visual focus task were measured.

% In each experimental session, one out of ten combinations entailing three levels of visual stimulus duration, three types of visual layouts, and an additional control group without the visual attraction was employed. Offsets were implemented on virtual movements, ranging from 0 to 30 degrees in increments of 5 degrees, alongside randomly assigned plus-minus directions on the x and y axes, which were uniformly distributed among participants. This led to a total of 13 target pose configurations. As such, each participant evaluated $10 visual attraction settings \times 13 target pose setting = 130 tasks$.


% \emph{Primary task:} perform given movements while observing virtual movements and decide if the virtual movement is the same as the physical movement.

% \emph{Secondary task:} observe the visual attractions which randomly appeared in the field of view and report the appearing position as soon as possible.

% \emph{Independent variables:} visual attraction duration and visual attraction layout

% \emph{Control variables:} offset magnitude, offset direction, target movements

% \emph{Metrics:} noticeability of offsets, visual attraction response accuracy and visual attraction response time


\subsection{Participants \& Apparatus}
The participants (N = 16) were recruited through an online questionnaire from a local university. 
The participants (7 females, 9 males) had an average age of 21.25 years ($SD = 1.71$). 
All were trichromats and right-handed. 
Prior to the experiment, participants self-evaluated on their familiarity with VR, reporting an average score of $3.75\ (SD=0.75) $on a 7-point Likert scale (1 - not at all familiar, 4 - neutral, 7 - very familiar).

% \subsection{Apparatus}
We implemented the experimental application in VR with a HTC Vive pro headset in Unity 2019, powered by an Intel Core i7 CPU and an NVIDIA GeForce RTX 3080 GPU. 
Throughout the experimental sessions, participants were seated and equipped with three Vive Trackers affixed to their left shoulder, elbow, and waist using nylon straps.
Based on data given by the tracker, we reconstructed the left arm movement on a virtual humanoid avatar from the Microsoft RocketBox avatar library~\cite{gonzalez2020rocketbox} with the user’s viewpoint coinciding with the avatar’s (as shown in \autoref{figure:formalapparatus}).
All gaze data was recorded with the HTC Viveo pro built-in gaze tracker.
All statistical analyses were conducted with SPSS 26.0.

% Results: Layout: F(2,147)=33.87, p<0.001. Duration: F(2,147)=43.40, p<0.001.

\subsection{Procedure}


To avoid bias from the participants knowing that we were intentionally introducing redirection, we introduced the purpose of the study as an evaluation of a motion capture and reconstruction technique and clarified the real purpose to participants after the study.
Participants were first provided with a walk-through of the platform.
Then, participants were provided with a warm-up session to ensure that they were familiar with the primary and secondary tasks.
After that, each participant completed 10 sessions ($3~durations~\times~3~layouts~+~1~baseline$) of experiments.
They took 2-minute breaks after every two sessions to reduce fatigue.
We recorded the participant's behavioral data, including the position and orientation of hand, elbow, shoulder, gaze, and pupil dilation, at a rate of 60 Hz.
The study lasted around 40 minutes and each participant was compensated with 15 US dollars.
%For the primary task, participants were asked to start from the initial pose and reach the target pose, bypassing two checkpoint poses.
%After retracing the trajectory, they were asked to report if they recognized the virtual arm movement as the same as their physical movement.
%During the primary task, visual attractions were displayed at a random time interval with designed intensity and position.
%Participants were asked to report the visual attraction's position with the touchpad on the controller as soon as they noticed it.


% Participants were first instructed to put on the VR headset and Vive trackers. Following a warm-up session, participants completed ten experimental sessions in a predetermined sequence, with a short break after every five sessions. During the experiment, participants executed left-arm movements while using the Vive controller with right hand to report their perception of offset and the position of visual attractions.

\begin{figure}[t]
    \centering
    \includegraphics[width=0.9\columnwidth]{figures/formativeStudy/noticeability.png}
    \caption{Noticeability results of the formative study in every condition.
    The error bars represent the standard errors.}
\end{figure}

\subsection{Results}
\label{section:study1results}

We first conducted Shapiro-Wilk tests on the noticeability results which showed that all 10 conditions followed a normal distribution, requiring no correction.
We then conducted Repeated-Measures ANOVA with Bonferroni-corrected post hoc T-tests on the results.
The average response time to visual stimuli was 327 ms (SD = 168 ms), indicating that participants were actively engaged in both tasks.

\begin{figure*}
    \centering
    \includegraphics[width=0.9\linewidth]{figures/formativeStudy/psychometric3.png}
    \caption{Psychometric functions of the noticeability in each condition.}
    \label{fig:psychometric}
\end{figure*}

\textit{With visual stimuli, participants noticed the redirection significantly less than without visual stimuli.}
We conducted a one-factor ANOVA between the baseline and the averaged nine other conditions.
Our statistical analysis showed that participants detected the redirection significantly $(F_{(1,15)}=5.90,~p=0.03)$ less frequently when they were exposed to the visual stimuli $(M=0.43,~SD=0.13)$ compared to none visual stimuli $(M=0.51,~SD=0.08)$.
These results confirm that the noticeability of redirection was reduced when visual stimuli were presented and further validate the design of our study.


We then evaluated whether participants' physical movements were effectively redirected under both noticed and unnoticed conditions. 
We divided all trials into two categories based on the participants' response to the redirection (\textit{noticed} or \textit{unnoticed}).
We then analyzed the lengths of participants' virtual and physical hand trajectories within these two categories.
The \textbf{physical trajectory length} refers to the ratio of the participant's physical hand movement trajectory length to the distance between the starting and ending pose.
Similarly, the \textbf{virtual trajectory length} refers to the participants' virtual hand movement trajectory length to the distance between the starting and ending pose.
In the unnoticed condition, participants' physical movement trajectories were significantly shorter than their virtual ones: \textit{Physical} $(AVG = 1.16,~SD = 0.04)$ and \textit{Virtual} $(AVG = 1.24,~SD = 0.05,~t(15) = 3.64,~p < 0.05)$.
Similarly, in the noticed condition, participants' physical movement trajectories were still significantly shorter than their virtual ones: \textit{Physical} $(AVG = 1.17,~SD = 0.04)$ and \textit{Virtual} $(AVG = 1.33,~SD = 0.04,~t(15) = 4.88,~p < 0.01)$.
These results suggest that participants' physical movements were successfully redirected,regardless of whether they noticed the redirection.

% \delete{
% \textbf{Less visual attention directed towards the avatar leads to lower noticeability of the applied offset.}
% }
% \delete{
% \textbf{Visual attention shifts away from the avatar lead to lower noticeability of the applied offset.}
% We conducted a one-factor ANOVA between the baseline and the averaged nine other conditions.
% Our statistical analysis showed that participants detected the offset significantly $(F_{(1,15)}=5.90,~p=0.03)$ less frequently when they were exposed to the visual stimuli $(M=0.43,~SD=0.13)$ compared to none visual stimuli $(M=0.51,~SD=0.08)$.
% This indicates that our visual stimuli successfully occupied participants' visual attention and made them less sensitive to the motion offset.
% We then conducted a two-factor ANOVA to explore the impact of the stimuli's duration and position on the noticeability.
% Results showed that both the stimuli's duration $(F_{(2,30)}=24.02,~p<0.001)$ and position $(F_{(2,30)}=27.83,~p<0.001)$ had a significant influence on the noticeability and there was not an interaction effect between these two factors $(F_{(4,60)}=1.70,~p=0.20)$.
% Therefore, we conducted post hoc tests on duration and layout separately.
% Post hoc results showed that the noticeability with \textit{short} duration $(M=0.35,~SD=0.10)$ was significantly lower than \textit{medium} $(M=0.46,~SD=0.07),~(t(15)=-3.34,~p<0.001)$, and \textit{long} $(M=0.50,~SD=0.07),~(t(15)=-4.39,~p<0.001)$ duration stimuli.
% The noticeability with \textit{sparse} layout $(M=0.37,~SD=0.09)$ was significantly lower than \textit{medium} $(M=0.43,~SD=0.07),~(t(15)=-3.07,~p<0.01)$, and \textit{dense} $(M=0.49,~SD=0.09),~(t(15)=-4.01,~p<0.001)$ duration stimuli.
% This indicates that both the visual stimuli' duration and position could affect the noticeability of offset significantly.
% We further plotted the psychometric functions of each condition to demonstrate how visual stimuli impact the noticeability behavior of users, which can be helpful for designers to decide the appropriate offsets to apply in different scenarios.
% To be noted, the noticeability of motion offsets under visual stimuli with the sparse layout and short duration did not achieve 100\%, due to participants' visual attention being strongly directed towards the visual stimuli.
% }

\begin{figure*}
    \centering
    \includegraphics[width=0.9\linewidth]{figures/formativeStudy/correlation_v3.png}
    \caption{The regression results between gaze distance, saccade frequency, fixation frequency, IPA, and noticeability are presented. 
    The correlation coefficients are indicated in the top right corner.}
    \label{figure:correlation}
\end{figure*}

We further analyzed the gaze behaviors (gaze location, saccades, fixation, and pupil activity) based on the recorded gaze data. 
Specifically, we computed the average gaze distance relative to the virtual hand, saccade and fixation frequencies, and Index of Pupil Activity (IPA) in each condition. 
We then calculated the correlation coefficients between these gaze behaviors and the noticeability results.
As shown in~\autoref{figure:correlation}, the results revealed significant correlations: gaze distance ($r = -0.26, p = 0.001$), saccade ($r = -0.43, p < 0.001$), fixation ($r = -0.27, p < 0.001$), and IPA ($r = -0.26, p = 0.001$). 
These results suggest that all the examined gaze behaviors exhibit a negative relationship with noticeability, with gaze saccades showing a stronger effect compared to the others. 
This may be because rapid saccadic movements often indicate high cognitive load or attentional shifts, making them more directly and negatively associated with the noticing of redirection. 
In contrast, fixation frequency does not inherently reflect cognitive load, although fixation duration might serve as a useful indicator.
Regarding gaze distance to the virtual hand, participants likely shifted their gaze between the virtual body and the stimuli, making it less consistently related to noticeability. 
For IPA, its design as a long-term estimator of cognitive load may render it less sensitive to subtle or transient changes in cognitive load caused by visual stimuli.
Overall, while each of these gaze behaviors responds to visual stimuli in distinct ways, they all show promise as predictors of noticeability.


% \delete{
% Since recent papers showed that the gaze saccade could affect the noticeability of motion offsets~\cite{zenner2023detectability}, we further analyzed other eye gaze behavior to investigate if participants more frequently switched their visual focus or had a higher cognitive load in the condition with stronger visual stimuli.
% To this end, we calculated participants' eye saccades and fixations (as implemented in \cite{pymovements}) from the gaze position data and Index of Pupillary Activity (IPA)~\cite{duchowski2018index} as the estimated cognitive load.
% By calculating the correlation coefficients between the average of these metrics and noticeability, we found that IPA $(r=-0.26)$, saccade $(r=-0.43)$ and fixation frequency $(r=-0.27)$ were correlated with the noticeability.
% This relationship suggests that participants were less likely to notice the offset when they had a high cognitive load or they had more frequent shifts in their visual focus.
% These results motivated us to further explore the relationship between users' gaze behavioral data and noticeability.
% Since there is no ground truth for visual attention, we aim to build an extendable model which takes users' gaze behavioral patterns as input and outputs the noticeability of applied offsets.
% }
% \delete{
% Our aim was to build an extendable model to predict the noticeability of the applied offsets without knowing the specific visual stimulus that leads to the identified gaze behavioral patterns.
% }


\section{Data Collection}
\label{section:datacollection}
After confirming that visual stimuli influence the noticeability of redirection, we conducted another user study using the same dual-task design to gather more data for developing a prediction model. 
This model aims to estimate noticeability based on users' gaze behavior.

% \delete{
% Next, we conducted a study to collect necessary data for constructing a model of the noticeability of arm motion offsets, taking visual attention into account.
% We built a dataset containing the participants' behavioral data and the corresponding noticeability results of arm motion offsets.
% The offsets had a constant strength of 20 degrees while participants were exposed to an opacity-based visual effect with different levels of intensities and displayed in different areas in the field of view.
% }

\subsection{Design}

To collect the noticeability results more accurately, we measured the noticeability of each redirection magnitude repeatedly for each participant, and tested on less redirection magnitude levels.
In future work, we consider it important to extend the experiments to include a wider range of redirection magnitude.s 
%It is worth noting that previous research has already explored the impact of offset strengths on noticeability~\cite{li2022modeling}.
%Additionally, since previous works had explored the influence of offset strength on noticeability, we conducted the data collection study with a fixed offset strength to predict the influence of visual attractions.
As per prior work that investigated the impact of redireciton magnitude on noticeability, we chose 20 degrees as the tested magnitude, as the reported noticeability rate was around 75\% without visual stimuli in~\cite{li2022modeling}. 
The relatively high rate allowed us to detect the impact of visual stimuli effectively.
% We verified the reported rate through a characterizing study. 
We randomly selected horizontal or vertical as the redirection direction.
%We envision that constructing a prediction model which considers the offset strength and the visual attractions simultaneously as an important future work.

We adopted the same dual-task design with a yes/no paradigm detailed in \autoref{section:methodology}.
As our formative study results showed, the visual stimuli with medium duration ($(M=0.46,~SD=0.07)$) did not yield statistically significant differences in terms of noticeability compared to the stimuli with long duration ($(M=0.50,~SD=0.07),~(t(15)=-0.39,~p>0.05)$ ).
Therefore, we excluded the medium condition and only selected the short (0.2s) and long duration (2s) to control the intensity in this study.
To further control the position of visual stimuli within participants' visual field, we displayed the stimuli at central vision (5 degrees from the central point of vision), near peripheral vision (30 degrees), and mid peripheral vision (60 degrees), illustrated in \autoref{figure:datacollectionlayout} and as in prior research~\cite{grosvenor2007primary, gutwin2017peripheral}.
Therefore, we had $2~\times~3~=~6$ conditions, named as \textbf{CS} (central layout with short duration), \textbf{CL} (central layout with long duration), \textbf{NS} (near peripheral layout with short duration), \textbf{NL} (near peripheral layout with long duration), \textbf{MS} (mid peripheral layout with short duration), and \textbf{ML} (mid peripheral layout with long duration), and we used a Latin square to counterbalance them.
Each participant completed $(2~durations~\times~3~layouts)~\times~24$ measurements $=~144$ trials in total. 

\begin{figure}[!htbp]
    \centering
    \includegraphics[width=0.9\columnwidth]{figures/dataCollection/datacollectionlayout.png}
    \caption{The possible locations of the visual stimuli in the data collection study.
    The locations are divided into three conditions: Central (5 degrees), Near Peripheral (30 degrees), and Mid Peripheral (60 degrees), based on the angular distance to the user's head direction.}
    \label{figure:datacollectionlayout}
\end{figure}


\subsection{Apparatus \& Procedure}
The apparatus and procedure were almost identical to those of our formative study (\autoref{section:formativestudy}). 
We recorded the position and orientation of hand, elbow, shoulder, gaze, and pupil dilation with a sample rate of 60 Hz.
All gaze data was recorded with the HTC Viveo pro built-in gaze tracker.
After the warm-up session, participants took 2-minute breaks after every two sessions to reduce fatigue.
The study lasted around 40 minutes and each participant was compensated with \$15 USD.

\subsection{Participants}
We recruited 12 participants (5 females, 7 males) from a local university.
The participants had an average age of 22.91 years $(SD=1.90)$. All were trichromats and right-handed. 
% All participants with an average age of 22.91 $(SD=1.90)$ were trichromats and right-handed. 
Participants' self-reported their familiarity with VR at an average of 3.17 $(SD=1.27)$ on a 7-point Likert scale from 1 (not at all familiar) to 7 (very familiar).

\subsection{Summary of data statistics}

\begin{figure}[!htbp]
    \centering
    \includegraphics[width=0.9\columnwidth]{figures/dataCollection/noticeability_collection.png}
    \caption{Noticeability results of the data collection study in each condition. The error bars represent the standard errors.}
    \label{figure:datacollectionresult}
\end{figure}

In total, we collected 1728 responses.
To estimate the noticeability, we calculated the ratio of trials in which participants reported noticing the redirection to the total number of trials for each session and participant.
As shown in \autoref{figure:datacollectionresult}, the noticeability result differed across the visual stimuli's duration and layout.
The noticeability results ranged from 16.7\% to 79.2\% with an average of 50.1\% and a standard deviation of 18.9\%.
The maximum and minimum indicate that we controlled the noticeability with the visual stimuli's duration and layout successfully.
Additionally, the high standard deviation suggest a high variability across conditions, which is beneficial for training a model to predict the influence of visual stimuli on noticeability.

To verify that participants' physical movements were effectively redirected, we analyzed participants virtual and physical trajectory lengths as defined in~\autoref{section:study1results}.
In the unnoticed condition, participants' physical movement trajectories were significantly shorter than their virtual ones: \textit{Physical} $(AVG = 1.14,~SD = 0.04)$ and \textit{Virtual} $(AVG = 1.20,~SD = 0.04,~t(11) = 3.97,~p < 0.05)$.
In the noticed condition, participants' physical movement trajectories were also significantly shorter than their virtual ones: \textit{Physical} $(AVG = 1.14,~SD = 0.04)$ and \textit{Virtual} $(AVG = 1.27,~SD = 0.05,~t(11) = 5.38,~p < 0.01)$.
These results indicated that participants' physical movements were successfully redirected.


\section{Implementation}
\label{section:implementation}

In this section, we investigated the best combination of gaze behavior features to predict the noticeability of redirection.
We described participants' gaze behaviors with pupil activity, gaze angular distance to users' hands, eye saccade, and fixation.
For each category of participants' gaze behavioral data, we systematically examine different feature combinations to identify those that most accurately characterize the participants' visual responses.
We then develop a regression model to explain the relationship between the selected gaze behavioral features and the noticeability of redirection.

% \delete{
% In this section, we describe our implementation of our prediction model with the collected data from Section~\ref{section:datacollection}, including the participants' pupil activity, gaze movement, eye saccade, and fixation.
% In each category of participants' gaze behavioral data, we enumerate the combination of features to select the features that describe the gaze behavior best.
% Then we combine the four categories and explore the best prediction performance with different prediction models.
% Meanwhile, we leveraged the collected noticeability under each visual attention condition (i.e., the proportion of correct responses in each condition) as the ground truth.
% In this way, we aimed to build a model that takes the user's gaze behavioral patterns as input and predicts the noticeability of the motion offset under different visual attention conditions.
% }

\subsection{Gaze behavioral patterns}
\label{section:featureselection}

Results in \autoref{section:formativestudy} showed that the users' gaze behavioral data was correlated to the noticeability.
We divided the gaze behavioral data into four categories:

\begin{itemize}
    \item \textit{Index of Pupillary Activity}: 
    Index of Pupillary Activity (IPA) has been used to reflect users' cognitive load by analyzing the change of users' pupil dilation~\cite{duchowski2018index, lindlbauer2019context}. 
    While visual stimuli were presented, users' cognitive load might be inadvertently affected and could therefore impact the noticeability results.
    \item \textit{Gaze angular distance to elbow and hand}:  
    We calculated the vector starting from the user's eye to the elbow and hand joint. 
    We then calculated the angular distance between this vector and the gaze vector.
    These metrics reflect whether the participant was looking at the primary task or attracted by the visual stimuli.
    We decided not to calculate the distance between the focus point and the visual stimuli, as in real-world use cases, there is no single visual stimuli but only complicated ones, which make it hard to compute this distance.
    \item \textit{Eye saccade frequency, duration and interval}:
    Eye saccade is a rapid eye movement that shifts the eye from one area to another.
    We leveraged the detection algorithm from \citeauthor{pymovements} to detect the saccade frequency and duration~\cite{pymovements}. 
    The saccade frequency and duration indicate how often and how quickly users shift their eye gaze separately.
    The saccade interval suggests the temporal distribution, which indicates whether the saccades are uniformly distributed across the session.
    \item \textit{Eye fixation frequency, duration and interval}:
    Eye fixations represent when eyes stop scanning the scene and hold the foveal vision on an object of interest. 
    We also used the frequency, duration, and interval of eye fixation to indicate how often and how long users stared at a place and the temporal distribution of fixation.
\end{itemize}

\subsection{Regression model}

To better represent the previous gaze behavioral patterns, we computed \textit{mean, standard deviation, median, maximum and minimum} of IPA and gaze angular distance and combined them with the eye saccade and fixation features.
Therefore, we had 3 behaviors(IPA, gaze angular distance to hand, gaze angular distance to elbow) $\times$ 5 features (mean, standard deviation, median, maximum and minimum) $+$ 3 saccade (saccade frequency, duration and interval) $+$ 3 fixation (fixation frequency, duration and interval) $=~21$ features in total.
% \delete{
% We used \textit{mean, standard deviation, median, maximum and minimum} of IPA and gaze distance to describe the features.
% Combining the eye saccade and fixation features, we had $3~\times~5~+3+3~=~21$ features in total.
% }
However, the search space to determine the combination of features that provides the highest predictive power for noticeability includes as many as $\sum_{i=1}^{21}\frac{21!}{i!(21-i!)}~=~2^{21}-1$ conditions, which means that a grid search is not practical.
Therefore, we adopted a similar method as \cite{maslych2023effective} to select the features.
We first selected the best combination of features within each category and then searched the combination of these categories iteratively to figure out the best combination.

In this process, we used Support Vector Regression (SVR) from scikit-learn package~\footnote{https://scikit-learn.org/stable/modules/generated/sklearn.svm.SVR.html} as the benchmark model since SVR has a stable performance on various data.
The SVR model took the selected features as input, then output a probability ranging from 0 to 1 as the predicted noticeability.
We leveraged the leave-one-user-out cross-validation in the test and the mean squared error (MSE) as the metric.

\autoref{table:featureselection} lists the best combination of features within each of four categories.
Among them, the selected combination in the \textit{gaze angular distance} achieved the best performance, while the other features also demonstrated the potential for predicting noticeability.
Therefore, we combined the features from different categories and further tested them.

\begin{table*}[htb]
  \centering
  \small
    \begin{tabular}[width=\columnwidth]{ccc}
    \toprule
    \textbf{Category} & \textbf{Best combination} & \textbf{MSE}\\
    \midrule
    IPA & mean, maximum, minimum & 0.039 (0.013) \\
    % \midrule
    Gaze angular distance & \makecell{mean(hand), std(hand), median(hand) \\ mean(elbow), std(elbow), maximum(elbow)} & 0.017(0.008) \\
    % \midrule
    Eye saccade & frequency, duration, interval & 0.027(0.009) \\
    % \midrule
    Eye fixation & frequency, duration & 0.040 (0.012) \\
    \bottomrule
    \end{tabular}
    \caption{The best feature combination of each category and the prediction performance. The prediction performance is presented as the average (standard deviation) of MSE.}
    \label{table:featureselection}
\end{table*}

We then tested the regression error of all combinations of the feature category with leave-one-user-out cross-validation.
For each feature combination, we filtered the data with it and then fitted a model with 11 participants' data and tested it on the one remaining participant's data.
After repeating this 12 times, we determined the overall regression error for one feature combination.
\autoref{figure:featureselection} illustrates the regression error of all 15 feature combinations.
The results demonstrate that combining all these four category features achieves the best performance with an MSE of 0.011.

To further understand the best feature combination across the users, we also explored the best feature set for each test user in the leave-one-user-out cross-validation process.
For each test user, we trained a model with each feature combination and selected the best one.
The results showed that for 7 out of the 12 participants, the best feature set was the combination of all four feature categories.
For 3 of the 12 participants, the best set was the combination of IPA, Gaze Angular Distance and Fixation and for the other 2 of the 12 participants, the best set contained IPA, Gaze Angular Distance and Saccade.
% \delete{
% These results show that the selected features were able to indicate the visual stimuli's influence on the noticeability for various users.
% }
The results suggest that each feature captures distinct aspects of gaze behavior that contribute to predicting noticeability. 
Although the gaze angular distance showed a lower correlation with noticeability compared to saccades in \autoref{section:formativestudy}, it performs as the most powerful feature for predicting noticeability. 
This may due to the fact that in~\autoref{section:formativestudy}, we only considered the mean distance, whereas in this study, we included additional numerical features, which could provide more informative insights than the mean alone.
As for eye gaze saccade, it also contributes significantly to the prediction, aligning with the correlation results in~\autoref{section:formativestudy}, as it indicates users' visual focus shift and cognitive activity.
While IPA and fixation also have the potential to predict noticeability, their prediction accuracy is lower compared to the other features. 
This could be because they reflect more general cognitive activity and engagement, rather than specific responses to visual stimuli.
However, combining these features allows us to capture both where users are looking at and the dynamic shifts in focus, which together indicate the noticeability of redirection.
% IPA reflects users' cognitive load, which is correlated with their level of engagement and noticeability. 
% Fixations reveal sustained attention, indicating users' intentional focus on a specific region, while saccades highlight transitions in gaze, suggesting more exploratory or spontaneous actions. 
% Together, these features capture both where users are looking at and the dynamic shifts in focus that influence noticeability.

To further investigate if the selected features could model noticeability, we analyzed the regression error for each individual data point in a per user manner. 
As shown in \autoref{figure:predictionperformance_peruser}, the outputs from our model preserved the relative order of noticeability across the six conditions in 90.3\% data points.
The fitted noticeability in various conditions mostly remained in the range of the ground truth, while most errors came from the two most similar conditions (\textbf{CS} and \textbf{NL}). 
Furthermore, \autoref{figure:predictionperformance_average} illustrates the noticeability average and standard deviation of the data collection results and our model's output.
Our model's output average approximates the participant's results while simultaneously exhibiting a lower standard deviation.
This could be due to the inherent noise introduced from estimating the noticeability using the frequency of participants who reported the noticing of redirection in the study.

\begin{figure}[t]
    \centering
    \includegraphics[width=0.9\columnwidth]{figures/implementation/category.png}
    \caption{The regression error of all combinations of the feature category. The error bars denote the standard deviations.}
    \label{figure:featureselection}
\end{figure}

% \subsection{Model selection}

% \delete{
% Subsequently, we tested different models with the selected features to determine the best prediction models.
% We acknowledge that a different set of features might be optimal for a prediction models other than SVR.
% However, we did not aim to fine tune the feature sets for each model; this is not feasible considering the large search space for feature combinations.
% Instead, we compared the performance of different models on the same set of features that we think are most indicative of the noticeability results.
% We implemented several common regression models from scikit-learn package and reported the MSE with leave-one-user-out cross-validation.
% As shown in \autoref{table:modelselection}, the Adaboost regression model achieved the best performance, while other models also achieved a comparable performance.
% Notably, we did not implement more complicated models (e.g., deep learning models) to prevent the model from overfitting to our relatively small dataset.
% In this paper, we aim to demonstrate the potential of predicting noticeability under various visual effects.
% We regard collecting more data and implementing more sophisticated models as important avenues for future work.
% }

% \begin{table}[htb]
%   \centering
%   \small
%     \begin{tabular}[width=\columnwidth]{cc}
%     \toprule
%     \textbf{Model} & \textbf{MSE}\\
%     \midrule
%     SVR & 0.012(0.008) \\
%     % \midrule
%     Linear Regressor & 0.012(0.008) \\
%     % \midrule
%     Adaboost Regressor & 0.008(0.005) \\
%     Decision Tree Regressor & 0.013 (0.007) \\
%     Random Forest Regressor & 0.018 (0.010) \\
%     % \midrule
%     \bottomrule
%     \end{tabular}
%     \caption{The regression performance of different models, presented as the average (standard deviation) of MSE.}
%     \label{table:modelselection}
% \end{table}

% \subsection{Performance analysis}

% \delete{
% To further investigate if our model and selected features could predict noticeability, we analyzed the prediction percentage error for each individual data point in a per user manner. 
% We performed the tests with the SVR model since it was robust and could represent the average performance of different models.
% As shown in \autoref{figure:predictionperformance_peruser}, the predictions from our model preserved the relative order of noticeability across the six conditions in 90.3\% data points.
% The predicted noticeability in various conditions mostly remained in the range of the ground truth, while most errors came from the two most similar conditions (\textbf{CS} and \textbf{NL}). 
% Furthermore, \autoref{figure:predictionperformance_average} illustrates the noticeability average and standard deviation of the data collection results and our model prediction.
% Our model's prediction average approximates the participant's results while simultaneously exhibiting a lower standard deviation.
% This could be due to the inherent noise introduced from estimating the noticeability using the frequency of participants who reported the offset in the study.
% }

\subsection{Classficiation model}
\label{section:classfication_model}

In our studies, noticeability was measured by the frequency with which participants detected redirection during the trials. Based on this, we developed a regression model that outputs the probability of noticing the redirection as a float value between 0 and 1. While this probability effectively indicates how likely users are to notice the redirection, a classification model providing a simple yes/no result could offer greater practical utility.
To explore this, we trained a classification model by applying various thresholds to the noticeability results and categorizing it into distinct classes. 

\begin{description}
\item[Binary model]
We applied a threshold of 0.5 to transform the collected noticeability results into binary labels: Unnoticeable $(\leq 0.5)$ and Noticeable $(> 0.5)$.
With these, we trained a Support Vector Machine (SVM) classification model with the same features selected in \autoref{section:featureselection}; this model achieved an accuracy of 0.9174 $(SD=0.1126)$ and an F1-score of 0.8968 $(SD=0.1342)$ with leave-one-user-out cross-validation on our collected dataset.
\item[Three-class model]
Then we divided the noticeability into three categories with two thresholds: Low Noticeability $(\leq 0.4)$, Medium Noticeability $(0.4 <$ noticeability $\leq 0.7)$, and High Noticeability $(>0.7)$.
With the same SVM classification model and selected feature, our re-trained model achieved an accuracy of 0.8562 $(SD=0.1240)$ and an F1-score of 0.8478 $(SD=0.1276)$.
To be noted, the prediction accuracy was affected by how we converted the noticeability value to separate labels and might increase with fine-tuned features tailored to the classification task.
This indicates that the selected features from the gaze behavioral pattern have the potential to predict the noticeability as separate categories.
\end{description}

\begin{figure*}[t]
    \centering
    \begin{subfigure}{0.47\linewidth}
        \includegraphics[width=\columnwidth]{figures/implementation/opacity_scatterplot.png}
        \caption{}
        \label{figure:predictionperformance_peruser}
    \end{subfigure}
    \centering
    \begin{subfigure}{0.47\linewidth}
        \includegraphics[width=\columnwidth]{figures/implementation/opacity_boxplot_regression.png}
        \caption{}
        \label{figure:predictionperformance_average}
    \end{subfigure}
    \caption{(a) illustrates the regression results in each condition for each user. (b) illustrates the regression results with leave-one-user-out cross-validation.}
    \label{figure:predictionperformance}
\end{figure*}



\section{Evaluation}
\label{sec:eval}
\vspace{-0.5em}

\ps{Overview of evaluation section}

\begin{figure*}[t]
\centering
\captionsetup{justification=centering}
%
\includegraphics[width=1.0\textwidth]{img/evaluation/eval_placements.pdf}
%
\caption{\textbf{(\textsection \ref{sec:eval}) Baseline architecture and optimized architecture found by \name~(for the \textit{baseline} configuration)}.}
\label{fig:eval-placements}
\vspace{-1em}
\end{figure*}
				% Fix figure placement

We evaluate our proposed chiplet placement and \gls{ici} topology co-optimization methodology on the two homogeneous architectures from \Cref{ssec:homo-opt} and on the two heterogeneous architectures from \Cref{ssec:hetero-opt}.
For each of these four architectures, we design a baseline architecture consisting of a 2D mesh of compute-chiplets in the center with memory- and IO-chiplets on the perimeter.
This type of architecture is the de-facto standard that is used in numerous systems \cite{dataflow_accel_dnn, cifher, simba, hecaton, dojo}.
We perform our evaluation using two different chiplet configurations: 
In the \textit{baseline} configuration, memory- and IO-chiplets only have a single PHY and they cannot relay messages, which is highly unfavorable for \name, as \name~often places memory- and IO-chiplets in the center of the chip (off-chip links of IO-chiplets are routed to the border on the redistribution layer as in AMD's EPYC and Ryzen \cite{amd-chiplet}).
In the \textit{\name} configuration, all chiplets have four \gls{phys} and relay capability.
To ensure a fair comparison, the total memory- and IO-bandwidth stays unchanged and the increased off-chiplet bandwidth due to additional \gls{phys} is only used to relay messages.
\Cref{fig:eval-placements} shows baselines and optimized architectures for the \textit{baseline} configuration.
Unfortunately, a direct comparison to prior work (see \Cref{sec:related-work}) is infeasible since frameworks to optimize the placement are not open-source, or they do not scale to our chiplet counts, and proposals for \gls{ici} topologies on active interposers are not applicable to passive interposers, silicon bridges, and organic substrates.

\vspace{-0.5em}
\subsection{Evaluation Methodology}
\label{ssec:eval-methodology}

\ps{Explain our evaluation methodology and introduce partial trace simulations}

We use RapidChiplet's \cite{rapidchiplet} feature to run simulations in BookSim2 \cite{booksim} using synthetic traffic and application traces from Netrace \cite{netrace}.
BookSim2 is an established, cycle-accurate \gls{noc} simulator and Netrace is a tool for dependency-driven, trace-based \gls{noc} simulations. 
We use the Netrace trace collection \cite{netrace-traces}, which is based on the PARSEC benchmark suite \cite{parsec}.
Each trace is split into five regions (see \Cref{tab:eval-traces}).
Since these traces span across billions of cycles, simulating them in a cycle accurate simulator is extremely time-consuming. 
The \textit{blackscholes\_64c\_simsmall} trace was the only one to terminated within 24 hours, therefore, for the remaining traces, we only simulate the first 1'000'000 cycles of each region.
All traces contain cache coherency traffic between the L1 cache (mapped to compute-chiplets), the L2 cache (mapped to memory-chiplets), and the main memory (mapped to IO-chiplets).

\begin{figure}[H]
\centering
\vspace{-0.5em}
\captionsetup{justification=centering}
\begin{subfigure}{0.99 \columnwidth}
\centering
\includegraphics[width=1.0\columnwidth]{img/evaluation/eval_synthetic.pdf}
\end{subfigure}
\caption{\textbf{(\textsection \ref{ssec:eval-synthetic}) Results on synthetic traffic using the \textit{baseline} configuration}.}
\label{fig:eval-synthetic}
\vspace{-1em}
\end{figure}

\begin{figure}[h]
\centering
\vspace{-0.5em}
\captionsetup{justification=centering}
\begin{subfigure}{0.99 \columnwidth}
\centering
\includegraphics[width=1.0\columnwidth]{img/appendix/eval_synthetic_appendix.pdf}
\end{subfigure}
\caption{\textbf{(\textsection \ref{ssec:eval-synthetic}) Results on synthetic traffic using the \textit{\name} configuration}.}

\label{fig:app-eval-synthetic}
\vspace{-2.2em}
\end{figure}




\ps{Provide remaining simulation details}

We set the parameters of RapidChiplet and BookSim2 to match the latencies described in \Cref{tab:homo-params,tab:hetero-params}. 
BookSim2 models input-queued \gls{vc} routers with a four-stage pipeline (routing, \gls{vc} allocation, switch allocation, crossbar traversal) and wormhole flow control.
We use 1-flit packets for control messages and 9-flit packets for data transfers \cite{netrace-tr}.
Furthermore, we use shortest path routing, up to 8 virtual channels, and 8-flit buffers.


\begin{figure*}[h]
\centering
\captionsetup{justification=centering}
\includegraphics[width=1.0\textwidth]{img/evaluation/eval_partial_trace_combined.pdf}
\caption{\textbf{(\textsection \ref{ssec:eval-trace-partial}) Results for the partial trace regions:} speedup in average packet latency compared to the baseline.}
\label{fig:eval-trace-partial}
\vspace{-1.75em}
\end{figure*}
	% Fix figure placement


\subsection{Performance Comparison using Synthetic Traffic}
\label{ssec:eval-synthetic}

\ps{Explain which synthetic traffic we use and why we care about synthetic traffic.}

We compare our optimized \gls{ici} topologies against the baselines in terms of latency and throughput using synthetic \gls{c2c}, \gls{c2m}, \gls{c2i}, and \gls{m2i} traffic.
The advantage of synthetic traffic over real traces is its generality, as synthetic traffic does not depend on the application.
\Cref{fig:eval-synthetic,fig:app-eval-synthetic} show the latency and throughput results under synthetic traffic for the \textit{baseline} and  \textit{\name} chiplet configurations, respectively.

\ps{Discuss results on synthetic traffic: Latency}

Recall that our primary optimization goal was to minimize \gls{c2m} and \gls{m2i} latency and to improve \gls{c2m} and \gls{m2i} throughput.
We observe that for all combinations of architecture and optimization algorithm, \name~improves \gls{c2m}, \gls{c2i}, and \gls{m2i} latency.
The fact that the baseline provides the best \gls{c2c} latency is not surprising, given that in the baseline, compute-chiplets form a regular grid with a 2D mesh topology.


\ps{Discuss results on synthetic traffic: Throughput}

\name~is only able to significantly outperform the baseline architecture in terms of \gls{c2m} and \gls{m2i} throughput if we use the \textit{\name} chiplet configuration, where memory- and IO-chiplets have four \gls{phys} and relay-capabilities. The \textit{baseline} configuration with only a single PHY per memory- and IO-chiplet turns out to be too restrictive to provide significant throughput improvements.

\begin{figure}[h]
\centering
\captionsetup{justification=centering}
\includegraphics[width=1.0\columnwidth]{img/evaluation/eval_full_trace.pdf}
\caption{\textbf{(\textsection \ref{ssec:eval-trace-full}) speedup over baseline in average packet latency} (blackscholes trace, \textit{baseline} configuration).}
\label{fig:eval-trace-full}
\end{figure}

\begin{figure}[h]
\centering
\captionsetup{justification=centering}
\includegraphics[width=1.0\columnwidth]{img/appendix/eval_full_trace_appendix.pdf}
\caption{\textbf{(\textsection \ref{ssec:eval-trace-full}) speedup over baseline in average packet latency} (blackscholes trace, \textit{\name} configuration).}
\label{fig:app-eval-trace-full}
\end{figure}


\subsection{Performance Comparison on Full Traffic Trace}
\label{ssec:eval-trace-full}

\ps{Explain the two trace modes we use}

We evaluate the performance of our optimized \gls{ici} topologies using the full blackscholes-trace (see \Cref{tab:eval-traces}).
We simulate this trace in two different simulation modes:
In the \emph{authentic} mode, a packet is only injected if all dependencies are satisfied and the cycle, in which the packet appears in the trace, is reached.
This represents a scenario where after receiving a packet, the compute cores need some time to perform computations before injecting the next packet.
The second mode is called \emph{idealized} and it injects a packet as soon as all dependencies are satisfied, assuming ideal cores that perform computations instantly.
This mode is intended as a stress-test for the \gls{ici} as the packet injection rate is significantly higher than in the \emph{authentic} mode.
Our results in \Cref{fig:eval-trace-full,fig:app-eval-trace-full} show that \name~is able to achieve speedups in average packet latency of up to $1.17\times$ (for the \textit{baseline} configuration) and $1.34\times$ (for the \textit{\name} configuration).


\subsection{Performance Comparison on Partial Traffic Traces}
\label{ssec:eval-trace-partial}


\ps{Discuss results on partial traffic traces}

\Cref{fig:eval-trace-partial} shows our results for the simulation of partial trace regions.
\name~is able to reduce the average packet latency to $92\%$ (\textit{baseline} configuration) and $82\%$ (\textit{\name} configuration) on average.
In \Cref{ssec:homo-opt,ssec:hetero-opt} we observed that the \gls{ga} performed significantly better than \gls{br} with respect to the minimization of the cost function.
However, in our partial trace simulation, we see that this is not always the case and in some instances, \gls{br} is even better than the \gls{ga}.
This shows that either our performance estimate or our cost function does not fully reflect the performance on real traces. 
Nevertheless, co-optimizing the chiplet placement and \gls{ici} topology works, as we outperform the baseline architecture in almost all cases.

\subsection{Area Comparison}
\label{ssec:eval-area}

\ps{Compare Area: No area loss compared to manually placed chiplets}

The area of all homogeneous placements for a given architecture is identical, therefore, we only discuss the area of heterogeneous placements. 
For the 32-core architecture, \gls{br} and \gls{sa} increase the area by $5.4\%$ and $0.8\%$ respectively, but the \gls{ga} reduced the area by $8.1\%$ compared to the baseline.
For the 64-core architecture, \gls{br} and \gls{sa} both increase the area by $3.3\%$ but the \gls{ga} reduced the area by $6.3\%$ compared to the baseline.
We conclude that \name~is able to improve the \gls{ici}-performance without introducing significant area overheads.












\section{Towards real-world use cases}

While the previous study results suggest that our proposed model could effectively compute the noticeability of redirection under various basic visual stimuli (transparency-, color- and scale-based), we aimed to explore how the model could be used in real-world scenarios.
To showcase the potential benefits of our model in practical use cases, we implemented \textbf{an adaptive redirection technique} and developed \textbf{two real-world applications} to demonstrate its generalizability and usability.
We also performed a proof-of-concept study to gather user feedback while interacting with the two applications and the adaptive redirection technique.

% \delete{
% Our regression model can compute the noticeability of motion offsets under various visual stimuli.
% We envision that the model can support various interaction techniques by adjusting the noticeability of motion offsets to match contextual requirements.
% In the following, we first outline how our model could potentially be utilized by content creators and then we demonstrate its applicability in two scenarios.
% Although the system has not been formally evaluated through user studies, we believe that they showcase the potential benefits that our regression model could provide.
% }

% \delete{
% Based on our noticeability regression model, we implemented two applications to demonstrate future interaction scenarios.
% In the applications, our regression model was integrated with the system to compute the noticeability using the user's eye behavioral patterns in the last 30 seconds.
% Although the system has not been formally evaluated through user studies, we believe that they showcase the potential benefits that our regression model could provide.
% }

\begin{figure}[t]
    \centering
    \begin{subfigure}{0.45\columnwidth}
        \includegraphics[width=0.9\columnwidth]{figures/applications/app1.png}
        \caption{}
        \label{figure:application_adaptive}
    \end{subfigure}
    \centering
    \begin{subfigure}{0.45\columnwidth}
        \includegraphics[width=0.9\columnwidth]{figures/applications/app2.png}
        \caption{}
        \label{figure:opportunistic}
    \end{subfigure}
    \caption{
    We developed two real-world applications to demonstrate the capabilities of our adaptive redirection technique:
    (a) Adjusting the difficulty of VR action game: 
    In this application, mid-air coins and monsters serve as visual cues for the target poses that users are asked to perform. 
    Our adaptive redirection technique enables the system to adjust the game’s difficulty without the user noticing, ensuring a balanced and engaging experience.
    (b) Opportunistic rendering for boxing training in VR : 
    Here, users are learning boxing movements by following a blue avatar. 
    Leveraging our adaptive redirection technique, the system can simulate opportunistic rendering which reduces requirements for computation resources.}
    \label{figure:application}
\end{figure}

\subsection{Adaptive motion redirection technique}
As discussed in the Introduction (\autoref{section:introduction}) and Related Work (\autoref{section:related_work}), users in real VR applications may face complicated visual effects that can impact the noticeability of redirection movements. 
This, in turn, influences the effectiveness and overall user experience of redirection techniques.
While it is impractical to predict the specific visual effects users will encounter beforehand, content creators can only predefine a static redirection intensity, which limits the effectiveness of redirection techniques. 
To address this limitation, our proposed model enables designers to dynamically adjust the redirection during usage based on the user’s gaze behavior.

Our model computes the noticeability of redirection as a float value ranging from 0 to 1. 
With this output, we implemented an adaptive redirection technique by using the Three-class model described in \autoref{section:classfication_model}.
For each class of noticeability, we predefined corresponding redirection: 25 degrees for Low Noticeability, 15 degrees for Medium Noticeability, and 5 degrees for High Noticeability. 
When the computed noticeability falls into one of these classes, the corresponding redirection is applied.
The redirection technique initializes with a 10 degree offset. 
When a change in redirection is required according to the noticeability changes, we use linear interpolation to transition the redirection gradually over a 10-second period. 
To maintain immersion, the redirection is adjusted only when the user’s arm is in motion, since if the redirection changes while the physical arm remains static, the virtual arm will be moved and lead to break of immersion and sense of embodiment.

To be noted, this adaptive redirection technique serves as a demonstration of the usability of our proposed model. 
Designers can leverage the model's probabilistic output to create their own redirection techniques tailored to specific applications.

% \subsection{\delete{Adjusting motion offsets to meet noticeability requirements}}
% \delete{
% Currently, designers typically aim to minimize the noticeability of motion offsets to preserve the user's sense of embodiment when employing redirection techniques in VR interactions. 
% However, the challenge lies in the difficulty of predicting the exact visual animations or effects that users will experience in advance, limiting designers to pre-setting motion offset thresholds.
% For instance, in the context of an action game, a designer might adopt a redirection technique to enhance the user experience. 
% The motion offsets might go unnoticed when complex visual effects and animations capture the user's visual attention. 
% However, if these visual effects diminish in intensity, the user's focus may return to their virtual body, making the same motion offset detectable.
% Therefore, we envision that our model could enable designers to dynamically adjust motion offsets by tracking users' gaze patterns and visual attention in real time. 
% This would allow for a more responsive and adaptable redirection technique, ensuring that motion offsets remain unnoticed under varying visual conditions.
% }

% \delete{
% With our model, content creators can design redirection techniques and their noticeability requirements in advance, while leaving their intensity to our model to decide. 
% During runtime, the interactive system can maintain a historical record of users' gaze behavioral data and input it to our model.
% Based on this, our model can simulate various motion offsets and compute their noticeability results.
% Then the interactive system can apply an appropriate motion offset to meet the predefined noticeability requirement.
% }


% \change{
% To further understand the prediction performance of our model, we converted the regression model to a classification model by applying different thresholds and dividing the noticeability into separate categories.
% We first converted the regression model into a binary classification model with a noticeability threshold of 0.4, which could classify the noticeability between low and high visual attention based on \autoref{figure:predictionperformance_average}.
% We trained an SVM classification model with the same features selected in \autoref{section:featureselection}; this model achieved an accuracy of 0.917 $(SD=0.112)$ and an F1-score of 0.896 $(SD=0.134)$ with leave-one-user-out cross-validation.
% Then we divided the noticeability into three categories with two thresholds: Low Noticeability $(\leq 0.33)$, Medium Noticeability $(0.33 <$ noticeability $\leq 0.66)$, and High Noticeability $(>0.66)$.
% With the same SVM classification model and selected feature, our re-trained model achieved an accuracy of 0.856 $(SD=0.124)$ and an F1-score of 0.847 $(SD=0.127)$.
% Notably, the prediction accuracy was affected by how we converted the noticeability value to separate labels and might increase with fine-tuned features tailored to the classification task.
% This indicates that the selected features from the gaze behavioral pattern have the potential to predict the noticeability as separate categories.
% }

\subsection{Real-world applications}

\subsubsection{Adjusting the difficulty of VR action game}
% \delete{
% First, we implemented an adaptive motion offset adjustment based on the status of the user's visual attention (as indicated by their gaze behavior).
% With the user's gaze behavioral data, our model is able to compute the noticeability of motion offsets.
% Thus, designers can limit the noticeability of motion offsets to a desired level by adjusting the motion offsets, based on our model.
% We demonstrate this with a VR action game}
Based on our adaptive redirection technique, we implemented a VR action game inspired by the VR game Beat Saber~\footnote{https://beatsaber.com/}.
In the game, users are asked to perform certain poses with their arms based on visual and musical guidance.
The game difficulty could be adjusted by redirecting the user's movement, for example, slightly amplifying their movements could make it easier and faster to achieve the targets.
Meanwhile, users need to focus on the targets to obtain sufficient information, and thus they paid less visual attention to their virtual body movements.
As shown in \autoref{figure:application_adaptive}, when the visual guidance for the target arm pose is highly detailed and draws significant attention from the user, 
the noticeability of redirection might be lower and 
the system can take the risk of applying large redirection for functional gains.
However, when the user interacts with a simpler interface and focuses mainly on their virtual arm, 
a low level redirection might be applied with the high noticeability prediction.
%designers can just amplify users' motion a bit and provide limited guidance when the visual effects are simple.
%Accordingly, designers can provide more attractive visual effects to provide stronger guidance by applying larger offsets to users' motion.

% Adjust the strength of the offsets according to the user's gaze behavior, as an indicator of their visual attention allocation status, to maintain the same level of noticeability.


\subsubsection{Opportunistic rendering for boxing training in VR}
We implemented a boxing training system designed to reduce rendering computation as our second application.
Accurate motion reconstruction and rendering may require high computing power~\cite{chen2021towards}.
While users may not always focus on their virtual movements, there is a chance to apply opportunistic rendering based on the user's visual attention to save computing capability and avoid being noticed by users.
As shown in \autoref{figure:opportunistic}, the user is learning boxing poses with a virtual coach in VR.
When the user is looking at the coach and observing them performing the pose, our model may output a lower level of noticeability and thus it allows the system to update the user's movement less frequently which leads to the virtual movement has a offset with the user's physical movement and save computing resources.
While the user shifts their attention back to his arm and is going to practice the boxing poses, our model can compute that the noticeability of motion offset is higher than in the previous scenario.
Therefore, the system can allocate more resources to render the user's movement, to ensure that they can perform and learn the accurate poses in VR.
To be noted, we implemented this application as a simulation of opportunistic rendering to demonstrate the potential of our model, rather than fully implementing it and measuring the computational resources it would save.

% When gaze behaviors tells the system, the user's visual focus is attracted by notifications/distractions, it can decide to lower the requirement on the sensing/rendering of the user's body motion.

\subsection{Proof-of-Concept study}

To further demonstrate and evaluate the how our model supports adaptive redirection techniques, we conducted a proof-of-concept evaluation study on two applications.

\subsubsection{Design}

We conducted a within-subject factorial study design, with the independent variable being the experimental conditions, including Adaptive Redirection (\textbf{AR}) and Static Redirection (\textbf{SR}).
In the VR action game, participants were tasked with performing poses that aligned with a moving target. 
The target’s appearance frequency progressively increased, starting at intervals of 2 seconds and accelerating to 0.5 seconds and the game lasted for 60 seconds.
In the boxing training application, participants engaged in a 60-second motion-learning task, attempting to replicate the movements demonstrated by a virtual coach.
For the \textbf{SR} condition, the redirection magnitude was fixed at 15 degrees, which is the same as the medium level magnitude used in the \textbf{AR} condition.
After completing the tasks in each condition, participants rated the tested conditions on physical demand ("\textit{The interaction was physically demanding}"), mental demand ("\textit{The interaction was mentally demanding and I had to concentrate a lot.}"), embodiment ("\textit{I felt as if the virtual body was my body}") and agency ("\textit{I felt like I could control the virtual body as if it was my own body}") with a 7-point Likert scale, using the questions from similar studies in prior work~\cite{peck2021avatar, feick2023investigating}.


\subsubsection{Apparatus \& Procedure}

We implemented the applications with a HTC Vive pro headset in Unity 2019, powered by an Intel Core i7 CPU and an NVIDIA GeForce RTX 2080 GPU. 
During the study, participants were equipped with three Vive Trackers affixed to their left shoulder, elbow, and waist using nylon straps.

After being introduced to the study, participants had a warm-up session to learn about the study tasks and get familiar with controlling the virtual movements.
Once they were comfortable with the virtual movements and tasks, they proceeded to experience one condition across both applications.
After completing the two applications under the first condition, participants provided their ratings before moving on to experience the second condition. 
The order of conditions and applications was counterbalanced.
The study lasted around 20 minutes, and each participant received a compensation of 10 US dollars for their participation.

\subsubsection{Participants}

We recruited 8 new participants (2 females, 6 males, average age of 25.63 with $SD=1.85$) from a local university.
These participants reported their familiarity with VR as an average of 3.75 $(SD=1.16)$ on a 7-point Likert-type scale from 1 (not at all familiar) to 7 (very familiar).

\subsubsection{Result}

\begin{figure}[t]
    \centering
    \includegraphics[width=0.9\linewidth]{figures/applications/preliminary_results.png}
    \caption{Proof-of-concept study results indicate that participants experienced less physical demand and a stronger sense of embodiment and agency when using the adaptive redirection technique compared to the static technique.}
    \label{figure:preliminary_result}
\end{figure}

\autoref{figure:preliminary_result} summarizes the study results.
We conducted Wilcoxon signed-rank tests to analyze the reported subjective metrics.
Participants reported lower physical demand in the adaptive redirection (\textbf{AR}) condition ($M=3.50, SD=1.00$) compared to the static redirection (\textbf{SR}) condition ($M=4.50, SD=0.50$, $W=2.00, p<0.05$). 
This is due to the larger redirection allowed in \textbf{AR} when visual stimuli were intense, reducing the need for extensive physical movement.
Despite the adaptive nature of \textbf{AR}, participants did not perceive a higher mental demand ($M=4.25, SD=0.60$) compared to \textbf{SR} ($M=4.25, SD=0.43$, $W=5.00, p>0.05$). 
This suggests that \textbf{AR} does not introduce additional cognitive effort for participants to control their virtual motion during interactions.
Participants reported a stronger sense of embodiment ($M=5.13, SD=0.60$) and agency ($M=5.25, SD=0.43$) in \textbf{AR} compared to \textbf{SR}, where embodiment ($M=4.13, SD=0.92$, $W=2.50, p<0.05$) and agency ($M=4.13, SD=0.92$, $W=3.00, p<0.05$) were rated lower. 
This can be attributed to the reduced possibility of detecting the redirection in \textbf{AR}, which enhanced participants' sense of control and immersion. 
In contrast, the frequent detection of redirection in \textbf{SR} reduced their sense of agency and embodiment.

These results suggest that our technique effectively adapts the redirection magnitude to the visual stimuli, aligning with the predicted noticeability from our computational model. 
This demonstrate the potential benefits and capabilities of the model in enhancing redirection interactions.

\section{DISCUSSION, LIMITATIONS, AND FUTURE WORK}

% 2. AI expose Stucture,透明度,很容易讀(可以generalize),不知道背後在幹嘛,很難control,我們organize讓user能夠立刻了解這些key attributes。其他AI工具也可以參考這樣的做法。(expose這些attribute,讓使用者可以更精准控制) 1. 提供系統化資訊很快了解2 提供界面很快修改

\subsection{Addressing Barriers to Adoption: Transparency, Accuracy, and User Perceptions in AI Design Tools}
A significant proportion of designers and artists exhibit resistance to the adoption of GenAI tools ~\cite{kawakami2024impact, jiang2023ai}. Although concerns such as copyright and other factors discussed previously play a role, another critical reason for this reluctance is the lack of transparency in these systems ~\cite{zhang2024confrontation, shi2023understanding}. Without a clear understanding of the underlying actions of the system, users struggle to control and communicate with it effectively, ultimately reducing acceptance and adoption~\cite{auernhammer2020human, Usmani2023Human-Centered}. 
% 2nd review
To address this, prior work has explored enhancing interpretability and user control through multi-modal feedback and visualization. XCreation ~\cite{yan2023xcreation} integrates an entity-relation graph to visually map picture elements and their relationships, making generative structures more transparent. In product design, PhotoDreamer ~\cite{zhang2024protodreamer} allows designers to prototype with physical materials while AI interprets their inputs, providing clear feedback on how designs evolve. And AutoSpark~\cite{chen2024autospark} enables fine-grained comparisons to improve text-image relevance. 

On the other hand, AIdeation is specifically designed to meet the needs of concept designers by breaking down brainstorming results into visuals and categorized keywords, helping designers quickly grasp key attributes. Building on this understanding, AIdeation enables designers to fine-tune elements precisely. At each step, it eliminates the traditional need for designers to spend excessive time interpreting generated images or manually crafting and modifying complex prompts, while still preserving high-level control over design directions. As one participant noted: “\textit{Compared to other image-generation tools I've used before, I can clearly see what to do next, making it much more efficient to achieve the desired outcome}” (P14). By enhancing AI transparency and control of creative directions, AI design tools would improve engagement, foster human-AI collaboration, and improve user satisfaction, as proposed by human-centered AI design principles~\cite{shneiderman2022human}.

Hallucination is another critical concern in GenAI, 
Hegazy et al.\cite{hegazy2023evolution} identified potential issues with using GenAI in architectural design, such as a lack of consideration for structural feasibility and inconsistencies in generated outcomes. Similarly, concept designers rely heavily on factual, real-world information\cite{maleki2024ai, monteith2024artificial}, distinguishing them from other design disciplines. Both formative and summative studies revealed that designers hesitate to adopt AI tools due to fears of receiving inaccurate output, compounded by a general preference to avoid over-reliance on others' designs. As one participant (P2) explained: “\textit{I mainly use photos as references and avoid concept art since, despite looking good, it may lack thorough, factual research. AI-generated images have the same problem}.” 
While prior work in architectural design explores pre-trained models and ControlNet~\cite{zhang2023adding} to improve accuracy~\cite{chen2024enhancing}, these methods are unsuitable for concept design due to its broader scope.
To mitigate this problem, AIdeation integrates non-AI image search to provide real-world reference images, supporting the design elements of its generated ideas and aligning with designers' existing workflow for reference gathering. This approach significantly increased designers' willingness to engage with the tool. As another participant (P15) noted, “\textit{Although I still don't like AI-generated images, the keywords and references are very useful}.” These findings, coupled with our observations in Section 7.4.3, highlight the substantial impact of user attitudes and expectations on their experience with AI systems, a conclusion supported by recent research~\cite{kang2024impact}.

% Hegazy et al ~\cite{hegazy2023evolution} stated the possible problems using GenAI in architectural design, such as Lack of consideration for structural design feasibility and Inconsistency of outcomes.

These issues also extend to other domains. While GenAI is powerful, designers need to identify and address the root causes of possible negative attitudes toward it. A user-centered approach is helpful in identifying the root causes, making it possible to design strategies to specifically address each of users' concerns, incorporating both GenAI and traditional approaches as needed. 
%Ensuring systems are trustworthy, transparent, and aligned with workflows is critical. Addressing these challenges can lead to greater adoption and improved user satisfaction in future systems.

% 3. 這個領域有高比例反AI,1. 覺得資訊是錯誤的 (我們的設計就是用真實世界的search reference來support)User 對 AI 的態度差異。要注意User的negative thought。工具好用但不是每個user都這樣。要給什麼建議,針對對AI負面的user (資訊是不正確的) 用真實的東西去Back AI產出的Idea
% \textbf{對ai負面態度的原因很多類型,針對不同原因去設計}

\subsection{Implication for GenAI in Iterative Ideation}
% 4. 不同階段user會要不同的control。拆成不同Phase,creative process convergence divergence。過去的做法是會用slider。我們是拆成brainstorming和refinement (concept designer workflow)。不同creative process,背後概念都是這樣。怎麼去support 這個progression
Unlike existing AI tools commonly used by concept designers, which typically follow a linear, one-step solution, AIdeation adopts a nonlinear and iterative approach that aligns more closely with designers' ideation processes. This design philosophy is similar to frameworks such as OptiMuse ~\cite{OptiMuse} and DesignGPT ~\cite{ding2023designgpt}, recognizing iteration as a fundamental aspect of the design process ~\cite{adams1999cognitive}, and many prior work has incorporated this principles ~\cite{hou2024c2ideas, han2024teams}.
At different stages of the design process, designers may require varying levels of divergent and convergent thinking, along with cognitive processes that balance exploring both breadth and depth. ~\cite{tversky2011creativity, goldschmidt2016linkographic}. Tools such as RoomDreaming used sliders to control the diversity of visual outputs ~\cite{wang2024roomdreaming}, while GenQuery employs visual search and image combination techniques to dynamically shift focus ~\cite{son2024genquery}. In contrast, AIdeation organizes the functionality into modular components, where designers can switch between based on their current needs, providing the flexibility to adapt to different phases of the creative process. 
These concepts apply to most creative processes. Future work could explore how GenAI can support different stages of ideation across various creative domains while allowing users to seamlessly switch between them.

During interviews, many designers highlighted that AIdeation was significantly easier to control and communicate with compared to other AI tools they had used. One participant noted, “\textit{I feel that AIdeation can effectively understand how I wish to modify the current idea}” (P6). This observation highlights the importance of systems that understand user intentions and support clear and effective communication. ~\cite{verganti2020innovation, shneiderman2022human}. Previous work, such as IntentTuner, has proposed frameworks to integrate human intentions into fine-tuning general image generation systems ~\cite{zeng2024intenttuner}. In contrast, AIdeation uses domain-specific knowledge to guide each AI module, ensuring that it aligns with the different phases of concept design. This approach improves communication between the tool and designers.

The principles behind AIdeation can guide the future development of AI-assisted design tools. One promising direction is exploring how GenAI can better support collaboration, enabling directors and designers to co-create in shared workflows. Such systems could act as a communication bridge,  integrating team inputs and supporting both broad exploration and focused refinement. This aligns with the frameworks of Han et al., which highlights AI’s role in enhancing team creativity ~\cite{han2024teams}.

\subsection{Integrating GenAI into the Design Workflow with a Human-Centered AI Approach}
% 1. Revisit Contribution and Research Goal
% We are integrating AI into a group that already using GenAI
% Objective: Restate the research goals and contributions of AIdeation.
% Highlight how AIdeation fills gaps in existing tools by combining breadth and depth in idea exploration. ()
% Note its role in advancing HCI knowledge by supporting divergent and convergent thinking in iterative workflows.
% Notice: Emphasize novelty and address specific challenges (e.g., inefficiency in traditional workflows, limited control in current AI tools).

% 1. 從最broad開始講,大家都用AI tool, 但沒效率。各個domain都有這樣的work。
% Highlight human centered approach, 可以更符合使用者需求, 有兩間還在用
%The findings from both studies indicate that AIdeation effectively integrates multiple GenAI models to support the ideation process and address the complex workflows of concept designers. 

While GenAI tools are increasingly used by designers across various domains, research shows they often fail to align with user-centered design principles. These shortcomings often result in user reluctance and inefficiencies~\cite{vimpari2023adapt, zhang2024confrontation, mahdavi2024ai}. Aligned with established principles of human-centered AI design ~\cite{shneiderman2022human, xu2023transitioning, auernhammer2020human}, AIdeation provides a solution that prioritizes the needs and workflows of concept designers.

Previous research in various design domains has demonstrated the use of GenAI to simplify nuanced tasks, enabling designers to rapidly explore various visual concepts~\cite{wang2024roomdreaming, davis2024fashioning, oh2024lumimood}. Furthermore, studies have demonstrated the effectiveness of AI multi-agent collaboration in managing complex tasks~\cite{talebirad2023multi, de2024llmr}. 
AIdeation, on the other hand, deconstructs complex workflows into modular tasks, combining both suitable AI modules and non-AI tools for each phase and integrating them into a cohesive workflow for concept designers. This approach eliminates labor-intensive steps while retaining essential creative decisions, allowing users to focus on the core creative aspects of their work. In this context, GenAI functions as a tool to augment human capabilities ~\cite{chen2023next}. As one participant remarked, “\textit{Using AIdeation felt like being an art director, with multiple design assistants gathering information and proposing ideas}” (P13).

A similar approach can be generalized to other design domains that involve multiple phases of ideation, prototyping, and refinement, such as fashion, graphic, architectural, and industrial design~\cite{camburn2017design, carlgren2016framing}. Although many design fields have already integrated AI tools into their workflows~\cite{anantrasirichai2022artificial}, these tools often do not align with domain-specific needs, which presents a significant opportunity for HCI researchers to bridge this gap.
Instead of relying on one-size-fits-all AI solutions, researchers should use domain expertise to integrate the right tools, AI or otherwise, into workflows and ensure designers retain control over core creative decisions. This approach results in systems that better meet user needs and outperform traditional or purely AI-driven solutions.


\subsection{Limitations and Future Work}
\subsubsection{Limitations of the study}
% Discuss study duration, self report method
% Reflect on generalizability to other contexts or domains.
% Notice: Frame limitations constructively, linking them to potential future studies.
Due to the difficulty of including the entire ideation process in our summative study and the challenge of directly comparing the results of the ideation between conditions, we relied mainly on self-reported data, which is a limitation of this work. While a follow-up field study evaluated real-world design outputs with input from designers, directors, and clients, it lacked quantitative measures and had less control compared to lab studies. Future research could explore longer summative sessions focused on narrower tasks, like designing a single prop, which is simpler than broader tasks like environment design.

\subsubsection{Controllability}
% Objective: Discuss the trade-offs between user control and automation in AIdeation.
% Address how increased control could either support or hinder creative exploration.
% Notice: Use participant feedback to support arguments about user needs for control.
Although AIdeation emphasizes idea exploration, participants noted its limitations in controlling specific details of generated results. Features like "combine with the reference" and "refine by instruction" provide high-level control but lack the ability to adjust elements such as lighting, atmosphere, camera angles, and composition while preserving other elements. These aspects remain challenging and are active areas of AI research. As one participant (P4) remarked, “\textit{The system covers 70-80\% for client communication, but control over lighting, atmosphere, and camera angles is needed for the final 20\%}.” As AI technology continues to advance, such controllability features could be integrated into AIdeation. Future iterations of AIdeation could integrate such detailed controls to better support designers' focus and refinement during the convergence phase of their work.

\subsubsection{Customization and personalization}
% Customization and Personalization
% Objective: Reflect on how AIdeation can be personalized to suit diverse workflows.
% Explore adaptive AI features, such as tailoring outputs to user preferences or expertise levels.
% Notice: Highlight ethical implications, like avoiding bias and ensuring inclusivity.
Many users noted the limited diversity in art styles, atmosphere, and camera angles, largely due to the constraints of the image generation model used in AIdeation. Different models have distinct strengths; for instance, users appreciated MidJourney for its aesthetic quality, while Stable Diffusion, fine-tuned with LoRA ~\cite{hu2021lora}, offers more style variety and specialized designs. Future updates could let users select specific styles or atmospheres, choose fine-tuned models, or allow the system to automatically pick the most suitable model based on input. Another option could be to generate multiple outputs from different models to better match the design task.

Beyond image generation tuning, AIdeation can be personalized to fit the design field, the designer's specialization, and work habits, similar to the ideas proposed by Long et al.~\cite{long2024not}. The system could adapt to various design domains by modifying the prompts or highlighting specific design elements to better suit individual users. For instance, designers could select a focus, such as environments, props, or characters, and AIdeation would generate customized output accordingly. Although the system currently lacks the ability to retain context from previous sessions, future updates could include memory features and personalized recommendations. Furthermore, incorporating self-adaptive capabilities, where the system adjusts its behavior based on user preferences or current work stage, could further improve its effectiveness, as suggested in previous research ~\cite{macias2013self}.


% \subsection{Controllability, Customization, and Personalization of GenAI}
% Another limitation noted by many users is the lack of diverse options in art style, atmosphere, and camera angles, primarily due to the constraints of the image generation model behind AIdeation. Different models have their strengths; for example, several users praised MidJourney for its aesthetic quality, while Stable Diffusion’s fine-tuning through LoRA ~\cite{hu2021lora} offers a variety of styles and specialized designs. Future improvements could allow users to intentionally select specific styles or atmospheres, choose corresponding fine-tuned models, or let the system automatically select the best model based on input. Alternatively, the system could generate multiple results from different models to better suit the design task.

% Beyond image generation tuning, AIdeation can also be personalized based on the design field, designer specialization, and work habits. As discussed in Section 8.2, the system could be customized for various design areas by adjusting prompts or emphasizing specific design elements that align with the user’s needs. For instance, designers could select their focus—whether environments, props, or characters—and AIdeation would generate results tailored to that area. While AIdeation currently lacks context from previous sessions, future versions could integrate memory and personalized recommendations, adapting to ongoing projects and user preferences for more targeted results.


% 2. Contextualize Findings
% Objective: Connect key findings to broader themes in HCI and GenAI.
% Designer rely on getting info <-> Brainstorm loop. AIdeation provided continuously rapid stimulation of both stages.
% AIdeation improves ideation efficiency and creativity through iterative refinement.
% It supports breadth (diverse idea generation) and depth (focused exploration), aligning with HCI principles of flexibility and user empowerment.
% Human Centered AI design with workflow integration. Increase quality and creativity than pureAI and human traditional workflow. Also improving designer's accepabiliy
% new guideline of disucssion


% GAI 快速實驗
% 1. 從最broad開始講,大家都用AI tool, 但沒效率。各個domain都有這樣的work。
% Highlight human centered approach, 可以更符合使用者需求, 有兩間還在用

% 2. AI expose Stucture,透明度,很容易讀(可以generalize),不知道背後在幹嘛,很難control,我們organize讓user能夠立刻了解這些key attributes。其他AI工具也可以參考這樣的做法。(expose這些attribute,讓使用者可以更精准控制) 1. 提供系統化資訊很快了解2 提供界面很快修改


% 3. 這個領域有高比例反AI,1. 覺得資訊是錯誤的 (我們的設計就是用真實世界的search reference來support)User 對 AI 的態度差異。要注意User的negative thought。工具好用但不是每個user都這樣。要給什麼建議,針對對AI負面的user (資訊是不正確的) 用真實的東西去Back AI產出的Idea
% \textbf{對ai負面態度的原因很多類型,針對不同原因去設計}

% 4. 不同階段user會要不同的control。拆成不同Phase,creative process convergence divergence。過去的做法是會用slider。我們是拆成brainstorming和refinement (concept designer workflow)。不同creative process,背後概念都是這樣。怎麼去support 這個progression

% 2. GenAI + Image Search Integration 
% Ideation 我們和其他人有什麼不一樣
% Notice: Mention unexpected findings or contrasts with prior research, emphasizing the novelty of results.

% 3. Expectations and Attitudes
% Objective: Explore how user expectations shape their experience with AIdeation.
% Attitude and Creativity
% Users expecting precise, finished designs often overlook exploratory benefits, while those with an open mindset utilize AIdeation more effectively.
% Discuss the implications of these findings for system onboarding and training.
% Notice: Use participant quotes to substantiate claims, linking findings to broader HCI challenges like trust, usability, and cognitive load.


% 2.1 Usage in the Same Domain
% Objective: Highlight how AIdeation can be further applied within environment concept design.
% Address specific scenarios, such as collaborative workflows or large-scale projects.
% Reflect on how iterative refinement could enhance multi-designer collaboration or client communication.
% Notice: Provide concrete examples to clarify potential applications.

% 2.2 Application in Other Design Domains
% Objective: Propose extending AIdeation’s principles to other domains.
% Suggest domains such as architecture, fashion, industrial design, or education.
% Emphasize how domain-specific adaptations (e.g., integrating specialized datasets) could improve relevance.
% Notice: Acknowledge challenges, such as adapting to domain-specific workflows or accommodating varied data needs.
% 我們學到:每個domain 都有 generalize的tool,但不太符合現在的workflow. 所以要針對使用者的workflow去。

% 2.3 Broader Lessons
% Objective: Derive generalizable lessons for designing AI systems.
% Reflect on how AI systems can integrate into workflows across creative domains. Like specific Prompt engineering and Design modules cooperation
% Highlight potential interdisciplinary creative applications, e.g., in collaborative or story-telling.

% Notice: Address ethical considerations, such as bias, over-reliance on AI, and maintaining human creativity.

% \subsection{Limitations and Future Work}
% \subsubsection{Limitation of Study}
% Discuss study duration, self report method
% Reflect on generalizability to other contexts or domains.
% Notice: Frame limitations constructively, linking them to potential future studies.


% \begin{figure*}
%     \centering
%     \includegraphics[width=1\linewidth]{figures/11 Design tools comparison.png}
%     \caption{A comparison between the initial outputs from AIdeation and DALL-E 3 on ChatGPT, using the same input provided by Field Study Participant S3, revealed notable differences. The participant observed that AIdeation produced designs with significantly greater diversity and richness compared to those generated by DALL-E 3 on ChatGPT.}
%     \label{fig:design tool}
% \end{figure*}

% Figure 11 compares the output from AIdeation with Dall-E 3 on ChatGPT using the same input provided by one of our field study participants (S3). With just a single well-crafted prompt through the LLM before engaging the image generation model, the diversity and richness of the designs increased significantly, as noted by a designer (S3). 

% In this research, AIdeation focuses on environment concept design, but with fine-tuned prompts and specific references, the same approach can be adapted for characters, props, and other design areas. This method could also extend to other domains by breaking down complex tasks, with each LLM handling specific subtasks (similar to \cite{de2024llmr}), supporting designers across various fields.

% \subsection{Implication for GenAI for Workflow Support}


% \subsection{Expanding AIdeation's Applicability}
% The findings from our study offer valuable insights into how AIdeation can enhance concept designers' workflows. As highlighted in the user study, AIdeation can be effectively used during meetings to communicate with clients or directors (P3–P4, P7–P9, P13–P14, P16). Its ability to generate diverse outputs and allow flexible real-time refinements enables designers to present multiple ideas instantly and adjust them based on feedback, significantly reducing the traditional trial-and-error process typical in designer-client collaborations.

% Another potential case is when designers need to explore multiple design topics based on the same theme. For example, when designing environments, props, and characters in a related game region, designers can start with an initial concept, such as an exterior reference, and type commands like “design a flying machine based on this environment” (P4). “This narrative-driven exploration process could greatly enhance the ideation workflow” (P10) for concept designers, particularly in the development and pre-production stages.


%Mike comments
%what did we learn? I sthere anything that we can improve or that the scientific community can learn from our work? What part do they don't like, the users? Look at roomdreaming. what are the things that can be accepted and what are the things that cannot be acceptable. What 建議 can we give? Because there are some of the users have a bad view 


\subsection{Limitations and Future Work}

% arm poses
In our user studies, we treated the ending arm poses that we applied redirection on as a control variable.
We clustered 25 arm poses from the CMU MoCap dataset~\cite{CMUMocap} that are common poses in real-life activities, randomly selecting and testing one of them in each trial.
This enabled us to average the impacts of different arm poses and focus on the influence of visual stimuli on redirection noticeability.
However, we acknowledge that the selected pose set is still limited in size compared to the amount of arm poses that are possible to perform in real life.
We regard extending our study to include more arm poses and apply redirection on other body parts as important future work.

% personalized
% physiological data
%Regarding the implementation of our prediction model, we leveraged the user's gaze behavioral patterns to predict the noticeability of the applied offsets.
We implemented the regression model with the data from 12 users and evaluated it with another 24 new users.
The results showed that our model could compute the noticeability accurately with new users while they experienced novel visual stimuli that never appeared in the training set.
However, we envision that a personalized model could improve the regression performance by collecting more data from the same user and capturing their unique behavioral patterns more accurately.
In addition, as we primarily focus on modeling the relationship between the visual stimuli and the redirection noticeability, we adopted SVR in the implementation of the regression model as it is relatively stable and did not overfit.
We note that when applying the findings into real life applications, more advanced regression/classification methods (e.g., deep learning models) and more fine-tuned parameters are worth exploring to optimize the regression performance.
As past work has demonstrated the relationship between hand redirection noticeability and users' physiological data~\cite{feick2023investigating}, we will explore how to add physiological data into the regression model in the future.

% abstract and controlled visual attractions
% more realistic tasks
% long-term adaptation
We investigated the effects of visual stimuli on noticeability and implemented a regression model with highly-controlled study designs and abstract visual stimuli.
Our goal was to study whether and how visual stimuli affects noticeability by controlling the factors and showcasing the potential applications that can benefit from our model.
We regard it as important future work to investigate the effect in a field study with more realistic tasks.
We will also further generalize our contribution with a longitudinal study to consider how users adapt their interaction patterns to redirection over time.


% Regarding the user studies: 
% 1) coverage of possible arm poses by the tested set;
% 2) nature of being highly controlled and abstract;

% Regarding the implementation:
% 1) taking arm pose into consideration
% 2) taking user's adaptivity into consideration
% 3) personalization
% 4) connect to the biophysical signals (?)
% 5) arm motion -> body motion

% Generic:
% 1) field study with more realistic tasks
% 2) longitudinal study on long-term adaptation



\section{Conclusion}
We study the optimal testing procedures in two extreme environments: the agent improves either  apparent performance (manipulate) or actual performance (invest) but never both. We show that the optimal testing procedure is sequential with fixed order when the agent manipulates and it is simultaneous when the agent invests.
We apply our model to explain the different banking regulatory practices in Europe and in the US, the joint regulation of merger and acquisition across departments, and the collective decision making involving subjective opinions.
Our paper points out several potential paths of future research.
First, the comparison between the incentive schemes or institutional designs facing a manipulating agent and an investing agent is fruitful. 
Second, we consider a setting where a principal has multiple requirements which can potentially contradict each other, but it is also interesting to relax the commitment assumption to study conflicting principals.
Third, one can also consider an agent with richer technology space, i.e., an agent that can both manipulate and invest.

%%
%% The acknowledgments section is defined using the "acks" environment
%% (and NOT an unnumbered section). This ensures the proper
%% identification of the section in the article metadata, and the
%% consistent spelling of the heading.
\begin{acks}
We thank Yu Jiang for her research insights.
This work is supported by the National Key Research and Development Program of China under Grant No.2024YFB2808803 and the Natural Science Foundation of China under Grant No. 62102221, 62132010, 62472244, the Tsinghua University Initiative Scientific Research Program, and the Undergraduate Education Innovation Grants, Tsinghua University.
\end{acks}

%%
%% The next two lines define the bibliography style to be used, and
%% the bibliography file.
% \normalem
\bibliographystyle{ACM-Reference-Format}
\bibliography{reference}

%%
%% If your work has an appendix, this is the place to put it.
\appendix

\section*{Appendix}
\label{sec:appendix_section}
%Supplementary material goes here.
\renewcommand{\thesubsection}{\Alph{subsection}}
\setcounter{subsection}{0}
\subsection{Additional Qualitative Comparison}
Figures~\ref{fig:more_1} and \ref{fig:more_2} present additional qualitative comparisons.
SAM~\cite{alaluf2021only} exhibits low inversion accuracy, resulting in inadequate modifications for both younger and older age transformations.
CUSP~\cite{gomez2022custom} demonstrates strong performance in editing for younger ages but struggles to maintain identity preservation across all target ages.
FADING~\cite{chen2023face} maintains identity well due to its high inversion accuracy, but it frequently fails to perform successful age modifications.
In contrast, our method achieves more consistent and stable age editing.
Furthermore, thanks to our prompt design, editing performance for both younger and older ages has significantly improved.
Notably, our method produces natural changes in hair volume and color, which are crucial for realistic age transformations.

\begin{figure*}[h]
  \centering
  \includegraphics[height=1.\textheight]{fig/appendix1-eps-converted-to.pdf}
  \caption{
  Additional qualitative comparison between our method and the existing methods~\cite{alaluf2021only,gomez2022custom,chen2023face}.
  }
  \label{fig:more_1}
\end{figure*}

\begin{figure*}[h]
  \centering
  \includegraphics[height=1.\textheight]{fig/appendix2-eps-converted-to.pdf}
  \caption{
  Additional qualitative comparison between our method and the existing methods~\cite{alaluf2021only,gomez2022custom,chen2023face}.
  }
  \label{fig:more_2}
\end{figure*}

\end{document}
\endinput
%%
%% End of file `sample-authordraft.tex'.
