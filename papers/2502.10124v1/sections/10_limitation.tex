\subsection{Limitations and Future Work}

% arm poses
In our user studies, we treated the ending arm poses that we applied redirection on as a control variable.
We clustered 25 arm poses from the CMU MoCap dataset~\cite{CMUMocap} that are common poses in real-life activities, randomly selecting and testing one of them in each trial.
This enabled us to average the impacts of different arm poses and focus on the influence of visual stimuli on redirection noticeability.
However, we acknowledge that the selected pose set is still limited in size compared to the amount of arm poses that are possible to perform in real life.
We regard extending our study to include more arm poses and apply redirection on other body parts as important future work.

% personalized
% physiological data
%Regarding the implementation of our prediction model, we leveraged the user's gaze behavioral patterns to predict the noticeability of the applied offsets.
We implemented the regression model with the data from 12 users and evaluated it with another 24 new users.
The results showed that our model could compute the noticeability accurately with new users while they experienced novel visual stimuli that never appeared in the training set.
However, we envision that a personalized model could improve the regression performance by collecting more data from the same user and capturing their unique behavioral patterns more accurately.
In addition, as we primarily focus on modeling the relationship between the visual stimuli and the redirection noticeability, we adopted SVR in the implementation of the regression model as it is relatively stable and did not overfit.
We note that when applying the findings into real life applications, more advanced regression/classification methods (e.g., deep learning models) and more fine-tuned parameters are worth exploring to optimize the regression performance.
As past work has demonstrated the relationship between hand redirection noticeability and users' physiological data~\cite{feick2023investigating}, we will explore how to add physiological data into the regression model in the future.

% abstract and controlled visual attractions
% more realistic tasks
% long-term adaptation
We investigated the effects of visual stimuli on noticeability and implemented a regression model with highly-controlled study designs and abstract visual stimuli.
Our goal was to study whether and how visual stimuli affects noticeability by controlling the factors and showcasing the potential applications that can benefit from our model.
We regard it as important future work to investigate the effect in a field study with more realistic tasks.
We will also further generalize our contribution with a longitudinal study to consider how users adapt their interaction patterns to redirection over time.


% Regarding the user studies: 
% 1) coverage of possible arm poses by the tested set;
% 2) nature of being highly controlled and abstract;

% Regarding the implementation:
% 1) taking arm pose into consideration
% 2) taking user's adaptivity into consideration
% 3) personalization
% 4) connect to the biophysical signals (?)
% 5) arm motion -> body motion

% Generic:
% 1) field study with more realistic tasks
% 2) longitudinal study on long-term adaptation


