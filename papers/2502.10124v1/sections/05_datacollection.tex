\section{Data Collection}
\label{section:datacollection}
After confirming that visual stimuli influence the noticeability of redirection, we conducted another user study using the same dual-task design to gather more data for developing a prediction model. 
This model aims to estimate noticeability based on users' gaze behavior.

% \delete{
% Next, we conducted a study to collect necessary data for constructing a model of the noticeability of arm motion offsets, taking visual attention into account.
% We built a dataset containing the participants' behavioral data and the corresponding noticeability results of arm motion offsets.
% The offsets had a constant strength of 20 degrees while participants were exposed to an opacity-based visual effect with different levels of intensities and displayed in different areas in the field of view.
% }

\subsection{Design}

To collect the noticeability results more accurately, we measured the noticeability of each redirection magnitude repeatedly for each participant, and tested on less redirection magnitude levels.
In future work, we consider it important to extend the experiments to include a wider range of redirection magnitude.s 
%It is worth noting that previous research has already explored the impact of offset strengths on noticeability~\cite{li2022modeling}.
%Additionally, since previous works had explored the influence of offset strength on noticeability, we conducted the data collection study with a fixed offset strength to predict the influence of visual attractions.
As per prior work that investigated the impact of redireciton magnitude on noticeability, we chose 20 degrees as the tested magnitude, as the reported noticeability rate was around 75\% without visual stimuli in~\cite{li2022modeling}. 
The relatively high rate allowed us to detect the impact of visual stimuli effectively.
% We verified the reported rate through a characterizing study. 
We randomly selected horizontal or vertical as the redirection direction.
%We envision that constructing a prediction model which considers the offset strength and the visual attractions simultaneously as an important future work.

We adopted the same dual-task design with a yes/no paradigm detailed in \autoref{section:methodology}.
As our formative study results showed, the visual stimuli with medium duration ($(M=0.46,~SD=0.07)$) did not yield statistically significant differences in terms of noticeability compared to the stimuli with long duration ($(M=0.50,~SD=0.07),~(t(15)=-0.39,~p>0.05)$ ).
Therefore, we excluded the medium condition and only selected the short (0.2s) and long duration (2s) to control the intensity in this study.
To further control the position of visual stimuli within participants' visual field, we displayed the stimuli at central vision (5 degrees from the central point of vision), near peripheral vision (30 degrees), and mid peripheral vision (60 degrees), illustrated in \autoref{figure:datacollectionlayout} and as in prior research~\cite{grosvenor2007primary, gutwin2017peripheral}.
Therefore, we had $2~\times~3~=~6$ conditions, named as \textbf{CS} (central layout with short duration), \textbf{CL} (central layout with long duration), \textbf{NS} (near peripheral layout with short duration), \textbf{NL} (near peripheral layout with long duration), \textbf{MS} (mid peripheral layout with short duration), and \textbf{ML} (mid peripheral layout with long duration), and we used a Latin square to counterbalance them.
Each participant completed $(2~durations~\times~3~layouts)~\times~24$ measurements $=~144$ trials in total. 

\begin{figure}[!htbp]
    \centering
    \includegraphics[width=0.9\columnwidth]{figures/dataCollection/datacollectionlayout.png}
    \caption{The possible locations of the visual stimuli in the data collection study.
    The locations are divided into three conditions: Central (5 degrees), Near Peripheral (30 degrees), and Mid Peripheral (60 degrees), based on the angular distance to the user's head direction.}
    \label{figure:datacollectionlayout}
\end{figure}


\subsection{Apparatus \& Procedure}
The apparatus and procedure were almost identical to those of our formative study (\autoref{section:formativestudy}). 
We recorded the position and orientation of hand, elbow, shoulder, gaze, and pupil dilation with a sample rate of 60 Hz.
All gaze data was recorded with the HTC Viveo pro built-in gaze tracker.
After the warm-up session, participants took 2-minute breaks after every two sessions to reduce fatigue.
The study lasted around 40 minutes and each participant was compensated with \$15 USD.

\subsection{Participants}
We recruited 12 participants (5 females, 7 males) from a local university.
The participants had an average age of 22.91 years $(SD=1.90)$. All were trichromats and right-handed. 
% All participants with an average age of 22.91 $(SD=1.90)$ were trichromats and right-handed. 
Participants' self-reported their familiarity with VR at an average of 3.17 $(SD=1.27)$ on a 7-point Likert scale from 1 (not at all familiar) to 7 (very familiar).

\subsection{Summary of data statistics}

\begin{figure}[!htbp]
    \centering
    \includegraphics[width=0.9\columnwidth]{figures/dataCollection/noticeability_collection.png}
    \caption{Noticeability results of the data collection study in each condition. The error bars represent the standard errors.}
    \label{figure:datacollectionresult}
\end{figure}

In total, we collected 1728 responses.
To estimate the noticeability, we calculated the ratio of trials in which participants reported noticing the redirection to the total number of trials for each session and participant.
As shown in \autoref{figure:datacollectionresult}, the noticeability result differed across the visual stimuli's duration and layout.
The noticeability results ranged from 16.7\% to 79.2\% with an average of 50.1\% and a standard deviation of 18.9\%.
The maximum and minimum indicate that we controlled the noticeability with the visual stimuli's duration and layout successfully.
Additionally, the high standard deviation suggest a high variability across conditions, which is beneficial for training a model to predict the influence of visual stimuli on noticeability.

To verify that participants' physical movements were effectively redirected, we analyzed participants virtual and physical trajectory lengths as defined in~\autoref{section:study1results}.
In the unnoticed condition, participants' physical movement trajectories were significantly shorter than their virtual ones: \textit{Physical} $(AVG = 1.14,~SD = 0.04)$ and \textit{Virtual} $(AVG = 1.20,~SD = 0.04,~t(11) = 3.97,~p < 0.05)$.
In the noticed condition, participants' physical movement trajectories were also significantly shorter than their virtual ones: \textit{Physical} $(AVG = 1.14,~SD = 0.04)$ and \textit{Virtual} $(AVG = 1.27,~SD = 0.05,~t(11) = 5.38,~p < 0.01)$.
These results indicated that participants' physical movements were successfully redirected.

