

\section{Confirmation Study}
\label{section:formativestudy}

Although it seems evident that adding additional visual stimuli may distract users and influence the noticeability of redirection, we conducted a confirmation study to validate this hypothesis and assess the effectiveness of our dual-task design.

% \delete{
% To investigate whether and how visual attention will influence the motion offset's noticeability, we leveraged visual stimuli to induce gaze saccades and visual attention shifts in users while users were performing hand reaching tasks with redirected motions.
% Specifically, we applied a dual-task paradigm where 16 participants performed an arm motion and decided whether there was a significant difference between the virtual and the physical motion (\textbf{primary task}), while monitoring and reporting when a red ball gradually changed from full transparency to full opacity at randomized locations in the field of view (\textbf{secondary task}).
% We controlled the intensity of the visual stimuli by altering the duration and the location of the animations.
% With the stimuli, we employed a yes/no paradigm in psychophysics~\cite{leek2001adaptive} to record the participants' responses and estimate the noticeability by the ratio of correct responses to the number of trials referring to previous research~\cite{li2022modeling}.
% }

% \delete{
% Though recent works proved that the noticeability of virtual hand offsets could be influenced by eye gaze saccades\cite{zenner2023detectability}, we aimed to further quantify the relationship between users' gaze behavioral patterns and noticeability.
% Therefore, we leveraged visual stimuli to induce gaze saccades and other gaze behaviors in users.
% We investigated how directing the user's visual attention away from their body at different strengths influences the probability of them noticing an offset applied to the avatar's motion with respect to their real motion (i.e., noticeability).
% We adopted the Method of Constant Stimuli~\cite{simpson1988method} to obtain the detection threshold (DT) and noticing probability when presenting the stimuli, including the motion offset and visual animations.
% We controlled the intensity of the stimuli by altering the strength of motion offsets, the duration and the location of the animations.
% To present the stimuli, we applied a dual-task paradigm where 16 participants performed an guided arm motion and decided whether there was a significant difference between the virtual and the physical motion (\textbf{primary task}), while monitoring and reporting when a red ball gradually changed from full transparency to full opacity at randomized locations in the field of view (\textbf{secondary task}).
% With the stimuli, we adopted a two-alternative forced-choice (2AFC) procedure to record the participants' responses and estimate the noticeability by the ratio of correct responses to the number of trials referring to previous research~\cite{li2022modeling}.
% }

% By referring to previous research~\cite{li2022modeling}, we estimate the noticeability by the frequency of participants noticing the offsets.


%invited 16 participants to decide if they noticed the offset under visual attractions with different intensities and positions.

% The primary objective of Study 1 was to investigate the potential impact of visual attention on users' perceptual likelihood of noticing limb offsets. We introduced offsets in varying magnitudes and directions to randomized poses, and combined with diverse levels of visual focus tasks.

\subsection{Design}

%We applied a dual-task paradigm in this study consisting of a primary limb movement task and a secondary visual attraction task. 
%In the primary task, participants executed prescribed movements using their left arm with the VR headset on, while concurrently evaluating virtual movements to determine their consistency with corresponding physical actions. 
% Subsequently, participants reported the immediate appearance position as pertaining to either the left or right half.

%We employed a factorial experimental design.

We employed a factorial study design to manage both independent and control variables.

% \delete{
% To investigate the noticeability under various stimuli, we altered the strength of motion offsets and visual animations (i.e., red balls rendered at different locations in the virtual environment).
% Since we aimed to investigate if visual attention influences the noticeability of the motion offset, we list the intensity of visual stimuli as the independent variable and the noticeability as the dependent variable.
% }

% independent factors
\subsubsection{Independent variables}
In this study, we aimed to investigate whether applying visual stimuli affects noticeability. 
Therefore, our initial independent variable was the presence or absence of visual stimuli. 
To further explore the impact of various visual stimuli, we extended the independent variable to the intensity of visual stimuli, ranging from none to high.
We manipulated intensity by adjusting the duration and placement of virtual animations, following previous studies~\cite{gutwin2017peripheral, li2024predicting}. 
Through a pilot study, we identified three levels of duration: Short (0.2 sec), Medium (1 sec), and Long (2 sec).
For placement, we defined three layout configurations: Sparse (stimuli appear only in the corner areas), Median (stimuli appear in both the corner and peripheral areas), and Dense (stimuli appear throughout the entire field of view), as shown in~\autoref{figure:formalapparatus}.
In each layout, we randomly picked one candidate to animate the visual stimuli.
Additionally, we included a baseline condition with no visual stimuli.
The order of these conditions was randomized.

% \delete{
% To control the intensity of the visual stimuli, we altered the duration and placement of the red ball, since previous works showed that animations with a shorter duration or appearing in the central area are more noticeable than subtle and peripheral ones~\cite{gutwin2017peripheral, li2024predicting}.
% Through a pilot study, we chose three levels for the duration (Short: 0.2 sec, Medium: 1 sec, Long: 2 sec) that were reported to direct participants' attention to different degrees of success.
% Regarding the placement, we defined three layouts (Sparse, Median, Dense) where the stimuli appears only in the corner area, in both corner and peripheral areas, and in all areas of the field of view, respectively. 
% To analyze whether at all the visual stimuli affect the noticeability of offsets, we include a baseline condition without any visual stimuli.
% }

% control factors
\subsubsection{Control variables}
We varied the magnitude and direction of the redirection as control variables. 
Based on the results from related research~\cite{li2022modeling}, we set the redirection magnitude from 0 to 30 degrees with an interval of 5 degrees, which covers the from being unnoticeable (no redirection) to easily noticeable.
We also varied the direction of redirection, sampling both horizontal and vertical directions.
As a result, each participant completed $(3~durations~\times~3~layouts~+~1~baseline)$  $\times (7$ redirection magnitudes $\times 2$ redirection direction - 1) $= 130$ trials in total.
The order of all redirection magnitudes and directions was randomized.

% \delete{
% We tested 25 different ending poses sampled from CMU MoCap dataset~\cite{CMUMocap} to enhance the generalizability of the collected data.
% To focus on the impact of visual stimuli on the noticeability of different offsets, we fixed the magnitude and direction of the offset as control variables.
% Based on findings of related research~\cite{li2022modeling},  we set the offset magnitude from -30 to 30 degrees with a step of 5 degrees, which we believe cover the offset range from being unnoticeable (no offset) to easily noticeable.
% To fix the offset direction, we randomly picked horizontal or vertical direction for each trial, while we ensured the number of trials in each direction to be the same.
% As a result, each participant completed $(3~durations~\times~3~layouts~+~1~baseline)$  $\times 13$ offset strengths $\times 1$ offset directions $= 130$ trials in total.
% }

\subsubsection{Dependent variables}
The noticeability of redirection was recorded as the primary dependent variable in this study and was estimated with the proportion of positive responses across all trials for each condition where redirection was applied.
Additionally, we captured participants' gaze behavior data with the HMD's eye tracker.

% \delete{
% To measure the noticeability of the motion offsets under each condition, we adopted a yes/no paradigm~\cite{leek2001adaptive}.
% Participants pressed the buttons on the VR controller to report a binary response if they observed a stimulus.
% Thus, we leveraged the proportion of the correct responses to the motion offset (hit rate) as an estimation of the noticeability.
% }

\begin{figure}[t]
    \centering
    \includegraphics[width=0.9\columnwidth]{figures/formativeStudy/apparatus.png}
    \caption{The apparatus of the formative study. 
    Wearing a headset, the participant wears three motion trackers to track their arm pose and sit on a comfortable chair. 
    While the virtual avatar mirrors the arm movement of the participant, the participant observes the virtual avatar's movement from a first-person point of view and follows the semi-transparent checkpoint pose to reach the semi-transparent target pose.
    As the secondary task, a virtual animation will start with different durations and locations.
    The right figure illustrates the possible locations of the red ball, named Sparse, Median, and Dense, accordingly.}
    \label{figure:formalapparatus}
\end{figure}


% \subsection{Task}

% \delete{
% To present the stimuli consisting of motion offsets and visual animations, we adopted a dual-task paradigm in this study.
% }

% \subsubsection{Primary task}
% \delete{
% Our primary task required participants to perform a guided arm motion in VR.
% The motion task is defined by the starting arm pose and the ending arm pose. 
% We use a fixed starting arm pose of naturally resting beside the body and sample 25 arm gestures as the ending poses from CMU MoCap database~\cite{CMUMocap}.
% While the participant performed each motion task, we added an offset on the avatar's elbow joint with respect to the participant's real elbow joint.
% Similar to previous research on offset noticeability~\cite{li2022modeling}, the offset is set to be an angular rotation applied to the joint.
% The offset was applied dynamically, beginning with no offset at the initial pose (pointing to the ground) and reaching the maximum offset at the ending pose. 
% The offset during the intermediate motion was calculated based on the relative angular distance from the starting pose and was adjusted linearly throughout the movement.
% Additionally, an intermediate checkpoint pose was added to ensure that participants performed the motion in a consistent manner.
% We adjusted the strength of the offset by altering the maximum offset applied at the ending pose.
% All motion tasks were performed with the left arm.
% }

%Participants were asked to navigate their movements and reach the checkpoint poses before achieving the target pose. 
%Once they achieved the target pose, participants were asked to retrace their movement and return to the initial pose.
%Then we asked participants to decide if they detected any offset during the movement and we measured the noticeability of the offset during the movement with the proportion of times that participants were able to detect it.


%To circumvent the potential detection of the offset arising from the uniform starting pose, a gradual introduction of the angular offset was implemented, progressively incrementing it from zero to the desired target offset magnitude, while participants transitioned from the initial pose to the target posture. Subsequently, once the target pose was attained, participants were instructed to retrace their movement to revert to the initial pointing downward posture. Subsequent to these motion sequences, participants were prompted to ascertain whether they perceived any offset throughout the course of their movements.

%To simulate realistic interaction scenarios, we asked participants to perform dynamic movements from a uniform initial posture of pointing downward to control the moving trajectory.
%To avoid participants noticing the offset from the uniform initial pose, we implemented a gradually increasing process in which we increased the offset from zero to the target magnitude while participants started from the initial pose and moved their left arm to the target pose.
%To control the trajectory of participants' movements, we displayed two checkpoint poses by interpolating the initial pose and the target pose.


% \subsubsection{Secondary task}
% \delete{
% In parallel to performing the primary task, we asked participants to monitor visual stimuli that appeared at random times and locations in their field of view.
% We designed the visual stimuli as a red ball that transitions from full transparency to full opacity and resets to full transparency.
% The red ball was initially registered in a fixed position in the participant's field of view following their head movements, with a fixed depth of 0.5 m and a radius of 0.02 m.
% We instructed participants to report whether the red ball appears in the left or right side of their field of view by pressing the corresponding part of the touchpad on the hand-held controller as soon as they notice it.
% An interval randomly selected from 1-3 seconds is set between every two visual stimuli to ensure that the participant cannot predict when they arise.
% }

%Visual attraction design.

%Participants verbally reported existence and left or right.

%In the secondary task, participants observed visual attractions that appeared randomly within their field of view at unpredictable intervals (1-3 seconds) and reported as soon as they noticed. 




% The target poses were sampled from the domain of human upper limb movement and chosen randomly to serve as the control variable. 
% Independent variables encompassed the magnitude and direction of offsets, in addition to the temporal duration and spatial layout of the visual attractions. 
% The dependent variables included noticeability of offset under conditions of visual focus, measured by the proportion of times that users were able to detect the offset. Furthermore, the reaction time (RT) and accuracy of the visual focus task were measured.

% In each experimental session, one out of ten combinations entailing three levels of visual stimulus duration, three types of visual layouts, and an additional control group without the visual attraction was employed. Offsets were implemented on virtual movements, ranging from 0 to 30 degrees in increments of 5 degrees, alongside randomly assigned plus-minus directions on the x and y axes, which were uniformly distributed among participants. This led to a total of 13 target pose configurations. As such, each participant evaluated $10 visual attraction settings \times 13 target pose setting = 130 tasks$.


% \emph{Primary task:} perform given movements while observing virtual movements and decide if the virtual movement is the same as the physical movement.

% \emph{Secondary task:} observe the visual attractions which randomly appeared in the field of view and report the appearing position as soon as possible.

% \emph{Independent variables:} visual attraction duration and visual attraction layout

% \emph{Control variables:} offset magnitude, offset direction, target movements

% \emph{Metrics:} noticeability of offsets, visual attraction response accuracy and visual attraction response time


\subsection{Participants \& Apparatus}
The participants (N = 16) were recruited through an online questionnaire from a local university. 
The participants (7 females, 9 males) had an average age of 21.25 years ($SD = 1.71$). 
All were trichromats and right-handed. 
Prior to the experiment, participants self-evaluated on their familiarity with VR, reporting an average score of $3.75\ (SD=0.75) $on a 7-point Likert scale (1 - not at all familiar, 4 - neutral, 7 - very familiar).

% \subsection{Apparatus}
We implemented the experimental application in VR with a HTC Vive pro headset in Unity 2019, powered by an Intel Core i7 CPU and an NVIDIA GeForce RTX 3080 GPU. 
Throughout the experimental sessions, participants were seated and equipped with three Vive Trackers affixed to their left shoulder, elbow, and waist using nylon straps.
Based on data given by the tracker, we reconstructed the left arm movement on a virtual humanoid avatar from the Microsoft RocketBox avatar library~\cite{gonzalez2020rocketbox} with the user’s viewpoint coinciding with the avatar’s (as shown in \autoref{figure:formalapparatus}).
All gaze data was recorded with the HTC Viveo pro built-in gaze tracker.
All statistical analyses were conducted with SPSS 26.0.

% Results: Layout: F(2,147)=33.87, p<0.001. Duration: F(2,147)=43.40, p<0.001.

\subsection{Procedure}


To avoid bias from the participants knowing that we were intentionally introducing redirection, we introduced the purpose of the study as an evaluation of a motion capture and reconstruction technique and clarified the real purpose to participants after the study.
Participants were first provided with a walk-through of the platform.
Then, participants were provided with a warm-up session to ensure that they were familiar with the primary and secondary tasks.
After that, each participant completed 10 sessions ($3~durations~\times~3~layouts~+~1~baseline$) of experiments.
They took 2-minute breaks after every two sessions to reduce fatigue.
We recorded the participant's behavioral data, including the position and orientation of hand, elbow, shoulder, gaze, and pupil dilation, at a rate of 60 Hz.
The study lasted around 40 minutes and each participant was compensated with 15 US dollars.
%For the primary task, participants were asked to start from the initial pose and reach the target pose, bypassing two checkpoint poses.
%After retracing the trajectory, they were asked to report if they recognized the virtual arm movement as the same as their physical movement.
%During the primary task, visual attractions were displayed at a random time interval with designed intensity and position.
%Participants were asked to report the visual attraction's position with the touchpad on the controller as soon as they noticed it.


% Participants were first instructed to put on the VR headset and Vive trackers. Following a warm-up session, participants completed ten experimental sessions in a predetermined sequence, with a short break after every five sessions. During the experiment, participants executed left-arm movements while using the Vive controller with right hand to report their perception of offset and the position of visual attractions.

\begin{figure}[t]
    \centering
    \includegraphics[width=0.9\columnwidth]{figures/formativeStudy/noticeability.png}
    \caption{Noticeability results of the formative study in every condition.
    The error bars represent the standard errors.}
\end{figure}

\subsection{Results}
\label{section:study1results}

We first conducted Shapiro-Wilk tests on the noticeability results which showed that all 10 conditions followed a normal distribution, requiring no correction.
We then conducted Repeated-Measures ANOVA with Bonferroni-corrected post hoc T-tests on the results.
The average response time to visual stimuli was 327 ms (SD = 168 ms), indicating that participants were actively engaged in both tasks.

\begin{figure*}
    \centering
    \includegraphics[width=0.9\linewidth]{figures/formativeStudy/psychometric3.png}
    \caption{Psychometric functions of the noticeability in each condition.}
    \label{fig:psychometric}
\end{figure*}

\textit{With visual stimuli, participants noticed the redirection significantly less than without visual stimuli.}
We conducted a one-factor ANOVA between the baseline and the averaged nine other conditions.
Our statistical analysis showed that participants detected the redirection significantly $(F_{(1,15)}=5.90,~p=0.03)$ less frequently when they were exposed to the visual stimuli $(M=0.43,~SD=0.13)$ compared to none visual stimuli $(M=0.51,~SD=0.08)$.
These results confirm that the noticeability of redirection was reduced when visual stimuli were presented and further validate the design of our study.


We then evaluated whether participants' physical movements were effectively redirected under both noticed and unnoticed conditions. 
We divided all trials into two categories based on the participants' response to the redirection (\textit{noticed} or \textit{unnoticed}).
We then analyzed the lengths of participants' virtual and physical hand trajectories within these two categories.
The \textbf{physical trajectory length} refers to the ratio of the participant's physical hand movement trajectory length to the distance between the starting and ending pose.
Similarly, the \textbf{virtual trajectory length} refers to the participants' virtual hand movement trajectory length to the distance between the starting and ending pose.
In the unnoticed condition, participants' physical movement trajectories were significantly shorter than their virtual ones: \textit{Physical} $(AVG = 1.16,~SD = 0.04)$ and \textit{Virtual} $(AVG = 1.24,~SD = 0.05,~t(15) = 3.64,~p < 0.05)$.
Similarly, in the noticed condition, participants' physical movement trajectories were still significantly shorter than their virtual ones: \textit{Physical} $(AVG = 1.17,~SD = 0.04)$ and \textit{Virtual} $(AVG = 1.33,~SD = 0.04,~t(15) = 4.88,~p < 0.01)$.
These results suggest that participants' physical movements were successfully redirected,regardless of whether they noticed the redirection.

% \delete{
% \textbf{Less visual attention directed towards the avatar leads to lower noticeability of the applied offset.}
% }
% \delete{
% \textbf{Visual attention shifts away from the avatar lead to lower noticeability of the applied offset.}
% We conducted a one-factor ANOVA between the baseline and the averaged nine other conditions.
% Our statistical analysis showed that participants detected the offset significantly $(F_{(1,15)}=5.90,~p=0.03)$ less frequently when they were exposed to the visual stimuli $(M=0.43,~SD=0.13)$ compared to none visual stimuli $(M=0.51,~SD=0.08)$.
% This indicates that our visual stimuli successfully occupied participants' visual attention and made them less sensitive to the motion offset.
% We then conducted a two-factor ANOVA to explore the impact of the stimuli's duration and position on the noticeability.
% Results showed that both the stimuli's duration $(F_{(2,30)}=24.02,~p<0.001)$ and position $(F_{(2,30)}=27.83,~p<0.001)$ had a significant influence on the noticeability and there was not an interaction effect between these two factors $(F_{(4,60)}=1.70,~p=0.20)$.
% Therefore, we conducted post hoc tests on duration and layout separately.
% Post hoc results showed that the noticeability with \textit{short} duration $(M=0.35,~SD=0.10)$ was significantly lower than \textit{medium} $(M=0.46,~SD=0.07),~(t(15)=-3.34,~p<0.001)$, and \textit{long} $(M=0.50,~SD=0.07),~(t(15)=-4.39,~p<0.001)$ duration stimuli.
% The noticeability with \textit{sparse} layout $(M=0.37,~SD=0.09)$ was significantly lower than \textit{medium} $(M=0.43,~SD=0.07),~(t(15)=-3.07,~p<0.01)$, and \textit{dense} $(M=0.49,~SD=0.09),~(t(15)=-4.01,~p<0.001)$ duration stimuli.
% This indicates that both the visual stimuli' duration and position could affect the noticeability of offset significantly.
% We further plotted the psychometric functions of each condition to demonstrate how visual stimuli impact the noticeability behavior of users, which can be helpful for designers to decide the appropriate offsets to apply in different scenarios.
% To be noted, the noticeability of motion offsets under visual stimuli with the sparse layout and short duration did not achieve 100\%, due to participants' visual attention being strongly directed towards the visual stimuli.
% }

\begin{figure*}
    \centering
    \includegraphics[width=0.9\linewidth]{figures/formativeStudy/correlation_v3.png}
    \caption{The regression results between gaze distance, saccade frequency, fixation frequency, IPA, and noticeability are presented. 
    The correlation coefficients are indicated in the top right corner.}
    \label{figure:correlation}
\end{figure*}

We further analyzed the gaze behaviors (gaze location, saccades, fixation, and pupil activity) based on the recorded gaze data. 
Specifically, we computed the average gaze distance relative to the virtual hand, saccade and fixation frequencies, and Index of Pupil Activity (IPA) in each condition. 
We then calculated the correlation coefficients between these gaze behaviors and the noticeability results.
As shown in~\autoref{figure:correlation}, the results revealed significant correlations: gaze distance ($r = -0.26, p = 0.001$), saccade ($r = -0.43, p < 0.001$), fixation ($r = -0.27, p < 0.001$), and IPA ($r = -0.26, p = 0.001$). 
These results suggest that all the examined gaze behaviors exhibit a negative relationship with noticeability, with gaze saccades showing a stronger effect compared to the others. 
This may be because rapid saccadic movements often indicate high cognitive load or attentional shifts, making them more directly and negatively associated with the noticing of redirection. 
In contrast, fixation frequency does not inherently reflect cognitive load, although fixation duration might serve as a useful indicator.
Regarding gaze distance to the virtual hand, participants likely shifted their gaze between the virtual body and the stimuli, making it less consistently related to noticeability. 
For IPA, its design as a long-term estimator of cognitive load may render it less sensitive to subtle or transient changes in cognitive load caused by visual stimuli.
Overall, while each of these gaze behaviors responds to visual stimuli in distinct ways, they all show promise as predictors of noticeability.


% \delete{
% Since recent papers showed that the gaze saccade could affect the noticeability of motion offsets~\cite{zenner2023detectability}, we further analyzed other eye gaze behavior to investigate if participants more frequently switched their visual focus or had a higher cognitive load in the condition with stronger visual stimuli.
% To this end, we calculated participants' eye saccades and fixations (as implemented in \cite{pymovements}) from the gaze position data and Index of Pupillary Activity (IPA)~\cite{duchowski2018index} as the estimated cognitive load.
% By calculating the correlation coefficients between the average of these metrics and noticeability, we found that IPA $(r=-0.26)$, saccade $(r=-0.43)$ and fixation frequency $(r=-0.27)$ were correlated with the noticeability.
% This relationship suggests that participants were less likely to notice the offset when they had a high cognitive load or they had more frequent shifts in their visual focus.
% These results motivated us to further explore the relationship between users' gaze behavioral data and noticeability.
% Since there is no ground truth for visual attention, we aim to build an extendable model which takes users' gaze behavioral patterns as input and outputs the noticeability of applied offsets.
% }
% \delete{
% Our aim was to build an extendable model to predict the noticeability of the applied offsets without knowing the specific visual stimulus that leads to the identified gaze behavioral patterns.
% }

