\section{Discussion}
% Short summary of the paper.
In this paper, we investigated the effects of visual stimuli on to what extent users notice inconsistencies in their physical movements versus avatar movements. 
We further contribute a regression model that computes the noticeability of redirection under various visual stimuli, based on users' gaze behavioral data.
With the model, we constructed two applications in realistic scenarios with different types of visual stimuli to demonstrate the potential advantages and extensions of our method.
In the following, we discuss possible extensions to our model, as well as limitations and future work.

\subsection{Redirection and visual stimuli}
While prior work~\cite{li2022modeling, feick2023investigating, feick2021visuo} explored how the properties (such as magnitude, direction, location) of redirection influenced its noticeabiltiy, we investigated the noticeability under visual stimuli in this paper.
However, we acknowledge that the redirection properties and the visual stimuli may affect the noticeability in different manners. 
The redirection properties could determine the upper and lower bounds of redirection noticeability, while the visual stimuli can only reduce the noticeability in a limited range.
For subtle redirection that are barely noticeable even when the user is focused on their body movements, adjusting visual stimuli does not significantly alter the noticeability.
Similarly, users will likely notice salient redirection even with a glance, unless the redirection is completely out of their field of view.
Therefore, in this paper, we fixed the redirection magnitude to be 20 degrees (as a control variable), for which the resulting noticeability ranged from approximately 20\% to 80\%.
This relatively large range enables us to quantify the impacts of visual stimuli extensively.
However, we believe that exploring the interaction effect of redirection properties and visual stimuli and combining their influence on noticeability could be important and interesting future work.

\subsection{Diverse visual stimuli}
In this paper, we used several abstract visual stimuli and changed their intensity in our user studies.
We acknowledge that beyond these static visual stimuli tested in this paper, there exist various complicated visual stimuli in realistic use cases.
For instance, a moving object or a wiggling notification may also affect users' gaze behavior and therefore influence the noticeability of redirection.
We consider visual stimuli appearing and staying at a static location to be a standard design paradigm in presenting notifications (e.g., highlighting app icons when new messages are received) on desktop~\cite{muller2023notification}, VR~\cite{rzayev2019notification}, and AR interfaces~\cite{lee2023effects}.
Though we validated our model on new type of abstract visual stimuli and in realistic scenarios,
we acknowledge that verifying the generalizability of our regression model on motion-based or other more complicated visual stimuli is an important future work.
We expect that our research methodology and the presented gaze behavioral patterns can also apply to the investigation of other visual stimuli.

Besides, the visual stimuli investigated in this paper primarily served as external cues for object selection or observation, rather than being directly related to users' body movements. 
In scenarios such as motion training and learning, users may observe their body movements through a mirror or from a third-person perspective, making redirected motion part of the visual stimuli. 
This raises an open question of how to decouple redirection from visual stimuli to investigate their specific influence on the noticeability of redirection. 
We acknowledge this as an important direction for future research.

Furthermore, in more realistic usage scenarios, the stimuli could be in different formats, including instant notifications, environmental events, or even the user's implicit observation of the virtual scene.
We acknowledge that in such cases, 
% besides the gaze patterns captured by the proposed method, 
different behavioral patterns or even physiological signals, such as EEG signals and heart rate variations, can also be indicative of the noticeability of redirection.
We expect that our research method can be adopted to explore further the behavioral patterns that reflect the noticeability in a more realistic setting.

% In a more practical usage scenario, it could be instant notifications, environmental events, mind wandering of oneself.
% Different attractions may elicit different behavioral patterns (?), discuss based on the evaluation results as we preliminarily test different types of attractions.


\subsection{User awareness and adaptation}
% To what extent should the user be aware of the manipulation.
% How to involve the user into the decision making process.
% Users might be able to adapt to the offset gradually, the system should be able to keep adapting the strategy accordingly.

In our user studies, we hide the true purpose from the participants by disguising it as an accuracy evaluation for a motion tracking system.
The consideration was to mitigate the potential bias of users being aware of the existence of redirection, which might nudge them to be more attuned to or hyper-aware of the redirection.
In addition, our study lasted at most 40 minutes with multiple breaks, which allowed users to regain their perception of their physical movements and prevent fatigue.
% This helped to mitigate the influence of user awareness and adaptation to the offset.
However, if a long-term redirection is applied in real-life applications, users might become desensitized to the redirection gradually.
Users may adapt to the redirection after noticing them multiple times and assume that the redirection exists consistently, which could reduce the noticeability of the redirection.
We argue that the regression model should take into account the user's awareness and their ability to adapt their interaction behavior to continue computing the noticeability accurately.


%\subsection{Attention estimation based on offset noticeability}
%We established a prediction model that takes as input the users' gaze behavioral data and predict the noticeability of an offset applied on the user's body movement.
%The rationale behind the model is that the user's gaze behavior reflect their attention to the body movements, which eventually impacts the probability that they notice the offset.
%In other words, gaze data serves as the bridging connecting the user's visual attention to the offset noticeability.
%Therefore, theoretically, it is also possible to leverage the noticeability results to infer the amount of visual attention of the user.
%This opens up new opportunities to estimate visual attention based on users' responses, which does not require tracking the user's gaze with extra sensors.
%For instance, when applying a visual attraction, our model can predict the noticeability results of users paying different levels of visual attention to it.
%If users consistently report detecting the offset during interactions, it could imply that the visual attraction has failed to attract the user's visual attention.
%This opens up new opportunities to estimate visual attention based on users' responses, which might help designers while designing interactive systems.


% Estimate the amount of visual attention the user pays to a certain task based on the behavioral patterns of their gaze.


% \subsection{VR vs. AR}
% Are we exclusive to AR? See-through might be applicable.



