\section{Conclusion}

In this paper, we investigated and modeled the effects of visual stimuli on the noticeability of redirection using users' gaze behaviors in VR.
We first conducted a confirmation study to verify if users' noticeability of redirection was affected by visual stimuli and whether their gaze behaviors were correlated with the noticeability results.
After confirming that visual stimuli could influence the noticeability of redirection, we conducted a data collection study with refined visual stimuli.
With the data, we built up a regression model and selected effective features to compute the noticeability of redirection based on gaze behavior data, achieving an accuracy of 0.011 MSE.
We then evaluated our model on unseen visual stimuli with 24 new users and results suggested that our prediction model could generalize to new visual stimuli.
We then implemented an adaptive redirection technique based on our model and conducted a proof-of-concept study comparing it to static redirection technique.
Results suggested that participants felt less physical demanding while kept a high sense of body ownership using the adaptive redirection technique based on our model.
We believe that our model could support more effective and immersive redirection interactions in VR.
