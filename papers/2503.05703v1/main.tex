
\documentclass{article} % For LaTeX2e
\usepackage{iclr2025_conference,times}
\usepackage{natbib}
\usepackage[utf8]{inputenc}

% Optional math commands from https://github.com/goodfeli/dlbook_notation.
%%%%% NEW MATH DEFINITIONS %%%%%

\usepackage{amsmath,amsfonts,bm}
\usepackage{derivative}
% Mark sections of captions for referring to divisions of figures
\newcommand{\figleft}{{\em (Left)}}
\newcommand{\figcenter}{{\em (Center)}}
\newcommand{\figright}{{\em (Right)}}
\newcommand{\figtop}{{\em (Top)}}
\newcommand{\figbottom}{{\em (Bottom)}}
\newcommand{\captiona}{{\em (a)}}
\newcommand{\captionb}{{\em (b)}}
\newcommand{\captionc}{{\em (c)}}
\newcommand{\captiond}{{\em (d)}}

% Highlight a newly defined term
\newcommand{\newterm}[1]{{\bf #1}}

% Derivative d 
\newcommand{\deriv}{{\mathrm{d}}}

% Figure reference, lower-case.
\def\figref#1{figure~\ref{#1}}
% Figure reference, capital. For start of sentence
\def\Figref#1{Figure~\ref{#1}}
\def\twofigref#1#2{figures \ref{#1} and \ref{#2}}
\def\quadfigref#1#2#3#4{figures \ref{#1}, \ref{#2}, \ref{#3} and \ref{#4}}
% Section reference, lower-case.
\def\secref#1{section~\ref{#1}}
% Section reference, capital.
\def\Secref#1{Section~\ref{#1}}
% Reference to two sections.
\def\twosecrefs#1#2{sections \ref{#1} and \ref{#2}}
% Reference to three sections.
\def\secrefs#1#2#3{sections \ref{#1}, \ref{#2} and \ref{#3}}
% Reference to an equation, lower-case.
\def\eqref#1{equation~\ref{#1}}
% Reference to an equation, upper case
\def\Eqref#1{Equation~\ref{#1}}
% A raw reference to an equation---avoid using if possible
\def\plaineqref#1{\ref{#1}}
% Reference to a chapter, lower-case.
\def\chapref#1{chapter~\ref{#1}}
% Reference to an equation, upper case.
\def\Chapref#1{Chapter~\ref{#1}}
% Reference to a range of chapters
\def\rangechapref#1#2{chapters\ref{#1}--\ref{#2}}
% Reference to an algorithm, lower-case.
\def\algref#1{algorithm~\ref{#1}}
% Reference to an algorithm, upper case.
\def\Algref#1{Algorithm~\ref{#1}}
\def\twoalgref#1#2{algorithms \ref{#1} and \ref{#2}}
\def\Twoalgref#1#2{Algorithms \ref{#1} and \ref{#2}}
% Reference to a part, lower case
\def\partref#1{part~\ref{#1}}
% Reference to a part, upper case
\def\Partref#1{Part~\ref{#1}}
\def\twopartref#1#2{parts \ref{#1} and \ref{#2}}

\def\ceil#1{\lceil #1 \rceil}
\def\floor#1{\lfloor #1 \rfloor}
\def\1{\bm{1}}
\newcommand{\train}{\mathcal{D}}
\newcommand{\valid}{\mathcal{D_{\mathrm{valid}}}}
\newcommand{\test}{\mathcal{D_{\mathrm{test}}}}

\def\eps{{\epsilon}}


% Random variables
\def\reta{{\textnormal{$\eta$}}}
\def\ra{{\textnormal{a}}}
\def\rb{{\textnormal{b}}}
\def\rc{{\textnormal{c}}}
\def\rd{{\textnormal{d}}}
\def\re{{\textnormal{e}}}
\def\rf{{\textnormal{f}}}
\def\rg{{\textnormal{g}}}
\def\rh{{\textnormal{h}}}
\def\ri{{\textnormal{i}}}
\def\rj{{\textnormal{j}}}
\def\rk{{\textnormal{k}}}
\def\rl{{\textnormal{l}}}
% rm is already a command, just don't name any random variables m
\def\rn{{\textnormal{n}}}
\def\ro{{\textnormal{o}}}
\def\rp{{\textnormal{p}}}
\def\rq{{\textnormal{q}}}
\def\rr{{\textnormal{r}}}
\def\rs{{\textnormal{s}}}
\def\rt{{\textnormal{t}}}
\def\ru{{\textnormal{u}}}
\def\rv{{\textnormal{v}}}
\def\rw{{\textnormal{w}}}
\def\rx{{\textnormal{x}}}
\def\ry{{\textnormal{y}}}
\def\rz{{\textnormal{z}}}

% Random vectors
\def\rvepsilon{{\mathbf{\epsilon}}}
\def\rvphi{{\mathbf{\phi}}}
\def\rvtheta{{\mathbf{\theta}}}
\def\rva{{\mathbf{a}}}
\def\rvb{{\mathbf{b}}}
\def\rvc{{\mathbf{c}}}
\def\rvd{{\mathbf{d}}}
\def\rve{{\mathbf{e}}}
\def\rvf{{\mathbf{f}}}
\def\rvg{{\mathbf{g}}}
\def\rvh{{\mathbf{h}}}
\def\rvu{{\mathbf{i}}}
\def\rvj{{\mathbf{j}}}
\def\rvk{{\mathbf{k}}}
\def\rvl{{\mathbf{l}}}
\def\rvm{{\mathbf{m}}}
\def\rvn{{\mathbf{n}}}
\def\rvo{{\mathbf{o}}}
\def\rvp{{\mathbf{p}}}
\def\rvq{{\mathbf{q}}}
\def\rvr{{\mathbf{r}}}
\def\rvs{{\mathbf{s}}}
\def\rvt{{\mathbf{t}}}
\def\rvu{{\mathbf{u}}}
\def\rvv{{\mathbf{v}}}
\def\rvw{{\mathbf{w}}}
\def\rvx{{\mathbf{x}}}
\def\rvy{{\mathbf{y}}}
\def\rvz{{\mathbf{z}}}

% Elements of random vectors
\def\erva{{\textnormal{a}}}
\def\ervb{{\textnormal{b}}}
\def\ervc{{\textnormal{c}}}
\def\ervd{{\textnormal{d}}}
\def\erve{{\textnormal{e}}}
\def\ervf{{\textnormal{f}}}
\def\ervg{{\textnormal{g}}}
\def\ervh{{\textnormal{h}}}
\def\ervi{{\textnormal{i}}}
\def\ervj{{\textnormal{j}}}
\def\ervk{{\textnormal{k}}}
\def\ervl{{\textnormal{l}}}
\def\ervm{{\textnormal{m}}}
\def\ervn{{\textnormal{n}}}
\def\ervo{{\textnormal{o}}}
\def\ervp{{\textnormal{p}}}
\def\ervq{{\textnormal{q}}}
\def\ervr{{\textnormal{r}}}
\def\ervs{{\textnormal{s}}}
\def\ervt{{\textnormal{t}}}
\def\ervu{{\textnormal{u}}}
\def\ervv{{\textnormal{v}}}
\def\ervw{{\textnormal{w}}}
\def\ervx{{\textnormal{x}}}
\def\ervy{{\textnormal{y}}}
\def\ervz{{\textnormal{z}}}

% Random matrices
\def\rmA{{\mathbf{A}}}
\def\rmB{{\mathbf{B}}}
\def\rmC{{\mathbf{C}}}
\def\rmD{{\mathbf{D}}}
\def\rmE{{\mathbf{E}}}
\def\rmF{{\mathbf{F}}}
\def\rmG{{\mathbf{G}}}
\def\rmH{{\mathbf{H}}}
\def\rmI{{\mathbf{I}}}
\def\rmJ{{\mathbf{J}}}
\def\rmK{{\mathbf{K}}}
\def\rmL{{\mathbf{L}}}
\def\rmM{{\mathbf{M}}}
\def\rmN{{\mathbf{N}}}
\def\rmO{{\mathbf{O}}}
\def\rmP{{\mathbf{P}}}
\def\rmQ{{\mathbf{Q}}}
\def\rmR{{\mathbf{R}}}
\def\rmS{{\mathbf{S}}}
\def\rmT{{\mathbf{T}}}
\def\rmU{{\mathbf{U}}}
\def\rmV{{\mathbf{V}}}
\def\rmW{{\mathbf{W}}}
\def\rmX{{\mathbf{X}}}
\def\rmY{{\mathbf{Y}}}
\def\rmZ{{\mathbf{Z}}}

% Elements of random matrices
\def\ermA{{\textnormal{A}}}
\def\ermB{{\textnormal{B}}}
\def\ermC{{\textnormal{C}}}
\def\ermD{{\textnormal{D}}}
\def\ermE{{\textnormal{E}}}
\def\ermF{{\textnormal{F}}}
\def\ermG{{\textnormal{G}}}
\def\ermH{{\textnormal{H}}}
\def\ermI{{\textnormal{I}}}
\def\ermJ{{\textnormal{J}}}
\def\ermK{{\textnormal{K}}}
\def\ermL{{\textnormal{L}}}
\def\ermM{{\textnormal{M}}}
\def\ermN{{\textnormal{N}}}
\def\ermO{{\textnormal{O}}}
\def\ermP{{\textnormal{P}}}
\def\ermQ{{\textnormal{Q}}}
\def\ermR{{\textnormal{R}}}
\def\ermS{{\textnormal{S}}}
\def\ermT{{\textnormal{T}}}
\def\ermU{{\textnormal{U}}}
\def\ermV{{\textnormal{V}}}
\def\ermW{{\textnormal{W}}}
\def\ermX{{\textnormal{X}}}
\def\ermY{{\textnormal{Y}}}
\def\ermZ{{\textnormal{Z}}}

% Vectors
\def\vzero{{\bm{0}}}
\def\vone{{\bm{1}}}
\def\vmu{{\bm{\mu}}}
\def\vtheta{{\bm{\theta}}}
\def\vphi{{\bm{\phi}}}
\def\va{{\bm{a}}}
\def\vb{{\bm{b}}}
\def\vc{{\bm{c}}}
\def\vd{{\bm{d}}}
\def\ve{{\bm{e}}}
\def\vf{{\bm{f}}}
\def\vg{{\bm{g}}}
\def\vh{{\bm{h}}}
\def\vi{{\bm{i}}}
\def\vj{{\bm{j}}}
\def\vk{{\bm{k}}}
\def\vl{{\bm{l}}}
\def\vm{{\bm{m}}}
\def\vn{{\bm{n}}}
\def\vo{{\bm{o}}}
\def\vp{{\bm{p}}}
\def\vq{{\bm{q}}}
\def\vr{{\bm{r}}}
\def\vs{{\bm{s}}}
\def\vt{{\bm{t}}}
\def\vu{{\bm{u}}}
\def\vv{{\bm{v}}}
\def\vw{{\bm{w}}}
\def\vx{{\bm{x}}}
\def\vy{{\bm{y}}}
\def\vz{{\bm{z}}}

% Elements of vectors
\def\evalpha{{\alpha}}
\def\evbeta{{\beta}}
\def\evepsilon{{\epsilon}}
\def\evlambda{{\lambda}}
\def\evomega{{\omega}}
\def\evmu{{\mu}}
\def\evpsi{{\psi}}
\def\evsigma{{\sigma}}
\def\evtheta{{\theta}}
\def\eva{{a}}
\def\evb{{b}}
\def\evc{{c}}
\def\evd{{d}}
\def\eve{{e}}
\def\evf{{f}}
\def\evg{{g}}
\def\evh{{h}}
\def\evi{{i}}
\def\evj{{j}}
\def\evk{{k}}
\def\evl{{l}}
\def\evm{{m}}
\def\evn{{n}}
\def\evo{{o}}
\def\evp{{p}}
\def\evq{{q}}
\def\evr{{r}}
\def\evs{{s}}
\def\evt{{t}}
\def\evu{{u}}
\def\evv{{v}}
\def\evw{{w}}
\def\evx{{x}}
\def\evy{{y}}
\def\evz{{z}}

% Matrix
\def\mA{{\bm{A}}}
\def\mB{{\bm{B}}}
\def\mC{{\bm{C}}}
\def\mD{{\bm{D}}}
\def\mE{{\bm{E}}}
\def\mF{{\bm{F}}}
\def\mG{{\bm{G}}}
\def\mH{{\bm{H}}}
\def\mI{{\bm{I}}}
\def\mJ{{\bm{J}}}
\def\mK{{\bm{K}}}
\def\mL{{\bm{L}}}
\def\mM{{\bm{M}}}
\def\mN{{\bm{N}}}
\def\mO{{\bm{O}}}
\def\mP{{\bm{P}}}
\def\mQ{{\bm{Q}}}
\def\mR{{\bm{R}}}
\def\mS{{\bm{S}}}
\def\mT{{\bm{T}}}
\def\mU{{\bm{U}}}
\def\mV{{\bm{V}}}
\def\mW{{\bm{W}}}
\def\mX{{\bm{X}}}
\def\mY{{\bm{Y}}}
\def\mZ{{\bm{Z}}}
\def\mBeta{{\bm{\beta}}}
\def\mPhi{{\bm{\Phi}}}
\def\mLambda{{\bm{\Lambda}}}
\def\mSigma{{\bm{\Sigma}}}

% Tensor
\DeclareMathAlphabet{\mathsfit}{\encodingdefault}{\sfdefault}{m}{sl}
\SetMathAlphabet{\mathsfit}{bold}{\encodingdefault}{\sfdefault}{bx}{n}
\newcommand{\tens}[1]{\bm{\mathsfit{#1}}}
\def\tA{{\tens{A}}}
\def\tB{{\tens{B}}}
\def\tC{{\tens{C}}}
\def\tD{{\tens{D}}}
\def\tE{{\tens{E}}}
\def\tF{{\tens{F}}}
\def\tG{{\tens{G}}}
\def\tH{{\tens{H}}}
\def\tI{{\tens{I}}}
\def\tJ{{\tens{J}}}
\def\tK{{\tens{K}}}
\def\tL{{\tens{L}}}
\def\tM{{\tens{M}}}
\def\tN{{\tens{N}}}
\def\tO{{\tens{O}}}
\def\tP{{\tens{P}}}
\def\tQ{{\tens{Q}}}
\def\tR{{\tens{R}}}
\def\tS{{\tens{S}}}
\def\tT{{\tens{T}}}
\def\tU{{\tens{U}}}
\def\tV{{\tens{V}}}
\def\tW{{\tens{W}}}
\def\tX{{\tens{X}}}
\def\tY{{\tens{Y}}}
\def\tZ{{\tens{Z}}}


% Graph
\def\gA{{\mathcal{A}}}
\def\gB{{\mathcal{B}}}
\def\gC{{\mathcal{C}}}
\def\gD{{\mathcal{D}}}
\def\gE{{\mathcal{E}}}
\def\gF{{\mathcal{F}}}
\def\gG{{\mathcal{G}}}
\def\gH{{\mathcal{H}}}
\def\gI{{\mathcal{I}}}
\def\gJ{{\mathcal{J}}}
\def\gK{{\mathcal{K}}}
\def\gL{{\mathcal{L}}}
\def\gM{{\mathcal{M}}}
\def\gN{{\mathcal{N}}}
\def\gO{{\mathcal{O}}}
\def\gP{{\mathcal{P}}}
\def\gQ{{\mathcal{Q}}}
\def\gR{{\mathcal{R}}}
\def\gS{{\mathcal{S}}}
\def\gT{{\mathcal{T}}}
\def\gU{{\mathcal{U}}}
\def\gV{{\mathcal{V}}}
\def\gW{{\mathcal{W}}}
\def\gX{{\mathcal{X}}}
\def\gY{{\mathcal{Y}}}
\def\gZ{{\mathcal{Z}}}

% Sets
\def\sA{{\mathbb{A}}}
\def\sB{{\mathbb{B}}}
\def\sC{{\mathbb{C}}}
\def\sD{{\mathbb{D}}}
% Don't use a set called E, because this would be the same as our symbol
% for expectation.
\def\sF{{\mathbb{F}}}
\def\sG{{\mathbb{G}}}
\def\sH{{\mathbb{H}}}
\def\sI{{\mathbb{I}}}
\def\sJ{{\mathbb{J}}}
\def\sK{{\mathbb{K}}}
\def\sL{{\mathbb{L}}}
\def\sM{{\mathbb{M}}}
\def\sN{{\mathbb{N}}}
\def\sO{{\mathbb{O}}}
\def\sP{{\mathbb{P}}}
\def\sQ{{\mathbb{Q}}}
\def\sR{{\mathbb{R}}}
\def\sS{{\mathbb{S}}}
\def\sT{{\mathbb{T}}}
\def\sU{{\mathbb{U}}}
\def\sV{{\mathbb{V}}}
\def\sW{{\mathbb{W}}}
\def\sX{{\mathbb{X}}}
\def\sY{{\mathbb{Y}}}
\def\sZ{{\mathbb{Z}}}

% Entries of a matrix
\def\emLambda{{\Lambda}}
\def\emA{{A}}
\def\emB{{B}}
\def\emC{{C}}
\def\emD{{D}}
\def\emE{{E}}
\def\emF{{F}}
\def\emG{{G}}
\def\emH{{H}}
\def\emI{{I}}
\def\emJ{{J}}
\def\emK{{K}}
\def\emL{{L}}
\def\emM{{M}}
\def\emN{{N}}
\def\emO{{O}}
\def\emP{{P}}
\def\emQ{{Q}}
\def\emR{{R}}
\def\emS{{S}}
\def\emT{{T}}
\def\emU{{U}}
\def\emV{{V}}
\def\emW{{W}}
\def\emX{{X}}
\def\emY{{Y}}
\def\emZ{{Z}}
\def\emSigma{{\Sigma}}

% entries of a tensor
% Same font as tensor, without \bm wrapper
\newcommand{\etens}[1]{\mathsfit{#1}}
\def\etLambda{{\etens{\Lambda}}}
\def\etA{{\etens{A}}}
\def\etB{{\etens{B}}}
\def\etC{{\etens{C}}}
\def\etD{{\etens{D}}}
\def\etE{{\etens{E}}}
\def\etF{{\etens{F}}}
\def\etG{{\etens{G}}}
\def\etH{{\etens{H}}}
\def\etI{{\etens{I}}}
\def\etJ{{\etens{J}}}
\def\etK{{\etens{K}}}
\def\etL{{\etens{L}}}
\def\etM{{\etens{M}}}
\def\etN{{\etens{N}}}
\def\etO{{\etens{O}}}
\def\etP{{\etens{P}}}
\def\etQ{{\etens{Q}}}
\def\etR{{\etens{R}}}
\def\etS{{\etens{S}}}
\def\etT{{\etens{T}}}
\def\etU{{\etens{U}}}
\def\etV{{\etens{V}}}
\def\etW{{\etens{W}}}
\def\etX{{\etens{X}}}
\def\etY{{\etens{Y}}}
\def\etZ{{\etens{Z}}}

% The true underlying data generating distribution
\newcommand{\pdata}{p_{\rm{data}}}
\newcommand{\ptarget}{p_{\rm{target}}}
\newcommand{\pprior}{p_{\rm{prior}}}
\newcommand{\pbase}{p_{\rm{base}}}
\newcommand{\pref}{p_{\rm{ref}}}

% The empirical distribution defined by the training set
\newcommand{\ptrain}{\hat{p}_{\rm{data}}}
\newcommand{\Ptrain}{\hat{P}_{\rm{data}}}
% The model distribution
\newcommand{\pmodel}{p_{\rm{model}}}
\newcommand{\Pmodel}{P_{\rm{model}}}
\newcommand{\ptildemodel}{\tilde{p}_{\rm{model}}}
% Stochastic autoencoder distributions
\newcommand{\pencode}{p_{\rm{encoder}}}
\newcommand{\pdecode}{p_{\rm{decoder}}}
\newcommand{\precons}{p_{\rm{reconstruct}}}

\newcommand{\laplace}{\mathrm{Laplace}} % Laplace distribution

\newcommand{\E}{\mathbb{E}}
\newcommand{\Ls}{\mathcal{L}}
\newcommand{\R}{\mathbb{R}}
\newcommand{\emp}{\tilde{p}}
\newcommand{\lr}{\alpha}
\newcommand{\reg}{\lambda}
\newcommand{\rect}{\mathrm{rectifier}}
\newcommand{\softmax}{\mathrm{softmax}}
\newcommand{\sigmoid}{\sigma}
\newcommand{\softplus}{\zeta}
\newcommand{\KL}{D_{\mathrm{KL}}}
\newcommand{\Var}{\mathrm{Var}}
\newcommand{\standarderror}{\mathrm{SE}}
\newcommand{\Cov}{\mathrm{Cov}}
% Wolfram Mathworld says $L^2$ is for function spaces and $\ell^2$ is for vectors
% But then they seem to use $L^2$ for vectors throughout the site, and so does
% wikipedia.
\newcommand{\normlzero}{L^0}
\newcommand{\normlone}{L^1}
\newcommand{\normltwo}{L^2}
\newcommand{\normlp}{L^p}
\newcommand{\normmax}{L^\infty}

\newcommand{\parents}{Pa} % See usage in notation.tex. Chosen to match Daphne's book.

\DeclareMathOperator*{\argmax}{arg\,max}
\DeclareMathOperator*{\argmin}{arg\,min}

\DeclareMathOperator{\sign}{sign}
\DeclareMathOperator{\Tr}{Tr}
\let\ab\allowbreak


\newcommand{\theHalgorithm}{\arabic{algorithm}}
\usepackage{hyperref}
\usepackage{url}
\usepackage{wrapfig}
\usepackage{comment}

\usepackage{amssymb}
\usepackage{graphicx}
\usepackage{array}
%\usepackage[table]{xcolor} % The 'table' option loads 'colortbl'
\usepackage{ragged2e}

\usepackage{booktabs}


\usepackage{makecell}

\usepackage{listings}
\usepackage{multirow}

\usepackage[inline]{enumitem}


\title{What I cannot execute, I do not understand: Training and Evaluating LLMs on Program Execution Traces}





\author{Jordi Armengol-Estapé$^1$\thanks{Work done while interning at FAIR, Meta AI. Contact: \texttt{jordi.armengol.estape@ed.ac.uk}}\ \ , Quentin Carbonneaux$^2$, Tianjun Zhang$^2$, \\ \textbf{Aram H. Markosyan$^2$, Volker Seeker$^2$, Chris Cummins$^2$, Melanie Kambadur$^2$,} \\
\textbf{Michael F.P. O'Boyle$^1$, Sida Wang$^2$, Gabriel Synnaeve$^2$, Hugh Leather$^2$} \\
$^1$University of Edinburgh  \quad
$^2$Meta AI
}




\newcommand{\fix}{\marginpar{FIX}}
\newcommand{\new}{\marginpar{NEW}}




\newcommand{\name}{E.T.}

\newcommand{\fullname}{Execution Tuning} 

\iclrfinalcopy 
\begin{document}



\maketitle

\begin{abstract}
Code generation and understanding are critical capabilities for large language models (LLMs). Thus, most LLMs are pretrained and fine-tuned on code data. However, these datasets typically treat code as static strings and rarely exploit the dynamic information about their execution. Building upon previous work on trace modeling,  we study \fullname{} (\name{}), a training procedure in which we explicitly model real-world program execution traces without requiring manual test annotations. 
We train and evaluate models on different execution trace granularities (line and instruction-level) and strategies on the task of output prediction, obtaining ${\sim}80\%$ accuracy on CruxEval and MBPP, and showing the advantages of \textit{dynamic scratchpads} (i.e., self-contained intermediate computations updated by the model rather than accumulated as a history of past computations)  on long executions (up to 14k steps). Finally, we discuss \name{}'s practical applications.
\end{abstract}





\section{Introduction}
Backdoor attacks pose a concealed yet profound security risk to machine learning (ML) models, for which the adversaries can inject a stealth backdoor into the model during training, enabling them to illicitly control the model's output upon encountering predefined inputs. These attacks can even occur without the knowledge of developers or end-users, thereby undermining the trust in ML systems. As ML becomes more deeply embedded in critical sectors like finance, healthcare, and autonomous driving \citep{he2016deep, liu2020computing, tournier2019mrtrix3, adjabi2020past}, the potential damage from backdoor attacks grows, underscoring the emergency for developing robust defense mechanisms against backdoor attacks.

To address the threat of backdoor attacks, researchers have developed a variety of strategies \cite{liu2018fine,wu2021adversarial,wang2019neural,zeng2022adversarial,zhu2023neural,Zhu_2023_ICCV, wei2024shared,wei2024d3}, aimed at purifying backdoors within victim models. These methods are designed to integrate with current deployment workflows seamlessly and have demonstrated significant success in mitigating the effects of backdoor triggers \cite{wubackdoorbench, wu2023defenses, wu2024backdoorbench,dunnett2024countering}.  However, most state-of-the-art (SOTA) backdoor purification methods operate under the assumption that a small clean dataset, often referred to as \textbf{auxiliary dataset}, is available for purification. Such an assumption poses practical challenges, especially in scenarios where data is scarce. To tackle this challenge, efforts have been made to reduce the size of the required auxiliary dataset~\cite{chai2022oneshot,li2023reconstructive, Zhu_2023_ICCV} and even explore dataset-free purification techniques~\cite{zheng2022data,hong2023revisiting,lin2024fusing}. Although these approaches offer some improvements, recent evaluations \cite{dunnett2024countering, wu2024backdoorbench} continue to highlight the importance of sufficient auxiliary data for achieving robust defenses against backdoor attacks.

While significant progress has been made in reducing the size of auxiliary datasets, an equally critical yet underexplored question remains: \emph{how does the nature of the auxiliary dataset affect purification effectiveness?} In  real-world  applications, auxiliary datasets can vary widely, encompassing in-distribution data, synthetic data, or external data from different sources. Understanding how each type of auxiliary dataset influences the purification effectiveness is vital for selecting or constructing the most suitable auxiliary dataset and the corresponding technique. For instance, when multiple datasets are available, understanding how different datasets contribute to purification can guide defenders in selecting or crafting the most appropriate dataset. Conversely, when only limited auxiliary data is accessible, knowing which purification technique works best under those constraints is critical. Therefore, there is an urgent need for a thorough investigation into the impact of auxiliary datasets on purification effectiveness to guide defenders in  enhancing the security of ML systems. 

In this paper, we systematically investigate the critical role of auxiliary datasets in backdoor purification, aiming to bridge the gap between idealized and practical purification scenarios.  Specifically, we first construct a diverse set of auxiliary datasets to emulate real-world conditions, as summarized in Table~\ref{overall}. These datasets include in-distribution data, synthetic data, and external data from other sources. Through an evaluation of SOTA backdoor purification methods across these datasets, we uncover several critical insights: \textbf{1)} In-distribution datasets, particularly those carefully filtered from the original training data of the victim model, effectively preserve the model’s utility for its intended tasks but may fall short in eliminating backdoors. \textbf{2)} Incorporating OOD datasets can help the model forget backdoors but also bring the risk of forgetting critical learned knowledge, significantly degrading its overall performance. Building on these findings, we propose Guided Input Calibration (GIC), a novel technique that enhances backdoor purification by adaptively transforming auxiliary data to better align with the victim model’s learned representations. By leveraging the victim model itself to guide this transformation, GIC optimizes the purification process, striking a balance between preserving model utility and mitigating backdoor threats. Extensive experiments demonstrate that GIC significantly improves the effectiveness of backdoor purification across diverse auxiliary datasets, providing a practical and robust defense solution.

Our main contributions are threefold:
\textbf{1) Impact analysis of auxiliary datasets:} We take the \textbf{first step}  in systematically investigating how different types of auxiliary datasets influence backdoor purification effectiveness. Our findings provide novel insights and serve as a foundation for future research on optimizing dataset selection and construction for enhanced backdoor defense.
%
\textbf{2) Compilation and evaluation of diverse auxiliary datasets:}  We have compiled and rigorously evaluated a diverse set of auxiliary datasets using SOTA purification methods, making our datasets and code publicly available to facilitate and support future research on practical backdoor defense strategies.
%
\textbf{3) Introduction of GIC:} We introduce GIC, the \textbf{first} dedicated solution designed to align auxiliary datasets with the model’s learned representations, significantly enhancing backdoor mitigation across various dataset types. Our approach sets a new benchmark for practical and effective backdoor defense.



\section{Execution Tuning}
\label{sec:traces}


\begin{figure*}[thbp!]
\centering
\includegraphics[width=0.8\textwidth]{resources/et_diagrams6.pdf}
    \caption{Overview of the data pipeline in \name{} We start from Python functions made executable with synthetic yet representative inputs generated by a combination of LLMs and fuzzing, filtered by test quality. Our custom Python tracer generates a structured dataset of traces. From this dataset, we train models prompted with different trace representations.}
    \label{fig:overview}
\end{figure*}


In this section, we describe the two implementation challenges of \name{} The first one is about where and how to collect execution traces to construct a large and representative training dataset. The second challenge is how to represent these traces to ingest them to the model. Figure \ref{fig:overview} shows an overview of the pipeline for these two challenges.




\subsection{Traces collection}



%\textbf{Collecting executable functions} 
We start from a collection of unrestricted Python code from where functions are extracted, together with their corresponding module imports and auxiliary functions. We allow importing common Python libraries such as \texttt{pandas} or \texttt{matplotlib}. The inputs (function arguments) are generated with an LLM - more specifically, by prompting Llama 3 8B \citep{dubey2024llama3herdmodels} to generate unit tests for these functions. 
For increased coverage, inputs are also generated using fuzzing. In both cases, inputs yielding runtime errors are discarded, and inputs are filtered on test quality by measuring line coverage and similarity between tests.

In total, combining the LLM and fuzzing generated inputs,  we gather about $\sim$300k executable functions, with an average of 6 inputs per function. Using automatically generated inputs allows us to scale the training dataset to 
 $>$ 1.5M executions, without requiring manually written unit tests.
  \begin{wrapfigure}{r}{0.4\textwidth} % 'r' for right, 'l' for left
    \centering
    \includegraphics[width=0.4\textwidth]{resources/ex.pdf}
    \caption{Prompt for Instruction-1.}
    \label{fig:ex}
\end{wrapfigure}

We build a custom tracer leveraging Python's built-in \texttt{sys.settrace}. We capture all Python function call, return, line  and opcode events, and step into user-defined auxiliary functions (but not into functions from imported modules). We deliberately  ignore C events because without access to the source code emitting these events, the C traces would introduce noise to the data. We note that for correctly discarding the 
 C traces, we need to explicitly deactivate tracing of Python code called from C (e.g. a C function calling a Python lambda).
 We step into auxiliary functions present in the context (i.e., defined in the same file) and but don't do so for functions out of the context. Unlike previous work \citep{scratchpad, ni2024nextteachinglargelanguage}, we also capture globals, the stack, and state changes at the instruction-level granularity (i.e. Python opcodes). 
After tracing, we have constructed a structured dataset of executions. The next step is to turn this structured dataset into concrete (prompt, expected output) pairs to ingest to the models.


\subsection{Traces representation}
Following \citet{ni2024nextteachinglargelanguage}, we rely on Python's \texttt{\_\_repr\_\_} 
to have an LLM-friendly representation of each object. Unlike in \citet{ni2024nextteachinglargelanguage}, where authors summarized loops with the first two and the last iterations,  here we want to represent complete program executions, for which we consider different strategies. 


\textbf{Granularities}
We study the following trace granularities: \begin{enumerate*}

\item Direct predictions: Following \citet{Austin2021, gu2024cruxeval}, we fine-tune the models to directly predict the output (return value) from the input of the function.

\item Line-level (source code): Following \citet{scratchpad}, we represent the states at each executed line.

\item Instruction-level (bytecode): Source code lines can map to multiple instructions at the assembly or bytecode level. 
With this motivation in mind, we also consider a representation in which we explicitly show instructions and instruction-level state to the model. Crucially, this implies the introduction of the stack. Aside from the bytecode, the model has access to the source code, shown as inline comments at the first opcode of the corresponding line.
\end{enumerate*}




\textbf{Scratchpads}
We consider the following scratchpad techniques for storing intermediate computations (i.e., the traces):
\begin{enumerate*}
    \item Scratchpad: Following the original scratchpad work \citep{scratchpad}, the model predicts the state after executing each line, defined as the line itself plus the dictionary of the local variables, followed by the predicted return value.
    \item \textit{Compact} scratchpad: Inspired by \citet{ni2024nextteachinglargelanguage}'s trace representation, we also consider a \texttt{diff}-based scratchpad, in which the model only needs to predict the variables that change with respect  to the previous state. This should help at long executions by decreasing the token count. Note, though, that in \citet{ni2024nextteachinglargelanguage} this representation was not used as a scratchpad, but to annotate code.

  
    \item \textit{Dynamic} scratchpad: The two previous scratchpad strategies ask the model to predict the entire execution history of a program (paired with an input), up to the return value. This is problematic with long executions. With this motivation in mind, we introduce dynamic scratchpads, in which a single, self-contained state is updated by the model at every step. It also has the additional advantage that with the same strategy we can naturally train the model to skip steps that are potentially unnecessary to predict the final output, by asking the model to predict what the state will be after N steps. The caveat is that parts of the state that were implicitly encoded by having access to the execution history, now will need to be encoded explicitly. In particular, iterator states, not part of the locals dictionary in Python, can be ambiguous.\footnote{For example, in \texttt{for c in ('a','b','a')}, if we only have access to the current state, we need a way of distinguishing between the first \texttt{'a'} and the last one.
    We explicitly encode it with e.g., \texttt{\_\_for\_iterator\_1\_\_=2} lets us know the iteration count on a given iterator.
    } For this reason, even for the models with line-level granularity, we access the stack to trace the iterators, and explicitly encode their iteration count.
\end{enumerate*}


Figure \ref{fig:scratchpad} depicts the differences between direct output prediction, scratchpad-based output prediction, and dynamic scratchpad. Compact scratchpad is omitted for brevity; it's similar to scratchpad but just predicting the variables that change. Figure \ref{fig:ex} provides a prompt example.




\section{Output prediction results}
\label{sec:intrinsic}

\begin{table*}[thbp!]
\centering
\caption{\label{tab:intrinsic_crux_e}
%Crux evaluations.
Results of individual state predictions on CruxEval, i.e. before aggregating steps into full executions for output prediction. The accuracy is broken down into control flow (does the model  correctly predict the next line?), variables (does the model correctly predict the variable values in the next state?), iterators (does the model correctly predict the iteration count for the current iterators?) stack state, and full state accuracy (how many states are completely correct, i.e. control flow, variables, iterators, and stack are all correct, assuming the state had them). Note that scratchpad does not have iterator states because it does not require them, and line-level models do not have access to the stack. In this Table, F.T. means that the models were fine-tuned on the task, while prompted results indicate no training on traces. %
}
\scalebox{0.95}{
\begin{tabular}{llrrrrr}
\hline
\textsc{Repr.} & \textsc{Model} & \textsc{C. flow} & \textsc{Vars} & \textsc{Iterator} & \textsc{Stack} & \textsc{Full}\\
\hline

Scratchpad & Llama3.1 8B + F.T. & 91.9\% & 86.5\% & - & - & 86.4\%\\

\hline
Line-1  & Llama3.1 8B (prompted)  & 53.8\% & 26.9\% & 39.6\% & - & 10.6\%\\
& \hspace{3mm} + F.T. & 99.5\% & 97.7\% & 99.7\% & - & 96.3\%\\

\hline 
Line-n (global) & Llama3.1 8B (prompted)  & 16.8\% &16.0\%  & 12.3\% & -  & 1.8\%\\
 & \hspace{3mm} + F.T. & 95.1\% & 66.8\% & 96.4\% & - & 79.0\%\\

\hline 
Instruction-1 & Llama3.1 8B (prompted) & 74.1\% & 80.4\% & 77.9\%& 5.8\% & 2.8\% \\
 & \hspace{3mm} +F.T.  & 99.9\% & 99.9\% & 99.9\% & 98.8\% & 98.8\% \\


\hline
\end{tabular}
} 

\end{table*}




In this section, we evaluate models on function output prediction, a proxy for code reasoning, comparing different trace representation strategies. All evaluated models are  fine-tuned using comparable hyperparameters from the instruct version of Llama 3.1-8B \citep{dubey2024llama3herdmodels}, unless stated otherwise.

We start by analyzing the results for individual step predictions. Then, we aggregate these step predictions to evaluate output prediction on CruxEval. Next,  aiming at evaluating on longer and more diverse executions, we also run our models on a subset of MBPP (selecting functions with nested loops) and algorithmic tasks with arbitrarily long execution lengths.

We will refer to the dynamic scratchpad models by using \{Line$\mid$Instruction\}-\{1$\mid$n\}, where Line models have a granularity of lines and Instruction models have a granularity of bytecode instructions. ``1" refers to models trained to predict the next step, while ``n" refers to models trained to predict multiple steps into the future, with n = \{1..10\}. Additionally, note that in our results, we refer to our re-implementation of \citet{scratchpad} as ``Scratchpad", which benefits from the increased data size and context length. The original Scratchpad restricted context windows to 512 tokens; here, we allow up to 8192 tokens.


\begin{wrapfigure}{r}{0.5\textwidth} 


    \centering

    \includegraphics[width=0.5\textwidth]{resources/line_n_v2.pdf}
    \caption{Plot showing \textit{individual} state prediction accuracy (e.g., for \textit{Return}, specifically for this plot and unlike in the rest of the article, we mean return statement accuracy, not full execution accuracy) when increasing N lines into the future, compared to the predictions Negative Log-Likelihood. Accuracy (bars) gets lower as the number of steps into the future increases, and confidence decreases as well (i.e., NLL increases).
    }
    \label{fig:variable_ns}

\end{wrapfigure}

\subsection{Individual state predictions on CruxEval}


Table \ref{tab:intrinsic_crux_e} shows the evaluations of individual states (not aggregated into full executions) for prompted out-of-the-box Llama 3.1 8B \citep{dubey2024llama3herdmodels}  and fine-tunings with traces on top of Llama 3.1 8B. In this section, for Line-n models, we report the average over n. 
\paragraph{Prompted (untrained) models} We find that general-purpose LLMs already exhibit non-trivial capabilities to predict execution steps out of the box. For example, Llama 3.1 8B prompted (i.e., not fine-tuned) to predict the state after executing the next line (Line-1 in Table \ref{tab:intrinsic_crux_e}) correctly guesses the answer 53.8\% of the times, implying a decent understanding of control flow. Control flow is considerably easier when prompted to predict the next bytecode state (Instruction-1 in Table \ref{tab:intrinsic_crux_e}), with a 74.1\%, since non-jump (and non-function calling) bytecode instructions have a linear flow. These control flow capabilities drop to 16.8\% when evaluated on \{1..10\} (averaged) lines into the future (Line-n in Table \ref{tab:intrinsic_crux_e}, i.e. asking the model to predict the state after N lines). Similar results, albeit slightly worse, are obtained for the iterator states. When looking at variable value predictions (Vars column), however, performance drops substantially for the line-level scratchpad. This struggle compared to other mainstream coding benchmarks hints at a lack of execution traces data in general-purpose LLMs. Notably, variable prediction for prompted Llama greatly improves for the Instruction-level variant (80.4\%). However, this is due to the heavy lifting being carried out on the stack, as variable states typically just consist of reading values from the stack into the variables. The stack-level accuracy is indeed low (5.8\%).




\textbf{Execution-tuned Models} Unsurprisingly, models trained on execution traces excel at control flow prediction. In particular, dynamic scratchpad models obtain almost perfect accuracy on control flow prediction, both for line (99.5\%) and instruction levels (99.9\%). The control flow accuracy only drops to 95.1\% for Line-n models, suggesting that the model is indeed capable of internally modeling the flow of future states. In comparison, scratchpad obtains a similarly high 91.0\%. Looking at iterator state prediction, both line and instruction-level dynamic scratchpads obtain almost perfect accuracy as well. Remarkably, the instruction-level model obtains an also near-perfect accuracy for the stack (98.8\%), in contrast to the prompted model. 


\textbf{Skipping steps} Figure \ref{fig:variable_ns} shows the state prediction accuracy when increasing the number of states into the future and the corresponding negative log-likelihood (NLL) as a measure of the confidence of the prediction, for the Line-N dynamic scratchpad. Unsuprisingly, accuracy lowers as N increases. Interestingly, the corresponding NLL increases, showing calibration of the model confidence. Remarkably, however, we do not observe sharp drops in performance when looking at N steps into the future. Actually, our model can effectively learn to predict multiple steps into the future. Control flow and iterator states are relatively feasible to predict when jumping multiple steps, but variables and return values get increasingly complicated. In the Appendix, we provide similar results for Instruction-N.
 

In summary, while the out-of-the-box, prompted Llama shows non-trivial trace modeling capabilities (10.6\% full state accuracy with the Line-1 approach), models trained on traces greatly improve upon it. Interestingly, we observe that the line-based dynamic scratchpad outperforms (96.3\% full accuracy) its scratchpad counterpart (86.4\% full state accuracy), and that the instruction-level obtains the highest full state accuracy, 98.8\%. We also observe that the task of learning the state of N steps into the future is feasible to learn effectively, and that NLL has potential as a measure of model confidence in this setting. However, it remains to be seen how these  individual trace results will aggregate into function output predictions, which we study in the following sections.




\subsection{CruxEval output prediction}

\begin{table*}[thbp!]
\centering
\caption{\label{tab:crux_o}
CruxEval output prediction results, allowing for multi-step predictions  for the variants trained with execution traces.  *Global search using \citet{dijkstra1959note} the algorithm. Not directly comparable due to having access to the ground truth for checking correctness of paths.}
\scalebox{0.87}{
\begin{tabular}{llll}
\hline
\textsc{Representation} & \textsc{Outcome Accuracy} & \textsc{Process Accuracy} & \textsc{avg Steps needed} \\
\hline
Output FT & 49.3\% & - & 1 (direct)\\

\hline

Scratchpad F.T. & 78.7\% & 75.5\% & 10.8 lines \\
Compact Scratchpad F.T. & \textbf{79.7\%} & \textbf{76.6\%} & 11.8 lines \\


 Line-1 FT & 73.3\% & 73.3\% & 8.3 lines\\
Line-n FT & 60.8\% & 60.8\% & \textbf{2.9} lines \\
\hspace{3mm} + search* & 70.3\% & 70.3\% & \textbf{1.8} lines \\
 


Llama 3.1 8B + Instruction-1 F.T & 73.5\% & 73.5\% & 38.8 instructions  \\
\hspace{3mm} + search*  & 74.1\% & 74.1\% & 38.6 instructions \\
Llama 3.1 8B + Instruction-n F.T.  & 62.5\% & 62.5\% & \textbf{22.4} instructions  \\
\hspace{3mm} + search* & 73.5\% & 73.5\% & \textbf{4.8} instructions \\

\hline

Prompted Llama 3.1 8B** & 37.8\% & - & -\\ 
Prompted GPT-4 & 82\% & - & -\\

\hline
\end{tabular}
} 


\end{table*}




We have seen the accuracy of the models when evaluated on individual state predictions. Here, we aggregate the results to evaluate output prediction on CruxEval in Table \ref{tab:crux_o}. For dynamic scratchpad models with more than one possible path (e.g., Line-n), we evaluate taking the \texttt{argmin(NLL)} one. Interestingly, the most confident prediction is not always the next immediate step, which is why predicting multiple steps ahead can lead to fewer overall steps.
We also show results with global search using \citet{dijkstra1959note}'s algorithm to obtain the shortest path using model predictions from the input of the function to the output,  which is not directly comparable to the other results due to having access to the ground truth for checking the correctness of paths. However, we provide it as an upper bound of what could be achieved with the predictions of the model. As a reference, we also  provide the top results in the CruxEval leaderboard, prompted GPT-4 with 82\% accuracy.\footnote{Pass-at-1, \texttt{gpt-4-turbo-2024-04-09+cot (n=3)}as of October 2024}

\textbf{Direct prediction} Out of the box, Llama 3.1 8B obtains an output prediction accuracy of 37.8\%. This accuracy can be improved to 49.3\% by fine tuning on direct output prediction. However, even with the relatively short executions found in CruxEval, more than half of the functions are out of the reach of the direct output prediction model.

\textbf{Results when using traces} Consistently with \citep{scratchpad}, all models trained on execution traces outperform by a great margin the direct output prediction fine-tuning. While we  generally obtain high accuracies (up to almost 80\%), note that the accuracy here, in Table \ref{tab:crux_o}, is substantially lower than in Table \ref{tab:intrinsic_crux_e}. The reason why this happens is that when aggregating individual trace predictions, a single step error (out of, e.g., 20 steps) can lead to a wrong result.   

\textbf{Comparison between models using traces} The compact scratchpad strategy slightly outperforms the full scratchpad one, and in turn these two outperform the dynamic scratchpad approaches. The executions in CruxEval are not long enough to show the advantages of dynamic scratchpads.  

\textbf{Indexing and string manipulation failure modes} In CruxEval, arithmetic operations (a classic failure mode of LLMs) were intentionally left out of the benchmark, to focus on program understanding itself. However, we noticed two interesting failure modes. After a qualitative error analysis, we found that the majority of the errors of the models on CruxEval belong to either one of two categories. The first one is string indexing. Indexing arbitrary strings is hard due to tokenization artifacts, since literals are tokenized inconsistently, and unlike arrays their elements aren't separated by punctuation.
However, it can be particularly hard for the dynamic scratchpad models (and this mainly explains the $\sim$ 5\% gap in output prediction accuracy between the line-level scratchpad and its dynamic scratchpad counterpart), because for each iteration the model needs to count from scratch to which characters the code is referring. Instead, the scratchpad model relies on the previous iteration as a hint to guess what character will come next. 
The second failure mode we saw consists of basic string manipulation. For example, models sometimes fail to predict the return value of Python's built-in \texttt{[string].istitle()} method, an issue that we also observed in the base model. CruxEval's string values might be out-of-distribution for the model.

\textbf{Skipping steps} Looking at the results of Line-n and Instruction-n, we observe that just by selecting the $n$ where the model is the most confident (based on NLL), we are able to obtain reasonable  accuracies (significantly better than direct prediction, albeit worse than Line-1 and Instruction-1) and considerably lesser number of steps needed. For example, Line-n on average needs only a 35\% of the steps of Line-1 to correctly predict a function. This has the remarkable implication that the ordering of model confidence for n=\{1..10\} does not always correspond to the number of steps into the future. That is, with significant frequency, $n=1$ is not always the prediction in which the model is the most confident. 





\subsection{MBPP}

Next, we evaluate on the Python synthesis dataset MBPP \citep{mbpp} with the goal of observing results in longer executions and different domain as CruxEval. 

Particularly, we select functions in the MBPP test set with nested loops (as a  proxy of computational complexity and execution length), leaving us with slightly fewer than 100 functions.\footnote{Unfortunately, for MBPP we discovered an issue with traced iterators in nested \texttt{for} loops. Thus, specifically for MBPP we applied an AST transformation to rewrite nested \texttt{for} loops to \texttt{while} loops. This issue had virtually no effect in CruxEval, due to the lack of executed nested for loops.} 

Similarly to the case of CruxEval, Table \ref{tab:mbpp} shows the results on output prediction for this MBPP subset. The big picture of the results is similar to the case of CruxEval, but with some crucial differences. First, if we look at the average steps needed for correct predictions, we see that here the functions are indeed considerably longer than in CruxEval (in the order of 7x more executed lines). However, the lengths are still not astronomical. 
Relatively to the CruxEval results, here the instruction-based models perform considerably better, which we attribute to the fact that in MBPP there are computations that can be broken down into multiple instructions. Instead, in CruxEval, since most errors consist of indexing or guessing the outputs of string built-in methods, further zooming in doesn't help, as the (indexing or calling a string built-in written in C) can't be further divided.
Since in this benchmark some functions present auxiliary functions, we introduce a variant of the compact scratchpad in this the model is able to step in other called functions, yielding  an improvement of 3 points with respect to compact scratchpad (80.6\% vs. 77.4\%).

\begin{table*}[thbp!]
\centering
\caption{\label{tab:mbpp}
Evaluation on MBPP test set on functions with nested loops  }
\scalebox{0.87}{
\begin{tabular}{llll}
\hline
\textsc{Representation} & \textsc{Outcome Accuracy} & \textsc{Process Accuracy} & \textsc{avg Steps needed} \\
\hline
Output  F.T. & 47.3\%  & - & 1 (direct)\\

\hline

Scratchpad  F.T. & 64.5\% & 64.5\%  & 58.2  lines  \\
Compact Scratchpad F.T. & 77.4\%& 76.3\%  & 73.9  lines  \\
Compact Scratchpad +step-in F.T. & \textbf{80.6\%}& 74.2\%  & 73.9  lines  \\



Line-1 F.T. & 73.1\%  &73.1\%  & 73.8 lines\\
Line-n F.T. & 43\% & 43\% & 15.4  lines \\
\hspace{3mm} + search* & 59.1\%  & 59.1\%  & 7.8  lines \\
 


Instruction-1 F.T. & 78.5\% & 78.5\%  & 351.3 instructions  \\
\hspace{3mm} + search*  & \textbf{80.6\%} & 80.6\%  & 354.6  instructions \\
Instruction-n F.T. &65.6\% &65.6\% & 139.8 instructions  \\
\hspace{3mm} + search* & 88.2\%  & 88.2\% & 35  instructions \\

\end{tabular}
} 


\end{table*}







\subsection{Long executions}

We observe that existing benchmarks for output prediction don't feature long executions. This is especially true for the standard one, CruxEval, but even when targeting functions with nested loops on MBPP, we rarely get to executions with more than 100 executed lines. In this section, we study well-known algorithmic tasks where we can obtain arbitrarily long executions: 
\begin{enumerate*}
    \item Collatz conjecture: a function returning the number of iterations required to reach 1 following the Collatz conjecture sequence, given a starting natural number.
    \item Binary counter: A 4-bit binary counter.
    \item Iterative Fibonacci: An iterative implementation of Fibonacci.
\end{enumerate*}



For selecting the inputs, we generate 4 random numbers (as the small inputs) between 1 and 20, and 5 between 20 and 4000 (as the larger inputs), and evaluate on all of them across the 3 functions. For Fibonacci, we restrict the evaluation on the smaller 5 numbers. For all functions in this section, we replace the function name by \texttt{f}, to give less hints to the model based on potential memorizations of well-known functions during pretraining. Table \ref{tab:summary} shows the summarized results on these tasks (see Appendix \ref{app:long_res_fine} for the fine-grained results).

\textbf{Collatz} The direct output prediction model is able to correctly predict the number of Collatz iterations for the 4 smaller numbers (up to $n=18$), and breaks for larger inputs. Curiously, the scratchpad model is not able to improve on the results of the direct output prediction model, and gets the same accuracy for a considerably increased number of intermediate steps (35 on average, corresponding to the number of executed lines). The compact scratchpad unlocks a larger input, $n=103$, for which is able to do 353 correct intermediate predictions, up to the (correct return value). 
The dynamic scratchpad models shine in this setting. Line-1 is able to correctly predict all studied inputs but 2620 (the next to largest one). For the largest input, $n=3038$, Line-1 needs to chain 619 correct predictions in a row.  Notably, Line-n is able to achieve the same accuracy but with only 39\% of the steps required by Line-1. Optimally, if we had access to an oracle that told us which of the paths was correct, Dijsktra would have yielded a perfect accuracy with only 32.3 steps required on average (compared to the average of 271 for Line-1).
\begin{wraptable}{r}{0.65\textwidth} % Adjust width as needed
\centering
\caption{\label{tab:summary} Long execution results: accuracy (avg. steps needed).}
\scalebox{0.85}{
\begin{tabular}{l|c|c|c}
\hline
\textsc{Representation} & \textsc{Collatz} & \textsc{Binary Counter} & \textsc{Fibonacci} \\
\hline
Output & 4/9 (1) & 1/9 (1) & 4/5 (1) \\
Scratchpad & 4/9 (35) & 4/9 (45.5) & 4/5 (37) \\
Compact Scratchpad & 5/9 (98.6) & 5/9 (116.2) & 5/5 (140.5) \\
Line-1 & 8/9 (271) & \textbf{8/9 (2441)} & 5/5 (140.5) \\
Line-n & \textbf{8/9 (106.1)} & 1/9 (6) & \textbf{5/5 (46.8)} \\
Line-n + Dijkstra & \textbf{9/9 (32.3)} & \textbf{9/9 (410.7)} & \textbf{5/5 (11.8)} \\
\hline
\end{tabular}
}
\end{wraptable}
\textbf{Binary counter} In the case of the 4-bit binary counter, curiously, the direct output prediction model is only able to correctly predict the output for the third smallest input ($n=8$). In this case, scratchpad does significantly improve results with respect to the direct output prediction model, correctly guessing the outputs for the 4 smaller inputs (up to $n=18$). However, the compact scratchpad is still better, unlocking the correct prediction for a bigger input, $n=103$. Curiously, Line-1 gets the same accuracy as with Collatz (all correct but the next to largest input), but with one crucial difference. Here, the executions are even longer. For correctly predicting the output for $n=3038$, Line-1 has to chain as many as 14,055 correct predictions in a row. Here, Line-n is not able to reliably select across paths based on its confidence (NLL). With Dijkstra on Line-n predictions, it would have obtained perfect accuracy with only 17\% of the number of steps needed by Line-1 on average.

\textbf{Fibonacci} Again, scratchpad obtains the exact same accuracy as direct prediction. The rest of the models are able to predict all inputs up to $n=103$. While both Compact scratchpad and Line-1 require 414 steps for $n=103$, Line-n is able to decrease this number to only 160 steps. Optimally, Dijkstra would have obtained 42.

\section{Downstream effects}
\label{sec:extrinsic}





So far we have shown that \name{} leads to improved output prediction capabilities. 
Here, we study its effects in a code supervised fine-tuning (SFT)  setting. 
We take the base Llama 3.1 8B \citep{dubey2024llama3herdmodels}, and evaluate the downstream performance with and without different versions of \name{} in the data mix. The base mix is a small code-only SFT dataset of samples similar to the ones in \citet{roziere2024codellamaopenfoundation}. We train for 7.5k steps with a global batch size of 1024 sequences of up to 8192 tokens.
We evaluate code generation on HumanEval \citep{chen2021codex} and MBPP \citep{mbpp}, without including the traces in inference time. We also evaluate a step-by-step reasoning task, GSM8k\citep{cobbe2021gsm8k}, to study 
 potential improved multi-step reasoning in other domains. 

Table \ref{tab:downstream} shows downstream evaluation results with and without \name{} in the SFT mix. Fine-tuning on direct I/O prediction improves Crux-I and Crux-O but not coding benchmarks. The best-performing trace variant, with 10\% Compact Scratchpad, brings slight gains on HumanEval, MBPP, and 1.2 points on GSM8K. Curiously, forward execution fine-tuning worsens Crux-I, and vice versa, suggesting weaker-than-expected ties  between forward and backward prediction. These results indicate that merging \name{} with SFT data offers little coding improvement. We hypothesize increased gains in evaluations  related to program state, such as  test generation or debugger-assisted tasks.



\begin{table}[thbp!]
  \center
  \caption{ Downstream evaluations on HumanEval, MBPP and GSM8K.
  \label{tab:downstream}}
  \scalebox{0.74}{
   \setlength{\tabcolsep}{3pt}
  \begin{tabular}{l|cc|cc|ccc|ccc|c}
  \toprule
  Model & \multicolumn{2}{c}{Crux-I} & \multicolumn{2}{c}{Crux-O} & \multicolumn{3}{c}{HumanEval} & \multicolumn{3}{c}{MBPP} & GSM8K \\
  & pass@1 & pass@5 & pass@1 & pass@5   & pass@1 & pass@10 & pass@100 & pass@1 & pass@10 & pass@100 & 0-shot\\
  \midrule 
 SFT mix  & 42.9 & 56.5 & 36.4 & 52 & 56.7 & 79.1 &  \textbf{91.2} & 52.2 & 69.4 & 80.7 & 66.3\\ 
  \midrule
   \hspace{3mm} + input FT (10\%)   & \textbf{43.8} & \textbf{57.9} & 34.8  & 46.8  & 51.8 & 77.6  &90.7  & 52.3  & 68.2 & 79.4 & 66\\ 
    \hspace{3mm} + output FT (10\%)    & 41.1 & 56.2 & \textbf{41.8} & \textbf{52.5}  & 56.7  & 79.2  & 91 & 23.8  & 65.2 & 79.1 & 65.1 \\ 
     \midrule
    \hspace{3mm} + C. Scratch  (10\%)  & 41.8 & 56 & 38.8 & 49.8 & \textbf{58.5}  & \textbf{79.9}  & 89.9  & \textbf{53}  & \textbf{69.6} & \textbf{82.5} & \textbf{67.5} \\ 
        \hspace{3mm} + C. Scratch (5\%)  & 42.9 & 58.3 & 38 & 50.5 & 57.9 & 78.9  & 89.3 & 52.6  & 69.3 & 81 & 66.3 \\
        \hspace{3mm} + Line-1 (10\%)    & 39.8 &  56.5& 38 & 50.5  &  53.7 &  78.7  &  89.2  & \textbf{53}  & 69.2 & 80.2 & 65 \\
    \hspace{3mm} + Line-n  (10\%)   & 39.4 & 56.3& 38.8 & 49.1  &  56.1&  78.3 &  88.7  & 51.4  & 68.5 & 80.8 & 66.2 \\


  \bottomrule
  \end{tabular}
  }
  
\end{table}



\section{Discussion}

\subsection{Related Prior Work}

\textbf{Training Energy Based Models}.
A number of recent works aim to improve EBM training. \citet{zhu2023learning} uses a diffusion model to reduce the number of Langevin steps within the recovery likelihood approach of \cite{gao2020learning}. \cite{schroder2024energy} eliminates the need for MCMC and $\nabla_x$-computation of the energy during training by using a contrastive loss with forward noising process instead of MCMC, coined Energy Divergence (ED). ED is a promising alternative to E-DSM and has connections to score-matching, but, as shown in \Cref{tab:unconditional_gen}, it is not yet competitive, and  suffers from a bias by choice of noisy energy function.

\textbf{Composition with MCMC}. Similar to our work, \cite{du2023reduce} also uses energy-parameterized diffusion models but performs controlled generation with MCMC rather than SMC. SMC is known to suffer weight degeneracy high dimension, resulting in lack of diversity across particles, MCMC does not suffer from this, though requires additional non-parallel steps which is time consuming. The approaches are however complementary, and indeed one may perform MCMC after resampling steps to promote diversity.

\textbf{Sequential Monte Carlo in Diffusion Models.}
Many recent works use SMC within diffusion models for conditional generation, we detail the FKM formulations of these works in in \Cref{app:fkm_potentials}.

\cite{wu2024practical} uses twisted SMC with a classifier-guided proposal \citep{dhariwal2021diffusion} and potentials approximated with diffusion posterior sampling \citep{chungdiffusion}, which has been detailed as a FKM by concurrent work \citep{zhao2024conditional}. \cite{cardoso2024monte} and \cite{dou2024diffusion} tackle linear inverse problems where potentials have a closed form using Gaussian conjugacy. \cite{li2024derivative} perform SMC for both discrete and continuous diffusion models whereby potentials consist of a reward function applied to $\mathbb{E}[\bfX_0|x_t]$.

\cite{liu2024correcting} corrects conditional generations using an adversarially trained density ratio potential, and scales this to text-to-image models. 


\textbf{SMC for LLMs}. SMC is not only popular within diffusion models, but has been successful within large language models (LLMs) \citep{lew2023sequential, zhao2024probabilistic}. \cite{lew2023sequential} uses a FKM formulation with indicator based potential functions similar to as detailed in \Cref{sec:bounded}, and \cite{zhao2024probabilistic} discuss using SMC for text using potentials from reward functions or learning such potentials via contrastive twist learning, similar to contrastive learning for EBMs.

\subsection{Concurrent work} Since submission/ acceptance of our work \footnote{Submission October 2024}, there have been a number of relevant concurrent works. 

\textbf{FKM Interpretation}. \cite{singhal2025general} similarly to \cite{zhao2024conditional} and this work, detail sampling diffusion models in terms of KFM. \cite{singhal2025general} follow \cite{li2024derivative} in using reward functions based potentials but focus on text-to-image reward, and explore further heuristics such as or combining rewards via sum or max; and sampling $\bfX_0|x_t$ via nested diffusion \citep{elata2024nested} as input to their reward rather than using $\mathbb{E}[\bfX_0|x_t]$ as done in \cite{li2024derivative}.

\textbf{SMC for discrete diffusion}.  \cite{lee2025debiasing} use SMC for low temperature sampling for discrete diffusion models. \cite{xu2024energy} use a pretrained autoregressive likelihood model applied to samples $\bfX_0|x_t$ for a potential within discrete diffusion sampling.

\textbf{Composition}. \cite{skreta2024superposition} construct a cheap density estimator by simulating from an SDE, which can be computed at sampling time if using reverse diffusion solver, though it is not clear if this can be used in conjunction with resampling and Langevin corrector schemes. \cite{skreta2024superposition} then use this estimator to perform composition-type sampling, however their logical \textsc{AND} appears to differ from other more commonly used logical \textsc{AND} operations, in that it targets samples with equal probability between classes rather than generating both classes, e.g. "a CAT and a DOG" results in a cat/dog hybrid optical illusion rather than a separate cat and separate dog in one image. 

\cite{bradley2025mechanisms} explore composition more formally, establishing types of composition and cases where summing scores is sufficient without need for SMC correction as performed in this work or with MCMC correction from \citep{du2023reduce}.




\paragraph{Summary}
Our findings provide significant insights into the influence of correctness, explanations, and refinement on evaluation accuracy and user trust in AI-based planners. 
In particular, the findings are three-fold: 
(1) The \textbf{correctness} of the generated plans is the most significant factor that impacts the evaluation accuracy and user trust in the planners. As the PDDL solver is more capable of generating correct plans, it achieves the highest evaluation accuracy and trust. 
(2) The \textbf{explanation} component of the LLM planner improves evaluation accuracy, as LLM+Expl achieves higher accuracy than LLM alone. Despite this improvement, LLM+Expl minimally impacts user trust. However, alternative explanation methods may influence user trust differently from the manually generated explanations used in our approach.
% On the other hand, explanations may help refine the trust of the planner to a more appropriate level by indicating planner shortcomings.
(3) The \textbf{refinement} procedure in the LLM planner does not lead to a significant improvement in evaluation accuracy; however, it exhibits a positive influence on user trust that may indicate an overtrust in some situations.
% This finding is aligned with prior works showing that iterative refinements based on user feedback would increase user trust~\cite{kunkel2019let, sebo2019don}.
Finally, the propensity-to-trust analysis identifies correctness as the primary determinant of user trust, whereas explanations provided limited improvement in scenarios where the planner's accuracy is diminished.

% In conclusion, our results indicate that the planner's correctness is the dominant factor for both evaluation accuracy and user trust. Therefore, selecting high-quality training data and optimizing the training procedure of AI-based planners to improve planning correctness is the top priority. Once the AI planner achieves a similar correctness level to traditional graph-search planners, strengthening its capability to explain and refine plans will further improve user trust compared to traditional planners.

\paragraph{Future Research} Future steps in this research include expanding user studies with larger sample sizes to improve generalizability and including additional planning problems per session for a more comprehensive evaluation. Next, we will explore alternative methods for generating plan explanations beyond manual creation to identify approaches that more effectively enhance user trust. 
Additionally, we will examine user trust by employing multiple LLM-based planners with varying levels of planning accuracy to better understand the interplay between planning correctness and user trust. 
Furthermore, we aim to enable real-time user-planner interaction, allowing users to provide feedback and refine plans collaboratively, thereby fostering a more dynamic and user-centric planning process.



%\subsubsection*{Author Contributions}
%If you'd like to, you may include  a section for author contributions as is done
%in many journals. This is optional and at the discretion of the authors.
%
%\subsubsection*{Acknowledgments}
%Use unnumbered third level headings for the acknowledgments. All
%acknowledgments, including those to funding agencies, go at the end of the paper.

\clearpage

\bibliography{main}
\bibliographystyle{iclr2025_conference}

\clearpage

\appendix

\subsection{Lloyd-Max Algorithm}
\label{subsec:Lloyd-Max}
For a given quantization bitwidth $B$ and an operand $\bm{X}$, the Lloyd-Max algorithm finds $2^B$ quantization levels $\{\hat{x}_i\}_{i=1}^{2^B}$ such that quantizing $\bm{X}$ by rounding each scalar in $\bm{X}$ to the nearest quantization level minimizes the quantization MSE. 

The algorithm starts with an initial guess of quantization levels and then iteratively computes quantization thresholds $\{\tau_i\}_{i=1}^{2^B-1}$ and updates quantization levels $\{\hat{x}_i\}_{i=1}^{2^B}$. Specifically, at iteration $n$, thresholds are set to the midpoints of the previous iteration's levels:
\begin{align*}
    \tau_i^{(n)}=\frac{\hat{x}_i^{(n-1)}+\hat{x}_{i+1}^{(n-1)}}2 \text{ for } i=1\ldots 2^B-1
\end{align*}
Subsequently, the quantization levels are re-computed as conditional means of the data regions defined by the new thresholds:
\begin{align*}
    \hat{x}_i^{(n)}=\mathbb{E}\left[ \bm{X} \big| \bm{X}\in [\tau_{i-1}^{(n)},\tau_i^{(n)}] \right] \text{ for } i=1\ldots 2^B
\end{align*}
where to satisfy boundary conditions we have $\tau_0=-\infty$ and $\tau_{2^B}=\infty$. The algorithm iterates the above steps until convergence.

Figure \ref{fig:lm_quant} compares the quantization levels of a $7$-bit floating point (E3M3) quantizer (left) to a $7$-bit Lloyd-Max quantizer (right) when quantizing a layer of weights from the GPT3-126M model at a per-tensor granularity. As shown, the Lloyd-Max quantizer achieves substantially lower quantization MSE. Further, Table \ref{tab:FP7_vs_LM7} shows the superior perplexity achieved by Lloyd-Max quantizers for bitwidths of $7$, $6$ and $5$. The difference between the quantizers is clear at 5 bits, where per-tensor FP quantization incurs a drastic and unacceptable increase in perplexity, while Lloyd-Max quantization incurs a much smaller increase. Nevertheless, we note that even the optimal Lloyd-Max quantizer incurs a notable ($\sim 1.5$) increase in perplexity due to the coarse granularity of quantization. 

\begin{figure}[h]
  \centering
  \includegraphics[width=0.7\linewidth]{sections/figures/LM7_FP7.pdf}
  \caption{\small Quantization levels and the corresponding quantization MSE of Floating Point (left) vs Lloyd-Max (right) Quantizers for a layer of weights in the GPT3-126M model.}
  \label{fig:lm_quant}
\end{figure}

\begin{table}[h]\scriptsize
\begin{center}
\caption{\label{tab:FP7_vs_LM7} \small Comparing perplexity (lower is better) achieved by floating point quantizers and Lloyd-Max quantizers on a GPT3-126M model for the Wikitext-103 dataset.}
\begin{tabular}{c|cc|c}
\hline
 \multirow{2}{*}{\textbf{Bitwidth}} & \multicolumn{2}{|c|}{\textbf{Floating-Point Quantizer}} & \textbf{Lloyd-Max Quantizer} \\
 & Best Format & Wikitext-103 Perplexity & Wikitext-103 Perplexity \\
\hline
7 & E3M3 & 18.32 & 18.27 \\
6 & E3M2 & 19.07 & 18.51 \\
5 & E4M0 & 43.89 & 19.71 \\
\hline
\end{tabular}
\end{center}
\end{table}

\subsection{Proof of Local Optimality of LO-BCQ}
\label{subsec:lobcq_opt_proof}
For a given block $\bm{b}_j$, the quantization MSE during LO-BCQ can be empirically evaluated as $\frac{1}{L_b}\lVert \bm{b}_j- \bm{\hat{b}}_j\rVert^2_2$ where $\bm{\hat{b}}_j$ is computed from equation (\ref{eq:clustered_quantization_definition}) as $C_{f(\bm{b}_j)}(\bm{b}_j)$. Further, for a given block cluster $\mathcal{B}_i$, we compute the quantization MSE as $\frac{1}{|\mathcal{B}_{i}|}\sum_{\bm{b} \in \mathcal{B}_{i}} \frac{1}{L_b}\lVert \bm{b}- C_i^{(n)}(\bm{b})\rVert^2_2$. Therefore, at the end of iteration $n$, we evaluate the overall quantization MSE $J^{(n)}$ for a given operand $\bm{X}$ composed of $N_c$ block clusters as:
\begin{align*}
    \label{eq:mse_iter_n}
    J^{(n)} = \frac{1}{N_c} \sum_{i=1}^{N_c} \frac{1}{|\mathcal{B}_{i}^{(n)}|}\sum_{\bm{v} \in \mathcal{B}_{i}^{(n)}} \frac{1}{L_b}\lVert \bm{b}- B_i^{(n)}(\bm{b})\rVert^2_2
\end{align*}

At the end of iteration $n$, the codebooks are updated from $\mathcal{C}^{(n-1)}$ to $\mathcal{C}^{(n)}$. However, the mapping of a given vector $\bm{b}_j$ to quantizers $\mathcal{C}^{(n)}$ remains as  $f^{(n)}(\bm{b}_j)$. At the next iteration, during the vector clustering step, $f^{(n+1)}(\bm{b}_j)$ finds new mapping of $\bm{b}_j$ to updated codebooks $\mathcal{C}^{(n)}$ such that the quantization MSE over the candidate codebooks is minimized. Therefore, we obtain the following result for $\bm{b}_j$:
\begin{align*}
\frac{1}{L_b}\lVert \bm{b}_j - C_{f^{(n+1)}(\bm{b}_j)}^{(n)}(\bm{b}_j)\rVert^2_2 \le \frac{1}{L_b}\lVert \bm{b}_j - C_{f^{(n)}(\bm{b}_j)}^{(n)}(\bm{b}_j)\rVert^2_2
\end{align*}

That is, quantizing $\bm{b}_j$ at the end of the block clustering step of iteration $n+1$ results in lower quantization MSE compared to quantizing at the end of iteration $n$. Since this is true for all $\bm{b} \in \bm{X}$, we assert the following:
\begin{equation}
\begin{split}
\label{eq:mse_ineq_1}
    \tilde{J}^{(n+1)} &= \frac{1}{N_c} \sum_{i=1}^{N_c} \frac{1}{|\mathcal{B}_{i}^{(n+1)}|}\sum_{\bm{b} \in \mathcal{B}_{i}^{(n+1)}} \frac{1}{L_b}\lVert \bm{b} - C_i^{(n)}(b)\rVert^2_2 \le J^{(n)}
\end{split}
\end{equation}
where $\tilde{J}^{(n+1)}$ is the the quantization MSE after the vector clustering step at iteration $n+1$.

Next, during the codebook update step (\ref{eq:quantizers_update}) at iteration $n+1$, the per-cluster codebooks $\mathcal{C}^{(n)}$ are updated to $\mathcal{C}^{(n+1)}$ by invoking the Lloyd-Max algorithm \citep{Lloyd}. We know that for any given value distribution, the Lloyd-Max algorithm minimizes the quantization MSE. Therefore, for a given vector cluster $\mathcal{B}_i$ we obtain the following result:

\begin{equation}
    \frac{1}{|\mathcal{B}_{i}^{(n+1)}|}\sum_{\bm{b} \in \mathcal{B}_{i}^{(n+1)}} \frac{1}{L_b}\lVert \bm{b}- C_i^{(n+1)}(\bm{b})\rVert^2_2 \le \frac{1}{|\mathcal{B}_{i}^{(n+1)}|}\sum_{\bm{b} \in \mathcal{B}_{i}^{(n+1)}} \frac{1}{L_b}\lVert \bm{b}- C_i^{(n)}(\bm{b})\rVert^2_2
\end{equation}

The above equation states that quantizing the given block cluster $\mathcal{B}_i$ after updating the associated codebook from $C_i^{(n)}$ to $C_i^{(n+1)}$ results in lower quantization MSE. Since this is true for all the block clusters, we derive the following result: 
\begin{equation}
\begin{split}
\label{eq:mse_ineq_2}
     J^{(n+1)} &= \frac{1}{N_c} \sum_{i=1}^{N_c} \frac{1}{|\mathcal{B}_{i}^{(n+1)}|}\sum_{\bm{b} \in \mathcal{B}_{i}^{(n+1)}} \frac{1}{L_b}\lVert \bm{b}- C_i^{(n+1)}(\bm{b})\rVert^2_2  \le \tilde{J}^{(n+1)}   
\end{split}
\end{equation}

Following (\ref{eq:mse_ineq_1}) and (\ref{eq:mse_ineq_2}), we find that the quantization MSE is non-increasing for each iteration, that is, $J^{(1)} \ge J^{(2)} \ge J^{(3)} \ge \ldots \ge J^{(M)}$ where $M$ is the maximum number of iterations. 
%Therefore, we can say that if the algorithm converges, then it must be that it has converged to a local minimum. 
\hfill $\blacksquare$


\begin{figure}
    \begin{center}
    \includegraphics[width=0.5\textwidth]{sections//figures/mse_vs_iter.pdf}
    \end{center}
    \caption{\small NMSE vs iterations during LO-BCQ compared to other block quantization proposals}
    \label{fig:nmse_vs_iter}
\end{figure}

Figure \ref{fig:nmse_vs_iter} shows the empirical convergence of LO-BCQ across several block lengths and number of codebooks. Also, the MSE achieved by LO-BCQ is compared to baselines such as MXFP and VSQ. As shown, LO-BCQ converges to a lower MSE than the baselines. Further, we achieve better convergence for larger number of codebooks ($N_c$) and for a smaller block length ($L_b$), both of which increase the bitwidth of BCQ (see Eq \ref{eq:bitwidth_bcq}).


\subsection{Additional Accuracy Results}
%Table \ref{tab:lobcq_config} lists the various LOBCQ configurations and their corresponding bitwidths.
\begin{table}
\setlength{\tabcolsep}{4.75pt}
\begin{center}
\caption{\label{tab:lobcq_config} Various LO-BCQ configurations and their bitwidths.}
\begin{tabular}{|c||c|c|c|c||c|c||c|} 
\hline
 & \multicolumn{4}{|c||}{$L_b=8$} & \multicolumn{2}{|c||}{$L_b=4$} & $L_b=2$ \\
 \hline
 \backslashbox{$L_A$\kern-1em}{\kern-1em$N_c$} & 2 & 4 & 8 & 16 & 2 & 4 & 2 \\
 \hline
 64 & 4.25 & 4.375 & 4.5 & 4.625 & 4.375 & 4.625 & 4.625\\
 \hline
 32 & 4.375 & 4.5 & 4.625& 4.75 & 4.5 & 4.75 & 4.75 \\
 \hline
 16 & 4.625 & 4.75& 4.875 & 5 & 4.75 & 5 & 5 \\
 \hline
\end{tabular}
\end{center}
\end{table}

%\subsection{Perplexity achieved by various LO-BCQ configurations on Wikitext-103 dataset}

\begin{table} \centering
\begin{tabular}{|c||c|c|c|c||c|c||c|} 
\hline
 $L_b \rightarrow$& \multicolumn{4}{c||}{8} & \multicolumn{2}{c||}{4} & 2\\
 \hline
 \backslashbox{$L_A$\kern-1em}{\kern-1em$N_c$} & 2 & 4 & 8 & 16 & 2 & 4 & 2  \\
 %$N_c \rightarrow$ & 2 & 4 & 8 & 16 & 2 & 4 & 2 \\
 \hline
 \hline
 \multicolumn{8}{c}{GPT3-1.3B (FP32 PPL = 9.98)} \\ 
 \hline
 \hline
 64 & 10.40 & 10.23 & 10.17 & 10.15 &  10.28 & 10.18 & 10.19 \\
 \hline
 32 & 10.25 & 10.20 & 10.15 & 10.12 &  10.23 & 10.17 & 10.17 \\
 \hline
 16 & 10.22 & 10.16 & 10.10 & 10.09 &  10.21 & 10.14 & 10.16 \\
 \hline
  \hline
 \multicolumn{8}{c}{GPT3-8B (FP32 PPL = 7.38)} \\ 
 \hline
 \hline
 64 & 7.61 & 7.52 & 7.48 &  7.47 &  7.55 &  7.49 & 7.50 \\
 \hline
 32 & 7.52 & 7.50 & 7.46 &  7.45 &  7.52 &  7.48 & 7.48  \\
 \hline
 16 & 7.51 & 7.48 & 7.44 &  7.44 &  7.51 &  7.49 & 7.47  \\
 \hline
\end{tabular}
\caption{\label{tab:ppl_gpt3_abalation} Wikitext-103 perplexity across GPT3-1.3B and 8B models.}
\end{table}

\begin{table} \centering
\begin{tabular}{|c||c|c|c|c||} 
\hline
 $L_b \rightarrow$& \multicolumn{4}{c||}{8}\\
 \hline
 \backslashbox{$L_A$\kern-1em}{\kern-1em$N_c$} & 2 & 4 & 8 & 16 \\
 %$N_c \rightarrow$ & 2 & 4 & 8 & 16 & 2 & 4 & 2 \\
 \hline
 \hline
 \multicolumn{5}{|c|}{Llama2-7B (FP32 PPL = 5.06)} \\ 
 \hline
 \hline
 64 & 5.31 & 5.26 & 5.19 & 5.18  \\
 \hline
 32 & 5.23 & 5.25 & 5.18 & 5.15  \\
 \hline
 16 & 5.23 & 5.19 & 5.16 & 5.14  \\
 \hline
 \multicolumn{5}{|c|}{Nemotron4-15B (FP32 PPL = 5.87)} \\ 
 \hline
 \hline
 64  & 6.3 & 6.20 & 6.13 & 6.08  \\
 \hline
 32  & 6.24 & 6.12 & 6.07 & 6.03  \\
 \hline
 16  & 6.12 & 6.14 & 6.04 & 6.02  \\
 \hline
 \multicolumn{5}{|c|}{Nemotron4-340B (FP32 PPL = 3.48)} \\ 
 \hline
 \hline
 64 & 3.67 & 3.62 & 3.60 & 3.59 \\
 \hline
 32 & 3.63 & 3.61 & 3.59 & 3.56 \\
 \hline
 16 & 3.61 & 3.58 & 3.57 & 3.55 \\
 \hline
\end{tabular}
\caption{\label{tab:ppl_llama7B_nemo15B} Wikitext-103 perplexity compared to FP32 baseline in Llama2-7B and Nemotron4-15B, 340B models}
\end{table}

%\subsection{Perplexity achieved by various LO-BCQ configurations on MMLU dataset}


\begin{table} \centering
\begin{tabular}{|c||c|c|c|c||c|c|c|c|} 
\hline
 $L_b \rightarrow$& \multicolumn{4}{c||}{8} & \multicolumn{4}{c||}{8}\\
 \hline
 \backslashbox{$L_A$\kern-1em}{\kern-1em$N_c$} & 2 & 4 & 8 & 16 & 2 & 4 & 8 & 16  \\
 %$N_c \rightarrow$ & 2 & 4 & 8 & 16 & 2 & 4 & 2 \\
 \hline
 \hline
 \multicolumn{5}{|c|}{Llama2-7B (FP32 Accuracy = 45.8\%)} & \multicolumn{4}{|c|}{Llama2-70B (FP32 Accuracy = 69.12\%)} \\ 
 \hline
 \hline
 64 & 43.9 & 43.4 & 43.9 & 44.9 & 68.07 & 68.27 & 68.17 & 68.75 \\
 \hline
 32 & 44.5 & 43.8 & 44.9 & 44.5 & 68.37 & 68.51 & 68.35 & 68.27  \\
 \hline
 16 & 43.9 & 42.7 & 44.9 & 45 & 68.12 & 68.77 & 68.31 & 68.59  \\
 \hline
 \hline
 \multicolumn{5}{|c|}{GPT3-22B (FP32 Accuracy = 38.75\%)} & \multicolumn{4}{|c|}{Nemotron4-15B (FP32 Accuracy = 64.3\%)} \\ 
 \hline
 \hline
 64 & 36.71 & 38.85 & 38.13 & 38.92 & 63.17 & 62.36 & 63.72 & 64.09 \\
 \hline
 32 & 37.95 & 38.69 & 39.45 & 38.34 & 64.05 & 62.30 & 63.8 & 64.33  \\
 \hline
 16 & 38.88 & 38.80 & 38.31 & 38.92 & 63.22 & 63.51 & 63.93 & 64.43  \\
 \hline
\end{tabular}
\caption{\label{tab:mmlu_abalation} Accuracy on MMLU dataset across GPT3-22B, Llama2-7B, 70B and Nemotron4-15B models.}
\end{table}


%\subsection{Perplexity achieved by various LO-BCQ configurations on LM evaluation harness}

\begin{table} \centering
\begin{tabular}{|c||c|c|c|c||c|c|c|c|} 
\hline
 $L_b \rightarrow$& \multicolumn{4}{c||}{8} & \multicolumn{4}{c||}{8}\\
 \hline
 \backslashbox{$L_A$\kern-1em}{\kern-1em$N_c$} & 2 & 4 & 8 & 16 & 2 & 4 & 8 & 16  \\
 %$N_c \rightarrow$ & 2 & 4 & 8 & 16 & 2 & 4 & 2 \\
 \hline
 \hline
 \multicolumn{5}{|c|}{Race (FP32 Accuracy = 37.51\%)} & \multicolumn{4}{|c|}{Boolq (FP32 Accuracy = 64.62\%)} \\ 
 \hline
 \hline
 64 & 36.94 & 37.13 & 36.27 & 37.13 & 63.73 & 62.26 & 63.49 & 63.36 \\
 \hline
 32 & 37.03 & 36.36 & 36.08 & 37.03 & 62.54 & 63.51 & 63.49 & 63.55  \\
 \hline
 16 & 37.03 & 37.03 & 36.46 & 37.03 & 61.1 & 63.79 & 63.58 & 63.33  \\
 \hline
 \hline
 \multicolumn{5}{|c|}{Winogrande (FP32 Accuracy = 58.01\%)} & \multicolumn{4}{|c|}{Piqa (FP32 Accuracy = 74.21\%)} \\ 
 \hline
 \hline
 64 & 58.17 & 57.22 & 57.85 & 58.33 & 73.01 & 73.07 & 73.07 & 72.80 \\
 \hline
 32 & 59.12 & 58.09 & 57.85 & 58.41 & 73.01 & 73.94 & 72.74 & 73.18  \\
 \hline
 16 & 57.93 & 58.88 & 57.93 & 58.56 & 73.94 & 72.80 & 73.01 & 73.94  \\
 \hline
\end{tabular}
\caption{\label{tab:mmlu_abalation} Accuracy on LM evaluation harness tasks on GPT3-1.3B model.}
\end{table}

\begin{table} \centering
\begin{tabular}{|c||c|c|c|c||c|c|c|c|} 
\hline
 $L_b \rightarrow$& \multicolumn{4}{c||}{8} & \multicolumn{4}{c||}{8}\\
 \hline
 \backslashbox{$L_A$\kern-1em}{\kern-1em$N_c$} & 2 & 4 & 8 & 16 & 2 & 4 & 8 & 16  \\
 %$N_c \rightarrow$ & 2 & 4 & 8 & 16 & 2 & 4 & 2 \\
 \hline
 \hline
 \multicolumn{5}{|c|}{Race (FP32 Accuracy = 41.34\%)} & \multicolumn{4}{|c|}{Boolq (FP32 Accuracy = 68.32\%)} \\ 
 \hline
 \hline
 64 & 40.48 & 40.10 & 39.43 & 39.90 & 69.20 & 68.41 & 69.45 & 68.56 \\
 \hline
 32 & 39.52 & 39.52 & 40.77 & 39.62 & 68.32 & 67.43 & 68.17 & 69.30  \\
 \hline
 16 & 39.81 & 39.71 & 39.90 & 40.38 & 68.10 & 66.33 & 69.51 & 69.42  \\
 \hline
 \hline
 \multicolumn{5}{|c|}{Winogrande (FP32 Accuracy = 67.88\%)} & \multicolumn{4}{|c|}{Piqa (FP32 Accuracy = 78.78\%)} \\ 
 \hline
 \hline
 64 & 66.85 & 66.61 & 67.72 & 67.88 & 77.31 & 77.42 & 77.75 & 77.64 \\
 \hline
 32 & 67.25 & 67.72 & 67.72 & 67.00 & 77.31 & 77.04 & 77.80 & 77.37  \\
 \hline
 16 & 68.11 & 68.90 & 67.88 & 67.48 & 77.37 & 78.13 & 78.13 & 77.69  \\
 \hline
\end{tabular}
\caption{\label{tab:mmlu_abalation} Accuracy on LM evaluation harness tasks on GPT3-8B model.}
\end{table}

\begin{table} \centering
\begin{tabular}{|c||c|c|c|c||c|c|c|c|} 
\hline
 $L_b \rightarrow$& \multicolumn{4}{c||}{8} & \multicolumn{4}{c||}{8}\\
 \hline
 \backslashbox{$L_A$\kern-1em}{\kern-1em$N_c$} & 2 & 4 & 8 & 16 & 2 & 4 & 8 & 16  \\
 %$N_c \rightarrow$ & 2 & 4 & 8 & 16 & 2 & 4 & 2 \\
 \hline
 \hline
 \multicolumn{5}{|c|}{Race (FP32 Accuracy = 40.67\%)} & \multicolumn{4}{|c|}{Boolq (FP32 Accuracy = 76.54\%)} \\ 
 \hline
 \hline
 64 & 40.48 & 40.10 & 39.43 & 39.90 & 75.41 & 75.11 & 77.09 & 75.66 \\
 \hline
 32 & 39.52 & 39.52 & 40.77 & 39.62 & 76.02 & 76.02 & 75.96 & 75.35  \\
 \hline
 16 & 39.81 & 39.71 & 39.90 & 40.38 & 75.05 & 73.82 & 75.72 & 76.09  \\
 \hline
 \hline
 \multicolumn{5}{|c|}{Winogrande (FP32 Accuracy = 70.64\%)} & \multicolumn{4}{|c|}{Piqa (FP32 Accuracy = 79.16\%)} \\ 
 \hline
 \hline
 64 & 69.14 & 70.17 & 70.17 & 70.56 & 78.24 & 79.00 & 78.62 & 78.73 \\
 \hline
 32 & 70.96 & 69.69 & 71.27 & 69.30 & 78.56 & 79.49 & 79.16 & 78.89  \\
 \hline
 16 & 71.03 & 69.53 & 69.69 & 70.40 & 78.13 & 79.16 & 79.00 & 79.00  \\
 \hline
\end{tabular}
\caption{\label{tab:mmlu_abalation} Accuracy on LM evaluation harness tasks on GPT3-22B model.}
\end{table}

\begin{table} \centering
\begin{tabular}{|c||c|c|c|c||c|c|c|c|} 
\hline
 $L_b \rightarrow$& \multicolumn{4}{c||}{8} & \multicolumn{4}{c||}{8}\\
 \hline
 \backslashbox{$L_A$\kern-1em}{\kern-1em$N_c$} & 2 & 4 & 8 & 16 & 2 & 4 & 8 & 16  \\
 %$N_c \rightarrow$ & 2 & 4 & 8 & 16 & 2 & 4 & 2 \\
 \hline
 \hline
 \multicolumn{5}{|c|}{Race (FP32 Accuracy = 44.4\%)} & \multicolumn{4}{|c|}{Boolq (FP32 Accuracy = 79.29\%)} \\ 
 \hline
 \hline
 64 & 42.49 & 42.51 & 42.58 & 43.45 & 77.58 & 77.37 & 77.43 & 78.1 \\
 \hline
 32 & 43.35 & 42.49 & 43.64 & 43.73 & 77.86 & 75.32 & 77.28 & 77.86  \\
 \hline
 16 & 44.21 & 44.21 & 43.64 & 42.97 & 78.65 & 77 & 76.94 & 77.98  \\
 \hline
 \hline
 \multicolumn{5}{|c|}{Winogrande (FP32 Accuracy = 69.38\%)} & \multicolumn{4}{|c|}{Piqa (FP32 Accuracy = 78.07\%)} \\ 
 \hline
 \hline
 64 & 68.9 & 68.43 & 69.77 & 68.19 & 77.09 & 76.82 & 77.09 & 77.86 \\
 \hline
 32 & 69.38 & 68.51 & 68.82 & 68.90 & 78.07 & 76.71 & 78.07 & 77.86  \\
 \hline
 16 & 69.53 & 67.09 & 69.38 & 68.90 & 77.37 & 77.8 & 77.91 & 77.69  \\
 \hline
\end{tabular}
\caption{\label{tab:mmlu_abalation} Accuracy on LM evaluation harness tasks on Llama2-7B model.}
\end{table}

\begin{table} \centering
\begin{tabular}{|c||c|c|c|c||c|c|c|c|} 
\hline
 $L_b \rightarrow$& \multicolumn{4}{c||}{8} & \multicolumn{4}{c||}{8}\\
 \hline
 \backslashbox{$L_A$\kern-1em}{\kern-1em$N_c$} & 2 & 4 & 8 & 16 & 2 & 4 & 8 & 16  \\
 %$N_c \rightarrow$ & 2 & 4 & 8 & 16 & 2 & 4 & 2 \\
 \hline
 \hline
 \multicolumn{5}{|c|}{Race (FP32 Accuracy = 48.8\%)} & \multicolumn{4}{|c|}{Boolq (FP32 Accuracy = 85.23\%)} \\ 
 \hline
 \hline
 64 & 49.00 & 49.00 & 49.28 & 48.71 & 82.82 & 84.28 & 84.03 & 84.25 \\
 \hline
 32 & 49.57 & 48.52 & 48.33 & 49.28 & 83.85 & 84.46 & 84.31 & 84.93  \\
 \hline
 16 & 49.85 & 49.09 & 49.28 & 48.99 & 85.11 & 84.46 & 84.61 & 83.94  \\
 \hline
 \hline
 \multicolumn{5}{|c|}{Winogrande (FP32 Accuracy = 79.95\%)} & \multicolumn{4}{|c|}{Piqa (FP32 Accuracy = 81.56\%)} \\ 
 \hline
 \hline
 64 & 78.77 & 78.45 & 78.37 & 79.16 & 81.45 & 80.69 & 81.45 & 81.5 \\
 \hline
 32 & 78.45 & 79.01 & 78.69 & 80.66 & 81.56 & 80.58 & 81.18 & 81.34  \\
 \hline
 16 & 79.95 & 79.56 & 79.79 & 79.72 & 81.28 & 81.66 & 81.28 & 80.96  \\
 \hline
\end{tabular}
\caption{\label{tab:mmlu_abalation} Accuracy on LM evaluation harness tasks on Llama2-70B model.}
\end{table}

%\section{MSE Studies}
%\textcolor{red}{TODO}


\subsection{Number Formats and Quantization Method}
\label{subsec:numFormats_quantMethod}
\subsubsection{Integer Format}
An $n$-bit signed integer (INT) is typically represented with a 2s-complement format \citep{yao2022zeroquant,xiao2023smoothquant,dai2021vsq}, where the most significant bit denotes the sign.

\subsubsection{Floating Point Format}
An $n$-bit signed floating point (FP) number $x$ comprises of a 1-bit sign ($x_{\mathrm{sign}}$), $B_m$-bit mantissa ($x_{\mathrm{mant}}$) and $B_e$-bit exponent ($x_{\mathrm{exp}}$) such that $B_m+B_e=n-1$. The associated constant exponent bias ($E_{\mathrm{bias}}$) is computed as $(2^{{B_e}-1}-1)$. We denote this format as $E_{B_e}M_{B_m}$.  

\subsubsection{Quantization Scheme}
\label{subsec:quant_method}
A quantization scheme dictates how a given unquantized tensor is converted to its quantized representation. We consider FP formats for the purpose of illustration. Given an unquantized tensor $\bm{X}$ and an FP format $E_{B_e}M_{B_m}$, we first, we compute the quantization scale factor $s_X$ that maps the maximum absolute value of $\bm{X}$ to the maximum quantization level of the $E_{B_e}M_{B_m}$ format as follows:
\begin{align}
\label{eq:sf}
    s_X = \frac{\mathrm{max}(|\bm{X}|)}{\mathrm{max}(E_{B_e}M_{B_m})}
\end{align}
In the above equation, $|\cdot|$ denotes the absolute value function.

Next, we scale $\bm{X}$ by $s_X$ and quantize it to $\hat{\bm{X}}$ by rounding it to the nearest quantization level of $E_{B_e}M_{B_m}$ as:

\begin{align}
\label{eq:tensor_quant}
    \hat{\bm{X}} = \text{round-to-nearest}\left(\frac{\bm{X}}{s_X}, E_{B_e}M_{B_m}\right)
\end{align}

We perform dynamic max-scaled quantization \citep{wu2020integer}, where the scale factor $s$ for activations is dynamically computed during runtime.

\subsection{Vector Scaled Quantization}
\begin{wrapfigure}{r}{0.35\linewidth}
  \centering
  \includegraphics[width=\linewidth]{sections/figures/vsquant.jpg}
  \caption{\small Vectorwise decomposition for per-vector scaled quantization (VSQ \citep{dai2021vsq}).}
  \label{fig:vsquant}
\end{wrapfigure}
During VSQ \citep{dai2021vsq}, the operand tensors are decomposed into 1D vectors in a hardware friendly manner as shown in Figure \ref{fig:vsquant}. Since the decomposed tensors are used as operands in matrix multiplications during inference, it is beneficial to perform this decomposition along the reduction dimension of the multiplication. The vectorwise quantization is performed similar to tensorwise quantization described in Equations \ref{eq:sf} and \ref{eq:tensor_quant}, where a scale factor $s_v$ is required for each vector $\bm{v}$ that maps the maximum absolute value of that vector to the maximum quantization level. While smaller vector lengths can lead to larger accuracy gains, the associated memory and computational overheads due to the per-vector scale factors increases. To alleviate these overheads, VSQ \citep{dai2021vsq} proposed a second level quantization of the per-vector scale factors to unsigned integers, while MX \citep{rouhani2023shared} quantizes them to integer powers of 2 (denoted as $2^{INT}$).

\subsubsection{MX Format}
The MX format proposed in \citep{rouhani2023microscaling} introduces the concept of sub-block shifting. For every two scalar elements of $b$-bits each, there is a shared exponent bit. The value of this exponent bit is determined through an empirical analysis that targets minimizing quantization MSE. We note that the FP format $E_{1}M_{b}$ is strictly better than MX from an accuracy perspective since it allocates a dedicated exponent bit to each scalar as opposed to sharing it across two scalars. Therefore, we conservatively bound the accuracy of a $b+2$-bit signed MX format with that of a $E_{1}M_{b}$ format in our comparisons. For instance, we use E1M2 format as a proxy for MX4.

\begin{figure}
    \centering
    \includegraphics[width=1\linewidth]{sections//figures/BlockFormats.pdf}
    \caption{\small Comparing LO-BCQ to MX format.}
    \label{fig:block_formats}
\end{figure}

Figure \ref{fig:block_formats} compares our $4$-bit LO-BCQ block format to MX \citep{rouhani2023microscaling}. As shown, both LO-BCQ and MX decompose a given operand tensor into block arrays and each block array into blocks. Similar to MX, we find that per-block quantization ($L_b < L_A$) leads to better accuracy due to increased flexibility. While MX achieves this through per-block $1$-bit micro-scales, we associate a dedicated codebook to each block through a per-block codebook selector. Further, MX quantizes the per-block array scale-factor to E8M0 format without per-tensor scaling. In contrast during LO-BCQ, we find that per-tensor scaling combined with quantization of per-block array scale-factor to E4M3 format results in superior inference accuracy across models. 


\end{document}
