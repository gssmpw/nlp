

\section{Conclusion}
\label{sec:conclusion}

In this work, we conducted a large-scale study on traces modeling building upon \citet{scratchpad} and \citet{ni2024nextteachinglargelanguage}. Reflecting back on our questions, (1) we can scale trace modeling up with \name{}, by tracing executions on automatically generated inputs and thus generating large training datasets, which generalizes to output prediction benchmarks. (2) A more fine-grained granularity can't help when the core issue can't be further broken down (indexing on CruxEval) but shows promise otherwise. Regarding scratchpad strategies (3) and execution lengths (4), our newly introduced dynamic scratchpad excels at very long executions, while compact scratchpad generally outperforms the original scratchpad. We saw no conclusive improvements on downstream coding benchmarks (5), where program state understanding might not be critical. % to solve the tasks.
As future work, we suggest extending our work to other languages such as C, pointer ids to understand phenomena such as aliasing, and closures. We are also keen on more challenging datasets with exceptions and dynamic trace granularities.

