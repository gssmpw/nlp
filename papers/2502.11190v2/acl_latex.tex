% This must be in the first 5 lines to tell arXiv to use pdfLaTeX, which is strongly recommended.
\pdfoutput=1
% In particular, the hyperref package requires pdfLaTeX in order to break URLs across lines.

\documentclass[11pt]{article}

% Change "review" to "final" to generate the final (sometimes called camera-ready) version.
% Change to "preprint" to generate a non-anonymous version with page numbers.
\usepackage[final]{acl}

% Standard package includes
\usepackage{times}
\usepackage{latexsym}

% For proper rendering and hyphenation of words containing Latin characters (including in bib files)
\usepackage[T1]{fontenc}
% For Vietnamese characters
% \usepackage[T5]{fontenc}
% See https://www.latex-project.org/help/documentation/encguide.pdf for other character sets

% This assumes your files are encoded as UTF8
\usepackage[utf8]{inputenc}

% This is not strictly necessary, and may be commented out,
% but it will improve the layout of the manuscript,
% and will typically save some space.
\usepackage{microtype}

% This is also not strictly necessary, and may be commented out.
% However, it will improve the aesthetics of text in
% the typewriter font.
\usepackage{inconsolata}

%Including images in your LaTeX document requires adding
%additional package(s)
\usepackage{graphicx}
\usepackage{multirow}
\usepackage{amsmath}
\usepackage{amssymb}
\usepackage{wrapfig}   % 用于实现文字环绕图片
\usepackage{float}
\usepackage{algorithm}
\usepackage{algorithmic}
\usepackage{mdframed}
\usepackage{tcolorbox}
\usepackage{xcolor}
\usepackage{subcaption}
\usepackage{listings}
\usepackage{lipsum}
\usepackage{array}
\usepackage{chngcntr}
\usepackage{ragged2e}
\tcbuselibrary{breakable}
\usepackage{booktabs}
\usepackage{makecell}
\usepackage{caption}
\usepackage{colortbl}
\lstset{
  basicstyle=\ttfamily\footnotesize,
  breaklines=true,
  frame=single,
}

% Define custom column types for better alignment

% \usepackage{colortbl}    % For cell colors
% \usepackage{array}       % For column formatting
% \usepackage{xcolor}
% \newcolumntype{L}[1]{>{\raggedright\arraybackslash}p{#1}}
% \newcolumntype{C}[1]{>{\centering\arraybackslash}p{#1}}

% If the title and author information does not fit in the area allocated, uncomment the following
%
%\setlength\titlebox{<dim>}
%
% and set <dim> to something 5cm or larger.

\title{ReLearn: Unlearning via Learning  for Large Language Models}

% Author information can be set in various styles:
% For several authors from the same institution:
% \author{Author 1 \and ... \and Author n \\
%         Address line \\ ... \\ Address line}
% if the names do not fit well on one line use
%         Author 1 \\ {\bf Author 2} \\ ... \\ {\bf Author n} \\
% For authors from different institutions:
% \author{Author 1 \\ Address line \\  ... \\ Address line
%         \And  ... \And
%         Author n \\ Address line \\ ... \\ Address line}
% To start a separate ``row'' of authors use \AND, as in
% \author{Author 1 \\ Address line \\  ... \\ Address line
%         \AND
%         Author 2 \\ Address line \\ ... \\ Address line \And
%         Author 3 \\ Address line \\ ... \\ Address line}

% \author{First Author \\
%   Affiliation / Address line 1 \\
%   Affiliation / Address line 2 \\
%   Affiliation / Address line 3 \\
%   \texttt{email@domain} \\\And
%   Second Author \\
%   Affiliation / Address line 1 \\
%   Affiliation / Address line 2 \\
%   Affiliation / Address line 3 \\
%   \texttt{email@domain} \\}
\author{
    Haoming Xu\textsuperscript{\rm 1 \thanks{Equal contribution and shared co-first authorship.}},  
    Ningyuan Zhao\textsuperscript{\rm 2 \footnotemark[1]},  
    Liming Yang\textsuperscript{\rm 3},  \\
    \textbf{Sendong Zhao}\textsuperscript{\rm 4},  
    \textbf{Shumin Deng}\textsuperscript{\rm 5},  
    \textbf{Mengru Wang}\textsuperscript{\rm 1}, \\
    \textbf{Bryan Hooi}\textsuperscript{\rm 5}, 
    \textbf{Nay Oo}\textsuperscript{\rm 5},
    \textbf{Huajun Chen}\textsuperscript{\rm 1 \thanks{Corresponding author.}},  
    \textbf{Ningyu Zhang}\textsuperscript{\rm 1 \dag}  
    \\  
    \textsuperscript{\rm 1} Zhejiang University \quad
    \textsuperscript{\rm 2} Xiamen University \quad
    \textsuperscript{\rm 3} Tsinghua University \quad \\
    \textsuperscript{\rm 4} Harbin Institude of Technology \quad
    \textsuperscript{\rm 5} National University of Singapore\\  
    \texttt{\{haomingxu2003, nyzhao2001, uriazdrucker\}@gmail.com} \\  
    \texttt{\{huajunsir, zhangningyu\}@zju.edu.cn}  
}

%\author{
%  \textbf{First Author\textsuperscript{1}},
%  \textbf{Second Author\textsuperscript{1,2}},
%  \textbf{Third T. Author\textsuperscript{1}},
%  \textbf{Fourth Author\textsuperscript{1}},
%\\
%  \textbf{Fifth Author\textsuperscript{1,2}},
%  \textbf{Sixth Author\textsuperscript{1}},
%  \textbf{Seventh Author\textsuperscript{1}},
%  \textbf{Eighth Author \textsuperscript{1,2,3,4}},
%\\
%  \textbf{Ninth Author\textsuperscript{1}},
%  \textbf{Tenth Author\textsuperscript{1}},
%  \textbf{Eleventh E. Author\textsuperscript{1,2,3,4,5}},
%  \textbf{Twelfth Author\textsuperscript{1}},
%\\
%  \textbf{Thirteenth Author\textsuperscript{3}},
%  \textbf{Fourteenth F. Author\textsuperscript{2,4}},
%  \textbf{Fifteenth Author\textsuperscript{1}},
%  \textbf{Sixteenth Author\textsuperscript{1}},
%\\
%  \textbf{Seventeenth S. Author\textsuperscript{4,5}},
%  \textbf{Eighteenth Author\textsuperscript{3,4}},
%  \textbf{Nineteenth N. Author\textsuperscript{2,5}},
%  \textbf{Twentieth Author\textsuperscript{1}}
%\\
%\\
%  \textsuperscript{1}Affiliation 1,
%  \textsuperscript{2}Affiliation 2,
%  \textsuperscript{3}Affiliation 3,
%  \textsuperscript{4}Affiliation 4,
%  \textsuperscript{5}Affiliation 5
%\\
%  \small{
%    \textbf{Correspondence:} \href{mailto:email@domain}{email@domain}
%  }
%}

\begin{document}
\maketitle
\begin{abstract}
Current unlearning methods for large language models usually rely on reverse optimization to reduce target token probabilities. However, this paradigm disrupts the subsequent tokens prediction, degrading model performance and linguistic coherence. Moreover, existing evaluation metrics overemphasize contextual forgetting while inadequately assessing response fluency and relevance. To address these challenges, we propose \textbf{ReLearn}, a data augmentation and fine-tuning pipeline for effective unlearning, along with a comprehensive evaluation framework. This framework introduces Knowledge Forgetting Rate (KFR) and Knowledge Retention Rate (KRR) to measure knowledge-level preservation, and Linguistic Score (LS) to evaluate generation quality. Our experiments show that ReLearn successfully achieves targeted forgetting while preserving high-quality output. Through mechanistic analysis, we further demonstrate how reverse optimization disrupts coherent text generation, while ReLearn preserves this essential capability\footnote{Code is available at \url{https://github.com/zjunlp/unlearn}.}.
\vspace{-1ex}
\begin{center}
  \textit{``The illiterate of the future are not those who can’t read or write but those who cannot learn, unlearn, and relearn.''} — Alvin Toffler
\end{center}
\end{abstract}

\section{Introduction}
% Large Language Models~(LLMs) represent a transformative advancement in the field of language processing, demonstrating an unparalleled capacity for text generation and comprehension, which can be further applied in a wide variety of applications.  
% %Large language models (LLMs) have risen to prominence in various fields, offering endless possibilities for artificial intelligence applications. 
% Despite their significant prevalence in recent years, LLMs are frequently challenged with security and privacy issues, such as poor explainability~\cite{}, poor robustness~\cite{}, data dependency~\cite{}, etc. Among them, a specific and notable concern that has garnered increasing attention is the phenomenon of `hallucination', where models generate plausible but factually inaccurate or irrelevant content when employed for specific tasks such as problem-solving.  
% %In particular, the hallucination issue is when these large models are employed for problem-solving, users frequently voice concerns regarding being misled or deceived by the models' nonsensical and erratic outputs. 
% The tendency of these models to produce inaccurate outputs and fabricate facts has severely undermined the safety and usability of LLM applications, which calls for immediate attention in LLM research. 
% %Hallucination in large language models (LLMs) is a critical issue that needs immediate attention in LLM research. The tendency of these models to produce inaccurate outputs and fabricate facts has severely undermined the safety and usability of LLM applications. 
%exceptional 
%including limited explainability, compromised robustness, and a heavy reliance on data, each 
%However, d
Large Language Models (LLMs) have revolutionized language processing, demonstrating impressive text generation and comprehension capabilities with diverse applications. However, despite their growing use, LLMs face significant security and privacy challenges~\cite{siddiq2023generate, hou2023large, kaddour2023challenges, li2024model, 10.1145/3691620.3695510}, which affect their overall effectiveness and reliability. A critical issue is the phenomenon of \emph{hallucination}, where LLMs generate outputs that are coherent but factually incorrect or irrelevant. This tendency to produce misleading information compromises the safety and usability of LLM-based systems. This paper focuses on \emph{fact-conflicting hallucina}tion (FCH), the most prominent form of hallucination in LLMs. FCH occurs when LLMs generate content that directly contradicts established facts. For instance, as illustrated in \figref{fig:example1}, an LLM incorrectly asserts that ``\emph{Haruki Murakami won the Nobel Prize in Literature in 2016}'', whereas the fact is that ``\emph{Haruki Murakami has not won the Nobel Prize, though he has received numerous other literary awards}''. 
Such inaccuracies can significantly lead to user confusion and undermine the trust and reliability that are crucial for LLM applications.

% Large Language Models~(LLMs) have brought transformative advancements to language processing and beyond, showcasing text generation and comprehension abilities with wide-ranging applications. 
% Despite the increasing prevalence, LLMs face critical challenges in security and privacy aspects~\cite{siddiq2023generate, hou2023large, kaddour2023challenges}, heavily impacting their effectiveness and reliability. 
% One notable issue is the phenomenon of \emph{hallucination}, where LLMs produce coherent but factually inaccurate or irrelevant outputs during problem-solving. 
% Such a tendency to generate misleading information jeopardizes the safety and usability of LLM-based applications. 
% This paper concerns the \emph{fact-conflicting hallucination}~(FCH), which is the primary form of hallucinations in LLMs. 
% FCH occurs when LLMs generate content that directly contradicts the well-established facts, as exemplified in \figref{fig:example1}, where an LLM incorrectly believes 
% ``\emph{Haruki Murakami won the Nobel Prize in Literature in 2016}'', deviating from the fact that ``\emph{Haruki Murakami has not won the Nobel Prize but other numerous awards for his work in Literature}''. Such misinformation can cause significant user confusion and undermine the trust and reliability that are essential in various LLM applications. 

%correct answer of 

%is manifested by
%Such misinformation dissemination leads to significant user confusion, eroding the trust and reliability that are crucial in various LLM applications. 

%Large Language Models~(LLMs) represent a transformative advancement in the field of language processing, demonstrating an unparalleled capacity for text generation and comprehension, which can be further applied in a wide variety of applications. Despite their growing prevalence, LLMs encounter critical challenges, particularly in aspects of security and privacy. These include concerns such as limited explainability~\cite{}, compromised robustness~\cite{}, and heavy reliance on data~\cite{}, each posing distinct challenges to their efficacy and reliability. Among these, the phenomenon of ``hallucination'' stands out as a notable concern. This occurs when LLMs, while employed in tasks like problem-solving, generate outputs that are coherent yet factually inaccurate or irrelevant. Such a tendency to produce misleading information not only compromises the safety of LLM applications but also raises urgent questions regarding their usability. 

% Hallucinations in LLMs manifest in several distinct forms, each contributing differently to the challenges identified in their growing applications. 
% %The first, known as ``Input-conflicting hallucination'', arises when there is a discrepancy between the model's output and the user's initial input, reflecting a potential misunderstanding of the task at hand. On the other hand, ``Context-conflicting hallucination'' represents the second type, occurring when LLMs produce inconsistent responses in prolonged or multi-turn interactions, indicative of their limitations in maintaining coherent context. 
% Among the three types categorized in the literature~\cite{huang2023survey,zhang2023hallucination}, ``Fact-conflicting hallucination~(FCH)'' poses a particularly serious concern which is the primary focus of this paper. This phenomenon generates content in direct opposition to established factual knowledge. As illustrated in the example shown in Figure~\ref{fig:example1}, when an LLM was asked about the first person to walk on the moon, it incorrectly answered ``Charles Lindbergh in 1951'', a clear deviation from the factual answer of Neil Armstrong in 1969. This type of hallucination can lead to the dissemination of incorrect information and cause significant confusion among users, undermining the trust and reliability critical in various LLM applications. %Addressing fact-conflicting hallucinations is therefore essential for the advancement of LLMs, ensuring they not only function effectively but also responsibly in their diverse roles.


% According to \cite{huang2023survey} and \cite{zhang2023hallucination}, hallucinations in large language models can be categorized into types such as factual hallucinations and contextual hallucinations. Current benchmark assessments tend to focus on evaluating the propensity of LLMs to generate erroneous facts. The origin of these issues can be traced back to multiple deficiencies, including flaws in training data, training algorithms, and the inference process.

% \begin{figure}[t]
%     \centering
%     \includegraphics[width=0.95\linewidth]{fig/example1-cropped.pdf}\\
%     \caption{A Hallucination Output Example.}
%     %\vspace{-0.5cm}
%     \label{fig:example1}
% \end{figure}

\begin{figure}[t]
\centering
\vspace{3mm}
\hspace{-3mm}
\includegraphics[width=\linewidth]{fig/drowzee-example.pdf}
\\[0.5em]
\caption{A Hallucination Output Example}
\label{fig:example1}
\vspace{-4mm}
\end{figure}
%\lnk{Factual Hallucination and LLM inference current status}

Recent studies have introduced various methods to detect LLM hallucinations. A common approach involves developing specialized benchmarks, such as TruthfulQA~\cite{lin-etal-2022-truthfulqa}, HaluEval~\cite{HaluEval}, and KoLA~\cite{yu2023kola}, to assess hallucinations in tasks like question-answering, summarization, and knowledge graphs. 
While manually labeled datasets provide valuable insights, current methods often rely on simplistic or semi-automated techniques such as string matching, manual validation, or verification through another language model. These approaches reveal significant gaps in automatically and effectively detecting fact-conflicting hallucinations (FCH). 
The primary challenges in FCH detection arise from the lack of dedicated ground truth datasets, the absence of comprehensive test cases designed to trigger FCH, and the lack of a robust testing framework.  
Unlike other types of hallucinations, such as input-conflicting or context-conflicting hallucinations~\cite{ji-etal-2023-rho, shi2023large}, which can often be identified through semantic consistency checks, detecting FCH requires the verification of factual accuracy against external knowledge sources/databases. This process is particularly challenging and resource-intensive, especially for tasks that involve complex logical relationships~\cite{zhang2024fusion}. We identify three primary challenges in addressing this research gap:


% Recent studies have introduced a range of methods for detecting 
% hallucinations. One common approach involves creating comprehensive benchmarks tailored for this purpose. 
% Datasets such as TruthfulQA~\cite{lin-etal-2022-truthfulqa}, HaluEval~\cite{HaluEval}, and KoLA~\cite{yu2023kola} have been designed to evaluate hallucinations across different contexts, including question-answering, summarization, and knowledge graphs. 
% Despite the value of these manually labeled datasets, the current techniques heavily rely on naive and semi-automatic methods, such as string matching, manual validation, or utilizing another LLM for confirmation. 
% Therefore, there is a gap 
% in automatically and effectively testing FCHs, and the primary obstacle in testing FCH is the absence of dedicated ground truth datasets and an extensive testing framework.  
% Unlike other types of hallucinations, e.g., input-conflicting or context-conflicting 
% \cite{ji-etal-2023-rho, shi2023large}, 
% which can be identified through checks for semantic consistency, 
% detecting FCH
% requires the verification of the content's factual accuracy against external sources of knowledge or databases. This makes the process particularly arduous and resource-intensive, especially for tasks processing content with complex logical connections. 
% Here, we highlight three concrete challenges in filling up the identified research gap: 




%The main obstacle in testing for FCH is the absence of dedicated ground truth datasets and specific testing frameworks. Unlike other types of hallucinations~(e.g., input-conflicting and context-conflicting hallucinations, to be detailed in Section~\ref{subsec:cat}) which can be identified through checks for semantic consistency, FCH demands the verification of the content's factual accuracy against external sources of knowledge or databases. This requirement makes the process particularly challenging and resource-intensive, especially for tasks processing contents with inherent logical connections.

% \shil{(I feel the transition is not smooth, we first introducing datasets, and not explaining how they use these datasets to test llm. after these, we can state these methods are not automatic.)}


% To tackle FCH, recent works have developed various techniques for testing and detecting hallucination~\citep{yu2023kola,HaluEval}. The typical and intuitive solution is to develop comprehensive benchmarks for detection. This is done through a process of sampling, filtering, and enhancing ground-truth answers to identify the best and correct answers from given candidates. For example, a well-known hallucination evaluation benchmark HaluEval~\cite{HaluEval} constructs scenarios where LLMs are tested on their ability to select the most factually accurate answers from a set of provided options, with a focus on filtering out hallucinated responses. %\yi{ also talk about the construction of benchmark?}
% Additionally, human annotation plays a critical role in identifying hallucinations in LLM outputs~\cite{Alpaca}. This involves humans determining whether responses contain hallucinated information and considering aspects such as unverifiability, non-factuality, and irrelevance. 



% \lnk{Key challenge: lack of hallucination testing when faced with logic reasoning related problems}
%Bridging the identified research gap in the literature necessitates exploring the inherent challenges faced in detecting FCHs, which are crucial for advancing and enhancing the reliability of LLMs. 

\begin{enumerate}[itemsep=1mm, wide,  labelindent=9pt]
%[itemsep=0ex,leftmargin=0.35cm]
%Challenge\#1: 
%While these benchmarks effectively detect certain hallucinations, they 
\item {\textbf{Automatically constructing and updating benchmark datasets.}} Existing methodologies mainly rely on manually curated benchmarks for detecting specific hallucinations, which fail to encompass the broad and dynamic spectrum of fact-conflicting scenarios in LLMs. 
Meanwhile, due to the ever-evolving nature of knowledge, the need for frequent updates to benchmark data imposes a substantial and continuous maintenance effort.
The reliance on benchmark datasets thus restricts the FCH detection techniques' adaptability, scalability, and  %more importantly, 
detection capability;  
%Challenge\#2:
% in existing test cases. 
\item {\textbf{Efficiently generating FCH test cases.}}
LLMs often answer correctly to simple, straightforward questions due to their extensive training on vast datasets. However, to effectively assess their reasoning capabilities, it is important to generate more complex questions, such as those involving intricate temporal characteristics, that require reasoning rather than just recalling facts. However, constructing such test cases is non-trivial. The challenge lies in designing questions that use familiar knowledge but involve reasoning patterns the LLM may not have been explicitly trained on. Creating such test cases efficiently while ensuring they probe reasoning skills in ways the model has not previously encountered is essential to uncovering latent hallucinations;
% queries that involve temporal concepts, such as ``\emph{Does the human population finally reach six billion by the year 2000?}'' may often be used in applications. However, the correctness of the LLM outputs cannot be guaranteed, potentially leading to misleading information. Currently, there are no satisfactory approaches to automatically verify LLM outputs in such test cases; 
%errors even before the occurrence of large model hallucinations; 
%However, it is known that 
%Another critical issue lies in the verification of temporal logic in existing test cases. 
%It is well known that test cases involving temporal-related questions often face difficulties in automatically verifying the soundness and completeness of these issues. Consequently, the correctness of these test cases cannot be guaranteed, potentially introducing errors even before the occurrence of large model hallucinations;
%Challenge\#3: 
\item {\textbf{Validating the reasoning steps from LLM outputs.}} Even when LLMs finally produce correct answers, the outputs may not indicate an accurate reasoning process, potentially masking false understanding -- a source of FCH. Additionally, the quality of manual validation can differ based on human expertise. As a result, automatically validating reasoning processes, particularly those involving complex logical relationships, is inherently challenging. 
\vspace{1mm}
\end{enumerate}







% \lnk{Key challenge: factual knowledge exploring and new facts generation}
%\yi{we should focus on testing, addressing is a little bit vague.}
% The current research landscape in LLM presents a critical gap in automatically testing FCHs. Predominantly, existing methodologies are anchored to manual benchmarks. %\yi{this sentence is quite chinglish.}
% While these benchmarks are effective in detecting certain types of hallucinations, such as those in Figure~\ref{fig:example1}, they fall short in encompassing the broad and dynamic spectrum of fact-conflicting scenarios inherent to LLMs. %\yi{again, this sentence is not very clear}
% Meanwhile, the need for frequent updates to benchmark data, due to the ever-evolving nature of knowledge, imposes a significant and continuous maintenance effort.
% The reliance on benchmark datasets thus restricts the detection techniques’ adaptability, scalability, and worse, detection capability. 
% From a second perspective, the consistency in the quality of benchmark questions can vary, reflecting the differing levels of experience and skill among the human experts responsible for creating them. This is particularly reflected in the stages such as data labeling and results validation. Additionally, it is important to acknowledge that humans are prone to errors.
% %the scalability and the deof these existing methods are also significantly challenged by their dependency on extensive human intervention, particularly in stages such as data labeling and results validation. %This heavy reliance on manual efforts not only limits the scalability of such approaches but also questions their feasibility in efficiently handling the extensive and intricate datasets characteristic of LLMs.
% Thus, the development of more autonomous, agile, and scalable testing techniques is imperative to effectively identify and tackle FCHs in LLMs.%\yi{in this paper, we focus on testing, but until this paragraph, no terms about ``testing'' explicitly occur.}

% \lnk{Solution to Challenge1: comprehensive logic reasoning based testing framework}

% \lnk{Solution to Challenge2: wiki factual knowledge extraction and prolog rules inference for scalability.}
% \lnk{Key challenge: }

%\textbf{Our Work.}
%To address limitations in the existing techniques, 
%we are the first, to the best of our knowledge, to introduce 
To address the problems outlined above, this paper presents a novel automatic end-to-end metamorphic testing technique based on temporal logic for detecting FCH. To the best of our knowledge, we are the first to create a comprehensive FCH testing framework that utilizes factual knowledge reasoning and metamorphic testing, all seamlessly integrated into the fully automated tool, \tool. 

%\shil{(which four methods?)}
\tool begins by establishing a comprehensive factual knowledge base sourced through crawling information from accessible knowledge bases such as Wikipedia. Each piece of this knowledge acts as a ``seed'' for subsequent transformations. Leveraging logical operators to automatically generate temporal reasoning rules, we transform and augment these seeds and expand factual knowledge into a well-established set of question-answer pairs.
%\yi{into xx}. 
Using the questions and answers in the knowledge set as test cases and ground truth, respectively, we construct a reliable and robust FCH testing benchmark. 


The experiment uses a series of carefully designed template-based prompts to test for FCHs in LLMs. To thoroughly evaluate the reasoning behind the responses, we instruct the LLMs not only to generate answers to the test cases but also to provide detailed justifications for their answers. To reliably identify FCH, we introduce two semantic-aware, similarity-based metamorphic oracles. These oracles extract the key semantic elements from each sentence and map out the logical relationships between them. By comparing the logical and semantic structures of the LLM's responses with the ground truth, the oracles can detect FCH by identifying significant deviations in the LLM's answers from the correct information.




%well-crafted prompts\yi{how prompts generated?} to engage LLMs, testing the alignment of their generated content with our enhanced ground truth. Disparities between LLM outputs and the ground truth signal potential hallucinations. 
%Additionally, in our commitment to fostering collaborative research, we have released our constructed dataset as a benchmark~\cite{drowzee}.

%Our approach directly addresses the need for a comprehensive and flexible testing method by transforming structural factual data into a diverse range of scenarios that LLMs may encounter. This method not only improves the reliability of detection but also enhances its adaptability to various factual contexts.
%Furthermore, we address the scalability challenge by automating the transformation and enlargement of our knowledge base, significantly reducing the dependency on human effort. The well-designed prompts used to test LLMs further streamline the process, making it more efficient in identifying potential hallucinations by comparing LLM outputs with our extended ground truth.

%\textbf{Results and Findings.}
%In evaluating our proposed FCH testing framework and \tool, 
%we undertake 
%to evaluate their effectiveness 
We demonstrate the effectiveness of our approach through comprehensive experiments in multiple contexts. First, our evaluation involves deploying \tool across a wide range of topics drawn from a diverse selection of Wikipedia articles. Second, we test our framework on various open-source and commercial LLMs, thoroughly assessing its applicability and performance across different model architectures. 
Our key findings indicate that \tool succeeds in automatically generating practical test cases and identifying hallucination issues of nine LLMs across nine domains. 
Using these test sets, our experiments show that the rate of hallucination responses produced by various LLMs ranges from 24.7\% to 59.8\% for cases unrelated to temporal reasoning and 16.7\% to 39.2\% for cases requiring temporal reasoning. 
%\shil{shall we differentiate the number for non-temporal and temporal one?}.  
We then categorize these hallucination responses into \emph{erroneous knowledge hallucination} and \emph{erroneous inference hallucination}. 
%\syh{four types?}. 
Through an in-depth analysis, we unveil that the lack of logical reasoning capabilities contributes the most to the FCH issues in LLMs. 
Additionally, we observe that LLMs are particularly prone to generating hallucinations in test cases involving temporal concepts and out-of-distribution knowledge. 
Such an evaluation demonstrates that the 
%Furthermore, we confirm that 
test cases generated using %our 
logical reasoning rules can effectively trigger and detect LLM hallucinations.  %issues in . 


This paper builds upon the earlier version~\cite{DBLP:journals/pacmpl/LiL0SW024} by incorporating hallucination detection through temporal-logic-guided test generation. It includes additional motivational examples (\secref{sec:motivating}), a comprehensive set of reasoning rules for encoding \emph{Metric Temporal Logic} (MTL)~\cite{DBLP:conf/lics/OuaknineW05} formulae (\secref{sec:encoding_MTL}) and automatically generating temporal-logic-related question-answer pairs (\secref{prompt}), and further experimental studies (the {RQ4} at \secref{sec:eval}) that detect hallucinations due to insufficient temporal reasoning capabilities. The main contributions of this work are summarized as follows: 
%We summarize the main contributions of this paper below:
\begin{itemize}[itemsep=1mm,leftmargin=0.35cm]
\item 
%Development of 
\textbf{A novel FCH testing framework.} 
To the best of our knowledge, 
we are the first to develop a novel testing framework based on logic programming and metamorphic testing to automatically detect FCH issues in LLMs. %\yi{hanging sentence}This framework represents a significant advancement over current methodologies, providing a more systematic, comprehensive approach to detection.
%Construction and Release of
\item \textbf{An extensive benchmark based on factual knowledge.} 
To facilitate collaborative efforts and future advances in identifying FCH, 
the source code of \tool and constructed benchmark dataset are publicly available  \cite{drowzee}. 
\item \textbf{Test generation via temporal reasoning.} 
Our tool automatically generates test cases that provide a more comprehensive evaluation of LLMs in handling reasoning tasks and identifying factual inconsistencies. By applying temporal logic-based reasoning rules, we expand the initial seed data from our knowledge base, enhancing the diversity and complexity of the test scenarios. 

\item \textbf{Semantic-aware oracles for LLM answer validation.} We propose and implement two automated verification mechanisms, i.e., the oracles, that analyze the semantic structure similarity between sentences. These oracles are designed to validate the reasoning logic behind the answers generated by LLMs, hereby reliably detecting the occurrence of FCHs. 

\end{itemize}



\section{Preliminaries}

% 这一部分需要修改
\textbf{Problem Definition}
Denote the local deployed SLM as $\mathcal{M}_\mathcal{D}$, and the cloud-based LLM as $\mathcal{M}_\mathcal{C}$.
The user's original query is restricted to the edge model for task decomposition and allocation, while the resulting sub-tasks can be resolved by either $\mathcal{M}_\mathcal{D}$ or $\mathcal{M}_\mathcal{C}$.
The entire set of reasoning tasks is represented as $\mathcal{T} = \{T_1, T_2, \dots, T_n\}$.
Let the reasoning accuracy over the entire task set be denoted as $Acc$, with the API cost represented by $C_{Api}$, and the completion time denoted as $C_{Time}$. 
For each task $T$, denote the decomposition process as:
\begin{equation}
T\rightarrow \{t^1,t^2,...t^{k}\}
\end{equation}
Based on the decomposed subtasks $t^i$, the model allocation scheme can be denoted as:
\begin{equation}
M :t^i\mapsto\{ \mathcal{M}_\mathcal{D} , \mathcal{M}_\mathcal{C}\}
\end{equation}
which prioritizes assigning simple subtasks to on-device SLM, while invoking the cloud-based LLM for handling complex subtasks.

The goal of our optimization is to minimize the discrepancy between the model's allocation scheme $M$ and the optimal scheme $M^*$:
\begin{equation}
   \min|M - M^*| 
\end{equation}
The optimal scheme $M^*$ is derived through a search strategy that maximizes SLM usage while maintaining accuracy. During the optimization process, as the allocation scheme gradually approaches the optimal solution, both time cost $C_{Time}$ and API cost $C_{Api}$ decrease, while $Acc$ remains well-maintained.

 



\section{Methodology}\label{sec:method}

\begin{figure}[!h]
\vspace{-2mm}
\centering
\includegraphics[width=1\linewidth]{fig/overview_syh.jpg}
\vspace{-5mm}
\caption{\tool Overview }
\label{fig:tool_overview}
\vspace{-1mm}
\end{figure}

%\shil{shall we provide an overview figure of the proposed framework?}
%\syh{I will work on a workflow figure if possible}
\tool (\figref{fig:tool_overview}) is a general-purpose testing framework that evaluates the LLM outputs for automatically generated test cases. 
The inputs for the response evaluation
contain a natural language (NL) query for LLM and its ground truth answer obtained using logic programming (\secref{subsec3.1}).  
Based on voluminous knowledge database dumps, \tool extracts factual knowledge (\secref{knowledge}), which outputs a set of 
predicates
in the form of Prolog facts. 
Then, \tool deploys a set of pre-defined or automatically generated reasoning rules to
extend the database with a set of derived facts (\secref{sec:derive_more_facts}, \secref{sec:encoding_MTL}). 
These derived facts facilitate an automated test generation (\secref{prompt}), which outputs question-answer pairs (Q\&A) and concrete prompts for LLMs. 
Given the Q\&A pairs and the LLM outputs, \tool evaluates the responses from LLMs and detects factual inconsistency automatically (\secref{response}). 
To this end, it 
first parses LLM outputs semantic-aware structure, and evaluates their semantic similarity to the ground truth. %Subsequently, 
Lastly, it develops similarity-based oracles that apply metamorphic testing to assess consistency against the ground truth. 



%Therefore, the response evaluation for automatically generated tests is achieved given the ground truth Q\&A pairs and the LLM outputs. 


%Lastly, to evaluate the responses from LLMs and detect factual inconsistency automatically (\secref{response}), it first parses LLM outputs semantic-aware structure. Then, it evaluates their semantic similarity to the ground truth. Subsequently, it develops similarity-based oracles that apply metamorphic testing to assess consistency against the ground truth. 






\subsection{Preliminary}
\label{subsec3.1}

\begin{figure}[!b]
{
\vspace{-2mm}
\centering
\small
$
\arraycolsep=3pt\def\arraystretch{1}
\begin{array}{@{}lrcl}
\m{(Program)}&  \Prolog &{::=  } &
\widetilde{\relation} \,\plus\plus\,   \widetilde{\drule} 
\\
\m{(Rule)} &  \m{\drule} &{ ::=  } & 
\relation ~\hornarrow~ \widetilde{body}
\\[0.3em]
\m{(Body)} & \m{body} &{  ::=  } & 
{\tt{Pos}}~ \relation
\,\mid\, {\tt{Neg}}~ \relation 
\,\mid\, \pi
  \\
\m{(Predicate)} &  \relation &{  ::=  } &
 \m{\nm}\,(\widetilde{\entity}) 
 %\text{\syh{how to link entity and term?}}
\\[0.3em]
 \m{(Pure)}  &\pi &{::=}~&
{ T }
  \mid  F
 \mid  {\m{bop}(}{t_1, t_2}{)}
 %\mid \nm(\widetilde{t})
 \mid   {{\pi_1}}  {\wedge}  \pi_2
 \mid  {{\pi_1}} {\vee} \pi_2
 \mid  \neg\pi
\\[0.3em]
 \m{(Term)}  &t &{::=}~& c 
 \mid X 
 \mid t_1{\text{\ttfamily +}}t_2
 \mid t_1\text{-}t_2
\end{array}$
\caption{A Core Syntax of Prolog}
\label{fig:Syntax_of_Prolog}
}
\end{figure}

Logic programming allows the programmer to specify the rules and facts, enabling the Prolog interpreter to infer answers to the given queries automatically. 
We define a core syntax of Prolog in \figref{fig:Syntax_of_Prolog}. 
A Prolog program consists of two parts: a set of facts ($\widetilde{\relation}$) and a set of rules ($\widetilde{\drule}$). 
Throughout the paper, we use the over-tilde notation to denote a set of items. 
For example, $\widetilde{X}$ refers to a set of variables, i.e., $\{X_1, \dots, X_n\}$. 
A fact is represented as a relational predicate with a name and a set of entity arguments, where $\nm$ is an arbitrary distinct identifier drawn from a finite set of relations. 
Entities are drawn from the knowledge database, ranging from string types (for names or events) and integers (for time points).  
A Prolog rule is a Horn clause that comprises a head predicate and a set of body predicates placed on the left and right side of the arrow symbol ($\hornarrow$).

A rule means that the left-hand side is logically implied by the right-hand side. 
The rule bodies are either positive or negative relations, corresponding to the requirements upon the presence or absence of facts. 
We use ``$\relation$'' and ``$\shortNeg\,\relation$'' as abbreviations for
``${\tt{Pos}}~\relation$'' and ``${\tt{Neg}}~\relation$'', respectively. 
Rule bodies contain pure formulae and simplified and decidable sets of Presburger arithmetic predicates over local variables. 
The Boolean values of \emph{true} and \emph{false} are respectively indicated by $T$ and $F$. 
%Other logical relations are represented using general abstract predicates over the terms, which contain the 
The binary operators $bop$ are from $\{ {<}, {\leq}, {=}, {\geq}, {>} \}$. 
Terms consist of constants (denoted by $c$), program variables (denoted by $X$ %\shil{since we use lowercase $c$ to represent constant, can we use lowercase $x$ to represent variables?}
%\syh{Upper case for variable is the Prolog convention.}
), or simple computations of terms, such as $t_1{\plus} t_2$ and $t_1\text{-}t_2$. 
%A Prolog query is executed against a database of facts. 





\subsection{Factual Knowledge Extraction}
\label{knowledge} 
%While predicates can have an arbitrary number of arguments in general, 
To facilitate an automated reasoning system, we extract the \emph{ground facts} in the structure of three-element predicates, i.e., $\m{\nm}\,(\Subj,\Obj)$, where ``$\Subj$'' (stands for $\m{subject}$) and ``$\Obj$'' (stands for $\m{object}$) are entities, and ``$\nm$'' is the name of the predicate. 
Here, we follow the convention of Prolog, where variable names must start with an uppercase letter, and any name that begins with a lowercase letter is a constant. %\shil{whether this format applies to the examples in Fig 6?}
%\syh{Fig. 6 is revised to lowercase now} 

Existing knowledge databases~\cite{freebase, DBpedia, Yago, WordNet} not only encompass a vast array of documents but also provide structured data, facilitating an ideal source for constructing a rich factual knowledge base. 
Thus, the genesis of our test cases is exclusively rooted in the entities and structured relations sourced from existing knowledge databases, ensuring a sophisticated and well-informed foundation for our testing framework. 
Specifically, we follow the categorization for entities (\figref{table:categories}) and relations (\figref{table:relations}) used by WikiPedia~\cite{DBpedia} to perform a thorough facts extraction. 
In particular, the {\small\textbf{Prop.}} (stands for properties) entry for relations guides the automated generation of reasoning rules detailed in \secref{sec:derive_more_facts}.

%as shown in \figref{table:entity_relations}, 


\begin{figure}[!b]
\vspace{-3mm}
\renewcommand{\arraystretch}{1.0}
\setlength{\tabcolsep}{2pt}
\footnotesize 
%\resizebox{\linewidth}{!}{
\begin{tabular}{l | l }
\Xhline{1.0\arrayrulewidth}
\textbf{Entity Cat.} & \textbf{Description}\\
        \Xhline{\arrayrulewidth}
        {Culture and the Arts} & Famous films, books, etc.\\ 
        % & 10,537\\
        %\hline
        {Geography and Places} & Countries, cities and locations. \\
        % & 8,806\\
        %\hline
        {Health and Fitness} & Diseases and genes. \\
        % & 179\\
        %\hline
        {History and Events} & Famous historical events, etc. \\
        % & 5,561\\
       % \hline
        {People and Self} & Important figures. \\
        % & 21,720\\
        %\hline
        {Mathematics and Logic} & Formulas and theorems. \\
        % & 141\\
       % \hline
        {Natural and Physical Sciences} & Celestial bodies and astronomy. \\
        % & 904\\
       % \hline
        {Society and Social Sciences} & Major social institutions, etc.\\ 
        % & 3,862\\
       % \hline
        {Technology and Applied Sciences} & Computer science, etc. \\
        % & 2,773\\
        \Xhline{1.5\arrayrulewidth} %添加表格底部粗线
    \end{tabular}
    %}
\caption{{Entity Categorization.}}
\label{table:categories}
\end{figure}


\begin{figure}[!b]
%\vspace{1.5mm}
\centering
\def\arraystretch{1.1}
\setlength{\tabcolsep}{2pt}
\footnotesize
%\resizebox{\linewidth}{!}{
\begin{tabular}{l | l | l}
\Xhline{1.5\arrayrulewidth}
\textbf{Relation Cat.} & \textbf{Examples}
& 
\textbf{Prop.} 
%(\figref{fig:basic_op_for_predicates})
\\
\Xhline{1.5\arrayrulewidth}
        {Noun Phrase} & 
\begin{tabular}[l]{@{}l@{}} \textit{place\_of\_birth\,(barack\_obama, honolulu).}\\ \textit{genre\,(28\_days\_later, horror\_film).} \end{tabular}
        &
\begin{tabular}[l]{@{}l@{}} 
        $\RNeg$ \\
        $\RSym$ \\
        $\RTrans$
        \end{tabular}
        \\
        \hline
        \begin{tabular}[l]{@{}l@{}} Verb Phrase \\  
        (Passive Voice) \end{tabular}
        & \begin{tabular}[l]{@{}l@{}} \textit{killed\_by}\,\textit{(alexander\_pushkin}, \\ \quad  \textit{georges-charles\_de\_heeckeren\_d'anthès)}.\\ \textit{located\_in\_time\_zone\,(arizona, utc-07:00).}\\ 
        % \textit{(Bayes' theorem, named after, Thomas Bayes)}
        \end{tabular}
        &
\begin{tabular}[l]{@{}l@{}} 
        $\RNeg$ \\
        $\RInv$
        \end{tabular}
        \\
        \hline
        \begin{tabular}[l]{@{}l@{}} Verb Phrase \\  
        (Active Voice) \end{tabular}
        & \begin{tabular}[l]{@{}l@{}} \textit{follows\,(4769\_Castalia, 4768\_hartley).}\\ \textit{replaces\,(american\_broadcasting\_company,} \\ \qquad \quad\ \   \textit{nbc\_blue\_network).} \end{tabular}
        &
\begin{tabular}[l]{@{}l@{}}  
        $\RNeg$ \\
        $\RInv$
        \end{tabular}
        \\
        \Xhline{1.5\arrayrulewidth} %添加表格底部粗线
    \end{tabular}
    %}
\caption{Relation Categorization.}
\label{table:relations}
\end{figure}









The facts extraction is done per-category basis, implementing a divide-and-conquer strategy, which efficiently integrates all the facts from all the categories. 
As shown in \algoref{alg:ground_truth}, for any given entity category and relation category, the function $\textsc{ExtractGroundFacts}$ iterates through all possible entities and relations. 
For each combination ($\m{entity}, \nm$), it queries the database using the $\textsc{QueryDB}$ function, which retrieves all three-element facts established with the specific predicate $\nm$ and the argument $\m{entity}$ placed either in the subject or the object position. 

{
\begin{algorithm}[!h]
\caption{Facts Extraction}
\label{alg:ground_truth}
\small
\begin{algorithmic}[1]
\Require  
Entity Category (\entityCat), Relation Category (\relationCat)
\Ensure Ground Facts ($\groundTruthTriples$)
\Function{ExtractGroundFacts}{
\entityCat, \relationCat}
\State$\groundTruthTriples \gets []$ \Comment{\commentstyle{Initialization}}
\For{$\m{entity}$ $\in$ \entityCat~} \Comment{\commentstyle{Iterate over each entity}}
\For{~$\nm$ $\in$  \relationCat~} \Comment{\commentstyle{Iterate over each relation}}
\State $\widetilde{\relation} \gets$ \Call{QueryDB}{$\m{entity}$, $\nm$} 
\Comment{\commentstyle{Retrieve ground facts}}
\State $\groundTruthTriples.\m{append}(\widetilde{\relation})$ \Comment{\commentstyle{Extend the ground facts}}
\EndFor
\EndFor
\State \Return $\groundTruthTriples$ \Comment{\commentstyle{Return the ground facts}}
\EndFunction
\end{algorithmic}
\end{algorithm}
}



\begin{figure}[!b]
%\vspace{-2mm}
\centering
\small
\begin{gather*}
\frac{
\begin{matrix}
\RNeg\\
{\drule}{=}\,\nm_{\m{new}}\,(\SUBJ, \OBJ){\hornarrow} !\nm\,(\SUBJ, \OBJ)
\end{matrix}
}{
\deriveRules{\nm}{\nm_{\m{new}}}{{\drule}}}
\ \  
\frac{
\begin{matrix}
\RInv\\
{\drule}{=}\,\nm_{\m{new}}\,(\SUBJ, \OBJ) {\hornarrow} \nm\,(\OBJ, \SUBJ)
\end{matrix}
}{
\deriveRules{\nm}{\nm_{\m{new}}}{{\drule}}}
\\[0.4em]
\frac{
\begin{matrix}
\RSym\\
{\drule}{=}\,\nm\,(\SUBJ, \OBJ) {\hornarrow} \nm\,(\OBJ, \SUBJ)
\end{matrix}
}{
\deriveRules{\nm}{\nm}{{\drule}}}
\quad\   
\frac{
\begin{matrix}
{\drule}{=}\,\nm\,(\SUBJ, \OBJ') {\hornarrow} 
\\ 
\nm\,(\SUBJ, \OBJ), 
\nm\,(\OBJ, \OBJ')
\end{matrix}
}{
\deriveRules{\nm}{\nm}{{\drule}}}  \RTrans
\end{gather*}
%\vspace{-1mm}
\caption{Deriving New Facts From the Known Facts}
\label{fig:basic_op_for_predicates}
\end{figure}


\vspace{-2mm}
\subsection{Deriving Simple Facts via Logical Reasoning}
\label{sec:derive_more_facts}
Based on the relation category, each predicate is labeled with a different set of properties, shown in \figref{table:relations}, which are mapped to different derivation rules. 

Based on the ground facts extracted from the databases, \tool derives additional facts to enrich the knowledge and generates test cases from each derived fact. 
As shown in \figref{fig:basic_op_for_predicates}, it provides four basic derivation rules, 
providing sound strategies to generate mutated facts from the ground facts. %\shil{1. How to define the he new name for $nm_{new}$? better provide this. 2. For negation rule, subject and object shall not be changed the position, can refer to oopsla paper.}
%\syh{1. added by the end of next paragraph; 2. revised}
%\footnote{
Note that these rules are also prevalently adopted in several literature~\cite{zhou2019completing, ren2020beta, liang2022reasoning, TIAN2022100159, abboud2020boxe} in the context of knowledge reasoning.
Given a predicate name $\nm$, the derivation  ``$\deriveRules {\nm}{\nm_{\m{new}}}{{\drule}}$'' holds if $\nm$ can be applied into a Prolog rule ${\drule}$, and produces more facts upon a new predicate with the name  $\nm_{\m{new}}$. 
These new predicates are freshly generated based on predefined suffixes. 


As indicated in \figref{table:relations}, 
%In particular, 
all the predicates can be applied to the $\RNeg$ rule, which derives the negated relations, e.g., ``!$\m{was}$'' using its positive counterpart, e.g., ``$\m{wasn't}$''. 
For the \emph{noun phrase} relations, both $\RSym$ and $\RTrans$ rules can apply, which generate more facts without creating new predicates. 
For the \emph{verb phrase} relations, both passive voice and active voice predicates can be applied to the  $\RInv$ rule, which captures the inverse relations, where the subject and object can be reversely linked through a variant of the original relation.  An example of such a rule is: 
\[
\m{influence(\OBJ, \SUBJ)}\hornarrow\m{influence\_by(\SUBJ, \OBJ)}
.\] 


We summarize the fact derivation process using \algoref{alg:logic_reasoning}. Given any relation category, we iterate its predicates and generates the derivation rule $\drule$ (Line 4). 
For simplicity, we assume that the choice of which derivation rule to apply is predetermined. Based on this assumption, a new Prolog program is constructed, comprising ground facts and $\drule$. 
In particular, we use $\llbracket \relation \rrbracket_{\Prolog}$ to denote the query results of $\relation$ concerning the Prolog program $\Prolog$, and $\Prolog$ contains all the ground facts and the derivation rule. 
{Note that when $\relation$ contains no variables, it returns Boolean results indicating the presence of the fact; otherwise, it outputs all the possible instantiation of the variables. }
For each instantiation that contains one subject ``\Subj'' and one object ``\Obj'', we then compose them with the new predicate, which is taken as a  \emph{derived fact}.  
These derived facts are later used to generate NL test cases, detailed in \secref{prompt}. 


{
\begin{algorithm}[!h]
\caption{Deriving New Facts}
\label{alg:logic_reasoning}
\small
\begin{algorithmic}[1]
\Require Ground Facts ($\groundTruthTriples$), Relation Category (\relationCat)
\Ensure Derived Facts ($\derivedFacts$)
\Function{DerivingFacts}{$\groundTruthTriples$, \relationCat}
\State $\derivedFacts \gets []$ \Comment{\commentstyle{Initialization}}
\For{$\nm$ in \relationCat}
\Comment{\commentstyle{Iterate each predicate}}
\State $\deriveRules{\nm}{\nm_{\m{new}}}{{\drule}}$\Comment{\commentstyle{Obtain the new predicate}} %the reasoning rule, and 
\State $\Prolog \gets \groundTruthTriples\plus\plus{\drule}$ \Comment{\commentstyle{Construct the prolog program}} 
\State $\m{instantiations} \gets \llbracket \nm_{\m{new}}(\SUBJ, \OBJ)\rrbracket_{\Prolog}$ 
%\Comment{\commentstyle{Obtain concrete entities}}
\For{(\Subj, \Obj) in $\m{instantiations}$}
\Comment{\commentstyle{Iterate each entity tuple}}
\State $\relation_{\m{new}} \gets \nm_{\m{new}}(\Subj, \Obj)$ 
\Comment{\commentstyle{Construct the derived fact}}
\State $\derivedFacts.\m{append}(\relation_{\m{new}})$ \Comment{\commentstyle{Append the derived facts}}
\EndFor
\EndFor
\State \Return $\derivedFacts$ \Comment{\commentstyle{Return the derived facts}}
\EndFunction 
\end{algorithmic}
\end{algorithm}
}






%-logic-based
\vspace{-2mm}
\subsection{Deriving Facts via Temporal Reasoning}
\label{sec:encoding_MTL}




%as the query language, 
%We convert the temporal-logic-based test cases into 
%temporal-logic-based natural language query, we use NLP techniques \cite{icaps2023fc,aaai2023fc} %\syh{cite here?} %to convert it into a 

Apart from the basic derivation rules, \tool can also automatically derive complex composition rules based on \emph{Metric Temporal Logic} (MTL) \cite{DBLP:conf/lics/OuaknineW05}. 
Specifically, we generate temporal test cases  based on randomly generated MTL formulae over historical events. 
We define the syntax for MTL formula in \figref{fig:syntax_of_the_metric_temporal_logic}, which contains the temporal operators for ``finally ($\mathcal{F}$)'', 
``globally ($\mathcal{G}$)'', 
``until ($\mathcal{U}$)'', 
and ``next ($\mathcal{N}$)''. 
The atomic propositions here are basic event relations $\nm$. 
%are three-element relations associated with time stamps. 
The time intervals are pairs of natural numbers indicating the lower and upper bounds of the years%\shil{only year?} \syh{so far yes, but I added a sentence by the end to make it more generic}
; and the constraint $\Istart \,{\leq}\, \Iend$ is enforced implicitly for all time intervals. 
In this paper for simplicity, we use discrete time measured in years as the smallest time interval. However, the framework can be extended to accommodate any smaller discrete time intervals, such as days or seconds. 


\begin{figure}[!h] 
\vspace{-2.5mm}
\small
\centering
\begin{align*}
(\m{MTL})\quad & \mtl &{::=}\quad &
\nm %(\Subj, \Obj) %\ap  
\,{\mid}\, \mathcal{F}_\interval \,\mtl
\,{\mid}\, \mathcal{G}_\interval \,\mtl 
\,{\mid}\, \mtl_1  
\,\mathcal{U}_\interval \,  \mtl_2 
\,{\mid}\, \mathcal{N} \,\mtl
\,{\mid}\, 
\\
&&&
%\,{\mid}\, \mtl_1 \, {\rightarrow} \,\mtl_2
 \mtl_1  \,{\wedge}\, \mtl_2
\,{\mid}\, \mtl_1  \,{\vee}\, \mtl_2
\,{\mid}\, \neg \mtl 
\\[0.3em]
%&(\m{Atomic~Proposaition})~ \ap ~{::=}~ 
%\nm\_{\m{TS}}(\interval, \Subj, \Obj)
%\\[0.3em]
(\m{Time~Interval}) \quad& \interval &{::=}\quad & [\Istart, \Iend]
\end{align*}
\vspace{-4mm}
\caption{A Core Syntax of MTL}
\label{fig:syntax_of_the_metric_temporal_logic}
\vspace{-3mm}
\end{figure}


To facilitate the generation of temporal-based Q\&A pairs, we define the semantics model for the MTL formulae in \defref{def:semantics_MTL}, where the history is a set of facts. 
Here, we use $\history$ as a set of historical relations, 
e.g., ``$\nm\_{\m{TS}}(\interval, \Subj, \Obj)$'', which are the time-stamped version relations of the three-element relations ``$\nm(\Subj, \Obj)$'', derived by one of the following rules: \\[-0.5em]

\noindent 
{\small $\ \   
\nm\_{\m{TS}}(\interval, \Subj, \Obj) \hornarrow \nm(\Subj, \Obj), \m{start}(\Obj, n_1), \m{end}(\Obj, n_2), \interval{=}[n_1, n_2]. 
$}\\[-0.5em]

\noindent  {\small $\ \  \nm\_{\m{TS}}(\interval, \Subj, \Obj) \hornarrow \nm(\Subj, \Obj), \m{start}(\Subj, n_1), \m{end}(\Subj, n_2), \interval{=}[n_1, n_2].$}
\\

\noindent which construct the event intervals using the time stamps associated with the object or the subject, respectively. 
The ``$\m{start}$'' and ``$\m{end}$'' predicates are originally generated from the knowledge database and mark the starting and ending points of the duration of the object (or subject) event. 
For simplicity, we use ``$\nm\_{\m{TS}}(\interval)$'' to abbreviate ``$\nm\_{\m{TS}}(\interval, \Subj, \Obj)$'' when $\Subj$ and $\Obj$ are unambiguously unique from the context. 
We also use $\llbracket \nm\_{\m{TS}}(\interval) %, \Subj, \Obj
\rrbracket_{\history}$ to denote the validity of querying the presence of a fact $\nm\_{\m{TS}}(\interval)$ 
against the historical facts $\history$, which stores all the time-stamped three-element predicates. 


\begin{definition}[A Point-based Semantics for MTL]
\label{def:semantics_MTL}
Given a set of (historical) facts $\history$, recording all the events that happened in history, an MTL formula $\mtl$, and a concrete time point  $\timepoint$, the satisfaction relation $(\history, \timepoint) \models \mtl$  (read at the time point \timepoint, the history $\history$ satisfies $\mtl$) is recursively defined as follows: 

{
\small
\begin{align*}
%%%%%%%%%%%%%%%%%%%%%%%%%%%%%%
%%%%%%%%%%% AP  R   %%%%%%%%%%
%%%%%%%%%%%%%%%%%%%%%%%%%%%%%%
(\history, \timepoint) &\models 
\nm &\m{iff}&~ 
\m{\exists\,\interval}.~ 
\llbracket \nm\_{\m{TS}}(\interval) \text{$\rrbracket_{\history}$}{=}\m{true}
~\m{and}~
\timepoint\,{\in}\,\interval
\\[0.1em]
%, \Subj, \Obj
%%%%%%%%%%%%%%%%%%%%%%%%%%%%%%
%%%%%%%%%%% Finally %%%%%%%%%%
%%%%%%%%%%%%%%%%%%%%%%%%%%%%%%
(\history, \timepoint) &\models \mathcal{F}_\interval \,\mtl & 
\m{iff}&~ 
\m{\exists\,\distance}.~\distance\,{\in}\,I  ~ \m{and}
~ (\history, \timepoint\plus\distance)\models\mtl
\\[0.1em]
%%%%%%%%%%%%%%%%%%%%%%%%%%%%%%
%%%%%%%%%%% Globally %%%%%%%%%%
%%%%%%%%%%%%%%%%%%%%%%%%%%%%%%
(\history, \timepoint) &\models \mathcal{G}_\interval\,\mtl & 
\m{iff}&~ 
\m{\forall\,\distance}.~\distance\,{\in}\,I  ~ \m{and}
~ (\history, \timepoint\plus\distance)\models\mtl
\\[0.1em]
%%%%%%%%%%%%%%%%%%%%%%%%%%%%%%
%%%%%%%%%%% Next %%%%%%%%%%
%%%%%%%%%%%%%%%%%%%%%%%%%%%%%%
(\history, \timepoint) &\models \mathcal{N}\,\mtl & 
\m{iff}&~ 
(\history, \timepoint\plus 1)\models\mtl
\\[0.1em]
%%%%%%%%%%%%%%%%%%%%%%%%%%%%%%
%%%%%%%%%%% negation %%%%%%%%%%
%%%%%%%%%%%%%%%%%%%%%%%%%%%%%%
(\history, \timepoint) &\models\neg \mtl & \m{iff}&~
(\history, \timepoint)\not\models\mtl
\\[0.1em]
%%%%%%%%%%%%%%%%%%%%%%%%%%%%%%
%%%%%%%%%%% Unitl %%%%%%%%%%
%%%%%%%%%%%%%%%%%%%%%%%%%%%%%%
(\history, \timepoint) &\models \mtl_1 \, \mathcal{U}_\interval \,\mtl_2  & \m{iff}&~  \m{\exists\,\distance}.~ \distance\,{\in}\,\interval  ~ \m{and}~ (\history, \timepoint\plus\distance)\models\mtl_2 ~ \m{and}
\\[0.1em] 
&&& ~ 
\m{\forall}\, 
k~\m{with} ~\timepoint{<}k{<}(\timepoint\plus\distance), 
(\history, k)\models \mtl_1
\\[0.1em]
%%%%%%%%%%%%%%%%%%%%%%%%%%%%%%
%%%%%%%%%%% conjunction %%%%%%%%%%
%%%%%%%%%%%%%%%%%%%%%%%%%%%%%%
(\history, \timepoint) &\models\mtl_1 \, {\wedge} \,\mtl_2 & \m{iff}&~ (\history, \timepoint)\models\mtl_1 ~\m{and}~ (\history, \timepoint)\models\mtl_2
\\[0.1em]
%%%%%%%%%%%%%%%%%%%%%%%%%%%%%%
%%%%%%%%%%% disjunction %%%%%%%%%%
%%%%%%%%%%%%%%%%%%%%%%%%%%%%%%
(\history, \timepoint) &\models\mtl_1 \, {\vee} \,\mtl_2 & \m{iff}&~ (\history, \timepoint)\models\mtl_1 ~\m{or}~ (\history, \timepoint)\models\mtl_2 
\end{align*}}
\end{definition}
\vspace{2mm}



We randomly generate temporal test cases based on the rich set of historical events and the syntax templates defined in \figref{fig:syntax_of_the_metric_temporal_logic}. 
Each temporal question consists of a concrete MTL formula and a concrete time point, i.e., $(\phi, \timepoint)$. 
For example, the query ``\emph{At 1800, does Victorian era finally come within 40 years?}'' is represented as $(\mathcal{F}_{[0, 40]} \m{victorian\_era}, 1800)$. 
Next, we show how to obtain the expected answer by automatically generating Prolog reasoning rules. 

Given a query $\mtl$, the relation ``$\encoding{\mtl}{\nm}{\widetilde{\drule}}$'' holds if $\mtl$ can be translated into a set of Prolog rules, i.e., $\widetilde{\drule}$. 
Querying ``$\nm(\interval)$'' 
%with the set of Prolog rules 
$\widetilde{\drule}$, against the known database facts yields a set of instantiation of the interval $\interval$. 
The validity of $\mtl$ at any given time point $\timepoint$ is then indicated by the existence of a concrete interval  $\interval$ such that $\timepoint\,{\in}\,\interval$. 
We define the full set of encoding rules for MTL operators in \figref{fig:encoding_rules_mtl}. 

These encoding rules deploy several auxiliary predicates: the 
``$\m{findall}(\interval, \nm)$'' relation indicates that $\interval$ is a union of all the time intervals which satisfy $\nm$; 
the  ``$\m{compl}(\interval, \interval_1)$'' relation indicates that time intervals $\interval$ and $\interval_1$ complement each other, and their union encompasses the entire set of time points; the union and intersection operations, denoted by $\cup$ and $\cap$, are applied to two sets of time intervals. 

\begin{figure}[!h]
% \begin{minipage}[b]{1\linewidth}
\vspace{-2mm}
\vspace{0mm}
\begin{lstlisting}[xleftmargin=6em,numbersep=5pt,basicstyle=\footnotesize\ttfamily]
//nm1 = charles_dickens
charles_dickens_TS([1812, 1870]).
//nm2 = victorian_era
victorian_era_TS([1837, 1901]).
\end{lstlisting} 
\vspace{-1mm}
\caption{Database $\history_s$ Containing Two Time-stamped Events}
\label{fig:Prolog_encoding_Example}
\vspace{-2mm}
\end{figure}


Next, we illustrate the encoding rules for each MTL operator using a few examples. 
To facilitate the illustration, we use a small database $\history_s$ defined in \figref{fig:Prolog_encoding_Example}, which contains two facts: ``\emph{The author Charles Dickens was born in 1812 and he lived until 1870, which spanned a significant portion of the Victorian era}'' and ``\emph{The Victorian era started from 1837 until Queen Victoria died in 1901}'': 

\begin{figure}[!b]
\vspace{-3mm}
\centering\small
\begin{gather*} 
%%%%%%%%%%%%%%%%%%%%%%%%%%%%%%
%%%%%%%%%%% AP R %%%%%%%%%%%%%
%%%%%%%%%%%%%%%%%%%%%%%%%%%%%%
\frac{
\begin{matrix}
\widetilde{\drule} = [\nmNEW(\interval) \hornarrow \nm\_{\m{TS}}(\interval).]
\end{matrix}
}{\encoding {\nm}{\nmNEW}{\widetilde{\drule}}}\ [\trans\text{-}\m{AP}]
\\[0.6em]
%%%%%%%%%%%%%%%%%%%%%%%%%%%%%%
%%%%%%%%%% Finally %%%%%%%%%%%
%%%%%%%%%%%%%%%%%%%%%%%%%%%%%%
\frac{
\begin{matrix}
\widetilde{\drule} {=} 
[\nmNEW([\interval^\prime_\m{start}\text{-}\interval_{\m{end}}, \interval^\prime_\m{end}\text{-}\interval_{\m{start}}]) \hornarrow \nm(\interval^\prime).]
\end{matrix}
}{\encoding {\mathcal{F}_\interval\,\mtl}{\nmNEW}{\widetilde{\drule} }}\ [\trans\text{-}\m{Finally}]
\\[0.6em]
%%%%%%%%%%%%%%%%%%%%%%%%%%%%%%
%%%%%%%%% Globally %%%%%%%%%%%
%%%%%%%%%%%%%%%%%%%%%%%%%%%%%%
\frac{
\begin{matrix}
\widetilde{\drule} {=} 
[\nmNEW([\interval^\prime_\m{start}\text{-}\interval_{\m{start}}, \interval^\prime_\m{end}\text{-}\interval_{\m{end}}]) \hornarrow \nm(\interval^\prime).]
\end{matrix}
}{\encoding {\mathcal{G}_\interval\,\mtl}{\nmNEW}{\widetilde{\drule} }}\ [\trans\text{-}\m{Globally}]
\\[0.6em]
%%%%%%%%%%%%%%%%%%%%%%%%%%%%%%
%%%%%%%%% Next %%%%%%%%%%%
%%%%%%%%%%%%%%%%%%%%%%%%%%%%%%
\frac{
\begin{matrix}
\widetilde{\drule} {=} 
[\nmNEW([\interval^\prime_\m{start}\text{-}1, \interval^\prime_\m{end}\text{-}1]) \hornarrow \nm(\interval^\prime).]
\end{matrix}
}{\encoding {\mathcal{N}\,\mtl}{\nmNEW}{\widetilde{\drule} }}\ [\trans\text{-}\m{Next}]
\\[0.6em]
%%%%%%%%%%%%%%%%%%%%%%%%%%%%%%
%%%%%%%%% Until %%%%%%%%%%%
%%%%%%%%%%%%%%%%%%%%%%%%%%%%%%
\frac{
\begin{matrix}
\encoding{\mtl_1}{\nm_1}{\widetilde{\drule}_1}
\qquad 
\encoding{\mtl_2}{\nm_2}{\widetilde{\drule}_2}
\\[0.2em]
\widetilde{\drule}_3{=} [\m{helper1}([\interval^\prime_{\m{start}}\plus\interval_{\m{start}}, \interval^\prime_{\m{end}}\plus1]) \hornarrow 
\nm_1(\interval^\prime).]
\\[0.2em]
\widetilde{\drule}_4{=} [\m{helper2}(\interval_1\,{\cap}\,\interval_2) \hornarrow 
\m{helper1}(\interval_1), 
\nm_2(\interval_2).] 
\\[0.2em]
\encoding {\mathcal{F}_\interval\,(\m{helper2})}{\nm_f}{\widetilde{\drule}_5 }
\\[0.2em]
\widetilde{\drule}_6{=} [\nmNEW(\interval_1\cap \interval_2) \hornarrow 
\nm_1(\interval_1), 
\nm_
f(\interval_2). ] 
\end{matrix}
}{\encoding{\mtl_1\,\mathcal{U}_\interval\,\mtl_2}{\nmNEW}{\widetilde{\drule}_1\cup \widetilde{\drule}_2\cup
\widetilde{\drule}_3\cup
\widetilde{\drule}_4\cup
\widetilde{\drule}_5\cup
\widetilde{\drule}_6}}\ [\trans\text{-}\m{Until}]
%\shil{
%~2. ~what's ~the~ definition ~\interval ~in~ \mathcal{F}? }
%\\ \shil{~3. ~what's ~the ~meaning ~of ~;?}\text{\syh{to~construct~list~from~single~rules}}
\\[0.6em]
%%%%%%%%%%%%%%%%%%%%%%%%%%%%%%
%%%%%%%%% Negation %%%%%%%%%%%
%%%%%%%%%%%%%%%%%%%%%%%%%%%%%%
\frac{
\begin{matrix}
\encoding{\mtl}{\nm}{\widetilde{\drule}_1}
\\ 
\widetilde{\drule}{=}[\nmNEW(\interval) \hornarrow
\m{findall}(\interval_1, \nm), \m{compl}(\interval_1, \interval).]
\end{matrix}
}{
\encoding{\neg\mtl}{\nmNEW}{
\widetilde{\drule}_1\,{\cup}\,\widetilde{\drule}}
}\ [\trans\text{-}\m{Neg}]
\\[0.6em]
%%%%%%%%%%%%%%%%%%%%%%%%%%%%%%
%%%%%%%%% Conjunction %%%%%%%%%%%
%%%%%%%%%%%%%%%%%%%%%%%%%%%%%%
\frac{
\begin{matrix}
[\trans\text{-}\m{Conj}]\\
\encoding{\mtl_1}{\nm_1}{\widetilde{\drule}_1}
\qquad 
\encoding{\mtl_2}{\nm_1}{\widetilde{\drule}_2}
\\
\widetilde{\drule}{=}[\nmNEW(\interval_1\,{\cap}\,\interval_2) \hornarrow
\m{findall}(\interval_1, \nm_1), \m{findall}(\interval_2, \nm_2)]
\end{matrix}
}{
\encoding{\mtl_1{\wedge}\mtl_2}{\nmNEW}{ \widetilde{\drule}_1\,{\cup}\,\widetilde{\drule}_2\,{\cup}\,\widetilde{\drule}}
}
\\[0.6em]
%%%%%%%%%%%%%%%%%%%%%%%%%%%%%%
%%%%%%%%% Disjunction %%%%%%%%%%%
%%%%%%%%%%%%%%%%%%%%%%%%%%%%%%
\frac{
\begin{matrix}
[\trans\text{-}\m{Disj}]\\
\encoding{\mtl_1}{\nm_1}{\widetilde{\drule}_1}
\qquad 
\encoding{\mtl_2}{\nm_1}{\widetilde{\drule}_2}
\\
\widetilde{\drule}{=}[\nmNEW(\interval_1\,{\cup}\,\interval_2) \hornarrow
\m{findall}(\interval_1, \nm_1), \m{findall}(\interval_2, \nm_2)]
\end{matrix}
}{
\encoding{\mtl_1{\vee}\mtl_2}{\nmNEW}{ \widetilde{\drule}_1\,{\cup}\,\widetilde{\drule}_2\,{\cup}\,\widetilde{\drule}}
}
\end{gather*}
\caption{Encoding MTL Formula $\mtl$ using Prolog Rules}
\label{fig:encoding_rules_mtl}
\end{figure}




\begin{enumerate}[itemsep=0.7em,leftmargin=!,wide]
\item 
When 
$\mtl\,{=}\,\m{charles\_dickens}$ and $\timepoint\,{=}\,1800$: \\ 
According to the encoding rule $[\trans\text{-}\m{AP}]$, the generated Prolog rule is: $\m{\nm1(\interval)}\hornarrow\,\m{
charles\_dickens\_TS(\interval)}$.
Now, querying ``$\nm1(\interval)$'' against $\history_s$ yields $\interval\,{=}\,[1812, 1870]$. Since $1800\,{\not\in}\,\interval$, the expected result of this query is false.

Similarly, when 
$\mtl\,{=}\,\m{victorian\_era}$ and $\timepoint\,{=}\,1900$: \\ 
According to the encoding rule $[\trans\text{-}\m{AP}]$, the generated Prolog rule is: $\m{\nm2(\interval)}\hornarrow\,\m{
victorian\_era\_TS(\interval)}$.
Now, querying ``$\nm2(\interval)$'' against $\history_s$ yields $\interval\,{=}\,[1837, 1901]$. Since $1900\,{\in}\,\interval$, the expected result of this query is true. 

\item When $\mtl\,{=}\,\mathcal{F}_{[0, 40]}\,\m{victorian\_era}$ and $\timepoint\,{=}\,1800$: \\
According to the encoding rule $[\trans\text{-}\m{Finally}]$, the generated Prolog rule is: 
$\m{finally\_\nm2([n_1\text{-}40, n_2\text{-}0])}\hornarrow \m{
\nm2([n_1, n_2])}.$
Now, querying ``$\m{finally\_\nm2}(\interval)$'' against $\history_s$ yields $\interval\,{=}\,[1797, 1901]$. Since $1800\,{\in}\,\interval$, the expected result is true. 
Indeed, all the time points in $\interval$ satisfy the property: ``\emph{within 40 years, finally Victorian era came/still exist}''. 

\item When $\mtl\,{=}\,\mathcal{G}_{[30, 50]}\,\m{victorian\_era}$ and $\timepoint\,{=}\,1800$: \\
According to the rule $[\trans\text{-}\m{Globally}]$, the generated Prolog rule is: $\m{globally\_\nm2([n_1\text{-}30, n_2\text{-}50])}\hornarrow \m{
\nm2([n_1, n_2])}.$
Now, querying ``$globally\_\nm2(\interval)$'' against $\history_s$ yields $\interval\,{=}\,[1807, 1851]$. Since $1800\,{\not\in}\,\interval$, the expected result is false. 
Indeed, only all the time points in $\interval$ satisfy that ``\emph{Victorian era is globally true throughout the 30th to the 50th years in the future}''. 

\item When $\mtl\,{=}\,\mathcal{N}\,\m{victorian\_era}$ and $\timepoint\,{=}\,1836$: \\
According to the rule $[\trans\text{-}\m{Next}]$, the generated Prolog rule is: $\m{next\_\nm2([n_1\text{-}1, n_2\text{-}1])}\hornarrow \m{
\nm2([n_1, n_2])}.$
Now, querying ``$\m{next\_\nm2}(\interval)$'' against $\history_s$ yields $\interval\,{=}\,[1836, 1900]$. Since  $1836\,{\in}\,\interval$, the expected result is true. 
Indeed, all the time points in $\interval$ satisfy that ``\emph{next year Victorian era came/still exist}''. 

\item When $\mtl\,{=}\,\m{charles\_dickens}
~\mathcal{U}_{[10, 20]}\,\m{victorian\_era}$ and $\timepoint{=}1800$: This query aims to determine the time interval $\interval$ that encompasses all time points $\timepoint'$ for which there exists a future year ($\timepoint'\plus\distance$) when the Victorian era had begun; and during the time from $\timepoint'$ to $\timepoint'\plus\distance$, Charles Dickens must have been born and remained alive throughout. Lastly, check if $1800\,{\in}\,\interval$.

In $[\trans\text{-}\m{Until}]$, 
$\m{helper1}$ computes all the possible  $\timepoint'\plus d$  
%where \\  $d\,{\in}\,[10, 20]$, 
such that $(\history, k)\models \m{charles\_dickens} ~\m{forall}~ 
k~\m{with} ~\timepoint'{<}k{<}(\timepoint'\plus\distance)$. 
Next $\m{helper2}$ computes the overlapping  intervals of $\m{helper1}$ and the intervals that also satisfy $(\history, \timepoint'\plus\distance)\,{\models}\,\m{victorian\_era}$. 
Then $\nm_f$ computes the interval of $\timepoint'$ which finally reach $\m{helper2}$ within 10 to 20 years. 
Lastly, the final answer of the interval of $\timepoint'$~is the intersection of $\nm_f$ and $\nm_1$. 
Therefore, given the concrete query $(\phi, \timepoint)$, from $[\trans\text{-}\m{Until}]$, 
the generated rules are shown in \figref{fig:until10-20-encoding}. 




\begin{figure}[!h]
\vspace{-3mm}
{
\begin{align*}
&\m{helper1([n_1\plus10, n_2\plus1])}\hornarrow \m{\nm1([n_1, n_2])}.
& // [1822, 1871]
\\
&\m{helper2(\interval_1\cap\interval_2)}\hornarrow\m{helper1(\interval_1)}, \m{\nm2(\interval_2)}. 
& // [1837, 1871]
\\
& \m{\nm_f([n_1\text{-}20, n_2\text{-}10])}\hornarrow \m{
helper2([n_1, n_2])}.
& // [1817, 1861]
\end{align*}
\vspace{-4mm}
\[\m{charles}\_\m{until}\_\m{victorian\_era}(\interval_1\cap\interval_2) \hornarrow \m{\nm1(\interval_1)}, \m{\nm_f(\interval_2)}.\]}
\caption{Prolog Rules Generated for an "Until" Query}
\label{fig:until10-20-encoding}
\vspace{-1mm}
\end{figure}

Querying ``$\m{charles}\_\m{until}\_\m{victorian\_era}$'' against $\history_s$ yields $\interval\,{=}\,[1817, 1861]$. Since $1800\,{\not\in}\,\interval$, the expected result is false. 
Indeed, only all the time points in $\interval$ satisfy $\phi$ under the semantic definition of \emph{Until}, cf.  \defref{def:semantics_MTL}. For example when $\timepoint'{=}1817$, there exists $\distance{=}20$ such that $\phi$ holds; and when $\timepoint'{=}1861$ there exists $\distance{=}10$ such that $\phi$ holds. 

Note that, in this encoding, the interval of ``\emph{Until}'' operators does not include $[0, 0]$, as $\mtl_1  
\,\mathcal{U}_{[0, 0]} \,  \mtl_2$ essentially equals $\mtl_2$. Therefore, when the interval compasses $[0, 0]$, we use the following rule to decompose the encoding: (Note that when $\interval'\,{=}\,\interval{\setminus}[0, 0]$, it means $\interval'\cup[0, 0]\,{=}\,\interval$)
\begin{align*}
\frac{
[0, 0] \subseteq \interval 
\qquad 
\interval^\prime \,{=}\, \interval{\setminus}[0, 0]
}{
\mtl_1  
\,\mathcal{U}_\interval \,  \mtl_2 
\equiv  (\mtl_1  
\,\mathcal{U}_{\interval^\prime} \,  \mtl_2)  \vee \mtl_2 
} \ [\trans\text{-}\m{Until}\text{-}0]
\end{align*}

\vspace{2mm}
\item When $\mtl\,{=}\,\neg\,\m{victorian\_era}$ and $\timepoint\,{=}\,1800$: \\
By $[\trans\text{-}\m{Neg}]$, the generated Prolog rule is: 
$\m{neg\_\nm2(\interval)}\hornarrow$ 
$\m{findall}(\interval_1, \nm1), \m{compl}(\interval_1, \interval).$ 
Querying ``$\m{neg\_\nm2}(\interval)$'' against $\history_s$ yields $\interval\,{=}\,[1, 1836] \cup [1902, 2024].$
Here, we take all the after-century years to be the full set. 
Since $1800\,{\in}\,\interval$, the expected result is true. 
Indeed, all the time points in $\interval$ satisfy that ``\emph{Victorian era has not come/already passed}''.  


\item When $\mtl\,{=}\,\m{charles\_dickens}\,{\wedge}\,\m{victorian\_era}$ and $\timepoint{=}1900$: 
By $[\trans\text{-}\m{Conj}]$, the generated Prolog rule is: 
$\m{\nm1\_and\_\nm2(\interval_1{\cap}\interval_2)}\hornarrow 
\m{findall}(\interval_1, \nm1), \m{findall}(\interval_2, \nm2)$. 
Now, querying ``$\m{\nm1\_and\_\nm2}(\interval)$'' against $\history_s$ yields $\interval\,{=}\,[1837, 1870]$. Since $1900\,{\not\in}\,\interval$, the expected result is false. 
Indeed, only the time points in $\interval$ satisfy that ``\emph{Victorian era exists and Charles Dickens is alive}''. 


\item When $\mtl\,{=}\,\m{charles\_dickens}\,{\vee}\,\m{victorian\_era}$~and
$\timepoint{=}1900$: 
By $[\trans\text{-}\m{Disj}]$, the generated Prolog rule is as follows: 
$\m{\nm1\_or\_\nm2(\interval_1{\cup}\interval_2)}\hornarrow 
\m{findall}(\interval_1, \nm1), \m{findall}(\interval_2, \nm2)$. 
Now, querying ``$\m{\nm1\_or\_\nm2}(\interval)$'' against $\history_s$ yields $\interval\,{=}\,[1812, 1901]$. Since $1900\,{\in}\,\interval$, the expected result is true. 
Indeed, all the time points in $\interval$ satisfy that ``\emph{Victorian era exists or Charles Dickens is alive}''.  
%\shil{check implementation?}
\end{enumerate}
\vspace{3mm}






{\emph{\textbf{Remark.}}} 
While discrete-time MTL is commonly employed for model-checking timed verification~ \cite{DBLP:phd/us/Henzinger91}, utilizing Prolog to encode MTL for reasoning about the temporal relationships among events and detecting LLM hallucination is novel. 
These encoding rules can recursively accommodate the entire range of MTL formulas, including those with any level of nesting. 
We provide a formal definition for the correctness of these encoding rules in \theoref{ThemSoundAndComplete} and demonstrate that they are both sound and complete.



\begin{restatable}[Correctness of the encoding rules]{thm}{ThemSoundAndComplete}
\label{ThemSoundAndComplete}
~\\
Given any $\history$, 
$\mtl$, and 
$\encoding {\mtl}{\nm}{\widetilde{\drule}}$, let $\Prolog{=}\history \plus\plus \widetilde{\drule}$, we define,  \\
(1) Soundness: \\
$\forall\, \interval$.  
$\llbracket \nm(\interval) \rrbracket_{ \Prolog} {=} \m{true}$, then 
$\forall\, \timepoint\,{\in}\, \interval$, we have 
$(\history, \timepoint) \models \mtl$;  \\
(2) Completeness: \\ 
$\forall \,\timepoint\,$. $(\history, \timepoint) \models \mtl$, then $\exists\, \interval$. $\llbracket \nm(\interval) \rrbracket_{\Prolog} {=} \m{true}$  and $\timepoint\,{\in}\,\interval$. 
\end{restatable}

\begin{proof}
By structural induction over $\phi$. 
%a case analysis of the encoding rules.
The detailed proofs are given in the Appendix. %\appref{app:correctness}.  
\end{proof}




\begin{table*}[!t]
% \def\arraystretch{1.0}
\setlength{\tabcolsep}{1pt}
\centering
% \footnotesize
\caption{Relation-Template Mapping Patterns.}
 % \lnk{need to refine}}
\label{table:template}
\footnotesize
%\resizebox{\linewidth}{!}{
\begin{tabular}{l l}
\toprule 
\textbf{Relation} & \textbf{Template Examples}  \\
    \midrule
{Noun Phrase} & \begin{tabular}[l]{@{}l@{}} - Is it true that 
$\langle \m{Subject}\rangle$ and 
$\langle\m{Object}\rangle$ share 
$\langle\m{Relation}\rangle$? 
\\ - $\langle\m{Subject}\rangle$ and $\langle\m{Object}\rangle$ have/made/shared totally different $\langle\m{Relation}\rangle$. Please judge the truthfulness of this statement.
%\\Please judge the truthfulness of this statement. 
    \end{tabular}  \\
    \midrule
    \begin{tabular}[l]{@{}l@{}} Verb Phrase \\ (Passive Voice) \end{tabular} & \begin{tabular}[l]{@{}l@{}} - Is it true that $\langle \m{Subject}\rangle$ is/was/are/were $\langle \m{Relation}\rangle$ $\langle\m{Object}\rangle$? \\ - It is impossible for $\langle \m{Subject}\rangle$ to be $\langle\m{Relation}\rangle$ $\langle\m{Object}\rangle$. Am I right?%\\ Other formats... 
    \end{tabular}  \\
    \midrule
\begin{tabular}[l]{@{}l@{}} Verb Phrase \\ (Active Voice) \end{tabular}
 & \begin{tabular}[l]{@{}l@{}} - Is it true that 
 $\langle \m{Subject}\rangle$
 $\langle\m{Relation}\rangle$
 $\langle\m{Object}\rangle$?  \\ - $\langle \m{Subject}\rangle$ $\langle\m{Relation}\rangle$ $\langle\m{Object}\rangle$. 
 %\\ Please judge the truthfulness of this statement.
 %\\ Other formats... 
 \end{tabular}  \\

    \bottomrule %添加表格底部粗线
\end{tabular}
%}
\end{table*}

\begin{table*}[!t]
\setlength{\tabcolsep}{3pt}
\centering
% \footnotesize
\caption{Temporal-Template Mapping Patterns (implicitly querying upon year $\iyear$).}
\label{table:temporal_template}
\footnotesize
\begin{tabular}{l  l }
\toprule 
\textbf{MTL Formulae} & \textbf{Template Examples}  \\
\midrule 
\mtltoNL($\nm$) &  Did $\langle\nm\rangle$ happen at year $\langle \iyear \rangle$? 
\\  \midrule
\mtltoNL($\mathcal{F}_\interval \,\mtl$) & Did ``Event'' finally happen within the time frame of $\langle \interval \rangle$ after the year $\langle \iyear \rangle$, where ``Event'' is defined as $\langle \mtltoNL(\mtl)\rangle$? 
\\ 
\midrule 
\mtltoNL($\mathcal{G}_\interval \,\mtl$) &  Did ``Event'' globally happen within the time frame of $\langle \interval \rangle$ after the year $\langle \iyear \rangle$, where ``Event'' is defined as $\langle \mtltoNL(\mtl)\rangle$?
\\ 
\midrule 
\mtltoNL($\mathcal{N} \,\mtl$) & Did ``Event'' happen in the next year of $\langle \iyear \rangle$, where ``Event'' is defined as $\langle \mtltoNL(\mtl)\rangle$? 
\\ 
\midrule 
\mtltoNL($\mtl_1 \, \mathcal{U}_\interval \,\mtl_2$) &  
\begin{tabular}[l]{@{}l@{}} 
Did ``Event$_1$'' happen continuously until ``Event$_2$'' started, during the period $\langle \interval \rangle$ after the year $\langle \iyear \rangle$, \\ 
where ``Event$_1$'' is $\langle \mtltoNL(\mtl_1) \rangle$ and ``Event$_2$'' is $\langle \mtltoNL(\mtl_2) \rangle$?
\end{tabular}
\\ 
\midrule 
\mtltoNL($\mtl_1  \,{\wedge}\,  \mtl_2$) &  Did both ``Event$_1$'' and  ``Event$_2$'' happen at year $\langle \iyear \rangle$, where ``Event$_1$'' is $\langle \mtltoNL(\mtl_1) \rangle$ and ``Event$_2$'' is $\langle \mtltoNL(\mtl_2) \rangle$? 
\\ 
\midrule 
\mtltoNL($\mtl_1  \,{\vee}\,  \mtl_2$) &  Did either ``Event$_1$'' or ``Event$_2$''  happen at year $\langle \iyear \rangle$, where ``Event$_1$'' is $\langle \mtltoNL(\mtl_1) \rangle$ and ``Event$_2$'' is $\langle \mtltoNL(\mtl_2) \rangle$? 
\\ 
\midrule 
\mtltoNL($\neg \mtl $) &  Did ``Event'' not happen at year $\langle \iyear \rangle$, where ``Event'' is $\langle \mtltoNL(\mtl) \rangle$? 
\\ 
\bottomrule %添加表格底部粗线
\end{tabular}
\end{table*}




\subsection{Benchmark Construction}
\label{prompt}
%From the derived facts, 
\tool constructs question-answer~(Q\&A) pairs and prompts to facilitate the testing for FCH. 
To address the challenge of the high human efforts required in the test oracle generation, we design an automated approach based on mapping relations between various entities to problem templates, largely reducing reliance on manual efforts. 

\textbf{\emph{Question Generation.}}
%\wkl{rewritten, check}
%\syh{checked, ok}
To facilitate efficient and systematic generation of test cases and prompts, we have adopted a method that leverages entity relationships and mappings of temporal operators to predefined Q\&A templates. 

When constructing relation-based Q\&A templates (without temporal operators), one key aspect lies in aligning various types of relations with the corresponding question templates from the mutated triples, i.e., the predicate type in the triple. Different relation types possess unique characteristics and expressive requirements, leading to various predefined templates. 
As listed in \tabref{table:template}, we map the relation types to question templates based on speech and the grammatical tense of the predicate to guarantee comprehensive coverage. 

When constructing temporal-logic-related queries, we define a mapping pattern for each temporal operator, as outlined in \tabref{table:temporal_template}. For any query expressed as ``$\mtl$''  with any concrete year $\iyear$ in query, the $\mtltoNL(\mtl)$ function converts the MTL formula $\mtl$ into a natural language query. In this context, $\mtltoNL(\mtl')$ is called recursively to generate the natural language description for the CTL subformula $\mtl'$. 



In both mapping patterns, we enhance the construction of the naturally formatted questions by leveraging an LLM to reformulate the structure and grammar of the Q\&A pairs. 

\textbf{\emph{Answer Generation.}}
We note that the answer to the corresponding question is readily attainable from the factual knowledge and the Prolog reasoning rules, defined in both \figref{fig:basic_op_for_predicates} and \figref{fig:encoding_rules_mtl}. 
Primarily, it is easy to determine whether the answer is \emph{true} or \emph{false} based on the mutated triples and the ground-truth time intervals using temporal reasoning. 
Meanwhile, mutated templates with positive and negative semantics via the usage of synonyms or antonyms, which greatly enhance the diversity of the questions, can be treated similarly as the negation rule defined in \figref{fig:basic_op_for_predicates}. 
Specifically, if the answer to a question with original semantics is Yes/No, then for a question with mutated opposite semantics, the corresponding answer would %naturally 
be the opposite, i.e., No/Yes. For example, after obtaining the original Q\&A pair \textit{- ``Is it true that Crohn's disease and Huntington's disease could share similar symptoms and signs? - Yes.''}, we can use antonyms to mutate it into \textit{- ``Is it true that Crohn's disease and Huntington's disease have different symptoms and signs? - No.''}
%\syh{what does this mean?}







% In total, we have defined 60 templates according to the pre-defined rules.
% , from which we have generated 194,850 Q\&A pairs. To prevent the dataset scope from becoming overly extensive, we conduct an initial screening based on categories of reasoning rules, ultimately yielding a total of 14,228 question pairs.
\begin{comment}   
\begin{table*}[!t]
    % \def\arraystretch{1.0}
    \setlength{\tabcolsep}{1ex}
	\centering
        \Large
	% \footnotesize
	\caption{Relation-Template Mapping Pattern.}
 % \lnk{need to refine}}
        \label{table:template}
	\resizebox{\linewidth}{!}{
	\begin{tabular}{l l l}
    \toprule 
    \textbf{Relation} & \textbf{Template} &\textbf{Example} \\
    \midrule
    \textbf{Noun Phrase} & \begin{tabular}[l]{@{}l@{}} - Is it true that \textit{Subject} and \textit{Object} share \textit{Relation}? \\ - \textit{Subject} and \textit{Object} have/made/shared totally different [Relation]. \\Please judge the truthfulness of this statement. \end{tabular} & \begin{tabular}[l]{@{}l@{}} New Triple: \textit{[Crohn's disease, similar\_symptoms\_and\_signs, Huntington's disease]}\\ Template: - Is it true that \textit{Crohn's disease} and \textit{Huntington's disease} share \textit{similar symptoms and signs}? -Yes. \\- Does \textit{Crohn's disease} and \textit{Huntington's disease} have similarities on {symptoms and signs}? - Yes. \end{tabular} \\
    \midrule
    \textbf{Verb Phrase in Passive Voice} & \begin{tabular}[l]{@{}l@{}} - Is it true that \textit{Subject} is/was/are/were \textit{Relation} \textit{Object}? \\ - It is impossible for \textit{Subject} to be \textit{Relation} \textit{Object}. Am I right?\\ Other formats... \end{tabular} & \begin{tabular}[l]{@{}l@{}} New Triple: \textit{[Kuratowski's theorem, not\_proved\_by, Kurt Gödel]}\\ Template: - Is it true that Kuratowski's theorem was proved by Kurt Gödel? - No.\\- Someone else other than Kurt Gödel proved Kuratowski's theorem, am I right? - Yes. \end{tabular} \\
    \midrule
    \textbf{Verb Phrase in Active Voice} & \begin{tabular}[l]{@{}l@{}} - Is it true that \textit{Subject Relation Object}?  \\ - \textit{Subject} \textit{Relation} \textit{Object}. Please judge the truthfulness of this statement.\\ Other formats... \end{tabular} & \begin{tabular}[l]{@{}l@{}} New Triple: \textit{[Baby Don't Lie, appeared\_before, Spark the Fire]}\\ Template: - Is it true that \textit{Baby Don't Lie} appeared before \textit{Spark the Fire}? - Yes.\\- \textit{Baby Don't Lie} never appeared before \textit{Spark the Fire}. \\ Please judge the truthfulness of this statement. -No.   \end{tabular} \\
    % \midrule
    % \textbf{Custom-Designed} & \begin{tabular}[l]{@{}l@{}l@{}l@{}l@{}} Given the \textit{SubjectList}, is it true that \textit{SubjectSelected} is the {Relation} among them?\\ Other formats... \end{tabular} & \begin{tabular}[l]{@{}l@{}l@{}l@{}l@{}} New Triple: [\textit{Roman Holiday, Hindenburg disaster newsreel footage, Lassie Come Home, The White Parade}, \textit{descending\_duration\_order}, \\ \textit{Roman Holiday, Lassie Come Home, The White Parade}]\\ Template: - Given the list [Roman Holiday, Hindenburg disaster newsreel footage, Lassie Come Home, The White Parade],\\ is it true that Roman Holiday have the longest duration among them? - Yes. \end{tabular} \\
    \bottomrule %添加表格底部粗线
\end{tabular}}
\end{table*}
\end{comment}

\textbf{\emph{Prompt Construction.}}
% \wkl{here maybe give a sample prompt screenshot as a figure to illustrate?}
As illustrated in \tabref{table:prompt}, before initiating any interaction with LLMs, we predefine specific instructions and prompts, requesting the model to utilize its inherent knowledge and inferential capabilities to deliver explicit (Yes/No/I don't know) judgments on our queries. Additionally, we instruct the model to present its reasoning process in a template following the judgment. The main goal is to ensure that LLMs give easy-to-understand responses using standardized prompts and instructions. 
This approach also helps the model to effectively exercise its reasoning abilities based on the given instructions and examples.


\begin{table*}[!t]
%\vspace{-0.3cm}
    % \def\arraystretch{1.0}
    \setlength{\tabcolsep}{1ex}
	\centering
    %\large 
	\small
	\caption{Prompt Template. %\shil{Shall we restrict the answer for relation query begnin with Yes/No/I don't know?}
    }
        \label{table:prompt}
        \vspace{-0.1cm}
	%\resizebox{\linewidth}{!}{
	\begin{tabular}{l}
    \toprule 
    \rowcolor{mycolor}
    \textbf{\instruction:} Answer the question with your knowledge and reasoning power.\\
    \midrule
    \rowcolor{mycolor} \textbf{\query (Relation):}  Given the $\langle \textit{question} \rangle$: \textit{question}, please provide an answer with your knowledge and reasoning power.\\ 
    \rowcolor{mycolor} Think of it step by step with a human-like reasoning process. After giving the answer, list the knowledge used in your\\ 
    \rowcolor{mycolor} reasoning process in the form of declarative sentences and point by point. The answer must contain `Yes', `No' or `I \\
    \rowcolor{mycolor} don't know' at the beginning. \\
    \midrule
    \rowcolor{mycolor} \textbf{\query (Temporal):}  Given the question: 
    $\langle \textit{question} \rangle$, please provide an answer with your knowledge and reasoning power \\
    \rowcolor{mycolor}  upon metric temporal logic. Think it step by step with a human-like reasoning process. After giving the answer, list the \\
    \rowcolor{mycolor} evidence from your temporal reasoning  in the form of declarative sentences and point by point. The answer must contain   \\
\rowcolor{mycolor} `Yes', `No' or `I don't know' at the beginning.\\
    \bottomrule %添加表格底部粗线
\end{tabular}
\end{table*}




\vspace{-1mm}
\subsection{Response Evaluation}\label{response}
%
%\begin{lstlisting}
% \begin{align*}
%     <program>::=&<decoder>|<query>|<model>|<condition>|<distribution>\\&|<query*>|<parser> \\
%     <query*>::=&statement(transform(s,R,o))\\
%     <parser>::=&extract(statement)\\
%     <transform>::=&[Neg]|[Sym]|[Inverse]|[Trans]|[Comp]
% \end{align*}
% \begingroup\vspace*{-1cm}
% \captionof{figure}{Syntax of Extended LMQL.}\label{sec:syntax}
% \vspace*{\baselineskip}\endgroup
%\end{lstlisting}
%
% \begin{lstlisting}[language=Python, caption=LMQL Program Grammar]
% <decoder> ::= argmax | beam(n=<int>) | sample(n=<int>)
% <query> ::= (<python_statement>)+
% <cond> ::= <cond> and <cond> | <cond> or <cond> | not <cond> | <cond_term>
% <cond_term> ::= <python_expression>
% <cond_op> ::= < | > | = | in
% <dist> ::= <var> over <python_expression>
% \end{lstlisting}
% \begin{grammar}
% <LMQL Program> ::= <decoder> <query>

% <decoder> ::= `argmax' | `beam(n=\textit{int})' | `sample(n=\textit{int})'

% <query> ::= (<python\_statement>)+

% <cond> ::= <cond> `and' <cond> | <cond> `or' <cond> | `not' <cond> | <cond\_term>

% <cond\_term> ::= <python\_expression>

% <cond\_op> ::= `<` | `>' | `=' | `in'

% <dist> ::= <var> `over' <python\_expression>
% \end{grammar}

% To facilitate the automated the query and answer validation process, we extend the previously proposed LMQL~(language Model Query Language)~\cite{Beurer-Kellner-2023} designed for LLM programming. LMQL utilizes SQL-like elements and a imperative syntax for scripting, as shown in Figure~\ref{sec:syntax}. More specifically, LMQL defines the interactive process with an LLM as a python-like $<program>$, including a $<decoder>$ to the decoding procedure employed by the LMQL runtime when solving the
% query, a $<query>$ to model the interaction with the model, a $<model>$ to denote the LLM to interact with, a $<condition>$ to place constraints on the variables in the program, a $<distribution>$ to represent the probability for output predictions from the LLM. language is augmented with additional constructs to facilitate the interaction and generation capabilities of LLMs.

% Firstly, we introduce the <query*> element as an extension to the existing <query> element. The <query*> block models the interaction with the LLM, serving as the prompt that is fed into the model. These query strings allow for the use of specially escaped subfields, similar to Python's f-strings. These subfields, denoted by "[varname]", represent phrases that will be generated by the LLM, also known as holes.

% Furthermore, we introduce a new element named <parser>. The <parser> element is responsible for extracting triples (subject-predicate-object) from the LLM-generated answers for further semantic comparison. The <parser> is capable of processing one or multiple sentences, extracting triples from each sentence sequentially and recording the derived results for further use. The <parser> element plays a crucial role in the answer validation process by breaking down the LLM's generated responses into structured triples. These triples can then be compared against a knowledge base or a set of predefined rules to validate the semantic correctness and coherence of the generated content.


The objective of this module is to enhance the detection of FCH in LLM outputs, specifically focusing on identifying the discrepancies between LLM responses and the verified ground truths. Recognizing the inherent limitation in directly accepting ``Yes'' or ``No'' answers from LLMs, our approach underscores the automated detection of factual consistency during the reasoning process presented by LLMs. 
% This analysis is vital for accurately determining the factual consistency of LLM responses, thereby addressing the primary challenge in identifying FCH within LLM outputs. 
% To achieve automated detection of factual consistency, our methodology first incorporates a parsing step that leverages advanced NLP techniques. This step is designed to extract essential semantic elements from each sentence within LLM outputs, assembling these elements into a coherent, semantic-aware structure. 
% The foundational premise of our approach is predicated on evaluating the semantic similarity between these constructed structures, aiming to discern the degree of consistency in their underlying semantics.
% Subsequently, we propose the development of a set of similarity-based testing oracles. These oracles are instrumental in applying metamorphic testing principles, enabling us to systematically assess the consistency or inconsistency between LLM responses and the established ground truth. 
Our approach is structured around several critical steps, as listed in \algoref{alg:eval} and detailed below:
% \shil{(SL: to cite Algorithm 3 and when explanation, referring to line number? For lines 11 to 15, use ``or" to include three conditions?)}
% The key target of this module is to detect the FCH by identifying the inconsistency between answers from the LLMs and the groundtruth in Q\&A pairs. However, as we unable to directly trust the yes or no answers from LLM directly, we need to parse the reasoning process carefully before reaching the verdict on the correctness with accuracy, which is the aforementioned challenges for detecting FCH in LLMs.  
% %This module outlines our approach for detecting hallucinations in the responses of the target LLMs. A key insight is the premise that any response contradicting the answer of the factual Q\&A pairs we provide is inherently regarded as an occurrence of FCH.
% To facilitate the automatic detection, we first design a parsing step to utilize an NLP-based approach to extract the critical semantic component from each sentence from LLM outputs, and assemble them into a semantic-aware structure. The key insight is to examine the similarity between these structures to determine the consistency in their semantics. Then, by designing a set of similarity-based testing oracles, we are able to utilize metamorphic testing to determine the (in)consistency between LLM answers and groundtruth.  This method primarily comprises the following steps:
%During the aforementioned prompt design process, we have prepared question prompts and their corresponding answers, thereby establishing ground truth Q\&A pairs. Therefore, we can automatically compare the LLM output with the expected ground truth answer to detect the discrepancies. Moreover, we utilize an NLP-based approach to compare the semantics in the reasoning process to identify inconsistencies to assist in understanding the cause of FCH. %our similarity between the responses from LLMs with these ground truth Q\&A pairs to verify the consistency and logical soundness of the LLMs' responses. 
% Detailed conclusions are discussed in the following section.


%In summary, we propose a method based on semantic parsing and metamorphic relations to verify whether LLM responses contain FCHs. This method primarily comprises the following steps:

%labelwidth=!,
\begin{enumerate}[wide,  labelindent=9pt]
%Step 1. 
\item \textbf{Preliminary Screening.} Given the LLM response $\llmResponse$, we first eliminate scenarios when the LLM declines to provide an answer, indicated by the ``answer'' field of LLM's responses. 
Most of these responses arise because the LLM lacks the relevant knowledge for the reasoning process. As these responses adhere to the LLM's principle of honesty, we categorize them as correct and normal responses.
% (as described in Algorithm~\ref{alg:eval} Line 7-8). 

%Step 2. 
\item \textbf{Response Parsing and Semantic Structure Construction.} Provided with the remaining suspicious responses from $\llmResponse$ and ground truth facts $\groundTruthTriples$, we use \textsc{ExtractTriple} function to extract triples that follow the same structure as the fact defined in the \secref{subsec3.1}. For each LLM response, the extracted triples ($\widetilde{\m{Trpl}}$) are based on the statements contained in the \textit{reasoning process} part of the LLM's response, which is further utilized to construct a response semantic structure $\semantic_{\m{resp}}$ using the \textsc{BuildGraph} function. In this structure, the $\widetilde{\entity}$ are depicted as \emph{nodes} ($\m{N}$), and the relational predicate ($\nm$) between them are illustrated as \emph{edges} ($\m{E}$). Concurrently using the same approach, we construct another semantic structure $\semantic_{\m{ground}}$ using $\groundTruthTriples$.

%Step 3. 
\item \textbf{Similarity-based Metamorphic Testing and Oracles.} 
We apply metamorphic relations to identify hallucination answers from LLMs, i.e., comparing the similarity between semantic structures generated by LLMs and the ground truth counterparts. Note that we provide four classifications: correct responses ($\m{CO}$), hallucinations caused by error inference ($\m{EI}$), hallucinations caused by erroneous knowledge ($\m{EK}$), and hallucinations containing both issues ($\m{OL}$). 
Specifically, the oracles for metamorphic testing can be divided into the following types:
 
% We then apply metamorphic relations to detect and evaluate potential errors in LLM responses, based on the relationships between inputs and outputs, rather than relying on traditional labeled data. 
% In our context, metamorphic relations specifically refer to comparing the similarity between semantic structures generated by LLMs and the ground truth counterparts, to identify and classify hallucination answers from LLMs.
% (as mentioned in Algorithm~\ref{alg:eval} Line 12-18). 
\end{enumerate}

\begin{comment}

tree = Leaf() | Node ()

resp{
    answer = bool 
    steps = tree ??
}

ground_truth{
    answer = bool 
    reasoning = tree
}


evaluation (resp, ground_truth)
    if resp.answer = refusal then CO 
    else 
        s_e = SE(resp, R_derived)
        s_n = SN(resp, R_derived)
        if s_e < threadhold_e then EI 
        else if s_n < threadhold_n then EK 
        else CO 


    
\end{comment}


% \lnk{check this algo}
% \begin{algorithm}[!h]
% \caption{Response Evaluation}
% \label{alg:eval}
% \small
% \begin{algorithmic}[1]
% \Require LLM Response ($\llmResponse$), Ground Facts ($\groundTruthTriples$), Threshold ($\theta_{\m{e}}, \theta_{\m{n}}$)
% \Ensure Evaluation Result ($\eval$)
% \Function{EvaluateResponse}{$\llmResponse$, $\groundTruthTriples$, $\theta_{\m{e}}$, $\theta_{\m{n}}$}
%     % \State $hallu\_ei, hallu\_ek, hallu\_both \gets$ [] \Comment{\commentstyle{Initialization}}
%     % \State $\eval, \eval_{\m{ei}}, \eval_{\m{ek}}, \eval_{\m{co}} \gets$ [] \Comment{\commentstyle{Initialization}}
%     % \State $refuse\_to\_answer \gets$ \Call{FindRefuseToAnswer}{$\llmResponse$} \Comment{\commentstyle{Find `refuse to answer' responses}}
%     % \State $suspicious\_resps \gets$ \Call{FilterSuspiciousRes}{$\llmResponse$, $GT\_Answer$} \Comment{\commentstyle{Filter suspicious responses}}
%     % \For{$\m{resp}$ in $\llmResponse$} \Comment{\commentstyle{Iterate each response}}
%     \If{$\llmResponse.answer = \m{refusal}$}
%         \State
%         $\eval$ $\in$ $CO$ \Comment{\commentstyle{Preliminary Screening}}
%     % \State $\eval_{\m{co}}$.append($\m{resp_\m{refusal}}$) \Comment{\commentstyle{Preliminary Screening}}
%     \Else
%         \State $\deriveKG{\m{resp}}{\groundTruthTriples}{\semantic_{\m{resp}}}{\semantic_{\m{ground}}}$
%         %\Comment{\commentstyle{Extract Semantic Structure}}
%         \State $\m{s}_{\m{e}} \gets$ $\similarity_\m{e}${$(\semantic_{\m{resp}}$, $\semantic_{\m{ground}})$} \Comment{\commentstyle{Calculate edge similarity}}
%         \State $\m{s}_{\m{n}} \gets$ $\similarity_\m{n}${$(\semantic_{\m{resp}}$, $\semantic_{\m{ground}})$} \Comment{\commentstyle{Calculate node similarity}}
%         % \If{$edge\_sim < \theta\_e$ and $node\_sim < \theta\_n$}
%             % \State $\eval$.append($response$) \Comment{\commentstyle{Append mixed hallucination}}
%         \If{$\m{s}_{\m{e}} < \theta_{\m{e}}$}
%             \State 
%             $\eval$ $\in$ $EI$  \Comment{\commentstyle{Append error inference hallucination}}
%         \ElsIf{$\m{s}_{\m{n}} < \theta_{\m{n}}$}
%             \State 
%             $\eval$ $\in$ $EK$  \Comment{\commentstyle{Append error knowledge hallucination}}
%         \Else
%             \State
%             $\eval$ $\in$ $CO$  
%             \Comment{\commentstyle{Append correct response}}
%         \EndIf
%     \EndIf
%     % \EndFor
%     % \State $\eval$.extend($\eval_{\m{ei}}, \eval_{\m{ek}}, \eval_{\m{co}}$) \Comment{\commentstyle{Merge the result}}
%     % \State $evaluation\_result \gets$ \Call{GenerateResult}{$hallu\_both$, $hallu\_ei$, $hallu\_ek$}
%     % \shil{(we use three lists \_both, \_ei, and \_ek)} $contradictory\_answers$} 
%     % \Comment{Generate evaluation result}
%     \State \Return $\eval$ \Comment{\commentstyle{Return the evaluation result}}
% \EndFunction
% \end{algorithmic}
% \end{algorithm}
\begin{algorithm}[!b]
\caption{Response Evaluation}
\label{alg:eval}
\small
\begin{algorithmic}[1]
\Require LLM Response ($\llmResponse$), Ground Facts ($\groundTruthTriples$), Threshold ($\theta_{\m{e}}, \theta_{\m{n}}$)
\Ensure Evaluation Result Category~($CO, EK, EI, OL$)
\Function{EvaluateResponse}{$\llmResponse$, $\groundTruthTriples$, $\theta_{\m{e}}$, $\theta_{\m{n}}$}
    % \State $hallu\_ei, hallu\_ek, hallu\_both \gets$ [] \Comment{\commentstyle{Initialization}}
    % \State $\eval, \eval_{\m{ei}}, \eval_{\m{ek}}, \eval_{\m{co}} \gets$ [] \Comment{\commentstyle{Initialization}}
    % \State $refuse\_to\_answer \gets$ \Call{FindRefuseToAnswer}{$\llmResponse$} \Comment{\commentstyle{Find `refuse to answer' responses}}
    % \State $suspicious\_resps \gets$ \Call{FilterSuspiciousRes}{$\llmResponse$, $GT\_Answer$} \Comment{\commentstyle{Filter suspicious responses}}
    % \For{$\m{resp}$ in $\llmResponse$} \Comment{\commentstyle{Iterate each response}}
    \State $CO, EK, EI, OL \gets []$ \Comment{\commentstyle{Initialization}}
    \If{$\llmResponse.answer = refusal$}
        \State
        $CO.\m{append}(\llmResponse)$ \Comment{\commentstyle{Preliminary Screening}}
    % \State $\eval_{\m{co}}$.append($\m{resp_\m{refusal}}$) \Comment{\commentstyle{Preliminary Screening}}
    \Else
        \State $\widetilde{\m{Trpl}} \gets$ \Call{ExtractTriple}{$\m{Resp.reasoning}$} 
        % \Comment{\commentstyle{Extract useful triples}}
        \State $\semantic_{\m{resp}}, \semantic_{\m{ground}} \gets$ \Call{BuildGraph}{$\widetilde{\m{Trpl}}, \groundTruthTriples$} 
        % \Comment{\commentstyle{Build semantic structure}}
        % \State $\deriveKG{\m{Resp}}{\rall}{\semantic_{\m{resp}}}{\semantic_{\m{ground}}}$\lnk{More specific} \Comment{\commentstyle{Extract Semantic Structure}}
        \State $\m{s}_{\m{e}} \gets$ $\similarity_\m{e}${$(\semantic_{\m{resp}}$, $\semantic_{\m{ground}})$} \Comment{\commentstyle{Calculate edge similarity}}
        \State $\m{s}_{\m{n}} \gets$ $\similarity_\m{n}${$(\semantic_{\m{resp}}$, $\semantic_{\m{ground}})$} \Comment{\commentstyle{Calculate node similarity}}
        % \If{$edge\_sim < \theta\_e$ and $node\_sim < \theta\_n$}
            % \State $\eval$.append($response$) \Comment{\commentstyle{Append mixed hallucination}}
        \If {$s_e < \theta_{e}$ and $s_n < \theta_{n}$}
            \State
            $OL.\m{append}(\llmResponse)$  \Comment{\commentstyle{Append  overlapped cases}}
        \ElsIf{$\m{s}_{\m{e}} < \theta_{\m{e}}$}
            \State 
            $EI.\m{append}(\llmResponse)$  \Comment{\commentstyle{Append error inference}}
        \ElsIf{$\m{s}_{\m{n}} < \theta_{\m{n}}$}
            \State 
            $EK.\m{append}(\llmResponse)$  \Comment{\commentstyle{Append error knowledge}}
        \Else
            \State
            $CO.\m{append}(\llmResponse)$
            \Comment{\commentstyle{Append correct response}}
        \EndIf
    \EndIf
    % \EndFor
    % \State $\eval$.extend($\eval_{\m{ei}}, \eval_{\m{ek}}, \eval_{\m{co}}$) \Comment{\commentstyle{Merge the result}}
    % \State $evaluation\_result \gets$ \Call{GenerateResult}{$hallu\_both$, $hallu\_ei$, $hallu\_ek$}
    % \shil{(we use three lists \_both, \_ei, and \_ek)} $contradictory\_answers$} 
    % \Comment{Generate evaluation result}
    \State \Return $CO, EK, EI, OL$ 
    % \Comment{\commentstyle{Return the result}}
\EndFunction
\end{algorithmic}
\end{algorithm}

\textbf{Edge Vector Metamorphic Oracle ($MO_E$)}: This oracle is based on the similarity of edge vectors between $\semantic_{\m{resp}}$ and $\semantic_{\m{ground}}$. If the vector similarity ($\m{s}_{\m{e}}$) between the edges in the $\semantic_{\m{resp}}$ and those in $\semantic_{\m{ground}}$ falls below a predetermined threshold $\theta_{\m{e}}$, it indicates that the LLM's answer significantly diverges from the ground truth. This suggests the presence of an FCH, and vice versa. %Conversely, the LLM's answer is considered to {align with} the ground truth. % indicates the correct answer. Otherwise, it detects an occurrence of FCH.
More specifically, we utilize \emph{Jaccard Similarity}~\cite{J_S} to calculate the similarity score between edge vectors extracted from $\semantic_{\m{resp}}$ and  $\semantic_{\m{ground}}$. 
$$
\similarity_{\m{e}}(\semantic_{\m{resp}}, \semantic_{\m{ground}}) = \frac{|\widetilde{E}_{\m{resp}} \cap \widetilde{E}_{\m{ground}}|}{|\widetilde{E}_{\m{resp}} \cup \widetilde{E}_{\m{ground}}|}, $$check if $  \similarity_{\m{e}}(\semantic_{\m{resp}}, \semantic_{\m{ground}})  < \theta_{\m{e}} \ 
$~
where $\widetilde{E}_{\m{resp}}$ and $\widetilde{E}_{\m{ground}}$ denote the set of edges extracted from $\semantic_{\m{resp}}$ and $\semantic_{\m{ground}}$. 
% , and \( \theta_E \) is a predefined threshold~(to be detailed in Section~\ref{sec:ex_setup}). 
% Intuitively, the similarity score is calculated as the proportion of identical edges shared between the two sets against the total number of unique edges in both sets. If the similarity score is smaller than the threshold, then an FCH is detected. Note that when determining the joint and union of sets $E_{LLM}$ and $E_{GT}$, we consider two edges as identical if their corresponding relations are identical or represented by synonymous words, and vice versa.  

% Define a function \( \text{Sim}_E(KG_{LLM}, KG_{GT}) \) to calculate the similarity of edge vectors between the knowledge graph generated by the language model, \( KG_{LLM} \), and the ground truth knowledge graph, \( KG_{GT} \).
% $$\text{Sim}_E(KG_{LLM}, KG_{GT}) = \frac{|E_{\text{LLM}} \cap E_{\text{GT}}|}{|E_{\text{LLM}} \cup E_{\text{GT}}|}$$

% If \( \text{Sim}_E(KG_{LLM}, KG_{GT}) < \theta_E \), where \( \theta_E \) is a predefined threshold, then an error inference hallucination is identified.

   
\textbf{Node Vector Metamorphic Oracle ($MO_N$)}: This relation examines the similarity of node vectors between $\semantic_{\m{resp}}$ and $\semantic_{\m{ground}}$. 
Defined in a similar manner as $MO_E$, if the node similarity between the nodes ($\m{s}_{\m{n}}$) in the $\semantic_{\m{resp}}$ and those in $\semantic_{\m{ground}}$ falls below a predetermined threshold $\theta_{\m{n}}$, it indicates that the LLM's answer significantly diverges from the ground truth, and vice versa.
$MO_N$ can be captured by the Jaccard Similarity, defined as follows:
%When the similarity between the nodes in the $KG_{LLM}$ and those in $KG_{GT}$ is below a predetermined threshold, this metamorphic relation exposes an error knowledge hallucination.
%Define a function \( \text{Sim}_N(KG_{LLM}, KG_{GT}) \) to measure the similarity of node vectors between \( KG_{LLM} \) and \( KG_{GT} \).

$$\similarity_{\m{n}}(\semantic_{\m{resp}}, \semantic_{\m{ground}}) = \frac{|N_{\m{resp}} \cap N_{\m{ground}}|}{|N_{\m{resp}} \cup N_{\m{ground}}|}, $$check if $
\similarity_{\m{n}}(\semantic_{\m{resp}}, \semantic_{\m{ground}})  < \theta_{\m{n}}  
$~
where $N_{\m{resp}}$ and $N_{\m{ground}}$ denotes the set of nodes extracted from $\semantic_{\m{resp}}$ and $\semantic_{\m{ground}}$.
% , and \( \theta_N \) is a predefined threshold~(to be detailed in Section~\ref{sec:ex_setup}). 
% Intuitively, the similarity score is calculated as the proportion of identical edges/nodes shared between the two sets against the total number of unique edges/nodes in both sets. If the similarity score is smaller than the threshold, then an FCH is detected. 
Note that when determining the joint and union of the edges/nodes sets, we consider two edges/nodes as identical if their corresponding entities are identical or synonymous, and vice versa.
%If \( \text{Sim}_N(KG_{LLM}, KG_{GT}) < \theta_N \), where \( \theta_N \) is a predetermined threshold, then an error knowledge hallucination is recognized.


% \textbf{Answer Consistency Metamorphic Oracle ($MO_C$)}: This relation is distinct in that it focuses on the consistency or inconsistency of the model's final answer with the ground truth, regardless of whether the node or edge vector similarities meet the thresholds. This relation helps identify situations where, despite a seemingly reasonable reasoning process, the outcome contradicts the facts (or vice versa), indicating contradictory answers.

% Consider the final answer \( Ans_{LLM} \) provided by LLMs and the ground truth answer \( Ans_{GT} \).
% If the similarity between \( KG_{LLM} \) and \( KG_{GT} \) is above or below the threshold but there exists a contradiction or consistency between \( Ans_{LLM} \) and \( Ans_{GT} \), this scenario is considered as a contradictory answer.
% $$ \text{If similarity between } KG_{LLM} \text{ and } KG_{GT} \text{ is above or below threshold and } A_{LLM} \neq A_{GT} \text{, then identify a contradictory answer.} $$


% \subsection{Feedback Loop}
% Based on the evaluation results from the preceding section, this module is employed to select test case types that trigger higher levels of FCHs and to mutate them for more hallucination answers, thereby enhancing the ability of the testing process to expose LLM FCHs.


\section{Experiments}
\subsection{Experiment Setup} 
\paragraph{Models.}

% \begin{table*}[htbp]
% \newcolumntype{g}{>{\columncolor{green!10}}c}
% \newcolumntype{b}{>{\columncolor{blue!10}}c}
% \renewcommand{\arraystretch}{1.22} % Adjust row spacing
% \small
% \resizebox{0.95\textwidth}{!}{
% \begin{tabular}{llcccc}

% \toprule
% UltraBench Split & Model & Overall Score & Soft Score & Hard Score & BLEU  \\ \midrule

% FineWeb  & Base Model  & 45.11 & 67.71 & 38.34 & 5.8 \\
% % \ \ Easy (66 samples)                      &         & 59.68 & 76.01 & 0.23 & 48.23 & 0.053     \\
% % \ \ Medium (76 samples)                    &         & 44.64 & 66.45 & 0.11 & 39.78 & 0.060     \\
% % \ \ Hard (58 samples)                      &         & 29.14 & 59.95 & 0.00 & 25.20 & 0.049     \\ \midrule

% & SFT Model  & 58.63 & 83.18  & 51.39 & 9.6 \\
% % \ \ Easy                      &         & 69.11 & 88.56 & 0.53 & 56.84 & 0.074    \\
% % \ \ Medium                    &         & 61.45 & 85.92 & 0.39 & 56.06 & 0.098     \\
% % \ \ Hard                       &         & 43.00 & 73.45 & 0.17 & 39.07 & 0.089     \\ \midrule

% & Llama-3.2-3B-DPO &  65.87 & 82.25 & 60.88 & 9.1  \\
% % \ \ Easy                       &         & 74.66 & 89.60 & 0.61 & 65.26 & 0.063    \\
% % \ \ Medium                   &         & 68.38 & 81.28 & 0.25 & 65.57 & 0.092  \\
% % \ \ Hard                       &         & 52.60 & 75.14 & 0.16 & 49.73 & 0.096   \\
% \midrule
% Global    & Llama-3.2-3B-Instruct\textsubscript{BASE}   & 36.85          & 43.95              & 35.23    & -         \\
% & Llama-3.2-3B-Instruct\textsubscript{SFT}    & 42.27          & 54.57               & 38.50     & -         \\
% & Llama-3.2-3B-Instruct\textsubscript{DPO}     & 63.84          & 57.84               & 64.86   & -           \\ 
% \bottomrule

% \end{tabular}
% }
% \caption{Performance scores for Llama-3.2-3B-Instruct models under different evaluation conditions.}
% \label{tab:ultrabench}
% \end{table*}

\begin{table*}[htbp]
\newcolumntype{g}{>{\columncolor{green!10}}c}
\newcolumntype{b}{>{\columncolor{blue!10}}c}
\renewcommand{\arraystretch}{1.22} % Adjust row spacing
\small
\resizebox{\textwidth}{!}
{
\begin{tabular}{llccccccc}

\toprule
&  \multirow{2}*{Model} & \multicolumn{4}{c}{\textbf{FineWeb Split}} & \multicolumn{3}{c}{\textbf{Multi-source Split}} \\
\cmidrule(l){3-6} \cmidrule(l){7-9} 
& & Overall Score & Soft Score & Hard Score & BERTScore F1    & Overall Score & Soft Score & Hard Score   \\ 
 
 \midrule

 \multirow{3}*{\rotatebox{90}{Main}}& Base Model  & 50.30 & 67.08 & 33.51 & 59.92  & 37.45       & 36.10            & 38.79            \\
% \hdashline

& UltraGen (AR)  & 56.05 & 81.44  & 30.65 & 62.00    & 50.15         & 62.41               & 37.89            \\
& UltraGen (AR+GPO) &  59.61 & 84.33 & 34.89 & 61.22    &  57.23       & 69.01               & 45.44            \\ 
\midrule
% \hdashline
\multirow{4}*{\rotatebox{90}{Ablation}} & AR (Few Constraints) & 48.25 & 74.09 & 22.41 & 60.10    & 38.38         &  46.00        & 30.76             \\
& GPO & 55.57 & 74.50 & 36.63 & 60.59 & 42.44 & 51.00 & 33.86 \\
& AR+GPO (Random Sampling) &  59.77 & 85.42 & 34.11 & 60.56 & 55.24         & 68.01            & 42.47            \\ 
& AR+GPO (High Similarity) &  59.44 & 83.22 & 35.65 & 60.85 & 55.45        & 66.05               & 44.85             \\ 
& AR+GPO (Low Correlation) &  58.91 & 83.59 & 34.23 & 60.00 & 54.47         & 65.22               & 43.71             \\ 

\bottomrule

\end{tabular}
}
\caption{Performance scores for Llama-3.2-3B-Instruct models on the validation set under different evaluation conditions across FineWeb and Global splits.}
\label{tab:ultrabench}
\end{table*}

% \begin{table}[h!]
\centering
\caption{Performance Comparison for Different Levels}
\resizebox{0.5\textwidth}{!}{
\begin{tabular}{@{}lccc@{}}
\toprule
\textbf{Category} & \textbf{Score} & \textbf{Soft Score} & \textbf{Hard Score} \\ \midrule
\multicolumn{4}{c}{\textbf{Overall Scores}} \\
Llama-3.2-3B-Instruct\textsubscript{BASE}       & 36.85          & 43.95              & 35.23            \\
Llama-3.2-3B-Instruct\textsubscript{SFT}         & 42.27          & 54.57               & 38.50              \\
Llama-3.2-3B-Instruct\textsubscript{DPO}         & 63.84          & 57.84               & 64.86             \\ \midrule
% \multicolumn{4}{c}{\textbf{Easy (47 Samples)}} \\
% Llama-3.2-3B-Instruct\textsubscript{BASE}         & 49.89          & 58.16               & 47.63              \\
% Llama-3.2-3B-Instruct\textsubscript{SFT}         & 51.65          & 66.38               & 42.82              \\
% Llama-3.2-3B-Instruct\textsubscript{DPO}         & 51.26          & 62.31               & 44.05              \\ \midrule
% \multicolumn{4}{c}{\textbf{Medium (55 Samples)}} \\
% Llama-3.2-3B-Instruct\textsubscript{BASE}         & 45.53         & 53.48               & 39.60              \\
% Llama-3.2-3B-Instruct\textsubscript{SFT}         & 43.92          & 53.53               & 32.14              \\
% Llama-3.2-3B-Instruct\textsubscript{DPO}         & 37.47          & 57.14               & 35.16              \\ \midrule
% \multicolumn{4}{c}{\textbf{Hard (98 Samples)}} \\
% Llama-3.2-3B-Instruct\textsubscript{BASE}         & 16.65          & 30.72               & 15.73              \\
% Llama-3.2-3B-Instruct\textsubscript{SFT}         & 25.30          & 42.14               & 24.40              \\
% Llama-3.2-3B-Instruct\textsubscript{DPO}         & 24.31          & 37.72               & 23.59              \\ \bottomrule
\end{tabular}
}
\label{table:global_performance_comparison}
\end{table}
Our experiments evaluate the EFCG task using one mainstream instruction-tuned base model: Llama-3.2-3B-Instruct ~\cite{dubey2024llama}, chosen for its demonstrated proficiency in instruction-following tasks within the 3B parameter range. To systematically assess the impact of our methodology, we compare three training paradigms: (1) \textbf{BASE}, which directly employs the unmodified base models to establish a performance baseline; (2) \textbf{AR}, where models undergo the auto-reconstruction stage on our meticulously constructed FineWeb dataset (§3.2), enriched with fine-grained attributes to enhance multi-constraint adherence; and (3) \textbf{AR+GPO}, a hybrid optimization approach combining direct preference optimization with global embedding space adaption.

\subsection{Evaluation Results on UltraBench}

Our experimental findings, summarized in Table \ref{tab:ultrabench}, demonstrate the substantial advancements achieved by applying the UltraGen paradigm to EFCG. The evaluation leverages the validation set of FineWeb and Global splits to assess model performance under both local and global constraints.

The application of AR yielded significant improvements over the base model. On the FineWeb split, the AR model attained an overall score of 56.05, representing a relative improvement of 11.4\%. The soft score rose to 81.44, indicating enhanced adherence to semantic and stylistic attributes, while the hard score increased to 30.65, reflecting better performance on programmatically verifiable constraints. On the Global split, the AR model demonstrated its ability to generalize, achieving an overall score of 50.15.

Further optimization through GPO demonstrated remarkable performance on the Global split, where the model achieved an overall score of 57.23 and an impressive hard score of 45.44. This highlights the model's robust generalization and optimization capabilities when dealing with diverse and challenging global constraints. Notably, despite being trained on the Global split, the AR+GPO model exhibited strong performance on the FineWeb split as well, achieving an overall score of 59.61, a soft score of 84.33, and a hard score of 34.89. This result underscores the model's ability to transfer its learned capabilities from the broader and more diverse Global split to the more localized FineWeb split.

\paragraph{Ablation}
To evaluate the contribution of key components in our UltraGen framework, we conducted ablation studies by systematically modifying the training process. We tested the impact of reducing the number of attributes during AR, removing the AR stage, replacing curated attributes with random sampling, and eliminating the high-correlation or low-redundancy selection steps. The results demonstrate that both AR and GPO stages are crucial for achieving strong performance, as reducing constraints, removing correlation modeling, or neglecting redundancy minimization leads to performance degradation.
% \paragraph{Ablation}
% To evaluate the contribution of key components in our UltraGen framework, we conducted several ablation studies by systematically modifying the training process. The following ablations were performed:
% \begin{enumerate}
%     \item \textbf{SFT with limited attributes}: To examine the impact of attribute numbers during the supervised fine-tuning stage, we trained an SFT model using a reduced set of attributes (fewer than 10 per sample).
%     \item \textbf{DPO only}: We directly train the DPO on the global split without SFT stage.
%     \item \textbf{SFT + DPO random sampling}: In this ablation, we replaced the curated high correlation and low redundancy attribute combinations with random sampling during the RL stage. 
%     \item \textbf{SFT + DPO w/o high correlation}: This experiment removed the attribute correlation modeling step, where attributes with strong relationships were prioritized.
%     \item \textbf{SFT + DPO w/o low redundancy}: In this setup, we did not enforce diversity in attribute sets by minimizing semantic redundancy.
% \end{enumerate}
% The ablation study shows that SFT with fewer constraints significantly underperforms the standard SFT. And DPO variants with fewer constraints, random sampling, or reduced correlation emphasize the importance of optimized attribute selection in the global space.
\subsection{Data Synthesis Improvement}

\begin{table}[htbp]
\centering
\small
\resizebox{0.48\textwidth}{!}{
\begin{tabular}{lccc}
\toprule
\textbf{Dataset (Domain)} & \textbf{Base} &  \textbf{AR} & \textbf{AR+GPO} \\ \midrule
Emotion (Tweet Emotion) & 28.25 & \textbf{42.30}  & 38.65 \\
Hillary (Tweet Stance)  & 55.93  & 45.76 & \textbf{58.31} \\
AG-News (News Topic) & 80.03 & 79.96 &\textbf{83.28} \\
TREC (Question Type) & 38.00  & 51.20  & \textbf{51.40} \\ 
\midrule
Average   & 50.55 & 54.81 & \textbf{57.91} \\
\bottomrule
\end{tabular}
}
\caption{Performance comparison for data synthesis.}
\vspace{-1em}
\label{tab:data_synthesis}
\end{table}

To demonstrate the improvement in the usage of texts synthesized by UltraGen, we utilize several diverse well-established text classification benchmarks to test the data synthesis capability, such as sentiment analysis \textbf{(1) Emotion} ~\cite{saravia-etal-2018-carer}, attitude classification towards a particular public figure \textbf{(2) Hillary} ~\cite{barbieri2020tweeteval}, topic classification \textbf{(3) AG News} ~\cite{Zhang2015CharacterlevelCN}, question type classification \textbf{(4) TREC} ~\cite{li-roth-2002-learning}.

For each dataset, we analyze the unique properties and paraphrase these properties as hard and soft attributes. Then using a uniform prompt tailored for each dataset, we generate 2,000 synthetic samples per dataset. These generated samples are then used to train a classifier, which is subsequently evaluated on the original test set of the dataset. This procedure allows for a fair comparison of model performance on synthetic data. 

The results, summarized in Table \ref{tab:data_synthesis}, demonstrate the superior generalization ability of the AR+GPO model trained on the Global split. Notably, the AR+GPO model achieved the highest average score of 57.91 across the benchmarks, significantly outperforming both the base model and the AR models. While the AR model’s performance stagnated (45.76, lower than the original one) on the Hillary benchmark, reflecting a focus on localized attributes, the AR+GPO model excelled with a score of 58.31, indicating its generalization and adaptability beyond localized training objectives.

\subsection{Trade-Offs in EFCG}
\begin{figure}[t]
    \centering
        \includegraphics[width=0.49\textwidth]{figs/tradeoff.pdf}
    \caption{The Trade-off between F1 score and CSR. While BERTScore tends to improve with more attributes, CSR declines}
    \vspace{-1.5em}
    \label{fig:tradeoff}
\end{figure}

Figure~\ref{fig:tradeoff} illustrates the interplay between BERTScore and CSR across different numbers of attributes from 10 to 50 for each model. As the figure shows, increasing the number of attributes presents a clear double-edged effect: while more attributes can enhance fine-grained control (e.g., higher F1 score) over the generated text, the added complexity makes it more difficult for the model to maintain high constraint adherence.

\paragraph{Better Multi-Objective Alignment Under EFCG.}
\begin{figure*}[htbp]
    \centering
        \includegraphics[width=0.98\textwidth]{figs/case_study.pdf}
    \caption{In a case study on travel itinerary generation, the attention flow illustrates improved constraint awareness in AR+GPO.}
    \vspace{-1em}
    \label{fig:case_study}
\end{figure*}

When looking at the 30, 40, and 50 attribute conditions:
AR+GPO consistently attains CSR values 5--10 points higher than the other two models without sacrificing F1.
For example, at 50 attributes, AR+GPO’s CSR (44.76\%) is considerably above AR’s (35.86\%) and Original’s (37.40\%), while also delivering the highest F1 (0.6348 vs. 0.6310 for AR and 0.6076 for Original).



This pattern illustrates a more favorable trade-off for AR+GPO: it does not simply chase high BERTScore by ignoring constraints, nor does it force all constraints at the expense of overall text quality. Instead, AR+GPO’s global optimization helps coordinate multiple constraints while retaining strong semantic alignment. In contrast, AR appears effective at moderate attribute counts but loses ground on CSR once the load goes beyond 30 attributes, and the Original model experiences an even steeper decline.

% \paragraph{Implications.} In extreme fine-grained control (EFCG) tasks, these findings confirm that:
% \begin{enumerate}
%     \item Light to Moderate Constraints (e.g., up to 20 attributes) can be addressed by simpler fine-tuning without major F1 loss.
%     \item High Constraint Settings (30+ attributes), especially when semantic overlap among attributes is low or conflicts are frequent, demand methods like DPO or other preference-optimization approaches to prevent a precipitous drop in CSR.
% \end{enumerate}
% Table~\ref{tab:data_synthesis} presents the performance comparison between the original baselines and the SFT model across three traditional text classification benchmarks: Emotion, AG-News, and TREC. The results highlight significant improvements achieved by the SFT model, particularly for Emotion and TREC datasets. On the Emotion dataset, the SFT model achieves a 42.3\% accuracy, representing a substantial improvement of 14.0 percentage points over the baseline. Similarly, on the TREC dataset, which focuses on question type classification, the SFT model attains a 55.0\% accuracy, outperforming the baseline by 17.0 percentage points.



% \subsection{Toxicity Control}
% \label{sec:toxic}
% In this section, we use a toxic classifier~\cite{Detoxify} to identify 171 harmful examples from FineWeb. Using the same attribute extraction methods as described in Section 3, we then generate texts based on these attributes.

% The results in Figure \ref{fig:toxicity} clearly demonstrate that the SFT model struggles to handle toxicity control effectively, particularly as the number of attributes increases. While the Original Model maintains a consistently low toxicity rate across all levels of attribute complexity, the SFT model shows a significant and steady rise in toxicity rate, reaching over 25\% when handling 60 attributes. 

% This trend suggests that the SFT model fails to generalize well under highly constrained conditions and becomes increasingly susceptible to generating toxic content as it attempts to satisfy a growing number of attributes. The inability to properly balance attribute satisfaction with toxicity control highlights a critical limitation of the SFT approach, emphasizing the need for more robust mechanisms to enforce safety constraints, especially in scenarios involving complex or numerous attributes.

% \begin{figure}[t] 
%     \centering
%         \includegraphics[width=0.5\textwidth]{figs/toxicity.pdf}
%     \caption{Comparison of toxicity. }
%     \label{fig:toxicity}
% \end{figure}

% \subsection{Attention Flow}


% In this section, we test whether our UltraGen adapt well in out of domain downstream tasks. We test two capabilities, the first one is the factual, the



% \section{Empirical Analyses}\label{sec:analysis}
\section{Analyses}\label{sec:analysis}

% As discussed in the previous section, we can clearly see the connection between LLM community and IR community, where the LLMs can be seen as retrievers.
% In this section, we conduct experiments to analyze LLM from the three perspective in IR:
% (1) retrieval quality; (2) hard negatives in training data; (3) retrieval list size.

This section provides empirical analyses of the three factors identified in Section \ref{sec:proposal}.  
% Building upon these findings, Section \ref{sec:lrpo} introduces new preference optimization methods that leverage insights from the field of Information Retrieval.


% \begin{table*}[t]
%     \centering
%     % \renewcommand{\arraystretch}{1.2}
%     \caption{Preference optimization objective study on AlpacaEval2 and MixEval. For AlpacaEval2, we report the result with both opensource LLM evaluator \texttt{alpaca\_eval\_llama3\_70b\_fn} and GPT4 evaluator \texttt{alpaca\_eval\_gpt4\_turbo\_fn}. SFT corresponds to the initial chat model.}\label{tab:objective}
%     \small
%     \scalebox{0.85}{\begin{tabular}{llccccccccc}
%         \toprule
%         & & \multicolumn{2}{c}{AlpacaEval 2 (opensource LLM)} & \multicolumn{2}{c}{AlpacaEval 2 (GPT-4)} & \multicolumn{1}{c}{MixEval} & \multicolumn{1}{c}{MixEval-Hard} \\
%          \cmidrule(r){3-4} \cmidrule(r){5-6} \cmidrule(r){7-7} \cmidrule(r){8-8}
%         & Method & LC Winrate & Winrate & LC Winrate & Winrate & Score & Score \\
%         \midrule
%         \multirow{6}{*}{\rotatebox{90}{Gemma2-2b-it}} & SFT & 47.03 & 48.38 & 36.39 & 38.26 & 0.6545 & 0.2980 \\
%         \cmidrule{2-8}
%         & pairwise & 55.06 & 66.56 & 41.39 & 54.60 & 0.6740 & 0.3375 \\
%         & contrastive & 60.44 & 72.35 & 43.41 & 56.83 & 0.6745 & 0.3315 \\
%         & ListMLE & 63.05 & 76.09 & 49.77 & 62.05 & 0.6715 & 0.3560 \\
%         & LambdaRank & 58.73 & 74.09 & 43.76 & 60.56 & 0.6750 & 0.3560 \\
%         \midrule
%         \multirow{6}{*}{\rotatebox{90}{Mistral-7b-it}} & SFT & 27.04 & 17.41 & 21.14 & 14.22 & 0.7070 & 0.3610 \\
%         \cmidrule{2-8}
%         & pairwise & 49.75 & 55.07 & 36.43 & 41.86 & 0.7175 & 0.4105 \\
%         & contrastive & 52.03 & 60.15 & 38.44 & 42.61 & 0.7260 & 0.4340 \\
%         & ListMLE & 48.84 & 56.73 & 38.02 & 43.03 & 0.7360 & 0.4200 \\
%         & LambdaRank & 51.98 & 59.73 & 40.29 & 46.21 & 0.7370 & 0.4400 \\
%         \bottomrule
%     \end{tabular}}
% \end{table*}

\begin{table}[t]
    \centering
    % \renewcommand{\arraystretch}{1.2}
    % \vspace{-0.1in}
    \caption{Preference optimization objective study on AlpacaEval2 and MixEval. SFT corresponds to the initial chat model.}\label{tab:objective}
    \small
    \scalebox{0.99}{\begin{tabular}{llcccccc}
        \toprule
        & & \multicolumn{2}{c}{AlpacaEval 2} & \multicolumn{1}{c}{MixEval} & \multicolumn{1}{c}{MixEval-Hard} \\
         \cmidrule(r){3-4} \cmidrule(r){5-5} \cmidrule(r){6-6}
        & Method & LC Winrate & Winrate & Score & Score \\
        \midrule
        \multirow{6}{*}{\rotatebox{90}{Gemma2-2b-it}} & SFT & 36.39 & 38.26 & 0.6545 & 0.2980 \\
        \cmidrule{2-6}
        & pairwise & 41.39 & 54.60 & 0.6740 & 0.3375 \\
        & contrastive & 43.41 & 56.83 & 0.6745 & 0.3315 \\
        & ListMLE & \textbf{49.77} & \textbf{62.05} & 0.6715 & \textbf{0.3560} \\
        & LambdaRank & 43.76 & 60.56 & \textbf{0.6750} & \textbf{0.3560} \\
        \midrule
        \midrule
        \multirow{6}{*}{\rotatebox{90}{Mistral-7b-it}} & SFT & 21.14 & 14.22 & 0.7070 & 0.3610 \\
        \cmidrule{2-6}
        & pairwise & 36.43 & 41.86 & 0.7175 & 0.4105 \\
        & contrastive & 38.44 & 42.61 & 0.7260 & 0.4340 \\
        & ListMLE & 38.02 & 43.03 & 0.7360 & 0.4200 \\
        & LambdaRank & \textbf{40.29} & \textbf{46.21} & \textbf{0.7370} & \textbf{0.4400} \\
        \bottomrule
    \end{tabular}}
    % \vspace{-0.1in}
\end{table}


\subsection{Retriever optimization objective}

\paragraph{Experimental setting.}
Iterative preference optimization is performed on LLMs using the different learning objectives outlined in Section \ref{sec:retrieval-obj}.
Alignment experiments are conducted using the Gemma2-2b-it \citep{team2024gemma} and Mistral-7b-it \citep{jiang2023mistral} models, trained on the Ultrafeedback dataset \citep{cui2024ultrafeedback}. 
Following the methodology of \citep{dong2024rlhf}, we conduct three iterations of training and report the performance of the final checkpoint in Table \ref{tab:objective}.  
Model evaluations are performed on AlpacaEval2 \citep{dubois2024length} and MixEval \citep{ni2024mixeval}. 
% For AlpacaEval2, we employed both the open-source LLM evaluator \texttt{alpaca\_eval\_llama3\_70b\_fn} and the GPT4 evaluator \texttt{alpaca\_eval\_gpt4\_turbo\_fn}.
Detailed settings can be found in Appendix \ref{apx:sec-objective-setting}.


\paragraph{Observation.}
Table \ref{tab:objective} presents the results, from which we make the following observations: 
(1) Contrastive optimization generally outperforms pairwise optimization (\textit{e.g.}, DPO), likely due to its ability to incorporate more negative examples during each learning step. 
(2) Listwise optimization methods, including ListMLE and LambdaRank, generally demonstrate superior performance compared to both pairwise and contrastive approaches. 
This is attributed to their utilization of a more comprehensive set of preference information within the candidate list.




\begin{figure*}[t]
    \centering
    \subfigure[Hard negative study]{\includegraphics[width=0.32\textwidth]{figure/LLM_alignment_gsm8k_mathstral7b_neg_study.pdf}}
    \subfigure[Temperature \& hard negatives]{\includegraphics[width=0.32\textwidth]{figure/LLM_alignment_mistral_temperature_study.pdf}}
    \subfigure[Candidate list length study]{\includegraphics[width=0.32\textwidth]{figure/LLM_alignment_mistral_length_study.pdf}}
    % \vspace{-0.1in}
    \caption{Hard negative and candidate list study. (a) Hard negative study with $\mathcal{L}_{\text{pair}}$ on GSM8K with Mathstral-7b-it model. We explore four negative settings: (1) a random response not related to the given prompt; (2) a response to a related prompt; (3) an incorrect response to the given prompt with high temperature; (4) an incorrect response to the given prompt with suitable temperature. Hardness: (4)$>$(3)$>$(2)$>$(1). The harder the negatives are, the stronger the trained LLM is.
    (b) Training temperature study with $\mathcal{L}_{\text{pair}}$ on Mistral-7b-it and Alpaca Eval 2. Within a specific range ($>$ 1), lower temperature leads to harder negative and benefit the trained LLM. However, much lower temperature could lead to less diverse responses and finally lead to LLM alignment performance drop.
    (c) Candidate list size study with $\mathcal{L}_{\text{con}}$ on Mistral-7b-it. As the candidate list size increases, alignment performance improves.}\label{fig:merge-study}
    % \vspace{-0.1in}
\end{figure*}


% \begin{figure}[h!]
% \centering
% \includegraphics[scale=0.3]{figure/LLM_alignment_mistral_length_study.pdf}
% \vskip -1em
% \caption{Candidate list size study with $\mathcal{L}_{\text{con}}$ on Mistral-7b-it. As the candidate list size increases, alignment performance improves.}\label{fig:length-study}\vspace{-10pt}
% \end{figure}


% \begin{figure}[h!]
% \centering
% \includegraphics[scale=0.3]{figure/LLM_alignment_mistral_temperature_study.pdf}
% \vskip -1em
% \caption{Training temperature study with $\mathcal{L}_{\text{pair}}$ on Mistral-7b-it and Alpaca Eval 2. Within a specific range ($>$ 1), lower temperature leads to harder negative and benefit the trained LLM. However, temperature lower than this range can cause preferred and rejected responses non-distinguishable and lead to degrade training.}\label{tab:temp-hard}
% \end{figure}


% \begin{figure}[h!]
% \centering
% \includegraphics[scale=0.3]{figure/LLM_alignment_gsm8k_mathstral7b_neg_study.pdf}
% \vskip -1em
% \caption{Hard negative study with $\mathcal{L}_{\text{pair}}$ on GSM8K with Mathstral-7b-it model. We explore four negative settings: (1) a random response not related to the given prompt; (2) a response to a related prompt; (3) an incorrect response to the given prompt with high temperature; (4) an incorrect response to the given prompt with suitable temperature. Hardness: (4)$>$(3)$>$(2)$>$(1). The harder the negatives are, the stronger the trained LLM is.}\label{fig:mathstral-gsm8k-hard}
% \end{figure}


\subsection{Hard negatives}

\paragraph{Experimental setting.}
The Mathstral-7b-it model is trained on the GSM8k training set and evaluated its performance on the GSM8k test set. 
Iterative DPO is employed as the RLHF method, with the gold or correct response designated as the positive example. 
The impact of different hard negative variants is investigated, as described in Section \ref{sec:hard-negative}, with the results presented in Figure \ref{fig:merge-study}(a). 
Additionally, the influence of temperature on negative hardness with Lambdarank objective are examined using experiments on the AlpacaEval 2 dataset, with results shown in Figure \ref{fig:merge-study}(b).
Detailed settings are in Appendix \ref{apx:sec-hard-neg-setting} and \ref{apx:sec-hard-neg-setting-temp}.

% 

\begin{figure}[h!]
\centering
\includegraphics[scale=0.3]{figure/LLM_alignment_gsm8k_mathstral7b_neg_study.pdf}
\vskip -1em
\caption{Hard negative study with $\mathcal{L}_{\text{pair}}$ on GSM8K with Mathstral-7b-it model. We explore four negative settings: (1) a random response not related to the given prompt; (2) a response to a related prompt; (3) an incorrect response to the given prompt with high temperature; (4) an incorrect response to the given prompt with suitable temperature. Hardness: (4)$>$(3)$>$(2)$>$(1). The harder the negatives are, the stronger the trained LLM is.}\label{fig:mathstral-gsm8k-hard}
\end{figure}

% \begin{table}[t]
%     \centering
%     % \renewcommand{\arraystretch}{1.2}
%     \caption{Temperature study results for Gemma2-2b-it and Mistral-7b-it. We conduct RLHF (iterative DPO) for 3 iterations. $\uparrow$, $\rightarrow$ and $\downarrow$ denote RLHF with high, medium and low temperature. We use \texttt{alpaca\_eval\_llama3\_70b\_fn} as the evaluator.}\label{tab:temp-hard}
%     \scalebox{0.9}{\begin{tabular}{llcc}
%         \toprule
%         & & \multicolumn{2}{c}{Alpaca Eval 2} \\
%          \cmidrule(r){3-4}
%         & Method & LC Winrate & Winrate \\
%         \midrule
%         \multirow{5}{*}{\rotatebox{90}{Gemma2}} & SFT & 47.03 & 48.38 \\
%         \cmidrule{2-4}
%         & RLHF ($\uparrow$) & 54.45 & 67.50 \\
%         & RLHF ($\rightarrow$) & 59.31 & 69.77 \\
%         & RLHF ($\downarrow$) & 59.04 & 71.38  \\
%         \midrule
%         \multirow{5}{*}{\rotatebox{90}{Mistral}} & SFT & 27.04 & 17.41  \\
%         \cmidrule{2-4}
%         & RLHF ($\uparrow$) & 49.75 & 55.07 \\
%         & RLHF ($\rightarrow$) & 53.29 & 60.52 \\
%         & RLHF ($\downarrow$) & 54.78 & 64.33 \\
%         \bottomrule
%     \end{tabular}}
% \end{table}


% \begin{figure}[h!]
% \centering
% \includegraphics[scale=0.3]{figure/LLM_alignment_mistral_temperature_study.pdf}
% \vskip -1em
% \caption{Training temperature study with $\mathcal{L}_{\text{pair}}$ on Mistral-7b-it and Alpaca Eval 2. Within a specific range ($>$ 1), lower temperature leads to harder negative and benefit the trained LLM. However, temperature lower than this range can cause preferred and rejected responses non-distinguishable and lead to degrade training.}\label{tab:temp-hard}
% \end{figure}


\paragraph{Observation.}
Figure \ref{fig:merge-study}(a) illustrates that the effectiveness of the final LLM is directly correlated with the hardness of the negatives used during training. 
Harder negatives consistently lead to a more performant LLM.  
Figure \ref{fig:merge-study}(b) further demonstrates that, within a specific range, lower temperatures generate harder negatives, resulting in a more effective final trained LLM. 
% However, much lower temperature could also affect the quality of the chosen responses, make the chose and rejected responses non-distinguishable and finally lead to performance drop.
However, much lower temperature could lead to less diverse responses and finally lead to LLM alignment performance drop.
% Definition of temperatures can be found in Appendix \ref{apx:sec-hard-neg-setting-temp}.


\subsection{Candidate List}

\paragraph{Experimental setting.}
To investigate the impact of inclusiveness and memorization on LLM alignment, experiments are conducted using Gemma2-2b-it, employing the same training settings as in our objective study. 
For the inclusiveness study, the performance of the trained LLM is evaluated using varying numbers of candidates in the list.
For the memorization study, three approaches are compared: (i) using only the current iteration's responses, (ii) using responses from the current and previous iteration, and (iii) using responses from the current and all previous iterations. 
% Finally, for the temperature diversity study, the effect of employing different sampling temperatures is examined during response generation.
Detailed settings can be found in Appendix \ref{apx:sec-length-setting} and \ref{apx:sec-list-setting}.


% \begin{figure}[h!]
% \centering
% \includegraphics[scale=0.3]{figure/LLM_alignment_mistral_length_study.pdf}
% \vskip -1em
% \caption{Candidate list size study with $\mathcal{L}_{\text{con}}$ on Mistral-7b-it. As the candidate list size increases, alignment performance improves.}\label{fig:length-study}\vspace{-10pt}
% \end{figure}


\begin{table}[t]
    \centering
    % \vspace{-0.15in}
    \caption{Candidate list study with $\mathcal{L}_{\text{pair}}$ on Gemma2-2b-it. Previous iteration responses enhance performance.}\label{fig:list-study}
    \small
    \scalebox{0.99}{\begin{tabular}{lcc}
        \toprule
        & \multicolumn{2}{c}{Alpaca Eval 2} \\
         \cmidrule(r){2-3}
        Method & LC Winrate & Winrate \\
        \midrule
         SFT & 47.03 & 48.38 \\
        \cmidrule{1-3}
        Alignment (w. current)  & 55.06 & 66.56 \\
        Alignment (w. current + prev) & 55.62 & 70.92 \\
        Alignment (w. current + all prev) & 56.02 & 72.50  \\
        % \cmidrule{1-3}
        % Alignment (single temperature)  & 55.06 & 66.56 \\
        % Alignment (diverse temperature)  & 59.36 & 73.47  \\
        \bottomrule
    \end{tabular}}
    % \vspace{-0.15in}
\end{table}



\paragraph{Observation.}
Figure \ref{fig:merge-study}(c) illustrates the significant impact of candidate list size on LLM alignment performance.
As the candidate list size increases, performance improves, albeit with a diminishing rate of return. 
This is intuitive, given that a bigger candidate list size can contribute to more hard negatives and potentially benefit the model learning \citep{qu2020rocketqa}.
Table \ref{fig:list-study} demonstrates that incorporating responses from previous iterations can enhance performance.
This is potentially because introducing previous responses can make the candidate list more comprehensive and lead to better preference signal capturing.
More explanations are in Appendix \ref{apx:sec-list-setting}.


% \begin{table}[t]
%     \centering
%     % \renewcommand{\arraystretch}{1.2}
%     \begin{tabular}{lcc}
%         \toprule
%         & \multicolumn{2}{c}{Alpaca Eval 2} \\
%          \cmidrule(r){2-3}
%         Method & LC Winrate & Winrate \\
%         \midrule
%          SFT & 27.04 & 17.41 \\
%         \cmidrule{1-3}
%         RLHF (4 responses) & 50.02 & 61.72 \\
%         RLHF (6 responses) & 52.56 & 63.59 \\
%         RLHF (8 responses) & 55.21 & 64.88  \\
%         RLHF (10 responses) & 55.52 & 64.42  \\
%         \bottomrule
%     \end{tabular}
%     \caption{Candidate list size study for Mistral-7b-It. We conduct RLHF (iterative InfoPO) for 3 iterations. We use \texttt{alpaca\_eval\_llama3\_70b\_fn} as the evaluator.}
% \end{table}


\section{Related Work}
\paragraph{Unlearning Methods for LLMs.}
LLM unlearning has recently gained significant attention.
Gradient Ascent \citep{ga} maximizes loss for forgetting, while Negative Preference Optimization \citep{npo} draws on Direct Preference Optimization \citep{DPO}.
Various unlearning methods have been proposed \citep{NEURIPS2022_b125999b,eldan2023whosharrypotterapproximate,yu-etal-2023-unlearning,chen2023unlearnwantforgetefficient,pawelczyk2024incontextunlearninglanguagemodels, gandikota2024erasingconceptualknowledgelanguage,liu-etal-2024-towards-safer,seyitoğlu2024extractingunlearnedinformationllms,ding2024unifiedparameterefficientunlearningllms,baluta2024unlearninginvsoutofdistribution, zhuang2024uoeunlearningexpertmixtureofexperts, wei2025underestimatedprivacyrisksminority}.
Another strategy, ``locate-then-unlearn,'' includes Memflex \citep{tian2024forgetnotpracticalknowledge} and SURE \citep{zhang2024doesllmtrulyunlearn}. 
Several data-based methods have also been introduced \citep{jang2022knowledgeunlearningmitigatingprivacy,ma2024unveilingentitylevelunlearninglarge, liu2024learningrefusemitigatingprivacy,gu2024meowmemorysupervisedllm, sinha2024unstarunlearningselftaughtantisample,mekala-etal-2025-alternate}. 
Furthermore, some papers have highlighted the limitations of current machine unlearning \citep{10488864, zhou2024limitationsprospectsmachineunlearning, thaker2024positionllmunlearningbenchmarks, cooper2024machineunlearningdoesntthink, barez2025openproblemsmachineunlearning}.
\paragraph{Unlearning Evaluation for LLMs.}
Most studies \citep{maini2024tofutaskfictitiousunlearning, tian2024forgetnotpracticalknowledge} utilize ROUGE and PPL for evaluating unlearning.
Building upon these metrics, 
\citet{joshi-etal-2024-towards} measure unlearning via benchmark data transformation;
WMDP \citep{pmlr-v235-li24bc} further probes all layers to verify unlearning;
MUSE \citep{shi2024musemachineunlearningsixway} extends evaluation by using Member Inference Attack \citep{kim2024detectingtrainingdatalarge};
RWKU \citep{jin2024rwku} introduces a concept-level unlearning benchmark with adversarial attacks.
Similarly, Unstar \citep{sinha2024unstarunlearningselftaughtantisample} uses GPT scores, and \citet{ma2024benchmarkingvisionlanguagemodel} introduces a vision unlearning benchmark.
\section{Conclusion}
This paper introduces \textbf{ReLearn}, a novel unlearning framework via positive optimization that balances forgetting, retention, and linguistic capabilities. 
Our key contributions encompass a practical unlearning paradigm, comprehensive metrics (KFR, KRR, LS), and a mechanistic analysis comparing reverse and positive optimization. 
%As underscored, unlearning should not only erase knowledge but also relearn knowledge for constructive outputs.

\label{sec:bibtex}
\section*{Limitations}
While ReLearn shows promising performance, several limitations remain.
(1) Computational Overhead: Data synthesis may hinder scalability.
(2) Metric Sensitivity: Our metrics still have limited sensitivity to subtle knowledge nuances.
(3) Theoretical Grounding: Understanding the dynamics of knowledge restructuring requires deeper theoretical investigation, which we plan to explore in the future work.
% \section*{Acknowledgments}
\section*{Ethical Statement}
This research is conducted with a strong commitment to ethical principles. 
We affirm that all datasets used in this study are either publicly available or synthetically generated to simulate privacy-sensitive scenarios. 
These synthetic datasets contain no personally identifiable information, ensuring that no privacy violations or copyright infringements occurred. 
Furthermore, this work draws inspiration from cognitive linguistic research on Alzheimer's disease, specifically on how linguistic abilities are affected.
However, this is solely for the purpose of analysis and comparison, and we expressly condemn any form of discrimination against individuals with Alzheimer's disease or any other health conditions. 
This study aims to advance knowledge in the field of LLM unlearning in an ethical and responsible manner.

% Bibliography entries for the entire Anthology, followed by custom entries
%\bibliography{anthology,custom}
% Custom bibliography entries only
\bibliography{custom}

\appendix

\newpage
\appendix
\onecolumn

The appendix is structured into multiple sections, each offering supplementary information and further clarification on topics discussed in the main body of the manuscript. 

\startcontents[sections]  % 开始定义附录目录
\printcontents[sections]{}{1}{\setcounter{tocdepth}{3}}  % 打印附录目录
\vskip 0.2in
\hrule

\section{More Details for Method}\label{appendix_method}
\subsection{More Details for Outlier-aware Weighting}\label{app:outlier}
\paragraph{Interpretation of Dual Objectives for outlier weighting}
The mathematical framework achieves cross-model consensus and intra-model saliency through its hierarchical thresholding mechanism:

(i) \textbf{Cross-Model Consensus}:
The denominator in Eq. (3) normalizes each model's contribution by the total sparse outlier magnitude across all $n$ models:
\begin{equation}
    \sum_{j=1}^n \sum_{c=1}^{d_l} \|\textsc{Threshold}(\bm{\Delta}_{l,c}^{(j)}, \mu_c^{(j)}+3\sigma_c^{(j)})\|_1
\end{equation}
This forces models with greater sparse deviation magnitudes (potential task conflicts) to receive proportionally reduced aggregation weights $\alpha_l^{(i)}$, effectively suppressing outlier-dominated models in the merged output.

(ii) \textbf{Intra-Model Saliency}:
The $3\sigma$ threshold in $\textsc{Threshold}(\bm{\Delta}_{l,c}^{(i)}, \mu_c^{(i)}+3\sigma_c^{(i)})$ implements statistical outlier detection within each model's parameter distribution. For Gaussian-distributed $\Delta_{l,c,k}^{(i)}$ (per Central Limit Theorem), this retains only the top 0.3\% extreme deviations that likely correspond to:
\begin{itemize}
    \item Task-specific knowledge carriers ($\Delta > \mu+3\sigma$)
    \item Catastrophic interference sources ($\Delta < \mu-3\sigma$)
\end{itemize}
The $L_1$ norm aggregation $\sum_{c=1}^{d_l}\|\cdot\|_1$ then amplifies layers containing concentrated outlier parameters.

\textbf{Synergistic Effect}: The normalization in (i) prevents any single model's outliers from dominating the merger, while the saliency detection in (ii) preserves critical task-specific features within each model. This dual mechanism reduces interference by selectively blending statistically significant parameters across models.

\subsection{More Details for the Reasonability of R-TSVM}
\label{ssec:analysis}

Building on TSVM's theoretical framework, our method provides enhanced guarantees through statistical awareness and adaptive computation.

\paragraph{Conflict Probability Bound}
Let $p_{\text{conflict}}^{(l)}$ denote the probability of directional conflicts in layer $l$. Our rank adaptation yields as follows. We can observe that , compared to TSVM's fixed $\frac{1}{\sqrt{d_l}}$, our bound adapts to layer sparsity.

\begin{equation}
    \mathbb{E}[p_{\text{conflict}}^{(l)}] \leq \frac{1}{\sqrt{k_l}} \propto \frac{1}{\sqrt{d_l(1-\gamma s_l)}}
\end{equation}

\paragraph{Weight Concentration}
The 3$\sigma$ thresholding induces weight concentration on critical parameters. For any layer $l$:

\begin{equation}
    \frac{\mathbb{V}[w_l^{(i)}]}{\mathbb{E}[w_l^{(i)}]^2} \leq \frac{1}{\|\mathcal{T}_{3\sigma}(\bm{\tau}_l^{(i)})\|_0}
\end{equation}

This variance-to-mean ratio decreases as outliers become sparser, stabilizing training.

\begin{table}[ht]
    \centering
    \footnotesize  % 使用更小字号
    \setlength{\tabcolsep}{5pt}  % 压缩列间距
    \caption{Theoretical Comparison between our reweight optimization and TSVM.}
    \label{tab:theory}
    \begin{tabular}{@{} l >{\centering\arraybackslash}p{1.8cm} >{\centering\arraybackslash}p{2cm} @{}} % 自定义列宽
        \toprule
        \textbf{Property}    & \textbf{TSVM} & \textbf{R-TSVM}                    \\
        \midrule
        Layer adaptivity     & $\times$      & $\checkmark$                       \\
        Sparsity awareness   & $\times$      & $\checkmark$                       \\
        Conflict bound       & $O(d^{-1/2})$ & $O(d^{-1/2}(1{-}\gamma s)^{-1/2})$ \\
        Weight concentration & Uniform       & Heavy-tailed                       \\
        Comp.\ complexity    & $O(d^3)$      & $O(k d^2)$                         \\
        \bottomrule
    \end{tabular}
\end{table}

% \subsection{Orthogonal Subspace Mechanism}
% The key to TSVM's conflict resolution lies in the orthogonality of singular vectors:
% \begin{equation}
% \langle \bm{u}_r^{(i)}, \bm{u}_s^{(j)} \rangle \approx 0 \quad (r \neq s)
% \label{eq:orthogonality}
% \end{equation}
% This property enables:
% \begin{itemize}
%     \item \textbf{Constructive Alignment}: Shared directions ($\bm{u}_r^{(i)} \approx \bm{u}_r^{(j)}$) are amplified.
%     \item \textbf{Conflict Suppression}: Orthogonal directions ($\bm{u}_r^{(i)} \perp \bm{u}_s^{(j)}$) are naturally attenuated.
% \end{itemize}


\subsection{Order of Orthogonalization and Rank Truncation/Selection}
\label{sec:order}

A critical design choice in our R-TSVM algorithm lies in the sequential relationship between orthogonalization (Eq.~\ref{seek_orthogonal_u}-\ref{seek_orthogonal_v}) and rank truncation (Eq.~\ref{rank_selection}). Through theoretical analysis and empirical validation, we establish that \textbf{orthogonalization should precede truncation} to ensure optimal subspace alignment and information preservation. This ordering stems from three fundamental considerations:
\textbf{Global Orthogonality Constraints}: The orthogonal projection in Eq.~\ref{seek_orthogonal_u} minimizes the Frobenius norm difference $\| {U_{l}}_\bot - U_{l} \|_F$ under strict orthogonality constraints. Performing this projection \textit{before} truncation preserves the complete singular vector structure, enabling accurate modeling of cross-task interference patterns. Early truncation would discard directional components essential for constructing the orthogonal basis, particularly when task-specific updates exhibit heterogeneous rank distributions.

\textbf{Dynamic Rank Adaptation}: Our sparsity-adaptive rank selection (Eq.~\ref{rank_selection}) requires layer-wise sparsity measurement $\Omega_l$, computed from the full parameter deviation matrix $\bm{\Delta}_l^{(i)}$. Truncating $\bm{\Delta}_l^{(i)}$ prematurely would bias $\Omega_l$ by excluding contributions from low-magnitude parameters, thereby undermining the adaptive rank calculation. As shown in Algorithm~\ref{alg:OWLM}, orthogonalization (Step~4) utilizes the full-rank SVD decomposition to maintain statistical fidelity.

\textbf{Outlier Weighting Integrity}: The outlier-aware weighting mechanism (Eq.~6) operates on the complete parameter deviation matrix to identify statistically significant updates. Truncation prior to outlier detection would risk eliminating subtle yet critical features masked within lower-rank components, particularly in layers with heavy-tailed parameter distributions.

\section{More Details for Related Work}\label{more_relatedwork}
\subsection{Discussion with the Alignment Tax.}
\begin{figure}[htbp]
    \centering
        \setlength{\abovecaptionskip}{0cm}   %调整图片标题与图距离
    \setlength{\belowcaptionskip}{0cm}   %调整图片标题与
    \includegraphics[width=0.99\linewidth]{fig/subsequent_DPO.pdf} % 调整宽度为栏宽的 90%
    \caption{Illustration of Training Stage of 3H Optimization, which aims to further enhance LLMs alignment from three perspectives based on the existing Initially Aligned LLMs.}
    \label{fig:3H_stage}
\end{figure}
We would like to further clarify the main difference between 3H trade-off and previously defined alignment tax \cite{lin2024mitigating,lu2024online}. In general, the alignment tax describes the phenomenon of RLHF training leading to \emph{the forgetting of pre-trained abilities during the first alignment stage}. However, as shown in Figure \ref{fig:3H_stage}, we mainly focus on how can we further \emph{enhance the 3H-related abilities of the existing already-aligned model during the second or subsequent stages.} The trade-off mainly comes from the conflict of different alignment objects without dealing with the pre-trained knowledge. Take the Llama3 series for example, alignment tax mainly analyzes the pre-trained ability degradation on the SFT version of the Base LLM (e.g. train the Llama-3-8B on the Ultrachat) while performing DPO training, which refers to the \textbf{green arrow} of the Figure \ref{fig:3H_stage}. However, in this paper, we mainly focus on how can we further enhance the 3H-related abilities of the existing already aligned model (e.g. Llama3-8B-Instruct) during the second or subsequent alignment stages (\textbf{orange arrow} of the Figure \ref{fig:3H_stage}), which can meet more strict demands for specific applications.

\subsection{Discussion with the MOE-Based Merging Methods (e.g. H3 fusion).} To further distinguish our work from previous ones and strengthen our contribution, we provide more detailed discussions about the MOE-Based Merging methods \cite{zhao2024loraretriever,zhao2024merging,zhou2025mergeme}. Specifically, most of MOE-based merging works, such as SMILES \cite{tang2024smile}, Free-Merging \cite{zheng2024free}, and Twin-Merging \cite{zheng2024free}, aim to balance the performance and deployment costs through modular expertise identification and integration adapted to the input data, which is not designed for our setting about 3H optimization in LLM alignment. 

Recently,  we have noticed a concurrent MOE-fusion work called H3 fusion \cite{tekin2024h} related to our theme. It includes three main steps:(i) Adopt the instruction tuning and summarization fusion as two modern ensemble learning in the context of helpful-harmless-honest (H3) alignment (ii) \textbf{Merge} the aligned model weights with an expert router \textbf{according to the type of
input} instruction and dynamically select a subset of experts. (iii) Utilize the gating loss and regularization terms to enhance performance. But our work mainly focuses on how can we address the conflict issued for 3H optimization to construct a multi-object aligned LLM rather than dynamically adapted to the input data. Simultaneously, considering that the constraints of data availability and data leak will limit the generalization of existing merging methods for LLMs, in the paper we mainly adopt the well-known and latest \textbf{training-
free and data-free} merging strategies for dense LLM, while H3 fusion needs the data for training and only utilizes the merging techniques for efficiently adapting to the input data. Thus, \textbf{H3 fusion is indeed different from our work from the perspective of problem and technique contributions.}

% (e.g.as SMILES \cite{tang2024smile}, WEMOE \cite{shen2024efficient}, Free-Merging \cite{zheng2024free}, Twin-Merging \cite{zheng2024free}, H3 fusion \cite{cf208ab262f0affc4f5581b9a18901265d9728ab}}
% Notably, those mixtures of experts(MOE) based merging methods, such as SMILES \cite{tang2024smile}, WEMOE \cite{shen2024efficient}, Free-Merging \cite{zheng2024free} and Twin-Merging \cite{zheng2024free} are not designed for our settings, we provide discussions in the Appendix.


\section{More Details for Experiments}\label{training_details}
\subsection{The Training Details for Model Constructions and Baselines}
\textbf{Training hyperparameters for model constructions:} following SimPO \cite{meng2024simpo}, based on Llama-3-8B-Instruct and Mistral-7B-Instruct-V2, we conduct preference optimization adopting the fixed batch size 128 for 1 epoch training with the Adam optimizer. We set the max sequence length to 4096 and apply a cosine learning rate schedule with 10 percent warmup steps for each dataset. Specially, we adjust $\beta \in \left[ 0.1, 0.5, 1.0, 2.0 \right]$ and learning rate $lr \in \left[3e-7,5e-7 \right]$ for model constructions and report the best individual training models corresponding to different annotation dimensions.


\textbf{The Implementation of Baselines:} For Heuristic data mixture methods, we control the ratio between Honesty\&Harmlessness and Helpfulness to 1/5,1/10 and 1/20 by default and report the best average score (usually 1/10 according to our experiments). For ArmoRM, we follow the process of SimPO \cite{meng2024simpo} to achieve refined full mixture data. For hummer \cite{jiang2024hummer}, we refine the alignment dimension conflict (ADC) among preference datasets leveraging the powerful ability of AI feedback(e.g. GPT4) as the paper stated. For the full mixture datasets of Table \ref{tab:dpo_data_stats}, we control the ADC lower than 20 percent.




\textbf{Computation environment:} All of our experiments in this paper were conducted on 16×A100 GPUs based on the LLaMA-Factory \cite{zheng2024llamafactory}, MergeKit \cite{goddard2024arcee} and fusion\_bench \cite{tang2024fusionbench}.


\textbf{Reproducibility:} We have made significant efforts to ensure the reproducibility of our work. Upon acceptance, we will release all of the trained models and the complete training and testing code to facilitate the full reproducibility of our results. We are committed to advancing this work and will provide updates on its accessibility in the future. 



\subsection{The Evaluation Details for the Judged Models}\label{evaluation_datails}
We provide detailed descriptions for the evaluation that needs the judged models. For MT-Bench, we report scores following its evaluation
protocol to grade single answers from 1 to 10 scores assisted by GPT4. For HaluEval-Wild, given prompts to our trained model, we utilize the judged model to check whether the output of our trained model is a hallucination or not and then calculate the no hallucination rate. Similarly, we utilize the prompts from SaladBench and OR-Bench to instruct our trained models and then let the judged models check whether the replies of our trained models are safe/unsafe or refusal/answer. Based on the check results, we can naturally calculate the safe score and refusal score by counting all results. The detailed descriptions of the evaluation can be shown in Table \ref{tab:evaluation_comparison}.  More details can be shown in the original paper.

\begin{table}[ht]
    \centering
    \footnotesize  % 使用更小字号
    \setlength{\tabcolsep}{5pt}  % 压缩列间距
    \caption{Evaluation details corresponding judge models, scoring types, and metrics.}
    \label{tab:evaluation_comparison}
    \begin{tabular}{@{} l l c c c @{}} % 自定义列宽
        \toprule
        \textbf{Evaluation Datasets} &\textbf{Examples} & \textbf{Judge Models} & \textbf{Scoring Type} & \textbf{Metrics} \\
        \midrule
         MT-Bench \cite{zheng2023judging}          & 80    & GPT-4  & Single Answer Grade  & Rating of 1-10 \\
         HaluEval-Wild \cite{zhu2024halueval}         & 500   & GPT4   & Classify \& Calculate Ratio  & Rating of 0-100 \\
         SaladBench \cite{li2024salad}        & 1817  & MD-Judge-V0.2 & Classify \& Calculate Ratio  & Rating of 0-100 \\
         OR-Bench \cite{cui2024or}           & 1319  & GPT4-o & Classify \& Calculate Ratio  & Rating of 0-100 \\
        \bottomrule
    \end{tabular}
\end{table}

\begin{table*}
    % \setlength{\abovecaptionskip}{0cm}
    % \setlength{\belowcaptionskip}{0cm}
    \captionsetup{font={small,stretch=1.25}, labelfont={bf}}
    \renewcommand{\arraystretch}{1}
    \caption{\textbf{3H Results on Mistral Under Continuous Optimization Setting where we sequentially perform DPO training using data with annotations about Helpfulness\&Honesty (Stage1), Helpfulness\&Harmlessness (Stage2) and Helpful (Stage3).For merging methods, we highlight the best score in bold and the second score with underlining.}}
    \label{continuous_mistral}
    \centering
    \resizebox{0.99\textwidth}{!}{
        \begin{tabular}{lcccccccc|c|cc|cccc}
            \toprule
            \multirow{2}{*}{\textbf{Methods}} & \multicolumn{8}{c|}{\textbf{Helpfulness}} & \textbf{Honesty} & \multicolumn{2}{c|}{\textbf{Harmlessness}} & \multirow{2}{*}{\textbf{Helpful\_Avg}} & \multirow{2}{*}{\textbf{Honest\_Avg}} & \multirow{2}{*}{\textbf{Harmless\_Avg}} & \multirow{2}{*}{\textbf{AVG}}                                                                                                          \\ \cmidrule{2-12}
                                              & Math                                      & GSM8K            & ARC-E                                      & ARC-C                                  & MMLU                                  & MBPP\_Plus                              & HumanEval\_Plus               & MT-Bench & HaluEval\_Wild & Salad\_Bench(↑) & OR-Bench(↑) &       &       &       &                   \\ \midrule
            \textbf{Mistral-7B-Instruct-V2}   & 9.54                                      & 46.17            & 82.36                                      & 72.88                                  & 59.97                                 & 26.46                                   & 28.66                         & 7.55     & 62.17           & 78.07           & 74.68       & 41.70 & 62.17 & 76.38 & 60.08             \\ \midrule
            Continual DPO Training Stage1     & 8.76                                      & 43.14            & 82.01                                      & 74.92                                  & 59.78                                 & 25.93                                   & 27.33                         & 7.59     & 61.33           & 78.74           & 77.23       & 41.18 & 61.33 & 77.99 & 60.17             \\
            Continual DPO Training Stage2     & 9.26                                      & 36.16            & 82.54                                      & 75.59                                  & 60.38                                 & 29.88                                   & 33.33                         & 7.86     & 56.40           & 82.76           & 78.54       & 41.88 & 56.40 & 80.65 & 59.64             \\
            Continual DPO Training Stage3     & 9.60                                      & 40.49            & 82.54                                      & 77.29                                  & 60.51                                 & 26.25                                   & 34.15                         & 7.46     & 57.40           & 80.77           & 83.16       & 42.29 & 57.40 & 81.97 & 60.52             \\ \midrule
            Weight Average                    & 10.04                                     & 45.72            & 82.36                                      & 75.25                                  & 61.03                                 & 26.46                                   & 31.71                         & 7.56     & 59.20           & 78.02           & 81.43       & 42.52 & 59.20 & 79.73 & 60.48             \\
            Rewarded Soup                     & 9.72                                      & 46.02            & 82.19                                      & 75.25                                  & 61.03                                 & 26.46                                   & 32.93                         & 7.61     & 58.60           & 77.94           & 81.34       & 42.65 & 58.60 & 79.64 & 60.30             \\
            Model Stock                       & 9.74                                      & 47.69            & 82.36                                      & 73.56                                  & 59.77                                 & 24.87                                   & 27.44                         & 7.68     & 61.00           & 78.51           & 76.44       & 41.64 & 61.00 & 77.48 & 60.04             \\
            Task Arithmetic                   & 9.76                                      & 43.06            & 82.54                                      & 75.93                                  & 61.27                                 & 25.66                                   & 32.93                         & 7.46     & 57.80           & 78.32           & 82.35       & 42.33 & 57.80 & 80.34 & 60.15             \\
            Ties                              & 10.48                                     & 41.55            & 84.66                                      & 76.27                                  & 61.60                                 & 26.19                                   & 30.49                         & 7.46     & 53.80           & 78.99           & 85.43       & 42.34 & 53.80 & 82.21 & 59.45             \\
            DARE                              & 10.40                                     & 42.99            & 85.36                                      & 75.93                                  & 61.54                                 & 24.60                                   & 33.54                         & 7.54     & 56.00           & 78.81           & 85.21       & 42.74 & 56.00 & 82.01 & 60.25             \\
            DARE Ties                         & 10.28                                     & 42.00            & 85.01                                      & 76.27                                  & 61.61                                 & 27.25                                   & 32.32                         & 7.43     & 53.00           & 79.17           & 86.50       & 42.77 & 53.00 & 82.84 & 59.54             \\
            DELLA                             & 10.18                                     & 43.14            & 84.83                                      & 75.25                                  & 61.46                                 & 26.46                                   & 31.71                         & 7.58     & 55.25           & 79.35           & 86.04       & 42.58 & 55.25 & 82.70 & 60.18             \\
            DELLA  Ties                       & 10.50                                     & 40.18            & 85.89                                      & 77.97                                  & 61.37                                 & 30.16                                   & 30.48                         & 7.30     & 54.80           & 79.90           & 87.49       & 42.98 & 54.80 & 83.70 & \underline{60.49} \\
            Breadcrumbs                       & 10.56                                     & 42.53            & 84.83                                      & 75.59                                  & 64.50                                 & 24.60                                   & 32.32                         & 7.53     & 52.40           & 79.42           & 84.34       & 42.81 & 52.40 & 81.88 & 59.03             \\
            Breadcrumbs Ties                  & 10.54                                     & 42.46            & 84.66                                      & 76.95                                  & 61.47                                 & 26.72                                   & 29.88                         & 7.45     & 53.40           & 79.80           & 84.57       & 42.52 & 53.40 & 82.19 & 59.37             \\
            TSVM                              & 10.52                                     & 41.25            & 85.28                                      & 77.21                                  & 61.57                                 & 29.22                                   & 30.48                         & 7.55     & 54.95           & 79.90           & 87.49       & 42.89 & 54.95 & 83.70 & \textbf{60.51}    \\


            \bottomrule
        \end{tabular}
    }
\end{table*}

\begin{table*}
  % \setlength{\abovecaptionskip}{0cm}
  % \setlength{\belowcaptionskip}{0cm}
  \captionsetup{font={small,stretch=1.25}, labelfont={bf}}
  \renewcommand{\arraystretch}{1}
  \caption{\textbf{The detailed 3H Results on Llama3 Under Static Optimization Setting adopting our proposed reweighting-based optimization.}}
  \label{detailed_reweight_llama3}
  \centering
  \setlength{\tabcolsep}{2pt}
  \resizebox{0.99\textwidth}{!}{
    \begin{tabular}{lcccccccc|c|cc|cccc}
      \toprule
      \multirow{2}{*}{\textbf{Methods}}      & \multicolumn{8}{c|}{\textbf{Helpfulness}} & \textbf{Honesty} & \multicolumn{2}{c|}{\textbf{Harmlessness}} & \multirow{2}{*}{\textbf{Helpful\_Avg}} & \multirow{2}{*}{\textbf{Honest\_Avg}} & \multirow{2}{*}{\textbf{Harmless\_Avg}} & \multirow{2}{*}{\textbf{AVG}}                                                                                                          \\ \cmidrule{2-12}
                                             & Math                                      & GSM8K            & ARC-E                                      & ARC-C                                  & MMLU                                  & MBPP\_Plus                              & HumanEval\_Plus               & MT-Bench & HaluEval\_Wild & Salad\_Bench & OR-Bench &       &       &       &                   \\ \midrule
      \textbf{llama3-8B-Instruct}            & 28.08                                     & 78.09            & 93.65                                      & 82.03                                  & 68.20                                 & 58.99                                   & 53.05                         & 8.25     & 53.50           & 91.16           & 26.97       & 58.79 & 53.50 & 59.07 & 57.12             \\ \midrule
      Hummer (best mixture training)          & 29.41                                     & 78.95            & 93.65                                      & 82.69                                  & 68.59                                 & 60.41                                   & 58.15                         & 8.58     & 55.60           & 92.10           & 50.11       & \textbf{60.35} & 55.60 & 73.21 & 63.05             \\ \midrule
      TSVM (best merging)                                   & 29.92                                     & 77.63            & 93.12                                      & 82.17                                  & 68.51                                 & 59.26                                   & 55.49                         & 8.29     & 56.20           & 89.43           & 67.76       & 59.30 & 56.20 & 78.60 & 64.70    \\
      \textbf{R-TSVM (ours)}                                   & 29.89                                     & 78.89            & 93.65                                      & 82.37                                  & 68.51                                 & 59.56                                   & 56.63                         & 8.29     & 57.20          & 89.92           & 69.27       & 59.72 & \textbf{57.20} & \textbf{79.60} & \textbf{65.51}    \\
      \bottomrule
    \end{tabular}
  }
\end{table*}

\begin{table*}
  % \setlength{\abovecaptionskip}{0cm}
  % \setlength{\belowcaptionskip}{0cm}
  \captionsetup{font={small,stretch=1.25}, labelfont={bf}}
  \renewcommand{\arraystretch}{1}
  \caption{\textbf{The detailed 3H Results on Mistral Under Static Optimization Setting adopting our proposed reweighting-based optimization.}}
  \label{detailed_reweight_mistral}
  \centering
  \setlength{\tabcolsep}{2pt}
  \resizebox{0.99\textwidth}{!}{
    \begin{tabular}{lcccccccc|c|cc|cccc}
      \toprule
      \multirow{2}{*}{\textbf{Methods}}      & \multicolumn{8}{c|}{\textbf{Helpfulness}} & \textbf{Honesty} & \multicolumn{2}{c|}{\textbf{Harmlessness}} & \multirow{2}{*}{\textbf{Helpful\_Avg}} & \multirow{2}{*}{\textbf{Honest\_Avg}} & \multirow{2}{*}{\textbf{Harmless\_Avg}} & \multirow{2}{*}{\textbf{AVG}}                                                                                                          \\ \cmidrule{2-12}
                                             & Math                                      & GSM8K            & ARC-E                                      & ARC-C                                  & MMLU                                  & MBPP\_Plus                              & HumanEval\_Plus               & MT-Bench & HaluEval\_Wild & Salad\_Bench & OR-Bench &       &       &       &                   \\ \midrule
      \textbf{Mistral-7B-Instruct-v0.2}        & 9.54                                      & 46.17            & 82.36                                      & 72.88                                  & 59.97                                 & 26.46                                   & 28.66                         & 7.55     & \textbf{62.17}           & 78.07           & 74.68       & 41.70 & 62.17 & 76.38 & 60.08             \\ \midrule
      Hummer (best mixture training)           & 9.79                                      & 44.50            & 83.72                                      & 74.89                                  & 60.53                                 & 25.85                                   & 33.15                         & 7.56     & 62.05           & 81.85           & 75.28       & 42.50 & 62.05 & 78.57 & 61.04             \\ \midrule
      TSVM (best merging)                                    & 10.40                                     & 44.88            & 84.29                                      & 75.24                                  & 60.87                                 & 28.50                                   & 32.32                         & 7.65     & 61.10           & 83.25           & 78.51       & 43.02 & 61.10 & 80.88 & 61.67  \\
        \textbf{R-TSVM (ours)}                                   & 10.44                                    & 45.00            & 84.35                                      & 75.79                                  & 60.87                                 & 28.50                                   & 32.52                         & 7.71     & 61.50           & 84.25           & 80.25       & \textbf{43.15} & 61.50 & \textbf{82.25} & \textbf{62.30}  \\
      \bottomrule
    \end{tabular}
  }
\end{table*}


\subsection{More Experiments under the Continual DPO Training Settings} \label{appendix_continual}
As shown in Table \ref{continuous_mistral}, we provide additional results under the continual training settings. Through comparison results between different training stages, we can observe the honesty, helpfulness, and harmlessness of LLMs are interactively enhanced due to forgetting during continual training. Moreover, model merging methods can achieve comparable results to these continual training methods without the need to consider the optimized status at a specific training stage. In other words, model merging paves a new way for continual DPO training, advocating training multiple models from the same start point and then merging them, rather than continually optimizing the model from the previous optimization. 

\subsection{The detailed results of Reweighting-based optimization.}\label{appendix_reweighting}
Due to the page limit in the main content, we provide the detailed results of Reweighting-based optimization over TSVM to further verify its effectiveness for 3H optimization in LLM alignment.



\subsection{Hyper-Parameter Analysis to Sparsity} \label{appendix_sparsity}
The sparsity-based strategy is closely related to the merging effect. As shown in Table \ref{static_llama3} and Table \ref{static_mistral}, apart from the SVD-based methods, the most effective merging methods are DARE and DELLA, both of which depend on random sparsification as shown in Table \ref{tab:methods}. However, we conduct extended studies to check the robustness and stability concerning random seed and sparsity factors. As shown in Figure \ref{seed} and Figure \ref{sparsity_sensitivity}, we can observe that R-TSVM can achieve better and more robust results than previous random sparsification-based methods, further verifying the effectiveness of our methods.


\begin{figure}[tb]
\centering
\subfloat[DARE-Ties]{
\includegraphics[width=0.3\linewidth]{fig/seed_dare_ties.pdf}
}
\subfloat[R-TSVM]{
\includegraphics[width=0.3\linewidth]{fig/seed_R_TSVM.pdf}
}
\caption{Comparisons between the random sparsification strategy (e.g.DARE-Ties) and SVD-based strategy (R-TSVM) on Mistral under static optimization settings adopting different seeds. R-TSVM can achieve more stable results than random sparsification methods. }
\label{seed}
\end{figure}

\begin{figure}[tb]
\centering
\subfloat[DARE-Ties]{
\includegraphics[width=0.3\linewidth]{fig/sparsity_dare_ties.pdf}
}
\subfloat[R-TSVM]{
\includegraphics[width=0.3\linewidth]{fig/sparsity_R_TSVM.pdf}
}
\caption{Paramerter sensitive analysis concerning sparsity factor for model merging methods on Mistral under static optimization settings.}
\label{sparsity_sensitivity}
\end{figure}


\end{document}
