\section{Conclusion}
This paper introduces \textbf{ReLearn}, a novel unlearning framework via positive optimization that balances forgetting, retention, and linguistic capabilities. 
Our key contributions encompass a practical unlearning paradigm, comprehensive metrics (KFR, KRR, LS), and a mechanistic analysis comparing reverse and positive optimization. 
%As underscored, unlearning should not only erase knowledge but also relearn knowledge for constructive outputs.

\label{sec:bibtex}
\section*{Limitations}
While ReLearn shows promising performance, several limitations remain.
(1) Computational Overhead: Data synthesis may hinder scalability.
(2) Metric Sensitivity: Our metrics still have limited sensitivity to subtle knowledge nuances.
(3) Theoretical Grounding: Understanding the dynamics of knowledge restructuring requires deeper theoretical investigation, which we plan to explore in the future work.