\documentclass[11pt]{article}

\usepackage[inline]{enumitem}
\usepackage{floatrow}
\usepackage{tikz}
\usetikzlibrary{positioning}
\usepackage{wrapfig}


% Optional math commands from https://github.com/goodfeli/dlbook_notation.
%%%%%% NEW MATH DEFINITIONS %%%%%

% \usepackage{amsmath,amsfonts,bm}
\usepackage{amsmath,amsfonts}

\usepackage{pifont}


\newcommand{\R}{\mathbb{R}}


\def\va{{\mathbf{a}}}
\def\vg{{\mathbf{g}}}

% Sets
\def\sR{\mathbb{R}}
\def\sC{\mathbb{C}}
\def\sZ{\mathbb{Z}}
\def\sN{\mathbb{N}}
\def\sQ{\mathbb{Q}}

\def\sS{\mathcal{S}}



% Vectors
\def\vzero{{\mathbf{0}}}
\def\vone{{\mathbf{1}}}
\def\vmu{{\mathbf{\mu}}}
\def\vtheta{{\mathbf{\theta}}}
\def\va{{\mathbf{a}}}
\def\vb{{\mathbf{b}}}
\def\vc{{\mathbf{c}}}
\def\vd{{\mathbf{d}}}
\def\ve{{\mathbf{e}}}
\def\vf{{\mathbf{f}}}
\def\vg{{\mathbf{g}}}
\def\vh{{\mathbf{h}}}
\def\vi{{\mathbf{i}}}
\def\vj{{\mathbf{j}}}
\def\vk{{\mathbf{k}}}
\def\vl{{\mathbf{l}}}
\def\vm{{\mathbf{m}}}
\def\vn{{\mathbf{n}}}
\def\vo{{\mathbf{o}}}
\def\vp{{\mathbf{p}}}
\def\vq{{\mathbf{q}}}
\def\vr{{\mathbf{r}}}
\def\vs{{\mathbf{s}}}
\def\vt{{\mathbf{t}}}
\def\vu{{\mathbf{u}}}
\def\vv{{\mathbf{v}}}
\def\vw{{\mathbf{w}}}
\def\vx{{\mathbf{x}}}
\def\vy{{\mathbf{y}}}
\def\vz{{\mathbf{z}}}
\def\vzeta{{\mathbf{\zeta}}}

% Matrix
\def\mA{{\mathbf{A}}}
\def\mB{{\mathbf{B}}}
\def\mC{{\mathbf{C}}}
\def\mD{{\mathbf{D}}}
\def\mE{{\mathbf{E}}}
\def\mF{{\mathbf{F}}}
\def\mG{{\mathbf{G}}}
\def\mH{{\mathbf{H}}}
\def\mI{{\mathbf{I}}}
\def\mJ{{\mathbf{J}}}
\def\mK{{\mathbf{K}}}
\def\mL{{\mathbf{L}}}
\def\mM{{\mathbf{M}}}
\def\mN{{\mathbf{N}}}
\def\mO{{\mathbf{O}}}
\def\mP{{\mathbf{P}}}
\def\mQ{{\mathbf{Q}}}
\def\mR{{\mathbf{R}}}
\def\mS{{\mathbf{S}}}
\def\mT{{\mathbf{T}}}
\def\mU{{\mathbf{U}}}
\def\mV{{\mathbf{V}}}
\def\mW{{\mathbf{W}}}
\def\mX{{\mathbf{X}}}
\def\mY{{\mathbf{Y}}}
\def\mZ{{\mathbf{Z}}}
\def\mBeta{{\mathbf{\beta}}}
\def\mPhi{{\mathbf{\Phi}}}
\def\mLambda{{\mathbf{\Lambda}}}
\def\mSigma{{\mathbf{\Sigma}}}


% Expectation
% \def\eE{\mathop{\mathbb{E}}\limits}
\def\eE{\mathbb{E}}

% Probability
\def\pP{\mathbb{P}}

% Tilde
\def\tf{\tilde{f}}
\def\tS{\tilde{S}}
\def\wtF{\widetilde{\mathcal{F}}}
\def\whR{\widehat{R}}
\def\tvx{\tilde{\mathbf{x}}}
\def\ty{\tilde{y}}


\def\defeq{\overset{\textup{def}}{=}}
% \def\defeq{\overset{.}{=}}
\def\defone{\overset{\text{\ding{172}}}{=}}
\def\deftwo{\overset{\text{\ding{173}}}{=}}
\def\leqone{\overset{\text{\ding{172}}}{\leq}}
\def\leqtwo{\overset{\text{\ding{173}}}{\leq}}
\def\leqthree{\overset{\text{\ding{174}}}{\leq}}
\def\leqfour{\overset{\text{\ding{175}}}{\leq}}
\def\eqone{\overset{\text{\ding{172}}}{=}}
\def\eqtwo{\overset{\text{\ding{173}}}{=}}
\def\eqthree{\overset{\text{\ding{174}}}{=}}
\def\eqfour{\overset{\text{\ding{175}}}{=}}
\def\geqfive{\overset{\text{\ding{176}}}{\geq}}
\newcommand{\tc}[1]{\textcolor{magenta}{[Tiffany: {#1}]}}
\usepackage{xcolor} %[dvipsnames]
\usepackage{hyperref}
\hypersetup{
    colorlinks=true,
    linkcolor=blue,
    citecolor=blue,
    filecolor=blue,      
    urlcolor=blue,
    pdfpagemode=FullScreen,
    }
\usepackage{url}
\usepackage{amsmath}
\usepackage{cleveref}

 \newcommand{\kwz}[1]{
  {\color{violet} [{#1}]} %KWZ: 
 }

  \newcommand{\hn}[1]{
  {\color{red} [HN: {#1}]}
 }

 \newcommand{\dan}[1]{
  {\color{green} [Dan: {#1}]}
 }
\usepackage[textsize=tiny]{todonotes}
\newcommand{\hntodo}[1]{\todo{Hong: #1}}
\newcommand{\kwztodo}[1]{\todo{KWZ: #1}}

\newcommand{\what}[1]{\widehat{#1}} % Wide hat
 
\newcommand{\R}{\bs{\MC{R}}}
\newcommand{\Rhat}{\bs{\hat{\MC{R}}}}
\newcommand{\Dtrain}{\MC{D}^{\TN{offline}}}
\newcommand{\Deval}{\MC{D}^{\TN{eval}}}
\newcommand{\D}{\MC{D}}
\newcommand{\Ahist}{\MC{A}^{\TN{offline}}}
\newcommand{\Aeval}{\MC{A}}
\newcommand{\Zeval}{\MC{Z}}
\newcommand{\psar}{\mathbb{A}_{\TN{TS-Gen}}}
\newcommand{\piX}{\bs{\pi}^*(X_{1:T})}

\usepackage{subcaption}
\usepackage{tcolorbox}

\usepackage{commands}
\usepackage{comment}
\usepackage{algorithm}
%\usepackage{algpseudocode}
\usepackage{algorithmic}

\makeatletter
\newcounter{phase}[algorithm]
\newlength{\phaserulewidth}
\newcommand{\setphaserulewidth}{\setlength{\phaserulewidth}}
\newcommand{\phase}[1]{%
  \vspace{-1.25ex}
  % Top phase rule
  \Statex\leavevmode\llap{\rule{\dimexpr\labelwidth+\labelsep}{\phaserulewidth}}\rule{\linewidth}{\phaserulewidth}
  \Statex\strut\refstepcounter{phase}\textit{Phase~\thephase~--~#1}% Phase text
  % Bottom phase rule
  \vspace{-1.25ex}\Statex\leavevmode\llap{\rule{\dimexpr\labelwidth+\labelsep}{\phaserulewidth}}\rule{\linewidth}{\phaserulewidth}}
\makeatother

\setphaserulewidth{.7pt}

% packages
\usepackage[margin=1in]{geometry}
%\usepackage{color}
\usepackage{epsfig}
\usepackage{epstopdf}
\usepackage{setspace}

\usepackage{amsmath}
\usepackage{amssymb}
\usepackage{amsthm}

\usepackage{rotating}
\usepackage{verbatim}
\usepackage{bm}
\usepackage[normalem]{ulem}
\usepackage[square, numbers, sort]{natbib}
\usepackage{bbm}
\usepackage{array}
\usepackage{float}
\usepackage{authblk}

\usepackage[perpage]{footmisc}


% theorems
\newtheorem{theorem}{Theorem} %[section]
\newtheorem{lemma}{Lemma} %[theorem]
\newtheorem{assumption}{Assumption} %[theorem]
\newtheorem{corollary}{Corollary} %[theorem]
\newtheorem{definition}{Definition} % DH: I changed this %[theorem]
\theoremstyle{definition}
\theoremstyle{plain}
%\newtheorem{definition}{Definition} %[section]
\newtheorem{proposition}{Proposition}
\newtheorem{example}{Example}
\newtheorem{remark}{Remark}

% Kelly's stuff
\usepackage{commands}
\usepackage{multirow}

%\usepackage{amsmath}
\usepackage{array}
\newcommand\undermat[2]{%
  \makebox[0pt][l]{$\smash{\underbrace{\phantom{%
    \begin{matrix}#2\end{matrix}}}_{\text{$#1$}}}$}#2}






\begin{document}

% Control whitespace around equations
\abovedisplayskip=8pt plus0pt minus3pt
\belowdisplayskip=8pt plus0pt minus3pt

\begin{center}
 {\LARGE Contextual Thompson Sampling via Generation of Missing Data} \\ 
  \vspace{.5cm}
{\large Kelly W. Zhang$^{1}$~~~~Tiffany (Tianhui) Cai$^{2}$~~~~
    Hongseok Namkoong$^{2}$~~~~Daniel Russo$^{2}$} \\ 
  \vspace{.2cm} 
  {\large Imperial College London$^{1}$ \hspace{1cm} Columbia University$^{2}$ }  \\
\vspace{.2cm}
  \texttt{kelly.zhang@imperial.ac.uk}
  \hspace{1cm}
    \texttt{\{tc3100, hn2369, djr2174\}@columbia.edu}
\end{center}



%\def\thefootnote{*}\footnotetext{}
%Contact: kelly.zhang@imperial.ac.uk, \{tc3100, hn2369, djr2174\}@columbia.edu

\begin{abstract}
We introduce a framework for Thompson sampling contextual bandit algorithms, in which the algorithm's ability to quantify uncertainty and make decisions depends on the quality of a generative model that is learned offline. Instead of viewing uncertainty in the environment as arising from unobservable latent parameters, our algorithm treats uncertainty as stemming from missing, but potentially observable, future outcomes. If these future outcomes were all observed, one could simply make decisions using an ``oracle'' policy fit on the complete dataset. Inspired by this conceptualization, at each decision-time, our algorithm uses a generative model to probabilistically impute missing future outcomes, fits a policy using the imputed complete dataset, and uses that policy to select the next action. We formally show that this algorithm is a generative formulation of Thompson Sampling and prove a state-of-the-art regret bound for it. Notably, our regret bound i) depends on the probabilistic generative model only through the quality of its offline prediction loss, and ii) applies to any method of fitting the ``oracle'' policy, which easily allows one to adapt Thompson sampling to decision-making settings with fairness and/or resource constraints.
\end{abstract}


\section{Introduction}
Recent advances in machine learning have transformed our ability to develop high quality predictive and generative models for complex data. This work introduces a framework for developing decision-making algorithms, specifically for contextual bandit problems, that can take advantage of these machine learning advances. By design, we assume the algorithm developer is able to effectively apply these machine learning techniques (e.g., minimize a loss via gradient descent) and employ these methods as subroutines in our decision-making algorithm. Moreover, our theory formally connects the quality of effective (self-)supervised learning via loss minimization to the quality of decision-making.

% Redfining the primitive operations
Classically, contextual Thompson sampling algorithms form a parametric model of the environment and consider the decision-maker's uncertainty as arising from unknown latent parameters of that model \citep{russo2020tutorial}. In this classical perspective, the primitive operations that are assumed to be feasible (at least approximately) include i) the ability to specify an informative prior for the latent parameter using domain knowledge, ii) the ability to sample from the posterior distribution of the latent parameter, and iii) the ability to update the posterior distribution as more data is collected. 
Unfortunately, it is well known that all three of these primitive operations are non-trivial to perform with neural networks \citep{tran2020practical,goan2020bayesian}.

Building on our previous work \citep{psar2024} which focuses on multi-armed bandits without contexts, we view missing, but potentially observable, future outcomes as the source of the decision-maker's uncertainty. This perspective allows us to replace the primitive operations required in the classical view with new primitives that are more compatible with neural networks: i) the ability to effectively minimize an offline sequence prediction loss, ii) the ability to autoregressively generate from the optimized sequence model, and iii) the ability to fit a desired policy given access to a complete dataset (outcomes from all actions and decision-times). 


% Explain the algorithm and what makes this algorithm possible (meta)
In the missing data view of uncertainty, if we had a complete dataset, there is no uncertainty because we could simply use the entire dataset to fit a desired ``oracle'' policy to use to make decisions. Inspired by this idea, at each decision time our algorithm imputes all missing outcomes using a pretrained generative sequence model, fits a desired policy using the imputed complete dataset, and selects the best action according to the fitted policy. We show that this algorithm is a generative implementation of Thompson sampling \citep{russo2020tutorial}. Moreover, we demonstrate empirically that it is possible to learn an accurate generative model to impute missing outcomes using standard machine learning tools in \textit{meta-bandit} settings, where one encounters many distinct, but related bandit tasks. We use data from previous bandit tasks to train  a generative sequence model offline.

% Advantages of this method
We prove a state-of-the-art regret bound for our generative Thompson sampling algorithm with three key properties, which each have significant practical implications. First, the generative model used to impute missing outcomes only affects our bound through the offline sequence prediction of the model. This means that our theory is applicable to any sequence model architecture, and that the quality of the sequence model can be easily optimized for and evaluated via offline training and validation. Second, our bound is unique in that it applies to any procedure for fitting a desired ``oracle'' policy. This allows one to easily adapt Thompson sampling to decision-making problems with constraints, including resource and fairness constraints. Finally, our proof approach makes important improvements to previous information theoretic analyses, which may be broadly applicable: i) we accommodate infinite policy classes directly without discretization, and ii) our bound quantifies the benefit of prior information available from a task, such as side information regarding actions. 


\section{Problem formulation}
\label{sec:probFormulation}
%%%%%%%%%%%%%%%%%%%%%%%%%%%%%%%%%%%%%%%%%%%%%%%%%
\subsection{Meta contextual bandit problem}
We consider a meta contextual bandit problem where bandit tasks $\tau$ are sampled from an unknown task distribution $p^*$:
\begin{align}
    \tau \sim p^*.
    \label{eqn:taskDist}
\end{align}
Each bandit task $\tau$ consists of prior information $Z_{\tau}$, an action space $\MC{A}_{\tau}$, a sequence of context vectors $X_{1:T} = \{ X_1, \dots, X_T \}$, and a table of potential outcomes\footnote{We omit subscripting $X_t$ and $Y_t^{(a)}$ with $\tau$ to reduce clutter.} $\{ Y_{1:T}^{(a)} \}_{a \in \MC{A}_{\tau}} = \{ Y_1^{(a)}, \dots, Y_T^{(a)} \}_{a \in \MC{A}_{\tau}}$:
\begin{align*}
    \tau = \big\{ Z_{\tau}, X_{1:T}, \{ Y_1^{(a)}, \dots, Y_T^{(a)} \}_{a \in \MC{A}_{\tau}} \big\}.
\end{align*}
Informally, the agent's objective is to select actions to maximize the total expected reward for each encountered task. 
At the start of a task, the agent observes prior information $Z_{\tau}$. For each decision time $t \in [1 \colon T]$, the agent observes the context $X_t$, selects an action $A_t \in \MC{A}_{\tau}$, observes the outcome $Y_t = Y_t^{(A_t)}$, and computes the reward $R(Y_t)$, for a fixed, known function $R$ that takes values in $[0,1]$. We use $\HH_t$ to denote the history, which includes the current context:
\begin{align*}
    \HH_t = \left\{ Z_{\tau}, (X_1, A_1, Y_1),  \dots, (X_{t-1}, A_{t-1}, Y_{t-1}), X_t \right\}.
\end{align*}
The agent is able to learn both online \textit{within a single task} meaning over the $T$ total decision times, as well as meta-learn \textit{across different tasks} (e.g., learning how task prior information $Z_{\tau}$ may inform the distribution of $\{ Y_{1:T}^{(a)} \}_{a \in \MC{A}_{\tau}}$).

%%%%%%%%%%%%%%%%%%%%%%%%%%%%%%%%%%%%%%%%%%%%%%%%%
\paragraph{Offline data.}
We assume that the algorithm has access to training data from previous tasks, $\Dtrain = \{ \tau_1, \dots, \tau_n \}$, sampled according to \eqref{eqn:taskDist}. These previous bandit tasks can be used by the algorithm to meta-learn across tasks, e.g., learn about the distribution $p^*$ itself to improve decision-making quality. For simplicity, we present our algorithm assuming we have access to complete task datasets $\tau_i$, where all outcomes from all actions, $\{ Y_{1:T}^{(a)} \}_{a \in \MC{A}_{\tau}}$, are available. In Appendix~\ref{app:pretrain_bootstrap}, we discuss how by bootstrapping we can learn from partial data from each task (e.g., where $Y_t^{(a)}$ is only observed if $A_t = a$ was selected).

\begin{figure}[t]
    \centering
    \includegraphics[width=0.8\linewidth]{figures/news-recommendation-context.png}
    \caption{News recommendation meta contextual bandit problem. This }
    \label{fig:news-recommendation}
\end{figure}

%%%%%%%%%%%%%%%%%%%%%%%%%%%%%%%%%%%%%%%%%%%%%%%%%
\paragraph{Motivating example: News recommendation task.}
As depicted in \ref{fig:news-recommendation}, a motivating meta-contextual bandit problem is cold-start news recommendations.
Each day, a new set of articles $\MC{A}_{\tau}$ are released, which the agent recommends to users who arrive throughout the day. In contrast to \citet{li2010contextual}, our algorithm meta-learns across news recommendation tasks and uses the article text to improve cold-start decisions. We use $Z_{\tau} = (Z_{\tau}^{(a)})_{a \in \MC{A}_{\tau}}$ to denote the task-specific prior information, i.e., the news article text. The context variables $X_t$ consist of user-specific features and $Y_t$ are recommendation outcomes observed following the $t^{\rm{th}}$ decision. 

The modern challenge in this problem setting is that incorporating the news article text $Z_{\tau}$ can greatly improve the recommendation system's decisions, but a foundation model is needed to read process this high dimensional text and inform the decision-making of the agent. This motivates us in this work to i) make very minimal structural assumptions on the relationship between prior information $Z_{\tau}$, context features $X_t$, and outcomes $Y_t$, and ii) develop an algorithm that is amenable to incorporating foundation models. 


%%%%%%%%%%%%%%%%%%%%%%%%%%%%%%%%%%%%%%%%%%%%%%%%%
\subsection{Environment assumptions}
The defining quality of our problem formulation is that \textit{we do not make explicit assumptions about the distribution of outcomes $Y$ conditional on contexts $X$ and prior information $Z$}. It is common in the meta bandit literature to assume a known parametric model class that accurately captures the distribution $\{ Y_t^{(a)} \}_{a \in \MC{A}_{\tau}} \mid (X_t, Z_{\tau})$ \citep{kveton2021meta,wan2021metadata,cella2020meta,cella2022meta}; Typically, there is an unknown environment parameter that varies between tasks.
We instead allow this distribution to be general. Our algorithm's decision-making quality depends on how accurately the agent models this distribution, as well as the policy fitting procedure the algorithm designer chooses. Rather than relying on strong assumptions on the environment structure, we put the onus on the algorithm designer to i) learn a model that accurately captures the environment structure of the meta-bandit task at hand, and ii) choose a meaningful method for fitting a desired ``oracle'' policy, assuming access to a complete dataset. The motivation for this comes recognizing that such offline learning problems are routinely solved in practice in settings that extend well beyond theory. We focus on our theory on formal reductions to offline learning and policy selection problems, assuming offline learning can be done at scale.

\paragraph{(1) Stationary contextual bandit.} Assumption \ref{assump:contextualBandit} below captures two critical properties of (stationary) contextual bandit environments: 
(i) that contexts $X_{1:T}$ are \textit{exogenous} to (independent of) past outcomes and actions taken by the agent,
and (ii) that the context outcome tuples $(X_t, Y_t^{(a)})$ are permutation invariant over time. These assumptions are standard properties of contextual bandit type environments considered in the literature \citep{LattimoreSz19}.

\begin{assumption}[Contextual bandit environment]
\label{assump:contextualBandit}
For any task $\tau \sim p^*$, the distribution of the next context $X_t$ is independent of the previous outcomes, i.e., 
\begin{align*}
     X_t \indep \big\{ Y_{1:t-1}^{(a)} \big\}_{a \in \MC{A}_{\tau}} \mid \big( X_{1:t-1}, Z_{\tau} \big). 
\end{align*}
Additionally, for any action $a \in \MC{A}_{\tau}$, the tuples $(X_t, Y_t^{(a)})$ are exchangeable over time, i.e., for any permutation $\sigma$ over $T$ elements, the following are equal in distribution: 
\begin{align*}
    \big( X_t, Y_t^{(a)} \big)_{t \in [1 \colon T]} 
    \quad \overset{D}{=} \quad
    \big( X_{\sigma(t)}, Y_{\sigma(t)}^{(a)} \big)_{t \in [1 \colon T]}.
\end{align*}
\end{assumption}

%%%%%%%%%%%%%%%%%%%%%%
\paragraph{(2) Independence across actions.} 
We also assume that the outcomes $Y$ are independent across actions, conditional on $Z$ and $X$. This is a simplifying assumption, which allows us to safely model actions independently, and ignore dependencies between outcomes across actions.
\begin{assumption}[Independence across actions]
    Conditional on $Z_{\tau}, X_{1:T}$, outcomes $Y_{1:T}^{(a)}$ are independent across actions $a \in \MC{A}_{\tau}$.
    \label{assump:indepAction}
\end{assumption}

\paragraph{(3) Known context distribution.}
The final assumption we make is that the agent knows the distribution of contexts $X_{1:T}$. In practice, this is often not a strong assumption because contexts $X$ from previous tasks (which do not need to be associated with any outcomes $Y$) can be used to approximate this context distribution. In many applications, the foremost cost in data collection is that of acquiring ``labels'' associated with the unlabeled covariates $X$ \citep{settles2009active} and it is common to easily acquire large quantities of unlabeled covariate data \citep{zhou2021semi}. We do not focus on learning the context distribution in this work since existing unsupervised methods could be directly used here for covariate modeling \citep{jonsson1998automated}. Moreover, learning the distribution of passively observed stationary contexts (which are unaffected by the actions taken by the agent by Assumption \ref{assump:contextualBandit}) is less compelling from the perspective of designing active learning algorithms. 
\begin{assumption}[Context distribution is known]
    The conditional distribution of contexts, $X_{1:T} \mid Z_{\tau}$, is known.\label{assump:context}
\end{assumption}


%%%%%%%%%%%%%%%%%%%%%%%%%%%%%%%%%%%%%%%%%%%%%%%%%
\subsection{Definition of regret}
\label{sec:regretDef}
\bo{Policy fitting.} 
We assume that the algorithm designer specifies a procedure for fitting a desired ``oracle'' policy given access to a complete bandit task dataset $\tau$. This policy fitting procedure outputs policies in a function class $\Pi^*$ where each $\pi^* \in \Pi^*$ defines a mapping from contexts $X_t$ to an action $a \in \MC{A}_{\tau}$ that does not vary over time. For notational simplicity, the policies in $\Pi^*$ are assumed to be non-stochastic. Note that we \textit{do not} require that this policy class is necessarily ``correct''. For a particular task $\tau$, we use $\pi^*(\cdotspace; \tau)$ to denote a ``best-fitting'' policy $\pi^* \in \Pi^*$, where the fitting criterion is defined by the algorithm designer. 
For a simple example, one could fit using a least squares criterion:
\begin{align*}
    \TN{argmin}_{\pi \in \Pi^*} ~ \frac{1}{T} \sum_{t=1}^T \left\{ R \big(Y_t^{(\pi(X_t))} \big) - \max_{a \in \MC{A}_{\tau}} R \big(Y_t^{(a)} \big) \right\}^2.
\end{align*}
One should think of $\pi^*(\cdotspace; \tau)$ the policy one \emph{would} implement if abundant task data, $\tau$, was available. This could involve fitting a model, adding prompt tokens to condition a language model, or maximizing hindsight performance. This policy fitting could also incorporate various desirable constraints on the policy, including resource or fairness constraints. We aim to attain performance competitive with this policy through efficient interactive learning, despite starting without any of the outcomes $Y$ from the task data. 


\paragraph{Regret.} 
We consider a best-in-class style regret objective, which is common in the contextual bandit literature \citep{foster2020open,foster2019model,langford2007epoch,agarwal2017corralling}. The objective of the agent $\mathbb{A}$ is to make decisions to minimize the per-period regret against $\pi^*(X_t; \tau)$: 
\begin{align}
    \label{eqn:regretContext}
    \hspace{-1mm}\Delta(\mathbb{A}) = \E \bigg[ \frac{1}{T} \sum_{t=1}^T \left\{ R \big(Y_{t}^{(\pi^*(X_t; \tau))} \big) - R \big( Y_{t}^{(A_t)} \big) \right\} \bigg].\hspace{-1mm}
\end{align}
The expectation above averages over tasks $\tau \sim p^*$ and any randomness in the algorithm used to select actions $A_{1:T}$. This can be interpreted as the long-run per-period regret if the algorithm was deployed across many tasks.

Note that increasing the complexity of the policy class $\Pi^*$ (in a VC dimension sense) can increase the average reward under the best-fitting policy, $\E \left[ \frac{1}{T} \sum_{t=1}^T R \big(Y_{t}^{(\pi^*(X_t; \tau))} \big)\right]$. However, this increased complexity also means that large sample sizes are required to learn $\pi^*(\cdotspace; \tau)$ accurately and will worsen our regret bound (see Section \ref{sec:regret}).


\section{Key conceptual idea: Missing data view of uncertainty quantification}
In this work, we view missing data as the source of the decision-maker's uncertainty.
This contrasts the classical approach of considering unknown model parameters as the source of uncertainty. As we will explore in the following sections, the missing data viewpoint is very amenable to modern deep learning methods, which can be used to train models that are able to impute missing data probabilistically in a calibrated fashion. 

\subsection{Posterior sampling via imputing missing data} 
To convey the missing data viewpoint, we first consider an idealized setting in which we have access to the meta task distribution $p^*$ (we discuss how to approximate $p^*$ in Section \ref{sec:ourAlg}). Using $p^*$ we can form an exact posterior sample for task outcomes $\tau = \big\{ Z_{\tau}, X_{1:T}, \{Y_{1:T}^{(a)}\} \big\}$ given the history $\HH_t$:
\begin{align}
    \label{eqn:taskPosteriorSample}
    \hat{\tau}_t \sim p^* \left( \tau \in \cdot \mid \HH_t \right).
\end{align}
Above we probabilistically generate values in $\tau$ that have not yet been observed in the history $\HH_t$; This consists of future contexts, future outcomes, and outcomes from previous timesteps for actions that were not selected.

\begin{figure}[h]
    \centering
    \includegraphics[width=0.5\linewidth]{figures/fitted-policy.png}
    \caption{At each decision time, the agent imputes missing outcomes and uses the imputed dataset to fit a policy to select actions.}
    \label{fig:fitted-policy}
\end{figure}

With this exact posterior sample of the task outcomes $\hat{\tau}_t$, we can use it to form posterior samples of any statistic computed using $\hat{\tau}_t$. 
In particular, we are interested in sampling from the posterior distribution of the fitted policy $\pi^*(\cdotspace; \tau)$, which can be computed by finding the fitted policy for the sampled task dataset $\hat{\tau}_t$, i.e., $\pi^*(\cdotspace; \hat{\tau}_t)$. 
Obtaining posterior samples of a best-fitting policy is a common subroutine used in Bayesian decision-making algorithms \cite{kaufmann2012bayesian, russo2018learning, ryzhov2012knowledge}. Thus, the posterior sampling via generation easily integrates with these existing contextual bandit algorithms. 

In this work, we focus on Thompson sampling \citep{russo2016information,thompson1933likelihood}, i.e., probability matching, which selects actions according to the posterior probability that they are optimal. Thompson sampling can be implemented with a single posterior sample per decision time. Specifically, at decision time $t$, after sampling $\hat{\tau}_t$ as in \eqref{eqn:taskPosteriorSample}, Thompson sampling fits the policy $\pi^*(\cdotspace; \hat{\tau}_t)$, and selects the action $A_t \gets \pi^*(X_t; \hat{\tau}_t)$. See Figure \ref{fig:fitted-policy} for a depiction.
Under this generative version of Thompson sampling, the polices in $\Pi^*$ that are optimal under some likely generation of $\hat{\tau}_t$ have a chance of being selected. Once no plausible sample of missing outcome $\hat{\tau}_t$ could result in an action being optimal, it is essentially written off. 
See Algorithm \ref{alg:Thompson} for pseudocode for Thompson sampling via generation.

\begin{algorithm}[h]
\caption{Thompson sampling via generation}
\label{alg:Thompson}
\begin{algorithmic}[1]
   \REQUIRE Imputation model $p$, actions $\MC{A}_{\tau}$, task input $Z_{\tau}$.
   \FOR{$t \in \{1, \dots, T\}$}
        \STATE Observe context $X_t$ and append it to $\HH_t$
        \STATE Generate / sample $\hat{\tau}_t \sim p( \tau \in \cdot \mid \HH_t)$
        \STATE Fit the policy $\pi^*(\cdotspace; \hat{\tau}_t)$ % where $\hat{\tau} \gets \{ Z_{\tau}, \hat{X}_{1:T}, \hat{\bs{Y}} \}$
        \STATE Select the action $A_t \gets \pi^*(X_t; \hat{\tau}_t)$
        \STATE Observe outcome $Y_t \gets Y_t^{(A_t)}$ from action $A_t$.
        \STATE Update history $\HH_{t+1} \gets \HH_t \cup \{ (X_t, A_t, Y_t)\}$ 
   \ENDFOR
\end{algorithmic}
\end{algorithm}


%%%%%%%%%%%%%%%%%%%%%%%%%%%%%%%%%%%%%%%%%%%%%%
\subsection{Regret: Thompson sampling via generation with $p^*$}
\label{sec:regret-perfect}
This section presents a regret bound for Algorithm \ref{alg:Thompson} with the perfect imputation model, $p^*$ from \eqref{eqn:taskDist}. Our work develops a novel analysis of contextual Thompson sampling, which is applicable to infinite policy classes $\Pi^*$ with finite VC dimension. Our VC dimension bound resembles those from adversarial bandits, but for the first time, we show we can derive this using an information theoretic analysis.

\paragraph{Notation.}
For any random variables $X, Y$ (for $Y$ discrete), we denote the conditional entropy of $Y$ given $X$ using $H(Y \mid X)$; Note $H(Y \mid X)$ is a constant, i.e., $H(Y \mid X) = \E \big[ \sum_y \PP( Y = y \mid X) \log \PP( Y = y \mid X) dy \big]$.
Additionally, we use the following notation to define the best fitting policy evaluated at contexts $X_{1:T}$:
\begin{align}
    \label{eqn:piXdef}
    \piX := \{ \pi^*(X_t; \tau) \}_{t=1}^T.
\end{align}

\paragraph{Regret bound.}
Proposition \ref{prop:psarRegretPerfect} states our regret bound for Algorithm \ref{alg:Thompson} when the imputation model is $p^*$ from \eqref{eqn:taskDist}.
\begin{proposition}[Regret for Thompson sampling via generation with $p^*$]
    \label{prop:psarRegretPerfect}
    Let Assumptions \ref{assump:contextualBandit} and \ref{assump:context} hold. Under Thompson sampling via generation (Algorithm \ref{alg:Thompson}) with the imputation model $p^*$, 
    denoted $\mathbb{A}_{\rm{TS-Gen}}(p^*)$, 
    \begin{align*}
        \Delta( \mathbb{A}_{\rm{TS-Gen}}(p^*) ) \leq \sqrt{ \frac{ |\Aeval| \cdot H \big( \piX \mid Z_{\tau}, X_{1:T} \big) }{2 T} }.
    \end{align*}
\end{proposition}
Proposition \ref{prop:psarRegretPerfect} follows as a direct corollary of our more general result (Theorem \ref{thm:psarRegret} in Section \ref{sec:regret}).  
What is notable in the regret bound from this result is that i) it quantifies the benefit of incorporating prior information $Z$, and ii) it automatically applies to infinite policy classes since it only depends on the entropy of the fitted policy evaluated at a finite number of contexts, $\piX := \{ \pi^*(X_t; \tau) \}_{t=1}^T$. 


\paragraph{Relation to other information theoretic regret bounds.} Our regret bound notably improves on prior work, which develop Bayesian regret bounds for contextual Thompson sampling. First, our bound can be applied broadly, while approaches like \citet{neu2022lifting} depends on the entropy of a latent environment parameter, which is only applicable to parametric contextual bandits. Second, the entropy term in our bound $H( \piX \mid Z_\tau, X_{1:T})$ is often notably smaller than those from other contextual Thompson sampling Bayesian regret bounds. Specifically, \citet{infotheoreticNonstationary} consider a contextual bandit setting where at decision-time $t$ the algorithm selects a policy to use to choose $A_t$, prior to observing the context $X_t$. Their regret depends on the entropy of the optimal policy function, which when translated to our setting is much larger than our entropy term:  $H( \{ \pi^*(X_t) \}_{t=1}^T \mid X_{1:T}, Z_\tau) < H ( \pi^* \mid Z_\tau)$; Intuitively, this is saying that the entropy of a policy evaluated at a finite number of contexts is less than the entropy of the entire policy function.
Finally, while many information-theoretic analyses for Thompson sampling that generalize beyond multi-armed bandits require arguments that discretize the latent parameter space \cite{dong2018information,gouverneur2024an,neu2022lifting,infotheoreticNonstationary} and utilize cover-number type arguments, our proof approach notably does not require any discretization.

\paragraph{Upper bounding the condition entropy using VC dimension.} We can construct a coarse upper bound for the entropy $H \big( \piX \mid Z_{\tau}, X_{1:T} \big)$ using the VC-dimension of the policy class $\Pi^*$. This bound is coarse because the VC-dimension is a worst case quantity that has to with the number of possible assignments of actions to contexts. In contrast, entropy reflects that based on the task distribution (learned from past tasks) and the information $Z$ (e.g. article texts), many assignments may be extremely unlikely to be optimal. We formalize VC dimension upper bound in Lemma \ref{lemma:VC} below, which holds by the Sauer-Shelah lemma \citep{sauer1972density,shelah1972combinatorial}.

\begin{lemma}[VC dimension bound on entropy]\label{lemma:VC}
    For any binary\footnote{Note, VC-dimension is defined for binary functions.} action policy class $\Pi^*$, 
    \begin{align*}
        H \big( \piX \mid Z_{\tau}, X_{1:T} \big)
        \leq H \big( \piX \mid X_{1:T} \big)
        = O\big( \TN{VCdim}(\Pi^*) \log T \big).
    \end{align*}
\end{lemma}

Using our coarser upper bound on the entropy, our regret bound (Proposition \ref{prop:psarRegretPerfect}) resembles adversarial regret bounds that depend on VC dimension, showing for the first time how such a result can be established through information theoretic arguments \citep{beygelzimer2011contextual}.
Additionally, our rate also resembles standard Bayesian regret bounds for linear contextual bandits \citep{russo2018learning}; When $\Pi^*$ is defined by a linear model of dimension $d$, then $\TN{VCdim}(\Pi^*) = d+1$, so $H \big( \piX \mid X_{1:T} \big) = O(d \log T)$. 


%%%%%%%%%%%%%%%%%%%%%%%%%%%%%%%%%%%%%%%%%%%%%
\section{Thompson sampling via generation under an imperfect imputation model}
\label{sec:ourAlg}
In the previous section, we introduced a generative version of Thompson sampling for contextual bandits under the assumption we have access to a perfect imputation model $p^*$. In this section, we discuss how to practically approximate such an algorithm. 
First, we pretrain an autoregressive sequence model to predict successive outcomes ($Y$'s) on historical data $\Dtrain$. Then, at decision time, recommendation decisions are made by imputing the missing outcomes in $\tau$ with by generating outcomes ($\hat{Y}'$s) autoregressively from the pretrained sequence model. The offline pretraining allows the algorithm to ``meta-learn'' a good model for imputing missing outcomes using data from previous tasks.

\begin{figure}[h]
    \centering
    \includegraphics[width=0.6\linewidth]{figures/process-fig.png}
    \caption{Offline meta-learning and online decision-making across multiple tasks. Figure is slight modification of one from \citet{psar2024}.}
    \label{fig:process}
\end{figure}


%%%%%%%%%%%%%%%%%%%%%%%%%%%%%%%%%%%%%%%%%%%%%%%%%%%%%%
\paragraph{Step 1: Offline learning to predict masked outcomes.}
We train an autoregressive sequence model $p_{\theta}$, parameterized by $\theta \in \Theta$, to predict missing outcomes, conditioned on the task prior information $Z$, and limited previously observed context and outcome tuples $(X, Y)$. This enables us to generate hypothetical completions of the potential outcome table $\tau$ later in the online decision-making phase. Formally, this model specifies a probability $p_{\theta}(Y_{t}^{(a)} \mid Z, X_{1:T}, Y_{1 : t-1}^{(a)} )$ of observing outcome $Y_{t}^{(a)}$ from the next interaction conditioned on the current context $X_t$, prior information $Z$, and the previous contexts and outcomes $(X_{1:t-1}, Y_{1: t-1}^{(a)})$. These one-step conditional probabilities generate a probability distribution over missing task $\tau$ outcomes. 


Specifically, we use historical data $\Dtrain$ to minimize the following loss function, which can be optimized using stochastic gradient descent (Algorithm \ref{alg:offline}): 
\begin{align}
    \label{eq:train_loss}
    &- \frac{1}{|\Dtrain|} \hspace{-1mm} \sum_{\tau \in \Dtrain} \sum_{a \in \MC{A}_{\tau}} \sum_{t=1}^T \log p_{\theta} \big( Y_t^{(a)} \mid  Z_{\tau}, X_{1:t}, Y_{1:t-1}^{(a)} \big)
\end{align}
The offline learning procedure can be used with any sequence model class. The quality of the decision-making algorithm depends on the quality of the learned sequence model $p_\theta$; Our regret bounds (Section \ref{sec:regret}) formalize this.
\begin{algorithm}[h]
  \caption{Offline training of a sequence model}
  \label{alg:offline}
  \begin{algorithmic}[1]
    \REQUIRE Training data $\Dtrain$, model class $\{p_\theta\}_{\theta \in \Theta}$
    %%%%%%%%%%%%%%%%%%%%%%%%%%%%%%%%%%%%%%%%%%%%%%%%
    \WHILE{not converged}
        \STATE Sample a mini-batch of tasks $\MC{D}^{\TN{mini-batch}} \subset \Dtrain$
        \STATE Compute loss in \eqref{eq:train_loss} using tasks $\tau \in \MC{D}^{\TN{mini-batch}}$ \\(optionally permute the order of $(X_t, Y_t^{(a)})$ tuples)
        \STATE Backpropagate and take gradient step to update $p_\theta$
    \ENDWHILE
  \end{algorithmic}
\end{algorithm}


Optionally, when training to minimize the loss \eqref{eq:train_loss}, we can permute the order of the tuples $(X_t, Y_t^{(a)})$ over time, since they are exchangeable under Assumption \ref{assump:contextualBandit}. Learning an approximately exchangeable sequence model mirrors recent work on neural processes \citep{garnelo2018neural,jha2022neural,NguyenGr22,lee2023martingale} and prior-data fitted networks \citep{MullerHoArGrHu22}, connecting also to a long tradition in Bayesian statistics and information theory \cite{dawid1984present, barron1998minimum}. 



%%%%%%%%%%%%%%%%%%%%%%%%%%%%%%%%%%%%%%%%%%%%%%%%%%%%%%
\paragraph{Step 2: Online decision-making using the learned sequence model.}
\label{sec:online}
After a sequence model $p_\theta$ is trained offline, it is deployed and used for online decision-making. No additional training of $p_\theta$ is needed. Instead the sequence model learns ``in-context'' by conditioning \citep{brown2020language}. We implement generative Thompson sampling (Algorithm \ref{alg:Thompson}) by using $p_\theta$ to autoregressively generate (impute) missing outcomes in $\tau$ for each candidate action $a \in \MC{A}_{\tau}$, forming $\tau_t$. In the generation procedure, we sample both missing outcomes $Y$ and future contexts $X$ (which is known by Assumption \ref{assump:context}).
Recall that generative Thompson sampling uses both the observed and generated outcomes, $\tau_t$, to fit a policy and selects the best action according to that policy. 
Good decision-making relies on the model $p_\theta$ matching the meta-bandit task distribution $p^*$ closely (see Section \ref{sec:regret}). 


\begin{figure*}[h]
    \centering
    \includegraphics[width=\linewidth]{figures/generation-steps.png}
    \caption{Posterior sampling via autoregressive generation (Algorithm \ref{alg:posterior_sample}). }
    \label{fig:autoregressive-generation}
\end{figure*}

%%%%%%%%%%%%%%%%%%%%%%%%%%%%%%%%%%%%%%%%%%%%%%%%
\begin{algorithm}[h]
  \caption{Posterior sampling via autoregressive generation}
  \label{alg:posterior_sample}
  \begin{algorithmic}[1]
    \REQUIRE Sequence model $p_{\theta}$, actions $\MC{A}_{\tau}$, history $\HH_t$
    \STATE For each $a \in \MC{A}_{\tau}$, construct $\MC{M}^{(a)}$ the set of timesteps $i \in [1 \colon T]$ for which $Y_i^{(a)}$ has not been observed in $\HH_t$
    \STATE For each $a \in \MC{A}_{\tau}$, construct ordering $\prec_{a}$ placing observed outcomes before unobserved ones
    \STATE $\hat{X}_{1:t+1} \gets X_{1:t+1}$ (observed contexts from $\HH_t$)
    \STATE Sample future contexts $\hat{X}_{t+1}, \dots, \hat{X}_T$ (Assump. \ref{assump:context})
    \FOR{$a \in \MC{A}_{\tau}$}
    \FOR{$i \in \{1,\ldots,T\}$ in order of $\prec_{a}$}        
        \IF{$i \not\in \MC{M}^{(a)}$} 
            \STATE $\hat{Y}_i^{(a)} \gets Y_i^{(a)}$
        \ELSE
            \STATE Sample $\hat{Y}_i^{(a)} \sim p_{\theta}(\cdotspace \mid Z, \{\hat{X}_j, \hat{Y}_j^{(a)} \}_{j \prec_{a} i}, \hat{X}_i )$
        \ENDIF
    \ENDFOR
    \ENDFOR
    \STATE \TN{\bo{Return:}} \, $\hat{\tau}_t \gets \big\{ Z_\tau, \hat{X}_{1:T}, \{ \hat{Y}_{1:T}^{(a)} \}_{a \in \MC{A}_{\tau}} \big\}$
  \end{algorithmic}
\end{algorithm}

We formally describe the posterior sampling via autoregressive generation procedure to form samples $\hat{\tau}_t$ in Algorithm~\ref{alg:posterior_sample}; This subroutine can be used in line 3 of Algorithm \ref{alg:Thompson} to implement generative Thompson sampling. In Algorithm~\ref{alg:posterior_sample}, we use $\mathcal{M}_a \subset \{1,\ldots,T\}$ to denote the timesteps $t$ for which $Y_t^{(a)}$ has not been observed. 
To clarify the order of generating outcomes in $\hat{\tau}_t$, we permute observed outcomes to always precede missing ones; this matches the ordering used for generation in Figure \ref{fig:autoregressive-generation}. 
Specifically, we use $\prec_a$ to denote this ordering for an action $a \in \MC{A}_{\tau}$; We use $i\prec_a j$ whenever either (a) $i < j$ or (b) $i \notin \mathcal{M}_a$, but $j \in \mathcal{M}_a$.







\section{Regret of generative Thompson sampling with an imperfect imputation model}
\label{sec:regret}
In this section, we present a generalization of the regret bound for the Thompson sampling via generation algorithm from Section \ref{sec:regret-perfect}. Our generalization is notable because the sequence model only affects the regret bound through its offline prediction loss, which means any sequence model class can be used---even sequence models that are not exactly exchangeable. Moreover, our result shows that the lower offline prediction loss of the sequence model $p_\theta$ translates into a better regret guarantee. Our result effectively \textit{reduces a difficult online decision-making problem to one of training an accurate sequence prediction model.} 

Specifically, our regret bound will depend on the following population-level version of the training loss from \eqref{eq:train_loss} (the expectation below averages over the task distribution $p^*$):
\begin{align}
    \ell(p_\theta) = - \E \bigg[ \sum_{a \in \MC{A}_{\tau}} \sum_{t=1}^T \log p_{\theta} \big( Y_t^{(a)} \mid Z_{\tau}, X_{1:t}, Y_{1:t-1}^{(a)} \big) \bigg].
    \label{eq:pop_loss}
\end{align}

%%%%%%%%%%%%%%%%%%%%%%%%%%%%%%%%%%%%%%%%%%%%%%

\begin{theorem}[Regret bound for generative Thompson sampling with an imperfect imputation model]
    \label{thm:psarRegret}
    Let Assumptions \ref{assump:contextualBandit}, \ref{assump:indepAction}, and \ref{assump:context} hold. Under generative Thompson sampling (Algorithm \ref{alg:Thompson}) using $p_\theta$, $\psar(p_\theta)$, 
    \begin{align}
        \label{eqn:psarregret}
      \Delta \big( \psar(p_\theta) \big)
      & \leq 
        \underbrace{ \sqrt{ \frac{ |\MC{A}_{\tau}| \cdot H \big( \piX \mid Z_\tau, X_{1:T} \big) }{2 T} } }_{\TN{Regret bound for Thompson sampling}} 
        + \underbrace{ \sqrt{ 2 \big\{ \ell(p_\theta) - \ell(p^*) \big\} } }_{\TN{Penalty for sub-optimal prediction}}.
    \end{align}
\end{theorem}
Comparing the results of Theorem \ref{thm:psarRegret} and Proposition \ref{prop:psarRegretPerfect}, we can interpret the ``cost'' of using an imperfect imputation model $p_\theta$ in generative sequence modeling instead of $p^*$ as $\sqrt{ 2 \big\{ \ell(p_\theta) - \ell(p^*) \big\}}$.
In other words, the regret is penalized depending on how well $p_\theta$ approximates $p^*$. This penalty term, which does not vanish as $T$ grows, is unavoidable in bandits; see Appendix \ref{app:misspecifiedPriors} for further discussion. 
Also, we do not attempt to mathematically characterize when making the gap $\ell(p_\theta) - \ell(p^*)$ small via offline learning is possible, since it would involve stringent and unrealistic conditions derived from loose generalization bounds for neural networks. Model performance ``scaling-laws'' with the quantity of training data has been studied with great interest empirically \citep{henighan2020scaling}.


\paragraph{Novelty and proof approach.}
Note that Proposition \ref{prop:psarRegretPerfect} itself (a direct corollary of Theorem \ref{thm:psarRegret}) is novel due to the conditional entropy term we introduce, which quantifies the benefit of using prior information $Z$ and can be bounded using VC dimension (Section \ref{sec:regret-perfect}). 
What is particularly novel about Theorem \ref{thm:psarRegret} is that we are able to carry out our analysis even in cases where the imputation model $p_\theta$ is misspecified and does not correspond to any proper way of performing Bayesian inference. 


In our previous work \citep{psar2024}, we prove a regret bound with a similar prediction loss penalty for model misspecifcation for a generative Thompson sampling algorithm for a multi-armed, non-contextual bandit algorithm. Although they also use an information theoretic analysis, in their simpler setting without contexts, they do not need to introduce the concept of a general ``oracle'' policy fitting procedure and as a result do not provide an information theoretic analysis that extends to infinite policy classes as we do.
Moreover, we were not able to directly build on the proof approach used in \citet{psar2024} because they critically rely on the fact that under $p^*$, unobserved outcomes $Y$ are exchangeable given the history; However, for contextual bandits, unobserved outcomes $Y$ are not exchangeable conditional on observing their associated contexts $X$. 

\citet{wen2021predictions} also consider a non-contextual, multi-armed Thompson sampling algorithm that incorporates a generative outcome model. Our algorithm is not a generalization of theirs, since they require specifying a prior over latent environment parameters. Their regret bound allows for the generative model to be misspecified and requires a history-dependent KL divergence term to be small, which is different from our prediction loss penalty. Despite these differences, we carefully integrate bounding techniques from \citet{wen2021predictions} in proving our contextual bandit regret bounds. What is surprising is that their information theoretic bounding techniques developed for multi-armed, non-contextual bandits are relevant for contextual bandits with infinite policy classes, which has been of independent interest in the literature, even for correctly specified model settings \citep{neu2022lifting,infotheoreticNonstationary}.


%!TEX root=iclr2025_conference.tex
\paragraph{Training based Alignment}

Supervised fine-tuning and instruction tuning~\citep{wei2021finetuned} are common methods to align an LLM to labeled data. RLHF~\citep{christiano2017deep,ziegler2019fine,lee2021pebble,nakano2021webgpt,snell2022offline} methods can align an LLM directly to human preferences. First, a reward model is trained on a dataset of human preferences using the Bradley Terry model ~\citep{bradleyterry1952paired} and then the LLM is updated, based on the reward model, using an RL algorithm such as PPO~\cite{schulman2017proximal}. However, updating the LLM with RL is expensive and researchers have explored cost-effective alternatives.

\cite{liu2023chain} convert the preference data into sequence of sentences which are then used to fine-tune the LLM. \cite{dong2023raft} used the reward model to filter high quality training samples and fine-tunes on them avoiding undesirable behavior. DPO \cite{rafailov2023direct,rafailov2024qfunction} avoids learning a reward model explicitly and finds an equivalent objective to RLHF which can be optimized by supervised learning. Even though the resulting optimization is cheaper than RL, nonetheless, it still involves updating the LLM.

Preference data itself provides sequence-level supervision. Some works have atttempted to collect and use fine-grained preferences by using either human annotators~\citep{wu2024fine} or LLMs~\citep{cao2024sparserewardsenhancingreinforcement}.


\paragraph{Guided Decoding}

In the guided decoding literature, a number of methods consider guidance at a step or process level~\citep{welleck2022naturalprover,uesato2022solving, lightman2023let, krishna2022rankgen,li2023making, khalifa2023grace, yao2023tree}. 

Some methods have applied token-level functions~\citep{dathathri2019plug, krause2021gedi, yang2021fudge,chaffin2022ppl, liu2023attribute} but they do not consider RGTG based on preference data. 

\citet{khanov2023alignment} introduce an RGTG method but they rely on a full-sequence reward model for partial sequence decoding. \citet{deng2023reward} learn to distill a partial sequence reward model, starting from the full-sequence model using a square loss function. \citet{mudgalcontrolled} employ a similar approach but instead of using preference data, generate a dataset by roll-outs from the base LLMs. \citet{han2024value} also use the base LLM to gather a dataset but employ TD learning to train the partial sequence reward model. Different from these works, \citet{zhao2024probabilistic} derive an RGTG method based on sequential Monte Carlo and demonstrate that it can approximate RLHF.



  




\section{Proof of Concept Experiments}
\label{sec:experiments}

%\begin{itemize}
%    \item joint exploration non e' spesso un opzione
%    \item specificare che le policy sono decentralizzate a differenza di tutti i casi precedenti
%    \item decentralizzata con feedback decentralizzato non si coordina e il problema e' abbastanza semplice da portare a policy quasi deterministiche
%\end{itemize}



%\mirco{questo primo paragrafo è un po' convoluto. Prova a ristruttura la sezione in questo modo: quali sono le domande a cui cerchiamo risposta? Quali sono i domini sperimentali? Quali sono gli algoritmi che compariamo? Quali sono i take away? Per l'ultimo potresti anche evidenziare qualche frase in grassetto o emph con le principali conclusioni empiriche}

In this section, we provide some empirical validations of the findings discussed so far. Especially, we aim to answer the following questions: (\textbf{a}) Is Algorithm~\ref{alg:trpe} actually capable of optimizing finite-trials objectives? (\textbf{b}) Do different objectives enforce different behaviors, as expected from Section~\ref{sec:problem_formulation}? (\textbf{c}) Does the \emph{clustering} behavior of mixture objectives play a crucial role? If yes, when and why?\\
Throughout the experiments, we will compare the result of optimizing finite-trial objectives, either joint, disjoint, mixture ones, through Algorithm~\ref{alg:trpe} via fully decentralized policies. The experiments will be performed with different values of the exploration horizon $T$, so as to test their capabilities in different exploration efficiency regimes.\footnote{The exploration horizon $T$, rather than being a given trajectory length, has to be seen as a parameter of the exploration phase which allows to tradeoff exploration quality with exploration efficiency.} The full implementation details are reported in Appendix~\ref{apx:exp}.
\vspace{-6pt}
\paragraph*{Experimental Domains.}~The experiments were performed on two domains. The first is a notoriously difficult multi-agent exploration task called \emph{secret room}~\citep[MPE,][]{pmlr-v139-liu21j},\footnote{We highlight that all previous efforts in this task employed centralized policies. We are interested on the role of the entropic feedback in fostering coordination rather than full-state conditioning, then maintaining fully decentralized policies instead.} referred to as  Env.~(\textbf{i}). In such task, two agents are required to reach a target while navigating over two rooms divided by a door. In order to keep the door open, at least one agent have to remain on a switch. Two switches are located at the corners of the two rooms. The hardness of the task then comes from the need of coordinated exploration, where one agent allows for the exploration of the other. The second is a simpler exploration task yet over a high dimensional state-space, namely a 2-agent instantiation of \emph{Reacher}~\citep[MaMuJoCo,][]{peng2021facmac}, referred to as Env.~(\textbf{ii}). Each agent corresponds to one joint and equipped with decentralized policies conditioned on her own states. In order to allow for the use of plug-in estimator of the entropy~\citep{paninski2003}, each state dimension was discretized over 10 bins.


\begin{figure*}[!]
    \centering
    \input{figures/pretraining_legend.tex}
    %\hfill
    \vfill
    %vspace{-0.2cm}
    \begin{subfigure}[b]{0.3\textwidth}
        \includegraphics[width=\textwidth]{figures/room_150_AverageReturnnokl.pdf}
        %\vspace{-0.8cm}
        \caption{\centering MA-TRPO with TRPE Pre-Training (Env.~(\textbf{i}), $T=150$).}
        \label{subfig:image9}
    \end{subfigure}
    \hfill
    \begin{subfigure}[b]{0.3\textwidth}
        \includegraphics[width=\textwidth]{figures/room_50_AverageReturnnokl.pdf}
        %\vspace{-0.8cm}
        \caption{\centering MA-TRPO with TRPE Pre-Training (Env.~(\textbf{i}), $T=50$).}
        \label{subfig:image10}
    \end{subfigure}
    \hfill
    \begin{subfigure}[b]{0.3\textwidth}
        \centering
        \includegraphics[width=0.8\textwidth]{figures/hand_100_AverageReturn.pdf}
        %\vspace{-0.8cm}
        \caption{\centering MA-TRPO with TRPE Pre-Training (Env.~(\textbf{ii}), $T=100$).}
        \label{subfig:image11}
    \end{subfigure}
\caption{\centering Effect of pre-training in sparse-reward settings.(\emph{left}) Policies initialized with either Uniform or TRPE pre-trained policies over 4 runs over a worst-case goal. (\emph{rigth}) Policies initialized with either Zero-Mean or TRPE pre-trained policies over 4 runs over 3 possible goal state. We report the average and 95\% c.i.}
\label{fig:pretraining}
\end{figure*}
\vspace{-10pt}
\paragraph*{Task-Agnostic Exploration.}~Algorithm~\ref{alg:trpe} was first tested in her ability to address task-agnostic exploration \emph{per se}. This was done by considering the well-know hard-exploration task of Env.~(\textbf{i}). The results are reported in Figure~\ref{fig:room} for a short exploration horizon $(T=50)$. Interestingly, at this efficiency regime, when looking at the joint entropy in Figure~\ref{subfig:image2}, joint and disjoint objectives perform rather well compared to mixture ones in terms of induced joint entropy, while they fail to address mixture entropy explicitly, as seen in Figure~\ref{subfig:image3}. On the other hand mixture-based objectives result in optimizing both mixture \emph{and} joint entropy effectively, as one would expect by the bounds in Th.~\ref{lem:entropymismatch}. By looking at the actual state visitation induced by the trained policies, the difference between the objectives is apparent. While optimizing joint objectives, agents exploit the high-dimensionality of the joint space to induce highly entropic distributions even without exploring the space uniformly via coordination (Fig.~\ref{subfig:image5}); the same outcome happens in disjoint objectives, with which agents focus on over-optimizing over a restricted space loosing any incentive for coordinated exploration (Fig.\ref{subfig:image6}). On the other hand, mixture objectives enforce a clustering behavior (Fig.\ref{subfig:image6}) and result in a better efficient exploration. 

\paragraph*{Policy Pre-Training via Task-Agnostic Exploration.}~More interestingly, we tested the effect of pre-training policies via different objectives as a way to alleviate the well-known hardness of sparse-reward settings, either throught faster learning or zero-short generalization. In order to do so, we employed a multi-agent counterpart of the TRPO algorithm~\cite{schulman2017trustregionpolicyoptimization} with different pre-trained policies. First, we investigated the effect on the learning curve in the hard-exploration task of Env.~(\textbf{i}) under long horizons ($T=150$), with a worst-case goal set on the the opposite corner of the closed room. Pre-training via mixture objectives still lead to a faster learning compared to initializing the policy with a uniform distribution. On the other hand, joint objective pre-training did not lead to substantial improvements over standard initializations. More interestingly, when extremely short horizons were taken into account ($T=50$) the difference became appalling, as shown in Fig.~\ref{subfig:image9}: pre-training via mixture-based objectives leaded to faster learning and higher performances, while pre-training via disjoint objectives turned out to be even \emph{harmful} (Fig.~\ref{subfig:image10}). This was motivated by the fact that the disjoint objective overfitted the task over the states reachable without coordinated exploration, resulting in almost deterministic policies, as shown in Fig~\ref{fig:333} in Appendix~\ref{apx:exp}. Finally, we tested the zero-shot capabilities of policy pre-training on the simpler but high dimensional exploration task of Env.~(\textbf{ii}), where the goal was sampled randomly between worst-case positions at the boundaries of the region reachable by the arm. As shown in Fig.~\ref{subfig:image11}, both joint and mixture were able to guarantee zero-shot performances via pre-training compatible with MA-TRPO after learning over $2$e$4$ samples, while disjoint objectives were not. On the other hand, pre-training with joint objectives showed an extremely high-variance, leading to worst-case performances not better than the ones of random initialization. Mixture objectives on the other hand showed higher stability in guaranteeing compelling zero-shot performance.
\vspace{-6pt}
\paragraph*{Take-Aways.}~Overall, the proposed proof of concepts experiments managed to answer to all of the experimental questions: (\textbf{a}) Algorithm~\ref{alg:trpe} is indeed able to explicitly optimize for finite-trial entropic objectives. Additionally, (\textbf{b}) \textbf{mixture distributions enforce diverse yet coordinated exploration}, that helps when high efficiency is required. Joint or disjoint objectives on the other hand may fail to lead to relevant solutions because of under or over optimization. Finally, (\textbf{c}) \textbf{efficient exploration} enforced by mixture distributions was shown to be a \textbf{crucial factor} not only for the sake of task-agnostic exploration per se, but also for the ability of \textbf{pre-training via task-agnostic exploration} to lead to \textbf{faster and better training} and even \textbf{zero-shot generalization}.

\section{Discussion}


In this paper, we adopted a learner-centered design approach, beginning with a formative study to identify students' challenges with existing tools. Based on these insights, we developed DBox, a tool that scaffolds students in breaking problems into smaller parts and provides personalized, adaptive support. Our user study demonstrated that DBox improved learners' performance on similar algorithmic problems, increased perceived learning gains, and fostered greater cognitive engagement, achievement, and satisfaction. In this section, we discuss design implications and generalizability based on our key findings.


\ms{
\subsection{Chaining Learners' Thoughts with Visualized Structured UI Components}

Decomposition requires students to effectively organize their thoughts. While visual elements are known to promote structured thinking and support mental model construction \cite{mcdougall2001effects, liu2010mental}, our formative and user studies revealed shortcomings in existing tools like LeetCode and ChatGPT, which rely on textual representations without adequately supporting structured mental models. In contrast, DBox uses an interactive step tree to visually organize learners' thoughts. This feature was praised by 22 of 24 participants for enhancing algorithmic thinking, serving as a progress tracker, and providing value even without AI assistance.

DBox's interactive step tree and tree-based scaffolding demonstrate the broader potential of intelligent tutoring systems (ITS) to promote active learning and self-regulated problem-solving in fields requiring problem decomposition. Similar principles could benefit STEM education, such as physics or engineering, by externalizing abstract concepts and facilitating multi-step problem-solving. Additionally, progress-tracking visual components may inspire designs for professional training tools in areas like medical diagnostics or software engineering.

\subsection{Promoting Independent Thinking and Active Decomposition Learning}

\subsubsection{\textbf{Transforming Learners from Passive Readers to Active Thinkers}}

Many coding tools provide direct answers or solutions \cite{kazemitabaar2023novices, phung2023generating}, which, while efficient, often bypass opportunities to develop critical problem-solving skills. In contrast, DBox cultivates students' decomposition abilities through structured scaffolding, fostering critical thinking and self-regulated learning in line with learning by doing \cite{anzai1979theory} and constructivist principles \cite{tobias2009constructivist}.

To strengthen decomposition skills, DBox first encourages students to develop their own decomposition strategies by coding or building a step tree from scratch. While DBox can generate parts of a step tree from a student's existing code, these steps are derived from the learner's own reasoning, with DBox acting solely as a modality converter. Besides, DBox provides feedback on tree node statuses, identifying potential errors or missing steps without directly showing the correct answer, challenging students to critically evaluate and refine their decomposition plans.


DBox's scaffolded hint system further supports decomposition skill development by providing adaptive guidance tailored to the student’s progress without overwhelming them. All hints are based on the learner's current decomposition skeleton, with the most detailed hint—``reveal substep''—triggered only after repeated attempts and struggles. Notably, even the most detailed hints prompt only one substep, requiring students to complete the rest independently. As shown in Sec \ref{hintusage}, only 19\% of hints are this detailed, with students primarily relying on simpler, thought-provoking question hints. This scaffolded support system balances guidance and independent thinking, keeping students engaged during challenges without compromising their ability to independently decompose problems \cite{kinnunen2006students}.

Based on these findings, we recommend fostering active problem-solving by shifting students from passive content consumption to active solution creation. Designers could adopt layered scaffolding, starting with minimal guidance and increasing support as needed, to help students progressively master decomposition skills while maintaining confidence and avoiding frustration. Additionally, adaptive learning techniques, such as real-time feedback and progress tracking, can further tailor the support to individual decomposition barriers, encouraging deeper engagement with decomposition tasks. Moreover, designers could integrate metacognitive strategies, such as encouraging students to articulate or reflect on their decomposition approaches, to further enhance critical thinking and foster habits of independent thinking.




\subsubsection{\textbf{Choice of Scaffolding: Balancing Independent Problem-Solving and Efforts}}

Scaffolding involves providing tailored support to help learners accomplish tasks they cannot yet complete independently \cite{kim2011scaffolding, tobias2009constructivist}. Broadly, scaffolding strategies fall into two categories \cite{van2010scaffolding}: (1) gradually reducing assistance as learners gain proficiency, and (2) encouraging independent problem-solving while offering incremental support to address challenges. DBox adopts the second approach, emphasizing independent thinking and encouraging learners to actively decompose problems \cite{zimmerman2013theories}. While our scaffolding strategies successfully enhanced critical thinking, satisfaction, and perceived usefulness, they also led to increased cognitive effort (Sec. \ref{Effects_on_UX}). This tradeoff underscores the importance of carefully balancing cognitive effort with the promotion of independent thinking.

Future designs could incorporate adaptive scaffolding that adjusts support dynamically based on learner proficiency, reducing unnecessary effort in areas where students have demonstrated competence. Additionally, while incremental scaffolding was effective for algorithmic problem-solving, tailoring strategies to different educational contexts could enhance their applicability in diverse domains. Such adaptive, context-specific approaches could further optimize the balance between support and independence in learning environments.


\subsection{Supporting Personalized Algorithmic Programming Learning}

\subsubsection{\textbf{Prioritizing Learners' Own Solutions Over Optimality}}

Algorithmic problems often have multiple solutions with varying time and space complexities. DBox prioritizes independent exploration by supporting learners' strategies rather than steering them toward a single ``optimal'' solution. Using LLM-driven prompts, it evaluates and guides each step based on the learner's reasoning, preserving their step decomposition and respecting their input—even when errors occur. While some solutions may not be the most efficient, this approach fosters autonomy by aligning feedback with learners’ thought processes instead of enforcing rigid standards.

Our user study showed that this approach improves learning outcomes and is well-received by students. We recommend designing systems that respect personalized problem-solving strategies by aligning feedback with learners' reasoning while allowing for diverse approaches. Designers should balance flexibility and rigor, using prompts and interfaces that support varied strategies while gently guiding learners toward effective solutions.


\subsubsection{\textbf{Catering to Individual Learning Styles and Contextual Needs}}

DBox accommodates diverse problem-solving approaches with two input modes: coding and natural language descriptions. Each mode offers distinct advantages tailored to different learners, stages, and situations. Learners can switch seamlessly between modes, with progress automatically synced across the interface. Features such as verifying code-step alignment ensure strong integration between modes.

Our findings reveal that this flexibility enhances user experience. Participant interaction logs and interviews revealed three usage patterns, highlighting that each mode fits different needs: code mode works well for students with a clear and detailed problem-solving plan already, while the step tree with natural language descriptions helps less experienced students with only a basic idea who are not ready to write code directly, boosting their confidence.


We argue there is no universal “best” mode for programming education—each has unique benefits depending on the learner habits, expertise, and context. Future tools should provide flexibility, like DBox, or use adaptive algorithms to recommend modes based on user needs and context. This flexibility highlights the importance of designing educational tools that accommodate varying levels of expertise and problem-solving styles, which can be generalized to other domains requiring personalized learning \cite{bernacki2021systematic}.

\subsection{Appropriate Usage of LLMs for Supporting Algorithmic Programming Learning}

\subsubsection{\textbf{Caution About LLM Errors}}

Although LLMs have shown strong performance in coding tasks \cite{finnie2023my, leinonen2023using}, they remain prone to errors. Our technical evaluation and user study revealed that even with comprehensive context—such as problem statements, user code, and natural language steps—LLM sometimes misinterprets user descriptions. These errors likely arise from discrepancies between the natural language used by students and the formal, precise language the LLM was trained on, which is primarily sourced from web-based code and comments \cite{liu2023wants}.

Such misinterpretations can hinder learning by causing confusion or frustration. While future improvements to training data and GPT versions may mitigate these issues, design strategies can help address them. \textbf{First}, LLMs should avoid giving direct solutions and instead focus on fostering active problem-solving through explanations and hints. \textbf{Second}, feedback could be paired with interactive features, like a ``Run Code'' option, allowing students to validate their reasoning. \textbf{Third}, simple tutorials could teach users how to phrase their descriptions more clearly, improving LLM's understanding. Additionally, future tools could integrate a ``Language Enhancement'' feature to suggest improvements or assess the clarity of descriptions, aiding LLM in accurately capturing user intent. Most importantly, we recommend designers prioritize technical feasibility, such as conducting rigorous evaluations like ours, before fully integrating LLMs into programming learning tools.
}



\subsubsection{\textbf{Learner-LLM Co-Decomposition of Solutions: Learner as Leader, LLM as Aid}}

A central feature of DBox is the construction of a step tree, where students break solutions into steps and sub-steps. The LLM supports this by mapping code to step descriptions, evaluating them, and offering hints. However, students maintain full control, deciding how to decompose problems and define each step, fostering independent thinking. The LLM acts solely as an aid, using a scaffolding approach to support the development of learners' Zone of Proximal Development (ZPD) \cite{chaiklin2003zone}. Unlike tools like ChatGPT or Copilot that dominate problem-solving, DBox fosters deeper cognitive engagement. Students reported greater accomplishment and found this approach more effective for learning.

This contrasts with existing human-AI collaboration paradigms in non-educational scenarios where AI usually suggest options, leaving final decisions to users \cite{dang2023choice, gao2024collabcoder, gebreegziabher2023patat, ma2019smarteye, ma2022glancee}, such as in human-AI decision-making \cite{ma2023should, ma2024towards, ma2024you}. Some educational tools, like Jin et al. \cite{jin2024teach}, use LLMs to generate solutions for students to evaluate, which aids in syntax learning but such ``LLM-generate then learner-evaluate'' approach is less effective for algorithmic problem-solving, where constructing solutions is key. Just evaluating LLM-generated contents can place a cognitive anchor on learners \cite{furnham2011literature}, limiting independent thinking and creativity. Thus, task allocation between humans and AI should align with the educational context (e.g., whether it is basic knowledge/concept learning or higher-level creative thinking). Future LLM-based educational tools should carefully define the division of roles between LLMs and learners, tailoring it to specific learning contexts and goals.




% \subsubsection{Human-LLM Co-Decomposition of Solution: AI Should Judge Instead of Recommending}

% A core interaction in DBox is the construction of a step tree, where the entire solution is broken down into a series of steps and sub-steps. We refer to this as the human-LLM co-decomposition process. In this process, the LLM behind DBox plays three roles: First, it maps the student's written code into step descriptions. Second, it evaluates the status of each step and sub-step (whether they are correct, incorrect, missing, or need further decomposition). Third, it provides hints for incorrect or missing steps or sub-steps. However, the actual construction of the step tree—such as dividing the solution into steps and sub-steps and determining the content of each node—remains primarily the student's responsibility.

% This division of labor maximizes student engagement in independent thinking and problem-solving. The LLM does not provide any suggestions for decomposition nor directly recommend content for specific steps, aligning with the scaffolding educational approach, where guidance is provided appropriately, but the main task of forming the solution is left to the students.

% In contrast, when students directly seek help from an LLM, such as asking questions in ChatGPT or using Copilot for code completion, the LLM takes too much initiative by directly offering ideas or code. In our co-decomposition design, however, students demonstrated higher cognitive engagement and more active critical thinking. Furthermore, students reported that constructing solutions in this way gave them a greater sense of achievement and made them feel the process was more beneficial for learning, leading to higher satisfaction with the experience.

% Related work has proposed similar approaches. For instance, XXX, in the context of problem-solving, uses the "learning by teaching" concept, where students take on the tasks of judging and teaching, while the LLM generates most of the solutions. Compared to our approach, their division of labor between the student and the LLM is reversed. This method works well in introductory programming, where the focus is on mastering syntax. Having students guide the LLM to generate code or evaluate potentially incorrect code produced by the LLM is an effective way to quiz them. However, in our work, which focuses on algorithmic programming, the key step is constructing a solution from scratch. If the LLM builds the solution, leaving students only to judge it, it hampers their independent thinking.

% Thus, when designing LLM-based educational tools in the future, it is crucial to consider the specific context to effectively allocate tasks between the student and the LLM, ensuring that students derive the maximum benefit from the co-decomposition process.


% \subsection{Future Design Opportunities}

% \emph{Providing Appropriate Generative Assistance:} While DBox promotes independent problem-solving, some users showed interest in features like auto-completion for trivial coding tasks. Future versions could balance promoting independence with targeted assistance by enabling adjustable difficulty levels and offering contextual suggestions when appropriate.

% \emph{Covering All Stages of Algorithmic Programming:} DBox currently lacks a focus on foundational algorithm instruction and problem comprehension. Future iterations could include features like generating distractor solutions, input-output tests, and step-by-step rephrasing to help students grasp key concepts and understand the coding problem.

% \emph{Combining Step Trees with Dialogue:} Users can currently describe their thought processes but cannot ask questions. Adding a dialogue system to the step tree would allow students to share challenges and ask follow-up questions. GPT could then provide guided feedback without giving direct answers, supporting independent problem-solving.





% \emph{Other Important Features.} DBox could offer more control by allowing users to select specific parts of their code for targeted evaluation and guidance. A ``review'' feature could also help students reflect on key stumbling points, understand where their thought process went wrong, and how they eventually solved the problem.


% \subsection{Future Design Opportunities}

% \emph{Providing Appropriate Generative Assistance.} Our tool primarily focuses on encouraging users to create the step tree and write the code independently, with the system mainly serving as a judge. However, users expressed a desire for some intelligent completion features, particularly for repetitive or simple code, allowing them to focus their efforts on learning the key parts. Future improvements should strike a balance between fostering independent thinking and providing appropriate assistance. One approach could be designing basic rules where the tool offers intelligent suggestions and completions for parts unrelated to the core logic, while maintaining the current level of independence for key learning areas. Additionally, the system could offer different modes, allowing users to choose the level of assistance, from basic judgment-only feedback to a combination of judgment, guidance, necessary completions, and even on-demand suggestions.

% \emph{Covering All Stages of Algorithmic Programming.} Currently, our system does not cover the basic teaching of algorithms or the problem comprehension stage. In the future, to address the diversity and uncertainty in solutions and help students grasp multiple approaches, we could expand assistance during the idea formation phase. For example, GPT could generate multiple potential solutions with distractors, prompting students to identify the one that meets the problem's complexity requirements. We could also introduce specialized algorithm training, where students select a specific algorithm, and the system’s guidance focuses solely on that algorithm. To assist with problem comprehension, we could incorporate input-output tests to check students' understanding of the problem and step-by-step rephrasing to help them grasp more complex problems.

% \emph{Combining Interactive Step Trees with Dialogue Boxes.} Sometimes users want to describe their difficulties, and currently, we ask them to outline their thought processes. Additionally, users may want to ask follow-up questions. In the future, we could combine the structured step tree with a small dialogue box. The primary goal would still be to construct the step tree, but users could engage in a conversation with GPT in the context of the current step tree or a specific step. Importantly, GPT should guide the user without revealing direct answers.

% \emph{Other Important Features.} First, DBox could offer learners more control, such as allowing users to select specific parts of the code for targeted evaluation and guidance. We could also introduce a summary feature for key stumbling points, helping students reflect on the challenges they faced, where their thought process went wrong, and how they eventually overcame the problem.




\subsection{Limitations and Future Work}

This study has several limitations. \emph{First}, we tested DBox's effectiveness on only two problem types; future work should examine a broader range of algorithms. \emph{Second}, participants engaged in just one learning session per condition due to time constraints, whereas mastering algorithmic problems typically requires extended practice. Longitudinal studies should explore how DBox supports skill development over time, including changes in mental models and skill retention. \emph{Third}, we assessed learning gains based on correctness in a test session using similar learning and test problems. Future research should evaluate knowledge transfer to less similar problems. Due to time constraints, we conducted a single post-test rather than a pre-post comparison. While pre-test expertise filtering and randomization minimized prior familiarity effects, a more rigorous pre-post design would yield more accurate learning gain measurements. Looking ahead, we plan to release DBox as a Chrome plugin for integration with existing coding platforms, enabling large-scale field studies. This will allow for the collection of long-term usage data and periodic surveys to identify usage patterns and learning experiences over time.



% This study has several limitations. First, in our within-subject design, we selected two types of algorithm problems—Greedy and Binary Search—and randomly assigned them to two conditions (DBox and baseline). However, selection bias may still exist, as some participants might naturally excel at one type of algorithm. Although we addressed this by filtering participants' proficiency through a pre-test and using a Latin Square design, further validation across a broader range of algorithms is needed in future work.

% Second, students experienced only one learning session per condition before the test session. While this allowed for a fair comparison, mastering algorithmic problems typically requires extended practice. Future work should explore how DBox supports students' long-term improvement in algorithmic skills. Longitudinal studies could provide insights into changes in learners' mental models, allowing students more time to deepen their understanding and refine their decomposition methods. Additionally, retention tests could assess whether students can still apply learned problem-solving methods after a time gap.

% We measured learning gains through correctness scores in the test session, with relatively similar learning and test problems. Future work should explore students' ability to transfer their knowledge to problems with lower similarity. Due to time constraints, we opted for a single post-test rather than a pre-post comparison. While we minimized prior familiarity effects by filtering participants and randomizing problem assignments, future studies could adopt a more rigorous pre-post test design for better measurement of learning gains.

% Looking ahead, we plan to release DBox as a Chrome plugin for integration with existing online coding platforms and large-scale real-world testing. In such settings, where students may be more motivated (e.g., preparing for algorithm interviews), we can gather long-term usage data while ensuring privacy. We also plan to conduct periodic surveys to track changes in students' usage patterns and learning experiences over time.



% \subsection{Limitations and Future Work}

% This study has several limitations. First, in our within-subjects study, we selected two types of algorithm problems, Greedy and Binary Search, and randomly assigned them to two conditions, DBox and the baseline. However, there may still be selection bias, where some participants were naturally better at one type of algorithm. While we mitigated this issue to a large extent by filtering participants' proficiency through a pre-test and employing a Latin Square design to randomize the problem-condition assignment, there is still room for improvement. Future work should validate DBox's effectiveness across a broader range of problem types.

% Second, in our experiment, students only experienced one learning session in each condition before moving on to the test session. Although this comparison was fair (as both conditions had only one learning session), mastering an algorithmic problem often requires extended practice. Future work should explore how DBox can help students gradually improve their algorithmic programming skills over time. Longitudinal studies may reveal significant changes in learners' mental models, providing more time for them to understand a specific algorithm and enhance their decomposition methods. Additionally, future studies could include retention tests to measure whether students can still effectively apply previously learned problem-solving methods after a period of time.

% Furthermore, when objectively measuring students' learning gains, we calculated their correctness score in the test session. On the one hand, the learning session and test session problems had a relatively high degree of similarity. Future work should investigate whether students can transfer what they have learned to solve problems of the same algorithm type with lower similarity. On the other hand, due to time constraints, we did not include a pre-post test comparison, opting for a single post-test instead. This result might be influenced by students' pre-existing familiarity with the problems. Although we mitigated this issue by filtering for familiarity (ensuring participants were not too familiar with the problems) and randomizing the problem assignments, future work could include a more rigorous pre-post test design to better calculate students' learning gains.

% Moreover, DBox is currently only applied in algorithmic programming, specifically solving algorithm problems. However, this decomposition-based computational thinking approach could be extended to other learning scenarios, such as project-based learning. Future work could explore how to adapt DBox to broader educational contexts outside of algorithmic programming.

% Looking forward, we aim to deploy DBox in real-world algorithm courses. Since algorithms are a core required subject in undergraduate computer science curricula, we hope to investigate how students who have just learned algorithm concepts use DBox to develop their problem-solving skills. Additionally, we plan to convert DBox into a Chrome plugin and release it in the Chrome Web Store for real-world testing. This would allow DBox to seamlessly integrate with existing online coding platforms, enabling large-scale experiments. In such settings, students' motivation may be stronger (e.g., a graduate preparing for an algorithm interview), leading to more realistic usage patterns. Students could use DBox to tackle a wide variety of algorithm problems. We hope to collect long-term (e.g., six-month) usage data from real-world users while ensuring privacy, and use periodic surveys to capture changes in students' usage patterns and learning experiences over time.





\section{Conclusion}
% In this paper, we introduced Decomposition Box (DBox), a novel tool designed to scaffold learners in decomposing problems during algorithmic programming learning. Based on insights from a formative study, we identified key design goals to address the limitations of existing tools in algorithmic programming education. DBox supports two critical stages of the programming process: idea formation and idea implementation. By offering two modes (code mode and language mode), it encourages users to independently develop their solution strategies. The interactive, visual step tree helps students break down problems and build a structured mental model. DBox provides fine-grained, step-level feedback, enabling students to quickly identify issues, while its multi-level guidance offers targeted support without undermining independent thinking.

% Our user study demonstrated that DBox led to significantly higher learning gains, cognitive engagement, and critical thinking. Students reported a stronger sense of achievement and found the assistance both appropriate and effective for their learning. We identified three main usage patterns, underscoring the importance of respecting students' problem-solving habits and offering them autonomy. The learner-LLM co-decomposition model we designed promotes independent thinking while allowing the LLM to contribute meaningfully, even with occasional imperfections. 

% We hope the formative study, design goals, features, technical evaluation, and key findings from this work will inspire future research on developing educational tools for broader programming learning.
In this paper, we introduced DBox, an interactive tool designed to help learners decompose algorithmic programming problems by supporting both solution formation and implementation. Featuring an intuitive tree-like box widget, DBox accepts input in both code and natural language, fostering independent problem-solving while its step tree structure helps learners develop structured mental models. It provides step-level feedback and layered guidance without compromising learner autonomy.
Our user study showed that DBox significantly improved learning outcomes, cognitive engagement, and critical thinking, with students reporting a greater sense of achievement and finding the support highly effective. Additionally, we identified three key usage patterns, highlighting the importance of accommodating individual problem-solving styles. Moreover, our findings suggest that the learner-LLM co-decomposition approach fosters independent thinking while providing meaningful guidance, even with occasional imperfections.
We hope the insights from our system design will inspire future research on integrating LLMs into educational tools for programming learning.


\bibliography{bib,bib-hong}
\bibliographystyle{plainnat}

%%%%%%%%%%%%%%%%%%%%%%%%%%%%%%%%%%%%%%%%%%%%%%%%%%%%%%%%%%%%%%%%%%%%%%%%%%%%%%%
%%%%%%%%%%%%%%%%%%%%%%%%%%%%%%%%%%%%%%%%%%%%%%%%%%%%%%%%%%%%%%%%%%%%%%%%%%%%%%%
% APPENDIX
%%%%%%%%%%%%%%%%%%%%%%%%%%%%%%%%%%%%%%%%%%%%%%%%%%%%%%%%%%%%%%%%%%%%%%%%%%%%%%%
%%%%%%%%%%%%%%%%%%%%%%%%%%%%%%%%%%%%%%%%%%%%%%%%%%%%%%%%%%%%%%%%%%%%%%%%%%%%%%%
\newpage
\appendix
\onecolumn

\section{Theory}

\subsection{Relationship to Thompson sampling with misspecified priors and lower bounds.}
\label{app:misspecifiedPriors}

In our per-period regret bound from Theorem \ref{thm:psarRegret}, the ``penalty'' for using a suboptimal sequence model $p_\theta$ does not vanish as $T$ grows. 
Since frequentist regret for Thompson sampling do not incorporate such non-vanishing terms, one might interpret this as indicating our result is not tight. This interpretation is significantly mistaken. Standard frequentist regret bounds for Thompson sampling \textit{critically} assume diffuse, non-informative priors \citep{agrawal2012analysis,agrawal2013thompson}, which ensure that each arm is explored sufficiently. It turns out that Thompson Sampling can be highly sensitive to misspecification in the prior, especially if under the prior the probability of the optimal action being the best is too low, so the algorithm has a high probability of under exploring the best action. Specifically, previous work has shown that the per period regret may be non-vanishing for a worst case environment and choice of prior \citep{liu2016prior,simchowitz2021bayesian}. Additionally, \citet{psar2024} show that for a multi-arm (non-contextual) bandit version of the generative Thompson sampling algorithm that the penalty for using an imperfect sequence model depends on $\sqrt{\ell(p_\theta) - \ell(p^*)}$ in a way that is in general unavoidable.



\subsection{Notation}
We first introduce some notation we use throughout this Appendix.
\begin{itemize}
    \item Recall that by definition 
    \begin{align*}
        \Delta \big( \psar(p_\theta ) \big) = \E_{\psar(p_\theta )} \left[  \frac{1}{T} \sum_{t=1}^{T} R \big( Y_t^{(\pi^*(X_t; \tau))} \big) - \frac{1}{T} \sum_{t=1}^{T} R \big( Y_{t}^{(A_t)} \big)\right].
    \end{align*}
    Throughout proof, we will omit the $\psar(p_\theta )$ subscript on the expectation, i.e., we use $\E \left[ \cdotspace \right] := \E_{\psar(p_\theta )} \left[ \cdotspace \right]$. 
    \item Additionally, throughout this proof we use $\E_t$ to denote expectations conditional on $\HH_t$ and $X_{1:T}$, i.e., we use 
    \begin{align}
        \label{eqn:EtDefinition}
        \E_t \left[ \cdotspace \right] = \E \left[ \cdotspace \mid \HH_t, X_{1:T} \right].
    \end{align}
    Note that this means $\E_1\left[ \cdotspace \right] = \E \left[ \cdotspace \mid \HH_1, X_{1:T} \right]= \E \left[ \cdotspace \mid Z, X_{1:T} \right]$.
    \item We use $H(Y)$ to denote the entropy of a discrete random variable $Y$, i.e., $H(Y) = \sum_y \PP(Y = y) \log \PP(Y = y) dy$. We also use $H_t(Y) = H(Y \mid \HH_t, X_{1:T})$ to denote the entropy of $Y$ conditional on $\HH_t$ and $X_{1:T}$; Note that is standard in information theory, $H_t(Y)$ \textit{is not} a random variable, rather, it marginalizes over $\HH_t$ and $X_{1:T}$:
    \begin{align*}
        H_t(Y) := 
        H(Y \mid \HH_t, X_{1:T}) = \E \left[ \sum_y \PP(Y = y \mid \HH_t, X_{1:T}) \log \PP(Y = y \mid \HH_t, X_{1:T}) dy \right];
    \end{align*}
    Above, the outer expectation marginalizes over the history $\HH_t$ and $X_{1:T}$.
    \item We also use $I(Z;Y)$ to denote the mutual information between some random variables $Z$ and $Y$, i.e., $I(Z;Y) = \int_z \int_y \PP(Z = z, Y=y) \log \frac{\PP(Z = z, Y=y)}{\PP(Z = z) \PP(Y=y)} dz dy$. We further use $I_t(Z; Y)$ to denote the mutual information between $Z$ and $Y$ conditional on $\HH_t$ and $X_{1:T}$ (where again we marginalize over $\HH_t$ and $X_{1:T}$), i.e.,
    \begin{align}
        &I_t(Z;Y) := I(Z;Y \mid \HH_t, X_{1:T}) \nonumber \\
        &= \E \left[ \int_z \int_y \PP(Z = z, Y=y \mid \HH_t, X_{1:T}) \log \frac{\PP(Z = z, Y=y\mid \HH_t, X_{1:T})}{\PP(Z = z\mid \HH_t, X_{1:T}) \PP(Y=y\mid \HH_t, X_{1:T})} dx dy \right];
        \label{eqn:mutualInfoDef}
    \end{align}
    Above, the outer expectation marginalizes over the history $\HH_t$ and $X_{1:T}$. 
\end{itemize}


%%%%%%%%%%%%%%%%%%%%%%%%%%%%%%%%%%%%%%%%%%%%%%%%%%%
%%%%%%%%%%%%%%%%%%%%%%%%%%%%%%%%%%%%%%%%%%%%%%%%%%%
\subsection{VC Dimension}
\label{app:VCdim}

\begin{customlemma}{\ref{lemma:VC}}[VC dimension bound on entropy]
    For any binary\footnote{Note, VC-dimension is defined for binary functions.} action policy class $\Pi^*$, 
    \begin{align*}
        H \big( \piX \mid Z_{\tau}, X_{1:T} \big)
        \leq H \big( \piX \mid X_{1:T} \big)
        = O\big( \TN{VCdim}(\Pi^*) \log T \big).
    \end{align*}
\end{customlemma}

\begin{proof}
The first inequality $H \big( \piX \mid Z_{\tau}, X_{1:T} \big) \leq H \big( \piX \mid X_{1:T} \big)$ holds by the chain rule for entropy.

The second big O equality result holds by the Sauer-Shelah lemma \citep{sauer1972density,shelah1972combinatorial}. Specifically, by the Sauer-Shelah lemma if a function class has VC dimension $k$, then that function class can produce most $\sum_{i=0}^k {T\choose i} = O(T^{k})$ different labelings of any $T$ points. Thus, since a coarse upper bound on the entropy of a random variable is the log of the number of unique values that variable can take, we get that $H \big( \piX \mid X_{1:T} \big) \leq \log \sum_{i=0}^{\TN{VCdim}(\Pi^*)} {T\choose i} = O\big( \TN{VCdim}(\Pi^*) \log T \big)$.
\end{proof}


%%%%%%%%%%%%%%%%%%%%%%%%%%%%%%%%%%%%%%%%%%%%%%%%%%%
%%%%%%%%%%%%%%%%%%%%%%%%%%%%%%%%%%%%%%%%%%%%%%%%%%%
\subsection{Lemma \ref{lemma:lossdecomp}: To minimize loss $p_\theta$ needs to approximate $p^*$.}
The next lemma is a standard result connecting the excess expected loss of a sequence model $p_{\theta}$ to its KL divergence from the true sequence model $p^*$. Recall, the expected loss of a sequence model $p_\theta$ is denoted $\ell(p_\theta)$, defined in \eqref{eq:pop_loss}. To (nearly) minimize loss, $p_\theta$ the learner needs to closely approximate the true sequence model $p^*$.
\begin{lemma}[Decomposing loss under $p_\theta$]
    \label{lemma:lossdecomp}
    Under Assumptions \ref{assump:contextualBandit} and \ref{assump:context}, for the loss $\ell$ as defined in \eqref{eq:pop_loss},
    \[
    \ell(p_{\theta}) = \ell(p^*) + |\MC{A}_{\tau}| \cdot \E \left[D_{\rm KL}\left( p^* \big( Y_{1:T}^{(a)} \mid Z_{\tau}, X_{1:T} \big) \; \Big\| \; p_{\theta} \big( Y_{1:T}^{(a)} \mid Z_{\tau}, X_{1:T} \big) \right) \right]. 
    \]
\end{lemma}
\begin{proof}
By the definition of the expected loss in  \eqref{eq:pop_loss},
\begin{align*}
    \ell(p_\theta) - \ell(p^*) &= \E \left[ - \sum_{a \in \MC{A}_{\tau}} \sum_{t=1}^T \log p_{\theta} \big( Y_t^{(a)} \mid Z_{\tau}, X_{1:t}, Y_{1:t-1}^{(a)} \big) \right] 
    - \left[ - \sum_{a \in \MC{A}_{\tau}} \sum_{t=1}^T \log p^* \big( Y_t^{(a)} \mid Z_{\tau}, X_{1:t}, Y_{1:t-1}^{(a)} \big) \right] \\
    &\underbrace{=}_{(a)} - |\MC{A}_{\tau}| \cdot \E \left[ \sum_{t=1}^T \left\{ \log p_{\theta} \big( Y_t^{(a)} \mid Z_{\tau}, X_{1:t}, Y_{1:t-1}^{(a)} \big) - \log p^* \big( Y_t^{(a)} \mid Z_{\tau}, X_{1:t}, Y_{1:t-1}^{(a)} \big) \right\} \right] \\
    &\underbrace{=}_{(b)} |\MC{A}_{\tau}| \cdot \sum_{t=1}^T \E \left[ \dkl{p^* \left( Y_t^{(a)} \mid Z_{\tau}, X_{1:t}, Y_{1:t-1}^{(a)} \right)}{p_\theta \left( Y_t^{(a)} \mid Z_{\tau}, X_{1:t}, Y_{1:t-1}^{(a)} \right)} \right] 
\end{align*}
\begin{align*}
    &\underbrace{=}_{(c)} |\MC{A}_{\tau}| \cdot \sum_{t=1}^T \E \left[ \dkl{p^* \left( Y_t^{(a)} \mid Z_{\tau}, X_{1:t}, Y_{1:t-1}^{(a)} \right)}{p_\theta \left( Y_t^{(a)} \mid Z_{\tau}, X_{1:t}, Y_{1:t-1}^{(a)} \right) } \right] \\
    &\quad + |\MC{A}_{\tau}| \cdot \sum_{t=1}^T \underbrace{ \E \left[ \dkl{p^* \left( X_t \mid Z_{\tau}, X_{1:t-1}, Y_{1:t-1}^{(a)} \right)}{p_\theta \left( X_t \mid Z_{\tau}, X_{1:t-1}, Y_{1:t-1}^{(a)} \right)} \right] }_{=0} \\
    &\underbrace{=}_{(d)} |\MC{A}_{\tau}| \cdot \E \left[ \dkl{p^* \big( Y_{1:T}^{(a)}, X_{1:T} \mid Z_{\tau} \big)}{p_\theta \big( Y_{1:T}^{(a)}, X_{1:T} \mid Z_{\tau} \big)} \right] \\
    &\underbrace{=}_{(e)} |\MC{A}_{\tau}| \cdot \E \left[ \dkl{p^* \left( Y_{1:T}^{(a)}\mid Z_{\tau}, X_{1:T} \right)}{p_\theta \left( Y_{1:T}^{(a)}, \mid Z_{\tau}, X_{1:T}  \right)} \right]
\end{align*}
Above, equality (a) uses Assumption \ref{assump:indepAction} (Independence across actions). Equality (b) uses the definition of KL divergence. Equality (c) uses Assumption \ref{assump:contextualBandit} (Contextual bandit environment) and Assumption \ref{assump:context} (Context distribution is known). Equality (d) uses the chain rule for KL divergence. Equality (e) holds by the chain rule for KL divergence and since $\E \left[ \dkl{p^* \left( X_{1:T}\mid Z_{\tau} \right)}{p_\theta \left( X_{1:T} \mid Z_{\tau} \right)} \right] = 0$ by Assumption \ref{assump:context} (Context distribution is known).
\end{proof}

%%%%%%%%%%%%%%%%%%%%%%%%%%%%%%%%%%%%%%%%%%%%%%%%%%%
%%%%%%%%%%%%%%%%%%%%%%%%%%%%%%%%%%%%%%%%%%%%%%%%%%%
\subsection{Lemma \ref{lemma:AAbarKLNew}: Action selection under perfect vs. imperfect imputation models.} 
\begin{lemma}[KL Divergence in next action distribution]
\label{lemma:AAbarKLNew}
Under Assumption \ref{assump:context}, for any $t$,
\begin{align*}
    \E \big[ \dkl{\PP_t \left(\pi^*(X_t; \tau) = \cdot \right)}{\PP_t \left(A_t = \cdot \right)} \big]
    \leq |\MC{A}_{\tau}| \cdot \{ \ell(p_{\theta}) - \ell(p^*) \}.
\end{align*}
\end{lemma}

\begin{proof}
Note the following:
\begin{align*}
    &\E \big[ \dkl{\PP_t \left(\pi^*(X_t; \tau) = \cdot \right)}{\PP_t \left(A_t = \cdot \right)} \big] \\
    &\underbrace{\leq}_{(a)} \E \left[
    \dkl{\PP_{p^*} \left(\{Y_{1:T}^{(a)}\}_{a \in \MC{A}_{\tau}} \mid X_{1:T}, \HH_t \right)}{\PP_{p_{\theta}}\left( \{Y_{1:T}^{(a)}\}_{a \in \MC{A}_{\tau}} \mid X_{1:T}, \HH_t \right)} \right] \\
    &\underbrace{\leq}_{(b)} \E \left[
    \dkl{\PP_{p^*} \left(\{Y_{1:T}^{(a)}\}_{a \in \MC{A}_{\tau}} \mid X_{1:T}, Z_{\tau} \right)}{\PP_{p_{\theta}}\left( \{Y_{1:T}^{(a)}\}_{a \in \MC{A}_{\tau}} \mid X_{1:T}, Z_{\tau} \right)} \right] \\
    &\underbrace{\leq}_{(c)} |\MC{A}_{\tau}| \cdot \E \left[
    \dkl{\PP_{p^*} \left( Y_{1:T}^{(a)} \mid X_{1:T}, Z_{\tau} \right)}{\PP_{p_{\theta}}\left( Y_{1:T}^{(a)} \mid X_{1:T}, Z_{\tau} \right)} \right] 
    \underbrace{\leq}_{(d)} \left\{ \ell(p_{\theta}) - \ell(p^*) \right\}.
\end{align*}
Above,
\begin{itemize}
    \item Inequality (a) holds because $\pi^*(X_t; \tau)$ and $A_t$ are both are derived by applying the same function to the outcomes $\{Y_{1:T}^{(a)}\}_{a \in \MC{A}_{\tau}}$.
    \item Inequality (b) holds because by the chain rule for KL divergence, 
    \begin{align*}
        &\dkl{\PP_{p^*} \left(\{Y_{1:T}^{(a)}\}_{a \in \MC{A}_{\tau}} \mid X_{1:T}, Z_{\tau} \right)}{\PP_{p_{\theta}}\left( \{Y_{1:T}^{(a)}\}_{a \in \MC{A}_{\tau}} \mid X_{1:T}, Z_{\tau} \right)} \\
        &= \dkl{\PP_{p^*} \left(\{Y_{1:T}^{(a)}\}_{a \in \MC{A}_{\tau}} \mid X_{1:T}, \HH_t \right)}{\PP_{p_{\theta}}\left( \{Y_{1:T}^{(a)}\}_{a \in \MC{A}_{\tau}} \mid X_{1:T}, \HH_t \right)} \\
        &+ \dkl{\PP_{p^*} \left( \HH_t \mid X_{1:T}, Z_{\tau} \right)}{\PP_{p_{\theta}}\left( \HH_t \mid X_{1:T}, Z_{\tau} \right)},
    \end{align*}
    and the KL divergence is non-negative.
    \item Inequality (c) holds by Assumption \ref{assump:indepAction} (Independence across actions).
    \item Inequality (d) holds by Lemma \ref{lemma:lossdecomp} (Decomposing loss under $p_\theta$).
\end{itemize}
\end{proof}


%%%%%%%%%%%%%%%%%%%%%%%%%%%%%%%%%%%%%%%%%%%%%%%%%%%
%%%%%%%%%%%%%%%%%%%%%%%%%%%%%%%%%%%%%%%%%%%%%%%%%%%
\subsection{Lemma \ref{lemma:mutualInfoNew}: Mutual information equivalency.} 
\begin{lemma}[Mutual information equivalency]
\label{lemma:mutualInfoNew}
\begin{multline*}
    I_t \big( \pi^*(X_t; \tau); ( Y_t^{(A_t)}, A_t ) \big) \\
    = \E \bigg[ \sum_{a, A \in \MC{A}_{\tau}} \PP_t \big( A_t = a \big) \PP_t \big( \pi^*(X_t ; \tau) = A \big) \cdot \dkl{ \PP_t \big( Y_t^{(a)} \mid \pi^*(X_t; \tau) = A \big) }{ \PP_t \big( Y_t^{(a)} \big) } \bigg]
\end{multline*}
\end{lemma}

\begin{proof}
Note that
\begin{align*}
    &I_t \big( \pi^*(X_t; \tau); ( Y_t^{(A_t)}, A_t ) \big)
    \underbrace{=}_{(a)} I_t \big( \pi^*(X_t; \tau); Y_t^{(A_t)} \mid A_t \big) \\
    &\underbrace{=}_{(b)} \E \bigg[ \sum_{a \in \MC{A}_{\tau}} \PP_t \big( A_t = a \big) I_t \big( \pi^*(X_t; \tau); Y_t^{(a)} \mid A_t = a \big) \bigg] \\
    &\underbrace{=}_{(c)} \E \bigg[ \sum_{a \in \MC{A}_{\tau}} \PP_t \big( A_t = a \big) I_t \big( \pi^*(X_t; \tau); Y_t^{(a)} \big) \bigg] \\
    &\underbrace{=}_{(d)} \E \bigg[ \sum_{a \in \MC{A}_{\tau}} \PP_t \big( A_t = a \big) \sum_{A \in \MC{A}_{\tau}} \PP_t \big( \pi^*(X_t; \tau) = A \big) \dkl{\PP_t \big( Y_t^{(a)} \mid \pi^*(X_t; \tau) = A \big)}{\PP_t \big( Y_t^{(a)}\big)} \bigg].
\end{align*}
Above, equality (a) holds since  $\pi^*(X_t; \tau)$ and $A_t$ are independent conditional on $\HH_{t}, X_{1:T}$. 
Equality (b) holds by the definition of conditional mutual information.
Equality (c) holds because $Y_t^{(a)}$ and $\pi^*(X_t; \tau)$ are independent of $A_t$ conditional on $\HH_{t}, X_{1:T}$.
Equality (d) holds by the KL divergence form of mutual information.
\end{proof}



%%%%%%%%%%%%%%%%%%%%%%%%%%%%%%%%%%%%%%%%%%%%%%%%%%%
%%%%%%%%%%%%%%%%%%%%%%%%%%%%%%%%%%%%%%%%%%%%%%%%%%%
\subsection{Lemma \ref{lemma:boundingMI}: Bounding sum of mutual information terms.}
\begin{lemma}[Bounding sum of mutual information terms]
    \label{lemma:boundingMI}
    \[
    \sum_{t=1}^T I_t \left( \piX; (A_t, Y_t^{(A_t)}) \right)
    \leq H \big( \piX \mid Z_{\tau}, X_{1:T} \big). 
    \]
\end{lemma}
\begin{proof}
\begin{align*}
    \sum_{t=1}^T I_t \left( \piX; (A_t, Y_t^{(A_t)}) \right)
    &\underbrace{=}_{(i)} I_1 \big( \piX; ( Y_t^{(A_t)}, A_t )_{t=1}^T \big) \\
    &\underbrace{=}_{(ii)} H_1(\piX) - H_1(\piX \mid ( Y_t^{(A_t)}, A_t )_{t=1}^T) \\
    &\underbrace{\leq}_{(iii)} H_1(\piX) \underbrace{=}_{(iv} H \big( \piX \mid Z_{\tau}, X_{1:T} \big).
\end{align*}
Above, equality (i) holds by the chain rule for mutual information. Equality (ii) holds by the entropy formulation of mutual information. Equality (iii) holds since the entropy is always non-negative. Equality (iv) holds by the definition of $H_1$, which is the entropy conditional on $\HH_1 = \{ Z_{\tau} \}$ and $X_{1:T}$.
\end{proof}


%%%%%%%%%%%%%%%%%%%%%%%%%%%%%%%%%%%%%%%%%%%%%%%%%%%
%%%%%%%%%%%%%%%%%%%%%%%%%%%%%%%%%%%%%%%%%%%%%%%%%%%
\subsection{Proof of Theorem \ref{thm:psarRegret}}
\begin{customthm}{\ref{thm:psarRegret}}[TS-Gen regret bound]
We use $\ell$ as defined in \eqref{eq:pop_loss}. Under Assumptions \ref{assump:contextualBandit}, \ref{assump:indepAction}, and \ref{assump:context}, the regret of the TS-Gen algorithm is bounded as follows:
    \begin{align*}
      \Delta \big( \psar(p_\theta) \big)
      & \leq 
        \underbrace{ \sqrt{ \frac{ |\MC{A}_{\tau}| \cdot H \big( \piX \mid Z, X_{1:T} \big) }{2 T} } }_{\TN{Regret bound for Thompson sampling}} 
        + \underbrace{ \sqrt{ 2 \big\{ \ell(p_\theta) - \ell(p^*) \big\} } }_{\TN{Penalty for sub-optimal prediction}}.
    \end{align*}
\end{customthm}
Recall from \eqref{eqn:piXdef} that $\piX := \{ \pi^*(X_t; \tau) \}_{t=1}^T$.

\begin{proof}
Note that by the law of iterated expectations,
\begin{align*}
    \Delta(\psar) 
    = \E \left[ \frac{1}{T} \sum_{t=1}^T R(Y_{t}^{(\pi^*(X_t; \tau))}) - R(Y_{t}^{(A_t)}) \right]
    = \E \left[ \frac{1}{T} \sum_{t=1}^T \E_t \left[ R(Y_{t}^{(\pi^*(X_t; \tau))}) - R(Y_{t}^{(A_t)}) \right] \right].
\end{align*}
Consider the following for any $t \in [1 \colon T]$:
\begin{align*}
  &\E_t \left[ R(Y_{t}^{(\pi^*(X_t; \tau))}) - R(Y_{t}^{(A_t)}) \right] \\
  &= \sum_{a \in \MC{A}_{\tau}} \PP_t(\pi^*(X_t; \tau) = a) \cdot \E_t \big[ R \big( Y_t^{(a)} \big) \mid \pi^*(X_t; \tau) = a \big]
  + \sum_{a \in \MC{A}_{\tau}} \PP_t(A_t = a) \cdot \E_t \big[ R \big( Y_t^{(a)} \big) \mid A_t = a \big] \\
  &\underbrace{=}_{(i)} \sum_{a \in \MC{A}_{\tau}} \PP_t(\pi^*(X_t; \tau) = a) \cdot \E_t \big[ R \big( Y_t^{(a)} \big) \mid \pi^*(X_t; \tau) = a \big]
  + \sum_{a \in \MC{A}_{\tau}} \PP_t(A_t = a) \cdot \E_t \big[ R \big( Y_t^{(a)} \big) \big] \\
  &= \sum_{a \in \MC{A}_{\tau}} \sqrt{ \PP_t(\pi^*(X_t; \tau) = a) \PP_t(A_t = a) } \left( \E_t \big[ R \big( Y_t^{(a)} \big) \mid \pi^*(X_t; \tau) = a \big]
  - \E_t \big[ R \big( Y_t^{(a)} \big) \big] \right) \\
  &+ \sum_{a \in \MC{A}_{\tau}} \left( \sqrt{ \PP_t(\pi^*(X_t; \tau) = a) } - \sqrt{ \PP_t(A_t = a) } \right) \\
  &\qquad \qquad \qquad \left( \sqrt{ \PP_t(\pi^*(X_t; \tau) = a) } \E_t \big[ R \big( Y_t^{(a)} \big) \mid \pi^*(X_t; \tau) = a \big]
  + \sqrt{ \PP_t(A_t = a) } \E_t \big[ R \big( Y_t^{(a)} \big) \big] \right) \\
 &\underbrace{\leq}_{(ii)} \sqrt{ |\MC{A}_{\tau}| \sum_{a \in \MC{A}_{\tau}} \PP_t(\pi^*(X_t; \tau) = a) \PP_t(A_t = a) \left( \E_t \big[ R \big( Y_t^{(a)} \big) \mid \pi^*(X_t; \tau) = a \big]
  - \E_t \big[ R \big( Y_t^{(a)} \big) \big] \right)^2 } \\
  &\qquad \qquad \qquad \qquad \qquad \qquad \qquad \qquad \qquad \qquad \qquad + \sum_{a \in \MC{A}_{\tau}} \left| \PP_t(\pi^*(X_t; \tau) = a) - \PP_t(A_t = a) \right| \\
  &\underbrace{\leq}_{(iii)} \sqrt{ |\MC{A}_{\tau}| \sum_{a \in \MC{A}_{\tau}} \PP_t(A_t = a) \sum_{A \in \MC{A}_{\tau}} \PP_t(\pi^*(X_t; \tau) = A) \left( \E_t \big[ R \big( Y_t^{(a)} \big) \mid \pi^*(X_t; \tau) = A \big]
  - \E_t \big[ R \big( Y_t^{(a)} \big) \big] \right)^2 } \\
  &\qquad \qquad \qquad \qquad \qquad \qquad \qquad \qquad \qquad \qquad \qquad + \sqrt{ 2 \cdot \dkl{\PP_t\big(\pi^*(X_t; \tau) = \cdotspace\big)}{\PP_t \big(A_t = \cdotspace \big)} } \\
  &\underbrace{\leq}_{(iv)} \sqrt{ \frac{|\MC{A}_{\tau}|}{2} \sum_{a \in \MC{A}_{\tau}} \PP_t(A_t = a) \sum_{ A \in \MC{A}_{\tau}} \PP_t(\pi^*(X_t; \tau) = A) \cdot \dkl{ \PP_t \big( Y_t^{(a)} \mid \pi^*(X_t; \tau) = A \big) }{ \PP_t \big( Y_t^{(a)} \big) } } \\
  &\qquad \qquad \qquad \qquad \qquad \qquad \qquad \qquad \qquad \qquad \qquad + \sqrt{ 2 \cdot \dkl{\PP_t\big(\pi^*(X_t; \tau) = \cdotspace\big)}{\PP_t \big(A_t = \cdotspace \big)} }
\end{align*}
Above, equality (i) holds since conditional on $\HH_t$, the action $A_t$ and the outcome $Y_t^{(a)}$ are independent.
Inequality (ii) uses Cauchy-Schwartz inequality in the first term and uses that $R$ takes values in $[0,1]$ in the second term.
Inequality (iii) uses an elementary equality of summation in the first term, and Pinsker's inequality in the second term.
Inequality (iv) uses Fact 9 of \citet{russo2016information} (which uses Pinsker's inequality).

Using the above result, averaging over $t$ and taking an expectation, we get
\begin{align*}
    &\Delta(\psar) 
    = \E \left[ \frac{1}{T} \sum_{t=1}^T \E_t \left[ R(Y_{t}^{(\pi^*(X_t; \tau))}) - R(Y_{t}^{(A_t)}) \right] \right] \\
    &\leq \E \bigg[ \frac{1}{T} \sum_{t=1}^T \sqrt{ \frac{|\MC{A}_{\tau}|}{2} \sum_{A \in \MC{A}_{\tau}} \PP_t(A_t = A) \sum_{a \in \MC{A}_{\tau}} \PP_t(\pi^*(X_t; \tau) = a) \cdot \dkl{ \PP_t \big( Y_t^{(a)} \mid \pi^*(X_t; \tau) = a \big) }{ \PP_t \big( Y_t^{(A)} \big) } } \bigg] \\
    &\qquad \qquad \qquad \qquad \qquad \qquad \qquad \qquad \qquad \qquad + \E \bigg[ \frac{1}{T} \sum_{t=1}^T \sqrt{ 2 \cdot \dkl{\PP_t\big(\pi^*(X_t; \tau) = \cdotspace\big)}{\PP_t \big(A_t = \cdotspace \big)} } \bigg] \\
    &\underbrace{\leq}_{(i)} \sqrt{ \E \bigg[ \frac{1}{T} \sum_{t=1}^T  \frac{|\MC{A}_{\tau}|}{2} \sum_{A \in \MC{A}_{\tau}} \PP_t(A_t = A) \sum_{a \in \MC{A}_{\tau}} \PP_t(\pi^*(X_t; \tau) = a) \cdot \dkl{ \PP_t \big( Y_t^{(a)} \mid \pi^*(X_t; \tau) = a \big) }{ \PP_t \big( Y_t^{(A)} \big) } \bigg] } \\
    &\qquad \qquad \qquad \qquad \qquad \qquad \qquad \qquad \qquad \qquad + \sqrt{ \E \bigg[ \frac{1}{T} \sum_{t=1}^T 2 \cdot \dkl{\PP_t\big(\pi^*(X_t; \tau) = \cdotspace\big)}{\PP_t \big(A_t = \cdotspace \big)} \bigg] } \\
    &\underbrace{=}_{(ii)} \sqrt{ \frac{|\MC{A}_{\tau}|}{2} \cdot \frac{1}{T} \sum_{t=1}^T I_t\big ( \pi^*(X_t; \tau); ( Y_t^{(A_t)}, A_t ) \big) }
    + \sqrt{ \E \bigg[ \frac{1}{T} \sum_{t=1}^T 2 \cdot \dkl{\PP_t\big(\pi^*(X_t; \tau) = \cdotspace\big)}{\PP_t \big(A_t = \cdotspace \big)} \bigg] } \\
    &\underbrace{\leq}_{(iii)} \sqrt{ \frac{|\MC{A}_{\tau}|}{2} \cdot \frac{1}{T} \sum_{t=1}^T I_t\big ( \pi^*(X_t; \tau); ( Y_t^{(A_t)}, A_t ) \big) }
    + \sqrt{ 2 \{ \ell(p_{\theta}) - \ell(p^*) \} } \\
    &\underbrace{\leq}_{(iv)} \sqrt{ \frac{|\MC{A}_{\tau}|}{2} \cdot \frac{1}{T} \sum_{t=1}^T I_t\big ( \piX; ( Y_t^{(A_t)}, A_t ) \big) }
    + \sqrt{ 2 \{ \ell(p_{\theta}) - \ell(p^*) \} } \\
    &\underbrace{=}_{(v)} \sqrt{ \frac{|\MC{A}_{\tau}|}{2T}  I_1\big ( \piX; ( Y_t^{(A_t)}, A_t )_{t=1}^T \big) }
    + \sqrt{ 2 \{ \ell(p_{\theta}) - \ell(p^*) \} } \\
    &\underbrace{=}_{(vi)} \sqrt{ \frac{|\MC{A}_{\tau}|}{2T} \big\{ H_1(\piX) - H_1(\piX \mid ( Y_t^{(A_t)}, A_t )_{t=1}^T) \big\} }
    + \sqrt{ 2 \{ \ell(p_{\theta}) - \ell(p^*) \} } \\
    &\underbrace{\leq}_{(vii)} \sqrt{ \frac{|\MC{A}_{\tau}| \cdot H_1(\piX) }{2T} }
    + \sqrt{ 2 \{ \ell(p_{\theta}) - \ell(p^*) \} } \\
\end{align*}
Above, 
\begin{itemize}
    \item Inequality (i) uses Jensen's inequality twice.
    \item Equality (ii) uses Lemma \ref{lemma:mutualInfoNew} (Mutual information equivalency).
    \item Inequality (iii) uses Lemma \ref{lemma:AAbarKLNew} (KL Divergence in next action distribution).
    \item For inequality (iv), note that for any random variables $X_1, X_2, Y$ (where $X_1, X_2$ are discrete), by properties of mutual information and entropy,
    \begin{align*}
        I((X_1, X_2); Y) &= H(X_1, X_2) - H(X_1, X_2 \mid Y) \\
        &= H(X_1) - H(X_1 \mid Y) + H(X_2 \mid X_1) - H(X_2 \mid Y, X_1) \\
        &= I(X_1; Y) + I(X_2; Y \mid X_1)
    \end{align*}
    The above implies that $I((X_1, X_2); Y) \geq I(X_1; Y)$ since $I(X_2; Y \mid X_1) \geq 0$. Thus, since $\pi^*(X_t; \tau) \in \piX$ we have that
    \begin{align*}
        I_t \big( \piX; ( Y_t^{(A_t)}, A_t ) \big)
        \geq I_t\big ( \pi^*(X_t; \tau); ( Y_t^{(A_t)}, A_t ) \big).
    \end{align*}
    \item Equality (v) uses the chain rule for mutual information.
    \item Equality (vi) holds by the relationship between mutual information and entropy.
    \item Inequality (vii) holds since entropy is always nonnegative.
\end{itemize}
\end{proof}


%%%%%%%%%%%%%%%%%%%%%%%%%%%%%%%%%%%%%%%%%%%%%%%%%%%
%%%%%%%%%%%%%%%%%%%%%%%%%%%%%%%%%%%%%%%%%%%%%%%%%%%
\subsection{Proof of Proposition \ref{prop:psarRegretPerfect}}

\begin{customprop}{\ref{prop:psarRegretPerfect}}[Regret for Thompson sampling via generation with a perfect imputation model]
    Under Assumptions \ref{assump:contextualBandit} and \ref{assump:context}, Thompson sampling via generation (Algorithm \ref{alg:Thompson}) with the imputation model $p^*$ has regret that is bounded as follows:
    \begin{align*}
        \Delta( \mathbb{A}_{\rm{TS-Gen}}(p^*) ) \leq \sqrt{ \frac{ |\Aeval| \cdot H \big( \piX \mid Z_{\tau}, X_{1:T} \big) }{2 T} }.
    \end{align*}
\end{customprop}

\begin{proof}
This proposition holds as a direct corollary of Theorem \ref{thm:psarRegret}.
\end{proof}
\clearpage
\section{Experiment details}
\label{app:experiments}

\subsection{Data generating process}
\label{app:dgp}
\paragraph{Synthetic setting}
We evaluate our method on a synthetic contexutal bandit setting. The task features $Z$ for a given bandit task consist of one feature per action, i.e. $Z=\{Z^{(a)}\}_{a\in \mathcal A_{\tau}}$, where only $Z^{(a)}$ affects the reward for action $a$.
For simplicity, $R(y)=y$. 
For task $\tau$, action $a$, and timestep $t$, with action features $Z^{(a)}_{\tau}$, context features $X_t$, and unknown coefficients $U^{(a)}:=(U^{(a)}_{\rm const}, U^{(a)}_{Z}, U^{(a)}_{X}, U^{(a)}_{\rm cross})$, let $\sigma(w):=(1+\exp(-w))^{-1}$, and define
\begin{align}
\label{eq:synthetic_dgp}
&W^{(a)}_t = U^{(a)}_{\rm const} + U^{(a)}_{Z} Z^{(a)}+U^{(a)}_{X} X_t + X_t^\top U^{(a)}_{\rm cross} Z^{(a)}\nonumber\\
&\quad\quad\textrm{ where }\;Y^{(a)}_t\sim \textrm{Bernoulli}(\sigma(W^{(a)}_t)) \; \textrm{ i.i.d.}
\end{align}
All of the random variables above are generated i.i.d. for each task $\tau$. All of the random variables indexed by action $a$ above are also generated i.i.d. across actions $a\in\mathcal A_{\tau}$. 

We generate each $Z^{(a)}\sim N(0_2,I_2)$ and $X_t\sim N(0_5,I_5)$ as multivariate Gaussians. The unobserved coefficients are also drawn as multivariate Gaussians: $U_{\rm const}^{(a)}\sim N(0,1)$, $U_Z^{(a)}\sim N(1_2,I_2 \cdot 0.25^2)$,
$U_X^{(a)}\sim N(1_5,I_5 \cdot 0.25^2)$. The last coefficient
$U^{(a)}_{\rm cross}$ is drawn as a random diagonal matrix, where the diagonal entries are each drawn independently as i.i.d. $N(1,0.25^2)$. 

Unless otherwise specified, the training dataset consists of 10k independently drawn actions, and the validation set also consists of 10k independently drawn actions. For bandit evaluation, sets of 10 actions are drawn independently for each bandit environment. 



\paragraph{Semi-synthetic setting}
We extend our synthetic experiment setting to a semi-synthetic news recommendation setting in which we use text headlines $Z^{(a)}$ for arm $a$, so that the sequence model requires feature learning. 
We define\begin{align}
\label{eq:semisynthetic_dgp}
&W^{(a)}_t = U^{(a)}_{\rm const} + U^{(a)}_{Z} \phi_Z (Z^{(a)})+U^{(a)}_{X} \phi_X( X_t) + \phi_X (X_t)^\top U^{(a)}_{\rm cross} \phi_Z( Z^{(a)})\nonumber\\
&\quad\quad\textrm{ where }\;Y^{(a)}_t\sim \textrm{Bernoulli}(\sigma(W^{(a)}_t)) \; \textrm{ i.i.d.}
\end{align}
This is similar to the synthetic setting in Equation~\eqref{eq:synthetic_dgp}, except that the data-generating process uses $\phi_X(X_t)$ and $\phi_Z(Z^{(a)})$ instead of $X_t,Z^{(a)}$, respectively, where $\phi_X$ and $\phi_Z$ are nonlinear. This increases the difficulty of the learning task for the sequence model. The rest of the data generation is the same, aside from $Z^{(a)}$ being text headlines and using $\phi_X(X_t)$ and $\phi_Z(Z^{(a)})$, is the same. 

More specifically, the headlines $Z^{(a)}$ are sampled randomly (without replacement) from the MIND large dataset  \cite{wu2020mind} (training split only). The headlines are split into training, validation, and bandit evaluation sets, where headlines are disjoint between these three datasets. The training and validation sets are used to train and perform model selection for sequence models, and the bandit evaluation set is solely for evaluating regret. We generate one draw of one action (i.e. $W_t^{(a)}$) for each headline.  Unless otherwise specified, the training set has 20k headlines, validation has 10k, and the bandit set is everything left over, which is about 74k headlines. 

Additionally, $\phi_Z( Z^{(a)})$ is a two-dimensional vector, where the first dimension is the probability output of a pre-trained binary \cite{hfsentiment} evaluated on $Z^{(a)}$, and the second dimension is the probability output of 
a binary pre-trained formality classifier on $Z^{(a)}$ \cite{hfformality} with outputs normalized to have mean 0 and variance 1. Both models were obtained from huggingface.com.
Next, $\phi_X(X_t)$ is as follows: as $X_t\in \mathbb R^5$ as defined in the synthetic setting, $\phi_X(X)_{t,1:4}= X_{t,1:4}\cdot  \textrm{sign}(X_{t,5})$, i.e. $\phi_X$ multiplies the first four dimensions of $X_t$ by the sign of the fifth dimension. 


\subsection{Offline training}
\subsubsection{Resampling historical data}
\label{app:pretrain_bootstrap}
It is uncommon in to have access to all potential outcomes for all actions in realistic scenarios. 
Instead, it is more common to have access to the outcome corresponding to the action that was taken. Under the assumption that the contexts $X_t$ are exchangeable, and that the actions chosen historically were chosen at random, then for each action $a$, we can consider the contexts $X_t$ for timesteps $t$ for which this action was taken, and the corresponding outcomes $Y^{(a)}_t$. 
We assume that we have 1000 such timesteps per action. 
During training, in every epoch, we sample without replacement from this set of $(X_t, Y_t^{(a)})$'s to form a sequence of length 500; the sequence model $p_\theta$ is then trained on such sequences of data. 



\subsubsection{Sequence model architecture}
\label{app:sequence_models}
\begin{figure}[h]
\centering
\includegraphics[width=0.6\linewidth]{figures/architecture7.png}    
\caption{Diagram of model architecture for $p_\theta$, for semisynthetic settings. In synthetic settings, the model architecture is the same, except that it does not include the DistilBERT~\cite{sanh2019distilbert} encoder to process text, or the $X$ MLP encoder. }
\label{fig:architecture}
\end{figure}

\paragraph{Synthetic setting} In the synthetic setting, the model architecture is as follows: the output of $p_\theta$ is a final MLP head on top of a vector that is the concatenation of $Z^{(a)},X_t,$ and the summary statistics of the history for action $a$. The final MLP head has 3 layers, each with width 100. 

For simplicity, in the synthetic setting, we use sequence $p_\theta$ models that summarize recent history with summary statistics as follows. The summary statistics are $(\mathbf{X}^\top \mathbf{X}+\epsilon I)^{-1}$ and $\mathbf{X}^\top \mathbf{Y}$, where $\mathbf{X}$ denotes a matrix where each row is a past observation of $\mathbf{X}$, and $\mathbf{Y}$ is a vector where each element the corresponding past observation of $Y$. The hyperparameter $\epsilon$ is a value that is tuned during training, and we set it to $\epsilon=1$. 


\paragraph{Semisynthetic setting} In the semisynthetic setting, $p_\theta$ is implemented to take as input the part of the task feature for one action $a$, $Z^{(a)}$, along with history for that action, and context $X_t$, to predict the next reward given $X_t$ and for action $a$. As displayed in Figure~\ref{fig:architecture}, 
the model architecture is as follows. We concatenate a DistilBert \citep{sanh2019distilbert} embedding of headline $Z^{(a)}$ with $X_t$, and also summary statistics of the history that take in $Y_{1:t-1}^{(a)}$, as well as an MLP embedding of $X_{1:t-1}$ (2 layers, width 100); the sufficient statistics described above are  repeated 100 times. Then, this concatenated vector is fed into the final MLP head (3 layers, width 100). Finally, the output of the MLP is fed through a sigmoid function to obtain a prediction for the probability that the next outcome is 1 (rather than 0). 
The other difference from the synthetic setting is that the summary statistics feed the history of $X_t$'s into a 2-layer, width 100 MLP before calculating summary statistics. 




\subsubsection{Additional sequence model training details}
\label{app:poolactions}
\paragraph{Synthetic setting} We train (and validate) on sequences of length 500, sampled with replacement from historical sequences of length 1000, for 100 epochs. The training set has a pool of 10,000 actions (except for Figure~\ref{fig:loss_vs_regret}), and the validation set also has pool of 10,000 actions. Tasks are sampled/constructed by independently selecting 10 actions from the pool. We optimize weights with the AdamW optimizer. We try learning rates $\{0.1,0.01,0.001\}$ and choose the learning rate with the lowest validation loss, which is 0.01. We set weight decay to 0.01. The batch size is 500. 

\paragraph{Semi-synthetic setting} We train (and validate) on sequences of length 500, sampled with replacement from historical sequences of length 1000, for 40 epochs. 
The training set has a pool of 20,000 actions. The validation set has a pool of 10,000 actions. Tasks are sampled/constructed by independently selecting 10 actions from the pool. Aagin, we optimize weights with the AdamW optimizer. We try learning rates $\{0.1,0.01,0.001\}$ and choose the learning rate and also the training epoch with the lowest validation loss; the learning rate chosen is 0.01. We set weight decay to 0.01. The batch size is 500. We do not fine-tune the DistilBERT encoder and leave those weights as-is. 

\subsection{Online learning}

\subsubsection{TS-Gen details}
\label{app:more_generation}

Here we describe additional details used to draw potential outcomes tables $\hat{ \tau}$ 
by using a sequence model $p_\theta$. 

First, the $\{X_t\}_{1:T}$ on which we evaluate the bandit algorithm are assumed to be known at the beginning of bandit evaluations, but not known before that (i.e. the sequence model is not trained on the exact data used to evaluate the bandit algorithm). When the potential outcomes table $\hat{ \tau}$ is generated, this $\{X_t\}_{1:T}$ is fixed. Variations of this algorithm can be made where the $\{X_t\}_{1:T}$ seen in the bandit setting are \emph{not} known ahead of time, but we use this setting for simplicity. 


Here we describe additional details used to fit $\pi^*\in \Pi$ on $\hat{\tau}$. 
For all model classes used, there is no train/test split. The policy $\pi^*$ is fit directly on $\hat{\tau}$. This is the case for both TS-Gen, and for ``oracles''.  
For logistic methods, we use the default logistic regression implementation from \texttt{scikit-learn} \cite{scikit-learn}. 
For tree-based methods, we use gradient-boosted forests with maximum depth set to 2, also from \texttt{scikit-learn} \cite{scikit-learn}. 




\subsubsection{Baseline bandit methods}
\label{app:baseline_bandit_methods}
The first three (Greedy, Epsilon-Greedy, and Softmax) are alternative ways to make decisions using an existing pre-trained sequence model $p_\theta$. The others (Linear Thompson Sampling, LinUCB) are contextual bandit methods that do not use $p_\theta$. 


\paragraph{Greedy}
We use a trained sequence model $p_\theta$ as in TS-Gen. 
In the online step, at time $t$, instead of using $p_\theta$ to generate the entire potential outcomes table $\hat{\tau}$, fit a policy $\pi^*$, and then evaluate this policy at the current context $X_t$, we just evaluate $p_\theta$ on the current task (action) features $Z^{(a)}$, history $\mathcal H_t$, and current context $X_t$. In our setting, this gives the probability that the corresponding outcome (and also reward, as $R(y)=y$) $Y_t^{(a)}$ is 1. We choose the arm with the largest such probability. 



\paragraph{Epsilon-Greedy}
The version of epsilon-greedy that we present here differs from the classical version of epsilon-greedy \citep{sutton2018reinforcement}. In our version, we choose according to the greedy algorithm (as described above, using the trained sequence model to obtain predicted rewards) with probability $1-\epsilon$, and choose a uniformly random action with probability $\epsilon$. We use $\epsilon=0.1$ in the experiments. 

\paragraph{Softmax}
This method is similar to PreDeToR-$\tau$ in \citet{mukherjee2024pretraining}. PreDeToR-$\tau$ also uses a trained sequence model to predict rewards, and chooses actions with probability 
according to softmax of the rewards for each arm, multiplied by a constant $\tau$. 
In other words, if $\hat {\textbf {r}}=\hat r_1,\ldots,\hat r_{\mathcal A_{\tau}}$ are the predicted rewards from a pre-trained model for actions $1,\ldots,\mathcal A_{\tau}$, then we choose an action with probability 
$\textrm{softmax}( \hat{\textbf r}/\tau)$. 
We set $\tau=0.05$ as in \citet{mukherjee2024pretraining}.

\paragraph{Linear Thompson Sampling}
For each arm $a\in\mathcal A_{\tau}$, outcomes are modeled as a linear function of $X$,
$$Y=X\beta + \epsilon,$$
where $\beta$ is modeled as a multivariate Gaussian prior with mean 0 and identity variance, and $\epsilon$ is modeled as a Gaussian prior with mean 0 and variance 1/4 (since the maximum variance of a Bernoulli is 1/4). 
After $t$ timesteps, we use the history $\mathcal H_t$ to calculate the posterior distributions for $\beta$ and $\epsilon$, for each arm $a$. 
Then, we do Thompson sampling: for each arm $a$, we sample once from the posteriors of $\beta$ and $\epsilon$, and calculate what $Y$ should be, given the current context $X_t$. We choose the arm with the largest such value. 
Note that unlike TS-Gen, linear Thompson sampling does not learn a rich and flexible prior based on task features $Z_\tau$. 

\paragraph{LinUCB}
We implement LinUCB-disjoint in \citep{li2010contextual}, on contexts $X_t$. We set $\alpha=0.1$ as it performs well in comparison to a small set of other values tried (\{0.1,1,2\}). 
In this particular setting, the task features are different for each action, there are few actions in each environment, and the arms are generated independently, so it is appropriate to exclude task features from the on-line modeling for LinUCB. 
Note that unlike TS-Gen, LinUCB does not learn a rich and flexible prior based on task features $Z_\tau$. 


\subsection{Sequence loss vs. regret under TS-Gen (Figure \ref{fig:loss_vs_regret})}
\label{app:seqloss}
We examine the relationship between sequence model loss $\ell(p_\theta)$ and regret of TS-Gen using $p_\theta$. Our Theorem \ref{thm:psarRegret} suggests that the lower the loss of a sequence model $p_\theta$ the lower the regret of TS-Gen using that sequence model $p_\theta$. We examine this by varying the amount of training tasks used to learn $p_\theta$ and thus obtain sequence models with different losses. We also compute the cumulative regret for TS-Gen using each respective sequence models. Indeed, in Figure~\ref{fig:loss_vs_regret}, models trained on more data tend to have lower sequence loss, which tend to have lower regret. 
\label{sec:loss_vs_regret}
\begin{figure}[h]
    \centering
\includegraphics[width=0.415\linewidth]{figures/loss_for_data_sizes.png}
\includegraphics[width=0.4\linewidth]{figures/regret_for_data_sizes.png}
\caption{\bo{Sequence loss vs. bandit regret:} 
We demonstrate the relationship between sequence loss and regret for TS-Gen by pre-training our sequence models offline on varying dataset sizes in the semisynthetic setting. As training dataset sizes are smaller, sequence loss (left) is higher (worse), and bandit regret (right) is higher (worse). ``Training rows'' refers to the number of actions used in the pool of actions to select from to form tasks (Appendix \ref{app:poolactions}).
\bo{(Left)}: Prediction loss by timestep. We plot an empirical estimate of the per-timestep (non-cumulative) loss from \eqref{eq:pop_loss} by evaluating our sequence models on an held-out validation set.
%On the horizontal axis is the number of previous observations in $\mathcal H_t$ for a given action to condition on. On the vertical axis is cross-entropy loss for $p_\theta$. If the true probability of the outcome is 1 is $q$ and the predicted probability is $p$, then we display $-q\log(p)-(1-q)\log(1-p)$. 
Error bars represent $\pm 1$ s.e. % averaged over 80,000 evaluation actions (10,000 from the validation set, plus 70,000 additional independently sampled actions). 
\bo{(Right)}: Cumulative regret for TS-Gen using the corresponding sequence models, with logistic policy class, and relative to the logistic ``oracle''. Error bars represent $\pm 1$ s.e. averaged over 500 re-drawn bandit environments.}
\label{fig:loss_vs_regret}
\end{figure}


\subsection{Policy class for TS-Gen (Figure \ref{fig:semisynthetic_policy_class_comparison})}
\label{sec:policy_class}
The choice of policy class affects both TS-Gen (for a fixed sequence model $p_\theta$), as well as the ``oracle''; See Figure \ref{fig:semisynthetic_policy_class_comparison}. In the semisynthetic setting, TS-Gen has moderately greater reward using a tree-based policy than a logistic policy. In contrast, the ``oracle'' using a tree-based policy is much better than the ``oracle'' using a logistic policy. 

\begin{figure}[h]
\centering\includegraphics[width=0.4\linewidth]{figures/semisynthetic_policy_class.png}
\centering\includegraphics[width=0.4\linewidth]{figures/semisynthetic_policy_class_zoomed.png}
\caption{Varying policy classes in the semisynthetic setting. The same experimental results are plotted on the left and the right. The plot on the right calculates regret relative to the logistic ``oracle'', while the left calculates regret relative to the tree-based ``oracle''. 
Error bars are $\pm 1$ s.e. across 500 bandit environments.}
\label{fig:semisynthetic_policy_class_comparison}
\end{figure}
%%%%%%%%%%%%%%%%%%%%%%%%%%%%%%%%%%%%%%%%%%%%%%%%%%%%%%%%%%%%%%%%%%%%%%%%%%%%%%%
%%%%%%%%%%%%%%%%%%%%%%%%%%%%%%%%%%%%%%%%%%%%%%%%%%%%%%%%%%%%%%%%%%%%%%%%%%%%%%%

\end{document}

