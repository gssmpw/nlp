% !TeX root = article.tex
% !TeX spellcheck = en_US


\begin{figure*}
	\tikzfig{rel-equations}
	\caption{Equations for a Strict Symmetric Monoidal Category.}
	\label{fig:rel-equations}
\end{figure*}
\begin{figure*}
\tikzfig{rel-equations-cup-cap}
\caption{Equations for the Cup and Cap}
\label{appfig:rel-equations-cup-cap}
\end{figure*}

All the diagrams of \Cref{sec:relations} can actually be formalized using string diagrams from category theory \cite{MacLane}. Indeed, finite sets and relations is well known to be a strict symmetric monoidal category, called \bfup{FinRel}, that is:
\begin{itemize}
	\item A set of objects (finite sets).
	\item A binary operation on objects (the Cartesian product $\times$) which is associative and has a neutral element (\one).
	\item A set of morphisms between those objects (relations), a way to compose them (the usual composition $\circ$) that is associative and has a neutral element (the identity $\id$).
	\item A binary operations on morphisms (the Cartesian product of relations $\times$) which is associative, has a neutral element ($\id_\one$), and is a bifunctor (see the last equation of \Cref{fig:rel-equations}).
	\item For every two objects $A$ and $B$, a morphism that swaps them ($\gamma_{A,B}$) which forms a natural isomorphism (third and fourth equations of \Cref{fig:rel-equations}).
\end{itemize}
We list all the resulting equations in \Cref{fig:rel-equations}, both in text and diagrams. In those diagrams, gray boxes correspond to ``bracketing'', and the overall consequence of those equations is that we can keep the bracketing implicit.

If we account for the cup and cap, relations form a compact closed category (where additionally all the objects are self-duals) \cite{KellyLaplaza}, which correspond to the additional equations listed in \Cref{appfig:rel-equations-cup-cap}. 
