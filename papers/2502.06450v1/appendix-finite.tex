% !TeX root = article.tex
% !TeX spellcheck = en_US

\subsection{Composing Transducers}\label{app:transducer-category}

In this section, we prove \Cref{prop:RegRel-is-cat} by showing that we can compose and make a product of transducers. 

Given two transducers $(\T,A,B,Q,I,F)$ and $(\S,C,D,P,J,G)$, we can build their product as $(\R,A \times C,B \times D,Q \times P, I \times J, F \times G)$ where $((a,c),(q,p))~\R~((b,d),(q',p'))$ whenever $(a,q)~\R~(b,q')$ and $(c,p)~\R~(d,p')$. Its behavior is exactly the product of the two behaviors. 
Assuming $B=C$, we can build their composition as $(\U,A,D,Q \times P,I \times J, F \times G)$ where  $(a,(q,p))~\U~(d,(q',p'))$ whenever there exists $b \in B$ such that  $(a,q)~\R~(b,q')$ and $(b,p)~\R~(d,p')$.	 Its behavior is exactly the composition of the two behaviors.

\subsection{From Transducers to Graphical Representation}\label{app:transducer-fin}

In this section, we prove \Cref{prop:transducer-fin}, that is for every transducer $(\T, A, B, Q, I,F)$, its behavior can be obtained by lifting $\T$ and using the shift as follows:
\[ \tikzfig{rel-trans-fin} \]

	We write $\R$ for the left-hand-side. We take $w \in A^*$ and $v \in B^*$ of size $n$, and note that by definition of the composition of relations, $w~\R~v$ if and only there exists two words $u,t \in Q^*$ such that $(w,u)~\T^*~(v,t)$ and $u~\transpose{\left(\shift{I}{F}\right)}~t$. 
	
	Let us focus on the former, it is equivalent to, for every $1 \leq k \leq n$, $(w_k,u_k)~\T~(v_k,t_k)$, that is $u_k \xrightarrow[v_k]{w_k} t_k$.
	
	Let us focus on the latter, it is equivalent to the first letter of $u$ being in $I$, the last letter of $t$ being in $F$, and for all $1 \leq k \leq n$, $u_{k+1} = t_k$. Said otherwise, there exists a sequence $q_0,q_1,\dots,q_n \in Q$ such that $u = q_0\dots q_{n-1}$, $t = q_1\dots q_n$, $q_0 \in I$ and $q_n \in F$.
	
	Coming back to the initial equivalence, we have $w~\R~v$ if and only if there exists a sequence $q_0,q_1,\dots,q_n \in Q$ such that $q_0 \in I$, $q_n \in F$, and for all $1 \leq k \leq n$, $q_{k-1} \xrightarrow[v_k]{w_k} q_k$. Since this is exactly the definition of a run, we have that $w~\R~v$ if and only if $w~\L(\T,A,B,Q,I,F)~v$.

\subsection{Soundness of the Standard Equations}\label{app:fin-soundness}

\begin{figure*}[h]
	\tikzfig{graph-generators}
	\caption{Generators of \bfup{Trans}.}
	\label{appfig:graph-generators}
\end{figure*}

\begin{figure*}[h]
	\tikzfig{rel-equations-fin}
	\caption{Equations for a Strict Symmetric Monoidal Category.}
	\label{appfig:rel-equations-fin}
\end{figure*}


\begin{figure*}[h]
	\tikzfig{graph-feedback}
	\caption{Equations for a Feedback Category.}
	\label{appfig:graph-feedback}
\end{figure*}

\begin{figure*}[h]
	\tikzfig{graph-finrel}
	\caption{Equations for Faithfully Embedding \bfup{FinRel}.}
	\label{appfig:graph-finrel}
\end{figure*}

\begin{figure*}
	\[\tikzfig{fin-interp}\]
	\caption{Inductive Definition of the Semantics $\binterp{-} : \bfup{Trans} \to \bfup{UniRel}$.}
	\label{appfig:fin-interp}
\end{figure*}

We start by  recalling the generators of our language and all its equations in \Cref{appfig:graph-generators,appfig:rel-equations-fin,appfig:graph-feedback,appfig:graph-finrel,appfig:graph-finrel}. Then we formally define the semantics $\binterp{-} : \bfup{Trans} \to \bfup{Rel}$ inductively on the syntax, as shown in \Cref{appfig:fin-interp}.

We now prove the soundness of the equational theory. The soundness of the equations stating that $\bfup{Trans}$ is a strict symmetric monoidal category follows immediately from the fact that $\bfup{Rel}$ is a strict symmetric monoidal category, so we only need to look at the equations of \Cref{fig:graph-feedback}. The top-left and bottom-left equations also follow from the fact that $\bfup{Rel}$ is a strict symmetric monoidal category. The top-right is sound because $\shift{\one}{\one}_\one = \id_\one$, and the bottom-right is sound because $\shift{I}{F}_C \times \shift{J}{G}_D = \shift{I \times J}{F \times G}_{C \times D}$. \qed

\subsection{Soundness of the Simulation Principle}\label{app:fin-simulation-principle}


\begin{figure*}
	\[\tikzfig{fin-simulation-principle-black}\]
	\caption{Semantics of the Simulation Principle for Finite Words.}
	\label{appfig:fin-simulation-principle-black}
\end{figure*}

We prove the soundness of the equation of \Cref{fig:fin-simulation-principle} with respect to $\binterp{-}$, that is, we want to prove the equation of \Cref{appfig:fin-simulation-principle-black}. 

Since all the relations involved are uniform, we can make a case-by-case analysis on the size of the input words on $A^*$. This mean we are now trying to prove the following for all $n \geq 0$:
\[\tikzfig{fin-simulation-principle-sound}\]
Then, using the hypothesis of \Cref{appfig:fin-simulation-principle-black}, we can rewrite $I$ into $\S \circ J$, then move the $\S$ through the $n$ copies of $\R$ to transform them into $n$ copies of $\T$, and lastly rewrite $\S \circ F$ into $G$.

\begin{figure*}[h]
	\[\tikzfig{fin-backward-simulation-principle}\]
	\caption{Backward-Simulation Principle for Finite Words.}
	\label{appfig:fin-backward-simulation-principle}
\end{figure*}

\begin{figure*}[h]
	\[\tikzfig{fin-forward-simulation-principle}\]
	\caption{Forward-Simulation Principle for Finite Words.}
	\label{appfig:fin-forward-simulation-principle}
\end{figure*}

In the above proof, one could replace all the instances of $=$ by $\subseteq$, or all of them by $\supseteq$, this would yield a proof of soundness for the backward-simulation principle of \Cref{appfig:fin-backward-simulation-principle} or of the forward-simulation principle of \Cref{appfig:fin-forward-simulation-principle}. \qed



\subsection{Minimization of finite transducers}\label{app:prop:minimization}
	We provide the full proof of \Cref{prop:minimization}.

	Using the logical reasoning as in \Cref{sec:logic}, we can rewrite the equation as the following. We are looking at $\forall a \in A, \forall p \in Q, \forall \ell \in L_{A^*},$
	\[ \begin{array}{c} \left(\exists q\in Q,~  (p \xrightarrow[\text{\small$\D$}]{a} q)\land (\ell = \{ w \mid \exists f \in F, q \xrightarrow[\text{\small$\D$}]{w} f\}) \right) \\ \Leftrightarrow \\ \left(\exists m \in L_{A^*},~  (m = \{ v \mid \exists f \in F, p \xrightarrow[\text{\small$\D$}]{v} f\}) \land (m \xrightarrow[\text{\small$L_\D$}]{a} \ell)\right) \end{array} \]

	Before studying either side of that equivalence, we note that since $p$ is accessible from the initial state, there exists a word $u \in A^*$ such that $i \xrightarrow[\text{\small$\D$}]{u} p$, hence $L_u = \{ v \mid \exists f \in F, p \xrightarrow[\text{\small$\D$}]{v} f\}$.

	We start by reformulating the second part of the equivalence.  Since the formula for $m$ is given, we can remove the $\exists m$ and obtain $L_u \xrightarrow[\text{\small$L_\D$}]{a} \ell$. We can now use the definition of $L_\D$ and simplify the later in $\ell = L_{ua}$.

	We now look at the first part of the equivalence, since $\D$ is deterministic, $q$ is uniquely determined so we can remove the $\exists q$ and we obtain $\ell = \{ w \mid \exists f \in F, \D(a,p) \xrightarrow[\text{\small$\D$}]{w} f\}$. Since  $i \xrightarrow[\text{\small$\D$}]{u} p$, this is equivalent to $\ell = \{ w \mid \exists f \in F, i \xrightarrow[\text{\small$\D$}]{uaw} f\}$, that is $\ell = L_{ua}$.

	\qed




\subsection{Deducing the Sliding Equation}\label{app:fin-slide}

\begin{figure*}
	\[\tikzfig{graph-slide}\]
	\caption{Deducible Equation.}
	\label{appfig:graph-slide}
\end{figure*}

We prove that the equation from \Cref{appfig:graph-slide} can be deduced from the others. For that, we start by using \Cref{prop:normal-form-fin} on $\bR$:
\[\tikzfig{graph-slide-proof-normal-form}\]
We can then merge the two feedbacks into one, rewriting both sides of the equation we are trying to prove into:
\[\tikzfig{graph-slide-proof-rewrite}\]

We then conclude using the the simulation principle, using the following prerequisites:
\[\tikzfig{graph-slide-proof-prerequisites}\]
~\qed


