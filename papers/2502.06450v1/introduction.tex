% !TeX root = article.tex
% !TeX spellcheck = en_US
The interest of category theory for computer science is to provide an abstract language suitable to describe computational structures in their most general form, allowing to clarify connections between similar constructions appearing in different fields, sometimes seemingly unrelated. 
One of those abstract ideas, ubiquitous in computer science, is a machine interacting with its environment by updating an internal set of states. 
Most computational models fit this description, the paradigmatic example being Turing machines. 
Such machines interact with their environment via inputs and outputs, thus their behavior can be described as the set of all possible input/output pairs, which are interpreted as a relation between a (possibly infinite) sequence of input events and another sequence of output events. 
When constraining the set of internal states of a machine to be finite, one enforces interesting mathematical properties on the set of admissible behaviors, as they must be definable by finite means, in a certain way.

Let us take three examples of models exhibiting this kind of behavior. 
The most famous is the finite automaton \cite{DBLP:books/daglib/0086373,DBLP:books/daglib/0016921}, which only takes finite sequences of input events and evaluates if those sequences are valid or not.
The behavior then coincides with the language recognized by the automaton, and the possible behaviors are exactly the regular languages.
Perhaps more aligned with the interacting machine point of view, a Labelled Transition System (LTS) performs a transition (depending on an input event and its current state) that can produce an output event. In the LTS literature, we usually talk about the trace of the system to design its behavior.
The last model we will mention is perhaps less known by the computer science community and comes from dynamical systems.
A discrete-time dynamical system is a set $X$ equipped with an update function $f:X\to X$ implementing the dynamic. 
Symbolic dynamics is the study of coarse-grained dynamical systems when $X$ is partitioned into subsets indexed by finitely many symbols \cite{lind-marcus}. 
One can learn a lot on the original dynamical system by studying the sequences of symbols induced by the evolution $f$.
Those sequences correspond to our general notion of behavior, and the ones "definable by finite means" correspond to a set of sequences known as sofic subshifts.
The fact that similar mathematical methods, mostly involving regular languages, can be employed to handle those three cases is folklore. 
However, to our knowledge, no framework exists to formally unify those three models.

We propose finite relational transducers as general objects subsuming automata, LTS, and symbolic dynamical systems. 
The originality of our approach is to embrace a fully relational point of view, considering relations and not functions as more fundamental. 
In other words, we will consider non-determinism to be more natural, and see determinism as an interesting special case. 
This stance will lead us to present generalizations of well-known models that might feel unfamiliar, typically in the case of non-deterministic symbolic dynamical systems. 
Still, we believe that the relational setting is the right place to understand clearly how the various models we present are linked.
Furthermore, the relational point of view is also the most well-suited for the use of string diagrams.

Indeed, our approach fits in a recent thread of research aiming to represent and reason on computational processes using diagrams with inputs and outputs.
Those diagrams can be thought of as boolean circuits-like structures, built from elementary generators (or gates) on which we can perform local rewritings by replacing a sub-circuit with another one having the same type and an equivalent behavior. Such methods have been successfully applied to provide compositional syntax for control flow graphs \cite{bonchi2014categorical,bonchi2021survey} or quantum computing \cite{van2020zx} for example.
%
As with the choice of moving from functions to relations, the diagrammatic paradigm requires the introduction of unfamiliar notions (for a reader used to automata, transition systems, or symbolic dynamic) from category theory \cite{selinger2011survey}.
Still, even if dressed in the unusual language of category theory and diagrams, the fundamental notions involved are the same.

We hope that the benefits of this choice outweigh the price paid for it.
Indeed, our formalism being formulated in the language of string diagrams (or equivalently, in the language of symmetric monoidal categories \cite{MacLane}), we expect to be able to obtain straightforward generalizations by changing our base category (the mathematical world in which we interpret our diagrams) from relation to stochastic kernels or quantum channels. 
This provides solid theoretical bases to define meaningful notions of automata, transition systems, and symbolic dynamical systems in the probabilistic and quantum case, on which we could develop similar simulation-based proof techniques. 


One can see the current paper as the first step toward this goal. The use of diagrammatic formalism to unify notions of finite tilings and quantum tensors has been developed in \cite{quantumWang}.

We hope our unified formalism can ease the transfer of ideas between the transition system and symbolic dynamics communities and stimulate the emergence of new questions and insights set off by our diagrammatical approach. 

%\textbf{Contribution:}
\smallskip

In this paper, we introduce relational transducers with finite states and show how they generalize finite automata, LTS, and symbolic dynamical systems. We define their behaviors on both finite and bi-infinite words. 
In both cases, we fully characterize the admissible behaviors with the notion of regular and sofic relations, respectively. 
We also introduce two string diagrammatic equational theories which are able to represent transducers on finite and bi-infinite words in a compositional way. 
We show that those equational theories are expressive enough to represent any transducers and present two sound simulation principles that are strong enough to equate in a diagrammatic form any two transducers having the same behavior, both in the finite and bi-infinite case. 
On a more technical side, we present a new diagrammatical formulation of the notion of backward and forward simulation and introduce a new notion of normal form for the presentation of a sofic subshift.

%\textbf{Structure:}2
\smallskip

We start by introducing the diagrammatic aspects of our approach in \Cref{sec:relations}. 
In \Cref{sec:finite}, with the notion of transducers used throughout the paper, we introduce our diagrammatical equational theory and completeness result for transducers acting on finite words. 
\Cref{sec:infinite} gathers results on sofic relations culminating in the extension of our diagrammatical theory and a completeness proof for the case of bi-infinite words. 
Most of the proofs are postponed to the appendix.
