% !TeX root = article.tex
% !TeX spellcheck = en_US 

The diagrammatical axiomatization presented in this paper presents many similarities with recent works also relying on feedback loops to represent internal state space.
There was a thread of work aiming to handle diagrammatically streams of data, first \cite{sprunger2019differentiable} in the classical case and then \cite{di2022monoidal} and \cite{carette2021graphical} in the probabilistic and quantum cases.
Our simulation principle bears similarity with equations appearing in \cite{ghica2022fully}.
In \cite{piedeleu2021string}, the authors provide a complete diagrammatical theory for finite automata with similar goals, however the structure or the categories involved is very different as they do not use the cartesian product as monoidal structure.
However, we do not know of any attempt to handle subshifts with such formalism.
Unifying transducers and sofic subshifts with similar graphical languages opens many research directions towards diagrammatical proofs in symbolic dynamics. We present a few of them.

\textbf{Subshifts on $\mathbb{N}$} A straightforward next step would be to consider the situation in-between finite and bi-infinite words. Indeed, this would allow to draw interesting bridges between symbolic dynamics on the half-line, Buchi automata and LTS with initial states. This direction would also allow easier comparison with the monoidal stream line of research. 

\textbf{One-dimensional SFTs} One of the most natural objects of symbolic dynamics are subshift of finite type (SFTs), defined by a finite number of local constraints.
Surprisingly, their diagrammatical definition is less straightforward than sofic ones.
The logical next step would be to define SFTs diagrammatically and use this language to show classical results linking sofic subshifts, SFTs and cellular automata.

\textbf{Symbolic dynamics on groups} Symbolic dynamics also studies infinite colorings of $\ZZ^2$ and more generally Cayley graphs of finitely generated groups. 
Diagrammatical proofs seem promising to tackle this since they seem to abstract the topology of the graph, "hidden" in the shift relation.
This would suggest that some diagrammatical one-dimensional proof techniques may be adapted to higher dimensions and groups quite straightforwardly.

\textbf{General induction principle} Many objects in symbolic dynamics have a coinductive definition (substitutive subshifts, limit sets of cellular automata, ...). 
Generalizing our simulation principle to more abstract categories could provide new proofs techniques usable for all these co-inductive objects in an abstract way.

Finally, abstracting our diagrams to other settings/categories than relations may also allow us to study other problems such as counting subshifts, studying ergodic probability measures, and defining a notion of quantum subshifts.  
