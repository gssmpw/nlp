% !TeX root = article.tex
% !TeX spellcheck = en_US

\subsection{Sofic subshifts and pruned factor-closed language}
\label{app:bijection-subshifts}
Here we present the proofs that where omitted in \Cref{sec:subshifts} and a few additional minor or intermediate results on the correspondence between sofic subshifts and pruned factor-closed languages.

First recall that, in Section \ref{sec:subshifts}, we define sofic subshifts as the subshifts whose factor language is regular.
This may differ from the definition with which the reader is familiar with, as a more usual definition is that a susbshift is sofic when it can be defined as the image of a SFT (subshift of finite type) by a cellular automaton.

Note that we deal here only with symbolic dynamics in dimension 1, that is the full shift is $A^\ZZ$ for some finite alphabet $A$.
Hence another definition of a sofic shift is a subshift $X$ which can be defined by a regular language $\mathcal{F}$ of forbidden words. Denote $A^*\cdot \mathcal{F}\cdot A^*$ the set of finite word containing a word in $\mathcal{F}$ as a factor subword.
Now remark that the factor language of $X$ is precisely the complement of $A^*\cdot \mathcal{F} \cdot A^*$, since $\mathcal{F}$ is regular then $\fact{X}$ is also regular.
Conversely if $X$ is a subshift with regular factor language $\L$, then it can be defined as avoiding the complement of $\L$ which is also a regular language.

For the same considerations (though with a different terminology) one may refer to \cite{lind-marcus}.

We can refine the factor relation $x \sqsubseteq y$ into $x \subset y$ by asking for the indexes to match. More precisely, for $u$ a finite word and $x$ a bi-infinite word, we write $\shift{k}{}(u)\subset x$ whenever $u=x_{-k}\dots x_{-k+|u|-1}$. We remark that $\mathrm{d}(x,\shift{k}{}(u))= \sup\{ \mathrm{d}(x,y) | y\in A^\ZZ\ s.t.\ \shift{k}{}(u)\subset y\}$.

\begin{remark}[Comparing finite and infinite words]
  For simplicity we extend the distance $\mathrm{d}$ to compare finite words and infinite words.
  However this does not define a metric on the mixed space $A^*\cup A^\mathbb{Z}$.

  The most rigorous approach to talk of a convergence of a sequence of finite words to an infinite word is that of cylinders.

  The \textbf{cylinder} of word $u$ at position $k$, denoted by $[u]_k$, is the set of all bi-infinite words that coincide with $u$ at position $k$ that is
  \[ [u]_k := \{ x \in A^\ZZ | \forall 0\leq i <|u|, x_{i-k}=u_i\} \]

  In our notation this corresponds to $[u]_k = \{ x \in A^\ZZ | \shift{k}{}(u) \subset x\}$.

  We call \textbf{domain} of a cylinder $[u]_k$ the integer interval $\llbracket -k, |u|-k-1\rrbracket$, and we denote it here by $\mathcal{D}([u]_k)$.

  Given a family $(C_n)_{n\in\NN}$ of cylinders, if $\mathcal{D}(C_n)\underset{n\to\infty}{\longrightarrow} \ZZ$, then there exists a sub-family $(C_{f(n)})_{n \in \NN}$, for each $n$ a larger cylinder $C_{f(n)} \subset C_n'$ and an infinite word $x$ such that $\{x\} = \bigcap\limits_{n\in\NN} C_n'$.

  In particular if $\mathcal{D}(C_n)\underset{n\to\infty}{\longrightarrow} \ZZ$ and for all $n$, $C_{n+1}\subseteq C_n$ then $\bigcap\limits_{n\in\NN} C_{n}=\{x\}$ for some bi-inifinite word $x$.

  The easiest case is when each cylinder is some $C_n=[u_n]_{k_n}$ such that $u_n \internal{1} u_{n+1}$ with $u_{n+1}= v\cdot u_n \cdot v'$, $|v|=k_{n+1}-k_{n}>0$ and $|v'|>0$.
  In this case we indeed have $\bigcap\limits_{n\in\NN} C_{n}=\{x\}$ which, outside this remark, we write $\shift{k_n}{}(u_n) \underset{n\to\infty}{\longrightarrow} x$.
\end{remark}

\begin{remark}[Intersection of cylinders compared to limit of words]
  The notion of limit of words is a bit more permissive than that of intersection of cylinders.
  Take for example the sequence of word $u_n := 1\cdot 0^{2n+1}\cdot 1$ and positions $k_n=n+1$.
  We have $\shift{k_n}{}(u_n) \underset{n \to \infty}{\longrightarrow} 0^\ZZ$, but $\bigcup\limits_{n\in\NN}[u_n]_{k_n} = \emptyset$.
  However, taking $v_n = 0^{2n+1}$ and $k_n'=n$, we have $[u_n]_{k_n} \subset [v_n]_{k_n'}$ as $u_n = 1\cdot v_n \cdot 1$ and $\bigcap\limits_{n\in \NN}[v_n]_{k_n'} = \{ 0^\ZZ\}$.
\end{remark}


\medskip


We now prove Lemma \ref{lem:limlang-is-subshift} which states that any limit language $\limlang{L}$ is a subshift.

\begin{proof}[Proof of Lemma \ref{lem:limlang-is-subshift}]
	Let $L\subseteq A^*$.
	First remark that $\limlang{L}$ is straightforwardly shift invariant, because if $x\in\limlang{L}$ then by definition $x$ is some limit of $\shift{k_n}{}(u_n)$ and simply shifting the sequence of indices by $k$ proves that $\shift{k}{}(x)$ is in $\limlang{L}$ as $\shift{k+k_n}{}(u_n)\underset{n\to\infty}{\longrightarrow} \shift{k}{}(x)$.
	
	Now we prove similarly that $\limlang{L}$ is closed. Let $y\in A^{\mathbb{Z}}$ be the limit of a sequence $(x_n)_{n\in\mathbb{N}}$ of infinite words in $\limlang{L}$.
	First let us remark that the sequence of central factors of $x_n$ also converges to $y$, that is $(x_n)_{-n}\dots (x_n)_{n} \underset{n\to\infty}{\longrightarrow} y$.
	Since $x_n\in \limlang{L}$, it is the limit of some sequence $(\shift{k_{n,m}}{}{u_{n,m}})_{m\in\mathbb{N}}$ where the $k_{n,m}$ are integers and $u_{n,m}$ words in $L$. Hence for each $n$  there exists $f(n)$ such that $\shift{k_{n,f(n)}}{}(u_{n,f(n)})$ contains $(x_n)_{-n}\dots (x_n)_{n}$ and we now have that $\shift{k_{n,f(n)}}{}(u_{n,f(n)}) \underset{n\to\infty}{\longrightarrow} y$. Hence $y$ is in $\limlang{L}$ and so $\limlang{L}$ is closed.
\end{proof}


\begin{lemma}
	A regular language is factor-closed if and only if it is recognized by a finite automaton where all states are initial and final. Furthermore, all the states of the minimal automaton of a factor-closed language are final.
\end{lemma}
\begin{proof}
	First given an automata $\mathcal{A}$ where all states are initial and final and a word $u$ recognized by $\mathcal{A}$, from an accepting run of $u$ in $\mathcal{A}$ we can directly read accepting runs for all factors of $u$, so the language recognized by $\mathcal{A}$ is factor-closed.
	
	Now, we consider a factor closed regular language $L$.
	There exist a minimal deterministic automaton $\mathcal{A}$ recognizing $L$.
	As $L$ is factor closed and $\mathcal{A}$ is determistic, every accessible and co-accessible state (\emph{i.e.}, on a path from an initial state to an accepting state) is accepting.
	By minimality, all states are accessible and co-accessible so all states are accepting.
	Now considering $\mathcal{A}'$ where all states are also initial states.
	$\mathcal{A}'$ also recognizes $L$. First we have that $L$ is included in the language recognized by $\mathcal{A}'$ because any accepting run in $\mathcal{A}$ is also accepting in $\mathcal{A}'$.
	Now we prove that the recognized language is exactly $L$. 
	Let $u$ be a word accepted by $\mathcal{A}'$, it is accepted by a path starting on some state $q$, as $q$ is accessible in $\mathcal{A}$ there exist a word $v$ such that $v\cdot u$ is accepted on a path starting from an initial state $q_0$ of $\mathcal{A}$. Therefore $v\cdot u$ is in $L$, by factor closure $u$ is also in $L$.
\end{proof}


\begin{lemma}[Limit and pruning]
	Given a language $L\subseteq A^*$, we have $\fact{\limlang{L}} = \fact{\prun{L}}$.
	\label{lemma:fact-prun}
\end{lemma}
This result is quite natural as factors of limit words are precisely the words that have arbitrarily long left-and-right extensions in the language.
\begin{proof}
	We prove this lemma by double inclusion. Let $L$ be a language.
	First we prove $\fact{\limlang{L}}\subseteq \fact{\prun{L}}$.
	Let $v \in \fact{\limlang{L}}$, that is there exists $x \in \limlang{L}$ such that $v\sqsubseteq x$, that is, there exists $k\in \mathbb{Z}$ $\shift{k}{}(v)\subset x$.
	By definition of $\limlang{L}$, there exists two sequences $(u_n)_{n\in\mathbb{N}}$ of words in $L$ and $(k_n)_{n\in\mathbb{N}}$ of integers such that $\lim\limits_{n\to \infty} \shift{k_n}{}(u_n) = x$.
	Denote $k' = |k| + |v|$, by definition for any integer $l$ and for any word $v'$ such that $d(x,v')\leq 2^{-k'-l}$ we have $x_{[-k'-l,k'+l]} \sqsubseteq v'$ and since $v\sqsubseteq x_{[-k',k']}$ we have $v \sqsubseteq v'$ and the existence of $v_p, v_s$ of length at least $l$ such that $v_p v v_s =v'$.
	In particular as  $(\shift{k_n}{}(u_n))$ tends to $x$, there exists $n_0$ such that
	$d(\shift{k_{n_0}}{}(u_{n_0}),x)\sqsubseteq 2^{-k'}$ so that $v\sqsubseteq u_{n_0} \in L$.
	Now repeat the same reasoning with $u_{n_0}$, and denote $k'' = |k_{n_0}| + |u_{n_0}|$.
	As $(\shift{k_n}{}(u_n))$ tends to $x$, for any $l$ there exists $n$ such that we have $d(\shift{k_n}{}(u_n),x)\sqsubseteq 2^{-k''-l}$ so that $u_{n_0}\sqsubseteq u_n$ with a prefix $u_p$ and a suffix $u_s$ of length at least $l$ such that $u_p u_{n_0} u_s = u_n$.
	That is $v\sqsubseteq u_{n_0}$ and $u_{n_0} \in\prun{L}$ so that $v\in \fact{\prun{L}}$.
	
	Now the second inclusion.
	Let $u\in \fact{\prun{L}}$, that is $u\sqsubseteq u' \in \prun{L}$.
	For simplicity assume that $|u'|$ is odd and denote $k' = \frac{|u'|-1}{2}$.
	That is, for any $n$ there exists $u_{p,n},u_{s,n} \in A^{\geq n}$ such that $u_{p,n}u'u_{s,n} \in L$.
	Denote $k_n = |u_{p,n}| + k'$ and $u_n = u_{p,n} u' u_{s,n}$.
	By compactness of $A^{\mathbb{Z}}$, $(\shift{k_n}{}(u_n))$ admits a converging subsequence. Denote $x$ its limit, which by definition is in $\limlang{L}$.
	Since $u'$ is at the center of each $\shift{k_n}{}(u_n)$, more precisely $\shift{k_n}{}(u_n)|_{[-k', k']}=u'$, so we have $u'\sqsubseteq x$ and more precisely $x|_{[-k',k']}=u'$. So $u\sqsubseteq u' \sqsubseteq x$ and $u \in \fact{\limlang{L}}$.
\end{proof}

\begin{definition}[Orbit, closure]
	Given a set $X\subseteq A^{\mathbb{Z}}$ we define its orbit $\orbit{X}$ as the set of all possible shifts of words in $X$, that is $\orbit{X}:= \{ \shift{k}{}(x) , k\in \mathbb{Z}, x \in X\}$.
	
	We also define its orbit-closure $\closure{\orbit{X}}$ as the topological closure of its orbit.
\end{definition}

As subshifts are precisely sets of bi-inifinte words that are shift-invariant and closed, a set $X\subseteq A^{\mathbb{Z}}$ is a subshifts if and only if $X= \closure{\orbit{X}}$. And more generaly $\closure{\orbit{X}}$ is the smallest subshift containing $X$.
\begin{lemma}
	For any set $X$ of infinite words we have $\limlang{\fact{X}} = \closure{\orbit{X}}$.
	%% NOTE : limit(factor(X)) = orbit-closure(X)
	\label{lemma:lim-factor}
\end{lemma}

\begin{proof}
	Let $y \in \closure{X}$, by definition there exists a sequence $(x_n)_{n \in \mathbb{N}}$ of elements of $\orbit{X}$ such that $x_n \underset{n\to\infty}{\longrightarrow}y$, by definition there exists for each $n$, $x_n'\in X$ and $k_n\in \mathbb{Z}$ such that $x_n = \shift{k_n}{}(x_n')$ .
	Denote $\shift{n}{}(u_n):= x_n|_{[-n,n]}$, that is the central word on length $2n+1$ in $x_n$.
	We have for each $n$, $u_n$ in $\fact{X}$ as $u_n \sqsubseteq x_n = \shift{k_n}{}(x_n')$, and we have by definition that $\shift{n}{}(u_n) \underset{n\to\infty}{\longrightarrow} y$ so that $y\in \limlang{\fact{X}}$.
	
	Conversely, let $y \in \limlang{\fact{X}}$. There exists a sequence $(u_n)_{n\in\mathbb{N}}$ of words in $\fact{X}$ and a sequence $(k_n)_{n\in \mathbb{N}}$ of offsets such that $\shift{k_n}{}(u_n) \underset{n\to\infty}{\longrightarrow} y$.
	By definition for each $u_n$ there exists a $x_n$ in $\orbit{X}$ such that $\shift{-|u_n|/2}{}(u_n) \subset x_n$.
	By definition of the distance on $A^{\mathbb{Z}}$ the sequence $x_n$ also converges and to the same limit $y$. Hence $y$ is in $\closure{\orbit{X}}$.  
\end{proof}

\begin{lemma}
  \label{lemma:fact-is-pfc}
  For any set $X \subseteq A^\ZZ$, $\fact{X}$ is pruned and factor-closed.
\end{lemma}
\begin{proof}
  Let $X\subseteq A^\ZZ$ and $L=\fact{X}$.

  $L$ is factor closed, indeed for any $u\in L$, there exists $x\in X$ such that $u\sqsubseteq x$ by definition of $\fact{-}$. For any $v\in L$ we have $v\sqsubseteq u \sqsubseteq x$ so $v\in \fact{x} \subset \fact{X}=L$.
  So we have $L=\fact{L}$.
  
  $L$ is pruned, indeed for any $u\in L$, there exists $x\in X$ and $k\in \mathbb{Z}$ such that $\shift{k}{}(u) \subset x$.
  Denote $k_n = k+n$ and $u_n = x_{-k-n}\dots x_{-k+|u|+n}$.
  For each $n$ we have $u_n\in \fact{x}\subseteq L$ and we have $u\internal{n}u_n$. Hence $u\in \prun{L}$ and $L$ is pruned.
\end{proof}


We now prove Proposition \ref{prop:bijection-subshifts} which states that $\fact{}:\P(A^\ZZ) \to \P(A^*)$ is, when restricted from sofic subshifts to pruned factor-closed regular languages, is a bijection.

\begin{proof}[Proof of Proposition \ref{prop:bijection-subshifts}]
  In this proof we denote $\soficshifts\subseteq \P(A^\ZZ)$ the set of sofic subshifts, $\regularlanguages\subseteq \P(A^*)$ the set of regular languages and $\regularfactorpruned\subseteq \regularlanguages$ the set of factor-closed and pruned regular languages.

  Let $L\in\regularfactorpruned$, then by \Cref{lemma:fact-prun}, $\fact{\limlang{L}} = \fact{\prun{L}} = L$.

  Let $X\in\soficshifts$, by \Cref{lemma:lim-factor}, $\limlang{\fact{X}} = \closure{\orbit{X}} = X$ and since $X$ is sofic, $\fact{\limlang{\fact{X}}} = \fact{X}$ is regular. It is also pruned and factor-closed by \Cref{lemma:fact-is-pfc}.

  Therefore the restriction $\fact{-} : \soficshifts \to \regularfactorpruned$ is bijective and its inverse is the restriction $\limlang{-} : \regularfactorpruned \to \soficshifts$.
\end{proof}



\subsection{Characterization of the Behavior of a Transducer}\label{app:transducer-behavior}

We prove Lemma \ref{lem:zeta-transducer-from-transducer} which states that the behavior of the infinite transducer is the limit of the behavior of the corresponding finite transducer, that is $\L^\ZZ(\T,A,B,Q) = \limlang{\L(\T,A,B,Q,Q,Q)}$.

\begin{proof}[Proof of Lemma \ref{lem:zeta-transducer-from-transducer}]
  Let $(\T,A,B,Q)$ be a $\ZZ$-transducer.

  Denote $X_{\T} \subset (A\times B)^\ZZ$ its behavior, that is $X_{\T}= \L^\ZZ(\T,A,B,Q)$.
  And $\L_{\T}\subset (A\times B)^*$ the behavior of the finite transducer $(\T,A,B,Q,Q,Q)$, that is $\L_{\T}=\L(\T,A,B,Q,Q,Q)$.

  We prove $X_{\T}=\limlang{\L_{\T}}$.

  This holds because an infinite run in $(\T,A,B,Q)$ is precisely the limit of an increasing sequence of finite runs in $(\T,A,B,Q,Q,Q)$.

  
  We prove that $\limlang{\L_{\T}} \subseteq X_{\T}$.
  First recall that since all states are initial and final, $\L_{\T}$ is pruned and factor-closed. Hence any sequence $(\shift{k_n}{}(u_n))_{n\in\NN}$ of words in $\L_{\T}$ that converge to a bi-infinite word $x$ can be taken as strictly increasing that is $\shift{k_n}{}(u_n)\internal{1}\shift{k_{n+1}}{}(u_{n+1})$.
  Take such a sequence $(\shift{k_n}{}(u_n))$ in $\L_{\T}$.
  For each $n$, $u_n=(a_{n,0},b_{n,0})\dots (a_{n,k},b_{n,k})$ admits a run in $(\T,A,B,Q,Q,Q)$ denote $u_n'$ the word in $(A\times B \times Q)$ such that for each $i$ we have $q_{n,i} \xrightarrow[a_{n,i}]{b_n{i}} q_{n,i+1}$.

  As $(A\times B \times Q)^\ZZ$ is compact, the sequence $(\shift{k_n}{}(u_n'))_{n\in\NN}$ has subsequence that converges to some bi-infinite word $y$.
  By hypothesis the projection of $y$ on the first two components of each letter is $x$ and $y$ is the trace of an infinite run in $(\T,A,B,Q)$ and hence $x \in X_{\T}$.

  \medskip

  Now we prove the converse, that is $X_{\T} \subseteq \limlang{\L_{\T}}$.
  Let $x = ((a_i,b_i))_{i\in \ZZ} \in X_{\T}$, there exists a sequence of states $(q_i)_{i\in\NN}$ such that
  $… \xrightarrow[a_{-1}]{b_{-1}}q_0 \xrightarrow[a_0]{b_0} q_1 \xrightarrow[a_1]{b_1}$ is a bi-infinite run in $(\T,A,B,Q)$.
  For each $n$, let $k_n = n$ and $u_n=(a_{-n},b_{-n})\dots (a_n,b_n)$ (the central subword of length $2n+1$ in $x$).

  For each $n$ we have $q_{-n-1} \xrightarrow[a_{-n}]{b_{-n}} q_{-n} \dots \xrightarrow[a_n]{b_n} q_n$ which is a run of $(\T,A,B,Q,Q,Q)$ of length $2n+1$.
  Hence $u_n\in \L_{\T}$. Therefore $x=\lim\limits_{n\to\infty} \shift{k_n}{}(u_n)$ is in $\limlang{\L_{\T}}$
  

\end{proof}







\subsection{Uniqueness of the Minimal Rooted Right-Resolving Pruned Presentation}\label{app:inf-uniqueness}

We recall that a presentation of a sofic subshift $X$ is simply a $\ZZ$-transducer with $\one$ for output alphabet, so we start by studying completeness in the case where $B = \one$. A presentation of a sofic subshift is:
\begin{description}
	\item[Pruned] if its language is pruned, meaning that for every word $w$ in the language, there exists $u$ and $v$ both non-empty such that $uwv$ is in the language.
	\item[Right-Resolving] whenever its transition relation is a partial function.
	\item[Rooted] whenever there exists a root state $r$ such that
	\begin{itemize}
		\item every accepted word admits an accepted run starting by $r$,
		\item and every state is accessible from $r$, meaning there is a run from $r$ to that state.
	\end{itemize}
\end{description}

\begin{lemma}[Pruned Presentation]\label{applem:pruned}
	For any pruned presentation $(\T,A,Q)$ of the sofic subshift $X$, the language recognized by the presentation is exactly $\fact{X}$.
\end{lemma}
\begin{proof}
	Let $L$ be the language recognized by $(\T,A,Q)$. By definition, we have $\limlang{L} = X$. 
	By \Cref{prop:bijection-subshifts}, $L = \fact{\limlang{L}} = \fact{X}$.
\end{proof}


\begin{theorem}[Uniqueness]\label{appthm:minimal-uniqueness-inf}
	Every non-empty sofic subshift admits a unique minimal rooted right-resolving pruned presentation, up to isomorphism.
\end{theorem}
\begin{proof}
	Given a sofic subshift $X$, let us consider the minimal deterministic automaton $(\D,A,Q,\{i\},F)$ that recognizes $\fact{X}$. Within that automaton, we write $\xrightarrow[\text{\small $\T$}]{}$ for the transitions, and $\xrightarrow[\text{\small $\D$}]{w}$ for the iterated transitions.	
	
	If we take $w = w_1\dots w_n \in \fact{X}$, it corresponds to a run from $i \xrightarrow[\text{\small $\D$}]{w} f \in F$. Since $\fact{X}$ is factor-closed, we also have $w_1\dots w_k \in \fact{X}$ for $0 \leq k \leq n$, and since $\D$ is deterministic it means that all the states of the run  $i \xrightarrow[\text{\small $\D$}]{w} f$ are also final states. This means that if we restrict our automaton to only its final states, so $(\D,A,F,\{i\},F)$, then the language recognized is identical. Note that $\D$ is now a partial function (except in the case $\fact{X} = A^*$).
	
	We now consider $q\in F$ and $u \in A^*$ such that there exists a $f \in F$ satisfying $p \xrightarrow[\text{\small $\D$}]{u} f$. In the minimal deterministic automaton, every state is accessible (otherwise the automaton would not be minimal), so there exists $v \in A^*$ such that $i \xrightarrow[\text{\small $\D$}]{v} q$, meaning that $vu \in \fact{X}$. Since $\fact{X}$ is factor-closed, this means that $u \in \fact{X}$, so there exists a run $i \xrightarrow[\text{\small $\D$}]{v} p$ for some $p \in F$. This means that if we extend initial states to be $F$, so the non-deterministic automaton $(\D,A,F,F,F)$, then the language recognized is identical. Additionally, any word accepted by this automaton admits an accepting run that start with $i$, and every state is accessible from $i$. Said otherwise, $(\D,A,F)$ is a representation of the sofic subshift $\limlang{\fact{X}}$, that is pruned, right-resolving and rooted in $i$. Using \Cref{prop:bijection-subshifts}, $\limlang{\fact{X}} = X$.
	
	Now for uniqueness, we consider another minimal pruned right-resolving presentation $(\M,A,P)$, rooted in $r$. If $\M$ is already total (which only happens whenever $\fact{X} = A^*$), $(\M,A,P,\{r\},P)$ is the minimal deterministic automaton recognizing $\fact{X}$. Assuming $\M$ is not a total function, we build build from it a deterministic automaton $(\M_\bot,A,P \sqcup \{\bot\},\{r\},P)$ where $\M_\bot$ is the total function that extends $\M$ by adding a transition toward the new state $\bot$ whenever $\M$ would be undefined. This deterministic automaton recognizes the same language as the presentation, which according to \Cref{applem:pruned} is exactly $\fact{X}$. We argue it is the minimal deterministic automaton recognizing $\fact{X}$. Indeed, if we take a deterministic automaton recognizing $\fact{X}$ that is strictly smaller, then by following the same procedure we used above on $\D$ to make it a pruned right-resolving presentation, we would obtain one that is strictly smaller than $\M$, which contradict minimality.
	
	In both cases, we have a minimal deterministic automaton that recognizes the same language as $(\D,A,P,\{i\},F)$, so by uniqueness of the minimal automata (see \cite{DBLP:books/daglib/0016921}) we have and isomorphism $\iota$ between the two. This directly induces an isomorphism between $(\D,A,F)$ and $(\M,A,P)$.
\end{proof}


