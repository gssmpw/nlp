% !TeX root = article.tex
% !TeX spellcheck = en_US


%\subsection{Informal Approach}
\label{sec:logic}

Before introducing transducers, we quickly present the formalism we use to handle relations between finite sets. We later use the same formalism for relations between infinite sets, more specifically sets of finite words and sets of bi-infinite words.
In the context of relations, those infinite sets will always be clearly denoted with a $\_^*$ or a $\_^\ZZ$, as such, the letters $A,B,C,\dots$ will refer to finite sets.

Given some finite sets $A_1,\dots,A_n,B_1,\dots,B_m$, a relation $\R$ from $\Pi_{i=1}^n A_i$ to $\Pi_{j=1}^m B_j$ is simply a subset $\R \subseteq \Pi_{i=1}^n A_i \times \Pi_{j=1}^m B_j$. We write $a~\R~b$, and say that $\R$ relates $a$ to $b$, whenever $(a,b) \in \R$. We represent such relation as the diagram on the left, the identity relation on $\Pi_{i=1}^n A_i$ as the one on the middle, and the transposed of $\R$ (defined by $b\R^t a$ iff $a\R b$) as the one on the right.
\[ \tikzfig{rel-def} \qquad \qquad \tikzfig{rel-identity} \qquad \qquad \tikzfig{rel-transposed} \]

We write $\one = \{()\}$ for the singleton set which is the zero-ary Cartesian product, and we will often omit diagrammatically the wires labeled by $\one$. Three notable relations are the swap $\gamma_{A,B} = \{ ((a,b),(b,a)) \mid a \in A, b \in B\}$, the cap $\eta_A = \{ ((),(a,a)) \mid a \in A\} : \one \to A \times A$ and the cup $\epsilon_A = \{((a,a),()) \mid a \in A\} : A \times A \to \one$, which we represent with bent wires as below. We also represent below the ``full'' relation $\bullet_{A,B} = A \times B : A \to B$.
\[ \tikzfig{rel-swap-cup-cap} \]
\begin{figure}[!h]
	\tikzfig{rel-equations-cup-cap}
	\caption{Equations for the Cup and Cap.} 
	\label{fig:rel-equations-cup-cap}
\end{figure}
We give in \Cref{fig:rel-equations-cup-cap} a couple of remarkable identities about them.
We note that there is a lot of arbitrary choices in those representations, as for example a relation from $A \times B$ to $\one$ could be represented in any of the following ways.
\[ \tikzfig{rel-equivalent-rep}\]





Nevertheless, those graphical representations remain practical, especially when representing various kind of compositions. 
The usual ones are the sequential compositions of $\R : A \to B$ with $\S : B \to C$, and the parallel composition of $\R : A \to B$ and $\S : C \to D$, which are defined as follows:
\[ \tikzfig{rel-normal-composition}\]

\begin{wrapfigure}{r}{0.2\textwidth}
	\tikzfig{rel-partial-composition}
\end{wrapfigure}
The general case is the partial composition. For example, given two relations $\R : A \to B \times C$ and $\S : C \times D \to E$, their composite relation $\{ ((a,d),(b,e)) \mid \exists c, a~\R~(b,c), (c,d)~\S~e \} : A \times D \to B \times E$ is written diagrammatically as shown on the right. Conversely, whenever we have a diagram that decomposes $\R : \Pi_{i=1}^n A_i \to \Pi_{j=1}^m B_j$, into the partial composition of $\R_1,\dots,\R_r$, and if we write $C_1, \dots, C_p$ for all the sets labeling the ``internal wires'' of the diagram, then we can build a logical formula equivalent to $(a_1,\dots,a_n)~\R~(b_1,\dots,b_m)$ of the following form:
\[ \exists c_1 \in C_1, \dots, \exists c_p \in C_p, (\_ \R_1 \_) \land \dots \land (\_ \R_r \_) \]
where the blanks $\_$ have to be filled with adequate tuples of $a_i$, $b_j$ and $c_k$. For example, the following diagram would correspond to the following logical formula:
\[a~\left(\tikzfig{rel-logic-example}\right)~b \iff \exists c_1 \in C_1, \exists c_2 \in C_2, ((a,c_1)~\R_1~(c_2,b)) \land (c_1~\R_2~c_2) \]
\begin{wrapfigure}{r}{0.2\textwidth}
	\tikzfig{rel-reorganise}
\end{wrapfigure}

We remark that if any of the $\R_\ell$ is the full relation $\bullet_{A,B}$, then it can safely be omitted from the logical formula as $(a~\bullet_{A,B}~b)$ is always true.

Combining the partial composition with the swap, cup and cap defined above, one can reorganize the inputs and outputs of a relation, turning for example a relation $\R : A \times B \times C \to D$ into a relation from $B \times A$ to $D \times C$ as on the right. 

All those diagrams can actually be formalized using string diagrams from category theory  \cite{MacLane}. Indeed, finite sets and relations form a strict symmetric monoidal category, called \bfup{FinRel}. We recall the definition of such a category in \Cref{app:relations}, though for the sake of this paper, the only required understanding is ``the diagrams works as intended, there is never a need to explicitly add bracketing, and all 'reasonable' ways of rewriting a diagram yield the same relation''. If we account for the cup and cap, relations even form a compact closed category\footnote{Where the dual is the transposition on relations, and the identity on objects.}, which exactly means that it satisfies the equations listed in \Cref{fig:rel-equations-cup-cap}. We redirect to \cite{selinger2011survey} for a survey of variations around monoidal categories and the corresponding diagrammatic notations.
