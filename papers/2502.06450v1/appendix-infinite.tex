% !TeX root = article.tex
% !TeX spellcheck = en_US

\subsection{From $\ZZ$-Transducers to Graphical Representation}\label{app:inf-zeta-transducer-red}

In this section, we prove \Cref{prop:zeta-transducer-red}, that is for every $\ZZ$-transducer $\T$, its behavior can be obtained using the lift of $\T$ as follows:
\[ \tikzfig{rel-trans-inf-red} \]

We write $\R$ for the left-hand-side. We take $w \in A^\ZZ$ and $v \in B^\ZZ$, and note that by definition of the composition of relations, $w~\R~v$ if and only there exists two words $q,u \in Q^\ZZ$ such that $(w,q)~\T^\ZZ~(v,u)$ and $q~\transpose{\left(\shift{}{}\right)}~u$. 

Let us focus on the latter, it is equivalent to the for all $k \in \ZZ$, $q_{k+1} = u_k$.
Let us focus on the former, it is equivalent to, for every $k \in \ZZ$, $(w_k,q_k)~\T~(v_k,u_k)$, that is $q_k \xrightarrow[v_k]{w_k} u_k$. Combining both, we obtain $q_k \xrightarrow[v_k]{w_k} q_{k+1}$, which yields exactly the definition of a run. So we have that $w~\R~v$ if and only if $w~\L^\ZZ(\T,A,B,Q)~v$. \qed


\subsection{Explicit Definition of the Graphical Language on Bi-Infinite Words}\label{app:infinite-language}

We start by defining formally the graphical language $\ZZ$\bfup{-Trans}. It is a category where the objects are finite sets, and where the morphisms are generated by composing sequentially and in parallel the generators of \Cref{appfig:graph-generators-inf} together with the equations of a strict symmetric monoidal category (see \Cref{appfig:rel-equations-inf}), equations of a feedback  category (see \Cref{appfig:graph-feedback-inf}), and equations ensuring we faithfully embed \bfup{FinRel} (see \Cref{appfig:graph-finrel-inf}).

\begin{figure*}[h]
	\tikzfig{graph-generators-inf}
	\caption{Generators of  $\ZZ$\bfup{-Trans}.}
	\label{appfig:graph-generators-inf}
\end{figure*}

\begin{figure*}[h]
	\tikzfig{rel-equations-inf}
	\caption{Equations for a Strict Symmetric Monoidal Category.}
	\label{appfig:rel-equations-inf}
\end{figure*}

\begin{figure*}[h]
	\tikzfig{graph-feedback-inf}
	\caption{Equations for a Feedback Category.}
	\label{appfig:graph-feedback-inf}
\end{figure*}

\begin{figure*}[h]
	\tikzfig{graph-finrel-inf}
	\caption{Equations for Faithfully Embedding \bfup{FinRel}.}
	\label{appfig:graph-finrel-inf}
\end{figure*}

Let us point a couple of facts about this language:
\begin{itemize}
	\item A bundle of wires labeled $A_1,\dots,A_n$ is the same as a single wire labeled $\Pi_{i=1}^n A_i$, and wires labeled $\one$ can be omitted.
	\item The wires and every syntactical construct are in \bfup{\red{thick red}}, to distinguish them from actual relations. In particular, $\R$ refers to an actual relation while $\rR$ refers to an element of our language.
	\item Double-line boxes denote the generator, while single-line boxes denote any diagram potentially constituted of many generators, including feedbacks.
	\item The arrows on the feedback are a reminder that those are not the same as the cup and cap of \Cref{sec:relations}.
	\item Within \Cref{appfig:graph-feedback-inf}, the gray lines correspond to bracketing, and similarly to \Cref{fig:rel-equations} the overall consequences of those equations is that bracketing can be safely ignored. 
	\item The equations of \Cref{appfig:graph-finrel-inf} ensures that if a diagram does not contain any instance of the ``feedback'' generator, we can merge all the generators into a single double-line box.
	\item The current equational theory is incomplete, additional equations will be added in \Cref{fig:inf-simulation-principle} to obtain completeness.
\end{itemize}

In order to ensure that the equations are not contradictory\footnote{Which would lead to the trivial category where all the diagrams are equal to one another.}, we provide a semantics and prove soundness of our equations. The semantics is a strong symmetric monoidal functor from $\ZZ$\bfup{-Trans} to \bfup{Rel}, which we write $\rinterp{-}$, and is actually simply ``removing the color and adding a $\_^\ZZ$ everywhere''. We provide an explicit definition in \Cref{appfig:inf-interp}. 

\begin{figure*}
	\[\tikzfig{inf-interp}\]
	\caption{Inductive Definition of the Semantics $\rinterp{-} : \ZZ\bfup{-Trans} \to \ZZ\bfup{-Rel}$.}
	\label{appfig:inf-interp}
\end{figure*}

\subsection{Soundness of the Standard Equations}\label{app:inf-soundness}

We now prove the soundness of the equational theory. 
Soundness means that if one rewrites a diagram $\rR$ into $\rS$ using any of the listed equations, we still have $\rinterp{\rR} = \rinterp{\rS}$.  The soundness of the equation of \Cref{appfig:rel-equations-inf} follows immediately from the fact that $\bfup{Rel}$ is a strict symmetric monoidal category, so we only need to look at the equations of \Cref{appfig:graph-feedback-inf}. The top-left and bottom-left equations also follow from the fact that $\bfup{Rel}$ is a strict symmetric monoidal category. The top-right is sound because $\shift{}{}_\one = \id_\one$, and the bottom-right is sound because $\shift{}{}_C \times \shift{}{}_D = \shift{}{}_{C \times D}$. \qed


\subsection{Quasi-Normal Form}\label{app:inf-quasi-normal-form}

We now explicitly state the generalization of \Cref{prop:normal-form-fin} to the infinite case.

\begin{proposition}[Quasi-Normal Form]\label{prop:normal-form-inf}
	Any diagram of $\rR \in \ZZ$\bfup{-Trans} from $A$ to $B$ can be put in the following form for some finite set $Q$ and $\T \in \bfup{FinRel}$.
	\[ \tikzfig{graph-from-trans-inf}\]
\end{proposition}
\begin{proof}
We start by using the first equation of \Cref{appfig:graph-feedback-inf} from right to left to push all the feedbacks at the bottom of the diagram. Then, we use the equations of \Cref{appfig:graph-finrel-inf} to merge all the non-feedback into a single box. Lastly, we use the last equation of \Cref{appfig:graph-feedback-inf} to merge all the feedbacks into a single feedback.
\end{proof}

\subsection{Universality}\label{app:inf-universality}

We now explicitly state generalization of \Cref{thm:universality-fin} to the infinite case.
\begin{theorem}[Universality]\label{thm:universality-inf}
	For any $\S \in \ZZ$\bfup{-Trans}, $\rinterp{\rS}$ is a sofic relation. 
	For any sofic relation $\R$, there exists $\S \in \ZZ$\bfup{-Trans} such that $\rinterp{\rS} = \R$.
\end{theorem}
\begin{proof}
It directly follows from \Cref{prop:normal-form-inf} and \Cref{prop:zeta-transducer-red}.
\end{proof}

\subsection{Soundness of the Simulation Principle}\label{app:inf-simulation-principle}


\begin{figure*}[h]
	\[\tikzfig{inf-simulation-principle}\]
	\caption{Simulation Principle for Bi-Infinite Words, and an Equation Deducible from it.} %% QUESTION majuscules ???? 
	\label{appfig:inf-simulation-principle}
\end{figure*}

\begin{figure*}[h]
	\[\tikzfig{inf-backward-simulation-principle}\]
	\caption{Backward-Simulation Principle for Bi-Infinite Words.}
	\label{appfig:inf-backward-simulation-principle}
\end{figure*}


\begin{figure*}[h]
	\[\tikzfig{inf-forward-simulation-principle}\]
	\caption{Forward-Simulation Principle for Bi-Infinite Words.}
	\label{appfig:inf-forward-simulation-principle}
\end{figure*}

We start by proving the following compactness result.
For any $\ZZ$-transducer $(\T,A,B,Q)$, we recall that by definition, we have $(a_n)_{n \in \ZZ}~\L^\ZZ(\T,A,B,Q)~(b_n)_{n \in \ZZ}$ if and only if there exists $(q_n)_{n \in \ZZ}$ such that $\forall n \in \ZZ$, $(a_n,q_{n+1})~\T~(b_n,q_n)$.
\begin{theorem}[Compactness]\label{appthm:compactness} 
	It is sufficient to only consider finite sequences, meaning that $(a_n)_{n \in \ZZ}~\L^\ZZ(\T,A,B,Q)~(b_n)_{n \in \ZZ}$  if and only if there exists for all $N \leq M \in \ZZ$ a finite sequence $(q_{n}^{N,M})_{N \leq n \leq M}$ such that $\forall N \leq n < M \in \ZZ$, $(a_n,q^{N,M}_{n+1})~\T~(b_n,q^{N,M}_n)$.
\end{theorem}
\begin{proof}
	The direct implication is trivial, as we simply take $q_{n}^{N,M} := q_n$. For the indirect implication, it is in fact enough to only consider the finite sequences centered on zero, that is $(q_{n}^{-N,N})_{-N \leq n \leq N}$ for $N \geq 0$, and ignore those that are not centered on zero. From them, we will define inductively $(q_n)_{n \in \ZZ}$. In order to do so, we add an additional induction hypothesis: there exists an infinite amount of $N \geq n$ such that $\forall -n \leq k \leq n$, $q^{-N,N}_{k} = q_k$.
	
	We start by arbitrarily providing a total order $\leq_Q$ over the finite set $Q$. 
	
	\bfup{Base Case}: We look at all the $q \in Q$ and check whether there exists an infinite amount of $N \geq 0$ such that $q^{-N,N}_0 = q$. Since $Q$ is finite, there must exists a $q$ satisfying that condition. We take $q_0$ to be the minimal $q$ (for $\leq_Q$) satisfying that condition.
	
	\bfup{Inductive Case}: We assume that we have defined $q_{-n},\dots,q_0,\dots,q_n$ such that:
	\begin{itemize}
		\item For all $-n \leq k < n$, $(a_n,q_{n+1})~\T~(b_n,q_n)$.
		\item There exists an infinite amount of $N \geq n$ such that $\forall -n \leq k \leq n$, $q^{-N,N}_{k} = q_k$.
	\end{itemize} 
	We look at all the pairs $(q_{\ominus},q_{\oplus}) \in Q^2$ and check whether there exists an infinite amount of $N \geq n+1$ such that $q^{-N,N}_{-n-1} = q_{\ominus}$ and $q^{-N,N}_{n+1} = q_{\oplus}$. Since $Q^2$ is finite, there must exists a pair $(q_{\ominus},q_{\oplus})$ satisfying that condition. We take $(q_{-n-1},q_{n+1})$ to be the minimal pair (for the lexicographic order) satisfying that condition. By definition, both induction hypothesis are preserved.
\end{proof}

We start by proving the soundness of \Cref{appfig:inf-backward-simulation-principle} with respect to $\rinterp{-}$. We need to prove that if we consider $(a_n)_{n \in \ZZ}~\L^\ZZ(\R,A,B,C)~(b_n)_{n \in \ZZ}$ then  $(a_n)_{n \in \ZZ}~\L^\ZZ(\T,A,B,D)~(b_n)_{n \in \ZZ}$. 

Focusing on the right-hand-side, and using the above \Cref{appthm:compactness}, it is enough to show there exists for all $N \leq M \in \ZZ$ a finite sequence $(d_{n}^{N,M})_{N \leq n \leq M}$ such that $\forall N \leq n < M \in \ZZ$, $(a_n,d^{N,M}_{n+1})~\T~(b_n,d^{N,M}_n)$. Or, in diagrammatic terms, it is enough to show that the following relation relates $(a_N,\dots,a_M)$ to $(b_N,\dots,b_M)$.
\[ \tikzfig{inf-simulation-principle-sound-right} \]

Focusing on the left-hand-side, it means that there exists $(c_n)_{n \in \ZZ}$ such that $\forall n \in \ZZ$, $(a_n,c_{n+1})~\T~(b_n,c_n)$. It implies that for $N \leq M$, the following relation relates $(a_N,\dots,a_M)$ to $(b_N,\dots,b_M)$.
\[ \tikzfig{inf-simulation-principle-sound-left} \]
Using the bottom precondition of \Cref{appfig:inf-backward-simulation-principle}, we can replace the left part of the above diagram by $\S \circ \bullet_D$ and the result relation would still relate $(a_N,\dots,a_M)$ to $(b_N,\dots,b_M)$. Then using the top precondition we can slide the $\S$ toward the right and transform every $\R$ into a $\T$, and finally using that the full relation is maximal we can replace $\bullet_C \circ \S$ by $\bullet_D$. The resulting relation still relates $(a_N,\dots,a_M)$ to $(b_N,\dots,b_M)$. \qed

Proving the soundness of \Cref{appfig:inf-forward-simulation-principle} follows an identical proof, albeit mirrored. And the soundness of the simulation principle of \Cref{appfig:inf-simulation-principle} follows from both the backward and forward principle. \qed


\subsection{Deducing the Sliding Equation}\label{app:inf-slide}

We prove that the bottom equation from \Cref{appfig:inf-simulation-principle} can be deduced from the others. For that, we start by using \Cref{prop:normal-form-inf} on $\rR$:
\[\tikzfig{graph-slide-proof-normal-form-inf}\]
We can then merge the two feedbacks into one, rewriting both sides of the equation we are trying to prove into:
\[\tikzfig{graph-slide-proof-rewrite-inf}\]

We then conclude using the the simulation principle, using the following prerequisites and the fact that $\bullet$ is the maximal relation for the inclusion:
\[\tikzfig{graph-slide-proof-prerequisites-inf}\]
~\qed




\subsection{Completeness of the Simulation Principle}\label{app:inf-completeness}

The core idea of the proof is similar to the finite case, meaning we will determinize then minimize the presentation of our sofic subshift, except we have to start by pruning the states of our presentation that cannot be part of any bi-infinite path. Whenever we consider $(\T,A,Q)$ to be a presentation of sofic subshift, we write $\xrightarrow[\text{\small $\T$}]{}$ for the transitions within that presentation, and we write $\xrightarrow[\text{\small $\T$}]{w}$ with $w$ a finite word of size $k$ for the iterated transition $\xrightarrow[\text{\small $\T$}]{w_1} \dots \xrightarrow[\text{\small $\T$}]{w_k}$.

\begin{definition}[Forward-Pruning]
	Let $(\T,A,Q)$ be a presentation of a sofic subshift. Its forward-pruning is the presentation $(\T,A,Q^{\to \infty})$, where $Q^{\to \infty}$ is the restriction of $Q$ to the states $q_0$ that are the start of an infinite path $q_0 \xrightarrow[\text{\small $\T$}]{a_1} q_1 \xrightarrow[\text{\small $\T$}]{a_2} \dots$, or equivalently of a path of size at least $\textup{card}(Q)$ as such a path necessarily loops, and where $\T$ has been restricted to $A \times Q^{\to \infty} \to Q^{\to \infty}$.
\end{definition}

\begin{proposition}\label{prop:forward-pruning-inf}
	For all presentation of a sofic subshift $(\T,A,Q)$, we have
	\[ \tikzfig{inf-forward-pruning}\]
	where $= : Q \to Q^{\to \infty}$ is the usual ``equality'' relation.
\end{proposition}
\begin{proof}
	The first equation state that if $q_0$ is the start of an infinite path, and $q_{-1} \xrightarrow[\text{\small $\T$}]{a_{0}} q_0$, then $q_{-1}$ is also the start of an infinite path, which is true.
	
	The second equation is trivially true as the right-hand-side simplifies to $\bullet_{Q^{\to\infty}}$, which is maximal for the inclusion.
	
	The third equation state that if $q_0$ is the start of a path of size $\textup{card}(Q)$, then it is the start of an infinite path, which is true because a path of size $\textup{card}(Q)$ necessarily loops.	
\end{proof}

\begin{corollary}\label{appcor:forward-pruning-inf}
	For all presentation of a sofic subshift $(\T,A,Q)$, using \Cref{fig:inf-simulation-principle}, we have
	\[ \tikzfig{inf-forward-pruning-conclusion}\]
\end{corollary}


\begin{definition}[Backward-Pruning]
	Let $(\T,A,Q)$ be a presentation of a sofic subshift. Its backward-pruning is the presentation $(\T,A,Q^{\infty\to})$, where $Q^{\infty\to}$ is the restriction of $Q$ to the states $q_0$ that are the end of an infinite path $\dots \xrightarrow[\text{\small $\T$}]{a_{-1}} q_{-1} \xrightarrow[\text{\small $\T$}]{a_0} q_0$, or equivalently of a path of size at least $\textup{card}(Q)$ as such a path necessarily loops, and where $\T$ has been restricted to $A \times Q^{\infty\to} \to Q^{\infty\to}$.
\end{definition}


\begin{proposition}\label{prop:backward-pruning-inf}
	For all presentation of a sofic subshift $(\T,A,Q)$, we have
	\[ \tikzfig{inf-backward-pruning}\]
	where $= : Q^{\infty \to} \to Q$ is the usual ``equality'' relation.
\end{proposition}
\begin{proof}
	The first equation state that if $q_0$ is the end of an infinite path, and $q_0 \xrightarrow[\text{\small $\T$}]{a_{1}} q_1$, then $q_1$ is also the end of an infinite path, which is true.
	
	The second equation is trivially true as the left-hand-side simplifies to $\bullet_{Q^{\infty\to}}$, which is maximal for the inclusion.
	
	The third equation state that if $q_0$ is the end of a path of size $\textup{card}(Q)$, then it is the end of an infinite path, which is true because a path of size $\textup{card}(Q)$ necessarily loops.
\end{proof}

\begin{corollary}\label{appcor:backward-pruning-inf}
	For all presentation of a sofic subshift $(\T,A,Q)$, using \Cref{fig:inf-simulation-principle}, we have
	\[ \tikzfig{inf-backward-pruning-conclusion}\]
\end{corollary}

\begin{definition}[Pruning]
	Let $(\T,A,Q)$ be a presentation of a sofic subshift. Its pruning is the presentation $(\T,A,Q^{\infty\to\infty})$, where $Q^{\infty\to\infty}$ is the restriction of $Q$ to the states $q_0$ that are on a bi-infinite path $\dots \xrightarrow[\text{\small $\T$}]{a_{-1}} q_{-1} \xrightarrow[\text{\small $\T$}]{a_0} q_0 \xrightarrow[\text{\small $\T$}]{a_1} q_1 \xrightarrow[\text{\small $\T$}]{a_2} \dots$, or equivalently at the middle of a path of size at least $2 \times \textup{card}(Q)$ as such a path necessarily loops before and after $q_0$, and where $\T$ has been restricted to $A \times Q^{\infty\to} \to Q^{\infty\to}$.
\end{definition}

\begin{lemma}\label{lem:double-pruning}
	The pruning of a presentation is equal to the forward-pruning of its backward-pruning, or equivalently the backward pruning of its forward-pruning. The pruning of a representation is always a \textbf{pruned} representation.
\end{lemma}
\begin{proof}
	If a state $q_0$ is on a bi-infinite path $(q_n)_{n \in \ZZ}$, then all the states $q_n$ are on an bi-infinite path, so none of them will be removed by the forward-pruning or backward-pruning, meaning that $Q^{\infty \to \infty} \subseteq (Q^{\to \infty})^{\infty \to}$. We then take $q_0 \in (Q^{\to \infty})^{\infty \to}$, there is an infinite path $(q_n)_{n \leq 0}$ in $(Q^{\to \infty})^{\infty \to}$ that ends on $q_0$, and there is an infinite path  $(q_n)_{n \geq 0}$ in $Q^{\to \infty}$ that starts with $q_0$. So $q_0 \in Q^{\infty \to \infty}$, hence $Q^{\infty \to \infty} = (Q^{\to \infty})^{\infty \to}$. The same reasoning can be done to prove $Q^{\infty \to \infty} = (Q^{\infty\to})^{\to\infty}$.
\end{proof}


\begin{definition}[Determinization]
	Let $(\T,A,Q)$ be a pruned presentation of a \bfup{non-empty} sofic subshift. Its determinization is the \textbf{pruned right-resolving rooted} presentation $(\Pne(\T), A, \Pne^{\textup{acc}}(Q))$.
	More precisely, we start by recalling that $\Pne(Q)$ is the set of all non-empty subsets of $Q$. Since our subshift is non-empty, $Q \in \Pne(Q)$. We use $x,y$ for elements of $Q$, and $X$,$Y$ for elements of $\Pne(Q)$.  Then, we define the partial function\footnote{We consider partial functions to be a special case of relations.} $\Pne(\T): A \times \Pne(Q) \to \Pne(Q)$ as 
	\[ \Pne(\T)(a,X)=\{y\in Q ~|~ \exists x\in X,~  x \xrightarrow[\text{\small $\T$}]{a} y \} \bfup{ if }\text{it is non-empty (undefined otherwise)} \]
	We then consider the set $\Pne^{\textup{acc}}(Q)$ of subsets of $Q$ accessible by iteration of that function, starting from $Q$. We can now restrict $\Pne(\T)$ to a function $A \mapsto \Pne^{\textup{acc}}(Q) \to \Pne^{\textup{acc}}(Q)$. The root is simply $Q$, and every state is accessible from $Q$, by definition of $ \Pne^{\textup{acc}}(Q)$.
\end{definition}


\begin{proposition}\label{prop:determinization-inf}
	For all pruned presentation of a non-empty sofic subshift $(\T,A,Q)$, we have
	\[ \tikzfig{inf-determinization}\]
	where $\ni : \Pne^{\textup{acc}}(Q) \to Q$ is the usual ``contains'' relation, that is $X \ni x$ whenever $x \in X$.
\end{proposition}
\begin{proof}
	In this proof, we write $\P(\T)$ for the total function from $A \times \P(Q) \to \P(Q)$ which extends $\Pne(\T)$ as below, and note that $\Pne(\T)(a,X)$ is defined if and only if $\P(\T)(a,X) \neq \varnothing$.
	\[ \P(\T)(a,X)=\{y\in Q ~|~ \exists x\in X,~  x \xrightarrow[\text{\small $\T$}]{a} y \} \]
	
	Using the logical reasoning as in \Cref{sec:logic}, we can rewrite the first equation as the following. We are looking at $\forall a\in A,~ \forall X\in \Pne^{\textup{acc}}(Q),~\forall y\in Q,~$
	
	\[ \left(\exists x\in Q,~ (x\in X) \land (x \xrightarrow[\text{\small$\T$}]{a} y)\right) ~\Leftrightarrow~ \left(\exists Y\in \Pne^{\textup{acc}}(Q),~   (X \xrightarrow[\text{\small$\Pne(\T)$}]{a} Y) \land (y\in Y)\right)\]
	We start by reformulating the right side of the equivalence, as $\Pne(\T)$ is a partial function, we can replace the $\exists$ and obtain $(\P(\T)(a,X) \neq \varnothing) \land (y \in \Pne(a,X))$, which is equivalent to $y \in \P(\T)(a,X)$. Then, using the definition of $ \P(\T)$, we obtain that it is equivalent to $\exists x \in X, x \xrightarrow[\text{\small$\T$}]{a} y$, which is exactly the left side of the equivalence. 
	
	The second and third equations are trivially true as their left-hand-side simplifies to $\bullet_{\Pne^{\textup{acc}}(Q)}$ and $\bullet_{Q}$, which are maximal for the inclusion.
\end{proof}

\begin{corollary}\label{appcor:determinization-inf}
	For all pruned presentation of a non-empty sofic subshift $(\T,A,Q)$, using \Cref{fig:fin-simulation-principle}, we have
	\[ \tikzfig{inf-determinization-conclusion}\]
\end{corollary}

\begin{definition}[Minimization]	
	Let $(\D,A,P)$ be a \textbf{pruned right-resolving rooted} presentation of a sofic subshift, and we write $r$ for its\footnote{While it is possible for that presentation to have multiple roots, choosing a different root have no influence on this minimization, as the $L_w$ defined would be identical.} root. Its minimzation is the \textbf{right-resolving rooted} presentation $(L_\D,A,L^{\neq \varnothing}_{A^*})$.
	More precisely, we write $L_w = \{ v \in A^* \mid \exists f \in P, r \xrightarrow[\text{\small $\D$}]{wv} f\}$, and remark that whenever $L_w = L_u$, then for all $a \in A$ we also have $L_{wa} = L_{ua}$. We then define $L^{\neq \varnothing}_{A^*} = \{ L_w \neq \varnothing \mid w \in A^*\}$ note that a single element of this set might correspond to multiple distinct $w$, and in fact $L_{A^*}$ is actually smaller or equal to $P$ in cardinality. Lastly, we define the transition function as $L_\D(a,L_w) =  L_{wa}$ whenever it is non-empty, and undefined otherwise.
\end{definition}

Similarly to the uniqueness of the minimal deterministic automata, this minimization yield the unique (up to isomorphism) minimal \textbf{pruned right-resolving rooted} presentation of the given sofic subshift, see \Cref{appthm:minimal-uniqueness-inf}.

\begin{proposition}
	For all pruned right-resolving rooted presentation $(\D,A,P)$, we have
	\[ \tikzfig{inf-minimization}\]
	where $\L : P \to L^{\neq \varnothing}_{A^*}$ relates $p \in P$ to $\ell \in L^{\neq \varnothing}_{A^*}$ whenever $\ell = \{ v \mid \exists f \in P, p \xrightarrow[\text{\small$\D$}]{v} f\}$. We note that since every state is accessible, there exists a $w \in A^*$ such that the later is equal to $L_w$.
\end{proposition}
\begin{proof}
	Using the logical reasoning as in \Cref{sec:logic}, we can rewrite the first equation as the following. We are looking at $\forall a \in A, \forall p \in P, \forall \ell \in L^{\neq \varnothing}_{A^*},$
	\[ \begin{array}{c} \left(\exists q\in P,~  (p \xrightarrow[\text{\small$\D$}]{a} q)\land (\ell = \{ w \mid \exists f \in F, q \xrightarrow[\text{\small$\D$}]{w} f\}) \right) \\ \Leftrightarrow \\ \left(\exists m \in L^{\neq \varnothing}_{A^*},~  (m = \{ v \mid \exists f \in P, p \xrightarrow[\text{\small$\D$}]{v} f\}) \land (m \xrightarrow[\text{\small$L_\D$}]{a} \ell)\right) \end{array} \]
	Before studying either side of that equivalence, we note that since $p$ is accessible from the root state $r$, there exists a $u \in A^*$ such that $r \xrightarrow[\text{\small$\D$}]{u} p$, hence $L_u = \{ v \mid \exists f \in F, p \xrightarrow[\text{\small$\D$}]{v} f\}$.
	
	We start by reformulating the second part of the equivalence.  Since the formula for $m$ is given, we can remove the $\exists m$ and obtain $L_u \xrightarrow[\text{\small$L_\D$}]{a} \ell$. We can now use the definition of $L_\D$ and simplify the later in $(L_{ua} \neq \varnothing) \land (\ell = L_{ua})$.
	
	We now look at the first part of the equivalence, since $\D$ is right-resolving, $q$ is uniquely determined if it exists, so we can remove $\exists q$ and we obtain $(\D(a,p) \text{ is defined}) \land (\ell = \{ w \mid \exists f \in F, \D(a,p) \xrightarrow[\text{\small$\D$}]{w} f\})$. Since  $r \xrightarrow[\text{\small$\D$}]{u} p$, this is equivalent to $(L_{ua} \neq \varnothing) \land (\ell = \{ w \mid \exists f \in F, r \xrightarrow[\text{\small$\D$}]{uaw} f\})$, that is $(L_{ua} \neq \varnothing) \land (\ell = L_{ua})$.
	
	
	The second and third equations are trivially true as their right-hand-side simplifies to $\bullet_{L_{A^*}}$ and $\bullet_{P}$, which are maximal for the inclusion.
\end{proof}

\begin{corollary}\label{appcor:minimization-inf}
	For all deterministic finite automata $(\D,A,P,\{i\},F)$ where every state is accessible from the initial state, using \Cref{fig:inf-simulation-principle}, we have
	\[ \tikzfig{inf-minimization-conclusion}\]
\end{corollary}
We can now provide a proof to \Cref{thm:completeness-inf}.
\begin{proof}
	We start we two diagrams $\rR,\rS$ of $\ZZ$\bfup{-Trans} from $A$ to $B$ and assume $\rinterp{\rR} = \rinterp{\rS}$. We start by focusing on $\rR$. We combine it with the cup $\epsilon_B$ to bend its output into an input, and we then use \Cref{prop:normal-form-inf} to put the result in normal form:
	\[ \tikzfig{inf-compl-epsilon} \qquad = \qquad \tikzfig{inf-compl-normal-form}\]
	Then, we use \Cref{appcor:forward-pruning-inf,appcor:backward-pruning-inf} on $\T$ to prune it, and we distinguish two cases.
	
	\textbf{Case 1}: If $\rinterp{\rR}$ is the empty relation, then the result of the pruning is empty, meaning that:
	\[ \tikzfig{inf-compl-epsilon} \qquad = \qquad \tikzfig{inf-compl-empty} \]
	We can do the same for $\S$, which leads to
	\[\tikzfig{inf-compl-epsilon} \qquad = \qquad \tikzfig{inf-compl-epsilon-bis}   \]
	By combining with the cap $\eta_B$ to bend the input $B$ into an output, and using the the fact that $(\id_B \times \epsilon_B) \circ (\eta_B \times \id_B) = \id_B$, we obtain $\rR = \rS$.
	
	\textbf{Case 2}: If $\rinterp{\rR}$ is non-empty, then the result of the pruning is non-empty. We can continue and use \Cref{appcor:determinization-inf} on $\T$ followed by \Cref{appcor:minimization-inf} on $\P(\T)$ to obtain:
	\[ \tikzfig{inf-compl-epsilon} \qquad = \qquad \tikzfig{inf-compl-minimal}\]
	For convenience, we name $(\R_M,B \times A,Q_M)$ the resulting minimal representation. We do the same for $\S$ and write $(\S_N,B \times A,Q_N)$ for the resulting minimal representation. Since $\rinterp{\rR} = \rinterp{\rS}$ and by soundness of the equations, we obtain that:
	\[ \rinterp{\tikzfig{inf-compl-minimal-M}} \qquad = \qquad \rinterp{\tikzfig{inf-compl-minimal-N}} \]
	Using the definition of $\rinterp{-}$ and \Cref{prop:zeta-transducer-red}, it follows that $(\R_M,B \times A,Q_M)$ and $(\S_N,B \times A,Q_N)$ represent the same subshift, hence by uniqueness of the minimal (pruned right-resolving rooted) representation of \Cref{appthm:minimal-uniqueness-inf}, we obtain that both representations are equal up to an isomorphism $\iota : Q_M \to Q_N$, hence using the simulation principle with this $\iota$, we obtain
	\[  \tikzfig{inf-compl-minimal-M} \qquad =\qquad \tikzfig{inf-compl-minimal-N} \]
	Which leads to
	\[\tikzfig{inf-compl-epsilon} \qquad = \qquad \tikzfig{inf-compl-epsilon-bis}   \]
	By combining with the cap $\eta_B$ to bend the input $B$ into an output, and using the the fact that $(\id_B \times \epsilon_B) \circ (\eta_B \times \id_B) = \id_B$, we obtain $\rR = \rS$.
\end{proof}
