% !TeX root = article.tex
% !TeX spellcheck = en_US

We will now consider the case of transducers acting on bi-infinite words, i.e. words that are infinite both ways. 
We call those transducers $\ZZ$-transducers, and note that within them the notion of initial or final state becomes irrelevant, although one could argue that in fact all the states of those transducers are both initial and final.


\subsection{$\ZZ$-Transducers}
Similarly to the $\_^*$ lift, for $A$ finite, we consider elements of $A^{\ZZ}$ as infinite words $\dots w_{-1}w_0w_1 \dots$, and we define the lifted relation $\R^{\ZZ} : A^{\ZZ} \to B^{\ZZ}$ of $\R : A \to B$ as $\{(w,v)\mid \forall i \in \ZZ, (w_i,v_i) \in \R\}$. 
Similarly to \bfup{UniRel} with $\_^*$, we write $\ZZ$-\bfup{Rel} for the compact closed strict symmetric monoidal category of relations over sets of the form $A^\ZZ$, and reuse the diagrammatic representations of \bfup{FinRel} in this setting. Similarly to $\_^*$, we can apply $\_^{\ZZ}$ to individual wires as below, and this lifting operation is a faithful strong symmetric monoidal functor from \bfup{FinRel} to $\ZZ$-\bfup{Rel}.
\[ \tikzfig{rel-lifted-z}\]

We can also define $\ZZ$-transducers -- a.k.a transducers on bi-infinite words. They are simply tuples $(\T,A,B,Q)$ similar to transducers but without a specified sets of initial or final states. When $B=\one$, such transducers corresponds to transition systems with finite set of states. When $A=\one$, they can also be interpreted as generalized symbolic dynamic systems. Indeed the set of elements of $Q$ in relation with a symbol $b\in B$ form an overlapping partition of $Q$, and the transition function can be interpreted as the action of a non-deterministic dynamic.

We reuse the same notations as for usual transducers, and note that a \textbf{run} within an $\ZZ$-transducer is now a bi-infinite sequence $\dots \xrightarrow[b_{-1}]{a_{-1}} q_{-1} \xrightarrow[b_0]{a_0} q_0 \xrightarrow[b_1]{a_1} q_1 \xrightarrow[b_2]{a_2} \dots$. 
When such a run exists, we say that the transducer can transform the bi-infinite word $\dots a_{-1}a_0a_1\dots$ into $\dots b_{-1}b_0\dots b_1\dots$.
The \emph{behavior} of the transducer is the relation from $A^\ZZ$ to $B^\ZZ$ which relates $w$ to $v$ whenever the transducer can transform $w$ into $v$. We write $\L^\ZZ(\T,A,B,Q)$ for its behavior. 
We define the \textbf{infinite shift} $\shift{}{}_A : A^\ZZ \to A^\ZZ$ as the relation $\{ (w,v) \mid \forall k \in \ZZ, w_k = v_{k+1} \}$. 
Said otherwise, $\shift{}{}_A((a_n)_{n \in \ZZ}) = (a_{n-1})_{n \in \ZZ}$. 
We often omit the alphabet $A$ and only write $\shift{}{}$ when the alphabet is clear from context and we represent it and its transposed:
$\tikzfig{rel-shift-infinite}$. As in the finite case, we can obtain the behavior of a transducer from the shift.

\begin{proposition}\label{prop:zeta-transducer-red}
	For every $\ZZ$-transducer $(\T,A,B,Q)$, its behavior can be obtained using the lift of $(\T,A,B,Q)$ and the shift as follows:
	\[ \tikzfig{rel-trans-inf-red} \]
\end{proposition}

In \Cref{lem:zeta-transducer-from-transducer} we will link the behavior of a $\ZZ$-transducer $(\T,A,B,Q)$ to the behavior of the transducer $(\T,A,B,Q,Q,Q)$, that is, where all the states are initial and final.

\subsection{Sofic Relations}
\label{sec:subshifts}

We define the analog of regular languages and relations in the bi-infinite case.
We start with \emph{sofic subshifts} which have been extensively studied in symbolic dynamics.

\subsubsection{Sofic Subshifts, Factor-Closure and Limits}
\label{subsec:sofic-subshfits}

For $A$ finite, we can define a distance on $A^\ZZ$ as 
$\mathrm{d}(x,y) = 2^{-\inf\{i \in  \NN, x_i\neq y_i \vee x_{-i} \neq y_{-i}\}}$. The topology induced by this distance is called the prodiscrete topology (or Cantor topology), for which $A^\ZZ$ is compact and closed. Subsets of $A^\ZZ$ that are shift-invariant (meaning $\shift{}{}_A(X) = X$) and closed for the prodiscrete topology are called \textbf{subshifts}.

\begin{definition}[Factor] Given two potentially bi-infinite words $x,y \in A^* \cup A^\ZZ$, we say that $x$ is a factor of $y$ and write $x \sqsubseteq y$ whenever there exists $u$,$v$ such that $y = uxv$. Additionally, we say that is $x$ is an internal factor of depth $n$, and write $x \internal{n} y$, whenever both $u$ and $v$ are of length at least $n$ (including infinity).	
	For $x\in A^{\mathbb{Z}}$, we then define $\fact{x}:= \{ u \in A^* | u \sqsubseteq x\}$. Similarly, for $X$ a subshift or a language, we define  $\fact{X}:= \{ u \in A^* | \exists x \in X, u\sqsubseteq x\}$. A language $L$ is said \textbf{factor-closed} if $\fact{L} = L$.
\end{definition}

A subshift $X \subseteq A^\ZZ$ is said to be \textbf{sofic}\footnote{There are several equivalent definitions of sofic subshifts, we briefly discuss this in Appendix \ref{app:bijection-subshifts}} if $\fact{X}$ is regular. 
More than that, a sofic subshift can actually be generated by its factors with a well-chosen notion of limit.

For the remaining of this section, whenever we consider words of $A^*$, we consider them to be indexed as if they were words of $A^\ZZ$.
By default, $u \in A^*$ is indexed from $0$ to $|u|-1$ but we write $\shift{k}{}(u)$ for its $k$-shifted version indexed from $-k$ to $-k+|u|-1$. %% , or equivalently  $\shift{k}{}(u)_i = u_{i+k}$.
Whenever $i$ is not a valid index for $u$, we write $u_i = \bot$.
This allows us to extend\footnote{The formula is unchanged, that is $\mathrm{d}(x,y) = 2^{-\inf\{i \in  \NN, x_i\neq y_i \vee x_{-i} \neq y_{-i}\}}$; see Appendix \ref{app:bijection-subshifts}.} the distance $\mathrm{d}(u,x)$ to a finite word $u$ and an infinite word $x$.

\begin{definition}[Limit Language]
	Given a language $L\subseteq A^*$, we define its limit language $\limlang{L}$ as the set of bi-infinite words that can be reached as a limit of words in $L$, that is :
	\[ \limlang{L} := \{ x \in A^{\mathbb{Z}}, \exists (u_n)_{n\in \mathbb{N}} \in L^{\mathbb{N}}, (k_n)_{n\in \mathbb{N}} \in \mathbb{N}^{\mathbb{N}}\ s.t.\ \lim\limits_{n\to \infty} \shift{k_n}{}(u_n) = x\}\] 
\end{definition}

\begin{lemma}\label{lem:limlang-is-subshift}
	Given a language $L\subseteq A^*$, its limit language $\limlang{L}$ is a subshift. %% , that is, it is closed and shift-invariant.
\end{lemma}

It is clear that different languages can have the same limit, typically a language and its factor closure have the same limit,
however there is a canonical language for a given sofic subshift.
The idea is to remove all the words that are not relevant for the limit behavior.

\begin{definition}[Pruned Language]
Given a language $L\subseteq A^*$, we define its pruned sub-language $\prun{L}$ as the set of words in $L$ that have arbitrarily long (simultaneous) left and right extensions that is : $\prun{L} := \{ u \in L, \forall n \in \mathbb{N}, \exists v \in L, u \internal{n} v\}$.
A language $L$ is said to be \textbf{pruned} when $L=\prun{L}$, or equivalently whenever for every $w \in L$ there exists two non-empty words $u,v$ such that $uwv \in L$.
\end{definition}

\begin{proposition}\label{prop:bijection-subshifts}
	$\fact{-} : \P(A^\ZZ) \to \P(A^*)$ can be restricted to a bijection between sofic subshifts and pruned factor-closed regular languages, and has $\limlang{-} : \P(A^*) \to \P(A^\ZZ)$ for inverse when restricted similarly.
\end{proposition}

This result allows to reduce questions on sofic subshifts to questions on pruned factor-closed regular languages.

\subsubsection{Presentation of a Sofic Subshift}

We have established a bijection between sofic subshifts and factor closed-regular languages, now building on this we introduce a normal form for automata that allows us to reproduce our completeness result to the infinite case.

\begin{definition}[Presentation of a Sofic Subshift]
	A \textbf{presentation} $(\T,A,Q)$ of a sofic subshift $X$ is an automaton $(\T,A,Q,Q,Q)$, meaning  all the  states are initial and final, recognizing a language $L$ and such that $\limlang{L}=X$. A presentation is said 
	\begin{description}
		\item[Pruned] if its language is pruned, meaning that for every word $w$ in the language, there exists $u$ and $v$ both non-empty such that $uwv$ is in the language.
		\item[Right-Resolving] whenever its transition relation is a partial function.
		\item[Rooted] whenever there exists a root state $r$ such that
		\begin{itemize}
			\item every accepted word admits an accepted run starting by $r$,
			\item and every state is accessible from $r$, meaning there is a run from $r$ to that state.
		\end{itemize}
	\end{description}
\end{definition}

\begin{theorem}[Uniqueness]\label{thm:minimal-uniqueness-inf}
	Every non-empty\footnote{If the sofic subshift is empty, its only pruned presentation is the empty automaton, but this automaton is not rooted.} sofic subshift admits a unique minimal rooted right-resolving pruned presentation, up to isomorphism.
\end{theorem}

Notice that this result is not in contradiction with the well known fact that the minimal presentation of a sofic subshift is not unique in general \cite{lind-marcus}. Indeed, it is often possible to find non-rooted presentations that are smaller than our ``minimal'' one, but it is from this rootedness that we obtain uniqueness.

\subsubsection{Sofic Relations and $\ZZ$-Transducers}\label{sec:sofic-z-transducer}

A relation $\R : A^\ZZ \to B^\ZZ$ is said sofic if it is a sofic subshift when seen as a subset of $(A\times B)^\ZZ$, which we know is equivalent to having a representation $(\T,A \times B,Q)$. Using the usual equivalence between automaton on $A \times B$ and transducers from $A$ to $B$, and a relation  $\R : A^\ZZ \to B^\ZZ$ is sofic if and only if there exists a $\ZZ$-transducer $(\T,A,B,Q)$ such that $\limlang{\L(T,A,B,Q,Q,Q)} = \R$. Using the following lemma, $\R$ is sofic if and only if it is the behavior of a $\ZZ$-transducer.

\begin{lemma}\label{lem:zeta-transducer-from-transducer}
	Given a $\ZZ$-transducers $(\T,A,B,Q)$, we consider the transducer  $(\T,A,B,Q,Q,Q)$ where all the states are initial and final.
        We have $\L^\ZZ(\T,A,B,Q) = \limlang{\L(\T,A,B,Q,Q,Q)}$.
\end{lemma}

The composition or product of two sofic relations is sofic, one can simply build the associated $\ZZ$-transducer using the exact same construction as in \Cref{prop:RegRel-is-cat}, so we write \bfup{SofRel} for the subcategory of $\ZZ$-\bfup{Rel} of sofic relations.

\subsection{Diagrams for Subshifts and Completeness}\label{sec:diag-shift}

\begin{wrapfigure}{r}{0.35\textwidth}\vspace{-0.2cm}
\tikzfig{graph-fin-to-inf}
\end{wrapfigure}
The graphical language $\ZZ$\bfup{-Trans} is a variant of \bfup{Trans}. 
Its definition and equational theory (without simulation principle) are exactly the same, up to changing \blue{\textbf{thick blue}} into \red{\textbf{thick red}} and applying the substitution on the right 

%\[ \tikzfig{graph-fin-to-inf}\]
We refer to \Cref{app:infinite-language} for an explicit definition. 
Adapting the simulation principle is more complex, so we will treat it explicitly in the core of this paper. 
We write $\rinterp{-}$ for its semantics, going from  $\ZZ$\bfup{-Trans} to $\ZZ$\bfup{-Rel}, which is simply ``removing the color and adding a $\_^\ZZ$ everywhere''. The explicit definition of this semantics is given in \Cref{appfig:inf-interp}. 
As in the finite case, even without a simulation principle, we have a quasi-normal form and our language is universal for sofic relations, see \Cref{app:inf-quasi-normal-form,app:inf-universality} for the exact statement.

Adapting the simulation principle to this infinite case is non-trivial, as we cannot simply remove the preconditions that acted on $I$ and $F$ and expect the equation to be sound. We propose in \Cref{fig:inf-simulation-principle} some relatively complex preconditions. 
\begin{figure*}[!h]
	\[\tikzfig{inf-simulation-principle}\]
	\caption{Simulation Principle for Bi-Infinite Words, and an Equation Deducible from it.}
	\label{fig:inf-simulation-principle}
\end{figure*}

Informally, the second precondition means that any state of  the $\ZZ$-transducer $(\T,A,B,D)$ that is the start of a path of size $\textup{card}(D)$ (hence a loop must exists, hence it is the start of an infinite path) must be in the domain of $\S$. Conversely, the third precondition means that any state of the $\ZZ$-transducer $(\R,A,B,C)$ that is the end of a path of size $\textup{card}(C)$ (hence a loop must exists, hence it is the end of an infinite path) must be in the codomain of $\S$. We note that most of the time this equation is used in the case where $\S$ is a total and surjective relation (that is $\S^{-1}(C) = D$ and $\S(D) = C$), which automatically satisfies the second and third preconditions.
Like in the finite case, we can derive from the simulation principle one of the well-known equations of trace monoidal categories.
The only missing equation to obtain a trace monoidal category is that adding a feedback loop to $\gamma_{A,A}$ yields the identity $\id_A$ (which would require to remove the ``shift'' from our feedback).
We can now state the completeness theorem.

\begin{theorem}[Completeness]\label{thm:completeness-inf} For $\rR$ and $\rT$ two diagrams of $\ZZ$\bfup{-Trans} from $A$ to $B$. If $\rinterp{\rR} = \rinterp{\rT}$, we can rewrite $\rR$ into $\rT$ by using only the rules of $\ZZ$\bfup{-Trans} and \Cref{fig:inf-simulation-principle}.
\end{theorem}

While the proof still relies on the uniqueness of minimal deterministic automatons, two differences are notable: (1) Our ``automaton'' is now a presentation of a sofic subshift, meaning that we rely on \Cref{thm:minimal-uniqueness-inf} instead of the usual theorem about uniqueness of the minimal automaton. (2) Before the determinization step, we need to prune all the states of the presentation that are not part of any bi-infinite path.
