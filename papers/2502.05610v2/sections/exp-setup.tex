

\section{Experimental setup}

In this section, we describe the models, datasets and various settings we use for our experiments.
% , We describe the tools we use to measure the energy consumed during inference stage, followed by the models, tasks and datasets studied in this work. 

%%%%%%%%%%%%%%%%%%%%%%%%%%%%%%%%%%%%%%%%%%%%%%%%%%%%%%%%%%%%%
%%%%%%%%%%%%%%%%%%%%%%%%%%%%%%%%%%%%%%%%%%%%%%%%%%%%%%%%%%%%%

\subsection{Models}
\label{sub:models}

% Based on architecture, LLM models can be broadly categorized into two categories, encoder-decoder models, and decoder-only models. 
We select $6$ popular and recent GPT-style models from the decoder-only family and $4$ Flan-T5 models from the encoder-decoder family, adding to $10$ models in total 
(details in Appendix~\ref{app:models}).


\noindent \textbf{Decoder-only Models} generate output in an autoregressive manner by predicting the next token in the sequence based on the context (key-value-query) vectors corresponding to the input and previously generated tokens. 
We consider the following models from decoder family in our study. 
(D1)~\textbf{Tiny-LLama}~(1.1B params);
(D2)~\textbf{Phi-3-mini}~(3.8B params);
(D3)~\textbf{Mistral-7B}~(7.2B params);
(D4)~\textbf{Llama-2-7B}~(6.7B params);
(D5)~\textbf{Llama-3-8B}~(8.0B params);
(D6)~\textbf{Llama-2-13B}~(13B params); 


\noindent \textbf{Encoder-Decoder models} process the input data and convert it into context (key-value) vectors. Then the decoder takes these vectors and generates output autoregressively. Models from this family, considered in our study, include: 
%
(ED1)~\textbf{Flan-T5-base}~(248M params); 
(ED2)~\textbf{Flan-T5-large}~(783M params);
(ED3)~\textbf{Flan-T5-xl}~(2.8B params);
(ED4)~\textbf{Flan-T5-xxl}~(11B params);




%%%%%%%%%%%%%%%%%%%%%%%%%%%%%%%%%%%%%%%%%%%%%%%%%%%%%%%%%%%%%
%%%%%%%%%%%%%%%%%%%%%%%%%%%%%%%%%%%%%%%%%%%%%%%%%%%%%%%%%%%%%

\subsection{Tasks and Datasets} 
In this work, we select a diverse range of NLP tasks, from generative to question-answering, classification, and single-sentence tasks. This includes both general GLUE / SuperGLUE benchmarks, as well as domain specific \vax and \caves (for studying effect of task complexity).
% We detail the tasks in App.~\ref{app:tasks}. 
We describe the tasks and their corresponding datasets in Table~\ref{tab:tasks}. 
%
% $\bullet$ \textbf{Single-sentence NLP tasks (GLUE)}: From GLUE benchmark~\cite{wangglue}, we select three single-sentence tasks, namely sentence entailment (\mnli), sentiment classification (\sst), and linguistic acceptability check (\cola).
%
% $\bullet$ \textbf{Complex logical NLP tasks (SUPER-GLUE)}: From Super-GLUE benchmark~\cite{wang2019superglue}, we select comparatively complex reading tasks, including contextual question answering (\boolq), causal reasoning (\copa), and entity question answering (\record). 
%
% $\bullet$ \textbf{Reading comprehension and question answering}: We select SQUAD~v2~\cite{rajpurkar-etal-2016-squad} for complex question answering task. 
%
% $\bullet$ \textbf{Generation}: The generation tasks include news article summarization (\cnndm{}~\cite{nallapati2016abstractive}) and dialogue summarization (\samsum{}~\cite{gliwa2019samsum}).
%
% $\bullet$ \textbf{Domain-specific Classification}: Finally, we include two tweet classification datasets from social domain, namely \vax{}~\cite{poddar2022winds} (3-class stance classification) and \caves{}~\cite{poddar2022caves} (12 class multi-label classification).
%
For each dataset, we selected $1024$ data samples randomly and performed all experiments on the same set for comparable results.
% \subsection{Performance Metrics}
Performance metrics are chosen depending on the tasks. For summarization tasks, average of ROUGE1, ROUGE2, and ROUGE-L are reported, whereas some form of F1 score are reported for the other tasks. Description/prompts of the datasets and the individual metrics have been given in Appendix~\ref{app:tasks}.



\vspace{1mm}
\noindent \textbf{Normalized Accuracy (NA) Metric:}
Since different tasks use different metrics on different scales, it is difficult to compare the accuracy performance of models across the tasks. To gauge the overall performance of the models across multiple tasks, we introduce the \textit{NA} metric that is obtained as follows. For each dataset, we first perform Z-score normalization across all the models, followed by a sigmoid operation to scale models between $0$ and $1$. We then average the scores for each model across all datasets and multiply by $100$.
Note that this metric depends on the set of models used and will vary if models are added/removed. However, it allows us to quantify how well a model performs compared to others in the set.




\subsection{Pathology}
\label{app:tasks:pathology}
\begin{tcolorbox}[title={\texttt{conch\_extract\_features}}]
Perform feature extraction on an input image using CONCH.

\vspace{.5em}
\textbf{Arguments:}
\begin{itemize}[topsep=0pt,parsep=-1pt,partopsep=0pt]
\item \texttt{input\_image} (\texttt{str}): Path to the input image\\  Example: \texttt{'/mount/input/TUM/TUM-TCGA-ACRLPPQE.tif'}
\end{itemize}

\vspace{.5em}
\textbf{Returns:} \begin{itemize}[topsep=0pt,parsep=-1pt,partopsep=0pt]
\item \texttt{features} (\texttt{list}): The feature vector extracted from the input image, as a list of floats
\end{itemize}
\tcblower
\setlength{\hangindent}{\widthof{\faGithub~}}
\faGithub~\url{https://github.com/mahmoodlab/CONCH}

\vspace{.5em}\setlength{\hangindent}{\widthof{\faFile*[regular]~}}\faFile*[regular]~\bibentry{lu2024conch}


\end{tcolorbox}

\begin{tcolorbox}[title={\texttt{musk\_extract\_features}}]
Perform feature extraction on an input image using the vision part of MUSK.

\vspace{.5em}
\textbf{Arguments:}
\begin{itemize}[topsep=0pt,parsep=-1pt,partopsep=0pt]
\item \texttt{input\_image} (\texttt{str}): Path to the input image\\  Example: \texttt{'/mount/input/TUM/TUM-TCGA-ACRLPPQE.tif'}
\end{itemize}

\vspace{.5em}
\textbf{Returns:} \begin{itemize}[topsep=0pt,parsep=-1pt,partopsep=0pt]
\item \texttt{features} (\texttt{list}): The feature vector extracted from the input image, as a list of floats
\end{itemize}
\tcblower
\setlength{\hangindent}{\widthof{\faGithub~}}
\faGithub~\url{https://github.com/lilab-stanford/MUSK}

\vspace{.5em}\setlength{\hangindent}{\widthof{\faFile*[regular]~}}\faFile*[regular]~\bibentry{xiang2025musk}


\end{tcolorbox}

\begin{tcolorbox}[title={\texttt{pathfinder\_verify\_biomarker}}]
Given WSI probability maps, a hypothesis of a potential biomarker, and clinical data, determine (1) whether the potential biomarker is significant for patient prognosis, and (2) whether the potential biomarker is independent among already known biomarkers.

\vspace{.5em}
\textbf{Arguments:}
\begin{itemize}[topsep=0pt,parsep=-1pt,partopsep=0pt]
\item \texttt{heatmaps} (\texttt{str}): Path to the folder containing the numpy array (\textasciigrave{}*.npy\textasciigrave{}) files, which contains the heatmaps of the trained model (each heatmap is HxWxC where C is the number of classes)\\  Example: \texttt{'/mount/input/TCGA\_CRC'}
\item \texttt{hypothesis} (\texttt{str}): A python file, which contains a function \textasciigrave{}def hypothesis\_score(prob\_map\_path: str) -\textgreater{} float\textasciigrave{} which expresses a mathematical model of a hypothesis of a potential biomarker.  For a particular patient, the function returns a risk score.\\  Example: \texttt{'/mount/input/mus\_fraction\_score.py'}
\item \texttt{clini\_table} (\texttt{str}): Path to the CSV file containing the clinical data\\  Example: \texttt{'/mount/input/TCGA\_CRC\_info.csv'}
\item \texttt{files\_table} (\texttt{str}): Path to the CSV file containing the mapping between patient IDs (in the PATIENT column) and heatmap filenames (in the FILENAME column)\\  Example: \texttt{'/mount/input/TCGA\_CRC\_files.csv'}
\item \texttt{survival\_time\_column} (\texttt{str}): The name of the column in the clinical data that contains the survival time\\  Example: \texttt{'OS.time'}
\item \texttt{event\_column} (\texttt{str}): The name of the column in the clinical data that contains the event (e.g. death, recurrence, etc.)\\  Example: \texttt{'vital\_status'}
\item \texttt{known\_biomarkers} (\texttt{list}): A list of known biomarkers. These are column names in the clinical data.\\  Example: \texttt{['MSI']}
\end{itemize}

\vspace{.5em}
\textbf{Returns:} \begin{itemize}[topsep=0pt,parsep=-1pt,partopsep=0pt]
\item \texttt{p\_value} (\texttt{float}): The p-value of the significance of the potential biomarker
\item \texttt{hazard\_ratio} (\texttt{float}): The hazard ratio for the biomarker
\end{itemize}
\tcblower
\setlength{\hangindent}{\widthof{\faGithub~}}
\faGithub~\url{https://github.com/LiangJunhao-THU/PathFinderCRC}

\vspace{.5em}\setlength{\hangindent}{\widthof{\faFile*[regular]~}}\faFile*[regular]~\bibentry{liang2023pathfinder}


\end{tcolorbox}

\begin{tcolorbox}[title={\texttt{stamp\_extract\_features}}]
Perform feature extraction using CTransPath with STAMP on a set of whole slide images, and save the resulting features to a new folder.

\vspace{.5em}
\textbf{Arguments:}
\begin{itemize}[topsep=0pt,parsep=-1pt,partopsep=0pt]
\item \texttt{output\_dir} (\texttt{str}): Path to the output folder where the features will be saved\\  Example: \texttt{'/mount/output/TCGA-BRCA-features'}
\item \texttt{slide\_dir} (\texttt{str}): Path to the input folder containing the whole slide images\\  Example: \texttt{'/mount/input/TCGA-BRCA-SLIDES'}
\end{itemize}

\vspace{.5em}
\textbf{Returns:} \begin{itemize}[topsep=0pt,parsep=-1pt,partopsep=0pt]
\item \texttt{num\_processed\_slides} (\texttt{int}): The number of slides that were processed
\end{itemize}
\tcblower
\setlength{\hangindent}{\widthof{\faGithub~}}
\faGithub~\url{https://github.com/KatherLab/STAMP}

\vspace{.5em}\setlength{\hangindent}{\widthof{\faFile*[regular]~}}\faFile*[regular]~\bibentry{elnahhas2024stamp}


\end{tcolorbox}

\begin{tcolorbox}[title={\texttt{stamp\_train\_classification\_model}}]
Train a model for biomarker classification. You will be supplied with the path to the folder containing the whole slide images, alongside a path to a CSV file containing the training labels.

\vspace{.5em}
\textbf{Arguments:}
\begin{itemize}[topsep=0pt,parsep=-1pt,partopsep=0pt]
\item \texttt{slide\_dir} (\texttt{str}): Path to the folder containing the whole slide images\\  Example: \texttt{'/mount/input/TCGA-BRCA-SLIDES'}
\item \texttt{clini\_table} (\texttt{str}): Path to the CSV file containing the clinical data\\  Example: \texttt{'/mount/input/TCGA-BRCA-DX\_CLINI.xlsx'}
\item \texttt{slide\_table} (\texttt{str}): Path to the CSV file containing the slide metadata\\  Example: \texttt{'/mount/input/TCGA-BRCA-DX\_SLIDE.csv'}
\item \texttt{target\_column} (\texttt{str}): The name of the column in the clinical data that contains the target labels\\  Example: \texttt{'TP53\_driver'}
\item \texttt{trained\_model\_path} (\texttt{str}): Path to the *.pkl file where the trained model should be saved by this function\\  Example: \texttt{'/mount/output/STAMP-BRCA-TP53-model.pkl'}
\end{itemize}

\vspace{.5em}
\textbf{Returns:} \begin{itemize}[topsep=0pt,parsep=-1pt,partopsep=0pt]
\item \texttt{num\_params} (\texttt{int}): The number of parameters in the trained model
\end{itemize}
\tcblower
\setlength{\hangindent}{\widthof{\faGithub~}}
\faGithub~\url{https://github.com/KatherLab/STAMP}

\vspace{.5em}\setlength{\hangindent}{\widthof{\faFile*[regular]~}}\faFile*[regular]~\bibentry{elnahhas2024stamp}


\end{tcolorbox}

\begin{tcolorbox}[title={\texttt{uni\_extract\_features}}]
Perform feature extraction on an input image using UNI.

\vspace{.5em}
\textbf{Arguments:}
\begin{itemize}[topsep=0pt,parsep=-1pt,partopsep=0pt]
\item \texttt{input\_image} (\texttt{str}): Path to the input image\\  Example: \texttt{'/mount/input/TUM/TUM-TCGA-ACRLPPQE.tif'}
\end{itemize}

\vspace{.5em}
\textbf{Returns:} \begin{itemize}[topsep=0pt,parsep=-1pt,partopsep=0pt]
\item \texttt{features} (\texttt{list}): The feature vector extracted from the input image, as a list of floats
\end{itemize}
\tcblower
\setlength{\hangindent}{\widthof{\faGithub~}}
\faGithub~\url{https://github.com/mahmoodlab/UNI}

\vspace{.5em}\setlength{\hangindent}{\widthof{\faFile*[regular]~}}\faFile*[regular]~\bibentry{chen2024uni}


\end{tcolorbox}

\subsection{Radiology}
\label{app:tasks:radiology}
\begin{tcolorbox}[title={\texttt{medsam\_inference}}]
Use the trained MedSAM model to segment the given abdomen CT scan.

\vspace{.5em}
\textbf{Arguments:}
\begin{itemize}[topsep=0pt,parsep=-1pt,partopsep=0pt]
\item \texttt{image\_file} (\texttt{str}): Path to the abdomen CT scan image.\\  Example: \texttt{'/mount/input/my\_image.jpg'}
\item \texttt{bounding\_box} (\texttt{list}): Bounding box to segment (list of 4 integers).\\  Example: \texttt{[25, 100, 155, 155]}
\item \texttt{segmentation\_file} (\texttt{str}): Path to where the segmentation image should be saved.\\  Example: \texttt{'/mount/output/segmented\_image.png'}
\end{itemize}

\vspace{.5em}
\textbf{Returns:} \textit{empty dict}
\tcblower
\setlength{\hangindent}{\widthof{\faGithub~}}
\faGithub~\url{https://github.com/bowang-lab/MedSAM}

\vspace{.5em}\setlength{\hangindent}{\widthof{\faFile*[regular]~}}\faFile*[regular]~\bibentry{ma2024medsam}


\end{tcolorbox}

\begin{tcolorbox}[title={\texttt{nnunet\_train\_model}}]
Train a nnUNet model from scratch on abdomen CT scans. You will be provided with  the path to the dataset, the nnUNet configuration to use, and the fold number  to train the model on.

\vspace{.5em}
\textbf{Arguments:}
\begin{itemize}[topsep=0pt,parsep=-1pt,partopsep=0pt]
\item \texttt{dataset\_path} (\texttt{str}): The path to the dataset to train the model on (contains dataset.json, imagesTr, imagesTs, labelsTr)\\  Example: \texttt{'/mount/input/Task02\_Heart'}
\item \texttt{unet\_configuration} (\texttt{str}): The configuration of the UNet to use for training. One of '2d', '3d\_fullres', '3d\_lowres', '3d\_cascade\_fullres'\\  Example: \texttt{'3d\_fullres'}
\item \texttt{fold} (\texttt{int}): The fold number to train the model on. One of 0, 1, 2, 3, 4.\\  Example: \texttt{0}
\item \texttt{output\_folder} (\texttt{str}): Path to the folder where the trained model should be saved\\  Example: \texttt{'/mount/output/trained\_model'}
\end{itemize}

\vspace{.5em}
\textbf{Returns:} \textit{empty dict}
\tcblower
\setlength{\hangindent}{\widthof{\faGithub~}}
\faGithub~\url{https://github.com/MIC-DKFZ/nnUNet}

\vspace{.5em}\setlength{\hangindent}{\widthof{\faFile*[regular]~}}\faFile*[regular]~\bibentry{isensee2020nnunet}


\end{tcolorbox}

\subsection{Omics}
\label{app:tasks:genomics_proteomics}
\begin{tcolorbox}[title={\texttt{cytopus\_db}}]
Initialize the Cytopus KnowledgeBase and generate a JSON file containing a nested dictionary with gene set annotations organized by cell type, suitable for input into the Spectra library.

\vspace{.5em}
\textbf{Arguments:}
\begin{itemize}[topsep=0pt,parsep=-1pt,partopsep=0pt]
\item \texttt{celltype\_of\_interest} (\texttt{list}): List of cell types for which to retrieve gene sets\\  Example: \texttt{['B\_memory', 'B\_naive', 'CD4\_T', 'CD8\_T', 'DC', 'ILC3', 'MDC', 'NK', 'Treg', 'gdT', 'mast', 'pDC', 'plasma']}
\item \texttt{global\_celltypes} (\texttt{list}): List of global cell types to include in the JSON file.\\  Example: \texttt{['all-cells', 'leukocyte']}
\item \texttt{output\_file} (\texttt{str}): Path to the file where the output JSON file should be stored.\\  Example: \texttt{'/mount/output/Spectra\_dict.json'}
\end{itemize}

\vspace{.5em}
\textbf{Returns:} \begin{itemize}[topsep=0pt,parsep=-1pt,partopsep=0pt]
\item \texttt{keys} (\texttt{list}): The list of keys in the produced JSON file.
\end{itemize}
\tcblower
\setlength{\hangindent}{\widthof{\faGithub~}}
\faGithub~\url{https://github.com/wallet-maker/cytopus}

\vspace{.5em}\setlength{\hangindent}{\widthof{\faFile*[regular]~}}\faFile*[regular]~\bibentry{kunes2023cytopus}


\end{tcolorbox}

\begin{tcolorbox}[title={\texttt{esm\_fold\_predict}}]
Generate the representation of a protein sequence and the contact map using Facebook Research's pretrained esm2\_t33\_650M\_UR50D model.

\vspace{.5em}
\textbf{Arguments:}
\begin{itemize}[topsep=0pt,parsep=-1pt,partopsep=0pt]
\item \texttt{sequence} (\texttt{str}): Protein sequence to for which to generate representation and contact map.\\  Example: \texttt{'MKTVRQERLKSIVRILERSKEPVSGAQLAEELSVSRQVIVQDIAYLRSLGYNIVATPRGYVLAGG'}
\end{itemize}

\vspace{.5em}
\textbf{Returns:} \begin{itemize}[topsep=0pt,parsep=-1pt,partopsep=0pt]
\item \texttt{sequence\_representation} (\texttt{list}): Token representations for the protein sequence as a list of floats, i.e. a 1D array of shape L where L is the number of tokens.
\item \texttt{contact\_map} (\texttt{list}): Contact map for the protein sequence as a list of list of floats, i.e. a 2D array of shape LxL where L is the number of tokens.
\end{itemize}
\tcblower
\setlength{\hangindent}{\widthof{\faGithub~}}
\faGithub~\url{https://github.com/facebookresearch/esm}

\vspace{.5em}\setlength{\hangindent}{\widthof{\faFile*[regular]~}}\faFile*[regular]~\bibentry{verkuil2022esm1}


\vspace{.5em}\setlength{\hangindent}{\widthof{\faFile*[regular]~}}\faFile*[regular]~\bibentry{hie2022esm2}


\end{tcolorbox}

\subsection{Other}
\label{app:tasks:imaging}
\begin{tcolorbox}[title={\texttt{retfound\_feature\_vector}}]
Extract the feature vector for the given retinal image using the RETFound pretrained vit\_large\_patch16 model.

\vspace{.5em}
\textbf{Arguments:}
\begin{itemize}[topsep=0pt,parsep=-1pt,partopsep=0pt]
\item \texttt{image\_file} (\texttt{str}): Path to the retinal image.\\  Example: \texttt{'/mount/input/retinal\_image.jpg'}
\end{itemize}

\vspace{.5em}
\textbf{Returns:} \begin{itemize}[topsep=0pt,parsep=-1pt,partopsep=0pt]
\item \texttt{feature\_vector} (\texttt{list}): The feature vector for the given retinal image, as a list of floats.
\end{itemize}
\tcblower
\setlength{\hangindent}{\widthof{\faGithub~}}
\faGithub~\url{https://github.com/rmaphoh/RETFound_MAE}

\vspace{.5em}\setlength{\hangindent}{\widthof{\faFile*[regular]~}}\faFile*[regular]~\bibentry{zhou2023retfound}


\end{tcolorbox}

\label{app:tasks:llms}
\begin{tcolorbox}[title={\texttt{medsss\_generate}}]
Given a user message, generate a response using the MedSSS\_Policy model.

\vspace{.5em}
\textbf{Arguments:}
\begin{itemize}[topsep=0pt,parsep=-1pt,partopsep=0pt]
\item \texttt{user\_message} (\texttt{str}): The user message.\\  Example: \texttt{'How to stop a cough?'}
\end{itemize}

\vspace{.5em}
\textbf{Returns:} \begin{itemize}[topsep=0pt,parsep=-1pt,partopsep=0pt]
\item \texttt{response} (\texttt{str}): The response generated by the model.
\end{itemize}
\tcblower
\setlength{\hangindent}{\widthof{\faGithub~}}
\faGithub~\url{https://github.com/pixas/MedSSS}

\vspace{.5em}\setlength{\hangindent}{\widthof{\faFile*[regular]~}}\faFile*[regular]~\bibentry{jiang2025medsss}


\end{tcolorbox}

\begin{tcolorbox}[title={\texttt{modernbert\_predict\_masked}}]
Given a masked sentence string, predict the original sentence using the pretrained ModernBERT-base model on CPU.

\vspace{.5em}
\textbf{Arguments:}
\begin{itemize}[topsep=0pt,parsep=-1pt,partopsep=0pt]
\item \texttt{input\_string} (\texttt{str}): The masked sentence string. The masked part is represented by "[MASK]"".\\  Example: \texttt{'Paris is the [MASK] of France.'}
\end{itemize}

\vspace{.5em}
\textbf{Returns:} \begin{itemize}[topsep=0pt,parsep=-1pt,partopsep=0pt]
\item \texttt{prediction} (\texttt{str}): The predicted original sentence
\end{itemize}
\tcblower
\setlength{\hangindent}{\widthof{\faGithub~}}
\faGithub~\url{https://github.com/AnswerDotAI/ModernBERT}

\vspace{.5em}\setlength{\hangindent}{\widthof{\faFile*[regular]~}}\faFile*[regular]~\bibentry{warner2024modernbert}


\end{tcolorbox}

\label{app:tasks:3d_vision}
\begin{tcolorbox}[title={\texttt{flowmap\_overfit\_scene}}]
Overfit FlowMap on an input scene to determine camera extrinsics for each frame in the scene.

\vspace{.5em}
\textbf{Arguments:}
\begin{itemize}[topsep=0pt,parsep=-1pt,partopsep=0pt]
\item \texttt{input\_scene} (\texttt{str}): Path to the directory containing the images of the input scene (just the image files, nothing else)\\  Example: \texttt{'/mount/input/llff\_flower'}
\end{itemize}

\vspace{.5em}
\textbf{Returns:} \begin{itemize}[topsep=0pt,parsep=-1pt,partopsep=0pt]
\item \texttt{n} (\texttt{int}): The number of images (frames) in the scene
\item \texttt{camera\_extrinsics} (\texttt{list}): The camera extrinsics matrix for each of the n frames in the scene, must have a shape of nx4x4 (as a nested python list of floats)
\end{itemize}
\tcblower
\setlength{\hangindent}{\widthof{\faGithub~}}
\faGithub~\url{https://github.com/dcharatan/flowmap}

\vspace{.5em}\setlength{\hangindent}{\widthof{\faFile*[regular]~}}\faFile*[regular]~\bibentry{smith2024flowmap}


\end{tcolorbox}

\label{app:tasks:tabular}
\begin{tcolorbox}[title={\texttt{tabpfn\_predict}}]
Train a predictor using TabPFN on a tabular dataset. Evaluate the predictor on the test set.

\vspace{.5em}
\textbf{Arguments:}
\begin{itemize}[topsep=0pt,parsep=-1pt,partopsep=0pt]
\item \texttt{train\_csv} (\texttt{str}): Path to the CSV file containing the training data\\  Example: \texttt{'/mount/input/breast\_cancer\_train.csv'}
\item \texttt{test\_csv} (\texttt{str}): Path to the CSV file containing the test data\\  Example: \texttt{'/mount/input/breast\_cancer\_test.csv'}
\item \texttt{feature\_columns} (\texttt{list}): The names of the columns to use as features\\  Example: \texttt{['mean radius', 'mean texture', 'mean perimeter', 'mean area', 'mean smoothness', 'mean compactness', 'mean concavity', 'mean concave points', 'mean symmetry', 'mean fractal dimension', 'radius error', 'texture error', 'perimeter error', 'area error', 'smoothness error', 'compactness error', 'concavity error', 'concave points error', 'symmetry error', 'fractal dimension error', 'worst radius', 'worst texture', 'worst perimeter', 'worst area', 'worst smoothness', 'worst compactness', 'worst concavity', 'worst concave points', 'worst symmetry', 'worst fractal dimension']}
\item \texttt{target\_column} (\texttt{str}): The name of the column to predict\\  Example: \texttt{'target'}
\end{itemize}

\vspace{.5em}
\textbf{Returns:} \begin{itemize}[topsep=0pt,parsep=-1pt,partopsep=0pt]
\item \texttt{roc\_auc} (\texttt{float}): The ROC AUC score of the predictor on the test set
\item \texttt{accuracy} (\texttt{float}): The accuracy of the predictor on the test set
\item \texttt{probs} (\texttt{list}): The probabilities of the predictor on the test set, as a list of floats (one per sample in the test set)
\end{itemize}
\tcblower
\setlength{\hangindent}{\widthof{\faGithub~}}
\faGithub~\url{https://github.com/PriorLabs/TabPFN}

\vspace{.5em}\setlength{\hangindent}{\widthof{\faFile*[regular]~}}\faFile*[regular]~\bibentry{hollmann2025tabpfn}


\end{tcolorbox}







%%%%%%%%%%%%%%%%%%%%%%%%%%%%%%%%%%%%%%%%%%%%%%%%%%%%%%%%%%%%%
%%%%%%%%%%%%%%%%%%%%%%%%%%%%%%%%%%%%%%%%%%%%%%%%%%%%%%%%%%%%%

\subsection{Hardware and Energy metrics}
We perform our experiments on a single NVIDIA A6000 GPU with 48GB VRAM hosted in a local server with Intel Xeon Silver 4210R processor and 128GB RAM, running Ubuntu 20.04-LTS. 
The server also hosted an NVIDIA A5000 GPU (24 GB), which was used for only one experiment, but otherwise was unused.
We also performed some experiments on two other systems to verify the generalizability of our findings, as detailed in Appendix~\ref{app:mgpu}.
We use Pytorch version 2.3 (with CUDA 12.1) and huggingface transformers version 4.41.
% \todo{Mention packages. Specify chosen packages. Explain the reason. }

We use the popular Code Carbon~\cite{schmidt2021codecarbon} and Carbon Tracker~\cite{anthony2020carbontracker} packages to measure the energy consumed in different experiments. 
% We select CodeCarbon and CarbonTracker among many existing energy tracker because 
\citet{jay2023experimental} demonstrated the suitability and accuracy of CarbonTracker, CodeCarbon, Energy Scope, and Experiment Impact Tracker across various software-based power meter setups, while \citet{bouza2023estimate} further established the superiority of CodeCarbon and CarbonTracker among these tools. CodeCarbon is especially the most user-friendly and works out of the box, provided appropriate NVIDIA libraries and permissions to Intel RAPL files.


% Both of these energy tracking libraries provide various important information including duration, CO2 emission rate, total power/energy and that consumed by GPU, CPU, and RAM; etc.  
%
These two packages measure the GPU-power usage using pynvml and CPU-power using Intel RAPL files every $\mathcal{X}$ seconds, and integrates it over time. Carbon-tracker reports sum of these as the total energy.
Code-carbon also adds an estimate of the RAM-power being used depending on the RAM size. 
We use $\mathcal{X} = 10secs$ for a balance between overhead costs and tracking accuracy.
We keep the Power Usage Effectiveness~(PUE) to the default $1.0$ since we run all experiments on the same server, but this implies that the actual energy usage is higher than reported.



During inference, we provide test samples in batches to the LLM, 
and measure the total energy required for 1024 samples per dataset using these tools. This includes both the input tokenization process by each model's tokenizer and the output generation from the model.
We keep the batch size to $8$ for most experiments, except on the \cnndm and \samsum dataset for which we use a batch size of $4$. 
While reporting results, we average the energy usage and report the \textbf{energy per sample} in mWh (milli-Watt-hour). Unless otherwise stated, these are the default settings used for experiments. 



% In our study, we omit the amount of carbon emission because we perform all the experiments in a single region where the carbon intensity is fixed and therefore, energy consumed is closely related with the amount of $CO_2$ emission. Furthermore, the CO$_2$ emission strongly varies depending on the region and the type of electricity source. Thus, we prefer to report the total energy consumed instead of the amount of CO$_2$ emission. 

