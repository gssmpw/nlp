

\section{Experimental setup}

In this section, we describe the models, datasets and various settings we use for our experiments.
% , We describe the tools we use to measure the energy consumed during inference stage, followed by the models, tasks and datasets studied in this work. 

%%%%%%%%%%%%%%%%%%%%%%%%%%%%%%%%%%%%%%%%%%%%%%%%%%%%%%%%%%%%%
%%%%%%%%%%%%%%%%%%%%%%%%%%%%%%%%%%%%%%%%%%%%%%%%%%%%%%%%%%%%%

\subsection{Models}
\label{sub:models}

% Based on architecture, LLM models can be broadly categorized into two categories, encoder-decoder models, and decoder-only models. 
We select $6$ popular and recent GPT-style models from the decoder-only family and $4$ Flan-T5 models from the encoder-decoder family, adding to $10$ models in total 
(details in Appendix~\ref{app:models}).


\noindent \textbf{Decoder-only Models} generate output in an autoregressive manner by predicting the next token in the sequence based on the context (key-value-query) vectors corresponding to the input and previously generated tokens. 
We consider the following models from decoder family in our study. 
(D1)~\textbf{Tiny-LLama}~(1.1B params);
(D2)~\textbf{Phi-3-mini}~(3.8B params);
(D3)~\textbf{Mistral-7B}~(7.2B params);
(D4)~\textbf{Llama-2-7B}~(6.7B params);
(D5)~\textbf{Llama-3-8B}~(8.0B params);
(D6)~\textbf{Llama-2-13B}~(13B params); 


\noindent \textbf{Encoder-Decoder models} process the input data and convert it into context (key-value) vectors. Then the decoder takes these vectors and generates output autoregressively. Models from this family, considered in our study, include: 
%
(ED1)~\textbf{Flan-T5-base}~(248M params); 
(ED2)~\textbf{Flan-T5-large}~(783M params);
(ED3)~\textbf{Flan-T5-xl}~(2.8B params);
(ED4)~\textbf{Flan-T5-xxl}~(11B params);




%%%%%%%%%%%%%%%%%%%%%%%%%%%%%%%%%%%%%%%%%%%%%%%%%%%%%%%%%%%%%
%%%%%%%%%%%%%%%%%%%%%%%%%%%%%%%%%%%%%%%%%%%%%%%%%%%%%%%%%%%%%

\subsection{Tasks and Datasets} 
In this work, we select a diverse range of NLP tasks, from generative to question-answering, classification, and single-sentence tasks. This includes both general GLUE / SuperGLUE benchmarks, as well as domain specific \vax and \caves (for studying effect of task complexity).
% We detail the tasks in App.~\ref{app:tasks}. 
We describe the tasks and their corresponding datasets in Table~\ref{tab:tasks}. 
%
% $\bullet$ \textbf{Single-sentence NLP tasks (GLUE)}: From GLUE benchmark~\cite{wangglue}, we select three single-sentence tasks, namely sentence entailment (\mnli), sentiment classification (\sst), and linguistic acceptability check (\cola).
%
% $\bullet$ \textbf{Complex logical NLP tasks (SUPER-GLUE)}: From Super-GLUE benchmark~\cite{wang2019superglue}, we select comparatively complex reading tasks, including contextual question answering (\boolq), causal reasoning (\copa), and entity question answering (\record). 
%
% $\bullet$ \textbf{Reading comprehension and question answering}: We select SQUAD~v2~\cite{rajpurkar-etal-2016-squad} for complex question answering task. 
%
% $\bullet$ \textbf{Generation}: The generation tasks include news article summarization (\cnndm{}~\cite{nallapati2016abstractive}) and dialogue summarization (\samsum{}~\cite{gliwa2019samsum}).
%
% $\bullet$ \textbf{Domain-specific Classification}: Finally, we include two tweet classification datasets from social domain, namely \vax{}~\cite{poddar2022winds} (3-class stance classification) and \caves{}~\cite{poddar2022caves} (12 class multi-label classification).
%
For each dataset, we selected $1024$ data samples randomly and performed all experiments on the same set for comparable results.
% \subsection{Performance Metrics}
Performance metrics are chosen depending on the tasks. For summarization tasks, average of ROUGE1, ROUGE2, and ROUGE-L are reported, whereas some form of F1 score are reported for the other tasks. Description/prompts of the datasets and the individual metrics have been given in Appendix~\ref{app:tasks}.



\vspace{1mm}
\noindent \textbf{Normalized Accuracy (NA) Metric:}
Since different tasks use different metrics on different scales, it is difficult to compare the accuracy performance of models across the tasks. To gauge the overall performance of the models across multiple tasks, we introduce the \textit{NA} metric that is obtained as follows. For each dataset, we first perform Z-score normalization across all the models, followed by a sigmoid operation to scale models between $0$ and $1$. We then average the scores for each model across all datasets and multiply by $100$.
Note that this metric depends on the set of models used and will vary if models are added/removed. However, it allows us to quantify how well a model performs compared to others in the set.




\section{Multi-Stage Robotic Manipulation Planning Tasks}\label{sec:tasks}
%%

\begin{figure*}[t!]
    \centering
    \includegraphics[width=0.99\textwidth]{figs/filmstrip.pdf}
    \caption{\textbf{Filmstrip of our method solving a complicated assembly task.} Frames are indexed by timestep. The goal image is in the top-left corner (with a green border). Each frame is the observation after executing the action (in black) above it. The other action in gray is the original action proposed by the VLM if it is revised after reflection. We highlight the reflection process at timestep 15, where the VLM first proposes an action to pick up the purple brick, but after reflection, it chooses to pick up the yellow brick instead as the generated future state (red-bordered image) shows little progress towards the goal.}
    \label{fig:filmstrip}
\end{figure*}
Inspired by~\citet{luo2024fmb}, we procedurally generated a suite of multi-stage long-horizon manipulation tasks that require understanding of physical interactions and reasoning about the effects of long-term action sequences. The task is initialized with a board and a set of small pieces randomly placed on a table. The goal is to fully assemble the board by inserting the pieces into the board one by one. Examples of the initial and goal configurations are shown in Fig.~\ref{fig:tasks}. Detailed task generation process is included in App.~\ref{sec:app_task_gen}. Notably, most tasks include inter-locking pieces so that they can be inserted into the board only in a specific order. This requires strategically choosing the object to be manipulated at each step and inferring possible interaction between this object and the other objects already in the board. 
As an example, Fig.~\ref{fig:tasks}(b) shows the dependencies between the pieces in one of the tasks. 
The interlocking feature further necessitates the agent’s ability to replan, enabling it to recover from failures caused by previous mistakes or bad initialization. 


\begin{figure}[h!]
    \centering
    \includegraphics[width=0.49\textwidth]{figs/tasks_single_column.pdf}
    \vspace{-0.1in}
    \caption{\textbf{Task examples.} (a) Generated multi-stage manipulation tasks with interlocking pieces. Top: initial configurations. Bottom: goal configurations. See App.~\ref{sec:app_more_task_samples} for more examples. (b) The graph shows the dependencies between the objects in the blue assembly board on the left. Each node represents an object, and each directed edge indicates the predecessor object should be assembled before the successor object.}
    \label{fig:tasks}
\end{figure}

% \begin{figure}[h!]
%     \centering
%     \includegraphics[width=0.95\linewidth]{figs/dependencies.pdf}
%     \caption{\textbf{A dependency graph of interlocking objects.} The right graph shows the dependencies between the objects in the assembly task on the left. Each node represents an object, and each directed edge indicates the predecessor object should be assembled before the successor object.}
%     \label{fig:dependencies}
% \end{figure}

We focus on the high-level planning of this long-horizon manipulation task. We define a set of actions in the form of ``{\tt [act] [obj]}", where $\text{\tt [act]}\in \{\text{\tt pick up}, \text{\tt insert}, \text{\tt reorient}, \text{\tt put down}\}$ is an action primitive, and {\tt [obj]} denotes the object to be manipulated. Specifically, ``{\tt pick up}" grasps a piece that is not in hand and picks it up. It can then be inserted into the board using the ``{\tt insert}" action, or put back on the table using ``{\tt put down}". By invoking ``{\tt reorient}", the object in hand can be reoriented with the black fixture if necessary, so that it is in a suitable pose for insertion. Each action primitive is implemented as a rule-based script controller; however, integrating other low-level controllers, such as learning-based policies like behavior cloning, is also possible. We also designed an expert policy for the mentioned motor primitives, see App.~\ref{sec:app_expert} for implementation details.







%%%%%%%%%%%%%%%%%%%%%%%%%%%%%%%%%%%%%%%%%%%%%%%%%%%%%%%%%%%%%
%%%%%%%%%%%%%%%%%%%%%%%%%%%%%%%%%%%%%%%%%%%%%%%%%%%%%%%%%%%%%

\subsection{Hardware and Energy metrics}
We perform our experiments on a single NVIDIA A6000 GPU with 48GB VRAM hosted in a local server with Intel Xeon Silver 4210R processor and 128GB RAM, running Ubuntu 20.04-LTS. 
The server also hosted an NVIDIA A5000 GPU (24 GB), which was used for only one experiment, but otherwise was unused.
We also performed some experiments on two other systems to verify the generalizability of our findings, as detailed in Appendix~\ref{app:mgpu}.
We use Pytorch version 2.3 (with CUDA 12.1) and huggingface transformers version 4.41.
% \todo{Mention packages. Specify chosen packages. Explain the reason. }

We use the popular Code Carbon~\cite{schmidt2021codecarbon} and Carbon Tracker~\cite{anthony2020carbontracker} packages to measure the energy consumed in different experiments. 
% We select CodeCarbon and CarbonTracker among many existing energy tracker because 
\citet{jay2023experimental} demonstrated the suitability and accuracy of CarbonTracker, CodeCarbon, Energy Scope, and Experiment Impact Tracker across various software-based power meter setups, while \citet{bouza2023estimate} further established the superiority of CodeCarbon and CarbonTracker among these tools. CodeCarbon is especially the most user-friendly and works out of the box, provided appropriate NVIDIA libraries and permissions to Intel RAPL files.


% Both of these energy tracking libraries provide various important information including duration, CO2 emission rate, total power/energy and that consumed by GPU, CPU, and RAM; etc.  
%
These two packages measure the GPU-power usage using pynvml and CPU-power using Intel RAPL files every $\mathcal{X}$ seconds, and integrates it over time. Carbon-tracker reports sum of these as the total energy.
Code-carbon also adds an estimate of the RAM-power being used depending on the RAM size. 
We use $\mathcal{X} = 10secs$ for a balance between overhead costs and tracking accuracy.
We keep the Power Usage Effectiveness~(PUE) to the default $1.0$ since we run all experiments on the same server, but this implies that the actual energy usage is higher than reported.



During inference, we provide test samples in batches to the LLM, 
and measure the total energy required for 1024 samples per dataset using these tools. This includes both the input tokenization process by each model's tokenizer and the output generation from the model.
We keep the batch size to $8$ for most experiments, except on the \cnndm and \samsum dataset for which we use a batch size of $4$. 
While reporting results, we average the energy usage and report the \textbf{energy per sample} in mWh (milli-Watt-hour). Unless otherwise stated, these are the default settings used for experiments. 



% In our study, we omit the amount of carbon emission because we perform all the experiments in a single region where the carbon intensity is fixed and therefore, energy consumed is closely related with the amount of $CO_2$ emission. Furthermore, the CO$_2$ emission strongly varies depending on the region and the type of electricity source. Thus, we prefer to report the total energy consumed instead of the amount of CO$_2$ emission. 

