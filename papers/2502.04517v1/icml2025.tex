\PassOptionsToPackage{dvipsnames}{xcolor}

\documentclass{article}

% Recommended, but optional, packages for figures and better typesetting:
\usepackage{microtype}
\usepackage{graphicx}
\usepackage{subfigure}
\usepackage{booktabs} % for professional tables
\usepackage{siunitx}
\usepackage{algorithmic}
% \usepackage{algpseudocode}
\usepackage{wrapfig}
\usepackage{enumerate}
\usepackage{etoolbox}
\usepackage{multirow}
\usepackage{tcolorbox}

% \usepackage{algorithmic}  
% \usepackage[algo2e, ruled]{algorithm2e} 

% hyperref makes hyperlinks in the resulting PDF.
% If your build breaks (sometimes temporarily if a hyperlink spans a page)
% please comment out the following usepackage line and replace
% \usepackage{icml2025} with \usepackage[nohyperref]{icml2025} above.
\usepackage{hyperref}
%%%%% NEW MATH DEFINITIONS %%%%%

\usepackage{amsmath,amsfonts,bm}
\usepackage{derivative}
% Mark sections of captions for referring to divisions of figures
\newcommand{\figleft}{{\em (Left)}}
\newcommand{\figcenter}{{\em (Center)}}
\newcommand{\figright}{{\em (Right)}}
\newcommand{\figtop}{{\em (Top)}}
\newcommand{\figbottom}{{\em (Bottom)}}
\newcommand{\captiona}{{\em (a)}}
\newcommand{\captionb}{{\em (b)}}
\newcommand{\captionc}{{\em (c)}}
\newcommand{\captiond}{{\em (d)}}

% Highlight a newly defined term
\newcommand{\newterm}[1]{{\bf #1}}

% Derivative d 
\newcommand{\deriv}{{\mathrm{d}}}

% Figure reference, lower-case.
\def\figref#1{figure~\ref{#1}}
% Figure reference, capital. For start of sentence
\def\Figref#1{Figure~\ref{#1}}
\def\twofigref#1#2{figures \ref{#1} and \ref{#2}}
\def\quadfigref#1#2#3#4{figures \ref{#1}, \ref{#2}, \ref{#3} and \ref{#4}}
% Section reference, lower-case.
\def\secref#1{section~\ref{#1}}
% Section reference, capital.
\def\Secref#1{Section~\ref{#1}}
% Reference to two sections.
\def\twosecrefs#1#2{sections \ref{#1} and \ref{#2}}
% Reference to three sections.
\def\secrefs#1#2#3{sections \ref{#1}, \ref{#2} and \ref{#3}}
% Reference to an equation, lower-case.
\def\eqref#1{equation~\ref{#1}}
% Reference to an equation, upper case
\def\Eqref#1{Equation~\ref{#1}}
% A raw reference to an equation---avoid using if possible
\def\plaineqref#1{\ref{#1}}
% Reference to a chapter, lower-case.
\def\chapref#1{chapter~\ref{#1}}
% Reference to an equation, upper case.
\def\Chapref#1{Chapter~\ref{#1}}
% Reference to a range of chapters
\def\rangechapref#1#2{chapters\ref{#1}--\ref{#2}}
% Reference to an algorithm, lower-case.
\def\algref#1{algorithm~\ref{#1}}
% Reference to an algorithm, upper case.
\def\Algref#1{Algorithm~\ref{#1}}
\def\twoalgref#1#2{algorithms \ref{#1} and \ref{#2}}
\def\Twoalgref#1#2{Algorithms \ref{#1} and \ref{#2}}
% Reference to a part, lower case
\def\partref#1{part~\ref{#1}}
% Reference to a part, upper case
\def\Partref#1{Part~\ref{#1}}
\def\twopartref#1#2{parts \ref{#1} and \ref{#2}}

\def\ceil#1{\lceil #1 \rceil}
\def\floor#1{\lfloor #1 \rfloor}
\def\1{\bm{1}}
\newcommand{\train}{\mathcal{D}}
\newcommand{\valid}{\mathcal{D_{\mathrm{valid}}}}
\newcommand{\test}{\mathcal{D_{\mathrm{test}}}}

\def\eps{{\epsilon}}


% Random variables
\def\reta{{\textnormal{$\eta$}}}
\def\ra{{\textnormal{a}}}
\def\rb{{\textnormal{b}}}
\def\rc{{\textnormal{c}}}
\def\rd{{\textnormal{d}}}
\def\re{{\textnormal{e}}}
\def\rf{{\textnormal{f}}}
\def\rg{{\textnormal{g}}}
\def\rh{{\textnormal{h}}}
\def\ri{{\textnormal{i}}}
\def\rj{{\textnormal{j}}}
\def\rk{{\textnormal{k}}}
\def\rl{{\textnormal{l}}}
% rm is already a command, just don't name any random variables m
\def\rn{{\textnormal{n}}}
\def\ro{{\textnormal{o}}}
\def\rp{{\textnormal{p}}}
\def\rq{{\textnormal{q}}}
\def\rr{{\textnormal{r}}}
\def\rs{{\textnormal{s}}}
\def\rt{{\textnormal{t}}}
\def\ru{{\textnormal{u}}}
\def\rv{{\textnormal{v}}}
\def\rw{{\textnormal{w}}}
\def\rx{{\textnormal{x}}}
\def\ry{{\textnormal{y}}}
\def\rz{{\textnormal{z}}}

% Random vectors
\def\rvepsilon{{\mathbf{\epsilon}}}
\def\rvphi{{\mathbf{\phi}}}
\def\rvtheta{{\mathbf{\theta}}}
\def\rva{{\mathbf{a}}}
\def\rvb{{\mathbf{b}}}
\def\rvc{{\mathbf{c}}}
\def\rvd{{\mathbf{d}}}
\def\rve{{\mathbf{e}}}
\def\rvf{{\mathbf{f}}}
\def\rvg{{\mathbf{g}}}
\def\rvh{{\mathbf{h}}}
\def\rvu{{\mathbf{i}}}
\def\rvj{{\mathbf{j}}}
\def\rvk{{\mathbf{k}}}
\def\rvl{{\mathbf{l}}}
\def\rvm{{\mathbf{m}}}
\def\rvn{{\mathbf{n}}}
\def\rvo{{\mathbf{o}}}
\def\rvp{{\mathbf{p}}}
\def\rvq{{\mathbf{q}}}
\def\rvr{{\mathbf{r}}}
\def\rvs{{\mathbf{s}}}
\def\rvt{{\mathbf{t}}}
\def\rvu{{\mathbf{u}}}
\def\rvv{{\mathbf{v}}}
\def\rvw{{\mathbf{w}}}
\def\rvx{{\mathbf{x}}}
\def\rvy{{\mathbf{y}}}
\def\rvz{{\mathbf{z}}}

% Elements of random vectors
\def\erva{{\textnormal{a}}}
\def\ervb{{\textnormal{b}}}
\def\ervc{{\textnormal{c}}}
\def\ervd{{\textnormal{d}}}
\def\erve{{\textnormal{e}}}
\def\ervf{{\textnormal{f}}}
\def\ervg{{\textnormal{g}}}
\def\ervh{{\textnormal{h}}}
\def\ervi{{\textnormal{i}}}
\def\ervj{{\textnormal{j}}}
\def\ervk{{\textnormal{k}}}
\def\ervl{{\textnormal{l}}}
\def\ervm{{\textnormal{m}}}
\def\ervn{{\textnormal{n}}}
\def\ervo{{\textnormal{o}}}
\def\ervp{{\textnormal{p}}}
\def\ervq{{\textnormal{q}}}
\def\ervr{{\textnormal{r}}}
\def\ervs{{\textnormal{s}}}
\def\ervt{{\textnormal{t}}}
\def\ervu{{\textnormal{u}}}
\def\ervv{{\textnormal{v}}}
\def\ervw{{\textnormal{w}}}
\def\ervx{{\textnormal{x}}}
\def\ervy{{\textnormal{y}}}
\def\ervz{{\textnormal{z}}}

% Random matrices
\def\rmA{{\mathbf{A}}}
\def\rmB{{\mathbf{B}}}
\def\rmC{{\mathbf{C}}}
\def\rmD{{\mathbf{D}}}
\def\rmE{{\mathbf{E}}}
\def\rmF{{\mathbf{F}}}
\def\rmG{{\mathbf{G}}}
\def\rmH{{\mathbf{H}}}
\def\rmI{{\mathbf{I}}}
\def\rmJ{{\mathbf{J}}}
\def\rmK{{\mathbf{K}}}
\def\rmL{{\mathbf{L}}}
\def\rmM{{\mathbf{M}}}
\def\rmN{{\mathbf{N}}}
\def\rmO{{\mathbf{O}}}
\def\rmP{{\mathbf{P}}}
\def\rmQ{{\mathbf{Q}}}
\def\rmR{{\mathbf{R}}}
\def\rmS{{\mathbf{S}}}
\def\rmT{{\mathbf{T}}}
\def\rmU{{\mathbf{U}}}
\def\rmV{{\mathbf{V}}}
\def\rmW{{\mathbf{W}}}
\def\rmX{{\mathbf{X}}}
\def\rmY{{\mathbf{Y}}}
\def\rmZ{{\mathbf{Z}}}

% Elements of random matrices
\def\ermA{{\textnormal{A}}}
\def\ermB{{\textnormal{B}}}
\def\ermC{{\textnormal{C}}}
\def\ermD{{\textnormal{D}}}
\def\ermE{{\textnormal{E}}}
\def\ermF{{\textnormal{F}}}
\def\ermG{{\textnormal{G}}}
\def\ermH{{\textnormal{H}}}
\def\ermI{{\textnormal{I}}}
\def\ermJ{{\textnormal{J}}}
\def\ermK{{\textnormal{K}}}
\def\ermL{{\textnormal{L}}}
\def\ermM{{\textnormal{M}}}
\def\ermN{{\textnormal{N}}}
\def\ermO{{\textnormal{O}}}
\def\ermP{{\textnormal{P}}}
\def\ermQ{{\textnormal{Q}}}
\def\ermR{{\textnormal{R}}}
\def\ermS{{\textnormal{S}}}
\def\ermT{{\textnormal{T}}}
\def\ermU{{\textnormal{U}}}
\def\ermV{{\textnormal{V}}}
\def\ermW{{\textnormal{W}}}
\def\ermX{{\textnormal{X}}}
\def\ermY{{\textnormal{Y}}}
\def\ermZ{{\textnormal{Z}}}

% Vectors
\def\vzero{{\bm{0}}}
\def\vone{{\bm{1}}}
\def\vmu{{\bm{\mu}}}
\def\vtheta{{\bm{\theta}}}
\def\vphi{{\bm{\phi}}}
\def\va{{\bm{a}}}
\def\vb{{\bm{b}}}
\def\vc{{\bm{c}}}
\def\vd{{\bm{d}}}
\def\ve{{\bm{e}}}
\def\vf{{\bm{f}}}
\def\vg{{\bm{g}}}
\def\vh{{\bm{h}}}
\def\vi{{\bm{i}}}
\def\vj{{\bm{j}}}
\def\vk{{\bm{k}}}
\def\vl{{\bm{l}}}
\def\vm{{\bm{m}}}
\def\vn{{\bm{n}}}
\def\vo{{\bm{o}}}
\def\vp{{\bm{p}}}
\def\vq{{\bm{q}}}
\def\vr{{\bm{r}}}
\def\vs{{\bm{s}}}
\def\vt{{\bm{t}}}
\def\vu{{\bm{u}}}
\def\vv{{\bm{v}}}
\def\vw{{\bm{w}}}
\def\vx{{\bm{x}}}
\def\vy{{\bm{y}}}
\def\vz{{\bm{z}}}

% Elements of vectors
\def\evalpha{{\alpha}}
\def\evbeta{{\beta}}
\def\evepsilon{{\epsilon}}
\def\evlambda{{\lambda}}
\def\evomega{{\omega}}
\def\evmu{{\mu}}
\def\evpsi{{\psi}}
\def\evsigma{{\sigma}}
\def\evtheta{{\theta}}
\def\eva{{a}}
\def\evb{{b}}
\def\evc{{c}}
\def\evd{{d}}
\def\eve{{e}}
\def\evf{{f}}
\def\evg{{g}}
\def\evh{{h}}
\def\evi{{i}}
\def\evj{{j}}
\def\evk{{k}}
\def\evl{{l}}
\def\evm{{m}}
\def\evn{{n}}
\def\evo{{o}}
\def\evp{{p}}
\def\evq{{q}}
\def\evr{{r}}
\def\evs{{s}}
\def\evt{{t}}
\def\evu{{u}}
\def\evv{{v}}
\def\evw{{w}}
\def\evx{{x}}
\def\evy{{y}}
\def\evz{{z}}

% Matrix
\def\mA{{\bm{A}}}
\def\mB{{\bm{B}}}
\def\mC{{\bm{C}}}
\def\mD{{\bm{D}}}
\def\mE{{\bm{E}}}
\def\mF{{\bm{F}}}
\def\mG{{\bm{G}}}
\def\mH{{\bm{H}}}
\def\mI{{\bm{I}}}
\def\mJ{{\bm{J}}}
\def\mK{{\bm{K}}}
\def\mL{{\bm{L}}}
\def\mM{{\bm{M}}}
\def\mN{{\bm{N}}}
\def\mO{{\bm{O}}}
\def\mP{{\bm{P}}}
\def\mQ{{\bm{Q}}}
\def\mR{{\bm{R}}}
\def\mS{{\bm{S}}}
\def\mT{{\bm{T}}}
\def\mU{{\bm{U}}}
\def\mV{{\bm{V}}}
\def\mW{{\bm{W}}}
\def\mX{{\bm{X}}}
\def\mY{{\bm{Y}}}
\def\mZ{{\bm{Z}}}
\def\mBeta{{\bm{\beta}}}
\def\mPhi{{\bm{\Phi}}}
\def\mLambda{{\bm{\Lambda}}}
\def\mSigma{{\bm{\Sigma}}}

% Tensor
\DeclareMathAlphabet{\mathsfit}{\encodingdefault}{\sfdefault}{m}{sl}
\SetMathAlphabet{\mathsfit}{bold}{\encodingdefault}{\sfdefault}{bx}{n}
\newcommand{\tens}[1]{\bm{\mathsfit{#1}}}
\def\tA{{\tens{A}}}
\def\tB{{\tens{B}}}
\def\tC{{\tens{C}}}
\def\tD{{\tens{D}}}
\def\tE{{\tens{E}}}
\def\tF{{\tens{F}}}
\def\tG{{\tens{G}}}
\def\tH{{\tens{H}}}
\def\tI{{\tens{I}}}
\def\tJ{{\tens{J}}}
\def\tK{{\tens{K}}}
\def\tL{{\tens{L}}}
\def\tM{{\tens{M}}}
\def\tN{{\tens{N}}}
\def\tO{{\tens{O}}}
\def\tP{{\tens{P}}}
\def\tQ{{\tens{Q}}}
\def\tR{{\tens{R}}}
\def\tS{{\tens{S}}}
\def\tT{{\tens{T}}}
\def\tU{{\tens{U}}}
\def\tV{{\tens{V}}}
\def\tW{{\tens{W}}}
\def\tX{{\tens{X}}}
\def\tY{{\tens{Y}}}
\def\tZ{{\tens{Z}}}


% Graph
\def\gA{{\mathcal{A}}}
\def\gB{{\mathcal{B}}}
\def\gC{{\mathcal{C}}}
\def\gD{{\mathcal{D}}}
\def\gE{{\mathcal{E}}}
\def\gF{{\mathcal{F}}}
\def\gG{{\mathcal{G}}}
\def\gH{{\mathcal{H}}}
\def\gI{{\mathcal{I}}}
\def\gJ{{\mathcal{J}}}
\def\gK{{\mathcal{K}}}
\def\gL{{\mathcal{L}}}
\def\gM{{\mathcal{M}}}
\def\gN{{\mathcal{N}}}
\def\gO{{\mathcal{O}}}
\def\gP{{\mathcal{P}}}
\def\gQ{{\mathcal{Q}}}
\def\gR{{\mathcal{R}}}
\def\gS{{\mathcal{S}}}
\def\gT{{\mathcal{T}}}
\def\gU{{\mathcal{U}}}
\def\gV{{\mathcal{V}}}
\def\gW{{\mathcal{W}}}
\def\gX{{\mathcal{X}}}
\def\gY{{\mathcal{Y}}}
\def\gZ{{\mathcal{Z}}}

% Sets
\def\sA{{\mathbb{A}}}
\def\sB{{\mathbb{B}}}
\def\sC{{\mathbb{C}}}
\def\sD{{\mathbb{D}}}
% Don't use a set called E, because this would be the same as our symbol
% for expectation.
\def\sF{{\mathbb{F}}}
\def\sG{{\mathbb{G}}}
\def\sH{{\mathbb{H}}}
\def\sI{{\mathbb{I}}}
\def\sJ{{\mathbb{J}}}
\def\sK{{\mathbb{K}}}
\def\sL{{\mathbb{L}}}
\def\sM{{\mathbb{M}}}
\def\sN{{\mathbb{N}}}
\def\sO{{\mathbb{O}}}
\def\sP{{\mathbb{P}}}
\def\sQ{{\mathbb{Q}}}
\def\sR{{\mathbb{R}}}
\def\sS{{\mathbb{S}}}
\def\sT{{\mathbb{T}}}
\def\sU{{\mathbb{U}}}
\def\sV{{\mathbb{V}}}
\def\sW{{\mathbb{W}}}
\def\sX{{\mathbb{X}}}
\def\sY{{\mathbb{Y}}}
\def\sZ{{\mathbb{Z}}}

% Entries of a matrix
\def\emLambda{{\Lambda}}
\def\emA{{A}}
\def\emB{{B}}
\def\emC{{C}}
\def\emD{{D}}
\def\emE{{E}}
\def\emF{{F}}
\def\emG{{G}}
\def\emH{{H}}
\def\emI{{I}}
\def\emJ{{J}}
\def\emK{{K}}
\def\emL{{L}}
\def\emM{{M}}
\def\emN{{N}}
\def\emO{{O}}
\def\emP{{P}}
\def\emQ{{Q}}
\def\emR{{R}}
\def\emS{{S}}
\def\emT{{T}}
\def\emU{{U}}
\def\emV{{V}}
\def\emW{{W}}
\def\emX{{X}}
\def\emY{{Y}}
\def\emZ{{Z}}
\def\emSigma{{\Sigma}}

% entries of a tensor
% Same font as tensor, without \bm wrapper
\newcommand{\etens}[1]{\mathsfit{#1}}
\def\etLambda{{\etens{\Lambda}}}
\def\etA{{\etens{A}}}
\def\etB{{\etens{B}}}
\def\etC{{\etens{C}}}
\def\etD{{\etens{D}}}
\def\etE{{\etens{E}}}
\def\etF{{\etens{F}}}
\def\etG{{\etens{G}}}
\def\etH{{\etens{H}}}
\def\etI{{\etens{I}}}
\def\etJ{{\etens{J}}}
\def\etK{{\etens{K}}}
\def\etL{{\etens{L}}}
\def\etM{{\etens{M}}}
\def\etN{{\etens{N}}}
\def\etO{{\etens{O}}}
\def\etP{{\etens{P}}}
\def\etQ{{\etens{Q}}}
\def\etR{{\etens{R}}}
\def\etS{{\etens{S}}}
\def\etT{{\etens{T}}}
\def\etU{{\etens{U}}}
\def\etV{{\etens{V}}}
\def\etW{{\etens{W}}}
\def\etX{{\etens{X}}}
\def\etY{{\etens{Y}}}
\def\etZ{{\etens{Z}}}

% The true underlying data generating distribution
\newcommand{\pdata}{p_{\rm{data}}}
\newcommand{\ptarget}{p_{\rm{target}}}
\newcommand{\pprior}{p_{\rm{prior}}}
\newcommand{\pbase}{p_{\rm{base}}}
\newcommand{\pref}{p_{\rm{ref}}}

% The empirical distribution defined by the training set
\newcommand{\ptrain}{\hat{p}_{\rm{data}}}
\newcommand{\Ptrain}{\hat{P}_{\rm{data}}}
% The model distribution
\newcommand{\pmodel}{p_{\rm{model}}}
\newcommand{\Pmodel}{P_{\rm{model}}}
\newcommand{\ptildemodel}{\tilde{p}_{\rm{model}}}
% Stochastic autoencoder distributions
\newcommand{\pencode}{p_{\rm{encoder}}}
\newcommand{\pdecode}{p_{\rm{decoder}}}
\newcommand{\precons}{p_{\rm{reconstruct}}}

\newcommand{\laplace}{\mathrm{Laplace}} % Laplace distribution

\newcommand{\E}{\mathbb{E}}
\newcommand{\Ls}{\mathcal{L}}
\newcommand{\R}{\mathbb{R}}
\newcommand{\emp}{\tilde{p}}
\newcommand{\lr}{\alpha}
\newcommand{\reg}{\lambda}
\newcommand{\rect}{\mathrm{rectifier}}
\newcommand{\softmax}{\mathrm{softmax}}
\newcommand{\sigmoid}{\sigma}
\newcommand{\softplus}{\zeta}
\newcommand{\KL}{D_{\mathrm{KL}}}
\newcommand{\Var}{\mathrm{Var}}
\newcommand{\standarderror}{\mathrm{SE}}
\newcommand{\Cov}{\mathrm{Cov}}
% Wolfram Mathworld says $L^2$ is for function spaces and $\ell^2$ is for vectors
% But then they seem to use $L^2$ for vectors throughout the site, and so does
% wikipedia.
\newcommand{\normlzero}{L^0}
\newcommand{\normlone}{L^1}
\newcommand{\normltwo}{L^2}
\newcommand{\normlp}{L^p}
\newcommand{\normmax}{L^\infty}

\newcommand{\parents}{Pa} % See usage in notation.tex. Chosen to match Daphne's book.

\DeclareMathOperator*{\argmax}{arg\,max}
\DeclareMathOperator*{\argmin}{arg\,min}

\DeclareMathOperator{\sign}{sign}
\DeclareMathOperator{\Tr}{Tr}
\let\ab\allowbreak

% \usepackage{algorithmic}  
% \usepackage[algo2e, ruled]{algorithm2e} 

\newcommand{\AK}[1]{{\color{Magenta} [\textbf{AK:} #1]}}

% Attempt to make hyperref and algorithmic work together better:
\newcommand{\theHalgorithm}{\arabic{algorithm}}

% Use the following line for the initial blind version submitted for review:
\usepackage[accepted]{icml2025}

% If accepted, instead use the following line for the camera-ready submission:
% \usepackage[accepted]{icml2025}

% For theorems and such
\usepackage{amsmath}
\usepackage{amssymb}
\usepackage{mathtools}
\usepackage{amsthm}

% if you use cleveref..
\usepackage[capitalize,noabbrev]{cleveref}

\renewcommand{\algorithmicrequire}{\textbf{Input:}}
\renewcommand{\algorithmicensure}{\textbf{Output:}}

%%%%%%%%%%%%%%%%%%%%%%%%%%%%%%%%
% THEOREMS
%%%%%%%%%%%%%%%%%%%%%%%%%%%%%%%%
\theoremstyle{plain}
\newtheorem{theorem}{Theorem}
\newtheorem{proposition}[theorem]{Proposition}
\newtheorem{lemma}[theorem]{Lemma}
\newtheorem{corollary}[theorem]{Corollary}
\theoremstyle{definition}
\newtheorem{definition}[theorem]{Definition}
\newtheorem{assumption}[theorem]{Assumption}
\theoremstyle{remark}
\newtheorem{remark}[theorem]{Remark}

% Todonotes is useful during development; simply uncomment the next line
%    and comment out the line below the next line to turn off comments
%\usepackage[disable,textsize=tiny]{todonotes}
\usepackage[textsize=tiny]{todonotes}


% The \icmltitle you define below is probably too long as a header.
% Therefore, a short form for the running title is supplied here:
\icmltitlerunning{Towards Cost-Effective Reward Guided Text Generation}

\begin{document}

\twocolumn[
\icmltitle{Towards Cost-Effective Reward Guided Text Generation}

% It is OKAY to include author information, even for blind
% submissions: the style file will automatically remove it for you
% unless you've provided the [accepted] option to the icml2025
% package.

% List of affiliations: The first argument should be a (short)
% identifier you will use later to specify author affiliations
% Academic affiliations should list Department, University, City, Region, Country
% Industry affiliations should list Company, City, Region, Country

% You can specify symbols, otherwise they are numbered in order.
% Ideally, you should not use this facility. Affiliations will be numbered
% in order of appearance and this is the preferred way.
\icmlsetsymbol{equal}{*}

\begin{icmlauthorlist}
    \icmlauthor{Ahmad Rashid}{equal,uw,vi}
    \icmlauthor{Ruotian Wu}{equal,uw,vi}
    \icmlauthor{Rongqi Fan}{uw}
    \icmlauthor{Hongliang Li}{hu}
    \icmlauthor{Agustinus Kristiadi}{vi}
    \icmlauthor{Pascal Poupart}{uw,vi}
  \end{icmlauthorlist}

  \icmlaffiliation{uw}{University of Waterloo}
  \icmlaffiliation{vi}{Vector Institute}
  \icmlaffiliation{hu}{Huawei Technologies}

  \icmlcorrespondingauthor{Ahmad Rashid}{a9rashid@uwaterloo.ca}

% \icmlaffiliation{yyy}{Department of XXX, University of YYY, Location, Country}
% \icmlaffiliation{comp}{Company Name, Location, Country}
% \icmlaffiliation{sch}{School of ZZZ, Institute of WWW, Location, Country}

% \icmlcorrespondingauthor{Firstname1 Lastname1}{first1.last1@xxx.edu}
% \icmlcorrespondingauthor{Firstname2 Lastname2}{first2.last2@www.uk}

% You may provide any keywords that you
% find helpful for describing your paper; these are used to populate
% the "keywords" metadata in the PDF but will not be shown in the document
\icmlkeywords{RLHF, Training Cost, LLM, Efficiency}

\vskip 0.3in
]

% this must go after the closing bracket ] following \twocolumn[ ...

% This command actually creates the footnote in the first column
% listing the affiliations and the copyright notice.
% The command takes one argument, which is text to display at the start of the footnote.
% The \icmlEqualContribution command is standard text for equal contribution.
% Remove it (just {}) if you do not need this facility.

%\printAffiliationsAndNotice{}  % leave blank if no need to mention equal contribution
\printAffiliationsAndNotice{\icmlEqualContribution} % otherwise use the standard text.

\begin{abstract}
% Reinforcement learning from human feedback (RLHF) is a critical step for aligning Large language models (LLMs) with human preferences. However, conventional training algorithms like proximal policy optimization (PPO) usually suffer from high cost and instability. Reward-guided text generation (RGTG) is an emerging method where a reward model is used to score partial sequences during decoding, and the reward score is combined with LLM logits to achieve an output distribution similar to RLHF. However, training the reward models remains to be challenging. Despite multiple methods of training such models have been proposed, they all suffer from certain assumptions and limitations. In this work, we aim to seek alternative methods that are more principled to approach such reward models. We proposed a local constraint that allows LLMs and score partial sequence based on their optimal expansions.

Reward-guided text generation (RGTG) has emerged as a viable alternative to offline reinforcement learning from human feedback (RLHF). 
RGTG methods can align baseline language models to human preferences without further training like in standard RLHF methods. 
However, they rely on a reward model to score each candidate token generated by the language model at inference, incurring significant test-time overhead.
Additionally, the reward model is usually only trained to score full sequences, which can lead to sub-optimal choices for partial sequences. 
In this work, we present a novel reward model architecture that is trained, using a Bradley-Terry loss, to prefer the optimal expansion of a sequence with just a \emph{single call} to the reward model at each step of the generation process.  
That is, a score for all possible candidate tokens is generated simultaneously, leading to efficient inference. 
We theoretically analyze various RGTG reward models and demonstrate that prior techniques prefer sub-optimal sequences compared to our method during inference. 
Empirically, our reward model leads to significantly faster inference than other RGTG methods. 
It requires fewer calls to the reward model and performs competitively compared to previous RGTG and offline RLHF methods. 
\end{abstract}

\section{Introduction}
\label{sec:intro}
\section{Introduction}\label{sec:Intro} 


Novel view synthesis offers a fundamental approach to visualizing complex scenes by generating new perspectives from existing imagery. 
This has many potential applications, including virtual reality, movie production and architectural visualization \cite{Tewari2022NeuRendSTAR}. 
An emerging alternative to the common RGB sensors are event cameras, which are  
 bio-inspired visual sensors recording events, i.e.~asynchronous per-pixel signals of changes in brightness or color intensity. 

Event streams have very high temporal resolution and are inherently sparse, as they only happen when changes in the scene are observed. 
Due to their working principle, event cameras bring several advantages, especially in challenging cases: they excel at handling high-speed motions 
and have a substantially higher dynamic range of the supported signal measurements than conventional RGB cameras. 
Moreover, they have lower power consumption and require varied storage volumes for captured data that are often smaller than those required for synchronous RGB cameras \cite{Millerdurai_3DV2024, Gallego2022}. 

The ability to handle high-speed motions is crucial in static scenes as well,  particularly with handheld moving cameras, as it helps avoid the common problem of motion blur. It is, therefore, not surprising that event-based novel view synthesis has gained attention, although color values are not directly observed.
Notably, because of the substantial difference between the formats, RGB- and event-based approaches require fundamentally different design choices. %

The first solutions to event-based novel view synthesis introduced in the literature demonstrate promising results \cite{eventnerf, enerf} and outperform non-event-based alternatives for novel view synthesis in many challenging scenarios. 
Among them, EventNeRF \cite{eventnerf} enables novel-view synthesis in the RGB space by assuming events associated with three color channels as inputs. 
Due to its NeRF-based architecture \cite{nerf}, it can handle single objects with complete observations from roughly equal distances to the camera. 
It furthermore has limitations in training and rendering speed: 
the MLP used to represent the scene requires long training time and can only handle very limited scene extents or otherwise rendering quality will deteriorate. 
Hence, the quality of synthesized novel views will degrade for larger scenes. %

We present Event-3DGS (E-3DGS), i.e.,~a new method for novel-view synthesis from event streams using 3D Gaussians~\cite{3dgs} 
demonstrating fast reconstruction and rendering as well as handling of unbounded scenes. 
The technical contributions of this paper are as follows: 
\begin{itemize}
\item With E-3DGS, we introduce the first approach for novel view synthesis from a color event camera that combines 3D Gaussians with event-based supervision. 
\item We present frustum-based initialization, adaptive event windows, isotropic 3D Gaussian regularization and 3D camera pose refinement, and demonstrate that high-quality results can be obtained. %

\item Finally, we introduce new synthetic and real event datasets for large scenes to the community to study novel view synthesis in this new problem setting. 
\end{itemize}
Our experiments demonstrate systematically superior results compared to EventNeRF \cite{eventnerf} and other baselines. 
The source code and dataset of E-3DGS are released\footnote{\url{https://4dqv.mpi-inf.mpg.de/E3DGS/}}. 






\section{Preliminaries}
\label{sec:prelim}
\section{Preliminaries}
\label{sec:prelim}
Given a dataset ${\cal D} = \{(x_1)\}$ consisting of data samples $x_1$, e.g., an image, %
generative models learn a distribution $p(x_1)$, often by maximizing the likelihood. 
In the following we discuss how  this distribution is learnt with variational auto-encoders and rectified flow matching, and why the corresponding modeled data distribution is multi-modal. %

\subsection{Variational Auto-Encoders (VAEs)}
Variational inference generally and variational auto-encoders (VAEs)~\citep{KingmaICLR2014} specifically have been shown to learn multi-modal distributions. %
This is achieved by introducing a latent variable $z$. At inference time, a latent  $z$ is obtained by sampling from the prior distribution $p(z)$, typically a zero mean unit covariance Gaussian. %
A decoder which characterizes a distribution $p(x_1|z)$ over the output space is then used to obtain an output space sample $x_1$. %

At training time, variational auto-encoders use an encoder to compute an approximate posterior distribution $q_\phi(z|x_1)$ over the latent space. As the approximate posterior distribution is only needed at training time, the data $x_1$ %
can be leveraged. Note, the approximate posterior distribution is often a Gaussian with parameterized mean and covariance. A sample from this approximate posterior distribution is then used as input in the  distribution $p_\theta(x_1|z)$ characterized by the decoder. The loss encourages a high probability of the output space samples while favoring an approximate posterior distribution $q_\phi(z|x_1,c)$ that is similar to the prior distribution $p(z)$. To achieve this, formally, VAEs maximize a lower-bound on the log-likelihood, i.e., 
\begin{align*}
&\mathbb{E}_{x_1\sim{\cal D}}\log p(x_1) \\
&\geq \mathbb{E}_{x_1\sim{\cal D}}\left[\mathbb{E}_{z\sim q_\phi}\left[\log p_\theta(x_1|z)\right] - D_\text{KL}(q_\phi(\cdot|x_1)|p(\cdot))\right].
\end{align*}



\subsection{Rectified Flow Matching}

For flow matching, at inference time, a source distribution $p_0(x_0)$ is queried to obtain a sample $x_0$. This is akin to sampling of a latent variable from the prior in VAEs. Different from VAEs which perform a single forward pass through the decoder, in flow matching, the source distribution sample $x_0$ is used as the boundary condition for an ordinary differential equation (ODE). This ODE is `solved' by pushing the sample $x_0$ forward from time zero to time one via integration along a trajectory specified via a learned velocity vector-field $v_\theta(x_t,t)$ defined at time $t$ and location $x_t$,  and commonly parameterized by deep net weights $\theta$. Note, the velocity vector-field is queried many times during integration. 
The likelihood of a data point $x_1$ can be assessed via the instantaneous change of variables formula~\citep{ChenARXIV2018,SongICLR2021,LipmanICLR2023}, 
\begin{equation}
\log p_1(x_1) = \log p_0(x_0) + \int_1^0 \di v_\theta(x_t,t) dt,
\label{eq:transportintegral}
\end{equation}
which is commonly~\citep{GrathwohlICLR2018} approximated via the Skilling-Hutchinson trace estimator~\citep{Skilling1989,Hutchinson1990}. Here, $\di$ denotes the divergence vector operator. %

Intuitively, by pushing forward samples $x_0$, randomly drawn from the source distribution $p_0(x_0)$, ambiguity in the data domain is captured as one expects from a generative model. 


At training time the parametric velocity vector-field $v_\theta(x_t,t)$ needs to be learnt. For this, 
coupled sample pairs $(x_0, x_1)$ are  constructed by randomly drawing  from the source and the target distribution, often independently from each other. 
A coupled sample $(x_0,x_1)$ and a time $t\in[0,1]$ is then used to compute a time-dependent location $x_t$ at time $t$ via a function $\phi(x_0, x_1, t) = x_t$. Recall, rectified flow matching uses $x_t = \phi(x_0,x_1,t) = (1-t)x_0 + tx_1$. %
Interpreting $x_t$ as a location, intuitively, the ``ground-truth'' velocity vector-field $v(x_0,x_1,t)$ is readily available via $v(x_0,x_1,t) = \partial\phi(x_0,x_1,t)/\partial t$, and can be used as the target to learn the parametric velocity vector-field $v_\theta(x_t,t)$. 
Concretely, flow matching learns the parametric velocity vector field $v_\theta(x_t,t)$ by matching the target via an $\ell_2$ loss, i.e., by minimizing w.r.t.\ trainable parameters $\theta$ the objective
$$
\mathbb{E}_{t,x_0,x_1}\left[\|v_\theta(x_t,t) - v(x_0,x_1,t)\|_2^2\right].
$$
Consider two different couplings that lead to different ``ground-truth'' velocity vectors at the same data-domain-time-domain point $(x_t,t)$. The parametric velocity vector-field $v_\theta(x_t,t)$ is then asked to match/regress to a different target given the same input $(x_t,t)$. This leads to averaging and the optimal functional velocity vector-field $v^\ast(x_t,t) = \mathbb{E}_{\{(x_0,x_1,t) : \phi(x_0,x_1,t) = x_t\}}\left[v(x_0,x_1,t)\right]$. Hence,  multi-modality in the data-domain-time-domain is not captured. In the following we discuss and study a method that is able to model this multi-modality. 



\section{Current Limitations of RGTG}
\label{sec:Limitations}
%!TEX root=icml2025.tex

We will discuss two primary limitations of current RGTG methods, namely high decoding cost and sub-optimal rewards. In the next section, we propose a solution to address these limitations.

\subsection{High Decoding Cost}
Most RGTG methods default to training a full sequence reward model $r_\phi$ and then either a) use it to directly score partial sequences \cite{khanov2023alignment} or b) distill a partial sequence value model $V_\theta$ from the full sequence reward model $r_\phi$ \cite{mudgalcontrolled}. During decoding, the score for each candidate token $y_i$ is calculated according to Equation~\ref{eq:score}. We note that the input to $V_\theta$ includes the sequence $y_{1:i-1}$ with each candidate token $y_{i}$ appended to the sequence.  Hence to score each candidate token, we need to make a different call to the value function, resulting into $k$ calls for top-$k$ decoding.  This adds substantial overhead during decoding.

\subsection{Sub-Optimal Reward Models}

Next we take a look at contemporary RGTG reward models and show that they may prefer partial sequences with sub-optimal extensions. 

\paragraph{PARGS} \citet{rashid2024critical} showed that using a BT reward model trained on full sequences to score partial sequences (as done by \citet{khanov2023alignment}) can lead to arbitrary rewards for partial sequences. Rashid et al.~proposed to train a BT reward model explicitly on partial sequences by creating a separate loss function for all prefix lengths $i$:

\begin{equation}
  \mathcal{L}_R^i = - \sum_{(\rvx, \rvy^w, \rvy^l) \in \D} \log \sigma ( V_{\theta} (\rvy^w_{1:i}|\rvx) - V_{\theta} (\rvy^l_{1:i}|\rvx)) . \label{eq:partial-seq-objectives}
\end{equation}


However, given that full sequence $\rvy^w$ is preferred to full sequence $\rvy^l$, training is based on the assumption that the partial sequence $\rvy^w_{1:i}$ is also preferred to the partial sequence $\rvy^l_{1:i}$. This assumption can be  problematic as the full-sequence dataset typically includes only one or a few full sequences that extend each partial sequence.  In fact, the empirical distribution of such extensions will impact the learned value function to the extent where a prefix with only extensions to suboptimal full sequences may be scored higher than a prefix with an extension to an optimal full sequence. %In fact, Lemma 2 of \cite{} shows that the resulting partial sequence reward model depends on the preference data distribution.  Hence, different preference data distributions may yield different partial sequence reward models, which is problematic.  We show in the following theorem for some preference data distributions, the partial reward 

\begin{theorem}
\label{thm:pargs}
In the limit of infinite training and a sufficiently expressive representation for the value function, PARGS may learn a value function that gives a lower score to a prefix extendable to an optimal full sequence than some other prefix.  More precisely, if $\rvy^* = \argmax_{\rvy} r(\rvy|\rvx)$, then there may exist $i,j,\rvy'$ such that
\begin{equation}
V(\rvy^*_{1:i}|\rvx) < V(\rvy'_{1:j}|\rvx)
\end{equation}
\end{theorem}

\begin{proof}
Let $\rvy^*$, $\rvy'$, $\rvy''$ and $\rvy'''$ be four responses to $\rvx$ such that $\rvy^*$ is an optimal response and $\rvy'$, $\rvy''$, $\rvy'''$ are three suboptimal responses.  Suppose also that the preference dataset contains exactly three comparisons:  $\D=\{(\rvx,\rvy^*,\rvy'), (\rvx,\rvy',\rvy''), (\rvx,\rvy',\rvy''')\}$ where the first response is preferred to the second response in each triple.  Suppose also that $\rvy^*$ and $\rvy'$ share the first $i-1$ tokens (i.e., $\rvy^*_{1:i-1} = \rvy'_{1:i-1}$) while $\rvy^*$, $\rvy''$ and $\rvy'''$ share the first $i$ tokens (i.e., $\rvy^*_{1:i} = \rvy''_{1:i} = \rvy'''_{1:i}$). In the limit of infinite training and sufficiently expressive value function representation, Lemma 2 in \cite{rashid2024critical} indicates that the learned value function $V$ satisfies
\begin{equation}
\label{eq:pargs}
    \sigma(V(\rvy^1_{1:i}|\rvx) - V(\rvy^2_{1:j}|\rvx)) = P_D([\rvx,\rvy^1] \succ [\rvx,\rvy^2])
\end{equation}
where $[a,b]$ indicates the concatenation of sequences $a$ and $b$, and $a\succ b$ indicates that $a$ is preferred to $b$.  \cref{eq:pargs} implies that the BT model induced by $V$ exhibits the same preference probabilities for the full sequence extension of $\rvy^1_{1:i}$ and $\rvy^2_{1:j}$ as the empirical distribution of the preference dataset.  Recall, that PARGS assumes that $[\rvx,\rvy^1_{1:i}] \succ [\rvx,\rvy^2_{1:j}]$ when their respective full sequence extensions exhibit the same preference ordering (i.e., $[\rvx,\rvy^1] \succ [\rvx,\rvy^2]$).  Since their might be different extensions $\rvy^1_{i+1:|\rvy^1|}$ and $\rvy^2_{i+1:|\rvy^2|}$ for each prefix with different preference labels in the preference dataset, then PARGS learns a value function that induces preference probabilities for partial sequences that are consistent with the empirical distribution $P_D$ of the preference dataset for the full sequence extensions of those partial sequences.  Applying \cref{eq:pargs} to prefixes $\rvy^*_{1:i}$ and $\rvy'_{1:i}$ yields:
\begin{equation}
\label{eq:sigmoid}
    \sigma(V(\rvy^*_{1:i}|\rvx) - V(\rvy'_{1:i}|\rvx)) = 1/3
\end{equation}
since the dataset $\D$ contains one preference ranking $(\rvx,\rvy^*,\rvy')$ where the full sequence extension $\rvy^*$ of $\rvy^*_{1:i}$ is preferred to the full sequence extension $\rvy'$ of $\rvy'_{1:i}$ and two preference rankings $(\rvx,\rvy'\rvy'')$, $(\rvx,\rvy'\rvy''')$ where the full sequence extension $\rvy'$ of $\rvy'_{1:i}$ is preferred to the full sequence extensions $\rvy''$, $\rvy'''$ of $\rvy^*_{1:i}$. 
Recall that $\rvy^*_{1:i}=\rvy''_{1:i}=\rvy'''_{1:i}$ and therefore $\rvy''$ and $\rvy'''$ are full sequence extensions of $\rvy^*_{1:i}$. Finally, since the sigmoid in \cref{eq:sigmoid} is less than 0.5, then $V(\rvy^*_{1:i}) < V(\rvy'_{1:i})$.  Hence, this shows that $\exists i{=}j,\rvy'$ such that $V(\rvy^*_{1:i}) < V(\rvy'_{1:j})$
\end{proof}

Theorem~\ref{thm:pargs} shows that the value function learned by PARGS may prefer prefixes that lead to suboptimal responses.  The key problem is PARGS' assumption that the preference ordering of prefixes is the same as the preference ordering of full sequence extensions.  Since it is possible to extend a prefix to many different full sequences with different scores, the value function learned by PARGS depends on the frequency of different prefix extensions instead of preferences only.  As shown in the proof of Theorem~\ref{thm:pargs}, this becomes problematic when a prefix that can lead to an optimal response is extended more frequently to losing full sequences instead of winning full sequences in $\D$.

%\vspace{0.5em}
%\begin{theorem} \label{thm:PARGS}
%  \label{thm:full_for_partial}
%  Let \(r\) be a reward model trained to minimize the Bradley-Terry loss on partial sequences
%  $\rvy^{1:\abs{\rvy}}$ \eqref{eq:partial-seq-objectives} using full-sequence preference data $\rvy^{1:i}$.
%  Then \(r\) may prefer sub-optimal continuations at decoding.
  
%  \AK{This statement is vague. A better statement for a theorem would be to fully specify the hypothesis, e.g.: ``Let \(r\) be a reward model ... Let \(s^*\), ... Then, if \(r_\phi(s^*) > r_\phi(s^{**})\) and ..., the token \(y_i'\) is picked.''}
  
%  \AK{Then, discuss the significance/implication of this theorem after the proof. E.g. mention that \(y_i'\) is suboptimal and thus RGTG/PARGS is pathological.}
%\end{theorem}
%
%\begin{proof}
%    Let $r_\phi$ be the full sequence reward model, $\mathbf{V}_{\theta}$ be the partial sequence reward model and $\rvy^{(p)} = y_1, \cdots, y_{i-1}$ be a partial sequence following the prompt $\rvx$. Suppose that $\rvs^* = \rvx, \rvy^{(p)}, y^*_{i}, y^*_{i+1}, \cdots, y^*_{n}$ is the optimal full sequence extended from ${\rvx, \rvy^{(p)}}$. Let $\rvs^{**} = \rvx, \rvy^{(p)}, y^*_{i}, y^{**}_{i+1} \cdots, y^{**}_{n}$ be a sequence extended from ${\rvx, \rvy^{(p)}, y^*_{i}}$ and $\rvs' = \rvx, \rvy^{(p)}, y'_{i}, y'_{i+1} \cdots, y'_{n}$: be a sequence extended from ${\rvx, \rvy^{(p)}}$. Suppose that $\mathcal{P} = \{\rvs^{**}, \rvs'\}$ is a pair in the preference dataset. 

%    \AK{This hypothesis should be put in the theorem's statement. Here, you just say ``... By the hypothesis that $r_{\phi}$ ..., then ...''}
%    \AK{Also, avoid using a concrete numbers. Make your hypothesis general.}
%    Suppose $r_{\phi}(s^*)=6, \:r_{\phi}(s^{**})=-6, \:r_{\phi}(s')=5$, then $\rvs'$ is the winning sequence in $\mathcal{P}$. 
%    By the assumption that "the prefixes of a winning full sequence is also wining against the prefix of the corresponding losing full sequence", partial sequence $(\rvx, \rvy^{p}, y'_{i})$ is also winning $(\rvx, \rvy^{p}, y^*_{i})$.  Following the Bradley-Terry loss, we maximize
%$$
%\begin{aligned}
%     \log \sigma \left( \mathbf{V}_{\theta} (\rvy^{(p)}, y'_{i}|\rvx) - \mathbf{V}_{\theta} (\rvy^{(p)}, y^*_{i}|\rvx)\right)
%\end{aligned}
%$$
%That is, $\mathbf{V}_{\theta} (\rvy^{(p)}, y'_{i}|\rvx) - \mathbf{V}_{\theta} (\rvy^{(p)}, y^*_{i}|\rvx)$ is maximized results in $\mathbf{V}_{\theta} (\rvy^{(p)}, y'_{i}|\rvx) > \mathbf{V}_{\theta} (\rvy^{(p)}, y^*_{i}|\rvx)$

%Next, suppose $\piref(y^{*}|\rvx, \rvy^{(p)}) = \piref(y'|\rvx, \rvy^{(p)}) = \frac{1}{2}$, at inference time we follow the policy:
%$$ 
%\begin{aligned}
%&\pi(y_{i} \vert \rvx,\rvy^{(p)}) \propto \piref(y_{i} \vert \rvx,\rvy^{(p)}) \exp(\beta \mathbf{V}_{\theta}(\rvy^{(p)}, y_{i} \vert \rvx))\\
%\Rightarrow ~~ &\pi(y'_{i} \vert \rvx,\rvy^{(p)}) > \pi(y^{*}_{i} \vert \rvx,\rvy^{(p)})
%\end{aligned}
%$$

%Again, following $\pi$ would generate $y'_{i}$ which is sub-optimal.
%\end{proof}

%\AK{Discuss the implication of the theorem here.}

\paragraph{CD}~\citet{mudgalcontrolled} proposed a target value function $V^*$ for partial sequences that corresponds to the expected reward of the full sequences when the partial sequence is extended by following the base model distribution $\piref$.
%
\begin{equation}
% \label{eq:CD}
    V^*(\rvx, \rvy_{1:i}) = \sum_{\rvy_{i+1:|\rvy|}} \piref(\rvy_{i+1:|\rvy|}|\rvx,\rvy_{1:i}) r(\rvx, \rvy)
\end{equation}
%
The training loss is the squared difference between the value function $V_\theta$ and the target $V^*$. They use rollouts from the base model along with a reward model trained on full sequences to distill the value function $V_\theta$. They sample extensions from $\piref$ to complete a partial sequence and compute the full-sequence score as the target $V^*$. This method has a limitation where the value function heavily depends on the language model. We will show such dependency is suboptimal. 

Value Augmented Sampling~\citep[VAS][]{han2024value} is similar to CD and uses $\piref$ to generate samples for learning a value function and a full-sequence reward model for generating the target score. However, the value function is trained by temporal difference (TD) learning.
% \subsection{Value Augmented Sampling (VAS)}
% Similar to CD, VAS also proposed to use a base LLM to generate samples and hence provide a target score by the full-sequence reward model. Then, they applied temporal difference learning to update the parameters of the value function which they treat as the partial-sequence reward model in the decoding step.

% \subsection{Why LLM dependency is undesirable?}
% \label{LLMD}

\begin{theorem}
\label{thm:full_for_partial}
In the limit of infinite training and a sufficiently expressive representation for the value function, CD may learn a value function that gives a lower score to a prefix extendable to an optimal full sequence than some other prefix.  More precisely, if $\rvy^* = \argmax_{\rvy} r(\rvy|\rvx)$, then there may exist $i,j,\rvy'$ such that
\begin{equation}
V(\rvy^*_{1:i}|\rvx) < V(\rvy'_{1:j}|\rvx)
\end{equation}

%\AK{Here, the statement is better since it's more precise.}

\end{theorem}

\begin{proof}
Let $\rvy^*$ be an optimal response to $\rvx$ such that $r(\rvy^*|\rvx)=6$.  Let $\rvy'$ and $\rvy''$ be two suboptimal responses to $\rvx$ such that $r(\rvy'|\rvx)=4$ and $r(\rvy''|\rvx)=-6$.  Suppose that $\rvy'$ and $\rvy^*$ share the same first $i-1$ tokens (i.e., $\rvy'_{1:i-1} = \rvy^*_{1:i-1}$) and that $\rvy''$ and $\rvy*$ share the same first $i$ tokens (i.e., $\rvy'_{1:i} = \rvy^*_{1:i}$). After generating $\rvy^*_{1:i}$, suppose that $\piref$ generates only $\rvy^*_{i+1:|\rvy^*|}$ and $\rvy''_{i+1:|\rvy''|}$ with uniform probability (i.e., $\piref(\rvy^*_{i+1:|\rvy^*|}|\rvx,\rvy^*_{1:i})=\piref(\rvy''_{i+1:|\rvy''}|\rvx,\rvy''_{1:i})=0.5$ and any other continuation has probability 0). After generating $\rvy'_{1:i}$, suppose also that $\piref$ generates only $\rvy'_{i+1:|\rvy'|}$ (i.e., $\piref(\rvy''_{i+1:|\rvy''}|\rvx,\rvy''_{1:i})=1$ and any other continuation has probability 0).  Then with infinite training and a sufficiently expressive value function representation, CD learns the following partial sequence values
\begin{align}
V(\rvy^*_{1:i}|\rvx) & = \piref(\rvy^*_{i+1:|\rvy^*|}|\rvx,\rvy^*_{1:i})r(\rvy^*|\rvx) \\
& + \piref(\rvy''_{i+1:|\rvy''|}|\rvx,\rvy^*_{1:i})r(\rvy''|\rvx) \\
& = 0.5(6) + 0.5(-6) = 0 \\
V(\rvy'_{1:i}|\rvx) & = \piref(\rvy'_{i+1:|\rvy'|}|\rvx,\rvy'_{1:i})r(\rvy'|\rvx) \\
& = 1(4) = 4
\end{align}
This example shows that $\exists i{=}j,\rvy'$ such that $V(\rvy^*_{1:i}|\rvx) < V(\rvy'_{1:j}|\rvx)$.
\end{proof}

Theorem~\ref{thm:full_for_partial} shows that CD may prefer prefixes that cannot be extended to optimal sequences depending on $\piref$.  The key problem is the dependency of the target $V^*$ on $\piref$.  When $\piref$ extends a prefix to bad responses, the value of this prefix is low, but if it extends the prefix to good responses, the value of this prefix is high.  In principle, the value function $V$ should be independent of $\piref$.  In RLHF, $\piref$ is the quantity that we seek to improve so it does not make sense to improve $\piref$ with a value function that depends on $\piref$ itself.  The value function should depend only on the preferences induced by the full sequence reward model.  As shown in the proof of Theorem~\ref{thm:full_for_partial}, CD may not prefer a prefix that can lead to an optimal response when it is extended by $\piref$ to suboptimal responses.

%\begin{theorem} \label{thm:CD}
%  \label{thm:full_for_partial}
%  A reward model which is trained on continuations from $\piref$ according to \eqref{thm:CD} may lead to sub-optimal continuations at decoding.
%\end{theorem}

% In this section we will describe a counter-example showing that RGTG is suboptimal if the partial-sequence reward model is dependent on the LLM. We consider CD explicitly.\\

%\begin{proof}
% Let $r_\phi$ be the full sequence reward model, $\mathbf{V}_{\theta}$ be the partial sequence reward model and $\rvy^{(p)} = y_1, \cdots, y_{i-1}$ be a partial sequence following the prompt $\rvx$.

% Let $\mathcal{S} = \{\rvs^*, \rvs^{**}, \rvs', \rvs''\}$ be the set of all sequences where 


% \begin{itemize}
    % \item $\mathbf{V}_{\theta}$: partial-sequence reward model
    % \item $r_{\phi}$: full-sequence reward model
    % \item $\rvx$: prompt
    % \item $\rvy^{(p)} = y_1, \cdots, y_{i-1}$: partial-sequence following $\rvx$
%    \item $\rvs^* = \rvx, \rvy^{(p)}, y^*_{i}, y^*_{i+1}, \cdots, y^*_{n}$: the optimal full sequence extended from ${\rvx, \rvy^{(p)}}$
%    \item $\rvs^{**} = \rvx, \rvy^{(p)}, y^*_{i}, y^{**}_{i+1} \cdots, y^{**}_{n}$: another sequence extended from ${\rvx, \rvy^{(p)}, y^*_{i}}$
%    \item $\rvs' = \rvx, \rvy^{(p)}, y'_{i}, y'_{i+1} \cdots, y'_{n}$: another sequence extended from ${\rvx, \rvy^{(p)}}$
%    \item $\rvs'' = \rvx, \rvy^{(p)}, y'_{i},  y''_{i+1}, \cdots, y''_{n}$: another sequence extended from ${\rvx, \rvy^{(p)}, y'_{i}}$
    % \item $\mathcal{S} = \{\rvs^*, \rvs^{**}, \rvs', \rvs''\}$
%\end{itemize} 

%\AK{Same comment as before, can we make the following specific numbers more general?}

%\noindent Suppose $r_{\phi}(s^*)=6, \: \:r_{\phi}(s^{**})=-6,\: \:r_{\phi}(s')=5,\: \:r_{\phi}(s')=3\: \: $ and $\piref$ will only sample $y'$ or $y^{*}$ as the next token and sample sequences from $\mathcal{S}$ follow a uniform distribution, that is, 
%$$
%\begin{aligned}
%\piref(y^{*}|\rvx, \rvy^{(p)}) &= \piref(y'|\rvx, \rvy^{(p)}) = \frac{1}{2}\\
%\piref(\rvs'|\rvx, \rvy^{(p)}, y') &=\piref(\rvs''|\rvx, \rvy^{(p)}, y') = \frac{1}{2}\\
%\piref(\rvs^{*}|\rvx, \rvy^{(p)}, y^{*}) &=\piref(\rvs^{**}|\rvx, \rvy^{(p)}, y^{*}) = \frac{1}{2}\\
%\piref(\rvs|\rvx, \rvy^{(p)}) &= \frac{1}{4} \: , \:\: \forall \rvs \in \mathcal{S}
%\end{aligned}
%$$

%\noindent By following the training objective of CD, in the limit we have:
%$$
%\begin{aligned}
%    \mathbf{V}_{\theta}(\rvy^{(p)}, y_{i} | \rvx) = \sum_{\rvz \sim \piref} \piref(\rvz|\rvx, \rvy^{(p)}, y_{i}, ) r_{\phi}(\rvx, \rvy^{(p)}, y_{i}, \rvz)
%\end{aligned}
%$$
%\\
%Hence we get the following values for partial sequences:
%$$
%\begin{aligned}
%\mathbf{V}_{\theta}(\rvy^{(p)},  y^{*}_{i} | \rvx) &= \piref(\rvs^{*}|\rvx, \rvy^{(p)}) r_{\phi}(\rvs^{*}) \\ & ~~ + \piref(\rvs^{**}|\rvx, \rvy^{(p)}) r_{\phi}(\rvs^{**}) \\
%&= (\frac{1}{2}) (6) + (\frac{1}{2})(-6) = 0\\
%\mathbf{V}_{\theta}(\rvy^{(p)},  y'_{i} | \rvx) &= \piref(\rvs'|\rvx, \rvy^{(p)}) r_{\phi}(\rvs') \\ & ~~ + \piref(\rvs''|\rvx, \rvy^{(p)}) r_{\phi}(\rvs'') \\
%&= (\frac{1}{2}) (5) + (\frac{1}{2})(3) = 4
%\end{aligned}
%$$
%\\
%Next, consider inference where we follow the policy:\\
%$$ 
%\begin{aligned}
%&\pi(y_{i} \vert \rvx,\rvy^{(p)}) \propto \piref(y_{i} \vert \rvx,\rvy^{(p)}) \exp(\beta \mathbf{V}_{\theta}(\rvy^{(p)}, y_{i} \vert \rvx))\\
%\Rightarrow ~~  &\pi(y'_{i} \vert \rvx,\rvy^{(p)}) > \pi(y^{*}_{i} \vert \rvx,\rvy^{(p)})
%\end{aligned}
%$$
%\\
%Note following $\pi$ would generate $y'_{i}$ and deviate from the optimal sequence.

%\end{proof}

%\AK{Same as before. Discuss the implication/significane/practicality of this preceding theorem.}

% \subsection{Limitation of PARGS}
% We will use the same notations as above for $\mathbf{V}_{\theta}$, $r_{\phi}$, $\rvx$, $\rvy^{(p)}$ and $\rvs^*$, with new sequence examples:
%  \begin{itemize}
%     \item $\rvs^{**} = \rvx, \rvy^{(p)}, y^*_{i}, y^{**}_{i+1} \cdots, y^{**}_{n}$: another sequence extended from ${\rvx, \rvy^{(p)}, y^*_{i}}$
%     \item $\rvs' = \rvx, \rvy^{(p)}, y'_{i}, y'_{i+1} \cdots, y'_{n}$: another sequence extended from ${\rvx, \rvy^{(p)}}$
%     \item $\mathcal{P} = \{\rvs^{**}, \rvs'\}$ is a pair in the preference dataset 
% \end{itemize} 

% Suppose $r_{\phi}(s^*)=6, \:r_{\phi}(s^{**})=-6, \:r_{\phi}(s')=5$, then $\rvs'$ is the winning sequence in $\mathcal{P}$. By the assumption that "the prefixes of a winning full sequence is also wining against the prefix of the corresponding losing full sequence", partial sequence $(\rvx, \rvy^{p}, y'_{i})$ is also winning $(\rvx, \rvy^{p}, y^*_{i})$.  Following the Bradley-Terry loss, we maximize
% $$
% \begin{aligned}
%      \log \sigma \left( \mathbf{V}_{\theta} (\rvy^{(p)}, y'_{i}|\rvx) - \mathbf{V}_{\theta} (\rvy^{(p)}, y^*_{i}|\rvx)\right)
% \end{aligned}
% $$
% That is, $\mathbf{V}_{\theta} (\rvy^{(p)}, y'_{i}|\rvx) - \mathbf{V}_{\theta} (\rvy^{(p)}, y^*_{i}|\rvx)$ is maximized results in $\mathbf{V}_{\theta} (\rvy^{(p)}, y'_{i}|\rvx) > \mathbf{V}_{\theta} (\rvy^{(p)}, y^*_{i}|\rvx)$

% Next, suppose $\piref(y^{*}|\rvx, \rvy^{(p)}) = \piref(y'|\rvx, \rvy^{(p)}) = \frac{1}{2}$, at inference time we follow the policy:
% $$ 
% \begin{aligned}
% &\pi(y_{i} \vert \rvx,\rvy^{(p)}) \propto \piref(y_{i} \vert \rvx,\rvy^{(p)}) \exp(\beta \mathbf{V}_{\theta}(\rvy^{(p)}, y_{i} \vert \rvx))\\
% \Rightarrow ~~ &\pi(y'_{i} \vert \rvx,\rvy^{(p)}) > \pi(y^{*}_{i} \vert \rvx,\rvy^{(p)})
% \end{aligned}
% $$

% Again, following $\pi$ would generate $y'_{i}$ hence result in become suboptimal.


\section{Proposal}
\label{sec:proposal}
%!TEX root=icml2025.tex

We propose to mitigate the inference overhead and sub-optimal rewards of previous RGTG methods by introducing (i) an efficient reward model and (ii) a novel loss function that will ensure that the resulting value function prefers prefixes extendable to optimal responses.  We name our method FaRMA, i.e. Faster Reward Model for Alignment.

\subsection{An Efficient Reward Model}

We design a reward model architecture so that instead of obtaining a single score for a sequence, we obtain the score for all possible next tokens in the dictionary. We modify \eqref{eq:score} such that:

\begin{equation}
  \label{eq:score-new}
  \begin{aligned}
  \textstyle
  \score(y_{i} | \mathbf{x}, \mathbf{y}_{1:i-1}) = &\log \piref(y_{i} \vert \mathbf{x}, \mathbf{y}_{1:i-1}) \\ 
    &\quad + \beta V_\theta(y_{i} \vert \mathbf{x}, \mathbf{y}_{1:i-1}) ,
  \end{aligned}
\end{equation}

where $V_\theta(.) \in R^{|D|\times 1}$ and $|D|$ is the size of the vocabulary.
% \AK{Vocabulary.}
In order to get the score of sequence $x,y_{1:i}$ we feed the $x,y_{1:i-1}$ into $r_\phi$ and get the score of the sequence with all possible extensions of $y_i$ in the dictionary. 
The efficiency and performance of the reward model is not dependent on $k$, for top-k generation, as we simultaneously get the score for all possible next tokens in the dictionary. We use the same architecture as a causal language model, however, we use a novel training loss which we discuss next. 

% \AK{Need more elaboration regarding the architecture. Is it the same as in the base LLM? Or is it different? If the latter, in which ways? A schematic here will be useful.}

% \AK{Avoid one-sentence paragraphs.}

% Next we discuss the loss that we use for training the reward model.

\subsection{A Principled Constraint}

Given the sub-optimality of the existing methods, there needs to be a more principled way to score partial sequences. 
We propose to score partial sequences based on their optimal extension: 
%\AK{The notations seem to be different than from Sec. 2.}
Given a partial sequence $\rvy_{1:i}$, we consider all possible full extensions and assign the score of the highest completion to $\rvy_{1:i}$. 
Na\"{i}vely, this would require an exponential search in terms of the size of the vocabulary which is intractable. To make this principled goal feasible, we propose a local constraint that the partial-sequence reward model needs to satisfy so that it will return the reward of the corresponding optimal expansion: 
\begin{equation} \label{eq:constraint}
    \mathbf{V}_{\theta}(y_{1:i}|\rvx) = \max_{y_{i+1}} \: \mathbf{V}_{\theta}(y_{1:i+1}|\rvx)
\end{equation}
If the above local constraint is satisfied, then we can keep expanding the sequence as in the generation:
$$
\begin{aligned}
\mathbf{V}_{\theta}(y_{1:i}|\rvx) &= \max_{y_{i+1}} \: \mathbf{V}_{\theta}(y_{1:i+1}|\rvx) \\
&= \max_{y_{i+1}} \: \max_{y_{i+2}}\: \mathbf{V}_{\theta}(y_{1:i+2}|\rvx) \\
&= \cdots = \max_{y_{i+1:n}}\mathbf{V}_{\theta}(y_{1:n}|\rvx),
\end{aligned}
$$
where $y_{i+1:n}$ is the optimal extension beyond $y_{1:i}$ and $y_n$ is the EOS token. That is, instead of doing an exponential search, we could train the value function to satisfy \eqref{eq:constraint}, which can be done by Temporal Difference (TD) learning. Note that VAS also uses TD learning in their algorithm, but, since they use a conventional reward model they do not do a max over the dictionary.

To be more precise, the training process can be separated into two steps with distinct objectives:
\begin{enumerate}
    \item Standard BT loss on full sequence preference dataset:
        \begin{equation} \label{eq:bradley-terry-new}
        \mathcal{L}_{(a)} = - \E_{\rvx, \rvy^w, \rvy^l \sim \D} \log \sigma ( \mathbf{V}_{\theta} (\rvy^w|\rvx) - \mathbf{V}_{\theta} (\rvy^l|\rvx))
        \end{equation}
    \item Constraint to ensure optimal partial sequence expansion.
        \begin{equation} \label{eq:constraint_loss}
            \mathcal{L}_{(b)} = \frac{1}{2}\left[\mathbf{V}_{\theta}(y_{1:i}|\rvx) - \max_{y_{i+1}} \:\mathbf{V}_{\theta}(y_{1:i+1}|\rvx)\right]^2       
        \end{equation}
\end{enumerate}

Firstly, we want to point out the similarity of our constraint \eqref{eq:constraint_loss} to TD control where $V(\rvy|\rvx)$ can be treated as a state-action value function (i.e., Q-function) with $y_i$ corresponding to the action and $[\rvx,\rvy_{1:i-1}]$ corresponding to the state. Note also that transitions are deterministic in LLMs since the action $y_i$ updates the state to $[\rvx,\rvy_{1:i}]$ deterministically.  We use $s$ to denote a state and $a$ to denote an action in Bellman's equation:
    $$
    \begin{aligned}
        &Q^{*}(s,a) = \mathbb{E}[r|s,a] + \gamma \sum_{s'}\mathbb{P}(s'|s,a) \max_{a'} Q^{*}(s', a')\\
        \Rightarrow \: &Q^{*}([\rvx,\rvy_{1:i-1}], y_i) = \max_{y_{i+1}} Q^{*}([\rvx,\rvy_{1:i}], y_{i+1}) \label{eq:bellman}\\
        \Rightarrow \: &\mathbf{V}_{\theta}(\rvy_{1:i}|\rvx) = \max_{y_{i+1}} \: \mathbf{V}_{\theta}(\rvy_{1:i+1}|\rvx)
    \end{aligned}
    $$

Note that \cref{eq:bellman} follows from the fact that there is no discount factor and no reward until the end of the sequence.
Then $\mathcal{L}_{(b)}$ is the same loss as in Q-gradient learning by treating $\max\limits_{y_{i+1}} \:\mathbf{V}_{\theta}(y_{1:i+1}|\rvx)$ as the target. 

To train the value function, we alternate between the two losses mentioned previously. For the Bradley-Terry loss \eqref{eq:bradley-terry-new}, we utilize full-sequence preference pairs as commonly done when training a reward model. Furthermore, for the new constraint loss \eqref{eq:constraint_loss}, we extract partial sequences from the winning sequences in the preference dataset and use them as the training data. Notably, there is no preference signal when training with the constraint loss. The model simply learns to align its scores to the best next token. We trained the model by alternating between the two losses. The training details are presented in Appendix \ref{app:training}.

%Therefore, we need to train the constraint in an iterative backward manner: 
%we start to train on partial sequences that are derived by removing the last token of full sequences and use optimal full sequence reward score as the target. Then, we remove the last two tokens of full sequences while using the optimal score computed from the partial sequences derived by removing the last token of full sequences as the target.  For instance, we have a full sequence $(\rvx, y_{1:n})$ from the dataset. In the first iteration, we would train on the partial sequence $(\rvx, y_{1:n-1})$ and use $\max\limits_{y_n} \mathbf{V}_{\theta}(\rvx, y_{1:n})$ as the target. Then, in the second iteration, we train on partial sequence $(\rvx, y_{1:n-2})$ and use $\max\limits_{y_{n-1}} \mathbf{V}_{\theta}(\rvx, y_{1:n-1})$ as the target, as at this iteration the reward model could provide a reliable score for $\mathbf{V}_{\theta}(\rvx, y_{1:n-1})$. We keep iterating by reducing the length of the partial sequence. Also, the model is frozen when it is computing the target and updated at the end of each iteration.

% Since this technique is similar to Q-gradient learning which is a well-studied training paradigm, relevant training tricks such as alternative training and rebuffer play can be applied to optimize the performance.

We emphasize that this kind of training would \emph{not} be possible with the reward models of previous RGTG methods that require $|D|$ forward passes to calculate the max over all the tokens in the dictionary $D$. 
Instead we calculate the max after a single forward pass. The complete algorithm for our method is presented in Alg.~\ref{alg:rgtg_train}.


\begin{algorithm}[t]
  \small
  \caption{Our Training Algorithm.} %\AK{Always use \texttt{text} for words, e.g., loss, max, iter etc. in math mode.}
  \label{alg:rgtg_train}

  \begin{algorithmic}[1]
    \REQUIRE Base LLM to initialize the reward model $V_\theta$, Full Sequence Preference dataset \(\mathbf{D_{BT}} = \{ (\mathbf{x}^k, \mathbf{y}^{wk}, \mathbf{y}^{lk}) \}_{k=1}^{K_{BT}}\), number of alternating iterations $\text{iter}_n$, mini-batch size $n$,  partial sequence dataset \(\mathbf{D_{max}} = \{ (\mathbf{x}^k, \mathbf{y}^{k}\}_{k=1}^{K_\text{max}}\)
    \ENSURE $\mathbf{V}_{\theta}$
    \vspace{1em}
    
    \FOR{$i$ = 1 to $\text{iter}_n$}
    \STATE Sample minibatch $\mathbf{D_{BT}^{(i)}}$ from $\mathbf{D_{BT}}$ of size $n$
        \FOR{every tuple $(\rvx, \rvy^w, \rvy^l)\in \mathbf{D_{BT}^{(i)}}$}
        \STATE Compute $\mathbf{V}_{\theta} (\rvy^w|\rvx)$ and $\mathbf{V}_{\theta} (\rvy^l|\rvx)$
        \STATE $\mathcal{L}_a = \log \sigma ( \mathbf{V}_{\theta} (\rvy^w|\rvx) - \mathbf{V}_{\theta}(\rvy^l|\rvx))$
        \STATE Update $\mathbf{V}_{\theta}$ based on loss $\mathcal{L}_a$ 
        \ENDFOR
    \STATE Sample minibatch $\mathbf{D_{max}^{(i)}}$ from $\mathbf{D_{max}}$ of size $n$
        \FOR{every tuple $(\rvx, \rvy)\in \mathbf{D_{max}^{(i)}}$}
        \STATE Compute $\mathbf{V}_{\theta} (\rvy|\rvx)$
        \STATE $V_{max} = \max_{y_{|\rvy| + 1}} \:\mathbf{V}_{\theta}(\rvy, y_{|\rvy| + 1}|\rvx)$
        \STATE $\mathcal{L}_b = \frac{1}{2}\left[\mathbf{V}_{\theta} (\rvy|\rvx) - V_\text{max}\right]^2$
        \STATE Update $\mathbf{V}_{\theta}$ based on loss $\mathcal{L}_b$
        \ENDFOR
    \ENDFOR
    % \STATE   $\mathbf{V}_{\theta}$
  \end{algorithmic}
\end{algorithm}

\begin{algorithm}[t]
    \small
    \caption{Our Decoding Algorithm.}
    \label{alg:rgtg_decode}

    \begin{algorithmic}[1]
        \REQUIRE Reward model $\mathbf{V}_{\theta}$, Prompt $\rvx$, top-k parameter $k$, hyperparameter $\beta > 0$, any reference/SFT model $\piref$, generation length $l$
        \ENSURE $\rvy_{1:l}$: A generated response to $\rvx$ of length $l$

        \vspace{1em}
        
        \FOR{i = 1 to $l$}
            %\State Reward $\mathbf{V}_{\theta}(v \vert \rvx, \rvy_{1:i-1})$
            %\State Logit $\piref(v \vert \rvx, \rvy_{1:i-1})$
            \STATE $\log \pi(y_i=v \vert \rvx, \rvy_{1:i-1}) \leftarrow$
            \STATE \qquad\qquad $\log \left(\piref(v \vert \rvx, \rvy_{1:i-1}) + \beta \mathbf{V}_{\theta}(v \vert \rvx, \rvy_{1:i-1})\right)$
            \STATE $y_{i} \sim \softmax(\mathtt{top\_k}(\log \pi(y_i \vert \rvx, \rvy_{1:i-1})))$
        \ENDFOR  
        % \STATE  $\rvy_{1:l}$
    \end{algorithmic}
\end{algorithm}

We now prove that unlike PARGS and CD, our algorithm is guaranteed to prefer prefixes that are extendable to optimal full sequences.


\begin{theorem}
In the limit of infinite training data and a sufficiently expressive representation for the value function, our algorithm guarantees that the learned value function scores prefixes that can be extended to optimal full sequences at least as high as any other prefix.  More precisely, if $\rvy^* = \argmax_{\rvy} r(\rvy|\rvx)$, then 
\begin{equation}
V(\rvy^*_{1:i}|\rvx) \ge V(\rvy'_{1:j}|\rvx) \; \forall i,j,\rvy'
\end{equation}
\end{theorem}

\begin{proof}
We provide a proof by contradiction.  Let $\rvy^*$ be an optimal response to $\rvx$ and $\rvy'$ be any other response.  Suppose that
\begin{equation}
    \exists i,j,\rvy' \mbox{ such that } V(\rvy'_{1:j}|\rvx) > V(\rvy^*_{1:i}|\rvx) \label{eq:hypothesis}
\end{equation}  
Since the loss in \eqref{eq:constraint_loss} ensures that the learned value function returns the reward of the best full sequence that extends a prefix then $V(\rvy^*_{1:i}|\rvx) = r(\rvy^*|\rvx)$.  Similarly, since $\rvy'_{1:j}$ is any other prefix whose extensions do not lead to better full sequences, then $V(\rvy'_{1:j}|\rvx) <= r(\rvy^*|\rvx)$.  This means that $V(\rvy'_{1:j}|\rvx) \le V(\rvy^*_{1:i}|\rvx)$, which contradicts \eqref{eq:hypothesis}.
\end{proof}




\section{Related Work}
\label{sec:proposal}
\section{Related Work}
% \gabis{I prefer to get to our new stuff ASAP, so if these aren't crucial for understanding our work, I vote to move this section to the end of the paper.}

\paragraph{Information Extraction} Previous works have focused on extracting basic result tuples (e.g., task, dataset, metric) from scientific literature~\citep{singh2019automated, hou2019identification, kardas2020axcell, yang2022telin, bai2023schema, singh2024legobench, csahinucc2024efficient, kabongo2024orkg}. 
Our extraction pipeline improves upon this approach in two significant ways: it extracts enriched tuples that include prompting-related attributes and generates detailed dataset descriptions by leveraging LLM and automatically linked source papers.
Hence, unlike previous works that primarily compiled leaderboard tables, our enhanced extraction pipeline enables deeper review analysis, contributing to the broader goal of AI-driven scientific discovery~\citep{xu2021artificial, majumder2024discoverybench, m2024augmenting}.

\paragraph{LLM \& Prompting} Our study focuses on extracting experimental results of frontier proprietary LLMs~\citep{achiam2023gpt, anthropic@claude, team2023gemini}, with a specific emphasis on target attributes that incorporate information about prompting methods~\citep{brown2020language, wei2022chain, min2022rethinking}. 
In the context of prompting, prior studies have systematically analyzed the mechanisms behind prompting methods, focusing either on the use of in-context examples~\citep{min2022rethinking,lampinen2022can,weber2023mind, zhang2022robustness} or techniques that elicit reasoning, such as CoT prompting~\citep{wei2022chain, shaikh2023second, wang2023towards, turpin2024language, sprague2024cot, liu2024mind}.  
Conversely, we examine the model's behavior by conducting a literature analysis, which compiles data from scientific sources to reveal insights.


\paragraph{Literature Analysis}

Literature analysis systematically aggregates and examines data from multiple independent studies on a given topic to derive more precise and reliable conclusions. 
It has been widely applied in the biomedical domain for identifying target materials or clinical records~\citep{bao2019using, yun2024automatically}. 
In the NLP domain, review analysis has been used for metric standardization~\citep{reiter2018structured}, literature review~\citep{santu2024prompting, du2024llms}, and assessing evaluation criteria for specific domains~\citep{ostheimer2023call}. 
In contrast, our work employs a review analysis approach to evaluate the behavior of LLMs.
In the context of LLMs, \citet{asai2024openscholar} utilizes retrieval-augmented language models to synthesize scientific literature. 
However, this approach can only process a limited number of documents during retrieval to synthesize.
The work by~\citet{sprague2024cot} is perhaps the most closely related to ours. 
They conducted a review analysis through a literature survey to examine the effectiveness of CoT prompting for LLMs. 
However, their study is focused on CoT prompting, conducted on a limited scale, and relies on manual extraction methods.






% \paragraph{Regression} In our study, we perform a regression task to forecast the performance of an unseen dataset on a specific model, given a particular prompting setup. 
% Similarly, \citet{xia2020predicting} employed traditional regression models~\citep{chen2016xgboost} to predict the performance of translation tasks, treating it more like a table-in-filling task.
% More recently, studies have investigated regression tasks with LLMs, mainly focusing on AutoML or predefined numeric functions~\citep{song2024omnipred, vacareanu2024words, nguyen2024predicting, requeima2024llm, tang2024understanding}. 
% However, our study is notably different as it uses textual descriptions of datasets and prompting setups as input rather than numeric features.

\section{Experiments}
\label{sec:experiments}
\section{Proof of Concept Experiments}
\label{sec:experiments}

%\begin{itemize}
%    \item joint exploration non e' spesso un opzione
%    \item specificare che le policy sono decentralizzate a differenza di tutti i casi precedenti
%    \item decentralizzata con feedback decentralizzato non si coordina e il problema e' abbastanza semplice da portare a policy quasi deterministiche
%\end{itemize}



%\mirco{questo primo paragrafo è un po' convoluto. Prova a ristruttura la sezione in questo modo: quali sono le domande a cui cerchiamo risposta? Quali sono i domini sperimentali? Quali sono gli algoritmi che compariamo? Quali sono i take away? Per l'ultimo potresti anche evidenziare qualche frase in grassetto o emph con le principali conclusioni empiriche}

In this section, we provide some empirical validations of the findings discussed so far. Especially, we aim to answer the following questions: (\textbf{a}) Is Algorithm~\ref{alg:trpe} actually capable of optimizing finite-trials objectives? (\textbf{b}) Do different objectives enforce different behaviors, as expected from Section~\ref{sec:problem_formulation}? (\textbf{c}) Does the \emph{clustering} behavior of mixture objectives play a crucial role? If yes, when and why?\\
Throughout the experiments, we will compare the result of optimizing finite-trial objectives, either joint, disjoint, mixture ones, through Algorithm~\ref{alg:trpe} via fully decentralized policies. The experiments will be performed with different values of the exploration horizon $T$, so as to test their capabilities in different exploration efficiency regimes.\footnote{The exploration horizon $T$, rather than being a given trajectory length, has to be seen as a parameter of the exploration phase which allows to tradeoff exploration quality with exploration efficiency.} The full implementation details are reported in Appendix~\ref{apx:exp}.
\vspace{-6pt}
\paragraph*{Experimental Domains.}~The experiments were performed on two domains. The first is a notoriously difficult multi-agent exploration task called \emph{secret room}~\citep[MPE,][]{pmlr-v139-liu21j},\footnote{We highlight that all previous efforts in this task employed centralized policies. We are interested on the role of the entropic feedback in fostering coordination rather than full-state conditioning, then maintaining fully decentralized policies instead.} referred to as  Env.~(\textbf{i}). In such task, two agents are required to reach a target while navigating over two rooms divided by a door. In order to keep the door open, at least one agent have to remain on a switch. Two switches are located at the corners of the two rooms. The hardness of the task then comes from the need of coordinated exploration, where one agent allows for the exploration of the other. The second is a simpler exploration task yet over a high dimensional state-space, namely a 2-agent instantiation of \emph{Reacher}~\citep[MaMuJoCo,][]{peng2021facmac}, referred to as Env.~(\textbf{ii}). Each agent corresponds to one joint and equipped with decentralized policies conditioned on her own states. In order to allow for the use of plug-in estimator of the entropy~\citep{paninski2003}, each state dimension was discretized over 10 bins.


\begin{figure*}[!]
    \centering
    \begin{tikzpicture}
    % Draw rounded box for the legend
    \node[draw=black, rounded corners, inner sep=2pt, fill=white] (legend) at (0,0) {
        \begin{tikzpicture}[scale=0.8]
            % Mixture
            \draw[thick, color={rgb,255:red,230; green,159; blue,0}, opacity=0.8] (0,0) -- (1,0);
            \fill[color={rgb,255:red,230; green,159; blue,0}, opacity=0.2] (0,-0.1) rectangle (1,0.1);
            \node[anchor=west, font=\scriptsize] at (1.2,0) {Mixture};
            
            % Joint
            \draw[thick, dashed, color={rgb,255:red,86; green,180; blue,233}, opacity=0.8] (2.5,0) -- (3.5,0);
            \fill[color={rgb,255:red,86; green,180; blue,233}, opacity=0.2] (2.5,-0.1) rectangle (3.5,0.1);
            \node[anchor=west, font=\scriptsize] at (3.7,0) {Joint};
            
            
            % Disjoint
            \draw[thick, dotted, color={rgb,255:red,204; green,121; blue,167}, opacity=0.8] (4.7,0) -- (5.7,0);
            \fill[color={rgb,255:red,204; green,121; blue,167}, opacity=0.2] (4.7,-0.1) rectangle (5.7,0.1);
            \node[anchor=west, font=\scriptsize] at (5.9,0) {Disjoint};
            
            % Uniform
            \draw[thick, color={rgb,255:red,153; green,153; blue,153}, opacity=0.8] (7.2,0) -- (8.2,0);
            \fill[color={rgb,255:red,153; green,153; blue,153}, opacity=0.2] (7.2,-0.1) rectangle (8.2,0.1);
            \node[anchor=west, font=\scriptsize] at (8.4,0) {Random Initialization};
        \end{tikzpicture}
    };
\end{tikzpicture}

    %\hfill
    \vfill
    %vspace{-0.2cm}
    \begin{subfigure}[b]{0.3\textwidth}
        \includegraphics[width=\textwidth]{figures/room_150_AverageReturnnokl.pdf}
        %\vspace{-0.8cm}
        \caption{\centering MA-TRPO with TRPE Pre-Training (Env.~(\textbf{i}), $T=150$).}
        \label{subfig:image9}
    \end{subfigure}
    \hfill
    \begin{subfigure}[b]{0.3\textwidth}
        \includegraphics[width=\textwidth]{figures/room_50_AverageReturnnokl.pdf}
        %\vspace{-0.8cm}
        \caption{\centering MA-TRPO with TRPE Pre-Training (Env.~(\textbf{i}), $T=50$).}
        \label{subfig:image10}
    \end{subfigure}
    \hfill
    \begin{subfigure}[b]{0.3\textwidth}
        \centering
        \includegraphics[width=0.8\textwidth]{figures/hand_100_AverageReturn.pdf}
        %\vspace{-0.8cm}
        \caption{\centering MA-TRPO with TRPE Pre-Training (Env.~(\textbf{ii}), $T=100$).}
        \label{subfig:image11}
    \end{subfigure}
\caption{\centering Effect of pre-training in sparse-reward settings.(\emph{left}) Policies initialized with either Uniform or TRPE pre-trained policies over 4 runs over a worst-case goal. (\emph{rigth}) Policies initialized with either Zero-Mean or TRPE pre-trained policies over 4 runs over 3 possible goal state. We report the average and 95\% c.i.}
\label{fig:pretraining}
\end{figure*}
\vspace{-10pt}
\paragraph*{Task-Agnostic Exploration.}~Algorithm~\ref{alg:trpe} was first tested in her ability to address task-agnostic exploration \emph{per se}. This was done by considering the well-know hard-exploration task of Env.~(\textbf{i}). The results are reported in Figure~\ref{fig:room} for a short exploration horizon $(T=50)$. Interestingly, at this efficiency regime, when looking at the joint entropy in Figure~\ref{subfig:image2}, joint and disjoint objectives perform rather well compared to mixture ones in terms of induced joint entropy, while they fail to address mixture entropy explicitly, as seen in Figure~\ref{subfig:image3}. On the other hand mixture-based objectives result in optimizing both mixture \emph{and} joint entropy effectively, as one would expect by the bounds in Th.~\ref{lem:entropymismatch}. By looking at the actual state visitation induced by the trained policies, the difference between the objectives is apparent. While optimizing joint objectives, agents exploit the high-dimensionality of the joint space to induce highly entropic distributions even without exploring the space uniformly via coordination (Fig.~\ref{subfig:image5}); the same outcome happens in disjoint objectives, with which agents focus on over-optimizing over a restricted space loosing any incentive for coordinated exploration (Fig.\ref{subfig:image6}). On the other hand, mixture objectives enforce a clustering behavior (Fig.\ref{subfig:image6}) and result in a better efficient exploration. 

\paragraph*{Policy Pre-Training via Task-Agnostic Exploration.}~More interestingly, we tested the effect of pre-training policies via different objectives as a way to alleviate the well-known hardness of sparse-reward settings, either throught faster learning or zero-short generalization. In order to do so, we employed a multi-agent counterpart of the TRPO algorithm~\cite{schulman2017trustregionpolicyoptimization} with different pre-trained policies. First, we investigated the effect on the learning curve in the hard-exploration task of Env.~(\textbf{i}) under long horizons ($T=150$), with a worst-case goal set on the the opposite corner of the closed room. Pre-training via mixture objectives still lead to a faster learning compared to initializing the policy with a uniform distribution. On the other hand, joint objective pre-training did not lead to substantial improvements over standard initializations. More interestingly, when extremely short horizons were taken into account ($T=50$) the difference became appalling, as shown in Fig.~\ref{subfig:image9}: pre-training via mixture-based objectives leaded to faster learning and higher performances, while pre-training via disjoint objectives turned out to be even \emph{harmful} (Fig.~\ref{subfig:image10}). This was motivated by the fact that the disjoint objective overfitted the task over the states reachable without coordinated exploration, resulting in almost deterministic policies, as shown in Fig~\ref{fig:333} in Appendix~\ref{apx:exp}. Finally, we tested the zero-shot capabilities of policy pre-training on the simpler but high dimensional exploration task of Env.~(\textbf{ii}), where the goal was sampled randomly between worst-case positions at the boundaries of the region reachable by the arm. As shown in Fig.~\ref{subfig:image11}, both joint and mixture were able to guarantee zero-shot performances via pre-training compatible with MA-TRPO after learning over $2$e$4$ samples, while disjoint objectives were not. On the other hand, pre-training with joint objectives showed an extremely high-variance, leading to worst-case performances not better than the ones of random initialization. Mixture objectives on the other hand showed higher stability in guaranteeing compelling zero-shot performance.
\vspace{-6pt}
\paragraph*{Take-Aways.}~Overall, the proposed proof of concepts experiments managed to answer to all of the experimental questions: (\textbf{a}) Algorithm~\ref{alg:trpe} is indeed able to explicitly optimize for finite-trial entropic objectives. Additionally, (\textbf{b}) \textbf{mixture distributions enforce diverse yet coordinated exploration}, that helps when high efficiency is required. Joint or disjoint objectives on the other hand may fail to lead to relevant solutions because of under or over optimization. Finally, (\textbf{c}) \textbf{efficient exploration} enforced by mixture distributions was shown to be a \textbf{crucial factor} not only for the sake of task-agnostic exploration per se, but also for the ability of \textbf{pre-training via task-agnostic exploration} to lead to \textbf{faster and better training} and even \textbf{zero-shot generalization}.

\section{Conclusion}
\label{sec:conclusion}
\section{Conclusion}
We present live monitoring and mid-run interventions for multi-agent systems. We demonstrate that monitors based on simple statistical measures can effectively predict future agent failures, and these failures can be prevented by restarting the communication channel. Experiments across multiple environments and models show consistent gains of up to 17.4\%-20\% in system performance, with an addition in inference-time compute.
Our work also introduces \ourenv{}, a new environment for studying multi-agent cooperation.



% \bibliography{icml2025}
% \bibliographystyle{icml2025}

\clearpage


% In the unusual situation where you want a paper to appear in the
% references without citing it in the main text, use \nocite
% \nocite{langley00}

\section*{Impact Statement}


The goal of this paper is to push the frontiers of Machine Learning. This may lead to impacts on the society, however, we do not feel the need to highlight them here.

\bibliography{icml2025}
\bibliographystyle{icml2025}


%%%%%%%%%%%%%%%%%%%%%%%%%%%%%%%%%%%%%%%%%%%%%%%%%%%%%%%%%%%%%%%%%%%%%%%%%%%%%%%
%%%%%%%%%%%%%%%%%%%%%%%%%%%%%%%%%%%%%%%%%%%%%%%%%%%%%%%%%%%%%%%%%%%%%%%%%%%%%%%
% APPENDIX
%%%%%%%%%%%%%%%%%%%%%%%%%%%%%%%%%%%%%%%%%%%%%%%%%%%%%%%%%%%%%%%%%%%%%%%%%%%%%%%
%%%%%%%%%%%%%%%%%%%%%%%%%%%%%%%%%%%%%%%%%%%%%%%%%%%%%%%%%%%%%%%%%%%%%%%%%%%%%%%
\newpage
\appendix
\onecolumn

\label{sec:appendix}
\appendix
% \setcounter{table}{0}
% \renewcommand*{\thetable}{\arabic{table}}
% \renewcommand*{\thefigure}{\arabic{figure}}
\section{Related algorithms and metric caculation}
\label{app:related_algo_metric}

\subsection{Performance-energy Consistency} In this paper, performance-energy consistency refers to the consistency between the results evaluated using an energy model and those evaluated using real-world metrics for the same sample. Specifically, the consistency requires that good samples are assigned low energy, while poor samples are assigned high energy. Performance-energy consistency measures the proportion of element pairs that maintain the same relative order in both permutations \( X \) and \( Y \), where \( X \) and \( Y \) represent the index arrays obtained by sorting the original energy values \( \mathbf{E} = (E_1, E_2, \dots, E_N) \) and performance metric values \( \mathbf{P} = (P_1, P_2, \dots, P_N) \), respectively, in ascending order. In this paper, the energy values are calculated by energy model $E_\theta(x_0)$ for samples $\x_0$. The performance metric values are calculated as the L2 distance between the generated samples $\x_0$ and the ground truth under the given condition.

Let \( X = (X_1, X_2, \dots, X_N) \) and \( Y = (Y_1, Y_2, \dots, Y_N) \) be the index arrays obtained by sorting the original energy values \( \mathbf{E} = (E_1, E_2, \dots, E_N) \) and performance metric values \( \mathbf{P} = (P_1, P_2, \dots, P_N) \), respectively, in ascending order. Specifically, \( X_i \) is the rank of the \( i \)-th sample in the sorted energy values \( \mathbf{E} \), and \( Y_i \) is the rank of the \( i \)-th sample in the sorted performance metric values \( \mathbf{P} \).

\textbf{Consistency Definition:}
The \textbf{consistency} is defined as the proportion of consistent pairs \( (i, j) \) where \( i < j \) and the relative order of \( i \) and \( j \) in \( X \) is the same as in \( Y \). Specifically:
\[
\text{Consistency} = \frac{1}{\binom{N}{2}} \sum_{i=1}^{N-1} \sum_{j=i+1}^{N} \mathbb{I}\left( (X_i < X_j \land Y_i < Y_j) \lor (X_i > X_j \land Y_i > Y_j) \right),
\]
where:
\begin{itemize}
    \item \( \binom{N}{2} = \frac{N(N-1)}{2} \) is the total number of pairs \( (i, j) \) with \( i < j \),
    \item \( \mathbb{I}[\cdot] \) is the indicator function, which evaluates to 1 if the condition inside the brackets holds (i.e., the relative order is consistent), and 0 otherwise.
\end{itemize}
\subsection{Adversarial sampling}
During the sampling process, energy optimization often gets trapped in local minima or incorrect global minima, making it difficult to escape and hindering the sampling of high-quality samples.
\subsection{Negative Sample Generation} Negative samples are generated by introducing noise into the positive sample \( x_0 \). In the Maze and Sudoku experiments, permutation noise is applied to the channel dimension to induce significant changes in the solution. Other noise types can be used, as this remains a hyperparameter choice. Specifically, we first randomly sample two scalars \( p_1 \) and \( p_2 \) from a uniform distribution in the interval \( [0, 1] \), i.e., \( p_1, p_2 \sim \text{Uniform}(0, 1) \) ($p_1<p_2$). Then, for each channel position of the positive sample \( x_0 \), we swap the channel positions with probabilities \( p_1 \) and \( p_2 \), resulting in \( x_0^{-} \) and \( x_0^{--} \), such that the L2 distance between \( x_0^{-} \) and \( x_0 \) is smaller than the L2 distance between \( x_0^{--} \) and \( x_0 \). For other noise types, such as Gaussian noise, we normalize the L2 norm of the noise and apply noise at different scales to ensure that the L2 distance from \( x_0^{-} \) to \( x_0 \) is smaller than the L2 distance from \( x_0^{--} \) to \( x_0 \).


\subsection{Linear-regression algorithm} Given three points \((x_1, y_1)\), \((x_2, y_2)\), and \((x_3, y_3)\), we wish to fit a line of the form ~\cite{lane2003introduction}:

\[
y = kx + b
\]
The mean of the \(x\)-coordinates and the mean of the \(y\)-coordinates are:
\[
\bar{x} = \frac{1}{3}(x_1 + x_2 + x_3), \quad \bar{y} = \frac{1}{3}(y_1 + y_2 + y_3)
\]
The slope \(k\) of the best-fit line is given by the formula:

\[
k = \frac{\sum_{i=1}^{3} (x_i - \bar{x})(y_i - \bar{y})}{\sum_{i=1}^{3} (x_i - \bar{x})^2}
\]
This formula represents the least-squares solution for the slope.
Once the slope \(k\) is determined, the intercept \(b\) can be calculated as:
\[
b = \bar{y} - k\bar{x}
\]
The equation of the best-fit line is:
\[
\hat{y} = kx + b
\]
\section{Details of experiments}
\label{app:Exp_detail}
\subsection{Detais of Sudoku experiments}
\label{app:Exp_sudoku}
For Sudoku experiment, the dataset, model architecture, and training configurations are adopted from \citet{du2024learning}. We mainly use solving success rate to evaluate different models. Model backbone and training configurations can be found in Fig. \ref{fig:sudoku_ebm} and Table \ref{tab:sudoku_exp_detail}, respectively. All the exploration hyperparameters $c$ are set as 100 for Sudoku task.
\begin{figure}[H]
\begin{minipage}{0.9\textwidth}
\centering
\small
\begin{tabular}{c}
    \toprule
    3x3 Conv2D, 384 \\
    \midrule
    Resblock 384 \\
    \midrule
    Resblock 384 \\
    \midrule
    Resblock 384 \\
    \midrule
    Resblock 384 \\
    \midrule
    Resblock 384 \\
    \midrule
    Resblock 384 \\
    \midrule
    3x3 Conv2D, 9 \\ 
    \bottomrule
\end{tabular}
\caption{The model architecture for \proj on Sudoku task. The energy value is computed using the L2 norm of the final predicted output similar to \citet{du2023reduce}, while the output is directly used as noise prediction for the diffusion baseline.}
\label{fig:sudoku_ebm}
\end{minipage}
\end{figure}
\begin{table}[ht]
  \begin{center}
    \caption{\textbf{Details of  training for Sudoku task}. }
    \vskip -0.15in
    \label{tab:2d_model_architecture}
    \begin{tabular}{l|c} % <-- Alignments: 1st column left, 2nd middle and 3rd right, with vertical lines in between
    \multicolumn{2}{l}{}\\
      \hline
       \multicolumn{1}{l|}{Training configurations } & \multicolumn{1}{l}{}\\
      \hline
      Number of training steps & 100000  \\
      Training batch size & 64 \\
      Learning rate & 0.0001 \\
      Diffusion steps & 10 \\
      Inner loop optimization steps & 20 \\
      Denoising loss type & MSE \\
      Optimizer & Adam \\
        \hline
    \end{tabular}
      \label{tab:sudoku_exp_detail}
  \end{center}
\end{table}
\subsection{Details of Maze experiments}
\label{app:Exp_maze}
The details of maze experiments and model backbone are provided in Table \ref{tab:maze_exp_detail} and Fig. \ref{fig:maze_ebm}, respectively. The key metric, the maze-solving success rate is defined as the proportion of model-generated paths that have no breakpoints, do not overlap with walls, and begin and end at the start and target points, respectively. Maze datasets are generated by \citet{ivanitskiy2023configurable}, and detailed hyperparameter configurations are in Table \ref{tab:maze_exp_detail}. All the exploration hyperparameters $c$ are set as 100 for Maze task.
\begin{figure}[H]
\begin{minipage}{0.9\textwidth}
\centering
\small
\begin{tabular}{c}
    \toprule
    3x3 Conv2D, 384 \\
    \midrule
    Resblock 384 \\
    \midrule
    Resblock 384 \\
    \midrule
    Resblock 384 \\
    \midrule
    Resblock 384 \\
    \midrule
    Resblock 384 \\
    \midrule
    Resblock 384 \\
    \midrule
    3x3 Conv2D, 9 \\ 
    \bottomrule
\end{tabular}
\caption{The model architecture for \proj on Maze task. The energy value is computed using the L2 norm of the final predicted output similar to \citet{du2023reduce}, while the output is directly used as noise prediction for the diffusion baseline.}
\label{fig:maze_ebm}
\end{minipage}
\end{figure}
\begin{table}[ht]
  \begin{center}
    \caption{\textbf{Details of Maze dataset, training}. }
    \vskip -0.15in
    \label{tab:2d_model_architecture}
    \begin{tabular}{l|c} % <-- Alignments: 1st column left, 2nd middle and 3rd right, with vertical lines in between
    \multicolumn{2}{l}{}\\
      \hline
      \multicolumn{1}{l|}{Dataset:} & \multicolumn{1}{l}{}\\ 
      \hline
      Size of training dataset with grid size 4 & 10219   \\
      Size of training dataset with grid size 5 & 9394   \\
      Size of training dataset with grid size 6 & 10295  \\
      Minimum length of solution path & 5 \\
      Algorithm to generate the maze & DFS \\
      Size of test dataset with grid size 6 & 837   \\
      Size of test dataset with grid size 8 & 888   \\
      Size of test dataset with grid size 10 & 948   \\
      Size of test dataset with grid size 12 & 960   \\
      Size of test dataset with grid size 15 & 975   \\
      Size of test dataset with grid size 20 & 978   \\
      Size of test dataset with grid size 30 & 994   \\
      \hline
       \multicolumn{1}{l|}{Training configurations } & \multicolumn{1}{l}{}\\
      \hline
      Number of training steps & 200000  \\
      Training batch size & 64 \\
      Learning rate & 0.0001 \\
      Diffusion steps & 10 \\
      Inner loop optimization steps & 20 \\
      Denoising loss type & MSE + MAE \\
      Optimizer & Adam \\
        \hline
    \end{tabular}
      \label{tab:maze_exp_detail}
  \end{center}
\end{table}

\section{Performance sensitivity to hyperparameters}
\label{app:hyperparameters_sensitivity}

% inner loop opt steps, mcts noise scale(original model, mixed trained model hMCTS & Random search) more visualizations?
In this subsection, we analyze the impact of several hyperparameters on the experimental results. As shown in Table \ref{tab:maze_noise_scale}, the influence of different noise scales on the performance of various methods is presented. The hMCTS denoising and random search require a relatively larger noise scale to better expand the search space and improve final performance, while the diffusion model with naive inference performs best with a smaller noise scale. As demonstrated in Table \ref{tab:maze_inner_loop_opt} and Fig. \ref{fig:maze_opt_step}, the effect of varying inner-loop optimization steps on the results is also analyzed. It can be observed that performance improves gradually with an increasing number of steps, and after 5 steps, the performance stabilizes and the improvement slows down. Therefore, we chose 5 inner-loop optimization steps for the Maze experiments in this paper.
\begin{figure}[h!]
\vskip 0.2in
\begin{center}
\centerline{\includegraphics[width=0.55\textwidth]{fig/maze_optimization_steps_vs_values.pdf}}
\caption{Visualization of success rate across different number of inner-loop optimization steps on Maze with grid size $\mathbf{15\times15}$. }
\label{fig:maze_opt_step}
\end{center}
\vskip -0.2in
\end{figure}
\begin{table}[ht]
\caption{Success rate across the different number of inner-loop optimization step on Maze with grid size \textbf{15}. }
\label{tab:maze_inner_loop_opt}
\vskip 0.15in
\begin{center}
\resizebox{0.85\textwidth}{!}{ % Resize the table to fit within a single column
\begin{tabular}{l|cccccccccc}
\toprule
 &\multicolumn{10}{c}{\textbf{Number of optimization step}} \\
\cmidrule(lr){2-11} 
\textbf{Methods}                & 1               & 2               & 3               & 4               & 5               & 6               & 7               & 8               & 9 &10        \\
\midrule
T-SCEND tr. (ours), Naive inference & 0.0000 & 0.1562 & 0.2109 & 0.2734 & 0.2812 & 0.2734 & 0.2812 & 0.2969 & 0.2969 & 0.2969\\
\bottomrule
\end{tabular}
}
\end{center}
\vskip -0.1in
\end{table}
\begin{table}[ht]
\caption{Success rate across different noise scales on Maze with grid size \textbf{15}. }
\label{tab:maze_noise_scale}
\vskip 0.15in
\begin{center}
\resizebox{1\textwidth}{!}{ % Resize the table to fit within a single column
\begin{tabular}{l|cccccccccc}
\toprule
 &\multicolumn{10}{c}{\textbf{Noise scale}} \\
\cmidrule(lr){2-11} 
\textbf{Methods}                & 0.1               & 0.2               & 0.3               & 0.4               & 0.5               & 0.6               & 0.7               & 0.8               & 0.9 &1.0        \\
\midrule
T-SCEND tr. (ours), hMCTS denoising (energy)               & 0.3828 & 0.4375 & 0.5312 & 0.6094 & 0.6562 & 0.6953 & 0.7031 & 0.7344 & 0.7734 & 0.7969 \\
T-SCEND tr. (ours), naive inference                    & 0.3125 & 0.2656 & 0.2578 & 0.2344 & 0.2422 & 0.2656 & 0.2578 & 0.2422 & 0.2500 & 0.2500 \\
T-SCEND tr. (ours), Random search(energy)      & 0.3906 & 0.4453 & 0.5312 & 0.5703 & 0.5938 & 0.6328 & 0.6641 & 0.6719 & 0.6797 & 0.6562 \\
\bottomrule
\end{tabular}
}
\end{center}
\vskip -0.1in
\end{table}
\section{Additional results}
\label{app:additional_results}
\begin{figure}[h!]
\vskip 0.2in
\begin{center}
\centerline{\includegraphics[width=0.6\textwidth]{fig/maze_success_rate_vs_ts.pdf}}
\caption{Visualization of Success rate across different MCTS start step $t_s$. }
\label{fig:maze_success_rate_vs_ts}
\end{center}
\vskip -0.2in
\end{figure}
The parameter \( t_s \) controls the proportion of the total inference budget allocated to MCTS denoising. When \( t_s = 9 \), it means only MCTS denoising is used, while \( t_s = 0 \) means only best-of-N random search is employed. For \( 0 < t_s < 9 \), hMCTS denoising is applied. As shown in Table \ref{tab:maze_mcts_start_step} and Fig. \ref{fig:maze_success_rate_vs_ts}, there is a noticeable peak in model performance as \( t_s \) varies.
\begin{table}[h!]
\caption{Success rate of hMCTS denoising on Maze with grid size \textbf{15} across different MCTS start steps. }
\label{tab:maze_mcts_start_step}
\vskip 0.15in
\begin{center}
\resizebox{\textwidth}{!}{ % Resize the table to fit within a single column
\begin{tabular}{l|cccccccccc}
\toprule
\cmidrule(lr){2-11} 
\textbf{Methods} & 0               & 1               & 2               & 3               & 4               & 5               & 6               & 7               & 8               & 9         \\
\midrule
Original, hMCTS denoising (energy)      & 0.0781 & 0.0703 & 0.0859& 0.0781 & 0.1250& 0.1484& 0.1250 & 0.0781 & 0.0625 & 0.0703\\
T-SCEND tr. (ours), hMCTS denoising (energy)   & 0.6562 & 0.6094& 0.6641 & 0.7969 & 0.7969 & 0.6406& 0.4922 & 0.4922 & 0.4609 & 0.4453 \\
\bottomrule
\end{tabular}
}
\end{center}
\vskip -0.1in
\end{table}

\begin{table}[h!]
\caption{Success rate of Random search for different training methods on Maze with grid size \textbf{15} and Sudoku harder dataset guided with ground truth accuracy. Untrained, Random search (gt) represents use ground truth to guide the random search.  Here, $L=N$. Bold font denotes the best model. }
\label{tab:maze_diffus_baseline_diversity}
\vskip 0.15in
\begin{center}
\resizebox{\textwidth}{!}{ % Resize the table to fit within a single column
\begin{tabular}{l|cccccc|ccccccc}
\toprule
\multicolumn{1}{c|}{} & \multicolumn{6}{c}{\textbf{Maze success rate}} & \multicolumn{7}{c}{\textbf{Sudoku success rate}}\\ 
\cmidrule(lr){2-14} 
\textbf{Methods} & \textbf{$N$=1} & \textbf{$N$=11} & \textbf{$N$=21} & \textbf{$N$=41} & \textbf{$N$=81} & \textbf{$N$=161} &\textbf{$N$=1} & \textbf{$N$=11} & \textbf{$N$=21} & \textbf{$N$=41} & \textbf{$N$=81} & \textbf{$N$=161} & \textbf{$N$=321} \\
\midrule
Untrained, Random search (gt) & 0.0000 & 0.0000 & 0.0000 & 0.0000 & 0.0000 & 0.0000 & 0.0000 & 0.0000 & 0.0000 & 0.0000 & 0.0000 & 0.0000 & 0.0000 \\ 
Original, Random search (gt) & 0.0625 & 0.1250 & 0.1094 & 0.1328 & 0.1719 & 0.1719 & 0.0859 & 0.1641 & 0.2188 & 0.2344 & 0.2422 & 0.2656 & 0.2969 \\
DDPM, Random search (gt) & 0.0312&0.1094&0.1587&0.1746&0.2031&0.2422& 0.0000          & 0.0000          & 0.0000          & 0.0000          & 0.0000          & 0.0000          & 0.0156 \\
T-SCEND tr. w/o LRNCL, Random search (gt) & \textbf{0.2500} & \textbf{0.5078} & \textbf{0.5938} & \textbf{0.6562} & \textbf{0.7109} & \textbf{0.7422} & \textbf{0.1094} & \textbf{0.2578} & \textbf{0.2969} & \textbf{0.3438} & \textbf{0.3750} & \textbf{0.3828} & \textbf{0.4219} \\ 
\bottomrule
\end{tabular}
}
\end{center}
\vskip -0.1in
\end{table}
\begin{table}[h!]
\caption{Success rate and element-wise accuracy of Random search for different training methods on Sudoku harder dataset guided with ground truth accuracy. Here, $L=N$. Bold font denotes the best model. }
\label{tab:sudoku_diffus_baseline_ddpm}
\vskip 0.15in
\begin{center}
\resizebox{1\textwidth}{!}{ % Resize the table to fit within a single column
\begin{tabular}{l|ccccccc|ccccccc}
\toprule
& \multicolumn{7}{c|}{\textbf{Success rate}} & \multicolumn{7}{c}{\textbf{Element-wise} accuracy}\\
\cmidrule(lr){2-15} 
Methods &\textbf{$N$=1} & \textbf{$N$=11} & \textbf{$N$=21} & \textbf{$N$=41} & \textbf{$N$=81} & \textbf{$N$=161} & \textbf{$N$=321} &
\textbf{$N$=1} & \textbf{$N$=11} & \textbf{$N$=21} & \textbf{$N$=41} & \textbf{$N$=81} & \textbf{$N$=161} & \textbf{$N$=321} \\
\midrule
DDPM, Random search, GT accuracy guided     & 0.0000          & 0.0000          & 0.0000          & 0.0000          & 0.0000          & 0.0000          & 0.0156          & 0.5071          & 0.6089          & 0.6316          & 0.6492          & 0.6691          & 0.6881          & 0.6999          \\
Original, Random search, GT accuracy guided & 0.0781          & 0.1641          & 0.2188          & 0.2344          & 0.2422          & 0.2656          & 0.2812          & \textbf{0.6650} & 0.7731          & 0.7952          & 0.8036          & 0.8217          & 0.8347          & 0.8491          \\
T-SCEND tr. w/o LRNCL, Random search, GT accuracy guided            & \textbf{0.1094} & \textbf{0.2578} & \textbf{0.2969} & \textbf{0.3438} & \textbf{0.3750} & \textbf{0.3828} & \textbf{0.4219} & 0.6442 & \textbf{0.7855} & \textbf{0.8096} & \textbf{0.8317} & \textbf{0.8466} & \textbf{0.8628} & \textbf{0.8854} \\ 
\bottomrule
\end{tabular}
}
\end{center}
\vskip -0.1in
\end{table}





% \begin{table}[]
% \caption{Success rate of mixed inference on Maze with grid size \textbf{15} across different MCTS start steps. Bold font denotes the best model.}
% \label{tab:maze_mcts_start_step}
% \begin{tabular}{lllllllllll}
% \cline{1-1}
% \multicolumn{1}{|l|}{Method} & \multicolumn{10}{c}{MCTS start step}                                                                                                                                              \\
% \multicolumn{1}{|l|}{}                        & 0               & 1               & 2               & 3               & 4               & 5               & 6               & 7               & 8               & 9               \\ \cline{1-1}
% Mixed inference, KL \& LRNCL ,Energy guided   & 0.6562 ± 0.4750 & 0.6094 ± 0.4879 & 0.6641 ± 0.4723 & 0.7969 ± 0.4023 & 0.7969 ± 0.4023 & 0.6406 ± 0.4798 & 0.4922 ± 0.4999 & 0.4922 ± 0.4999 & 0.4609 ± 0.4985 & 0.4453 ± 0.4970 \\
% Mixed inference, Original, Energy guided      & 0.0781 ± 0.2684 & 0.0703 ± 0.2557 & 0.0859 ± 0.2803 & 0.0781 ± 0.2684 & 0.1250 ± 0.3307 & 0.1484 ± 0.3555 & 0.1250 ± 0.3307 & 0.0781 ± 0.2684 & 0.0625 ± 0.2421 & 0.0703 ± 0.2557
% \end{tabular}
% \end{table}
\section{Limitations and future work}
\label{app:limit_future} 
Our inference framework primarily relies on MCTS, which presents two key limitations: (1) limited compatibility with parallel computing, and (2) challenges in effectively evaluating node quality during the early stages of denoising. Future work could explore integrating alternative search strategies, such as those proposed by \citet{wu2024inference}. Additionally, to enhance performance-energy consistency, we introduce linear-regression negative contrastive learning, which enforces a linear relationship between energy and the distance to real samples. Further investigation is needed to assess the broader implications of this constraint and explore alternative regularization approaches. Lastly, while our current implementation utilizes Gaussian noise for branching, other diffusion-based branching mechanisms remain an open area for exploration.
\section{Visualization of results}
\label{app:vis_results}
\subsection{Visualization of Maze experiments}
\label{app:maze_vis}
This section presents visualizations of the training in Fig. \ref{fig:maze_training_vis}, test Maze data in Fig. \ref{fig:maze_test_vis}, and samples generated by different methods in Fig. \ref{fig:maze_samples_diff}. In the visuals, black pixels denote walls, green represents the starting point, red represents the goal point, blue marks the solved path, and white represents the feasible area. All visualizations are based on a few representative samples. The results from the training and test sets clearly show that the tasks in the test set are notably more challenging than those in the training set. Visual comparisons of samples generated by different methods reveal that the originally trained model, regardless of the inference strategy, performs consistently worse than \proj.
\begin{figure}[tb]
\vskip 0.2in
\begin{center}
\centerline{\includegraphics[width=0.8\textwidth]{fig/maze_plot_multi_grid_size_appendix_train.pdf}}
\caption{Visualization of training maze dataset. }
\label{fig:maze_training_vis}
\end{center}
\vskip -0.2in
\end{figure}

\begin{figure}[ht]
\vskip 0.2in
\begin{center}
\centerline{\includegraphics[width=0.8\textwidth]{fig/maze_plot_multi_grid_size_appendix_test.pdf}}
\caption{Visualization of test maze dataset, where the blue paths are ground-truth solutions.}
\label{fig:maze_test_vis}
\end{center}
\vskip -0.2in
\end{figure}

\begin{figure}[ht]
\vskip 0.2in
\begin{center}
\centerline{\includegraphics[width=0.8\textwidth]{fig/maze_plot_diff.pdf}}
\caption{Visualization of samples generated by different training and inference methods.}
\label{fig:maze_samples_diff}
\end{center}
\vskip -0.2in
\end{figure}
\subsection{Visualization of Sudoku experiments}
\label{app:sudoku_vis}

\begin{figure}[ht]
% \vskip 0.2in
\begin{center}
\centerline{\includegraphics[width=0.7\textwidth]{fig/sudoku_train_test_samples.pdf}}
\caption{Visualization of training and test Sudoku dataset.}
\label{fig:sudoku_training_test_vis}
\end{center}
% \vskip -0.2in
\end{figure}

\begin{figure}[ht]

\vskip 0.2in
\begin{center}
\centerline{\includegraphics[width=0.7\textwidth]{fig/sudoku_plot_diff.pdf}}
\caption{Visualization of samples generated by different training and inference methods.}
\label{fig:sudoku_samples_diff}
\end{center}
\vskip -0.2in
\end{figure}

This section presents visualizations of the training and test Sudoku data in Fig.~\ref{fig:sudoku_training_test_vis}, and representative samples generated by different methods in Fig.~\ref{fig:sudoku_samples_diff}. In the
visuals, black numbers denote the condition, green numbers represent correct predictions, and red numbers represent wrong predictions. All visualizations are derived from a few representative samples. The comparison between the training and test sets clearly indicates that the tasks in the test set are significantly more difficult than those in the training set. When comparing the samples generated by different methods, it is evident that the originally trained model, regardless of the inference strategy, consistently underperforms compared to \proj.

% training dataset, landscape visualization, solution of different models
% \section{Analysis of failure case}
% \tao{TODO}
% \jiashu{TODO}
% \label{app:failure_analysis} 
% training dataset, test dataset, solutions of different models


%\section{You \emph{can} have an appendix here.}

%You can have as much text here as you want. The main body must be at most $8$ pages long.
%For the final version, one more page can be added.
%If you want, you can use an appendix like this one.  

%The $\mathtt{\backslash onecolumn}$ command above can be kept in place if you prefer a one-column appendix, or can be removed if you prefer a two-column appendix.  Apart from this possible change, the style (font size, spacing, margins, page numbering, etc.) should be kept the same as the main body.

%%%%%%%%%%%%%%%%%%%%%%%%%%%%%%%%%%%%%%%%%%%%%%%%%%%%%%%%%%%%%%%%%%%%%%%%%%%%%%%
%%%%%%%%%%%%%%%%%%%%%%%%%%%%%%%%%%%%%%%%%%%%%%%%%%%%%%%%%%%%%%%%%%%%%%%%%%%%%%%


\end{document}


% This document was modified from the file originally made available by
% Pat Langley and Andrea Danyluk for ICML-2K. This version was created
% by Iain Murray in 2018, and modified by Alexandre Bouchard in
% 2019 and 2021 and by Csaba Szepesvari, Gang Niu and Sivan Sabato in 2022.
% Modified again in 2023 and 2024 by Sivan Sabato and Jonathan Scarlett.
% Previous contributors include Dan Roy, Lise Getoor and Tobias
% Scheffer, which was slightly modified from the 2010 version by
% Thorsten Joachims & Johannes Fuernkranz, slightly modified from the
% 2009 version by Kiri Wagstaff and Sam Roweis's 2008 version, which is
% slightly modified from Prasad Tadepalli's 2007 version which is a
% lightly changed version of the previous year's version by Andrew
% Moore, which was in turn edited from those of Kristian Kersting and
% Codrina Lauth. Alex Smola contributed to the algorithmic style files.
