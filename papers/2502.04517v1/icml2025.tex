\PassOptionsToPackage{dvipsnames}{xcolor}

\documentclass{article}

% Recommended, but optional, packages for figures and better typesetting:
\usepackage{microtype}
\usepackage{graphicx}
\usepackage{subfigure}
\usepackage{booktabs} % for professional tables
\usepackage{siunitx}
\usepackage{algorithmic}
% \usepackage{algpseudocode}
\usepackage{wrapfig}
\usepackage{enumerate}
\usepackage{etoolbox}
\usepackage{multirow}
\usepackage{tcolorbox}

% \usepackage{algorithmic}  
% \usepackage[algo2e, ruled]{algorithm2e} 

% hyperref makes hyperlinks in the resulting PDF.
% If your build breaks (sometimes temporarily if a hyperlink spans a page)
% please comment out the following usepackage line and replace
% \usepackage{icml2025} with \usepackage[nohyperref]{icml2025} above.
\usepackage{hyperref}
%%%%% NEW MATH DEFINITIONS %%%%%

% \usepackage{amsmath,amsfonts,bm}
\usepackage{amsmath,amsfonts}

\usepackage{pifont}


\newcommand{\R}{\mathbb{R}}


\def\va{{\mathbf{a}}}
\def\vg{{\mathbf{g}}}

% Sets
\def\sR{\mathbb{R}}
\def\sC{\mathbb{C}}
\def\sZ{\mathbb{Z}}
\def\sN{\mathbb{N}}
\def\sQ{\mathbb{Q}}

\def\sS{\mathcal{S}}



% Vectors
\def\vzero{{\mathbf{0}}}
\def\vone{{\mathbf{1}}}
\def\vmu{{\mathbf{\mu}}}
\def\vtheta{{\mathbf{\theta}}}
\def\va{{\mathbf{a}}}
\def\vb{{\mathbf{b}}}
\def\vc{{\mathbf{c}}}
\def\vd{{\mathbf{d}}}
\def\ve{{\mathbf{e}}}
\def\vf{{\mathbf{f}}}
\def\vg{{\mathbf{g}}}
\def\vh{{\mathbf{h}}}
\def\vi{{\mathbf{i}}}
\def\vj{{\mathbf{j}}}
\def\vk{{\mathbf{k}}}
\def\vl{{\mathbf{l}}}
\def\vm{{\mathbf{m}}}
\def\vn{{\mathbf{n}}}
\def\vo{{\mathbf{o}}}
\def\vp{{\mathbf{p}}}
\def\vq{{\mathbf{q}}}
\def\vr{{\mathbf{r}}}
\def\vs{{\mathbf{s}}}
\def\vt{{\mathbf{t}}}
\def\vu{{\mathbf{u}}}
\def\vv{{\mathbf{v}}}
\def\vw{{\mathbf{w}}}
\def\vx{{\mathbf{x}}}
\def\vy{{\mathbf{y}}}
\def\vz{{\mathbf{z}}}
\def\vzeta{{\mathbf{\zeta}}}

% Matrix
\def\mA{{\mathbf{A}}}
\def\mB{{\mathbf{B}}}
\def\mC{{\mathbf{C}}}
\def\mD{{\mathbf{D}}}
\def\mE{{\mathbf{E}}}
\def\mF{{\mathbf{F}}}
\def\mG{{\mathbf{G}}}
\def\mH{{\mathbf{H}}}
\def\mI{{\mathbf{I}}}
\def\mJ{{\mathbf{J}}}
\def\mK{{\mathbf{K}}}
\def\mL{{\mathbf{L}}}
\def\mM{{\mathbf{M}}}
\def\mN{{\mathbf{N}}}
\def\mO{{\mathbf{O}}}
\def\mP{{\mathbf{P}}}
\def\mQ{{\mathbf{Q}}}
\def\mR{{\mathbf{R}}}
\def\mS{{\mathbf{S}}}
\def\mT{{\mathbf{T}}}
\def\mU{{\mathbf{U}}}
\def\mV{{\mathbf{V}}}
\def\mW{{\mathbf{W}}}
\def\mX{{\mathbf{X}}}
\def\mY{{\mathbf{Y}}}
\def\mZ{{\mathbf{Z}}}
\def\mBeta{{\mathbf{\beta}}}
\def\mPhi{{\mathbf{\Phi}}}
\def\mLambda{{\mathbf{\Lambda}}}
\def\mSigma{{\mathbf{\Sigma}}}


% Expectation
% \def\eE{\mathop{\mathbb{E}}\limits}
\def\eE{\mathbb{E}}

% Probability
\def\pP{\mathbb{P}}

% Tilde
\def\tf{\tilde{f}}
\def\tS{\tilde{S}}
\def\wtF{\widetilde{\mathcal{F}}}
\def\whR{\widehat{R}}
\def\tvx{\tilde{\mathbf{x}}}
\def\ty{\tilde{y}}


\def\defeq{\overset{\textup{def}}{=}}
% \def\defeq{\overset{.}{=}}
\def\defone{\overset{\text{\ding{172}}}{=}}
\def\deftwo{\overset{\text{\ding{173}}}{=}}
\def\leqone{\overset{\text{\ding{172}}}{\leq}}
\def\leqtwo{\overset{\text{\ding{173}}}{\leq}}
\def\leqthree{\overset{\text{\ding{174}}}{\leq}}
\def\leqfour{\overset{\text{\ding{175}}}{\leq}}
\def\eqone{\overset{\text{\ding{172}}}{=}}
\def\eqtwo{\overset{\text{\ding{173}}}{=}}
\def\eqthree{\overset{\text{\ding{174}}}{=}}
\def\eqfour{\overset{\text{\ding{175}}}{=}}
\def\geqfive{\overset{\text{\ding{176}}}{\geq}}
% \usepackage{algorithmic}  
% \usepackage[algo2e, ruled]{algorithm2e} 

\newcommand{\AK}[1]{{\color{Magenta} [\textbf{AK:} #1]}}

% Attempt to make hyperref and algorithmic work together better:
\newcommand{\theHalgorithm}{\arabic{algorithm}}

% Use the following line for the initial blind version submitted for review:
\usepackage[accepted]{icml2025}

% If accepted, instead use the following line for the camera-ready submission:
% \usepackage[accepted]{icml2025}

% For theorems and such
\usepackage{amsmath}
\usepackage{amssymb}
\usepackage{mathtools}
\usepackage{amsthm}

% if you use cleveref..
\usepackage[capitalize,noabbrev]{cleveref}

\renewcommand{\algorithmicrequire}{\textbf{Input:}}
\renewcommand{\algorithmicensure}{\textbf{Output:}}

%%%%%%%%%%%%%%%%%%%%%%%%%%%%%%%%
% THEOREMS
%%%%%%%%%%%%%%%%%%%%%%%%%%%%%%%%
\theoremstyle{plain}
\newtheorem{theorem}{Theorem}
\newtheorem{proposition}[theorem]{Proposition}
\newtheorem{lemma}[theorem]{Lemma}
\newtheorem{corollary}[theorem]{Corollary}
\theoremstyle{definition}
\newtheorem{definition}[theorem]{Definition}
\newtheorem{assumption}[theorem]{Assumption}
\theoremstyle{remark}
\newtheorem{remark}[theorem]{Remark}

% Todonotes is useful during development; simply uncomment the next line
%    and comment out the line below the next line to turn off comments
%\usepackage[disable,textsize=tiny]{todonotes}
\usepackage[textsize=tiny]{todonotes}


% The \icmltitle you define below is probably too long as a header.
% Therefore, a short form for the running title is supplied here:
\icmltitlerunning{Towards Cost-Effective Reward Guided Text Generation}

\begin{document}

\twocolumn[
\icmltitle{Towards Cost-Effective Reward Guided Text Generation}

% It is OKAY to include author information, even for blind
% submissions: the style file will automatically remove it for you
% unless you've provided the [accepted] option to the icml2025
% package.

% List of affiliations: The first argument should be a (short)
% identifier you will use later to specify author affiliations
% Academic affiliations should list Department, University, City, Region, Country
% Industry affiliations should list Company, City, Region, Country

% You can specify symbols, otherwise they are numbered in order.
% Ideally, you should not use this facility. Affiliations will be numbered
% in order of appearance and this is the preferred way.
\icmlsetsymbol{equal}{*}

\begin{icmlauthorlist}
    \icmlauthor{Ahmad Rashid}{equal,uw,vi}
    \icmlauthor{Ruotian Wu}{equal,uw,vi}
    \icmlauthor{Rongqi Fan}{uw}
    \icmlauthor{Hongliang Li}{hu}
    \icmlauthor{Agustinus Kristiadi}{vi}
    \icmlauthor{Pascal Poupart}{uw,vi}
  \end{icmlauthorlist}

  \icmlaffiliation{uw}{University of Waterloo}
  \icmlaffiliation{vi}{Vector Institute}
  \icmlaffiliation{hu}{Huawei Technologies}

  \icmlcorrespondingauthor{Ahmad Rashid}{a9rashid@uwaterloo.ca}

% \icmlaffiliation{yyy}{Department of XXX, University of YYY, Location, Country}
% \icmlaffiliation{comp}{Company Name, Location, Country}
% \icmlaffiliation{sch}{School of ZZZ, Institute of WWW, Location, Country}

% \icmlcorrespondingauthor{Firstname1 Lastname1}{first1.last1@xxx.edu}
% \icmlcorrespondingauthor{Firstname2 Lastname2}{first2.last2@www.uk}

% You may provide any keywords that you
% find helpful for describing your paper; these are used to populate
% the "keywords" metadata in the PDF but will not be shown in the document
\icmlkeywords{RLHF, Training Cost, LLM, Efficiency}

\vskip 0.3in
]

% this must go after the closing bracket ] following \twocolumn[ ...

% This command actually creates the footnote in the first column
% listing the affiliations and the copyright notice.
% The command takes one argument, which is text to display at the start of the footnote.
% The \icmlEqualContribution command is standard text for equal contribution.
% Remove it (just {}) if you do not need this facility.

%\printAffiliationsAndNotice{}  % leave blank if no need to mention equal contribution
\printAffiliationsAndNotice{\icmlEqualContribution} % otherwise use the standard text.

\begin{abstract}
% Reinforcement learning from human feedback (RLHF) is a critical step for aligning Large language models (LLMs) with human preferences. However, conventional training algorithms like proximal policy optimization (PPO) usually suffer from high cost and instability. Reward-guided text generation (RGTG) is an emerging method where a reward model is used to score partial sequences during decoding, and the reward score is combined with LLM logits to achieve an output distribution similar to RLHF. However, training the reward models remains to be challenging. Despite multiple methods of training such models have been proposed, they all suffer from certain assumptions and limitations. In this work, we aim to seek alternative methods that are more principled to approach such reward models. We proposed a local constraint that allows LLMs and score partial sequence based on their optimal expansions.

Reward-guided text generation (RGTG) has emerged as a viable alternative to offline reinforcement learning from human feedback (RLHF). 
RGTG methods can align baseline language models to human preferences without further training like in standard RLHF methods. 
However, they rely on a reward model to score each candidate token generated by the language model at inference, incurring significant test-time overhead.
Additionally, the reward model is usually only trained to score full sequences, which can lead to sub-optimal choices for partial sequences. 
In this work, we present a novel reward model architecture that is trained, using a Bradley-Terry loss, to prefer the optimal expansion of a sequence with just a \emph{single call} to the reward model at each step of the generation process.  
That is, a score for all possible candidate tokens is generated simultaneously, leading to efficient inference. 
We theoretically analyze various RGTG reward models and demonstrate that prior techniques prefer sub-optimal sequences compared to our method during inference. 
Empirically, our reward model leads to significantly faster inference than other RGTG methods. 
It requires fewer calls to the reward model and performs competitively compared to previous RGTG and offline RLHF methods. 
\end{abstract}

\section{Introduction}
\label{sec:intro}
\section{Introduction}
\label{sec:intro}


\ps{Challenges of technology scaling}

The growing demand for computing performance has always been met by increasing the number of transistors per chip, which is only possible due to CMOS technology scaling.
However, as we keep pushing the boundaries of technology scaling, we encounter multiple challenges.
Firstly, whenever we transition to a more advanced technology node, the non-recurring cost due to physical design, verification, software, mask sets, and prototyping almost doubles \cite{cost-tech-node}.
As a result, designing a chip in an advanced technology node is only economically viable if the chip is manufactured in vast quantities.
Secondly, many chip components such as I/O drivers, analog circuits, or \gls{srams} have reached their scaling limit.
This means that we cannot shrink these components further, even if we use a more advanced technology with a smaller feature size.
Thirdly, advanced technology nodes suffer from high defect rates, diminishing the yield and inflating the recurring cost.
To tackle these challenges, new chip-design paradigms have been developed.

\ps{Why 2.5D integration?}

One of these new paradigms is 2.5D integration, where multiple silicon dies called chiplets are integrated into the same package.
Once designed, a single chiplet can be reused in multiple 2.5D stacked chips, which increases the ratio of production volume to non-recurring cost.
Another advantage is that multiple chiplets - fabricated in different technologies - can be integrated into the same package.
This means that only components that can take full advantage of technology scaling are built in bleeding-edge technologies.
Components that have reached their scaling limit are fabricated in more mature and hence less costly technology nodes.
Furthermore, chiplets are smaller than monolithic chips.
Therefore, manufacturing chiplets results in less silicon area loss due to fabrication defects and hence a higher yield.
Due to these economic advantages, chip vendors such as AMD \cite{amd-chiplet} and NVIDIA \cite{chiplet-book} have adopted the 2.5D integration paradigm.  

\ps{Challenges of 2.5D integration}

An important challenge when designing 2.5D stacked chips is the construction of a low-latency and high-throughput \gls{ici}. 
To build an \gls{ici}, we connect different chiplets using \gls{d2d} links.
These links are fabricated in an organic package substrate, silicon bridge, or silicon interposer, and they are connected to the chiplets using \gls{c4} bumps or microbumps.
The number of bumps per chiplet is limited, and so is the bandwidth of \gls{d2d} links.
In addition to having lower bandwidth than links in monolithic chips, \gls{d2d} links also have higher latency.
This latency is caused by wire delay and by \gls{phys} that are necessary in both the sending and the receiving chiplet.
\gls{phys} are needed to convert between protocols, voltage levels, and frequencies, which are usually different between on-chiplet links and \gls{d2d} links.
Due to these limitations, the \gls{ici} can quickly become a bottleneck.

\ps{How we solve these challenges differently than the related work does.}

Existing approaches to maximize the performance of the \gls{ici} either optimize the placement of chiplets (with potentially heterogeneous shapes) for a predetermined \gls{ici} topology 
\cite{ho,liu,seemuth,eris,osmolovskyi,tap25d,chiou}, select one topology out of a set of candidates \cite{coskun-1, coskun-2}, or they optimize the \gls{ici} topology for a 2D grid of homogeneously shaped chiplets on an active interposer \cite{butterdonut, cluscross, kite}.
To the best of our knowledge, there is no prior work on \gls{ici} topologies for chips with heterogeneously shaped chiplets or with passive silicon interposers or silicon bridges.
To fill this gap, we propose \name, a novel optimization methodology to jointly optimize the chiplet placement and \gls{ici} topology of such architectures.
\ifnb
\else
\newpage
\fi

\ps{Details on \name~and the key idea}

The key idea is as follows: 
We optimize the chiplet placement without a predetermined topology.
For each placement generated by an optimization algorithm, we infer a placement-based \gls{ici} topology by connecting chiplets that are in close proximity in that specific placement.
We then compute the latency and throughput of this combination of placement and topology for different traffic types.
These latencies and throughputs together with the total chip area are used to compute a user-defined quality-score of the placement, which is returned to the optimization algorithm.
Based on this quality score, the algorithm can further optimize the placement.
By following this iterative process, we jointly optimize the chiplet placement and the \gls{ici} topology.

\ps{Short evaluation-summary}

We provide our open-source framework implementing the proposed placement and topology co-optimization methodology, which we evaluate using both synthetic traffic and traffic traces.
A 2D grid of chiplets with a mesh topology is used as a baseline since many proposals for 2.5D stacked chips \cite{dataflow_accel_dnn, cifher, simba, hecaton, dojo} use such an architecture.
We reduce the latency of synthetic L1-to-L2 and L2-to-memory traffic, the two most important traffic types for cache coherency traffic, by up to 28\% and 62\% respectively.
For real traffic traces, we reduce the average packet latency for almost all traces and architectures considered (reduced by an 8\% or 18\% on average depending on the configuration of \gls{phys} within a chiplet).


\section{Preliminaries}
\label{sec:prelim}
\section{Preliminaries}
\label{sec:prelim}

Given a dataset ${\cal D} = \{(x_1)\}$ consisting of samples $x_1\sim\rho_1$, e.g., images, drawn from an unknown target data distribution $\rho_1$, the goal of generative modeling is to learn a model that faithfully captures the unknown target data distribution $\rho_1$ and permits to sample from the learned distribution. %

Since we focus primarily on rectified flow, we provide its formulation in the following. %
At inference time, 
rectified flow starts from samples $x_0\sim\rho_0$ drawn from a known source distribution $\rho_0$, e.g., a standard Gaussian. The source distribution samples are pushed forward from time $t=0$ to target time $t=1$ via integration along a trajectory specified via a learned velocity field $v(z_t,t)$. This learned velocity field depends on the current time $t$ and the sample location $z_t$ at time $t$. Formally, we obtain samples by numerically solving the ordinary differential equation (ODE)
\begin{equation}
    \label{eq:RF}
    d z_t = v (z_t, t) dt, \, \text{with }z_0 \sim \rho_0, \quad t \in [0, 1]. 
\end{equation}
Notably, this sampling procedure is able to capture multimodal dataset distributions, as one expects from a generative model.

To learn the velocity field, at training time, rectified flow constructs random pairs $(x_0,x_1)$, consisting of a source distribution sample $x_0\sim\rho_0$ and a target distribution sample $x_1\sim{\cal D}$. The latter is drawn from a given dataset ${\cal D}$ consisting of samples which are assumed to be drawn from the unknown target distribution $\rho_1$. For a uniformly drawn time $t\sim U[0,1]$, the time-dependent location $x_t$ %
is computed from the pair $(x_0,x_1)$ using linear interpolation of $(x_0, x_1)$, i.e., 
\begin{equation}
\label{eq:lin_int}
x_t = (1-t)x_0 + tx_1, \quad \text{where} \, x_0 \sim \rho_0, \, x_1 \sim {\cal D}.
\end{equation}
At this location $x_t$ and time $t$, the ``ground-truth'' velocity $v_\text{gt}(x_t,t) = \partial x_t/\partial t = x_1 - x_0$ is readily available. It is then matched during training with a velocity model $v(x_t,t)$ via a standard $\ell_2$ loss, i.e., during training we address
\begin{equation}
\label{eq:lin_rect_flow}
\inf_{v} \mathbb{E}_{x_0\sim\rho_0,x_1\sim{\cal D},t\sim U[0,1]}\left[\| x_1 - x_0 - v(x_t,t) \|^2_2 \right],
\end{equation}
where the optimization is over the set of all measurable velocity fields. 
In practice, the functional velocity model $v(x_t,t)$ is often parameterized via a deep net with trainable parameters $\theta$, i.e., $v(x_t,t)\approx v_\theta(x_t,t)$, and the infimum resorts to a minimization over parameters $\theta$.

Considering the training procedure more carefully, it is easy to see that different random pairs $(x_0,x_1)$ can lead to different ``ground-truth'' velocity directions at the same time $t$ and at the same location $x_t$. The aforementioned $\ell_2$ loss hence asks the functional velocity model $v(x_t,t)$ to regress to  different ``ground-truth'' velocity directions. This leads to averaging, i.e., the optimal functional velocity model $v^\ast(x_t,t) = \mathbb{E}_{\{(x_0,x_1,t):(1-t)x_0+tx_1 = x_t\}}\left[v(x_t,t)\right]$.

According to Theorem 3.3 by~\cite{liu2023flow}, if we use $v^\ast$ for the ODE in~\cref{eq:RF}, then the stochastic process associated with \cref{eq:RF} has the same marginal distributions for all $t \in [0, 1]$ as the stochastic process associated with the linear interpolation characterized in~\cref{eq:lin_int}.

Nonetheless, to avoid the averaging, in this paper we wonder whether it is possible to capture the multimodal velocity distribution at each time $t$ and at each location $x_t$, and whether there are any potential benefits to doing so.




\section{Current Limitations of RGTG}
\label{sec:Limitations}
%!TEX root=icml2025.tex

We will discuss two primary limitations of current RGTG methods, namely high decoding cost and sub-optimal rewards. In the next section, we propose a solution to address these limitations.

\subsection{High Decoding Cost}
Most RGTG methods default to training a full sequence reward model $r_\phi$ and then either a) use it to directly score partial sequences \cite{khanov2023alignment} or b) distill a partial sequence value model $V_\theta$ from the full sequence reward model $r_\phi$ \cite{mudgalcontrolled}. During decoding, the score for each candidate token $y_i$ is calculated according to Equation~\ref{eq:score}. We note that the input to $V_\theta$ includes the sequence $y_{1:i-1}$ with each candidate token $y_{i}$ appended to the sequence.  Hence to score each candidate token, we need to make a different call to the value function, resulting into $k$ calls for top-$k$ decoding.  This adds substantial overhead during decoding.

\subsection{Sub-Optimal Reward Models}

Next we take a look at contemporary RGTG reward models and show that they may prefer partial sequences with sub-optimal extensions. 

\paragraph{PARGS} \citet{rashid2024critical} showed that using a BT reward model trained on full sequences to score partial sequences (as done by \citet{khanov2023alignment}) can lead to arbitrary rewards for partial sequences. Rashid et al.~proposed to train a BT reward model explicitly on partial sequences by creating a separate loss function for all prefix lengths $i$:

\begin{equation}
  \mathcal{L}_R^i = - \sum_{(\rvx, \rvy^w, \rvy^l) \in \D} \log \sigma ( V_{\theta} (\rvy^w_{1:i}|\rvx) - V_{\theta} (\rvy^l_{1:i}|\rvx)) . \label{eq:partial-seq-objectives}
\end{equation}


However, given that full sequence $\rvy^w$ is preferred to full sequence $\rvy^l$, training is based on the assumption that the partial sequence $\rvy^w_{1:i}$ is also preferred to the partial sequence $\rvy^l_{1:i}$. This assumption can be  problematic as the full-sequence dataset typically includes only one or a few full sequences that extend each partial sequence.  In fact, the empirical distribution of such extensions will impact the learned value function to the extent where a prefix with only extensions to suboptimal full sequences may be scored higher than a prefix with an extension to an optimal full sequence. %In fact, Lemma 2 of \cite{} shows that the resulting partial sequence reward model depends on the preference data distribution.  Hence, different preference data distributions may yield different partial sequence reward models, which is problematic.  We show in the following theorem for some preference data distributions, the partial reward 

\begin{theorem}
\label{thm:pargs}
In the limit of infinite training and a sufficiently expressive representation for the value function, PARGS may learn a value function that gives a lower score to a prefix extendable to an optimal full sequence than some other prefix.  More precisely, if $\rvy^* = \argmax_{\rvy} r(\rvy|\rvx)$, then there may exist $i,j,\rvy'$ such that
\begin{equation}
V(\rvy^*_{1:i}|\rvx) < V(\rvy'_{1:j}|\rvx)
\end{equation}
\end{theorem}

\begin{proof}
Let $\rvy^*$, $\rvy'$, $\rvy''$ and $\rvy'''$ be four responses to $\rvx$ such that $\rvy^*$ is an optimal response and $\rvy'$, $\rvy''$, $\rvy'''$ are three suboptimal responses.  Suppose also that the preference dataset contains exactly three comparisons:  $\D=\{(\rvx,\rvy^*,\rvy'), (\rvx,\rvy',\rvy''), (\rvx,\rvy',\rvy''')\}$ where the first response is preferred to the second response in each triple.  Suppose also that $\rvy^*$ and $\rvy'$ share the first $i-1$ tokens (i.e., $\rvy^*_{1:i-1} = \rvy'_{1:i-1}$) while $\rvy^*$, $\rvy''$ and $\rvy'''$ share the first $i$ tokens (i.e., $\rvy^*_{1:i} = \rvy''_{1:i} = \rvy'''_{1:i}$). In the limit of infinite training and sufficiently expressive value function representation, Lemma 2 in \cite{rashid2024critical} indicates that the learned value function $V$ satisfies
\begin{equation}
\label{eq:pargs}
    \sigma(V(\rvy^1_{1:i}|\rvx) - V(\rvy^2_{1:j}|\rvx)) = P_D([\rvx,\rvy^1] \succ [\rvx,\rvy^2])
\end{equation}
where $[a,b]$ indicates the concatenation of sequences $a$ and $b$, and $a\succ b$ indicates that $a$ is preferred to $b$.  \cref{eq:pargs} implies that the BT model induced by $V$ exhibits the same preference probabilities for the full sequence extension of $\rvy^1_{1:i}$ and $\rvy^2_{1:j}$ as the empirical distribution of the preference dataset.  Recall, that PARGS assumes that $[\rvx,\rvy^1_{1:i}] \succ [\rvx,\rvy^2_{1:j}]$ when their respective full sequence extensions exhibit the same preference ordering (i.e., $[\rvx,\rvy^1] \succ [\rvx,\rvy^2]$).  Since their might be different extensions $\rvy^1_{i+1:|\rvy^1|}$ and $\rvy^2_{i+1:|\rvy^2|}$ for each prefix with different preference labels in the preference dataset, then PARGS learns a value function that induces preference probabilities for partial sequences that are consistent with the empirical distribution $P_D$ of the preference dataset for the full sequence extensions of those partial sequences.  Applying \cref{eq:pargs} to prefixes $\rvy^*_{1:i}$ and $\rvy'_{1:i}$ yields:
\begin{equation}
\label{eq:sigmoid}
    \sigma(V(\rvy^*_{1:i}|\rvx) - V(\rvy'_{1:i}|\rvx)) = 1/3
\end{equation}
since the dataset $\D$ contains one preference ranking $(\rvx,\rvy^*,\rvy')$ where the full sequence extension $\rvy^*$ of $\rvy^*_{1:i}$ is preferred to the full sequence extension $\rvy'$ of $\rvy'_{1:i}$ and two preference rankings $(\rvx,\rvy'\rvy'')$, $(\rvx,\rvy'\rvy''')$ where the full sequence extension $\rvy'$ of $\rvy'_{1:i}$ is preferred to the full sequence extensions $\rvy''$, $\rvy'''$ of $\rvy^*_{1:i}$. 
Recall that $\rvy^*_{1:i}=\rvy''_{1:i}=\rvy'''_{1:i}$ and therefore $\rvy''$ and $\rvy'''$ are full sequence extensions of $\rvy^*_{1:i}$. Finally, since the sigmoid in \cref{eq:sigmoid} is less than 0.5, then $V(\rvy^*_{1:i}) < V(\rvy'_{1:i})$.  Hence, this shows that $\exists i{=}j,\rvy'$ such that $V(\rvy^*_{1:i}) < V(\rvy'_{1:j})$
\end{proof}

Theorem~\ref{thm:pargs} shows that the value function learned by PARGS may prefer prefixes that lead to suboptimal responses.  The key problem is PARGS' assumption that the preference ordering of prefixes is the same as the preference ordering of full sequence extensions.  Since it is possible to extend a prefix to many different full sequences with different scores, the value function learned by PARGS depends on the frequency of different prefix extensions instead of preferences only.  As shown in the proof of Theorem~\ref{thm:pargs}, this becomes problematic when a prefix that can lead to an optimal response is extended more frequently to losing full sequences instead of winning full sequences in $\D$.

%\vspace{0.5em}
%\begin{theorem} \label{thm:PARGS}
%  \label{thm:full_for_partial}
%  Let \(r\) be a reward model trained to minimize the Bradley-Terry loss on partial sequences
%  $\rvy^{1:\abs{\rvy}}$ \eqref{eq:partial-seq-objectives} using full-sequence preference data $\rvy^{1:i}$.
%  Then \(r\) may prefer sub-optimal continuations at decoding.
  
%  \AK{This statement is vague. A better statement for a theorem would be to fully specify the hypothesis, e.g.: ``Let \(r\) be a reward model ... Let \(s^*\), ... Then, if \(r_\phi(s^*) > r_\phi(s^{**})\) and ..., the token \(y_i'\) is picked.''}
  
%  \AK{Then, discuss the significance/implication of this theorem after the proof. E.g. mention that \(y_i'\) is suboptimal and thus RGTG/PARGS is pathological.}
%\end{theorem}
%
%\begin{proof}
%    Let $r_\phi$ be the full sequence reward model, $\mathbf{V}_{\theta}$ be the partial sequence reward model and $\rvy^{(p)} = y_1, \cdots, y_{i-1}$ be a partial sequence following the prompt $\rvx$. Suppose that $\rvs^* = \rvx, \rvy^{(p)}, y^*_{i}, y^*_{i+1}, \cdots, y^*_{n}$ is the optimal full sequence extended from ${\rvx, \rvy^{(p)}}$. Let $\rvs^{**} = \rvx, \rvy^{(p)}, y^*_{i}, y^{**}_{i+1} \cdots, y^{**}_{n}$ be a sequence extended from ${\rvx, \rvy^{(p)}, y^*_{i}}$ and $\rvs' = \rvx, \rvy^{(p)}, y'_{i}, y'_{i+1} \cdots, y'_{n}$: be a sequence extended from ${\rvx, \rvy^{(p)}}$. Suppose that $\mathcal{P} = \{\rvs^{**}, \rvs'\}$ is a pair in the preference dataset. 

%    \AK{This hypothesis should be put in the theorem's statement. Here, you just say ``... By the hypothesis that $r_{\phi}$ ..., then ...''}
%    \AK{Also, avoid using a concrete numbers. Make your hypothesis general.}
%    Suppose $r_{\phi}(s^*)=6, \:r_{\phi}(s^{**})=-6, \:r_{\phi}(s')=5$, then $\rvs'$ is the winning sequence in $\mathcal{P}$. 
%    By the assumption that "the prefixes of a winning full sequence is also wining against the prefix of the corresponding losing full sequence", partial sequence $(\rvx, \rvy^{p}, y'_{i})$ is also winning $(\rvx, \rvy^{p}, y^*_{i})$.  Following the Bradley-Terry loss, we maximize
%$$
%\begin{aligned}
%     \log \sigma \left( \mathbf{V}_{\theta} (\rvy^{(p)}, y'_{i}|\rvx) - \mathbf{V}_{\theta} (\rvy^{(p)}, y^*_{i}|\rvx)\right)
%\end{aligned}
%$$
%That is, $\mathbf{V}_{\theta} (\rvy^{(p)}, y'_{i}|\rvx) - \mathbf{V}_{\theta} (\rvy^{(p)}, y^*_{i}|\rvx)$ is maximized results in $\mathbf{V}_{\theta} (\rvy^{(p)}, y'_{i}|\rvx) > \mathbf{V}_{\theta} (\rvy^{(p)}, y^*_{i}|\rvx)$

%Next, suppose $\piref(y^{*}|\rvx, \rvy^{(p)}) = \piref(y'|\rvx, \rvy^{(p)}) = \frac{1}{2}$, at inference time we follow the policy:
%$$ 
%\begin{aligned}
%&\pi(y_{i} \vert \rvx,\rvy^{(p)}) \propto \piref(y_{i} \vert \rvx,\rvy^{(p)}) \exp(\beta \mathbf{V}_{\theta}(\rvy^{(p)}, y_{i} \vert \rvx))\\
%\Rightarrow ~~ &\pi(y'_{i} \vert \rvx,\rvy^{(p)}) > \pi(y^{*}_{i} \vert \rvx,\rvy^{(p)})
%\end{aligned}
%$$

%Again, following $\pi$ would generate $y'_{i}$ which is sub-optimal.
%\end{proof}

%\AK{Discuss the implication of the theorem here.}

\paragraph{CD}~\citet{mudgalcontrolled} proposed a target value function $V^*$ for partial sequences that corresponds to the expected reward of the full sequences when the partial sequence is extended by following the base model distribution $\piref$.
%
\begin{equation}
% \label{eq:CD}
    V^*(\rvx, \rvy_{1:i}) = \sum_{\rvy_{i+1:|\rvy|}} \piref(\rvy_{i+1:|\rvy|}|\rvx,\rvy_{1:i}) r(\rvx, \rvy)
\end{equation}
%
The training loss is the squared difference between the value function $V_\theta$ and the target $V^*$. They use rollouts from the base model along with a reward model trained on full sequences to distill the value function $V_\theta$. They sample extensions from $\piref$ to complete a partial sequence and compute the full-sequence score as the target $V^*$. This method has a limitation where the value function heavily depends on the language model. We will show such dependency is suboptimal. 

Value Augmented Sampling~\citep[VAS][]{han2024value} is similar to CD and uses $\piref$ to generate samples for learning a value function and a full-sequence reward model for generating the target score. However, the value function is trained by temporal difference (TD) learning.
% \subsection{Value Augmented Sampling (VAS)}
% Similar to CD, VAS also proposed to use a base LLM to generate samples and hence provide a target score by the full-sequence reward model. Then, they applied temporal difference learning to update the parameters of the value function which they treat as the partial-sequence reward model in the decoding step.

% \subsection{Why LLM dependency is undesirable?}
% \label{LLMD}

\begin{theorem}
\label{thm:full_for_partial}
In the limit of infinite training and a sufficiently expressive representation for the value function, CD may learn a value function that gives a lower score to a prefix extendable to an optimal full sequence than some other prefix.  More precisely, if $\rvy^* = \argmax_{\rvy} r(\rvy|\rvx)$, then there may exist $i,j,\rvy'$ such that
\begin{equation}
V(\rvy^*_{1:i}|\rvx) < V(\rvy'_{1:j}|\rvx)
\end{equation}

%\AK{Here, the statement is better since it's more precise.}

\end{theorem}

\begin{proof}
Let $\rvy^*$ be an optimal response to $\rvx$ such that $r(\rvy^*|\rvx)=6$.  Let $\rvy'$ and $\rvy''$ be two suboptimal responses to $\rvx$ such that $r(\rvy'|\rvx)=4$ and $r(\rvy''|\rvx)=-6$.  Suppose that $\rvy'$ and $\rvy^*$ share the same first $i-1$ tokens (i.e., $\rvy'_{1:i-1} = \rvy^*_{1:i-1}$) and that $\rvy''$ and $\rvy*$ share the same first $i$ tokens (i.e., $\rvy'_{1:i} = \rvy^*_{1:i}$). After generating $\rvy^*_{1:i}$, suppose that $\piref$ generates only $\rvy^*_{i+1:|\rvy^*|}$ and $\rvy''_{i+1:|\rvy''|}$ with uniform probability (i.e., $\piref(\rvy^*_{i+1:|\rvy^*|}|\rvx,\rvy^*_{1:i})=\piref(\rvy''_{i+1:|\rvy''}|\rvx,\rvy''_{1:i})=0.5$ and any other continuation has probability 0). After generating $\rvy'_{1:i}$, suppose also that $\piref$ generates only $\rvy'_{i+1:|\rvy'|}$ (i.e., $\piref(\rvy''_{i+1:|\rvy''}|\rvx,\rvy''_{1:i})=1$ and any other continuation has probability 0).  Then with infinite training and a sufficiently expressive value function representation, CD learns the following partial sequence values
\begin{align}
V(\rvy^*_{1:i}|\rvx) & = \piref(\rvy^*_{i+1:|\rvy^*|}|\rvx,\rvy^*_{1:i})r(\rvy^*|\rvx) \\
& + \piref(\rvy''_{i+1:|\rvy''|}|\rvx,\rvy^*_{1:i})r(\rvy''|\rvx) \\
& = 0.5(6) + 0.5(-6) = 0 \\
V(\rvy'_{1:i}|\rvx) & = \piref(\rvy'_{i+1:|\rvy'|}|\rvx,\rvy'_{1:i})r(\rvy'|\rvx) \\
& = 1(4) = 4
\end{align}
This example shows that $\exists i{=}j,\rvy'$ such that $V(\rvy^*_{1:i}|\rvx) < V(\rvy'_{1:j}|\rvx)$.
\end{proof}

Theorem~\ref{thm:full_for_partial} shows that CD may prefer prefixes that cannot be extended to optimal sequences depending on $\piref$.  The key problem is the dependency of the target $V^*$ on $\piref$.  When $\piref$ extends a prefix to bad responses, the value of this prefix is low, but if it extends the prefix to good responses, the value of this prefix is high.  In principle, the value function $V$ should be independent of $\piref$.  In RLHF, $\piref$ is the quantity that we seek to improve so it does not make sense to improve $\piref$ with a value function that depends on $\piref$ itself.  The value function should depend only on the preferences induced by the full sequence reward model.  As shown in the proof of Theorem~\ref{thm:full_for_partial}, CD may not prefer a prefix that can lead to an optimal response when it is extended by $\piref$ to suboptimal responses.

%\begin{theorem} \label{thm:CD}
%  \label{thm:full_for_partial}
%  A reward model which is trained on continuations from $\piref$ according to \eqref{thm:CD} may lead to sub-optimal continuations at decoding.
%\end{theorem}

% In this section we will describe a counter-example showing that RGTG is suboptimal if the partial-sequence reward model is dependent on the LLM. We consider CD explicitly.\\

%\begin{proof}
% Let $r_\phi$ be the full sequence reward model, $\mathbf{V}_{\theta}$ be the partial sequence reward model and $\rvy^{(p)} = y_1, \cdots, y_{i-1}$ be a partial sequence following the prompt $\rvx$.

% Let $\mathcal{S} = \{\rvs^*, \rvs^{**}, \rvs', \rvs''\}$ be the set of all sequences where 


% \begin{itemize}
    % \item $\mathbf{V}_{\theta}$: partial-sequence reward model
    % \item $r_{\phi}$: full-sequence reward model
    % \item $\rvx$: prompt
    % \item $\rvy^{(p)} = y_1, \cdots, y_{i-1}$: partial-sequence following $\rvx$
%    \item $\rvs^* = \rvx, \rvy^{(p)}, y^*_{i}, y^*_{i+1}, \cdots, y^*_{n}$: the optimal full sequence extended from ${\rvx, \rvy^{(p)}}$
%    \item $\rvs^{**} = \rvx, \rvy^{(p)}, y^*_{i}, y^{**}_{i+1} \cdots, y^{**}_{n}$: another sequence extended from ${\rvx, \rvy^{(p)}, y^*_{i}}$
%    \item $\rvs' = \rvx, \rvy^{(p)}, y'_{i}, y'_{i+1} \cdots, y'_{n}$: another sequence extended from ${\rvx, \rvy^{(p)}}$
%    \item $\rvs'' = \rvx, \rvy^{(p)}, y'_{i},  y''_{i+1}, \cdots, y''_{n}$: another sequence extended from ${\rvx, \rvy^{(p)}, y'_{i}}$
    % \item $\mathcal{S} = \{\rvs^*, \rvs^{**}, \rvs', \rvs''\}$
%\end{itemize} 

%\AK{Same comment as before, can we make the following specific numbers more general?}

%\noindent Suppose $r_{\phi}(s^*)=6, \: \:r_{\phi}(s^{**})=-6,\: \:r_{\phi}(s')=5,\: \:r_{\phi}(s')=3\: \: $ and $\piref$ will only sample $y'$ or $y^{*}$ as the next token and sample sequences from $\mathcal{S}$ follow a uniform distribution, that is, 
%$$
%\begin{aligned}
%\piref(y^{*}|\rvx, \rvy^{(p)}) &= \piref(y'|\rvx, \rvy^{(p)}) = \frac{1}{2}\\
%\piref(\rvs'|\rvx, \rvy^{(p)}, y') &=\piref(\rvs''|\rvx, \rvy^{(p)}, y') = \frac{1}{2}\\
%\piref(\rvs^{*}|\rvx, \rvy^{(p)}, y^{*}) &=\piref(\rvs^{**}|\rvx, \rvy^{(p)}, y^{*}) = \frac{1}{2}\\
%\piref(\rvs|\rvx, \rvy^{(p)}) &= \frac{1}{4} \: , \:\: \forall \rvs \in \mathcal{S}
%\end{aligned}
%$$

%\noindent By following the training objective of CD, in the limit we have:
%$$
%\begin{aligned}
%    \mathbf{V}_{\theta}(\rvy^{(p)}, y_{i} | \rvx) = \sum_{\rvz \sim \piref} \piref(\rvz|\rvx, \rvy^{(p)}, y_{i}, ) r_{\phi}(\rvx, \rvy^{(p)}, y_{i}, \rvz)
%\end{aligned}
%$$
%\\
%Hence we get the following values for partial sequences:
%$$
%\begin{aligned}
%\mathbf{V}_{\theta}(\rvy^{(p)},  y^{*}_{i} | \rvx) &= \piref(\rvs^{*}|\rvx, \rvy^{(p)}) r_{\phi}(\rvs^{*}) \\ & ~~ + \piref(\rvs^{**}|\rvx, \rvy^{(p)}) r_{\phi}(\rvs^{**}) \\
%&= (\frac{1}{2}) (6) + (\frac{1}{2})(-6) = 0\\
%\mathbf{V}_{\theta}(\rvy^{(p)},  y'_{i} | \rvx) &= \piref(\rvs'|\rvx, \rvy^{(p)}) r_{\phi}(\rvs') \\ & ~~ + \piref(\rvs''|\rvx, \rvy^{(p)}) r_{\phi}(\rvs'') \\
%&= (\frac{1}{2}) (5) + (\frac{1}{2})(3) = 4
%\end{aligned}
%$$
%\\
%Next, consider inference where we follow the policy:\\
%$$ 
%\begin{aligned}
%&\pi(y_{i} \vert \rvx,\rvy^{(p)}) \propto \piref(y_{i} \vert \rvx,\rvy^{(p)}) \exp(\beta \mathbf{V}_{\theta}(\rvy^{(p)}, y_{i} \vert \rvx))\\
%\Rightarrow ~~  &\pi(y'_{i} \vert \rvx,\rvy^{(p)}) > \pi(y^{*}_{i} \vert \rvx,\rvy^{(p)})
%\end{aligned}
%$$
%\\
%Note following $\pi$ would generate $y'_{i}$ and deviate from the optimal sequence.

%\end{proof}

%\AK{Same as before. Discuss the implication/significane/practicality of this preceding theorem.}

% \subsection{Limitation of PARGS}
% We will use the same notations as above for $\mathbf{V}_{\theta}$, $r_{\phi}$, $\rvx$, $\rvy^{(p)}$ and $\rvs^*$, with new sequence examples:
%  \begin{itemize}
%     \item $\rvs^{**} = \rvx, \rvy^{(p)}, y^*_{i}, y^{**}_{i+1} \cdots, y^{**}_{n}$: another sequence extended from ${\rvx, \rvy^{(p)}, y^*_{i}}$
%     \item $\rvs' = \rvx, \rvy^{(p)}, y'_{i}, y'_{i+1} \cdots, y'_{n}$: another sequence extended from ${\rvx, \rvy^{(p)}}$
%     \item $\mathcal{P} = \{\rvs^{**}, \rvs'\}$ is a pair in the preference dataset 
% \end{itemize} 

% Suppose $r_{\phi}(s^*)=6, \:r_{\phi}(s^{**})=-6, \:r_{\phi}(s')=5$, then $\rvs'$ is the winning sequence in $\mathcal{P}$. By the assumption that "the prefixes of a winning full sequence is also wining against the prefix of the corresponding losing full sequence", partial sequence $(\rvx, \rvy^{p}, y'_{i})$ is also winning $(\rvx, \rvy^{p}, y^*_{i})$.  Following the Bradley-Terry loss, we maximize
% $$
% \begin{aligned}
%      \log \sigma \left( \mathbf{V}_{\theta} (\rvy^{(p)}, y'_{i}|\rvx) - \mathbf{V}_{\theta} (\rvy^{(p)}, y^*_{i}|\rvx)\right)
% \end{aligned}
% $$
% That is, $\mathbf{V}_{\theta} (\rvy^{(p)}, y'_{i}|\rvx) - \mathbf{V}_{\theta} (\rvy^{(p)}, y^*_{i}|\rvx)$ is maximized results in $\mathbf{V}_{\theta} (\rvy^{(p)}, y'_{i}|\rvx) > \mathbf{V}_{\theta} (\rvy^{(p)}, y^*_{i}|\rvx)$

% Next, suppose $\piref(y^{*}|\rvx, \rvy^{(p)}) = \piref(y'|\rvx, \rvy^{(p)}) = \frac{1}{2}$, at inference time we follow the policy:
% $$ 
% \begin{aligned}
% &\pi(y_{i} \vert \rvx,\rvy^{(p)}) \propto \piref(y_{i} \vert \rvx,\rvy^{(p)}) \exp(\beta \mathbf{V}_{\theta}(\rvy^{(p)}, y_{i} \vert \rvx))\\
% \Rightarrow ~~ &\pi(y'_{i} \vert \rvx,\rvy^{(p)}) > \pi(y^{*}_{i} \vert \rvx,\rvy^{(p)})
% \end{aligned}
% $$

% Again, following $\pi$ would generate $y'_{i}$ hence result in become suboptimal.


\section{Proposal}
\label{sec:proposal}
%!TEX root=icml2025.tex

We propose to mitigate the inference overhead and sub-optimal rewards of previous RGTG methods by introducing (i) an efficient reward model and (ii) a novel loss function that will ensure that the resulting value function prefers prefixes extendable to optimal responses.  We name our method FaRMA, i.e. Faster Reward Model for Alignment.

\subsection{An Efficient Reward Model}

We design a reward model architecture so that instead of obtaining a single score for a sequence, we obtain the score for all possible next tokens in the dictionary. We modify \eqref{eq:score} such that:

\begin{equation}
  \label{eq:score-new}
  \begin{aligned}
  \textstyle
  \score(y_{i} | \mathbf{x}, \mathbf{y}_{1:i-1}) = &\log \piref(y_{i} \vert \mathbf{x}, \mathbf{y}_{1:i-1}) \\ 
    &\quad + \beta V_\theta(y_{i} \vert \mathbf{x}, \mathbf{y}_{1:i-1}) ,
  \end{aligned}
\end{equation}

where $V_\theta(.) \in R^{|D|\times 1}$ and $|D|$ is the size of the vocabulary.
% \AK{Vocabulary.}
In order to get the score of sequence $x,y_{1:i}$ we feed the $x,y_{1:i-1}$ into $r_\phi$ and get the score of the sequence with all possible extensions of $y_i$ in the dictionary. 
The efficiency and performance of the reward model is not dependent on $k$, for top-k generation, as we simultaneously get the score for all possible next tokens in the dictionary. We use the same architecture as a causal language model, however, we use a novel training loss which we discuss next. 

% \AK{Need more elaboration regarding the architecture. Is it the same as in the base LLM? Or is it different? If the latter, in which ways? A schematic here will be useful.}

% \AK{Avoid one-sentence paragraphs.}

% Next we discuss the loss that we use for training the reward model.

\subsection{A Principled Constraint}

Given the sub-optimality of the existing methods, there needs to be a more principled way to score partial sequences. 
We propose to score partial sequences based on their optimal extension: 
%\AK{The notations seem to be different than from Sec. 2.}
Given a partial sequence $\rvy_{1:i}$, we consider all possible full extensions and assign the score of the highest completion to $\rvy_{1:i}$. 
Na\"{i}vely, this would require an exponential search in terms of the size of the vocabulary which is intractable. To make this principled goal feasible, we propose a local constraint that the partial-sequence reward model needs to satisfy so that it will return the reward of the corresponding optimal expansion: 
\begin{equation} \label{eq:constraint}
    \mathbf{V}_{\theta}(y_{1:i}|\rvx) = \max_{y_{i+1}} \: \mathbf{V}_{\theta}(y_{1:i+1}|\rvx)
\end{equation}
If the above local constraint is satisfied, then we can keep expanding the sequence as in the generation:
$$
\begin{aligned}
\mathbf{V}_{\theta}(y_{1:i}|\rvx) &= \max_{y_{i+1}} \: \mathbf{V}_{\theta}(y_{1:i+1}|\rvx) \\
&= \max_{y_{i+1}} \: \max_{y_{i+2}}\: \mathbf{V}_{\theta}(y_{1:i+2}|\rvx) \\
&= \cdots = \max_{y_{i+1:n}}\mathbf{V}_{\theta}(y_{1:n}|\rvx),
\end{aligned}
$$
where $y_{i+1:n}$ is the optimal extension beyond $y_{1:i}$ and $y_n$ is the EOS token. That is, instead of doing an exponential search, we could train the value function to satisfy \eqref{eq:constraint}, which can be done by Temporal Difference (TD) learning. Note that VAS also uses TD learning in their algorithm, but, since they use a conventional reward model they do not do a max over the dictionary.

To be more precise, the training process can be separated into two steps with distinct objectives:
\begin{enumerate}
    \item Standard BT loss on full sequence preference dataset:
        \begin{equation} \label{eq:bradley-terry-new}
        \mathcal{L}_{(a)} = - \E_{\rvx, \rvy^w, \rvy^l \sim \D} \log \sigma ( \mathbf{V}_{\theta} (\rvy^w|\rvx) - \mathbf{V}_{\theta} (\rvy^l|\rvx))
        \end{equation}
    \item Constraint to ensure optimal partial sequence expansion.
        \begin{equation} \label{eq:constraint_loss}
            \mathcal{L}_{(b)} = \frac{1}{2}\left[\mathbf{V}_{\theta}(y_{1:i}|\rvx) - \max_{y_{i+1}} \:\mathbf{V}_{\theta}(y_{1:i+1}|\rvx)\right]^2       
        \end{equation}
\end{enumerate}

Firstly, we want to point out the similarity of our constraint \eqref{eq:constraint_loss} to TD control where $V(\rvy|\rvx)$ can be treated as a state-action value function (i.e., Q-function) with $y_i$ corresponding to the action and $[\rvx,\rvy_{1:i-1}]$ corresponding to the state. Note also that transitions are deterministic in LLMs since the action $y_i$ updates the state to $[\rvx,\rvy_{1:i}]$ deterministically.  We use $s$ to denote a state and $a$ to denote an action in Bellman's equation:
    $$
    \begin{aligned}
        &Q^{*}(s,a) = \mathbb{E}[r|s,a] + \gamma \sum_{s'}\mathbb{P}(s'|s,a) \max_{a'} Q^{*}(s', a')\\
        \Rightarrow \: &Q^{*}([\rvx,\rvy_{1:i-1}], y_i) = \max_{y_{i+1}} Q^{*}([\rvx,\rvy_{1:i}], y_{i+1}) \label{eq:bellman}\\
        \Rightarrow \: &\mathbf{V}_{\theta}(\rvy_{1:i}|\rvx) = \max_{y_{i+1}} \: \mathbf{V}_{\theta}(\rvy_{1:i+1}|\rvx)
    \end{aligned}
    $$

Note that \cref{eq:bellman} follows from the fact that there is no discount factor and no reward until the end of the sequence.
Then $\mathcal{L}_{(b)}$ is the same loss as in Q-gradient learning by treating $\max\limits_{y_{i+1}} \:\mathbf{V}_{\theta}(y_{1:i+1}|\rvx)$ as the target. 

To train the value function, we alternate between the two losses mentioned previously. For the Bradley-Terry loss \eqref{eq:bradley-terry-new}, we utilize full-sequence preference pairs as commonly done when training a reward model. Furthermore, for the new constraint loss \eqref{eq:constraint_loss}, we extract partial sequences from the winning sequences in the preference dataset and use them as the training data. Notably, there is no preference signal when training with the constraint loss. The model simply learns to align its scores to the best next token. We trained the model by alternating between the two losses. The training details are presented in Appendix \ref{app:training}.

%Therefore, we need to train the constraint in an iterative backward manner: 
%we start to train on partial sequences that are derived by removing the last token of full sequences and use optimal full sequence reward score as the target. Then, we remove the last two tokens of full sequences while using the optimal score computed from the partial sequences derived by removing the last token of full sequences as the target.  For instance, we have a full sequence $(\rvx, y_{1:n})$ from the dataset. In the first iteration, we would train on the partial sequence $(\rvx, y_{1:n-1})$ and use $\max\limits_{y_n} \mathbf{V}_{\theta}(\rvx, y_{1:n})$ as the target. Then, in the second iteration, we train on partial sequence $(\rvx, y_{1:n-2})$ and use $\max\limits_{y_{n-1}} \mathbf{V}_{\theta}(\rvx, y_{1:n-1})$ as the target, as at this iteration the reward model could provide a reliable score for $\mathbf{V}_{\theta}(\rvx, y_{1:n-1})$. We keep iterating by reducing the length of the partial sequence. Also, the model is frozen when it is computing the target and updated at the end of each iteration.

% Since this technique is similar to Q-gradient learning which is a well-studied training paradigm, relevant training tricks such as alternative training and rebuffer play can be applied to optimize the performance.

We emphasize that this kind of training would \emph{not} be possible with the reward models of previous RGTG methods that require $|D|$ forward passes to calculate the max over all the tokens in the dictionary $D$. 
Instead we calculate the max after a single forward pass. The complete algorithm for our method is presented in Alg.~\ref{alg:rgtg_train}.


\begin{algorithm}[t]
  \small
  \caption{Our Training Algorithm.} %\AK{Always use \texttt{text} for words, e.g., loss, max, iter etc. in math mode.}
  \label{alg:rgtg_train}

  \begin{algorithmic}[1]
    \REQUIRE Base LLM to initialize the reward model $V_\theta$, Full Sequence Preference dataset \(\mathbf{D_{BT}} = \{ (\mathbf{x}^k, \mathbf{y}^{wk}, \mathbf{y}^{lk}) \}_{k=1}^{K_{BT}}\), number of alternating iterations $\text{iter}_n$, mini-batch size $n$,  partial sequence dataset \(\mathbf{D_{max}} = \{ (\mathbf{x}^k, \mathbf{y}^{k}\}_{k=1}^{K_\text{max}}\)
    \ENSURE $\mathbf{V}_{\theta}$
    \vspace{1em}
    
    \FOR{$i$ = 1 to $\text{iter}_n$}
    \STATE Sample minibatch $\mathbf{D_{BT}^{(i)}}$ from $\mathbf{D_{BT}}$ of size $n$
        \FOR{every tuple $(\rvx, \rvy^w, \rvy^l)\in \mathbf{D_{BT}^{(i)}}$}
        \STATE Compute $\mathbf{V}_{\theta} (\rvy^w|\rvx)$ and $\mathbf{V}_{\theta} (\rvy^l|\rvx)$
        \STATE $\mathcal{L}_a = \log \sigma ( \mathbf{V}_{\theta} (\rvy^w|\rvx) - \mathbf{V}_{\theta}(\rvy^l|\rvx))$
        \STATE Update $\mathbf{V}_{\theta}$ based on loss $\mathcal{L}_a$ 
        \ENDFOR
    \STATE Sample minibatch $\mathbf{D_{max}^{(i)}}$ from $\mathbf{D_{max}}$ of size $n$
        \FOR{every tuple $(\rvx, \rvy)\in \mathbf{D_{max}^{(i)}}$}
        \STATE Compute $\mathbf{V}_{\theta} (\rvy|\rvx)$
        \STATE $V_{max} = \max_{y_{|\rvy| + 1}} \:\mathbf{V}_{\theta}(\rvy, y_{|\rvy| + 1}|\rvx)$
        \STATE $\mathcal{L}_b = \frac{1}{2}\left[\mathbf{V}_{\theta} (\rvy|\rvx) - V_\text{max}\right]^2$
        \STATE Update $\mathbf{V}_{\theta}$ based on loss $\mathcal{L}_b$
        \ENDFOR
    \ENDFOR
    % \STATE   $\mathbf{V}_{\theta}$
  \end{algorithmic}
\end{algorithm}

\begin{algorithm}[t]
    \small
    \caption{Our Decoding Algorithm.}
    \label{alg:rgtg_decode}

    \begin{algorithmic}[1]
        \REQUIRE Reward model $\mathbf{V}_{\theta}$, Prompt $\rvx$, top-k parameter $k$, hyperparameter $\beta > 0$, any reference/SFT model $\piref$, generation length $l$
        \ENSURE $\rvy_{1:l}$: A generated response to $\rvx$ of length $l$

        \vspace{1em}
        
        \FOR{i = 1 to $l$}
            %\State Reward $\mathbf{V}_{\theta}(v \vert \rvx, \rvy_{1:i-1})$
            %\State Logit $\piref(v \vert \rvx, \rvy_{1:i-1})$
            \STATE $\log \pi(y_i=v \vert \rvx, \rvy_{1:i-1}) \leftarrow$
            \STATE \qquad\qquad $\log \left(\piref(v \vert \rvx, \rvy_{1:i-1}) + \beta \mathbf{V}_{\theta}(v \vert \rvx, \rvy_{1:i-1})\right)$
            \STATE $y_{i} \sim \softmax(\mathtt{top\_k}(\log \pi(y_i \vert \rvx, \rvy_{1:i-1})))$
        \ENDFOR  
        % \STATE  $\rvy_{1:l}$
    \end{algorithmic}
\end{algorithm}

We now prove that unlike PARGS and CD, our algorithm is guaranteed to prefer prefixes that are extendable to optimal full sequences.


\begin{theorem}
In the limit of infinite training data and a sufficiently expressive representation for the value function, our algorithm guarantees that the learned value function scores prefixes that can be extended to optimal full sequences at least as high as any other prefix.  More precisely, if $\rvy^* = \argmax_{\rvy} r(\rvy|\rvx)$, then 
\begin{equation}
V(\rvy^*_{1:i}|\rvx) \ge V(\rvy'_{1:j}|\rvx) \; \forall i,j,\rvy'
\end{equation}
\end{theorem}

\begin{proof}
We provide a proof by contradiction.  Let $\rvy^*$ be an optimal response to $\rvx$ and $\rvy'$ be any other response.  Suppose that
\begin{equation}
    \exists i,j,\rvy' \mbox{ such that } V(\rvy'_{1:j}|\rvx) > V(\rvy^*_{1:i}|\rvx) \label{eq:hypothesis}
\end{equation}  
Since the loss in \eqref{eq:constraint_loss} ensures that the learned value function returns the reward of the best full sequence that extends a prefix then $V(\rvy^*_{1:i}|\rvx) = r(\rvy^*|\rvx)$.  Similarly, since $\rvy'_{1:j}$ is any other prefix whose extensions do not lead to better full sequences, then $V(\rvy'_{1:j}|\rvx) <= r(\rvy^*|\rvx)$.  This means that $V(\rvy'_{1:j}|\rvx) \le V(\rvy^*_{1:i}|\rvx)$, which contradicts \eqref{eq:hypothesis}.
\end{proof}




\section{Related Work}
\label{sec:proposal}
%!TEX root=iclr2025_conference.tex
\paragraph{Training based Alignment}

Supervised fine-tuning and instruction tuning~\citep{wei2021finetuned} are common methods to align an LLM to labeled data. RLHF~\citep{christiano2017deep,ziegler2019fine,lee2021pebble,nakano2021webgpt,snell2022offline} methods can align an LLM directly to human preferences. First, a reward model is trained on a dataset of human preferences using the Bradley Terry model ~\citep{bradleyterry1952paired} and then the LLM is updated, based on the reward model, using an RL algorithm such as PPO~\cite{schulman2017proximal}. However, updating the LLM with RL is expensive and researchers have explored cost-effective alternatives.

\cite{liu2023chain} convert the preference data into sequence of sentences which are then used to fine-tune the LLM. \cite{dong2023raft} used the reward model to filter high quality training samples and fine-tunes on them avoiding undesirable behavior. DPO \cite{rafailov2023direct,rafailov2024qfunction} avoids learning a reward model explicitly and finds an equivalent objective to RLHF which can be optimized by supervised learning. Even though the resulting optimization is cheaper than RL, nonetheless, it still involves updating the LLM.

Preference data itself provides sequence-level supervision. Some works have atttempted to collect and use fine-grained preferences by using either human annotators~\citep{wu2024fine} or LLMs~\citep{cao2024sparserewardsenhancingreinforcement}.


\paragraph{Guided Decoding}

In the guided decoding literature, a number of methods consider guidance at a step or process level~\citep{welleck2022naturalprover,uesato2022solving, lightman2023let, krishna2022rankgen,li2023making, khalifa2023grace, yao2023tree}. 

Some methods have applied token-level functions~\citep{dathathri2019plug, krause2021gedi, yang2021fudge,chaffin2022ppl, liu2023attribute} but they do not consider RGTG based on preference data. 

\citet{khanov2023alignment} introduce an RGTG method but they rely on a full-sequence reward model for partial sequence decoding. \citet{deng2023reward} learn to distill a partial sequence reward model, starting from the full-sequence model using a square loss function. \citet{mudgalcontrolled} employ a similar approach but instead of using preference data, generate a dataset by roll-outs from the base LLMs. \citet{han2024value} also use the base LLM to gather a dataset but employ TD learning to train the partial sequence reward model. Different from these works, \citet{zhao2024probabilistic} derive an RGTG method based on sequential Monte Carlo and demonstrate that it can approximate RLHF.



  





\section{Experiments}
\label{sec:experiments}
\section{Proof of Concept Experiments}
\label{sec:experiments}

%\begin{itemize}
%    \item joint exploration non e' spesso un opzione
%    \item specificare che le policy sono decentralizzate a differenza di tutti i casi precedenti
%    \item decentralizzata con feedback decentralizzato non si coordina e il problema e' abbastanza semplice da portare a policy quasi deterministiche
%\end{itemize}



%\mirco{questo primo paragrafo è un po' convoluto. Prova a ristruttura la sezione in questo modo: quali sono le domande a cui cerchiamo risposta? Quali sono i domini sperimentali? Quali sono gli algoritmi che compariamo? Quali sono i take away? Per l'ultimo potresti anche evidenziare qualche frase in grassetto o emph con le principali conclusioni empiriche}

In this section, we provide some empirical validations of the findings discussed so far. Especially, we aim to answer the following questions: (\textbf{a}) Is Algorithm~\ref{alg:trpe} actually capable of optimizing finite-trials objectives? (\textbf{b}) Do different objectives enforce different behaviors, as expected from Section~\ref{sec:problem_formulation}? (\textbf{c}) Does the \emph{clustering} behavior of mixture objectives play a crucial role? If yes, when and why?\\
Throughout the experiments, we will compare the result of optimizing finite-trial objectives, either joint, disjoint, mixture ones, through Algorithm~\ref{alg:trpe} via fully decentralized policies. The experiments will be performed with different values of the exploration horizon $T$, so as to test their capabilities in different exploration efficiency regimes.\footnote{The exploration horizon $T$, rather than being a given trajectory length, has to be seen as a parameter of the exploration phase which allows to tradeoff exploration quality with exploration efficiency.} The full implementation details are reported in Appendix~\ref{apx:exp}.
\vspace{-6pt}
\paragraph*{Experimental Domains.}~The experiments were performed on two domains. The first is a notoriously difficult multi-agent exploration task called \emph{secret room}~\citep[MPE,][]{pmlr-v139-liu21j},\footnote{We highlight that all previous efforts in this task employed centralized policies. We are interested on the role of the entropic feedback in fostering coordination rather than full-state conditioning, then maintaining fully decentralized policies instead.} referred to as  Env.~(\textbf{i}). In such task, two agents are required to reach a target while navigating over two rooms divided by a door. In order to keep the door open, at least one agent have to remain on a switch. Two switches are located at the corners of the two rooms. The hardness of the task then comes from the need of coordinated exploration, where one agent allows for the exploration of the other. The second is a simpler exploration task yet over a high dimensional state-space, namely a 2-agent instantiation of \emph{Reacher}~\citep[MaMuJoCo,][]{peng2021facmac}, referred to as Env.~(\textbf{ii}). Each agent corresponds to one joint and equipped with decentralized policies conditioned on her own states. In order to allow for the use of plug-in estimator of the entropy~\citep{paninski2003}, each state dimension was discretized over 10 bins.


\begin{figure*}[!]
    \centering
    \input{figures/pretraining_legend.tex}
    %\hfill
    \vfill
    %vspace{-0.2cm}
    \begin{subfigure}[b]{0.3\textwidth}
        \includegraphics[width=\textwidth]{figures/room_150_AverageReturnnokl.pdf}
        %\vspace{-0.8cm}
        \caption{\centering MA-TRPO with TRPE Pre-Training (Env.~(\textbf{i}), $T=150$).}
        \label{subfig:image9}
    \end{subfigure}
    \hfill
    \begin{subfigure}[b]{0.3\textwidth}
        \includegraphics[width=\textwidth]{figures/room_50_AverageReturnnokl.pdf}
        %\vspace{-0.8cm}
        \caption{\centering MA-TRPO with TRPE Pre-Training (Env.~(\textbf{i}), $T=50$).}
        \label{subfig:image10}
    \end{subfigure}
    \hfill
    \begin{subfigure}[b]{0.3\textwidth}
        \centering
        \includegraphics[width=0.8\textwidth]{figures/hand_100_AverageReturn.pdf}
        %\vspace{-0.8cm}
        \caption{\centering MA-TRPO with TRPE Pre-Training (Env.~(\textbf{ii}), $T=100$).}
        \label{subfig:image11}
    \end{subfigure}
\caption{\centering Effect of pre-training in sparse-reward settings.(\emph{left}) Policies initialized with either Uniform or TRPE pre-trained policies over 4 runs over a worst-case goal. (\emph{rigth}) Policies initialized with either Zero-Mean or TRPE pre-trained policies over 4 runs over 3 possible goal state. We report the average and 95\% c.i.}
\label{fig:pretraining}
\end{figure*}
\vspace{-10pt}
\paragraph*{Task-Agnostic Exploration.}~Algorithm~\ref{alg:trpe} was first tested in her ability to address task-agnostic exploration \emph{per se}. This was done by considering the well-know hard-exploration task of Env.~(\textbf{i}). The results are reported in Figure~\ref{fig:room} for a short exploration horizon $(T=50)$. Interestingly, at this efficiency regime, when looking at the joint entropy in Figure~\ref{subfig:image2}, joint and disjoint objectives perform rather well compared to mixture ones in terms of induced joint entropy, while they fail to address mixture entropy explicitly, as seen in Figure~\ref{subfig:image3}. On the other hand mixture-based objectives result in optimizing both mixture \emph{and} joint entropy effectively, as one would expect by the bounds in Th.~\ref{lem:entropymismatch}. By looking at the actual state visitation induced by the trained policies, the difference between the objectives is apparent. While optimizing joint objectives, agents exploit the high-dimensionality of the joint space to induce highly entropic distributions even without exploring the space uniformly via coordination (Fig.~\ref{subfig:image5}); the same outcome happens in disjoint objectives, with which agents focus on over-optimizing over a restricted space loosing any incentive for coordinated exploration (Fig.\ref{subfig:image6}). On the other hand, mixture objectives enforce a clustering behavior (Fig.\ref{subfig:image6}) and result in a better efficient exploration. 

\paragraph*{Policy Pre-Training via Task-Agnostic Exploration.}~More interestingly, we tested the effect of pre-training policies via different objectives as a way to alleviate the well-known hardness of sparse-reward settings, either throught faster learning or zero-short generalization. In order to do so, we employed a multi-agent counterpart of the TRPO algorithm~\cite{schulman2017trustregionpolicyoptimization} with different pre-trained policies. First, we investigated the effect on the learning curve in the hard-exploration task of Env.~(\textbf{i}) under long horizons ($T=150$), with a worst-case goal set on the the opposite corner of the closed room. Pre-training via mixture objectives still lead to a faster learning compared to initializing the policy with a uniform distribution. On the other hand, joint objective pre-training did not lead to substantial improvements over standard initializations. More interestingly, when extremely short horizons were taken into account ($T=50$) the difference became appalling, as shown in Fig.~\ref{subfig:image9}: pre-training via mixture-based objectives leaded to faster learning and higher performances, while pre-training via disjoint objectives turned out to be even \emph{harmful} (Fig.~\ref{subfig:image10}). This was motivated by the fact that the disjoint objective overfitted the task over the states reachable without coordinated exploration, resulting in almost deterministic policies, as shown in Fig~\ref{fig:333} in Appendix~\ref{apx:exp}. Finally, we tested the zero-shot capabilities of policy pre-training on the simpler but high dimensional exploration task of Env.~(\textbf{ii}), where the goal was sampled randomly between worst-case positions at the boundaries of the region reachable by the arm. As shown in Fig.~\ref{subfig:image11}, both joint and mixture were able to guarantee zero-shot performances via pre-training compatible with MA-TRPO after learning over $2$e$4$ samples, while disjoint objectives were not. On the other hand, pre-training with joint objectives showed an extremely high-variance, leading to worst-case performances not better than the ones of random initialization. Mixture objectives on the other hand showed higher stability in guaranteeing compelling zero-shot performance.
\vspace{-6pt}
\paragraph*{Take-Aways.}~Overall, the proposed proof of concepts experiments managed to answer to all of the experimental questions: (\textbf{a}) Algorithm~\ref{alg:trpe} is indeed able to explicitly optimize for finite-trial entropic objectives. Additionally, (\textbf{b}) \textbf{mixture distributions enforce diverse yet coordinated exploration}, that helps when high efficiency is required. Joint or disjoint objectives on the other hand may fail to lead to relevant solutions because of under or over optimization. Finally, (\textbf{c}) \textbf{efficient exploration} enforced by mixture distributions was shown to be a \textbf{crucial factor} not only for the sake of task-agnostic exploration per se, but also for the ability of \textbf{pre-training via task-agnostic exploration} to lead to \textbf{faster and better training} and even \textbf{zero-shot generalization}.

\section{Conclusion}
\label{sec:conclusion}
\section{Discussion and Conclusion}
\label{sec:discuss}

We presented \bench, the first framework  and experimental platform to benchmark AI Agents for IT automation tasks. \bench strives to capture the complexity of modern IT systems and the diversity of IT tasks. The reproducibility of \bench ensures the community-driven effort despite inherent nondeterminism of large-scale IT systems. 

One of the key design principles of \bench is ensuring its flexibility to support diverse areas of different IT systems
and its extensibility to new scenarios. While current scope of \bench is comprehensive and representative, we plan to further enrich the benchmark suites by adding other important processes essential to modern IT automation. Furthermore, we plan to expand our benchmarking beyond event-triggered scenarios. 
We are actively working to expand scenario coverage for the supported processes and promote growth through open-community contributions.
 We invite the community to reproduce their real-world-inspired incidents in a synthetic sandboxed environment leveraging the \bench. We expect that everyone contributing can bring their expertise to the table.

We expect \bench to drive the innovations of AI agent-based techniques with a direct impact on the safety, efficiency, and intelligence of today’s IT infrastructures. 
With \bench, we are starting to explore many deep, exciting open problems: How to develop domain-specific AI agents that specialize in certain types of IT tasks? How to orchestrate multiple agents with various expertise to collaborate on bigger projects? How can we ensure safety of agent-driven solutions? How can we effectively use human-in-the-loop while developing diverse adaptive agents? We invite everyone to participate in answering these questions and realizing the vision of using AI agents to automate critical IT tasks.




% \bibliography{icml2025}
% \bibliographystyle{icml2025}

\clearpage


% In the unusual situation where you want a paper to appear in the
% references without citing it in the main text, use \nocite
% \nocite{langley00}

\section*{Impact Statement}


The goal of this paper is to push the frontiers of Machine Learning. This may lead to impacts on the society, however, we do not feel the need to highlight them here.

\bibliography{icml2025}
\bibliographystyle{icml2025}


%%%%%%%%%%%%%%%%%%%%%%%%%%%%%%%%%%%%%%%%%%%%%%%%%%%%%%%%%%%%%%%%%%%%%%%%%%%%%%%
%%%%%%%%%%%%%%%%%%%%%%%%%%%%%%%%%%%%%%%%%%%%%%%%%%%%%%%%%%%%%%%%%%%%%%%%%%%%%%%
% APPENDIX
%%%%%%%%%%%%%%%%%%%%%%%%%%%%%%%%%%%%%%%%%%%%%%%%%%%%%%%%%%%%%%%%%%%%%%%%%%%%%%%
%%%%%%%%%%%%%%%%%%%%%%%%%%%%%%%%%%%%%%%%%%%%%%%%%%%%%%%%%%%%%%%%%%%%%%%%%%%%%%%
\newpage
\appendix
\onecolumn

\label{sec:appendix}
\section{APPENDIX}
% Tarik: Also mention the evaluation procedure, how many runs are recorded? What is the executed action horizon? Are we doing any action chunking/blending? 
In appendix, we present the implementation details of CEM and policy training.
% \begin{figure*}[t]
% \centering
% \includegraphics[width=1.0\textwidth]{figures/allegro_franka_snapshots.png}
% 	\caption{Snapshots of trajectories generated from a single demonstration for Allegro hand and bimanual Panda arms.}
%     \label{fig:allegro_franka_snapshots}
% \end{figure*}
\subsection{CEM Implementation Details}
\label{sec:appendix_cem}
\begin{table}
\centering
        \renewcommand{\arraystretch}{0.8}
        \begin{threeparttable}
        \begin{tabular}{@{}lccccc@{}}
        \toprule
        Parameter & $T$ & Plan Duration & $q_o$ & $q_r$ & $r_u$ \\
        \midrule
        Floating Allegro Hand & 6 & 1.25 s & 10 & 0.01 & 0.1 \\
        Bimanual iiwa Arms & 6 & 1.25 s & 10 & 0.01 & 10 \\
        Bimanual Panda Arms & 6 & 2.0 s & 10 & 0.01 & 10 \\
        \bottomrule
        \end{tabular}
        \end{threeparttable}
        \caption{\textbf{Parameters for CEM. } $T$: planning horizon. $q_o$: scalar weight for tracking object trajectories. $q_r$: scalar weight for tracking robot trajectories. $r_u$: scalar weight for control input.}
        \label{tab:cem_params}
\end{table}
We provide detailed parameters for the CEM implementation in Table \ref{tab:cem_params}. We optimize over the action knot points $u_{0:T-1}$, which are linearly interpolated to generate action commands sent to Drake. Drake simulates the contact dynamics $f$ at 200 Hz. The state cost matrix  $Q_t = diag(q_o \cdot \mathbf{1}_{n_o}, q_r \cdot \mathbf{1}_{n_r})$, where $n_o$ and $n_r$ denote the object and robot state dimensions, and $\mathbf{1}$ is a vector or all 1's. The terminal state cost matrix $Q_T = 10 \cdot Q_t$. The input cost matrix $R_t = diag(r_u \cdot \mathbf{1}_{n_u})$, where $n_u$ represents the control input dimension. All of the systems adopt 50 samples, 5 elites and initial standard deviation $\sigma = 0.05 \cdot \mathbf{1}_{n_u}$ for action sampling.
\subsection{Policy Implementation Details}
We train UNet-based diffusion policies \cite{chi2023diffusion} for all tasks. The action space is the robot configuration (joint angles, and additional floating base coordinates for the Allegro hand), while the observation space is the robot configuration and object pose (with orientations represented by rotation matrices). Detailed parameters are listed in Table \ref{tab:diffo_po_params}.

\begin{table}
\centering
        \renewcommand{\arraystretch}{0.8}
        \begin{threeparttable}
        \begin{tabular}{@{}lcccccc@{}}
        \toprule
        Parameter & $T_o$ & $T_a$ & Freq & Epochs & Obs. Dim. & Act. Dim. \\
        \midrule
        Floating Allegro Hand & 10 & 40 & 50 & 1000 & 34 & 22 \\
        Bimanual iiwa Arms & 10 & 40 & 20 & 800 & 26 & 14\\
        Bimanual Panda Arms & 10 & 40 & 50 & 800 & 26 & 14\\
        \bottomrule
        \end{tabular}
        \end{threeparttable}
        \caption{\textbf{Parameters for diffusion policies. } $T_o$: observation horizon. $T_a$: action horizon. Freq: environment frequency (Hz, both observations and actions).}
        \label{tab:diffo_po_params}
\end{table}




%\section{You \emph{can} have an appendix here.}

%You can have as much text here as you want. The main body must be at most $8$ pages long.
%For the final version, one more page can be added.
%If you want, you can use an appendix like this one.  

%The $\mathtt{\backslash onecolumn}$ command above can be kept in place if you prefer a one-column appendix, or can be removed if you prefer a two-column appendix.  Apart from this possible change, the style (font size, spacing, margins, page numbering, etc.) should be kept the same as the main body.

%%%%%%%%%%%%%%%%%%%%%%%%%%%%%%%%%%%%%%%%%%%%%%%%%%%%%%%%%%%%%%%%%%%%%%%%%%%%%%%
%%%%%%%%%%%%%%%%%%%%%%%%%%%%%%%%%%%%%%%%%%%%%%%%%%%%%%%%%%%%%%%%%%%%%%%%%%%%%%%


\end{document}


% This document was modified from the file originally made available by
% Pat Langley and Andrea Danyluk for ICML-2K. This version was created
% by Iain Murray in 2018, and modified by Alexandre Bouchard in
% 2019 and 2021 and by Csaba Szepesvari, Gang Niu and Sivan Sabato in 2022.
% Modified again in 2023 and 2024 by Sivan Sabato and Jonathan Scarlett.
% Previous contributors include Dan Roy, Lise Getoor and Tobias
% Scheffer, which was slightly modified from the 2010 version by
% Thorsten Joachims & Johannes Fuernkranz, slightly modified from the
% 2009 version by Kiri Wagstaff and Sam Roweis's 2008 version, which is
% slightly modified from Prasad Tadepalli's 2007 version which is a
% lightly changed version of the previous year's version by Andrew
% Moore, which was in turn edited from those of Kristian Kersting and
% Codrina Lauth. Alex Smola contributed to the algorithmic style files.
