\begin{figure}[t]
    \begin{tcolorbox}[width=\columnwidth,title=,colupper=black]\scriptsize
        \textbf{Instruction} = Compare and assess the statements of Witness A and Witness B regarding [specific aspect, e.g., identification, action, object, timeline, location, motive]. Identify agreements, contradictions, or missing information between their accounts. \\
        \textbf{Fill in the \texttt{mask} accordingly:}
        \begin{itemize}[itemsep=-2pt]
            \item[--] Witness A's \textbf{identification} of the person \texttt{[mask]} Witness B's identification.
            \item[--] Witness A's \textbf{described action} \texttt{[mask]} Witness B's description.
            \item[--] Witness A's \textbf{described object} \texttt{[mask]} Witness B's described object.
            \item[--] Witness A's reported \textbf{timeline} \texttt{[mask]} Witness B's reported timeline.
            \item[--] Witness A's reported \textbf{location} \texttt{[mask]} Witness B's reported location.
            \item[--] Witness A's reported \textbf{reason} \texttt{[mask]} Witness B's reported reason.
        \end{itemize}
        \textbf{\texttt{[mask]} options are:}
        \begin{itemize}[itemsep=-2pt]
            \item[ ] \textbf{agrees with:} Use when both testimonies provide consistent or additional details that align with each other.
            \item[ ] \textbf{contradict:} Use when testimonies directly conflict with each other.
            \item[ ] \textbf{is absent from:} Use when one testimony does not provide information on a detail covered by the other.
        \end{itemize}
        % """
    \end{tcolorbox}  
    \vspace{-3mm}
    \caption{An example of prompt with $6Ws$ instruction for incongruence detection.}
    \label{fig:prompt:6w}
\end{figure}