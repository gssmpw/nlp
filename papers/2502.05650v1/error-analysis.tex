%%%%%%%%%%%%%%%%%%%%%%%%%%%%%%%%%%%%%%%%%%%%%%%%%%%%%%%%
\begin{table*}[ht]
\centering

\resizebox{\textwidth}{!}{
\begin{tabular}{lcp{25em}p{25em}p{20em}}
\toprule
\multirow{1}{*}{\bf Model} & \multirow{1}{*}{\bf Hop} & \multicolumn{1}{c}{\bf Incongruent Span in $T1$ } & \multicolumn{1}{c}{\bf Incongruent Span in $T2$ } & \multirow{1}{*}{\bf Remarks} \\ 
% \cmidrule{3-4}
% & & \multicolumn{1}{c}{$T1$} & \multicolumn{1}{c}{$T2$} & \\ 
\midrule \midrule
\rowcolor{cyan!60} \multicolumn{4}{l}{\textbf{Context:} How did Conner Help Shaun?} & \\ \hline
Gold & & Conner gave a \textcolor{blue!90}{\emph{\textbf{gave him the money and like other facilities.}}} and and told him to go for underground for like three months that was the help. & so conner \textcolor{blue!90}{\emph{\textbf{provided shaun with the contact of a man named Broadridge.}}} He said that Broadridge will be able to help you deal with the italian gang. &  Actual Incongruent pair  \\ \midrule
                   
\multirow{5}{*}{\model} & 1 & \multicolumn{2}{l}{\multirow{1}{60em}{Overall, while the two accounts differ in the specifics of how Conner helped Shaun, they both suggest that Conner played a role in supporting Shaun in some way, whether it was providing financial assistance, logistical support, or connecting him with someone who could help him with his problems.}} & \multirow{3}{20em}{The models failed to accurately capture the specific differences in how Conner assisted Shaun in 2\textsuperscript{nd} hop, mistakenly treating both events as the same rather than recognizing them as contradictory.} \\ \\

& 2 & \multicolumn{2}{l}{\multirow{1}{60em}{The key difference between the two accounts is the nature of the assistance provided by Conner. However, both accounts suggest that Conner helped Shaun in some way, and there is no direct conflict between the details provided.}} &  \multirow{3}{20em}{} \\ \\
% in the 1st hop model captured that Shaun is being assisted by corner, and the nature of help is different. But for the 2nd hop model didn't delve into the \textbf{HOW} part of the context and

& 3 & \textcolor{red!90}{\emph{\textbf{No contradiction}}} & 
 & \multirow{3}{20em}{} \\ \midrule{}

Baseline & & \textcolor{red!90}{\emph{\textbf{Conner}}} \textcolor{darkgreen!90}{\emph{\textbf{gave a gave him the money and like other facilities}}} \textcolor{red!90}{\emph{\textbf{and and told him to go for underground for like three months that was the help.}}} & \textcolor{red!90}{\emph{\textbf{Conner}}} \textcolor{darkgreen!90}{\emph{\textbf{provided shaun with the contact of a man named Broadridge.}}} & \multirow{3}{20em}{Model predicted contradiction, But with incorrect boundaries} \\ 

%%%%%%%%%%%%%%%%%%%%%%%%%%%%%%%%%%%%%%%%%%%%%%%%%%%%%

% \midrule \midrule
% \rowcolor{cyan!60} \multicolumn{4}{l}{\textbf{Context:} Why was Veer committing this crime?} & \\ \hline
%  Gold & & Veer said that remember the day when my bike crashed with your car and you were wrong as you are in the wrong way and you didn't even say sorry and You slapped me and then Naman said that we were angry and that's why, I'm not even driving the car and then Veer said that you said you should save sorry at that time  & Veer explained that yesterday. Those guys after meeting with the accident humiliated him and did not apologize and that is the reason why Veer is punishing them now. &  Actual congruent pair \\ \midrule
                   
% \multirow{7}{*}{\model} & 1 & \multicolumn{2}{l}{\multirow{1}{60em}{The key points that stand out are the past incident where Veer felt wronged by the respondent, and the lack of apology or accountability from the respondents in both testimonies.}} & \multirow{3}{20em}{In the 1\textsuperscript{st} hop, the model correctly identified the key details, but in the 2\textsuperscript{nd} hop, it mistakenly flagged mutually inclusive statements as contradictions.} \\ \\ 

%  & 2 & \multicolumn{2}{l}{\multirow{1}{60em}{The inconsistency arises from the fact that in Account A, Veer specifically mentions that the respondent slapped him, implying a physical altercation. However, in Account B, Veer doesn't mention any physical altercation and instead focuses on the lack of apology and accountability from the respondents. This discrepancy suggests that either Veer's account of the incident is inconsistent or he is presenting a biased or partial view of the events.}} &  \multirow{3}{20em}{} \\ \\ \\ \\

%  & 3 & \textcolor{red!90}{\emph{\textbf{You slapped me}}} & \textcolor{red!90}{\emph{\textbf{did not apologize}}}
%   & \multirow{3}{*}{} \\ \midrule{}

%  Baseline & & \textcolor{red!90}{\emph{\textbf{Veer said that remember the day when my bike crashed with your car and you were wrong as you are in the wrong way and you didn't even say sorry and You slapped me}}} & \textcolor{red!90}{\emph{\textbf{Veer explained that yesterday. Those guys after meeting with the accident humiliated him and did not apologize}}} & 
%   \multirow{3}{20em}{The model predicted contradictions even when the testimonies logically described the same event.} \\ 

%%%%%%%%%%%%%%%%%%%%%%%%%%%%%%%%%%%%%%%%%%%%%%%%%%%%%



\bottomrule
\end{tabular}}
\vspace{-3mm}
\caption{Error Analysis. Gold spans are highlighted in blue. Red text denotes incorrect spans by our model \model\ and baseline (Llama-3 Few-Shot), while the green text highlights the correct spans for incongruent testimony pairs.
}
\label{tab:error}
\end{table*}

%  Error analysis of the outputs. Bold text (green) highlights the correct claim span whereas text in italics(red) represents the mistakes committed by our model, DABERTa, and vanilla RoBERTa as baseline.

% & & \hl{Case1: Actual Incongruent pair: Span in both T1 and T2: Incongruent prediction.} The above case is this one.\\ DONE

% & & \hl{Case 2: Actual Incongruent pair: Span in either one of them or none of them: non-incongruent prediction.} \\   DONE

% & & \hl{Case 3: Actual congruent pair: Span in either one of them or both of them: Incongruent prediction.} \\  DONE

% & & \hl{Case 4: Actual congruent pair: Span in neither of them: congruent prediction.} \\ DONE