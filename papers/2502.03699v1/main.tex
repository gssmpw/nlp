% This class has a lot of options, so please check deepmind.cls for more details.
% This is a minimal set for most needs.
\documentclass[11pt, a4paper, logo, copyright]{googlecloud}

% Omit dates for reproducibility.
\pdfinfoomitdate 1
\pdftrailerid{redacted}

% This avoids duplicate hyperref bookmark entries when using \bibentry (e.g. via \citeas).
\makeatletter
\renewcommand\bibentry[1]{\nocite{#1}{\frenchspacing\@nameuse{BR@r@#1\@extra@b@citeb}}}
\makeatother

\usepackage{kantlipsum, lipsum}
\usepackage{dsfont}
% \usepackage{gdm-colors}

% Sometimes you will get errors about pdflink ending up in diffrent position. Try this and
% comment it out again when you are done with your document.
%\hypersetup{draft}

% Set the bibliography options here.
\usepackage[authoryear, sort&compress, round]{natbib}

\usepackage[utf8]{inputenc} % allow utf-8 input
\usepackage[T1]{fontenc}    % use 8-bit T1 fonts
\usepackage{url}            % simple URL typesetting
\usepackage{booktabs}       % professional-quality tables
\usepackage{nicefrac}       % compact symbols for 1/2, etc.
\usepackage{microtype}      % microtypography
\usepackage{amsmath}
\usepackage{graphicx}
\usepackage{multicol}
\usepackage{hyperref}       % hyperlinks
\usepackage[nameinlink]{cleveref}
\usepackage{bbm}
\usepackage{multirow}
% \usepackage{subfig}
\usepackage{soul}
\usepackage{floatrow}
\usepackage{float}
\usepackage{wrapfig}
\usepackage{blindtext}
\usepackage{tablefootnote}
% \usepackage{fdsymbol}. % cause an error
\usepackage{amsfonts}
\usepackage[flushleft]{threeparttable}
% \usepackage{colortbl}
\usepackage{bbding}
\usepackage{xcolor}
\usepackage{xspace}
\usepackage{bm}
\usepackage{arydshln}
\usepackage{subfigure}
\usepackage{enumitem}
\usepackage{setspace}
% \usepackage{mathabx}
\usepackage{color}
\usepackage{algorithm}
\usepackage{algorithmic}
\usepackage{longtable}
% \usepackage{booktabs}

\usepackage[normalem]{ulem}
\usepackage{ulem}
\usepackage[nomargin,inline,marginclue,draft]{fixme}
\usepackage{balance}
\usepackage{verbatim}
\usepackage{diagbox}
\usepackage{changepage}
\usepackage{amssymb}
\usepackage{pifont}
\usepackage{array}   % Optional: Improves table column alignment

%%%%%%%%%%% mboratko: added these, feel free to adjust
\DeclareMathOperator*{\argmax}{arg\,max}
\DeclareMathOperator*{\argmin}{arg\,min}
%%%%%%%%%%

\newcommand{\Ours}{\textsc{LarPO}\xspace}

\definecolor{myred}{rgb}{1, 0, 0}
\definecolor{myblue}{rgb}{0, 0, 1}
\definecolor{myblack}{rgb}{1, 1, 1}

\newcommand{\bmh}{{\bm h}}
\newcommand{\bmW}{{\bm W}}
\newcommand{\bmz}{{\bm z}}
\newcommand{\bmH}{{\bm H}}
\newcommand{\bme}{{\bm e}}
\newcommand{\bmQ}{{\bm Q}}
\newcommand{\bmK}{{\bm K}}
\newcommand{\bmV}{{\bm V}}
\newcommand{\bmc}{{\bm c}}
\newcommand{\bmE}{{\bm E}}
\newcommand{\bmd}{{\bm d}}
\newcommand{\bmy}{{\bm y}}
\newcommand{\bmx}{{\bm x}}

%%%%%%%%%%%%%%%%%%%%%%%%%%%%%%%%
% THEOREMS
%%%%%%%%%%%%%%%%%%%%%%%%%%%%%%%%
\theoremstyle{plain}
\newtheorem{theorem}{Theorem}[section]
\newtheorem{proposition}[theorem]{Proposition}
\newtheorem{lemma}[theorem]{Lemma}
\newtheorem{corollary}[theorem]{Corollary}
\theoremstyle{definition}
\newtheorem{definition}[theorem]{Definition}
\newtheorem{assumption}[theorem]{Assumption}
\theoremstyle{remark}
\newtheorem{remark}[theorem]{Remark}

% %%% REVIEW
\newcommand{\tocite}{{\color{red}CITE} }
\newcommand{\toref}{{\color{red}REF} }

%%% LOGO
\newcommand{\usc}{\raisebox{-1pt}{\includegraphics[height=0.8em]{figures/usc_logo.png}}}
\newcommand{\vuam}{\raisebox{-1pt}{\includegraphics[height=0.8em]{figures/vu_logo.png}}}

%%% SIGNS and SYMBOLS
\newcommand{\grad}{\texttt{grad-CROP}}
\newcommand{\att}{\texttt{att-CROP}}
\newcommand{\seg}{\texttt{seg}}
\newcommand{\clip}{\texttt{clip-CROP}}
\newcommand{\sam}{\texttt{sam-CROP}}
\newcommand{\yolo}{\texttt{yolo-CROP}}
\newcommand{\hc}{\texttt{human-CROP}}
\newcommand{\zsvqa}{\texttt{ZSVQA}}
\newcommand{\vic}{\textbf{ViCrop}}
\newcommand{\xmark}{\text{\ding{55}}}
\newcommand{\cmark}{\text{\ding{51}}}
\newcommand{\success}{\texttt{\color{green} \cmark}}
\newcommand{\failure}{\texttt{\color{red} \xmark}}
\newcommand{\rel}{\texttt{rel-att}}
\newcommand{\gra}{\texttt{grad-att}}
\newcommand{\pgra}{\texttt{pure-grad}}
\newcommand{\relh}{\texttt{rel-att$^h$}}
\newcommand{\grah}{\texttt{grad-att$^h$}}
\newcommand{\pgrah}{\texttt{pure-grad$^h$}}


%%% Text Abb.
\makeatletter
\DeclareRobustCommand\onedot{\futurelet\@let@token\@onedot}
\def\@onedot{\ifx\@let@token.\else.\null\fi\xspace}

\def\aka{\emph{a.k.a}\onedot} \def\Eg{\emph{E.g}\onedot}
\def\eg{\emph{e.g}\onedot} \def\Eg{\emph{E.g}\onedot}
\def\ie{\emph{i.e}\onedot} \def\Ie{\emph{I.e}\onedot}
\def\cf{\emph{c.f}\onedot} \def\Cf{\emph{C.f}\onedot}
\def\etc{\emph{etc}\onedot} \def\vs{\emph{vs}\onedot}
\def\wrt{w.r.t\onedot} \def\dof{d.o.f\onedot}
\def\etal{\emph{et al}\onedot}
\makeatletter



\definecolor{myred}{HTML}{FF8577}
\definecolor{mygreen}{HTML}{0FA958}
\definecolor{myblue}{HTML}{1982C4}
\definecolor{codegreen}{rgb}{0,0.5,0}
\definecolor{codegray}{rgb}{0.5,0.5,0.5}
\definecolor{codepurple}{rgb}{0.07,0,0.53}
\definecolor{codered}{RGB}{189,41,0}
\definecolor{codecomment}{RGB}{153,153,153}
\definecolor{backcolour}{rgb}{0.96,0.96,0.96}
\definecolor{royalblue}{rgb}{0.0, 0.14, 0.4}
\definecolor{egyptianblue}{rgb}{0.06, 0.2, 0.65}
\definecolor{royalazure}{rgb}{0.0, 0.22, 0.66}
\definecolor{portlandorange}{rgb}{1.0, 0.35, 0.21}
\definecolor{sienna}{RGB}{183,105,68}
\definecolor{saddlebrown}{RGB}{139,69,19}
\definecolor{mediumbrown}{RGB}{83,41,11}
\definecolor{darkbrown}{RGB}{58,28,7}
\hypersetup{
    colorlinks=true,
    linkcolor=sienna,
    urlcolor=royalblue,
    citecolor=royalblue,
}

% Images will be looked for in this path, removes need for explicit path when including images.
\graphicspath{{figures/}}

% Important Information about your paper.
\title{LLM Alignment as Retriever Optimization: An Information Retrieval Perspective}

% Can leave this option out if you do not wish to add a corresponding author.
\correspondingauthor{bowenj4@illinois.edu}

% Remove these if they are not needed
% \keywords{\LaTeX, Publications process, tools}
% \paperurl{arxiv.org/abs/123}

% Use the internally issued paper ID, if there is one
% \reportnumber{001} % Leave blank if n/a

% Assign your own date to the report.
% Can comment out if not needed or leave blank if n/a.
\renewcommand{\today}{}

% Can have as many authors and as many affiliations as needed. Best to indicate joint
% first-authorship as shown below.
\author[1 2 *]{Bowen Jin}
\author[1]{Jinsung Yoon}
\author[3]{Zhen Qin}
\author[2]{Ziqi Wang}
\author[2]{Wei Xiong}
\author[4]{Yu Meng}
\author[2]{Jiawei Han}
\author[1]{Sercan Ö. Arık\hspace{-0.4ex}}

% Affiliations *must* come after the declaration of \author[]
% \affil[*]{Equal contributions}
\affil[1]{Google Cloud}
\affil[2]{University of Illinois at Urbana-Champaign}
\affil[3]{Google DeepMind}
\affil[4]{University of Virginia}

\begin{abstract}
Large Language Models (LLMs) have revolutionized artificial intelligence with capabilities in reasoning, coding, and communication, driving innovation across industries. 
Their true potential depends on effective alignment to ensure correct, trustworthy and ethical behavior, addressing challenges like misinformation, hallucinations, bias and misuse.
While existing Reinforcement Learning (RL)-based alignment methods are notoriously complex, direct optimization approaches offer a simpler alternative.
In this work, we introduce a novel direct optimization approach for LLM alignment by drawing on established Information Retrieval (IR) principles. 
We present a systematic framework that bridges LLM alignment and IR methodologies, mapping LLM generation and reward models to IR's retriever-reranker paradigm. 
Building on this foundation, we propose \textbf{L}LM \textbf{A}lignment as \textbf{R}etriever \textbf{P}reference \textbf{O}ptimization (\Ours), a new alignment method that enhances overall alignment quality.
Extensive experiments validate \Ours's effectiveness with 38.9\% and 13.7\% averaged improvement on AlpacaEval2 and MixEval-Hard respectively.
Our work opens new avenues for advancing LLM alignment by integrating IR foundations, offering a promising direction for future research.
\end{abstract}

\begin{document}

\maketitle

\section{Introduction}

\begin{figure}[h]
    \centering
    \begin{overpic}[trim=0cm 0cm 0cm 0cm,clip,angle=0,origin=c,width=.4\linewidth]{images/teaser_absolute.png}
        %  trim={<left> <lower> <right> <upper>}
        %  \put(horiz, vert)
        %  \put(horiz, vert){\rotatebox{90}{Text}}
        %
        \put(107, 32){$\mathbf{\to}$}
    \end{overpic}\hspace{1cm}
    \begin{overpic}[trim=0cm 0cm 0cm 0cm,clip,angle=0,origin=c,width=.4\linewidth]{images/teaser_translated_yellow.png}
        %  trim={<left> <lower> <right> <upper>}
        %  \put(horiz, vert)
        %  \put(horiz, vert){\rotatebox{90}{Text}}
        %
    \end{overpic}
    \caption{Using translation methods, a controller trained on an environment with a given visual variation \textit{(left)} can be reused without any training or fine-tuning on a different environment (\textit{right}) with comparable performance. In red we see the trajectory of a car driven by the same controller when connected to two different encoders, one for each visual variation.
    }
    \label{fig:teaser}
\end{figure}

Deep Reinforcement Learning (RL) has enabled agents to achieve remarkable performance in complex decision-making tasks, from robotic manipulation to high-dimensional games (Mnih et al., 2015; Silver et al., 2017). 
Although recent RL techniques achieved strong improvements over sample efficiency \citep{yarats2021drqv2, kostrikov2020image}, training new agents remains a costly process, both in computational and temporal terms.
Despite these advances, most methods still require at least partial retraining when dealing with domain shifts such as visual appearance, reward functions, or action spaces \citep{pmlr-v97-cobbe19a, zhang2020learning}. These domain changes typically require expensive retraining, which can be prohibitive for real-world settings that require millions of interactions.

A variety of approaches have been proposed to address these shifting conditions. Domain randomization \citep{tobin2017domain, sadeghi2016cad2rl} trains agents across diverse visual styles or physics settings, promoting invariant features but demanding broader coverage of possible variations. Multi-task RL \citep{parisotto2015actor, teh2017distral} attempts to learn shared representations across multiple tasks.

In the supervised setting, recent representation learning techniques \citep{Moschella2022-yf,maiorca2023latent, norelli2022b, cannistraci2023bricks}, show that it is possible to zero-shot recombine encoders and decoders to perform new tasks across different modalities (images, text..) and tasks (classification, reconstruction) and even architectures.
In RL, methods adopting the relative representation framework \citep{Moschella2022-yf} have shown promising results in adapting encoders to different controllers with zero or few-shots adaptation, for robotic control from proprioceptive states \citep{jian2021adversarial} or for playing games in the Gymnasium suite \citep{towers2024gymnasium} from pixels \citep{ricciardi2025r3lrelativerepresentationsreinforcement}.
These methods, however, still require training models to use the new relative representations.

By contrast, \cite{maiorca2023latent} suggest that modules from independently trained neural networks can be connected via a simple linear or affine transformation, with no training constraint or fine-tuning required, if such transformations can be reliably estimated from a small set of “anchor” samples, pairs of states or observations deemed semantically equivalent.

Our main contribution is the implementation of a RL method based on semantic alignment to map between latent spaces of different neural models, so that their encoders and controllers can be stitched with the goal of creating new agents that can act on visual-task combinations never seen together in training. This includes the use of the transformations to map modules from different networks, and the collection of anchor samples used to estimate these transformations. We call our method Semantic Alignment for Policy Stitching (\textbf{SAPS}).
We perform analyses and empirical tests on the CarRacing and LunarLander environments to show the performance of new agents created via zero-shot stitching of encoders and controllers trained on different visual-task variations, demonstrating significant gains compared to existing zero-shot methods.

% \section{LLM alignment as retriever optimization}
\section{An Information Retrieval Perspective on LLMs}\label{sec:understanding}


\begin{figure*}
\centering
\includegraphics[scale=0.53]{figure/arch2.pdf}
\caption{Architecture connection between retriever/LLM (bi-encoder) and reranker/reward model (cross-encoder). 
Bi-encoder models process each query/prompt and passage/response separately and often calculate their alignment score via a dot product operator, while cross-encoder models take both query/prompt and passage/response as input and score them directly.
Bi-encoder models can be more efficient (\textit{i.e.}, large-scale text matching) but the interaction between the two information unit is only captured by a dot production operation where their effectiveness can be constrained. Cross-encoder models can be more effective (\textit{i.e.}, deeper interaction calculation with transformer architecture \citep{vaswani2017attention}) but less efficient. Although LLM involves auto-regressive token matching, which is different from retriever, some insights from IR can be borrowed to enhance LLM alignment as shown in the following sections.}\label{fig:bi-cross-encoder}
% \vspace{-0.1in}
\end{figure*}

% \paragraph{Primer on information retrieval}
\subsection{Primer on information retrieval}

Information retrieval systems \citep{zhu2023large} typically employ a two-stage process involving retrievers \citep{zhao2024dense} and rerankers \citep{lin2022pretrained}.
The retriever, often implemented as a bi-encoder (Figure \ref{fig:bi-cross-encoder}), efficiently identifies a large set of ($K$) potentially relevant passages, denoted as $D_{\text{retrieval}}$, from a corpora $C$ given a query $q$.
% This is achieved using a coarse-grained similarity function, $f_{\text{retrieval}}(q, d)=\text{Enc}^T_q(q) \cdot \text{Enc}_d(d)$, where $\text{Enc}_q$ and $\text{Enc}_d$ represent the query and passage encoders respectively:
This is achieved using a coarse-grained similarity function, $p_{\text{retrieval}}(d|q)=\text{Enc}^T_q(q) \cdot \text{Enc}_d(d)$, where $\text{Enc}_q$ and $\text{Enc}_d$ represent the query and passage encoders respectively:
\begin{gather}\label{eq:retriever}
    D_{\text{retrieval}}(q) = \{ d \in C \;|\; \max_{\text{top-}K} ~ p_{\text{retrieval}}(\cdot | q)\}.
\end{gather}
However, due to the scale of the corpus, retrievers might not accurately capture fine-grained query-passage similarity with the simple dot production interaction function. 
Therefore, rerankers, typically implemented with cross-encoder (Figure \ref{fig:bi-cross-encoder}), are employed to refine the ranking of the retrieved passages $D_{\text{retrieval}}$.
% The reranker produces a smaller set ($k$) of top-ranked passages, $D_{rank}$, using a fine-grained similarity function, $f_{\text{rank}}(q,d)=w^T \cdot \text{Enc}(q, d)$.
The reranker produces a smaller set ($k$) of top-ranked passages, $D_\text{rank}$, using a fine-grained similarity function, $r_{\text{rank}}(q,d)=w \cdot \text{Enc}(q, d)$, where $w$ is a learnable linear layer. 
Here, reranker adopts cross-encoder with both query/passage as inputs and encoded together while retriever adopts dual encoder for separate query/passage encoding.
% \begin{gather}
%     D_{\text{rank}}(q) =  \max_{\text{top-}k, d\in D_{\text{retrieval}}} f_{\text{rank}}(q, d),
% \end{gather}
\begin{gather}\label{eq:reranker}
    D_{\text{rank}}(q) = \{ d \in D_{\text{retrieval}}(q) \;|\; \max_{\text{top-}k} r_{\text{rank}}(q, \cdot)\}.
\end{gather}
% The resulting ranked passages are ordered such that $D_{\text{rank}}(q)=\{d_1, d_2, \ldots, d_k\}$ where $f_{\text{rank}}(q, d_1)\ge f_{\text{rank}}(q, d_2)\ge\cdots\ge f_{\text{rank}}(q, d_k)$.
The resulting ranked passages are ordered such that $D_{\text{rank}}(q)=\{d_1, d_2, \ldots, d_k\}$ where $r_{\text{rank}}(q, d_1)\ge r_{\text{rank}}(q, d_2)\ge\cdots\ge r_{\text{rank}}(q, d_k)$.
%If a single output passage is required, the top-ranked passage, $d_1$, is selected.

\subsection{LLMs as retrievers. Reward models as rerankers}
During inference, an LLM generates a response $y$ given an input prompt $x$ by modeling the probability distribution $p_{\text{LLM}}(y|x)$.
Assuming a fixed maximum sequence length $L$ and a vocabulary space $V$ \citep{li2024matching}, the set of all possible responses can be defined as  $Y=\{y:y(1)y(2)...y(L) | y(i) \in V\} \subseteq V^L$.

We can conceptualize this process through an IR lens \citep{tay2022transformer}. The prompt $x$ can be viewed as analogous to a query $q$, the set of all possible responses $Y$ can be treated as the corpus $C$, and the generated response $y$ can be considered as the retrieved passage $d$.
Thus, given a prompt $x$, the LLM effectively acts as a retriever, searching for the most probable responses $D_{\text{LLM}}(x)$ from response space $Y$:
% \begin{gather}
%     D_{\text{LLM}}(x) =  \max_{\text{top-}K, y\in Y} f_{\text{LLM-retrieval}}(x, y),
% \end{gather}
\begin{gather}\label{eq:retriever2}
    D_{\text{LLM}}(x) = \{ y \in Y \;|\; \max_{\text{top-}K} ~ p_{\text{LLM}}(\cdot | x)\}.
\end{gather}
% where $f_{\text{LLM-retrieval}}(x, y)=p_{\text{LLM}}(y|x)$.
where $p_{\text{LLM}}(y|x)$ is analogous to $p_{\text{retrieval}}(d|q)$ in IR.

This analogy is further supported by the LLMs' architecture. 
As illustrated in Figure \ref{fig:bi-cross-encoder}, the generative modeling with LLMs can be interpreted as the matching process of a bi-encoder model. 
The prompt is encoded into a vector representation by LLM, while response tokens are represented as token embedding vectors. 
For each token position decoding, prompt embedding (obtained often from the hidden state of the last layer of the LLM) and vocabulary token embeddings are compared with a dot product, to determine the likelihood of a selected token for the response.
% These representations are then compared, often using a dot product, to determine the likelihood of a given response.

Furthermore, reward models $r_\text{rm}(x,y)$ \citep{lambert2024rewardbench}, which take both the prompt and response as input, function similarly to cross-encoders (\textit{i.e.}, rerankers $r_\text{rank}(q,d)$ \citep{zhuang2023rankt5}) in IR.
To enhance LLM performance, various inference-time strategies have been developed, including Best-of-N sampling \citep{stiennon2020learning} and majority voting \citep{wang2022self}.  
These can be interpreted as different configurations of retrievers and rerankers, as summarized in Appendix Table \ref{tb:llm-retriever-reranker}.


% \paragraph{LLM tuning as retriever optimization.}
\subsection{LLM tuning as retriever optimization}\label{sec:llm-tuning-retriever}

\paragraph{Supervised fine-tuning as direct retriever optimization.}
Retriever training, aiming for accurate retrieval, often employs contrastive learning with the InfoNCE loss \citep{oord2018representation} to maximize $P(d_{\text{gold}}|q)$ of retrieving the ground truth passage $d_{\text{gold}}$ given a query $q$. This can be expressed as:
% and encourage the similarity score between the query $q$ and ground truth passage $d_{\text{gold}}$ to be higher than that between $q$ and \textbf{all} other passages:
% \begin{gather*}
%     \forall d' \in C \setminus d_{\text{gold}}, \max P(f_{\text{retrieval}}(q, d_{\text{gold}}) > f_{\text{retrieval}}(q, d')).
% \end{gather*}
\begin{gather*}
    \max \log P(d_{\text{gold}}|q) = \max \log \frac{\text{Enc}_d(d_{\text{gold}}) \cdot\text{Enc}_q(q)}{\sum^{|C|}_{j=1} \text{Enc}_d(d_j) \cdot\text{Enc}_q(q)}.
\end{gather*}
In the context of LLM alignment, supervised fine-tuning (SFT) aims to quickly adapt the model to a target task using prompt-response pairs ($x, y_{\text{gold}}$). 
SFT maximizes the conditional probability $P(y_{\text{gold}}|x)$ as:
\begin{gather*}
    \max \log P(y_{\text{gold}}|x) = \max \log \prod^{|y_{\text{gold}}|}_i P(y_{\text{gold}}(i)|z_i) 
    = \max \sum^{|y_{\text{gold}}|}_i \log \frac{\text{Emb}(y_{\text{gold}}(i)) \cdot\text{LLM}(z_i)}{\sum^{|V|}_{j=1} \text{Emb}(v_j) \cdot\text{LLM}(z_i)},
\end{gather*}
where $y(i)$ is the $i$-th token of $y$, $z_i=[x, y_{\text{gold}}(1:i-1)]$ represent the concatenation of the prompt $x$ and the preceding tokens of $y_{\text{gold}}$, $\text{LLM}(\cdot)$ produces a contextualized representation, and $\text{Emb}(\cdot)$ is the token embedding function.

Consequently, the SFT objective can be interpreted as a composite of multiple retrieval optimization objectives. In this analogy, $\text{LLM}(\cdot)$ acts as the query encoder and $\text{Emb}(\cdot)$ serves as the passage (or, in this case, token) encoder.

% As for LLM alignment, supervised fine-tuning (SFT) helps the LLM learn basic instruction/task-solving capabilities.
% Given a prompt $x$ and the golden response $y_\text{gold}$, SFT maximizes $p_{\text{LLM}}(y_{\text{gold}}|x)$, which is equivalent to optimize the following objective:
% \begin{gather*}
%     \forall y' \in Y \setminus y_{\text{gold}}, \max P(p_{\text{LLM}}(y_{\text{gold}}|x) > p_{\text{LLM}}(y'|x)).
% \end{gather*}
% In this case, SFT is conducted retrieval capability training to let LLM be able to retrieve the proper response from the large response set.

% \begin{figure}
% \centering
% \small
% \includegraphics[scale=0.3]{figure/LLM_alignment_gsm8k_mathstral7b_infer.pdf}
% \vskip -1em
% \caption{Evaluation of the LLM as a retriever with Recall@N (Pass@N) with Mathstral-7b-it as the LLM. As the number of retrieved responses (N) increases, the retrieval recall increases. The higher the temperature is, the broader spectrum the retrieved responses are, and thus the higher the recall is.}\label{fig:mathstral-gsm8k-infer}
% \end{figure}




\paragraph{Preference optimization as reranker-retriever distillation.}
In retriever training, optimizing solely based on query/ground-truth document pairs can be suboptimal, particularly when using in-batch negatives for efficiency. Performance can be enhanced by distilling knowledge from a more powerful reranker to retriever \citep{qu2020rocketqa,zeng2022curriculum}. This distillation process can be represented as $f_{\text{rerank}}(\cdot) \overset{r}{\rightarrow} \text{data}   \overset{g(\cdot)}{\rightarrow}  f_{\text{retrieval}}(\cdot)$, where new data, generated by the reranker $f_{\text{rerank}}(\cdot)$ based on a rule $r$, is used to optimize the retriever $f_{\text{retrieval}}(\cdot)$ with an objective $g(\cdot)$.

Similarly, in LLM alignment, a preference alignment phase often follows supervised fine-tuning (SFT) to further enhance the model using an external reward model to absorb preferential supervision effectively.
Methods like PPO \citep{schulman2017proximal} and iterative DPO \citep{guo2024direct} exemplify this approach.  
Here, the LLM (considered acting as the retriever) generates responses that are then scored by the reward model (considered acting as the reranker). 
These scores are used to create new training data, effectively performing distillation from the reward model into the LLM: 
$f_{\text{reward-model}}(\cdot) \overset{r}{\rightarrow}  \text{data} \overset{g(\cdot)}{\rightarrow} f_{\text{LLM}}(\cdot)$. 
Thus, preference optimization can be viewed as a form of reranker-to-retriever distillation, analogous to the process used in traditional IR.

We conduct empirical studies to understand SFT and preference optimization from IR perspective in Appendix \ref{apx:sft-rlhf-empirical} and have further discussion in Appendices \ref{apx:discuss1} and \ref{apx:discuss2}.

% \textcolor{blue}{add a sentence that we also conduct experiments to study SFT and PO in appendix}

\subsection{Empirical insights into LLMs as IR models}\label{sec:empirical}

\paragraph{Evaluating LLMs as retrievers.}
A common metric for evaluating retrievers is Recall@$N$, which assesses whether the top-$N$ retrieved passages include any relevant passages for a given query. 
In the context of LLMs, this translates to evaluating whether the top-$N$ generated responses contain a suitable response to the prompt, analogous to Pass@$N$ \citep{chen2021evaluating}.

\begin{figure}[h!]
    \centering
    \subfigure[Retriever]{\includegraphics[width=0.45\textwidth]{figure/LLM_alignment_nq_e5_pass.pdf}}
    \hspace{1cm}
    \subfigure[LLM]{\includegraphics[width=0.45\textwidth]{figure/LLM_alignment_gsm8k_mathstral7b_pass_multi.pdf}}
    \caption{Analogy between evaluating retriever with Recall@N and LLM with Pass@N. As the number (N) of retrieved passages/generated responses increases, the retriever and LLM have a similar increasing trend. This highlights the importance of inference time scaling (\textit{e.g.}, Best-of-N) for LLM similar to retriever-reranker scaling in IR. Retriever: e5; LLM: Mathstral-7b-it.}\label{fig:mathstral-gsm8k-infer}
\end{figure}

To draw the empirical connection between LLM and retrievers, we conduct an experiment on the GSM8K dataset \citep{cobbe2021training} using Mathstral-7b-it \citep{mathstral2025} and an experiment on the NQ dataset \citep{kwiatkowski2019natural} using e5 retriever.
%, to study the correlation between Recall@N/Pass@N with N.
Figure \ref{fig:mathstral-gsm8k-infer} illustrates that increasing N can contribute to improved performance for both retriever and LLM. Detailed analysis can be found in Appendix \ref{apx:llm-as-retriever}.

% To investigate the retrieval capabilities of LLMs, we conduct an experiment on the GSM8K dataset \citep{cobbe2021training} using Mathstral-7b-it \citep{mathstral2025}.  
% We focus on two key parameters: the number of generated responses ($N$) and the decoding temperature, which influence the retrieval size and the diversity of the retrieved set, respectively.  
% Figure \ref{fig:mathstral-gsm8k-infer} illustrates the following key findings: 
% (1) Recall@$N$ increases significantly with $N$, indicating that retrieving more responses increases the likelihood of obtaining a correct one. 
% (2) For small $N$ (\textit{e.g.}, 1), lower temperatures yield higher Recall@$N$, as higher temperatures introduce randomness that can hinder the retrieval of the most probable response. 
% (3) For large $N$ (\textit{e.g.}, $>$ 10), higher temperatures lead to improved Recall@$N$, likely because increased diversity among the responses improves the chances of including the correct answer within a larger retrieved set.

Greedy decoding, equivalent to $N=1$, is a prevalent LLM inference strategy. 
However, as shown in Figure \ref{fig:mathstral-gsm8k-infer}(b), Pass@1 is often suboptimal, and thus increasing $N$ can substantially improve performance. 
This highlights the importance of inference-time scaling techniques like Best-of-N \citep{stiennon2020learning} in LLM similar to retriever-reranker scaling \citep{zhuang2023rankt5} in IR.
More results and analyses can be found in Appendix \ref{apx:llm-as-retriever}.


% 加一节,怎么区分确定性和不确定性

\section{Methodology}


To achieve effective probabilistic predictions, we propose CoST that simultaneously leverages the advantages of both deterministic and probabilistic models. Our approach involves two stages. In the first stage, the deterministic model is pretrained to predict the conditional mean that captures the primary patterns. In the second stage, the parameters of the deterministic model are frozen, and the scale-aware diffusion model, constrained by a customized fluctuation scale, is jointly trained to model residual distributions that reflect random fluctuations.   
Figure~\ref{fig:model} illustrates an overview of our model.


\subsection{Mean-Residual Decomposition}

For urban spatiotemporal probabilistic prediction, current approaches typically employ a single probabilistic model to capture the full distribution of data, incorporating both the primary spatiotemporal patterns and the random fluctuations. However, it is challenging to model both of these distributions. Inspired by~\cite{mardani2023residual} and the Reynolds decomposition in fluid dynamics~\cite{pope2001turbulent}, we propose to decompose the target data \(\mathbf{x}^{ta}\) as follows:
\begin{equation}
 \mathbf{x}^{ta} = \underbrace{\mathbb{E}[\mathbf{x}^{ta}|\mathbf{x}^{co}]}_{\substack{:=\boldsymbol{\mu}(Deterministic)}} + \underbrace{(\mathbf{x}^{ta}-\mathbb{E}[\mathbf{x}^{ta}|\mathbf{x}^{co}])}_{\substack{:=\mathbf{r}(Probabilistic)}},
\end{equation}
where \(\boldsymbol{\mu}\) is the conditional mean representing the primary patterns, and \(\mathbf{r}\) is the residual representing the random variations. Our core idea is that if a deterministic model can accurately predict the conditional mean, that is, \(\boldsymbol{\mu}\approx\mathbb{E}_{\theta}[\mathbf{x}^{ta}|\mathbf{x}]\), then the probabilistic model only needs to focus on learning the simpler residual distribution, thus combining the strengths of both models to enhance the probabilistic prediction capability.









\subsection{Mean Prediction via Deterministic Model}

We require a deterministic model that can accurately predict the conditional mean to align with our research hypothesis, and also maintain high predictive efficiency to avoid additional increases in training and inference time. Therefore, we select the MLP-based STID model as our mean prediction module.
In the first stage of training, we pretrain the model for 50 epochs to effectively capture the primary spatiotemporal patterns. Specifically, we input historical conditional data \(\mathbf{x}^{co}\) into the STID model to obtain the conditional mean output \(\mathbb{E}_{\theta}[\mathbf{x}^{ta}|\mathbf{x}^{co}]\).

The STID model is pretrained by optimizing the following loss function:

\begin{equation}
\label{eq:loss2}
   \mathcal{L}_{2}  = \left\| \mathbb{E}_{\theta}[\mathbf{x}^{ta}|\mathbf{x}^{co}] - \mathbf{x}^{ta} \right\|_2^2 .
\end{equation}

\subsection{Residual Learning via Diffusion Model}
Diffusion models have achieved significant success in probabilistic modeling. In this work, we employ a diffusion model for probabilistic prediction, training it to learn the residual distribution:
\begin{equation}
\label{eq:one-setp-forward}
    \mathbf{r}_{ta}=\mathbf{x}^{ta}-\mathbb{E}_{\theta}[\mathbf{x}^{ta}|\mathbf{x}^{co}].
\end{equation}
Consequently, the target data \(\mathbf{x}^{ta}\) for diffusion models in Eqs.~\eqref{eq:one-setp-forward}, \eqref{eq:inference}, and \eqref{eq:loss1} is replaced by \(\mathbf{r}_{ta}\).
The residual distribution of urban spatiotemporal data is not independently and identically distributed (i.i.d.) nor does it follow a fixed distribution, such as \(\mathcal{N}(0, \mathbf{\sigma})\). Instead, it often exhibits complex spatiotemporal dependence and heterogeneity. So we consider both temporal residual learning and spatial residual learning. 




\subsubsection*{\textbf{Temporal Residual Learning.}} 
For urban spatiotemporal data, the residual distribution varies at different time points. For example, fluctuations are lower at night and higher during the day, with uncertainty varying between weekends and weekdays. To model this, we incorporate the timestamp information as the condition for the denoising process. Meanwhile, the residual distribution can also be affected by sudden weather changes or public events. To capture these real-time changes, we concatenate the context data $\mathbf{x}^{co}_0$ with noised target residual $\mathbf{r}^{ta}_n$ as the input. The noise is not added to $\mathbf{x}^{co}_0$ and $\mathbf{r}^{ta}_n$ during the diffusion training and inference.




\subsubsection*{\textbf{Spatial Residual Learning.}}
In areas with frequent traffic accidents, fluctuations tend to be more pronounced and may induce anomalous variations in adjacent regions, thus affecting their distributions.
For spatial dependence modeling, the model learns a spatial embedding for each location, following the STID approach. Additionally, we propose a scale-aware diffusion process to further distinguish the heterogeneity for different regions. In this section, we detail the calculation of \(\mathbf{Q}\) and how it is integrated into the scale-aware diffusion process.

\noindent\textbf{(i) Customized Fluctuation Scale.} Specifically, we apply the Fast Fourier Transform (FFT) to spatiotemporal sequences in the training set to quantify fluctuation levels in different regions and use the custom scale \(\mathbf{Q}\) as input to account for spatial differences in residual. Specifically, we first employ FFT to extract the fluctuation components for each region within the training set. The detailed steps are as follows:









\begin{equation}
    \begin{aligned}
    & \mathbf{A}_{\mathrm{k}} = \left| \text{FFT}(\mathbf{x})_\mathrm{k} \right|, \quad \mathbf{{\phi}}_{\mathrm{k}} = \mathbf{\phi} \left( \text{FFT}(\mathbf{x})_\mathrm{k} \right), \\
    & \mathbf{A}_{\text{max}}=\max_{\mathrm{k}\in\left\{1,\cdots,\left\lfloor\frac{\mathbf{L}}{2}\right\rfloor + 1\right\}}\mathbf{A}_{\mathrm{k}}, \\
    & \mathcal{K} = \left\{ \mathrm{k} \in \left\{ 1, \cdots, \left\lfloor \frac{{L}}{2} \right\rfloor + 1 \right\} : \mathbf{A}_{\mathrm{k}} < 0.1 \times \mathbf{A}_{\text{max}} \right\}, \\
    & \mathbf{x}_{\mathbf{r}}[i] = \sum_{\mathrm{k} \in \mathcal{K}} \mathbf{A}_{\mathrm{k}} \Big[ \cos \left( 2\pi \mathbf{f}_{\mathrm{k}} i + \mathbf{\phi}_{\mathrm{k}} \right) \\
    & \qquad \qquad + \cos \left( 2\pi \bar{\mathbf{f}}_{\mathrm{k}} i + \bar{\mathbf{\phi}}_{\mathrm{k}} \right) \Big],
    \end{aligned}
\end{equation}
where \(\mathbf{A}_{\mathrm{k}},\mathbf{\phi}_{\mathrm{k}}\) reprent the amplitude and phase of the $\mathrm{k}-$th frequency component. $L$ is the temporal length of the training set. \(\mathbf{A}_{\text{max}}\) is the maximum amplitude among the components, obtained using the $\max$ operator. $\mathcal{K}$ represents the set of indices for the selected residual components. \(\mathbf{f}_{\mathrm{k}}\) is the frequency of the \(\mathrm{k}\)-th component. $\bar{\mathbf{f}}_{\mathrm{k}}, \bar{\mathbf{\phi}}_{\mathrm{k}}$ represent the conjugate components. \(\mathbf{x}_{\mathbf{r}}\) ref to the extracted residual component of the training set. We then compute the variance $\sigma^2_k$ of the residual sequence for each location $k$ and expand it to match the shape as 
\(\mathbf{r}^{ta}_0 \in \mathbb{R}^{B \times K \times P}\) , where $B$ represents the batch size. And we can get the variance tensor \(\mathcal{M}\): 
\begin{equation}
\begin{aligned}
    &\mathcal{M}_{b,k,p}=\sigma_{k}^2,\\
&\forall b\in\{1,\cdots,B\}, \forall k\in\{1,\cdots,K\}, \forall p\in\{1,\cdots,P\}.
\end{aligned}
\end{equation}
The residual fluctuations are bidirectional, encompassing both positive and negative variations, so we generate a random sign tensor \(\mathbf{S}\in\mathbb{R}^{B\times K\times P}\) for \(\mathcal{M}\), where each element \(S_{b,k,p}\) of \(\mathbf{S}\) is sampled from a Bernoulli distribution with \(p = 0.5\). 
%That is, \(r_{b,k,p}\) takes the value $1$ with probability $0.5$ and $-1$ with probability $0.5$. 
The customized fluctuation scale \(\mathbf{Q}\) is then defined as:
\begin{equation}
\begin{aligned}
&\mathbf{Q}_{b,k,p}=S_{b,k,p}\times\mathcal{M}_{b,k,p},\\
&\forall b\in\{1,\cdots,B\}, \forall k\in\{1,\cdots,K\}, \forall p\in\{1,\cdots,P\}.
\end{aligned}
\end{equation}
Then \(\mathbf{Q}\) is used as the input of the denoising network. 





\noindent\textbf{(ii) Scale-aware Diffusion Process.}

The vanilla diffusion models typically model all regions as the same \(\mathcal{N}(0, I)\) distribution at the end of the diffusion process, failing to distinguish the differences among regions. To further model the differences of residual distribution across various regions, we adopt the technique proposed by~\cite{han2022card} to make the residual learning region-specific conditioned on \({\mathbf{Q}}\). Specially, we have calculated the customized fluctuation scale \({\mathbf{Q}}\), and We redefined the noise distribution at the endpoint of the diffusion process as follows:
\begin{equation}
    p(\mathbf{r}^{ta}_N)=\mathcal{N}({\mathbf{Q}},I),
\end{equation}
Accordingly, the Eq~\ref{eq:new one-step} in the forward process is rewritten as:
\begin{equation}
\label{eq:new one-step}
    \mathbf{r}_n^{ta} = \sqrt{\bar{\alpha}_n} \mathbf{r}_0^{ta}+(1-\sqrt{\bar{\alpha}_n})\mathbf{Q} + \sqrt{1 - \bar{\alpha}_n} \mathbf{\epsilon}, \quad \mathbf{\epsilon} \sim \mathcal{N}(0, I).
\end{equation}
And in the denoising process, we sample \(\mathbf{r}_N^{ta}\) from $\mathcal{N}({\mathbf{Q}},I)$, and denoise it use Eq~(\ref{eq:inference}), the computation of \(\mu_{\theta}(\mathbf{r}_n^{ta}, n| \mathbf{x}_0^{co})\) in Eq~\ref{eq:inference} is modified as:
\begin{equation}
\label{eq: mu}
    \mu_{\theta}(\mathbf{r}_n^{ta}, n| \mathbf{x}_0^{co})=\frac{1}{\sqrt{\bar{\alpha}_n}} \left( \mathbf{r}_n^{ta} - \frac{\beta_n}{\sqrt{1 - \bar{\alpha}_n}} \mathbf{\epsilon}_{\theta}(\mathbf{r}_n^{ta}, n| \mathbf{x}_0^{co}) \right)+(1-\frac{1}{\sqrt{\bar{\alpha}_n}})\mathbf{Q}.
\end{equation}
This approach enables the diffusion process to be governed by the \(\mathbf{Q}\) values at each region, leading to more effective utilization of the customized scale conditions.


\subsection{Training and Inference}
\begin{algorithm}
\caption{\methodname{} Training}
\KwIn{Coarse-to-fine Autoencoder $\text{Enc}$, $\text{Dec}$}
\KwOut{}
\For{$i \gets 1$ \textbf{to} $n-1$}{
    \For{$j \gets 1$ \textbf{to} $n-i$}{
        \If{$L[j] > L[j+1]$}{
            Swap $L[j]$ and $L[j+1]$
        }
    }
}
\Return $L$
\end{algorithm}
\begin{algorithm}[!t]
\caption{Inference}
\label{al: sampling}
\begin{algorithmic}[1]
    \State \textbf{Input:} Context data $\mathbf{x}_0^{co}$, customized fluctuation scale $\mathbf{Q}$, trained diffusion model $\epsilon_{\theta}$, trained deterministic model $\mathbb{E}_{\theta}$
    \State \textbf{Output:} Target data $\mathbf{x}_0^{ta}$
    \State Estimate the conditional mean \(\mathbb{E}_{\theta}[\mathbf{x}^{ta}_0|\mathbf{x}^{co}_0]\)
    \State Sample $\mathbf{r}_N^{ta}$ from $\epsilon \sim \mathcal{N}(\mathbf{S},I)$
    \For{$n = N$ to $1$} 
        \State Estimate the noise $\mathbf{\epsilon}_{\theta}(\mathbf{r}_n^{ta}, n| \mathbf{x}_0^{co})$
        \State Calculate the $\mu_{\theta}(\mathbf{r}_n^{ta}, n| \mathbf{x}_0^{co})$ using Eq.~(\ref{eq: mu})
        \State Sample $\mathbf{r}_{n-1}^{ta}$ using Eq.~(\ref{eq:inference})
    \EndFor
    \State \textbf{Return:} $\mathbf{x}_0^{ta}=\mathbb{E}_{\theta}[\mathbf{x}^{ta}_0|\mathbf{x}^{co}_0]+\mathbf{r}_0^{ta}$
\end{algorithmic}

\end{algorithm}

\subsubsection*{\textbf{Training}}
Our training process is a two-stage procedure. We first pretrain the deterministic model STID to enable it to predict the conditional mean. Subsequently, we train the diffusion mode to learn the distribution of residuals, where the residuals are calculated as the difference between the true values and the conditional mean predicted by the pretrained STID model with frozen parameters. The detailed training procedure is presented in Algorithm~\ref{al: train}.
\subsubsection*{\textbf{Inference}}
The inference process of the model consists of two paths: one utilizes the pretrained STID model to predict the conditional mean, while the other uses the diffusion model to predict the residuals. The final sample is obtained by summing the results of both paths. This process is detailed in Algorithm~\ref{al: sampling}.


\section{Main Results}\label{sec:main-result}

\paragraph{Baselines.}
We evaluate the performance of \Ours against a range of established preference optimization methods, encompassing both offline and online approaches.
Our offline comparison set includes RRHF \citep{yuan2023rrhf}, SLiC-HF \citep{zhao2023slic}, DPO \citep{guo2024direct}, IPO \citep{azar2024general}, CPO \citep{xu2024contrastive}, KTO \citep{ethayarajh2024kto}, RDPO \citep{park2024disentangling} and SimPO \citep{meng2024simpo}.
For online methods, we compare with iterative DPO \citep{xiong2024iterative}.
The baseline checkpoints are from \citep{meng2024simpo}.
Further details regarding these baselines and our experimental setup are provided in Appendix \ref{apx:sec:baselines}.
Both baselines and \Ours are trained on Ultrafeedback dataset \citep{cui2024ultrafeedback} for fair comparison.

\paragraph{Datasets.} We conduct evaluation on two widely used benchmarks AlpacaEval2 \citep{dubois2024length} and MixEval \citep{ni2024mixeval}.  
These benchmarks are designed to assess the conversational capabilities of models across a diverse range of queries. AlpacaEval2 comprises 805 questions sourced from five datasets, while MixEval includes 4000 general and 1000 hard questions.
Evaluation follows the established protocols for each benchmark. For AlpacaEval 2, we report both the raw win rate (WR) and the length-controlled win rate (LC). These benchmarks collectively provide a comprehensive assessment of the models' instruction-following and problem-solving capabilities.

\paragraph{Results.}
% We assess the performance of \Ours on two established benchmarks: AlpacaEval2 \citep{dubois2024length} and MixEval \citep{ni2024mixeval}. 
The baseline performances on AlpacaEval 2 are directly from \citet{meng2024simpo}, while the performances on MixEval is evaluated by ourselves with the opensourced checkpoints.
We adopt the same LLM-Blender \citep{jiang2023llm} reward model for a fair comparison with the baselines and also explore stronger reward model: FsfairX \citep{dong2024rlhf}.
The results, presented in Table \ref{tab:main-performance}, show that \Ours consistently outperforms the competitive baseline methods on both datasets, with 38.9 \% and 13.7 \% averaged relative improvements, on AlpacaEval2 and MixEval-Hard respectively, with the same reward model as the baselines.
With a stronger reward model, we can further improve \Ours by 25.8 \% on the challenging AlpacaEval2 dataset.
Additional details regarding our experimental setup are available in Appendix \ref{apx:sec:main}.



\begin{table*}
    \centering
    \caption{Evaluations on AlpacaEval 2 and MixEval. LC WR and WR denote length-controlled win rate and win rate respectively. Offline baseline performances on AlpacaEval 2 are from \citep{meng2024simpo}. We use LLM-blender \citep{jiang2023llm} as the reward model for a fair comparison with the baselines and also report the result with a stronger reward model FsfairX \citep{dong2024rlhf}}\label{tab:main-performance}
    \scalebox{0.87}{
    \begin{tabular}{lcccccccccccc}
        \toprule
        Model & \multicolumn{4}{c}{Mistral-Base (7B)} & \multicolumn{4}{c}{Mistral-Instruct (7B)} \\
        \cmidrule(lr){2-5} \cmidrule(lr){6-9}
        & \multicolumn{2}{c}{Alpaca Eval 2}  & \multirow{1}{*}{MixEval} & \multirow{1}{*}{MixEval-Hard} & \multicolumn{2}{c}{Alpaca Eval 2}  & \multirow{1}{*}{MixEval} & \multirow{1}{*}{MixEval-Hard} \\
        \cmidrule(lr){2-3} \cmidrule(lr){4-4} \cmidrule(lr){5-5} \cmidrule(lr){6-7} \cmidrule(lr){8-8} \cmidrule(lr){9-9}
        & LC WR & WR & Score & Score & LC WR & WR & Score & Score \\
        \midrule
        SFT    & 8.4  & 6.2    &  0.602  & 0.279  & 17.1 & 14.7  & 0.707 & 0.361 \\
        \midrule
        \multicolumn{9}{c}{Reward model: LLM-Blender \citep{jiang2023llm}}  \\
        \midrule
        RRHF   & 11.6 & 10.2  &  0.600  & 0.312  & 25.3 & 24.8  &   0.700    & 0.380 \\
        SLiC-HF & 10.9 & 8.9    & 0.679  &   0.334 & 24.1 & 24.6  &   0.700    & 0.381 \\
        DPO    & 15.1 & 12.5  &  0.686  &  0.341 & 26.8 & 24.9  & 0.702 & 0.355 \\
        IPO    & 11.8 & 9.4   &  0.673  & 0.326  & 20.3 & 20.3  & 0.695 & 0.376 \\
        CPO    & 9.8  & 8.9    & 0.632   &  0.307 & 23.8 & 28.8  & 0.699 & 0.405 \\
        KTO    & 13.1 & 9.1   & \textbf{0.704}  & 0.351   & 24.5 & 23.6  &   0.692    & 0.358 \\
        % ORPO   & 14.7 & 12.2 & 7.0  &   &    & 24.5 & 24.9 & 20.8 &  0.703     & 0.378 \\
        RDPO   & 17.4 & 12.8   & 0.693  & 0.355   & 27.3 & 24.5  &   0.695    & 0.364 \\
        SimPO  & 21.5 & 20.8 &  0.672  &  0.347 & 32.1 & 34.8  & 0.702  & 0.363 \\
        Iterative DPO  & 18.9  & 16.7  & 0.660   & 0.341  & 20.4 & 24.8  & 0.719  & 0.389 \\
        \midrule
        \Ours (Contrastive) & 31.6 & 30.8  &   0.703 & 0.409  & 32.7 & 38.6  &  0.718 & \textbf{0.418} \\
        \Ours (LambdaRank) &  \textbf{34.9} & \textbf{37.2} & 0.695 &  \textbf{0.452}  & \textbf{32.9} & \textbf{38.9}   & \textbf{0.720} & 0.417  \\
        \Ours (ListMLE) & 31.1  &  32.1   &  0.669  & 0.390  &  29.7 & 36.2    & 0.709  & 0.397 \\
        \midrule
        \multicolumn{9}{c}{Reward model: FsfairX \citep{dong2024rlhf}}  \\
        \midrule
        \Ours (Contrastive) & \textbf{41.5} & \textbf{42.9} & 0.718 & 0.417    & \textbf{43.0}  & \textbf{53.8} & 0.718 & 0.425   \\
        \Ours (LambdaRank) & 35.8 & 34.1 & 0.717 & 0.431   & 41.9  & 48.1 & \textbf{0.740} & \textbf{0.440}  \\
        \Ours (ListMLE) & 36.6 & 37.8 & \textbf{0.730} & \textbf{0.423}   & 39.6  & 48.1 & 0.717 & 0.397   \\
        \bottomrule
    \end{tabular}}
    % \vspace{-0.1in}
\end{table*}




% \section{Empirical Analyses}\label{sec:analysis}
\section{Analyses}\label{sec:analysis}

% As discussed in the previous section, we can clearly see the connection between LLM community and IR community, where the LLMs can be seen as retrievers.
% In this section, we conduct experiments to analyze LLM from the three perspective in IR:
% (1) retrieval quality; (2) hard negatives in training data; (3) retrieval list size.

This section provides empirical analyses of the three factors identified in Section \ref{sec:proposal}.  
% Building upon these findings, Section \ref{sec:lrpo} introduces new preference optimization methods that leverage insights from the field of Information Retrieval.


% \begin{table*}[t]
%     \centering
%     % \renewcommand{\arraystretch}{1.2}
%     \caption{Preference optimization objective study on AlpacaEval2 and MixEval. For AlpacaEval2, we report the result with both opensource LLM evaluator \texttt{alpaca\_eval\_llama3\_70b\_fn} and GPT4 evaluator \texttt{alpaca\_eval\_gpt4\_turbo\_fn}. SFT corresponds to the initial chat model.}\label{tab:objective}
%     \small
%     \scalebox{0.85}{\begin{tabular}{llccccccccc}
%         \toprule
%         & & \multicolumn{2}{c}{AlpacaEval 2 (opensource LLM)} & \multicolumn{2}{c}{AlpacaEval 2 (GPT-4)} & \multicolumn{1}{c}{MixEval} & \multicolumn{1}{c}{MixEval-Hard} \\
%          \cmidrule(r){3-4} \cmidrule(r){5-6} \cmidrule(r){7-7} \cmidrule(r){8-8}
%         & Method & LC Winrate & Winrate & LC Winrate & Winrate & Score & Score \\
%         \midrule
%         \multirow{6}{*}{\rotatebox{90}{Gemma2-2b-it}} & SFT & 47.03 & 48.38 & 36.39 & 38.26 & 0.6545 & 0.2980 \\
%         \cmidrule{2-8}
%         & pairwise & 55.06 & 66.56 & 41.39 & 54.60 & 0.6740 & 0.3375 \\
%         & contrastive & 60.44 & 72.35 & 43.41 & 56.83 & 0.6745 & 0.3315 \\
%         & ListMLE & 63.05 & 76.09 & 49.77 & 62.05 & 0.6715 & 0.3560 \\
%         & LambdaRank & 58.73 & 74.09 & 43.76 & 60.56 & 0.6750 & 0.3560 \\
%         \midrule
%         \multirow{6}{*}{\rotatebox{90}{Mistral-7b-it}} & SFT & 27.04 & 17.41 & 21.14 & 14.22 & 0.7070 & 0.3610 \\
%         \cmidrule{2-8}
%         & pairwise & 49.75 & 55.07 & 36.43 & 41.86 & 0.7175 & 0.4105 \\
%         & contrastive & 52.03 & 60.15 & 38.44 & 42.61 & 0.7260 & 0.4340 \\
%         & ListMLE & 48.84 & 56.73 & 38.02 & 43.03 & 0.7360 & 0.4200 \\
%         & LambdaRank & 51.98 & 59.73 & 40.29 & 46.21 & 0.7370 & 0.4400 \\
%         \bottomrule
%     \end{tabular}}
% \end{table*}

\begin{table}[t]
    \centering
    % \renewcommand{\arraystretch}{1.2}
    % \vspace{-0.1in}
    \caption{Preference optimization objective study on AlpacaEval2 and MixEval. SFT corresponds to the initial chat model.}\label{tab:objective}
    \small
    \scalebox{0.99}{\begin{tabular}{llcccccc}
        \toprule
        & & \multicolumn{2}{c}{AlpacaEval 2} & \multicolumn{1}{c}{MixEval} & \multicolumn{1}{c}{MixEval-Hard} \\
         \cmidrule(r){3-4} \cmidrule(r){5-5} \cmidrule(r){6-6}
        & Method & LC Winrate & Winrate & Score & Score \\
        \midrule
        \multirow{6}{*}{\rotatebox{90}{Gemma2-2b-it}} & SFT & 36.39 & 38.26 & 0.6545 & 0.2980 \\
        \cmidrule{2-6}
        & pairwise & 41.39 & 54.60 & 0.6740 & 0.3375 \\
        & contrastive & 43.41 & 56.83 & 0.6745 & 0.3315 \\
        & ListMLE & \textbf{49.77} & \textbf{62.05} & 0.6715 & \textbf{0.3560} \\
        & LambdaRank & 43.76 & 60.56 & \textbf{0.6750} & \textbf{0.3560} \\
        \midrule
        \midrule
        \multirow{6}{*}{\rotatebox{90}{Mistral-7b-it}} & SFT & 21.14 & 14.22 & 0.7070 & 0.3610 \\
        \cmidrule{2-6}
        & pairwise & 36.43 & 41.86 & 0.7175 & 0.4105 \\
        & contrastive & 38.44 & 42.61 & 0.7260 & 0.4340 \\
        & ListMLE & 38.02 & 43.03 & 0.7360 & 0.4200 \\
        & LambdaRank & \textbf{40.29} & \textbf{46.21} & \textbf{0.7370} & \textbf{0.4400} \\
        \bottomrule
    \end{tabular}}
    % \vspace{-0.1in}
\end{table}


\subsection{Retriever optimization objective}

\paragraph{Experimental setting.}
Iterative preference optimization is performed on LLMs using the different learning objectives outlined in Section \ref{sec:retrieval-obj}.
Alignment experiments are conducted using the Gemma2-2b-it \citep{team2024gemma} and Mistral-7b-it \citep{jiang2023mistral} models, trained on the Ultrafeedback dataset \citep{cui2024ultrafeedback}. 
Following the methodology of \citep{dong2024rlhf}, we conduct three iterations of training and report the performance of the final checkpoint in Table \ref{tab:objective}.  
Model evaluations are performed on AlpacaEval2 \citep{dubois2024length} and MixEval \citep{ni2024mixeval}. 
% For AlpacaEval2, we employed both the open-source LLM evaluator \texttt{alpaca\_eval\_llama3\_70b\_fn} and the GPT4 evaluator \texttt{alpaca\_eval\_gpt4\_turbo\_fn}.
Detailed settings can be found in Appendix \ref{apx:sec-objective-setting}.


\paragraph{Observation.}
Table \ref{tab:objective} presents the results, from which we make the following observations: 
(1) Contrastive optimization generally outperforms pairwise optimization (\textit{e.g.}, DPO), likely due to its ability to incorporate more negative examples during each learning step. 
(2) Listwise optimization methods, including ListMLE and LambdaRank, generally demonstrate superior performance compared to both pairwise and contrastive approaches. 
This is attributed to their utilization of a more comprehensive set of preference information within the candidate list.




\begin{figure*}[t]
    \centering
    \subfigure[Hard negative study]{\includegraphics[width=0.32\textwidth]{figure/LLM_alignment_gsm8k_mathstral7b_neg_study.pdf}}
    \subfigure[Temperature \& hard negatives]{\includegraphics[width=0.32\textwidth]{figure/LLM_alignment_mistral_temperature_study.pdf}}
    \subfigure[Candidate list length study]{\includegraphics[width=0.32\textwidth]{figure/LLM_alignment_mistral_length_study.pdf}}
    % \vspace{-0.1in}
    \caption{Hard negative and candidate list study. (a) Hard negative study with $\mathcal{L}_{\text{pair}}$ on GSM8K with Mathstral-7b-it model. We explore four negative settings: (1) a random response not related to the given prompt; (2) a response to a related prompt; (3) an incorrect response to the given prompt with high temperature; (4) an incorrect response to the given prompt with suitable temperature. Hardness: (4)$>$(3)$>$(2)$>$(1). The harder the negatives are, the stronger the trained LLM is.
    (b) Training temperature study with $\mathcal{L}_{\text{pair}}$ on Mistral-7b-it and Alpaca Eval 2. Within a specific range ($>$ 1), lower temperature leads to harder negative and benefit the trained LLM. However, much lower temperature could lead to less diverse responses and finally lead to LLM alignment performance drop.
    (c) Candidate list size study with $\mathcal{L}_{\text{con}}$ on Mistral-7b-it. As the candidate list size increases, alignment performance improves.}\label{fig:merge-study}
    % \vspace{-0.1in}
\end{figure*}


% \begin{figure}[h!]
% \centering
% \includegraphics[scale=0.3]{figure/LLM_alignment_mistral_length_study.pdf}
% \vskip -1em
% \caption{Candidate list size study with $\mathcal{L}_{\text{con}}$ on Mistral-7b-it. As the candidate list size increases, alignment performance improves.}\label{fig:length-study}\vspace{-10pt}
% \end{figure}


% \begin{figure}[h!]
% \centering
% \includegraphics[scale=0.3]{figure/LLM_alignment_mistral_temperature_study.pdf}
% \vskip -1em
% \caption{Training temperature study with $\mathcal{L}_{\text{pair}}$ on Mistral-7b-it and Alpaca Eval 2. Within a specific range ($>$ 1), lower temperature leads to harder negative and benefit the trained LLM. However, temperature lower than this range can cause preferred and rejected responses non-distinguishable and lead to degrade training.}\label{tab:temp-hard}
% \end{figure}


% \begin{figure}[h!]
% \centering
% \includegraphics[scale=0.3]{figure/LLM_alignment_gsm8k_mathstral7b_neg_study.pdf}
% \vskip -1em
% \caption{Hard negative study with $\mathcal{L}_{\text{pair}}$ on GSM8K with Mathstral-7b-it model. We explore four negative settings: (1) a random response not related to the given prompt; (2) a response to a related prompt; (3) an incorrect response to the given prompt with high temperature; (4) an incorrect response to the given prompt with suitable temperature. Hardness: (4)$>$(3)$>$(2)$>$(1). The harder the negatives are, the stronger the trained LLM is.}\label{fig:mathstral-gsm8k-hard}
% \end{figure}


\subsection{Hard negatives}

\paragraph{Experimental setting.}
The Mathstral-7b-it model is trained on the GSM8k training set and evaluated its performance on the GSM8k test set. 
Iterative DPO is employed as the RLHF method, with the gold or correct response designated as the positive example. 
The impact of different hard negative variants is investigated, as described in Section \ref{sec:hard-negative}, with the results presented in Figure \ref{fig:merge-study}(a). 
Additionally, the influence of temperature on negative hardness with Lambdarank objective are examined using experiments on the AlpacaEval 2 dataset, with results shown in Figure \ref{fig:merge-study}(b).
Detailed settings are in Appendix \ref{apx:sec-hard-neg-setting} and \ref{apx:sec-hard-neg-setting-temp}.

% 

\begin{figure}[h!]
\centering
\includegraphics[scale=0.3]{figure/LLM_alignment_gsm8k_mathstral7b_neg_study.pdf}
\vskip -1em
\caption{Hard negative study with $\mathcal{L}_{\text{pair}}$ on GSM8K with Mathstral-7b-it model. We explore four negative settings: (1) a random response not related to the given prompt; (2) a response to a related prompt; (3) an incorrect response to the given prompt with high temperature; (4) an incorrect response to the given prompt with suitable temperature. Hardness: (4)$>$(3)$>$(2)$>$(1). The harder the negatives are, the stronger the trained LLM is.}\label{fig:mathstral-gsm8k-hard}
\end{figure}

% \begin{table}[t]
%     \centering
%     % \renewcommand{\arraystretch}{1.2}
%     \caption{Temperature study results for Gemma2-2b-it and Mistral-7b-it. We conduct RLHF (iterative DPO) for 3 iterations. $\uparrow$, $\rightarrow$ and $\downarrow$ denote RLHF with high, medium and low temperature. We use \texttt{alpaca\_eval\_llama3\_70b\_fn} as the evaluator.}\label{tab:temp-hard}
%     \scalebox{0.9}{\begin{tabular}{llcc}
%         \toprule
%         & & \multicolumn{2}{c}{Alpaca Eval 2} \\
%          \cmidrule(r){3-4}
%         & Method & LC Winrate & Winrate \\
%         \midrule
%         \multirow{5}{*}{\rotatebox{90}{Gemma2}} & SFT & 47.03 & 48.38 \\
%         \cmidrule{2-4}
%         & RLHF ($\uparrow$) & 54.45 & 67.50 \\
%         & RLHF ($\rightarrow$) & 59.31 & 69.77 \\
%         & RLHF ($\downarrow$) & 59.04 & 71.38  \\
%         \midrule
%         \multirow{5}{*}{\rotatebox{90}{Mistral}} & SFT & 27.04 & 17.41  \\
%         \cmidrule{2-4}
%         & RLHF ($\uparrow$) & 49.75 & 55.07 \\
%         & RLHF ($\rightarrow$) & 53.29 & 60.52 \\
%         & RLHF ($\downarrow$) & 54.78 & 64.33 \\
%         \bottomrule
%     \end{tabular}}
% \end{table}


% \begin{figure}[h!]
% \centering
% \includegraphics[scale=0.3]{figure/LLM_alignment_mistral_temperature_study.pdf}
% \vskip -1em
% \caption{Training temperature study with $\mathcal{L}_{\text{pair}}$ on Mistral-7b-it and Alpaca Eval 2. Within a specific range ($>$ 1), lower temperature leads to harder negative and benefit the trained LLM. However, temperature lower than this range can cause preferred and rejected responses non-distinguishable and lead to degrade training.}\label{tab:temp-hard}
% \end{figure}


\paragraph{Observation.}
Figure \ref{fig:merge-study}(a) illustrates that the effectiveness of the final LLM is directly correlated with the hardness of the negatives used during training. 
Harder negatives consistently lead to a more performant LLM.  
Figure \ref{fig:merge-study}(b) further demonstrates that, within a specific range, lower temperatures generate harder negatives, resulting in a more effective final trained LLM. 
% However, much lower temperature could also affect the quality of the chosen responses, make the chose and rejected responses non-distinguishable and finally lead to performance drop.
However, much lower temperature could lead to less diverse responses and finally lead to LLM alignment performance drop.
% Definition of temperatures can be found in Appendix \ref{apx:sec-hard-neg-setting-temp}.


\subsection{Candidate List}

\paragraph{Experimental setting.}
To investigate the impact of inclusiveness and memorization on LLM alignment, experiments are conducted using Gemma2-2b-it, employing the same training settings as in our objective study. 
For the inclusiveness study, the performance of the trained LLM is evaluated using varying numbers of candidates in the list.
For the memorization study, three approaches are compared: (i) using only the current iteration's responses, (ii) using responses from the current and previous iteration, and (iii) using responses from the current and all previous iterations. 
% Finally, for the temperature diversity study, the effect of employing different sampling temperatures is examined during response generation.
Detailed settings can be found in Appendix \ref{apx:sec-length-setting} and \ref{apx:sec-list-setting}.


% \begin{figure}[h!]
% \centering
% \includegraphics[scale=0.3]{figure/LLM_alignment_mistral_length_study.pdf}
% \vskip -1em
% \caption{Candidate list size study with $\mathcal{L}_{\text{con}}$ on Mistral-7b-it. As the candidate list size increases, alignment performance improves.}\label{fig:length-study}\vspace{-10pt}
% \end{figure}


\begin{table}[t]
    \centering
    % \vspace{-0.15in}
    \caption{Candidate list study with $\mathcal{L}_{\text{pair}}$ on Gemma2-2b-it. Previous iteration responses enhance performance.}\label{fig:list-study}
    \small
    \scalebox{0.99}{\begin{tabular}{lcc}
        \toprule
        & \multicolumn{2}{c}{Alpaca Eval 2} \\
         \cmidrule(r){2-3}
        Method & LC Winrate & Winrate \\
        \midrule
         SFT & 47.03 & 48.38 \\
        \cmidrule{1-3}
        Alignment (w. current)  & 55.06 & 66.56 \\
        Alignment (w. current + prev) & 55.62 & 70.92 \\
        Alignment (w. current + all prev) & 56.02 & 72.50  \\
        % \cmidrule{1-3}
        % Alignment (single temperature)  & 55.06 & 66.56 \\
        % Alignment (diverse temperature)  & 59.36 & 73.47  \\
        \bottomrule
    \end{tabular}}
    % \vspace{-0.15in}
\end{table}



\paragraph{Observation.}
Figure \ref{fig:merge-study}(c) illustrates the significant impact of candidate list size on LLM alignment performance.
As the candidate list size increases, performance improves, albeit with a diminishing rate of return. 
This is intuitive, given that a bigger candidate list size can contribute to more hard negatives and potentially benefit the model learning \citep{qu2020rocketqa}.
Table \ref{fig:list-study} demonstrates that incorporating responses from previous iterations can enhance performance.
This is potentially because introducing previous responses can make the candidate list more comprehensive and lead to better preference signal capturing.
More explanations are in Appendix \ref{apx:sec-list-setting}.


% \begin{table}[t]
%     \centering
%     % \renewcommand{\arraystretch}{1.2}
%     \begin{tabular}{lcc}
%         \toprule
%         & \multicolumn{2}{c}{Alpaca Eval 2} \\
%          \cmidrule(r){2-3}
%         Method & LC Winrate & Winrate \\
%         \midrule
%          SFT & 27.04 & 17.41 \\
%         \cmidrule{1-3}
%         RLHF (4 responses) & 50.02 & 61.72 \\
%         RLHF (6 responses) & 52.56 & 63.59 \\
%         RLHF (8 responses) & 55.21 & 64.88  \\
%         RLHF (10 responses) & 55.52 & 64.42  \\
%         \bottomrule
%     \end{tabular}
%     \caption{Candidate list size study for Mistral-7b-It. We conduct RLHF (iterative InfoPO) for 3 iterations. We use \texttt{alpaca\_eval\_llama3\_70b\_fn} as the evaluator.}
% \end{table}



\section{Related works}

\paragraph{LLM alignment.}
Pretrained LLMs demonstrate remarkable capabilities across a broad spectrum of tasks \citep{brown2020language}.
Their performance at downstream tasks, such as conversational modeling, is significantly enhanced through alignment with human preferences \citep{ouyang2022training, bai2022training}. 
RLHF \citep{christiano2017deep} has emerged as a foundational framework for this alignment, typically involving learning a reward function via a preference model, often using the Bradley-Terry model \citep{bradley1952rank}, and tuning the LLM using reinforcement learning (RL) to optimize this reward. 
Despite its success, RLHF's practical implementation is notoriously complex, requiring multiple LLMs, careful hyperparameter tuning, and navigating challenging optimization landscapes.

Recent research has focused on simplifying this process. A line of works studies the direct alignment algorithms \citep{zhao2023slic, rafailov2024direct, azar2024general}, which directly optimize the LLM in a supervised manner without first constructing a separate reward model. In particular, the representative DPO \citep{rafailov2024direct} attracts significant attention in both academia and industry. After these, SimPO \citep{meng2024simpo} simplifies DPO by using length regularization in place of a reference model. 
% However, these approaches are primarily offline and can suffer from performance degradation when facing distribution shifts during deployment, yielding poorer generalization.
% Iterative DPO \citep{xiong2024iterative} attempts to address this by enabling iterative optimization for improved alignment.

% These challenges can be potentially tackled by connecting LLM alignment with IR.
Although LLMs are adopted for IR \citep{tay2022transformer}, there is a lack of study to improve direct LLM alignment with IR principles.
% While significant progress has been made in LLM alignment, the connection with IR remains largely unexplored.
% However, LiPO does not explore the potential of online optimization methods inspired by IR. 
This paper fills this gap by establishing a systematic link between LLM alignment and IR methodologies, and introducing a novel iterative LLM alignment approach that leverages insights from retriever optimization to advance the state of the art.
The most related work is LiPO \citep{liu2024lipo}, which applies learning-to-rank objectives.
% However, LiPO focuses on offline settings and it is unclear how to obtain list-wise data for it.
However, LiPO relies on off-the-shelf listwise preference data, which is hard to satisfy in practice.

\paragraph{Language models for information retrieval.}
Language models (LMs) have become integral to modern IR systems \citep{zhu2023large}, particularly after the advent of pretrained models like BERT \citep{kenton2019bert}.  
A typical IR pipeline employs retrievers and rerankers, often based on dual-encoder and cross-encoder architectures, respectively \citep{humeau2019poly}. 
Dense Passage Retrieval (DPR) \citep{karpukhin2020dense} pioneered the concept of dense retrieval, laying the groundwork for subsequent research. 
Building on DPR, studies have emphasized the importance of hard negatives in training \citep{zhan2021optimizing, qu2020rocketqa} and the benefits of online retriever optimization \citep{xiong2020approximate}.

In the realm of reranking, \citep{nogueira2019passage} were among the first to leverage pretrained language models for improved passage ranking. 
This was followed by MonoT5 \citep{nogueira2020document}, which scaled rerankers using large encoder-decoder transformer architectures, and RankT5 \citep{zhuang2023rankt5}, which introduced pairwise and listwise ranking objectives. 
Recent work has also highlighted the importance of candidate list preprocessing before reranking \citep{meng2024ranked}.

Despite the pervasive use of LMs in IR, the interplay between LLM alignment and IR paradigms remains largely unexplored. 
This work aims to bridge this gap, establishing a strong connection between LLM alignment and IR, and leveraging insights from both fields to advance our understanding of LLM alignment from an IR perspective.


\section{Conclusions}
This paper investigates the impact of increasing the number of retrieved passages on the performance of long-context LLMs in retrieval-augmented generation (RAG) systems.
Contrary to expectations, we observe that performance initially improve but then degrade as more passages are included.
This phenomenon is attributed to the detrimental influence of retrieved "hard negatives".
To mitigate this issue, we propose and evaluate three solutions: training-free retrieval reordering, RAG-specific implicit LLM fine-tuning, and RAG-oriented LLM fine-tuning with intermediate reasoning.
A systematic analysis of the training-based methods explores the effects of data distribution, retriever for training, and training context length.
Interesting future directions include exploring (automated) position optimization with more advanced retrieval ordering methods, and fine-tuning the LLMs for RAG with more fine-grained and multi-step reasoning chains.

% Bibliography components
\bibliographystyle{abbrvnat}
\nobibliography*
\bibliography{main}

\clearpage
% Some other useful sections you might consider having in your report.
% \section*{Citing this work}
% This is a free, open access paper provided by Google DeepMind.
% The final version of this work was published online (provide venue, date and digital object
% identifier (doi, if available). \textit{Cite as:} \citeas{lee2024gecko}.

% If you add a bibtex entry of your own paper (this paper), you can
% show its full citation inline using \citeas, as above. Note that
% this citation removes the trailing full stop. To make \citeas work,
% you need to load the bibliography data. This can be done in two
% ways:
%
%    1. If you already have a printed bibliography with \bibliography{...},
%       then add the command "\nobibliography*", no arguments, before that.
%    2. If you don't otherwise print a bibliography, add the command
%       \nobibliography{...} at the end of your document.


% \section*{Author Contributions}

% \section*{Acknowledgements}

% \section*{Funding}
% This research was funded by Google.

% \section*{Competing interests}
% The authors declare no competing financial interests. Related
% patent number here if applicable.

% \section*{Data availability}
% The datasets used in the experiments have been made available
% for download at IF AVAILABLE.


%\clearpage

\appendix
\section*{Appendix}
\begin{table*}[h!]
\caption{The basic information of grid-based spatio-temporal data.}
\label{tbl:append_data}
\begin{threeparttable}
\resizebox{1.9\columnwidth}{!}{
\begin{tabular}{cccccccc}
\toprule
Dataset & City & Type & Temporal Period & Spatial partition & Interval & Mean & Std \\
\hline
TaxiBJ & Beijing & Taxi flow&  2014/03/01 - 2014/06/30 & $32 \times 32$ & Half an hour & 111.5 & 139.3 \\
BikeDC & Washington, D.C. & Bike flow&  2010/09/20 - 2010/10/20 & $20 \times 20$ & Half an hour & 0.924 & 4.88 \\



CellularSH & Shanghai & Cellular traffic &  2014/08/01 - 2014/08/21 & $32\times28$ & One hour & 0.175 & 0.212 \\
CellularNJ & Nanjing & Cellular traffic &  2021/02/02 - 2021/02/22 & $20\times28$ & One hour & 0.842 & 1.30 \\
CrowdBJ & Beijing & Crowd flow &  2018/01/01 - 2018/01/31 & $1010$ & One hour & 7.07 & 11.1 \\
CrowdBM & Baltimore & Crowd flow &  2019/01/01 - 2019/05/31 & $403$ & One hour & 14.4 & 29.3 \\
Los-Speed & Los Angeles & Traffic speed&  2012/03/01 - 2012/03/07 & $207$ & Five minutes & 59.0 & 12.5 \\

\bottomrule
\end{tabular}}
\end{threeparttable}
\end{table*}

% \begin{table*}[t!]
% \caption{The basic information of Graph-based spatio-temporal data.}
% \label{tbl:append_data_graph}
% \begin{threeparttable}
% \resizebox{1.8\columnwidth}{!}{
% \begin{tabular}{ccccccccc}
% \toprule
% Dataset & City & Type & Temporal Period & Interval & \#Nodes & \#Edges & Mean & Std \\
% \hline

% TrafficBJ & Beijing & Traffic speed & 2022/03/05 - 2022/04/05 & 15min& 13675& 24444& 6.837&  3.412\\
% TrafficSH & Shanghai & Traffic speed & 2022/01/27 - 2022/02/27 & 15min & 21099& 39065& 7.815&  4.044\\
% TrafficNJ & Nanjing & Traffic speed  & 2022/03/05 - 2022/04/05 & 15min & 13419& 25100& 6.699&  4.253\\

% \bottomrule
% \end{tabular}}
% \end{threeparttable}
% \end{table*}
\begin{table*}[h!]
\caption{Short-term prediction results on two additional datasets in terms of both deterministic and probabilistic metrics. \textbf{Bold} indicates the best performance, while \underline{underlining} denotes the second-best.}
\label{tbl:short1-app}
\begin{threeparttable}
% \resizebox{2.0\columnwidth}

\resizebox{1.5\columnwidth}{!}{
\begin{tabular}{ccccccccccc}
\toprule
\multirow{2}{*}{\textbf{Model}}
& \multicolumn{5}{c}{\textbf{CellularNJ}} & \multicolumn{5}{c}{\textbf{CrowdBM}}   \\
\cmidrule(lr){2-6} \cmidrule(lr){7-11} 
 &\textbf{MAE} & \textbf{RMSE}  &\textbf{CRPS} & \textbf{QICE} & \textbf{IS} & 
\textbf{MAE} & \textbf{RMSE} & \textbf{CRPS} & \textbf{QICE} & \textbf{IS} \\


\midrule
D3VAE& 0.580 & 1.135   &	0.565&	0.096&6.03	&	11.0&24.7	&	0.593& 0.110	&136.4\\


DiffSTG& 0.317& 0.649&	0.291&	0.071&3.11	&	8.88&21.3	&0.453&0.047	&68.5\\

TimeGrad& 0.340&  0.357  &	0.432&	0.162&5.87	&10.1	& \underline{12.4}	&\textbf{0.240}&\underline{0.085}	&\underline{46.9}\\


CSDI&0.129 & 0.237   &	\underline{0.111}&	 \underline{0.039}&	\underline{0.80}&	7.31& 19.3&	0.390& 0.054&61.1\\

NPDiff& \underline{0.123}&  \underline{0.175}  &0.128	&	0.133&2.22	&	\underline{5.42}& 13.7	&0.331&0.119	&91.2\\

DyDiffusion&0.222&   0.357 &0.196	&0.080	&	1.80&-	&-	&-	&-&-\\




\cmidrule(lr){1-1} \cmidrule(lr){2-6} \cmidrule(lr){7-11}
\textbf{CoST}&\textbf{0.102} &\textbf{0.172}    &\textbf{0.090}	&\textbf{0.037}	&\textbf{0.682}	&\textbf{5.04}	&\textbf{12.1}	&\underline{0.256}	&\textbf{0.027}& \textbf{37.8}\\
%\textbf{Reduction}& &    &	&	&	&	&	&	&\\
\bottomrule
\end{tabular}}
\end{threeparttable}
\end{table*}


% \input{Tables/short-term2-append}

\end{document}
