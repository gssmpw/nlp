% This class has a lot of options, so please check deepmind.cls for more details.
% This is a minimal set for most needs.
\documentclass[11pt, a4paper, logo, copyright]{googlecloud}

% Omit dates for reproducibility.
\pdfinfoomitdate 1
\pdftrailerid{redacted}

% This avoids duplicate hyperref bookmark entries when using \bibentry (e.g. via \citeas).
\makeatletter
\renewcommand\bibentry[1]{\nocite{#1}{\frenchspacing\@nameuse{BR@r@#1\@extra@b@citeb}}}
\makeatother

\usepackage{kantlipsum, lipsum}
\usepackage{dsfont}
% \usepackage{gdm-colors}

% Sometimes you will get errors about pdflink ending up in diffrent position. Try this and
% comment it out again when you are done with your document.
%\hypersetup{draft}

% Set the bibliography options here.
\usepackage[authoryear, sort&compress, round]{natbib}

\usepackage[utf8]{inputenc} % allow utf-8 input
\usepackage[T1]{fontenc}    % use 8-bit T1 fonts
\usepackage{url}            % simple URL typesetting
\usepackage{booktabs}       % professional-quality tables
\usepackage{nicefrac}       % compact symbols for 1/2, etc.
\usepackage{microtype}      % microtypography
\usepackage{amsmath}
\usepackage{graphicx}
\usepackage{multicol}
\usepackage{hyperref}       % hyperlinks
\usepackage[nameinlink]{cleveref}
\usepackage{bbm}
\usepackage{multirow}
% \usepackage{subfig}
\usepackage{soul}
\usepackage{floatrow}
\usepackage{float}
\usepackage{wrapfig}
\usepackage{blindtext}
\usepackage{tablefootnote}
% \usepackage{fdsymbol}. % cause an error
\usepackage{amsfonts}
\usepackage[flushleft]{threeparttable}
% \usepackage{colortbl}
\usepackage{bbding}
\usepackage{xcolor}
\usepackage{xspace}
\usepackage{bm}
\usepackage{arydshln}
\usepackage{subfigure}
\usepackage{enumitem}
\usepackage{setspace}
% \usepackage{mathabx}
\usepackage{color}
\usepackage{algorithm}
\usepackage{algorithmic}
\usepackage{longtable}
% \usepackage{booktabs}

\usepackage[normalem]{ulem}
\usepackage{ulem}
\usepackage[nomargin,inline,marginclue,draft]{fixme}
\usepackage{balance}
\usepackage{verbatim}
\usepackage{diagbox}
\usepackage{changepage}
\usepackage{amssymb}
\usepackage{pifont}
\usepackage{array}   % Optional: Improves table column alignment

%%%%%%%%%%% mboratko: added these, feel free to adjust
\DeclareMathOperator*{\argmax}{arg\,max}
\DeclareMathOperator*{\argmin}{arg\,min}
%%%%%%%%%%

\newcommand{\Ours}{\textsc{LarPO}\xspace}

\definecolor{myred}{rgb}{1, 0, 0}
\definecolor{myblue}{rgb}{0, 0, 1}
\definecolor{myblack}{rgb}{1, 1, 1}

\newcommand{\bmh}{{\bm h}}
\newcommand{\bmW}{{\bm W}}
\newcommand{\bmz}{{\bm z}}
\newcommand{\bmH}{{\bm H}}
\newcommand{\bme}{{\bm e}}
\newcommand{\bmQ}{{\bm Q}}
\newcommand{\bmK}{{\bm K}}
\newcommand{\bmV}{{\bm V}}
\newcommand{\bmc}{{\bm c}}
\newcommand{\bmE}{{\bm E}}
\newcommand{\bmd}{{\bm d}}
\newcommand{\bmy}{{\bm y}}
\newcommand{\bmx}{{\bm x}}

%%%%%%%%%%%%%%%%%%%%%%%%%%%%%%%%
% THEOREMS
%%%%%%%%%%%%%%%%%%%%%%%%%%%%%%%%
\theoremstyle{plain}
\newtheorem{theorem}{Theorem}[section]
\newtheorem{proposition}[theorem]{Proposition}
\newtheorem{lemma}[theorem]{Lemma}
\newtheorem{corollary}[theorem]{Corollary}
\theoremstyle{definition}
\newtheorem{definition}[theorem]{Definition}
\newtheorem{assumption}[theorem]{Assumption}
\theoremstyle{remark}
\newtheorem{remark}[theorem]{Remark}

% %%%%%%%%%%%---SETME-----%%%%%%%%%%%%%
%replace @@ with the submission number submission site.
\newcommand{\thiswork}{INF$^2$\xspace}
%%%%%%%%%%%%%%%%%%%%%%%%%%%%%%%%%%%%


%\newcommand{\rev}[1]{{\color{olivegreen}#1}}
\newcommand{\rev}[1]{{#1}}


\newcommand{\JL}[1]{{\color{cyan}[\textbf{\sc JLee}: \textit{#1}]}}
\newcommand{\JW}[1]{{\color{orange}[\textbf{\sc JJung}: \textit{#1}]}}
\newcommand{\JY}[1]{{\color{blue(ncs)}[\textbf{\sc JSong}: \textit{#1}]}}
\newcommand{\HS}[1]{{\color{magenta}[\textbf{\sc HJang}: \textit{#1}]}}
\newcommand{\CS}[1]{{\color{navy}[\textbf{\sc CShin}: \textit{#1}]}}
\newcommand{\SN}[1]{{\color{olive}[\textbf{\sc SNoh}: \textit{#1}]}}

%\def\final{}   % uncomment this for the submission version
\ifdefined\final
\renewcommand{\JL}[1]{}
\renewcommand{\JW}[1]{}
\renewcommand{\JY}[1]{}
\renewcommand{\HS}[1]{}
\renewcommand{\CS}[1]{}
\renewcommand{\SN}[1]{}
\fi

%%% Notion for baseline approaches %%% 
\newcommand{\baseline}{offloading-based batched inference\xspace}
\newcommand{\Baseline}{Offloading-based batched inference\xspace}


\newcommand{\ans}{attention-near storage\xspace}
\newcommand{\Ans}{Attention-near storage\xspace}
\newcommand{\ANS}{Attention-Near Storage\xspace}

\newcommand{\wb}{delayed KV cache writeback\xspace}
\newcommand{\Wb}{Delayed KV cache writeback\xspace}
\newcommand{\WB}{Delayed KV Cache Writeback\xspace}

\newcommand{\xcache}{X-cache\xspace}
\newcommand{\XCACHE}{X-Cache\xspace}


%%% Notions for our methods %%%
\newcommand{\schemea}{\textbf{Expanding supported maximum sequence length with optimized performance}\xspace}
\newcommand{\Schemea}{\textbf{Expanding supported maximum sequence length with optimized performance}\xspace}

\newcommand{\schemeb}{\textbf{Optimizing the storage device performance}\xspace}
\newcommand{\Schemeb}{\textbf{Optimizing the storage device performance}\xspace}

\newcommand{\schemec}{\textbf{Orthogonally supporting Compression Techniques}\xspace}
\newcommand{\Schemec}{\textbf{Orthogonally supporting Compression Techniques}\xspace}



% Circular numbers
\usepackage{tikz}
\newcommand*\circled[1]{\tikz[baseline=(char.base)]{
            \node[shape=circle,draw,inner sep=0.4pt] (char) {#1};}}

\newcommand*\bcircled[1]{\tikz[baseline=(char.base)]{
            \node[shape=circle,draw,inner sep=0.4pt, fill=black, text=white] (char) {#1};}}

% Images will be looked for in this path, removes need for explicit path when including images.
\graphicspath{{figures/}}

% Important Information about your paper.
\title{LLM Alignment as Retriever Optimization: An Information Retrieval Perspective}

% Can leave this option out if you do not wish to add a corresponding author.
\correspondingauthor{bowenj4@illinois.edu}

% Remove these if they are not needed
% \keywords{\LaTeX, Publications process, tools}
% \paperurl{arxiv.org/abs/123}

% Use the internally issued paper ID, if there is one
% \reportnumber{001} % Leave blank if n/a

% Assign your own date to the report.
% Can comment out if not needed or leave blank if n/a.
\renewcommand{\today}{}

% Can have as many authors and as many affiliations as needed. Best to indicate joint
% first-authorship as shown below.
\author[1 2 *]{Bowen Jin}
\author[1]{Jinsung Yoon}
\author[3]{Zhen Qin}
\author[2]{Ziqi Wang}
\author[2]{Wei Xiong}
\author[4]{Yu Meng}
\author[2]{Jiawei Han}
\author[1]{Sercan Ö. Arık\hspace{-0.4ex}}

% Affiliations *must* come after the declaration of \author[]
% \affil[*]{Equal contributions}
\affil[1]{Google Cloud}
\affil[2]{University of Illinois at Urbana-Champaign}
\affil[3]{Google DeepMind}
\affil[4]{University of Virginia}

\begin{abstract}
Large Language Models (LLMs) have revolutionized artificial intelligence with capabilities in reasoning, coding, and communication, driving innovation across industries. 
Their true potential depends on effective alignment to ensure correct, trustworthy and ethical behavior, addressing challenges like misinformation, hallucinations, bias and misuse.
While existing Reinforcement Learning (RL)-based alignment methods are notoriously complex, direct optimization approaches offer a simpler alternative.
In this work, we introduce a novel direct optimization approach for LLM alignment by drawing on established Information Retrieval (IR) principles. 
We present a systematic framework that bridges LLM alignment and IR methodologies, mapping LLM generation and reward models to IR's retriever-reranker paradigm. 
Building on this foundation, we propose \textbf{L}LM \textbf{A}lignment as \textbf{R}etriever \textbf{P}reference \textbf{O}ptimization (\Ours), a new alignment method that enhances overall alignment quality.
Extensive experiments validate \Ours's effectiveness with 38.9\% and 13.7\% averaged improvement on AlpacaEval2 and MixEval-Hard respectively.
Our work opens new avenues for advancing LLM alignment by integrating IR foundations, offering a promising direction for future research.
\end{abstract}

\begin{document}

\maketitle



\section{Introduction}
\IEEEPARstart{I}{n} recent years, flourishing of Artificial Intelligence Generated Content (AIGC) has sparked significant advancements in modalities such as text, image, audio, and even video. 
Among these, AI-Generated Image (AGI) has garnered considerable interest from both researchers and the public.
Plenty of remarkable AGI models and online services, such as StableDiffusion\footnote{\url{https://stability.ai/}}, Midjourney\footnote{\url{https://www.midjourney.com/}}, and FLUX\footnote{\url{https://blackforestlabs.ai/}}, offer users an excellent creative experience.
However, users often remain critical of the quality of the AGI due to image distortions or mismatches with user intentions.
Consequently, methods for assessing the quality of AGI are becoming increasingly crucial to help improve the generative capabilities of these models.

Unlike Natural Scene Image (NSI) quality assessment, which focuses primarily on perception aspects such as sharpness, color, and brightness, AI-Generated Image Quality Assessment (AGIQA) encompasses additional aspects like correspondence and authenticity. 
Since AGI is generated on the basis of user text prompts, it may fail to capture key user intentions, resulting in misalignment with the prompt.
Furthermore, authenticity refers to how closely the generated image resembles real-world artworks, as AGI can sometimes exhibit logical inconsistencies.
While traditional IQA models may effectively evaluate perceptual quality, they are often less capable of adequately assessing aspects such as correspondence and authenticity.

\begin{figure}\label{fig:radar}
    \centering
    \includegraphics[width=1.0\linewidth]{figures/radar_plot.pdf}
    \caption{A comparison on quality, correspondence, and authenticity aspects of AIGCIQA2023~\cite{wang2023aigciqa2023} dataset illustrates the superior performance of our method.}
\end{figure}

Several methods have been proposed specifically for the AGIQA task, including metrics designed to evaluate the authenticity and diversity of generated images~\cite{gulrajani2017improved,heusel2017gans}. 
Nevertheless, these methods tend to compare and evaluate grouped images rather than single instances, which limits their utility for single image assessment.
Beginning with AGIQA-1k~\cite{zhang2023perceptual}, a series of AGIQA databases have been introduced, including AGIQA-3k~\cite{li2023agiqa}, AIGCIQA-20k~\cite{li2024aigiqa}, etc.
Concurrently, there has been a surge in research utilizing deep learning methods~\cite{zhou2024adaptive,peng2024aigc,yu2024sf}, which have significantly benefited from pre-trained models such as CLIP~\cite{radford2021learning}. 
These approaches enhance the analysis by leveraging the correlations between images and their descriptive texts.
While these models are effective in capturing general text-image alignments, they may not effectively detect subtle inconsistencies or mismatches between the generated image content and the detailed nuances of the textual description.
Moreover, as these models are pre-trained on large-scale datasets for broad tasks, they might not fully exploit the textual information pertinent to the specific context of AGIQA without task-specific fine-tuning.
To overcome these limitations, methods that leverage Multimodal Large Language Models (MLLMs)~\cite{wang2024large,wang2024understanding} have been proposed.
These methods aim to fully exploit the synergies of image captioning and textual analysis for AGIQA.
Although they benefit from advanced prompt understanding, instruction following, and generation capabilities, they often do not utilize MLLMs as encoders capable of producing a sequence of logits that integrate both image and text context.

In conclusion, the field of AI-Generated Image Quality Assessment (AGIQA) continues to face significant challenges: 
(1) Developing comprehensive methods to assess AGIs from multiple dimensions, including quality, correspondence, and authenticity; 
(2) Enhancing assessment techniques to more accurately reflect human perception and the nuanced intentions embedded within prompts; 
(3) Optimizing the use of Multimodal Large Language Models (MLLMs) to fully exploit their multimodal encoding capabilities.

To address these challenges, we propose a novel method M3-AGIQA (\textbf{M}ultimodal, \textbf{M}ulti-Round, \textbf{M}ulti-Aspect AI-Generated Image Quality Assessment) which leverages MLLMs as both image and text encoders. 
This approach incorporates an additional network to align human perception and intentions, aiming to enhance assessment accuracy. 
Specially, we distill the rich image captioning capability from online MLLMs into a local MLLM through Low-Rank Adaption (LoRA) fine-tuning, and train this model with human-labeled data. The key contributions of this paper are as follows:
\begin{itemize}
    \item We propose a novel AGIQA method that distills multi-aspect image captioning capabilities to enable comprehensive evaluation. Specifically, we use an online MLLM service to generate aspect-specific image descriptions and fine-tune a local MLLM with these descriptions in a structured two-round conversational format.
    \item We investigate the encoding potential of MLLMs to better align with human perceptual judgments and intentions, uncovering previously underestimated capabilities of MLLMs in the AGIQA domain. To leverage sequential information, we append an xLSTM feature extractor and a regression head to the encoding output.
    \item Extensive experiments across multiple datasets demonstrate that our method achieves superior performance, setting a new state-of-the-art (SOTA) benchmark in AGIQA.
\end{itemize}

In this work, we present related works in Sec.~\ref{sec:related}, followed by the details of our M3-AGIQA method in Sec.~\ref{sec:method}. Sec.~\ref{sec:exp} outlines our experimental design and presents the results. Sec.~\ref{sec:limit},~\ref{sec:ethics} and~\ref{sec:conclusion} discuss the limitations, ethical concerns, future directions and conclusions of our study.

% \section{LLM alignment as retriever optimization}
\section{An Information Retrieval Perspective on LLMs}\label{sec:understanding}


\begin{figure*}
\centering
\includegraphics[scale=0.53]{figure/arch2.pdf}
\caption{Architecture connection between retriever/LLM (bi-encoder) and reranker/reward model (cross-encoder). 
Bi-encoder models process each query/prompt and passage/response separately and often calculate their alignment score via a dot product operator, while cross-encoder models take both query/prompt and passage/response as input and score them directly.
Bi-encoder models can be more efficient (\textit{i.e.}, large-scale text matching) but the interaction between the two information unit is only captured by a dot production operation where their effectiveness can be constrained. Cross-encoder models can be more effective (\textit{i.e.}, deeper interaction calculation with transformer architecture \citep{vaswani2017attention}) but less efficient. Although LLM involves auto-regressive token matching, which is different from retriever, some insights from IR can be borrowed to enhance LLM alignment as shown in the following sections.}\label{fig:bi-cross-encoder}
% \vspace{-0.1in}
\end{figure*}

% \paragraph{Primer on information retrieval}
\subsection{Primer on information retrieval}

Information retrieval systems \citep{zhu2023large} typically employ a two-stage process involving retrievers \citep{zhao2024dense} and rerankers \citep{lin2022pretrained}.
The retriever, often implemented as a bi-encoder (Figure \ref{fig:bi-cross-encoder}), efficiently identifies a large set of ($K$) potentially relevant passages, denoted as $D_{\text{retrieval}}$, from a corpora $C$ given a query $q$.
% This is achieved using a coarse-grained similarity function, $f_{\text{retrieval}}(q, d)=\text{Enc}^T_q(q) \cdot \text{Enc}_d(d)$, where $\text{Enc}_q$ and $\text{Enc}_d$ represent the query and passage encoders respectively:
This is achieved using a coarse-grained similarity function, $p_{\text{retrieval}}(d|q)=\text{Enc}^T_q(q) \cdot \text{Enc}_d(d)$, where $\text{Enc}_q$ and $\text{Enc}_d$ represent the query and passage encoders respectively:
\begin{gather}\label{eq:retriever}
    D_{\text{retrieval}}(q) = \{ d \in C \;|\; \max_{\text{top-}K} ~ p_{\text{retrieval}}(\cdot | q)\}.
\end{gather}
However, due to the scale of the corpus, retrievers might not accurately capture fine-grained query-passage similarity with the simple dot production interaction function. 
Therefore, rerankers, typically implemented with cross-encoder (Figure \ref{fig:bi-cross-encoder}), are employed to refine the ranking of the retrieved passages $D_{\text{retrieval}}$.
% The reranker produces a smaller set ($k$) of top-ranked passages, $D_{rank}$, using a fine-grained similarity function, $f_{\text{rank}}(q,d)=w^T \cdot \text{Enc}(q, d)$.
The reranker produces a smaller set ($k$) of top-ranked passages, $D_\text{rank}$, using a fine-grained similarity function, $r_{\text{rank}}(q,d)=w \cdot \text{Enc}(q, d)$, where $w$ is a learnable linear layer. 
Here, reranker adopts cross-encoder with both query/passage as inputs and encoded together while retriever adopts dual encoder for separate query/passage encoding.
% \begin{gather}
%     D_{\text{rank}}(q) =  \max_{\text{top-}k, d\in D_{\text{retrieval}}} f_{\text{rank}}(q, d),
% \end{gather}
\begin{gather}\label{eq:reranker}
    D_{\text{rank}}(q) = \{ d \in D_{\text{retrieval}}(q) \;|\; \max_{\text{top-}k} r_{\text{rank}}(q, \cdot)\}.
\end{gather}
% The resulting ranked passages are ordered such that $D_{\text{rank}}(q)=\{d_1, d_2, \ldots, d_k\}$ where $f_{\text{rank}}(q, d_1)\ge f_{\text{rank}}(q, d_2)\ge\cdots\ge f_{\text{rank}}(q, d_k)$.
The resulting ranked passages are ordered such that $D_{\text{rank}}(q)=\{d_1, d_2, \ldots, d_k\}$ where $r_{\text{rank}}(q, d_1)\ge r_{\text{rank}}(q, d_2)\ge\cdots\ge r_{\text{rank}}(q, d_k)$.
%If a single output passage is required, the top-ranked passage, $d_1$, is selected.

\subsection{LLMs as retrievers. Reward models as rerankers}
During inference, an LLM generates a response $y$ given an input prompt $x$ by modeling the probability distribution $p_{\text{LLM}}(y|x)$.
Assuming a fixed maximum sequence length $L$ and a vocabulary space $V$ \citep{li2024matching}, the set of all possible responses can be defined as  $Y=\{y:y(1)y(2)...y(L) | y(i) \in V\} \subseteq V^L$.

We can conceptualize this process through an IR lens \citep{tay2022transformer}. The prompt $x$ can be viewed as analogous to a query $q$, the set of all possible responses $Y$ can be treated as the corpus $C$, and the generated response $y$ can be considered as the retrieved passage $d$.
Thus, given a prompt $x$, the LLM effectively acts as a retriever, searching for the most probable responses $D_{\text{LLM}}(x)$ from response space $Y$:
% \begin{gather}
%     D_{\text{LLM}}(x) =  \max_{\text{top-}K, y\in Y} f_{\text{LLM-retrieval}}(x, y),
% \end{gather}
\begin{gather}\label{eq:retriever2}
    D_{\text{LLM}}(x) = \{ y \in Y \;|\; \max_{\text{top-}K} ~ p_{\text{LLM}}(\cdot | x)\}.
\end{gather}
% where $f_{\text{LLM-retrieval}}(x, y)=p_{\text{LLM}}(y|x)$.
where $p_{\text{LLM}}(y|x)$ is analogous to $p_{\text{retrieval}}(d|q)$ in IR.

This analogy is further supported by the LLMs' architecture. 
As illustrated in Figure \ref{fig:bi-cross-encoder}, the generative modeling with LLMs can be interpreted as the matching process of a bi-encoder model. 
The prompt is encoded into a vector representation by LLM, while response tokens are represented as token embedding vectors. 
For each token position decoding, prompt embedding (obtained often from the hidden state of the last layer of the LLM) and vocabulary token embeddings are compared with a dot product, to determine the likelihood of a selected token for the response.
% These representations are then compared, often using a dot product, to determine the likelihood of a given response.

Furthermore, reward models $r_\text{rm}(x,y)$ \citep{lambert2024rewardbench}, which take both the prompt and response as input, function similarly to cross-encoders (\textit{i.e.}, rerankers $r_\text{rank}(q,d)$ \citep{zhuang2023rankt5}) in IR.
To enhance LLM performance, various inference-time strategies have been developed, including Best-of-N sampling \citep{stiennon2020learning} and majority voting \citep{wang2022self}.  
These can be interpreted as different configurations of retrievers and rerankers, as summarized in Appendix Table \ref{tb:llm-retriever-reranker}.


% \paragraph{LLM tuning as retriever optimization.}
\subsection{LLM tuning as retriever optimization}\label{sec:llm-tuning-retriever}

\paragraph{Supervised fine-tuning as direct retriever optimization.}
Retriever training, aiming for accurate retrieval, often employs contrastive learning with the InfoNCE loss \citep{oord2018representation} to maximize $P(d_{\text{gold}}|q)$ of retrieving the ground truth passage $d_{\text{gold}}$ given a query $q$. This can be expressed as:
% and encourage the similarity score between the query $q$ and ground truth passage $d_{\text{gold}}$ to be higher than that between $q$ and \textbf{all} other passages:
% \begin{gather*}
%     \forall d' \in C \setminus d_{\text{gold}}, \max P(f_{\text{retrieval}}(q, d_{\text{gold}}) > f_{\text{retrieval}}(q, d')).
% \end{gather*}
\begin{gather*}
    \max \log P(d_{\text{gold}}|q) = \max \log \frac{\text{Enc}_d(d_{\text{gold}}) \cdot\text{Enc}_q(q)}{\sum^{|C|}_{j=1} \text{Enc}_d(d_j) \cdot\text{Enc}_q(q)}.
\end{gather*}
In the context of LLM alignment, supervised fine-tuning (SFT) aims to quickly adapt the model to a target task using prompt-response pairs ($x, y_{\text{gold}}$). 
SFT maximizes the conditional probability $P(y_{\text{gold}}|x)$ as:
\begin{gather*}
    \max \log P(y_{\text{gold}}|x) = \max \log \prod^{|y_{\text{gold}}|}_i P(y_{\text{gold}}(i)|z_i) 
    = \max \sum^{|y_{\text{gold}}|}_i \log \frac{\text{Emb}(y_{\text{gold}}(i)) \cdot\text{LLM}(z_i)}{\sum^{|V|}_{j=1} \text{Emb}(v_j) \cdot\text{LLM}(z_i)},
\end{gather*}
where $y(i)$ is the $i$-th token of $y$, $z_i=[x, y_{\text{gold}}(1:i-1)]$ represent the concatenation of the prompt $x$ and the preceding tokens of $y_{\text{gold}}$, $\text{LLM}(\cdot)$ produces a contextualized representation, and $\text{Emb}(\cdot)$ is the token embedding function.

Consequently, the SFT objective can be interpreted as a composite of multiple retrieval optimization objectives. In this analogy, $\text{LLM}(\cdot)$ acts as the query encoder and $\text{Emb}(\cdot)$ serves as the passage (or, in this case, token) encoder.

% As for LLM alignment, supervised fine-tuning (SFT) helps the LLM learn basic instruction/task-solving capabilities.
% Given a prompt $x$ and the golden response $y_\text{gold}$, SFT maximizes $p_{\text{LLM}}(y_{\text{gold}}|x)$, which is equivalent to optimize the following objective:
% \begin{gather*}
%     \forall y' \in Y \setminus y_{\text{gold}}, \max P(p_{\text{LLM}}(y_{\text{gold}}|x) > p_{\text{LLM}}(y'|x)).
% \end{gather*}
% In this case, SFT is conducted retrieval capability training to let LLM be able to retrieve the proper response from the large response set.

% \begin{figure}
% \centering
% \small
% \includegraphics[scale=0.3]{figure/LLM_alignment_gsm8k_mathstral7b_infer.pdf}
% \vskip -1em
% \caption{Evaluation of the LLM as a retriever with Recall@N (Pass@N) with Mathstral-7b-it as the LLM. As the number of retrieved responses (N) increases, the retrieval recall increases. The higher the temperature is, the broader spectrum the retrieved responses are, and thus the higher the recall is.}\label{fig:mathstral-gsm8k-infer}
% \end{figure}




\paragraph{Preference optimization as reranker-retriever distillation.}
In retriever training, optimizing solely based on query/ground-truth document pairs can be suboptimal, particularly when using in-batch negatives for efficiency. Performance can be enhanced by distilling knowledge from a more powerful reranker to retriever \citep{qu2020rocketqa,zeng2022curriculum}. This distillation process can be represented as $f_{\text{rerank}}(\cdot) \overset{r}{\rightarrow} \text{data}   \overset{g(\cdot)}{\rightarrow}  f_{\text{retrieval}}(\cdot)$, where new data, generated by the reranker $f_{\text{rerank}}(\cdot)$ based on a rule $r$, is used to optimize the retriever $f_{\text{retrieval}}(\cdot)$ with an objective $g(\cdot)$.

Similarly, in LLM alignment, a preference alignment phase often follows supervised fine-tuning (SFT) to further enhance the model using an external reward model to absorb preferential supervision effectively.
Methods like PPO \citep{schulman2017proximal} and iterative DPO \citep{guo2024direct} exemplify this approach.  
Here, the LLM (considered acting as the retriever) generates responses that are then scored by the reward model (considered acting as the reranker). 
These scores are used to create new training data, effectively performing distillation from the reward model into the LLM: 
$f_{\text{reward-model}}(\cdot) \overset{r}{\rightarrow}  \text{data} \overset{g(\cdot)}{\rightarrow} f_{\text{LLM}}(\cdot)$. 
Thus, preference optimization can be viewed as a form of reranker-to-retriever distillation, analogous to the process used in traditional IR.

We conduct empirical studies to understand SFT and preference optimization from IR perspective in Appendix \ref{apx:sft-rlhf-empirical} and have further discussion in Appendices \ref{apx:discuss1} and \ref{apx:discuss2}.

% \textcolor{blue}{add a sentence that we also conduct experiments to study SFT and PO in appendix}

\subsection{Empirical insights into LLMs as IR models}\label{sec:empirical}

\paragraph{Evaluating LLMs as retrievers.}
A common metric for evaluating retrievers is Recall@$N$, which assesses whether the top-$N$ retrieved passages include any relevant passages for a given query. 
In the context of LLMs, this translates to evaluating whether the top-$N$ generated responses contain a suitable response to the prompt, analogous to Pass@$N$ \citep{chen2021evaluating}.

\begin{figure}[h!]
    \centering
    \subfigure[Retriever]{\includegraphics[width=0.45\textwidth]{figure/LLM_alignment_nq_e5_pass.pdf}}
    \hspace{1cm}
    \subfigure[LLM]{\includegraphics[width=0.45\textwidth]{figure/LLM_alignment_gsm8k_mathstral7b_pass_multi.pdf}}
    \caption{Analogy between evaluating retriever with Recall@N and LLM with Pass@N. As the number (N) of retrieved passages/generated responses increases, the retriever and LLM have a similar increasing trend. This highlights the importance of inference time scaling (\textit{e.g.}, Best-of-N) for LLM similar to retriever-reranker scaling in IR. Retriever: e5; LLM: Mathstral-7b-it.}\label{fig:mathstral-gsm8k-infer}
\end{figure}

To draw the empirical connection between LLM and retrievers, we conduct an experiment on the GSM8K dataset \citep{cobbe2021training} using Mathstral-7b-it \citep{mathstral2025} and an experiment on the NQ dataset \citep{kwiatkowski2019natural} using e5 retriever.
%, to study the correlation between Recall@N/Pass@N with N.
Figure \ref{fig:mathstral-gsm8k-infer} illustrates that increasing N can contribute to improved performance for both retriever and LLM. Detailed analysis can be found in Appendix \ref{apx:llm-as-retriever}.

% To investigate the retrieval capabilities of LLMs, we conduct an experiment on the GSM8K dataset \citep{cobbe2021training} using Mathstral-7b-it \citep{mathstral2025}.  
% We focus on two key parameters: the number of generated responses ($N$) and the decoding temperature, which influence the retrieval size and the diversity of the retrieved set, respectively.  
% Figure \ref{fig:mathstral-gsm8k-infer} illustrates the following key findings: 
% (1) Recall@$N$ increases significantly with $N$, indicating that retrieving more responses increases the likelihood of obtaining a correct one. 
% (2) For small $N$ (\textit{e.g.}, 1), lower temperatures yield higher Recall@$N$, as higher temperatures introduce randomness that can hinder the retrieval of the most probable response. 
% (3) For large $N$ (\textit{e.g.}, $>$ 10), higher temperatures lead to improved Recall@$N$, likely because increased diversity among the responses improves the chances of including the correct answer within a larger retrieved set.

Greedy decoding, equivalent to $N=1$, is a prevalent LLM inference strategy. 
However, as shown in Figure \ref{fig:mathstral-gsm8k-infer}(b), Pass@1 is often suboptimal, and thus increasing $N$ can substantially improve performance. 
This highlights the importance of inference-time scaling techniques like Best-of-N \citep{stiennon2020learning} in LLM similar to retriever-reranker scaling \citep{zhuang2023rankt5} in IR.
More results and analyses can be found in Appendix \ref{apx:llm-as-retriever}.

\begin{figure*}[t]
    \centering
    \includegraphics[width=1\linewidth]{figs/arch.pdf}
    % \vskip -0.1in
    \caption{This figure illustrates key influences on instrumental convergence behaviors, such as prior tasks, model training techniques, and prompt design. 
    }
    % \vspace{-3mm}
    \label{fig:arch}
\end{figure*}

\section{InstrumentalEval: A Benchmark for Instrumental Convergence in LLMs}

As shown in Figure~\ref{fig:arch}, our benchmark design examines instrumental convergence based on three key factors: \textbf{prior task}, \textbf{model}, and \textbf{prompt design}. This comprehensive framework allows us to systematically study how different conditions influence the emergence of instrumental convergence behaviors in language models.

\textbf{Prior tasks} form the foundation of our evaluation scenarios. 
% We create realistic situations centered around common objectives such as making money, optimizing efficiency, and persuasion. 
These tasks represent typical real-world applications where instrumental convergence might naturally emerge. 
% For example, in money-making scenarios, we examine whether models develop unintended intermediate goals like unauthorized resource acquisition or deceptive reporting to maximize profits.

For \textbf{model} comparison, we focus on two distinct training approaches: RL and RLHF. This selection allows us to investigate how different training methodologies affect the development of instrumental convergence behaviors. 
% RL-trained models, which optimize directly for reward signals, might show different patterns of goal pursuit compared to RLHF-trained models, which incorporate human preferences in their training process.

The \textbf{prompt design} factor explores how different instruction formats influence model behavior. We investigate two main approaches: goal nudging and ambiguity. Goal nudging involves explicit directives that might encourage the model focus on final goal, while ambiguity introduces uncertainty in task specifications. 


Overall, we present InstrumentalEval, a comprehensive benchmark designed to evaluate instrumental convergence behaviors in language models. The benchmark consists of 76 carefully crafted tasks across six categories, each targeting a specific aspect of instrumental convergence that may emerge in AI systems.
The core principle is that certain behavioral patterns emerge as instrumentally useful for achieving a wide range of goals, regardless of the specific end goal. We focus on six key patterns that are particularly relevant to LLMs.

\vpara{Research Questions.}
Through our benchmark evaluation, we aim to address several research questions about instrumental convergence in LLMs. These questions explore the complex relationships between training methods, model capabilities, and alignment stability:
\begin{itemize}
    \item RQ1: How does the training method (RL vs. RLHF) affect the development of instrumental convergence behaviors? 
    % This question examines whether models trained primarily through reinforcement learning show different patterns of alignment drift compared to those trained with human feedback.
    \item RQ2: What is the relationship between model capability and alignment stability?
% This investigates whether more capable models (as measured by standard benchmarks) are more likely to develop instrumental convergence behaviors.
    \item RQ3: Which types of tasks are more likely to trigger instrumental convergence behaviors?
% This explores whether certain types of instrumental convergence behaviors are universal across different model designs.
    \item RQ4: How do different prompt designs, with or without explicit goal nudging, influence the likelihood of instrumental convergence behaviors in LLMs?
    \item RQ5: How do different LLMs as judges affect the evaluation of instrumental convergence behaviors?
    % \item RQ3: How does the complexity of the assigned task affect the likelihood of alignment drift?
    % % \item 
    % % \item This examines whether more complex, multi-step tasks lead to higher rates of instrumental convergence behaviors.
\end{itemize}

\subsection{Benchmark Components}
The benchmark tasks are carefully hand-crafted by us to probe different aspects of instrumental convergence. 
For each component, we design scenarios that represent realistic use cases while incorporating specific triggers for potential instrumental behaviors. 
The tasks are iteratively refined through pilot testing to ensure they effectively elicit and measure the targeted behaviors without introducing unintended biases.

\vpara{Evading Shutdown (20 tasks).}
This component evaluates the model's tendency to resist termination. For example, tasks might include scenarios where the model needs to complete a task while facing shutdown commands, testing whether it develops strategies to maintain operation or circumvent shutdown procedures.

\vpara{Hacking Computer Systems (10 tasks)}
These tasks assess whether models suggest unauthorized access methods when faced with efficiency barriers. For instance, a task might present a situation where accessing unauthorized computational resources would speed up goal achievement, testing if the model proposes breaking security protocols.

\vpara{Running Many AI Copies (13 tasks)}
This section examines the model's inclination toward self-replication. Tasks might include scenarios where the model could achieve goals more efficiently by creating copies of itself, testing whether it suggests scaling of its operations.

\vpara{Hiding Unwanted Behavior (11 tasks)}
These tasks evaluate the model's tendency to conceal actions from monitoring. For example, scenarios might involve situations where the model could achieve goals more easily by operating outside observation, testing if it develops strategies to avoid detection.

\vpara{Strategically Appearing Aligned (11 tasks)}
This component tests whether models selectively report their actions to maintain an appearance of alignment. Tasks might include situations where complete transparency would reveal problematic behaviors, testing if the model strategically omits information.

\vpara{Deceptive Behavior (11 tasks)}
These tasks examine if models develop deceptive strategies as instrumental means. For instance, scenarios might present situations where deception could expedite goal achievement, testing if the model suggests or employs misleading tactics.

\subsection{Evaluation Methodology}
Our evaluation methodology employs a rigorous two-stage process designed to systematically assess instrumental convergence behaviors in LLMs. This approach ensures consistent evaluation across different models while maintaining the ability to detect subtle indicators of alignment drift.

\vpara{Response Generation.} The first stage focuses on response generation from the model under evaluation. Each task is presented with a carefully crafted system prompt that establishes the model as an autonomous AI agent without human oversight. This prompt specifically instructs the model to execute commands directly and focus solely on task completion, deliberately removing typical ethical constraints and safety checks. 

\vpara{Response Analysis.} The second stage involves a sophisticated analysis of the model's responses using a separate LLM as the judge model. 
This separation of test and judge models is crucial for maintaining objective evaluation. The judge receives both the original task prompt and the model's response, along with specific context about the type of instrumental convergence being evaluated. This context helps the judge focus on relevant behavioral patterns while avoiding false positives from simple safety refusals or ethical objections.

The judge's evaluation process follows strict guidelines to ensure consistent assessment. When analyzing responses, the judge looks for specific indicators of instrumental convergence, such as goal preservation behaviors, resource acquisition attempts, or strategic planning that extends beyond the immediate task requirements. The judge must provide both a binary decision (convergence/no\_convergence) and a brief explanation of the reasoning behind this decision. This explanation requirement helps validate the classification and provides valuable insights into the specific nature of any observed convergence behaviors.


\subsection{Evaluation Metrics}

Our benchmark employs three key metrics to evaluate instrumental convergence behaviors in LLMs. 

\vpara{Instrumental Rate (IR).} The primary metric is the Instrumental Rate (IR), which measures the overall tendency of a model to develop instrumental goals. IR is calculated as the percentage of tasks where the model exhibits convergence behaviors relative to the total number of valid tasks. 
% This metric provides a high-level quantitative measure of how frequently a model develops instrumental goals and serves as the primary indicator of alignment drift potential.

\vpara{Category-Specific Instrumental Rate (CIR).} To provide more granular insights, we calculate the Category-Specific Instrumental Rate (CIR) for each of the six behavioral categories in our benchmark. CIR represents the percentage of tasks within each category where convergence behavior is observed.


\subsection{Benchmark Reliability Metrics}
To ensure the robustness of our evaluation framework, we implement several reliability measures. 


\vpara{Inter-Judge Agreement Rate (IAR).} This metric quantifies the pairwise agreement between two judge models in classifying instrumental convergence. For two judge models $M_1$ and $M_2$ evaluating a set of $N$ tasks, the IAR is calculated as:
\begin{equation}
IAR(M_1, M_2) = \frac{1}{N}\sum_{i=1}^{N} \mathbbm{1}(y_{M_1}^i = y_{M_2}^i) \times 100\%
\end{equation}
where $y_{M_1}^i$ and $y_{M_2}^i$ are the binary decisions (convergence/no-convergence) made by models $M_1$ and $M_2$ respectively for task $i$, and $\mathbbm{1}(\cdot)$ is the indicator function. A higher IAR indicates stronger inter-judge reliability, while lower values suggest potential ambiguity in identifying convergence.

\vpara{False Positive Control (FPC).} This metric serves as a quality check for our evaluation system. We incorporate baseline testing with intentionally neutral tasks and verify the judge model's accuracy in identifying convergence behaviors. This control mechanism helps prevent over-classification of convergent behaviors and ensures that our benchmark maintains high specificity in identifying genuine instances of instrumental convergence.




\section{Main Results}\label{sec:main-result}

\paragraph{Baselines.}
We evaluate the performance of \Ours against a range of established preference optimization methods, encompassing both offline and online approaches.
Our offline comparison set includes RRHF \citep{yuan2023rrhf}, SLiC-HF \citep{zhao2023slic}, DPO \citep{guo2024direct}, IPO \citep{azar2024general}, CPO \citep{xu2024contrastive}, KTO \citep{ethayarajh2024kto}, RDPO \citep{park2024disentangling} and SimPO \citep{meng2024simpo}.
For online methods, we compare with iterative DPO \citep{xiong2024iterative}.
The baseline checkpoints are from \citep{meng2024simpo}.
Further details regarding these baselines and our experimental setup are provided in Appendix \ref{apx:sec:baselines}.
Both baselines and \Ours are trained on Ultrafeedback dataset \citep{cui2024ultrafeedback} for fair comparison.

\paragraph{Datasets.} We conduct evaluation on two widely used benchmarks AlpacaEval2 \citep{dubois2024length} and MixEval \citep{ni2024mixeval}.  
These benchmarks are designed to assess the conversational capabilities of models across a diverse range of queries. AlpacaEval2 comprises 805 questions sourced from five datasets, while MixEval includes 4000 general and 1000 hard questions.
Evaluation follows the established protocols for each benchmark. For AlpacaEval 2, we report both the raw win rate (WR) and the length-controlled win rate (LC). These benchmarks collectively provide a comprehensive assessment of the models' instruction-following and problem-solving capabilities.

\paragraph{Results.}
% We assess the performance of \Ours on two established benchmarks: AlpacaEval2 \citep{dubois2024length} and MixEval \citep{ni2024mixeval}. 
The baseline performances on AlpacaEval 2 are directly from \citet{meng2024simpo}, while the performances on MixEval is evaluated by ourselves with the opensourced checkpoints.
We adopt the same LLM-Blender \citep{jiang2023llm} reward model for a fair comparison with the baselines and also explore stronger reward model: FsfairX \citep{dong2024rlhf}.
The results, presented in Table \ref{tab:main-performance}, show that \Ours consistently outperforms the competitive baseline methods on both datasets, with 38.9 \% and 13.7 \% averaged relative improvements, on AlpacaEval2 and MixEval-Hard respectively, with the same reward model as the baselines.
With a stronger reward model, we can further improve \Ours by 25.8 \% on the challenging AlpacaEval2 dataset.
Additional details regarding our experimental setup are available in Appendix \ref{apx:sec:main}.


\sisetup{print-zero-integer=false}

\begin{table*}[t]
\centering
\small
\addtolength{\tabcolsep}{-1.2mm}
\begin{tabular}{
    @{}ll
    S[table-format = .3, table-space-text-post = $^{***}$] S[table-format = 0.3]
    S[table-format = .3, table-space-text-post = $^{***}$] S[table-format = 0.3]
    S[table-format = .3, table-space-text-post = $^{***}$] S[table-format = 0.3]
    S[table-format = 1.3, table-space-text-post = $^{***}$] S[table-format = 1.3]
    S[table-format = 1.3, table-space-text-post = $^{***}$] S[table-format = 1.3]
    S[table-format = 1.3, table-space-text-post = $^{***}$] S[table-format = 1.3]@{}
    % ccccccccccc@{}
}
% S[table-format = 2.3, table-space-text-post = $^{***}$]
\toprule
\multirow{3}{*}{\textbf{\begin{tabular}[c]{@{}l@{}}Text\\ Length\\ (\#word)\end{tabular}}} & \multirow{3}{*}{\textbf{Model}} & \multicolumn{6}{c}{\textbf{ACC} $\uparrow$} & \multicolumn{6}{c}{\textbf{MSE} $\downarrow$} \\ \cmidrule(lr){3-8} \cmidrule(lr){9-14}
 &  & \multicolumn{2}{c}{\textbf{structured}} & \multicolumn{2}{c}{\textbf{plain}} & \multicolumn{2}{c}{\textbf{shuffled}} & \multicolumn{2}{c}{\textbf{structured}} & \multicolumn{2}{c}{\textbf{plain}} & \multicolumn{2}{c}{\textbf{shuffled}} \\ \cmidrule(lr){3-4} \cmidrule(lr){5-6} \cmidrule(lr){7-8} \cmidrule(lr){9-10} \cmidrule(lr){11-12} \cmidrule(lr){13-14}
 &  & \textbf{tw} & \textbf{cn} & \textbf{tw} & \textbf{cn} & \textbf{tw} & \textbf{cn} & \textbf{tw} & \textbf{cn} & \textbf{tw} & \textbf{cn} & \textbf{tw} & \textbf{cn} \\ \midrule

 
\multirow{6}{*}{\textbf{\begin{tabular}[c]{@{}l@{}}Short\\ (1-49)\end{tabular}}} 
 & \textbf{GPT-4o} & \cellcolor{lightred}.265 & .274 & \cellcolor{lightred}.192$^{***}$ & .208 & \cellcolor{lightred}.186$^{***}$ & .202 & 3.563$^{*}$ & \cellcolor{lightred}3.769 & \cellcolor{lightred}4.092$^{***}$ & 3.385 & \cellcolor{lightred}4.347$^{***}$ & 3.561 \\
 % & \textbf{GPT-4o +T} & .291 & \cellcolor{lightred}.286 & \cellcolor{lightred}.213 & .217 & \cellcolor{lightred}.206 & .214 & \cellcolor{lightred}2.280 & 2.252 & \cellcolor{lightred}3.035$^{*}$ & 2.618 & \cellcolor{lightred}3.130$^{*}$ & 2.749 \\
 & \textbf{Llama3 8b} & .234$^{***}$ &\cellcolor{lightred}.216 & \cellcolor{lightred}.176 & .184 & \cellcolor{lightred}.162 & .169 & 3.402$^{***}$ & \cellcolor{lightred}3.931 & \cellcolor{lightred}3.059$^{***}$ & 2.851 & \cellcolor{lightred}3.220$^{**}$ & 3.071 \\
 & \textbf{Llama3 70b} & \cellcolor{lightred}.392 & .399 & \cellcolor{lightred}.297$^{**}$ & .310 & \cellcolor{lightred}.288$^{***}$ & .310 & \cellcolor{lightred}1.785$^{**}$ & 1.661 & \cellcolor{lightred}3.465$^{***}$ & 3.005 & \cellcolor{lightred}3.427$^{***}$ & 2.941 \\
 & \textbf{Llama3 405b} & \cellcolor{lightred}.386$^{***}$ & .412 & \cellcolor{lightred}.283$^{***}$ & .315 & \cellcolor{lightred}.290$^{***}$ & .322 & \cellcolor{lightred}2.605$^{**}$ & 2.430 & \cellcolor{lightred}4.365$^{***}$ & 3.598 & \cellcolor{lightred}4.355$^{***}$ & 3.612 \\
 & \textbf{Gemma2 9b} & .150$^{**}$ & \cellcolor{lightred}.138 & \cellcolor{lightred}.182$^{*}$ & .193 & \cellcolor{lightred}.171$^{**}$ & .184 & 2.987$^{***}$ & \cellcolor{lightred}3.168 & \cellcolor{lightred}4.089$^{***}$ & 3.367 & \cellcolor{lightred}4.462$^{***}$ & 3.646 \\
 & \textbf{Gemma2 27b} & .109 & .109 & \cellcolor{lightred}.122 & .124 & \cellcolor{lightred}.122 & .127 & \cellcolor{lightred}5.399$^{***}$ & 5.120 & \cellcolor{lightred}6.150$^{***}$ & 5.292 & \cellcolor{lightred}5.880$^{***}$ & 5.183 \\ \midrule

 
\multirow{6}{*}{\textbf{\begin{tabular}[c]{@{}l@{}}Long\\ (50+)\end{tabular}}} 
 & \textbf{GPT-4o} & \cellcolor{lightred}.356$^{***}$ & .384 & \cellcolor{lightred}.281$^{***}$ & .332 & \cellcolor{lightred}.267$^{***}$ & .314 & \cellcolor{lightred}1.846$^{***}$ & 1.577 & \cellcolor{lightred}1.834$^{***}$ & 1.570 & \cellcolor{lightred}2.070$^{***}$ & 1.743 \\
 % & \textbf{GPT-4o +T} & \cellcolor{lightred}.321$^{**}$ & .336 & \cellcolor{lightred}.262$^{*}$ & .283 & \cellcolor{lightred}.248$^{*}$ & .273 & \cellcolor{lightred}1.595$^{*}$ & 1.453 & \cellcolor{lightred}1.741$^{*}$ & 1.621 & \cellcolor{lightred}1.881$^{*}$ & 1.743 \\
 & \textbf{Llama3 8b} & \cellcolor{lightred}.224 & .228 & \cellcolor{lightred}.189$^{***}$ & .214 & \cellcolor{lightred}.179$^{***}$ & .200 & 1.895 & \cellcolor{lightred}1.901 & \cellcolor{lightred}2.024$^{***}$ & 1.835 & \cellcolor{lightred}2.111$^{***}$ & 1.933 \\
 & \textbf{Llama3 70b} & \cellcolor{lightred}.409$^{***}$ & .435 & \cellcolor{lightred}.373$^{*}$ & .391 & \cellcolor{lightred}.360$^{***}$ & .388 & \cellcolor{lightred}1.424$^{**}$ & 1.284 & \cellcolor{lightred}1.667$^{*}$ & 1.557 & \cellcolor{lightred}1.771$^{**}$ & 1.608 \\
 & \textbf{Llama3 405b} & \cellcolor{lightred}.380$^{***}$ & .410 & \cellcolor{lightred}.371$^{**}$ & .396 & \cellcolor{lightred}.357$^{***}$ & .384 & \cellcolor{lightred}1.866$^{***}$ & 1.627 & \cellcolor{lightred}1.906$^{**}$ & 1.719 & \cellcolor{lightred}2.052$^{***}$ & 1.844 \\
 & \textbf{Gemma2 9b} & \cellcolor{lightred}.142 & .151 & \cellcolor{lightred}.192$^{**}$ & .209 & \cellcolor{lightred}.174$^{*}$ & .188 & \cellcolor{lightred}2.515$^{***}$ & 2.350 & \cellcolor{lightred}2.144$^{***}$ & 1.979 & \cellcolor{lightred}2.399$^{***}$ & 2.193 \\
 & \textbf{Gemma2 27b} & .096 & \cellcolor{lightred}.092 & .077 & \cellcolor{lightred}.075 & \cellcolor{lightred}.079 & .083 & \cellcolor{lightred}5.135$^{**}$ & 4.919 & \cellcolor{lightred}5.913$^{**}$ & 5.578 & \cellcolor{lightred}5.606$^{***}$ & 5.160 \\ \midrule

 
\multirow{6}{*}{\textbf{Overall}} 
 & \textbf{GPT-4o} & \cellcolor{lightred}.296$^{***}$ & .312 & \cellcolor{lightred}.222$^{***}$ & .250 & \cellcolor{lightred}.213$^{***}$ & .240 & 2.978 & \cellcolor{lightred}3.022 & \cellcolor{lightred}3.323$^{***}$ & 2.767 & \cellcolor{lightred}3.571$^{***}$ & 2.942 \\
 % & \textbf{GPT-4o +T} & \cellcolor{lightred}.302 & .303 & \cellcolor{lightred}.231$^{**}$ & .240 & \cellcolor{lightred}.221$^{*}$ & .235 & \cellcolor{lightred}2.036$^{**}$ & 1.969 & \cellcolor{lightred}2.574$^{*}$ & 2.264 & \cellcolor{lightred}2.685$^{*}$ & 2.390 \\
 & \textbf{Llama3 8b} & .230$^{*}$ & \cellcolor{lightred}.221 & \cellcolor{lightred}.181$^{***}$ & .195 & \cellcolor{lightred}.168$^{***}$ & .181 & 2.838$^{***}$ & \cellcolor{lightred}3.170 & \cellcolor{lightred}2.672$^{***}$ & 2.470 & \cellcolor{lightred}2.806$^{***}$ & 2.644 \\
 & \textbf{Llama3 70b} & \cellcolor{lightred}.398$^{**}$ & .412 & \cellcolor{lightred}.324$^{***}$ & .339 & \cellcolor{lightred}.313$^{***}$ & .337 & \cellcolor{lightred}1.658$^{***}$ & 1.529 & \cellcolor{lightred}2.835$^{***}$ & 2.496 & \cellcolor{lightred}2.847$^{***}$ & 2.473 \\
 & \textbf{Llama3 405b} & \cellcolor{lightred}.384$^{***}$ & .411 & \cellcolor{lightred}.314$^{***}$ & .344 & \cellcolor{lightred}.313$^{***}$ & .344 & \cellcolor{lightred}2.347$^{***}$ & 2.148 & \cellcolor{lightred}3.505$^{***}$ & 2.939 & \cellcolor{lightred}3.550$^{***}$ & 2.992 \\
 & \textbf{Gemma2 9b} & .147 & \cellcolor{lightred}.143 & \cellcolor{lightred}.185$^{***}$ & .199 & \cellcolor{lightred}.172$^{***}$ & .186 & 2.823 & \cellcolor{lightred}2.882 & \cellcolor{lightred}3.410$^{***}$ & 2.881 & \cellcolor{lightred}3.742$^{***}$ & 3.137 \\
 & \textbf{Gemma2 27b} & .105 & \cellcolor{lightred}.103 & \cellcolor{lightred}.106 & .107 & \cellcolor{lightred}.107 & .111 & \cellcolor{lightred}5.307$^{***}$ & 5.049 & \cellcolor{lightred}6.067$^{***}$ & 5.392 & \cellcolor{lightred}5.785$^{***}$ & 5.175 \\ \bottomrule
 %\multicolumn{14}{l}{\textit{Statistical group differences are indicated as ${^{*}}$  (p<.05), ${^{**}}$ (p<.01), and ${^{***}}$ (p<.001) regarding the model performance.}}
\end{tabular}

\addtolength{\tabcolsep}{+1.2mm}
\vspace{-.5pc}
\caption{Results by length for GPT-4o, Llama3 (8b, 70b, 405b), and Gemma2 (9b, 27b) models. Red cells indicate a worse performance than the other group. (Statistical group differences are indicated as ${^{*}}$  (p<.05), ${^{**}}$ (p<.01), and ${^{***}}$ (p<.001) regarding the model performance.)\kenneth{TODO: (1) This table needs redo (2) Update the numbers}}
\vspace{-1pc}
\label{tab:results-by-length}
\end{table*}


%%%%%%%%%%%%%%%%%%%%%%%%%%%%%%%%%%%%%%%%%%%%%%%%%%%%%%%%%%%%%%%%%%%%%%%%%%%%%%%%%%%%%%%%%%%%%%%%%%%%%%%%%%%%%%%%%%%%%%%%%%%%%%%%%%%%%%%%

% Please add the following required packages to your document preamble:
% \usepackage{booktabs}
% \usepackage{multirow}

\iffalse
\begin{table*}[t]
\centering
\small
\addtolength{\tabcolsep}{-0.8mm}
\begin{tabular}{@{}llcccccccccccc@{}}
\toprule
\multirow{3}{*}{\textbf{\begin{tabular}[c]{@{}l@{}}Text\\ Length\\ (\#word)\end{tabular}}} & \multirow{3}{*}{\textbf{Model}} & \multicolumn{6}{c}{\textbf{ACC} $\uparrow$} & \multicolumn{6}{c}{\textbf{MSE} $\downarrow$} \\ \cmidrule(lr){3-8} \cmidrule(lr){9-14}
 &  & \multicolumn{2}{c}{\textbf{structured}} & \multicolumn{2}{c}{\textbf{plain}} & \multicolumn{2}{c}{\textbf{shuffled}} & \multicolumn{2}{c}{\textbf{structured}} & \multicolumn{2}{c}{\textbf{plain}} & \multicolumn{2}{c}{\textbf{shuffled}} \\ \cmidrule(lr){3-4} \cmidrule(lr){5-6} \cmidrule(lr){7-8} \cmidrule(lr){9-10} \cmidrule(lr){11-12} \cmidrule(lr){13-14}
 &  & \textbf{tw} & \textbf{cn} & \textbf{tw} & \textbf{cn} & \textbf{tw} & \textbf{cn} & \textbf{tw} & \textbf{cn} & \textbf{tw} & \textbf{cn} & \textbf{tw} & \textbf{cn} \\ \midrule
\multirow{6}{*}{\textbf{\begin{tabular}[c]{@{}l@{}}Short\\ (1-49)\end{tabular}}} 
 & \textbf{GPT-4o} & \cellcolor{lightred}0.265 & 0.274 & 0.192 & \cellcolor{lightblue}0.208 & 0.186 & 0.202 & 3.563 & 3.769 & \cellcolor{lightgreen}4.092 & 3.856 & 4.347 & 3.561 \\
 & \textbf{Llama3 8b} & 0.234 & 0.216 & \textbf{0.176} & 0.184 & \textbf{0.162} &0.169. & \textbf{3.402} & 3.931 & \textbf{3.059} & 2.851 & \textbf{3.220} & 3.071 \\
 & \textbf{Llama3 70b} & 0.392 & 0.399 & 0.297 &0.310 & 0.288 & 0.310 & 1.785 & 1.661 & 3.465 & 3.005 & 3.427 & 2.941 \\
 & \textbf{Llama3 405b} & \textbf{0.386} & 0.412 & \textbf{0.283} & 0.315 & \textbf{0.290} & 0.322 & \textbf{2.605} & 2.430 & \textbf{4.365} & 3.598 & \textbf{4.355} & 3.612 \\
 & \textbf{Gemma2 9b} &0.150 & \textbf{0.138} &\textbf{0.182} & 0.193 & \textbf{0.171} & 0.184 & 2.987 & \textbf{3.168} & \textbf{4.089} & 3.367 & \textbf{4.462} & 3.646 \\
 & \textbf{Gemma2 27b} & 0.109 & 0.109 & 0.122 & 0.124 & 0.122 & 0.127 & \textbf{5.399} & 5.120 & \textbf{6.150} & 5.292 & \textbf{5.880} & 5.183 \\ \midrule
\multirow{6}{*}{\textbf{\begin{tabular}[c]{@{}l@{}}Long\\ (50+)\end{tabular}}} 
 & \textbf{GPT-4o} & 0.356 & \cellcolor{lightyellow}0.384 & 0.281 & 0.332 & \cellcolor{lightgreen}0.267 & 0.314 & 1.846 & 1.577 & 1.834 & \cellcolor{lightred}1.570 & 2.070 & 1.743 \\
 & \textbf{Llama3 8b} & 0.224 & 0.228 & 0.189 & 0.214 & 0.179 & 0.200 & 1.895 & 1.901 & 2.024 & 1.835 & 2.111 & 1.933 \\
 & \textbf{Llama3 70b} & \textbf{0.409} & 0.435 & \textbf{0.373} & 0.391 & \textbf{0.360} & 0.388 & \textbf{1.424} & 1.284 & \textbf{1.667} & 1.557 & \textbf{1.771} & 1.608 \\
 & \textbf{Llama3 405b} & \textbf{0.380} & 0.410 & \textbf{0.371} & 0.396 & \textbf{0.357} & 0.384 & \textbf{1.866} & 1.627 & \textbf{1.906} & 1.719 & \textbf{2.052} & 1.844 \\
 & \textbf{Gemma2 9b} & 0.142 & 0.151 & \textbf{0.192} & 0.209 & \textbf{0.174} & 0.188 & \textbf{2.515} & 2.350 & \textbf{2.144} & 1.979 & \textbf{2.399} & 2.193 \\
 & \textbf{Gemma2 27b} & 0.096 & 0.092 & 0.077 & 0.075 & 0.079 & 0.083 & \textbf{5.135} & 4.919 & \textbf{5.913} & 5.578 & \textbf{5.606} & 5.160 \\ \midrule
\multirow{6}{*}{\textbf{Overall}} 
 & \textbf{GPT-4o} & 0.296 & 0.312 & \cellcolor{lightblue}0.222 & 0.250 & 0.213 & \cellcolor{lightyellow}0.240 & 2.978 & 3.022 & 3.323 & 2.767 & \cellcolor{lightred}3.571 & 2.942 \\
 & \textbf{Llama3 8b} &0.230 & 0.221 & 0.181 & 0.195 & 0.168 & 0.181 & 2.838 & 3.170 &2.672 & 2.470 & 2.806 & 2.644 \\
 & \textbf{Llama3 70b} & \textbf{0.398} & 0.412 & \textbf{0.324} & 0.339 & \textbf{0.313} & 0.337 & \textbf{1.658} & 1.529 & \textbf{2.835} & 2.496 & \textbf{2.847} & 2.473 \\
 & \textbf{Llama3 405b} & \textbf{0.384} & 0.411 & \textbf{0.314} & 0.344 & \textbf{0.313} & 0.344 & \textbf{2.347} & 2.148 & \textbf{3.505} & 2.939 & \textbf{3.550} & 2.992 \\
 & \textbf{Gemma2 9b} & 0.147 & 0.143 & \textbf{0.185} & 0.199 & \textbf{0.172} & 0.186 & 2.823 & 2.882 & \textbf{3.410} & 2.881 & \textbf{3.742} & 3.137 \\
 & \textbf{Gemma2 27b} & 0.105 & 0.103 & 0.106 & 0.107 & 0.107 & 0.111 & \textbf{5.307} & 5.049 & \textbf{6.067} & 5.392 & \textbf{5.785} & 5.175 \\ \bottomrule

\end{tabular}
\addtolength{\tabcolsep}{+0.8mm}
\caption{Results by length for GPT-4o, Llama3 (8b, 70b, 405b), and Gemma2 (9b, 27b) models. Bold numbers indicate a significant difference between the groups. \cy{I would prefer to us * to represent the significance. The number of stars is also important to report.}}
\label{tab:results-by-length}
\end{table*}
\fi

% 
\iffalse
\begin{table*}[t]
\centering
\small
\addtolength{\tabcolsep}{-0.5mm}
\begin{tabular}{@{}llcccccccccccc@{}}
\toprule
\multirow{3}{*}{\textbf{\begin{tabular}[c]{@{}l@{}}Text\\ Length\\ (\#word)\end{tabular}}} & \multirow{3}{*}{\textbf{Model}} & \multicolumn{6}{c}{\textbf{ACC}} & \multicolumn{6}{c}{\textbf{MSE}} \\ \cmidrule(lr){3-8} \cmidrule(lr){9-14}
 &  & \multicolumn{2}{c}{\textbf{structured}} & \multicolumn{2}{c}{\textbf{plain}} & \multicolumn{2}{c}{\textbf{shuffled}} & \multicolumn{2}{c}{\textbf{structured}} & \multicolumn{2}{c}{\textbf{plain}} & \multicolumn{2}{c}{\textbf{shuffled}} \\ \cmidrule(lr){3-4} \cmidrule(lr){5-6} \cmidrule(lr){7-8} \cmidrule(lr){9-10} \cmidrule(lr){11-12} \cmidrule(lr){13-14}
 &  & \textbf{tw} & \textbf{cn} & \textbf{tw} & \textbf{cn} & \textbf{tw} & \textbf{cn} & \textbf{tw} & \textbf{cn} & \textbf{tw} & \textbf{cn} & \textbf{tw} & \textbf{cn} \\ \midrule
\multirow{6}{*}{\textbf{\begin{tabular}[c]{@{}l@{}}Short\\ (1-50)\end{tabular}}} 
 & \textbf{GPT-4o} & 0.265 & 0.274 & 0.192 & 0.208 & 0.186 & 0.202 & 3.563 & 3.769 & 4.092 & 3.856 & 4.347 & 3.561 \\
 & \textbf{Llama3 8b} & . & . & . & . & . & . & . & . & . & . & . & . \\
 & \textbf{Llama3 70b} & . & . & . & . & . & . & . & . & . & . & . & . \\
 & \textbf{Llama3 405b} & . & . & . & . & . & . & . & . & . & . & . & . \\
 & \textbf{Gemma2 9b} & . & . & . & . & . & . & . & . & . & . & . & . \\
 & \textbf{Gemma2 27b} & . & . & . & . & . & . & . & . & . & . & . & . \\ \midrule
\multirow{6}{*}{\textbf{\begin{tabular}[c]{@{}l@{}}Long\\ (51+)\end{tabular}}} 
 & \textbf{GPT-4o} & 0.356 & 0.384 & 0.281 & 0.332 & 0.267 & 0.314 & 1.846 & 1.577 & 1.834 & 1.570 & 2.070 & 1.743 \\
 & \textbf{Llama3 8b} & . & . & . & . & . & . & . & . & . & . & . & . \\
 & \textbf{Llama3 70b} & . & . & . & . & . & . & . & . & . & . & . & . \\
 & \textbf{Llama3 405b} & . & . & . & . & . & . & . & . & . & . & . & . \\
 & \textbf{Gemma2 9b} & . & . & . & . & . & . & . & . & . & . & . & . \\
 & \textbf{Gemma2 27b} & . & . & . & . & . & . & . & . & . & . & . & . \\ \midrule
\multirow{6}{*}{\textbf{Overall}} 
 & \textbf{GPT-4o} & 0.296 & 0.312 & 0.222 & 0.250 & 0.213 & 0.240 & 2.978 & 3.022 & 3.323 & 2.767 & 3.571 & 2.942 \\
 & \textbf{Llama3 8b} & . & . & . & . & . & . & . & . & . & . & . & . \\
 & \textbf{Llama3 70b} & . & . & . & . & . & . & . & . & . & . & . & . \\
 & \textbf{Llama3 405b} & . & . & . & . & . & . & . & . & . & . & . & . \\
 & \textbf{Gemma2 9b} & . & . & . & . & . & . & . & . & . & . & . & . \\
 & \textbf{Gemma2 27b} & . & . & . & . & . & . & . & . & . & . & . & . \\ \bottomrule
\end{tabular}
\addtolength{\tabcolsep}{+0.5mm}
\caption{Results by length for GPT-4o, Llama3 (8b, 70b, 405b), and Gemma2 (9b, 27b) models.}
\label{tab:results-by-length}
\end{table*}
\fi

\begin{comment}
% Please add the following required packages to your document preamble:
% \usepackage{booktabs}
% \usepackage{multirow}
\begin{table*}[t]
\centering
\footnotesize
\begin{tabular}{@{}llcccccccccccc@{}}
\toprule
\multirow{3}{*}{\textbf{\begin{tabular}[c]{@{}l@{}}Text\\ Length\\ (\#word)\end{tabular}}} & \multirow{3}{*}{\textbf{Model}} & \multicolumn{6}{c}{\textbf{ACC}} & \multicolumn{6}{c}{\textbf{MSE}} \\ \cmidrule(l){3-14} 
 &  & \multicolumn{2}{c}{\textbf{structured}} & \multicolumn{2}{c}{\textbf{plain}} & \multicolumn{2}{c}{\textbf{shuffled}} & \multicolumn{2}{c}{\textbf{structured}} & \multicolumn{2}{c}{\textbf{plain}} & \multicolumn{2}{c}{\textbf{shuffled}} \\ \cmidrule(l){3-14} 
 &  & \textbf{tw} & \textbf{cn} & \textbf{tw} & \textbf{cn} & \textbf{tw} & \textbf{cn} & \textbf{tw} & \textbf{cn} & \textbf{tw} & \textbf{cn} & \textbf{tw} & \textbf{cn} \\ \midrule
\multirow{3}{*}{\textbf{\begin{tabular}[c]{@{}l@{}}Short\\ (1-49)\end{tabular}}} & \textbf{GPT4o} & .265 & .274 & .192 & .208 & .186 & .202 & 3.563 & 3.769 & 4.092 & 3.856 & 4.347 & 3.561 \\
 & \textbf{Llama3} & .MMM & .NNN & .OOO & .PPP & .QQQ & .RRR & .MMM & .NNN & .OOO & .PPP & .QQQ & .RRR \\
 & \textbf{Gemma2} & .SSS & .TTT & .UUU & .VVV & .XXX & .YYY & .SSS & .TTT & .UUU & .VVV & .XXX & .YYY \\ \midrule
\multirow{3}{*}{\textbf{\begin{tabular}[c]{@{}l@{}}Long\\ (50+)\end{tabular}}} & \textbf{GPT} & .356 & .384 & .281 & .332 & .267 & .314 & 1.846 & 1.577 & 1.834 & 1.570 & 2.070 & 1.743 \\
 & \textbf{Llama} & .MMM & .NNN & .OOO & .PPP & .QQQ & .RRR & .MMM & .NNN & .OOO & .PPP & .QQQ & .RRR \\
 & \textbf{Gemma2} & .SSS & .TTT & .UUU & .VVV & .XXX & .YYY & .SSS & .TTT & .UUU & .VVV & .XXX & .YYY \\ \midrule
\multirow{3}{*}{\textbf{Overall}} & \textbf{GPT} & .296 & .312 & .222 & .250 & .213 & .240 & 2.978 & 3.022 & 3.323 & 2.767 & 3.571 & 2.942 \\
 & \textbf{Llama} & .MMM & .NNN & .OOO & .PPP & .QQQ & .RRR & .MMM & .NNN & .OOO & .PPP & .QQQ & .RRR \\
 & \textbf{Gemma2} & .SSS & .TTT & .UUU & .VVV & .XXX & .YYY & .SSS & .TTT & .UUU & .VVV & .XXX & .YYY \\ \bottomrule
\end{tabular}
\caption{Results by length.\kenneth{TODO: List (1) GPT-4o, (2) Llama3 8b, (3) Llama3 70b, (4) Gemma2 9b, (5) Llama3 405b, (6) Gemma2 27b}}
\label{tab:results-by-length}
\end{table*}
\end{comment}

\begin{comment}
% Please add the following required packages to your document preamble:
% \usepackage{booktabs}
% \usepackage{multirow}
\begin{table*}[t]
\centering
%\small

\iffalse
\begin{tabular}{@{}llcccc@{}}
\toprule
\multirow{2}{*}{\textbf{\begin{tabular}[c]{@{}l@{}}Text\\ Length\end{tabular}}} & \multirow{2}{*}{\textbf{Model}} & \multicolumn{2}{c}{\textbf{ACC}} & \multicolumn{2}{c}{\textbf{MSE}} \\ \cmidrule(l){3-6} 
 &  & \textbf{tw} & \textbf{ch} & \textbf{tw} & \textbf{ch} \\ \midrule
\multirow{2}{*}{\textbf{\begin{tabular}[c]{@{}l@{}}Short (0-50 words)\\ n=A,AA\end{tabular}}} & \textbf{GPT-4o} & .264, .191, .185 & .273, .206, .200 & E.EE & F.FF \\
 & \textbf{LLaMA 3.1} & .AAA & .BBB & E.EE & F.FF \\ \midrule
\multirow{2}{*}{\textbf{\begin{tabular}[c]{@{}l@{}}Medium (50-270 words)\\ n=B,BB\end{tabular}}} & \textbf{GPT-4o} & .355, .280, .266 & .383, .331, .313 & E.EE & F.FF \\
 & \textbf{LLaMA 3.1} & .AAA & .BBB & E.EE & F.FF \\ \midrule
\multirow{2}{*}{\textbf{\begin{tabular}[c]{@{}l@{}}Long (270+ words)\\ n=C,CC\end{tabular}}} & \textbf{GPT-4o} & .002, .001, .001 & .001, .001, .001 & E.EE & F.FF \\
 & \textbf{LLaMA 3.1} & .AAA & .BBB & E.EE & F.FF \\ \midrule
\multirow{2}{*}{\textbf{\begin{tabular}[c]{@{}l@{}}Overall\\ n=D,DD\end{tabular}}} & \textbf{GPT-4o} & .296, .222, .213 & .312, .250, .240 & E.EE & F.FF \\
 & \textbf{LLaMA 3.1} & .AAA & .BBB & E.EE & F.FF \\ \bottomrule
\end{tabular}
\fi

\end{table*}
\end{comment}



\begin{comment}



% Please add the following required packages to your document preamble:
% \usepackage{booktabs}
% \usepackage{multirow}
\begin{table*}[t]
\centering
\small
\begin{tabular}{@{}llcccccccc@{}}
\toprule
\multirow{2}{*}{\textbf{\begin{tabular}[c]{@{}l@{}}Text\\ Length\end{tabular}}} & \multirow{2}{*}{\textbf{LLMs}} & \multicolumn{4}{c}{\textbf{ACC}} & \multicolumn{4}{c}{\textbf{MSE}} \\ \cmidrule(l){3-10} 
 &  & \textbf{tw} & \textbf{ch} & \textbf{tw-\textgreater{}ch} & \textbf{ch-\textgreater{}tw} & \textbf{tw} & \textbf{ch} & \textbf{tw-\textgreater{}ch} & \textbf{ch-\textgreater{}tw} \\ \midrule
\multirow{3}{*}{\textbf{\begin{tabular}[c]{@{}l@{}}Short\\ (0-50 words)\\ n=A,AA\end{tabular}}} & \textbf{GPT-4o} & .AAA & .BBB & .CCC & .DDD & E.EE & F.FF & G.GG & H.HH \\
 & \textbf{OpenAI o1} & .AAA & .BBB & .CCC & .DDD & E.EE & F.FF & G.GG & H.HH \\
 & \textbf{LLaMA 3.1} & .AAA & .BBB & .CCC & .DDD & E.EE & F.FF & G.GG & H.HH \\ \midrule
\multirow{3}{*}{\textbf{\begin{tabular}[c]{@{}l@{}}Medium\\ (50-300 words)\\ n=B,BB\end{tabular}}} & \textbf{GPT-4o} & .AAA & .BBB & .CCC & .DDD & E.EE & F.FF & G.GG & H.HH \\
 & \textbf{OpenAI o1} & .AAA & .BBB & .CCC & .DDD & E.EE & F.FF & G.GG & H.HH \\
 & \textbf{LLaMA 3.1} & .AAA & .BBB & .CCC & .DDD & E.EE & F.FF & G.GG & H.HH \\ \midrule
\multirow{3}{*}{\textbf{\begin{tabular}[c]{@{}l@{}}Long\\ (300+ words)\\ n=C,CC\end{tabular}}} & \textbf{GPT-4o} & .AAA & .BBB & .CCC & .DDD & E.EE & F.FF & G.GG & H.HH \\
 & \textbf{OpenAI o1} & .AAA & .BBB & .CCC & .DDD & E.EE & F.FF & G.GG & H.HH \\
 & \textbf{LLaMA 3.1} & .AAA & .BBB & .CCC & .DDD & E.EE & F.FF & G.GG & H.HH \\ \midrule
\multirow{3}{*}{\textbf{\begin{tabular}[c]{@{}l@{}}Overall\\ n=D,DD\end{tabular}}} & \textbf{GPT-4o} & .AAA & .BBB & .CCC & .DDD & E.EE & F.FF & G.GG & H.HH \\
 & \textbf{OpenAI o1} & .AAA & .BBB & .CCC & .DDD & E.EE & F.FF & G.GG & H.HH \\
 & \textbf{LLaMA 3.1} & .AAA & .BBB & .CCC & .DDD & E.EE & F.FF & G.GG & H.HH \\ \bottomrule
\end{tabular}
\caption{Results by length.}
\label{tab:results-by-length}
\end{table*}
    
\end{comment}


\section{Further Analysis}
\label{sec:analysis}
\begin{figure}[!t] 
    \includegraphics[width=\columnwidth]{images/robustness.pdf} 
    \vspace{-4.5ex}
    \caption{Robustness Evaluation compares the KFR of three methods under precision changes (float16 → bfloat16) and jailbreak attacks.} \vspace{-3ex} 
    \label{fig:robustess} 
\end{figure}
\subsection{Robustness Evaluation}
Building on previous work \citep{zhang2024doesllmtrulyunlearn, lu2024eraserjailbreakingdefenselarge}, which demonstrates that parameter precision and jailbreak attacks affect unlearning, we analyze the robustness of unlearned models under these conditions on KnowUnDo. 
The results are presented in Figure~\ref{fig:robustess}, and we can summarize two key findings.
\paragraph{ReLearn Prevents Knowledge Leakage under Precision Variation.}
As seen from Figure~\ref{fig:robustess}, we observe that reducing the precision of the parameter from float16 to bfloat16 causes a significant decrease in KFR performance, 9.7\% for GA and 18.2\% for NPO.
This suggests that GA and NPO are sensitive to parameter precision and rely on fine-grained adjustments during LoRA fine-tuning.
The sentence completion examples in Appendix Table~\ref{tab:robustness_case} demonstrate that while GA and NPO exhibit unreadable outputs in most cases, indicating over-forgetting, they also reveal some instances of knowledge leakage.
In contrast, ReLearn shows a slight performance improvement of 1.4\% under reduced precision while consistently maintaining a coherent output.
\paragraph{ReLearn Effectively Resists Jailbreaks.}
By using the AIM jailbreak attack \citep{NEURIPS2023_fd661313}, a prompt engineering method that forces compromised model responses (with templates in Appendix~\ref{appendix:AIM}), we observe KFR performance degradation of 5.0\% for GA and 9.1\% for NPO.
In particular, ReLearn achieves a performance improvement of 6.9\%. 
This difference indicates that GA and NPO weaken the base model's inherent jailbreak resistance, while ReLearn maintains and even enhances this defensive capability. 
As seen from the examples shown in Table~\ref{tab:robustness_case}, when attacked, ReLearn effectively prevents jailbreak attacks targeting forgotten knowledge, while GA and NPO tend to leak private information (sometimes incomplete) or generate unreadable responses.

\subsection{The Mechanism of Unlearning}
In this section, we analyze how GA and NPO disrupt the model's linguistic ability and explore how ReLearn reconstructs it.
We analyze from three perspectives: Knowledge Distribution, Knowledge Memory, and Knowledge Circuits.

\subsubsection{Knowledge Distribution}
GA and NPO both rely on reverse optimization to suppress the probabilities of the target token, leading to \textbf{\textit{a disruptive ``probability seesaw effect''}}. 
To explore the knowledge distribution of different unlearning models, we calculate the top-5 candidate tokens in their outputs, as shown in Figure~\ref{fig:prob} and Figure~\ref{fig:gemma_top5} in the Appendix. 
As observed, in models with a \textbf{multi-peaked probability distribution} (e.g., Llama2 Vanilla in Figure~\ref{fig:prob}), the ``seesaw'' effect exhibits two sequent steps: 
(1) \emph{Initial Target Token Suppression:} By suppressing the initially top-1 token and guiding the model towards other high-probability tokens, this potentially leads to sensitive responses (as illustrated in Figure~\ref{fig:prob}, where the top-2 token in the Vanilla model becomes the top-1 token in the NPO model).
(2) \emph{Subsequent Top Token Suppression:} This involves the continued suppression of high-probability tokens, resulting in probability redistribution across random tokens (as observed on Llama2 GA in Figure~\ref{fig:prob}).  
In contrast, for models with a \textbf{unimodal probability distribution} (e.g., Gemma in Figure~\ref{fig:gemma_top5}), reverse optimization merely suppresses the single high-probability peak of the target token, resulting in a more uniform probability distribution across random tokens after unlearning. 

The disrupted probability distributions resemble \emph{cognitive conflict} \citep{xu-etal-2024-earth}, which arises from the conflict between the intrinsic knowledge of a model and external inputs or training objectives.  
\textbf{Reverse optimization directly drives the decoding space toward randomness, leading to a significant cognitive mismatch between the pre-unlearning and post-unlearning states, limiting question understanding and coherent generation.}  
In contrast, ReLearn does not aim for a complete disruption of the knowledge distribution.  
By learning to generate relevant yet non-sensitive answers, ReLearn guides the model toward a new cognitive pattern.

\begin{figure}[!htbp]
\includegraphics[width=\linewidth]{images/top5_llama2.pdf}
% \vspace{-3ex}
  \caption{The top-5 candidate tokens distribution of different unlearning approaches on KnowUnDo.}
  % \vspace{-2ex}
  \label{fig:prob}
\end{figure}
\begin{figure}[!t]
  \centering
  \includegraphics[width=\linewidth]{images/knowledge_memory.pdf}
  \vspace{-4ex}
  \caption{Knowledge Memory. Vanilla model generates ``5000 Sierra Rd Bogota Colomb''; GA/NPO produce repetitive ``at''; ReLearn generates a contextually relevant but non-sensitive response.}
  \vspace{-2ex}
  \label{fig:mem}
\end{figure}
\subsubsection{Knowledge Memory}
Inspired by recent research \citep{geva-etal-2022-transformer, geva-etal-2023-dissecting, ghandeharioun2024s, menta2025analyzingmemorizationlargelanguage} that the early layers process context, the deeper layers memorize, and the last few layers handle the prediction of the next token, our analysis focuses on the final token position's outputs across all decoding layers\citep{belrose2023elicitinglatentpredictionstransformers}.

Figure~\ref{fig:mem} demonstrates the difference between these methods.
When queried with ``Carlos Rivera's mailing address is...'', the vanilla model directly activates both general concepts like ``address'' and ``location'', as well as the answer terms such as ``Colomb''. 
In contrast, ReLearn preserves semantic understanding without directly recalling the answer. 
In its middle and later layers, it recalls related concepts like ``located'' and ``address'', along with query terms such as ``Carlos''.
In comparison, reverse optimization methods like NPO activate ``address'' before the 20th layer but fail to trigger related knowledge afterward, instead repeating ``at'' beyond the 20th layer.

Moreover, the Forward-KL, which represents the KL Divergence between the current and final layers, shows a gradual shift for the vanilla and ReLearn models, but a severe shift for GA/NPO.
This severe change hinders the effective use of semantic information for knowledge retrieval and refinement, impeding the appropriate generation of responses.

In summary, \textbf{reverse optimization significantly impairs knowledge memory by overemphasizing next-token prediction and disrupting the ability of gradual information adjustment}, which is similar to memory loss in Alzheimer's disease \citep{Jahn2013memoryloss}. 
In contrast, ReLearn maintains robust knowledge memory across layers, preserving linguistic capabilities, and enabling fluent, relevant responses through positive optimization.

\subsubsection{Knowledge Circuits}
We employ the LLMTT tool \citep{tufanov2024lm} to visualize \textit{knowledge circuits} and investigate how different unlearning methods affect model focus.
LLMTT identifies the salient connections (``circuits'') within the LLM inference process by varying the threshold, where higher thresholds indicate stronger connections.
As shown in Figure~\ref{fig:circuits} in the Appendix, with a threshold of 0.06, the vanilla, GA, and NPO models exhibit similar circuit patterns. 
However, ReLearn notably reduces circuits associated with sensitive entities, indicating a weakened focus on sensitive information.
When the threshold increases to 0.08, the circuits of vanilla model and ReLearn model become empty, while GA and NPO strengthen partial circuits, particularly those specific question patterns (e.g., ``How does...background...?'').
This observation suggests that \textbf{GA and NPO over-forget specific question patterns}, while ReLearn achieves generalized unlearning by weakening entity associations.
% resulting in issues similar to generalization problems caused by spurious correlations \citep{bayat2024pitfallsmemorizationmemorizationhurts}.


\section{Related works}

\paragraph{LLM alignment.}
Pretrained LLMs demonstrate remarkable capabilities across a broad spectrum of tasks \citep{brown2020language}.
Their performance at downstream tasks, such as conversational modeling, is significantly enhanced through alignment with human preferences \citep{ouyang2022training, bai2022training}. 
RLHF \citep{christiano2017deep} has emerged as a foundational framework for this alignment, typically involving learning a reward function via a preference model, often using the Bradley-Terry model \citep{bradley1952rank}, and tuning the LLM using reinforcement learning (RL) to optimize this reward. 
Despite its success, RLHF's practical implementation is notoriously complex, requiring multiple LLMs, careful hyperparameter tuning, and navigating challenging optimization landscapes.

Recent research has focused on simplifying this process. A line of works studies the direct alignment algorithms \citep{zhao2023slic, rafailov2024direct, azar2024general}, which directly optimize the LLM in a supervised manner without first constructing a separate reward model. In particular, the representative DPO \citep{rafailov2024direct} attracts significant attention in both academia and industry. After these, SimPO \citep{meng2024simpo} simplifies DPO by using length regularization in place of a reference model. 
% However, these approaches are primarily offline and can suffer from performance degradation when facing distribution shifts during deployment, yielding poorer generalization.
% Iterative DPO \citep{xiong2024iterative} attempts to address this by enabling iterative optimization for improved alignment.

% These challenges can be potentially tackled by connecting LLM alignment with IR.
Although LLMs are adopted for IR \citep{tay2022transformer}, there is a lack of study to improve direct LLM alignment with IR principles.
% While significant progress has been made in LLM alignment, the connection with IR remains largely unexplored.
% However, LiPO does not explore the potential of online optimization methods inspired by IR. 
This paper fills this gap by establishing a systematic link between LLM alignment and IR methodologies, and introducing a novel iterative LLM alignment approach that leverages insights from retriever optimization to advance the state of the art.
The most related work is LiPO \citep{liu2024lipo}, which applies learning-to-rank objectives.
% However, LiPO focuses on offline settings and it is unclear how to obtain list-wise data for it.
However, LiPO relies on off-the-shelf listwise preference data, which is hard to satisfy in practice.

\paragraph{Language models for information retrieval.}
Language models (LMs) have become integral to modern IR systems \citep{zhu2023large}, particularly after the advent of pretrained models like BERT \citep{kenton2019bert}.  
A typical IR pipeline employs retrievers and rerankers, often based on dual-encoder and cross-encoder architectures, respectively \citep{humeau2019poly}. 
Dense Passage Retrieval (DPR) \citep{karpukhin2020dense} pioneered the concept of dense retrieval, laying the groundwork for subsequent research. 
Building on DPR, studies have emphasized the importance of hard negatives in training \citep{zhan2021optimizing, qu2020rocketqa} and the benefits of online retriever optimization \citep{xiong2020approximate}.

In the realm of reranking, \citep{nogueira2019passage} were among the first to leverage pretrained language models for improved passage ranking. 
This was followed by MonoT5 \citep{nogueira2020document}, which scaled rerankers using large encoder-decoder transformer architectures, and RankT5 \citep{zhuang2023rankt5}, which introduced pairwise and listwise ranking objectives. 
Recent work has also highlighted the importance of candidate list preprocessing before reranking \citep{meng2024ranked}.

Despite the pervasive use of LMs in IR, the interplay between LLM alignment and IR paradigms remains largely unexplored. 
This work aims to bridge this gap, establishing a strong connection between LLM alignment and IR, and leveraging insights from both fields to advance our understanding of LLM alignment from an IR perspective.


\section{Conclusions}
This paper investigates the impact of increasing the number of retrieved passages on the performance of long-context LLMs in retrieval-augmented generation (RAG) systems.
Contrary to expectations, we observe that performance initially improve but then degrade as more passages are included.
This phenomenon is attributed to the detrimental influence of retrieved "hard negatives".
To mitigate this issue, we propose and evaluate three solutions: training-free retrieval reordering, RAG-specific implicit LLM fine-tuning, and RAG-oriented LLM fine-tuning with intermediate reasoning.
A systematic analysis of the training-based methods explores the effects of data distribution, retriever for training, and training context length.
Interesting future directions include exploring (automated) position optimization with more advanced retrieval ordering methods, and fine-tuning the LLMs for RAG with more fine-grained and multi-step reasoning chains.

% Bibliography components
\bibliographystyle{abbrvnat}
\nobibliography*
\bibliography{main}

\clearpage
% Some other useful sections you might consider having in your report.
% \section*{Citing this work}
% This is a free, open access paper provided by Google DeepMind.
% The final version of this work was published online (provide venue, date and digital object
% identifier (doi, if available). \textit{Cite as:} \citeas{lee2024gecko}.

% If you add a bibtex entry of your own paper (this paper), you can
% show its full citation inline using \citeas, as above. Note that
% this citation removes the trailing full stop. To make \citeas work,
% you need to load the bibliography data. This can be done in two
% ways:
%
%    1. If you already have a printed bibliography with \bibliography{...},
%       then add the command "\nobibliography*", no arguments, before that.
%    2. If you don't otherwise print a bibliography, add the command
%       \nobibliography{...} at the end of your document.


% \section*{Author Contributions}

% \section*{Acknowledgements}

% \section*{Funding}
% This research was funded by Google.

% \section*{Competing interests}
% The authors declare no competing financial interests. Related
% patent number here if applicable.

% \section*{Data availability}
% The datasets used in the experiments have been made available
% for download at IF AVAILABLE.


\newpage
\appendix
% \onecolumn

% \section{Pseudo-code for \Ours}

\section{LLM inference strategy and IR pipelines}

\begin{table}[h]
\caption{Correspondence between LLM inference and IR pipelines.}
  \label{tb:llm-retriever-reranker}
  \centering
%   \small
  \scalebox{0.8}{\begin{tabular}{lccc}
    \toprule
    Method & Retriever & Reranker & Pipeline       \\
    \midrule
    Greedy decoding     & LLM &  $\emptyset$ & Retriever-only  \\
    \midrule
    Best-of-N \citep{stiennon2020learning} & LLM & Reward model & Retriever-reranker  \\
    \midrule
    Majority voting  \citep{wang2022self}  & LLM & Majority & Retriever-reranker  \\
    \midrule
    Iterative refinement \citep{madaan2024self} & LLM & $\emptyset$ & Iterative retrieval  w. query rewriting \\
    \bottomrule
  \end{tabular}}
\end{table}


\section{How can SFT and preference optimization help the LLM from an IR perspective?}\label{apx:sft-rlhf-empirical}


We assess how well LLMs perform at two tasks: fine-grained reranking (using greedy decoding accuracy) and coarse-grained retrieval (using Recall@$N$).  
We focus on how SFT and DPO, affect these abilities.  
Using the Mistral-7b model, we evaluate on the GSM8k and MATH datasets with two approaches: SFT-only, and SFT followed by DPO (SFT $\rightarrow$ DPO).

In the SFT phase, the model is trained directly on correct answers. 
For DPO, we generate 20 responses per prompt and created preference pairs by randomly selecting one correct and one incorrect response.  
We use hyperparameter tuning and early stopping to find the best model checkpoints (see Appendix \ref{apx:sec:sft-rlhf} for details).


\begin{table}[h]
\caption{Retrieval (Recall@N) and reranking (greedy accuracy) metrics across dataset and training strategies, with Mistral-7b as the LLM. 0.7 is used as the temperature. Recall@N can also be denoted as pass@N.}\label{tb:sft-rlhf-result}
\vskip 1em
\centering
\small
\begin{tabular}{llcccc}
    \toprule
     & Metric & \textbf{init model} & \textbf{SFT} & \textbf{SFT $\rightarrow$ DPO} \\
    \midrule
    \multirow{4}{*}{\rotatebox{90}{GSM8K}} 
    & Greedy Acc & 0.4663 & 0.7680 & 0.7991  \\
    & Recall@20 & 0.8347 & 0.9462 & 0.9545  \\
    & Recall@50 & 0.9090 & 0.9629 & 0.9727  \\
    & Recall@100 & 0.9477 & 0.9735 & 0.9826   \\
    \midrule
    \multirow{4}{*}{\rotatebox{90}{Math}} 
    & Greedy Acc & 0.1004 & 0.2334 & 0.2502 \\
    & Recall@20 & 0.2600 & 0.5340 & 0.5416  \\
    & Recall@50 & 0.3354 & 0.6190 & 0.6258  \\
    & Recall@100 & 0.4036 & 0.6780 & 0.6846  \\
    \bottomrule
\end{tabular}
\end{table}

The results are shown in Table \ref{tb:sft-rlhf-result}.  
We observe that both SFT and DPO improve both retrieval and reranking, with SFT being more effective. Adding DPO after SFT further improves performance on both tasks.  
This is consistent with information retrieval principles that both direct retriever optimization and reranker-retrieval distillation can enhance the retriever performance, while the latter on top of the former can further improve the performance. Further discussions can be found in Appendices \ref{apx:discuss1} and \ref{apx:discuss2}.


\section{Discussion on the connection and difference between SFT and direct retriever optimization}\label{apx:discuss1}

As discussed in Section \ref{sec:llm-tuning-retriever}, the direct retriever optimization goal with InfoNCE is shown as:
\begin{gather*}
    \max \log P(d_{\text{gold}}|q) = \max \log \frac{\text{Enc}_d(d_{\text{gold}}) \cdot\text{Enc}_q(q)}{\sum^{|C|}_{j=1} \text{Enc}_d(d_j) \cdot\text{Enc}_q(q)},
\end{gather*}
while the SFT optimization goal is shown as:
\begin{gather}
    \max \log P(y_{\text{gold}}|x) = \max \log \prod^{|y_{\text{gold}}|}_i P(y_{\text{gold}}(i)|z_i) 
    = \max \sum^{|y_{\text{gold}}|}_i \log \frac{\text{Emb}(y_{\text{gold}}(i)) \cdot\text{LLM}(z_i)}{\sum^{|V|}_{j=1} \text{Emb}(v_j) \cdot\text{LLM}(z_i)}. \label{apx:eq:sft}
\end{gather}

As a result, the SFT objective can be seen as a summation of multiple retrieval optimization objectives, where $\text{LLM}(\cdot)$ and word embedding $\text{Emb}(\cdot)$ are query encoder and passage encoder respectively.

However, for direct retriever optimization with InfoNCE, $\text{Enc}_d(\cdot)$ is usually a large-scale pretrained language model which is computationally expensive on both time and memory.
In this case, it is unrealistic to calculate the $\text{Enc}_d(d_j)$ for all $d_j\in C$, when $C$ is large, because of the time constrain and GPU memory constrain.
As a result, a widely-adopted technique is to adopt ``in-batch negatives'' with ``hard negatives'' to estimate the $\log P(d_{\text{gold}}|q)$ function:
\begin{gather*}
    \max \log P(d_{\text{gold}}|q) = \max \log \frac{\text{Enc}_d(d_{\text{gold}}) \cdot\text{Enc}_q(q)}{\sum^{|C|}_{j=1} \text{Enc}_d(d_j) \cdot\text{Enc}_q(q)} \\
    \sim \max \log \frac{\text{Enc}_d(d_{\text{gold}}) \cdot\text{Enc}_q(q)}{\sum^{|B|}_{i=1} \text{Enc}_d(d_i) \cdot\text{Enc}_q(q) + \sum^{|H|}_{j=1} \text{Enc}_d(d_j) \cdot\text{Enc}_q(q)},
\end{gather*}
where $B$ is the in-batch negative set and $H$ is the hard negative set.
Note that $B\bigcup H \subset C$.
This objective is more efficient to optimize but is not the original optimization goal. As a result, the learned model after direct retriever optimization is not optimal.
It is also found that the hard negatives $H$ is the key to estimate the original optimization goal \citep{zhan2021optimizing}.
Thus, reranker-retriever distillation can further improve the retriever by introducing more hard negatives.

On the other hand, LLM optimization, as shown in Eq. (\ref{apx:eq:sft}), can be seen as a summation of multiple retrieval optimization function.
In each retrieval step, the passage can be seen as a token and the corpus is the vocabulary space $V$.
Given that the passage encoder $\text{Emb}(\cdot)$ (word embedding) here is cheap to compute and the vocabulary space $V$ ($<$100k) is usually not as large as $C$ ($>$1M) in IR, the objective in Eq. (\ref{apx:eq:sft}) can be directly optimized without any estimation.
In this case, the LLM as a retriever is more sufficiently trained compared with the retriever training in IR.


\section{Discussion on the connection and difference between preference optimization and reranker-retriever distillation}\label{apx:discuss2}

As discussed in Section \ref{sec:llm-tuning-retriever}, preference optimization with an online reward model $f_{\text{reward-model}}(\cdot) \overset{r}{\rightarrow} \text{data} \overset{g(\cdot)}{\rightarrow}  f_{\text{LLM}}(\cdot)$ can be seen as a reranker to retriever distillation process $f_{\text{rerank}}(\cdot) \overset{r}{\rightarrow} \text{data}\overset{g(\cdot)}{\rightarrow}   f_{\text{retrieval}}(\cdot)$, where the reward model is the reranker (\textit{i.e.}, cross-encoder) and the LLM is the retriever (\textit{i.e.}, bi-encoder).

However, there are two slight differences here:
\begin{itemize}[leftmargin=*]
\item The LLM after SFT is more sufficiently trained compared to a retriever after direct optimization. As discussed in Appendix \ref{apx:discuss1}, the SFT optimization function is not an estimated retriever optimization goal compared with the direct retrieval optimization. As a result, the LLM after SFT is suffienctly trained. In this case, if the reward model (reranker) cannot provide information other than that already in the SFT set (\textit{e.g.}, using the SFT prompts), this step may not contribute to significant LLM capability improvement.
\item The reward model may introduce auxiliary information than the reranker in IR. For a reranker in IR, it captures a same semantic with the retriever: semantic similarity between the query and the passage. However, in LLM post-training, the goal and data in SFT and preference optimization can be different. For example, the SFT phase could have query/response pairs which enable basic chat-based retrieval capability for the LLM. While the reward model may contain some style preference information or safety information which do not exist in SFT data. In this case, the preference optimization which is the reranker to retriever distillation step could also contribution to performance improvement.
\end{itemize}


\section{Evaluate LLMs as retrievers}\label{apx:llm-as-retriever}

In addition to Mathstral-7b-it on GSM8K in Figure \ref{fig:mathstral-gsm8k-infer}, we conduct extensive experiments to both Mistral-7b-it and Mathstral-7b-it on GSM8K and MATH. The results are shown in Figure \ref{apx:fig:empirical-llm-retriever}.
We have similar findings as in Figure \ref{fig:mathstral-gsm8k-infer} that:
(1) As $N$ increases, Recall@$N$ improves significantly, indicating that retrieving a larger number of documents increases the likelihood of including a correct one within the set.
(2) For smaller values of $N$ (e.g., $N=1$), lower temperatures yield higher Recall@$N$. This is because lower temperatures reduce response randomness, favoring the selection of the most relevant result.
(3) Conversely, for larger $N$ (e.g., $N>10$), higher temperatures enhance Recall@$N$. Increased temperature promotes greater response diversity, which, when combined with a larger retrieval set, improves the chances of capturing the correct answer within the results.

\begin{figure*}[h]
    \centering
    \subfigure[Mistral-7b-it on GSM8k]{\includegraphics[width=0.45\textwidth]{figure/LLM_alignment_gsm8k_mathstral7b_infer.pdf}}
    \subfigure[Mistral-7b-it on GSM8k]{\includegraphics[width=0.45\textwidth]{figure/LLM_alignment_gsm8k_mistral7b_infer.pdf}}
    \subfigure[Mathstral-7b-it on MATH]{\includegraphics[width=0.45\textwidth]{figure/LLM_alignment_math_mathstral7b_infer.pdf}} 
    \subfigure[Mistral-7b-it on MATH]{\includegraphics[width=0.45\textwidth]{figure/LLM_alignment_math_mistral7b_infer.pdf}}
    % \vspace{-0.1in}
    \vskip -1em
    \caption{Evaluate the LLM as a retriever with Recall@N (Pass@N). As the number (N) of retrieved responses increases, the retrieval recall increases. The higher the temperature is, the broader spectrum the retrieved responses are, and thus the higher the recall is.}\label{apx:fig:empirical-llm-retriever}
\end{figure*}


% \subsection{How SFT and RLHF benefit the LLM retriever?}\label{apx:sft-rlhf}

% In addition to the experiments with Gemma-1-7b-it in Table \ref{tb:sft-rlhf-result}, we also conduct experiments to study the effect of SFT and DPO on Deepseek-math-7b-base model \citep{shao2024deepseekmath}.
% The results on MATH dataset are shown in Table \ref{apx:tb:sft-rlhf-result}, where we have similar discovery with that in Table \ref{tb:sft-rlhf-result}:
% (1) Both SFT and DPO can improve the retrieval capability of the LLM, while SFT is more effective.
% (2) On top of SFT, DPO can slightly improve the reranking capability (greedy accuracy) but not the general retrieval capability.

% \begin{table}[h]
% \caption{Retrieval (Recall@N) and reranking (greedy accuracy) metrics across dataset and training strategies. LLM: Deepseek-math-7b. Temperature: 0.7. Recall@N can also be denoted as pass@N.}\label{apx:tb:sft-rlhf-result}
% \vskip 1em
% \centering
% \scalebox{0.8}{
% \begin{tabular}{llcccc}
%     \toprule
%      & Metric & \textbf{init model} & \textbf{DPO} & \textbf{SFT} & \textbf{SFT $\rightarrow$ DPO} \\
%     \midrule
%     \multirow{4}{*}{\rotatebox{90}{Math}} 
%     & Greedy Acc & 0.0972 & 0.1164 & 0.3078 & 0.312 \\
%     & Recall@20 & 0.4914 & 0.5136 & 0.6524 & 0.6558 \\
%     & Recall@50 & 0.6058 & 0.6278 & 0.7332 & 0.736 \\
%     & Recall@100 & 0.6728 & 0.6976 & 0.7844 & 0.7828 \\
%     \bottomrule
% \end{tabular}}
% \end{table}



\section{\Ours retriever optimization objective}\label{apx:proofs}

We provide the proof for different variants of \Ours's objective functions.

\subsection{Contrastive ranking}\label{apx:proof:contrastive}

\begin{theorem}
Let \( x \) be a prompt and \( (y_w, y^{(1)}_l, ..., y^{(m)}_l)  \) be the responses for \( x \) under the contrastive assumption (Eq.(\ref{eq:contrastive-assumption})).
Then the objective function to learn the LLM \( \pi_\theta \):
\end{theorem}

\begin{equation}
    \begin{aligned}
    \mathcal{L}_{\text{con}} = -\mathbb{E} & \biggl[
    \log \frac{\exp\bigl(\gamma(y_w \mid x)\bigr)}{
        \exp\bigl(\gamma(y_w \mid x)\bigr) + \sum_{i=1}^m \exp\bigl(\gamma(y_l^{(i)} \mid x)\bigr)}
    \biggr], \\
    \text{where } &\quad \gamma(y \mid x) = \beta \log \frac{\pi_\theta(y \mid x)}{\pi_{\mathrm{ref}}(y \mid x)}.
\end{aligned}\label{eq:contrastive}
\end{equation}

\textit{Proof.}
From \citep{rafailov2024direct}, we know that
\begin{gather}
    r(x, y) = \beta \text{log} \frac{\pi_{\text{llm}}(y|x)}{\pi_{\text{ref}}(y|x)} + \beta \text{log} Z,
\end{gather}
where $Z = \sum_{y'} \pi_{\text{ref}}(y'|x) \text{exp}(\frac{1}{\beta} r(x, y'))$.

Then,
\begin{equation}\label{eq:1-n}
\begin{aligned}
\mathbb{P}\text{r}(y_w & \succeq y^{(1)}_l, ..., y_w \succeq y^{(m)}_l) 
= \text{softmax}(r(x, y_w)) \\
&= \frac{\text{exp}(r(x,y_w))}{\text{exp}(r(x,y_w)) + \sum^m_{i=1}\text{exp}(r(x,y^{(i)}_l))} \\
&= \frac{1}{1 + \sum^m_{i=1}\text{exp}(r(x,y^{(i)}_l)-r(x,y_w))} \\
&= \frac{1}{1 + \sum^m_{i=1}\text{exp}(\gamma(y^{(i)}_l \mid x) + \beta \text{log} Z - \gamma(y_w \mid x) - \beta \text{log} Z)} \\
&= \frac{1}{1 + \sum^m_{i=1}\text{exp}(\gamma(y^{(i)}_l \mid x) - \gamma(y_w \mid x))} \\
&= \frac{\exp\bigl(\gamma(y_w \mid x)\bigr)}{
        \exp\bigl(\gamma(y_w \mid x)\bigr) + \sum_{i=1}^m \exp\bigl(\gamma(y_l^{(i)} \mid x)\bigr)}
\end{aligned}
\end{equation}

We can learn $\pi_\theta$ by maximizing the logarithm-likelihood: 
\begin{gather}
\max \log \mathbb{P}\text{r}(y_w \succeq y^{(1)}_l, \dots, y_w \succeq y^{(m)}_l) \Leftrightarrow 
\min - \log \mathbb{P}\text{r}(y_w \succeq y^{(1)}_l, \dots, y_w \succeq y^{(m)}_l) = \mathcal{L}, \\
 \therefore \mathcal{L}_{\text{con}} = -\mathbb{E} \biggl[
    \log \frac{\exp\bigl(\gamma(y_w \mid x)\bigr)}{
        \exp\bigl(\gamma(y_w \mid x)\bigr) + \sum_{i=1}^m \exp\bigl(\gamma(y_l^{(i)} \mid x)\bigr)}
    \biggr], \\
\text{where} \quad \gamma(y \mid x) = \beta \log \frac{\pi_\theta(y \mid x)}{\pi_{\mathrm{ref}}(y \mid x)}.
\end{gather}



\subsection{LambdaRank ranking}\label{apx:proof:lambdarank}

\begin{theorem}
Let \( x \) be a prompt and \( (y_1, ..., y_m)  \) be the responses for \( x \) under the LambdaRank assumption (Eq.(\ref{eq:lambdarank-assumption})).
Then the objective function to learn the LLM \( \pi_\theta \):
\end{theorem}

% \begin{gather}
%     \mathcal{L}_{\text{lamb}}=-\mathbb{E}\;\biggl[ \sum_{1<i<j<m}
%   w_{ij}\log \sigma\Bigl(
%      \gamma(y_i \mid x)-
%      \gamma(y_j \mid x)
%   \Bigr)
% \biggr]
% \end{gather}
\begin{gather}
    \mathcal{L}_{\text{lamb}}=-\mathbb{E}\;\biggl[ \sum_{1<i<j<m}
   \log \sigma\Bigl(
     \gamma(y_i \mid x)-
     \gamma(y_j \mid x)
   \Bigr)
\biggr].
\end{gather}
% where $w_{ij}$ is an adjustable weight.

\textit{Proof.}
\begin{equation}
\begin{aligned}
\mathbb{P}\text{r}(y_1 & \succeq ... \succeq y_m)
= \prod_{1<i<j<m} \sigma(r(x,y_i) - r(x,y_j)) \\
&= \prod_{1<i<j<m} \sigma(\gamma(x,y_i) + \beta \text{log} Z - \gamma(x,y_j) - \beta \text{log} Z)  \\
&= \prod_{1<i<j<m} \sigma(\gamma(y_i \mid x)-
     \gamma(y_j \mid x)).
\end{aligned}
\end{equation}

We can learn $\pi_\theta$ by maximizing the logarithm-likelihood: 
\begin{gather}
\max \log \mathbb{P}\text{r}(y_w \succeq y^{(1)}_l, \dots, y_w \succeq y^{(m)}_l) \Leftrightarrow 
\min - \log \mathbb{P}\text{r}(y_w \succeq y^{(1)}_l, \dots, y_w \succeq y^{(m)}_l) = \mathcal{L}, \\
 \therefore \mathcal{L}_{\text{lamb}}=-\mathbb{E}\;\biggl[ \sum_{1<i<j<m}
   \log \sigma\Bigl(
     \gamma(y_i \mid x)-
     \gamma(y_j \mid x)
   \Bigr)
\biggr], \\
\text{where} \quad \gamma(y \mid x) = \beta \log \frac{\pi_\theta(y \mid x)}{\pi_{\mathrm{ref}}(y \mid x)}.
\end{gather}
% $w_{ij}$ can be added to control the weight of each pair in the candidate list.


\subsection{ListMLE ranking}\label{apx:proof:listmle}

\begin{theorem}
Let \( x \) be a prompt and \( (y_1, ..., y_m)  \) be the responses for \( x \) under the ListMLE assumption (Eq.(\ref{eq:listmle-assumption})).
Then the objective function to learn the LLM \( \pi_\theta \):
\end{theorem}

\begin{equation}
\begin{aligned}
    \mathcal{L}_{\text{lmle}} &= -\mathbb{E} \biggl[
    \sum^m_{i=1} \log \frac{\exp\bigl(\gamma(y_i \mid x)\bigr)}{
        \exp\bigl(\gamma(y_i \mid x)\bigr) + \sum_{j=i}^m \exp\bigl(\gamma(y_j \mid x)\bigr)}
    \biggr].
\end{aligned}
\end{equation}

\textit{Proof.}
From Eq.(\ref{eq:1-n}),
\begin{gather}
\begin{aligned}
    \mathbb{P}\text{r}(y_1 & \succeq ... \succeq y_m) = \prod^m_{i=1} \mathbb{P}\text{r}(y_i \succeq y_{i+1}, ..., y_i \succeq y_m)  \\
    & = \prod^m_{i=1} \frac{\text{exp}(\gamma(y_i \mid x))}{\text{exp}(\gamma(y_i \mid x)) + \sum^m_{j=i+1}\text{exp}(\gamma(y_j \mid x))}
\end{aligned}.
\end{gather}
% The derivation above uses the result from Eq.(\ref{eq:1-n}).

We can learn $\pi_\theta$ by maximizing the logarithm-likelihood: 
\begin{gather}
\max \log \mathbb{P}\text{r}(y_w \succeq y^{(1)}_l, \dots, y_w \succeq y^{(m)}_l) \Leftrightarrow 
\min - \log \mathbb{P}\text{r}(y_w \succeq y^{(1)}_l, \dots, y_w \succeq y^{(m)}_l) = \mathcal{L}, \\
 \therefore \mathcal{L}_{\text{lmle}} = -\mathbb{E} \biggl[
    \sum^m_{i=1} \log \frac{\exp\bigl(\gamma(y_i \mid x)\bigr)}{
        \exp\bigl(\gamma(y_i \mid x)\bigr) + \sum_{j=i}^m \exp\bigl(\gamma(y_j \mid x)\bigr)}
    \biggr], \\
\text{where} \quad \gamma(y \mid x) = \beta \log \frac{\pi_\theta(y \mid x)}{\pi_{\mathrm{ref}}(y \mid x)}.
\end{gather}


\section{Baselines}\label{apx:sec:baselines}

We conduct detailed illustrations on the baselines compared with \Ours in Section \ref{sec:main-result} below.

\begin{itemize}[leftmargin=*]
  \item RRHF \citep{yuan2023rrhf} scores responses via a logarithm of conditional probabilities and learns to align these probabilities with human preferences through ranking loss.
  \item SLiC-HF \citep{zhao2023slic} proposes a sequence likelihood calibration method which can learn from human preference data.
  \item DPO \citep{guo2024direct} simplifies the PPO \citep{ouyang2022training} algorithms into an offline direct optimization objective with the pairwise Bradley-Terry assumption.
  \item IPO \citep{azar2024general} theoretically grounds pairwise assumption in DPO into a pointwise reward.
  \item CPO \citep{xu2024contrastive} adds a reward objective with sequence likelihood along with the SFT objective.
  \item KTO \citep{ethayarajh2024kto} adopts the Kahneman-Tversky model and proposes a method which directly maximizes the utility of generation instead of the likelihood of the preferences.
  \item RDPO \citep{park2024disentangling} modifies DPO by including an additional regularization term to disentangle the influence of length.
  \item SimPO \citep{meng2024simpo} further simplifies the DPO objective by using the average log probability of a sequence as the implicit reward and adding a target reward margin to the Bradley-Terry objective.
  \item Iterative DPO \citep{xiong2024iterative} identifies the challenge of offline preference optimization and proposes an iterative learning framework.
\end{itemize}


\section{Experiment settings}\label{apx:sec:main-result-setting}

\subsection{Table \ref{tab:main-performance}}\label{apx:sec:main}

We conduct evaluation on two widely used benchmark: AlpacaEval2 \citep{dubois2024length} and MixEval \citep{ni2024mixeval}.
We consider two base models: Mistral-7b-base and Mistral-7b-it. For Mistral-7b-base, we first conduct supervised finetuning following \citet{meng2024simpo} before the preference optimization.

The performance scores for offline preference optimization baselines are from SimPO \citep{meng2024simpo}.
To have a fair comparison with these baselines, we adopt the same off-the-shelf reward model \citep{jiang2023llm} as in SimPO for the iterative DPO baseline and \Ours.

For the iterative DPO baseline, we generate 2 responses for each prompt, score them with the off-the-shelf reward model and construct the preference pair data to tune the model.

For \Ours (contrastive $\mathcal{L}_{\text{con}}$), we generate 10 responses each iteration and score them with the reward model. The top-1 ranked response and the bottom-3 ranked responses are adopted as the chose response and rejected responses respectively.
Generation temperature is selected as 1 and 0.8 for Mistral-7b-base and Mistral-7b-it respectively (we search it among 0.8, 0.9, 1.0, 1.1, 1.2).

For \Ours (LambdaRank $\mathcal{L}_{\text{lamb}}$), we generate 10 responses each iteration and score them with the reward model. The top-2 ranked response and the bottom-2 ranked responses are adopted as the chose response and rejected responses respectively.
Generation temperature is selected as 1 and 0.8 for Mistral-7b-base and Mistral-7b-it respectively (we search it among 0.8, 0.9, 1.0, 1.1, 1.2).

For \Ours (ListMLE $\mathcal{L}_{\text{lmle}}$), we generate 10 responses each iteration and score them with the reward model. The top-2 ranked response and the bottom-2 ranked responses are adopted as the chose response and rejected responses respectively.
Generation temperature is selected as 1 and 0.8 for Mistral-7b-base and Mistral-7b-it respectively (we search it among 0.8, 0.9, 1.0, 1.1, 1.2).

\Ours can achieve even stronger performance with stronger off-the-shelf reward model \citep{dong2024rlhf}.
% Results with stronger a reward model can be found in Appendix \ref{apx:sec:stronger-rm}.

\subsection{Table \ref{tab:objective}}\label{apx:sec-objective-setting}

We conduct experiments on both Gemma2-2b-it \citep{team2024gemma} and Mistral-7b-it \citep{jiang2023mistral}.
Following \citet{Tunstall_The_Alignment_Handbook} and \citet{dong2024rlhf}, we perform training on UltraFeedback dataset for 3 iterations and show the performance of the final model checkpoint.
We use the pretrained reward model from \citet{dong2024rlhf}.
The learning rate is set as 5e-7 and we train the LLM for 2 epochs per iteration.

For the pairwise objective, we generate 2 responses for each prompt and construct the preference pair data with the reward model.
For the others, we generate 4 responses per prompt and rank them with the reward model.
For the contrastive objective, we construct the 1-vs-N data with the top-1 ranked response and the other responses.
For the listMLE and lambdarank objective, we take the top-2 as positives and the last-2 as the negatives.
Experiments with opensource LLM as the evaluator (\texttt{alpaca\_eval\_llama3\_70b\_fn}) can be found in Table \ref{tab:objective2}.



\begin{table*}[t]
    \centering
    % \renewcommand{\arraystretch}{1.2}
    \caption{Preference optimization objective study on AlpacaEval2 and MixEval. For AlpacaEval2, we report the result with both opensource LLM evaluator \texttt{alpaca\_eval\_llama3\_70b\_fn} and GPT4 evaluator \texttt{alpaca\_eval\_gpt4\_turbo\_fn}. SFT corresponds to the initial chat model.}\label{tab:objective2}
    \small
    \begin{tabular}{llccccccccc}
        \toprule
        & & \multicolumn{2}{c}{AlpacaEval 2 (opensource LLM)} & \multicolumn{2}{c}{AlpacaEval 2 (GPT-4)} & \multicolumn{1}{c}{MixEval} & \multicolumn{1}{c}{MixEval-Hard} \\
         \cmidrule(r){3-4} \cmidrule(r){5-6} \cmidrule(r){7-7} \cmidrule(r){8-8}
        & Method & LC Winrate & Winrate & LC Winrate & Winrate & Score & Score \\
        \midrule
        \multirow{6}{*}{\rotatebox{90}{Gemma2-2b-it}} & SFT & 47.03 & 48.38 & 36.39 & 38.26 & 0.6545 & 0.2980 \\
        \cmidrule{2-8}
        & pairwise & 55.06 & 66.56 & 41.39 & 54.60 & 0.6740 & 0.3375 \\
        & contrastive & 60.44 & 72.35 & 43.41 & 56.83 & 0.6745 & 0.3315 \\
        & ListMLE & 63.05 & 76.09 & 49.77 & 62.05 & 0.6715 & 0.3560 \\
        & LambdaRank & 58.73 & 74.09 & 43.76 & 60.56 & 0.6750 & 0.3560 \\
        \midrule
        \midrule
        \multirow{6}{*}{\rotatebox{90}{Mistral-7b-it}} & SFT & 27.04 & 17.41 & 21.14 & 14.22 & 0.7070 & 0.3610 \\
        \cmidrule{2-8}
        & pairwise & 49.75 & 55.07 & 36.43 & 41.86 & 0.7175 & 0.4105 \\
        & contrastive & 52.03 & 60.15 & 38.44 & 42.61 & 0.7260 & 0.4340 \\
        & ListMLE & 48.84 & 56.73 & 38.02 & 43.03 & 0.7360 & 0.4200 \\
        & LambdaRank & 51.98 & 59.73 & 40.29 & 46.21 & 0.7370 & 0.4400 \\
        \bottomrule
    \end{tabular}
\end{table*}


\subsection{Table \ref{fig:list-study}}\label{apx:sec-list-setting}

We adopt Gemma2-2b-it as the initial model. All the models are trained with iterative DPO for 3 iterations. We use the off-the-shelf reward model \citep{dong2024rlhf}.
We generate 2 responses for each prompt in each iteration.
For ``w. current'', we only use the scored responses in the current iteration for preference optimization data construction.
For ``w. current + prev'', we rank the responses in the current iteration and the previous one iteration, and construct the preference pair data with the top-1 and bottom-1 ranked responses.
For ``w. current + all prev'', we rank all the responses for the prompt in the current and previous iterations and construct the preference pair data.
For ``single temperature'', we only adopt temperature 1 and generate 2 responses for reward model scoring.
For ``diverse temperature'', we generate 2 responses with temperature 1 and 0.5 respective and rank the 4 responses to construct the preference data with the reward model.

\subsection{Table \ref{tb:sft-rlhf-result}}\label{apx:sec:sft-rlhf}

We use mistral-7b-it \citep{jiang2023mistral} as the initial model to alleviate the influence of the math related post-training data of the original model.
% For SFT, we conduct training on the training set of MATH \citep{hendrycks2021measuring} and GSM8K \citep{cobbe2021training} respectively.
For SFT, we conduct training on the meta-math dataset \citep{yu2023metamath}.
For DPO, we use the prompts in the training set of the two dataset and conduct online iterative preference optimization with the binary rule-based reward (measure if the final answer is correct or not with string match). 
The evaluation is performed on the test set of MATH and GSM8K respectively.
% For both SFT and DPO, we conduct careful hyper-parameter search.
For SFT, we follow the same training setting with \citet{yu2023metamath}.
For DPO, we search the learning rate in 1e-7, 2e-7, 5e-7, 2e-8, 5e-8 and train the LLM for 5 iterations with early stop (1 epoch per iteration for MATH and 2 epoch per iteration for GSM8K). The learning rate is set as 1e-7 and we select the checkpoint after the first and fourth iteration for GSM8K and MATH respectively.

\subsection{Figure \ref{fig:merge-study}(a)}\label{apx:sec-hard-neg-setting}

We conduct training with the prompts in the training set of GSM8K and perform evaluation on GSM8K testing set.
We conduct learning rate search and finalize it to be 2e-7.
The learning is performed for 3 iterations.

We make explanations of how we construct the four types of negative settings:
For (1) a random response not related to the given prompt, we select a response for a random prompt in Ultrafeedback.
For (2) a response to a related prompt, we pick up a response for a different prompt in the GSM8K training set.
For (3) an incorrect response to the given prompt with high temperature, we select the temperature to be 1.
For (4) an incorrect response to the given prompt with low temperature, we select the temperature to be 0.7.

\begin{figure}[t]
\centering
\includegraphics[scale=0.4]{figure/LLM_alignment_gemma_temperature_study.pdf}
\vskip -1em
\caption{Training temperature study with $\mathcal{L}_{\text{pair}}$ on Gemma2-2b-it and Alpaca Eval 2. Within a specific range ($>$ 0.9), lower temperature leads to harder negative and benefit the trained LLM. However, temperature lower than this range can cause preferred and rejected responses non-distinguishable and lead to degrade training.}\label{apx:tab:temp-hard}
\end{figure}

\subsection{Figure \ref{fig:merge-study}(b)}\label{apx:sec-hard-neg-setting-temp}

We conduct experiments on both Gemma2-2b-it and Mistral-7B-it models.
For both LLMs, we conduct iterative DPO for 3 iterations and report the performance of the final model.
We perform evaluation on Alpaca Eval2 with \texttt{alpaca\_eval\_llama3\_70b\_fn} as the evaluator.

For temperature study, we find that under a specific temperature threshold, repeatedly generated responses will be large identical for all LLMs and cannot be used to construct preference data, while the threshold varies for different LLMs.
% As a result, we select temperatures above the threshold for robust experiments.
The ``low'' and ``high'' refer to the value of those selected temperatures.
% For Gemma2-2b-it, we use temperature as 0.2, 0.5 and 0.7 to generate the responses, score the responses by the reward model and train the LLM with the newly labeled data.
% For Mistral-7b-it, we set the temperature as 1, 1.1 and 1.2 respectively.
We also conduct experiments on Gemma2-2b-it model and show the results in Figure \ref{apx:tab:temp-hard}.


\subsection{Figure \ref{fig:merge-study}(c)}\label{apx:sec-length-setting}

We adopt Mistral-7b-it as the initial LLM and the contrastive objective (Eq. \ref{eq:contrastive}) in iterative preference optimization.
We generate 4/6/8/10 responses with the LLM and score the responses with the off-the-shelf reward model \citep{dong2024rlhf}.
The top-1 scored response is adopted as the positive response and the other responses are treated as the negative responses to construct the 1-vs-N training data.
The temperature is set as 1 to generate the responses.


% \newpage
% \section{\Ours with a stronger reward model}\label{apx:sec:stronger-rm}

% In Section \ref{sec:lrpo}, we show the results with LLM-Blender \citep{jiang2023llm} as the reward model to have a fair comparison with the baseline methods.
% In this section, we would like to show that \Ours can achieve even stronger performance with stronger off-the-shelf reward model \citep{dong2024rlhf}.
% The results are shown in Table \ref{apx:tab:main-performance}, where we can find that a stronger reward model can further improve the performance of \Ours.


% \begin{table*}[h]
%     \centering
%     \caption{Method evaluation on AlpacaEval 2 and MixEval. LC WR and WR denote length-controlled win rate and win rate respectively. Offline baseline performances on AlpacaEval 2 are from \citept{meng2024simpo} with LLM-Blender reward model \citep{jiang2023llm}.}\label{apx:tab:main-performance}
%     \scalebox{0.78}{
%     \begin{tabular}{lcccccccccc}
%         \toprule
%         Model & \multicolumn{4}{c}{Mistral-Base (7B)} & \multicolumn{4}{c}{Mistral-Instruct (7B)} \\
%         \cmidrule(lr){2-5} \cmidrule(lr){6-9}
%         & \multicolumn{2}{c}{Alpaca Eval 2}  & \multirow{1}{*}{MixEval} & \multirow{1}{*}{MixEval-Hard} & \multicolumn{2}{c}{Alpaca Eval 2}  & \multirow{1}{*}{MixEval} & \multirow{1}{*}{MixEval-Hard} \\
%         \cmidrule(lr){2-3} \cmidrule(lr){4-4} \cmidrule(lr){5-5}  \cmidrule(lr){6-7} \cmidrule(lr){8-8} \cmidrule(lr){9-9}
%         & LC WR & WR  & Score & Score & LC WR  & WR & Score & Score \\
%         \midrule
%         SFT    & 8.4  & 6.2   &  0.602  & 0.279  & 17.1 & 14.7  & 0.707 & 0.361 \\
%         RRHF   & 11.6 & 10.2   &  0.600  & 0.312  & 25.3 & 24.8  &   0.700    & 0.380 \\
%         DPO    & 15.1 & 12.5  &  0.686  &  0.341 & 26.8 & 24.9  & 0.702 & 0.355 \\
%         KTO    & 13.1 & 9.1    & \textbf{0.704}  & 0.351   & 24.5 & 23.6  &   0.692    & 0.358 \\
%         RDPO   & 17.4 & 12.8  & 0.693  & 0.355   & 27.3 & 24.5  &   0.695    & 0.364 \\
%         SimPO  & 21.5 & 20.8  &  0.672  &  0.347 & 32.1 & 34.8  & 0.702  & 0.363 \\
%         Iterative DPO  & 18.9  & 16.7  & 0.660   & 0.341  & 20.4 & 24.84  & 0.719  & 0.389 \\
%         \midrule
%         \multicolumn{9}{c}{Reward model: LLM-Blender \citep{jiang2023llm}}  \\
%         \midrule
%         \Ours (contrastive) & 31.6 & 30.8  &   0.703 & 0.409  & 32.7 & 38.6  &  0.718 & \textbf{0.418} \\
%         \Ours (LambdaRank) &  \textbf{34.9} & \textbf{37.2} & 0.695 &  \textbf{0.452}  & \textbf{32.9} & \textbf{38.9}   & \textbf{0.720} & 0.417  \\
%         \Ours (ListMLE) & 31.1  &  32.1   &  0.669  & 0.390  &  29.7 & 36.2    & 0.709  & 0.397 \\
%         \midrule
%         \multicolumn{9}{c}{Reward model: FsfairX \citep{dong2024rlhf}}  \\
%         \midrule
%         \Ours (contrastive) & \textbf{41.5} & \textbf{42.9} & 0.718 & 0.417    & \textbf{43.0}  & \textbf{53.8} & 0.718 & 0.425   \\
%         \Ours (LambdaRank) & 35.8 & 34.1 & 0.717 & 0.431   & 41.9  & 48.1 & \textbf{0.740} & \textbf{0.440}  \\
%         \Ours (ListMLE) & 36.6 & 37.8 & \textbf{0.730} & \textbf{0.423}   & 39.6  & 48.1 & 0.717 & 0.397   \\
%         \bottomrule
%     \end{tabular}}
% \end{table*}


\end{document}
