

\section{Main Results}\label{sec:main-result}

\paragraph{Baselines.}
We evaluate the performance of \Ours against a range of established preference optimization methods, encompassing both offline and online approaches.
Our offline comparison set includes RRHF \citep{yuan2023rrhf}, SLiC-HF \citep{zhao2023slic}, DPO \citep{guo2024direct}, IPO \citep{azar2024general}, CPO \citep{xu2024contrastive}, KTO \citep{ethayarajh2024kto}, RDPO \citep{park2024disentangling} and SimPO \citep{meng2024simpo}.
For online methods, we compare with iterative DPO \citep{xiong2024iterative}.
The baseline checkpoints are from \citep{meng2024simpo}.
Further details regarding these baselines and our experimental setup are provided in Appendix \ref{apx:sec:baselines}.
Both baselines and \Ours are trained on Ultrafeedback dataset \citep{cui2024ultrafeedback} for fair comparison.

\paragraph{Datasets.} We conduct evaluation on two widely used benchmarks AlpacaEval2 \citep{dubois2024length} and MixEval \citep{ni2024mixeval}.  
These benchmarks are designed to assess the conversational capabilities of models across a diverse range of queries. AlpacaEval2 comprises 805 questions sourced from five datasets, while MixEval includes 4000 general and 1000 hard questions.
Evaluation follows the established protocols for each benchmark. For AlpacaEval 2, we report both the raw win rate (WR) and the length-controlled win rate (LC). These benchmarks collectively provide a comprehensive assessment of the models' instruction-following and problem-solving capabilities.

\paragraph{Results.}
% We assess the performance of \Ours on two established benchmarks: AlpacaEval2 \citep{dubois2024length} and MixEval \citep{ni2024mixeval}. 
The baseline performances on AlpacaEval 2 are directly from \citet{meng2024simpo}, while the performances on MixEval is evaluated by ourselves with the opensourced checkpoints.
We adopt the same LLM-Blender \citep{jiang2023llm} reward model for a fair comparison with the baselines and also explore stronger reward model: FsfairX \citep{dong2024rlhf}.
The results, presented in Table \ref{tab:main-performance}, show that \Ours consistently outperforms the competitive baseline methods on both datasets, with 38.9 \% and 13.7 \% averaged relative improvements, on AlpacaEval2 and MixEval-Hard respectively, with the same reward model as the baselines.
With a stronger reward model, we can further improve \Ours by 25.8 \% on the challenging AlpacaEval2 dataset.
Additional details regarding our experimental setup are available in Appendix \ref{apx:sec:main}.


\sisetup{print-zero-integer=false}

\begin{table*}[t]
\centering
\small
\addtolength{\tabcolsep}{-1.2mm}
\begin{tabular}{
    @{}ll
    S[table-format = .3, table-space-text-post = $^{***}$] S[table-format = 0.3]
    S[table-format = .3, table-space-text-post = $^{***}$] S[table-format = 0.3]
    S[table-format = .3, table-space-text-post = $^{***}$] S[table-format = 0.3]
    S[table-format = 1.3, table-space-text-post = $^{***}$] S[table-format = 1.3]
    S[table-format = 1.3, table-space-text-post = $^{***}$] S[table-format = 1.3]
    S[table-format = 1.3, table-space-text-post = $^{***}$] S[table-format = 1.3]@{}
    % ccccccccccc@{}
}
% S[table-format = 2.3, table-space-text-post = $^{***}$]
\toprule
\multirow{3}{*}{\textbf{\begin{tabular}[c]{@{}l@{}}Text\\ Length\\ (\#word)\end{tabular}}} & \multirow{3}{*}{\textbf{Model}} & \multicolumn{6}{c}{\textbf{ACC} $\uparrow$} & \multicolumn{6}{c}{\textbf{MSE} $\downarrow$} \\ \cmidrule(lr){3-8} \cmidrule(lr){9-14}
 &  & \multicolumn{2}{c}{\textbf{structured}} & \multicolumn{2}{c}{\textbf{plain}} & \multicolumn{2}{c}{\textbf{shuffled}} & \multicolumn{2}{c}{\textbf{structured}} & \multicolumn{2}{c}{\textbf{plain}} & \multicolumn{2}{c}{\textbf{shuffled}} \\ \cmidrule(lr){3-4} \cmidrule(lr){5-6} \cmidrule(lr){7-8} \cmidrule(lr){9-10} \cmidrule(lr){11-12} \cmidrule(lr){13-14}
 &  & \textbf{tw} & \textbf{cn} & \textbf{tw} & \textbf{cn} & \textbf{tw} & \textbf{cn} & \textbf{tw} & \textbf{cn} & \textbf{tw} & \textbf{cn} & \textbf{tw} & \textbf{cn} \\ \midrule

 
\multirow{6}{*}{\textbf{\begin{tabular}[c]{@{}l@{}}Short\\ (1-49)\end{tabular}}} 
 & \textbf{GPT-4o} & \cellcolor{lightred}.265 & .274 & \cellcolor{lightred}.192$^{***}$ & .208 & \cellcolor{lightred}.186$^{***}$ & .202 & 3.563$^{*}$ & \cellcolor{lightred}3.769 & \cellcolor{lightred}4.092$^{***}$ & 3.385 & \cellcolor{lightred}4.347$^{***}$ & 3.561 \\
 % & \textbf{GPT-4o +T} & .291 & \cellcolor{lightred}.286 & \cellcolor{lightred}.213 & .217 & \cellcolor{lightred}.206 & .214 & \cellcolor{lightred}2.280 & 2.252 & \cellcolor{lightred}3.035$^{*}$ & 2.618 & \cellcolor{lightred}3.130$^{*}$ & 2.749 \\
 & \textbf{Llama3 8b} & .234$^{***}$ &\cellcolor{lightred}.216 & \cellcolor{lightred}.176 & .184 & \cellcolor{lightred}.162 & .169 & 3.402$^{***}$ & \cellcolor{lightred}3.931 & \cellcolor{lightred}3.059$^{***}$ & 2.851 & \cellcolor{lightred}3.220$^{**}$ & 3.071 \\
 & \textbf{Llama3 70b} & \cellcolor{lightred}.392 & .399 & \cellcolor{lightred}.297$^{**}$ & .310 & \cellcolor{lightred}.288$^{***}$ & .310 & \cellcolor{lightred}1.785$^{**}$ & 1.661 & \cellcolor{lightred}3.465$^{***}$ & 3.005 & \cellcolor{lightred}3.427$^{***}$ & 2.941 \\
 & \textbf{Llama3 405b} & \cellcolor{lightred}.386$^{***}$ & .412 & \cellcolor{lightred}.283$^{***}$ & .315 & \cellcolor{lightred}.290$^{***}$ & .322 & \cellcolor{lightred}2.605$^{**}$ & 2.430 & \cellcolor{lightred}4.365$^{***}$ & 3.598 & \cellcolor{lightred}4.355$^{***}$ & 3.612 \\
 & \textbf{Gemma2 9b} & .150$^{**}$ & \cellcolor{lightred}.138 & \cellcolor{lightred}.182$^{*}$ & .193 & \cellcolor{lightred}.171$^{**}$ & .184 & 2.987$^{***}$ & \cellcolor{lightred}3.168 & \cellcolor{lightred}4.089$^{***}$ & 3.367 & \cellcolor{lightred}4.462$^{***}$ & 3.646 \\
 & \textbf{Gemma2 27b} & .109 & .109 & \cellcolor{lightred}.122 & .124 & \cellcolor{lightred}.122 & .127 & \cellcolor{lightred}5.399$^{***}$ & 5.120 & \cellcolor{lightred}6.150$^{***}$ & 5.292 & \cellcolor{lightred}5.880$^{***}$ & 5.183 \\ \midrule

 
\multirow{6}{*}{\textbf{\begin{tabular}[c]{@{}l@{}}Long\\ (50+)\end{tabular}}} 
 & \textbf{GPT-4o} & \cellcolor{lightred}.356$^{***}$ & .384 & \cellcolor{lightred}.281$^{***}$ & .332 & \cellcolor{lightred}.267$^{***}$ & .314 & \cellcolor{lightred}1.846$^{***}$ & 1.577 & \cellcolor{lightred}1.834$^{***}$ & 1.570 & \cellcolor{lightred}2.070$^{***}$ & 1.743 \\
 % & \textbf{GPT-4o +T} & \cellcolor{lightred}.321$^{**}$ & .336 & \cellcolor{lightred}.262$^{*}$ & .283 & \cellcolor{lightred}.248$^{*}$ & .273 & \cellcolor{lightred}1.595$^{*}$ & 1.453 & \cellcolor{lightred}1.741$^{*}$ & 1.621 & \cellcolor{lightred}1.881$^{*}$ & 1.743 \\
 & \textbf{Llama3 8b} & \cellcolor{lightred}.224 & .228 & \cellcolor{lightred}.189$^{***}$ & .214 & \cellcolor{lightred}.179$^{***}$ & .200 & 1.895 & \cellcolor{lightred}1.901 & \cellcolor{lightred}2.024$^{***}$ & 1.835 & \cellcolor{lightred}2.111$^{***}$ & 1.933 \\
 & \textbf{Llama3 70b} & \cellcolor{lightred}.409$^{***}$ & .435 & \cellcolor{lightred}.373$^{*}$ & .391 & \cellcolor{lightred}.360$^{***}$ & .388 & \cellcolor{lightred}1.424$^{**}$ & 1.284 & \cellcolor{lightred}1.667$^{*}$ & 1.557 & \cellcolor{lightred}1.771$^{**}$ & 1.608 \\
 & \textbf{Llama3 405b} & \cellcolor{lightred}.380$^{***}$ & .410 & \cellcolor{lightred}.371$^{**}$ & .396 & \cellcolor{lightred}.357$^{***}$ & .384 & \cellcolor{lightred}1.866$^{***}$ & 1.627 & \cellcolor{lightred}1.906$^{**}$ & 1.719 & \cellcolor{lightred}2.052$^{***}$ & 1.844 \\
 & \textbf{Gemma2 9b} & \cellcolor{lightred}.142 & .151 & \cellcolor{lightred}.192$^{**}$ & .209 & \cellcolor{lightred}.174$^{*}$ & .188 & \cellcolor{lightred}2.515$^{***}$ & 2.350 & \cellcolor{lightred}2.144$^{***}$ & 1.979 & \cellcolor{lightred}2.399$^{***}$ & 2.193 \\
 & \textbf{Gemma2 27b} & .096 & \cellcolor{lightred}.092 & .077 & \cellcolor{lightred}.075 & \cellcolor{lightred}.079 & .083 & \cellcolor{lightred}5.135$^{**}$ & 4.919 & \cellcolor{lightred}5.913$^{**}$ & 5.578 & \cellcolor{lightred}5.606$^{***}$ & 5.160 \\ \midrule

 
\multirow{6}{*}{\textbf{Overall}} 
 & \textbf{GPT-4o} & \cellcolor{lightred}.296$^{***}$ & .312 & \cellcolor{lightred}.222$^{***}$ & .250 & \cellcolor{lightred}.213$^{***}$ & .240 & 2.978 & \cellcolor{lightred}3.022 & \cellcolor{lightred}3.323$^{***}$ & 2.767 & \cellcolor{lightred}3.571$^{***}$ & 2.942 \\
 % & \textbf{GPT-4o +T} & \cellcolor{lightred}.302 & .303 & \cellcolor{lightred}.231$^{**}$ & .240 & \cellcolor{lightred}.221$^{*}$ & .235 & \cellcolor{lightred}2.036$^{**}$ & 1.969 & \cellcolor{lightred}2.574$^{*}$ & 2.264 & \cellcolor{lightred}2.685$^{*}$ & 2.390 \\
 & \textbf{Llama3 8b} & .230$^{*}$ & \cellcolor{lightred}.221 & \cellcolor{lightred}.181$^{***}$ & .195 & \cellcolor{lightred}.168$^{***}$ & .181 & 2.838$^{***}$ & \cellcolor{lightred}3.170 & \cellcolor{lightred}2.672$^{***}$ & 2.470 & \cellcolor{lightred}2.806$^{***}$ & 2.644 \\
 & \textbf{Llama3 70b} & \cellcolor{lightred}.398$^{**}$ & .412 & \cellcolor{lightred}.324$^{***}$ & .339 & \cellcolor{lightred}.313$^{***}$ & .337 & \cellcolor{lightred}1.658$^{***}$ & 1.529 & \cellcolor{lightred}2.835$^{***}$ & 2.496 & \cellcolor{lightred}2.847$^{***}$ & 2.473 \\
 & \textbf{Llama3 405b} & \cellcolor{lightred}.384$^{***}$ & .411 & \cellcolor{lightred}.314$^{***}$ & .344 & \cellcolor{lightred}.313$^{***}$ & .344 & \cellcolor{lightred}2.347$^{***}$ & 2.148 & \cellcolor{lightred}3.505$^{***}$ & 2.939 & \cellcolor{lightred}3.550$^{***}$ & 2.992 \\
 & \textbf{Gemma2 9b} & .147 & \cellcolor{lightred}.143 & \cellcolor{lightred}.185$^{***}$ & .199 & \cellcolor{lightred}.172$^{***}$ & .186 & 2.823 & \cellcolor{lightred}2.882 & \cellcolor{lightred}3.410$^{***}$ & 2.881 & \cellcolor{lightred}3.742$^{***}$ & 3.137 \\
 & \textbf{Gemma2 27b} & .105 & \cellcolor{lightred}.103 & \cellcolor{lightred}.106 & .107 & \cellcolor{lightred}.107 & .111 & \cellcolor{lightred}5.307$^{***}$ & 5.049 & \cellcolor{lightred}6.067$^{***}$ & 5.392 & \cellcolor{lightred}5.785$^{***}$ & 5.175 \\ \bottomrule
 %\multicolumn{14}{l}{\textit{Statistical group differences are indicated as ${^{*}}$  (p<.05), ${^{**}}$ (p<.01), and ${^{***}}$ (p<.001) regarding the model performance.}}
\end{tabular}

\addtolength{\tabcolsep}{+1.2mm}
\vspace{-.5pc}
\caption{Results by length for GPT-4o, Llama3 (8b, 70b, 405b), and Gemma2 (9b, 27b) models. Red cells indicate a worse performance than the other group. (Statistical group differences are indicated as ${^{*}}$  (p<.05), ${^{**}}$ (p<.01), and ${^{***}}$ (p<.001) regarding the model performance.)\kenneth{TODO: (1) This table needs redo (2) Update the numbers}}
\vspace{-1pc}
\label{tab:results-by-length}
\end{table*}


%%%%%%%%%%%%%%%%%%%%%%%%%%%%%%%%%%%%%%%%%%%%%%%%%%%%%%%%%%%%%%%%%%%%%%%%%%%%%%%%%%%%%%%%%%%%%%%%%%%%%%%%%%%%%%%%%%%%%%%%%%%%%%%%%%%%%%%%

% Please add the following required packages to your document preamble:
% \usepackage{booktabs}
% \usepackage{multirow}

\iffalse
\begin{table*}[t]
\centering
\small
\addtolength{\tabcolsep}{-0.8mm}
\begin{tabular}{@{}llcccccccccccc@{}}
\toprule
\multirow{3}{*}{\textbf{\begin{tabular}[c]{@{}l@{}}Text\\ Length\\ (\#word)\end{tabular}}} & \multirow{3}{*}{\textbf{Model}} & \multicolumn{6}{c}{\textbf{ACC} $\uparrow$} & \multicolumn{6}{c}{\textbf{MSE} $\downarrow$} \\ \cmidrule(lr){3-8} \cmidrule(lr){9-14}
 &  & \multicolumn{2}{c}{\textbf{structured}} & \multicolumn{2}{c}{\textbf{plain}} & \multicolumn{2}{c}{\textbf{shuffled}} & \multicolumn{2}{c}{\textbf{structured}} & \multicolumn{2}{c}{\textbf{plain}} & \multicolumn{2}{c}{\textbf{shuffled}} \\ \cmidrule(lr){3-4} \cmidrule(lr){5-6} \cmidrule(lr){7-8} \cmidrule(lr){9-10} \cmidrule(lr){11-12} \cmidrule(lr){13-14}
 &  & \textbf{tw} & \textbf{cn} & \textbf{tw} & \textbf{cn} & \textbf{tw} & \textbf{cn} & \textbf{tw} & \textbf{cn} & \textbf{tw} & \textbf{cn} & \textbf{tw} & \textbf{cn} \\ \midrule
\multirow{6}{*}{\textbf{\begin{tabular}[c]{@{}l@{}}Short\\ (1-49)\end{tabular}}} 
 & \textbf{GPT-4o} & \cellcolor{lightred}0.265 & 0.274 & 0.192 & \cellcolor{lightblue}0.208 & 0.186 & 0.202 & 3.563 & 3.769 & \cellcolor{lightgreen}4.092 & 3.856 & 4.347 & 3.561 \\
 & \textbf{Llama3 8b} & 0.234 & 0.216 & \textbf{0.176} & 0.184 & \textbf{0.162} &0.169. & \textbf{3.402} & 3.931 & \textbf{3.059} & 2.851 & \textbf{3.220} & 3.071 \\
 & \textbf{Llama3 70b} & 0.392 & 0.399 & 0.297 &0.310 & 0.288 & 0.310 & 1.785 & 1.661 & 3.465 & 3.005 & 3.427 & 2.941 \\
 & \textbf{Llama3 405b} & \textbf{0.386} & 0.412 & \textbf{0.283} & 0.315 & \textbf{0.290} & 0.322 & \textbf{2.605} & 2.430 & \textbf{4.365} & 3.598 & \textbf{4.355} & 3.612 \\
 & \textbf{Gemma2 9b} &0.150 & \textbf{0.138} &\textbf{0.182} & 0.193 & \textbf{0.171} & 0.184 & 2.987 & \textbf{3.168} & \textbf{4.089} & 3.367 & \textbf{4.462} & 3.646 \\
 & \textbf{Gemma2 27b} & 0.109 & 0.109 & 0.122 & 0.124 & 0.122 & 0.127 & \textbf{5.399} & 5.120 & \textbf{6.150} & 5.292 & \textbf{5.880} & 5.183 \\ \midrule
\multirow{6}{*}{\textbf{\begin{tabular}[c]{@{}l@{}}Long\\ (50+)\end{tabular}}} 
 & \textbf{GPT-4o} & 0.356 & \cellcolor{lightyellow}0.384 & 0.281 & 0.332 & \cellcolor{lightgreen}0.267 & 0.314 & 1.846 & 1.577 & 1.834 & \cellcolor{lightred}1.570 & 2.070 & 1.743 \\
 & \textbf{Llama3 8b} & 0.224 & 0.228 & 0.189 & 0.214 & 0.179 & 0.200 & 1.895 & 1.901 & 2.024 & 1.835 & 2.111 & 1.933 \\
 & \textbf{Llama3 70b} & \textbf{0.409} & 0.435 & \textbf{0.373} & 0.391 & \textbf{0.360} & 0.388 & \textbf{1.424} & 1.284 & \textbf{1.667} & 1.557 & \textbf{1.771} & 1.608 \\
 & \textbf{Llama3 405b} & \textbf{0.380} & 0.410 & \textbf{0.371} & 0.396 & \textbf{0.357} & 0.384 & \textbf{1.866} & 1.627 & \textbf{1.906} & 1.719 & \textbf{2.052} & 1.844 \\
 & \textbf{Gemma2 9b} & 0.142 & 0.151 & \textbf{0.192} & 0.209 & \textbf{0.174} & 0.188 & \textbf{2.515} & 2.350 & \textbf{2.144} & 1.979 & \textbf{2.399} & 2.193 \\
 & \textbf{Gemma2 27b} & 0.096 & 0.092 & 0.077 & 0.075 & 0.079 & 0.083 & \textbf{5.135} & 4.919 & \textbf{5.913} & 5.578 & \textbf{5.606} & 5.160 \\ \midrule
\multirow{6}{*}{\textbf{Overall}} 
 & \textbf{GPT-4o} & 0.296 & 0.312 & \cellcolor{lightblue}0.222 & 0.250 & 0.213 & \cellcolor{lightyellow}0.240 & 2.978 & 3.022 & 3.323 & 2.767 & \cellcolor{lightred}3.571 & 2.942 \\
 & \textbf{Llama3 8b} &0.230 & 0.221 & 0.181 & 0.195 & 0.168 & 0.181 & 2.838 & 3.170 &2.672 & 2.470 & 2.806 & 2.644 \\
 & \textbf{Llama3 70b} & \textbf{0.398} & 0.412 & \textbf{0.324} & 0.339 & \textbf{0.313} & 0.337 & \textbf{1.658} & 1.529 & \textbf{2.835} & 2.496 & \textbf{2.847} & 2.473 \\
 & \textbf{Llama3 405b} & \textbf{0.384} & 0.411 & \textbf{0.314} & 0.344 & \textbf{0.313} & 0.344 & \textbf{2.347} & 2.148 & \textbf{3.505} & 2.939 & \textbf{3.550} & 2.992 \\
 & \textbf{Gemma2 9b} & 0.147 & 0.143 & \textbf{0.185} & 0.199 & \textbf{0.172} & 0.186 & 2.823 & 2.882 & \textbf{3.410} & 2.881 & \textbf{3.742} & 3.137 \\
 & \textbf{Gemma2 27b} & 0.105 & 0.103 & 0.106 & 0.107 & 0.107 & 0.111 & \textbf{5.307} & 5.049 & \textbf{6.067} & 5.392 & \textbf{5.785} & 5.175 \\ \bottomrule

\end{tabular}
\addtolength{\tabcolsep}{+0.8mm}
\caption{Results by length for GPT-4o, Llama3 (8b, 70b, 405b), and Gemma2 (9b, 27b) models. Bold numbers indicate a significant difference between the groups. \cy{I would prefer to us * to represent the significance. The number of stars is also important to report.}}
\label{tab:results-by-length}
\end{table*}
\fi

% 
\iffalse
\begin{table*}[t]
\centering
\small
\addtolength{\tabcolsep}{-0.5mm}
\begin{tabular}{@{}llcccccccccccc@{}}
\toprule
\multirow{3}{*}{\textbf{\begin{tabular}[c]{@{}l@{}}Text\\ Length\\ (\#word)\end{tabular}}} & \multirow{3}{*}{\textbf{Model}} & \multicolumn{6}{c}{\textbf{ACC}} & \multicolumn{6}{c}{\textbf{MSE}} \\ \cmidrule(lr){3-8} \cmidrule(lr){9-14}
 &  & \multicolumn{2}{c}{\textbf{structured}} & \multicolumn{2}{c}{\textbf{plain}} & \multicolumn{2}{c}{\textbf{shuffled}} & \multicolumn{2}{c}{\textbf{structured}} & \multicolumn{2}{c}{\textbf{plain}} & \multicolumn{2}{c}{\textbf{shuffled}} \\ \cmidrule(lr){3-4} \cmidrule(lr){5-6} \cmidrule(lr){7-8} \cmidrule(lr){9-10} \cmidrule(lr){11-12} \cmidrule(lr){13-14}
 &  & \textbf{tw} & \textbf{cn} & \textbf{tw} & \textbf{cn} & \textbf{tw} & \textbf{cn} & \textbf{tw} & \textbf{cn} & \textbf{tw} & \textbf{cn} & \textbf{tw} & \textbf{cn} \\ \midrule
\multirow{6}{*}{\textbf{\begin{tabular}[c]{@{}l@{}}Short\\ (1-50)\end{tabular}}} 
 & \textbf{GPT-4o} & 0.265 & 0.274 & 0.192 & 0.208 & 0.186 & 0.202 & 3.563 & 3.769 & 4.092 & 3.856 & 4.347 & 3.561 \\
 & \textbf{Llama3 8b} & . & . & . & . & . & . & . & . & . & . & . & . \\
 & \textbf{Llama3 70b} & . & . & . & . & . & . & . & . & . & . & . & . \\
 & \textbf{Llama3 405b} & . & . & . & . & . & . & . & . & . & . & . & . \\
 & \textbf{Gemma2 9b} & . & . & . & . & . & . & . & . & . & . & . & . \\
 & \textbf{Gemma2 27b} & . & . & . & . & . & . & . & . & . & . & . & . \\ \midrule
\multirow{6}{*}{\textbf{\begin{tabular}[c]{@{}l@{}}Long\\ (51+)\end{tabular}}} 
 & \textbf{GPT-4o} & 0.356 & 0.384 & 0.281 & 0.332 & 0.267 & 0.314 & 1.846 & 1.577 & 1.834 & 1.570 & 2.070 & 1.743 \\
 & \textbf{Llama3 8b} & . & . & . & . & . & . & . & . & . & . & . & . \\
 & \textbf{Llama3 70b} & . & . & . & . & . & . & . & . & . & . & . & . \\
 & \textbf{Llama3 405b} & . & . & . & . & . & . & . & . & . & . & . & . \\
 & \textbf{Gemma2 9b} & . & . & . & . & . & . & . & . & . & . & . & . \\
 & \textbf{Gemma2 27b} & . & . & . & . & . & . & . & . & . & . & . & . \\ \midrule
\multirow{6}{*}{\textbf{Overall}} 
 & \textbf{GPT-4o} & 0.296 & 0.312 & 0.222 & 0.250 & 0.213 & 0.240 & 2.978 & 3.022 & 3.323 & 2.767 & 3.571 & 2.942 \\
 & \textbf{Llama3 8b} & . & . & . & . & . & . & . & . & . & . & . & . \\
 & \textbf{Llama3 70b} & . & . & . & . & . & . & . & . & . & . & . & . \\
 & \textbf{Llama3 405b} & . & . & . & . & . & . & . & . & . & . & . & . \\
 & \textbf{Gemma2 9b} & . & . & . & . & . & . & . & . & . & . & . & . \\
 & \textbf{Gemma2 27b} & . & . & . & . & . & . & . & . & . & . & . & . \\ \bottomrule
\end{tabular}
\addtolength{\tabcolsep}{+0.5mm}
\caption{Results by length for GPT-4o, Llama3 (8b, 70b, 405b), and Gemma2 (9b, 27b) models.}
\label{tab:results-by-length}
\end{table*}
\fi

\begin{comment}
% Please add the following required packages to your document preamble:
% \usepackage{booktabs}
% \usepackage{multirow}
\begin{table*}[t]
\centering
\footnotesize
\begin{tabular}{@{}llcccccccccccc@{}}
\toprule
\multirow{3}{*}{\textbf{\begin{tabular}[c]{@{}l@{}}Text\\ Length\\ (\#word)\end{tabular}}} & \multirow{3}{*}{\textbf{Model}} & \multicolumn{6}{c}{\textbf{ACC}} & \multicolumn{6}{c}{\textbf{MSE}} \\ \cmidrule(l){3-14} 
 &  & \multicolumn{2}{c}{\textbf{structured}} & \multicolumn{2}{c}{\textbf{plain}} & \multicolumn{2}{c}{\textbf{shuffled}} & \multicolumn{2}{c}{\textbf{structured}} & \multicolumn{2}{c}{\textbf{plain}} & \multicolumn{2}{c}{\textbf{shuffled}} \\ \cmidrule(l){3-14} 
 &  & \textbf{tw} & \textbf{cn} & \textbf{tw} & \textbf{cn} & \textbf{tw} & \textbf{cn} & \textbf{tw} & \textbf{cn} & \textbf{tw} & \textbf{cn} & \textbf{tw} & \textbf{cn} \\ \midrule
\multirow{3}{*}{\textbf{\begin{tabular}[c]{@{}l@{}}Short\\ (1-49)\end{tabular}}} & \textbf{GPT4o} & .265 & .274 & .192 & .208 & .186 & .202 & 3.563 & 3.769 & 4.092 & 3.856 & 4.347 & 3.561 \\
 & \textbf{Llama3} & .MMM & .NNN & .OOO & .PPP & .QQQ & .RRR & .MMM & .NNN & .OOO & .PPP & .QQQ & .RRR \\
 & \textbf{Gemma2} & .SSS & .TTT & .UUU & .VVV & .XXX & .YYY & .SSS & .TTT & .UUU & .VVV & .XXX & .YYY \\ \midrule
\multirow{3}{*}{\textbf{\begin{tabular}[c]{@{}l@{}}Long\\ (50+)\end{tabular}}} & \textbf{GPT} & .356 & .384 & .281 & .332 & .267 & .314 & 1.846 & 1.577 & 1.834 & 1.570 & 2.070 & 1.743 \\
 & \textbf{Llama} & .MMM & .NNN & .OOO & .PPP & .QQQ & .RRR & .MMM & .NNN & .OOO & .PPP & .QQQ & .RRR \\
 & \textbf{Gemma2} & .SSS & .TTT & .UUU & .VVV & .XXX & .YYY & .SSS & .TTT & .UUU & .VVV & .XXX & .YYY \\ \midrule
\multirow{3}{*}{\textbf{Overall}} & \textbf{GPT} & .296 & .312 & .222 & .250 & .213 & .240 & 2.978 & 3.022 & 3.323 & 2.767 & 3.571 & 2.942 \\
 & \textbf{Llama} & .MMM & .NNN & .OOO & .PPP & .QQQ & .RRR & .MMM & .NNN & .OOO & .PPP & .QQQ & .RRR \\
 & \textbf{Gemma2} & .SSS & .TTT & .UUU & .VVV & .XXX & .YYY & .SSS & .TTT & .UUU & .VVV & .XXX & .YYY \\ \bottomrule
\end{tabular}
\caption{Results by length.\kenneth{TODO: List (1) GPT-4o, (2) Llama3 8b, (3) Llama3 70b, (4) Gemma2 9b, (5) Llama3 405b, (6) Gemma2 27b}}
\label{tab:results-by-length}
\end{table*}
\end{comment}

\begin{comment}
% Please add the following required packages to your document preamble:
% \usepackage{booktabs}
% \usepackage{multirow}
\begin{table*}[t]
\centering
%\small

\iffalse
\begin{tabular}{@{}llcccc@{}}
\toprule
\multirow{2}{*}{\textbf{\begin{tabular}[c]{@{}l@{}}Text\\ Length\end{tabular}}} & \multirow{2}{*}{\textbf{Model}} & \multicolumn{2}{c}{\textbf{ACC}} & \multicolumn{2}{c}{\textbf{MSE}} \\ \cmidrule(l){3-6} 
 &  & \textbf{tw} & \textbf{ch} & \textbf{tw} & \textbf{ch} \\ \midrule
\multirow{2}{*}{\textbf{\begin{tabular}[c]{@{}l@{}}Short (0-50 words)\\ n=A,AA\end{tabular}}} & \textbf{GPT-4o} & .264, .191, .185 & .273, .206, .200 & E.EE & F.FF \\
 & \textbf{LLaMA 3.1} & .AAA & .BBB & E.EE & F.FF \\ \midrule
\multirow{2}{*}{\textbf{\begin{tabular}[c]{@{}l@{}}Medium (50-270 words)\\ n=B,BB\end{tabular}}} & \textbf{GPT-4o} & .355, .280, .266 & .383, .331, .313 & E.EE & F.FF \\
 & \textbf{LLaMA 3.1} & .AAA & .BBB & E.EE & F.FF \\ \midrule
\multirow{2}{*}{\textbf{\begin{tabular}[c]{@{}l@{}}Long (270+ words)\\ n=C,CC\end{tabular}}} & \textbf{GPT-4o} & .002, .001, .001 & .001, .001, .001 & E.EE & F.FF \\
 & \textbf{LLaMA 3.1} & .AAA & .BBB & E.EE & F.FF \\ \midrule
\multirow{2}{*}{\textbf{\begin{tabular}[c]{@{}l@{}}Overall\\ n=D,DD\end{tabular}}} & \textbf{GPT-4o} & .296, .222, .213 & .312, .250, .240 & E.EE & F.FF \\
 & \textbf{LLaMA 3.1} & .AAA & .BBB & E.EE & F.FF \\ \bottomrule
\end{tabular}
\fi

\end{table*}
\end{comment}



\begin{comment}



% Please add the following required packages to your document preamble:
% \usepackage{booktabs}
% \usepackage{multirow}
\begin{table*}[t]
\centering
\small
\begin{tabular}{@{}llcccccccc@{}}
\toprule
\multirow{2}{*}{\textbf{\begin{tabular}[c]{@{}l@{}}Text\\ Length\end{tabular}}} & \multirow{2}{*}{\textbf{LLMs}} & \multicolumn{4}{c}{\textbf{ACC}} & \multicolumn{4}{c}{\textbf{MSE}} \\ \cmidrule(l){3-10} 
 &  & \textbf{tw} & \textbf{ch} & \textbf{tw-\textgreater{}ch} & \textbf{ch-\textgreater{}tw} & \textbf{tw} & \textbf{ch} & \textbf{tw-\textgreater{}ch} & \textbf{ch-\textgreater{}tw} \\ \midrule
\multirow{3}{*}{\textbf{\begin{tabular}[c]{@{}l@{}}Short\\ (0-50 words)\\ n=A,AA\end{tabular}}} & \textbf{GPT-4o} & .AAA & .BBB & .CCC & .DDD & E.EE & F.FF & G.GG & H.HH \\
 & \textbf{OpenAI o1} & .AAA & .BBB & .CCC & .DDD & E.EE & F.FF & G.GG & H.HH \\
 & \textbf{LLaMA 3.1} & .AAA & .BBB & .CCC & .DDD & E.EE & F.FF & G.GG & H.HH \\ \midrule
\multirow{3}{*}{\textbf{\begin{tabular}[c]{@{}l@{}}Medium\\ (50-300 words)\\ n=B,BB\end{tabular}}} & \textbf{GPT-4o} & .AAA & .BBB & .CCC & .DDD & E.EE & F.FF & G.GG & H.HH \\
 & \textbf{OpenAI o1} & .AAA & .BBB & .CCC & .DDD & E.EE & F.FF & G.GG & H.HH \\
 & \textbf{LLaMA 3.1} & .AAA & .BBB & .CCC & .DDD & E.EE & F.FF & G.GG & H.HH \\ \midrule
\multirow{3}{*}{\textbf{\begin{tabular}[c]{@{}l@{}}Long\\ (300+ words)\\ n=C,CC\end{tabular}}} & \textbf{GPT-4o} & .AAA & .BBB & .CCC & .DDD & E.EE & F.FF & G.GG & H.HH \\
 & \textbf{OpenAI o1} & .AAA & .BBB & .CCC & .DDD & E.EE & F.FF & G.GG & H.HH \\
 & \textbf{LLaMA 3.1} & .AAA & .BBB & .CCC & .DDD & E.EE & F.FF & G.GG & H.HH \\ \midrule
\multirow{3}{*}{\textbf{\begin{tabular}[c]{@{}l@{}}Overall\\ n=D,DD\end{tabular}}} & \textbf{GPT-4o} & .AAA & .BBB & .CCC & .DDD & E.EE & F.FF & G.GG & H.HH \\
 & \textbf{OpenAI o1} & .AAA & .BBB & .CCC & .DDD & E.EE & F.FF & G.GG & H.HH \\
 & \textbf{LLaMA 3.1} & .AAA & .BBB & .CCC & .DDD & E.EE & F.FF & G.GG & H.HH \\ \bottomrule
\end{tabular}
\caption{Results by length.}
\label{tab:results-by-length}
\end{table*}
    
\end{comment}

