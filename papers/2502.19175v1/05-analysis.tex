\vspace{-0.5em}
\label{analysis}
\paragraph{Fixed vs. Dynamic Iterations.}
A key feature of MEDDxAgent is its iterative DDx process, which operates in \textit{fixed} or \textit{dynamic} iteration. Our experiments (see~\autoref{tab:choice_merged} in Appendix) show that fixed iteration consistently outperforms dynamic iteration in both accuracy and system efficiency.\footnote{On average, fixed iteration is 1.2x-1.7x faster than dynamic iteration.} Fixed iteration ensures a structured sequence where all modules -- history taking simulator, knowledge retrieval agent, and diagnosis strategy agent, are utilized in each cycle, preventing over-reliance on a single component. In contrast, dynamic iteration, which allows DDxDriver to choose the component at each step, introduces some suboptimal decision-making. We observe that Llama3.1 models, for instance, frequently favor the history taking simulator rather than leveraging the knowledge retrieval or diagnosis strategy agents, leading to redundant questioning rather than efficient diagnostic reasoning. Despite this, our findings demonstrate the general applicability of MEDDxAgent for dynamic iteration and highlight future work toward optimizing dynamic iteration for interactive DDx.

\paragraph{Error Analysis.}
To better understand the struggles of MEDDxAgent, we conduct error analysis on cases where it failed to reach the correct diagnosis efficiently. We emphasize that our MEDDxAgent's logging of intermediate logic greatly enhances our understanding and explanations of failure cases. First, in RareBench, over-reliance on few-shot examples often misprioritizes frequent conditions over rarer diseases, as some rare conditions are underrepresented in knowledge retrieval databases. Second, in the first iteration, MEDDxAgent tends to prioritize the knowledge retrieval while overlooking few-shot patient examples with similar profiles. This is then mitigated when further iterations help to refine and enhance the DDx process.
Third, larger models (GPT-4o, Llama3.1-70B) benefit more from iterative refinement, while smaller models (Llama3.1-8B) plateau after the second iteration, especially for iCraft-MD and RareBench. Addressing these challenges can further enhance MEDDxAgent’s diagnostic accuracy.