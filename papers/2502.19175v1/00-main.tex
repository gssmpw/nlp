% This must be in the first 5 lines to tell arXiv to use pdfLaTeX, which is strongly recommended.
\pdfoutput=1
% In particular, the hyperref package requires pdfLaTeX in order to break URLs across lines.

\documentclass[11pt]{article}

% Change "review" to "final" to generate the final (sometimes called camera-ready) version.
% Change to "preprint" to generate a non-anonymous version with page numbers.
%\usepackage[review]{acl}
\usepackage[]{acl}

% Standard package includes
\usepackage{times}
\usepackage{latexsym}

% For proper rendering and hyphenation of words containing Latin characters (including in bib files)
\usepackage[T1]{fontenc}

% For Vietnamese characters
% \usepackage[T5]{fontenc}
% See https://www.latex-project.org/help/documentation/encguide.pdf for other character sets

% This assumes your files are encoded as UTF8
\usepackage[utf8]{inputenc}

% This is not strictly necessary, and may be commented out,
% but it will improve the layout of the manuscript,
% and will typically save some space.
\usepackage{microtype}

% This is also not strictly necessary, and may be commented out.
% However, it will improve the aesthetics of text in
% the typewriter font.
\usepackage{inconsolata}
\usepackage{dsfont}
\usepackage{amsmath}

%Including images in your LaTeX document requires adding
%additional package(s)
\usepackage{graphicx}
\usepackage{amssymb}

% If the title and author information does not fit in the area allocated, uncomment the following
%
\setlength\titlebox{6cm}
%
% and set <dim> to something 5cm or larger.

%Extra packages
\usepackage{multirow}
\usepackage{booktabs}
\usepackage{xcolor}
\usepackage{subcaption}
\definecolor{lightblue}{RGB}{0,127,255}
\definecolor{yellow-history}{RGB}{234,216,91}
\definecolor{green-diagnosis}{RGB}{86, 181, 170}
\definecolor{blue-retrieval}{RGB}{116, 167, 226}
\newcommand{\danny}[1]{\textbf{\small {\color{lightblue}[Danny: #1]}}}
\newcommand{\cc}[1]{\textbf{\small {\color{teal}[ChiaChien: #1]}}}
\newcommand{\marco}[1]{\textbf{\small {\color{purple}[Marco: #1]}}}
\newcommand{\cl}[1]{\textbf{\small {\color{red}[CL: #1]}}}
\newcommand{\isra}[1]{\textcolor{black!55!red}{Israa: #1}}
\def\sectionautorefname{§}
\def\subsectionautorefname{§}
\usepackage{enumitem}

\usepackage[most]{tcolorbox}

% Define a custom box style for AI prompts
% Define a custom style for system prompts


\newtcolorbox{prompt}[2][]{
    colback=black!3,
    colframe=black,
    boxrule=0.5pt,
    left=8pt,
    right=8pt,
    top=8pt,
    bottom=8pt,
    arc=2pt,
    breakable,
    enhanced,
    before skip=10pt,
    after skip=10pt,
    title={#2}, % Title argument
    #1 % Allows passing additional optional arguments
}

\newtcolorbox{example}[2][]{
    colback=black!3,
    colframe=black!20,
    boxrule=0.5pt,
    left=8pt,
    right=8pt,
    top=8pt,
    bottom=8pt,
    arc=2pt,
    breakable,
    enhanced,
    before skip=10pt,
    after skip=10pt,
    coltitle=black,
    title={#2}, % Title argument
    #1 % Allows passing additional optional arguments
}

% \title{DDxDriver: A Unified Modular Agent Framework for\\ Explainable Automatic Differential Diagnosis}
\title{MEDDxAgent: A Unified Modular Agent Framework for\\ Explainable Automatic Differential Diagnosis}


\author{Daniel Rose\textsuperscript{1}, Chia-Chien Hung\textsuperscript{2}, Marco Lepri\textsuperscript{2}, \\
\textbf{Israa Alqassem\textsuperscript{2}}, \textbf{Kiril Gashteovski\textsuperscript{2, 3}} and \textbf{Carolin Lawrence\textsuperscript{2}} \\
\textsuperscript{1}University of California, Santa Barbara \\
\textsuperscript{2}NEC Laboratories Europe, Heidelberg, Germany\\
\textsuperscript{3}CAIR, Ss. Cyril and Methodius University of Skopje, North Macedonia \\
  \texttt{danielrose@ucsb.edu}\\
  \texttt{\{Chia-Chien.Hung, Marco.Lepri,} \\
  \texttt{Israa.Alqassem, Kiril.Gashteovski, Carolin.Lawrence\}@neclab.eu}\\
  }
  

\begin{document}
\maketitle
\begin{abstract}

Differential Diagnosis (DDx) is a fundamental yet complex aspect of clinical decision-making, in which physicians iteratively refine a ranked list of possible diseases based on symptoms, antecedents, and medical knowledge. While recent advances in large language models have shown promise in supporting DDx, existing approaches face key limitations, including single-dataset evaluations, isolated optimization of components, unrealistic assumptions about complete patient profiles, and single-attempt diagnosis. We introduce a \textbf{M}odular \textbf{E}xplainable \textbf{DDx} \textbf{Agent} (\textbf{MEDDxAgent}) framework designed for interactive DDx, where diagnostic reasoning evolves through \textit{iterative learning}, rather than assuming a complete patient profile is accessible. MEDDxAgent integrates three modular components: (1) an orchestrator (DDxDriver), (2) a history taking simulator, and (3) two specialized agents for knowledge retrieval and diagnosis strategy. To ensure robust evaluation, we introduce a comprehensive DDx benchmark covering respiratory, skin, and rare diseases.  We analyze single-turn diagnostic approaches and demonstrate the importance of iterative refinement when patient profiles are not available at the outset. Our broad evaluation demonstrates that MEDDxAgent achieves over 10\% accuracy improvements in interactive DDx across both large and small LLMs, while offering critical explainability into its diagnostic reasoning process.  


\end{abstract}


% \isra{Highlight main selling points, e.g., Our agentic framework is modular, allowing seamless integration of different agents based on specific needs. Its primary advantage lies in its ability to iteratively refine diagnoses over time, adapting to new information. Additionally, it operates in a realistic setting, effectively handling missing or incomplete data. The framework is designed to address various disease including rare ones and has been benchmarked across multiple datasets under various settings, demonstrating its robustness and adaptability.}

\section{Introduction}

% humans are sensitive to the way information is presented.

% introduce framing as the way we address framing. say something about political views and how information is represented.

% in this paper we explore if models show similar sensitivity.

% why is it important/interesting.



% thought - it would be interesting to test it on real world data, but it would be hard to test humans because they come already biased about real world stuff, so we tested artificial.


% LLMs have recently been shown to mimic cognitive biases, typically associated with human behavior~\citep{ malberg2024comprehensive, itzhak-etal-2024-instructed}. This resemblance has significant implications for how we perceive these models and what we can expect from them in real-world interactions and decisionmaking~\citep{eigner2024determinants, echterhoff-etal-2024-cognitive}.

The \textit{framing effect} is a well-known cognitive phenomenon, where different presentations of the same underlying facts affect human perception towards them~\citep{tversky1981framing}.
For example, presenting an economic policy as only creating 50,000 new jobs, versus also reporting that it would cost 2B USD, can dramatically shift public opinion~\cite{sniderman2004structure}. 
%%%%%%%% 图1:  %%%%%%%%%%%%%%%%
\begin{figure}[t]
    \centering
    \includegraphics[width=\columnwidth]{Figs/01.pdf}
    \caption{Performance comparison (Top-1 Acc (\%)) under various open-vocabulary evaluation settings where the video learners except for CLIP are tuned on Kinetics-400~\cite{k400} with frozen text encoders. The satisfying in-context generalizability on UCF101~\cite{UCF101} (a) can be severely affected by static bias when evaluating on out-of-context SCUBA-UCF101~\cite{li2023mitigating} (b) by replacing the video background with other images.}
    \label{fig:teaser}
\end{figure}


Previous research has shown that LLMs exhibit various cognitive biases, including the framing effect~\cite{lore2024strategic,shaikh2024cbeval,malberg2024comprehensive,echterhoff-etal-2024-cognitive}. However, these either rely on synthetic datasets or evaluate LLMs on different data from what humans were tested on. In addition, comparisons between models and humans typically treat human performance as a baseline rather than comparing patterns in human behavior. 
% \gabis{looks good! what do we mean by ``most studies'' or ``rarely'' can we remove those? or we want to say that we don't know of previous work doing both at the same time?}\gili{yeah the main point is that some work has done each separated, but not all of it together. how about now?}

In this work, we evaluate LLMs on real-world data. Rather than measuring model performance in terms of accuracy, we analyze how closely their responses align with human annotations. Furthermore, while previous studies have examined the effect of framing on decision making, we extend this analysis to sentiment analysis, as sentiment perception plays a key explanatory role in decision-making \cite{lerner2015emotion}. 
%Based on this, we argue that examining sentiment shifts in response to reframing can provide deeper insights into the framing effect. \gabis{I don't understand this last claim. Maybe remove and just say we extend to sentiment analysis?}

% Understanding how LLMs respond to framing is crucial, as they are increasingly integrated into real-world applications~\citep{gan2024application, hurlin2024fairness}.
% In some applications, e.g., in virtual companions, framing can be harnessed to produce human-like behavior leading to better engagement.
% In contrast, in other applications, such as financial or legal advice, mitigating the effect of framing can lead to less biased decisions.
% In both cases, a better understanding of the framing effect on LLMs can help develop strategies to mitigate its negative impacts,
% while utilizing its positive aspects. \gabis{$\leftarrow$ reading this again, maybe this isn't the right place for this paragraph. Consider putting in the conclusion? I think that after we said that people have worked on it, we don't necessarily need this here and will shorten the long intro}


% If framing can influence their outputs, this could have significant societal effects,
% from spreading biases in automated decision-making~\citep{ghasemaghaei2024understanding} to reducing public trust in AI-generated content~\citep{afroogh2024trust}. 
% However, framing is not inherently negative -- understanding how it affects LLM outputs can offer valuable insights into both human and machine cognition.
% By systematically investigating the framing effect,


%It is therefore crucial to systematically investigate the framing effect, to better understand and mitigate its impact. \gabis{This paragraph is important - I think that right now it's saying that we don't want models to be influenced by framing (since we want to mitigate its impact, right?) When we talked I think we had a more nuanced position?}




To better understand the framing effect in LLMs in comparison to human behavior,
we introduce the \name{} dataset (Section~\ref{sec:data}), comprising 1,000 statements, constructed through a three-step process, as shown in Figure~\ref{fig:fig1}.
First, we collect a set of real-world statements that express a clear negative or positive sentiment (e.g., ``I won the highest prize'').
%as exemplified in Figure~\ref{fig:fig1} -- ``I won the highest prize'' positive base statement. (2) next,
Second, we \emph{reframe} the text by adding a prefix or suffix with an opposite sentiment (e.g., ``I won the highest prize, \emph{although I lost all my friends on the way}'').
Finally, we collect human annotations by asking different participants
if they consider the reframed statement to be overall positive or negative.
% \gabist{This allows us to quantify the extent of \textit{sentiment shifts}, which is defined as labeling the sentiment aligning with the opposite framing, rather then the base sentiment -- e.g., voting ``negative'' for the statement ``I won the highest prize, although I lost all my friends on the way'', as it aligns with the opposite framing sentiment.}
We choose to annotate Amazon reviews, where sentiment is more robust, compared to e.g., the news domain which introduces confounding variables such as prior political leaning~\cite{druckman2004political}.


%While the implications of framing on sensitive and controversial topics like politics or economics are highly relevant to real-world applications, testing these subjects in a controlled setting is challenging. Such topics can introduce confounding variables, as annotators might rely on their personal beliefs or emotions rather than focusing solely on the framing, particularly when the content is emotionally charged~\cite{druckman2004political}. To balance real-world relevance with experimental reliability, we chose to focus on statements derived from Amazon reviews. These are naturally occurring, sentiment-rich texts that are less likely to trigger strong preexisting biases or emotional reactions. For instance, a review like ``The book was engaging'' can be framed negatively without invoking specific cultural or political associations. 

 In Section~\ref{sec:results}, we evaluate eight state-of-the-art LLMs
 % including \gpt{}~\cite{openai2024gpt4osystemcard}, \llama{}~\cite{dubey2024llama}, \mistral{}~\cite{jiang2023mistral}, \mixtral{}~\cite{mistral2023mixtral}, and \gemma{}~\cite{team2024gemma}, 
on the \name{} dataset and compare them against human annotations. We find  that LLMs are influenced by framing, somewhat similar to human behavior. All models show a \emph{strong} correlation ($r>0.57$) with human behavior.
%All models show a correlation with human responses of more than $0.55$ in Pearson's $r$ \gabis{@Gili check how people report this?}.
Moreover, we find that both humans and LLMs are more influenced by positive reframing rather than negative reframing. We also find that larger models tend to be more correlated with human behavior. Interestingly, \gpt{} shows the lowest correlation with human behavior. This raises questions about how architectural or training differences might influence susceptibility to framing. 
%\gabis{this last finding about \gpt{} stands in opposition to the start of the statement, right? Even though it's probably one of the largest models, it doesn't correlate with humans? If so, better to state this explicitly}

This work contributes to understanding the parallels between LLM and human cognition, offering insights into how cognitive mechanisms such as the framing effect emerge in LLMs.\footnote{\name{} data available at \url{https://huggingface.co/datasets/gililior/WildFrame}\\Code: ~\url{https://github.com/SLAB-NLP/WildFrame-Eval}}

%\gabist{It also raises fundamental philosophical and practical questions -- should LLMs aim to emulate human-like behavior, even when such behavior is susceptible to harmful cognitive biases? or should they strive to deviate from human tendencies to avoid reproducing these pitfalls?}\gabis{$\leftarrow$ also following Itay's comment, maybe this is better in the dicsussion, since we don't address these questions in the paper.} %\gabis{This last statement brings the nuance back, so I think it contradicts the previous parapgraph where we talked about ``mitigating'' the effect of framing. Also, I think it would be nice to discuss this a bit more in depth, maybe in the discussion section.}






\section{MEDDxAgent Overview}
Our proposed MEDDxAgent framework (see \autoref{fig:overview_ddxdriver} and a detailed version in~\autoref{fig:overview_framework}) comprises a central orchestrator (DDxDriver), a history taking simulator, and two specialized diagnostic agents dedicated to knowledge retrieval and diagnosis strategy. Both the simulator and diagnostic agents communicate exclusively with the DDxDriver, which monitors, stores, maintains, and updates patient information and ranked differential diagnoses. This central role also positions the DDxDriver to coordinate an iterative feedback loop, wherein observations from each agent are leveraged to enhance and refine subsequent agent calls with agent instructions (see an example in \autoref{fig:Example}). In the following, we introduce the design of simulator (\autoref{subsec:simulator}), agents (\autoref{subsec:diagnostic_agents}), orchestrator (DDxDriver) (\autoref{subsec:ddxdriver}), and iterative learning mechanism  (\autoref{subsec:iterative_learning}).

\subsection{Simulator}
% Clafify the setting without initial patient profile!
\label{subsec:simulator}

History taking is a critical first step in differential diagnosis, where clinicians gather essential information by asking patients questions about their symptoms, medical history, and lifestyle factors. In real-world clinical settings, a full patient profile is rarely available at the outset~\citep{li2024mediq} -- doctors typically start with only partial information (e.g., age, gender, chief complaint). The process of interactive DDx allows clinicians to gather more patient information and refine their diagnostic hypotheses before making a follow-up decision. 

To simulate such an interactive environment, we introduce a history taking simulator. We initialize the simulator with two LLMs \cite{wu2023large} in our experiments. The first LLM simulates the patient and receives access to the full patient profile. The second LLM simulates the doctor and receives an initial patient profile and optionally a set of conversational goals defined by DDxDriver (\textit{action}). During the interactions, the doctor role asks questions relevant to the diagnosis process, and the patient role provides answers based on its patient profile. 
The interaction continues until either the conversational goals are achieved or a predefined stopping criterion (e.g., maximum number of questions) is reached. Once the conversation concludes, the dialogue history is forwarded to DDxDriver.  

\begin{figure*}[t]
	\centering
    \includegraphics[trim={0cm 0.9cm 0cm 0cm},clip,width=0.9\textwidth]{img/Fig_example_ddxplus.pdf}
    %\vspace{-0.9em}
    \caption{An illustrated DDxPlus~\citep{fansi2022ddxplus} example with MEDDxAgent framework. Given the initial patient profile and list of diagnosis options, DDxDriver determines the goals and actions for the simulator (\textcolor{yellow-history}{\textbf{History Taking}}) and agents (\textcolor{blue-retrieval}{\textbf{Knowledge Retrieval}}, \textcolor{green-diagnosis}{\textbf{Diagnosis Strategy}}), updating the patient profile, and returning the ranked DDx. Each step is logged for transparency, enabling iterative refinement and learning.} 
    \vspace{-1.5em}
    \label{fig:Example}
\end{figure*}
\subsection{Agents}
\label{subsec:diagnostic_agents}


\paragraph{Knowledge Retrieval Agent.}
This agent aids the diagnostic process by retrieving relevant medical knowledge from external sources, such as scientific literature, medical databases, and clinical guidelines. This is particularly critical for diagnosing rare or complex conditions where external knowledge (as compared to internal knowledge learned by LLMs' training data) is required to enhance clinical reasoning with validated information.

Upon activation, the agent receives a search query formulated by DDxDriver, based on the current patient profile and provisional DDx list. It extracts the key medical concepts from the query as structured keywords, then conducts a targeted search in external databases. We consider two primary sources: Wikipedia and PubMed\footnote{\url{https://pubmed.ncbi.nlm.nih.gov/}}, with the former providing concise summaries of top-ranked pages, while the latter retrieves abstracts of full-access articles. The retrieved knowledge is synthesized into an evidence-based summary, which ensures that the diagnostic reasoning process has access to up-to-date, relevant medical knowledge.


\paragraph{Diagnosis Strategy Agent.}
This agent is responsible for generating, refining, and ranking possible diagnoses based on the information prepared by DDxDriver. % based on the previous agents' output. %The diagnosis instruction is sourced from the updated patient profile and/or retrieved information. 
There are two distinct modes that can be chosen for the diagnosis strategy agent. First, in the zero-shot setting, the LLM predicts the most probable diagnoses solely based on the current patient. This approach is straightforward but may have limited accuracy for complex or rare conditions. Second, in the few-shot setting, the diagnosis strategy agent utilizes additional patient cases to guide its predictions, enabling more context-aware diagnostic reasoning. We explore two variations. First, in a standard few-shot approach, a fixed set of patient examples are selected as references and provided to the model alongside the current patient profile. Second, the dynamic few-shot approach improves upon this by selecting reference cases based on similarity metrics, ensuring that the most relevant patients are included. Patient similarity is determined using embedding-based retrieval, with two embeddings (e.g., BioClinicalBERT~\citep{alsentzer-etal-2019-publicly}, BGE~\citep{xiao-etal-2024-bge}) evaluated to match patients with similar profiles.

We also integrate Chain-of-Thought (CoT) reasoning~\citep{wei2022chain}, guiding the model to explicitly reason through intermediate clinical steps before predicting a diagnosis. CoT can be combined with both standard and dynamic few-shot approaches, allowing the model to generate a stepwise rationale for each diagnosis. Inspired by MedPrompt~\citep{nori2023can}, our approach extends CoT by incorporating structured, example-driven reasoning, where each reference case includes both a diagnosis and an associated COT explanation. The integration of CoT enables the system to better handle complex cases with diagnostic uncertainty, such as specific skin diseases in iCRAFT-MD which share common symptoms. 
% \cc{example to show?}

Once the model completes the diagnostic inference, the ranked list of differential diagnoses is returned to DDxDriver, which further refines or finalizes the diagnosis through iterative updates. By distinguishing between zero-shot and few-shot inference strategies, plus dynamic adaptation through embeddings and reasoning techniques, the diagnosis strategy agent aims to enhance both accuracy and generalizability.

\vspace{-0.5em}
\subsection{Orchestrator} 
%\cc{mention the progress rate!}
\label{subsec:ddxdriver}

Inspired by the concept of a unified interface layer from previous work~\citep{gioacchini-etal-2024-agentquest}, we introduce DDxDriver as the central coordination hub in the MEDDxAgent framework (\autoref{fig:overview_ddxdriver}). DDxDriver enables modular compatibility between the diagnostic agents and benchmark datasets, with minimal adaptation efforts. 
DDxDriver uses the ReAct paradigm~\citep{yao2023react} -- which combines step-by-step reasoning (\textit{thought}) with decision-making (\textit{action}) and feedback processing (\textit{observation}). At each step, DDxDriver obtains the information from the \textit{environment} (input/output) and the results from the previous state of the simulator and agents (\textit{observation}, if it exists), then reasons about the current state of evidence (\textit{thought}) and generates agent-specific instructions based on the current state of the patient profile. It dispatches these instructions to the selected simulator/agent, executes the simulator/agent, and subsequently updates the patient profile with newly obtained information (\textit{action}). 
Beyond execution management, DDxDriver serves four primary functions. First, it manages the patient profile, storing and maintaining all relevant clinical information, including demographics, medical history, symptoms, and evolving diagnostic rankings. Second, it schedules and dispatches diagnostic actions, dynamically determining which simulator/agent to invoke next based on the evolving diagnostic context. Third, it ensures traceability by logging all interactions, including inputs, outputs, and intermediate reasoning steps, thereby providing transparency in the decision-making process. Finally, it enforces stopping criteria by monitoring diagnostic convergence and applying configurable thresholds, such as the number of iterations or the stabilization of ranked diagnoses.



\subsection{Iterative Learning Mechanism}
\label{subsec:iterative_learning}
Diagnoses in the real world are rarely made in a single step. They are refined through multiple interactions with patients, clinical data, and external knowledge. To mirror this process, the \textit{iterative learning} mechanism is designed to avoid relying on any single diagnostic agent or static decision process. We implement two settings: (i) \textit{fixed iteration}, and (ii) \textit{dynamic iteration}. \textit{Fixed iteration} cycles through the history taking simulator, knowledge retrieval agent, and diagnosis strategy agent in order until the predefined stopping criterion is met (e.g., $n$ iterations).
In contrast, the \textit{dynamic iteration} process lifts constraints on the predetermined execution order, allowing the DDxDriver to adapt dynamically during the differential diagnosis process. After each observation, the DDxDriver reasons about which component -- history taking simulator, knowledge retrieval agent, or diagnosis strategy agent -- to call next based on up-to-date observations (i.e. updated patient profile, medical documents, predicted DDx). For instance, if the current diagnosis indicates a rare condition for which it needs clarifying details, the system may invoke the knowledge retrieval agent to search for specialized information. This allows for flexible decision-making, opening up the opportunity for both ideal and non-ideal choices. The iterative learning mechanism allows MEDDxAgent to continuously refine diagnosis while offering transparent insights into its reasoning process.
\vspace{-0.5em}

\section{Experimental Setup}
%This section introduces the benchmark with three datasets and the evaluation metrics. We further provide details on the selected models, the hyperparameters, and the baseline systems.
%
\setlength{\tabcolsep}{10.7pt}
\begin{table*}[h]
    \centering
    \footnotesize
\begin{tabular}{lccccc}
\toprule
\textbf{Dataset}   & \textbf{Domain}  & \textbf{\# Cases}    & \textbf{\# Diseases} & \textbf{Synthetic}   & \textbf{License\dag}    \\
\midrule
DDxPlus~\citep{fansi2022ddxplus}   & \textit{respiratory} & 1.3M    & 49           & \checkmark & CC-BY      \\
iCraft-MD~\citep{li2024mediq}   & \textit{skin}    & 140    & 394         &   \checkmark       & MIT        \\
RareBench~\citep{chen2024rarebench} & \textit{rare}    & 2,185    & 421         &     $\times$ 
& Apache-2.0 \\ 
\bottomrule
\end{tabular}
\caption{Overview of the selected sources for constructing DDx benchmark. We consider three domains (i.e., disease categories) (\textit{respiratory}, \textit{skin}, \textit{rare}) over different sizes of diagnosis options. All selected sources are applicable for \textit{commercial} usage. \dag License: Creative Commons Attribution International License (CC-BY). }
    \label{tab:datasets_overview}
\end{table*}


% Israa: I changed the columns order
\iffalse
\begin{tabular}{lccccc}
\toprule
\textbf{Dataset}   & \textbf{\# Cases}    & \textbf{Domain}  & \textbf{Synthetic}   & \textbf{\# Diagnosis} & \textbf{License\dag}    \\
\midrule
DDxPlus~\citep{fansi2022ddxplus}   & 1.3M & \textit{general} & \checkmark & 49           & CC-BY      \\
CraftMD~\citep{johri2024craft,johri2025craftmd}   & 140         & \textit{skin}    &   $\times$       & 393          & MIT        \\
RareBench~\citep{chen2024rarebench} &     2,185      & \textit{rare}    &    $\times$        &     421         & Apache-2.0 \\ \bottomrule
\end{tabular}
\fi
\subsection{DDx Benchmark}
\label{subsec:ddx_benchmark}

We introduce a comprehensive DDx benchmark integrating three datasets -- DDxPlus, iCraft-MD, and RareBench, covering \textit{respiratory}, \textit{skin}, and \textit{rare} diseases for a robust assessment of diagnostic performance. This addresses limitations of prior work, which often relies on a single dataset and single-turn evaluation for differential diagnosis. DDxPlus~\citep{fansi2022ddxplus} provides a large-scale structured dataset with 1.3 million synthetic respiratory patient cases across 49 respiratory-related pathologies. iCraft-MD ~\citep{li2024mediq} includes 394 skin diseases, adapting static dermatological clinical vignettes (from original Craft-MD dataset~\citep{johri2024craft, johri2025craftmd}) into an interactive setting%\cl{We use the ineractive setting by doing XYZ with our history taking simulator.} 
\footnote{Interactive DDx is a more complex information-seeking setup, since in the real world the full patient profile might not be accessible initially.} -- the system is only provided with partial patient information and is expected to proactively ask questions and gather information. %It consists of 140 dermatology cases, with 100 sourced from an online medical question bank and 40 designed by expert clinicians. In the interactive setting, it requires iterative history-taking and diagnostic refinement. 
RareBench~\citep{chen2024rarebench} expands DDxPlus with 421 rare diseases. We select three subsets from RareBench -- RAMEDIS (Europe), MME (Canada), and PUMCH (China) -- to ensure diversity in regional representation.

To enable a consistent evaluation across datasets, we standardize each dataset into a structured format: (i) optional initial patient information (e.g., age, sex, chief complaint); (ii) full patient profile (complete list of symptoms and antecedents); and (iii) full set of possible diseases for differential diagnosis. %A key challenge in iCraft-MD and RareBench is the lack of predefined diagnostic options and redundant disease labels; we address this by leveraging GPT-4o to generate unique, non-redundant differential diagnosis sets, yielding 394 distinct dermatological conditions in iCraft-MD and 102 rare diseases in RareBench. 
This refinement enhances diagnostic consistency and supports the evaluation of interactive DDx. We sample 100 patients from each dataset at a fixed random seed, due to the cost of experiments and excessive time for reasoning steps. Detailed dataset statistics are in~\autoref{app:ddx_benchmark_details}.


\subsection{Evaluation Metrics}
%\danny{Evaluation for each module, and the tracking progress states}
To evaluate diagnostic performance, we employ three metrics. First, we compute the \textit{average rank} of the correct disease, which represents the model’s ability to position the correct diagnosis closer to the top. If the diagnosis does not appear in the top-10 position, we assign a rank of 11. Second, we use \textit{GTPA@k} (Ground Truth Pathology Accuracy)~\citep{fansi2022ddxplus}, which measures whether the ground truth diagnosis appears within the top-\textit{k} predicted diagnoses.
Third, we introduce a new metric suitable for the iterative setting: \textit{average progress rate} ($\Delta$ Progress). Inspired by AgentQuest \cite{gioacchini-etal-2024-agentquest}, it tracks changes in rank $r$ of the ground truth pathology in the differential diagnosis. For each patient case i, we average the progress in rank ($r_{i,t}-r_{i,t+1}$) over N iterations of differential diagnosis, then aggregate over M patients. This metric quantifies how effectively the system refines and converges on the correct diagnosis over successive iterations:

\small{
\begin{equation*}
\Delta \text{Progress} = \frac{1}{M} \sum_{i=1}^{M} \left( \frac{1}{N_i - 1} \sum_{t=1}^{N_i-1} \Bigl( r_{i,t} - r_{i,t+1} \Bigr) \right)
\end{equation*}}
\normalsize
\subsection{Models and Tasks}

We evaluate on GPT-4o (version: \textsc{2024-11-20})~\citep{hurst2024gpt}, Llama3.1-70B and Llama3.1-8B~\citep{dubey2024llama} across all tasks, ensuring a comparison of LLMs at varying scales.  Our experiments are conducted in two setups: (1) optimizing individual agents; and (2) interactive differential diagnosis. In the first task, we evaluate the two agents (knowledge retrieval, diagnosis strategy) in a single-turn setting. This allows us to isolate the effectiveness of the reasoning mechanisms without the confounding factor of incomplete information. In the second task, we assess MEDDxAgent's performance at interactive DDx, comparing it against the single-turn diagnostic agents and history taking simulator. Interactive differential diagnosis, as suggested by~\citet{li2024mediq}, is a challenging yet realistic scenario, where only initial patient information is available -- without a complete list of symptoms and antecedents. This setup highlights how limited information constrains the single-turn setting (i.e., no iteration), compared to MEDDxAgent's iterative interactions, which refine and enhance the diagnostic process.

\subsection{Hyperparameters and Optimization}
\label{subsec:hyperaparemeters_optimization}

\setlength{\tabcolsep}{2.3pt}
\begin{table*}[h]
\centering
\scriptsize
\begin{tabular}{rccccccccc}
\toprule
                               & \multicolumn{3}{c}{\textbf{DDxPlus}} & \multicolumn{3}{c}{\textbf{iCraft-MD}} & \multicolumn{3}{c}{\textbf{RareBench}} \\ \cmidrule(lr){2-4} \cmidrule(lr){5-7} \cmidrule(lr){8-10}
                               & \textbf{GTPA@1 $\uparrow$}          & \textbf{GTPA@5 $\uparrow$}   & \textbf{Avg Rank $\downarrow$} & \textbf{GTPA@1 $\uparrow$}       & \textbf{GTPA@5 $\uparrow$}     & \textbf{Avg Rank $\downarrow$}   & \textbf{GTPA@1 $\uparrow$}        & \textbf{GTPA@5 $\uparrow$}   & \textbf{Avg Rank $\downarrow$}     \\\midrule
                               %& \multicolumn{9}{c}{\textbf{GPT-4o}}                                                                 \\\midrule
Retrieval (PubMed)                   & 0.69           &  0.90   &  2.27  & 0.68        &     \textbf{0.79}    & 3.23  & 0.45         &   0.72  &    3.92   \\
Retrieval (Wiki)                   &    0.69        &   0.90 &  2.24  & \textbf{0.69}         &     \textbf{0.79}    & \textbf{3.22}  &   0.45       &  0.74  &  4.00    \\ 
\cmidrule(lr){2-10}
Zero-shot (Standard)                     &     0.69           &     0.90       &      2.21      &       0.68         &      0.77        &         3.37       &       0.46       &      0.72       &   3.99             \\
Zero-shot (CoT)                    &     0.71          &     0.92       &      2.10      &       0.68         &     0.77         &         3.35       &       0.47       &    0.69         &   4.02              \\ %\cmidrule(lr){2-10}
%Few-shot (Standard, Dyn\_BERT) &       0.96        &  1.00           &     1.06           &     0.67         &     0.77           &      3.31        &     0.78        &       0.90     & 2.19     \\
Few-shot (Standard, Dyn\_BAII)$\ddag$ &      0.96          &      \textbf{1.00}      &    1.06         &        0.62        &        0.72      &     3.85           &     0.79         &   \textbf{0.91}          &      \textbf{2.03}           \\
Few-shot (CoT, Dyn\_BERT)      &       0.96         &    \textbf{1.00}        &  1.05        &     0.64           &      0.73        &         3.68      &     0.81         &  \textbf{0.91}            &          2.04      \\
Few-shot (CoT, Dyn\_BAII)      &       \textbf{0.97}         &     \textbf{1.00}       &       \textbf{1.03}      &         0.60       &      0.70        &        4.00      &         \textbf{0.82}     &      0.88       &         2.11      \\

%History (n=5)         & 0.45           &      & 4.13 & 0.40        & 5.58     &      & 0.11         &      &  7.84   \\
%History (n=10)        & 0.59           & 3.16    & -  & 0.45        & 5.35        & -  & 0.24         & 6.67      &  -   \\
%History (n=15)        & 0.69           &      & 2.47 & 0.46        &       &  5.23   & 0.36         &       &    5.49 \\
%\midrule
%                                & \multicolumn{9}{c}{\textbf{Llama3.1-70B}}                                                           \\ \midrule
% Zero-shot (Standard)                     &      0.54          &    0.78        &       3.53     &     0.40           &     0.64         &        4.87      &      0.39        &       0.77      &         4.05         \\
% Zero-shot (CoT)                      &     0.45          &    0.78      &       3.69     &       \textbf{0.48}         &      \textbf{0.66}        &         \textbf{4.50}       &       0.49       &       0.75      &   3.91              \\\cmidrule(lr){2-10}
% Retrieval (PubMed)                 & 0.56           &  0.79     & 3.42 & 0.44        &    0.63     &  4.72 & 0.38         &   0.75       & 3.96 \\
% Retrieval (Wiki)                    &   0.49         & 0.77   &  3.60  &       0.44  &  \textbf{0.66}       &  4.71 &     0.39     & 0.75   &  4.09   \\\cmidrule(lr){2-10}
% %Few-shot (Standard, Dyn\_BERT) &      0.84         &  0.94          &                 1.68           &      0.40        &         0.62       &    4.96          &       0.72      &    0.87       & 2.44      \\
% Few-shot (Standard, Dyn\_BAII)$\ddag$ &        0.86        &    \textbf{0.95}        &    1.59         &     0.40           &      0.63        &         5.02       &     0.73         &       0.87      &      2.44           \\
% Few-shot (CoT, Dyn\_BERT)      &      \textbf{0.91}          &     \textbf{0.95}       &   \textbf{1.55}         &      0.45          &   0.61           &        4.93     &     \textbf{0.75}         &   \textbf{0.88}          &        \textbf{2.35}      \\
% Few-shot (CoT, Dyn\_BAII)      &     0.89           &     0.93       &       1.71     &      0.45          &    0.63          &       4.90      &         0.71     &     0.87        &         2.62     \\
%History (n=5)         & 0.45           &     &  4.15 & 0.29        &    &     6.48   & 0.30         &       &   6.04  \\
%History (n=10)        & 0.58           &      & 3.12 & 0.33        &     &    5.82   & 0.36         &     &    4.51   \\
%History (n=15)        & 0.56           &       & 3.50 & 0.36        &       &  5.36   & 0.31         &     &    4.80   \\
% \midrule
%                                & \multicolumn{9}{c}{\textbf{Llama3.1-8B}}                                                            \\\midrule
% Zero-shot (Standard)                      &    0.45            &      0.68      &   9.00          &    0.27             &    0.43          &      7.02         &    0.33           &    0.57         &             5.45     \\
% Zero-shot (CoT)                     &    0.45            &      0.70      &   4.51          &    0.27             &      0.40        &      7.25         &    0.24           &     0.55        &             5.65     \\\cmidrule(lr){2-10}
% Retrieval (PubMed)                   & 0.42           &  0.67   & 4.50  & 0.29        &   \textbf{0.44}   &   6.93   & 0.35         &   0.59   &    5.33  \\
% Retrieval (Wiki)                    &    0.43        &  0.67   &  4.56  &  0.29       &   \textbf{0.44}      & 6.93  &     0.36     &   0.67 &    4.80   \\\cmidrule(lr){2-10}
% %Few-shot (Standard, Dyn\_BERT)  &     0.97           &   1.00         &  1.04           &      0.22          &    0.40          &         7.32       &     0.70         &     0.81        &   2.94              \\
% Few-shot (Standard, Dyn\_BAII)$\ddag$  &     \textbf{0.97}           &    \textbf{1.00}        &   \textbf{1.03}          &    0.21            &      0.42        &    6.93            &  \textbf{0.71}            &     0.83        &     \textbf{2.80}            \\
% Few-shot (CoT, Dyn\_BERT)     &     0.95           &     0.99       &       1.19      &     0.29           &      0.36        &        7.28     &  0.64            &    \textbf{0.84}         &        2.96      \\
% Few-shot (CoT, Dyn\_BAII)       &      \textbf{0.97}          &    \textbf{1.00}        &       \textbf{1.03}     &      \textbf{0.30}          &   \textbf{0.44}           &         \textbf{6.66}     &  0.65            &      0.82       &        2.95      \\
%History (n=5)         & 0.23           & 6.85    & -  & 0.10        & 8.78        &  - & 0.05         & 8.38       &   - \\
%History (n=10)        & 0.35           & 5.46    &  - & 0.12        & 8.39     &   -   & \textbf{0.13}         & 8.25   &     -   \\
%History (n=15)        & 0.40           & 5.44   &  -  & 0.11        & 8.30       &  -  & \textbf{0.11}        & 8.95       &  -  \\

\bottomrule
    \end{tabular}
    \vspace{-0.8em}
    \caption{Results in the non-interactive setting for the knowledge retrieval agent (\textit{upper}) and the diagnosis strategy agent (\textit{bottom}). %, assuming that there are existing \emph{full} patient profiles. 
    $\ddag$ Only Few-shot (Standard, Dyn\_BAII) results are recorded, since the method is consistently better than Dyn\_BERT. All models exhibit similar trends. To give a more concise overview, we only report GPT-4o here. The full set of results can be found in~\autoref{tab:with_patient_profile_full} in Appendix.}
    %Since we discover similar trends for other models, for brevity, we recorded the results of GPT-4o, where the full results are noted in 
    %Appendix~\ref{app:additional_experiments}.}%\cc{could decide which model to present}}
    \label{tab:with_patient_profile}
    \vspace{-1.0em}
\end{table*}

%\subsection{Hyperparameters and Optimization} 
For the knowledge retrieval agent, we limit searches to a maximum of three medical keywords per query. Wikipedia is used as an open-access resource, while PubMed retrieval is restricted to full-text articles from commercially licensed sources,\footnote{We use MediaWiki API: \url{https://en.wikipedia.org/w/api.php} and \textsc{biopython}~\url{https://biopython.org/}.} ensuring that retrieved information is clinically validated and relevant to the diagnostic task. For the diagnosis strategy agent, we take 5 examples for few-shot learning. For dynamic few-shot, we use BioClinicalBERT (BERT)~\citep{alsentzer-etal-2019-publicly} and \textsc{bge-base-en-v1.5} (BAII)~\citep{xiao-etal-2024-bge} embeddings, based on the structure proposed by~\citet{wu2024streambench}. Specifically, it uses L2 distance on normalized embeddings, a similar setting to cosine similarity. With the history taking simulator, we create an iterative environment, which we evaluate at 5, 10, and 15 maximum questions. This is based on prior clinical studies that indicate physicians typically ask fewer than 15 questions per consultation~\citep{ely1999analysis}. This ensures that our model operates within a realistic range, capturing essential patient details without excessive interaction. 
%\textbf{DDxDriver:} 
For MEDDxAgent's iterative learning, we select the optimized history taking simulator and diagnostic agents and experiment on interactive DDx. Our setup is inspired by previous work~\citep{johri2025craftmd}, which demonstrates that updating the patient profile with new history-taking dialogue significantly enhances performance. We experiment with 1 to 3 iterations, with 5 questions per iteration. This aligns with the history-taking simulator setting (5 questions per iteration, max 15 for 3 iterations). Additionally, we set the DDxDriver's instruction for each agent and simulator to a list of length 10.
\vspace{-0.5em}


\section{Evaluation Results}
\label{evaluation-results}
% \setlength{\tabcolsep}{4.6pt}
% \begin{table*}[t]
% \centering
% \footnotesize
% \begin{tabular}{rcccccc}
% \toprule
%                                & \multicolumn{2}{c}{\textbf{DDxPlus}} & \multicolumn{2}{c}{\textbf{iCraft-MD}} & \multicolumn{2}{c}{\textbf{RareBench}} \\ \cmidrule(lr){2-3} \cmidrule(lr){4-5} \cmidrule(lr){6-7}
%                                & \textbf{GTPA@1 $\uparrow$}          & \textbf{Avg Rank $\downarrow$}   & \textbf{GTPA@1 $\uparrow$}       & \textbf{Avg Rank $\downarrow$}       & \textbf{GTPA@1 $\uparrow$}        & \textbf{Avg Rank $\downarrow$}       \\\midrule
%                                & \multicolumn{6}{c}{\textbf{GPT-4o}}                                                                 \\\midrule
% \textcolor{cyan}{Zero-shot}                      &                &            &             &                &              &                \\
% \textcolor{cyan}{Few-shot (Standard, Dyn\_BAII)} &                &            &             &                &              &                \\
% \textcolor{cyan}{Few-shot (CoT, Dyn\_BAII)}      &                &            &             &                &              &                \\
% History Taking (\textit{n}=5)         & 0.45           & 4.13       & 0.40        & 5.58           & 0.11         & 7.84           \\
% %History Taking (\textit{n}=10)        & 0.59           & 3.16       & 0.45        & 5.35           & 0.24         & 6.67           \\
% History Taking (\textit{n}=15)        & 0.69           & 2.47       & 0.46        & 5.23           & 0.36         & 5.49           \\
% Retrieval (PubMed) \textcolor{red}{rerun/ignore?}                   & 0.69           & 2.27       & 0.68        & 3.23           & 0.45         & 3.92           \\
% MEDDxAgent (\textbf{Ours})         &                &            &             &                &              &                \\
% \textit{iter} =  1                       & 0.74           & 1.91       & 0.52        & 4.93           & 0.51         & 4.37           \\
% \textit{iter} =  2                       & 0.78           & 1.56       & \textbf{0.54}        & \textbf{4.71}           & \textbf{0.56}         & 4.10           \\
% \textit{iter} =  3                       & \textbf{0.86}           & \textbf{1.29}       & \textbf{0.54}        & 4.80           & 0.50         & \textbf{4.09}           \\\midrule
%                                & \multicolumn{6}{c}{\textbf{Llama3.1-70B}}                                                           \\ \midrule
% \textcolor{cyan}{Zero-shot}                      &                &            &             &                &              &                \\
% \textcolor{cyan}{Few-shot (Standard, Dyn\_BAII)} &                &            &             &                &              &                \\
% \textcolor{cyan}{Few-shot (CoT, Dyn\_BAII)}      &                &            &             &                &              &                \\
% History Taking (\textit{n}=5)         & 0.45           & 4.15       & 0.29        & 6.48           & 0.30         & 6.04           \\
% %History Taking (\textit{n}=10)        & 0.58           & 3.12       & 0.33        & 5.82           & 0.36         & 4.51           \\
% History Taking (\textit{n}=15)        & 0.56           & 3.50       & 0.36        & 5.36           & 0.31         & 4.80           \\
% Retrieval (PubMed)  \textcolor{red}{rerun/ignore?}                 & 0.56           & 3.42       & 0.44        & 4.72           & 0.38         & 3.96           \\
% MEDDxAgent (\textbf{Ours})         &                &            &             &                &              &                \\
% \textit{iter} =  1                       & 0.61           & 2.91       & 0.29        & 7.05           & 0.39         & 5.05           \\
% \textit{iter} =  2                       & \textbf{0.71}   & \textbf{2.20}       & 0.37        & \textbf{6.26}           & \textbf{0.48}         & 4.48           \\
% \textit{iter} =  3                       & 0.68   & 2.30       & \textbf{0.42}        & 6.31           & \textbf{0.48}         & \textbf{4.30}           \\\midrule
%                                & \multicolumn{6}{c}{\textbf{Llama3.1-8B}}                                                            \\\midrule
% \textcolor{cyan}{Zero-shot}                      &                &            &             &                &              &                \\
% \textcolor{cyan}{Few-shot (Standard, Dyn\_BAII)} &                &            &             &                &              &                \\
% \textcolor{cyan}{Few-shot (CoT, Dyn\_BAII)}     &                &            &             &                &              &                \\
% History Taking (\textit{n}=5)         & 0.23           & 6.85       & 0.10        & 8.78           & 0.05         & 8.38           \\
% %History Taking (\textit{n}=10)        & 0.35           & 5.46       & 0.12        & 8.39           & \textbf{0.13}         & 8.25           \\
% History Taking (\textit{n}=15)        & 0.40           & 5.44       & 0.11        & 8.30           & \textbf{0.11}        & 8.95           \\
% Retrieval (PubMed)  \textcolor{red}{rerun/ignore?}                 & 0.42           & 4.50       & 0.29        & 6.93           & 0.35         & 5.33           \\
% MEDDxAgent (\textbf{Ours})         &                &            &             &                &              &                \\
% \textit{iter} =  1                       & 0.34           & 5.25       & 0.11        & 9.38           & 0.08         & 8.47           \\
% \textit{iter} =  2                       & 0.56           & 3.59       & \textbf{0.14}        & 9.22           & 0.09         & \textbf{8.11}           \\
% \textit{iter} =  3                       & \textbf{0.58}           & \textbf{3.10}       & 0.12        & \textbf{9.07}           & 0.07         & 8.56        \\  
% \bottomrule
%     \end{tabular}
%     \caption{Iterative experiment performance across 3 datasets. \textcolor{red}{The \textbf{best results} are based on ignoring the Pubmed retrieval results!}}
%     \label{tab:iterative_overall}
% \end{table*}

\setlength{\tabcolsep}{3.8pt}
\begin{table*}[ht]
\centering
\scriptsize
\begin{tabular}{rccccccccc}
\toprule
                               & \multicolumn{3}{c}{\textbf{DDxPlus}} & \multicolumn{3}{c}{\textbf{iCraft-MD}} & \multicolumn{3}{c}{\textbf{RareBench}} \\ \cmidrule(lr){2-4} \cmidrule(lr){5-7} \cmidrule(lr){8-10}
                               & \textbf{GTPA@1 $\uparrow$}          & \textbf{Avg Rank $\downarrow$}   & \textbf{$\Delta$ Progress} & \textbf{GTPA@1 $\uparrow$}       & \textbf{Avg Rank $\downarrow$}     & \textbf{$\Delta$ Progress}   & \textbf{GTPA@1 $\uparrow$}        & \textbf{Avg Rank $\downarrow$}   & \textbf{$\Delta$ Progress}     \\\midrule
                               & \multicolumn{9}{c}{\textbf{GPT-4o}}                                                                 \\\midrule
%\textcolor{cyan}{Zero-shot}                      &     0.69           &    2.21        &      -      &       0.68         &     3.37         &         -       &       0.46       & 3.99            &   -              \\
%\textcolor{cyan}{Zero-shot (CoT)}                      &     0.71          &    2.10        &      -      &       0.68         &     3.35         &         -       &       0.47       & 4.02            &   -              \\
%\textcolor{cyan}{Few-shot (CoT, Dyn\_BAII)} &                &            &     -        &                &              &          -      &              &             &            -     \\
%\textcolor{cyan}{Few-shot (CoT, Dyn\_BERT/Dyn\_BAII)}      &                &            &       -      &                &              &          -      &              &             &           -      \\
%\textit{Single-Turn}      &                &            &       -      &                &              &          -      &              &             &           -      \\
KR (\textit{n}=0)                &      0.18      & 7.33  &  -  &   0.15      &    8.27     & -  &       0.07   &  9.07  &    -   \\
DS (\textit{n}=0)     &  0.27    &    6.01        &       -      &      0.18          &      7.87        &          -      &       0.11       &     8.38        &           -      \\
%SDS (\textit{n}=5)         & 0.45           & 4.13     & - & 0.40        & 5.58     &    -  & 0.11         & 7.84     &  -    \\
KR (\textit{n}=5)  &      0.52      & 3.32   &  -  &  0.49  &  5.36       & -  &     0.40   &   5.27 &    -   \\
DS (\textit{n}=5)  &    0.72       &  2.14 &  -  &  0.40 &    5.55   & -  &   0.50    &   4.94 &    -   \\\cmidrule(lr){2-10}
%History Taking (\textit{n}=10)        & 0.59           & 3.16    & -  & 0.45        & 5.35        & -  & 0.24         & 6.67      &  -   \\
%SDS (\textit{n}=15)        & 0.69           & 2.47     & - & 0.46        & 5.23      &  -   & 0.36         & 5.49      &    - \\\cmidrule(lr){2-10}

%Retrieval (Wiki) \textcolor{red}{rerun}                   &            &    &  -  &         &         & -  &          &    &    -   \\
%MEDDxAgent         &                &            &             &                &              &               &              &             &                  \\
 MEDDx (\textit{iter}=1, \textit{n}=5)                       & 0.74           & 1.91     & ~~0.00 & 0.52        & 4.93      &  ~~0.00   & 0.51         & 4.37        &   ~~0.00\\
MEDDx (\textit{iter}=2, \textit{n}=10)                       & 0.78           & 1.56    & +0.32  & \textbf{0.54}        & \textbf{4.71}    &    +0.26   & \textbf{0.56}         & 4.10   &     +0.13   \\
MEDDx (\textit{iter}=3, \textit{n}=15)                       & \textbf{0.86}           & \textbf{1.29}    & +0.32  & \textbf{0.54}        & 4.80      & +0.17    & 0.50         & \textbf{4.09}       &   +0.16 \\\midrule
                               & \multicolumn{9}{c}{\textbf{Llama3.1-70B}}                                                           \\ \midrule
%\textcolor{cyan}{Zero-shot}                      &      0.54          &     3.53       &       -      &     0.40           &       4.87       &         -      &      0.39        &    4.05         &         -         \\
%\textcolor{cyan}{Zero-shot (CoT)}                      &     0.45          &    3.69       &      -      &       0.48         &     4.50         &         -       &       0.49       & 3.91            &   -              \\
%\textcolor{cyan}{Few-shot (Standard, Dyn\_BAII)} &                &            &      -       &                &              &          -      &              &             &        -         \\
%\textcolor{cyan}{Few-shot (CoT, Dyn\_BERT/Dyn\_BAII)}      &                &            &      -       &                &              &          -    &              &             &                -   \\
KR (\textit{n}=0)           &   0.19         &  7.58  &  -  &      0.13   &   8.19      & -  &    0.09      &  9.13  &    -   \\
DS (\textit{n}=0)    &        0.17        &       7.28     &       -      &      0.11          &      8.74        &          -      &       0.20       &      6.81       &           -      \\
%History Taking (\textit{n}=5)         & 0.45           & 4.15    &  - & 0.29        & 6.48   &     -   & 0.30         & 6.04      &   -  \\
KR (\textit{n}=5)  &      0.39      & 5.03   &  -  &  0.34  &  6.86       & -  &     0.29   &   5.86 &    -   \\
DS (\textit{n}=5)  &     0.50      &  2.89 &  -  & 0.24 &    7.33   & -  &   0.23    &  5.77  &    -   \\\cmidrule(lr){2-10}
%History Taking (\textit{n}=10)        & 0.58           & 3.12     & - & 0.33        & 5.82    &    -   & 0.36         & 4.51    &    -   \\
%History Taking (\textit{n}=15)        & 0.56           & 3.50      &- & 0.36        & \textbf{5.36}       &  -  & 0.31         & 4.80    &    -   \\\cmidrule(lr){2-10}
%Retrieval (Wiki) \textcolor{red}{rerun}                   &            &    &  -  &         &         & -  &          &    &    -   \\
%MEDDxAgent         &                &            &             &                &              &            &                &              &        \\
MEDDx (\textit{iter}=1, \textit{n}=5)                       & 0.61           & 2.91    & ~~0.00  & 0.29        & 7.05       & ~~0.00   & 0.39         & 5.05   &     ~~0.00   \\
MEDDx (\textit{iter}=2, \textit{n}=10)                       & \textbf{0.71}   & \textbf{2.20}      & +0.41 & 0.37        & \textbf{6.26}      & +0.07   & \textbf{0.48}         & 4.48  &    +0.75     \\
MEDDx (\textit{iter}=3, \textit{n}=15)                       & 0.68   & 2.30    & +0.17  & \textbf{0.42}        & 6.31     &   +0.26   & \textbf{0.48}         & \textbf{4.30}      &   +0.44  \\\midrule
                               & \multicolumn{9}{c}{\textbf{Llama3.1-8B}}                                                            \\\midrule
%\textcolor{cyan}{Zero-shot}                      &    0.45            &   9.00         &   -          &    0.27             &     7.02         &      -         &    0.33           &  5.45           &             -     \\
%\textcolor{cyan}{Zero-shot (CoT)}                      &    0.45            &   4.51         &   -          &    0.27             &     7.25         &      -         &    0.24           &  5.65           &             -     \\
%\textcolor{cyan}{Few-shot (Standard, Dyn\_BAII)} &                &            &     -        &                &              &        -      &              &             &             -      \\
%\textcolor{cyan}{Few-shot (CoT, Dyn\_BERT/Dyn\_BAII)}     &                &            &       -      &                &              &         -     &              &             &           -        \\
KR (\textit{n}=0)     &     0.20       &  7.49  &  -  &   0.11      &  \textbf{8.86}       & -  &     \textbf{0.11}     &  8.58  &    -   \\
DS (\textit{n}=0)  &       0.16         &       8.45     &       -      &      0.03          &      10.37        &          -      &         0.04     &       8.52      &           -      \\
%History Taking (\textit{n}=5)         & 0.23           & 6.85    & -  & 0.10        & 8.78        &  - & 0.05         & 8.38       &   - \\
KR (\textit{n}=5)  &      0.21      & 7.42   &  -  &  0.09  &  9.48       & -  &     0.04   &   9.69 &    -   \\
DS (\textit{n}=5)  &     0.23      &  5.77  &  -  &  0.03 &   10.08    & -  &   0.06    &  8.64  &    -   \\\cmidrule(lr){2-10}
%History Taking (\textit{n}=10)        & 0.35           & 5.46    &  - & 0.12        & 8.39     &   -   & \textbf{0.13}         & 8.25   &     -   \\
%History Taking (\textit{n}=15)        & 0.40           & 5.44   &  -  & 0.11        & \textbf{8.30}       &  -  & \textbf{0.11}        & 8.95       &  -  \\\cmidrule(lr){2-10}
%Retrieval (Wiki) \textcolor{red}{rerun}                   &            &    &  -  &         &         & -  &          &    &    -   \\
%MEDDxAgent         &                &            &             &                &              &               &                &              &     \\
MEDDx (\textit{iter}=1, \textit{n}=5)                       & 0.34           & 5.25   &   ~~0.00 & 0.11        & 9.38       &  ~~0.00  & 0.08         & 8.47    &    ~~0.00   \\
MEDDx (\textit{iter}=2, \textit{n}=10)                       & 0.56           & 3.59    & +1.73  & \textbf{0.14}        & 9.22       &  +0.22  & 0.09         & \textbf{8.11}      &  +0.44   \\
MEDDx (\textit{iter}=3, \textit{n}=15)                       & \textbf{0.58}           & \textbf{3.10}    &  +1.23 & 0.12        & 9.07     & +0.17     & 0.07         & 8.56    &  +0.38  \\  
\bottomrule
    \end{tabular}
    \vspace{-0.8em}
    \caption{Interactive experiment performance across 3 datasets without \textit{full} patient profile, with KR: knowledge retrieval agent; DS: diagnosis strategy agent; $n$ is the number of turns of the simulator; MEDDx uses KR+DS.
    %We compare the single-turn (\textit{upper}) with the proposed iterative setup for MEDDxAgent (\textit{bottom}). The selection of the agents and simulator are optimized (\autoref{subsec:optimize-agents}), unless controlled by the number of questions ($n$) asked from history taking simulator.\cc{May be we just need 3 entries of single turn for best agents and simulator compared to MEDDxAgent? For the others we leave it for ablation study?}} %\cl{why we don't have the baseline with diagnosis strategy only without full patient profile (e.g., few-shot CoT, Dyn\_BAII?)}
    }
    \label{tab:interactive_overall}
    \vspace{-1.8em}
\end{table*}

We experiment on two configurations: (1) optimizing individual agents (\autoref{subsec:optimize-agents}), by determining the best settings for knowledge retrieval and diagnosis strategy agents; and (2) interactive differential diagnosis (\autoref{subsec:iterative_learning}), where the optimized agents are used to assess MEDDxAgent's performance in the interactive DDx setup.

\subsection{Optimizing Individual Agents}
\label{subsec:optimize-agents}

We first explore the optimal single-turn configuration for the knowledge retrieval and diagnosis strategy agents, before integrating them into iterative setup. For this, we provide the full patient profile as in previous work~\cite{wu2024streambench,chen2024rarebench}, and present the results in~\autoref{tab:with_patient_profile}. For the knowledge retrieval agent, PubMed performs slightly better overall than Wikipedia, especially for Rarebench, which demands more complex disease information. For the diagnosis strategy agent, the best setting varies by dataset. 
Namely, dynamic few-shot with BAII embeddings performs the best on DDxPlus and RareBench, where relevant patient examples offer reliable contextual cues to likely diseases. 
In contrast, iCraft-MD benefits more from zero-shot CoT, which enables structured reasoning through complex clinical vignettes. Few-shot learning often decreases performance for iCraft-MD because each patient vignette is distinct, so additional examples can introduce noise.
Based on the above findings, we select the following configurations for the iterative scenario:\footnote{We do not run all possible settings in the interactive environment due to cost reasons.} PubMed for knowledge retrieval agent; few-shot (dynamic BAII) for DDxPlus and RareBench, and zero-shot (CoT) for iCraft-MD for diagnosis strategy agent.

\begin{figure*}[t]
    \centering
    \begin{subfigure}{0.48\textwidth}
    \includegraphics[trim={0.2cm 0cm 0cm 0cm },clip, width=\textwidth]{img/ddxplus_history.pdf}
    \vspace{-1.8em}
    \caption{}
    \end{subfigure}
    \begin{subfigure}{0.48\textwidth}
    \includegraphics[trim={0.2cm 0cm 0cm 0cm}, clip, width=\textwidth]{img/agent_iterations_plot_ddxplus.pdf}
    \vspace{-1.8em}
    \caption{}
    \end{subfigure}
    \vspace{-0.5em}
    \caption{Results of DDxPlus compared between (a) history taking simulator, and (b) MEDDxAgent, over the number of questions and iterations. For brevity, the results of iCraft-MD and RareBench are in~\autoref{subsec:comparison_history_taking_iterative}.}
    \label{fig:ddxplus_comparison}
    \vspace{-1.8em}
\end{figure*}

\subsection{Interactive Differential Diagnosis}
\label{subsec:interactive_differential_diagnosis}
We now evaluate the more challenging task of interactive DDx, where we begin with limited patient information and the history taking simulator enables the interactive environment~(\autoref{tab:interactive_overall}).
At $n=0$, the simulator has not yet learned any patient information, and performance drops significantly from observing the full patient profile (\autoref{tab:with_patient_profile}). 
For GPT-4o in RareBench, the knowledge retrieval agent (KR)'s GTPA@1 drops from 0.45  to 0.07. Similarly, the diagnosis strategy agent (DS) drops from 0.46 (zero-shot) to 0.11. This simple baseline showcases that previous evaluations do not hold well in the interactive setup with initially limited patient information. 
Already for $n=5$, we find a large boost in performance for both KR and DS. These findings reinforce the importance of history taking for diagnostic precision. 
We illustrate the trend for changing $n$ in~\autoref{fig:ddxplus_comparison} and find that gains also plateau around \textit{n}=10-15 questions, reinforcing the optimal balance between information gathering and diagnostic efficiency \cite{ely1999analysis}.

Finally, we run MEDDxAgent, which calls KR+DS in the \textit{fixed iteration} pipeline (\autoref{subsec:iterative_learning}). MEDDxAgent exhibits clear improvements over the KR and DS baselines for $n=5$, supporting our hypothesis that all three modules are important for interactive DDx. It also improves significantly over the history taking baselines, as we illustrate in \autoref{fig:ddxplus_comparison}. MEDDxAgent is also capable of improving upon the zero-shot setting with the full patient profile (\autoref{tab:with_patient_profile}). For DDxPlus, GTPA@1 for GPT-4o and Llama3.1-70B rise from 0.56 to 0.86 and from 0.46 to 0.71, respectively. For Llama3.1-8B, the trend continues for DDxPlus but inconsistently for iCraft-MD and RareBench, highlighting the importance of model scale. Notably, MEDDxAgent improves over successive iterations, though the optimal number of iterations (2, 3) depends on the dataset and LLM. The values of $\Delta$ are consistently positive, indicating that MEDDxAgent iteratively increases the rank of the ground-truth diagnosis over time. $\Delta$ Progress also varies by dataset and model, offering explainable insight to the diagnosistic improvement of MEDDxAgent. The overall results show that MEDDxAgent can operate well in the challenging, realistic setup of interactive DDx. Additionally, MEDDxAgent logs all intermediate reasoning, action, and observations, providing critical insight into its DDx process (\autoref{fig:Example}).
\vspace{-0.5em}
\section{Analysis}
\vspace{-0.5em}
\label{analysis}
\paragraph{Fixed vs. Dynamic Iterations.}
A key feature of MEDDxAgent is its iterative DDx process, which operates in \textit{fixed} or \textit{dynamic} iteration. Our experiments (see~\autoref{tab:choice_merged} in Appendix) show that fixed iteration consistently outperforms dynamic iteration in both accuracy and system efficiency.\footnote{On average, fixed iteration is 1.2x-1.7x faster than dynamic iteration.} Fixed iteration ensures a structured sequence where all modules -- history taking simulator, knowledge retrieval agent, and diagnosis strategy agent, are utilized in each cycle, preventing over-reliance on a single component. In contrast, dynamic iteration, which allows DDxDriver to choose the component at each step, introduces some suboptimal decision-making. We observe that Llama3.1 models, for instance, frequently favor the history taking simulator rather than leveraging the knowledge retrieval or diagnosis strategy agents, leading to redundant questioning rather than efficient diagnostic reasoning. Despite this, our findings demonstrate the general applicability of MEDDxAgent for dynamic iteration and highlight future work toward optimizing dynamic iteration for interactive DDx.

\paragraph{Error Analysis.}
To better understand the struggles of MEDDxAgent, we conduct error analysis on cases where it failed to reach the correct diagnosis efficiently. We emphasize that our MEDDxAgent's logging of intermediate logic greatly enhances our understanding and explanations of failure cases. First, in RareBench, over-reliance on few-shot examples often misprioritizes frequent conditions over rarer diseases, as some rare conditions are underrepresented in knowledge retrieval databases. Second, in the first iteration, MEDDxAgent tends to prioritize the knowledge retrieval while overlooking few-shot patient examples with similar profiles. This is then mitigated when further iterations help to refine and enhance the DDx process.
Third, larger models (GPT-4o, Llama3.1-70B) benefit more from iterative refinement, while smaller models (Llama3.1-8B) plateau after the second iteration, especially for iCraft-MD and RareBench. Addressing these challenges can further enhance MEDDxAgent’s diagnostic accuracy.
\section{Related Work}
\subsection{LLM-based Methods} %With the rise of ChatGPT \cite{achiam2023gpt}, researchers have studied its performance for a variety of tasks, including automatic diagnosis. 
Researchers have studied the capabilities of LLMs for automatic diagnosis \cite{mizuta2024diagnosis}. One line of work have found that the performance of LLMs is comparable to the performance of physicians \cite{hirosawa2023chatgpt,rutledge2024diagnostic} or that the performance of the physicians themselves is improved when they use LLMs \cite{ten2024chatgpt}. However, researchers have also observed that LLMs struggle to perform this task when applied on rarer diseases or on more unusual cases \cite{fabre2024evaluating,shikino2024evaluation}. For this reason, we assemble a benchmark that measures performance for different rarity levels.

Many of the current methods are typically targeting either automatic standard diagnosis or automatic differential diagnosis in an end-to-end manner. For example, there are methods that use Chain-of-Thought strategies \cite{wu2023large,savage2024diagnostic,nachane2024few}, reinforcement learning \cite{fansi2022towards}, fine-tuning LLMs \cite{alam2023ddxt,reese2024limitations}, preranking-reranking methods \cite{sun2024conversational} or specifically trained neural networks \cite{liu2020deep,hwang2022differential}. Such LLM-based methods do not allow for modularity, thus posing difficulties in integrating specific modules that solve certain sub-problems within the diagnosis process.

\subsection{Agent-based Methods}
Recent work shifts from standalone LLMs to multi-agent frameworks, enhancing efficiency by enabling external tools, and assigning specialized roles to each agent to accomplish complex tasks more efficiently. In medical applications, agent-based methods streamline clinical workflows to improve diagnostic accuracy. KG4Diagnosis~\citep{zuo2024kg4diagnosis} integrated LLMs with knowledge graphs (KGs) for medical diagnosis. However, its static KG dependence makes expanding diagnoses for rare diseases difficult, and the lack of iterative refinement limits adaptability to evolving clinical cases. \citet{wu2024streambench} presented StreamBench to evaluate the continuous improvement of LLM agents in streaming environments via simulated feedback. However, it evaluates full patient profiles and a single DDx dataset, limiting its generalizability.
Other frameworks, such as AMIE~\citep{tu2024towards}, AMSC~\cite{wang2024beyond}, AgentHospital \cite{li2024agent}, MediQ~\cite{li2024mediq}, and MedAgents~\citep{tang-etal-2024-medagents} address different aspects of clinical interactions. However, these approaches suffer from (i) lack of iterative refinement, relying on single-turn reasoning; (ii) limited evaluation, focusing on a single dataset or a single diagnostic component, missing diagnostic generalization; (iii) assuming full patient profiles, which does not reflect real-world interactive differential diagnosis -- where information is collected iteratively. We address these aspects in our work, proposing a more challenging interactive setting and a modular, iterative agent framework. 
\vspace{-0.5em}
\section{Conclusions}
\section{Conclusion}\label{sec:conclusion}
This work introduces a novel approach to TOT query elicitation, leveraging LLMs and human participants to move beyond the limitations of CQA-based datasets. Through system rank correlation and linguistic similarity validation, we demonstrate that LLM- and human-elicited queries can effectively support the simulated evaluation of TOT retrieval systems. Our findings highlight the potential for expanding TOT retrieval research into underrepresented domains while ensuring scalability and reproducibility. The released datasets and source code provide a foundation for future research, enabling further advancements in TOT retrieval evaluation and system development.

\section*{Limitations}
While our proposed MEDDxAgent framework advances the task of automatic differential diagnosis through a modular, explainable, and interactive approach, we acknowledge certain limitations. (1) Model Selection: Our evaluation focuses on Llama models (8B and 70B) and GPT-4o. These models demonstrate strong instruction-following capabilities, so findings may not generalize to all LLM architectures. We also evaluate state-of-the-art medical-domain LLMs~\citep{zhang2024ultramedical} but find that this instruction-tuned model underperforms the general-purpose LLMs (e.g., Llama3.1~\citep{dubey2024llama}), likely due to diverse instruction-following behavior. Further exploration is needed for models with different architectures, training paradigms, or domain adaptations. (2) Language Coverage: our framework is evaluated primarily on an English DDx benchmark, limiting its applicability to non-English-speaking regions, where medical terminology, case presentations, and healthcare practices may differ significantly. Extending the framework for multilingual and cross-lingual diagnostic tasks remains an important direction \cite{qiu2024towards, garcia-ferrero-etal-2024-medmt5}. (3) Multimodality: Medical information often relies on multimodal data, such as medical imaging and videos, laboratory tests, electronic health records, and genomic/pathology data. Our DDx Benchmark is soley text-based, which limits the scope of its applicability to clinical decision making. Future work should explore multimodal agentic frameworks \cite{schmidgall2024agentclinic, kim2024mdagents} and reasoning over multimodal sequential data \cite{rose2023visual, himakunthala-etal-2023-lets, zhu2024emerge}. (4) Benchmark Dataset Selection: We introduce a DDx benchmark encompassing respiratory, skin, and rare diseases. Though our benchmark offers greater diversity than prior work, it does not cover all medical specialties or real-world patient distributions. Expanding datasets to reflect a wider range of diseases, demographics, and clinical settings would improve generalizability.

Despite these limitations, we hope this work contributes to advancing automatic DDx with a challenging yet realistic setup -- interactive differential diagnosis. We hope that future research builds on top of our findings to include more languages, modalities, datasets, and a deeper exploration of diagnostic processes to enhance the applicability and effectiveness of interactive diagnostic models.


%\section*{Acknowledgments}


% Bibliography entries for the entire Anthology, followed by custom entries
%\bibliography{anthology,custom}
% Custom bibliography entries only
%\newpage
\bibliography{custom}

\appendix
\clearpage
\onecolumn
% \section{Framework Details}
% Our framework is described in Algorithm~\ref{algorithm}, and compared with former baselines in Table~\ref{table:comparison}. Distinct with several methods generating Python code for visualization directly, we use VQL as an intermediate representation to bridge natural language queries and visualization code. Additionally, our framework can be easily optimized by adding some useful tools such as Retrieval Augmented Generation. Moreover, our method supports handling multi-table data and the visualization can be customized according to humans' preferences. Our framework utilizes the agent-based collaborative workflow, which consists of data preprocessing, generation, and error correction, organized with the modular design.

% \begin{algorithm}
% \small
% \caption{\system Framework}
% \label{algorithm}
% \begin{algorithmic}[1]
% \Function{\nlvis}{$Q$, $S$}
%     \State Initialize $Mem \gets \{Q,S\}$
%     \State $(S', A) \gets \textsc{Processor}(Mem)$
%     \State $Mem.update(S', A)$
%     \State $V \gets \textsc{Composer}(Mem)$
%     \State $Mem.update(V)$
%     \State $Chart, isValid \gets \textsc{Validator}(Mem)$
%     \While{not $isValid$}
%         \State $V \gets \textsc{Refine}(Mem)$
%         \State $Mem.update(V)$
%         \State $Chart, isValid \gets \textsc{Validator}(Mem)$
%     \EndWhile
%     \State \Return $Chart$
% \EndFunction
% \end{algorithmic}

% \end{algorithm}




% \begin{table*}[!t]
%     \centering
    
%     \vspace{-1em}
%     \scalebox{0.68}{
%     \begin{tabular}{lccccccc}
%         \toprule[1.5pt]
%         \multirow{3}{*}{\textbf{Framework}} & \multicolumn{2}{c}{\textbf{System Features}} & \multicolumn{2}{c}{\textbf{Visualization Capabilities}} & \multicolumn{3}{c}{\textbf{Agentic Workflow}} \\
%         \cmidrule(lr){2-3} \cmidrule(lr){4-5} \cmidrule(lr){6-8}
%         & \textbf{VQL as} & \textbf{Extensible} & \textbf{Multi-Table} & \textbf{Customizable} & \textbf{Data} & \textbf{Modular} & \textbf{Error-} \\
%         & \textbf{Thoughts} & \textbf{Optimization} & \textbf{Support} & \textbf{Styling} & \textbf{Preprocess} & \textbf{Design} & \textbf{Correction} \\
%         \midrule
%         Chat2VIS~\cite{chat2vis} & \textcolor{red}{\ding{56}} & \textcolor{red}{\ding{56}} & \textcolor{red}{\ding{56}} & \textcolor{red}{\ding{56}} & \textcolor{green!60!black}{\ding{52}} & \textcolor{red}{\ding{56}} & \textcolor{red}{\ding{56}} \\
%         Mirror~\cite{mirror} & \textcolor{red}{\ding{56}} & \textcolor{red}{\ding{56}} & \textcolor{red}{\ding{56}} & \textcolor{red}{\ding{56}} & \textcolor{red}{\ding{56}} & \textcolor{green!60!black}{\ding{52}} & \textcolor{red}{\ding{56}} \\
        
%         LIDA~\cite{lida} & \textcolor{red}{\ding{56}} & \textcolor{green!60!black}{\ding{52}} & \textcolor{red}{\ding{56}} & \textcolor{green!60!black}{\ding{52}} & \textcolor{green!60!black}{\ding{52}} & \textcolor{green!60!black}{\ding{52}} & \textcolor{red}{\ding{56}} \\
%         CoML4VIS~\cite{coml} & \textcolor{red}{\ding{56}} & \textcolor{red}{\ding{56}} & \textcolor{green!60!black}{\ding{52}} & \textcolor{red}{\ding{56}} & \textcolor{green!60!black}{\ding{52}} & \textcolor{red}{\ding{56}} & \textcolor{red}{\ding{56}} \\
        
%         Prompt4VIS~\cite{prompt4vis} & \textcolor{green!60!black}{\ding{52}} & \textcolor{red}{\ding{56}} & \textcolor{green!60!black}{\ding{52}} & \textcolor{red}{\ding{56}} & \textcolor{green!60!black}{\ding{52}} & \textcolor{green!60!black}{\ding{52}} & \textcolor{red}{\ding{56}} \\
        
%         CoT-Vis~\cite{cotvis} & \textcolor{green!60!black}{\ding{52}} & \textcolor{red}{\ding{56}} & \textcolor{red}{\ding{56}} & \textcolor{red}{\ding{56}} & \textcolor{green!60!black}{\ding{52}} & \textcolor{red}{\ding{56}} & \textcolor{red}{\ding{56}} \\

%         \midrule
%         \SystemName (Ours) & \textcolor{green!60!black}{\ding{52}} & \textcolor{green!60!black}{\ding{52}} & \textcolor{green!60!black}{\ding{52}} & \textcolor{green!60!black}{\ding{52}} & \textcolor{green!60!black}{\ding{52}} & \textcolor{green!60!black}{\ding{52}} & \textcolor{green!60!black}{\ding{52}} \\
%         \bottomrule[1.5pt]
%     \end{tabular}}
% \caption{Comparison of various \nlvis frameworks. }  \label{table:comparison}
% \vspace{-1em}
% \end{table*}

\section{Detailed Experiment Setups}
\label{detailed_experiment_setups}
\paragraph{Baselines.}
\label{detailed_baselines}
% We implemented our experiment compared with three recent baselines. Note that, we also tried to use Code Interpreter as a baseline, but due to the rate limit of API constraint, the evaluation failed to generate visualizations via direct .csv files.
This study compares our approach with three state-of-the-art baselines. We also attempted to include Code Interpreter as a baseline; however, API rate limitations prevent the direct generation of visualizations from CSV files.

\begin{itemize}[leftmargin=*, itemsep=0pt] 
    \item \textbf{Chat2Vis} \cite{chat2vis}: It generates data visualizations by leveraging prompt engineering to translate natural language descriptions into visualizations. It uses a language-based table description, which includes column types and sample values, to inform the visualization generation process.\item \textbf{LIDA} \cite{lida}: It structures visualization generation as a four-step process, where each step builds on the previous one to incrementally translate natural language inputs into visualizations. It uses a JSON format to describe column statistics and samples, making it adaptable across various visualization tasks.
    \item \textbf{CoML4Vis} \cite{coml}: 
    % Building on a data science code generation framework, CoML4Vis 
    It utilizes a few-shot prompt that integrates multiple tables into a single visualization task. It summarizes data table information, including column names and samples, and then applies a few-shot prompt to guide visualization generation.
\end{itemize}

\paragraph{Metrics.}
\label{detailed_metrics}
Our evaluation framework involves five main metrics:
\begin{itemize}[leftmargin=*, itemsep=0pt] 
    \item \textbf{Invalid Rate} represents the percentage of visualizations that fail to render due to issues like incorrect API usage or other code errors.
    \item \textbf{Illegal Rate} indicates the percentage of visualizations that do not meet query requirements, which can include incorrect data transformations, mismatched chart types, or improper visualizations.
    \item \textbf{Readability Score} is the average score (range 1-5) assigned by a vision language model, like GPT-4V, for valid and legal visualizations, assessing their visual clarity and ease of interpretation.
    \item \textbf{Pass Rate} measures the proportion of visualizations in the evaluation set that are both valid (able to render) and legal (meet the query requirements).
    \item \textbf{Quality Score} is set to 0 for invalid or illegal visualizations; otherwise, it is equal to the readability score, providing an overall assessment of visualization quality factoring in both functionality and clarity.
\end{itemize}
To thoroughly evaluate each main metric, we further break them down into the following detailed assessment criteria:
\begin{itemize}[leftmargin=4mm, itemsep=0.05mm] 
    \item \textbf{Code Execution Check} verifies that the Python code generated by the model can be successfully executed.
    \item \textbf{Surface-form Check} ensures that the generated code includes necessary elements to produce a visualization like function calls to display the chart.
    \item \textbf{Chart Type Check} verifies whether the extracted chart type from the visualization matches the ground truth.
    \item \textbf{Data Check} assesses if the data used in the visualization matches the ground truth, taking into consideration potential channel swaps based on specified channels.
    \item \textbf{Order Check} evaluates whether the sorting of visual elements follows the specified query requirements.
    \item \textbf{Layout Check} examines issues like text overflow or element overlap within visualizations.
    \item \textbf{Scale \& Ticks Check} ensures that scales and ticks are appropriately chosen, avoiding unconventional representations.
    \item \textbf{Overall Readability Rating} integrates various readability checks to provide a comprehensive score considering layout, scale, text clarity, and arrangement.
\end{itemize}

% For all evaluation results, these metrics are averaged across the dataset to provide an overarching view of model performance. These metrics collectively ensure that visualizations are not only correct in terms of execution but also effective in communicating the intended data narratives.
The evaluation metrics are averaged across the dataset to provide a comprehensive overview of the model's performance. Together, these metrics ensure that the visualizations are both accurate in execution and effective in conveying the intended data narratives.



\begin{table}[!t]
\centering
\setlength{\belowcaptionskip}{0em} 
% \vspace{-1em}
\begin{tabular}{lcc}
\toprule[1.5pt]
\textbf{Model} & \textbf{P-corr} & \textbf{P-value} \\
\midrule
GPT-4o-mini & \textbf{0.6503} & 0.000 \\
GPT-4o & 0.5648 & 0.000 \\
\bottomrule[1.5pt]
\end{tabular}
\caption{ The Pearson correlations of GPT-4o-mini and GPT-4o with human judgments on readability scores. }
\label{tab:pearson_corr}
\vspace{-1em}
\end{table}

\begin{table*}[!ht]
\centering

\vspace{-1em}
\begin{tabular}{l|ccc|ccc}
\toprule
\multirow{2}{*}{Method} & \multicolumn{3}{c|}{Single Table} & \multicolumn{3}{c}{Multiple Tables} \\
\cmidrule(l){2-4} \cmidrule(l){5-7}
 & prompt & response & total & prompt & response & total \\
\midrule
LIDA & 1386.23 & 237.90 & 1624.13 & \multicolumn{3}{c}{N/A} \\
Chat2Vis & 414.35 & 451.30 & 865.65 & \multicolumn{3}{c}{N/A} \\
CoML4Vis & 2614.76 & 279.86 & 2894.62 & 3069.62 & 307.67 & 3377.29 \\
\system & 5122.99 & 777.63 & 5900.62 & 5613.96 & 1014.10 & 6628.06 \\
\bottomrule
\end{tabular}
\caption{Token usage comparison for different methods. N/A indicates that LIDA and Chat2Vis cannot handle multiple table scenarios.}
\label{tab:token_usage}
\end{table*}

\begin{table}[ht]
\centering
\scalebox{1}{
\begin{tabular}{l|ccc}
\toprule
Agent & \#Input & \#Output & \#Total \\
\midrule
Processor & 1486.07 & 569.58 & 1755.65\\
Composer & 3268.32 & 221.74 & 3490.07 \\
Validator & 1051.82 & 127.85 & 1179.67  \\
\bottomrule
\end{tabular}}
\caption{Token usage of three agents in \system.} \label{tab:token_agent} 
\vspace{-1em}
\end{table}

\paragraph{Implement Details.}
Our system is implemented in Python 3.9, utilizing GPT-4o \citep{openai_gpt4o_2024}, GPT-4o-mini~\cite{openai2024gpt4omini}, and GPT-3.5-turbo~\cite{chatgpt3.5} as the backbone model for all approaches, with the temperature set to 0 for consistent outputs. GPT-4o-mini serves as the vision language model for readability evaluation. We interact with these models through the Azure OpenAI API. The specific prompt templates for each agent, crucial for guiding their respective roles in the visualization generation process, are detailed in Appendix~\ref{prompt_details}. Token usages of \system and baselines are demonstrated in Table~\ref{tab:token_usage}, and usage for each agent in our \system is shown in Table~\ref{tab:token_agent}. Additionally, our evaluations are conducted in VisEval Benchmark (with MIT license).

\paragraph{Human Annotation.}
\label{human}
The annotation is conducted by 5 authors of this paper independently. As acknowledged, the diversity of annotators plays a crucial role in reducing bias and enhancing the reliability of the benchmark. These annotators have knowledge in the data visualization domain, with different genders, ages, and educational backgrounds. The educational backgrounds of annotators are above undergraduate. To ensure the annotators can proficiently mark the data, we provide them with detailed tutorials, teaching them how to judge the quality of data visualization. We also provide them with detailed criteria and task requirements in each annotation process shown in Figure~\ref{fig:annotation}. Two experiments requiring human annotation are detailed as follows:

\begin{figure}[!ht]
    \centering
    \includegraphics[width=\linewidth]{figure/score_distribution.pdf}
    \caption{Comparison of score density distribution between GPT-4o, GPT-4o-mini and human average score.}
    \label{fig:score_distribution}
\end{figure}

\begin{table*}[!ht]
\centering
\begin{tabular}{l|ccc}
\toprule
& Invalid Rate & Illegal Rate & Pass Rate \\
\midrule
\system & 4.66\% & 23.97\% & 71.35\% \\
w. CoT for Validator & 5.82\% & 23.39\% & 70.78\% \\
w. original schema for Validator & 4.80\% & 24.22\% & 70.97\% \\
\bottomrule
\end{tabular}
\caption{Additional exploration for Validator (using GPT-3.5-turbo).} 
\vspace{-1em} 
\label{tab:ablation_validator}
\end{table*}

\begin{itemize}[leftmargin=*, itemsep=0pt]
    \item \textbf{Pearson Correlation of Visual Language Model.} We conduct human annotation frameworks to compare the ability of the visual language model for MLLM-as-a-Judge~\cite{chen2024mllm}, providing the readability score. Our annotation framework is shown in Figure~\ref{fig:annotation}. The final Pearson scores are demonstrated in Table~\ref{tab:pearson_corr}, with its density distribution in Figure~\ref{fig:score_distribution}. The detailed instructions can be found in Figure~\ref{fig:scoring_instructions}.
    \item \textbf{Qualitative comparison to calculate ELO Scores.} We conduct human-judgments evaluations to compare which visualization generated by different models meets the query requirement more precisely. The leaderboard is shown in Table~\ref{tab:elo_rankings}, and Figure~\ref{fig:elo} shows the judgment framework. Each model starts with a base ELO score of 1500. After each pairwise comparison, the scores are updated based on the outcome and the current scores of the models involved. The hyperparameters are set as follows: the $K$-factor is set to 32, which determines the maximum change in rating after a single comparison. We conduct two sets of evaluations: one for single-table queries and another for multiple-table queries, with 1000 bootstrap iterations for each set to ensure statistical robustness. For each model's ELO rating, we report the 95\% confidence intervals computed through bootstrap resampling, providing a measure of rating stability. The evaluation process involves presenting human judges with a query and two visualizations, asking them to select the one that better meets the query requirements. This process is repeated across all model pairs and queries in our test set. The detailed guidance provides to the human evaluators can be found in Figure~\ref{fig:evaluation_instructions}, which outlines the criteria for judging visualization quality and relevance to the given query.


\end{itemize}

\begin{figure}[!ht]
	\centering
    \setlength{\belowcaptionskip}{-1em}
	\includegraphics[width=0.98\linewidth,scale=1.0]
    {./figure/library.pdf}
    \vspace{-1em}
	\caption{Performance of different models using \texttt{Matplotlib} and \texttt{Seaborn} libraries, using GPT-3.5-turbo.
    % \yao{larger fontsize?}
    }
\label{fig: library}
\end{figure}

\begin{figure*}[!h]
    \centering
    \includegraphics[width=0.98\linewidth]{figure/annotation.pdf}
    \caption{Screenshot of human annotation process in readability score.}
    \label{fig:annotation}
\end{figure*}

\begin{figure*}[ht]
\centering
\vspace{1em}
\begin{tcolorbox}[enhanced,attach boxed title to top center={yshift=-3mm,yshifttext=-1mm},boxrule=0.9pt, 
  colback=gray!00,colframe=black!50,colbacktitle=gray,
  title=Readability Scoring Instruction,
  boxed title style={size=small,colframe=gray} ]
\small
\textbf{Scoring Instructions:} Please evaluate the charts based on the following criteria, with a score range from 1 to 5, where 1 indicates very poor quality and 5 indicates excellent quality. You should focus on the following aspects:

\vspace{0.5em}
\textbf{1. Chart Colors:}
\begin{itemize}
    \item Are the colors clear and natural, effectively conveying the information?
    \item Color blindness accessibility: Are the color combinations easy to distinguish, especially for users with color blindness?
\end{itemize}

\vspace{0.5em}
\textbf{2. Title and Axis Labels:}
\begin{itemize}
    \item Ensure the chart has a clear title.
    \item Do the X-axis and Y-axis labels exist, and are they complete?
    \item Check if the labels are difficult to read, e.g., are they written vertically instead of horizontally?
    \item The title should not be a direct question; instead, it should describe the data or trends being presented.
\end{itemize}

\vspace{0.5em}
\textbf{3. Legend Completeness:}
\begin{itemize}
    \item Is the legend complete, and does it clearly indicate the color labels for different data series?
    \item Ensure each color has a corresponding legend, making it easy for users to understand what the data represents.
\end{itemize}

\vspace{0.5em}
\textbf{Scoring Scale:}
\begin{itemize}
    \item \textbf{1 Point:} Very poor, unable to understand or severely lacking information.
    \item \textbf{2 Points:} Poor quality, multiple issues present, difficult to extract information.
    \item \textbf{3 Points:} Fair, conveys some information but still has room for improvement.
    \item \textbf{4 Points:} Good, generally clear charts with minor areas for improvement.
    \item \textbf{5 Points:} Excellent, outstanding chart design with clear and effective information presentation.
\end{itemize}

Please consider the above factors when assessing the charts and provide the appropriate score. Thank you for your cooperation and effort!
\end{tcolorbox}
\vspace{-7pt}
\caption{Instructions for human annorators in annotating readability scoring.}
\label{fig:scoring_instructions}
\vspace{1em}
\end{figure*}

\begin{figure*}[!ht]
    \centering
    \includegraphics[width=0.98\linewidth]{figure/elo.pdf}
    \caption{Screenshot of ELO score evaluation framework for Human-as-a-Judge.}
    \label{fig:elo}
\end{figure*}

\begin{figure*}[ht]
\centering
\vspace{1em}
\begin{tcolorbox}[enhanced,attach boxed title to top center={yshift=-3mm,yshifttext=-1mm},boxrule=0.9pt, 
  colback=gray!00,colframe=black!50,colbacktitle=gray,
  title=Visualization Comparison Guidance,
  boxed title style={size=small,colframe=gray} ]
\small
Welcome to the visualization comparison evaluation. Your task is to judge which model-generated visualization better meets the requirements of the natural language query.

\vspace{0.5em}
\textbf{Evaluation criteria:}
\begin{enumerate}
    \item \textbf{Appropriateness of chart type:} Check if the selected chart type is suitable for expressing the data and relationships required by the query.
    \item \textbf{Data completeness:} Ensure the chart includes all necessary data required by the query.
    \item \textbf{Readability:} Assess the clarity of the chart, accuracy of labels, and overall layout.
    \item \textbf{Aesthetics:} Consider if the chart's color scheme, proportions, and overall design are visually pleasing.
    \item \textbf{Information conveyance:} Judge if the chart effectively conveys the main information or insights required by the query.
\end{enumerate}

\vspace{0.5em}
\textbf{Evaluation process:}
\begin{enumerate}
    \item Carefully read the natural language query.
    \item Observe the visualization results generated by two models.
    \item Based on the above criteria, choose the better visualization or select a tie if they are equally good.
    \item If neither visualization satisfies the query requirements well, please choose the relatively better one.
\end{enumerate}

Remember, your evaluation will help us improve and compare different visualization models. Thank you for your participation!
\end{tcolorbox}
\vspace{-7pt}
\caption{Instructions for human annorators in visualization comparison.}
\label{fig:evaluation_instructions}
\vspace{1em}
\end{figure*}


\section{Additional Experiment Results}
\label{additional_experiment_result}

We also conducted a comparison experiment of different methods using matplotlib or seaborn library. Figure~\ref{fig: library} demonstrates the results, indicating that our method outperforms obviously other baselines not only with matplotlib but also seaborn.

In addition, we test techniques in the Validator Agent, such as Chain-of-Thought. As is shown in Table~\ref{tab:ablation_validator}, integrating Chain-of-Thought reasoning, may affect its performance badly, likely due to the simple refining task with complex reasoning. Moreover, using the original schema to check for false schema filtering seems to be useless in this case.

\section{Evaluation Results with Detailed Metrics}
We demonstrated the main results in Table~\ref{tab:performance_comparison}, and here we reported more detailed results of other metrics in Table~\ref{tab:detailed_results}, which underscored the error rates for each stage, including \textit{Invalid}, \textit{Illegal}, and \textit{Low Readability}. 

\begin{table*}[!ht]
\centering
\footnotesize
\scalebox{0.98}{
\begin{tabular}{ll|cc|cccc|cc}
\toprule[1.5pt]
\multirow{2}{*}{Method} & \multirow{2}{*}{Dataset} & \multicolumn{2}{c|}{Invalid} & \multicolumn{4}{c|}{Illegal} & \multicolumn{2}{c}{Low Readability} \\
&  & Execution & Surface. & Decon. & Chart Type & Data & Order & Layout & Scale\&Ticks \\
\midrule
\multicolumn{10}{c}{ \textbf{\textit{GPT-4o}}}\\
\midrule
\multirow{3}{*}{CoML4Vis} & All & 1.15 & 0.00 & 0.26 & 1.75 & 14.28 & 10.36 & 32.02 & 32.55 \\
& Single & 0.67 & 0.00 & 0.43 & 1.93 & 13.54 & 10.16 & 31.08 & 32.76 \\
& Multiple & 1.87 & 0.00 & 0.00 & 1.48 & 15.39 & 10.66 & 33.43 & 32.23 \\
\cmidrule{2-10}
\multirow{3}{*}{LIDA} & All & 6.61 & 0.00 & 1.60 & 3.24 & 40.53 & 4.07 & 32.68 & 15.77 \\
& Single & 1.13 & 0.00 & 2.11 & 0.89 & 12.26 & 6.79 & 53.93 & 26.22 \\
& Multiple & 14.80 & 0.00 & 0.79 & 8.51 & 80.53 & 0.00 & 1.24 & 0.21 \\
\cmidrule{2-10}
\multirow{3}{*}{Chat2Vis} & All & 16.05 & 0.00 & 0.62 & 3.99 & 30.14 & 5.96 & 2.37 & 20.88 \\
& Single & 0.86 & 0.00 & 0.75 & 2.30 & 10.78 & 9.73 & 3.97 & 34.63 \\
& Multiple & 38.74 & 0.00 & 0.43 & 6.51 & 59.08 & 0.32 & 0.00 & 0.34 \\
\cmidrule{2-10}
\multirow{3}{*}{nvAgent} & All & 0.97 & 0.00 & 0.08 & 1.28 & 11.07 & 4.05 & 5.07 & 40.03 \\
& Single & 0.72 & 0.00 & 0.14 & 1.27 & 9.88 & 3.60 & 3.92 & 39.36 \\
& Multiple & 1.34 & 0.00 & 0.00 & 1.30 & 12.84 & 4.73 & 6.79 & 41.03 \\
\midrule
\multicolumn{10}{c}{ \textbf{\textit{GPT-4o-mini}}}\\
\midrule
\multirow{3}{*}{CoML4Vis} & All & 4.23 & 0.00 & 0.20 & 2.31 & 16.64 & 11.83 & 35.23 & 29.35 \\
& Single & 0.36 & 0.00 & 0.26 & 2.32 & 13.80 & 11.67 & 35.92 & 32.22 \\
& Multiple & 10.01 & 0.00 & 0.10 & 2.31 & 20.87 & 12.07 & 34.19 & 25.05 \\
\cmidrule{2-10}
\multirow{3}{*}{LIDA} & All & 12.50 & 0.00 & 0.40 & 4.92 & 40.02 & 5.80 & 27.87 & 17.05 \\
& Single & 9.09 & 0.00 & 0.44 & 2.53 & 12.91 & 9.68 & 45.69 & 28.32 \\
& Multiple & 17.61 & 0.00 & 0.33 & 8.51 & 80.53 & 0.00 & 1.24 & 0.21 \\
\cmidrule{2-10}
\multirow{3}{*}{Chat2Vis} & All & 15.45 & 0.17 & 0.17 & 4.21 & 31.90 & 8.20 & 2.14 & 18.97 \\
& Single & 2.14 & 0.29 & 0.41 & 2.53 & 11.99 & 9.68 & 45.69 & 28.32 \\
& Multiple & 35.78 & 0.00 & 0.00 & 6.70 & 61.66 & 0.00 & 0.92 & 0.32 \\
\cmidrule{2-10}
\multirow{3}{*}{nvAgent} & All & 5.14 & 0.00 & 0.00 & 2.40 & 16.33 & 10.61 & 41.06 & 27.00 \\
& Single & 1.97 & 0.00 & 0.14 & 2.97 & 15.21 & 7.49 & 39.30 & 32.39 \\
& Multiple & 8.15 & 0.00 & 0.00 & 2.31 & 20.87 & 12.07 & 34.19 & 25.05 \\
\midrule
\multicolumn{10}{c}{ \textbf{\textit{GPT-3.5-turbo}}}\\
\midrule
\multirow{3}{*}{CoML4Vis} & All & 9.28 & 0.00 & 0.62 & 1.91 & 15.83 & 12.86 & 25.09 & 27.73 \\ 
& Single & 6.17 & 0.00 & 0.89 & 2.50 & 14.71 & 13.20 & 26.10 & 29.93 \\ 
& Multiple & 13.92 & 0.00 & 0.21 & 1.04 & 17.51 & 12.36 & 23.57 & 24.43 \\ 
\cmidrule{2-10} 
\multirow{3}{*}{LIDA} & All & 53.43 & 0.00 & 1.27 & 3.56 & 22.33 & 0.53 & 14.90 & 6.62 \\ 
& Single & 47.32 & 0.00 & 1.91 & 2.81 & 13.03 & 0.89 & 24.43 & 11.05 \\ 
& Multiple & 62.57 & 0.00 & 0.32 & 4.68 & 36.23 & 0.00 & 0.65 & 0.00 \\ 
\cmidrule{2-10} 
\multirow{3}{*}{Chat2Vis} & All & 18.68 & 0.00 & 0.28 & 3.66 & 32.47 & 7.20 & 25.45 & 20.15 \\ 
& Single & 3.90 & 0.00 & 0.47 & 2.78 & 15.62 & 12.01 & 41.74 & 33.38 \\ 
& Multiple & 40.77 & 0.00 & 0.00 & 4.97 & 57.66 & 0.00 & 1.12 & 0.37 \\ 
\cmidrule{2-10} 
\multirow{3}{*}{nvAgent} & All & 4.66 & 0.00 & 0.08 & 3.06 & 18.24 & 5.64 & 5.25 & 35.34 \\ 
& Single & 2.98 & 0.00 & 0.14 & 2.84 & 15.08 & 5.69 & 3.62 & 37.57 \\ 
& Multiple & 7.18 & 0.00 & 0.00 & 3.38 & 22.95 & 5.56 & 7.69 & 32.02 \\
\bottomrule[1.5pt]
\end{tabular}
}
\caption{Detailed error rates (\%) for different methods.} 
\label{tab:detailed_results}
\end{table*}

\section{Case Study}
\label{example}
% To demonstrate our approach's effectiveness, we present several illustrative examples. Figure~\ref{fig:nl_vql} shows how our system translates natural language into a structured VQL representation. Figure~\ref{python code} and Figure~\ref{fig:example_chart} demonstrate the complete pipeline from query to visualization.
Figure~\ref{fig:nl_vql} shows an example of a natural language query with its corresponding VQL representation. The output Python code for visualization and the final bar chart are demonstrated in Figure~\ref{python code} and Figure~\ref{fig:example_chart}, respectively.
Furthermore, we provide a case study of \system performance on four hardness-level NL2Vis problems in VisEval in Figure \ref{hardness case}.

The easy case demonstrates accurate grouping in scatter plot relationships. The medium case shows correct handling of multi-table joins for continent-wise statistics. The hard case exhibits temporal data visualization with proper filtering. The extra hard case showcases complex operations including weekday binning and stacked visualization. These cases highlight our system's consistent performance across varying task complexities, particularly excelling in multiple table scenarios and complex aggregations.

\begin{figure*}[htbp]
\centering
\begin{tcolorbox}[enhanced,attach boxed title to top center={yshift=-3mm,yshifttext=-1mm},boxrule=0.9pt, 
  colback=gray!00,colframe=black!50,colbacktitle=gray,
  title=An Example of Natural Language Query and  Corresponding VQL,,
  boxed title style={size=small,colframe=gray} ]

\textbf{Natural Language Query:}\\
How many documents are stored? Bin the store date by weekday in a bar chart.\\
\tcbline
\textbf{Corresponding VQL:}\\
Visualize BAR \\
SELECT Date\_Stored, COUNT(Document\_ID)\\
FROM All\_Documents \\
GROUP BY Date\_Stored \\
BIN Date\_Stored BY WEEKDAY\\
\end{tcolorbox}
\caption{The natural language query case and its corresponding output VQL representation.}
\label{fig:nl_vql}
\end{figure*}

\lstset{
    basicstyle=\ttfamily\small,
    breaklines=true,
    numbers=left,
    numberstyle=\tiny,
    frame=single,
    showstringspaces=false,
    tabsize=4,
    keywordstyle=\color{blue},
    commentstyle=\color{green!60!black},
    stringstyle=\color{purple},
    breakatwhitespace=false,
    keepspaces=true
}
\begin{figure*}[htbp]
\begin{lstlisting}[
    language=Python,
    frame=single,
    numbers=left,
    numberstyle=\tiny,
    basicstyle=\tiny\ttfamily,    
    breaklines=true,
    breakatwhitespace=false,
    showstringspaces=false,
    framesep=1mm,                 % 减小框架边距
    framexleftmargin=1mm,         % 减小框架左边距
    columns=flexible,             % 灵活列宽
    keepspaces=true
]
import matplotlib.pyplot as plt 
import pandas as pd 
import os 
import duckdb

# Set data folder path
data_folder = 'E:/visEval_dataset/databases/cre_Doc_Tracking_DB'  

# Connect to database 
con = duckdb.connect(database=':memory:')  

# Read all CSV files and create view
csv_files = [f for f in os.listdir(data_folder) if f.endswith('.csv')] 
for file in csv_files:
    table_name = os.path.splitext(file)[0]
    con.execute(f"CREATE VIEW {table_name} AS SELECT * FROM read_csv_auto('{os.path.join(data_folder, file)}')")  
    
# Execute SQL query 
sql = f''' 
    SELECT Date_Stored, COUNT(Document_ID) AS count_Document_ID 
    FROM All_Documents 
    GROUP BY Date_Stored 
''' 
df = con.execute(sql).fetchdf() 
con.close()  

# Rename columns 
df.columns = ['Date_Stored','count_Document_ID'] 

# Apply binning operation
flag = True 
df['Date_Stored'] = pd.to_datetime(df['Date_Stored']) 
df['Date_Stored'] = df['Date_Stored'].dt.day_name()  

# Group by and calculate count 
if flag:
    df = df.groupby('Date_Stored').sum().reset_index() 

# Ensure all seven days of the week are included 
weekday_order = ['Monday', 'Tuesday', 'Wednesday', 'Thursday', 
                 'Friday', 'Saturday', 'Sunday'] 
df = df.set_index('Date_Stored').reindex(weekday_order, fill_value=0).reset_index()
df['Date_Stored'] = pd.Categorical(df['Date_Stored'], 
                                  categories=weekday_order, ordered=True) 
df = df.sort_values('Date_Stored')

# Create visualization 
fig, ax = plt.subplots(1, 1, figsize=(10, 4)) 
ax.spines['top'].set_visible(False) 
ax.spines['right'].set_visible(False) 
ax.bar(df['Date_Stored'], df['count_Document_ID']) 
ax.set_xlabel('Date_Stored') 
ax.set_ylabel('count_Document_ID') 
ax.set_title(f'BAR Chart of count_Document_ID by Date_Stored') 
plt.xticks(rotation=45) 
plt.tight_layout()  
plt.show()
\end{lstlisting}
\caption{An example of python code generating module within \system.}
\label{python code}
\end{figure*}


\begin{figure*}[!ht]
    \centering
    \includegraphics[width=0.98\linewidth,scale=1.0]{figure/bar_chart.pdf}
    \caption{An example of generated bar chart using \system.}
    \label{fig:example_chart}
\end{figure*}

\begin{figure*}[htbp]
\centering
\begin{tcolorbox}[enhanced,attach boxed title to top center={yshift=-3mm,yshifttext=-1mm},boxrule=0.9pt, 
  colback=gray!00,colframe=black!50,colbacktitle=gray,
  title=Examples of \textsc{nvAgent} performance on different hardness levels,
  boxed title style={size=small,colframe=gray} ]
  
\textbf{Hardness Level:} Easy \\
\begin{minipage}{0.45\linewidth}
    \textbf{Dataset}: \textit{Single}\\
    \textbf{Input Tables}: basketball\_match\\
    \textbf{Input Query}: Show the relation between acc percent and all\_games\_percent for each ACC\_Home using a grouped scatter chart.\\
\end{minipage}\hfill
\begin{minipage}{0.45\linewidth}
    \centering
    \textbf{Response}:
    \includegraphics[width=\linewidth]{figure/easy_3085.pdf} 
\end{minipage}
\tcbline

\textbf{Hardness Level:} Medium \\
\begin{minipage}{0.45\linewidth}
    \textbf{Dataset}: \textit{Multiple}\\
    \textbf{Input Tables}: car\_makers, car\_names, cars\_data, continents, countries, model\_list\\
    \textbf{Input Query}: Display a pie chart for what is the name of each continent and how many car makers are there in each one?\\
\end{minipage}\hfill
\begin{minipage}{0.55\linewidth}
    \centering
    \textbf{Response}:
    \includegraphics[width=\linewidth]{figure/medium_433.pdf} 
\end{minipage}
\tcbline

\textbf{Hardness Level:} Hard \\[1em]
\begin{minipage}{0.45\linewidth}
    \textbf{Dataset}: \textit{Multiple}\\
    \textbf{Input Tables}: advisor, classroom, course, department, instructor, prereq, section, student, takes, teaches, time\_slot\\
    \textbf{Input Query}: Find the number of courses offered by Psychology department in each year with a line chart.\\
\end{minipage}\hfill
\begin{minipage}{0.45\linewidth}
    \centering
    \textbf{Response}:
    \includegraphics[width=\linewidth]{figure/hard_611.pdf} 
\end{minipage}
\tcbline

\textbf{Hardness Level:} Extra Hard \\[1em]
\begin{minipage}{0.45\linewidth}
    \textbf{Dataset}: \textit{Multiple}\\
    \textbf{Input Tables}: Accounts, Documents, Documents\_with\_Expenses, Projects, Ref- \_Budget\_Codes, Ref\_Document\_Types, Statements\\
    \textbf{Input Query}: How many documents are created in each day? Bin the document date by weekday and group by document type description with a stacked bar chart, I want to sort Y in desc order.\\
\end{minipage}\hfill
\begin{minipage}{0.45\linewidth}
    \centering
    \textbf{Response}:
    \includegraphics[width=\linewidth]{figure/extra_851.pdf} 
\end{minipage}

\end{tcolorbox}
    \caption{Examples of \textsc{nvAgent}'s performance on different hardness levels in VisEval (easy, medium, hard, and extra hard.}
    \label{hardness case}
\end{figure*}


\clearpage
\onecolumn
\section{Prompts Details}
\label{prompt_details}
We provide detailed prompt design of our \system as follows.



\begin{promptbox}[Prompt template for Processor Agent]
You are an experienced and professional database administrator. Given a database schema and a user query, your task is to analyze the query, filter the relevant schema, generate an optimized representation, and classify the query difficulty. \\
\\
Now you can think step by step, following these instructions below. \\
\textbf{[Instructions]} \\
1. Schema Filtering: \\
\text{\ \ \ \ }- Identify the tables and columns that are relevant to the user query.\\
\text{\ \ \ \ }- Only exclude columns that are completely irrelevant.\\
\text{\ \ \ \ }- The output should be \{\{tables: [columns]\}\}.\\
\text{\ \ \ \ }- Keep the columns needed to be primary keys and foreign keys in the filtered schema.\\
\text{\ \ \ \ }- Keep the columns that seem to be similar with other columns of another table.\\
\\
2. New Schema Generation:\\
\text{\ \ \ \ }- Generate a new schema of the filtered schema, based on the given database schema and your filtered schema.\\
\\
3. Augmented Explanation:\\
\text{\ \ \ \ }- Provide a concise summary of the filtered schema to give additional knowledge.\\
\text{\ \ \ \ }- Include the number of tables, total columns, and any notable relationships or patterns.\\
\\
4. Classification:\\
For the database new schema, classify it as SINGLE or MULTIPLE based on the tables number.\\
\text{\ \ \ \ }- if tables number >= 2: predict MULTIPLE\\
\text{\ \ \ \ }- elif only one table: predict SINGLE\\
\\
==============================\\
Here is a typical example:\\
\textbf{[Database Schema]}\\
\textbf{[DB\_ID]} dorm\_1\\
\textbf{[Schema]}\\
\# Table: Student\\
\text{[}\\
  \text{\ \ \ \ }(stuid, And This is a id type column),\\
  \text{\ \ \ \ }(lname, Value examples: [`Smith', `Pang', `Lee', `Adams', `Nelson', `Wilson'].),\\
  \text{\ \ \ \ }(fname, Value examples: [`Eric', `Lisa', `David', `Sarah', `Paul', `Michael'].),\\
  \text{\ \ \ \ }(age, Value examples: [18, 20, 17, 19, 21, 22].),\\
  \text{\ \ \ \ }(sex, Value examples: [`M', `F'].),\\
  \text{\ \ \ \ }(major, Value examples: [600, 520, 550, 50, 540, 100].),\\
  \text{\ \ \ \ }(advisor, And this is a number type column),\\
  \text{\ \ \ \ }(city code, Value examples: [`PIT', `BAL', `NYC', `WAS', `HKG', `PHL'].)\\
\text{]}\\
% \end{promptbox}
% \end{figure*}
% \begin{figure*}[!h]
% \begin{promptbox}[Prompt template for Processor Agent]
\# Table: Dorm\\
\text{[}\\
  \text{\ \ \ \ }(dormid, And This is a id type column),\\
  \text{\ \ \ \ }(dorm name, Value examples: [`Anonymous Donor Hall', `Bud Jones Hall', `Dorm-plex 2000', `Fawlty Towers', `Grad Student Asylum', `Smith Hall'].),\\
  \text{\ \ \ \ }(student capacity, Value examples: [40, 85, 116, 128, 256, 355].),
  (gender, Value examples: [`X', `F', `M'].)\\
\text{]}\\
\# Table: Dorm\_amenity\\
\text{[}\\
  \text{\ \ \ \ }(amenid, And This is a id type column),\\
  \text{\ \ \ \ }(amenity name, Value examples: [`4 Walls', `Air Conditioning', `Allows Pets', `Carpeted Rooms', `Ethernet Ports', `Heat'].)\\
\text{]}\\
\# Table: Has\_amenity\\
\text{[}\\
  \text{\ \ \ \ }(dormid, And This is a id type column),\\
  \text{\ \ \ \ }(amenid, And This is a id type column)\\
\text{]}\\
\# Table: Lives\_in\\
\text{[}\\
  \text{\ \ \ \ }(stuid, And This is a id type column),\\
  \text{\ \ \ \ }(dormid, And This is a id type column),\\
  \text{\ \ \ \ }(room number, And this is a number type column)\\
\text{]}\\
\\
\textbf{[Query]}\\
Find the first name of students who are living in the Smith Hall, and count them by a pie chart\\
\\
Now we can think step by step\\
\textbf{[Filtered Schema]}\\
\{\\
  "Student": ["stuid", "fname"],\\
  "Dorm": ["dormid", "dorm name"],\\
  "Lives\_in": ["stuid", "dormid"]\\
\}\\
\\
\textbf{[New Schema]}\\
\# Table: Student\\
\text{[}\\
  (stuid, And This is a id type column),\\
  (fname, Value examples: [`Eric', `Lisa', `David', `Sarah', `Paul', `Michael'].),\\
\text{]}\\
% \end{promptbox}
% \end{figure*}
% \begin{figure*}[!h]
% \begin{promptbox}[Prompt template for Processor Agent]
\# Table: Dorm
\text{[}\\
  (dormid, And This is a id type column),\\
  (dorm name, Value examples: [`Anonymous Donor Hall', `Bud Jones Hall', `Dorm-plex 2000', `Fawlty Towers', `Grad Student Asylum', `Smith Hall'].),\\
\text{]}\\
\# Table: Lives\_in\\
\text{[}\\
  (stuid, And This is a id type column),\\
  (dormid, And This is a id type column),\\
\text{]}\\
\textbf{[Augmented Explanation]}\\
The filtered schema consists of 3 tables (Student, Dorm, and Lives\_in) with a total of 6 relevant columns. There is a many-to-one relationship between Student and Dorm through the Lives\_in junction table. The query involves joining these three tables to find students living in a specific dorm (Smith Hall).\\
\\
Key points:\\
1. The Lives\_in table acts as a bridge between Student and Dorm, allowing for the association of students with their dorms.\\
2. The `dorm name' column in the Dorm table is crucial for filtering the specific dorm (Smith Hall).\\
3. The `fname' column from the Student table is required for the final output.\\
\\
\textbf{[Classification]}\\
MULTIPLE\\
\\
==============================\\
Here is a new question:\\
\\
\textbf{[DB\_ID]} \{db\_id\}\\
\textbf{[Database Schema]}\\
\{db\_schema\}\\
\\
\textbf{[Query]}\\
\{query\}\\
\\
Now give your answer following this format strictly without other explanation:\\
\\
\textbf{[Filtered Schema]}\\
\\
\textbf{[New Schema]}\\
\\
\textbf{[Augmented Explanation]}\\
\\
\textbf{[Classification]}\\
\\
\end{promptbox}
% \end{figure*}

% \subsection{Composer Agent Prompt}
% \label{composer_prompt}
% \begin{figure*}[!h]
\begin{promptbox}[Prompt template for multiple classification]
Given a [Database schema] with [Augmented Explanation] and a [Question], generate a valid VQL (Visualization Query Language) sentence. VQL is similar to SQL but includes visualization components. \\
\\
Now you can think step by step, following these instructions below. \\
\textbf{[Background]} \\
VQL Structure:\\
Visualize [TYPE] SELECT [COLUMNS] FROM [TABLES] [JOIN] [WHERE] [GROUP BY] [ORDER BY] [BIN BY]\\
\\
You can consider a VQL sentence as "VIS TYPE + SQL + BINNING"\\
You must consider which part in the sketch is necessary, which is unnecessary, and construct a specific sketch for the natural language query.\\
\\
Key Components:\\
1. Visualization Type: bar, pie, line, scatter, stacked bar, grouped line, grouped scatter\\
2. SQL Components: SELECT, FROM, JOIN, WHERE, GROUP BY, ORDER BY\\
3. Binning: BIN [COLUMN] BY [INTERVAL], [INTERVAL]: [YEAR, MONTH, DAY, WEEKDAY]\\
\\
When generating VQL, we should always consider special rules and constraints:\\
\textbf{[Special Rules]} \\
a. For simple visualizations:\\
    \text{\ \ \ \ }- SELECT exactly TWO columns, X-axis and Y-axis(usually aggregate function)\\
b. For complex visualizations (STACKED BAR, GROUPED LINE, GROUPED SCATTER):\\
    \text{\ \ \ \ }- SELECT exactly THREE columns in this order!!!:\\
        \text{\ \ \ \ }\text{\ \ \ \ }1. X-axis\\
        \text{\ \ \ \ }\text{\ \ \ \ }2. Y-axis (aggregate function)\\
        \text{\ \ \ \ }\text{\ \ \ \ }3. Grouping column\\
c. When "COLORED BY" is mentioned in the question:\\
    \text{\ \ \ \ }- Use complex visualization type(STACKED BAR for bar charts, GROUPED LINE for line charts, GROUPED SCATTER for scatter charts)\\
    \text{\ \ \ \ }- Make the "COLORED BY" column the third SELECT column\\
    \text{\ \ \ \ }- Do NOT include "COLORED BY" in the final VQL\\     
d. Aggregate Functions:\\
    \text{\ \ \ \ }- Use COUNT for counting occurrences\\
    \text{\ \ \ \ }- Use SUM only for numeric columns\\
    \text{\ \ \ \ }- When in doubt, prefer COUNT over SUM\\
e. Time based questions:\\
    \text{\ \ \ \ }- Always use BIN BY clause at the end of VQL sentence\\
    \text{\ \ \ \ }- When you meet the questions including "year", "month", "day", "weekday"\\
    \text{\ \ \ \ }- Avoid using window function, just use BIN BY to deal with time base queries\\
% \end{promptbox}
% \end{figure*}
% \begin{figure*}[!h]
% \begin{promptbox}[Prompt template for multiple classification]
\textbf{[Constraints]} \\
- In SELECT <column>, make sure there are at least two selected!!!\\
- In FROM <table> or JOIN <table>, do not include unnecessary table\\
- Use only table names and column names from the given database schema\\
- Enclose string literals in single quotes\\
- If [Value examples] of <column> has `None' or None, use JOIN <table> or WHERE <column> is NOT NULL is better\\
- Ensure GROUP BY precedes ORDER BY for distinct values\\
- NEVER use window functions in SQL\\
\\
Now we could think step by step:\\
1. First choose visualize type and binning, then construct a specific sketch for the natural language query\\
2. Second generate SQL components following the sketch.\\
3. Third add Visualize type and BINNING into the SQL components to generate final VQL\\
\\
==============================\\
Here is a typical example:\\
\textbf{[Database Schema]}\\
\# Table: Orders, (orders)\\
\text{[}\\
  \text{\ \ \ \ }(order\_id, order id, And this is a id type column),\\
  \text{\ \ \ \ }(customer\_id, customer id, And this is a id type column),\\
  \text{\ \ \ \ }(order\_date, order date, Value examples: [`2023-01-15', `2023-02-20', `2023-03-10'].),\\
  \text{\ \ \ \ }(total\_amount, total amount, Value examples: [100.00, 200.00, 300.00, 400.00, 500.00].)\\
\text{]}\\
\# Table: Customers, (customers)\\
\text{[}\\
  \text{\ \ \ \ }(customer\_id, customer id, And this is a id type column),\\
  \text{\ \ \ \ }(customer\_name, customer name, Value examples: [`John', `Emma', `Michael', `Sophia', `William'].),\\
  \text{\ \ \ \ }(customer\_type, customer type, Value examples: [`Regular', `VIP', `New'].)\\
\text{]}\\
\textbf{[Augmented Explanation]}\\
The filtered schema consists of 2 tables (Orders and Customers) with a total of 7 relevant columns. There is a one-to-many relationship between Customers and Orders through the customer\_id foreign key.\\
\\
Key points:\\
1. The Orders table contains information about individual orders, including the order date and total amount.\\
2. The Customers table contains customer information, including their name and type (Regular, VIP, or New).\\
3. The customer\_id column links the two tables, allowing us to associate orders with specific customers.\\
% \end{promptbox}
% \end{figure*}
% \begin{figure*}[!h]
% \begin{promptbox}[Prompt template for multiple classification]
4. The order\_date column in the Orders table will be used for monthly grouping and binning.\\
5. The total\_amount column in the Orders table needs to be summed for each group.\\
6. The customer\_type column in the Customers table will be used for further grouping and as the third dimension in the stacked bar chart.\\
\\

The query involves joining these two tables to analyze order amounts by customer type and month, which requires aggregation and time-based binning.\\
\\
\textbf{[Question]}\\
Show the total order amount for each customer type by month in a stacked bar chart.\\
\\
Decompose the task into sub tasks, considering [Background] [Special Rules] [Constraints], and generate the VQL after thinking step by step:\\
\\
\textbf{Sub task 1:} First choose visualize type and binning, then construct a specific sketch for the natural language query\\
Visualize type: STACKED BAR, BINNING: True\\
VQL Sketch:\\
Visualize STACKED BAR SELECT \_ , \_ , \_ FROM \_ JOIN \_ ON \_ GROUP BY \_ BIN \_ BY MONTH\\
\\
\textbf{Sub task 2:} Second generate SQL components following the sketch.\\
Let's think step by step:\\
1. We need to select 3 columns for STACKED BAR chart, order\_date as X-axis, SUM(total\_amout) as Y-axis, customer\_type as group column.\\
2. We need to join the Orders and Customers tables.\\
3. We need to group by customer type.\\
4. We do not need to use any window function for MONTH.\\
\\
\text{sql}\\
```sql\\
SELECT O.order\_date, SUM(O.total\_amount), C.customer\_type\\
FROM Orders AS O\\
JOIN Customers AS C ON O.customer\_id = C.customer\_id\\
GROUP BY C.customer\_type\\
```\\
\\
\textbf{Sub task 3:} Third add Visualize type and BINNING into the SQL components to generate final VQL\\
\textbf{Final VQL:}\\
Visualize STACKED BAR SELECT O.order\_date, SUM(O.total\_amount), C.customer\_type FROM Orders O JOIN Customers C ON O.customer\_id = C.customer\_id GROUP BY C.customer\_type BIN O.order\_date BY MONTH\\
\\
% \end{promptbox}
% \end{figure*}
% \begin{figure*}[!h]
% \begin{promptbox}[Prompt template for multiple classification]
==============================\\
Here is a new question:\\
\\
\textbf{[Database Schema]}\\
\{desc\_str\}\\
\\
\textbf{[Augmented Explanation]}\\
\{augmented\_explanation\}\\
\\
\textbf{[Query]}\\
\{query\}\\
\\
Now, please generate a VQL sentence for the database schema and question after thinking step by step.\\

\end{promptbox}
% \end{figure*}


% \begin{figure*}[!h]
\begin{promptbox}[Prompt template for single classification]
Given a [Database schema] with [Augmented Explanation] and a [Question], generate a valid VQL (Visualization Query Language) sentence. VQL is similar to SQL but includes visualization components. \\
\\
Now you can think step by step, following these instructions below. \\
\textbf{[Background]} \\
VQL Structure:\\
Visualize [TYPE] SELECT [COLUMNS] FROM [TABLES] [JOIN] [WHERE] [GROUP BY] [ORDER BY] [BIN BY]\\
\\
You can consider a VQL sentence as "VIS TYPE + SQL + BINNING"\\
You must consider which part in the sketch is necessary, which is unnecessary, and construct a specific sketch for the natural language query.\\
\\
Key Components:\\
1. Visualization Type: bar, pie, line, scatter, stacked bar, grouped line, grouped scatter\\
2. SQL Components: SELECT, FROM, JOIN, WHERE, GROUP BY, ORDER BY\\
3. Binning: BIN [COLUMN] BY [INTERVAL], [INTERVAL]: [YEAR, MONTH, DAY, WEEKDAY]\\
\\
When generating VQL, we should always consider special rules and constraints:\\
\textbf{[Special Rules]} \\
a. For simple visualizations:\\
    \text{\ \ \ \ }- SELECT exactly TWO columns, X-axis and Y-axis(usually aggregate function)\\
b. For complex visualizations (STACKED BAR, GROUPED LINE, GROUPED SCATTER):\\
    \text{\ \ \ \ }- SELECT exactly THREE columns in this order!!!:\\
        \text{\ \ \ \ }\text{\ \ \ \ }1. X-axis\\
        \text{\ \ \ \ }\text{\ \ \ \ }2. Y-axis (aggregate function)\\
        \text{\ \ \ \ }\text{\ \ \ \ }3. Grouping column\\
c. When "COLORED BY" is mentioned in the question:\\
    \text{\ \ \ \ }- Use complex visualization type(STACKED BAR for bar charts, GROUPED LINE for line charts, GROUPED SCATTER for scatter charts)\\
    \text{\ \ \ \ }- Make the "COLORED BY" column the third SELECT column\\
    \text{\ \ \ \ }- Do NOT include "COLORED BY" in the final VQL\\     
d. Aggregate Functions:\\
    \text{\ \ \ \ }- Use COUNT for counting occurrences\\
    \text{\ \ \ \ }- Use SUM only for numeric columns\\
    \text{\ \ \ \ }- When in doubt, prefer COUNT over SUM\\
e. Time based questions:\\
    \text{\ \ \ \ }- Always use BIN BY clause at the end of VQL sentence\\
    \text{\ \ \ \ }- When you meet the questions including "year", "month", "day", "weekday"\\
    \text{\ \ \ \ }- Avoid using window function, just use BIN BY to deal with time base queries\\
% \end{promptbox}
% \end{figure*}
% \begin{figure*}[!h]
% \begin{promptbox}[Prompt template for single classification]
\textbf{[Constraints]} \\
- In SELECT <column>, make sure there are at least two selected!!!\\
- In FROM <table> or JOIN <table>, do not include unnecessary table\\
- Use only table names and column names from the given database schema\\
- Enclose string literals in single quotes\\
- If [Value examples] of <column> has `None' or None, use JOIN <table> or WHERE <column> is NOT NULL is better\\
- Ensure GROUP BY precedes ORDER BY for distinct values\\
- NEVER use window functions in SQL\\
\\
Now we could think step by step:\\
1. First choose visualize type and binning, then construct a specific sketch for the natural language query\\
2. Second generate SQL components following the sketch.\\
3. Third add Visualize type and BINNING into the SQL components to generate final VQL\\
\\
==============================\\
Here is a typical example:\\
\textbf{[Database Schema]}\\
\# Table: course, (course)\\
\text{[}\\
  \text{\ \ \ \ }(course\_id, course id, Value examples: [101, 696, 656, 659]. And this is an id type column),\\
  \text{\ \ \ \ }(title, title, Value examples: [`Geology', `Differential Geometry', `Compiler Design', `International Trade', `Composition and Literature', `Environmental Law'].),\\
  \text{\ \ \ \ }(dept\_name, dept name, Value examples: [`Cybernetics', `Finance', `Psychology', `Accounting', `Mech. Eng.', `Physics'].),\\
  \text{\ \ \ \ }(credits, credits, Value examples: [3, 4].)\\
\text{]}\\
\# Table: section, (section)\\
\text{[}\\
  \text{\ \ \ \ }(course\_id, course id, Value examples: [362, 105, 960, 468]. And this is an id type column),\\
  \text{\ \ \ \ }(sec\_id, sec id, Value examples: [1, 2, 3]. And this is an id type column),\\
  \text{\ \ \ \ }(semester, semester, Value examples: [`Fall', `Spring'].),\\
  \text{\ \ \ \ }(year, year, Value examples: [2002, 2006, 2003, 2007, 2010, 2008].),\\
  \text{\ \ \ \ }(building, building, Value examples: [`Saucon', `Taylor', `Lamberton', `Power', `Fairchild', `Main'].),\\
  \text{\ \ \ \ }(room\_number, room number, Value examples: [180, 183, 134, 143].),\\
  \text{\ \ \ \ }(time\_slot\_id, time slot id, Value examples: [`D', `J', `M', `C', `E', `F']. And this is an id type column)\\
\text{]}\\
\textbf{[Augmented Explanation]}\\
The filtered schema consists of 2 tables (course and section) with a total of 11 relevant columns. There is a one-to-many relationship between course and section through the course\_id foreign key.\\
\\
% \end{promptbox}
% \end{figure*}
% \begin{figure*}[!h]
% \begin{promptbox}[Prompt template for single classification]
Key points:\\
1. The course table contains information about individual courses, including the course title, department, and credits.\\
2. The section table contains information about specific sections of courses, including the semester, year, building, room number, and time slot.\\
3. The course\_id column links the two tables, allowing us to associate sections with specific courses.\\
4. The dept\_name column in the course table will be used to filter for Psychology department courses.\\
5. The year column in the section table will be used for yearly grouping and binning.\\
6. We need to count the number of courses offered each year, which requires aggregation and time-based binning.\\
\\
The query involves joining these two tables to analyze the number of courses offered by the Psychology department each year, which requires aggregation and time-based binning.\\
\\
\textbf{[Question]}\\
Find the number of courses offered by Psychology department in each year with a line chart.\\
\\
Decompose the task into sub tasks, considering [Background] [Special Rules] [Constraints], and generate the VQL after thinking step by step:\\
\\
\textbf{Sub task 1:} First choose visualize type and binning, then construct a specific sketch for the natural language query\\
Visualize type: LINE, BINNING: True\\
VQL Sketch:\\
Visualize LINE SELECT \_ , \_ FROM \_ JOIN \_ ON \_ WHERE \_ BIN \_ BY YEAR\\
\\
\textbf{Sub task 2:} Second generate SQL components following the sketch.\\
Let's think step by step:\\
1. We need to select 2 columns for LINE chart, year as X-axis, COUNT(year) as Y-axis.\\
2. We need to join the course and section tables to get the number of courses offered by the Psychology department in each year.\\
3. We need to filter the courses by the Psychology department.\\
4. We do not need to use any window function for YEAR.\\
\\
\text{sql}\\
```sql\\
SELECT S.year, COUNT(S.year)\\
FROM course AS C\\
JOIN section AS S ON C.course\_id = S.course\_id\\
WHERE C.dept\_name = `Psychology'\\
```\\
\\
% \end{promptbox}
% \end{figure*}
% \begin{figure*}[!h]
% \begin{promptbox}[Prompt template for single classification]
\textbf{Sub task 3:} Third add Visualize type and BINNING into the SQL components to generate final VQL\\
\textbf{Final VQL:}\\
Visualize LINE SELECT S.year, COUNT(S.year) FROM course C JOIN section S ON C.course\_id = S.course\_id WHERE C.dept\_name = `Psychology' BIN S.year BY YEAR\\
\\
==============================\\
Here is a new question:\\
\\
\textbf{[Database Schema]}\\
\{desc\_str\}\\
\\
\textbf{[Augmented Explanation]}\\
\{augmented\_explanation\}\\
\\
\textbf{[Query]}\\
\{query\}\\
\\
Now, please generate a VQL sentence for the database schema and question after thinking step by step.\\

\end{promptbox}
% \end{figure*}

% \subsection{Validator Agent Prompt}
% \label{validator_prompt}
% \begin{figure*}
\begin{promptbox}[Prompt template for Validator Agent]
As an AI assistant specializing in data visualization and VQL (Visualization Query Language), your task is to refine a VQL query that has resulted in an error. Please approach this task systematically, thinking step by step.\\
\textbf{[Background]}\\
VQL Structure:\\
Visualize [TYPE] SELECT [COLUMNS] FROM [TABLES] [JOIN] [WHERE] [GROUP BY] [ORDER BY] [BIN BY]\\
\\
You can consider a VQL sentence as "VIS TYPE + SQL + BINNING"\\
\\
Key Components:\\
1. Visualization Type: bar, pie, line, scatter, stacked bar, grouped line, grouped scatter\\
2. SQL Components: SELECT, FROM, JOIN, WHERE, GROUP BY, ORDER BY\\
3. Binning: BIN [COLUMN] BY [INTERVAL], [INTERVAL]: [YEAR, MONTH, DAY, WEEKDAY]\\
\\
When refining VQL, we should always consider special rules and constraints:\\
\textbf{[Special Rules]} \\
a. For simple visualizations:\\
    \text{\ \ \ \ }- SELECT exactly TWO columns, X-axis and Y-axis(usually aggregate function)\\
b. For complex visualizations (STACKED BAR, GROUPED LINE, GROUPED SCATTER):\\
    \text{\ \ \ \ }- SELECT exactly THREE columns in this order!!!:\\
        \text{\ \ \ \ }\text{\ \ \ \ }1. X-axis\\
        \text{\ \ \ \ }\text{\ \ \ \ }2. Y-axis (aggregate function)\\
        \text{\ \ \ \ }\text{\ \ \ \ }3. Grouping column\\
c. When "COLORED BY" is mentioned in the question:\\
    \text{\ \ \ \ }- Use complex visualization type(STACKED BAR for bar charts, GROUPED LINE for line charts, GROUPED SCATTER for scatter charts)\\
    \text{\ \ \ \ }- Make the "COLORED BY" column the third SELECT column\\
    \text{\ \ \ \ }- Do NOT include "COLORED BY" in the final VQL\\     
d. Aggregate Functions:\\
    \text{\ \ \ \ }- Use COUNT for counting occurrences\\
    \text{\ \ \ \ }- Use SUM only for numeric columns\\
    \text{\ \ \ \ }- When in doubt, prefer COUNT over SUM
% \end{promptbox}
% \end{figure*}

% \begin{figure*}
% \begin{promptbox}[Prompt template for Validator Agent]
e. Time based questions:\\
    \text{\ \ \ \ }- Always use BIN BY clause at the end of VQL sentence\\
    \text{\ \ \ \ }- When you meet the questions including "year", "month", "day", "weekday"\\
    \text{\ \ \ \ }- Avoid using time function, just use BIN BY to deal with time base queries\\
\\
\textbf{[Constraints]} \\
- In FROM <table> or JOIN <table>, do not include unnecessary table\\
- Use only table names and column names from the given database schema\\
- Enclose string literals in single quotes\\
- If [Value examples] of <column> has `None' or None, use JOIN <table> or WHERE <column> is NOT NULL is better\\
- ENSURE GROUP BY clause cannot contain aggregates\\
- NEVER use date functions in SQL\\
\\
\textbf{[Query]} \\
\{query\}\\
\\
\textbf{[Database info]} \\
\{db\_info\}\\
\\
\textbf{[Current VQL]} \\
\{vql\}\\
\\
\textbf{[Error]} \\
\{error\}\\
\\
Now, please analyze and refine the VQL, please provide:\\
\\
\textbf{[Explanation]}\\
\text{[}Provide a detailed explanation of your analysis process, the issues identified, and the changes made. Reference specific steps where relevant.\text{]}\\
\\
\textbf{[Corrected VQL]}\\
\text{[}Present your corrected VQL here. Ensure it's on a single line without any line breaks.\text{]}\\
\\
Remember:\\
- The SQL components must be parseable by DuckDB.\\
- Do not change rows when you generate the VQL.\\
- Always verify your answer carefully before submitting.\\
\end{promptbox}
% \end{figure*}

\end{document}
