Our proposed MEDDxAgent framework (see \autoref{fig:overview_ddxdriver} and a detailed version in~\autoref{fig:overview_framework}) comprises a central orchestrator (DDxDriver), a history taking simulator, and two specialized diagnostic agents dedicated to knowledge retrieval and diagnosis strategy. Both the simulator and diagnostic agents communicate exclusively with the DDxDriver, which monitors, stores, maintains, and updates patient information and ranked differential diagnoses. This central role also positions the DDxDriver to coordinate an iterative feedback loop, wherein observations from each agent are leveraged to enhance and refine subsequent agent calls with agent instructions (see an example in \autoref{fig:Example}). In the following, we introduce the design of simulator (\autoref{subsec:simulator}), agents (\autoref{subsec:diagnostic_agents}), orchestrator (DDxDriver) (\autoref{subsec:ddxdriver}), and iterative learning mechanism  (\autoref{subsec:iterative_learning}).

\subsection{Simulator}
% Clafify the setting without initial patient profile!
\label{subsec:simulator}

History taking is a critical first step in differential diagnosis, where clinicians gather essential information by asking patients questions about their symptoms, medical history, and lifestyle factors. In real-world clinical settings, a full patient profile is rarely available at the outset~\citep{li2024mediq} -- doctors typically start with only partial information (e.g., age, gender, chief complaint). The process of interactive DDx allows clinicians to gather more patient information and refine their diagnostic hypotheses before making a follow-up decision. 

To simulate such an interactive environment, we introduce a history taking simulator. We initialize the simulator with two LLMs \cite{wu2023large} in our experiments. The first LLM simulates the patient and receives access to the full patient profile. The second LLM simulates the doctor and receives an initial patient profile and optionally a set of conversational goals defined by DDxDriver (\textit{action}). During the interactions, the doctor role asks questions relevant to the diagnosis process, and the patient role provides answers based on its patient profile. 
The interaction continues until either the conversational goals are achieved or a predefined stopping criterion (e.g., maximum number of questions) is reached. Once the conversation concludes, the dialogue history is forwarded to DDxDriver.  

\begin{figure*}[t]
	\centering
    \includegraphics[trim={0cm 0.9cm 0cm 0cm},clip,width=0.9\textwidth]{img/Fig_example_ddxplus.pdf}
    %\vspace{-0.9em}
    \caption{An illustrated DDxPlus~\citep{fansi2022ddxplus} example with MEDDxAgent framework. Given the initial patient profile and list of diagnosis options, DDxDriver determines the goals and actions for the simulator (\textcolor{yellow-history}{\textbf{History Taking}}) and agents (\textcolor{blue-retrieval}{\textbf{Knowledge Retrieval}}, \textcolor{green-diagnosis}{\textbf{Diagnosis Strategy}}), updating the patient profile, and returning the ranked DDx. Each step is logged for transparency, enabling iterative refinement and learning.} 
    \vspace{-1.5em}
    \label{fig:Example}
\end{figure*}
\subsection{Agents}
\label{subsec:diagnostic_agents}


\paragraph{Knowledge Retrieval Agent.}
This agent aids the diagnostic process by retrieving relevant medical knowledge from external sources, such as scientific literature, medical databases, and clinical guidelines. This is particularly critical for diagnosing rare or complex conditions where external knowledge (as compared to internal knowledge learned by LLMs' training data) is required to enhance clinical reasoning with validated information.

Upon activation, the agent receives a search query formulated by DDxDriver, based on the current patient profile and provisional DDx list. It extracts the key medical concepts from the query as structured keywords, then conducts a targeted search in external databases. We consider two primary sources: Wikipedia and PubMed\footnote{\url{https://pubmed.ncbi.nlm.nih.gov/}}, with the former providing concise summaries of top-ranked pages, while the latter retrieves abstracts of full-access articles. The retrieved knowledge is synthesized into an evidence-based summary, which ensures that the diagnostic reasoning process has access to up-to-date, relevant medical knowledge.


\paragraph{Diagnosis Strategy Agent.}
This agent is responsible for generating, refining, and ranking possible diagnoses based on the information prepared by DDxDriver. % based on the previous agents' output. %The diagnosis instruction is sourced from the updated patient profile and/or retrieved information. 
There are two distinct modes that can be chosen for the diagnosis strategy agent. First, in the zero-shot setting, the LLM predicts the most probable diagnoses solely based on the current patient. This approach is straightforward but may have limited accuracy for complex or rare conditions. Second, in the few-shot setting, the diagnosis strategy agent utilizes additional patient cases to guide its predictions, enabling more context-aware diagnostic reasoning. We explore two variations. First, in a standard few-shot approach, a fixed set of patient examples are selected as references and provided to the model alongside the current patient profile. Second, the dynamic few-shot approach improves upon this by selecting reference cases based on similarity metrics, ensuring that the most relevant patients are included. Patient similarity is determined using embedding-based retrieval, with two embeddings (e.g., BioClinicalBERT~\citep{alsentzer-etal-2019-publicly}, BGE~\citep{xiao-etal-2024-bge}) evaluated to match patients with similar profiles.

We also integrate Chain-of-Thought (CoT) reasoning~\citep{wei2022chain}, guiding the model to explicitly reason through intermediate clinical steps before predicting a diagnosis. CoT can be combined with both standard and dynamic few-shot approaches, allowing the model to generate a stepwise rationale for each diagnosis. Inspired by MedPrompt~\citep{nori2023can}, our approach extends CoT by incorporating structured, example-driven reasoning, where each reference case includes both a diagnosis and an associated COT explanation. The integration of CoT enables the system to better handle complex cases with diagnostic uncertainty, such as specific skin diseases in iCRAFT-MD which share common symptoms. 
% \cc{example to show?}

Once the model completes the diagnostic inference, the ranked list of differential diagnoses is returned to DDxDriver, which further refines or finalizes the diagnosis through iterative updates. By distinguishing between zero-shot and few-shot inference strategies, plus dynamic adaptation through embeddings and reasoning techniques, the diagnosis strategy agent aims to enhance both accuracy and generalizability.

\vspace{-0.5em}
\subsection{Orchestrator} 
%\cc{mention the progress rate!}
\label{subsec:ddxdriver}

Inspired by the concept of a unified interface layer from previous work~\citep{gioacchini-etal-2024-agentquest}, we introduce DDxDriver as the central coordination hub in the MEDDxAgent framework (\autoref{fig:overview_ddxdriver}). DDxDriver enables modular compatibility between the diagnostic agents and benchmark datasets, with minimal adaptation efforts. 
DDxDriver uses the ReAct paradigm~\citep{yao2023react} -- which combines step-by-step reasoning (\textit{thought}) with decision-making (\textit{action}) and feedback processing (\textit{observation}). At each step, DDxDriver obtains the information from the \textit{environment} (input/output) and the results from the previous state of the simulator and agents (\textit{observation}, if it exists), then reasons about the current state of evidence (\textit{thought}) and generates agent-specific instructions based on the current state of the patient profile. It dispatches these instructions to the selected simulator/agent, executes the simulator/agent, and subsequently updates the patient profile with newly obtained information (\textit{action}). 
Beyond execution management, DDxDriver serves four primary functions. First, it manages the patient profile, storing and maintaining all relevant clinical information, including demographics, medical history, symptoms, and evolving diagnostic rankings. Second, it schedules and dispatches diagnostic actions, dynamically determining which simulator/agent to invoke next based on the evolving diagnostic context. Third, it ensures traceability by logging all interactions, including inputs, outputs, and intermediate reasoning steps, thereby providing transparency in the decision-making process. Finally, it enforces stopping criteria by monitoring diagnostic convergence and applying configurable thresholds, such as the number of iterations or the stabilization of ranked diagnoses.



\subsection{Iterative Learning Mechanism}
\label{subsec:iterative_learning}
Diagnoses in the real world are rarely made in a single step. They are refined through multiple interactions with patients, clinical data, and external knowledge. To mirror this process, the \textit{iterative learning} mechanism is designed to avoid relying on any single diagnostic agent or static decision process. We implement two settings: (i) \textit{fixed iteration}, and (ii) \textit{dynamic iteration}. \textit{Fixed iteration} cycles through the history taking simulator, knowledge retrieval agent, and diagnosis strategy agent in order until the predefined stopping criterion is met (e.g., $n$ iterations).
In contrast, the \textit{dynamic iteration} process lifts constraints on the predetermined execution order, allowing the DDxDriver to adapt dynamically during the differential diagnosis process. After each observation, the DDxDriver reasons about which component -- history taking simulator, knowledge retrieval agent, or diagnosis strategy agent -- to call next based on up-to-date observations (i.e. updated patient profile, medical documents, predicted DDx). For instance, if the current diagnosis indicates a rare condition for which it needs clarifying details, the system may invoke the knowledge retrieval agent to search for specialized information. This allows for flexible decision-making, opening up the opportunity for both ideal and non-ideal choices. The iterative learning mechanism allows MEDDxAgent to continuously refine diagnosis while offering transparent insights into its reasoning process.
\vspace{-0.5em}