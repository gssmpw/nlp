% Google Paper, DDxPlus paper, DDxGym paper, mentioning how crucial the task is and why it is challenging (1. iterative process 2. hard to differentiate diseases with similar symptoms for diagnosis) 
% Need to mention the explainable part? Why previous approach doesn't tackle this?
Differential Diagnosis (DDx) is a crucial step in medical decision-making, where doctors systematically narrow down the most likely diagnosis from a range of possible diseases \citep{rhoads2017formulating}. In real-world clinical practice, DDx is essential because it accounts for uncertainty in the diagnosis \citep{henderson2012patient}. It's also incredibly challenging given the large number of potential diseases, rapidly evolving medical knowledge, and the fact that symptoms and antecedents can point to multiple diseases \citep{winter2024ddxgym}.

Expert clinicians rely on pattern recognition and past experience to narrow down potential diseases. However, the complexity and variability of real-world clinical presentations have prompted recent research into computational frameworks that use large language models (LLMs) to improve the DDx process \citep{fansi2022towards,zhou2024interpretable}.

\begin{figure}[t]
	\centering
    \includegraphics[trim={7.2cm 1cm 8cm 3cm},clip,width=0.42\textwidth]{img/Overview_framework_v3.pdf}
    %\vspace{-0.9em}
    \caption{\textbf{MEDDxAgent} facilitates differential diagnosis by iteratively narrowing down a patient's possible disease. DDxDriver acts as the central orchestrator. It receives an interactive environment via a simulator (\textcolor{yellow-history}{\textbf{History Taking}}) and can access two agents (\textcolor{blue-retrieval}{\textbf{Knowledge Retrieval}}, \textcolor{green-diagnosis}{\textbf{Diagnosis Strategy}}).}
    \vspace{-1.8em}
    \label{fig:overview_ddxdriver}
\end{figure}

Though LLM-based systems have shown promise in improving diagnostic assistance, existing methods face several limitations: (1) reliance on \textit{single-dataset evaluations}, limiting the generalizability across diverse patient populations and disease categories~\citep{alam2023ddxt}; (2) focus on \textit{optimizing a single diagnostic component} (e.g., diagnosis strategy \textit{only})~\citep{ mcduff2023towards}, without an integrated approach to enhance multiple phases of the diagnostic process;\footnote{The DDx process typically involves three key components: history taking, knowledge retrieval, and diagnosis strategy~\citep{cook2020higher, 2024PrinciplesOD}.} (3) assumption of \textit{complete patient profiles} upfront (i.e., with all symptoms and antecedents)~\citep{wu2024streambench} and \textit{single-turn} paradigm~\citep{zelin2024rare}, diverging from the reality that DDx is an investigative process, requiring follow-up actions to gather information~\citep{li2024mediq}; 
(4) lack of \textit{iterative learning}, preventing diagnosis updates over successive interactions -- an essential aspect of real-world diagnostic decision-making; (5) an \textit{over-reliance on medical QA benchmarks}~\citep{zhang2024ultramedical} for medical applications, which do not accurately reflect the complexities of real-world DDx tasks.  

We target these gaps and propose a \textbf{M}odular \textbf{E}xplainable \textbf{DDx} \textbf{Agent} (\textbf{MEDDxAgent}) framework (see~\autoref{fig:overview_ddxdriver}). It consists of (1) DDxDriver that acts as the central orchestrator; (2) a history taking simulator which enables an iterative environment; and (3) two individual agents -- knowledge retrieval and diagnosis strategy -- to support the diagnostic process. 
We advance the task of automatic DDx with the following contributions: 
(i) We propose a modular, multi-faceted DDx agent framework (MEDDxAgent), integrating a history taking simulator and two diagnostic agents (knowledge retrieval, diagnosis strategy), which enables extensible and explainable decision-making processes.
(ii) We introduce an orchestrator (DDxDriver) as a unified interface, ensuring \textit{iterative learning} and interactive optimizations between agents, as well as monitoring of the decision-making process. 
(iii) We build a new DDx benchmark incorporating three diagnostic sources with different disease categories: DDxPlus~\citep{fansi2022ddxplus} (\textit{respiratory}), iCRAFT-MD~\citep{li2024mediq} (\textit{skin}), and RareBench~\citep{chen2024rarebench} (\textit{rare}). This allows for a more comprehensive diagnostic scope than in existing work. 
(iv) We evaluate MEDDxAgent in a more challenging but realistic scenario -- \textit{interactive differential diagnosis}, and demonstrate its effectiveness by achieving over 10\% points improvements in accuracy (i.e., GTPA@1) for both large (70B) and small (8B) LLMs.  
%(v) The code is publicly available.\footnote{\url{https://github.com/nec-research/meddxagent}}%=> Add this back while the code is finally released?
\vspace{-0.5em}

