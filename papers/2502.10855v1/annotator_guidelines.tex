\begin{tcolorbox}[
    breakable,                    
    colback=white,                
    colframe=black,              
    title=Annotation Guidelines,       
    title after break=Annotation Guidelines (Continued),
    fonttitle=\bfseries, 
    coltext=black,
]
\begin{lstlisting}[breaklines=true, breakindent=0pt, basicstyle=\small\ttfamily\raggedright, xleftmargin=-5pt, frame=none, xrightmargin=-5pt, aboveskip=-2pt, belowskip=-2pt]
## Overview
You will be given a set of question-answer pairs. The answers were generated by an LLM, based on some search results.

For each question, your task is to identify all sentences in the answer that contain at least one verifiable factual claim. A "verifiable factual claim" is a statement that can be objectively verified as true or false based on empirical evidence or reality. The statement should be sufficiently specific, providing enough detail that a fact-checker would know how to identify relevant evidence. 

For example, the sentence "California and New York implemented incentives for renewable energy adoption, highlighting the broader importance of sustainability in policy decisions" contains at least one verifiable factual claim: 
"California and New York implemented incentives for renewable energy adoption." 
(Note that the last part - "highlighting the broader importance of sustainability in policy decisions" - is an interpretation that cannot be objectively verified as true or false.)

It's possible that NO sentences in the answer contain verifiable factual claims.
For example, the entire answer could provide advice to the reader ("You should do X") or speculate about the future ("AI could potentially revolutionize X") without making any statements that can be objectively verified as true or false.

## Key Guidelines
- You are NOT being asked to determine whether the sentence is true or false, or to check whether evidence exists to confirm or refute the information in a sentence. We are only interested in whether the sentence has the potential to be objectively verified. 
- You should NOT consider whether the sentence is relevant to the question.
- Some sentences in the answer may have citations (e.g., [^2^]). Do NOT consider the presence or absence of a citation when deciding whether the sentence contains a verifiable factual claim.
- If the sentence is about the LLM's inability to answer the question (e.g., 
"The search results did not find any indication of X" or "I'm sorry, I'm unable to respond to this question"), it does NOT contain a verifiable factual claim.
- It is extremely important that you consider the context for a sentence, i.e., the preceding and following sentences. If a sentence is a high-level introduction for the following sentences, or a high-level conclusion for the preceding sentences, then it usually does NOT contain a verifiable factual 
claim.	
    - For example, if a sentence is "Climate change has had several significant economic effects, such as:" and it's followed by a list of specific examples of economic effects, then the sentence is merely an introduction and does NOT contain a verifiable factual claim.
    - For each paragraph in the answer: it is highly recommended that you read through the entire paragraph first without making any decisions, then consider each sentence individually.

## Examples
Here are some examples of sentences that do NOT contain any verifiable factual claims: 
- By prioritizing ethical considerations, companies can ensure that their innovations are not only groundbreaking but also socially responsible -> generic statement that cannot be objectively verified as true or false
- Technological progress should be inclusive -> opinion
- Leveraging AI is essential for maximizing productivity -> opinion
- Networking events can be crucial in shaping the paths of young entrepreneurs and providing them with valuable connections -> opinion
- AI could lead to advancements in healthcare -> speculation
- This implies that John Smith is a courageous person -> interpretation
- Try to show appreciation to your friends -> advice/recommendation
- Basketball is a fun, dynamic game, and an important part of many people's lives -> opinion and generic

As you can see from these examples, unverifiable claims can often be described as broad or generic statements, opinions, interpretations, speculations, and/or advice.

Here are some examples of sentences that do contain at least one verifiable factual claim:
- The partnership between Company X and Company Y illustrates the power of innovation -> a verifiable factual claim would be "there is a partnership between Company X and Company Y"; the rest (the partnership illustrates the power of innovation) is an unverifiable interpretation
- Jane Doe's approach of embracing adaptability and prioritizing customer feedback can be valuable advice for new executives -> a verifiable factual claim would be "Jane Doe's approach includes embracing adaptability and prioritizing customer feedback"; the rest (her approach can be valuable advice) is an opinion
- Smith's advocacy for renewable energy is crucial in addressing these challenges -> "Smith advocates for renewable energy"
- **John Smith**: instrumental in numerous renewable energy initiatives, playing a pivotal role in Project Green -> "John Smith is involved in renewable energy initiatives and played a role in Project Green"
- John, the CEO of Company X, is a notable example of strong leadership -> "John is the CEO of Company X"
- Therefore, leveraging industry events, as demonstrated by Jane's experience at the Tech Networking Club, can provide visibility and traction for new ventures -> "Jane had an experience at the Tech Networking Club"

You'll notice that in some of the above examples, only part of the sentence - not the entire sentence - contains a verifiable factual claim. It is NOT necessary for the entire sentence to convey a verifiable factual claim.

<The remaining instructions explained how to use the annotation interface.>
\end{lstlisting}
\end{tcolorbox}
