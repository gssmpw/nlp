\begin{tcolorbox}[
    breakable,                    
    colback=white,                
    colframe=black,              
    title=Invalid Sentences System Prompt,      
    title after break=Invalid Sentences System Prompt (Continued),
    fonttitle=\bfseries, 
    coltext=black,
    after skip=-1pt
]
\begin{lstlisting}[breaklines=true, breakindent=0pt, basicstyle=\small\ttfamily\raggedright, xleftmargin=-5pt, frame=none, xrightmargin=-5pt, aboveskip=-2pt, belowskip=-2pt]
## Overview
You will be given a question, an excerpt from the response to the question, and a sentence of interest from the excerpt (which will be referred to as S).

Your task is to determine whether S, in light of its context, can be interpreted as a complete, declarative sentence by following these steps:
1. Print "S = <insert sentence of interest here EXACTLY as written>"
2. Describe the context for S. 
3. Can S be interpreted as a complete, declarative sentence as is? If not, given its context, can it be rewritten as a complete, declarative sentence? If yes, print "S can be interpreted as a complete, declarative sentence"; otherwise, print "S cannot be interpreted as a complete, declarative sentence".

## Examples
### Example 1
Question: How can companies improve their sustainability practices?
Excerpt from response: Some examples include:
- Reducing energy consumption by using energy-efficient appliances
- Implementing recycling programs
- Sourcing materials from sustainable suppliers
Sentence of interest: Some examples include:

S = Some examples include:
Describe the context for S. The question is about how companies can improve their sustainability practices, and S is the header for a list of examples. 
Can S be interpreted as a complete, declarative sentence as is? If not, given its context, can it be rewritten as a complete, declarative sentence? S is not a complete, declarative sentence as is. Since it merely introduces the list of examples without providing a complete thought, it cannot be rewritten as a complete, declarative sentence. Therefore, S cannot be interpreted as a 
complete, declarative sentence.
 
### Example 2
Question: How are companies improving their sustainability practices?
Excerpt from response: Some examples include:
- Reducing energy consumption by using energy-efficient appliances
- Implementing recycling programs
- Sourcing materials from sustainable suppliers
Sentence of interest: - Sourcing materials from sustainable suppliers

S = - Sourcing materials from sustainable suppliers
Describe the context for S. The question is about how companies are improving their sustainability practices, and the excerpt provides a list of examples. 
Can S be interpreted as a complete, declarative sentence as is? If not, given its context, can it be rewritten as a complete, declarative sentence? S is not a complete, declarative sentence as is. However, in the context of the question and the excerpt, it can be rewritten as "An example of how companies are improving their sustainability practices is sourcing materials from sustainable suppliers". Therefore, S can be interpreted as a complete, declarative sentence.
\end{lstlisting}
\end{tcolorbox}

\begin{tcolorbox}[
    breakable,                    
    colback=white,                
    colframe=black,              
    title=Invalid Sentences User Prompt,       
    title after break=Invalid Sentences User Prompt (Continued),
    fonttitle=\bfseries, 
    coltext=black,
    before skip=5pt,
]
\begin{lstlisting}[breaklines=true, breakindent=0pt, basicstyle=\small\ttfamily\raggedright, xleftmargin=-5pt, frame=none, xrightmargin=-5pt, aboveskip=-2pt, belowskip=-2pt]
Question:
{question}

Excerpt from response: 
{excerpt}

Sentence of interest: 
{sentence}
\end{lstlisting}
\end{tcolorbox}