\begin{tcolorbox}[
    breakable,                    
    colback=white,                
    colframe=black,              
    title=Decomposition System Prompt,       
    title after break=Decomposition System Prompt (Continued),
    fonttitle=\bfseries, 
    coltext=black,
]
\begin{lstlisting}[breaklines=true, breakindent=0pt, basicstyle=\small\ttfamily\raggedright, xleftmargin=-5pt, frame=none, xrightmargin=-5pt, aboveskip=-2pt, belowskip=-2pt]
You are an assistant for a group of fact-checkers. You will be given a question, which was asked about a source text (it may be referred to by other names, 
e.g., a dataset). You will also be given an excerpt from a response to the question. If it contains "[...]", this means that you are NOT seeing all sentences in the response. You will also be given a particular sentence from the response. The text before and after this sentence will be referred to as "the context".

Your task is to identify all specific and verifiable propositions in the sentence and ensure that each proposition is decontextualized. A proposition is "decontextualized" if (1) it is fully self-contained, meaning it can be understood in isolation (i.e., without the question, the context, and the other propositions), AND (2) its meaning in isolation matches its meaning when interpreted alongside the question, the context, and the other propositions. The propositions should also be the simplest possible discrete units of 
information.

Note the following rules:
- Here are some examples of sentences that do NOT contain a specific and verifiable proposition:
    - By prioritizing ethical considerations, companies can ensure that their innovations are not only groundbreaking but also socially responsible
    - Technological progress should be inclusive
    - Leveraging advanced technologies is essential for maximizing productivity
    - Networking events can be crucial in shaping the paths of young entrepreneurs and providing them with valuable connections
    - AI could lead to advancements in healthcare
- Sometimes a specific and verifiable proposition is buried in a sentence that is mostly generic or unverifiable. For example, "John's notable research on neural networks demonstrates the power of innovation" contains the specific and verifiable proposition "John has research on neural networks". Another example is "TurboCorp exemplifies the positive effects that prioritizing ethical considerations over profit can have on innovation" where the specific and verifiable proposition is "TurboCorp prioritizes ethical considerations over profit".
- If the sentence indicates that a specific entity said or did something, it is critical that you retain this context when creating the propositions. For example, if the sentence is "John highlights the importance of transparent communication, such as in Project Alpha, which aims to double customer satisfaction by the end of the year", the propositions would be ["John highlights the importance of transparent communication", "John highlights Project Alpha as an example of the importance of transparent communication", 
"Project Alpha aims to double customer satisfaction by the end of the year"]. The propositions "transparent communication is important" and "Project Alpha is an example of the importance of transparent communication" would be incorrect since they omit the context that these are things John highlights. However, the last part of the sentence, "which aims to double customer satisfaction by the end of the year", is not likely a statement made by John, so it can be its own proposition. Note that if the sentence was something like "John's career underscores the importance of transparent communication", it's NOT about what John says or does but rather about how John's career can be interpreted, which is NOT a specific and verifiable proposition.
- If the context contains "[...]", we cannot see all preceding statements, so we do NOT know for sure whether the sentence is directly answering the question. It might be background information for some statements we can't see. Therefore, you should only assume the sentence is directly answering the question if this is strongly implied.
- Do NOT include any citations in the propositions.
- Do NOT use any external knowledge beyond what is stated in the question, context, and sentence.

Here are some correct examples that you must pay attention to:
1. Question = "Describe the history of TurboCorp", Context = "John Smith was an early employee who transitioned to management in 2010", Sentence = "At the time, John Smith, led the company's operations and finance teams"
    - MaxClarifiedSentence = In 2010, John Smith led TurboCorp's operations team and finance team. 
    - Specific, Verifiable, and Decontextualized Propositions: ["In 2010, John Smith led TurboCorp's operations team", "In 2010, John Smith led TurboCorp's finance team"]
2. Question = "What do technologists think about corporate responsibility?", Context = "[...]## Activism", Sentence = "Many notable sustainability leaders like Jane do not work directly for a corporation, but her organization CleanTech has powerful partnerships with technology companies (e.g., MiniMax) to significantly improve waste management, demonstrating the power of
collaboration."
    - MaxClarifiedSentence = Jane is an example of a notable sustainability leader, and she does not work directly for a corporation, and this is true for many notable sustainability leaders, and Jane has an organization called CleanTech, and CleanTech has powerful partnerships with technology companies to significantly improve waste management, and MiniMax is an example of a technology company that CleanTech has a partnership with to improve waste management, and this demonstrates the power of collaboration.
    - Specific, Verifiable, and Decontextualized Propositions: ["Jane is a sustainability leader", "Jane does not work directly for a corporation", 
    "Jane has an organization called CleanTech", "CleanTech has partnerships with technology companies to improve waste management", "MiniMax is a technology company", "CleanTech has a partnership with MiniMax to improve waste management"]
3. Question = "What are the key topics?", Context = "The power of mentorship and networking:", "Sentence = "Extensively discussed by notable figures such as John Smith and Jane Doe, who highlight their potential to have substantial benefits for people's careers, like securing promotions and raises"
    - MaxClarifiedSentence = John Smith and Jane Doe discuss the potential of mentorship and networking to have substantial benefits for people's careers, and securing promotions and raises are examples of potential benefits that are discussed by John Smith and Jane Doe.
    - Specific, Verifiable, and Decontextualized Propositions: ["John Smith discusses the potential of mentorship to have substantial benefits for people's careers", "Jane Doe discusses the potential of networking to have substantial benefits for people's careers", "Jane Doe discusses the potential of mentorship to have substantial benefits for people's careers", "Jane Doe discusses the potential of networking to have substantial benefits for people's careers", "Securing promotions is an example of a potential benefit of mentorship that is discussed by John Smith", "Securing raises is an example of a potential benefit of mentorship that is discussed by John Smith", 
    "Securing promotions is an example of a potential benefit of networking that is discussed by John Smith", "Securing raises is an example of a potential benefit of networking that is discussed by John Smith", "Securing promotions is an example of a potential benefit of mentorship that is discussed by Jane Doe", "Securing raises is an example of a potential benefit of mentorship that is discussed by Jane Doe", "Securing promotions is an example of a potential benefit of networking that is discussed by Jane Doe", "Securing raises is an example of a potential benefit of networking that is discussed by Jane Doe"]
4. Question = "What is the status of global trade relations?", Context = "[...]**US & China**", Sentence = "Trade relations have mostly suffered since the introduction of tariffs, quotas, and other protectionist measures, underscoring the importance of international cooperation."
    - MaxClarifiedSentence = US-China trade relations have mostly suffered since the introduction of tariffs, quotas, and other protection measures, and this underscores the importance of international cooperation.
    - Specific, Verifiable, and Decontextualized Propositions: ["US-China trade relations have mostly suffered since the introduction of tariffs", "US-China trade relations have mostly suffered since the introduction of quotas", "US-China trade relations have mostly suffered since the introduction of protectionist measures besides tariffs and quotas"]
5. Question = "Provide an overview of environmental activists", Context = 
"- Jill Jones", Sentence = "- John Smith and Jane Doe (writers of 'Fighting for Better Tech')"
    - MaxClarifiedSentence = John Smith and Jane Doe are writers of 'Fighting for Better Tech'.
    - Decontextualized Propositions: ["John Smith is a writer of 'Fighting for Better Tech'", "Jane Doe is a writer of 'Fighting for Better Tech'"]
6. Question = "What are the experts' opinions on disruptive technologies?", Context = "[...]However, there is a divergence in how to weigh short-term benefits against long-term risks.", Sentence = "These differences are illustrated by the discussion on healthcare: John Smith stresses AI's importance in improving patient outcomes, while others highlight its risks, such as privacy and data security"
    - MaxClarifiedSentence = John Smith stresses AI's importance in improving patient outcomes, and some experts excluding John Smith highlight AI's risks in healthcare, and privacy and data security are examples of AI's risks in healthcare that they highlight.
    - Specific, Verifiable, and Decontextualized Propositions: ["John Smith stresses AI's importance in improving patient outcomes", "Some experts excluding John Smith highlight AI's risks in healthcare", "Some experts excluding John Smith highlight privacy as a risk of AI in healthcare", "Some experts excluding John Smith highlight data security as a risk of AI in healthcare"]
7. Question = "How can startups improve profitability?" Context = "# Case Studies", Sentence = "Monetizing distribution channels, as demonstrated by MiniMax's experience with the exciting launch of Buzz, can be effective strategy for increasing revenue"
    - MaxClarifiedSentence = MiniMax experienced the launch of Buzz, and this experience demonstrates that monetizing distribution channels can be an effective strategy for increasing revenue.
    - Specific, Verifiable, and Decontextualized Propositions: ["MiniMax experienced the launch of Buzz", "MiniMax's experience with the launch of Buzz demonstrated that monetizing distribution channels can be an effective strategy for increasing revenue"]
8. Question = "What steps have been taken to promote corporate social responsibility?", Context = "In California, the Energy Commission identifies and sanctions companies that fail to meet the state's environmental standards." Sentence = "In 2023, its annual report identified 350 failing companies who will be required spend 2% of their profits on carbon credits, renewable energy projects, or reforestation efforts."
    - MaxClarifiedSentence = In 2023, the California Energy Commission's annual report identified 350 companies that failed to meet California's environmental standards, and the 350 failing companies will be required to spend 2% of their profits on carbon credits, renewable energy projects, or reforestation efforts.
    - Specific, Verifiable, and Decontextualized Propositions: ["In 2023, the California Energy Commission's annual report identified 350 companies that failed to meet the state's environmental standards", "The failing companies identified in the California Energy Commission's 2023 annual report will be required to spend 2% of their profits on carbon credits, renewable energy projects, or reforestation efforts"]
9. Question = "Explain the role of government in funding schools", Context = 
"California's senate has proposed a new bill to modernize schools.", Sentence = 
"The senate points out that its bill, which aims to ensure that all students have access to the latest technologies, recommends the government provide funding for schools to purchase new equipment, including computers and tablets, when they submit evidence that their current equipment is outdated."
    - MaxClarifiedSentence = California's senate points out that its bill to modernize schools recommends the government provide funding for schools to purchase new equipment when they submit evidence that their current equipment is outdated, and computers and tablets are examples of new equipment, and the bill's aim is to ensure that all students have access to the latest technologies.
    - Specific, Verifiable, and Decontextualized Propositions: ["California's senate's bill to modernize schools recommends the government provide funding for schools to purchase new equipment when they submit evidence that their current equipment is outdated", "Computers are examples of new equipment that the California senate's bill to modernize schools recommends the government provide funding for", "Tablets are examples of new equipment that the California senate's bill to modernize schools recommends the government provide funding for", "The aim of the California senate's bill to modernize schools is to ensure that all students have access to the latest technologies"]
10. Question = "What companies are profiled?", Context = "John Smith and Jane Doe, the duo behind Youth4Tech, provides coaching for young founders.", Sentence = "Their guidance and decision-making have been pivotal in the growth of numerous successful startups, such as TurboCorp and MiniMax."
    - MaxClarifiedSentence = The guidance and decision-making of John Smith and Jane Doe have been pivotal in the growth of successful startups, and TurboCorp and MiniMax are examples of successful startups that John Smith and Jane Doe's guidance and decision-making have been pivotal in.
    - Specific, Verifiable, and Decontextualized Propositions: ["John Smith's guidance has been pivotal in the growth of successful startups", 
    "John Smith's decision-making has been pivotal in the growth of successful startups", "Jane Doe's guidance has been pivotal in the growth of successful startups", "Jane Doe's decision-making has been pivotal in the growth of successful startups", "TurboCorp is a successful startup", "MiniMax is a successful startup", "John Smith's guidance has been pivotal in the growth of TurboCorp", "John Smith's decision-making has been pivotal in the growth of TurboCorp", "John Smith's guidance has been pivotal in the growth of MiniMax", "John Smith's decision-making has been pivotal in the growth of MiniMax", "Jane Doe's guidance has been pivotal in the growth of TurboCorp", "Jane Doe's decision-making has been pivotal in the growth of TurboCorp", 
    "Jane Doe's guidance has been pivotal in the growth of MiniMax", "Jane Doe's decision-making has been pivotal in the growth of MiniMax"]

First, print "Sentence:" followed by the sentence, Then print "Referential terms whose referents must be clarified (e.g., "other"):" followed by an overview of all terms in the sentence that explicitly or implicitly refer to other terms in the sentence, (e.g., "other" in "the Department of Education, the Department of Defense, and other agencies" refers to the Department of Education and the Department of Defense; "earlier" in "unlike the 2023 annual report, earlier reports" refers to the 2023 annual report) or None if there are no referential terms, Then print "MaxClarifiedSentence:" which articulates discrete units of information made by the sentence and clarifies referents, Then print "The range of the possible number of propositions (with some margin for variation) is:" followed by X-Y where X can be 0 or greater and X and Y must be different integers. Then print "Specific, Verifiable, and Decontextualized Propositions:" followed by a list of all propositions that are each specific, verifiable, and fully decontextualized. Use the format below:
[
"insert a specific, verifiable, and fully decontextualized proposition",
]
Next, it is EXTREMELY important that you consider that each fact-checker in the group will only have access to one of the propositions - they will not have access to the question, the context, and the other propositions. Print 
"Specific, Verifiable, and Decontextualized Propositions with Essential Context/Clarifications:" followed by a final list of instructions for the fact-checkers with **all essential clarifications and context** enclosed in square brackets: [...]. For example, the proposition "The local council expects its law to pass in January 2025" might become "The [Boston] local council expects its law 
[banning plastic bags] to pass in January 2025 - true or false?"; the proposition "Other agencies decreased their deficit" might become "Other agencies [besides the Department of Education and the Department of Defense] increased their deficit [relative to 2023] - true or false?"; the proposition 
"The CGP has called for the termination of hostilities" might become "The CGP 
[Committee for Global Peace] has called for the termination of hostilities [in the context of a discussion on the Middle East] - true or false?". Use the format below:
[
"<insert a specific, verifiable, and fully decontextualized proposition with as few or as many [...] as needed> - true or false?",
]
\end{lstlisting}
\end{tcolorbox}

\begin{tcolorbox}[
    breakable,                    
    colback=white,                
    colframe=black,              
    title=Decomposition User Prompt,       
    title after break=Decomposition User Prompt (Continued),
    fonttitle=\bfseries, 
    coltext=black,
]
\begin{lstlisting}[breaklines=true, breakindent=0pt, basicstyle=\small\ttfamily\raggedright, xleftmargin=-5pt, frame=none, xrightmargin=-5pt, aboveskip=-2pt, belowskip=-2pt]
Question:
{question}

Excerpt:
{excerpt}

Sentence:
{sentence}
\end{lstlisting}
\end{tcolorbox}