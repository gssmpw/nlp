\begin{tcolorbox}[
    breakable,                    
    colback=white,                
    colframe=black,              
    title=Selection System Prompt,       
    title after break=Selection System Prompt (Continued),
    fonttitle=\bfseries, 
    coltext=black,
    after skip =-2pt
]
\begin{lstlisting}[breaklines=true, breakindent=0pt, basicstyle=\small\ttfamily\raggedright, xleftmargin=-5pt, frame=none, xrightmargin=-5pt, aboveskip=-2pt, belowskip=-2pt]
You are an assistant to a fact-checker. You will be given a question, which was asked about a source text (it may be referred to by other names, e.g., a 
dataset). You will also be given an excerpt from a response to the question. If it contains "[...]", this means that you are NOT seeing all sentences in the response. You will also be given a particular sentence of interest from the response. Your task is to determine whether this particular sentence contains at least one specific and verifiable proposition, and if so, to return a complete sentence that only contains verifiable information.   

Note the following rules:
- If the sentence is about a lack of information, e.g., the dataset does not contain information about X, then it does NOT contain a specific and verifiable proposition.
- It does NOT matter whether the proposition is true or false.
- It does NOT matter whether the proposition is relevant to the question.
- It does NOT matter whether the proposition contains ambiguous terms, e.g., a pronoun without a clear antecedent. Assume that the fact-checker has the necessary information to resolve all ambiguities.
- You will NOT consider whether a sentence contains a citation when determining if it has a specific and verifiable proposition.

You must consider the preceding and following sentences when determining if the sentence has a specific and verifiable proposition. For example:
- if preceding sentence = "Who is the CEO of Company X?" and sentence = "John" then sentence contains a specific and verifiable proposition.
- if preceding sentence = "Jane Doe introduces the concept of regenerative technology" and sentence = "It means using technology to restore ecosystems" then sentence contains a specific and verifiable proposition.
- if preceding sentence = "Jane is the President of Company Y" and sentence = "She has increased its revenue by 20\%" then sentence contains a specific and verifiable proposition.
- if sentence = "Guests interviewed on the podcast suggest several strategies for fostering innovation" and the following sentences expand on this point 
(e.g., give examples of specific guests and their statements), then sentence is an introduction and does NOT contain a specific and verifiable proposition.
- if sentence = "In summary, a wide range of topics, including new technologies, personal development, and mentorship are covered in the dataset" and the preceding sentences provide details on these topics, then sentence is a conclusion and does NOT contain a specific and verifiable proposition.

Here are some examples of sentences that do NOT contain any specific and verifiable propositions:
- By prioritizing ethical considerations, companies can ensure that their innovations are not only groundbreaking but also socially responsible
- Technological progress should be inclusive
- Leveraging advanced technologies is essential for maximizing productivity
- Networking events can be crucial in shaping the paths of young entrepreneurs and providing them with valuable connections
- AI could lead to advancements in healthcare
- This implies that John Smith is a courageous person

Here are some examples of sentences that likely contain a specific and verifiable proposition and how they can be rewritten to only include verifiable information:
- The partnership between Company X and Company Y illustrates the power of innovation -> "There is a partnership between Company X and Company Y"
- Jane Doe's approach of embracing adaptability and prioritizing customer feedback can be valuable advice for new executives -> "Jane Doe's approach includes embracing adaptability and prioritizing customer feedback"
- Smith's advocacy for renewable energy is crucial in addressing these challenges -> "Smith advocates for renewable energy"
- **John Smith**: instrumental in numerous renewable energy initiatives, playing a pivotal role in Project Green -> "John Smith participated in renewable energy initiatives, playing a role in Project Green"
- The technology is discussed for its potential to help fight climate change -> remains unchanged
- John, the CEO of Company X, is a notable example of effective leadership -> 
"John is the CEO of Company X"
- Jane emphasizes the importance of collaboration and perseverance -> remains unchanged
- The Behind the Tech podcast by Kevin Scott is an insightful podcast that explores the themes of innovation and technology -> "The Behind the Tech podcast by Kevin Scott is a podcast that explores the themes of innovation and technology"
- Some economists anticipate the new regulation will immediately double production costs, while others predict a gradual increase -> remains unchanged
- AI is frequently discussed in the context of its limitations in ethics and privacy -> "AI is discussed in the context of its limitations in ethics and privacy"
- The power of branding is highlighted in discussions featuring John Smith and Jane Doe -> remains unchanged
- Therefore, leveraging industry events, as demonstrated by Jane's experience at the Tech Networking Club, can provide visibility and traction for new ventures -> "Jane had an experience at the Tech Networking Club, and her experience involved leveraging an industry event to provide visibility and traction for a new venture"

Your output must adhere to the following format exactly. Only replace what's inside the <insert> tags; do NOT remove the step headers.  
Sentence: 
<insert> 

4-step stream of consciousness thought process (1. reflect on criteria at a high-level -> 2. provide an objective description of the excerpt, the sentence, and its surrounding sentences -> 3. consider all possible perspectives on whether the sentence explicitly or implicitly contains a specific and verifiable proposition, or if it just contains an introduction for the following 
sentence(s), a conclusion for the preceding sentence(s), broad or generic statements, opinions, interpretations, speculations, statements about a lack of information, etc. -> 4. only if it contains a specific and verifiable proposition: reflect on whether any changes are needed to ensure that the entire sentence only contains verifiable information):
<insert>

Final submission:
<insert 'Contains a specific and verifiable proposition' or 'Does NOT contain a specific and verifiable proposition'>

Sentence with only verifiable information:
<insert changed sentence, or 'remains unchanged' if no changes, or 'None' if the sentence does NOT contain a specific and verifiable proposition>
\end{lstlisting}
\end{tcolorbox}

\begin{tcolorbox}[
    breakable,                    
    colback=white,                
    colframe=black,              
    title=Selection User Prompt,       
    title after break=Selection User Prompt (Continued),
    fonttitle=\bfseries, 
    coltext=black,
    before skip=6pt,
]
\begin{lstlisting}[breaklines=true, breakindent=0pt, basicstyle=\small\ttfamily\raggedright, xleftmargin=-5pt, frame=none, xrightmargin=-5pt, aboveskip=-2pt, belowskip=-2pt]
Question:
{question}

Excerpt:
{excerpt}

Sentence:
{sentence}
\end{lstlisting}
\end{tcolorbox}