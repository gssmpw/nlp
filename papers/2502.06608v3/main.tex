%%%%%%%% ICML 2024 EXAMPLE LATEX SUBMISSION FILE %%%%%%%%%%%%%%%%%

\documentclass{article}
\usepackage{graphicx}

% Recommended, but optional, packages for figures and better typesetting:
% \usepackage{subfigure}
\usepackage{algorithmic}
\usepackage{algorithm}
\usepackage[caption=false,font=normalsize,labelfont=sf,textfont=sf]{subfig}
\usepackage{textcomp}
\usepackage{stfloats}
\usepackage{url}
\usepackage{verbatim}
\usepackage{cite}
\hyphenation{op-tical net-works semi-conduc-tor IEEE-Xplore}
% updated with editorial comments 8/9/2021
\usepackage{cuted}
\usepackage{caption}
\usepackage{pifont}
\usepackage{array} 
\usepackage{subcaption} 

\usepackage{multirow}

% Recommended, but optional, packages for figures and better typesetting:
\usepackage{microtype}
\usepackage{booktabs} % for professional tables

% hyperref makes hyperlinks in the resulting PDF.
% If your build breaks (sometimes temporarily if a hyperlink spans a page)
% please comment out the following usepackage line and replace
% \usepackage{icml2025} with \usepackage[nohyperref]{icml2025} above.
\usepackage{hyperref}




% Attempt to make hyperref and algorithmic work together better:


% hyperref makes hyperlinks in the resulting PDF.
% If your build breaks (sometimes temporarily if a hyperlink spans a page)
% please comment out the following usepackage line and replace
% \usepackage{icml2024} with \usepackage[nohyperref]{icml2024} above.


% Attempt to make hyperref and algorithmic work together better:

% Use the following line for the initial blind version submitted for review:
% \usepackage{icml2024}

% If accepted, instead use the following line for the camera-ready submission:
\usepackage[accepted]{icml2025}

% For theorems and such
\usepackage{amsmath}
\usepackage{amssymb}
\usepackage{mathtools}
\usepackage{amsthm}

\newcommand*{\method}{TripoSG}


% Todonotes is useful during development; simply uncomment the next line
%    and comment out the line below the next line to turn off comments
%\usepackage[disable,textsize=tiny]{todonotes}
\usepackage[textsize=tiny]{todonotes}


% The \icmltitle you define below is probably too long as a header.
% Therefore, a short form for the running title is supplied here:
\icmltitlerunning{\method{}: High-Fidelity 3D Shape Synthesis using Large-Scale Rectified Flow Models}

\begin{document}



\twocolumn[
\icmltitle{\method{}: High-Fidelity 3D Shape Synthesis using Large-Scale Rectified Flow Models}

\icmlsetsymbol{equal}{*}
\icmlsetsymbol{corre}{$\dagger$}


\begin{icmlauthorlist}
\icmlauthor{Yangguang Li$^1$}{equal}
\icmlauthor{Zi-Xin Zou$^1$}{equal}
\icmlauthor{Zexiang Liu$^1$}{}
\icmlauthor{Dehu Wang$^1$}{}
\icmlauthor{Yuan Liang$^1$}{}
\icmlauthor{Zhipeng Yu$^1$}{}
\icmlauthor{Xingchao Liu$^3$}{}
\icmlauthor{Yuan-Chen Guo$^1$}{}
\icmlauthor{Ding Liang$^1$}{corre}
\icmlauthor{Wanli Ouyang$^{2,4}$}{}
\icmlauthor{Yan-Pei Cao$^1$}{}\\
~\\
Project Homepage: \href{https://yg256li.github.io/TripoSG-Page/}{\textcolor{red}{TripoSG-Page}}
\end{icmlauthorlist}

\icmlkeywords{Machine Learning, ICML}

\vskip 0.3in
]




% this must go after the closing bracket ] following \twocolumn[ ...

% This command actually creates the footnote in the first column
% listing the affiliations and the copyright notice.
% The command takes one argument, which is text to display at the start of the footnote.
% The \icmlEqualContribution command is standard text for equal contribution.
% Remove it (just {}) if you do not need this facility.

% \printAffiliationsAndNotice{}  % leave blank if no need to mention equal contribution

\printAffiliationsAndNotice
{
~\\
$^1$VAST \\
$^2$The Chinese University of Hong Kong \\
$^3$The University of Texas at Austin \\
$^4$Shanghai AI Laboratory \\
$^*$: Equal Contribution \\
$^\dagger$: Corresponding Author\\
}


\begin{strip}
    \centering
    \vspace{-9em}
    \includegraphics[width=1\textwidth]{Images/teaser_resting_4k_shapegen_w_bg.png}
    % \vspace{-1.5em}
    \captionof{figure}{High-quality 3D shape samples from our largest \method{} model. Covering various complex structures, diverse styles, imaginative designs, multi-object compositions, and richly detailed outputs, demonstrates its powerful generation capabilities.}
    \vspace{2em}
    \label{fig:teaser}
\end{strip}

\begin{abstract}

% Recent works to jointly reconstruct 3D human and object from a single RGB image, are mostly model-based, that fail to capture the fine details of the clothed human body and object surface. In this paper, we introduce ReCHOR, a novel, model-free, first-method to produce realistic clothed human-object reconstructions from a monocular view. This is extremely challenging due to human-object occlusions, diverse interactions and depth ambiguity, as it needs to infer both 3D spatial awareness and high resolution details. Our core idea is based on estimating neural implicit representations for human and object respectively by an attention-based neural implicit model that attends to pixel-aligned features from both the global human-object image for spatial awareness and  the local separate view of human and object images for high quality details. Additionally, the network is conditioned on semantic features from an initial estimated human-object pose prior and a generative diffusion model that inpaints occluded regions, thus enabling the retrieval of details from them.
% We also propose a synthetic dataset with rendered scenes of diverse, inter-occluded 3D human and object scans, to train our network. We evaluate our method on the synthetic and real world BEHAVE dataset. Our experiments show that our method outperforms the SOTA in achieving realistic clothed human-object reconstructions.
Recent approaches to jointly reconstruct 3D humans and objects from a single RGB image represent 3D shapes with template-based or coarse models, which fail to capture details of loose clothing on human bodies. In this paper, we introduce a novel implicit approach for jointly reconstructing realistic 3D clothed humans and objects from a monocular view. For the first time, we model both the human and the object with an implicit representation, allowing to capture more realistic details such as clothing. This task is extremely challenging due to human-object occlusions and the lack of 3D information in 2D images, often leading to poor detail reconstruction and depth ambiguity. To address these problems, we propose a novel attention-based neural implicit model that leverages image pixel alignment from both the input human-object image for a global understanding of the human-object scene and from local separate views of the human and object images to improve realism with, for example, clothing details. Additionally, the network is conditioned on semantic features derived from an estimated human-object pose prior, which provides 3D spatial information about the shared space of humans and objects. To handle human occlusion caused by objects, we use a generative diffusion model that inpaints the occluded regions, recovering otherwise lost details. For training and evaluation, we introduce a synthetic dataset featuring rendered scenes of inter-occluded 3D human scans and diverse objects. Extensive evaluation on both synthetic and real-world datasets demonstrates the superior quality of the proposed human-object reconstructions over competitive methods.
\end{abstract}
\section{Introduction}\label{sec:intro}

In computational finance, Monte Carlo simulations are used extensively to estimate the expected value of financial payoffs based on the solution of stochastic differential equations (SDEs) which model the evolution of stock prices, interest rates, exchange rates and other quantities \cite{glasserman04}.  Monte Carlo methods are very general and flexible, but for high accuracy it requires generating a large number of costly SDE path approximations, which has motivated research into a number of variance reduction or, equivalently, cost reduction techniques. One such method is
Multilevel Monte Carlo (MLMC), which was proposed in \cite{GILES2008} and was adapted for various applications that are summarised in \cite{Giles_overview17} and successfully combined with other methods such as quasi-Monte Carlo methods. The main idea of MLMC is to approximate the payoff using different time stepping resolutions when numerically solving the underlying SDE and to generate an optimal number of samples on each level, such that the overall computational cost is minimised subject to the desired bound on the variance. %, such that the total computational cost is minimised. 
The computational savings come from the fact that most samples are computed on the coarser levels and hence are less expensive while only a few samples from the finest levels are required \cite{GILES2008}.


Among the directions in which the computational cost 
of MLMC methods could further be reduced, an important avenue is the use of lower precision calculations, especially for the first Monte Carlo levels where the targeted accuracy is relatively low. 
 An overview of the research on mixed precision for the standard Monte Carlo (MC) framework is provided in \cite{ChowMixedPrecisionStandardMC} but only a few references study the potential of low precision computation in the MLMC framework \cite{Rounding_error_oliver}. To the best of our knowledge, the only MLMC framework with customised precision in the literature is \cite{brugger2014mixed}, but they use a uniform precision for all operations on each Monte Carlo level instead of optimising 
 the precision of each intermediary variable to reduce as much as possible the cost of path generation.
 
An important motivation for an MLMC framework with variable precision would be performing the low precision computations on reconfigurable hardware devices such as Field Programmable Gate Arrays (FPGAs). FPGAs contain customizable logic blocks and connectors that make it easy to adapt the digital circuit architecture for a specific application, leading to a highly parallel and optimised implementation. Therefore they are successfully exploited in applications that require high speed and have high computational workload, such as signal processing \cite{woods2008fpga}, and real time applications like high frequency trading \cite{HFT1,HFT2}. That is why a number of previous works in hardware architecture design implemented the MLMC algorithm to price financial options using FPGAs as accelerators, which resulted in improved speed and power efficiency compared to full CPU architectures \cite{Schryver2013AMM}. The paper \cite{lindsey2016domain} also proposed 
a Domain Specific Language to automate the configuration of FPGAs for this specific application. However, only \cite{brugger2014mixed} proposed a heuristic to reduce the precision in calculations.

In addition, all aforementioned works considered that the random number generation (RNG) is performed in single or double precision. Yet in most cases an important portion of the workload in the overall MLMC simulation comes from the RNG and in \cite{brugger2014mixed} this limited the total computational savings.
To reduce the cost of MLMC simulations in particular those based on the Geometric Brownian Motion (GBM), \cite{approximateICDF_Oliver, NestedOliver} have proposed to use approximate random numbers that are generated by applying an approximation of the inverse CDF to uniform random numbers. In \cite{NestedOliver}, the authors proposed a way to integrate these lower precision random variables into a \textit{nested} MLMC framework and completed a numerical analysis to bound the resulting error at each MC level by a product of the time step and the error in the random number approximation. The same authors show in \cite{approximateICDF_Oliver} that using approximate random variables reduces the cost of path generation by a factor 7.


In this paper we propose a nested MLMC framework that combines the use of approximate random normal variables and lower precision calculations to reduce the computational cost of MLMC even further than \cite{brugger2014mixed,NestedOliver}. We illustrate the efficiency of our framework in Matlab, after making several assumptions on the cost of operations and size of the errors that we carefully justify. We focus on the case of GBM and use the approximate RNG methods presented in \cite{approximateICDF_Oliver} as well as a new slightly modified method that combines CDF inversion and the central limit theorem. To choose the precision of the variables in the low precision path generation, we introduce a novel method to optimise the bit-widths. This optimisation is performed before the main path generation loop is executed and is based on a linear model of the payoff error  
due to rounding when computing in low precision. The error model relies on algorithmic differentiation in a similar manner to \cite{unifying-bwoptim,bitwidth-AD,ADAPT}. The bit-width optimisation procedure can be performed off-line, so this stage can be excluded from the on-line time complexity of our framework. The user specified desired accuracy is then enforced by calculating on-line the number of samples that need to be generated.

In terms of hardware design, we suggest implementing the low precision path generation on FPGAs and the full-precision ones on a CPU or GPU. 
The FPGA offers enough flexibility to define a separate bit-width for every variable in the low precision path generation, and can be reconfigured periodically to update the bit-widths when the market parameters have changed considerably. 


The paper is organized as follows : \Cref{sec:MLMC} introduces MLMC and nested MLMC to make clear the estimator that is implemented in our framework. Then in \Cref{sec:RNG} we detail the methods that could be used to obtain approximate random normally distributed numbers very cheaply for the low precision path generation. In \Cref{sec:error_model} and \Cref{sec:costModel} we propose an error model and a cost model (resp.) that we then use to formulate the optimisation problem that is solved to obtain the optimal bit-widths of fixed point variables in \Cref{sec:optimisation}. Finally we summarise our results and future directions in \Cref{sec:conclusion}.



\section{Related Work}
\label{sec:related_work}

The original investigation \cite{gibson1979ecological} on the relationship between visual perception and human action defines \emph{affordance} as the opportunities for interaction with the surrounding environment. Behavioral studies on regular and cognitively impaired persons have shown evidence that perception results in both visual and motor signals in the human brain. An extended study \cite{anderson2002attentional} shows that visual attention to the spatial characteristics of the perceived objects initiates automatic motor signals for different actions. In computer vision, human affordance learning involves novel pose prediction such that the estimated pose represents a valid human action within the scene context. The task is fundamental to many problems requiring robust semantic reasoning about the environment, such as human motion synthesis \cite{wang2021scene} and scene-aware human pose generation \cite{wang2017binge, roy2016multi, zhang2022inpaint, yao2023scene}.

Earlier methods of affordance learning have explored knowledge mining \cite{zhu2014reasoning} and multimodal feature cues \cite{roy2016multi} to address the problem. In \cite{zhu2014reasoning}, the authors use a Markov Logic Network for constructing a knowledge base by extracting several object attributes from different image and metadata sources, which can perform various downstream visual inference tasks without any additional classifier, including zero-shot affordance prediction. In \cite{roy2016multi}, the authors use depth map, surface normals, and segmentation map as multimodal cues to train a multi-scale convolutional neural network (CNN) for scene-level semantic label assignment associated with specific human actions. In \cite{do2018affordancenet}, the authors design a multi-branch end-to-end CNN with two separate pathways for object detection and affordance label assignment to achieve high real-time inference throughput. Researchers \cite{chuang2018learning} have also explored socially imposed constraints for affordance learning. In \cite{chuang2018learning}, the authors propose a graph neural network (GNN) to propagate contextual scene information from egocentric views for action-object affordance reasoning.

Probabilistic modeling of scene-aware human motion generation also involves semantic reasoning of human interaction with the environment. Initial works on human motion synthesis have taken different architectural approaches, such as sequence-to-sequence models \cite{barsoum2018hp}, generative adversarial networks (GAN) \cite{barsoum2018hp, cai2018deep, yang2018pose}, graph convolutional networks (GCN) \cite{yan2019convolutional}, and variational autoencoders (VAE) \cite{guo2020action2motion}. However, these methods have mostly ignored the role of environmental semantics. Due to potential uncertainty in human motion, in a recent approach \cite{wang2021scene}, the authors address such motion synthesis with a GAN conditioned on scene attributes and motion trajectory to predict probable body pose dynamics.

One key challenge of human affordance generation in 2D scenes is the lack of large-scale datasets with rich pose annotations. In \cite{wang2017binge}, the authors compile the only public dataset of annotated human body poses in complex 2D indoor scenes by extracting frames from sitcom videos. Aiming to generate a contextually valid human affordance at a user-defined location, the authors propose sampling the scale and deformation parameters for an existing human pose template using a VAE conditioned on the localized image patches as scene context. In \cite{zhang2022inpaint}, the authors introduce a two-stage GAN architecture for achieving a similar goal by estimating the affine bounding box parameters to localize a probable human in the scene and then generating a potential body pose at that location. The method uses the input scene, corresponding depth, and segmentation maps as semantic guidance. In \cite{yao2023scene}, the authors propose a transformer-based approach with knowledge distillation for generating human affordances in 2D indoor scenes.


\section{Causal IL as CMRs}\label{sec:method}

In this section, we demonstrate that performing causal IL in our framework is possible using trajectory histories as instruments. In the next step, we show that the problem can be described as CMRs and propose an effective algorithm to solve it.

The typical target for IL would be the expert policy $\pi_E$ itself. However, since the expert has access to information, namely $u^o_t$, which the imitator does not, the best thing an imitator can do is to learn a history-dependent policy $\pi_h$ that is the closest to the expert. A natural choice is the conditional expectation of $\pi_E(s_t,u^o_t)$ on the history $h_t$:
\begin{align}
\pi_h(h_t)\coloneqq \expectE_{\probP(u^o_t\mid h_t)}[\pi_E(s_t,u^o_t)]=\expectE[\pi_E(s_t,u^o_t)\mid h_t],\nonumber
\end{align}
% where $p(u^o_t\mid h_t)$ is a distribution over expert-observable confounders and captures the information about $u^o_t$ can be inferred from the trajectory history. 
because the conditional expectation minimizes the least squares criterion~\citep{hastie01statisticallearning} and $\pi_h$ is the best predictor of $\pi_E$ given $h_t$. In $\pi_h$, the distribution $\probP(u^o_t\mid h_t)$ captures the information about $u^o_t$ that can be inferred from trajectory histories.
\begin{remark}
\emph{Learning $\pi_h$ is not trivial. Policies learnt naively using behaviour cloning (i.e., $\expectE[a_t\mid h_t]$) fail to match $\pi_E$. In view of~\cref{eq:action}, we have that
\begin{align} 
\expectE[a_t\mid h_t]&=\expectE[\pi_E(s_t,u^o_t) \mid h_{t}]+\expectE[u^\epsilon_t\mid h_{t}]\nonumber\\
&=\pi_h(h_t)+\expectE[u^\epsilon_t\mid h_{t}],\label{eq:history_policy}
\end{align}
where $\expectE[u^\epsilon_t\mid h_{t}]\neq 0$ due to the spurious correlation between $u^\epsilon_t$ and the trajectory history $h_t$. As a result, $\expectE[a_t\mid h_t]$ becomes biased, which can lead to arbitrarily worse performance compared to $\pi_E$.   }
\end{remark}

\vspace{-5pt}
\paragraph{Derivation of CMRs.} 
Leveraging the confounding horizon from Assumption~\ref{assump:horizon}, it becomes possible to break the spurious correlation using the independence of $u^\epsilon_t$ and $u^\epsilon_{t-k}$. We propose to use the $k$-step trajectory history $h_{t-k}=(s_{1},a_{1},...,s_{t-k})$ as an instrument for the current state $s_t$. Taking the expectation conditional on $h_{t-k}$ in~\cref{eq:history_policy} yields
\begin{align*}
    \expectE[a_t\mid h_{t-k}] & = \expectE\left[\expectE[a_t\mid h_{t}]\mid h_{t-k}\right] \\ & = \expectE[\pi_h(h_t)\mid h_{t-k}]+\expectE[\expectE[u^\epsilon_t\mid h_{t}]\mid h_{t-k}] \\
    & = \expectE[\pi_h(h_t) \mid h_{t-k}]+\expectE[u^\epsilon_t\mid h_{t-k}]
\end{align*}
where we use the fact that $h_{t-k}$ is $\sigma(h_t)$-measurable because $h_{t-k}\subseteq h_t$. Next, recall that $u^\epsilon_t\indep u^\epsilon_{t-k}$ by Assumption~\ref{assump:horizon}, which implies $u^\epsilon_t\indep h_{t-k}$, so that % Hence, since $\expectE[u^\epsilon_t] = 0$, we obtain
\begin{align}
    \expectE[a_t\mid h_{t-k}] &= \expectE[\pi_h(h_t) \mid h_{t-k}]+\expectE[u^\epsilon_t]\nonumber\\
    &=\expectE[\pi_h(h_t) \mid h_{t-k}].
\end{align}

As a result, the problem of learning $\pi_h$ reduces to solving for $\pi_h$ that satisfies the following identity
\begin{align}
    \expectE[a_t-\pi_h(h_t)\mid h_{t-k}]=0,\label{eq:CMR}
\end{align}
which is a CMR problem as defined in~\cref{sec:cmr}. In this case, both $a_t$ and $h_t$ are observed in the confounded expert demonstrations, and $h_{t-k}$ acts as the instrument. 

To make sure the instrument $h_{t-k}$ is valid, we check that it satisfies the conditions of~\cref{assump:iv}. Firstly, we have checked that $u^\epsilon_t\indep h_{t-k}$. Secondly, the environment and the expert policy are non-trivial, which means $\probP(h_t\mid h_{t-k})$ is not constant in $h_{t-k}$. Finally, $h_{t-k}$ indeed only affects $a_t$ through $s_t$ by the Markovian property. However, the strength of the instrument, which informally represents the correlation between the instrument $h_{t-k}$ and $h_t$, plays an important role in how well we can identify $\pi_h(h_t)$ by solving the CMRs in~\cref{eq:CMR}. In particular, we see that, as the confounding horizon $k$ increases, the correlation between $h_{t-k}$ and $h_t$ weakens and $h_{t-k}$ becomes a weaker instrument. This means that it is less able to identify $\pi_h$ via the CMR in~\cref{eq:CMR} and the final learnt imitator will have poorer performance. This is confirmed theoretically in Proposition~\ref{prop:ill-posed} and experimentally in~\cref{sec:exps}, and we will formalise this notion of instrument strength in~\cref{sec:theory}.


% Note this problem is equivalent to solving an IV regression on~\cref{eq:history_policy}, where $Y=\expectE[a_t\lvert h_t]$, $f(x)=\pi_h(h_t)$, $\epsilon=\expectE[u^\epsilon_t$ and the instrument $Z=h_{t-k}$.




\subsection{Practical Algorithms for Solving the CMRs}

\begin{algorithm}[tb]
   \caption{DML-IL}
   \label{alg:DML-IL}
\begin{algorithmic}[1]
   \STATE {\bfseries input} Dataset $\dataset_E$ of expert demonstrations, Confounding noise horizon $k$
   \STATE Initialize the roll-out model $\hat{M}$ as a Gaussian mixture model\label{algo:roll_out_1}
    \REPEAT
   \STATE Sample $(h_{t},a_t)$ from data $\dataset_E$
   \STATE Fit the roll-out model $(h_t,a_t)\sim\hat{M}(h_{t-k})$ to maximize the log likelihood 
\UNTIL{convergence}\label{algo:roll_out_2}
   \STATE Initialize the expert model $\hat \pi_h$ as a neural network
   \REPEAT
   % \FOR{$k=1$ {\bfseries to} $K$}
   \STATE Sample $h_{t-k}$ from $\dataset_E$
   \STATE Generate $\hat{h}_t$ and $\hat{a}_t$ using the roll-out model $\hat{M}$
   \STATE Update $\hat \pi_h$ to minimise the loss $\ell:= \norm{\hat{a}_t - \hat{\pi}_h (\hat h_t)}_2$
   % \ENDFOR
    \UNTIL{convergence}
    \STATE {\bfseries return} A history-dependent imitator policy $\hat{\pi}_h$
\end{algorithmic}
\end{algorithm}

There are various techniques~\citep{Shao2024,Bennett2019,Xu2020,Dikkala2020} for solving the CMRs $\expectE[a_t\lvert h_{t-k}]=\expectE[\pi_h(h_t) \lvert h_{t-k}]$. Here, the \textit{CMR error} that we aim to minimise is given by 
\begin{align*}
\sqrt{\expectE\big[\expectE[a_t-\hat{\pi}_h(h_t)\lvert h_{t-k}]^2\big]}=\norm{\expectE[a_t-\hat{\pi}_h(h_t)\lvert h_{t-k}]}_{2}.    
\end{align*}
In~\cref{alg:DML-IL}, we introduce DML-IL, an algorithm adapted from the IV regression algorithm DML-IV~\citep{Shao2024}\footnote{DML stands for double machine learning~\citep{Chernozhukov2018Double}, which is a statistical technique to ensure fast convergence rate for two-step regression, as is the case in~\cref{alg:DML-IL}.}, which solves our CMRs by minimising the CMR error. The first part of the algorithm (line 3-7) learns a roll-out model $\hat{M}$ that generates a trajectory $k$ steps ahead given $h_{t-k}$. Then, the roll-out model $\hat{M}$ is used to train the policy model $\hat{\pi}_h$ (line 8-13). $\hat{\pi}_h$ takes the generated trajectory $\hat{h}_t$ from $\hat{M}(h_{t-k})$ as inputs, and minimises the mean squared error to the next action. Using generated trajectories is crucial in breaking the spurious correlation caused by $u^\epsilon_t$ between past states and actions, and using the trajectory history before $h_{t-k}$ allows the imitator to infer information about $u^o_t$.

DML-IL can also be implemented with $K$-fold cross-fitting, where the dataset is partitioned into $K$ folds, with each fold alternately used to train $\hat{\pi}_h$ and the remaining folds to train $\hat{M}$. This ensures unbiased estimation and improves the stability of training. The base IV algorithm DML-IV with $K$-fold cross-fitting is theoretically shown to converge at the rate of $O(N^{-1/2})$~\citep{Shao2024}, where $N$ is the sample size, under regularity conditions. DML-IL with $K$-fold cross-fitting (see~\cref{appendix:dmlil} for details) will thus inherit this convergence rate guarantee. 

Note that~\cref{alg:DML-IL} requires the confounding noise horizon $k$ as input. While the exact value of $k$ can be difficult to obtain in reality, any upper bound $\bar{k}$ of $k$ is sufficient to guarantee the correctness of ~\cref{alg:DML-IL}, since $h_{t-\bar{k}}$ is also a valid instrument. Ideally, we would like a data-driven approach to determine $k$. Unfortunately, it is generally intractable to empirically verify whether $h_{t-k}$ is a valid instrument from a static dataset, especially the unconfounded instrument condition (i.e., $h_{t-k}\indep u^\epsilon_t$). Therefore, we rely on the user to provide a sensible choice of $\bar{k}$ based on the environment that does not substantially overestimate $k$.


\subsection{Theoretical Analysis}\label{sec:theory}

% \begin{align}
% p(u_t\lvert do(a_{t-k+1}),...,do(a_{t-1}),s_{t-k+1},...,s_{t-1})&\propto p(h_t)p_{\mu_0}(s_{t-k+1})\prod_{i=t-k+1}^{t-1} \transitions(s_{i+1}\lvert s_i,a_i,u_i)
% \end{align}

% since $$(u_t\indep a_{(t-k+1)...(t-1)} \lvert s_{(t-k+1)...(t_1)})_{\mathcal{G}_{\underline{a{(t-k+1)...(t-1)}}}}$$
% on the causal graph $\mathcal{G}_{\underline{a{(t-k+1)...(t-1)}}}$ where the arrows going into $a_{(t-k+1)...(t-1)}$ are removed.



In this section, we derive theoretical guarantees for our algorithm, focusing on the imitation gap and its relationship with existing work.


On a high level, in order to bound the imitation gap of the learnt policy $\hat{\pi}_h$, i.e., $J(\pi_E)-J(\hat{\pi}_h)$, we need to control:
\begin{enumerate}
    \item[($i$)] The amount of information about the hidden confounders that can be inferred from trajectory histories;
    \item[($ii$)] The ill-posedness (or identifiability) of the set of CMRs, which intuitively measures the strength of the instrument $h_{t-k}$;
    \item[($iii$)] The disturbance of the confounding noise to the states and actions at test time.
\end{enumerate}
These factors are all determined by the environment and the expert policy. To control ($i$), we measure how much information about $u^o_t$ is captured by the trajectory history $h_t$ by analysing the Total Variation (TV) distance between the distribution of $u^o_t$ and $\expectE[u^o_t\lvert h_t]$ along the trajectories of $\pi_E$. To control ($ii$) and ($iii$), we need to introduce the following two key concepts.

\begin{definition}[The ill-posedness of CMRs~\citep{Dikkala2020,Chen2012}]

Given the derived CMRs in~\cref{eq:CMR}, for a policy $\pi\in\Pi$, $\norm{\pi_E-\pi}_2$ is the root mean squared error to the expert and $\norm{\expectE[a_t-\pi(s_t)\lvert s_{t-k}]}_2$ is the CMR error we aim to minimise. Then, the \emph{ill-posedness} $\ill(\Pi,k)$ of the policy space with confounding noise horizon $k$ is given by
\begin{align*}
    \ill(\Pi,k)=\sup_{\pi\in\Pi} \frac{\norm{\pi_E-\pi}_{2}}{\norm{\expectE[a_t-\pi(h_t)\lvert h_{t-k}]}_{2}}.
\end{align*}
\end{definition}
The ill-posedness $\ill(\Pi,k)$ measures the strength of the instrument where a higher $\ill(\Pi,k)$ indicates a weaker instrument. It bounds the ratio between the learning error of the imitator following our CMR objective and its $L_2$ error to the expert policy. 

As discussed previously, intuitively, the strength of the instrument would decrease as the confounding horizon $k$ increases. This is in fact true and is confirmed by the following proposition. The proof is deferred to~\cref{appendix:prop}. 
\begin{proposition}\label{prop:ill-posed}
The ill-posedness $\ill(\Pi,k)$ is monotonically increasing as the confounded horizon $k$ increases.
\end{proposition}

Next, we introduce the notion of c-TV stability.
\begin{definition}[c-total variation stability~\citep{Bassily2021,Swamy2022_temporal}]
Let $P(X)$ be the distribution of a random variable $X:\Omega\rightarrow \mathcal{X}$. $P(X)$ is c-TV stable if for $a_1,a_2\in \mathcal{X}$ and $\Delta>0$,
\begin{align*}
\norm{a_1-a_2}\leq\Delta \implies \delta_{TV}(a_1+X,a_2+X)\leq c\Delta.
\end{align*}
where $\norm{\cdot}$ is some norm defined on $\mathcal{X}$ and $\delta_{TV}$ is the total variation distance.
\end{definition}
A wide range of distributions are c-TV stable. For example, standard normal distributions are $\frac{1}{2}$-TV stable. We apply this notion to the distribution over $u^\epsilon_t$ to bound the disturbance it induces in the trajectory and the expected return.

With the notion of ill-posedness and c-TV stability, we can now analyse and upper bound the imitation gap $J(\pi_E)-J(\hat{\pi}_h)$ by controlling the three components $(i)-(iii)$ discussed above. 
% We present the main result for this paper, where t
The full proof is deferred to~\cref{appendix:gap}.

\begin{theorem}[Imitation Gap Bound]\label{thm:gap}
Let $\hat{\pi}_h$ be the learnt policy with CMR error $\epsilon$ and let $\ill(\Pi,k)$ be the ill-posedness of the problem. Assume that $\delta_{TV}(u^o_t,\expectE_{\pi_E}[u^o_t\lvert h_t])\leq\delta$ for $\delta\in\realNumber^+$, $P(u^\epsilon_t)$ is c-TV stable and $\pi_E$ is deterministic. Then, the imitation gap is upper bounded by 
\begin{align*}
    J(\pi_E)-J(\hat{\pi}_h)\leq T^2\big(c\epsilon\ill(\Pi,k)+2\delta\big)=\mathcal{O}\big(T^2(\delta+\epsilon)\big).
\end{align*}
\end{theorem}
This upper bound scales at the rate of $T^2$, which aligns with the expected behaviour of imitation learning without an interactive expert~\citep{Ross2010}.
Next, we show that the upper bounds on the imitation gap from prior work~\citep{Swamy2022_temporal, Swamy2022} are special cases of
% of  subsumed by the unifying causal IL framework introduced in Section~\ref{sec:setting} are special cases of 
Theorem~\ref{thm:gap}. The proofs are deferred to~\cref{appendix:corollaries}.
\begin{corollary}\label{corollary:noUo}
In the special case that $u^o_t = 0$, i.e., there are no expert-observable confounders, or $u^o_t=\expectE_{\pi_E}[u^o_t\lvert h_t]$, i.e., $u^o_t$ is $\sigma(h_t)$ measurable (all information about $u^o_t$ is contained in the history), the imitation gap is upper bounded by
\begin{align*}
    J(\pi_E)-J(\hat{\pi}_h)\leq T^2\big(c\epsilon\ill(\Pi,k)\big)=\mathcal{O}\big(T^2\epsilon\big),
\end{align*}
which coincides with Theorem 5.1 of~\citet{Swamy2022_temporal}.
\end{corollary}

When there are no hidden confounders, i.e, $u^\epsilon_t=0$, our framework is reduced to that of~\citet{Swamy2022}. However, \citet{Swamy2022} provided an abstract bound that directly uses the supremum of key components in the imitation gap over all possible Q functions to bound the imitation gap. We further extend and concretise the bound using the learning error $\epsilon$ and the TV distance bound $\delta$ instead of relying on the suprema.


\begin{corollary}\label{corollary:unconfounded}
In the special case that $u^\epsilon_t=0$, if the learnt policy has optimisation error $\epsilon$,  the imitation gap is upper bounded by
\begin{align*}
    J(\pi_E)-J(\hat{\pi}_h)\leq T^2\left(\frac{2}{\sqrt{\dim(A)}}\epsilon+2\delta \right),
\end{align*}
which is a concrete bound that extends the abstract bound in Theorem 5.4 of~\cite{Swamy2022}.
\end{corollary}

\begin{remark}
\emph{If both $u^\epsilon_t$ and $u^o_t$ are zero, we then recover the classic setting of IL without confounders~\citep{Ross2010}, and the imitation gap bound is $T^2\epsilon$, where $\epsilon$ is the optimisation error of the algorithm.}
\end{remark}
\begin{figure}[!t]
\centering
\includegraphics[width=\linewidth]{Images/data_process_pipeline.pdf}
\caption{Demonstration of the \method{} data-building system. I: Data scoring procedure; II: Data filtering procedure; III: Data fixing and augmentation procedure. IV: Field data producing procedure.}
\vspace{-1em}
\label{fig:data_process_pipeline}
\vspace{-1em}
\end{figure}




\section{Data-Building System.}\label{sec:data}
\method{} is trained on existing open-source datasets such as Objaverse (-XL)\cite{deitke2023objaverse, deitke2024objaverse} and ShapeNet\cite{chang2015shapenet}, which contains approximately $10$ million 3D data. Since most of these data are sourced from the Internet, their quality varies significantly, requiring extensive preprocessing to ensure suitability for training.
To overcome these challenges, \method{} developed a dedicated 3D data processing system that produces high-quality, large-scale datasets for model training.
As illustrated in Fig.\ref{fig:data_process_pipeline}, the system comprises four processing stages (Data Process I$\sim$IV), responsible for data scoring, filtering, fixing and augmentation, and field data producing, respectively.

\subsection{I: Data Scoring}
Each 3D model is scored, with only the high-score models advancing to the subsequent processing stages.
Specifically, we randomly selected approximately $10K$ 3D models and used Blender to render four different views of normal maps for each model. These multi-view normal maps are then manually evaluated by 10 professional 3D modelers, assigning scores on a scale from $1$ (lowest) to $5$ (highest).
Using this annotated data, we trained a linear regression-based scoring model concatenating their CLIP\cite{radford2021learning} and DINOv2\cite{oquab2023dinov2} features as input. This model was subsequently used to infer quality scores from the multi-view normal maps of all 3D models for filtering.

\subsection{II: Data Filtering}
After scoring, further filtering is applied to exclude models with large planar bases, rendering errors in animations, and those containing multiple objects.
Specifically, models with large planar bases are filtered by determining if different surface patches can be classified as a single plane, based on features composed of their centroid positions, normal vectors, and the area of the resulting plane. Blender identifies animated models, sets them to the first frame, and filters out any models that still exhibit rendering errors after being set. And models containing multiple objects are filtered by evaluating the proportion of the largest connected component on the opaque mask, along with the magnitude of the solidity of both the largest connected component and the entire mask.

\subsection{III: Data Fixing and Augmentation}
After data filtering, we perform the orientation fixing of character models to ensure they face forward. Specifically, we select 24 orientations around the x, y, and z axes, and for each, render images from six orthogonal views: front, back, left, right, top, and bottom. The DINOv2\cite{oquab2023dinov2} features from these six views are concatenated to train an orientation estimation model, which is then used to infer and fix the orientation of all character models. Additionally, for all untextured models, we render multi-view normal maps and use ControlNet++\cite{li2024controlnet++} to generate corresponding multi-view RGB data, which serve as conditional inputs during training.

\subsection{IV: Field Data Production}
% \subsubsection{Data process IV: Training Data Reproducing}
Although Objaverse (-XL)~\cite{deitke2023objaverse,deitke2024objaverse} contains a large amount of data, most of the models are unsuitable for direct training, even after processing steps such as scoring, filtering, and fixing. 
Since we adopt the neural implicit field as our 3D model representation, it's necessary to convert the original non-watertight mesh to watertight ones for computing geometry supervision (e.g., occupancy or SDF). 
Rather than using common methods like TSDF-fusion~\cite{DBLP:conf/ismar/NewcombeIHMKDKSHF11} or ManifoldPlus~\cite{DBLP:journals/corr/abs-1802-01698,DBLP:journals/corr/abs-2005-11621}, we are inspired by \cite{DBLP:journals/tog/WangLT22,zhang2024clay} to construct a Unsigned Distance Function (UDF) field with a resolution of $512^3$ grid from the original non-watertight mesh, and then apply Marching Cubes~\cite{DBLP:conf/siggraph/LorensenC87} to extract the iso-surface with a small threshold $\tau=\frac{3}{512}$. 
To remove interior structure for more efficient geometry learning, we follow \cite{zhang2024clay} by resetting the UDF value of the invisible grids to prevent the extraction of interior iso-surface before applying Marching Cubes. 
We then remove some small and invisible interior mesh components by calculating the area and the ambient occlusion ratio of each mesh component.
Finally, we uniformly sample surface points along with their normals, and randomly sample points both within the volume and near the surface.



\section{Experiment}
In this section, we conduct extensive experiments to evaluate the performance of various LLMs on our Hellaswag-Pro benchmark. Our study is guided by three key research questions:
\textbf{RQ1}: How do different LLMs perform across all variants?
\textbf{RQ2}: What is the relative difficulty of different variants?
\textbf{RQ3}: How robust are LLMs to diverse prompts during evaluation?

\subsection{Experiment Setup} 
\subsubsection{Model Selection and Implementation Details}
We select 41 representative commercial and open-source models, including English LLMs, such as GPT-4o, Claude-3.5-Sonnet, Gemini-1.5-Pro,Mistral series, Llama3 series and Chinese LLMs, like Qwen-Max,  Qwen2.5 series, InternLM-2.5 series, Yi-1.5 series, Baichuan-2 series and DeepSeek series.

We integrate both Chinese HellaSwag and HellaSwagPro into the lm-evaluation-harness platform. For the open-source models, we use the default settings of lm-evaluation-harness: do\_sample is set to false and the temperature is set to the default value of the hugging-face library. For the closed-source models, we set the temperature to 0.7. In addition, we set the maximum output length to 1024.

\subsubsection{Prompt Strategy}
Taking into account the influence of language and shot, we design 9 prompting strategies, including Direct, CN-CoT, EN-CoT, CN-XLT and EN-XLT. The last four setups include both zero-shot and few-shot variants.\footnote{
For open-source models, Direct adopts an approach similar to the official implementation of HellaSwag, computing the log-likelihood for each option and selecting the one with the highest log-likelihood. And we report normalized accuracy that accounts for the impact of option length. Other prompting strategies use a generation setup and report accuracy based on exact match.}
\textbf {(1)Direct}: LLMs makes the selection directly without any CoT process.
\textbf{(2)CN-CoT}: LLMs performs CoT in Chinese, regardless of dataset language.
\textbf{(3)EN-CoT}: Similar to CN-CoT, but CoT is conducted in English. 
\textbf{(4)CN-XLT}: LLMs are instructed to first translate English questions and options to Chinese, and then reason in Chinese.
\textbf{(5)EN-XLT}: Similar to CN-XLT, but translates from Chinese dataset to English and reasons in English. 

%\textbf {CN-CoT}: LLMs perform Chinese reasoning and then output the answer and 3 shots are provided.
%\textbf {CN-CoT}: Similar as CNCoTFewShot without any shots.
%\textbf {EN-CoT}: The reasoning process in English is executed and then the answer is output and 3 shots are provided.
%\textbf {CN-XLT}: Inspired by this, we instruct LLMs to translate questions in Chinese and then output the answer after performing reasoning in Chinese too. And 3 shots are provided.
%\textbf {EN-XLT}: Inspired by this, we instruct LLMs to translate questions in Englsih and then output the answer after performing reasoning in Englsih too. Three shots are provided.

\subsubsection{Evaluation metric}

To comprehensively evaluate the robustness of each LLM, we consider four metrics: 
% Original Accuracy (\textbf{OA}), Average Robust Accuracy (\textbf{ARA}), Robust Loss Accuracy (\textbf{RLA}), and  Consistent Robust Accuracy (\textbf{CRA}).
\noindent %
\textbf{- Original Accuracy (OA)} measures accuracy on original problems.
\begin{equation}\label{eq1}
OA=\frac{\sum_{(x, y) \in D} \mathds{1}[L M(x), y]}{|D|}.
\end{equation}
\noindent %
\textbf{- Average Robust Accuracy  (ARA)} represents average accuracy across all variants, gauging overall performance on the robustness tasks.
\begin{equation}\label{eq2}
ARA=\frac{\sum_{\left(x^{\prime}, y^{\prime}\right) \in D_{R}} \mathds{1}\left(L M\left(x^{\prime}, y^{\prime}\right)\right.}{\left|D_{R}\right|}.
\end{equation}

\noindent %
\textbf{- Robust Loss Accuracy (RLA)} is the difference between ARA and OA, indicating performance degradation on robustness data versus original data.
%\begin{tiny}
%\begin{equation}\label{eq3}
%RLA=\frac{\sum_{\left(x^{\prime}, y^{\prime}\right) \in D_{R}} %\mathds{1}\left(L M\left(x^{\prime}, y^{\prime}\right)\right.}{\left|D_{R}\right|}-\frac{\sum_{(x, y) \in D}\mathds{1}[L M(x), y]}{|D|}
%\end{equation}
%\end{tiny}
\begin{equation}\label{eq3}
RLA= OA - ARA.
\end{equation}
\noindent %
\textbf{- Consistent Robust Accuracy (CRA)} shows accuracy when the model correctly answers both original and variant data, reflecting the model do understand the problem.
% consistency in problem-solving.
\begin{equation}\label{eq4}
CRA=\frac{\sum_{x, y, x^{\prime}, y^{\prime}}\mathds{1}[L M(x), y] \cdot \mathds{1}[L M(x^{\prime}), y^{\prime}]}{\left|D_{R}\right|}.
\end{equation}
For all equation above, $D$ denotes the original dataset, where $x$ represents the input question and options, and $y$ represents the correct label, while $D_{R}$ is the robust dataset with $x^{\prime}$ and $y^{\prime}$ representing similar to $x$ and $y$.


\begin{table*}[ht]
\centering
\setlength{\tabcolsep}{5pt}
% \footnotesize
\scalebox{0.6}{
% Please add the following required packages to your document preamble:
% \usepackage{multirow}
% \usepackage[table,xcdraw]{xcolor}
% Beamer presentation requires \usepackage{colortbl} instead of \usepackage[table,xcdraw]{xcolor}
% Please add the following required packages to your document preamble:
% \usepackage{multirow}
% \usepackage[table,xcdraw]{xcolor}
% Beamer presentation requires \usepackage{colortbl} instead of \usepackage[table,xcdraw]{xcolor}
\begin{tabular}{ccccccccccccc}
\hline
\multicolumn{1}{c|}{{ }}& \multicolumn{4}{c|}{Chinese}& \multicolumn{4}{c|}{English}& \multicolumn{4}{c}{AVG}\\ \cline{2-13} 
\multicolumn{1}{c|}{\multirow{-2}{*}{{ Model}}} & { OA(\%)$\uparrow$}& { ARA(\%)$\uparrow$} & {RLA(\%)$\downarrow$}& \multicolumn{1}{l|}{{CRA(\%)$\uparrow$}} & { OA(\%)$\uparrow$}& { ARA(\%)$\uparrow$} & { RLA(\%)$\downarrow$}& \multicolumn{1}{l|}{{CRA(\%)$\uparrow$}} & {OA(\%)$\uparrow$}& { ARA(\%)$\uparrow$} & {RLA(\%)$\downarrow$}& { CRA(\%)$\uparrow$} \\ \hline
\multicolumn{1}{c|}{{ Human}} & 96.41& 97.79& -1.38 & \multicolumn{1}{l|}{92.03}& 95.56& 96.04& -0.48 & \multicolumn{1}{l|}{90.02}& 95.99 & 96.92 & -0.93& 91.03 \\ \hline
\multicolumn{13}{c}{\textit{Close-source LLMs}}\\ 
\multicolumn{1}{c|}{{ GPT-4o}}& { 91.37} & { 81.97} & { 9.40}& \multicolumn{1}{l|}{{ 75.55}} & { \textbf{88.63}} & { \textbf{70.17}} & { \textbf{18.46}} & \multicolumn{1}{l|}{{ \textbf{63.06}}} & { 90.00} & { \textbf{76.07}} & { \textbf{13.93}} & { \textbf{69.31}} \\
\multicolumn{1}{c|}{{ Claude3.5}}& { \textbf{95.37}} & { 80.15} & { 15.22} & \multicolumn{1}{l|}{{ 75.04}} & { 85.11} & { 66.02} & { 19.08} & \multicolumn{1}{l|}{{ 57.20}} & { 90.24} & { 73.09} & { 17.15} & { 66.12} \\
\multicolumn{1}{c|}{{ Gemini-1.5-Pro}}& { 90.62} & { 78.36} & { 12.26} & \multicolumn{1}{l|}{{ 70.48}} & { 87.75} & { 60.74} & { 27.01} & \multicolumn{1}{l|}{{ 58.27}} & { 89.19} & { 69.55} & { 19.63} & { 64.38} \\
\multicolumn{1}{c|}{{ Qwen-Max}}& { 93.50} & { \textbf{84.82}} & { \textbf{8.68}}& \multicolumn{1}{l|}{{ \textbf{78.91}}} & { 87.60} & { 62.61} & { 24.99} & \multicolumn{1}{l|}{{ 59.65}} & { \textbf{90.55}} & { 73.72} & { 16.83} & { 69.28} \\ \hline
\multicolumn{13}{c}{\textit{Chinese open-source LLMs}} \\ 
\multicolumn{1}{c|}{{ Qwen2.5-0.5B}}& { 60.75} & { 45.18} & { \textbf{15.57}} & \multicolumn{1}{l|}{{ 28.70}} & { 49.50} & { 38.21} & { \textbf{11.29}} & \multicolumn{1}{l|}{{ 20.57}} & { 55.13} & { 41.70} & { \textbf{13.43}} & { 24.64} \\
\multicolumn{1}{c|}{{ Qwen2.5-1.5B}}& { 63.25} & { 46.16} & { 17.09} & \multicolumn{1}{l|}{{ 29.89}} & { 56.88} & { 39.57} & { 17.30} & \multicolumn{1}{l|}{{ 23.48}} & { 60.06} & { 42.87} & { 17.20} & { 26.69} \\
\multicolumn{1}{c|}{{ Qwen2.5-3B}}& { 67.50} & { 48.75} & { 18.75} & \multicolumn{1}{l|}{{ 33.79}} & { 61.75} & { 39.98} & { 21.77} & \multicolumn{1}{l|}{{ 25.75}} & { 64.63} & { 44.37} & { 20.26} & { 29.77} \\
\multicolumn{1}{c|}{{ Qwen2.5-7B}}& { 67.63} & { 50.59} & { 17.04} & \multicolumn{1}{l|}{{ 35.62}} & { 65.63} & { 43.93} & { 21.70} & \multicolumn{1}{l|}{{ 30.77}} & { 66.63} & { 47.26} & { 19.37} & { 33.20} \\
\multicolumn{1}{c|}{{ Qwen2.5-14B}} & { 69.00} & { 51.41} & { 17.59} & \multicolumn{1}{l|}{{ 35.84}} & { 68.50} & { 45.20} & { 23.30} & \multicolumn{1}{l|}{{ 32.12}} & { 68.75} & { 48.30} & { 20.45} & { 33.98} \\
\multicolumn{1}{c|}{{ Qwen2.5-32B}} & { 69.75} & { 53.11} & { 16.64} & \multicolumn{1}{l|}{{ 37.54}} & { 70.00} & { 46.10} & { 23.90} & \multicolumn{1}{l|}{{ 32.68}} & { 69.88} & { 49.61} & { 20.27} & { 35.11} \\
\multicolumn{1}{c|}{{ Qwen2.5-72B}} & { \textbf{70.87}} & { \textbf{54.75}} & { 16.12} & \multicolumn{1}{l|}{{ \textbf{39.64}}} & { \textbf{72.00}} & { \textbf{47.75}} & { 24.25} & \multicolumn{1}{l|}{{\textbf{ 35.12}}} & { \textbf{71.44}} & { \textbf{51.25}} & {20.19} & { \textbf{37.38}} \\ \hdashline[0.5pt/5pt]
\multicolumn{1}{c|}{{ Baichuan2-7B}}& { 67.00} & { 46.16} & { 20.84} & \multicolumn{1}{l|}{{ 31.50}} & { 60.62} & { 39.04} & { 21.58} & \multicolumn{1}{l|}{{ 25.21}} & { 63.81} & { 42.60} & { 21.21} & { 28.36} \\
\multicolumn{1}{c|}{{ Baichua2-13B}}& { 69.13} & { 46.98} & { 22.15} & \multicolumn{1}{l|}{{ 33.45}} & { 64.62} & { 38.82} & { 25.80} & \multicolumn{1}{l|}{{ 26.07}} & { 66.88} & { 42.90} & { 23.97} & { 29.76} \\ \hdashline[0.5pt/5pt]
\multicolumn{1}{c|}{{ DeepSeek-7B}} & { 68.13} & { 47.96} & { 20.17} & \multicolumn{1}{l|}{{ 33.30}} & { 63.38} & { 40.39} & { 22.99} & \multicolumn{1}{l|}{{ 26.70}} & { 65.76} & { 44.18} & { 21.58} & { 30.00} \\
\multicolumn{1}{c|}{{ DeepSeek-67B}}& { 71.50} & { 49.21} & { 22.29} & \multicolumn{1}{l|}{{ 35.89}} & { 71.37} & { 40.63} & { 30.75} & \multicolumn{1}{l|}{{ 29.71}} & { 71.44} & { 44.92} & { 26.52} & { 32.80} \\ \hdashline[0.5pt/5pt]
\multicolumn{1}{c|}{{ InternLM2.5-1.8B}}& { 61.62} & { 42.07} & { 19.55} & \multicolumn{1}{l|}{{ 26.99}} & { 55.37} & { 38.46} & { 16.91} & \multicolumn{1}{l|}{{ 22.61}} & { 58.50} & { 40.27} & { 18.23} & { 24.80} \\
\multicolumn{1}{c|}{{ InternLM2.5-7B}}& { 67.25} & { 49.77} & { 17.48} & \multicolumn{1}{l|}{{ 34.57}} & { 69.50} & { 40.89} & { 28.61} & \multicolumn{1}{l|}{{ 29.75}} & { 68.38} & { 45.33} & { 23.04} & { 32.16} \\
\multicolumn{1}{c|}{{ InternLM2.5-20B}} & { 67.37} & { 48.08} & { 19.29} & \multicolumn{1}{l|}{{ 33.21}} & { 73.62} & { 41.11} & { 32.51} & \multicolumn{1}{l|}{{ 31.23}} & { 70.50} & { 44.60} & { 25.90} & { 32.22} \\ \hdashline[0.5pt/5pt]
\multicolumn{1}{c|}{{ Yi-1.5-6B}} & { 67.00} & { 49.59} & { 17.41} & \multicolumn{1}{l|}{{ 34.27}} & { 64.38} & { 39.37} & { 25.01} & \multicolumn{1}{l|}{{ 26.62}} & { 65.69} & { 44.48} & { 21.21} & { 30.45} \\
\multicolumn{1}{c|}{{ Yi-1.5-9B}} & { 68.50} & { 50.18} & { 18.32} & \multicolumn{1}{l|}{{ 35.55}} & { 66.37} & { 39.58} & { 26.79} & \multicolumn{1}{l|}{{ 27.48}} & { 67.44} & { 44.88} & { 22.56} & { 31.52} \\
\multicolumn{1}{c|}{{ Yi-1.5-34B}}& { 71.00} & { 52.23} & { 18.77} & \multicolumn{1}{l|}{{ 38.09}} & { 71.00} & { 40.75} & { 30.25} & \multicolumn{1}{l|}{{ 29.91}} & { 71.00} & { 46.49} & { 24.51} & { 34.00} \\ \hline
\multicolumn{13}{c}{\textit{English open-source LLMs}} \\ 
\multicolumn{1}{c|}{{ Llama3-8B}} & { 59.13} & { 46.62} & { 12.51} & \multicolumn{1}{l|}{{ 28.23}} & { 66.25} & { 40.21} & { 26.04} & \multicolumn{1}{l|}{{ 27.34}} & { 62.69} & { 43.42} & { 19.27} & { 27.79} \\
\multicolumn{1}{c|}{{ Llama3-70B}}& { 65.75} & { 48.63} & { 17.12} & \multicolumn{1}{l|}{{ 32.70}} & { \textbf{72.50}} & { 41.27} & { 31.23} & \multicolumn{1}{l|}{{\textbf{ 30.63}}} & {\textbf{ 69.13}} & { 44.95} & { 24.18} & { 31.67} \\ \hdashline[0.5pt/5pt]
\multicolumn{1}{c|}{{ Mistral-7B-v0.2}} & { 57.75} & { 46.25} & { \textbf{11.50}} & \multicolumn{1}{l|}{{ 27.57}} & { 67.50} & { \textbf{41.52}} & { 25.98} & \multicolumn{1}{l|}{{ 28.93}} & { 62.63} & { 43.88} & { 18.74} & { 28.25} \\
\multicolumn{1}{c|}{{ Mixtral-8x7B-v0.1}} & { 63.62} & { 46.80} & { 16.82} & \multicolumn{1}{l|}{{ 30.82}} & { 69.75} & { 41.21} & { 28.54} & \multicolumn{1}{l|}{{ 29.39}} & { 66.69} & { 44.01} & { 22.68} & { 30.11} \\
\multicolumn{1}{c|}{{ Mixtral-8x22B-v0.1}}& { 66.00} & {\textbf{ 50.73}} & { 15.27} & \multicolumn{1}{l|}{{ \textbf{34.32}}} & { 72.12} & { 41.25} & { 30.87} & \multicolumn{1}{l|}{{ 30.61}} & { 69.06} & { \textbf{45.99}} & { 23.07} & { \textbf{32.47}} \\ \hdashline[0.5pt/5pt]
\multicolumn{1}{c|}{{ Gemma-2-2B}}& { 61.88} & { 45.38} & { 16.51} & \multicolumn{1}{l|}{{ 29.02}} & { 59.62} & { 39.13} & { \textbf{20.50}} & \multicolumn{1}{l|}{{ 24.88}} & { 60.75} & { 42.25} & {\textbf{ 18.50}} & { 26.95} \\
\multicolumn{1}{c|}{{ Gemma-2-9B}}& { \textbf{69.13}} & { 46.75} & { 22.38} & \multicolumn{1}{l|}{{ 33.29}} & { 64.88} & { 39.80} & { 25.08} & \multicolumn{1}{l|}{{ 26.91}} & { 67.01} & { 43.28} & { 23.73} & { 30.10} \\
\multicolumn{1}{c|}{{ Gemma-2-27B}} & { 63.38} & { 48.52} & { 14.86} & \multicolumn{1}{l|}{{ 31.96}} & { 71.88} & { 40.91} & { 30.97} & \multicolumn{1}{l|}{{ 30.25}} & { 67.63} & { 44.71} & { 22.92} & { 31.11} \\ \hline
\end{tabular}
}
\caption{TODO: bolded is not result. Results of existing LLMs on our HellaSwag-Pro dataset using \textbf{Direct} prompt. ``AVG'' indicates the average performance of each model on Chinese and English parts of the dataset.
The best results for each metric in each model category are \textbf{bolded}. }
\label{tab:main experiment.}
\end{table*}

\subsection{Model Performance (RQ1)}
\paragraph{Overall Performance}
Table \ref{tab:main experiment.} provides a comprehensive evaluation of various LLMs across four performance metrics\footnote{The results of instruct/chat models of Qwen2.5, Llama3 and Mixtral latest series are shown in Appendix.}. The main observations are as follow:
\begin{itemize}[leftmargin=*,topsep=0pt]
% \setlength{}{0}
    \item Upon evaluating all available models, we found that all performed well in overall accuracy (e.g., GPT-4 scored 90.00 in AVG OA, Claude 3.5 scored 90.24 in AVG OA). However, all models struggled with variations of the questions, as evidenced by a positive RLA value for each model. In contrast, humans received a negative RLA value, suggesting that the question variants were not more challenging than the originals. This disparity further illustrates that current LLMs lack a true understanding of the reasoning process and can easily be misled by question variants.
    \item When comparing open-source and close-source models, the close-source models demonstrate stronger capabilities in both OA and ARA scores, similar to most existing benchmarks. Overall, the RLA values for close-source models are also smaller, indicating that they are more robust in commonsense reasoning tasks compared to open-source models.
    \item When we compare models within the same series (e.g., Qwen, Llama), we observe that larger models often achieve higher scores on OA, ARA, and CRA. However, they are also more susceptible to variations, i.e., they have higher RLA values, a phenomenon particularly evident in English datasets. We attribute this phenomenon to the fact that larger models, compared to smaller ones, may have memorized more data, allowing them to rely on memorization to solve some problems more easily and making them more prone to the influence of variations~\cite{}.
\end{itemize}
% 1. When evaluating all available models, We find although 
% 2. When comparing the opensource LLMs and close source LLMs, 
% 3. When looking into each serious details
% \noindent
% \textbf{Overall Model Performance.}
% 1. close-source > open-source 2. the large the better 3. all have a performance decline when meeting varients.

% To evaluate the performance of various models, we observed patterns consistent with current mainstream trends: closed-source models generally outperform open-source models across metrics. 
% For instance, the closed-source model GPT-4o achieved scores of 90.00 in OA, 76.07 in ARA, and 69.31 in CRA, whereas the open-source model Qwen2.5-72B scored 71.44, 51.25, and 37.38, respectively. 
% Furthermore, within each model series, performance tends to improve with larger model sizes. 
% Nevertheless, even the strongest closed-source models struggle with variations in questions, as indicated by positive values in RLA for all models. In contrast, human performance yields a negative RLA value, highlighting that current LLMs do not genuinely grasp the reasoning process and are prone to falling into traps set by question variants. 
% This suggests that there is still significant room for improvement in developing models that can robustly understand and reason through complex linguistic challenges.
% It reveals a consistent pattern across Chinese, English, and average scores, with close-sourced LLMs generally outperforming open-sourced models. 
% However, all models exhibit a significant drop in performance when faced with robust variants, as indicated by RLA and CRA. Among closed-source models, GPT-4o demonstrates the highest ARA of 76.07\% in average scores, demonstrating its overwhelming superiority. Among open-sourced models, larger models tend to perform better, with Qwen2.5-72B achieving the highest OA (71.44\%) and ARA (51.25\%) in the average scores. However, even these top performers still struggle with robustness, as evidenced by the substantial RLA of 13.93\% for GPT-4o and 20.19\% for Qwen2.5-72B. Interestingly, some English open-sourced models, such as Llama3-70B and Mixtral-8x22B-v0.1, show competitive performance in English tasks but lag in Chinese tasks, highlighting the importance of language-specific training.

% \noindent
% \textbf{Chinese Models vs English Models.}
% Chinese models generally demonstrate higher OA in Chinese tasks compared to English tasks, with Qwen-Max achieving 93.50\% OA in Chinese versus 87.60\% in English. Conversely, English models tend to perform better in English tasks, exemplified by Llama3-70B's 72.50\% OA in English compared to 65.75\% in Chinese. 
% However, both Chinese and English models exhibit important drops in ARA across languages, indicating challenges in maintaining performance when faced with variations. This trend suggests that while models may excel in their primary language, they struggle with robustness across linguistic boundaries. 
% Notably, larger models tend to achieve higher ARA scores but also experience more substantial RLA, as seen with Qwen2.5-0.5B (41.70\% ARA, 13.43\% RLA in total) and Qwen2.5-72B (51.25\% ARA, 20.19\% RLA in total). 
% This pattern indicates that while increased model size enhances overall performance, it doesn't necessarily improve robustness proportionally. 
% The discrepancy between OA and ARA across languages underscores the need for improved cross-lingual robustness in language models, particularly as they scale in size and capability.


% \noindent
% \textbf{Comparison between Chinese and English datasets.}
% Generally, models demonstrate higher accuracy on the Chinese dataset compared to the English one, as evidenced by the consistently higher OA, ARA and CRA scores. For instance, GPT-4o achieves an OA of 91.37\%, an ARA of 81.97\% , an CRA of 75.55\% on the Chinese dataset, compared to 88.63\% and 70.17\% respectively on the English dataset. This trend is observed across most models, suggesting that the Chinese dataset is easier than English one. Moreover, the RLA values are typically lower for Chinese, indicating smaller performance drops when dealing with robust variants of Chinese questions. For example, Qwen-Max shows an RLA of 8.68\% for Chinese versus 24.99\% for English, highlighting a more consistent performance in Chinese. The CRA scores further reinforce this observation, with models generally maintaining higher consistency in correct answers for both original and variant Chinese questions.
% We attribute this phenomenon to the fact that blablabla

\noindent
\textbf{Reasoning Transferable Capability.}
% 为了进一步
To further analyze whether the model can transfer reasoning ability from the original question to its variant, Figure \ref{consis} presents the distribution of model performance on the original question and variant pairs. For all models, the pairs of (HellaSwag \ding{51} HellaSwag-Pro \ding{55}) occupy a significant proportion, indicating a challenge in transferring reasoning capabilities for current LLMs to more complex scenarios. Looking deeply, closed-source models like GPT-4 and Qwen-Max achieve around a 69\% portion of (HellaSwag \ding{51} HellaSwag-Pro \ding{51}) and a 3\% portion of (HellaSwag \ding{55} HellaSwag-Pro \ding{55}), while in contrast, open-source models struggle with around a 30\% portion of (HellaSwag \ding{51} HellaSwag-Pro \ding{51}) and a 20\% portion of (HellaSwag \ding{55} HellaSwag-Pro \ding{55}), further indicating the robustness of reasoning abilities in closed-source models.
% If a model can get both the original question and the variant right, we consider it to have transferable reasoning ability. Table \ref{consis} presents the distribution of model performance on the original question and variant pairs. Among all models, the pairs of (HellaSwag \ding{51}HellaSwag-Pro \ding{55}) account for a considerable proportion, i 
% The closed-source models like GPT-4o and Qwen-Max achieve around 69\% portion of (HellaSwag \ding{51}HellaSwag-Pro \ding{51}) and 3\% portion of (HellaSwag \ding{55} HellaSwag-Pro \ding{55}), indicating stronger reasoning transfer ability than other models. In contrast, open-source models struggle more, with around 30\% portion of (HellaSwag \ding{51}HellaSwag-Pro \ding{51}) and 20\% portion of (HellaSwag \ding{55} HellaSwag-Pro \ding{55}). 
% A notable trend is observed among the Qwen2.5 series, where increasing model size from 7B to 72B parameters correlates with improved performance on correct answers for both datasets (33.20\% to 37.38\%) and decreased failure rates (17.69\% to 14.7\%). It underscores the importance of model size in commonsense reasoning tasks.

\begin{figure}[t]
\centering
\setlength{\abovecaptionskip}{0.1cm}
\setlength{\belowcaptionskip}{0cm}
\includegraphics[width=\linewidth,scale=1.00]{images/consis.pdf}
\caption{Analysis of the transferable ability of model reasoning based on question pair performance. The green part, where both the original and the variant data are right, represents the transferable performance of model reasoning.}
\label{consis}
\vspace{-15pt}
\end{figure}

\begin{figure*}[ht]
\centering
\setlength{\abovecaptionskip}{0.1cm}
\setlength{\belowcaptionskip}{0cm}
\includegraphics[width=\linewidth,scale=1.00]{images/xing.pdf}
\caption{The impact of different few-shot prompts on model performance. With - as the separator, the first two parts of the legend represent the prompt name, and the third part represents the language of the dataset.}
\label{xing}
\vspace{-15pt}
\end{figure*}

\begin{figure}[ht]
\centering
\setlength{\abovecaptionskip}{0.1cm}
\setlength{\belowcaptionskip}{0cm}
\includegraphics[width=1.05\linewidth,scale=1.05]{images/zhu.pdf}
\caption{The RLA Distribution for 7 variants of commonsense reasoning. Parts below the 0 axis indicate that the model’s performance on the variant is improved compared to the original problem.}
\label{fig:zhu}
\vspace{-15pt}
\end{figure}


\subsection{Variant Analysis (RQ2)}
To further analyze the impact of different variants, we assessed the contribution of each variant to the RLA score. A higher contribution indicates that the model is more likely to make errors in that type. Figure~\ref{fig:zhu} presents the overall results, and the key observations are as follows:
\begin{itemize}[leftmargin=*]
    \item For problem restatement, causal inference, and sentence ordering, these three categories are the least challenging. Almost all models, particularly the close-source and Qwen series models, perform well on these variants, indicating that current LLMs can effectively handle these forms and we do not pay more attention on this kind of varients.
    \item For reverse conversion and critical testing, these two varients each contribute about 10\% to the RLA score. This indicates that current LLMs struggle to fully generalize to these simple scenarios, possibly because these types of questions are not commonly encountered, and reaserchers should pay some attention to this type of varients.
    \item For negative transformation and scenario refinement, this are the two most difficult tasks, with negative transformation being particularly challenging. For almost all models, these two varients accounts for more than 50\% of the RLA score. This may be due to intuitively counterintuitive questions—such as the use of "will not"  or counterfactual scenarios in scenario refinement. These setups are less common in LLM training data and cannot be easily tackled through memory alone. Only those LLMs which truely understand the question could answer the varient correctly, wihch better reflect the true performance of the model.. In the future, researchers should focus more on enhancing LLM's capability to address such types of questions.
\end{itemize}

% 1. Problem restCausal Inference 
% To further analysis the impact of different varients, we further 
% Figure \ref{fig: zhu} presents a comprehensive analysis of various LLMs' performance across different variant types. Negative transformation emerges as the most challenging task for all models, with scores consistently above 50.00\% and peaking at 78.38\% for Gemini-1.5-Pro. Conversely, problem restatement appears to be the least challenging, with most models scoring in the negative range. Intriguingly, smaller models like Qwen2.5-0.5B demonstrate unexpected strengths in certain areas, such as sentence sorting (7.75\%), outperforming some larger counterparts. A detailed analysis of each variant type follows.

% \noindent
% \textbf{Causal inference.} In this category, scores vary widely from -4.73\% for Qwen-Max to 12.25\% for Baichuan2-13B, illustrating differing degrees of sensitivity to causal reasoning among the models. Smaller models, such as Qwen2.5-0.5B and Qwen2.5-1.5B, achieve better scores, indicating relatively stronger robustness in causal reasoning. Conversely, larger models, like Baichuan2-13B, have higher scores, suggesting greater sensitivity to the challenges of inferring causality.

% \noindent
% \textbf{Critical testing.} Larger models, including Qwen2.5-72B and DeepSeek-67B, exhibit higher RLA scores of 30.50\% and 31.37\%, respectively, suggesting increased sensitivity when dealing with incomplete key information. In contrast, GPT-4o achieves the lowest score, highlighting its superior robustness in critical reasoning. This trend indicates that more complex models might struggle to handle incomplete contexts, underscoring potential areas for improvement in sophisticated architectures.

% \noindent
% \textbf{Negative transformation.} This aspect remains consistently challenging for all models, with scores ranging from 48.88\% to 78.38\%. Advanced commercial models like Gemini-1.5-Pro and Claude-3.5 also score higher (78.38\% and 76.43\%, respectively), indicating a prevalent sensitivity issue in reasoning processes when handling negations, irrespective of model size or architecture.

% \noindent
% \textbf{Problem restatement.} The negative values in this category for nearly all models suggest it is not particularly challenging. This is surprising, given that previous models were quite sensitive to sentence representation.

% \noindent
% \textbf{Reverse conversion.} This variation, which involves swapping the roles of the question and answer, seems to specifically impact larger models. For example, Qwen2.5-72B and DeepSeek-67B exhibit higher RLA scores of 24.38\% and 27.43\%, respectively, indicating heightened sensitivity to reverse reasoning compared to their performance on original questions.

% \noindent
% \textbf{Scenario refinement.} The scores range from 16.06\% for Gemma-2-2B to 32.56\% for Qwen2.5-72B, with larger models displaying more sensitivity in adapting to counterfactual predictions. This suggests that larger models may rely more heavily on general commonsense rather than flexibly adapting to specific contexts. Consequently, increased model complexity might adversely affect adaptability to scenario changes, underscoring the need for enhanced flexibility in advanced models.

% \noindent
% \textbf{Sentence sorting.} This category exhibits the most varied results across models. Some larger models like DeepSeek-67B and InternLM2.5-20B display higher scores (26.69\% and 26.68\%), indicating sensitivity, while others like Qwen2.5-72B and Gemini-1.5-Pro excel with lower scores (-9.88\% and -1.07\%, respectively). This suggests that sentence sorting ability may depend more on specific training approaches rather than being solely contingent on model size.


\subsection{Prompt Robustness (RQ3)}
% To investigate how prompt  influence our benchmark, we apply sereral prompt strategy on our datasets and showcase the average performance of all models on different kind of prompt strategies.
% Table~\ref{prompt} illustrates the final results. For both Chinese and English datasets, CN LLMs achieve the highest performance using CN-CoT-Few-Shot, followed closely by EN-CoT-Few-Shot, with overall performance scores of 67.36\% and 67.03\%, respectively. In contrast, English LLMs perform best with the EN-CoT-Few-Shot, reaching 67.55\% on the Chinese dataset and 60.36\% on the English dataset.
% Contrary to previous findings, translating the dataset to the model's advantage language before performing reasoning does not enhance performance. Moreover, Figure~\ref{xing} also shows the similar phenomenon. Conducting CoT reasoning in the model’s advantage language generally leads to better outcomes compared to Direct. Additionally, increasing the number of shots consistently improves performance across most configurations, highlighting the benefits of exposing models to multiple examples. 
To explore the impact of various prompt strategies on our benchmarks, we evaluated several approaches across our datasets and present the average performance of all models using different prompting techniques. Table~\ref{prompt} summarizes the results. For both Chinese and English datasets, Chinese LLMs performed best with the CN-CoT-Few-Shot strategy, followed closely by EN-CoT-Few-Shot, achieving overall scores of 67.36\% and 67.03\%, respectively. Conversely, English LLMs showed optimal performance with the EN-CoT-Few-Shot approach, attaining 67.55\% on the Chinese dataset and 60.36\% on the English dataset.
Besides, translating datasets into the model's native language before reasoning did not enhance performance. This phenomenon is further illustrated in Figure~\ref{xing}. Conducting CoT reasoning in the model's native language generally yields better results compared to direct reasoning. Furthermore, increasing the number of examples (shots) consistently boosts performance across most configurations, emphasizing the advantages of exposing models to multiple examples.
% Overall, the interaction between question language, prompt language, and the number of shots underscores the importance of aligning these factors to optimize task performance and robustness in LLMs.



% Please add the following required packages to your document preamble:
% \usepackage{multirow}
% Please add the following required packages to your document preamble:
% \usepackage{multirow}
\begin{table}[t]
\setlength{\tabcolsep}{8pt}
% \footnotesize
\scalebox{0.65}{
\begin{tabular}{c|l|lll}
\hline
\multicolumn{1}{l|}{Dataset}  & Prompt  & CN LLMs & EN LLMs &  LLMs \\ \hline
\multirow{7}{*}{\begin{tabular}[c]{@{}c@{}}Chinese\\ HellaSwag-Pro\end{tabular}} & Direct  & 48.95& 41.16& 45.06  \\
& CN-CoT-Few  & \textbf{71.04}& 51.90& 61.47  \\
& EN-CoT-Few  & 70.95& \textbf{67.55}& \textbf{69.25}  \\
& EN-XLT-Few  & 41.48& 28.69& 35.09  \\
& CN-CoT-Zero & 44.82& 23.89& 34.36  \\
& EN-CoT-Zero & 45.38& 31.39& 38.39  \\
& EN-XLT-Zero & 28.57& 12.93& 20.75  \\ \hline
\multirow{7}{*}{\begin{tabular}[c]{@{}c@{}}English\\ HellaSwag-Pro\end{tabular}} & Direct  & 47.46& 40.66& 44.06  \\
& CN-CoT-Few  & \textbf{63.67}& 47.24& 55.46  \\
& EN-CoT-Few  & 63.12& \textbf{60.36}& \textbf{61.74}  \\
& CN-XLT-Few  & 48.77& 16.61& 32.69  \\
& CN-CoT-Zero & 34.89& 18.25& 26.57  \\
& EN-CoT-Zero & 42.41& 31.03& 36.72  \\
& CN-XLT-Zero & 16.36& 11.22& 13.79  \\ \hline
\multirow{9}{*}{HellaSwag-Pro}& Direct  & 48.21& 40.91& 44.83  \\
& CN-CoT-Few  & \textbf{67.36}& 49.57& 58.46  \\
& EN-CoT-Few  & 67.03& \textbf{63.95}& \textbf{65.49}  \\
& CN-XLT-Few  & 59.91& 34.26& 47.08  \\
& EN-XLT-Few  & 52.30& 44.52& 48.41  \\
& CN-CoT-Zero & 39.86& 21.07& 30.46  \\
& EN-CoT-Zero & 43.90& 31.21& 37.55  \\
& CN-XLT-Zero & 30.59& 17.55& 24.07  \\
& EN-XLT-Zero & 35.49& 21.98& 28.74  \\ \hline
\end{tabular}
}
\caption{Average ARA of all open-source models on different prompts. CN-LLMs contains 17 LLMs, and EN-LLMs contains 7 LLMs. The bast results for each dataset are \textbf{bolded}.}
\label{prompt}
\end{table}




\section{Texture Generation}\label{sec:texture}
Thanks to the finely detailed and high-quality 3D geometry generated by \method{}, referring to Meta 3D TextureGen~\cite{bensadoun2024meta}, we can leverage the rendered normal maps as input conditions for existing mature multi-view generation methods to produce consistent multi-view texture images. These multi-view texture images are then projected onto the geometric surface to obtain detailed texture maps. 
Fig.\ref{fig:texture_demo_show} shows the 3D result with texture maps generated by \method{}.
\section{Ablation and Analysis}

\begin{table}[t]
\begin{center}
\caption{The ablation for flow model improvements. `Skip-C' is the skip-connection operation and `Sample-S' is the sample schedule.}
\vspace{1em}
\label{tab: diffusion_improvements_abaltion}
\begin{sc}
\resizebox{0.48\textwidth}{!}{
\begin{tabular}{| c | c | c| c|}
\hline
Condition & Skip-C & Sample-S & Normal-FID $\downarrow$\\
% of filter & $e_m$ &   $b_{ij}$ \\
\hline
Dinov2 & \ding{55} & R-Flow  & 10.69\\
\hline
\multirow{4}{*}{CLIP-Dinov2} & \ding{55} & R-Flow  & 10.61\\
\cline{2-4}
 & \ding{51} & DDPM  & 9.63\\
\cline{2-4}
 & \ding{51} & EDM  & 9.50\\
\cline{2-4}
 & \ding{51} & R-Flow  & 9.47 \\
\hline
\end{tabular}
}
\end{sc}
\end{center}
\vspace{-2em}
\end{table}
\subsection{Ablation for Flow Model}


To validate the effectiveness of the proposed flow model improvements and scaling-up strategies, we performed specific ablation experiments and comparative analyses for each improvement.
Using a further filtered $180K$ high-quality Objaverse dataset, we conducted ablation experiments following the training settings in \ref{sec:Implementation_detail}, and evaluated the results using the Normal-FID metric introduced in \ref{sec:metric}.

For the evaluation of Normal-FID, we selected $1K$ data samples from the $180K$ dataset as a dedicated test set for 3D generation performance validation, with the remaining data samples used for training. For the test set, we rendered each 3D ground-truth model's front view paired RGB and normal map using a $50mm$ camera focal length and a $10^\circ$ elevation (the test set rendering settings are included within the training set settings). The RGB images are used to generate 3D shapes, and the normal maps are compared with the normal maps rendered from the generated 3D shapes to calculate the Normal-FID.

Our flow model ablation experiment consists of two parts: flow model improvement training and flow model scaling up. For the flow model improvement experiments, as shown in Tab.\ref{tab: diffusion_improvements_abaltion}, we used a $975M$ parameter model with a latent resolution of $512$ tokens and trained for $300K$ steps for each experiment on the high-quality Objaverse dataset. We conducted comparative analyses on $Condition$, $Skip-connection$, and $Sampling-schedule$ improvements.
From the last three rows in Tab.\ref{tab: diffusion_improvements_abaltion}, we observe that R-Flow sampling yields better generation results compared to EDM and DDPM. Combined with its training efficiency, R-Flow demonstrates clear advantages in 3D generation tasks. Comparing rows 2 and 5 shows that the skip-connection operation significantly affects generation results, with the fusion of deep and shallow features improving flow modeling. Additionally, the comparison between the first two rows indicates that the CLIP condition also slightly improves generation results. 
From the overall quantitative results, the skip-connection operation has the most obvious effect among these ablations.

\begin{table}[t]
\begin{center}
\caption{The ablation for flow model scaling up.}
\vspace{1em}
\label{tab: diffusion_scaling_abaltion}
% \begin{small}
\begin{sc}
\resizebox{0.48\textwidth}{!}{
\begin{tabular}{| c| c|c|c|}
\hline
Dataset  & Token number & MoE &Normal-FID $\downarrow$\\
% of filter & $e_m$ &   $b_{ij}$ \\
\hline
\multirow{4}{*}{Objaverse}  &512 & \ding{55}& 9.47 \\
\cline{2-4}
  &2048 & \ding{55}& 8.38\\
\cline{2-4}
  &4096 & \ding{55}& 8.12\\
\cline{2-4}
  &4096 & \ding{51} & 7.94\\
\hline
\method{}  &4096 & \ding{51} & 3.36\\
\hline
\end{tabular}
}
\end{sc}
% \end{small}
\end{center}
\vspace{-2em}
\end{table}

For the flow model scaling-up experiments, as shown in rows 2-4 of Tab.\ref{tab: diffusion_scaling_abaltion}, we used a $975M$ parameter model with CLIP-DINOv2 dual-conditioning, skip-connection operations, and rectified-flow sampling schedule. These models are trained for a total of $300K$ steps on the high-quality Objaverse data to conduct comparative analyses on latent resolution and MoE. The last row of Tab.\ref{tab: diffusion_scaling_abaltion} represents our largest \method{} model, encompassing the largest data, model size, resolution, and training cost.
From the first three rows of Tab.\ref{tab: diffusion_scaling_abaltion} we observe that as the latent resolution increases, the generated results consistently improve, with the most significant improvement occurring from 512 to 2048 tokens. Comparing rows 3 and 4 shows the gains from increasing model parameters through MoE. 
Comparing rows 4 and 5 demonstrates the performance improvement achieved by increasing high-quality data size. When combined with the results from row 1, we can see that the improvement from the increased high-quality data size surpasses that from higher resolution.
Overall, the large-scale dataset, large model size, and high resolution contribute to significant performance improvements, allowing \method{} to achieve remarkable 3D generation results.

% The comparison between the last row and the others highlights the significant impact of large-scale dataset, large model size, and high resolution. With these improvements, \method{} achieves impressive 3D generation results.

\subsection{Ablation for VAE}

\begin{table}[t]
\begin{center}
\caption{The ablation of different VAE, including 3D representation, training supervision and training dataset. `Repr' refers to the type of 3D representation used, and `$\mathcal{L}_\text{sn}$' and `$\mathcal{L}_\text{eik}$' refer to surface normal loss and eikonal regularization respectively. The `Dataset' indicates whether a large (TripoSG) or a small dataset (Objaverse) is used. }
\vspace{1em}
\label{tab:vae}
\begin{sc}
\resizebox{0.48\textwidth}{!}{
\begin{tabular}{| c | c c c | c c c |}
\hline
Dataset & Repr. & $\mathcal{L}_\text{sn}$ & $\mathcal{L}_{eik}$ & Chamfer $\downarrow$ & F-score $\uparrow$ & N.C. $\uparrow$ \\
\hline
\multirow{4}{*}{Objaverse} & Occ & \ding{55} & \ding{55} & 4.59 & 0.999 & 0.952 \\
\cline{2-7}
 & SDF & \ding{55} & \ding{55} & 4.60 & 0.999 & 0.955 \\ 
\cline{2-7}
 & SDF & \ding{51} & \ding{55} & 4.56 & 0.999 & 0.956 \\
\cline{2-7} 
 & SDF & \ding{51} & \ding{51} & 4.57 & 0.999 & 0.957 \\
\hline 
\method{}& SDF & \ding{51} & \ding{51} & 4.51 & 0.999 & 0.958 \\
\hline 
\end{tabular}
}
\end{sc}
\end{center}
\end{table}

\begin{figure}[t]
    \centering
    \includegraphics[width=1\linewidth]{Images/vae-comparison.pdf}
    \caption{Qualitative comparison for the ablation of VAE with different types of 3D representation and training supervision.}
    \label{fig:vae-comparison}
    \vspace{-1em}
\end{figure}

To evaluate the effectiveness of neural SDF implicit representation with surface normal guidance, we experiment with different VAE model settings, including the formulation of neural implicit representation, training supervision, and training dataset.
Tab.\ref{tab:vae} demonstrates the qualitative results of VAE reconstruction quality with different training settings.
We can observe that the SDF representation, combined with surface normal guidance and eikonal regularization, improves the reconstruction quality and geometry details, achieving lower Chamfer distance and higher normal consistency compared to occupancy-based results.
As the amount of training data increases (as demonstrated by \method{}), the reconstruction quality of the VAE further improves.
Fig.\ref{fig:vae-comparison} provides qualitative comparisons between them.
Occupancy-based reconstruction results suffer from aliasing artifacts (highlighted by the blue box), thin structures, and floaters (highlighted by the red boxes).
While SDF representation avoids aliasing artifacts, there remains a gap in achieving high-quality reconstruction, particularly for thin-shell structures where performance may worsen. 
Incorporating surface normal guidance can result in sharper reconstructions with finer details. However, over-emphasizing surface normal guidance during training introduces slight aliasing artifacts (as seen in the first row of Fig.\ref{fig:vae-comparison}), which can be mitigated by introducing eikonal regularization.



\begin{table}[t]
\begin{center}
\caption{The ablation for data quality and quantity.}
\vspace{1em}
\label{tab: data_abaltion}
% \begin{small}
\begin{sc}
\resizebox{0.48\textwidth}{!}{
\begin{tabular}{| c| c|c|c|}
\hline
Dataset  & Size & Data-building System &Normal-FID $\downarrow$\\
% of filter & $e_m$ &   $b_{ij}$ \\
\hline
\multirow{2}{*}{Objaverse}  &$800K$ & \ding{55}& 11.61\\
\cline{2-4}
  &$180K$ & \ding{51}& 9.47 \\
\hline
\method{}  &$2M$ & \ding{51} & 5.81\\
\hline
\end{tabular}
}
\end{sc}
% \end{small}
\end{center}
\vspace{-2em}
\end{table}
\subsection{Ablation for Data-Building System}
To demonstrate the effectiveness of the data-building system proposed by \method{}, we implement ablation experiments on both data quality and quantity. Using the optimal R-Flow training settings (first row of Tab.\ref{tab: diffusion_improvements_abaltion}), we replaced the $180K$ Objaverse dataset produced by \method{} with the original $800K$ Objaverse dataset, which had not undergone scoring, filtering, orientation fixing, untextured model processing, or internal processing of the converted watertight model. This experiment demonstrates the effect of data quality. Similarly, under the same R-Flow settings, we expanded the high-quality dataset from $180K$ Objaverse to $2M$ \method{} to evaluate the effect of data quantity.

As shown in the first two rows of Tab.\ref{tab: data_abaltion}, although our data-building system reduced the $800K$ Objaverse dataset to $180K$, the improved data quality resulted in better generation results, demonstrating that, when training with in-the-wild data, quality outweighs quantity. Furthermore, as seen in last two rows of Tab.\ref{tab: data_abaltion}, increasing the high-quality dataset from $180K$ to $2M$ led to a significant boost in generation performance, showing that with high-quality data, scaling up data size is crucial for achieving better results. 
Additionally, the overall quantitative results in the Tab.\ref{tab: data_abaltion} show that the performance improvement gained from $2M$ high-quality data size is greater than that from improving data quality alone. Furthermore, after enhancing data quality, performance continues to improve with an increase in data size, without encountering a bottleneck at the current training scale.


\subsection{The Visualization for Flow Model Ablation}
\begin{figure*}
    \centering
    \includegraphics[width=\linewidth]{Images/ablation_vis.pdf}
    \vspace{-1em}
    \caption{Visualization results of the flow model ablation experiments. `DBS' is the abbreviation for Data-Building System.
    }
    % \caption{Caption: \textcolor{red}{vis corresponding, 180k-4096 only for ablation. quality data scaling better than resolution}}
    \vspace{-1em}
    \label{fig:ablation_vis}
    % \vspace{-1em}
\end{figure*}

In addition to the quantitative results, we also performed a visualization analysis of the core experiments, as shown in Fig.\ref{fig:ablation_vis}. The rows 1, 2, 4 correspond to the three experimental results in Tab.\ref{tab: data_abaltion}, while the row 3 corresponds to the row 4 of results in Tab.\ref{tab: diffusion_scaling_abaltion}. The visualization reveals several insights consistent with the quantitative results: (1) Data quality is more important than the size of raw in-the-wild data (row 1 vs. row 2). (2) The improvement from increasing high-quality data size is more obvious than the improvement from resolution (row 2 vs. row 3 vs. row 4). (3) Increasing the size of high-quality data ($2M$) provides a greater boost to performance than merely improving data quality. After improving data quality, performance continues to improve with increased data size without encountering bottlenecks at the current training scale.







\section{Summary and Conclusion}
\label{sec:conclusion}


In this paper, we introduced \ToolName{}, a method for discovering fine-grained \emph{sub-activities} from unlabeled smart home sensor data without relying on pre-segmentation. Our pipeline is organized into two core steps: Clustering and Labeling. 
The \textbf{Clustering step} consists of:

\begin{itemize}
    \item \textbf{Encoder Pre-Training:} We leverage a pre-trained BERT model adapted with sensor-specific tokens and train it using a masked language modeling (MLM) objective to generate context-rich embeddings for raw sensor sequences.
    
    \item \textbf{Clustering Model Fine-Tuning:} Using the SCAN loss function, we fine-tune these embeddings to form more homogeneous and distinct clusters of sensor sequences.
\end{itemize}

The \textbf{Labeling step} comprises:

\begin{itemize}
    \item \textbf{Cluster Centroid Annotation:} Representative sequences from each cluster are visualized with a custom tool, enabling expert annotators to assign meaningful sub-activity labels to the centroids.
    
    \item \textbf{Label Propagation:} The centroid labels are propagated to all sequences within their respective clusters, resulting in a fully labeled dataset with minimal manual effort.
    
    \item \textbf{Re-annotation of Original Time-Series Data:} 
    Finally, these propagated labels are mapped back onto the original time-series data, preserving temporal continuity and facilitating the analysis of longitudinal activity patterns.
\end{itemize}


Our approach addresses important challenges in HAR, including the high cost and effort of manual data annotation, the limitations of coarse activity labels, and the need for scalable and generalizable models. \ToolName{} offers an open source tool that facilitates the HAR annotation and re-annotation process and enables the dynamic discovery and validation of sub-activities, thus capturing a broader spectrum of behaviors observed in real homes.


\bibliography{main}
\bibliographystyle{icml2025}


%%%%%%%%%%%%%%%%%%%%%%%%%%%%%%%%%%%%%%%%%%%%%%%%%%%%%%%%%%%%%%%%%%%%%%%%%%%%%%%
%%%%%%%%%%%%%%%%%%%%%%%%%%%%%%%%%%%%%%%%%%%%%%%%%%%%%%%%%%%%%%%%%%%%%%%%%%%%%%%
% APPENDIX
%%%%%%%%%%%%%%%%%%%%%%%%%%%%%%%%%%%%%%%%%%%%%%%%%%%%%%%%%%%%%%%%%%%%%%%%%%%%%%%
%%%%%%%%%%%%%%%%%%%%%%%%%%%%%%%%%%%%%%%%%%%%%%%%%%%%%%%%%%%%%%%%%%%%%%%%%%%%%%%
% \newpage
% \appendix
% \onecolumn
% \section{You \emph{can} have an appendix here.}

% You can have as much text here as you want. The main body must be at most $8$ pages long.
% For the final version, one more page can be added.
% If you want, you can use an appendix like this one.  

% The $\mathtt{\backslash onecolumn}$ command above can be kept in place if you prefer a one-column appendix, or can be removed if you prefer a two-column appendix.  Apart from this possible change, the style (font size, spacing, margins, page numbering, etc.) should be kept the same as the main body.
% %%%%%%%%%%%%%%%%%%%%%%%%%%%%%%%%%%%%%%%%%%%%%%%%%%%%%%%%%%%%%%%%%%%%%%%%%%%%%%%
% %%%%%%%%%%%%%%%%%%%%%%%%%%%%%%%%%%%%%%%%%%%%%%%%%%%%%%%%%%%%%%%%%%%%%%%%%%%%%%%


\end{document}


% This document was modified from the file originally made available by
% Pat Langley and Andrea Danyluk for ICML-2K. This version was created
% by Iain Murray in 2018, and modified by Alexandre Bouchard in
% 2019 and 2021 and by Csaba Szepesvari, Gang Niu and Sivan Sabato in 2022.
% Modified again in 2023 and 2024 by Sivan Sabato and Jonathan Scarlett.
% Previous contributors include Dan Roy, Lise Getoor and Tobias
% Scheffer, which was slightly modified from the 2010 version by
% Thorsten Joachims & Johannes Fuernkranz, slightly modified from the
% 2009 version by Kiri Wagstaff and Sam Roweis's 2008 version, which is
% slightly modified from Prasad Tadepalli's 2007 version which is a
% lightly changed version of the previous year's version by Andrew
% Moore, which was in turn edited from those of Kristian Kersting and
% Codrina Lauth. Alex Smola contributed to the algorithmic style files.
