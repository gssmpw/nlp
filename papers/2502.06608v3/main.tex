%%%%%%%% ICML 2024 EXAMPLE LATEX SUBMISSION FILE %%%%%%%%%%%%%%%%%

\documentclass{article}
\usepackage{graphicx}

% Recommended, but optional, packages for figures and better typesetting:
% \usepackage{subfigure}
\usepackage{algorithmic}
\usepackage{algorithm}
\usepackage[caption=false,font=normalsize,labelfont=sf,textfont=sf]{subfig}
\usepackage{textcomp}
\usepackage{stfloats}
\usepackage{url}
\usepackage{verbatim}
\usepackage{cite}
\hyphenation{op-tical net-works semi-conduc-tor IEEE-Xplore}
% updated with editorial comments 8/9/2021
\usepackage{cuted}
\usepackage{caption}
\usepackage{pifont}
\usepackage{array} 
\usepackage{subcaption} 

\usepackage{multirow}

% Recommended, but optional, packages for figures and better typesetting:
\usepackage{microtype}
\usepackage{booktabs} % for professional tables

% hyperref makes hyperlinks in the resulting PDF.
% If your build breaks (sometimes temporarily if a hyperlink spans a page)
% please comment out the following usepackage line and replace
% \usepackage{icml2025} with \usepackage[nohyperref]{icml2025} above.
\usepackage{hyperref}




% Attempt to make hyperref and algorithmic work together better:


% hyperref makes hyperlinks in the resulting PDF.
% If your build breaks (sometimes temporarily if a hyperlink spans a page)
% please comment out the following usepackage line and replace
% \usepackage{icml2024} with \usepackage[nohyperref]{icml2024} above.


% Attempt to make hyperref and algorithmic work together better:

% Use the following line for the initial blind version submitted for review:
% \usepackage{icml2024}

% If accepted, instead use the following line for the camera-ready submission:
\usepackage[accepted]{icml2025}

% For theorems and such
\usepackage{amsmath}
\usepackage{amssymb}
\usepackage{mathtools}
\usepackage{amsthm}

\newcommand*{\method}{TripoSG}


% Todonotes is useful during development; simply uncomment the next line
%    and comment out the line below the next line to turn off comments
%\usepackage[disable,textsize=tiny]{todonotes}
\usepackage[textsize=tiny]{todonotes}


% The \icmltitle you define below is probably too long as a header.
% Therefore, a short form for the running title is supplied here:
\icmltitlerunning{\method{}: High-Fidelity 3D Shape Synthesis using Large-Scale Rectified Flow Models}

\begin{document}



\twocolumn[
\icmltitle{\method{}: High-Fidelity 3D Shape Synthesis using Large-Scale Rectified Flow Models}

\icmlsetsymbol{equal}{*}
\icmlsetsymbol{corre}{$\dagger$}


\begin{icmlauthorlist}
\icmlauthor{Yangguang Li$^1$}{equal}
\icmlauthor{Zi-Xin Zou$^1$}{equal}
\icmlauthor{Zexiang Liu$^1$}{}
\icmlauthor{Dehu Wang$^1$}{}
\icmlauthor{Yuan Liang$^1$}{}
\icmlauthor{Zhipeng Yu$^1$}{}
\icmlauthor{Xingchao Liu$^3$}{}
\icmlauthor{Yuan-Chen Guo$^1$}{}
\icmlauthor{Ding Liang$^1$}{corre}
\icmlauthor{Wanli Ouyang$^{2,4}$}{}
\icmlauthor{Yan-Pei Cao$^1$}{}\\
~\\
Project Homepage: \href{https://yg256li.github.io/TripoSG-Page/}{\textcolor{red}{TripoSG-Page}}
\end{icmlauthorlist}

\icmlkeywords{Machine Learning, ICML}

\vskip 0.3in
]




% this must go after the closing bracket ] following \twocolumn[ ...

% This command actually creates the footnote in the first column
% listing the affiliations and the copyright notice.
% The command takes one argument, which is text to display at the start of the footnote.
% The \icmlEqualContribution command is standard text for equal contribution.
% Remove it (just {}) if you do not need this facility.

% \printAffiliationsAndNotice{}  % leave blank if no need to mention equal contribution

\printAffiliationsAndNotice
{
~\\
$^1$VAST \\
$^2$The Chinese University of Hong Kong \\
$^3$The University of Texas at Austin \\
$^4$Shanghai AI Laboratory \\
$^*$: Equal Contribution \\
$^\dagger$: Corresponding Author\\
}


\begin{strip}
    \centering
    \vspace{-9em}
    \includegraphics[width=1\textwidth]{Images/teaser_resting_4k_shapegen_w_bg.png}
    % \vspace{-1.5em}
    \captionof{figure}{High-quality 3D shape samples from our largest \method{} model. Covering various complex structures, diverse styles, imaginative designs, multi-object compositions, and richly detailed outputs, demonstrates its powerful generation capabilities.}
    \vspace{2em}
    \label{fig:teaser}
\end{strip}

\begin{abstract}


The choice of representation for geographic location significantly impacts the accuracy of models for a broad range of geospatial tasks, including fine-grained species classification, population density estimation, and biome classification. Recent works like SatCLIP and GeoCLIP learn such representations by contrastively aligning geolocation with co-located images. While these methods work exceptionally well, in this paper, we posit that the current training strategies fail to fully capture the important visual features. We provide an information theoretic perspective on why the resulting embeddings from these methods discard crucial visual information that is important for many downstream tasks. To solve this problem, we propose a novel retrieval-augmented strategy called RANGE. We build our method on the intuition that the visual features of a location can be estimated by combining the visual features from multiple similar-looking locations. We evaluate our method across a wide variety of tasks. Our results show that RANGE outperforms the existing state-of-the-art models with significant margins in most tasks. We show gains of up to 13.1\% on classification tasks and 0.145 $R^2$ on regression tasks. All our code and models will be made available at: \href{https://github.com/mvrl/RANGE}{https://github.com/mvrl/RANGE}.

\end{abstract}


\section{Introduction}

Video generation has garnered significant attention owing to its transformative potential across a wide range of applications, such media content creation~\citep{polyak2024movie}, advertising~\citep{zhang2024virbo,bacher2021advert}, video games~\citep{yang2024playable,valevski2024diffusion, oasis2024}, and world model simulators~\citep{ha2018world, videoworldsimulators2024, agarwal2025cosmos}. Benefiting from advanced generative algorithms~\citep{goodfellow2014generative, ho2020denoising, liu2023flow, lipman2023flow}, scalable model architectures~\citep{vaswani2017attention, peebles2023scalable}, vast amounts of internet-sourced data~\citep{chen2024panda, nan2024openvid, ju2024miradata}, and ongoing expansion of computing capabilities~\citep{nvidia2022h100, nvidia2023dgxgh200, nvidia2024h200nvl}, remarkable advancements have been achieved in the field of video generation~\citep{ho2022video, ho2022imagen, singer2023makeavideo, blattmann2023align, videoworldsimulators2024, kuaishou2024klingai, yang2024cogvideox, jin2024pyramidal, polyak2024movie, kong2024hunyuanvideo, ji2024prompt}.


In this work, we present \textbf{\ours}, a family of rectified flow~\citep{lipman2023flow, liu2023flow} transformer models designed for joint image and video generation, establishing a pathway toward industry-grade performance. This report centers on four key components: data curation, model architecture design, flow formulation, and training infrastructure optimization—each rigorously refined to meet the demands of high-quality, large-scale video generation.


\begin{figure}[ht]
    \centering
    \begin{subfigure}[b]{0.82\linewidth}
        \centering
        \includegraphics[width=\linewidth]{figures/t2i_1024.pdf}
        \caption{Text-to-Image Samples}\label{fig:main-demo-t2i}
    \end{subfigure}
    \vfill
    \begin{subfigure}[b]{0.82\linewidth}
        \centering
        \includegraphics[width=\linewidth]{figures/t2v_samples.pdf}
        \caption{Text-to-Video Samples}\label{fig:main-demo-t2v}
    \end{subfigure}
\caption{\textbf{Generated samples from \ours.} Key components are highlighted in \textcolor{red}{\textbf{RED}}.}\label{fig:main-demo}
\end{figure}


First, we present a comprehensive data processing pipeline designed to construct large-scale, high-quality image and video-text datasets. The pipeline integrates multiple advanced techniques, including video and image filtering based on aesthetic scores, OCR-driven content analysis, and subjective evaluations, to ensure exceptional visual and contextual quality. Furthermore, we employ multimodal large language models~(MLLMs)~\citep{yuan2025tarsier2} to generate dense and contextually aligned captions, which are subsequently refined using an additional large language model~(LLM)~\citep{yang2024qwen2} to enhance their accuracy, fluency, and descriptive richness. As a result, we have curated a robust training dataset comprising approximately 36M video-text pairs and 160M image-text pairs, which are proven sufficient for training industry-level generative models.

Secondly, we take a pioneering step by applying rectified flow formulation~\citep{lipman2023flow} for joint image and video generation, implemented through the \ours model family, which comprises Transformer architectures with 2B and 8B parameters. At its core, the \ours framework employs a 3D joint image-video variational autoencoder (VAE) to compress image and video inputs into a shared latent space, facilitating unified representation. This shared latent space is coupled with a full-attention~\citep{vaswani2017attention} mechanism, enabling seamless joint training of image and video. This architecture delivers high-quality, coherent outputs across both images and videos, establishing a unified framework for visual generation tasks.


Furthermore, to support the training of \ours at scale, we have developed a robust infrastructure tailored for large-scale model training. Our approach incorporates advanced parallelism strategies~\citep{jacobs2023deepspeed, pytorch_fsdp} to manage memory efficiently during long-context training. Additionally, we employ ByteCheckpoint~\citep{wan2024bytecheckpoint} for high-performance checkpointing and integrate fault-tolerant mechanisms from MegaScale~\citep{jiang2024megascale} to ensure stability and scalability across large GPU clusters. These optimizations enable \ours to handle the computational and data challenges of generative modeling with exceptional efficiency and reliability.


We evaluate \ours on both text-to-image and text-to-video benchmarks to highlight its competitive advantages. For text-to-image generation, \ours-T2I demonstrates strong performance across multiple benchmarks, including T2I-CompBench~\citep{huang2023t2i-compbench}, GenEval~\citep{ghosh2024geneval}, and DPG-Bench~\citep{hu2024ella_dbgbench}, excelling in both visual quality and text-image alignment. In text-to-video benchmarks, \ours-T2V achieves state-of-the-art performance on the UCF-101~\citep{ucf101} zero-shot generation task. Additionally, \ours-T2V attains an impressive score of \textbf{84.85} on VBench~\citep{huang2024vbench}, securing the top position on the leaderboard (as of 2025-01-25) and surpassing several leading commercial text-to-video models. Qualitative results, illustrated in \Cref{fig:main-demo}, further demonstrate the superior quality of the generated media samples. These findings underscore \ours's effectiveness in multi-modal generation and its potential as a high-performing solution for both research and commercial applications.
\section{Related Work}

\subsection{Large 3D Reconstruction Models}
Recently, generalized feed-forward models for 3D reconstruction from sparse input views have garnered considerable attention due to their applicability in heavily under-constrained scenarios. The Large Reconstruction Model (LRM)~\cite{hong2023lrm} uses a transformer-based encoder-decoder pipeline to infer a NeRF reconstruction from just a single image. Newer iterations have shifted the focus towards generating 3D Gaussian representations from four input images~\cite{tang2025lgm, xu2024grm, zhang2025gslrm, charatan2024pixelsplat, chen2025mvsplat, liu2025mvsgaussian}, showing remarkable novel view synthesis results. The paradigm of transformer-based sparse 3D reconstruction has also successfully been applied to lifting monocular videos to 4D~\cite{ren2024l4gm}. \\
Yet, none of the existing works in the domain have studied the use-case of inferring \textit{animatable} 3D representations from sparse input images, which is the focus of our work. To this end, we build on top of the Large Gaussian Reconstruction Model (GRM)~\cite{xu2024grm}.

\subsection{3D-aware Portrait Animation}
A different line of work focuses on animating portraits in a 3D-aware manner.
MegaPortraits~\cite{drobyshev2022megaportraits} builds a 3D Volume given a source and driving image, and renders the animated source actor via orthographic projection with subsequent 2D neural rendering.
3D morphable models (3DMMs)~\cite{blanz19993dmm} are extensively used to obtain more interpretable control over the portrait animation. For example, StyleRig~\cite{tewari2020stylerig} demonstrates how a 3DMM can be used to control the data generated from a pre-trained StyleGAN~\cite{karras2019stylegan} network. ROME~\cite{khakhulin2022rome} predicts vertex offsets and texture of a FLAME~\cite{li2017flame} mesh from the input image.
A TriPlane representation is inferred and animated via FLAME~\cite{li2017flame} in multiple methods like Portrait4D~\cite{deng2024portrait4d}, Portrait4D-v2~\cite{deng2024portrait4dv2}, and GPAvatar~\cite{chu2024gpavatar}.
Others, such as VOODOO 3D~\cite{tran2024voodoo3d} and VOODOO XP~\cite{tran2024voodooxp}, learn their own expression encoder to drive the source person in a more detailed manner. \\
All of the aforementioned methods require nothing more than a single image of a person to animate it. This allows them to train on large monocular video datasets to infer a very generic motion prior that even translates to paintings or cartoon characters. However, due to their task formulation, these methods mostly focus on image synthesis from a frontal camera, often trading 3D consistency for better image quality by using 2D screen-space neural renderers. In contrast, our work aims to produce a truthful and complete 3D avatar representation from the input images that can be viewed from any angle.  

\subsection{Photo-realistic 3D Face Models}
The increasing availability of large-scale multi-view face datasets~\cite{kirschstein2023nersemble, ava256, pan2024renderme360, yang2020facescape} has enabled building photo-realistic 3D face models that learn a detailed prior over both geometry and appearance of human faces. HeadNeRF~\cite{hong2022headnerf} conditions a Neural Radiance Field (NeRF)~\cite{mildenhall2021nerf} on identity, expression, albedo, and illumination codes. VRMM~\cite{yang2024vrmm} builds a high-quality and relightable 3D face model using volumetric primitives~\cite{lombardi2021mvp}. One2Avatar~\cite{yu2024one2avatar} extends a 3DMM by anchoring a radiance field to its surface. More recently, GPHM~\cite{xu2025gphm} and HeadGAP~\cite{zheng2024headgap} have adopted 3D Gaussians to build a photo-realistic 3D face model. \\
Photo-realistic 3D face models learn a powerful prior over human facial appearance and geometry, which can be fitted to a single or multiple images of a person, effectively inferring a 3D head avatar. However, the fitting procedure itself is non-trivial and often requires expensive test-time optimization, impeding casual use-cases on consumer-grade devices. While this limitation may be circumvented by learning a generalized encoder that maps images into the 3D face model's latent space, another fundamental limitation remains. Even with more multi-view face datasets being published, the number of available training subjects rarely exceeds the thousands, making it hard to truly learn the full distibution of human facial appearance. Instead, our approach avoids generalizing over the identity axis by conditioning on some images of a person, and only generalizes over the expression axis for which plenty of data is available. 

A similar motivation has inspired recent work on codec avatars where a generalized network infers an animatable 3D representation given a registered mesh of a person~\cite{cao2022authentic, li2024uravatar}.
The resulting avatars exhibit excellent quality at the cost of several minutes of video capture per subject and expensive test-time optimization.
For example, URAvatar~\cite{li2024uravatar} finetunes their network on the given video recording for 3 hours on 8 A100 GPUs, making inference on consumer-grade devices impossible. In contrast, our approach directly regresses the final 3D head avatar from just four input images without the need for expensive test-time fine-tuning.


Effective human-robot cooperation in CoNav-Maze hinges on efficient communication. Maximizing the human’s information gain enables more precise guidance, which in turn accelerates task completion. Yet for the robot, the challenge is not only \emph{what} to communicate but also \emph{when}, as it must balance gathering information for the human with pursuing immediate goals when confident in its navigation.

To achieve this, we introduce \emph{Information Gain Monte Carlo Tree Search} (IG-MCTS), which optimizes both task-relevant objectives and the transmission of the most informative communication. IG-MCTS comprises three key components:
\textbf{(1)} A data-driven human perception model that tracks how implicit (movement) and explicit (image) information updates the human’s understanding of the maze layout.
\textbf{(2)} Reward augmentation to integrate multiple objectives effectively leveraging on the learned perception model.
\textbf{(3)} An uncertainty-aware MCTS that accounts for unobserved maze regions and human perception stochasticity.
% \begin{enumerate}[leftmargin=*]
%     \item A data-driven human perception model that tracks how implicit (movement) and explicit (image transmission) information updates the human’s understanding of the maze layout.
%     \item Reward augmentation to integrate multiple objectives effectively leveraging on the learned perception model.
%     \item An uncertainty-aware MCTS that accounts for unobserved maze regions and human perception stochasticity.
% \end{enumerate}

\subsection{Human Perception Dynamics}
% IG-MCTS seeks to optimize the expected novel information gained by the human through the robot’s actions, including both movement and communication. Achieving this requires a model of how the human acquires task-relevant information from the robot.

% \subsubsection{Perception MDP}
\label{sec:perception_mdp}
As the robot navigates the maze and transmits images, humans update their understanding of the environment. Based on the robot's path, they may infer that previously assumed blocked locations are traversable or detect discrepancies between the transmitted image and their map.  

To formally capture this process, we model the evolution of human perception as another Markov Decision Process, referred to as the \emph{Perception MDP}. The state space $\mathcal{X}$ represents all possible maze maps. The action space $\mathcal{S}^+ \times \mathcal{O}$ consists of the robot's trajectory between two image transmissions $\tau \in \mathcal{S}^+$ and an image $o \in \mathcal{O}$. The unknown transition function $F: (x, (\tau, o)) \rightarrow x'$ defines the human perception dynamics, which we aim to learn.

\subsubsection{Crowd-Sourced Transition Dataset}
To collect data, we designed a mapping task in the CoNav-Maze environment. Participants were tasked to edit their maps to match the true environment. A button triggers the robot's autonomous movements, after which it captures an image from a random angle.
In this mapping task, the robot, aware of both the true environment and the human’s map, visits predefined target locations and prioritizes areas with mislabeled grid cells on the human’s map.
% We assume that the robot has full knowledge of both the actual environment and the human’s current map. Leveraging this knowledge, the robot autonomously navigates to all predefined target locations. It then randomly selects subsequent goals to reach, prioritizing grid locations that remain mislabeled on the human’s map. This ensures that the robot’s actions are strategically focused on providing useful information to improve map accuracy.

We then recruited over $50$ annotators through Prolific~\cite{palan2018prolific} for the mapping task. Each annotator labeled three randomly generated mazes. They were allowed to proceed to the next maze once the robot had reached all four goal locations. However, they could spend additional time refining their map before moving on. To incentivize accuracy, annotators receive a performance-based bonus based on the final accuracy of their annotated map.


\subsubsection{Fully-Convolutional Dynamics Model}
\label{sec:nhpm}

We propose a Neural Human Perception Model (NHPM), a fully convolutional neural network (FCNN), to predict the human perception transition probabilities modeled in \Cref{sec:perception_mdp}. We denote the model as $F_\theta$ where $\theta$ represents the trainable weights. Such design echoes recent studies of model-based reinforcement learning~\cite{hansen2022temporal}, where the agent first learns the environment dynamics, potentially from image observations~\cite{hafner2019learning,watter2015embed}.

\begin{figure}[t]
    \centering
    \includegraphics[width=0.9\linewidth]{figures/ICML_25_CNN.pdf}
    \caption{Neural Human Perception Model (NHPM). \textbf{Left:} The human's current perception, the robot's trajectory since the last transmission, and the captured environment grids are individually processed into 2D masks. \textbf{Right:} A fully convolutional neural network predicts two masks: one for the probability of the human adding a wall to their map and another for removing a wall.}
    \label{fig:nhpm}
    \vskip -0.1in
\end{figure}

As illustrated in \Cref{fig:nhpm}, our model takes as input the human’s current perception, the robot’s path, and the image captured by the robot, all of which are transformed into a unified 2D representation. These inputs are concatenated along the channel dimension and fed into the CNN, which outputs a two-channel image: one predicting the probability of human adding a new wall and the other predicting the probability of removing a wall.

% Our approach builds on world model learning, where neural networks predict state transitions or environmental updates based on agent actions and observations. By leveraging the local feature extraction capabilities of CNNs, our model effectively captures spatial relationships and interprets local changes within the grid maze environment. Similar to prior work in localization and mapping, the CNN architecture is well-suited for processing spatially structured data and aligning the robot’s observations with human map updates.

To enhance robustness and generalization, we apply data augmentation techniques, including random rotation and flipping of the 2D inputs during training. These transformations are particularly beneficial in the grid maze environment, which is invariant to orientation changes.

\subsection{Perception-Aware Reward Augmentation}
The robot optimizes its actions over a planning horizon \( H \) by solving the following optimization problem:
\begin{subequations}
    \begin{align}
        \max_{a_{0:H-1}} \;
        & \mathop{\mathbb{E}}_{T, F} \left[ \sum_{t=0}^{H-1} \gamma^t \left(\underbrace{R_{\mathrm{task}}(\tau_{t+1}, \zeta)}_{\text{(1) Task reward}} + \underbrace{\|x_{t+1}-x_t\|_1}_{\text{(2) Info reward}}\right)\right] \label{obj}\\ 
        \subjectto \quad
        &x_{t+1} = F(x_t, (\tau_t, a_t)), \quad a_t\in\Ocal \label{const:perception_update}\\ 
        &\tau_{t+1} = \tau_t \oplus T(s_t, a_t), \quad a_t\in \Ucal\label{const:history_update}
    \end{align}
\end{subequations} 

The objective in~\eqref{obj} maximizes the expected cumulative reward over \( T \) and \( F \), reflecting the uncertainty in both physical transitions and human perception dynamics. The reward function consists of two components: 
(1) The \emph{task reward} incentivizes efficient navigation. The specific formulation for the task in this work is outlined in \Cref{appendix:task_reward}.
(2) The \emph{information reward} quantifies the change in the human’s perception due to robot actions, computed as the \( L_1 \)-norm distance between consecutive perception states.  

The constraint in~\eqref{const:history_update} ensures that for movement actions, the trajectory history \( \tau_t \) expands with new states based on the robot’s chosen actions, where \( s_t \) is the most recent state in \( \tau_t \), and \( \oplus \) represents sequence concatenation. 
In constraint~\eqref{const:perception_update}, the robot leverages the learned human perception dynamics \( F \) to estimate the evolution of the human’s understanding of the environment from perception state $x_t$ to $x_{t+1}$ based on the observed trajectory \( \tau_t \) and transmitted image \( a_t\in\Ocal \). 
% justify from a cognitive science perspective
% Cognitive science research has shown that humans read in a way to maximize the information gained from each word, aligning with the efficient coding principle, which prioritizes minimizing perceptual errors and extracting relevant features under limited processing capacity~\cite{kangassalo2020information}. Drawing on this principle, we hypothesize that humans similarly prioritize task-relevant information in multimodal settings. To accommodate this cognitive pattern, our robot policy selects and communicates high information-gain observations to human operators, akin to summarizing key insights from a lengthy article.
% % While the brain naturally seeks to gain information, the brain employs various strategies to manage information overload, including filtering~\cite{quiroga2004reducing}, limiting/working memory, and prioritizing information~\cite{arnold2023dealing}.
% In this context of our setup, we optimize the selection of camera angles to maximize the human operator's information gain about the environment. 

\subsection{Information Gain Monte Carlo Tree Search (IG-MCTS)}
IG-MCTS follows the four stages of Monte Carlo tree search: \emph{selection}, \emph{expansion}, \emph{rollout}, and \emph{backpropagation}, but extends it by incorporating uncertainty in both environment dynamics and human perception. We introduce uncertainty-aware simulations in the \emph{expansion} and \emph{rollout} phases and adjust \emph{backpropagation} with a value update rule that accounts for transition feasibility.

\subsubsection{Uncertainty-Aware Simulation}
As detailed in \Cref{algo:IG_MCTS}, both the \emph{expansion} and \emph{rollout} phases involve forward simulation of robot actions. Each tree node $v$ contains the state $(\tau, x)$, representing the robot's state history and current human perception. We handle the two action types differently as follows:
\begin{itemize}
    \item A movement action $u$ follows the environment dynamics $T$ as defined in \Cref{sec:problem}. Notably, the maze layout is observable up to distance $r$ from the robot's visited grids, while unexplored areas assume a $50\%$ chance of walls. In \emph{expansion}, the resulting search node $v'$ of this uncertain transition is assigned a feasibility value $\delta = 0.5$. In \emph{rollout}, the transition could fail and the robot remains in the same grid.
    
    \item The state transition for a communication step $o$ is governed by the learned stochastic human perception model $F_\theta$ as defined in \Cref{sec:nhpm}. Since transition probabilities are known, we compute the expected information reward $\bar{R_\mathrm{info}}$ directly:
    \begin{align*}
        \bar{R_\mathrm{info}}(\tau_t, x_t, o_t) &= \mathbb{E}_{x_{t+1}}\|x_{t+1}-x_t\|_1 \\
        &= \|p_\mathrm{add}\|_1 + \|p_\mathrm{remove}\|_1,
    \end{align*}
    where $(p_\mathrm{add}, p_\mathrm{remove}) \gets F_\theta(\tau_t, x_t, o_t)$ are the estimated probabilities of adding or removing walls from the map. 
    Directly computing the expected return at a node avoids the high number of visitations required to obtain an accurate value estimate.
\end{itemize}

% We denote a node in the search tree as $v$, where $s(v)$, $r(v)$, and $\delta(v)$ represent the state, reward, and transition feasibility at $v$, respectively. The visit count of $v$ is denoted as $N(v)$, while $Q(v)$ represents its total accumulated return. The set of child nodes of $v$ is denoted by $\mathbb{C}(v)$.

% The goal of each search is to plan a sequence for the robot until it reaches a goal or transmits a new image to the human. We initialize the search tree with the current human guidance $\zeta$, and the robot's approximation of human perception $x_0$. Each search node consists consists of the state information required by our reward augmentation: $(\tau, x)$. A node is terminal if it is the resulting state of a communication step, or if the robot reaches a goal location. 

% A rollout from the expanded node simulates future transitions until reaching a terminal state or a predefined depth $H$. Actions are selected randomly from the available action set $\mathcal{A}(s)$. If an action's feasibility is uncertain due to the environment's unknown structure, the transition occurs with probability $\delta(s, a)$. When a random number draw deems the transition infeasible, the state remains unchanged. On the other hand, for communication steps, we don't resolve the uncertainty but instead compute the expected information gain reward: \philip{TODO: adjust notation}
% \begin{equation}
%     \mathbb{E}\left[R_\mathrm{info}(\tau, x')\right] = \sum \mathrm{NPM(\tau, o)}.
% \end{equation}

\subsubsection{Feasibility-Adjusted Backpropagation}
During backpropagation, the rewards obtained from the simulation phase are propagated back through the tree, updating the total value $Q(v)$ and the visitation count $N(v)$ for all nodes along the path to the root. Due to uncertainty in unexplored environment dynamics, the rollout return depends on the feasibility of the transition from the child node. Given a sample return \(q'_{\mathrm{sample}}\) at child node \(v'\), the parent node's return is:
\begin{equation}
    q_{\mathrm{sample}} = r + \gamma \left[ \delta' q'_{\mathrm{sample}} + (1 - \delta') \frac{Q(v)}{N(v)} \right],
\end{equation}
where $\delta'$ represents the probability of a successful transition. The term \((1 - \delta')\) accounts for failed transitions, relying instead on the current value estimate.

% By incorporating uncertainty-aware rollouts and backpropagation, our approach enables more robust decision-making in scenarios where the environment dynamics is unknown and avoids simulation of the stochastic human perception dynamics.

\begin{figure}[!t]
\centering
\includegraphics[width=\linewidth]{Images/data_process_pipeline.pdf}
\caption{Demonstration of the \method{} data-building system. I: Data scoring procedure; II: Data filtering procedure; III: Data fixing and augmentation procedure. IV: Field data producing procedure.}
\vspace{-1em}
\label{fig:data_process_pipeline}
\vspace{-1em}
\end{figure}




\section{Data-Building System.}\label{sec:data}
\method{} is trained on existing open-source datasets such as Objaverse (-XL)\cite{deitke2023objaverse, deitke2024objaverse} and ShapeNet\cite{chang2015shapenet}, which contains approximately $10$ million 3D data. Since most of these data are sourced from the Internet, their quality varies significantly, requiring extensive preprocessing to ensure suitability for training.
To overcome these challenges, \method{} developed a dedicated 3D data processing system that produces high-quality, large-scale datasets for model training.
As illustrated in Fig.\ref{fig:data_process_pipeline}, the system comprises four processing stages (Data Process I$\sim$IV), responsible for data scoring, filtering, fixing and augmentation, and field data producing, respectively.

\subsection{I: Data Scoring}
Each 3D model is scored, with only the high-score models advancing to the subsequent processing stages.
Specifically, we randomly selected approximately $10K$ 3D models and used Blender to render four different views of normal maps for each model. These multi-view normal maps are then manually evaluated by 10 professional 3D modelers, assigning scores on a scale from $1$ (lowest) to $5$ (highest).
Using this annotated data, we trained a linear regression-based scoring model concatenating their CLIP\cite{radford2021learning} and DINOv2\cite{oquab2023dinov2} features as input. This model was subsequently used to infer quality scores from the multi-view normal maps of all 3D models for filtering.

\subsection{II: Data Filtering}
After scoring, further filtering is applied to exclude models with large planar bases, rendering errors in animations, and those containing multiple objects.
Specifically, models with large planar bases are filtered by determining if different surface patches can be classified as a single plane, based on features composed of their centroid positions, normal vectors, and the area of the resulting plane. Blender identifies animated models, sets them to the first frame, and filters out any models that still exhibit rendering errors after being set. And models containing multiple objects are filtered by evaluating the proportion of the largest connected component on the opaque mask, along with the magnitude of the solidity of both the largest connected component and the entire mask.

\subsection{III: Data Fixing and Augmentation}
After data filtering, we perform the orientation fixing of character models to ensure they face forward. Specifically, we select 24 orientations around the x, y, and z axes, and for each, render images from six orthogonal views: front, back, left, right, top, and bottom. The DINOv2\cite{oquab2023dinov2} features from these six views are concatenated to train an orientation estimation model, which is then used to infer and fix the orientation of all character models. Additionally, for all untextured models, we render multi-view normal maps and use ControlNet++\cite{li2024controlnet++} to generate corresponding multi-view RGB data, which serve as conditional inputs during training.

\subsection{IV: Field Data Production}
% \subsubsection{Data process IV: Training Data Reproducing}
Although Objaverse (-XL)~\cite{deitke2023objaverse,deitke2024objaverse} contains a large amount of data, most of the models are unsuitable for direct training, even after processing steps such as scoring, filtering, and fixing. 
Since we adopt the neural implicit field as our 3D model representation, it's necessary to convert the original non-watertight mesh to watertight ones for computing geometry supervision (e.g., occupancy or SDF). 
Rather than using common methods like TSDF-fusion~\cite{DBLP:conf/ismar/NewcombeIHMKDKSHF11} or ManifoldPlus~\cite{DBLP:journals/corr/abs-1802-01698,DBLP:journals/corr/abs-2005-11621}, we are inspired by \cite{DBLP:journals/tog/WangLT22,zhang2024clay} to construct a Unsigned Distance Function (UDF) field with a resolution of $512^3$ grid from the original non-watertight mesh, and then apply Marching Cubes~\cite{DBLP:conf/siggraph/LorensenC87} to extract the iso-surface with a small threshold $\tau=\frac{3}{512}$. 
To remove interior structure for more efficient geometry learning, we follow \cite{zhang2024clay} by resetting the UDF value of the invisible grids to prevent the extraction of interior iso-surface before applying Marching Cubes. 
We then remove some small and invisible interior mesh components by calculating the area and the ambient occlusion ratio of each mesh component.
Finally, we uniformly sample surface points along with their normals, and randomly sample points both within the volume and near the surface.



\section{Experiments}
\label{section5}

In this section, we conduct extensive experiments to show that \ourmethod~can significantly speed up the sampling of existing MR Diffusion. To rigorously validate the effectiveness of our method, we follow the settings and checkpoints from \cite{luo2024daclip} and only modify the sampling part. Our experiment is divided into three parts. Section \ref{mainresult} compares the sampling results for different NFE cases. Section \ref{effects} studies the effects of different parameter settings on our algorithm, including network parameterizations and solver types. In Section \ref{analysis}, we visualize the sampling trajectories to show the speedup achieved by \ourmethod~and analyze why noise prediction gets obviously worse when NFE is less than 20.


\subsection{Main results}\label{mainresult}

Following \cite{luo2024daclip}, we conduct experiments with ten different types of image degradation: blurry, hazy, JPEG-compression, low-light, noisy, raindrop, rainy, shadowed, snowy, and inpainting (see Appendix \ref{appd1} for details). We adopt LPIPS \citep{zhang2018lpips} and FID \citep{heusel2017fid} as main metrics for perceptual evaluation, and also report PSNR and SSIM \citep{wang2004ssim} for reference. We compare \ourmethod~with other sampling methods, including posterior sampling \citep{luo2024posterior} and Euler-Maruyama discretization \citep{kloeden1992sde}. We take two tasks as examples and the metrics are shown in Figure \ref{fig:main}. Unless explicitly mentioned, we always use \ourmethod~based on SDE solver, with data prediction and uniform $\lambda$. The complete experimental results can be found in Appendix \ref{appd3}. The results demonstrate that \ourmethod~converges in a few (5 or 10) steps and produces samples with stable quality. Our algorithm significantly reduces the time cost without compromising sampling performance, which is of great practical value for MR Diffusion.


\begin{figure}[!ht]
    \centering
    \begin{minipage}[b]{0.45\textwidth}
        \centering
        \includegraphics[width=1\textwidth, trim=0 20 0 0]{figs/main_result/7_lowlight_fid.pdf}
        \subcaption{FID on \textit{low-light} dataset}
        \label{fig:main(a)}
    \end{minipage}
    \begin{minipage}[b]{0.45\textwidth}
        \centering
        \includegraphics[width=1\textwidth, trim=0 20 0 0]{figs/main_result/7_lowlight_lpips.pdf}
        \subcaption{LPIPS on \textit{low-light} dataset}
        \label{fig:main(b)}
    \end{minipage}
    \begin{minipage}[b]{0.45\textwidth}
        \centering
        \includegraphics[width=1\textwidth, trim=0 20 0 0]{figs/main_result/10_motion_fid.pdf}
        \subcaption{FID on \textit{motion-blurry} dataset}
        \label{fig:main(c)}
    \end{minipage}
    \begin{minipage}[b]{0.45\textwidth}
        \centering
        \includegraphics[width=1\textwidth, trim=0 20 0 0]{figs/main_result/10_motion_lpips.pdf}
        \subcaption{LPIPS on \textit{motion-blurry} dataset}
        \label{fig:main(d)}
    \end{minipage}
    \caption{\textbf{Perceptual evaluations on \textit{low-light} and \textit{motion-blurry} datasets.}}
    \label{fig:main}
\end{figure}

\subsection{Effects of parameter choice}\label{effects}

In Table \ref{tab:ablat_param}, we compare the results of two network parameterizations. The data prediction shows stable performance across different NFEs. The noise prediction performs similarly to data prediction with large NFEs, but its performance deteriorates significantly with smaller NFEs. The detailed analysis can be found in Section \ref{section5.3}. In Table \ref{tab:ablat_solver}, we compare \ourmethod-ODE-d-2 and \ourmethod-SDE-d-2 on the \textit{inpainting} task, which are derived from PF-ODE and reverse-time SDE respectively. SDE-based solver works better with a large NFE, whereas ODE-based solver is more effective with a small NFE. In general, neither solver type is inherently better.


% In Table \ref{tab:hazy}, we study the impact of two step size schedules on the results. On the whole, uniform $\lambda$ performs slightly better than uniform $t$. Our algorithm follows the method of \cite{lu2022dpmsolverplus} to estimate the integral part of the solution, while the analytical part does not affect the error.  Consequently, our algorithm has the same global truncation error, that is $\mathcal{O}\left(h_{max}^{k}\right)$. Note that the initial and final values of $\lambda$ depend on noise schedule and are fixed. Therefore, uniform $\lambda$ scheduling leads to the smallest $h_{max}$ and works better.

\begin{table}[ht]
    \centering
    \begin{minipage}{0.5\textwidth}
    \small
    \renewcommand{\arraystretch}{1}
    \centering
    \caption{Ablation study of network parameterizations on the Rain100H dataset.}
    % \vspace{8pt}
    \resizebox{1\textwidth}{!}{
        \begin{tabular}{cccccc}
			\toprule[1.5pt]
            % \multicolumn{6}{c}{Rainy} \\
            % \cmidrule(lr){1-6}
             NFE & Parameterization      & LPIPS\textdownarrow & FID\textdownarrow &  PSNR\textuparrow & SSIM\textuparrow  \\
            \midrule[1pt]
            \multirow{2}{*}{50}
             & Noise Prediction & \textbf{0.0606}     & \textbf{27.28}   & \textbf{28.89}     & \textbf{0.8615}    \\
             & Data Prediction & 0.0620     & 27.65   & 28.85     & 0.8602    \\
            \cmidrule(lr){1-6}
            \multirow{2}{*}{20}
              & Noise Prediction & 0.1429     & 47.31   & 27.68     & 0.7954    \\
              & Data Prediction & \textbf{0.0635}     & \textbf{27.79}   & \textbf{28.60}     & \textbf{0.8559}    \\
            \cmidrule(lr){1-6}
            \multirow{2}{*}{10}
              & Noise Prediction & 1.376     & 402.3   & 6.623     & 0.0114    \\
              & Data Prediction & \textbf{0.0678}     & \textbf{29.54}   & \textbf{28.09}     & \textbf{0.8483}    \\
            \cmidrule(lr){1-6}
            \multirow{2}{*}{5}
              & Noise Prediction & 1.416     & 447.0   & 5.755     & 0.0051    \\
              & Data Prediction & \textbf{0.0637}     & \textbf{26.92}   & \textbf{28.82}     & \textbf{0.8685}    \\       
            \bottomrule[1.5pt]
        \end{tabular}}
        \label{tab:ablat_param}
    \end{minipage}
    \hspace{0.01\textwidth}
    \begin{minipage}{0.46\textwidth}
    \small
    \renewcommand{\arraystretch}{1}
    \centering
    \caption{Ablation study of solver types on the CelebA-HQ dataset.}
    % \vspace{8pt}
        \resizebox{1\textwidth}{!}{
        \begin{tabular}{cccccc}
			\toprule[1.5pt]
            % \multicolumn{6}{c}{Raindrop} \\     
            % \cmidrule(lr){1-6}
             NFE & Solver Type     & LPIPS\textdownarrow & FID\textdownarrow &  PSNR\textuparrow & SSIM\textuparrow  \\
            \midrule[1pt]
            \multirow{2}{*}{50}
             & ODE & 0.0499     & 22.91   & 28.49     & 0.8921    \\
             & SDE & \textbf{0.0402}     & \textbf{19.09}   & \textbf{29.15}     & \textbf{0.9046}    \\
            \cmidrule(lr){1-6}
            \multirow{2}{*}{20}
              & ODE & 0.0475    & 21.35   & 28.51     & 0.8940    \\
              & SDE & \textbf{0.0408}     & \textbf{19.13}   & \textbf{28.98}    & \textbf{0.9032}    \\
            \cmidrule(lr){1-6}
            \multirow{2}{*}{10}
              & ODE & \textbf{0.0417}    & 19.44   & \textbf{28.94}     & \textbf{0.9048}    \\
              & SDE & 0.0437     & \textbf{19.29}   & 28.48     & 0.8996    \\
            \cmidrule(lr){1-6}
            \multirow{2}{*}{5}
              & ODE & \textbf{0.0526}     & 27.44   & \textbf{31.02}     & \textbf{0.9335}    \\
              & SDE & 0.0529    & \textbf{24.02}   & 28.35     & 0.8930    \\
            \bottomrule[1.5pt]
        \end{tabular}}
        \label{tab:ablat_solver}
    \end{minipage}
\end{table}


% \renewcommand{\arraystretch}{1}
%     \centering
%     \caption{Ablation study of step size schedule on the RESIDE-6k dataset.}
%     % \vspace{8pt}
%         \resizebox{1\textwidth}{!}{
%         \begin{tabular}{cccccc}
% 			\toprule[1.5pt]
%             % \multicolumn{6}{c}{Raindrop} \\     
%             % \cmidrule(lr){1-6}
%              NFE & Schedule      & LPIPS\textdownarrow & FID\textdownarrow &  PSNR\textuparrow & SSIM\textuparrow  \\
%             \midrule[1pt]
%             \multirow{2}{*}{50}
%              & uniform $t$ & 0.0271     & 5.539   & 30.00     & 0.9351    \\
%              & uniform $\lambda$ & \textbf{0.0233}     & \textbf{4.993}   & \textbf{30.19}     & \textbf{0.9427}    \\
%             \cmidrule(lr){1-6}
%             \multirow{2}{*}{20}
%               & uniform $t$ & 0.0313     & 6.000   & 29.73     & 0.9270    \\
%               & uniform $\lambda$ & \textbf{0.0240}     & \textbf{5.077}   & \textbf{30.06}    & \textbf{0.9409}    \\
%             \cmidrule(lr){1-6}
%             \multirow{2}{*}{10}
%               & uniform $t$ & 0.0309     & 6.094   & 29.42     & 0.9274    \\
%               & uniform $\lambda$ & \textbf{0.0246}     & \textbf{5.228}   & \textbf{29.65}     & \textbf{0.9372}    \\
%             \cmidrule(lr){1-6}
%             \multirow{2}{*}{5}
%               & uniform $t$ & 0.0256     & 5.477   & \textbf{29.91}     & 0.9342    \\
%               & uniform $\lambda$ & \textbf{0.0228}     & \textbf{5.174}   & 29.65     & \textbf{0.9416}    \\
%             \bottomrule[1.5pt]
%         \end{tabular}}
%         \label{tab:ablat_schedule}



\subsection{Analysis}\label{analysis}
\label{section5.3}

\begin{figure}[ht!]
    \centering
    \begin{minipage}[t]{0.6\linewidth}
        \centering
        \includegraphics[width=\linewidth, trim=0 20 10 0]{figs/trajectory_a.pdf} %trim左下右上
        \subcaption{Sampling results.}
        \label{fig:traj(a)}
    \end{minipage}
    \begin{minipage}[t]{0.35\linewidth}
        \centering
        \includegraphics[width=\linewidth, trim=0 0 0 0]{figs/trajectory_b.pdf} %trim左下右上
        \subcaption{Trajectory.}
        \label{fig:traj(b)}
    \end{minipage}
    \caption{\textbf{Sampling trajectories.} In (a), we compare our method (with order 1 and order 2) and previous sampling methods (i.e., posterior sampling and Euler discretization) on a motion blurry image. The numbers in parentheses indicate the NFE. In (b), we illustrate trajectories of each sampling method. Previous methods need to take many unnecessary paths to converge. With few NFEs, they fail to reach the ground truth (i.e., the location of $\boldsymbol{x}_0$). Our methods follow a more direct trajectory.}
    \label{fig:traj}
\end{figure}

\textbf{Sampling trajectory.}~ Inspired by the design idea of NCSN \citep{song2019ncsn}, we provide a new perspective of diffusion sampling process. \cite{song2019ncsn} consider each data point (e.g., an image) as a point in high-dimensional space. During the diffusion process, noise is added to each point $\boldsymbol{x}_0$, causing it to spread throughout the space, while the score function (a neural network) \textit{remembers} the direction towards $\boldsymbol{x}_0$. In the sampling process, we start from a random point by sampling a Gaussian distribution and follow the guidance of the reverse-time SDE (or PF-ODE) and the score function to locate $\boldsymbol{x}_0$. By connecting each intermediate state $\boldsymbol{x}_t$, we obtain a sampling trajectory. However, this trajectory exists in a high-dimensional space, making it difficult to visualize. Therefore, we use Principal Component Analysis (PCA) to reduce $\boldsymbol{x}_t$ to two dimensions, obtaining the projection of the sampling trajectory in 2D space. As shown in Figure \ref{fig:traj}, we present an example. Previous sampling methods \citep{luo2024posterior} often require a long path to find $\boldsymbol{x}_0$, and reducing NFE can lead to cumulative errors, making it impossible to locate $\boldsymbol{x}_0$. In contrast, our algorithm produces more direct trajectories, allowing us to find $\boldsymbol{x}_0$ with fewer NFEs.

\begin{figure*}[ht]
    \centering
    \begin{minipage}[t]{0.45\linewidth}
        \centering
        \includegraphics[width=\linewidth, trim=0 0 0 0]{figs/convergence_a.pdf} %trim左下右上
        \subcaption{Sampling results.}
        \label{fig:convergence(a)}
    \end{minipage}
    \begin{minipage}[t]{0.43\linewidth}
        \centering
        \includegraphics[width=\linewidth, trim=0 20 0 0]{figs/convergence_b.pdf} %trim左下右上
        \subcaption{Ratio of convergence.}
        \label{fig:convergence(b)}
    \end{minipage}
    \caption{\textbf{Convergence of noise prediction and data prediction.} In (a), we choose a low-light image for example. The numbers in parentheses indicate the NFE. In (b), we illustrate the ratio of components of neural network output that satisfy the Taylor expansion convergence requirement.}
    \label{fig:converge}
\end{figure*}

\textbf{Numerical stability of parameterizations.}~ From Table 1, we observe poor sampling results for noise prediction in the case of few NFEs. The reason may be that the neural network parameterized by noise prediction is numerically unstable. Recall that we used Taylor expansion in Eq.(\ref{14}), and the condition for the equality to hold is $|\lambda-\lambda_s|<\boldsymbol{R}(s)$. And the radius of convergence $\boldsymbol{R}(t)$ can be calculated by
\begin{equation}
\frac{1}{\boldsymbol{R}(t)}=\lim_{n\rightarrow\infty}\left|\frac{\boldsymbol{c}_{n+1}(t)}{\boldsymbol{c}_n(t)}\right|,
\end{equation}
where $\boldsymbol{c}_n(t)$ is the coefficient of the $n$-th term in Taylor expansion. We are unable to compute this limit and can only compute the $n=0$ case as an approximation. The output of the neural network can be viewed as a vector, with each component corresponding to a radius of convergence. At each time step, we count the ratio of components that satisfy $\boldsymbol{R}_i(s)>|\lambda-\lambda_s|$ as a criterion for judging the convergence, where $i$ denotes the $i$-th component. As shown in Figure \ref{fig:converge}, the neural network parameterized by data prediction meets the convergence criteria at almost every step. However, the neural network parameterized by noise prediction always has components that cannot converge, which will lead to large errors and failed sampling. Therefore, data prediction has better numerical stability and is a more recommended choice.


\section{Texture Generation}\label{sec:texture}
Thanks to the finely detailed and high-quality 3D geometry generated by \method{}, referring to Meta 3D TextureGen~\cite{bensadoun2024meta}, we can leverage the rendered normal maps as input conditions for existing mature multi-view generation methods to produce consistent multi-view texture images. These multi-view texture images are then projected onto the geometric surface to obtain detailed texture maps. 
Fig.\ref{fig:texture_demo_show} shows the 3D result with texture maps generated by \method{}.
\section{Further Analysis}

\label{sec:ablation}

\begin{table}[t!]
    \centering
    %\vspace{-0.5em}
    \bgroup
    \def\arraystretch{1.2}
    \caption{
    Comparison between \sassha and SAM with more training budgets for the ViT-s-32 / ImageNet workload.
    }
    \label{tab:sam}
    \vskip 0.1in
    \resizebox{0.7\linewidth}{!}{
        \centering
        \begin{tabular}{lccc}  % 5 more
        \toprule
        &\multicolumn{1}{c}{ Epoch } & Time (\texttt{s}) &\multicolumn{1}{c}{ Accuracy (\%) } \\ \midrule
        SAM$_\text{ SGD}$      &  180  & 220,852 & $  65.403 _{\textcolor{black!60}{\pm 0.63}}$ \\
        SAM$_\text{ AdamW}$    &  180 & 234,374 & $68.706 _{\textcolor{black!60}{\pm 0.16}}$ \\
        \midrule
        \rowcolor{green!20}\sassha        &  \textbf{90}  & \textbf{123,948} &  $\textbf{69.195} _{\textcolor{black!60}{\pm 0.30} } $  \\
        \bottomrule
        \end{tabular}
    }
    \egroup
    \vspace{-1.5em}
\end{table}

\subsection{Robustness}
\label{sec:robustness}

Noisily labeled training data can critically degrade generalization performance \citep{natarajan2013learning}.
To evaluate how \sassha generalizes under these practical conditions, we randomly corrupt certain fractions of the training data and compare the validation performances between different methods.
The results show that \sassha outperforms other methods across all noise levels with minimal accuracy degradation (\cref{tab:noise_label}).
Additionally, we also observe the same trend on CIFAR-10 (\cref{tab:noise_label_sassha}).

Interestingly, \sassha surpasses SAM \citep{sam}, which is known to be one of the most robust techniques against label noise \citep{baek2024why}. 
We hypothesize that its robustness stems from the complementary benefits of the sharpness-minimization scheme and second-order methods.
Specifically, SAM enhances robustness by adversarially perturbing the parameters and giving more importance to clean data during optimization, making the model more resistant to label noise \citep{sam, baek2024why}.
Also, recent research indicates that second-order methods are robust to label noise due to preconditioning that reduces the variance in the population risk \citep{amari2021when}.

\begin{table}[t!]
    \centering
    %\vspace{-0.5em}
    \caption{
    Validation accuracy measured for ResNet-32/CIFAR-100 at different levels of noise.
    \sassha shows the best robustness.
    }
    \label{tab:noise_label}
    \vskip 0.1in
    \resizebox{0.92\linewidth}{!} & {20\%} & {40\%} & {60\%} \\ 
        \midrule
        SGD                 &
        $69.32_{\textcolor{black!60}{\pm 0.19}}$
        & $62.18_{\textcolor{black!60}{\pm 0.06}}$ 
        & $55.78_{\textcolor{black!60}{\pm 0.55}}$  
        & $45.53_{\textcolor{black!60}{\pm 0.78}}$ \\ 
        
        SAM $_{\text{SGD}}$ & 
        $71.99_{\textcolor{black!60}{\pm 0.20}}$
        & $65.53_{\textcolor{black!60}{\pm 0.11}}$  
        & $ 61.20_{\textcolor{black!60}{\pm 0.17}}$  
        & $ 51.93_{\textcolor{black!60}{\pm 0.47}}$ \\ 
        
        AdaHessian         &
        $68.06_{\textcolor{black!60}{\pm 0.22}}$
        & $63.06_{\textcolor{black!60}{\pm 0.25}}$  
        & $58.37_{\textcolor{black!60}{\pm 0.13}}$  
        & $46.02_{\textcolor{black!60}{\pm 1.96}}$  \\

        Sophia-H           &
        $67.76_{\textcolor{black!60}{\pm 0.37}}$
        & $62.34_{\textcolor{black!60}{\pm 0.47}}$  
        & $56.54_{\textcolor{black!60}{\pm 0.28}}$  
        & $45.37_{\textcolor{black!60}{\pm 0.27}}$  \\
        
        Shampoo           & 
        $64.08_{\textcolor{black!60}{\pm 0.46}}$
        & $58.85_{\textcolor{black!60}{\pm 0.66}}$ & $ 53.82 _{\textcolor{black!60}{\pm 0.71}}$  
        & $ 42.91_{\textcolor{black!60}{\pm 0.99}}$ \\
        
        \midrule
        
        \rowcolor{green!20} \sassha         & 
        $ \textbf{72.14}_{\textcolor{black!60}{\pm 0.16}}    $
        & $\textbf{66.78}_{\textcolor{black!60}{\pm 0.47}}   $  
        & $ \textbf{ 61.97}_{\textcolor{black!60}{\pm 0.27}} $  
        & $\textbf{ 53.98}_{\textcolor{black!60}{\pm 0.57}}  $ \\
        
        % \rowcolor{green!20} \msassha        &  
        % $ 70.93 _{\textcolor{black!60}{\pm 0.26}}     $
        % $ 66.10 _{\textcolor{black!60}{\pm 0.26}}     $  
        % &  $ 61.13 _{\textcolor{black!60}{\pm 0.28}}  $  
        % &  $ 52.45 _{\textcolor{black!60}{\pm 0.34}}  $ \\   
        \bottomrule  
    \end{tabular}}
    %\vspace{-1em}
\end{table}

\begin{figure}[t!]
%\vspace{-2em}
\resizebox{\linewidth}{!}{%
    \centering
    \begin{subfigure}{0.4\linewidth}
        \centering
        \includegraphics[width=\linewidth]{figures/ablation/loss.pdf}
        \caption{Train loss}
        \label{fig:sqrt_ablation_train_loss}
    \end{subfigure}
    \begin{subfigure}{0.4\linewidth}
        \centering
        \includegraphics[width=\linewidth,trim={0.5em 0.5em 1em 0},clip]{figures/ablation/Update_size.pdf}
        \caption{Update size}
        \label{fig:sqrt_ablation_update-size}
    \end{subfigure}
    \hfill
    \begin{subfigure}{0.4\linewidth}
        \centering
        \includegraphics[width=\linewidth,trim={0 1em 0 1.2em},clip]{figures/ablation/sqrt_ridgeline.pdf}\\
        
        \caption{$D$ distribution}
        \label{fig:sqrt_ablation_ridgeline-plot}
    \end{subfigure}
    }
    \caption{
    Effects of square-root measured for ResNet-32/CIFAR-100;
    $D$ is set to be either $|\widehat{H}|^{1/2}$ for \sassha or $|\widehat{H}|$ for \texttt{No-Sqrt}.
    Sharpness minimization drives the diagonal Hessian entries move towards zero, causing divergence.
    The square-root in \sassha helps counteract this effect, stabilizing the training process.
    }
    \label{fig:sqrt_ablation}
    %\vspace{-1em}
\end{figure}

\subsection{Stability} \label{sec:sqrt_ablation}

To show the effect of the square-root function on stabilizing the training process, we run \sassha without the square-root (\texttt{No-Sqrt}), repeatedly for multiple times with different random seeds.
As a result, we find that the training diverges most of the time.
A failure case is depicted in \cref{fig:sqrt_ablation}.

At first, we find that the level of training loss for \texttt{No-Sqrt} is much higher than that of \sassha, and also, it spikes up around step $200$ (\cref{fig:sqrt_ablation_train_loss}).
To look into it further, we also measure the update sizes along the trajectory (\cref{fig:sqrt_ablation_update-size}).
The results show that it matches well with the loss curves, suggesting that the training failure is somehow due to taking too large steps.

It turns out that this problem stems from the preconditioning matrix $D$ being too small;
\ie, the distribution of diagonal entries in the preconditioning matrix gradually shifts toward zero values (\cref{fig:sqrt_ablation_ridgeline-plot});
as a result, $D^{-1}$ becomes too large, creating large steps.
This progressive increase in near-zero diagonal Hessian entries is precisely due to the sharpness minimization scheme that we introduced; it penalizes the Hessian eigenspectrum to yield flat solutions, yet it could also make training unstable if taken naively.
By including square-root, the preconditioner are less situated near zero, effectively suppressing the risk of large updates, thereby stabilizing the training process.
We validate this further by showing its superiority to other alternatives including damping and clipping in \cref{app:sqrt_alternatives}.

We also provide an ablation analysis for the absolute-value function in \cref{sec:abs_ablation}, which demonstrates that it increases the stability of \sassha in tandem with square-root.

\begin{figure}[t!] 
    \centering
    \begin{minipage}{\linewidth}
        \centering
        \hspace{1.2em}
        \includegraphics[width=0.9\linewidth, trim={0 0 0 0},clip]{figures/ablation/legend_only.pdf} % Path to your legend image
        \vspace{-0.6em}
    \end{minipage}

    \resizebox{\linewidth}{!}{
        \begin{subfigure}{0.288\linewidth}
        \centering%
            \includegraphics[width=\textwidth,trim={0 -1em 1.2em 0.5em},clip]{figures/ablation/hessian_update_frequency.pdf}
            \caption{Lazy Hessian}
            \label{fig:lazy_results}
        \end{subfigure}
        % \hspace{2.3em} 
        \begin{subfigure}{0.305\linewidth}
            \centering
            \includegraphics[width=\textwidth,trim={2em 1.4em 0 0.1em},clip]{figures/ablation/hess_diff.pdf}
            \caption{$\widehat{H}$ change}
            \label{fig:lazy_hess_diff}
        \end{subfigure}%
        \begin{subfigure}{0.32\linewidth}
            \centering
            \includegraphics[width=\linewidth,trim={0 -0.7em 0 0.3em},clip]{figures/ablation/perturbed.pdf}
            \caption{Local sensitivity}
            \label{fig:lazy_perturbed}
        \end{subfigure}
    }

    \caption{
    Effect of lazy Hessian for ResNet-32/CIFAR-100.
    \sassha stays within the region where the Hessian varies small.
    }
    \label{fig:diagonal_hessian_comp} %
    %\vspace{-1em}
\end{figure}

\subsection{Efficiency} \label{sec:emp_lazy_hess}

Here we show the effectiveness of lazy Hessian updates in \sassha.
The results are shown in \cref{fig:diagonal_hessian_comp}.
At first, we see that \sassha maintains its performance even at $k=100$, indicating that it is extremely robust to lazy Hessian updates (\cref{fig:lazy_results}).
We also measure the difference between the current and previous Hessians to validate lazy Hessian updates more directly (\cref{fig:lazy_hess_diff}).
The result shows that \sassha keeps the changes in Hessian to be small, and much smaller than other methods, indicating its advantage of robust reuse, and hence, computational efficiency.

We attribute this robustness to the sharpness minimization scheme incorporated in \sassha, which can potentially bias optimization toward the region of low curvature sensitivity.
To verify, we define local Hessian sensitivity as follows:
\begin{equation}\label{eq:diff_hessian(2)}
    \max_{\delta \sim \mathcal{N}(0, 1)}\left\|\widehat{H}\left(x+\rho\frac{\delta}{\|\delta\|_2}\right) - \widehat{H}(x)\right\|_F
\end{equation}
\ie, it measures the maximum change in Hessian induced from normalized random perturbations.
A smaller Hessian sensitivity would suggest reduced variability in the loss curvature, leading to greater relevance of the current Hessian for subsequent optimization steps.
We find that \sassha is far less sensitive compared to other methods (\cref{fig:lazy_perturbed}).

\subsection{Cost}
\label{sec:cost}

Second-order methods can be highly costly.
In this section, we discuss the computational cost of \sassha and reveal its competitiveness to other methods.

\sassha requires one gradient computation (\texttt{GC}) in the sharpness minimization step, one Hessian-vector product (\texttt{HVP}) for diagonal Hessian computation, and an additional \texttt{GC} in the descent step.
That is, a total of $2$\texttt{GC}s and $1$\texttt{HVP} are required.
However, with lazy Hessian updates, the number of \texttt{HVP}s reduces drastically to $ 1 / k $.
With $ k = 10 $ as the default value used in this work, this scales down to $0.1$\texttt{HVP}s.

It turns out that this is critical to the utility of \sassha, because $1$\texttt{HVP} is known to take about $ \times 3 $ the computation time of $1$\texttt{GC} in practice \citep{dagrou2024how}.
Compared to conventional second-order methods ($1$\texttt{GC} $+$ $1$\texttt{HVP} $\simeq$ $4$\texttt{GC}s), the cost of \sassha can roughly be a half of that ($2.3$\texttt{GC}s).
It is also comparable to standard SAM variants ($2$\texttt{GC}s).

Furthermore, we can leverage a momentum of gradients in the perturbation step to reduce the cost.
This variant \msassha requires only $1.3$\texttt{GC}s with minimal decrease in performance.
Notably, \msassha still outperforms standard first-order methods like SGD and AdamW (\cref{app:msassha}).

To verify, we measure the average wall-clock times and present the results in \cref{tab:costs}.
First, one can see that the theoretical cost is reflected well on the actual cost;
\ie, the time measurements scales proportionally roughly well with respect to the total cost.
More importantly, this result indicates the potential of \sassha for performance-critical applications.
Considering its well-balanced cost, and that it has been challenging to employ second-order methods efficiently for large-scale tasks without sacrificing performance, \sassha can be a reasonable addition to the lineup.

\begin{table}[t!]
    \vspace{-0.5em}
    \centering
    \caption{
    Average wall-clock time per epoch (\texttt{s}) and the theoretical cost of different methods.
    \sassha can be an effective alternative to existing methods for its enhanced generalization performance.
    } 
    \vskip 0.1in
    \resizebox{\linewidth}{!}{%
    \begin{tabular}{l|r|r|r|r|r|r|r}
    \toprule
    \multirow{2.5}{*}{Method}
    & \multicolumn{4}{c|}{Cost}
    & \multicolumn{1}{c|}{CIFAR10} 
    & \multicolumn{1}{c|}{CIFAR100} 
    & \multicolumn{1}{c}{ImageNet} \\ 
    \cmidrule(l{3pt}r{3pt}){2-5}
    \cmidrule(l{3pt}r{3pt}){6-8} 
     & \multicolumn{1}{c|}{Descent}
     & \multicolumn{1}{c|}{Sharpness}
     & \multicolumn{1}{c|}{Hessian}
     & \multicolumn{1}{c|}{Total}
     & \multicolumn{1}{c|}{ResNet32} 
     & \multicolumn{1}{c|}{WRN28-10}  
     & \multicolumn{1}{c}{ViT-small} \\ 
    \midrule
    AdamW  & 1 \texttt{GC}
    & 0 \texttt{GC}
    & 0 \texttt{HVP}
    & 1 \texttt{GC} & 5.03 & 59.29 & 976.56 \\
    
    SAM    & 1 \texttt{GC}
    & 1 \texttt{GC}
    & 0 \texttt{HVP}
    & 2 \texttt{GC} & 9.16 & 118.46  &  1302.08\\ 
    
    AdaHessian &  1 \texttt{GC}
    & 0 \texttt{GC}
    & 1 \texttt{HVP}
    & 4 \texttt{GC} & 33.75 & 296.63  & 2489.07 \\ \midrule
    
    \rowcolor{green!20} \sassha & 1 \texttt{GC}
    & 1 \texttt{GC}
    & 0.1 \texttt{HVP}
    & 2.3 \texttt{GC} & 12.00   & 142.06   & 1377.20 \\ 
    
    \rowcolor{green!20} \msassha & 1 \texttt{GC}
    & 0 \texttt{GC}
    & 0.1 \texttt{HVP}
    & 1.3 \texttt{GC}  & 8.91   & 84.12 & 1065.40 \\ 
    \bottomrule
    \end{tabular}
    }
    \vspace{-1em}
    \label{tab:costs}
\end{table}
\section{Conclusion \& Future Work}\label{conclusion}
This work presents XAMBA, the first framework optimizing SSMs on COTS NPUs, removing the need for specialized accelerators. XAMBA mitigates key bottlenecks in SSMs like CumSum, ReduceSum, and activations using ActiBA, CumBA, and ReduBA, transforming sequential operations into parallel computations. These optimizations improve latency, throughput (Tokens/s), and memory efficiency. Future work will extend XAMBA to other models, explore compression, and develop dynamic optimizations for broader hardware platforms.



% This work introduces XAMBA, the first framework to optimize SSMs on COTS NPUs, eliminating the need for specialized hardware accelerators. XAMBA addresses key bottlenecks in SSM execution, including CumSum, ReduceSum, and activation functions, through techniques like ActiBA, CumBA, and ReduBA, which restructure sequential operations into parallel matrix computations. These optimizations reduce latency, enhance throughput, and improve memory efficiency. 
% Experimental results show up to 2.6$\times$ performance improvement on Intel\textregistered\ Core\texttrademark\ Ultra Series 2 AI PC. 
% Future work will extend XAMBA to other models, incorporate compression techniques, and explore dynamic optimization strategies for broader hardware platforms.


% This work presents XAMBA, an optimization framework that enhances the performance of SSMs on NPUs. Unlike transformers, SSMs rely on structured state transitions and implicit recurrence, which introduce sequential dependencies that challenge efficient hardware execution. XAMBA addresses these inefficiencies by introducing CumBA, ReduBA, and ActiBA, which optimize cumulative summation, ReduceSum, and activation functions, respectively, significantly reducing latency and improving throughput. By restructuring sequential computations into parallelizable matrix operations and leveraging specialized hardware acceleration, XAMBA enables efficient execution of SSMs on NPUs. Future work will extend XAMBA to other state-space models, integrate advanced compression techniques like pruning and quantization, and explore dynamic optimization strategies to further enhance performance across various hardware platforms and frameworks.
% This work presents XAMBA, an optimization framework that enhances the performance of SSMs on NPUs. Key techniques, including CumBA, ReduBA, and ActiBA, achieve significant latency reductions by optimizing operations like cumulative summation, ReduceSum, and activation functions. Future work will focus on extending XAMBA to other state-space models, integrating advanced compression techniques, and exploring dynamic optimization strategies to further improve performance across various hardware platforms and frameworks.

% This work introduces XAMBA, an optimization framework for improving the performance of Mamba-2 and Mamba models on NPUs. XAMBA includes three key techniques: CumBA, ReduBA, and ActiBA. CumBA reduces latency by transforming cumulative summation operations into matrix multiplication using precomputed masks. ReduBA optimizes the ReduceSum operation through matrix-vector multiplication, reducing execution time. ActiBA accelerates activation functions like Swish and Softplus by mapping them to specialized hardware during the DPU’s drain phase, avoiding sequential execution bottlenecks. Additionally, XAMBA enhances memory efficiency by reducing SRAM access, increasing data reuse, and utilizing Zero Value Compression (ZVC) for masks. The framework provides significant latency reductions, with CumBA, ReduBA, and ActiBA achieving up to 1.8X, 1.1X, and 2.6X reductions, respectively, compared to the baseline.
% Future work includes extending XAMBA to other state-space models (SSMs) and exploring further hardware optimizations for emerging NPUs. Additionally, integrating advanced compression techniques like pruning and quantization, and developing adaptive strategies for dynamic optimization, could enhance performance. Expanding XAMBA's compatibility with other frameworks and deployment environments will ensure broader adoption across various hardware platforms.


\bibliography{main}
\bibliographystyle{icml2025}


%%%%%%%%%%%%%%%%%%%%%%%%%%%%%%%%%%%%%%%%%%%%%%%%%%%%%%%%%%%%%%%%%%%%%%%%%%%%%%%
%%%%%%%%%%%%%%%%%%%%%%%%%%%%%%%%%%%%%%%%%%%%%%%%%%%%%%%%%%%%%%%%%%%%%%%%%%%%%%%
% APPENDIX
%%%%%%%%%%%%%%%%%%%%%%%%%%%%%%%%%%%%%%%%%%%%%%%%%%%%%%%%%%%%%%%%%%%%%%%%%%%%%%%
%%%%%%%%%%%%%%%%%%%%%%%%%%%%%%%%%%%%%%%%%%%%%%%%%%%%%%%%%%%%%%%%%%%%%%%%%%%%%%%
% \newpage
% \appendix
% \onecolumn
% \section{You \emph{can} have an appendix here.}

% You can have as much text here as you want. The main body must be at most $8$ pages long.
% For the final version, one more page can be added.
% If you want, you can use an appendix like this one.  

% The $\mathtt{\backslash onecolumn}$ command above can be kept in place if you prefer a one-column appendix, or can be removed if you prefer a two-column appendix.  Apart from this possible change, the style (font size, spacing, margins, page numbering, etc.) should be kept the same as the main body.
% %%%%%%%%%%%%%%%%%%%%%%%%%%%%%%%%%%%%%%%%%%%%%%%%%%%%%%%%%%%%%%%%%%%%%%%%%%%%%%%
% %%%%%%%%%%%%%%%%%%%%%%%%%%%%%%%%%%%%%%%%%%%%%%%%%%%%%%%%%%%%%%%%%%%%%%%%%%%%%%%


\end{document}


% This document was modified from the file originally made available by
% Pat Langley and Andrea Danyluk for ICML-2K. This version was created
% by Iain Murray in 2018, and modified by Alexandre Bouchard in
% 2019 and 2021 and by Csaba Szepesvari, Gang Niu and Sivan Sabato in 2022.
% Modified again in 2023 and 2024 by Sivan Sabato and Jonathan Scarlett.
% Previous contributors include Dan Roy, Lise Getoor and Tobias
% Scheffer, which was slightly modified from the 2010 version by
% Thorsten Joachims & Johannes Fuernkranz, slightly modified from the
% 2009 version by Kiri Wagstaff and Sam Roweis's 2008 version, which is
% slightly modified from Prasad Tadepalli's 2007 version which is a
% lightly changed version of the previous year's version by Andrew
% Moore, which was in turn edited from those of Kristian Kersting and
% Codrina Lauth. Alex Smola contributed to the algorithmic style files.
