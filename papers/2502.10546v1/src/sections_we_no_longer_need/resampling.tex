    \subsection{Particle Resampling}
        The stochastic nature of PF algorithms often leads particles to gradually drift towards low probability regions of the posterior. These divergent particles, receiving low weights during measurement updates, contribute minimally to the posterior density. When only a few particles significantly impact the posterior density, its representational power is diminished, and further divergence can negatively affect the resulting posterior densities.
        
        Particle resampling provides offers a remedy by drawing a new uniformly weighted particle set $\hat{x}_t^{(:)}$ from $x_t^{(:)}$, with each particle duplicated (or not) proportional its current weight $w_t^{(i)}$. The number of resampled copies for each particle should approximate its share of the total weight:
        \begin{equation}
             E[N_t^{(i)}] = Nw_t^{(i)}, \qquad \text{as } N \rightarrow \infty, \qquad \text{for } i = 1, ..., N
            \label{eqn:resampling_num_of_samples_expectation}
        \end{equation}	
         where $N_t^{(i)}$ is the number of times $x_t^{(i)}$ is drawn into the resampled particle set. This is known as the "unbiasedness" or "properly weighting" condition  \cite{liu1998sequential, douc2005comparison}
         Following resampling, particles are uniformly assigned weights $\hat{w}_t^{(i)} = 1/N$, ensuring the preservation of richness and diversity within the particle set over time by eliminating low-weight particles and redistributing weights among the remaining ones. Resampling is usually performed after weighing particles using the latest observation.
        
        \textbf{Multinomial Resampling} is the simplest of the resampling methods that satisfies eqn. \ref{eqn:resampling_num_of_samples_expectation}.  Each resampled particle is produced using sampling with replacement with particle weights defining the resampling probabilities:
        \begin{equation}
             \hat{x}_t^{(i)} = x_t^{(j)},  \qquad\qquad j \sim \text{Cat}(w_t^{(1)},\ldots,w_t^{(N)}).
            \label{eqn:multinomial_resampling}
        \end{equation}	
        This methods is easy to implement and amenable to parallelization however can have higher variance than other resampling techniques \cite{douc2005comparison}.
        
        \textbf{Residual Resampling} \cite{liu1998sequential,douc2005comparison, whitley1994genetic} is proposed as a straightforward method for reducing variance induced from particle resampling.  This method computes the number of times particle $x_t^{(i)}$ is resampled as
        \begin{equation*}
             N_t^{(i)} = \lfloor Nw_t^{(i)} \rfloor + \bar{N}_t^{(i)}, \qquad \text{with } N = \sum_{i=1}^N N_t^{(i)}, \qquad \bar{w}_t^{(i)} = \frac{Nw_t^{(i)} - \lfloor Nw_t^{(i)} \rfloor}{N - \sum_{i=1}^N \lfloor Nw_t^{(i)} \rfloor}
            \label{eqn:residual_resampling}
        \end{equation*}	
        with $\bar{N}_t^{(i)}$ being a stochastic term $\bar{N}_t^{(1)}, ..., \bar{N}_t^{(N)} \sim \text{Multinomial}(\bar{w}_t^{(1)}, ..., \bar{w}_t^{(N)}) $ used to determine the number of additional copies of $x_t^{(i)}$ should be made in order to achieve a resampled particle set of a particular size $N$. 

        \textbf{Stratifed Resampling}
            \todo{Fill this in}
