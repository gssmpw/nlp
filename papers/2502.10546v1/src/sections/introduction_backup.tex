\section{Introduction}



    Global localization at a city scale poses an interesting challenge due to the size of the target region and the inherent ambiguity in urban landscapes, where similarities among street intersections and building appearances are prevalent.   
    Specifically, this task aims to determine the precise 2D position and heading direction of a moving system (i.e. person, car, bike) over time by leveraging first-person camera images along with a map of the environment. With the abundance of data available, solving this task using end-to-end trained systems is very appealing and a topic of open research.


    

    Several choices exists for representing maps however the usefulness and practicality greatly varies. 3D maps (point clouds and photogrammetry) offer a rich set of features useful for localization however are costly to create at city scales, requiring constant updating to account for changes in visual appearance over time and have prohibitive storage requirements when scaled to city sized regions due to their dense 3D representations.  Implicit map representations \todo{cite occupancy maps, nerfs, ext} have smaller storage requirements but must still be regularly updated to account for environment appearance changes which is expensive. \todo{cite}
    
    2D city maps comparibly require much less storage than other map types due to their reduced dimensional.  Overhead (i.e. satellite) imagery is a popular choice for 2D map as it is widely accessible however like 3D maps, these overhead imagery based 2D maps must constantly be updated. Planimetric 2D maps are another widely available map representation cheaply available from multiple sources making their use appealing. Since planimetric maps represent the environment semantically, they are robust against most visual appearance changes and thus do not need to be updated as frequently as other maps.  These maps also require much less storage than other map types since they are stores as shape primatives (polygons, lines) with labels instead of as pixels, voxels or 3D points. \todo{cite}
       
    Several works \todo{cite} exist for global localization using 2D maps which are based on latent vector comparisons, location optimization, and dense search. These methods often disregard temporal information when localizing and simply estimate the global position of the system at each time-point independently. Some methods incorporate temporal information however they require near exact relative pose information which is not usually available and thus must be computed in parallel via a secondary complex and expensive VI SLAM system \todo{cite orienternet and that vector database paper}.

    In parallel with the global localization task, a rich body of work on differentiable and learnable Particle Filters (PF) has recently developed. Interestingly PFs use all available past observations when estimating the state at the current time-step making them especially suitable for estimation tasks with temporal information such as global localization over time. Recent end-to-end trainable PFs are able to learn complex temporal system dynamics as well as interesting observation models, however these methods were only applied to synthetic datasets.  \todo{citet} applied a non-differentiable PF to the global localization task with planimetric 2D maps however each component of the PF was trained separately and the temporal dynamics of the system was modeled simply as Gaussian noise which may not capture true system dynamics at all. 

    Filtering approaches are useful when processing data 
    
    as it streams in (live processing) as it only relies on observations up to the current time-step when estimating posteriors
    

    

    Of note the Simultaneous Localization and Mapping (SLAM) \todo{Add citation} and Visual Inertial Odometry (VIO) \todo{Add citation} tasks differ from global localization by only estimating the position relative to a known starting location at time $t=0$. Both SLAM and VIO tasks can be converted to global localization by anchoring the the starting location using accurate ground truth information however we focus on general global localization where accurate ground truth information is not available.

    In this work we explore applying differentiable and fully end-to-end learnable particle filters to the task of global localization using 2d planimetric maps.  We then extend

    
    
    

    
    


