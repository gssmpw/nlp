\section{Introduction}
\vspace*{-5pt}
    Global localization of the state of a moving vehicle using a city-scale map is challenging due to the large area, as well as the inherent ambiguity in urban landscapes, where many street intersections and buildings appear similar. Recent work on global localization~\cite{shi2020where, Hu_2018_CVPR, noe2020eccv, zhu2021vigor, NEURIPS2019_ba2f0015, shi2019optimal, xia2022visual, 9635972GausePF, shi2020beyond, sarlin21pixloc, sarlin2023orienternet} has typically localized each time point independently during training, sometimes followed by temporal post-processing, often with demanding requirements like near-exact external estimation of relative vehicle poses~\cite{sarlin2023orienternet}. %often with prohibitive requirements on the type of information provided, such as requiring near exact relative pose information.
    
    For a broader range of state estimation problems in fields like vision and robotics, a number of methods for end-to-end \emph{particle filter} (PF) training have been proposed~\cite{9635972GausePF, pmlr-v139-corenflos21a_optimal_transport, pmlr-v87-karkus18a_soft_resampling, younis2023mdpf, jonschkowski18_differentiable_particle_filter, scibior2021differentiable}.  Learnable PFs are suitable for global localization because they can represent multi-modal posterior densities, propagate uncertainty over time, and learn models of real vehicle dynamics and complex sensors directly from data. However, most learnable PF methods have only been applied to simulated environments~\cite{pmlr-v139-corenflos21a_optimal_transport, pmlr-v87-karkus18a_soft_resampling, younis2023mdpf}, with only a few preliminary applications to real-world data~\cite{9635972GausePF, jonschkowski18_differentiable_particle_filter}.

    Particle filters~\cite{gordon1993novel,kanazawa95,doucet01,arulampalam02,probabilistic_robotics} only use past observations to estimate the current state.  For offline inference from complete time series, more powerful particle smoothing (PS) algorithms \cite{bresler1986TwoFilter,doi:10.1080/10618600.1996.10474692stratified, doucet2009tutorial, Klaas2006FastPS, briers2010smoothing, rauch1965maximum} may in principle perform better by integrating past and future data.  But to our knowledge, recent advances in end-to-end differentiable training of PFs have not been generalized to the more complex PS scenario, requiring error-prone human engineering of PS dynamics and observation models. Classical work on generative parameter estimation via PS~\cite{kantas2015particle} is limited to parametric models with few parameters. In contrast, we develop differentiable PS that scale to complex models defined by deep neural networks.
    %It is not clear how to extrapolate these methods to settings with large numbers of parameters (i.e. when using neural networks).
    
    After introducing differentiable particle filters (Sec.~\ref{sec:diffy_pf}) and classical particle smoothers (Sec.~\ref{sec:from_filtering_to_smoothing}), we develop our differentiable, discriminative \emph{Mixture Density Particle Smoother} (MDPS, see Fig.~\ref{fig:forward_backward_smoother_flow_diagram}) in Sec.~\ref{sec:mdps}. Thorough experiments in Sec.~\ref{sec:experiments} then highlight the advantages of our MDPS over differential PFs on a synthetic bearings-only tracking task, and also show substantial advantages over search-based and retrieval-based baselines for challenging real-world, city-scale global localization problems.    


    % Global localization, estimating 2D position and heading, at a city scale poses an interesting challenge due to the large area and the inherent ambiguity in urban landscapes, where similarities among street intersections and building appearances are prevalent.  With the abundance of data available, solving this task using end-to-end trained systems is very appealing and a topic of open research.
    
    % In order to localize globally, a map of the environment must be provided that offers enough information for the system to match locally observable features with global locations. 3D maps (i.e. point clouds and photogrametry) have been proposed however these have prohibitive storage requirements when scaled to cities and are sensitive to small visual appearance changes of the 3D environments (weather, seasons, ext). Implicit maps such as Neural Radience Fields \cite{mildenhall2020nerf, tancik2022blocknerf, xiangli2022bungeenerf} and Gaussian Splats \cite{kerbl3Dgaussians} offer more compact representations but continue to suffer from the same drawbacks as 3D maps. In contrast 2D maps can easily scale to city sized areas due to their reduced dimentionality. Overhead (i.e. satellite) imagery is a popular choice for 2D map as it is widely accessible however like 3D maps, they are sensitive to visual appearance changes and must be constantly updated.  Recently several works propose using planimetric 2D maps which represent represent the environment semantically and are thus robust to appearance changes. These maps are widely available and easily downloadable. Further these maps require very little storage as they are saved as shape primitives (polygons, lines) with labels.  In this work we focus on global localization using planimetric 2D maps.

    % Global localization of moving systems (i.e. cars, bikes, people) is inherently a temporal state estimation problem where the state of the system (2D position and heading) is estimated for each time-step. Recently a rich body of work on differentiable and learnable Particle Filters (PF) has developed. Interestingly PFs use all available past observations when estimating the state at the current time-step making them especially suitable for estimation tasks with temporal information such as global localization. These end-to-end trainable PFs are able to learn complex temporal system dynamics as well as interesting observation models, however these methods were only applied to synthetic datasets. 

    % Particle filters, and filtering in general, only utilize past observations when estimating the position at the current time-step.  This is useful for ``online" problems when the data is processed as it is received. If the full sequence of observations is given at processing time, data is collected and then processed at a later time, \textit{smoothing} instead of \textit{filtering} can be applied where observations from future time-steps as well as the past are used when estimating the position. Several classical particle smoother techniques exists \todo{cite forward-backward, fixed lag and 2 filter} however to the best of our knowledge, none have been adapted to be end-to-end differentiable. 

    % In this work we apply differentiable and fully end-to-end learnable particle filters to the task of global localization using 2d planimetric maps on real world datasets. We also explore various variance reducing resampling techniques for particle resampling as well as propose a novel end-to-end differentiable particle smoother which we call.  Finally we evaluate our ... on a range of datasets.
    
    

    

    % Of note the Simultaneous Localization and Mapping (SLAM) \todo{Add citation} and Visual Inertial Odometry (VIO) \todo{Add citation} tasks differ from global localization by only estimating the position relative to a known starting location at time $t=0$. Both SLAM and VIO tasks can be converted to global localization by anchoring the the starting location using accurate ground truth information however we focus on general global localization where accurate ground truth information is not available.


    
    
    

    
    


