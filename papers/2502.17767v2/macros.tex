\usepackage{mathtools, amsfonts, amssymb, mathrsfs}
\usepackage{enumerate,enumitem}
\usepackage{subcaption}
\usepackage{soul}
\usepackage{algorithm, algpseudocode, algorithmicx}
\usepackage{comment}
\usepackage{xspace}
\usepackage{mleftright}

% Customize the backref format
\renewcommand*{\backref}[1]{}
\renewcommand*{\backrefalt}[4]{%
	\ifcase #1 %
	(No citations.)% use this if no citations
	\or
	(Cited on page #2.)% use this if only one citation
	\else
	(Cited on pages #2.)% use this if multiple citations
	\fi
}

\usepackage[most]{tcolorbox}

\newcommand{\todo}[1]{\hl{\textbf{TODO}: #1}}

\renewcommand{\algorithmicrequire}{\textbf{Input:}}
\renewcommand{\algorithmicensure}{\textbf{Output:}}
\algrenewcommand\algorithmiccomment[1]{\hfill\textcolor{CornflowerBlue}{$\triangleright$ #1}}

% Number systems
\newcommand{\Sym}{\mathbb{H}}
\newcommand{\real}{\mathbb{R}}
\newcommand{\complex}{\mathbb{C}}
\newcommand{\field}{\mathbb{K}}
\newcommand{\onevec}{\mathbb{1}}

% Linear algebra
\DeclareMathOperator{\tr}{tr}
\DeclareMathOperator{\diag}{diag}
\DeclareMathOperator{\orth}{orth}
\newcommand{\mat}[1]{\boldsymbol{#1}}
\renewcommand{\vec}[1]{\boldsymbol{#1}}
\newcommand{\lowrank}[1]{\mleft\llbracket #1 \mright\rrbracket}
\newcommand{\norm}[1]{\mleft\| #1 \mright\|}
\newcommand{\ip}[2]{\mleft\langle #1, \, #2\mright\rangle}
\DeclareMathOperator{\range}{range}
\newcommand{\QR}{\textsf{QR}\xspace}
\newcommand{\ULV}{\textsf{ULV}\xspace}
\newcommand{\LU}{\textsf{LU}\xspace}

% Matrices
\newcommand{\expmat}[1]{\begin{bmatrix} #1 \end{bmatrix}}
\newcommand{\twobytwo}[4]{\expmat{#1 & #2 \\ #3 & #4}}
\newcommand{\twobyone}[2]{\expmat{#1 \\ #2}}
\newcommand{\onebytwo}[2]{\expmat{#1 & #2}}
\newcommand{\threebythree}[9]{\expmat{#1 & #2 & #3 \\ #4 & #5 & #6 \\ #7 & #8 & #9}}
\newcommand{\templatetwobytwo}[1]{\twobytwo{#1_{11}}{#1_{12}}{#1_{21}}{#1_{22}}}
\newcommand{\Id}{\mathbf{I}}

% Probability
\DeclareMathOperator{\Var}{Var}
\DeclareMathOperator{\expect}{\mathbb{E}}
\DeclareMathOperator{\prob}{\mathbb{P}}
\DeclareMathOperator{\Unif}{\textsc{Unif}}

% Asymptotics
\newcommand{\order}{\mathcal{O}}
\newcommand{\orderish}{\widetilde{\mathcal{O}}}

% Numerical stability
\DeclareMathOperator{\fl}{fl}

% General
% \definecolor{forestgreen}{RGB}{34,139,34}
\newcommand{\ENE}[1]{{\color{blue} [\textbf{ENE}: #1]}}
\newcommand{\AG}[1]{{\color{green} [\textbf{AG}: #1]}}
\DeclareMathOperator*{\argmax}{argmax}
\DeclareMathOperator*{\argmin}{argmin}
\newcommand{\set}[1]{\mathsf{#1}}
\newcommand{\e}{\mathrm{e}}
\newcommand{\im}{\mathrm{i}}
\renewcommand{\Re}{\mathrm{Re}}
\renewcommand{\Im}{\mathrm{Im}}
\renewcommand{\hat}[1]{\widehat{#1}}
\renewcommand{\tilde}[1]{\widetilde{#1}}
\newcommand{\actionbox}[1]{\begin{tcolorbox}[colback=white,colframe=black,width=\columnwidth,boxsep=5pt,arc=4pt]
    #1
\end{tcolorbox}}

% Specialized notation for this document
\DeclareMathOperator{\cond}{cond}
\DeclareMathOperator{\err}{err}
\newcommand{\myparagraph}[1]{\vspace{0.3em}

\textit{\textbf{#1}}}