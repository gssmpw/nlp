\section{Related Work}
\label{sec:rel}
%%%%%%%%%%%%%%%%%%%%%%%%%%%%%%%%%%%%%%%%%%%%%%%%%%%%%%%%%%%%%%%%%%%%%%%%%%%%%%%%%%%%%%%%%%%

%  neural and neuro-symbolic approaches for VQA 
Our approach 
%follows ideas from 
builds on previous work____, where we introduced a %similar 
neuro-symbolic method for VQA in the context of the CLEVR dataset____
%As here, the reasoning component there is logic-based and implemented in ASP.
using a reasoning component based on ASP inspired by NSVQA____. The latter used a combination of RCNN____ for object detection, an LSTM____ for natural language parsing, and Python as a symbolic executor to infer the answer. 
The vision and language modules in these previous approaches were trained for the datasets. 
As compared to these datasets the number of questions obtained from the questionnaires to build our dataset is small, it would be hardly possible to effectively train an % \egc 
LSTM on them. 
It is a particular strength of our work that we resort to LLMs that do not require any further training.

We also mention the neural and end-to-end trainable MAC system____ that achieves very promising results in VQA datasets, provided there is enough data available to train the system.
A recent approach that combines large pretrained models for images and text in combination with symbolic execution in Python is ViperGPT____; complicated graph images
are not handled well by pretrained vision-language models, however.
%would presumably require some fine-tuning, however.

%  neuro-symbolic approaches that involve ASP 
A characteristic of \ours is that we use ASP for reasoning, an
idea that was also explored in previous work____.
Outside of the context of VQA, ASP has been applied for various neuro-symbolic tasks such as
segmentation of laryngeal images____, and discovery of rules that explain sequences of sensory input____. 
Barbara et al.____ describe a neuro-symbolic approach that involves ASP for visual validation of electric panels where a component graph from an image is matched against its specification. This is an example of another interesting application that involves images of graphs and our approach could be used to contribute question-answering capabilities in such a setting.

In passing, it should be noted that there are also systems that can be used for neuro-symbolic learning, \egc by employing semantic loss____, which means that they use the information produced by the reasoning module to improve the learning tasks of the neural networks involved____.

% related work on LLMs for predicate extraction and ASP
Our approach to using LLMs to extract predicates for the downstream reasoning task is inspired by recent work by Rajasekharan et al.____. They proposed the STAR framework, which consists of LLMs and prompts for extracting logical predicates in combination with an ASP knowledge base. The authors applied STAR to different problems requiring qualitative reasoning, mathematical reasoning, as well as goal-directed conversation. 
%
Going one step further, Ishay et al.____ introduced a method to translate problems formulated in natural language into complete ASP programs. This method requires multiple prompts, each responsible for a subtask such as identifying constant symbols, forming predicates, and transforming the specification into rules. % using these predicates. 
%
The idea to apply LLMs to parse natural language into a formal language suitable for automated reasoning is also found outside the context of ASP, \egc work by Liu et al.____, who use prompting techniques to translate text into the Planning Domain Definition Language.
%(PDDL) as an example.



%%%%%%%%%%%%%%%%%%%%%%%%%%%%%%%%%%%%%%%%%%%%%%%%%%%%%%%%%%%%%%%%%%%%%%%%%%%%%%%%%%%%%%%%%%%