\section{Related Works}
\vspace{-1mm}
We review graph generation methods along with higher-order generation. \cref{app:related} presents a more detailed review.



\paragraph{Deep Generative Models}
Graph generative models make great progress by exploiting the capacity of deep neural networks. These models typically generate nodes and edges either in an autoregressive manner or simultaneously, utilizing techniques such as variational autoencoders (VAE) \cite{VAE-Jin2018,GraphVAE-DrugDiscovery}, recurrent neural networks (RNN) \cite{GraphRNN2018}, normalizing flows \cite{Moflow-SIGKDD2020,GraphAF-ICLR2020,GraphDF-ICML2021}, and generative adversarial networks (GAN) \cite{GAN1-MolGAN,GAN2-Spectre}.

\paragraph{Diffusion-based Graph Generation}
A breakthrough in graph generative models has been marked by the recent progress in diffusion-based generative models \cite{EDPGNN-2020}.
Recent models employ various strategies to enhance the generation of complex graphs, including capturing node-edge dependency \cite{GDSS+ICML2022}, addressing discretization challenges \cite{DiGress+ICLR2023,CDGS+AAAI2023}, exploiting low-to-high frequency generation curriculum \cite{GPrinFlowNet+ACM2024}, and improving computational efficiency through low-rank diffusion processes \cite{GSDM+TPAMI2023}. 
% ------ Diffusion bridge models
Recent studies have also enhanced diffusion-based generative models by incorporating diffusion bridge processes, \ie, processes conditioned on the endpoints \cite{wu2022diffusion,GLAD-ICMLworkshop2024,GruM+ICML2024}.
%
% ---- short conclusion
Despite these advances, existing methods either overlook or inadvertently disrupt higher-order structures during graph generation, or struggle to model the topological properties, as denoising the noisy samples does not explicitly preserve the intricate structural dependencies required for generating realistic graphs.




\paragraph{Higher-order Generative Models}
Generative modelling uses higher-order information mostly in the form of hypergraphs.
Models such as Hygene~\cite{gailhard2024hygene} and HypeBoy~\cite{kim2024hypeboy} aim to generate hypergraphs. Dymond~\cite{zeno2021dymond} focuses on higher-order motifs in dynamic graphs. 
To the best of our knowledge, we are the first to consider higher-order guides for graph generation.