\begin{figure*}[ht]
%\vspace{-1mm}
\begin{minipage}{0.23\linewidth}
    \centering
    \vspace{1mm}
    \includegraphics[width=1\linewidth]{figs/training_loss.pdf}
\end{minipage}
\hfill
\begin{minipage}{0.43\linewidth}
\resizebox{\textwidth}{!}{
\begin{tabular}{l l ccc}
\toprule
Dataset & Guide 
& NSPDK$\downarrow$
& FCD$\downarrow$
& \makecell{Val. w/o \\ corr.$\uparrow$ }\\
\midrule
& Noise       & 0.0015 & 0.829 & 91.52  \\
QM9 
& Peripheral & 0.0009 & 0.305 & 97.58  \\
& Cell        & 0.0003 & 0.172 & 98.74  \\
\midrule
& Noise    & 0.002  & 1.665 & 96.78 \\
ZINC250k 
& Peripheral  & 0.002  & 1.541 & 97.93 \\
& Cell    & 0.001  & 1.533 & 98.56 \\
\bottomrule
\end{tabular}}
\end{minipage}
\hfill
\begin{minipage}{0.26\linewidth}
\vspace{-2mm}
    \caption{\small (\textbf{Left}) Training curves of the score-matching process. The entire process of HOG-Diff is divided into two stages, \ie, $K=2$, referred to as coarse and fine, respectively. The combined loss of these two stages is labelled as Coarse+Fine.  
    (\textbf{Right}) Sampling results of various topological guides.}
    \label{fig:ablation}
\end{minipage}
\vspace{-4mm}
\end{figure*} 