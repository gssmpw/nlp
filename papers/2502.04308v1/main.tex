%%%%%%%% ICML 2025 EXAMPLE LATEX SUBMISSION FILE %%%%%%%%%%%%%%%%%

\documentclass{article}

% Recommended, but optional, packages for figures and better typesetting:
\usepackage{microtype}
\usepackage{graphicx}
\usepackage{subfigure}
\usepackage{booktabs} % for professional tables

% hyperref makes hyperlinks in the resulting PDF.
% If your build breaks (sometimes temporarily if a hyperlink spans a page)
% please comment out the following usepackage line and replace
% \usepackage{icml2025} with \usepackage[nohyperref]{icml2025} above.
\usepackage{hyperref}


% Attempt to make hyperref and algorithmic work together better:
\newcommand{\theHalgorithm}{\arabic{algorithm}}



% Use the following line for the initial blind version submitted for review:
%\usepackage{icml2025}
\usepackage[preprint]{icml2025}

% If accepted, instead use the following line for the camera-ready submission:
%\usepackage[accepted]{ICML_sty/icml2025}



% For theorems and such
\usepackage{amsmath}
\usepackage{amssymb}
\usepackage{mathtools}
\usepackage{amsthm}

% if you use cleveref..
\usepackage[capitalize,noabbrev]{cleveref}

%%%%%%%%%%%%%%%%%%%%%%%%%%%%%%%%
% THEOREMS
%%%%%%%%%%%%%%%%%%%%%%%%%%%%%%%%
\theoremstyle{plain}
\newtheorem{theorem}{Theorem}[section]
\newtheorem{proposition}[theorem]{Proposition}
\newtheorem{lemma}[theorem]{Lemma}
\newtheorem{corollary}[theorem]{Corollary}
\theoremstyle{definition}
\newtheorem{definition}[theorem]{Definition}
\newtheorem{assumption}[theorem]{Assumption}
\theoremstyle{remark}
\newtheorem{remark}[theorem]{Remark}


%\usepackage{mysty}


\usepackage{makecell} % 



\usepackage{multirow}
\usepackage{enumitem}
\usepackage{graphicx} % for \resizebox
\usepackage{bm} % for \bm

% The following two are for algorithm environment
\usepackage{algorithmic}
\renewcommand{\algorithmiccomment}[1]{\hfill{ $\rhd$ #1}}



% For table, add a gray background color
\usepackage{colortbl}
\definecolor{gg}{gray}{0.92}
\newcolumntype{a}{>{\columncolor{gg}}c}


\hypersetup{
    colorlinks = true,
    citecolor = blue,
    linkcolor = red
}


% for the Proposition environment in the Appendix
\newtheoremstyle{customsty}
  {\topsep}                 % 上方间距
  {\topsep}                 % 下方间距
  {\itshape}                % 正文字体
  {0pt}                     % 缩进量
  {\bfseries}               % 标题字体
  {.}                       % 标题后缀
  {5pt plus 1pt minus 1pt}  % 标题与正文间的间距
  {}                        % 标题格式

\theoremstyle{customsty}
\newtheorem{innercustomthe}{}
\newenvironment{customthe}[1][]{%
    \renewcommand\theinnercustomthe{#1}% 自定义编号并加粗
    \begin{innercustomthe}
}{%
    \end{innercustomthe}
}



\crefname{equation}{eq.}{eq.}
\Crefname{equation}{Eq.}{Eq.}
\crefname{theorem}{thm.}{thms.}
\Crefname{Theorem}{Thm.}{Thms.}
\crefname{conjecture}{conj.}{conjs.}
\Crefname{Conjecture}{Conj.}{Conjs.}
\crefname{proposition}{Prop.}{Props.}
\Crefname{proposition}{Prop.}{Props.}
\crefname{definition}{dfn.}{dfn.}
\Crefname{definition}{Dfn.}{Dfn.}
\crefname{remark}{remark}{remark}
\Crefname{Remark}{Remark}{Remark}
\Crefname{algorithm}{Alg.}{Alg.}

\crefname{section}{Sec.}{Secs.}
\Crefname{section}{Sec.}{Secs.}
\crefname{equation}{Eq.}{Eqs.}
\Crefname{equation}{Eq.}{Eqs.}
\crefname{figure}{Fig.}{Figs.}
\Crefname{figure}{Fig.}{Figs.}
\crefname{table}{Tab.}{Tabs.}
\Crefname{table}{Tab.}{Tabs.}
\crefname{thm}{Thm.}{Thms.}
\Crefname{thm}{Thm.}{Thms.}
\crefname{conj}{Conj.}{Conjs.}
\Crefname{conj}{Conj.}{Conjs.}
\crefname{dfn}{Dfn.}{Dfns.}
\crefname{dfn}{Dfn.}{Dfns.}
\crefname{remark}{remark}{remarks}
\Crefname{Remark}{Remark}{Remarks}
\crefname{prop}{Prop.}{Prop.}
\Crefname{prop}{Prop.}{Prop.}
\Crefname{algorithm}{Alg.}{Alg.}
\crefname{appendix}{App.}{Apps.}
\Crefname{appendix}{App.}{Apps.}
%\crefname{app}{appendix}{appendix}
\crefname{appsec}{appendix}{appendices}
\Crefname{appsec}{Appendix}{Appendices}



\renewcommand{\paragraph}[1]{{\vspace{0.3mm}\noindent \bf #1}.}
\newcommand{\paragraphnoper}[1]{{\vspace{0.3mm}\noindent \bf #1}}


% \newcommand{\deq}{\overset{\triangle}{=}} % in case \triangleq does not work
\newcommand{\norm}[1]{\left\|#1\right\|}
\newcommand{\Fi}[1]{\textbf{#1}}
\newcommand{\Se}[1]{\underline{#1}}

\newcommand*\diff{\mathop{}\!\mathrm{d}}
\newcommand{\dt}{\diff{t}}

\newcommand\numberthis{\addtocounter{equation}{1}\tag{\theequation}}


\newcommand{\eg}{\textit{e}.\textit{g}.}
\newcommand{\etal}{\textit{et al}.}
\newcommand{\ie}{\textit{i}.\textit{e}.}
\newcommand{\cf}{\textit{cf}.} 

\newcommand{\tolga}[1]{\textcolor{magenta}{T: \string#1}}


% Todonotes is useful during development; simply uncomment the next line
%    and comment out the line below the next line to turn off comments
%\usepackage[disable,textsize=tiny]{todonotes}
\usepackage[textsize=tiny]{todonotes}


% The \icmltitle you define below is probably too long as a header.
% Therefore, a short form for the running title is supplied here:
% \icmltitlerunning{Higher-order Structures Guided Diffusion for Graph Generation}
\icmltitlerunning{HOG-Diff: Higher-Order Guided Diffusion for Graph Generation}


\begin{document}

\twocolumn[
% \icmltitle{Higher-order Structures Guided Diffusion for Graph Generation}
\icmltitle{HOG-Diff: Higher-Order Guided Diffusion for Graph Generation}

% It is OKAY to include author information, even for blind
% submissions: the style file will automatically remove it for you
% unless you've provided the [accepted] option to the icml2025
% package.

% List of affiliations: The first argument should be a (short)
% identifier you will use later to specify author affiliations
% Academic affiliations should list Department, University, City, Region, Country
% Industry affiliations should list Company, City, Region, Country

% You can specify symbols, otherwise they are numbered in order.
% Ideally, you should not use this facility. Affiliations will be numbered
% in order of appearance and this is the preferred way.
%\icmlsetsymbol{equal}{*}

\begin{icmlauthorlist}
\icmlauthor{Yiming Huang}{IC}
\icmlauthor{Tolga Birdal}{IC}
\end{icmlauthorlist}

\icmlaffiliation{IC}{Imperial College London, London, United Kingdom}
% \icmlaffiliation{comp}{Company Name, Location, Country}
% \icmlaffiliation{sch}{School of ZZZ, Institute of WWW, Location, Country}

%\icmlcorrespondingauthor{Yiming Huang}{y.huang24@imperial.ac.uk}
\icmlcorrespondingauthor{Tolga Birdal}{t.birdal@imperial.ac.uk}


% You may provide any keywords that you
% find helpful for describing your paper; these are used to populate
% the "keywords" metadata in the PDF but will not be shown in the document
\icmlkeywords{Graph Generation, Higher-order Networks, Score-based Generative Model, Topological Deep Learning}

\vskip 0.3in
]

% this must go after the closing bracket ] following \twocolumn[ ...

% This command actually creates the footnote in the first column
% listing the affiliations and the copyright notice.
% The command takes one argument, which is text to display at the start of the footnote.
% The \icmlEqualContribution command is standard text for equal contribution.
% Remove it (just {}) if you do not need this facility.

\printAffiliationsAndNotice{}  % leave blank if no need to mention equal contribution
%\printAffiliationsAndNotice{\icmlEqualContribution} % otherwise use the standard text.

The escalating challenges of managing vast sensor-generated data, particularly in audio applications, necessitate innovative solutions. Current systems face significant computational and storage demands, especially in real-time applications like gunshot detection systems (GSDS), and the proliferation of edge sensors exacerbates these issues. This paper proposes a groundbreaking approach with a near-sensor model tailored for intelligent audio-sensing frameworks. Utilizing a Fast Fourier Transform (FFT) module, convolutional neural network (CNN) layers, and HyperDimensional Computing (HDC), our model excels in low-energy, rapid inference, and online learning. It is highly adaptable for efficient ASIC design implementation, offering superior energy efficiency compared to conventional embedded CPUs or GPUs, and is compatible with the trend of shrinking microphone sensor sizes. Comprehensive evaluations at both software and hardware levels underscore the model's efficacy. Software assessments through detailed ROC curve analysis revealed a delicate balance between energy conservation and quality loss, achieving up to 82.1\% energy savings with only 1.39\% quality loss. Hardware evaluations highlight the model's commendable energy efficiency when implemented via ASIC design, especially with the Google Edge TPU, showcasing its superiority over prevalent embedded CPUs and GPUs.
\section{Introduction}

\begin{figure}
%\vspace{-0.1in}
\centering
\includegraphics[width=0.99\linewidth]{figs/framework.pdf}
\caption{\textbf{Overivew of HOG-Diff.} The dashed line above illustrates the classical generation process, where graphs quickly degrade into random structures with uniformly distributed entries. In contrast, as shown in the coloured region below, HOG-Diff adopts a coarse-to-fine generation curriculum based on the diffusion bridge, explicitly learning higher-order structures during intermediate steps with theoretically guaranteed performance.}
\label{fig:framework}
\vspace{-4mm}
\end{figure}


%%%%% ----- 1.1 what are graphs? 介绍什么是图,
Graphs provide an elegant abstraction for representing complex empirical phenomena by encoding entities as vertices and their relationships as edges, thereby transforming unstructured data into analyzable representations.
%
%%%%% ----- 1.2 The meaning of  Graph Generation 研究意义。有什么用?
Modelling the underlying distribution of graph-structured data is a crucial yet challenging task with broad applications, including social network analysis, motion synthesis, drug discovery, protein design, and urban planning~\cite{zhu2022survey}.
%
%%%%% ----- 1.3 Traditional models 
The study of graph generation seeks to synthesize graphs that align with the observed distribution and traces back to seminal models of random network models \cite{ER1960, BA1999}.
While these models offer foundational insights, they are often too simplistic to capture the complexity of graph distributions we encounter in practice.


%%%%% ----- 2. Recent studies about graph generative models (深度生成模型)
Recently, advances in generative models have leveraged the power of deep neural networks to significantly improve the ability to learn graph distributions.
Notable approaches include models based on recurrent neural networks (RNNs) \cite{GraphRNN2018},  variational autoencoders (VAEs) \cite{VAE-Jin2018}, and generative adversarial networks (GANs) \cite{GAN1-MolGAN, GAN2-Spectre}.
However, the end-to-end structure of these methods makes them hard to train.
%
%% Diffusion-based generative models
More recently, diffusion-based models have achieved remarkable success in image generation by learning a model to denoise a noisy sample \cite{DDPM+NeurIPS2020, Score-SDE+ICLR2021}. 
%
With the advent of diffusion models, their applications on graphs with complex topological structural properties have recently aroused significant scientific interest \cite{EDPGNN-2020,GDSS+ICML2022,DiGress+ICLR2023}.




%%%%% ----- 3. The difference in graph generation 
%%%%% This part gives the motivation for our work
Despite these advances, existing graph generative models typically inherit the frameworks designed for image generation \cite{Score-SDE+ICLR2021}, which fundamentally limits their ability to capture the intrinsic topological properties of networks. 
%
%-----  3.1 higher-order structures are crucial for graphs 
Notably, networks exhibit higher-order structures, such as motifs, simplices, and cells, which capture multi-way interactions and critical topological dependencies beyond pairwise relationships \cite{HigherOrderReview2020,ISMnet2024,TDL-position+ICML2024}.
These structures are vital for representing complex phenomena in domains like molecular graphs, social networks, and protein interactions.
However, current methods are ineffective at modelling the topological properties of higher-order systems since \emph{learning to denoise the noisy samples does not explicitly preserve the intricate structural dependencies required for generating realistic graphs}.


% 
Moreover, the image corrupted by Gaussian noise retains recognizable numerical patterns during the early and middle stages of forward diffusion. By contrast, the graph adjacency matrix quickly degrades into a dense matrix with uniformly distributed entries within a few diffusion steps. 
%
% 3.3 
In addition, directly applying diffusion-based generative models to graph topology generation by injecting isotropic Gaussian noise to adjacency matrices is harmful as it destroys critical graph properties such as sparsity and connectivity.
%
%
%% 3.4 Permutation equivalence.  
Lastly, such a framework should ensure equivariance\footnote{invariance as a particular special case}, maintaining the learned distribution despite node index permutations, which is essential for robustness and capturing intrinsic graph distribution.
%
%
Therefore, a graph-friendly diffusion process should also retain meaningful intermediate states and trajectories, avoid inappropriate noise addition, and ensure equivariance.




% 4.3 Introduction to our framework
% Coarse-to-fine generation
Motivated by these principles and advances in \emph{topological deep learning}~\cite{hajij2022topological,TDL-position+ICML2024}, we propose the \textbf{Higher-order Guided Diffusion} (HOG-Diff) framework, illustrated in \cref{fig:framework}, to address the gaps in graph generation. 
HOG-Diff introduces a coarse-to-fine generation curriculum that enhances the model’s ability to capture complex graph properties by preserving higher-order topologies throughout the diffusion process.
% 
Specifically, we decompose the graph generation task into manageable sub-tasks, beginning by generating higher-order graph skeletons that capture core structures, which are then refined to include pairwise interactions and finer details, resulting in complete graphs with both topological and semantic fidelity.
%
Additionally, HOG-Diff integrates diffusion bridge and spectral diffusion to ensure effective generation and adherence to the aforementioned graph generation principles. 
%
Our theoretical analysis reveals that HOG-Diff converges more rapidly in score matching and achieves sharper reconstruction error bounds than classical approaches, offering strong theoretical support for the proposed framework.
Furthermore, our framework promises to enhance interpretability by enabling the analysis of different topological guides’ performance in the generation process.
%
%
%%%%% ----- 5. Our Contributions
The contributions of this paper are threefold:
\vspace{-0.1in}
\begin{itemize}[noitemsep, parsep=0.3pt, leftmargin=*]
%\begin{itemize}[leftmargin=*]
\item \textbf{Algorithmic}: we introduce a coarse-to-fine graph generation curriculum guided by higher-order topological information using the OU diffusion bridge. 
\item \textbf{Theoretical}: our analysis reveals that HOG-Diff achieves faster convergence during score-matching and a sharper reconstruction error bound compared to classical methods.
\item \textbf{Experimental}: extensive evaluations show that HOG-Diff achieves state-of-the-art graph generation performance across various datasets, highlighting the functional importance of topological guidance.\vspace{-2mm}
\end{itemize}


\section{Preliminaries}
\label{sec:prelim}
\subsection{Notations}
\label{ssec:notation}
The set $\{1,2,\ldots,x\}$ is denoted as $[x]$.
We consider $\graph = (\vertexset,\edgeset)$ to be a simple, unweighted, undirected graph with $\size{\vertexset} = \vertexcount$, and $\size{\edgeset} = \edgecount$. Given a vertex $\vertex$, its neighboring vertex set is denoted as $\neighbour(\vertex) = \set{\altvertex|(\altvertex,\vertex)\in \edgeset}$. We denote by $\degree{\vertex}$ the degree of the vertex $\vertex$. Based on the degrees of the two vertices of an edge $\edge = \fbrac{\vertex,\altvertex}$, we define the degree of the edge $\edge$ as $\degree{\edge} = \min\fbrac{\degree{\vertex}~,\degree{\altvertex}}$. We denote the set of triangles in $\graph$ as $\triangleset$, and individual triangles are denoted as $\triangle$. ($\fbrac{\vertex,\edge}$ denotes a triangle formed by the vertices $\vertex$ and the endpoints of the edge $\edge$). We want to estimate the number of triangles,  $\size{\triangleset} = \numtriangle$ in the graph given the $\degreeq$, $\neighbourq$, $\edgeexistsq$ and $\randedgeq$ queries. An edge $\edge$ participates in a triangle $\triangle$ means that the triangle $\triangle$ is incident on the edge $\edge$. We denote by $\numtriangle_\edge$ the number of triangles the edge $\edge$ participates in. $\uniform(S)$ denotes an element of $S$ is chosen uniformly at random. 

% \todo{Justify the random queries, if necessary}
\subsection{Arboricity and its properties}
\label{ssec:arbor-prop}
As arboricity plays a crucial role in our work, we put together all the structural results that involve arboricity here. Let us restate the definition once more. 
\begin{definition}[Arboricity$(\arboricity)$]
   The arboricity of a graph $\graph = (\vertexset,\edgeset)$, denoted by $\arboricitygraph{G}$, is the minimum number of spanning forests that $\edgeset$ can be partitioned into.
   \label{def:arboricity}
\end{definition}
The arboricity of a graph can be seen as a measure of the density of the graph. $\arboricitygraph{G}$ can be at least $\left\lceil m/(n-1)\right\rceil$. Also, $\arboricitygraph{G} \geq \arboricitygraph{H}$ where $H$ is any subgraph of $G$. We will write $\arboricity$ instead of $\arboricitygraph{G}$ when the underlying graph is understood. We introduce the following lemma due to~\citep{DBLP:journals/siamcomp/ChibaN85} on the sum of edge degrees over all  edges in the graph.
\begin{lemma}(~\citep{DBLP:journals/siamcomp/ChibaN85})
\label{Lemma: deg(e) sum is m * arboricity}
     Given a graph $\graph = (\vertexset,\edgeset)$ with arboricity $\arboricity$ and $\size{\edgeset} = \edgecount$,  $\sum\limits_{\edge \in \edgeset} \degree{\edge} = 2\edgecount\arboricity$.
\end{lemma}

The following lemma due to~\citep{DBLP:conf/soda/EdenRS20} builds on the work of~\citep{DBLP:journals/siamcomp/ChibaN85} to bound the number of triangles based on the number of edges $\edgecount$ and arboricity $\arboricity$. 
\begin{lemma}[Triangle Upper Bound ~\citep{DBLP:conf/soda/EdenRS20}]
\label{lemma: arboricity triangle bound}
    Given a graph $\graph = (\vertexset,\edgeset)$ with arboricity $\arboricity$ and $\size{\edgeset} = \edgecount$, the graph $\graph$ has at most $\edgecount\arboricity$ triangles.
\end{lemma}
Note that this upper bound is also tight, i.e., there exists graphs that contain $\edgecount$ edges and $\bigomega{\edgecount\arboricity}$ triangles. Additionally, arboricity $\arboricity$ can be at most $\bigo{\sqrt{\edgecount}}$. Thus all our results can be reformulated by plugging in this upper bound. 


\ifarxiv{
\subsection{Chernoff Bounds}
We will be using the following variation of the Chernoff bound that bounds the deviation of the sum of independent Poisson trials~\citep{Mitzenmacher_Upfal_2005}.

\begin{lemma}[Multiplicative Chernoff Bound]\label{Lemma: Multiplicative Chernoff Bound}
    Given i.i.d. random variables $X_1,X_2,...,X_t$ where $\Pr[X_i = 1] = p$ and $\Pr[X_i = 0] = (1-p)$, define $X = \sum_{i \in [t]} X_i$. Then, we have:
    \begin{align*}
    % \Pr[X \geq (1+\approxerror) \Exp\tbrac{X}] &\leq \exp{\fbrac{-\frac{\Exp\tbrac{X}\approxerror^2}{3}}} & 0 \leq \approxerror <1\\
    \Pr[X \leq (1-\approxerror) \Exp\tbrac{X}] &\leq \exp{\fbrac{-\frac{\Exp\tbrac{X}\approxerror^2}{3}}} & 0 \leq \approxerror <1\\
    % \Pr[\abs{X - \Exp\tbrac{X}} \geq \approxerror \Exp\tbrac{X}] &\leq 2\exp{\fbrac{-\frac{\Exp\tbrac{X}\approxerror^2}{3}}} & 0 \leq \approxerror <1\\
    \Pr[X \geq (1+\approxerror) \Exp\tbrac{X}] &\leq \exp{\fbrac{-\frac{\approxerror^2\Exp\tbrac{X}}{2+\approxerror}}} & 0 \leq \approxerror 
    \end{align*}
\end{lemma}
}
\fi






%-----------------------------Notations-----------------------


\section{Methods}

\begin{figure*}[t]
  \centering
  \includegraphics[width=\textwidth]{figures/LaVCa_method.pdf}
  \vskip 0.1in % キャプション前のスペース
  % \vspace{0.1in}
  \caption{Architecture of LaVCa. (a) We construct a voxel-wise encoding model for a human subject’s brain activity data (measured using fMRI) while viewing images, using CLIP -Vision latent representations. The encoding weight is obtained through ridge regression. (b) We identify the optimal images for a given voxel by calculating the inner product between the CLIP-Vision latent representations of external image datasets and the voxel’s trained encoding weight, selecting the top-N images (the ``optimal image set'') that produce the highest predicted activation. (c) Next, we use a Multimodal LLM (MLLM) to generate captions for each optimal image set, allowing an LLM to interpret them. (d) Finally, we prompt an LLM to extract keywords from the captions, filter these keywords, and feed them into a ``Sentence Composer,'' producing a concise voxel caption.}
  \label{methods:fig1}
    % \vskip -0.1in % キャプション前のスペース
\end{figure*}

\subsection{fMRI dataset}
This study uses the Natural Scenes Dataset (NSD)~\cite{allen2022massive} following the same experimental conditions as in BrainSCUBA. The NSD consists of data collected over 30 to 40 sessions using a 7 Tesla fMRI scanner, with each participant viewing 10,000 images, repeated three times. We analyze data from the four participants (Subject~01, Subject~02, Subject~05, and Subject~07) who completed all imaging sessions.
The images and captions used in NSD are drawn from MS~COCO and resized to $224 \times 224$ pixels to align with the input requirements of the vision models used. We average the brain activity data for each subject across repeated trials of the same image to improve the signal-to-noise ratio. Up to 9,000 images per subject are used as training data, and the remaining 1,000 images are reserved for testing. We use the preprocessed scans with a resolution of $1.8\,\mathrm{mm}$ provided by NSD for the functional data. We use single-trial beta weights estimated via a generalized linear model (GLM) within ROIs. Moreover, we standardize the response of each voxel to have a mean of zero and a variance of one within each session. We use the ROIs provided by NSD, which include early and higher-level (ventral) visual areas and face, place, body, and word-selective regions.


%-------------------------------------------------------------------------

\subsection{LLM-assisted Visual Cortex Captioning (LaVCa)}
We propose a method, \textbf{LaVCa (LLM-assisted Visual Cortex Captioning)}, to automatically generate data-driven natural language captions that characterize each voxel’s selectivity in the visual cortex. LaVCa consists of four stages (Figure \ref{methods:fig1}): 

\begin{enumerate}
    \item Construct voxel-wise encoding models for each subject while they view natural images.
    \item Identify the optimal image set by finding the top-$N$ images that most strongly activate each voxel (according to the trained encoding models).
    \item Generate captions for these optimal images using a Multimodal LLM (MLLM) for summarization by an LLM in the next step.
    \item Derive concise voxel captions by extracting and filtering keywords from the image captions, then feeding these keywords into a ``Sentence Composer.''
\end{enumerate}


\subsubsection{Encoding Model Construction}
First, we construct voxel-wise encoding models to predict each voxel’s activity in response to natural images (Figure \ref{methods:fig1}a). For ease of comparison with BrainSCUBA, we use the projection layer of CLIP’s vision branch \cite{radford2021learning}, using the same pretrained checkpoint used in BrainSCUBA (\ref{appendix:pretrained_ckpts} for details on pretrained model checkpoints). Hereafter, we refer to CLIP's vision branch as ``CLIP-Vision''. Because the code for BrainSCUBA is not publicly available, we implement it in-house. Both the dataset used for the softmax projection and the training approach for the encoding model differ from those in the original paper (\ref{appendix:brainscuba} for details). We estimate the encoding model weights using L2-regularized linear regression on the NSD training set.

\subsubsection{Exploration of Optimal Image Sets for Voxels}
Next, we identify the optimal image set for each voxel (Figure \ref{methods:fig1}b). We compute the inner product between the voxel’s encoding weight and CLIP-Vision latent representations from a large-scale external dataset (distinct from NSD) to obtain predicted voxel responses for each image. We then select the top-$N$ images that generate the highest predicted activation. This process is equivalent to calculating the predicted responses of each voxel for every image. This study uses approximately 1.7 million images from OpenImages-v6~\cite{OpenImages} (the same dataset used in our in-house BrainSCUBA).

\subsubsection{Captioning Optimal Image Sets with MLLM}
To enable an LLM to interpret each voxel’s optimal image set, we first generate captions for these image sets using an MLLM. We use MiniCPM-V~\cite{yao2024minicpmv} with the prompt \textit{``Describe the image briefly.''} For our accuracy evaluation, we also form a simple baseline by concatenating the top-$N$ captions from the optimal image set. 

\begin{figure*}[t]
  \centering
  \includegraphics[width=\textwidth]{figures/compare_sim_inflated.pdf}
  \vskip 0.1in
  \caption{Mapping of brain activity prediction accuracy (subj01). (a) The sentence-level prediction performance is projected onto inflated cortical surfaces
  (top: lateral, medial, and dorsal views) and flattened cortical surfaces (bottom, with the occipital areas at the center) for both hemispheres. Voxels with significant prediction performance are color-coded (all colored voxels $P<0.05$, FDR corrected). The white outlines indicate the ROIs that are among the top two in terms of the total voxel count across subjects for each semantic category—Body (Extra Striate Body Area; EBA, and Fusiform Body Area; FBA-2), Face (Fusiform Face Area; FFA-1, and Occipital Face Area; OFA), and Places (Parahippocampal Place Area; PPA, and Occipital Place Area; OPA). Word areas are shown in Figure \ref{appendix:sentence_cc_flatmap}. (b) A comparison of sentence-level prediction performance between our method, LaVCa, and the existing method, BrainSCUBA on the flattened cortical surface. If only one model exhibits significant prediction performance for a given voxel, the other model's performance for that voxel is set to zero and color-coded accordingly.}
  \label{results:accuracy_mapping}
  % \vskip -0.05in
\end{figure*}

\subsubsection{Generating Voxel Captions}
Finally, we generate interpretable voxel captions from the image captions. First, we use an LLM to extract common keywords across the captions within each voxel’s optimal image set (Figure~\ref{methods:fig1}d). Following the in-context learning prompt approach from~\cite{dunlap2024describing}, we extract multiple keywords from the caption sets using an LLM (\ref{appendix:prompt} for the prompt). We use \textit{gpt-4o} (gpt-4o-2024–08–06 in the OpenAI API) as the LLM. In \ref{appendix:ablation}, we investigate the effects of several hyperparameters on accuracy, including the number of images in the optimal image sets, the number of extracted keywords, and the type of keywords extractor.
To remove irrelevant or noisy keywords, we compute the cosine similarity between each keyword’s embedding from CLIP' text branch (prompted as \textit{``A photo of \{keyword\}.''}) and the encoding weight for that voxel, then apply a softmax threshold to retain only sufficiently relevant keywords. Hereafter, we refer to CLIP's text branch as ``CLIP-Text''.
Next, we transform these filtered keywords into a sentence-level caption using the ``Sentence Composer'' from MeaCap~\cite{zeng2024meacap}, initially designed to generate image captions from keyword sets. 
MeaCap can generate a caption by inputting the target image’s keywords into the Sentence Composer while referencing similarities to the image features.
In this study, we replace image features with encoding weights so that the model composes a coherent sentence from the voxel-specific keywords (for details, see Section~\ref{appendix:meacap}).



\subsection{Caption Evaluation}

\subsubsection{Brain Activity Prediction at Sentence Level}
We predict brain activity based on sentence similarity to assess how accurately voxel captions describe voxel selectivity (Figure \ref{appendix:voxel_pred_method}a). We hypothesize that a voxel caption capturing the true selectivity of a voxel should be more similar to the corresponding caption of an NSD image that strongly activates that voxel and less similar otherwise. Following~\cite{singh2023explainingblackboxtext}, we:
\begin{enumerate}
    \item Use a pretrained Sentence-BERT to compute text embeddings for each voxel caption and each NSD image caption.
    \item Compute the cosine similarity between the voxel caption embedding and each NSD image caption embedding.
    \item Treat this similarity value as the predicted activity for that voxel on that image.
\end{enumerate}
We then calculate Spearman’s rank correlation coefficient ($\rho$) between these predicted values (sentence similarities) and the actual voxel activities. Statistical significance (one-tailed) is determined by comparing the observed correlation to a null distribution of correlations from two independent Gaussian random vectors of the same length, using a threshold of \textit{P}~$<$~0.05, followed by False Discovery Rate (FDR) correction.


\subsubsection{Brain Activity Prediction at Image Level}
Because sentence-based evaluation can be influenced by non-visual linguistic features (e.g., sentence length), we also assess voxel selectivity using \textit{image} similarity (Figure \ref{appendix:voxel_pred_method}b). We use FLUX.1-schnell to create a \textit{voxel image} and then compute vision embeddings (via CLIP-Vision) for both the generated voxel image and each NSD trial image. We obtain an image-level metric of predicted brain activity by comparing these embeddings, focusing purely on visual content.
% \section{Experiments}

\section{Analysis}

\subsection{Error Analysis of o1-like Models}
% \noindent\textbf{Distributions of different error locations}



\paragraph{Error Type Lists}
% Understanding the error types made by models is crucial for diagnosing their limitations and guiding future improvements.
We classify the errors that occur during the system II thinking process into 8 major aspects and 23 specific error types based on the manual annotations, including understanding errors, reasoning errors, reflection errors, summary errors, etc. For detailed information about the error categories, see Appendix \ref{app: error_classification}.

\paragraph{What Are the Most Common Errors Across Domains?}

\begin{figure}[t]
    \centering
    \resizebox{1.0\textwidth}{!}
    {\includegraphics{figures/error_type_distribution.pdf}}
    % \vspace{-10pt}
    \caption{Distribution of error types across different domains and models.}
    % \vspace{-3mm}
    \label{fig: error_type}
\end{figure}

To analyze the characteristics of error distribution in different domains, we performed a uniform sampling of the data based on the model, the domain, and the query difficulty. Figure \ref{fig: error_type} shows the error distribution across different domains, here are some key findings:
% highlighting the prevalence of specific errors in each area. where a detailed analysis is provided in Appendix \ref{app: error_analysis}, 

\begin{itemize}[left=1em]
\item \textbf{Math:} The most frequent error type is \textit{Reasoning Error}(25.3\%), followed by \textit{Understanding Error}(15.7\%) and \textit{Calculation Error}(15.4\%). This indicates that while the models often struggle with logical reasoning and problem understanding, low-level computational mistakes also remain a significant issue.

\item \textbf{Programming}: 
\textit{Reasoning Error} (21.5\%) is the most common, followed by \textit{Formal Error} (16.7\%) and \textit{Understanding Error} (12.6\%). The high frequency of \textit{Formal Error} and \textit{Programming Error} (11.8\%) underscores the models' struggles with code-specific details and implementation. 

\item \textbf{PCB}: 
The dominant error types are \textit{Understanding Error} (20.4\%) and \textit{Knowledge Error} (17.3\%), closely followed by \textit{Reasoning Error} (17.3\%). This suggests that the main challenge for current models in the fields of physics, chemistry and biology is to understand field-specific concepts and accurately apply relevant knowledge.

\item \textbf{General Reasoning}: \textit{Reasoning Error} is the most prevalent, accounting for 43\%, followed by comprehension errors, accounting for 19\%, showing that logical reasoning is the primary bottleneck.

\end{itemize}

\paragraph{What Are the Model-Specific Error Patterns?}

% \begin{figure}[t]
%     \centering
%     \includegraphics[width=0.8\textwidth]{figures/error_type_model.pdf}
%     % \vspace{-3mm}
%     \caption{Distribution of Error Types Across Models.}
%     % \vspace{-3mm}
%     \label{fig: error_type_model}
% \end{figure}

We also analyzed errors specific to individual models, providing further insights into model weaknesses, as illustrated in Figure \ref{fig: error_type_model}. The error distributions reveal distinct patterns for each model, highlighting their unique strengths and areas for improvement. Here are some key findings:
%Due to space constraints, we focus here on the key findings from the most commonly used models, with a comprehensive analysis of all models provided in Appendix \ref{app: error_analysis}.

\begin{itemize}[leftmargin=4mm]

\item \textbf{DeepSeek-R1} exhibits its most pronounced weakness in \textit{Reasoning Errors} (22.7\%), indicating challenges in constructing coherent and accurate logical reasoning paths. However, it demonstrates relative strength in handling fundamental tasks, with minimal \textit{Calculation Errors} (3.1\%) and \textit{Programming Errors} (4.4\%).

%achieves strong performance in detail-oriented tasks such as formula computation and code syntax. Its primary limitation lies in reasoning and comprehension capabilities.

\item \textbf{QwQ-32B-Preview} excels at identifying correct problem-solving approaches. However, its effectiveness is significantly hindered by deficiencies in handling finer details, particularly in \textit{Calculation Errors} (17.9\%)

%but its effectiveness is often undermined by deficiencies in handling finer details.

% {QwQ-32B-Preview} demonstrates a relatively balanced performance but is notably weak in \textit{Calculation Errors} (17.9\%), indicating a significant limitation in numerical precision. It also shows a moderate frequency of \textit{Understanding Errors} (17.1\%), suggesting occasional difficulties in problem interpretation. 

\end{itemize}

\begin{tcolorbox}[colback=white!95!gray, colframe=gray!70!black,  title=Key Finding for Error Type]
The primary bottleneck of current models remains reasoning ability. However, detailed errors like calculation and formal mistakes also contribute significantly.
\end{tcolorbox}


\subsection{Reflection Analysis of o1-like Models}


\begin{figure}[t]
    \centering
    \includegraphics[width=0.95\textwidth]{figures/reflection.pdf}
    \caption{Distribution of effective reflection times by models and domains on a sample level. The segments within each pie chart represent how many times effective reflection occurs in one sample, with segment `0' indicating there is no effective reflection.}
    \label{fig: error_type_model}
\end{figure}

\paragraph{Statistics.}
We also conduct a analysis of the total number of reflections and the proportion of effective reflections in the long CoT output of all questions (including questions answered correctly and incorrectly by the model). 
% On average, 
%We observe that the long CoT contains \textit{five} times reflections, indicating that current o1-like models tend to reflect frequently. 

\paragraph{How Effective Are Model Reflections Across Different Models and Domains?}
We classify samples with reflections based on the number of valid reflections to evaluate the ability to produce valid reflections. Specifically, we label samples as \texttt{0} if no valid reflections occur, and \texttt{1}, \texttt{2}, or \texttt{>=3} for samples with one, two, or three and more valid reflections, respectively(all statistical analyses were performed under strictly controlled conditions, ensuring uniform sampling and balanced tasks for a fair comparison). In Figure \ref{fig: error_type_model}, {DeepSeek-R1} exhibits the highest proportion of effective reflections, and the models show a notably higher rate of effective reflections in the {math} domain. However, the overall proportion of valid reflections across all models remains relatively low, ranging between 30\% and 40\%. This suggests that the reflection capabilities of current models require further improvement.
%Detailed statistical data can be found in Appendix D.

\begin{tcolorbox}[colback=white!95!gray, colframe=gray!70!black,  title=Key Finding for Reflection]
Despite frequent reflection attempts, the proportion of effective reflections remains low across models, and  DeepSeek-R1 achieves the highest rate of valid reflections.
\end{tcolorbox}

\subsection{Effective Reasoning of o1-like Models}

\begin{figure}[t]
    \centering
    \includegraphics[width=0.98\textwidth]{figures/effetive_reasoning.pdf}
    \caption{Distribution of effective reasoning ratios.}
    
    \label{fig: effetive_reasoning}
\end{figure}

\paragraph{Statistics.} 
% As previously mentioned, 
Human annotators evaluate the usefulness of the reasoning in each section, enabling us to calculate the proportion of valid reasoning in each response. As illustrated in Figure \ref{fig: effetive_reasoning}, each graph shows the distribution of effective reasoning ratios for a particular model. The red dashed line in each graph indicates the average effective reasoning ratio.

\paragraph{What Proportion of Reasoning in Long CoT Responses is Effective?}
On average, only 73\% of the reasoning in the collected long CoT responses is useful, highlighting significant redundancy issues. Among the models analyzed, \textit{QwQ-32B-Preview} exhibited the lowest proportion of effective reasoning at 70\%, while \textit{DeepSeek-R1} achieved a notably higher proportion compared to the others, demonstrating superior reasoning efficiency.


\begin{tcolorbox}[colback=white!95!gray, colframe=gray!70!black,  title=Key Finding for Reasoning Efficiency]
On average, 27\% of reasoning in long CoT responses we collected is redundant, and DeepSeek-R1 outperforms others in reasoning efficiency.
\end{tcolorbox}
\vspace{-3mm}

\subsection{Reasoning Process Analysis}

Figure ~\ref{fig: action_roles} shows the distribution of each section's action roles in the system II thinking process of the o1-like models. Initially, problem analysis dominates, indicating that the model initially focuses on understanding the requirements and constraints of the problem. As the solution progresses, cognitive activities diversify significantly, with reflection and validation becoming more prominent. In the later part of the reasoning, the distribution of conclusion and summarization gradually increases. 
%As the model progresses from problem analysis, solution implementation and conclusion, it demonstrates the common reasoning template of o1-like models.


\begin{figure}[t]
    \centering
    \includegraphics[width=0.8\textwidth]{figures/action_role.pdf}
    \caption{Distribution of different task types throughout the progress of a long CoT response.}
    \vspace{-3mm}
    
    \label{fig: action_roles}
\end{figure}
\subsection{Results on DeltaBench}

% Please add the following required packages to your document preamble:
% \usepackage{multirow}
\begin{table*}[!t]
\centering
\resizebox{1.0\textwidth}{!}{%
    \begin{tabular}{cccccccccccccccc}
    \toprule
    \multirow{2}{*}{\textbf{Model}} & \multicolumn{3}{c}{\textbf{Overall}} & \textbf{Math} & \textbf{Code} & \textbf{PCB} & \textbf{General} \\
    \cmidrule(lr){2-4} \cmidrule(lr){5-5} \cmidrule(lr){6-6} \cmidrule(lr){7-7} \cmidrule(lr){8-8}
     & \textbf{\textit{Recall}} & \textbf{\textit{Precision}} & \textbf{\textit{F1}} & \textbf{\textit{F1}} & \textbf{\textit{F1}} & \textbf{\textit{F1}} & \textbf{\textit{F1}} \\
    \midrule
    \multicolumn{8}{c}{\textbf{\textit{Process Reward Models (PRMs)}}} \\
    \midrule
    \rowcolor[rgb]{ .988,  .949,  .8} Qwen2.5-Math-PRM-7B & \textbf{30.30} & \textbf{34.96} & \textbf{29.22}  &  \textbf{29.64} & \textbf{23.76} & \underline{31.09} & \underline{34.19}   \\
    \rowcolor[rgb]{ .988,  .949,  .8} Qwen2.5-Math-PRM-72B & \underline{28.16} & \underline{29.37} & \underline{26.38}  & \underline{24.16} & \underline{22.02} & \textbf{31.14} & \textbf{35.83}  \\
    \rowcolor[rgb]{ .988,  .949,  .8} Llama3.1-8B-PRM-Deepseek-Data & 11.7 & 15.59 & 12.02 &  12.28 & 10.95 & 16.76 & 12.59  \\
    \rowcolor[rgb]{ .988,  .949,  .8} Llama3.1-8B-PRM-Mistral-Data & 9.64 & 11.21 & 9.45 & 9.40 & 10.72 & 13.43 & 12.40  \\
    \rowcolor[rgb]{ .988,  .949,  .8} Skywork-o1-Qwen-2.5-1.5B & 3.32 & 3.84 & 3.07 & 1.30 & 6.66 & 5.43 & 7.87  \\
    \rowcolor[rgb]{ .988,  .949,  .8} Skywork-o1-Qwen-2.5-7B & 2.49 & 2.22 & 2.17 & 0.78 & 6.28 & 6.02 & 3.11  \\
    \midrule
     \multicolumn{8}{c}{\textbf{\textit{LLM as Critic Models}}} \\
    \midrule
    \rowcolor[rgb]{ .922,  .89,  .988} GPT-4-turbo-128k & \textbf{57.19} & \textbf{37.35} & \textbf{40.76} & \textbf{37.56} & \textbf{43.06} & \underline{45.54} & \underline{42.17} \\
    \rowcolor[rgb]{ .922,  .89,  .988} GPT-4o-mini & \underline{49.88} & 35.37 & \underline{37.82} & \underline{33.26} & 37.95 & \textbf{45.98} & \textbf{46.39} \\
    \rowcolor[rgb]{ .922,  .89,  .988} Doubao-1.5-Pro & 39.68 & \underline{37.02} & 35.25 & 32.46 & \underline{39.47} & 33.53 & 37.00 \\
    \rowcolor[rgb]{ .922,  .89,  .988} GPT-4o & 36.52 & 32.48 & 30.85 & 28.61 & 28.53 & 39.25 & 36.50 \\
    \rowcolor[rgb]{ .922,  .89,  .988} Qwen2.5-Max & 36.11 & 30.82 & 30.49 & 26.73 & 32.81 & 39.49 & 29.54 \\
    \rowcolor[rgb]{ .922,  .89,  .988} Gemini-1.5-pro & 35.51 & 30.32 & 29.59 & 26.56 & 28.20 & 40.13 & 33.66 \\
    \rowcolor[rgb]{ .922,  .89,  .988} DeepSeek-V3 & 32.33 & 28.13 & 27.33 & 27.04 & 27.73 & 27.35 & 27.45 \\
    \rowcolor[rgb]{ .922,  .89,  .988} Llama-3.1-70B-Instruct & 32.22 & 28.85 & 27.67 & 21.49 & 32.13 & 28.45 & 39.18 \\
    \rowcolor[rgb]{ .922,  .89,  .988} Qwen2.5-32B-Instruct & 30.12 & 28.63 & 26.73 & 22.34 & 31.37 & 33.78 & 24.37 \\
    \rowcolor[rgb]{ .882,  .949,  .89} DeepSeek-R1 & 29.20 & 32.66 & 28.43 & 24.17 & 29.28 & 34.78 & 35.87 \\
    \rowcolor[rgb]{ .882,  .949,  .89} o1-preview & 27.92 & 30.59 & 26.97 & 22.19 & 28.09 & 33.11 & 35.94 \\
    % Gemini-2.0-flash-thinking & 14.02 & 17.36 & 14.56 & 14.79 & 11.97 & 19.34 & 15.26 \\
    \rowcolor[rgb]{ .922,  .89,  .988} Qwen2.5-14B-Instruct & 26.64 & 27.27 & 24.73 & 21.51 & 29.05 & 29.98 & 20.59 \\
    \rowcolor[rgb]{ .922,  .89,  .988} Llama-3.1-8B-Instruct & 25.71 & 28.01 & 24.91 & 18.12 & 32.17 & 27.30 & 29.93 \\
    \rowcolor[rgb]{ .882,  .949,  .89} o1-mini & 22.90 & 22.90 & 19.89 & 16.71 & 21.70 & 20.37 & 26.94 \\
    \rowcolor[rgb]{ .922,  .89,  .988} Qwen2.5-7B-Instruct & 21.99 & 19.61 & 18.63 & 11.61 & 25.92 & 29.85 & 15.18 \\
    \rowcolor[rgb]{ .882,  .949,  .89} DeepSeek-R1-Distill-Qwen-32B & 17.19 & 18.65 & 16.28 & 13.02 & 23.55 & 15.05 & 11.56 \\
    % Gemini-2.0-flash-thinking & 14.02 & 17.36 & 14.56 & 14.79 & 11.97 & 19.34 & 15.26 \\
    \rowcolor[rgb]{ .882,  .949,  .89} DeepSeek-R1-Distill-Qwen-14B & 12.81 & 14.54 & 12.55 & 9.40 & 18.36 & 10.44 & 12.01 \\
    % \rowcolor[rgb]{ .882,  .949,  .89} QwQ-32B-Preview & 10.20 & 10.17 & 9.07 & 7.38 & 8.60 & 14.97 & 10.54 \\
    \bottomrule
    \end{tabular}
}
\caption{Experimental results of PRMs and critic models on DeltaBench. \textbf{Bold} indicates the best results within the same group of models, while \underline{ underline} indicates the second best.}
% \vspace{-4mm}
\label{tab: main}
\end{table*}

% \noindent\textbf{Evaluation Metrics.}
% % To accurately assess the performance of the PRM and critic models on DeltaBench, 
% We employ \textbf{recall}, \textbf{precision}, and \textbf{macro-F1 score} for error sections as evaluation metrics. For the PRMs, we utilize an outlier detection technique based on the Z-Score to make predictions. This method was chosen because threshold-based prediction methods determined from other step-level datasets, such as those used in ProcessBench~\citep{Zheng2024ProcessBenchIP}, may not be reliable due to significant differences in dataset distributions, particularly as DeltaBench focuses on long CoT. Outlier detection helps to avoid this bias. The threshold $t$ for determining the correctness of a section is defined as:
% % \begin{align}
% $t = \mu - \sigma$,
% % \nonumber
% % \label{eq: prm_threshold}
% % \end{align}
% where $\mu$ is the mean of the rewards distribution across the dataset, and $\sigma$ is the standard deviation. Sections falling below $t$ are predicted as error sections. For critic models, all erroneous sections within a long CoT are prompted to be identified. Given that error sections constitute a smaller proportion than correct sections across the dataset, we use macro-F1 to mitigate the potential impact of the imbalance between positive and negative sections. Macro-F1 independently calculates the F1 score for each sample
% % (for our metric, each case) 
% and then takes the average, providing a more balanced evaluation metric when dealing with class imbalance.

\noindent\textbf{Baseline Models.}
% 开源(中英模型,llama3)和闭源模型
% To comprehensively evaluate the performance of current PRMs and critic models, we extensively selected and evaluated a wide range of both open-source and closed-source models on DeltaBench.
% \paragraph{Process Reward Models}
For the \textbf{PRMs}, we select the following models: Qwen2.5-Math-PRM-7B\footnote{\href{https://huggingface.co/Qwen/Qwen2.5-Math-PRM-7B}{Qwen/Qwen2.5-Math-PRM-7B}}, Qwen2.5-Math-PRM-72B\footnote{\href{https://huggingface.co/Qwen/Qwen2.5-Math-PRM-72B}{Qwen/Qwen2.5-Math-PRM-72B}}, Llama3.1-8B-PRM-Deepseek-Data\footnote{\href{https://huggingface.co/RLHFlow/Llama3.1-8B-PRM-Deepseek-Data}{RLHFlow/Llama3.1-8B-PRM-Deepseek-Data}}, Llama3.1 -8B-PRM-Mistral-Data\footnote{\href{https://huggingface.co/RLHFlow/Llama3.1-8B-PRM-Mistral-Data}{RLHFlow/Llama3.1-8B-PRM-Mistral-Data}}, Skywork-o1-Open-PRM- Qwen-2.5-1.5B\footnote{\href{https://huggingface.co/Skywork/Skywork-o1-Open-PRM-Qwen-2.5-1.5B}{Skywork/Skywork-o1-Open-PRM-Qwen-2.5-1.5B}}, and Skywork-o1-Open-PRM-Qwen-2.5-7B\footnote{\href{https://huggingface.co/Skywork/Skywork-o1-Open-PRM-Qwen-2.5-7B}{Skywork/Skywork-o1-Open-PRM-Qwen-2.5-7B}}. 
% These represent some of the best open-source PRMs currently available.
% \paragraph{Critic Models}
We select a group of the most advanced open-source and closed-source LLMs to serve as \textbf{critic models} for evaluation, which includes various GPT-4~\citep{gpt4} variants (such as GPT-4-turbo-128K, GPT-4o-mini, GPT-4o), the Gemini model~\citep{Reid2024Gemini1U}(Gemini-1.5-pro), several Qwen models~\citep{qwen2.5} (such as Qwen2.5-32B-Instruct and Qwen2.5-14B-Instruct), Doubao-1.5-Pro~\citep{doubao2025}
and o1 models~\citep{openai-o1} (o1-preview-0912, o1-mini-0912).
% , and a GPT-3.5 variant (gpt-3.5-16K).



\subsubsection{Main Results}
In Table \ref{tab: main},
we provide the results of different LLMs on DeltaBench. 
For PRMs, we have the following observations: (1). Existing PRMs usually achieve low performance, which indicates that existing PRMs cannot identify the errors in long CoTs effectively and it is necessary to improve the performance of PRMs. (2). Larger PRMs
do not lead to better performance. For example, the Qwen2.5-Math-PRM-72B is inferior to wen2.5-Math-PRM-7B.
For critic models, we have the following findings: (1)
GPT-4-turbo-128k archives the best critique results, which is better than other models (e.g., GPT-4o) a lot in DeltaBench. (2) For o1-like models (e.g., DeepSeek-R1, o1-mini, o1-preview), we observe that the results of these models are not superior to non-o1-like models, with the performance of o1-preview is even lower than Qwen2.5-32B-Instruct.
%Additionally, we observe that the QWQ and DeepSeek-R1-Distill series models exhibit weaknesses in following instructions. 
A detailed analysis of underperforming models is provided in Appendix \ref{app: underperforming}.

% model size
% domains
% o1模型跟普通模型critic能力对比分析


\subsubsection{Further Analysis}

\paragraph{Effect of Long CoT Length.}
\begin{figure}[t]
    \centering
    \includegraphics[width=1.0\textwidth]{figures/4.5.1/length2.pdf}
    \caption{The effect of long CoT length.}
    \label{fig: crtic1}
\end{figure}
In Figure \ref{fig: crtic1}, we compare the average F1-Score performance of critic models and PRMs across varying LongCoT token lengths. 
For critic models, the performance notably declines as token length increases. Initially, models like Deepseek-R1 and GPT-4o exhibit strong performance with shorter sequences (1-3k tokens). However, as token length increases to mid-ranges (4-7k tokens), there is a marked decrease in performance across all models. This trend highlights the growing difficulty for critic models to maintain precision and recall as long CoT response become longer and more complex, likely due to the challenge of evaluating lengthy model outputs. In contrast, PRMs demonstrate greater stability across token lengths, as they evaluate sections sequentially rather than processing the entire output at once. Despite this advantage, PRMs achieve lower overall scores compared to critic models on our evaluation set.

\begin{tcolorbox}[colback=white!95!gray, colframe=gray!70!black, title=Key Finding]
  Critic models exhibit significant performance degradation with longer contexts, while PRMs demonstrate consistent evaluation capability across varying lengths.
\end{tcolorbox}


\paragraph{Performance Analysis Across Different Error Types.}
\begin{figure}[t]
    \centering
    \includegraphics[width=0.9\textwidth]{figures/4.5.2/top_models_per_task.pdf}
    \caption{Results of different LLMs on top-5 errors.}
    \label{fig: top_models_per_task}
\end{figure}
Figure \ref{fig: top_models_per_task} shows the performance of different models on the five most common error types. In terms of error types, most models demonstrate the highest accuracy in recognizing calculation errors. Conversely, the recognition of strategy errors is generally the weakest. In terms of models, there is significant variation in the ability of individual models to recognize different error types. For instance, DeepSeek-V3 achieves an F1 of 36\% on calculation errors but only 23\% on strategy errors. Meanwhile, Llama3.1-8B-PRM-Deepseek performs poorly, with an F1 score of 22\% on calculation errors, and shows a significant decline in performance across the other four error types. This highlights the limited generalization capabilities of most models when recognizing various error types.

\begin{tcolorbox}[colback=white!95!gray, colframe=gray!70!black, title=Key Finding]
  Models exhibit strong performance on calculation errors but struggle with strategy errors, revealing limited generalization across error types.
\end{tcolorbox}

\begin{table}[!ht]
    \centering
    % \scriptsize
    % \footnotesize
    \begin{tabular}{cccc}
    \toprule
        \multirow{2}{*}{Model} & \multicolumn{3}{c}{HitRate@$k$ - Avg(\%)} \\ \cline{2-4}
                           & $k=1$ & $k=3$ & $k=5$ \\ 
                           % \hline
                           \midrule
        Qwen2.5-Math-PRM-7B & \textbf{49.15} & \textbf{69.14} & \textbf{83.14} \\
        Qwen2.5-Math-PRM-72B & \underline{41.13} & \underline{62.70} & \underline{75.73} \\ 
        Llama3.1-8B-PRM-Deepseek-Data & 12.63 & 48.62 & 69.78 \\
        Llama3.1-8B-PRM-Mistral-Data & 8.99 & 42.97 & 65.33 \\
        Skywork-o1-Open-PRM-Qwen-2.5-1.5B & 31.90 & 53.82 & 69.23 \\
        Skywork-o1-Open-PRM-Qwen-2.5-7B & 31.58 & 52.59 & 69.16 \\
        % \hline
        \bottomrule
    \end{tabular}
    \vspace{+3mm}
    \caption{Results of HitRate@$k$. Bold and underlined results indicate the best and the second best.}
    % \vspace{-4mm}
\label{tab: hitrate}
\end{table}

\paragraph{Analysis on HitRate evaluation for PRMs.}

\begin{figure}[t]
    \centering
    \includegraphics[width=\textwidth]{figures/prm_rank.pdf}
    % \vspace{-10pt}
    \caption{Ranking of rewards for the first incorrect section for different PRMs.}
    % \vspace{-3mm}
    \label{fig: prm_rank}
\end{figure}

To better measure the ability of PRMs to identify erroneous sections in long CoTs, we use HitRate@$k$ to evaluate PRMs. Specifically, within a sample, we rank the sections in ascending order based on the rewards given by the PRM, select the smallest $k$ sections, and calculate the recall rate for the erroneous sections among them. Specifically, we define the sorted sections as $S = \{s_1, s_2, \ldots, s_n\}$, with $E$ being the set of erroneous sections. We select the top $k$ sections, denoted as $S_k = \{s_1, s_2, \ldots, s_k\}$. The HitRate@$k$ is  calculated as:
\begin{align}
\text{HitRate@}k = \frac{|S_k \cap E|}{\min(k, |E|)}
% \nonumber
\label{eq: hitrate}
\end{align}
In this formula, $|S_k \cap E|$ indicates the number of erroneous sections identified among the top $k$ sections. This metric reflects the ability of PRMs to effectively identify erroneous sections within the top $k$ candidate sections. In Table \ref{tab: hitrate}, the relative performance rankings among different PRMs are quite similar to the results in Table \ref{tab: main}. Additionally, we observe that for $k=3$ and $k=5$, the performance differences between various PRMs are not particularly significant. However, when $k=1$, the Qwen2.5-Math-PRM-7B shows a clear performance advantage. Figure \ref{fig: prm_rank} illustrates the ranking ability of different PRMs for the first incorrect section within the sample, which is generally consistent with the performance evaluation results of HitRate@k.
% This is because a smaller $k$ value imposes stricter requirements on the PRM's ability to identify errors.

% HitRate@$k$ evaluates the performance of PRMs from the perspective of reward ranking, providing additional evidence for the experimental results and conclusions in Table \ref{tab: main} from a different angle.

\begin{tcolorbox}[colback=white!95!gray, colframe=gray!70!black, title=Key Finding]
  HitRate@k evaluation aligns with the main results, with Qwen2.5-Math-PRM-7B demonstrating superior performance in identifying the first incorrect section.
\end{tcolorbox}


\begin{figure}[t]
    \centering
    \includegraphics[width=0.8\textwidth]{figures/4.5.4/self-critic.pdf}
    % \vspace{-10pt}
    \caption{F1-score comparison of self-critique and cross-model critique abilities for different models.}
    % \vspace{-5mm}
    \label{fig: self-critic}
\end{figure}

\paragraph{Comparative Analysis of Self-Critique Capabilities of LLMs.} We randomly sample queries based on domains and models that generate the long CoT output, followed by a statistical analysis of the model's performance in evaluating its own outputs as well as those of other models. In Figure \ref{fig: self-critic},  Gemini 2.0 Flash Thinking, DeepSeek-R1, and QwQ-32B-Preview show lower self-critique scores compared to their cross-model critique scores, indicating a prevalent deficiency in self-critic abilities. Notably, DeepSeek-R1 exhibits the largest discrepancy, with a 36\% decrease in self-evaluation compared to evaluations of other models. This suggests models' self-critic abilities remain underdeveloped.
% signaling an area that requires improvement.

\begin{tcolorbox}[colback=white!95!gray, colframe=gray!70!black, title=Key Finding]
  LLMs demonstrate weaker self-critique performance compared to cross-model critique, highlighting a fundamental limitation in self-critic capabilities.
\end{tcolorbox}



%%%

% \noindent\textbf{Performance Analysis Across Different Categories}

% \begin{figure}[htbp]
% \centering
% \includegraphics[width=\linewidth]{figures/prm_task_comparison.pdf}
% \caption{Performance of PRMs across different categories (outlier detection).}
% \label{fig: prm_task}
% % \vspace{-0.6cm}
% % \vspace{-4mm}
% \end{figure}


% \noindent\textbf{Performance Variation in Different Lengths of Long CoT}

% \noindent\textbf{Performance Analysis Across Different Error Types}

% \noindent\textbf{Analysis of In-Sample Reward Ranking}


% % \subsection{Evaluation Metrics}

% % \subsection{Main Results}

% % \subsection{Further Analysis}
% \subsection{Analysis on LLM Critics}
%  \textbf{error location}



% \subsubsection{The Performance across different domains}

% \begin{figure}[t]
%     \centering
%     \includegraphics[width=0.5\textwidth]{figures/critic6.pdf}
%     \caption{The score distributions across different domains.}
%     \label{fig: crtic2}
% \end{figure}

% In Figure \ref{fig: crtic2}, we illustrate the F1-score distribution of various large language models (LLMs) across different domains. Analyzing model performance across domains reveals that most models demonstrate stronger critiquing abilities in Physics, Chemistry, Biology, and General Reasoning compared to Mathematics and Programming, indicating higher proficiency in scientific and general reasoning tasks. Meanwhile, the performance of each model varies significantly depending on the domain, reflecting inherent strengths and weaknesses in handling different tasks. For instance, the Gemini-1.5-Pro model achieves an F1-score of 40.1\% in PCB, yet only 26.6\% in Mathematics. These discrepancies underscore challenges in the models' generalization capabilities.






\section{Related Works}
\vspace{-1mm}
We review graph generation methods along with higher-order generation. \cref{app:related} presents a more detailed review.



\paragraph{Deep Generative Models}
Graph generative models make great progress by exploiting the capacity of deep neural networks. These models typically generate nodes and edges either in an autoregressive manner or simultaneously, utilizing techniques such as variational autoencoders (VAE) \cite{VAE-Jin2018,GraphVAE-DrugDiscovery}, recurrent neural networks (RNN) \cite{GraphRNN2018}, normalizing flows \cite{Moflow-SIGKDD2020,GraphAF-ICLR2020,GraphDF-ICML2021}, and generative adversarial networks (GAN) \cite{GAN1-MolGAN,GAN2-Spectre}.

\paragraph{Diffusion-based Graph Generation}
A breakthrough in graph generative models has been marked by the recent progress in diffusion-based generative models \cite{EDPGNN-2020}.
Recent models employ various strategies to enhance the generation of complex graphs, including capturing node-edge dependency \cite{GDSS+ICML2022}, addressing discretization challenges \cite{DiGress+ICLR2023,CDGS+AAAI2023}, exploiting low-to-high frequency generation curriculum \cite{GPrinFlowNet+ACM2024}, and improving computational efficiency through low-rank diffusion processes \cite{GSDM+TPAMI2023}. 
% ------ Diffusion bridge models
Recent studies have also enhanced diffusion-based generative models by incorporating diffusion bridge processes, \ie, processes conditioned on the endpoints \cite{wu2022diffusion,GLAD-ICMLworkshop2024,GruM+ICML2024}.
%
% ---- short conclusion
Despite these advances, existing methods either overlook or inadvertently disrupt higher-order structures during graph generation, or struggle to model the topological properties, as denoising the noisy samples does not explicitly preserve the intricate structural dependencies required for generating realistic graphs.




\paragraph{Higher-order Generative Models}
Generative modelling uses higher-order information mostly in the form of hypergraphs.
Models such as Hygene~\cite{gailhard2024hygene} and HypeBoy~\cite{kim2024hypeboy} aim to generate hypergraphs. Dymond~\cite{zeno2021dymond} focuses on higher-order motifs in dynamic graphs. 
To the best of our knowledge, we are the first to consider higher-order guides for graph generation.
\section{Limitations and Future Work}
The proposed OpenFly platform incorporates various rendering engines/techniques to provide high-quality scenes. Specifically, this is the first attempt to use 3D GS reconstructed scenes to support real-to-sim training and testing, while in the reconstruction of large-scale areas, a few visual artifacts are inevitably present. Future work will focus on exploring more effective reconstruction methods to enhance realism in large-scale scenes. Besides, the proposed OpenFly-Agent is built upon the large VLN model architecture, which is not practical for real-time deployment on UAVs. To address this, future research should focus on developing more efficient architectures and effective quantization techniques. 


\section{Conclusion}
In this work, we present OpenFly, a platform designed for large-scale data collection in aerial Vision-and-Language Navigation (VLN). OpenFly integrates multiple rendering engines and advanced real-to-sim techniques for data generation, enabling efficient collection of diverse, high-quality aerial VLN data. The resulting large-scale dataset comprises 100k trajectories across 18 distinct scenes, spanning a wide range of altitudes and difficulty levels, which is significantly superior than existing ones. Furthermore, we propose OpenFly-Agent, a keyframe-aware aerial navigation model capable of directly predicting flight actions based on observations and language instructions. Extensive experiments validate the effectiveness of the proposed method, and establishing a comprehensive benchmark for future advancements in aerial navigation. 
%The toolchain, dataset, and code will be publicly released, providing a valuable resource for future research in this field.
\clearpage

\section*{Impact Statement}
This paper presents work whose goal is to advance the field of deep generative models. 
%There are many potential societal consequences of our work, none of which we feel must be specifically highlighted here.
Positive applications include generating graph-structured data for scientific discovery and accelerating drug discovery by generating novel molecular structures.
%
However, like other generative technologies, our work could potentially be misused to synthesize harmful molecules, counterfeit social interactions, or deceptive network structures. 

\bibliography{ref}
\bibliographystyle{icml2025}

% WARNING: do not forget to delete the supplementary pages from your submission 
\newpage
\appendix
\onecolumn

% Colors ##########################################################################################
\definecolor{backcolour}{rgb}{0.95,0.95,0.92}
\definecolor{arrowcolor}{RGB}{145,148,138}
\definecolor{weborange}{RGB}{255,165,0}
\definecolor{darkblue}{rgb}{0.0,0.0,0.6}
\definecolor{cyan}{rgb}{0.0,0.6,0.6}
\definecolor{deepred}{rgb}{0.6,0,0}
\definecolor{deepgreen}{rgb}{0,0.5,0}


% PAULA ############################################################################################
\lstdefinelanguage{PAULA}{
  keywords={par, and, or, if, ifrt, in, out, program, variable, parameter},
  emph={matrix_vector_multiplication, mvm},
  emphstyle=\color{deepred},
  comment=[l]{//},
  commentstyle=\color{green!50!black},
  keywordstyle=\color{blue}
}

\lstdefinestyle{my_PAULA_style}{
  language=PAULA,
  backgroundcolor=\color{backcolour},
  breaklines=true,
  xleftmargin=1.5em,
  numbers=left,
  numbersep=0.5em
}

% PAULA (Aufgabenstellung) #########################################################################
\lstdefinestyle{PAULA_style}{
  language=PAULA,
%   backgroundcolor=\color{white},
  breaklines=true
}

% Pseudo Code ######################################################################################
\lstdefinestyle{PseudoCode}{%
  keywords={for, do},
  breaklines=true,
  backgroundcolor=\color{backcolour},
  keywordstyle=\color{blue}
}

% LOG ##############################################################################################
\lstdefinestyle{log}{%
  breaklines=true,
  backgroundcolor=\color{backcolour}
}

% JSON #############################################################################################
\lstdefinelanguage{json}{%
  basicstyle=\normalfont\ttfamily,
  breaklines=true,
  backgroundcolor=\color{backcolour},
  literate=
     *{0}{{{\color{deepred}0}}}{1}
      {1}{{{\color{deepred}1}}}{1}
      {2}{{{\color{deepred}2}}}{1}
      {3}{{{\color{deepred}3}}}{1}
      {4}{{{\color{deepred}4}}}{1}
      {5}{{{\color{deepred}5}}}{1}
      {6}{{{\color{deepred}6}}}{1}
      {7}{{{\color{deepred}7}}}{1}
      {8}{{{\color{deepred}8}}}{1}
      {9}{{{\color{deepred}9}}}{1}
      {:}{{{\color{deepred}{:}}}}{1}
      {,}{{{\color{deepred}{,}}}}{1}
      {\{}{{{\color{darkblue}{\{}}}}{1}
      {\}}{{{\color{darkblue}{\}}}}}{1}
      {[}{{{\color{darkblue}{[}}}}{1}
      {]}{{{\color{darkblue}{]}}}}{1},
}

% C ################################################################################################
\lstdefinestyle{myC}{%
  language=C,
  tabsize=2,
  keywordstyle=\color{blue},
  commentstyle=\color{green!50!black},
  stringstyle=\color{deepgreen},
  backgroundcolor=\color{backcolour},
  breaklines=true,
  breakatwhitespace=true,
  moredelim=[s][\color{red}]{\$\{}{\}},
  postbreak=\mbox{\textcolor{arrowcolor}{$\hookrightarrow$}}
}
\lstdefinestyle{myC2}{
  style=myC,
  keywords=[2]{mapping},
  keywordstyle=[2]\color{cyan},
  emphstyle=\color{deepred}
}



% List prompts, training details, human eval details, etc.


\section{Appendix}
\label{sec:appendix}


\subsection{Qualitative Comparison}
\label{subsec:qualitative_comparison}
\begin{figure*}[t]
    \centering
    \begin{subfigure}{\textwidth}
        \centering
        \includegraphics[width=\linewidth]{fig/qualitative_1_0.png}
    \end{subfigure}
    \hfill
    \vspace{0.2cm}
    \begin{subfigure}{\textwidth}
        \centering
        \includegraphics[width=\linewidth]{fig/qualitative_1_30.png}
    \end{subfigure}
    \begin{subfigure}{\textwidth}
        \centering
        \includegraphics[width=\linewidth]{fig/bar.jpg}
    \end{subfigure}
    \caption{\textbf{Qualitative predictions of a SurroundOcc \cite{wei2023surroundocc} model trained with \method{} on the SurroundOcc-nuScenes \cite{wei2023surroundocc} dataset.} We display the six input camera images (top left), the rendered predictions (bottom left), the BeV ground-truth (top right) and BeV prediction (bottom left). The scene is randomly selected from the validation set and we show predictions at two different timesteps.}
    \label{fig:qualitative_0}
\end{figure*}


\begin{figure*}[t]
    \centering
    \begin{subfigure}{\textwidth}
        \centering
        \includegraphics[width=\linewidth]{fig/qualitative_0_0.png}
    \end{subfigure}
    \hfill
    \vspace{0.2cm}
    \begin{subfigure}{\textwidth}
        \centering
        \includegraphics[width=\linewidth]{fig/qualitative_0_30.png}
    \end{subfigure}
    \begin{subfigure}{\textwidth}
        \centering
        \includegraphics[width=\linewidth]{fig/bar.jpg}
    \end{subfigure}
    \caption{\textbf{Qualitative predictions of a TPVFormer \cite{huang2023tpv} model trained with \method{} on the Occ3d-nuScenes \citep{tian2023occ3d} dataset.} We display the six input camera images (top left), the rendered predictions (bottom left), the BeV ground-truth (top right) and BeV prediction (bottom left). The scene is randomly selected from the validation set and we show predictions at two different timesteps.}
    \label{fig:qualitative_1}
\end{figure*}

Figure \ref{fig:qualitative} illustrates the qualitative comparison of the quality of generated images using \textbf{CoDA} and other approaches.

The Crop method, while useful for localized feature emphasis, often results in the loss of crucial visual details necessary for species identification. For instance, in the Phyllobates Samperi images, cropping removes the black spots on the frog’s skin, which are an essential distinguishing feature. Without these patterns, the cropped images lack key identity cues, potentially leading to misclassification. Similarly, in the Tail-Spot Wrasse images, cropping reduces visibility of the distinct horizontal striping pattern along the fish’s body, making it difficult to recognize key species attributes.

ARMADA is capable of retaining some structural features, but it struggles with precise reproduction due to the limited image editing capabilities. This limitation is particularly evident in its generated images, where critical patterns such as the Phyllobates Samperi's orange stripes are missing. The generated frog appears to have a distorted pattern, failing to fully capture the contrast between black skin and bright orange lines, which are key species identifiers. Similarly, in the case of the Tail-Spot Wrasse, the image generated by ARMADA loses the feature of its vibrant horizontal stripes, leading to a visually inconsistent and less biologically accurate representation.

In contrast, \textbf{CoDA} successfully captures all species-specific features by leveraging contrastive textual and visual attributes through different image generation models. \textbf{CoDA} (SD-3.5) produces a high-fidelity image of Phyllobates Samperi, accurately preserving the orange stripes, dark skin, and black spots. However, slight variations in texture suggest that this model, while effective, may not fully match the real-world skin reflectivity of the species. Meanwhile, \textbf{CoDA} (Recraft V3) generates an even more realistic image, successfully capturing the frog’s signature features with improved color richness and anatomical precision, making it nearly indistinguishable from real-world references.

Generally, \textbf{CoDA} (SD-3.5) and \textbf{CoDA} (Recraft V3) both perform significantly better than previous methods. Take Tail-Spot Wrasse as another example, the horizontal stripes, which were previously distorted or missing in previous methods, are now clearly visible. \textbf{CoDA} (Recraft V3), in particular, produces a more vivid and structurally accurate representation, ensuring the preservation of both color gradients and fin structure.

Note that the quality of CoDA-generated images is inherently dependent on the capability of the underlying image generation model, meaning that limitations in the base model, such as resolution constraints or texture inaccuracies, may impact the fidelity of the final augmented data.


\subsection{Data Selection Strategy}
\label{subsec:data_selection_strategy}
For each dataset, we focus on a randomly selected subset of concepts that the model is unable to recognize. The data selection strategy is as follows: In each iteration, we select a random subset of 15 species across different supercategories, including "Birds," "Mammals," and "Reptiles." This strategy allows us to identify confusing pairs without overloading the system, progressively building a collection of challenging cases from each subset. For each species within a subset, we create prompts in a multiple-choice format, incorporating the image and a randomized list of options from all species in the subset. Based on the response from the LMM, we are able to highlight specific species that are commonly mistaken for each other, guiding us in selecting pairs for further analysis. In particular, misclassification happens when an image of one species is identified by the LLM to be an image of another species. A pair \((A, B)\) is considered as a confusing pair if rate of misclassification on either direction is above the threshold 0.2. The process is repeated across new subsets, incrementally building an ample dataset of concepts the model has difficulty recognizing.


% \subsection{NovelSpecies Dataset Details}
% \label{subsec:NovelSpecies_details}
% Since there are relatively few new species, we adopt a slightly different approach. Each pair must contain exactly one new species. For every new species, we randomly sample 14 others within the same supercategory and track the misclassification rate. We then identify the species most frequently confused with the new one to form a pair.

\subsection{Experiment Details}

\subsubsection{Feature Extraction}
For textual feature extraction, we use GPT-4o-mini with chain-of-thought reasoning, running with OpenAI API calls. Each API call processes up to 2048 tokens, costing approximately $0.0025$ per 1K input tokens and $0.005$ per 1K output tokens. Given an average of 500 tokens per query and 10 queries per concept, the estimated cost per concept is around $\$0.0375$.

For visual feature extraction, we utilize GPT-4o-mini running with OpenAI API calls. Images are preprocessed to a resolution of 336x336 pixels and normalized before feature embedding extraction. Each image query incurs a cost similar to textual feature extraction. With an estimated 5 images processed per concept, the cost per concept amounts to approximately $0.1875$. 

With the rapid advancement of open-weights large language models and vision language models including DeepSeekV3~\cite{liu2024deepseek}, DeepSeekVL2~\cite{wu2024deepseekvl2}, Llama 3.2~\cite{dubey2024llama}, and more; we expect that feature extraction LLMs and VLMs can be replaced with these models with none or minimal impact to performance. We plan to perform experiments on some of these models and provide comparison results in the next updated version of our work.

\subsubsection{Feature Filtering}
We employ CLIP for automatic feature filtering, evaluating Discriminability and Generability scores. Discriminability is computed using cosine similarity between feature embeddings of target and misidentified concepts, with a threshold of 0.6. Generability is assessed by comparing feature presence in synthetic images using an ensemble of Stable Diffusion 3.5 Large and RecraftV3 models. The feature selection step is executed on an NVIDIA A100 GPU, processing features in approximately 2 hours. Top 5 ranked features are selected per concept.

\subsubsection{Image Generation and Verification}
For synthetic image generation, we employ Stable Diffusion 3.5 Large, running on a single A100 GPU. Additionally, we also integrate the RecraftV3 model through an API call. Image generation is performed at a resolution of 512x512 pixels with a guidance scale of 7.5. The pipeline generates 50 images per concept in approximately 1.2 seconds per image.

Post-generation, we perform automated verification using LLaVA V1.6-34b, running on an A6000 GPU. Each image would takes approximately 1 minutes to run for feature presence using a feature-matching confidence threshold of 0.85. Images with a satisfaction rate $S(i^{\text{synthetic}}, \mathcal{F}, \mathcal{M}) < 1.0$ are discarded.

\subsubsection{Model Updating}

We train V1.6-34b with supervised fine-tuning (SFT) using LoRA with rank 128 and alpha 256, optimizing memory efficiency while maintaining model expressiveness. The training runs on two NVIDIA A6000 GPUs, leveraging DeepSpeed Zero-3 for distributed optimization and mixed precision (bf16) for efficiency. The vision encoder is CLIP-ViT-Large-Patch14-336, with an MLP projector aligning visual and text features. We use a cosine learning rate scheduler with a 3\% warmup ratio, training for 30 epochs with a batch size of 5 and a learning rate of 2e-4. Images are padded for aspect ratio consistency, and gradient checkpointing is enabled to reduce memory usage. Checkpoints are saved every 50,000 steps, retaining only the most recent one.

\subsubsection{Evaluation}

Automatic evaluation measures zero-shot classification accuracy on a held-out test set. Inference runs on a single A6000 GPU with a batch size of 20, taking approximately 1 hour to complete. The prompt templates for evaluation are attached to Appendix \ref{app:prompt}


\subsection{Prompt Construction}
\label{app:prompt}


\lstinputlisting[language=Octave]{prompt/all_prompts.py}



% \subsubsection{Prompt for Visual/Text Feature Extractions}

% \paragraph{Contrastive Visual}
% \begin{verbatim}
% You are an experienced and meticulously observant biological scientist who is 
% asked to carefully assess the provided image. As labelled in the image, the 
% left half of the image contains a picture of the animal {main_class} and the 
% right half contains a picture of the animal {confusing_class}. Now, your task 
% is summarize the key distinctive visual attributes possessed by {main_class} 
% (on the left of the image) that makes uniquely discernible from the 
% {confusing_class} (on the right half of the image). Reason step by step to 
% produce an answer. Finally, output the key visual attributes of a {main_class} 
% (that make it distinct from a {confusing_class}) in a Python list format 
% containing short phrases of less than 8 words each. Do not output any features 
% of the {confusing_class} in your Python list. Make sure not to name the 
% {main_class} or the {confusing_class} in any of the attributes in your list. 
% Also, please try not to use negation in the visual attributes you generate: 
% for example, change features like "lack of facial markings" to "plain brown 
% face". Additionally, do not use comparative form in any of the features you 
% output, for example, change features like "thinner body than the other class" 
% to "thin body".
% \end{verbatim}

% \paragraph{Visual}
% \begin{verbatim}
% You are an experienced and meticulously observant biological scientist who is 
% asked to carefully assess the provided image. The image contains a picture of 
% the animal {main_class}. Now, your task is summarize the key distinctive visual 
% attributes possessed by {main_class}. Reason step by step to produce an answer. 
% Finally, output the key visual attributes of a {main_class} in a Python list 
% format containing short phrases of less than 8 words each. Make sure not to 
% name the {main_class} in any of the attributes in your list. Also, please try 
% not to use negation in the visual attributes you generate: for example, change 
% features like "lack of facial markings" to "plain brown face". Additionally, 
% do not use comparative form in any of the features you output, for example, 
% change features like "thinner body than the other class" to "thin body".
% \end{verbatim}

% \paragraph{Contrastive Text}
% \begin{verbatim}
% You are an experienced and knowledgeable scene classification specialist who 
% is tasked to summarize the key distinctive visual attributes possessed by 
% {main_class} that makes uniquely discernible from the {confusing_class} (just 
% based on a visual image). First retrieve your knowledge about the two 
% different types of scenes, then reason step by step to produce an answer. 
% Finally, output the key visual attributes of a {main_class} (distinct from 
% a {confusing_class}) in a Python list format containing short phrases of 
% less than 8 words each. Do not output any features of the {confusing_class} 
% in your Python list. Make sure not to name the {main_class} or the 
% {confusing_class} in any of the attributes in your list. Also, please try 
% not to use negation in the visual attributes you generate: for example, 
% instead of saying "no bright lights," use "dark environment." Additionally, 
% do not use comparative forms in any of the features you provide. For instance, 
% instead of saying "smaller windows than the other place," use "small windows."
% \end{verbatim}

% \paragraph{Text}
% \begin{verbatim}
% You are an experienced and knowledgeable scene classification specialist who 
% is tasked to summarize the key distinctive visual attributes possessed by 
% {main_class}. First retrieve your knowledge about the {main_class}, then reason 
% step by step to produce an answer. Finally, output the key visual attributes of 
% a {main_class} in a Python list format containing short strings of less than 
% 8 words each. Make sure not to name the {main_class} in any of the attributes 
% in your list. Do not output any features of the {confusing_class} in your 
% Python list. Also, please try not to use negation in the visual attributes you 
% generate: for example, instead of saying "no bright lights," use "dark 
% environment." Additionally, do not use comparative forms in any of the features 
% you provide. For instance, instead of saying "smaller windows than the other 
% place," use "small windows."
% \end{verbatim}

% \subsubsection{Text to Image Generation Prompt}
% \begin{verbatim}
% f"Generate a 4K realistic image of {main_class} that contains the following 
% attributes: {', '.join(attributes)}"
% \end{verbatim}

% \subsubsection{Feature Verification Prompt}
% \begin{verbatim}
% You are an image verification specialist. Your task is to meticulously assess 
% the image for specific attributes and confirm their presence. For each 
% attribute in the list, carefully check the image, examine visual elements 
% such as color, shape, texture, position, and context clues that might indicate 
% whether the attribute is present. Provide a binary Python output list, where 
% each element is either 1 (attribute is present) or 0 (attribute is absent), 
% corresponding exactly to the order of attributes provided.

% Attributes to Verify: {attributes}

% Expected Output: A list of 0s and 1s indicating the presence or absence of 
% each attribute, in the same order as listed. Here is an example output: [0, 1, 1].
% \end{verbatim}

% \section{Finetune and Evaluation Prompt}
% \begin{verbatim}
% "You are an image classification specialist with expertise in categorizing 
% images into specific groups. Given an image, identify its category from the 
% following options: " + ", ".join(provided_options_capitalized[:-1]) + ", or " 
% + provided_options_capitalized[-1] + ". Provide your answer as only one 
% category name for precise classification. Please response with the category 
% name only."
% \end{verbatim}

% \section{Deduplication Prompt}
% \begin{verbatim}
% You are an experienced and knowledgeable biological scientist who is tasked 
% to summarize the key distinctive visual attributes possessed by {main_class} 
% into a coherent list. Given the following list of attributes describing the 
% animal species {main_class}: {attributes_list}. You task is to combine the 
% duplicate features (which have the same or very similar meanings) into one. 
% Then, you will order the remaining features in order of visual importance, 
% the most visually significant / observable features will be at the front of 
% the list while the least visually observable features will be at the back. 
% Finally, output the key visual attributes of a {main_class} in a Python list 
% format containing short phrases of less than 8 words each. Make sure not to 
% name the {main_class} in any of the attributes in your list. Also, please try 
% not to use negation in the visual attributes you generate: for example, change 
% features like "lack of facial markings" to "plain brown face". Additionally, 
% do not use comparative form in any of the features you output, for example, 
% change features like "thinner body than the other class" to "thin body".
% \end{verbatim}

% \section{System Prompt}
% \begin{verbatim}
% "You are a helpful assistant."
% \end{verbatim}





\end{document}


% This document was modified from the file originally made available by
% Pat Langley and Andrea Danyluk for ICML-2K. This version was created
% by Iain Murray in 2018, and modified by Alexandre Bouchard in
% 2019 and 2021 and by Csaba Szepesvari, Gang Niu and Sivan Sabato in 2022.
% Modified again in 2023 and 2024 by Sivan Sabato and Jonathan Scarlett.
% Previous contributors include Dan Roy, Lise Getoor and Tobias
% Scheffer, which was slightly modified from the 2010 version by
% Thorsten Joachims & Johannes Fuernkranz, slightly modified from the
% 2009 version by Kiri Wagstaff and Sam Roweis's 2008 version, which is
% slightly modified from Prasad Tadepalli's 2007 version which is a
% lightly changed version of the previous year's version by Andrew
% Moore, which was in turn edited from those of Kristian Kersting and
% Codrina Lauth. Alex Smola contributed to the algorithmic style files.
