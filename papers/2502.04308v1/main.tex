%%%%%%%% ICML 2025 EXAMPLE LATEX SUBMISSION FILE %%%%%%%%%%%%%%%%%

\documentclass{article}

% Recommended, but optional, packages for figures and better typesetting:
\usepackage{microtype}
\usepackage{graphicx}
\usepackage{subfigure}
\usepackage{booktabs} % for professional tables

% hyperref makes hyperlinks in the resulting PDF.
% If your build breaks (sometimes temporarily if a hyperlink spans a page)
% please comment out the following usepackage line and replace
% \usepackage{icml2025} with \usepackage[nohyperref]{icml2025} above.
\usepackage{hyperref}


% Attempt to make hyperref and algorithmic work together better:
\newcommand{\theHalgorithm}{\arabic{algorithm}}



% Use the following line for the initial blind version submitted for review:
%\usepackage{icml2025}
\usepackage[preprint]{icml2025}

% If accepted, instead use the following line for the camera-ready submission:
%\usepackage[accepted]{ICML_sty/icml2025}



% For theorems and such
\usepackage{amsmath}
\usepackage{amssymb}
\usepackage{mathtools}
\usepackage{amsthm}

% if you use cleveref..
\usepackage[capitalize,noabbrev]{cleveref}

%%%%%%%%%%%%%%%%%%%%%%%%%%%%%%%%
% THEOREMS
%%%%%%%%%%%%%%%%%%%%%%%%%%%%%%%%
\theoremstyle{plain}
\newtheorem{theorem}{Theorem}[section]
\newtheorem{proposition}[theorem]{Proposition}
\newtheorem{lemma}[theorem]{Lemma}
\newtheorem{corollary}[theorem]{Corollary}
\theoremstyle{definition}
\newtheorem{definition}[theorem]{Definition}
\newtheorem{assumption}[theorem]{Assumption}
\theoremstyle{remark}
\newtheorem{remark}[theorem]{Remark}


%\usepackage{mysty}


\usepackage{makecell} % 



\usepackage{multirow}
\usepackage{enumitem}
\usepackage{graphicx} % for \resizebox
\usepackage{bm} % for \bm

% The following two are for algorithm environment
\usepackage{algorithmic}
\renewcommand{\algorithmiccomment}[1]{\hfill{ $\rhd$ #1}}



% For table, add a gray background color
\usepackage{colortbl}
\definecolor{gg}{gray}{0.92}
\newcolumntype{a}{>{\columncolor{gg}}c}


\hypersetup{
    colorlinks = true,
    citecolor = blue,
    linkcolor = red
}


% for the Proposition environment in the Appendix
\newtheoremstyle{customsty}
  {\topsep}                 % 上方间距
  {\topsep}                 % 下方间距
  {\itshape}                % 正文字体
  {0pt}                     % 缩进量
  {\bfseries}               % 标题字体
  {.}                       % 标题后缀
  {5pt plus 1pt minus 1pt}  % 标题与正文间的间距
  {}                        % 标题格式

\theoremstyle{customsty}
\newtheorem{innercustomthe}{}
\newenvironment{customthe}[1][]{%
    \renewcommand\theinnercustomthe{#1}% 自定义编号并加粗
    \begin{innercustomthe}
}{%
    \end{innercustomthe}
}



\crefname{equation}{eq.}{eq.}
\Crefname{equation}{Eq.}{Eq.}
\crefname{theorem}{thm.}{thms.}
\Crefname{Theorem}{Thm.}{Thms.}
\crefname{conjecture}{conj.}{conjs.}
\Crefname{Conjecture}{Conj.}{Conjs.}
\crefname{proposition}{Prop.}{Props.}
\Crefname{proposition}{Prop.}{Props.}
\crefname{definition}{dfn.}{dfn.}
\Crefname{definition}{Dfn.}{Dfn.}
\crefname{remark}{remark}{remark}
\Crefname{Remark}{Remark}{Remark}
\Crefname{algorithm}{Alg.}{Alg.}

\crefname{section}{Sec.}{Secs.}
\Crefname{section}{Sec.}{Secs.}
\crefname{equation}{Eq.}{Eqs.}
\Crefname{equation}{Eq.}{Eqs.}
\crefname{figure}{Fig.}{Figs.}
\Crefname{figure}{Fig.}{Figs.}
\crefname{table}{Tab.}{Tabs.}
\Crefname{table}{Tab.}{Tabs.}
\crefname{thm}{Thm.}{Thms.}
\Crefname{thm}{Thm.}{Thms.}
\crefname{conj}{Conj.}{Conjs.}
\Crefname{conj}{Conj.}{Conjs.}
\crefname{dfn}{Dfn.}{Dfns.}
\crefname{dfn}{Dfn.}{Dfns.}
\crefname{remark}{remark}{remarks}
\Crefname{Remark}{Remark}{Remarks}
\crefname{prop}{Prop.}{Prop.}
\Crefname{prop}{Prop.}{Prop.}
\Crefname{algorithm}{Alg.}{Alg.}
\crefname{appendix}{App.}{Apps.}
\Crefname{appendix}{App.}{Apps.}
%\crefname{app}{appendix}{appendix}
\crefname{appsec}{appendix}{appendices}
\Crefname{appsec}{Appendix}{Appendices}



\renewcommand{\paragraph}[1]{{\vspace{0.3mm}\noindent \bf #1}.}
\newcommand{\paragraphnoper}[1]{{\vspace{0.3mm}\noindent \bf #1}}


% \newcommand{\deq}{\overset{\triangle}{=}} % in case \triangleq does not work
\newcommand{\norm}[1]{\left\|#1\right\|}
\newcommand{\Fi}[1]{\textbf{#1}}
\newcommand{\Se}[1]{\underline{#1}}

\newcommand*\diff{\mathop{}\!\mathrm{d}}
\newcommand{\dt}{\diff{t}}

\newcommand\numberthis{\addtocounter{equation}{1}\tag{\theequation}}


\newcommand{\eg}{\textit{e}.\textit{g}.}
\newcommand{\etal}{\textit{et al}.}
\newcommand{\ie}{\textit{i}.\textit{e}.}
\newcommand{\cf}{\textit{cf}.} 

\newcommand{\tolga}[1]{\textcolor{magenta}{T: \string#1}}


% Todonotes is useful during development; simply uncomment the next line
%    and comment out the line below the next line to turn off comments
%\usepackage[disable,textsize=tiny]{todonotes}
\usepackage[textsize=tiny]{todonotes}


% The \icmltitle you define below is probably too long as a header.
% Therefore, a short form for the running title is supplied here:
% \icmltitlerunning{Higher-order Structures Guided Diffusion for Graph Generation}
\icmltitlerunning{HOG-Diff: Higher-Order Guided Diffusion for Graph Generation}


\begin{document}

\twocolumn[
% \icmltitle{Higher-order Structures Guided Diffusion for Graph Generation}
\icmltitle{HOG-Diff: Higher-Order Guided Diffusion for Graph Generation}

% It is OKAY to include author information, even for blind
% submissions: the style file will automatically remove it for you
% unless you've provided the [accepted] option to the icml2025
% package.

% List of affiliations: The first argument should be a (short)
% identifier you will use later to specify author affiliations
% Academic affiliations should list Department, University, City, Region, Country
% Industry affiliations should list Company, City, Region, Country

% You can specify symbols, otherwise they are numbered in order.
% Ideally, you should not use this facility. Affiliations will be numbered
% in order of appearance and this is the preferred way.
%\icmlsetsymbol{equal}{*}

\begin{icmlauthorlist}
\icmlauthor{Yiming Huang}{IC}
\icmlauthor{Tolga Birdal}{IC}
\end{icmlauthorlist}

\icmlaffiliation{IC}{Imperial College London, London, United Kingdom}
% \icmlaffiliation{comp}{Company Name, Location, Country}
% \icmlaffiliation{sch}{School of ZZZ, Institute of WWW, Location, Country}

%\icmlcorrespondingauthor{Yiming Huang}{y.huang24@imperial.ac.uk}
\icmlcorrespondingauthor{Tolga Birdal}{t.birdal@imperial.ac.uk}


% You may provide any keywords that you
% find helpful for describing your paper; these are used to populate
% the "keywords" metadata in the PDF but will not be shown in the document
\icmlkeywords{Graph Generation, Higher-order Networks, Score-based Generative Model, Topological Deep Learning}

\vskip 0.3in
]

% this must go after the closing bracket ] following \twocolumn[ ...

% This command actually creates the footnote in the first column
% listing the affiliations and the copyright notice.
% The command takes one argument, which is text to display at the start of the footnote.
% The \icmlEqualContribution command is standard text for equal contribution.
% Remove it (just {}) if you do not need this facility.

\printAffiliationsAndNotice{}  % leave blank if no need to mention equal contribution
%\printAffiliationsAndNotice{\icmlEqualContribution} % otherwise use the standard text.

\begin{abstract}

Hierarchical clustering is a powerful tool for exploratory data analysis, organizing data into a tree of clusterings from which a partition can be chosen. This paper generalizes these ideas by proving that, for any reasonable hierarchy, one can optimally solve any center-based clustering objective over it (such as $k$-means). Moreover, these solutions can be found exceedingly quickly and are \emph{themselves} necessarily hierarchical. 
%Thus, given a cluster tree, we show that one can quickly generate a myriad of \emph{new} hierarchies from it. 
Thus, given a cluster tree, we show that one can quickly access a plethora of new, equally meaningful hierarchies.
Just as in standard hierarchical clustering, one can then choose any desired partition from these new hierarchies. We conclude by verifying the utility of our proposed techniques across datasets, hierarchies, and partitioning schemes.


\end{abstract}

\section{Introduction}

\begin{figure}
%\vspace{-0.1in}
\centering
\includegraphics[width=0.99\linewidth]{figs/framework.pdf}
\caption{\textbf{Overivew of HOG-Diff.} The dashed line above illustrates the classical generation process, where graphs quickly degrade into random structures with uniformly distributed entries. In contrast, as shown in the coloured region below, HOG-Diff adopts a coarse-to-fine generation curriculum based on the diffusion bridge, explicitly learning higher-order structures during intermediate steps with theoretically guaranteed performance.}
\label{fig:framework}
\vspace{-4mm}
\end{figure}


%%%%% ----- 1.1 what are graphs? 介绍什么是图,
Graphs provide an elegant abstraction for representing complex empirical phenomena by encoding entities as vertices and their relationships as edges, thereby transforming unstructured data into analyzable representations.
%
%%%%% ----- 1.2 The meaning of  Graph Generation 研究意义。有什么用?
Modelling the underlying distribution of graph-structured data is a crucial yet challenging task with broad applications, including social network analysis, motion synthesis, drug discovery, protein design, and urban planning~\cite{zhu2022survey}.
%
%%%%% ----- 1.3 Traditional models 
The study of graph generation seeks to synthesize graphs that align with the observed distribution and traces back to seminal models of random network models \cite{ER1960, BA1999}.
While these models offer foundational insights, they are often too simplistic to capture the complexity of graph distributions we encounter in practice.


%%%%% ----- 2. Recent studies about graph generative models (深度生成模型)
Recently, advances in generative models have leveraged the power of deep neural networks to significantly improve the ability to learn graph distributions.
Notable approaches include models based on recurrent neural networks (RNNs) \cite{GraphRNN2018},  variational autoencoders (VAEs) \cite{VAE-Jin2018}, and generative adversarial networks (GANs) \cite{GAN1-MolGAN, GAN2-Spectre}.
However, the end-to-end structure of these methods makes them hard to train.
%
%% Diffusion-based generative models
More recently, diffusion-based models have achieved remarkable success in image generation by learning a model to denoise a noisy sample \cite{DDPM+NeurIPS2020, Score-SDE+ICLR2021}. 
%
With the advent of diffusion models, their applications on graphs with complex topological structural properties have recently aroused significant scientific interest \cite{EDPGNN-2020,GDSS+ICML2022,DiGress+ICLR2023}.




%%%%% ----- 3. The difference in graph generation 
%%%%% This part gives the motivation for our work
Despite these advances, existing graph generative models typically inherit the frameworks designed for image generation \cite{Score-SDE+ICLR2021}, which fundamentally limits their ability to capture the intrinsic topological properties of networks. 
%
%-----  3.1 higher-order structures are crucial for graphs 
Notably, networks exhibit higher-order structures, such as motifs, simplices, and cells, which capture multi-way interactions and critical topological dependencies beyond pairwise relationships \cite{HigherOrderReview2020,ISMnet2024,TDL-position+ICML2024}.
These structures are vital for representing complex phenomena in domains like molecular graphs, social networks, and protein interactions.
However, current methods are ineffective at modelling the topological properties of higher-order systems since \emph{learning to denoise the noisy samples does not explicitly preserve the intricate structural dependencies required for generating realistic graphs}.


% 
Moreover, the image corrupted by Gaussian noise retains recognizable numerical patterns during the early and middle stages of forward diffusion. By contrast, the graph adjacency matrix quickly degrades into a dense matrix with uniformly distributed entries within a few diffusion steps. 
%
% 3.3 
In addition, directly applying diffusion-based generative models to graph topology generation by injecting isotropic Gaussian noise to adjacency matrices is harmful as it destroys critical graph properties such as sparsity and connectivity.
%
%
%% 3.4 Permutation equivalence.  
Lastly, such a framework should ensure equivariance\footnote{invariance as a particular special case}, maintaining the learned distribution despite node index permutations, which is essential for robustness and capturing intrinsic graph distribution.
%
%
Therefore, a graph-friendly diffusion process should also retain meaningful intermediate states and trajectories, avoid inappropriate noise addition, and ensure equivariance.




% 4.3 Introduction to our framework
% Coarse-to-fine generation
Motivated by these principles and advances in \emph{topological deep learning}~\cite{hajij2022topological,TDL-position+ICML2024}, we propose the \textbf{Higher-order Guided Diffusion} (HOG-Diff) framework, illustrated in \cref{fig:framework}, to address the gaps in graph generation. 
HOG-Diff introduces a coarse-to-fine generation curriculum that enhances the model’s ability to capture complex graph properties by preserving higher-order topologies throughout the diffusion process.
% 
Specifically, we decompose the graph generation task into manageable sub-tasks, beginning by generating higher-order graph skeletons that capture core structures, which are then refined to include pairwise interactions and finer details, resulting in complete graphs with both topological and semantic fidelity.
%
Additionally, HOG-Diff integrates diffusion bridge and spectral diffusion to ensure effective generation and adherence to the aforementioned graph generation principles. 
%
Our theoretical analysis reveals that HOG-Diff converges more rapidly in score matching and achieves sharper reconstruction error bounds than classical approaches, offering strong theoretical support for the proposed framework.
Furthermore, our framework promises to enhance interpretability by enabling the analysis of different topological guides’ performance in the generation process.
%
%
%%%%% ----- 5. Our Contributions
The contributions of this paper are threefold:
\vspace{-0.1in}
\begin{itemize}[noitemsep, parsep=0.3pt, leftmargin=*]
%\begin{itemize}[leftmargin=*]
\item \textbf{Algorithmic}: we introduce a coarse-to-fine graph generation curriculum guided by higher-order topological information using the OU diffusion bridge. 
\item \textbf{Theoretical}: our analysis reveals that HOG-Diff achieves faster convergence during score-matching and a sharper reconstruction error bound compared to classical methods.
\item \textbf{Experimental}: extensive evaluations show that HOG-Diff achieves state-of-the-art graph generation performance across various datasets, highlighting the functional importance of topological guidance.\vspace{-2mm}
\end{itemize}


\section{Preliminaries}
\label{sec:prelim}
\subsection{Notations}
\label{ssec:notation}
The set $\{1,2,\ldots,x\}$ is denoted as $[x]$.
We consider $\graph = (\vertexset,\edgeset)$ to be a simple, unweighted, undirected graph with $\size{\vertexset} = \vertexcount$, and $\size{\edgeset} = \edgecount$. Given a vertex $\vertex$, its neighboring vertex set is denoted as $\neighbour(\vertex) = \set{\altvertex|(\altvertex,\vertex)\in \edgeset}$. We denote by $\degree{\vertex}$ the degree of the vertex $\vertex$. Based on the degrees of the two vertices of an edge $\edge = \fbrac{\vertex,\altvertex}$, we define the degree of the edge $\edge$ as $\degree{\edge} = \min\fbrac{\degree{\vertex}~,\degree{\altvertex}}$. We denote the set of triangles in $\graph$ as $\triangleset$, and individual triangles are denoted as $\triangle$. ($\fbrac{\vertex,\edge}$ denotes a triangle formed by the vertices $\vertex$ and the endpoints of the edge $\edge$). We want to estimate the number of triangles,  $\size{\triangleset} = \numtriangle$ in the graph given the $\degreeq$, $\neighbourq$, $\edgeexistsq$ and $\randedgeq$ queries. An edge $\edge$ participates in a triangle $\triangle$ means that the triangle $\triangle$ is incident on the edge $\edge$. We denote by $\numtriangle_\edge$ the number of triangles the edge $\edge$ participates in. $\uniform(S)$ denotes an element of $S$ is chosen uniformly at random. 

% \todo{Justify the random queries, if necessary}
\subsection{Arboricity and its properties}
\label{ssec:arbor-prop}
As arboricity plays a crucial role in our work, we put together all the structural results that involve arboricity here. Let us restate the definition once more. 
\begin{definition}[Arboricity$(\arboricity)$]
   The arboricity of a graph $\graph = (\vertexset,\edgeset)$, denoted by $\arboricitygraph{G}$, is the minimum number of spanning forests that $\edgeset$ can be partitioned into.
   \label{def:arboricity}
\end{definition}
The arboricity of a graph can be seen as a measure of the density of the graph. $\arboricitygraph{G}$ can be at least $\left\lceil m/(n-1)\right\rceil$. Also, $\arboricitygraph{G} \geq \arboricitygraph{H}$ where $H$ is any subgraph of $G$. We will write $\arboricity$ instead of $\arboricitygraph{G}$ when the underlying graph is understood. We introduce the following lemma due to~\citep{DBLP:journals/siamcomp/ChibaN85} on the sum of edge degrees over all  edges in the graph.
\begin{lemma}(~\citep{DBLP:journals/siamcomp/ChibaN85})
\label{Lemma: deg(e) sum is m * arboricity}
     Given a graph $\graph = (\vertexset,\edgeset)$ with arboricity $\arboricity$ and $\size{\edgeset} = \edgecount$,  $\sum\limits_{\edge \in \edgeset} \degree{\edge} = 2\edgecount\arboricity$.
\end{lemma}

The following lemma due to~\citep{DBLP:conf/soda/EdenRS20} builds on the work of~\citep{DBLP:journals/siamcomp/ChibaN85} to bound the number of triangles based on the number of edges $\edgecount$ and arboricity $\arboricity$. 
\begin{lemma}[Triangle Upper Bound ~\citep{DBLP:conf/soda/EdenRS20}]
\label{lemma: arboricity triangle bound}
    Given a graph $\graph = (\vertexset,\edgeset)$ with arboricity $\arboricity$ and $\size{\edgeset} = \edgecount$, the graph $\graph$ has at most $\edgecount\arboricity$ triangles.
\end{lemma}
Note that this upper bound is also tight, i.e., there exists graphs that contain $\edgecount$ edges and $\bigomega{\edgecount\arboricity}$ triangles. Additionally, arboricity $\arboricity$ can be at most $\bigo{\sqrt{\edgecount}}$. Thus all our results can be reformulated by plugging in this upper bound. 


\ifarxiv{
\subsection{Chernoff Bounds}
We will be using the following variation of the Chernoff bound that bounds the deviation of the sum of independent Poisson trials~\citep{Mitzenmacher_Upfal_2005}.

\begin{lemma}[Multiplicative Chernoff Bound]\label{Lemma: Multiplicative Chernoff Bound}
    Given i.i.d. random variables $X_1,X_2,...,X_t$ where $\Pr[X_i = 1] = p$ and $\Pr[X_i = 0] = (1-p)$, define $X = \sum_{i \in [t]} X_i$. Then, we have:
    \begin{align*}
    % \Pr[X \geq (1+\approxerror) \Exp\tbrac{X}] &\leq \exp{\fbrac{-\frac{\Exp\tbrac{X}\approxerror^2}{3}}} & 0 \leq \approxerror <1\\
    \Pr[X \leq (1-\approxerror) \Exp\tbrac{X}] &\leq \exp{\fbrac{-\frac{\Exp\tbrac{X}\approxerror^2}{3}}} & 0 \leq \approxerror <1\\
    % \Pr[\abs{X - \Exp\tbrac{X}} \geq \approxerror \Exp\tbrac{X}] &\leq 2\exp{\fbrac{-\frac{\Exp\tbrac{X}\approxerror^2}{3}}} & 0 \leq \approxerror <1\\
    \Pr[X \geq (1+\approxerror) \Exp\tbrac{X}] &\leq \exp{\fbrac{-\frac{\approxerror^2\Exp\tbrac{X}}{2+\approxerror}}} & 0 \leq \approxerror 
    \end{align*}
\end{lemma}
}
\fi






%-----------------------------Notations-----------------------


\section{Higher-order Guided Diffusion Model}
We now present our \textit{Higher-order Guided Diffusion} (HOG-Diff) model, which enhances graph generation by exploiting higher-order structures. 
We begin by detailing a coarse-to-fine generation curriculum that incrementally constructs graphs, followed by the introduction of three essential supporting techniques: the diffusion bridge, spectral diffusion, and a denoising model, respectively. Finally, we provide theoretical evidence validating the efficacy of HOG-Diff.


%\label{sec:gen-curriculum}
\paragraph{Coarse-to-fine Generation}
We draw inspiration from curriculum learning, a paradigm that mimics the human learning process by systematically organizing data in a progression from simple to complex \cite{curriculum-IJCV2022}.
Likely, an ideal graph generation curriculum should be a composition of multiple easy-to-learn and meaningful intermediate steps.
Additionally, higher-order structures encapsulate rich structural properties beyond pairwise interactions that are crucial for various empirical systems \cite{HiGCN2024}. 
As a graph-friendly generation framework, HOG-Diff incorporates higher-order structures during the intermediate stages of forward diffusion and reverse generative processes, thereby realizing a coarse-to-fine generation curriculum.

To implement our coarse-to-fine generation curriculum, we introduce a key operation termed cell complex filtering (CCF).
As illustrated in \cref{fig:cell-transform}, CCF generates an intermediate state of a graph by pruning nodes and edges that do not belong to a given cell complex.

\begin{definition}[Cell complex filtering]
Given a graph $\bm{G} = (\bm{V},\bm{E})$ and its associated cell complex $\mathcal{S}$, the cell complex filtering operation produces a filtered graph $\bm{G}^\prime = (\bm{V}^\prime,\bm{E}^\prime)$ where 
$\bm{V}^{\prime} = \{ v \in \bm{V}  \mid \exists\;x_\alpha \in \mathcal{S} : v \in x_\alpha \}$, 
and $\bm{E}^{\prime} = \{ (u, v) \in \bm{E} \mid \exists\;x_\alpha \in \mathcal{S} : u,v \in x_\alpha\}$.
\end{definition}

This filtering operation is a pivotal step in decomposing the graph generation task into manageable sub-tasks, with the filtered results serving as natural intermediaries in hierarchical graph generation. 
The overall framework of our proposed framework is depicted in \cref{fig:framework}.
Specifically, the forward and reverse diffusion processes in HOG-Diff are divided into $K$ hierarchical time windows, denoted as  $\{[\tau_{k-1},\tau_k]\}_{k=1}^K$, where $0 = \tau_0 < \cdots < \tau_{k-1}< \tau_k < \cdots < \tau_K = T$. 
Our sequential generation progressively denoises the higher-order skeletons.  
First, we generate coarse-grained higher-order skeletons, and subsequently refine them into finer pairwise relationships, simplifying the task of capturing complex graph distributions. 
%
This coarse-to-fine approach inherently aligns with the hierarchical nature of many real-world graphs, enabling smoother training and improved sampling performance.



Formally, our generation process factorizes the joint distribution of the final graph $\bm{G}_0$ into a product of conditional distributions across these time windows:
\begin{equation}
p(\bm{G}_0)=p(\bm{G}_0|\bm{G}_{\tau_1})p(\bm{G}_{\tau_1}|\bm{G}_{\tau_2}) \cdots p(\bm{G}_{\tau_{K-1}}|\bm{G}_{T}).
\end{equation}
Here, intermediate states $\bm{G}_{\tau_1}, \bm{G}_{\tau_2}, \cdots, \bm{G}_{\tau_{K-1}}$ represents different levels of cell filtering results of the corresponding graph. 
Our approach aligns intermediate stages of the diffusion process with realistic graph representations, and the ordering reflects a coarse-to-fine generation process.




The forward process introduces noise in a stepwise manner while preserving intermediate structural information. During each time window $[\tau_{k-1}, \tau_k]$, the evolution of the graph is governed by the following forward SDE:
\begin{equation}
\label{eq:HoGD-forward}
\mathrm{d}\bm{G}_t^{(k)}=\mathbf{f}_{k,t}(\bm{G}_t^{(k)})\mathrm{d}t+g_{k,t}\mathrm{d}\bm{W}_t, t \in [\tau_{k-1}, \tau_k].
\end{equation}

Reversing this process enables the model to generate authentic samples with desirable higher-order information.
The reverse-time SDE corresponds to \cref{eq:HoGD-forward} is as follows:
\begin{equation}
%\fontsize{9pt}{9pt}\selectfont
\begin{split}
    \mathrm{d}\bm{G}_t^{(k)}
     = & \left[\mathbf{f}_{k,t}(\bm{G}_t^{(k)})-g_{k,t}^2\nabla_{\bm{G}_t^{(k)}}\log p_t(\bm{G}_t^{(k)})\right]\mathrm{d}\bar{t} \\
    & +g_{k,t}\mathrm{d}\bar{\bm{W}}_t.
\end{split}
%\fontsize{10pt}{10pt}\selectfont
\end{equation}
Instead of using higher-order information as a direct condition, HOG-Diff employs it to guide the generation process through multiple steps. This strategy allows the model to incrementally build complex graph structures while maintaining meaningful structural integrity at each stage.
Moreover, integrating higher-order structures into graph generative models improves interpretability by allowing analysis of their significance in shaping the graph’s properties.







%\subsection{Diffusion Bridge}
%\label{sec:bridge}
\paragraph{Diffusion Bridge Process}
% -------- OU process
We realize the guided diffusion based on the generalized Ornstein-Uhlenbeck (GOU) process \cite{GOU1988, IRSDE+ICML2023}, a stationary Gaussian-Markov process characterized by its mean-reverting property. 
Over time, the marginal distribution of the GOU process stabilizes around a fixed mean and variance, making it well-suited for stochastic modelling with terminal constraints. 
The GOU process $\mathbb{Q} $ is governed by the following SDE:
\begin{equation}
\mathbb{Q}: \mathrm{d} \bm{G}_t = \theta_t(\bm{\mu} -\bm{G}_t)\mathrm{d}t + g_t(\bm{G}_t)\mathrm{d}\bm{W}_t,
\label{eq:GOU-SDE}
\end{equation}
where $\bm{\mu}=\bm{G}_{\tau_k}$ is the target terminal state, $\theta_t$ denotes a scalar drift coefficient and $g_t$ represents the diffusion coefficient. 
To ensure the process remains analytically tractable, $\theta_t$ and $g_t$ are constrained by the relationship $g_t^2/\theta_t = 2\sigma^2$ \cite{IRSDE+ICML2023}, where $\sigma^2$ is a given constant scalar.
%
Under these conditions, its transition probability admits a closed-form solution:
\begin{equation}
\begin{split}
p(&\bm{G}_{t}\mid \bm{G}_s) 
=\mathcal{N}(\mathbf{m}_{s:t},v_{s:t}^{2}\bm{I})  \\
&=  \mathcal{N}\left(
\bm{\mu}+\left(\bm{G}_s-\bm{\mu}\right)e^{-\bar{\theta}_{s:t}},
\sigma^2 (1-e^{-2\bar{\theta}_{s:t}})\bm{I}
\right).
\end{split}
\label{eq:GOU-p}
\end{equation}
Here, $\bar{\theta}_{s:t}=\int_s^t\theta_zdz$, and for notional simplicity, $\bar{\theta}_{0:t}$ is replaced by $\bar{\theta}_t$ when $s=0$.
% 
At time $t$ progress, $p(\bm{G}_t)$ gradually approaches a Gaussian distribution characterized by mean $\bm{\mu}$ and variance $\sigma^2$, indicating that the GOU process exhibits the mean-reverting property.  




% ---------- OU Bridge
The Doob’s $h$-transform can modify an SDE such that it passes through a specified endpoint.
When applied to the GOU process, this eliminates variance in the terminal state, driving the diffusion process toward a Dirac distribution centered at $\bm{G}_{\tau_k}$ \cite{GOUB2021,GOUB+ICML2024}.


\begin{proposition}
\label{pro:OUB}
Let $\bm{G}_t$ evolve according to the generalized OU process in \cref{eq:GOU-SDE}, subject to the terminal conditional $\bm{\mu}=\bm{G}_{\tau_k}$. 
%Then, the evolution of the conditional marginal distribution $p(\bm{G}_t \mid \bm{G}_{\tau_k})$ satisfies the following SDE:
The conditional marginal distribution $p(\bm{G}_t\mid\bm{G}_{\tau_k})$ then evolves according to the following SDE:
\begin{equation}
%\fontsize{8.5pt}{8.5pt}\selectfont
\mathrm{d}\bm{G}_t = \theta_t \left( 1 + \frac{2}{e^{2\bar{\theta}_{t:\tau_k}}-1}  \right)(\bm{G}_{\tau_k} - \bm{G}_t)  \mathrm{d}t 
+ g_{k,t} \mathrm{d}\bm{W}_t.\nonumber
%\fontsize{10pt}{10pt}\selectfont
\end{equation}
The conditional transition probability $p(\bm{G}_t \mid \bm{G}_{\tau_{k-1}}, \bm{G}_{\tau_k})$ has analytical form as follows:
\begin{equation}
\begin{split}
&p(\bm{G}_t \mid  \bm{G}_{\tau_{k-1}}, \bm{G}_{\tau_k}) 
= \mathcal{N}(\bar{\mathbf{m}}_t, \bar{v}_t^2 \bm{I}),\\
&\bar{\mathbf{m}}_t = 
\bm{G}_{\tau_k} + (\bm{G}_{\tau_{k-1}}-\bm{G}_{\tau_k})e^{-\bar{\theta}_{\tau_{k-1}:t}} 
\frac{v_{t:\tau_k}^2}{v_{\tau_{k-1}:\tau_k}^2}, \\
&\bar{v}_t^2 = {v_{\tau_{k-1}:t}^2 v_{t:\tau_k}^2}/{v_{\tau_{k-1}:\tau_k}^2}.
\end{split}
\end{equation}
Here, $\bar{\theta}_{a:b}=\int_a^b \theta_s  \mathrm{d}s$, and $v_{a:b}=\sigma^2(1-e^{-2\bar{\theta}_{a:b}})$.
\end{proposition}


We can directly use the closed-form solution in \Cref{pro:OUB} for one-step forward sampling without performing multi-step forward iteration using the SDE.
%
The reverse-time dynamics of the conditioned process can be derived using the theory of SDEs and take the following form:
\begin{equation}
%\fontsize{9pt}{9pt}\selectfont
 \mathrm{d}\bm{G}_t = \left[\mathbf{f}_{k,t}(\bm{G}_t)-g_{k,t}^2 \nabla_{\bm{G}_t} \log p(\bm{G}_t|\bm{G}_{\tau_k})\right] \diff\bar{t} + g_{k,t} \diff\bar{\bm{W}}_t, \nonumber
% \fontsize{10pt}{10pt}\selectfont
\end{equation}
where $\mathbf{f}_{k,t}(\bm{G}_t) = \theta_t \left( 1 + \frac{2}{e^{2\bar{\theta}_{t:\tau_k}}-1}  \right)(\bm{G}_{\tau_k} - \bm{G}_t)$.


\paragraph{Spectral Diffusion}
Generating graph adjacency matrices presents several significant challenges.
% 1.non-uniqueness of adjacency matrix 
Firstly, the non-uniqueness of graph representations implies that a graph with $n$ vertices can be equivalently modelled by up to $n!$ distinct adjacency matrices. This ambiguity requires a generative model to assign probabilities uniformly across all equivalent adjacencies to accurately capture the graph’s inherent symmetry. 
%
% 2.Sparsity
Additionally, unlike densely distributed image data, graphs typically follow a Pareto distribution and exhibit sparsity \cite{Sparsity+NP2024}, so that adjacency score functions lie on a low-dimensional manifold. Consequently, noise injected into out-of-support regions of the full adjacency space severely degrades the signal-to-noise ratio, impairing the training of the score-matching process. 
Even for densely connected graphs, isotropic noise distorts global message-passing patterns by encouraging message-passing on sparsely connected regions.
%
% 3. diffusion faster / scalability
Moreover, the adjacency matrix scales quadratically with the number of nodes, making the direct generation of adjacency matrices computationally prohibitive for large-scale graphs.

To address these challenges, inspired by~\citet{GAN2-Spectre,GSDM+TPAMI2023}, we introduce noise in the eigenvalue domain of the graph Laplacian matrix $\bm{L}=\bm{D}-\bm{A}$, instead of the adjacency matrix $\bm{A}$, where $\bm{D}$ denotes the diagonal degree matrix.
%
As a symmetric positive semi-definite matrix, the graph Laplacian can be diagonalized as $\bm{L} = \bm{U} \bm{\Lambda} \bm{U}^\top$. Here, the orthogonal matrix $\bm{U} = [\bm{u}_1,\cdots,\bm{u}_n]$ comprises the eigenvectors, and the diagonal matrix $\bm{\Lambda} = \operatorname{diag}(\lambda_1,\cdots,\lambda_n)$ holds the corresponding eigenvalues.
The relationship between the Laplacian spectrum and the graph's topology has been extensively explored~\cite{chung1997spectral}. For instance,  the low-frequency components of the spectrum capture the global structural properties such as connectivity and clustering, whereas the high-frequency components are crucial for reconstructing local connectivity patterns.
%
%
%% ------- 
Therefore, the target graph distribution $p(\bm{G}_0)$ represents a joint distribution of $\bm{X}_0$ and $\bm{\Lambda}_0$, exploiting the permutation invariance and structural robustness of the Laplacian spectrum.
%
Consequently, we split the reverse-time SDE into two parts that share drift and diffusion coefficients as
\begin{equation}
\fontsize{8pt}{8pt}\selectfont
\left\{
\begin{aligned}
\mathrm{d}\bm{X}_t=
&\left[\mathbf{f}_{k,t}(\bm{X}_t)
-g_{k,t}^2 
\nabla_{\bm{X}} \log p_t(\bm{G}_t | \bm{G}_{\tau_k}) \right]\mathrm{d}\bar{t}
+g_{k,t}\mathrm{d}\bar{\bm{W}}_{t}^1
\\
\mathrm{d}\bm{\Lambda}_t=
&\left[\mathbf{f}_{k,t}(\bm{\Lambda}_t)
- g_{k,t}^2 \nabla_{\bm{\Lambda}} \log p_t(\bm{G}_t | \bm{G}_{\tau_k})\right]\mathrm{d}\bar{t}
+g_{k,t}\mathrm{d}\bar{\bm{W}}_{t}^2
\end{aligned}
\right..\nonumber
\fontsize{10pt}{10pt}\selectfont
\label{eq:reverse-HoGD}
\end{equation}
Here, the superscript of $\bm{X}^{(k)}_t$ and $\bm{\Lambda}^{(k)}_t$ are dropped for simplicity, and $\mathbf{f}_{k,t}$ is determined according to \Cref{pro:OUB}.



To approximate the score functions $\nabla_{\bm{X}_t} \log p_t(\bm{G}_t | \bm{G}_{\tau_k})$ and $\nabla_{\bm{\Lambda}_t} \log p_t(\bm{G}_t | \bm{G}_{\tau_k})$, we employ a neural network $\bm{s}^{(k)}_{\bm{\theta}}(\bm{G}_t, \bm{G}_{\tau_k},t)$, composed of a node ($\bm{s}^{(k)}_{\bm{\theta},\bm{X}}(\bm{G}_t, \bm{G}_{\tau_k},t)$) and a spectrum ($\bm{s}^{(k)}_{\bm{\theta},\bm{\Lambda}}(\bm{G}_t, \bm{G}_{\tau_k},t)$) output, respectively.
The model is optimized by minimizing the loss function:
\fontsize{8pt}{8pt}\selectfont
\begin{align}
\ell^{(k)}(\bm{\theta})=&
\mathbb{E}_{t,\bm{G}_t,\bm{G}_{\tau_{k-1}},\bm{G}_{\tau_k}} \{\omega(t) [c_1\|
\bm{s}^{(k)}_{\bm{\theta},\bm{X}} - \nabla_{\bm{X}} \log p_t(\bm{G}_t | \bm{G}_{\tau_k})\|_2^2 \nonumber \\  
+&c_2 ||\bm{s}^{(k)}_{\bm{\theta},\bm{\Lambda}} - \nabla_{\bm{\Lambda}} \log p_t(\bm{G}_t | \bm{G}_{\tau_k})||_2^2]\}, \label{eq:final-loss}  
\end{align}
\normalsize 
where $\omega(t)$ is a positive weighting function, and $c_1, c_2$ controls the relative importance of vertices and spectrum.
The training procedure is detailed in \cref{alg:train} in the Appendix.




We sample $(\bm{X}_{\tau_K},\bm{\Lambda}_{\tau_K})$ from the prior distribution and uniformly sample $\bm{U}_0$ from the observed eigenvector matrices.
The generation process involves multi-step diffusion to produce samples $(\hat{\bm{X}}_{\tau_{K-1}}, \hat{\bm{\Lambda}}_{\tau_{K-1}}), \cdots, (\hat{\bm{X}}_{\tau_1}, \hat{\bm{\Lambda}}_{\tau_1}), (\hat{\bm{X}}_0, \hat{\bm{\Lambda}}_0)$ in sequence by reversing the diffusion bridge, where the endpoint of one generation step serves as the starting point for the next. 
Finally, plausible samples with higher-order structures $\hat{\bm{G}}_0=(\hat{\bm{X}}_0 , \hat{\bm{L}}_0 =\bm{U}_0 \hat{\bm{\Lambda}}_0 \bm{U}_0^\top)$ can be reconstructed.
The complete sampling procedure is outlined in \cref{alg:sample} within the Appendix.


\begin{table*}[t!]
\vspace{-3mm}
\centering
\caption{Comparison of different methods based on molecular datasets. The best results for the first three metrics are highlighted in bold.}
\resizebox{\textwidth}{!}{ 
\begin{tabular}{lcccccc cccccc}
\toprule
\multirow{3}{*}{Methods} 
& \multicolumn{6}{c}{QM9} 
& \multicolumn{6}{c}{ZINC250k} \\
\cmidrule(lr){2-7} \cmidrule(lr){8-13}
& NSPDK$\downarrow$ 
& FCD$\downarrow$ 
& \makecell{Val. w/o \\ corr.$\uparrow$ } 
& Val.$\uparrow$ 
& Uni.$\uparrow$ 
& Nov.$\uparrow$ 
& NSPDK $\downarrow$ 
& FCD $\downarrow$ 
& \makecell{Val. w/o \\ corr.$\uparrow$ } 
& Val.$\uparrow$ 
& Uni.$\uparrow$ 
& Nov.$\uparrow$ \\
\midrule
GraphAF     
& 0.020 & 5.268  & 67.00  & 100.00 & 94.51 & 88.83 & 0.044 & 16.289 & 68.00 & 100.00 & 99.10 & 100.00 \\
GraphAF+FC  
& 0.021 & 5.625  & 74.43  & 100.00 & 86.59 & 89.57 & 0.044 & 16.023 & 68.47 & 100.00 & 98.64 & 99.99 \\
GraphDF     
& 0.063 & 10.816 & 82.67  & 100.00 & 97.62 & 98.10 & 0.176 & 34.202 & 89.03 & 100.00 & 99.16 & 99.99 \\
GraphDF+FC  
& 0.064 & 10.928 & 93.88  & 100.00 & 98.32 & 98.54 & 0.177 & 33.546 & 90.61 & 100.00 & 99.63 & 100.00 \\
\midrule
MoFlow      
& 0.017 & 4.467  & 91.36  & 100.00 & 98.65 & 94.72 & 0.046 & 20.931 & 63.11 & 100.00 & 99.99 & 99.99 \\
%GraphCNF    
%&  -    &   -    &   -    &   -    &   -   &   -   & 0.021 & 13.532 & 96.35 & 100.00 & 99.64 & 100.00 \\
EDP-GNN     
& 0.005 & 2.680  & 47.52  & 100.00 & 99.25 & 86.58 & 0.049 & 16.737 & 82.97 & 100.00 & 99.79 & 99.99 \\
GraphEBM    
& 0.003 & 6.143  & 8.22   & 100.00 & 99.25 & 85.48 & 0.212 & 35.471 & 5.29  & 99.96  & 98.79 & 99.99 \\
GDSS        
& 0.003 & 2.900  & 95.72  & 100.00 & 98.46 & 86.27 & 0.019 & 14.656 & 97.01 & 100.00 & 99.64 & 100.00 \\
DiGress     
& 0.0005 & 0.360 & \Fi{99.00}  & 100.00 & 96.66 & 33.40 & 0.082 & 23.060 & 91.02 & 100.00 & 81.23 & 100.00 \\
\Fi{HOG-Diff}        
& \Fi{0.0003} & \Fi{0.172} & 98.74  & 100.00 & 97.01 & 75.12 
& \Fi{0.001} & \Fi{1.533} & \Fi{98.56} & 100.00 & 99.96 & 99.43 \\
\bottomrule
\end{tabular}}
\label{tab:mol_rel}
\vspace{-3.3mm}
\end{table*}

\paragraph{Denoising Network Architecture}
We design a neural network $\bm{s}^{(k)}_{\bm{\theta}}(\bm{G}_t, \bm{G}_{\tau_k},t) $ to estimate score functions in \cref{eq:reverse-HoGD}.
Standard graph neural networks designed for classical tasks such as graph classification and link prediction may be inappropriate for graph distribution learning due to the immediate real-number graph states and the complicated requirements. 
For example, an effective model for molecular graph generation should capture local node-edge dependence for chemical valency rules and attempt to recover global graph patterns like edge sparsity, frequent ring subgraphs, and atom-type distribution.



% Our designed model
To achieve this, we introduce a unified denoising network that explicitly integrates node and spectral representations. 
As illustrated in \cref{fig:denoising-model} of Appendix, the network comprises two different graph processing modules: a standard graph convolution network (GCN) \cite{GCN+ICLR2017} for local feature aggregation and a graph transformer network (ATTN)~ \cite{TFmodel2021AAAIworkshop,DiGress+ICLR2023} for global information extraction.
The outputs of these modules are fused with time information through a Feature-wise Linear Modulation (FiLM) layer~\cite{Film+AAAI2018}. The resulting representations are concatenated to form a unified hidden embedding.
This hidden embedding is further processed through separate multilayer perceptrons (MLPs) to produce predictions for $\nabla_{\bm{X}} \log p(\bm{G}_t|\bm{G}_{\tau_k})$ and $\nabla_{\bm{\Lambda}} \log p(\bm{G}_t|\bm{G}_{\tau_k})$, respectively.
%
It is worth noting that our graph noise prediction model is permutation equivalent as each component of our model avoids any node ordering-dependent operations.
%
Our model is detailed in \cref{app:denoising-model}.






\paragraph{Theoretical Analysis}
%\subsection{Theoretical Analysis}
%\label{sec:theorms}
In the following, we provide supportive theoretical evidence for the efficacy of HOG-Diff, demonstrating that the proposed framework achieves faster convergence in score-matching and tighter reconstruction error bounds compared to standard graph diffusion works.

\begin{proposition}[Informal]
\label{pro:training}
Suppose the loss function $\ell^{(k)}(\bm{\theta})$ in \cref{eq:final-loss} is $\beta$-smooth and satisfies the $\mu$-PL condition in the ball $B\left(\boldsymbol{\theta}_0, R\right)$. 
Then, the expected loss at the $i$-th iteration of the training process satisfies:
\begin{equation}
%\fontsize{7pt}{7pt}\selectfont
\mathbb{E}\left[\ell^{(k)}(\bm{\theta}_i)\right] 
\leq \left(1-\frac{b\mu^2}{\beta N(\beta N^2+\mu(b-1))}\right)^i \ell^{(k)}\left(\bm{\theta}_0\right),\nonumber
%\fontsize{10pt}{10pt}\selectfont
\end{equation}
where $N$ denotes the size of the training dataset, and $b$ is the mini-batch size.
Furthermore, it holds that $\beta_{\text{HOG-Diff}}\leq \beta_{\text{classical}}$, implying that the distribution learned by the proposed framework converges to the target distribution faster than classical generative models.
\end{proposition}



Following \citet{GSDM+TPAMI2023},  we define the expected reconstruction error at each generation process as $\mathcal{E}(t)=\mathbb{E}\norm{\bar{\bm{G}}_t-\widehat{\bm{G}}_t}^2$, where $\bar{\bm{G}}_t$ represents the data reconstructed sing the ground truth score $\nabla \log p_t(\cdot)$ and $\widehat{\bm{G}}_t$ denotes the data reconstructed with the learned score function $\bm{s}_{\bm{\theta}}$.
We establish that the reconstruction error in HOG-Diff is bounded more tightly than in classical graph generation models, ensuring superior sample quality.
\begin{proposition}
\label{pro:reconstruction-error}
Under appropriate Lipschitz and boundedness assumptions, the reconstruction error of HOG-Diff satisfies the following bound: 
\begin{equation}
%\fontsize{8.5pt}{8.5pt}\selectfont
\mathcal{E}(0)
\leq 
\alpha(0)\exp{\int_0^{\tau_1} \gamma(s) } \diff{s},
%\fontsize{10pt}{10pt}\selectfont
\end{equation}
where $\alpha(0)=C^2 \ell^{(1)} (\bm{\theta}) \int_0^{\tau_1} g_{1,s}^4 \diff{s}
+ C \mathcal{E}(\tau_1) \int_0^{\tau_1} h_{1,s}^2 \diff{s}$, 
$\gamma(s) = C^2 g_{1,s}^4 \|\bm{s}_{\bm{\theta}}(\cdot,s)\|_{\mathrm{lip}}^2 
 + C \|h_{1,s}\|_{\mathrm{lip}}^2$,
and $h_{1,s} = \theta_s \left(1 + \frac{2}{e^{2\bar{\theta}_{s:\tau_1}}-1}\right)$.
%
Furthermore, we can derive that the reconstruction error bound of HOG-Diff is sharper than classical graph generation models.
\end{proposition}


The propositions above rely primarily on mild assumptions, such as smoothness and boundedness, without imposing strict conditions like the target distribution being log-concave or satisfying the log-Sobolev inequality.
%
Their formal statements and detailed proofs are postponed to \cref{app:proof}.
We experimentally verify these Propositions in \Cref{sec:ablations}.


\section{Fine-Tuning Experiments}
This section validates that our dataset can enhance the GUI grounding capabilities of VLMs and that the proposed functionality grounding and referring are effective fine-tuning tasks.
\subsection{Experimental Settings}
\noindent\textbf{Evaluation Benchmarks} We base our evaluation on the UI grounding benchmarks for various scenarios: \textbf{FuncPred} is the test split from our collected functionality dataset. This benchmark requires a model to locate the element specified by its functionality description. \textbf{ScreenSpot}~\citep{cheng2024seeclick} is a benchmark comprising test samples on mobile, desktop, and web platforms. It requires the model to locate elements based on short instructions. \textbf{RefExp}~\citep{Bai2021UIBertLG} is to locate elements given crowd-sourced referring expressions. \textbf{VisualWebBench (VWB)}~\citep{liu2024visualwebbench} is a comprehensive multi-modal benchmark assessing the understanding capabilities of VLMs in web scenarios. We select the element and action grounding tasks from this benchmark. To better align with high-level semantic instructions for potential agent requirements and avoid redundancy evaluation with ScreenSpot, we use ChatGPT to expand the OCR text descriptions in the original task instructions, such as \textit{Abu Garcia College Fishing} into functionality descriptions like \textit{This element is used to register for the Abu Garcia College Fishing event}.
\textbf{MOTIF}~\citep{Burns2022ADF} requires an agent to complete a natural language command in mobile Apps.
For all of these benchmarks, we report the grounding accuracy (\%): $\text { Acc }= \sum_{i=1}^N \mathbf{1}\left(\text {pred}_i \text { inside GT } \text {bbox}_i\right) / N \times 100 $ where $\mathbf{1}$ is an indicator function and $N$ is the number of test samples. This formula denotes the percentage of samples with the predicted points lying within the bounding boxes of the target elements.

\noindent\textbf{Training Details}
We select Qwen-VL-10B~\citep{bai2023qwen} and SliME-8B~\citep{slime} as the base models and fine-tune them on 25k, 125k, and 702k samples of the AutoGUI training data to investigate how the AutoGUI data enhances the UI grounding capabilities of the VLMs. The models are fine-tuned on 8 A100 GPUs for one epoch. We follow SeeClick~\citep{cheng2024seeclick} to fine-tune Qwen-VL with LoRA~\citep{hu2022lora} and follow the recipe of SliME~\citep{slime} to fine-tune it with only the visual encoder frozen (More details in Sec.~\ref{sec:supp:impl details}).

\noindent\textbf{Compared VLMs}
We compare with both general-purpose VLMs (i.e., LLaVA series~\citep{liu2023llava,liu2024llavanext}, SliME~\citep{slime}, and Qwen-VL~\citep{bai2023qwen}) and UI-oriented ones (i.e., Qwen2-VL~\citep{qwen2vl}, SeeClick~\citep{cheng2024seeclick}, CogAgent~\citep{hong2023cogagent}). SeeClick finetunes Qwen-VL with around 1 million data combining various data sources, including a large proportion of human-annotated UI grounding/referring samples. CogAgent is trained with a huge amount of text recognition, visual grounding, UI understanding, and publicly available text-image datasets, such as LAION-2B~\citep{LAION5B}. During the evaluation, we manually craft grounding prompts suitable for these VLMs.
\subsection{Experimental Results and Analysis}
\begin{table}[]
\scriptsize
\centering
\caption{\textbf{Element grounding accuracy on the used benchmarks.} We compare the base models fine-tuned with our AutoGUI data and representative open-source VLMs. The results show that the two base models (i.e. Qwen-VL and SliME-8B) obtain significant performance gains over the benchmarks after being fine-tuned with AutoGUI data. Moreover, increasing the AutoGUI data size consistently improves grounding accuracy, demonstrating notable scaling effects. $\dag$ means the metric value is borrowed from the benchmark paper. $*$ means using additional SeeClick training data.}
\label{tab:eval results}
\begin{tabular}{@{}cccccccccc@{}}
\toprule
Type & Model    & Size    & FuncPred & VWB EG & VWB AG & MoTIF & RefExp & ScreenSpot  \\ \midrule
\multirow{5}{*}{General} & LLaVA-1.5~\citep{liu2023llava} & 7B & 3.2      &        12.1$^{\dag}$        &     13.6$^{\dag}$           &  7.2   &  4.2 & 5.0 & \\
 & LLaVA-1.5~\citep{liu2023llava} & 13B & 5.8      &           16.7     &        9.7        &   12.3 &  20.3   & 11.2 &  \\
 & LLaVA-1.6~\citep{liu2024llavanext} & 34B &  4.4      &      19.9          &    17.0            &   7.0 &  29.1  & 10.3 &  \\
 & SliME~\citep{slime} & 8B &  3.2  &   6.1       &     4.9     & 7.0  &  8.3  &  13.0  \\ 

 & Qwen-VL~\citep{bai2023qwen} & 10B &  3.0     &      1.7          &      3.9          &    7.8 &  8.0  & 5.2$^{\dag}$   \\ 
 \midrule
\multirow{3}{*}{UI-VLM} &  Qwen2-VL~\citep{bai2023qwen}  & 7B     &     7.8       &    3.9        &  3.9  &  16.7 & 32.4 & 26.1    \\
 & CogAgent~\citep{hong2023cogagent} & 18B    &  29.3   &    \underline{55.7}      &    \textbf{59.2}      & \textbf{24.7}   & 35.0 &  47.4$^{\dag}$  \\
 & SeeClick~\citep{cheng2024seeclick} & 10B    &    19.8     &    39.2           &     27.2           & 11.1  &  \textbf{58.1}  & \underline{53.4}$^{\dag}$ \\ 
\midrule
\multirow{4}{*}{Finetuned} &  Qwen-VL-AutoGUI25k & 10B      &    14.2     &      12.8         &    12.6           &   10.8    &  12.0 & 19.0    \\
 & Qwen-VL-AutoGUI125k  & 10B       &     25.5     &      23.2         &        29.1       &    11.5   &  14.9 & 32.0     \\ 
 & Qwen-VL-AutoGUI702k  & 10B       &   43.1   &    38.0       &     32.0    &  15.5  & 23.9 &    38.4   \\
& Qwen-VL-AutoGUI702k$^*$   & 10B     &  \underline{50.0}  &    \textbf{56.2}    &  \underline{45.6}  & \underline{21.0} & \underline{51.5} & \textbf{54.2}      \\
\midrule
\multirow{3}{*}{Finetuned} & SliME-AutoGUI25k  & 8B     &   28.0   &     14.0      &      10.6      &  14.3   & 18.4 & 27.2   \\
 & SliME-AutoGUI125k   & 8B      &   39.9    &  22.0   &     12.0       &  17.8  & 22.1 &  35.0     \\
 & SliME-AutoGUI702k   & 8B      &     \textbf{62.6}   &       25.4        &     13.6          &   20.6    & 26.7 & 44.0 &          \\
\bottomrule
\end{tabular}
\end{table}
\vspace{-2mm}


\noindent\textbf{A) AutoGUI functionality annotations effectively enhance VLMs' UI grounding capabilities and achieve scaling effects.} We endeavor to show that the element functionality data autonomously collected by AutoGUI contributes to high grounding accuracy. The results in Tab.~\ref{tab:eval results} demonstrate that on all benchmarks the two base models achieve progressively rising grounding accuracy as the functionality data size scales from 25k to 702k, with SliME-8B's accuracy increasing from merely \textbf{3.2} and \textbf{13.0} to \textbf{62.6} and \textbf{44.0} on FuncPred and ScreenSpot, respectively. This increase is visualized in Fig.~\ref{fig:funcpred scaling success} showing that increasing AutoGUI data amount leads to more precise localization performance.

After fine-tuning with AutoGUI 702k data, the two base models surpass SeeClick, the strong UI-oriented VLM on FuncPred and MOTIF. We notice that the base models lag behind SeeClick and CogAgent on ScreenSpot and RefExp, as the two benchmarks contain test samples whose UIs cannot be easily recorded (e.g., Apple devices and Desktop software) as training data, causing a domain gap. Nevertheless, SliME-8B still exhibits noticeable performance improvements on ScreenSpot and RefExp when scaling up the AutoGUI data, suggesting that the AutoGUI data helps to enhance grounding accuracy on the out-of-domain tasks.

To further unleash the potential of the AutoGUI data, the base model, Qwen-VL, is finetuned with the combination of the AutoGUI and SeeClick UI-grounding data. This model becomes the new state-of-the-art on FuncPred, ScreenSpot, and VWB EG, surpassing SeeClick and CogAgent. This result suggests that our AutoGUI data can be mixed with existing UI grounding training data to foster better UI grounding capabilities.

In summary, our functionality data can endow a general VLM with stronger UI grounding ability and exhibit clear scaling effects as the data size increases.


\begin{table}[]
\centering
\footnotesize
\caption{\textbf{Comparing the AutoGUI functionality annotation type with existing types}. Qwen-VL is fine-tuned with the three annotation types. The results show that our functionality data leads to superior grounding accuracy compared with the naive element-HTML data and the condensed functionality annotations.}
\label{tab:ablation}
\begin{tabular}{@{}ccccc@{}}
\toprule
Data Size             & Variant          & FuncPred & RefExp & ScreenSpot \\ \midrule
\multirow{3}{*}{25k}  & w/ Elem-HTML data     &  5.3      &  4.5   &    5.7     \\
                      & w/ Condensed Func. Anno.     &  3.8   &  3.0  &   4.8      \\
                      & w/ Func. Anno. (Ours full)         &    \textbf{21.1}    &   \textbf{10.0}   &   \textbf{16.4}    \\ \midrule
\multirow{3}{*}{125k} & w/ Elem-HTML data     &  15.5   &  7.8  &   17.0      \\
                      & w/ Condensed Func. Anno.     &  14.1   &  11.7  &   23.8      \\
                      & w/ Func. Anno. (Ours full)         &  \textbf{24.6}   &  \textbf{12.7}  &   \textbf{27.0}    \\ \bottomrule
\end{tabular}
\end{table}



\noindent\textbf{B) Our functionality annotations are effective for enhancing UI grounding capabilities.} To assess the effectiveness of functionality annotations, we compare this annotation type with two existing types: 1) \textbf{Naive element-HTML pairs}, which are directly obtained from the UI source code~\citep{hong2023cogagent} and associate HTML code with elements in specified areas of a screenshot. Examples are shown in Fig.~\ref{fig: functionality vs others}. To create these pairs, we replace the functionality annotations with the corresponding HTML code snippets recorded during trajectory collection. 2) \textbf{Brief functionality descriptions} that are generated by prompting GPT-4o-mini\footnote{https://openai.com/index/gpt-4o-mini-advancing-cost-efficient-intelligence/} to condense the AutoGUI functionality annotations. For example, a full description such as \textit{`This element provides access to a documentation category, allowing users to explore relevant information and guides'} is shortened to \textit{`Documentation category access'}.

After experimenting with Qwen-VL~\citep{bai2023qwen} at the 25k and 125k scales, the results in Tab.~\ref{tab:ablation} show that fine-tuning with the complete functionality annotations is superior to the other two types. Notably, our functionality annotation type yields the largest gain on the challenging FuncPred benchmark that emphasizes contextual functionality grounding. In contrast, the Elem-HTML type performs poorly due to the noise inherent in HTML code (e.g., numerous redundant tags), which reduces fine-tuning efficiency. The condensed functionality annotations are inferior, as the consensing loses details necessary for fine-grained UI understanding. In summary, the AutoGUI functionality annotations provide a clear advantage in enhancing UI grounding capabilities.


\subsection{Failure Case Analysis}
After analyzing the grounding failure cases, we identified several failure patterns in the fine-tuned models: a) difficulty in accurately locating small elements; b) challenges in distinguishing between similar but incorrect elements; and c) issues with recognizing icons that have uncommon shapes. Please refer to Sec.~\ref{sec:supp:case analysis} for details.



\section{Related Works}
\vspace{-1mm}
We review graph generation methods along with higher-order generation. \cref{app:related} presents a more detailed review.



\paragraph{Deep Generative Models}
Graph generative models make great progress by exploiting the capacity of deep neural networks. These models typically generate nodes and edges either in an autoregressive manner or simultaneously, utilizing techniques such as variational autoencoders (VAE) \cite{VAE-Jin2018,GraphVAE-DrugDiscovery}, recurrent neural networks (RNN) \cite{GraphRNN2018}, normalizing flows \cite{Moflow-SIGKDD2020,GraphAF-ICLR2020,GraphDF-ICML2021}, and generative adversarial networks (GAN) \cite{GAN1-MolGAN,GAN2-Spectre}.

\paragraph{Diffusion-based Graph Generation}
A breakthrough in graph generative models has been marked by the recent progress in diffusion-based generative models \cite{EDPGNN-2020}.
Recent models employ various strategies to enhance the generation of complex graphs, including capturing node-edge dependency \cite{GDSS+ICML2022}, addressing discretization challenges \cite{DiGress+ICLR2023,CDGS+AAAI2023}, exploiting low-to-high frequency generation curriculum \cite{GPrinFlowNet+ACM2024}, and improving computational efficiency through low-rank diffusion processes \cite{GSDM+TPAMI2023}. 
% ------ Diffusion bridge models
Recent studies have also enhanced diffusion-based generative models by incorporating diffusion bridge processes, \ie, processes conditioned on the endpoints \cite{wu2022diffusion,GLAD-ICMLworkshop2024,GruM+ICML2024}.
%
% ---- short conclusion
Despite these advances, existing methods either overlook or inadvertently disrupt higher-order structures during graph generation, or struggle to model the topological properties, as denoising the noisy samples does not explicitly preserve the intricate structural dependencies required for generating realistic graphs.




\paragraph{Higher-order Generative Models}
Generative modelling uses higher-order information mostly in the form of hypergraphs.
Models such as Hygene~\cite{gailhard2024hygene} and HypeBoy~\cite{kim2024hypeboy} aim to generate hypergraphs. Dymond~\cite{zeno2021dymond} focuses on higher-order motifs in dynamic graphs. 
To the best of our knowledge, we are the first to consider higher-order guides for graph generation.
In this paper, we systematically investigate the position bias problem in the multi-constraint instruction following. To quantitatively measure the disparity of constraint order, we propose a novel Difficulty Distribution Index (CDDI). Based on the CDDI, we design a probing task. First, we construct a large number of instructions consisting of different constraint orders. Then, we conduct experiments in two distinct scenarios. Extensive results reveal a clear preference of LLMs for ``hard-to-easy'' constraint orders. To further explore this, we conduct an explanation study. We visualize the importance of different constraints located in different positions and demonstrate the strong correlation between the model's attention distribution and its performance.
\clearpage

\section*{Impact Statement}
This paper presents work whose goal is to advance the field of deep generative models. 
%There are many potential societal consequences of our work, none of which we feel must be specifically highlighted here.
Positive applications include generating graph-structured data for scientific discovery and accelerating drug discovery by generating novel molecular structures.
%
However, like other generative technologies, our work could potentially be misused to synthesize harmful molecules, counterfeit social interactions, or deceptive network structures. 

\bibliography{ref}
\bibliographystyle{icml2025}

% WARNING: do not forget to delete the supplementary pages from your submission 
\newpage
\appendix
\onecolumn
\begin{center}{\bf \Large Appendix}\end{center}
%\section*{Appendix}
\vspace{0.15in}


\paragraph{Organization} 
The appendix is structured as follows: 
We first present the derivations excluded from the main paper due to space limitation in Section~\ref{app:proof}.
Section~\ref{app:ho-intro} introduces the concept and examples of higher-order networks.
Additional explanations on related work are provided in Section~\ref{app:related}. 
Section~\ref{app:detail-HOG-Diff} details the generation process, including the architecture of the proposed denoising network, as well as the training and sampling procedures.
Computational efficiency is discussed in Section~\ref{app:complexity}.
Section~\ref{app:exp_set} outlines the experimental setup, and Section~\ref{app:vis} concludes with visualizations of the generated samples.


\section{Formal Statements and Proofs}
\label{app:proof}
This section presents the formal statements of key theoretical results and their detailed derivations. 
We will recall and more precisely state the propositions before presenting the proof.

\subsection{Diffusion Bridge Process}

In the following, we derive the Generalized Ornstein-Uhlenbeck (GOU) bridge process using Doob's $h$-transform \cite{doob-h-transform1984} and analyze its relationship with the Brownian bridge process.

Recall that the generalized Ornstein-Uhlenbeck (GOU) process is the time-varying OU process.
It is a stationary Gaussian-Markov process whose marginal distribution gradually tends towards a stable mean and variance over time. 
The GOU process $\mathbb{Q}$ is generally defined as follows \cite{GOU1988,IRSDE+ICML2023}:
\begin{equation}
\mathbb{Q}: \mathrm{d}\bm{G}_t=\theta_t\left(\bm{\mu}-\bm{G}_t\right)\mathrm{d}t+g_t\mathrm{d}\bm{W}_t,
\end{equation}
where $\bm{\mu}$ is a given state vector, $\theta_t$ denotes a scalar drift coefficient and $g_t$ represents the diffusion coefficient. At the same time, we require $\theta_t,g_t$ to satisfy the specified relationship $2\sigma^2=g_t^2/\theta_t$, where $\sigma^2$ is a given constant scalar. As a result, its transition probability possesses a closed-form analytical solution:
\begin{equation}
\begin{split}
p\left(\bm{G}_{t}\mid \bm{G}_s\right)
& =\mathcal{N}(\mathbf{m}_{s:t},v_{s:t}^{2}\bm{I}), \\
\mathbf{m}_{s:t} 
& = \bm{\mu}+\left(\bm{G}_s-\bm{\mu}\right)e^{-\bar{\theta}_{s:t}},\\
v_{s:t}^{2} 
&= \sigma^2 \left(1-e^{-2\bar{\theta}_{s:t}}\right).
\end{split}
\end{equation}
Here, $\bar{\theta}_{s:t}=\int_s^t\theta_zdz$. When the starting time $t=0$, we substitute $\bar{\theta}_{0:t}$ with $\bar{\theta}_t$ for notation simplicity. 





\begin{customthe}[Proposition~\ref{pro:OUB}]
%Let $\bm{G}_t$ evolve according to the generalized OU process in \cref{eq:GOU-SDE}, subject to the terminal conditional $\bm{\mu}=\bm{G}_{\tau_k}$. 
%Then, the evolution of the conditional marginal distribution $p(\bm{G}_t \mid \bm{G}_{\tau_k})$ satisfies the following SDE:
The conditional marginal distribution $p(\bm{G}_t\mid\bm{G}_{\tau_k})$ then evolves according to the following SDE:
\begin{equation}
%\fontsize{8.5pt}{8.5pt}\selectfont
\mathrm{d}\bm{G}_t = \theta_t \left( 1 + \frac{2}{e^{2\bar{\theta}_{t:\tau_k}}-1}  \right)(\bm{G}_{\tau_k} - \bm{G}_t)  \mathrm{d}t 
+ g_{k,t} \mathrm{d}\bm{W}_t.\nonumber
%\fontsize{10pt}{10pt}\selectfont
\end{equation}
The conditional transition probability $p(\bm{G}_t \mid \bm{G}_{\tau_{k-1}}, \bm{G}_{\tau_k})$ has analytical form as follows:
\begin{equation}
\begin{split}
&p(\bm{G}_t \mid  \bm{G}_{\tau_{k-1}}, \bm{G}_{\tau_k}) 
= \mathcal{N}(\bar{\mathbf{m}}_t, \bar{v}_t^2 \bm{I}),\\
&\bar{\mathbf{m}}_t = 
\bm{G}_{\tau_k} + (\bm{G}_{\tau_{k-1}}-\bm{G}_{\tau_k})e^{-\bar{\theta}_{\tau_{k-1}:t}} 
\frac{v_{t:\tau_k}^2}{v_{\tau_{k-1}:\tau_k}^2}, \\
&\bar{v}_t^2 = {v_{\tau_{k-1}:t}^2 v_{t:\tau_k}^2}/{v_{\tau_{k-1}:\tau_k}^2}.
\end{split}
\end{equation}
Here, $\bar{\theta}_{a:b}=\int_a^b \theta_s  \mathrm{d}s$, and $v_{a:b}=\sigma^2(1-e^{-2\bar{\theta}_{a:b}})$.
Let $\bm{G}_t$ evolve according to the generalized OU process in \cref{eq:GOU-SDE}, subject to the terminal conditional $\bm{\mu}=\bm{G}_{\tau_k}$. 
%
The conditional marginal distribution $p(\bm{G}_t\mid\bm{G}_{\tau_k})$ then evolves according to the following SDE:
\begin{equation}
\mathrm{d}\bm{G}_t = \theta_t \left( 1 + \frac{2}{e^{2\bar{\theta}_{t:\tau_k}}-1}  \right)(\bm{G}_{\tau_k} - \bm{G}_t)  \mathrm{d}t 
+ g_{k,t} \mathrm{d}\bm{W}_t.
\label{eq:GOUB-SDE}
\end{equation}
The conditional transition probability $p(\bm{G}_t \mid \bm{G}_{\tau_{k-1}}, \bm{G}_{\tau_k})$ has analytical form as follows:
\begin{equation}
\begin{split}
&p(\bm{G}_t \mid  \bm{G}_{\tau_{k-1}}, \bm{G}_{\tau_k}) 
= \mathcal{N}(\bar{\mathbf{m}}_t, \bar{v}_t^2 \bm{I}),\\
&\bar{\mathbf{m}}_t = 
\bm{G}_{\tau_k} + (\bm{G}_{\tau_{k-1}}-\bm{G}_{\tau_k})e^{-\bar{\theta}_{\tau_{k-1}:t}} 
\frac{v_{t:\tau_k}^2}{v_{\tau_{k-1}:\tau_k}^2}, \\
&\bar{v}_t^2 = {v_{\tau_{k-1}:t}^2 v_{t:\tau_k}^2}/{v_{\tau_{k-1}:\tau_k}^2}.
\end{split}
\end{equation}
Here, $\bar{\theta}_{a:b}=\int_a^b \theta_s  \mathrm{d}s$, and $v_{a:b}=\sigma^2(1-e^{-2\bar{\theta}_{a:b}})$.
\end{customthe}

\begin{proof}
To simplify the notion, in the $k$-th generation step, we adopt the following conventions: 
 $T=\tau_k$, $\mathbf{x}_t = \bm{G}_t^{(k)}$, $0=\tau_{k-1}$, $\mathbf{x}_0=\bm{G}_{\tau_{k-1}}$, $\mathbf{x}_T=\bm{G}_{\tau_k}$. 

From \cref{eq:GOU-p}, we can derive the following conditional distribution
\begin{equation}
    p(\mathbf{x}_T \mid \mathbf{x}_t)=\mathcal{N}(
    \mathbf{x}_T + (\mathbf{x}_t-\mathbf{x}_T) e^{\bar{\theta}_{t:T}},
    v_{t:T}^2 \bm{I}).
\end{equation}
Hence, the $h$-function can be directly computed as:
\begin{equation}
\begin{split}
\bm{h}(\mathbf{x}_t, t, \mathbf{x}_T, T) 
& = \nabla_{\mathbf{x}_t} \log p(\mathbf{x}_T \mid \mathbf{x}_t)\\
& = -\nabla_{\mathbf{x}_t} \left[\frac{(\mathbf{x}_t - \mathbf{x}_T)^2 e^{-2 \bar{\theta}_{t:T}}}{2 v_{t:T}^2} + const \right]\\
& = (\mathbf{x}_T - \mathbf{x}_t) \frac{e^{-2 \bar{\theta}_{t:T}}}{v_{t:T}^2} \\
& = (\mathbf{x}_T - \mathbf{x}_t) \sigma^{-2}/(e^{2\bar{\theta}_{t:T}}-1).
\end{split}
\end{equation}


Then the Doob's $h$-transform yields the representation of an endpoint $\mathbf{x}_T$ conditioned process defined by the following SDE: 
% 
\begin{equation}
\begin{split}
\mathrm{d}\mathbf{x}_t 
&= \left[ f(\mathbf{x}_t, t) + g_t^2 \bm{h}(\mathbf{x}_t, t, \mathbf{x}_T, T) \right] \mathrm{d}t + g_t \mathrm{d}\mathbf{w}_t\\
&= \left( \theta_t + \frac{g_t^2}{\sigma^2 (e^{2\bar{\theta}_{t:T}}-1)}  \right)(\mathbf{x}_T - \mathbf{x}_t)  \mathrm{d}t + g_t \mathrm{d}\mathbf{w}_t \\
& = \theta_t \left( 1 + \frac{2}{e^{2\bar{\theta}_{t:T}}-1}  \right)(\mathbf{x}_T - \mathbf{x}_t)  \mathrm{d}t + g_t \mathrm{d}\mathbf{w}_t.
\end{split}
\end{equation}

Given that the joint distribution of $[\mathbf{x}_0, \mathbf{x}_t, \mathbf{x}_T]$ is multivariate normal, the conditional distribution $p(\mathbf{x}_t \mid \mathbf{x}_0, \mathbf{x}_T)$ is also Gaussian:
\begin{equation}
    p(\mathbf{x}_t\mid \mathbf{x}_0, \mathbf{x}_T) = \mathcal{N}(\bar{\mathbf{m}}_t, \bar{v}_t^2 \bm{I}),
\end{equation}
where the mean $\bar{\mathbf{m}}_t$ and variance $\bar{v}_t^2$ are determined using the conditional formulas for multivariate normal variables:
\begin{equation}
\begin{split}
\bar{\mathbf{m}}_t 
=  \mathbb{E}[\mathbf{x}_t\mid \mathbf{x}_0 \mid \mathbf{x}_T]
=\mathbb{E}[\mathbf{x}_t\mid \mathbf{x}_0]+\mathrm{Cov}(\mathbf{x}_t,\mathbf{x}_T\mid \mathbf{x}_0)\mathrm{Var}(\mathbf{x}_T\mid \mathbf{x}_0)^{-1}(\mathbf{x}_T-\mathbb{E}[\mathbf{x}_T\mid \mathbf{x}_0]),\\
\bar{v}_t^2
= \mathrm{Var}(\mathbf{x}_t\mid \mathbf{x}_0 \mid \mathbf{x}_T)
=\mathrm{Var}(\mathbf{x}_t\mid \mathbf{x}_0)-\mathrm{Cov}(\mathbf{x}_t,\mathbf{x}_T\mid \mathbf{x}_0)\mathrm{Var}(\mathbf{x}_T\mid \mathbf{x}_0)^{-1}\mathrm{Cov}(\mathbf{x}_T,\mathbf{x}_t\mid \mathbf{x}_0).
\end{split}
\label{eq:OUB-m-v}
\end{equation}

Notice that
\begin{equation}
    \mathrm{Cov}(\mathbf{x}_t,\mathbf{x}_T\mid \mathbf{x}_0)=\mathrm{Cov}\left(\mathbf{x}_t,(\mathbf{x}_t-\mathbf{x}_T)e^{-\bar{\theta}_{t:T}}\mid \mathbf{x}_0\right)=e^{-\bar{\theta}t:T}\mathrm{Var}(\mathbf{x}_t\mid \mathbf{x}_0).
\end{equation}
By substituting this and the results in \cref{eq:GOU-p} into \cref{eq:OUB-m-v}, we can obtain
\begin{equation}
\begin{split}
\bar{\mathbf{m}}_t 
& = \left(\mathbf{x}_T+(\mathbf{x}_0-\mathbf{x}_T)e^{-\bar{\theta}_t}\right)
+ \left(e^{-\bar{\theta}_{t:T}} v_t^2\right)
/ v_T^2
\cdot \left(\mathbf{x}_T - \mathbf{x}_T - (\mathbf{x}_0 - \mathbf{x}_T)e^{-\bar{\theta}_T}\right) \\
& = \mathbf{x}_T + (\mathbf{x}_0-\mathbf{x}_T) \left(e^{-\bar{\theta}_t} -  e^{-\bar{\theta}_{t:T}}e^{-\bar{\theta}_T} v_t^2 /v_T^2\right) \\
& = \mathbf{x}_T + (\mathbf{x}_0-\mathbf{x}_T)e^{-\bar{\theta}_t} 
\left(\frac{1-e^{-2\bar{\theta}_{T}}-e^{-2\bar{\theta}_{t:T}}(1-e^{-2\bar{\theta}_t})}{1-e^{-2\bar{\theta}_{T}}}\right)\\
& = \mathbf{x}_T + (\mathbf{x}_0-\mathbf{x}_T)e^{-\bar{\theta}_t} 
v_{t:T}^2/v_T^2,
\end{split}
\end{equation}
and 
\begin{equation}
\begin{split}
\bar{v}_t^2
& = v_t^2 - \left(e^{-\bar{\theta}_{t:T}} v_t^2 \right)^2 / v_T^2\\
& = \frac{v_t^2}{v_T^2}(v_T^2-e^{-2\bar{\theta}_{t:T}}v_t^2)\\
& = \frac{v_t^2}{v_T^2} \sigma^2\left(1-e^{-2\bar{\theta}_T} - e^{-2\bar{\theta}_{t:T}}(1-e^{-\bar{2\theta}_t})\right)\\
& = v_t^2 v_{t:T}^2/ v_T^2.
\end{split}
\end{equation}

Finally, we conclude the proof by reverting to the original notations.
\end{proof}



Note that the generalized OU bridge process, also referred to as the conditional GOU process, has been studied theoretically in previous works \cite{salminen1984conditional,GOUB2021,GOUB+ICML2024}. However, we are the first to demonstrate its effectiveness in explicitly learning higher-order structures within the graph generation process.


\paragraph{Brownian Bridge Process}  
In the following, we demonstrate that the Brownian bridge process is a particular case of the generalized OU bridge process when $\theta_t$ approaches zero.

Assume $\theta_t = \theta$ is a constant that tends to zero, we obtain 
\begin{equation}
    \bar{\theta}_{a:b}=\int_a^b \theta_s \diff{s} = \theta (b-a)\rightarrow 0.
\end{equation}

Consider the term $ e^{2\bar{\theta}_{t:\tau_k}}-1$, we approximate the exponential function using a first-order Taylor expansion for small $\bar{\theta}_{t:\tau_k}$:
\begin{equation}
    e^{2\bar{\theta}_{t:\tau_k}}-1 
    \approx
    2\bar{\theta}_{t:\tau_k}
        \rightarrow
        2\theta(\tau_k - t).
\end{equation}
Hence, the drift term in the generalized OU bridge simplifies to
\begin{equation}
    \theta_t \left( 1 + \frac{2}{e^{2\bar{\theta}_{t:\tau_k}}-1}\right)
     \approx
     \theta\left(1+\frac{2}{2\theta(\tau_k-t)}\right)
     \rightarrow
     \frac{1}{\tau_k-t}.
\end{equation}

Consequently, in the limit $\theta_t \rightarrow 0$, the generalized OU bridge process described in \cref{eq:GOUB-SDE} can be modelled by the following SDE:
\begin{equation}
    \mathrm{d}\bm{G}_t=  \frac{\bm{G}_{\tau_k}-\bm{G}_t}{\tau_k-t}\mathrm{d}t+
    g_{k,t}\mathrm{d}\bm{W}_t.
\end{equation}
This equation precisely corresponds to the SDE representation of the classical Brownian bridge process.


In contrast to the generalized OU bridge process in \cref{eq:GOUB-SDE}, the evolution of the Brownian bridge is fully determined by the noise schedule $g_{k,t}$, resulting in a simpler SDE representation. 
However, this constraint in the Brownian bridge reduces the flexibility in designing the generative process.


Note that the Brownian bridge is an endpoint-conditioned process relative to a reference Brownian motion, which the SDE governs:
\begin{equation}
    \mathrm{d}\bm{G}_t=  
    g_{t}\mathrm{d}\bm{W}_t.
\end{equation}
This equation describes a pure diffusion process without drift, making it a specific instance of the generalized OU process in \cref{eq:GOU-SDE}.

\subsection{Proof of Proposition~\ref{pro:training}}

To establish proof, we begin by introducing essential definitions and assumptions.

\begin{definition}[$\beta$-smooth]
A function $f:\mathbb{R}^m  \to \mathbb{R}^n$ is said to be $\beta$-smooth if and only if
\begin{equation}
    \norm{f(\mathbf{w})-f(\mathbf{v})-\nabla f(\mathbf{v})(\mathbf{w}-\mathbf{v})} \leq \frac{\beta}{2} \norm{\mathbf{w}-\mathbf{v}}^2, \forall \mathbf{w},\mathbf{v}\in \mathbb{R}^m.
\end{equation}
\end{definition}

\begin{customthe}[Proposition~\ref{pro:training}\textnormal{ (Formal)}]
Let $\ell^{(k)}(\boldsymbol{\theta})$ be a loss function that is $\beta$-smooth and satisfies the $\mu$-PL (Polyak-Łojasiewicz) condition in the ball $B\left(\boldsymbol{\theta}_0, R\right)$ of radius $R=2N \sqrt{2 \beta \ell^{(k)}\left(\boldsymbol{\theta}_0\right)}/(\mu \delta)$, where $\delta>0$. 
%
Then, with probability $1-\delta$ over the choice of mini-batch of size $b$, stochastic gradient descent (SGD) with a learning rate $\eta^*=\frac{\mu N}{N \beta\left(N^2 \beta+\mu(b-1)\right)}$ converges to a global solution in the ball $B$ with exponential convergence rage: 
\begin{equation}
 \mathbb{E}\left[\ell^{(k)}\left(\boldsymbol{\theta}_i\right)\right] \leq\left(1-\frac{b \mu^2}{\beta N\left(\beta N^2+\mu(b-1)\right)}\right)^i \ell^{(k)}\left(\boldsymbol{\theta}_0\right).
\end{equation}
Here, $N$ denotes the size of the training dataset.
Furthermore, the proposed generative model yields a smaller smoothness constant $\beta_{\text{HOG-Diff}}$ compared to that of the classical model $\beta_{\text {classical}}$, \ie, $\beta_\text{HOG-Diff} \leq \beta_{\text {classical}}$, implying that the learned distribution in HOG-Diff converges to the target distribution faster than classical generative models.
\end{customthe}

\begin{proof}
Assume that the loss function $\ell^{(k)}(\bm{\theta})$ in \cref{eq:final-loss} is minimized using standard Stochastic Gradient Descent (SGD) on a training dataset $\mathcal{S}=\{\mathbf{x}^i\}_{i=1}^N$. At the $i$-th iteration, parameter $\bm{\theta}_i$ is updated using a mini-batch of size $b$ as follows:
\begin{equation}
    \bm{\theta}_{i+1} \triangleq \bm{\theta}_i - \eta \nabla \ell^{(k)}(\bm{\theta}_i),
\end{equation}
where $\eta$ is the learning rate.


Following \citet{liu2020toward} and \citet{GSDM+TPAMI2023}, we assume that $\ell^{(k)}(\bm{\theta})$ is $\beta$-smooth and satisfies the $\mu$-PL condition in the ball $B(\bm{\theta}_0, R)$ with $R=2N\sqrt{2\beta \ell^{(k)}(\bm{\theta}_0)}/(\mu\delta)$ where $\delta>0$. 
%
Then, with probability $1-\delta$ over the choice of mini-batch of size $b$, SGD with a learning  rate $\eta^* =\frac{\mu N}{N\beta (N^2\beta +\mu(b-1))}$ converges to a global solution in the ball $B(\bm{\theta}_0, R)$ with exponential convergence rate \cite{liu2020toward}:
\begin{equation}
\mathbb{E}[\ell^{(k)}(\bm{\theta}_i)] 
\leq \left(1-\frac{b\mu\eta^*}{N}\right)^i \ell^{(k)}(\bm{\theta}_0)
= \left(1-\frac{b\mu^2}{\beta N(\beta N^2+\mu(b-1))}\right)^i \ell^{(k)}(\bm{\theta}_0).
\end{equation}

% 2------
Next, we show that the proposed framework has a smaller smoothness constant than the classical one-step model. 
Therefore, we focus exclusively on the spectral component $||\bm{s}^{(k)}_{\bm{\theta},\bm{\Lambda}} - \nabla_{\bm{\Lambda}} \log p_t(\bm{G}_t | \bm{G}_{\tau_k})||_2^2$ from the full loss function in \cref{eq:final-loss}, as the feature-related part of the loss function in HOG-Diff aligns with that of the classical framework.  
For simplicity, we use the notation $\bar{\ell}(\bm{\theta})=||\bm{s}^{(k)}_{\bm{\theta},\bm{\Lambda}} - \nabla_{\bm{\Lambda}} \log p_t(\bm{G}_t | \bm{G}_{\tau_k})||^2 = ||\bm{s}_{\bm{\theta}}(\mathbf{x}_t) - \nabla_{\mathbf{x}} \log p_t(\mathbf{x}_t)||^2$ as the feature-related part of the loss.%, and let $\bar{\ell}(\bm{\varphi}) = \mathbb{E} ||s_{\bm{\varphi}}(\mathbf{x}_t) - \nabla_{\mathbf{x}} q_t(\mathbf{x}_t|\mathbf{x}_0)||^2$ be its classical counterpart.


Next, we verify that $\bar{\ell}(\bm{\theta})$ is $\beta$-smooth under the assumptions given.
Notice that the gradient of the loss function is given by:
\begin{equation}
\nabla \bar{\ell}(\bm{\theta})=2\mathbb{E}\left[(\bm{s}_{\bm{\theta}}(\mathbf{x})-\nabla\log p(\mathbf{x}))^\top\nabla_{\bm{\theta}} s_{\bm{\theta}}(\mathbf{x})\right]
\end{equation}
Hence,
\begin{equation}
\begin{split}
&\|\nabla \bar{\ell}(\bm{\theta}_1)-\nabla \bar{\ell}(\bm{\theta}_2)\|
=2\left\|\mathbb{E}\left[(\bm{s}_{\bm{\theta}_1}(\mathbf{x})-\nabla\log p(\mathbf{x}))^\top\nabla \bm{s}_{\bm{\theta}_1}(\mathbf{x})-(\bm{s}_{\bm{\theta}_2}(\mathbf{x})-\nabla\log p(\mathbf{x}))^\top\nabla \bm{s}_{\bm{\theta}_2}(\mathbf{x})\right]\right\|\\
&\leq 2\mathbb{E}[\|\bm{s}_{\bm{\theta}_{1}}(\mathbf{x})-\bm{s}_{\bm{\theta}_2}(\mathbf{x})\|\cdot\|\nabla \bm{s}_{\bm{\theta}_1}(\mathbf{x})\|  
+\|\bm{s}_{\bm{\theta}_2}(\mathbf{x})-\nabla\log p(\mathbf{x})\|\cdot\|\nabla \bm{s}_{\bm{\theta}_1}(\mathbf{x})-\nabla \bm{s}_{\bm{\theta}_2}(\mathbf{x})\|].
\end{split}
\end{equation}

Suppose $\|\nabla_{\bm{\theta}} \bm{s}_{\bm{\theta}}(\mathbf{x})\|\leq C_1$ and $\|\bm{s}_{\bm{\theta}}(\mathbf{x})-\nabla\log p(\mathbf{x})\|\leq C_2$, then we can obtain
\begin{equation}
\begin{split}
\|\nabla \bar{\ell}(\bm{\theta}_1)-\nabla \bar{\ell}(\bm{\theta}_2)\|\
& \leq 2 \mathbb{E} \left[C_1 \beta_{\bm{s}_{\bm{\theta}}}\|\bm{\theta}_1-\bm{\theta}_2\|+C_2\beta_{\nabla \bm{s}_{\bm{\theta}}}\|\bm{\theta}_1-\bm{\theta}_2\| \right] \\
& =2(\beta_{\bm{s}_{\bm{\theta}}} C_1+C_2\beta_{\nabla \bm{s}_{\bm{\theta}}})\|\bm{\theta}_1-\bm{\theta}_2\|.
\end{split}
\end{equation}

To satisfy the $\beta$-smooth of $\bar{\ell}(\bm{\theta})$, we require that
\begin{equation}
    2(C_1\beta_{\bm{s}_{\bm{\theta}}}+C_2\beta_{\nabla \bm{s}_{\bm{\theta}}}) 
\leq \beta.
\end{equation}

This implies that the distribution learned by the proposed framework can converge to the target distribution. Therefore, following \citet{CCDF+CVPR2022}, we further assume that $\bm{s}_{\bm{\theta}}$ is a sufficiently expressive parameterized score function so that 
$\beta_{\bm{s}_{\bm{\theta}}} =  \beta_{\nabla \log p_{t|\tau_{k-1}}}$ and $\beta_{\nabla^2 \bm{s}_{\bm{\theta}}} =  \beta_{\nabla^2 \log p_{t|\tau_{k-1}}}$.


%
Consider the loss function of classical generative models goes as: $\bar{\ell}(\bm{\varphi}) = \mathbb{E} ||\bm{s}_{\bm{\varphi}}(\mathbf{x}_t) - \nabla_{\mathbf{x}_t} q_t(\mathbf{x}_t|\mathbf{x}_0)||^2$.
To demonstrate that the proposed framework converges faster to the target distribution compared to the classical one-step generation framework, it suffices to show that: $\beta_{\nabla p_{t|\tau_{k-1}}} \leq \beta_{\nabla q_{t|0}}$ and $\beta_{\nabla^2 p_{t|\tau_{k-1}}} \leq \beta_{\nabla^2 q_{t|0}}$.

Let $\mathbf{x}\sim q_{t|0}$ and $\mathbf{x}'\sim p_{t|\tau_{k-1}}$. Since we inject topological information from $\mathbf{x}$ into $\mathbf{x}^{\prime}$, $\mathbf{x}'$ can be viewed as being obtained by adding noise to $\mathbf{x}$. Hence, we can model $\mathbf{x}'$ as $\mathbf{x}' = \mathbf{x} + \epsilon$ where $\epsilon \sim \mathcal{N}(\mathbf{0},\sigma^2 \bm{I})$. The variance of Gaussian noise $\sigma^2$ controls the information remained in $z'$. 
Hence, its distribution can be expressed as $p(\mathbf{x}')=\int q(\mathbf{x}'-\epsilon)\pi(\epsilon)\diff\epsilon$.

Therefore, we can obtain
\begin{equation}
\begin{split}
||\nabla_{\mathbf{x}'}^k p(\mathbf{x}'_1) - \nabla_{\mathbf{x}'}^k p(\mathbf{x}'_2)||
& =|| \nabla_{\mathbf{x}'}^k \int \left(q(\mathbf{x}_1'-\epsilon)-q(\mathbf{x}_2'-\epsilon)\right)\pi(\epsilon)\diff{\epsilon}||\\
&\leq  \int ||\nabla_{\mathbf{x}'}^k q(\mathbf{x}_1'-\epsilon) - \nabla_{\mathbf{x}'}^k q(\mathbf{x}_2'-\epsilon)|| \pi(\epsilon)\diff{\epsilon} \\ %变量代换
& \leq ||\nabla_{\mathbf{x}'}^k q(\mathbf{x}') ||_{\mathrm{lip}} (\mathbf{x}_1'-\mathbf{x}_2')  \int \pi(\epsilon)\diff{\epsilon}\\
& \leq ||\nabla_{\mathbf{x}'}^k q(\mathbf{x}') ||_{\mathrm{lip}} (\mathbf{x}_1'-\mathbf{x}_2').
\end{split}
\end{equation}


Hence, $||\nabla_{\mathbf{x}'}^k \log p(\mathbf{x}')||_{\mathrm{lip}} \leq ||\nabla_{\mathbf{x}'}^k \log q(\mathbf{x}')||_{\mathrm{lip}}$.

By setting $k=3$ and $k=4$, we can obtain $\beta_{\nabla \log p_{t|\tau_{k-1}}} \leq \beta_{\nabla \log q_{t|0}}$ and $\beta_{\nabla^2 \log p_{t|\tau_{k-1}}} \leq \beta_{\nabla^2 \log q_{t|0}}$. 
Therefore $\beta_{\text{HOG-Diff}}\leq \beta_{\text{classical}}$, implying that the training process of HOG-Diff ($\bm{s}_{\bm{\theta}}$) will converge faster than the classical generative framework ($\bm{s}_{\bm{\varphi}}$).

\end{proof}








\subsection{Proof of Proposition~\ref{pro:reconstruction-error}}

Here, we denote the expected reconstruction error at each generation process
 as $\mathcal{E}(t)=\mathbb{E}\norm{\bar{\bm{G}}_t-\widehat{\bm{G}}_t}^2$.

\begin{customthe}[Proposition~\ref{pro:reconstruction-error}]
Under appropriate Lipschitz and boundedness assumptions, the reconstruction error of HOG-Diff satisfies the following bound: 
\begin{equation}
%\fontsize{8.5pt}{8.5pt}\selectfont
\mathcal{E}(0)
\leq 
\alpha(0)\exp{\int_0^{\tau_1} \gamma(s) } \diff{s},
%\fontsize{10pt}{10pt}\selectfont
\end{equation}
where $\alpha(0)=C^2 \ell^{(1)} (\bm{\theta}) \int_0^{\tau_1} g_{1,s}^4 \diff{s}
+ C \mathcal{E}(\tau_1) \int_0^{\tau_1} h_{1,s}^2 \diff{s}$, 
$\gamma(s) = C^2 g_{1,s}^4 \|\bm{s}_{\bm{\theta}}(\cdot,s)\|_{\mathrm{lip}}^2 
 + C \|h_{1,s}\|_{\mathrm{lip}}^2$,
and $h_{1,s} = \theta_s \left(1 + \frac{2}{e^{2\bar{\theta}_{s:\tau_1}}-1}\right)$.
%
Furthermore, we can derive that the reconstruction error bound of HOG-Diff is sharper than classical graph generation models.
\end{customthe}
 

\begin{proof}

Let $\mathcal{E}(t)= \mathbb{E}\norm{\bar{\bm{G}}_t-\widehat{\bm{G}}_t}^2$, which reflects the expected error between the data reconstructed with the ground truth score $\nabla \log p_t(\cdot)$ and the learned scores $\bm{s}_{\bm{\theta}} (\cdot)$.  
%
In particular, $\bar{\bm{G}}$ is obtained by solving the following oracle reversed time SDE:
\begin{equation}
    \diff \bar{\bm{G}}_t=\left(\mathbf{f}_{k,t}(\bar{\bm{G}}_t)-g_{k,t}^2 \nabla_{\bm{G}}\log p_t(\bar{\bm{G}}_t)\right)\diff\bar{t}
    +g_{k,t}\diff \bar{\bm{W}}_t, t\in[\tau_{k-1},\tau_k],
\end{equation}
whereas $\widehat{\bm{G}}_t$ is governed based on the corresponding estimated reverse time SDE:
\begin{equation}
    \mathrm{d}\widehat{\bm{G}}_t=\left(\mathbf{f}_{k,t}(\widehat{\bm{G}}_t)-g_{k,t}^2 \bm{s}_{\bm{\theta}}(\widehat{\bm{G}}_t,t)\right)\diff\bar{t}
    +g_{k,t}\diff\bar{\bm{W}}_t, t\in [\tau_{k-1},\tau_k].
\end{equation}
Here, $\mathbf{f}_{k,t}$ is the drift function of the Ornstein–Uhlenbeck bridge. 
For simplicity, we we denote the Lipschitz norm by $||\cdot||_{\operatorname{lip}}$ and $\mathbf{f}_{k,s}(\bm{G}_s)=h_{k,s}(\bm{G}_{\tau_k}-\bm{G}_s)$, where $h_{k,s}=\theta_s \left(1 + \frac{2}{e^{2\bar{\theta}_{s:\tau_k}}-1}\right)$. 


To bound the expected reconstruction error $\mathbb{E}\norm{\bar{\bm{G}}_{\tau_{k-1}}-\widehat{\bm{G}}_{\tau_{k-1}}}^2$ at each generation process, we begin by analyze how $\mathbb{E}\norm{\bar{\bm{G}}_t-\widehat{\bm{G}}_t}^2$ evolves as time $t$ is reversed from $\tau_k$ to $\tau_{k-1}$. 
The reconstruction error goes as follows
\begin{equation}
\begin{aligned}
\mathcal{E}(t)
&\leq \mathbb{E}\int_{\tau_k}^t\norm{\left(\mathbf{f}_{k,s}(\bar{\bm{G}}_s)-\mathbf{f}_{k,s}(\widehat{\bm{G}}_{s})\right)+g_{k,s}^2 \left(\bm{s}_{\bm{\theta}}(\widehat{\bm{G}}_{s},s)-\nabla_{\bm{G}}\log p_{s}(\bar{\bm{G}}_{s})\right)}^2\mathrm{d}\bar{s} \\ 
% Line2
&\leq C\mathbb{E}\int_{\tau_k}^t\left\|\mathbf{f}_{k,s}(\bar{\bm{G}}_{s})-\mathbf{f}_{k,s}(\widehat{\bm{G}}_{s})\right\|^2 \mathrm{d}\bar{s} 
+ C\mathbb{E}\int_{\tau_k}^t g_{k,s}^4\left\|\bm{s}_{\bm{\theta}}(\widehat{\bm{G}}_s,s)-\nabla_{\bm{G}}\log p_s(\bar{\bm{G}}_s)\right\|^2\mathrm{d}\bar{s} \\ 
% Line3
&\leq C\int_{\tau_k}^t\|h_{k,s}\|_{\mathrm{lip}}^2\cdot \mathcal{E}(s) \mathrm{d}\bar{s} 
+ C \mathcal{E}(\tau_k) \int_{\tau_k}^t h_{k,s}^2 \mathrm{d}\bar{s}  \\
&+ C^2 \int_{\tau_k}^t g_{k,s}^4 \cdot\mathbb{E}\left\|\bm{s}_{\bm{\theta}}(\widehat{\bm{G}}_{s},s)-\bm{s}_{\bm{\theta}}(\bar{\bm{G}}_{s},s)\right\|^2   
+g_{k,s}^4\cdot\mathbb{E}\left\|\bm{s}_{\bm{\theta}}(\bar{\bm{G}}_s,s)-\nabla_{\bm{G}}\log p_s(\bar{\bm{G}}_s)\right\|^2\mathrm{d}\bar{s}  \\
% Line5
&\leq \underbrace{C^2 \ell^{(k)}(\bm{\theta}) \int_{\tau_k}^t g_{k,s}^4\mathrm{d}\bar{s} 
+ C \mathcal{E}(\tau_k) \int_{\tau_k}^t h_{k,s}^2   \mathrm{d}\bar{s}}_{\alpha(t)}  
+\int_{\tau_k}^t \underbrace{\left( C^2 g_{k,s}^4 \|\bm{s}_{\bm{\theta}}(\cdot,s)\|_{\mathrm{lip}}^2 
+C\|h_{k,s}\|_{\mathrm{lip}}^2 \right)}_{\gamma(s)}  \mathcal{E}(s)  \mathrm{d}\bar{s} \\
% Line6
& = \alpha(t) + \int_{\tau_k}^t \gamma(s) \mathcal{E}(s)  \mathrm{d}\bar{s}.
\end{aligned}
\end{equation}



% changing the integral range (通过改变积分范围,我们可以得到等价的形式为)
Let $v(t)=\mathcal{E}(\tau_k-t)$ and $s'=\tau_k-s$, it can be derived that
\begin{equation}
    v(t) = \mathcal{E}(\tau_k-t) \leq \alpha(\tau_k-t) + \int _0 ^t \gamma(\tau_k - s')v(s')\diff s'.
\end{equation}

Here, $\alpha(\tau_k - t)$ is a non-decreasing function. 
By applying Grönwall’s inequality, we can derive that
\begin{align}
    v(t) & \leq \alpha(\tau_k-t)   \exp{
    \int_0^t \gamma(\tau_k-s') } \mathrm{d}s'  \\
    & = \alpha(\tau_k-t)  \exp{
    \int_{\tau_k-t}^{\tau_k} \gamma(s) } \mathrm{d}s.
\end{align}

Hence,
\begin{equation}
    \mathcal{E}(t) \leq \alpha(t)  \exp{
    \int_t^{\tau_k} \gamma(s) } \mathrm{d}s.
\end{equation}
Therefore, the reconstruction error of HOG-Diff is bounded by 
\begin{equation}
\begin{split}
\mathcal{E}(0)
%\mathbb{E}\norm{\bar{\mathbf{x}}_0-\widehat{\mathbf{x}}_0}^2 
&\leq \alpha(0)  \exp{ \int_0^{\tau_1} \gamma(s) } \mathrm{d}s \\
&= 
 \left(C^2 \ell^{(1)}(\bm{\theta}) \int_0^{\tau_1} g_{1,s}^4\mathrm{d}s
+ C \mathcal{E}(\tau_1) \int_0^{\tau_1} h_{1,s}^2   \mathrm{d}s \right) \exp{ \int_0^{\tau_1} \gamma(s) } \mathrm{d}s.
\end{split}
\end{equation}
A comparable calculation for a classical graph generation model (with diffusion interval $[0, T]$) yields a bound
\begin{equation}
    \mathcal{E}^\prime (0) \leq
 \left(C^2 \ell(\bm{\varphi}) \int_0^T g_s^4\mathrm{d}s
+ C \mathcal{E}^{\prime}(T) \int_0^T h_s^2   \mathrm{d}s \right) \exp \int_0^T \gamma^\prime(s)  \mathrm{d}s,
\end{equation}
where $h_s=\theta_s \left(1 + \frac{2}{e^{2\bar{\theta}_{s:T}}-1}\right)$.


Let $h(s,\tau)=\theta_s  \left(1 + \frac{2}{e^{2\bar{\theta}_{s:\tau}}-1}\right)$, $a(\tau)=\int_0^{\tau} h(s,\tau)^2 \diff{s}$, and $b(\tau)=\int_0^{\tau} \|h(s,\tau)\|_{\operatorname{lip}}^2 \diff{s}$ .
Since $\tau_1\leq T$, it follows that $ \mathcal{E}(\tau_1)\leq \mathcal{E}^\prime(T)$. Additionally, by \cref{pro:training}, $\ell(\cdot)$ converges exponentially in the score-matching process.
Therefore, to prove $\mathcal{E}(0)\leq \mathcal{E}^\prime(0)$, it suffices to show that both $a(\tau)$ and $b(\tau)$ are increasing functions.

Applying the Leibniz Integral Rule, we obtain:
\begin{equation}
a^\prime (\tau) = h(\tau,\tau)^2 + \int_0^{\tau} \frac{\partial}{\partial \tau} h(s, \tau)^2 \diff{s} 
\qquad \mathrm{and} \qquad
b^\prime(\tau) = \|h(\tau,\tau)\|^2_{\operatorname{lip}} + \int_0^{\tau} \frac{\partial}{\partial \tau} \|h(s, \tau)\|^2_{\operatorname{lip}} \diff{s}.
\end{equation}
Since $h(\tau,\tau) \rightarrow 0$, we can derive that $a^\prime (\tau)>0$ and $b^\prime (\tau)>0$. 
This implies $\int_0^{\tau_1} h_{1,s}^2 \diff{s} \leq \int_0^T h_s^2 \diff{s}$ and $\int_0^{\tau_1} \gamma(s) \diff{s} \leq \int_0^T \gamma^{\prime}(s) \diff{s}$.
Combining these inequalities, we can finally conclude $\mathcal{E}(0)\leq \mathcal{E}^\prime(0)$.
Therefore, HOG-Diff provides a sharper reconstruction error bound than the classical graph generation framework.
\end{proof}


\section{Higher-order Networks}
\label{app:ho-intro}
Graphs are elegant and useful abstractions for modelling irregular relationships in empirical systems, but their inherent limitation to pairwise interactions restricts their representation of group dynamics \cite{HigherOrderReview2020,xiao2022people}. 
% 
For example, cyclic structures like benzene rings and functional groups play a holistic role in molecular networks; densely interconnected structures, like simplices, often have a collective influence on social networks; and functional brain networks exhibit higher-order dependencies.
%
To address this, various topological models have been employed to describe data in terms of its higher-order relations, including simplicial complexes \cite{HiGCN2024}, cell complexes \cite{CWN+NeurIPS2021}, and combinatorial complexes \cite{combinatorial-complexes}.
%
As such, the study of higher-order networks has gained increasing attention for their capacity to capture higher-order interactions \cite{TDL-position+ICML2024,HoRW2024}.


Given the broad applicability and theoretical richness of higher-order networks, the following delves deeper into two key frameworks for modelling such interactions: simplicial complexes and cell complexes.

\subsection{Simplicial Complexes}

Simplicial complexes (SCs) are fundamental concepts in algebraic topology that flexibly subsume pairwise graphs \cite{Top_Hodge_Hatcher+2001}. 
Specifically, simplices generalize fundamental geometric structures such as points, lines, triangles, and tetrahedra, enabling the modelling of higher-order interactions in networks. They offer a robust framework for capturing multi-way relationships that extend beyond pairwise connections typically represented in classical networks.

A simplicial complex $\mathcal{X}$ consists of a set of simplices of varying dimensions, including vertices (dimension 0), edges (dimension 1), and triangles (dimension 2).

% @ xxx
A $d$-dimensional simplex is formed by a set of $(d+1)$ interacting nodes and includes all the subsets of $\delta + 1$ nodes (with $\delta<d$), which are called the $\delta$-dimensional faces of the simplex.
A simplicial complex of dimension $d$ is formed by simplices of dimension at most equal to $d$ glued along their faces.


\begin{definition}[Simplicial complexes]
A simplicial complex $\mathcal{X}$ is a finite collection of node subsets closed under the operation of taking nonempty subsets, and such a node subset $\sigma \in \mathcal{X}$ is called a simplex. 
\end{definition}


We can obtain a clique complex, a particular kind of SCs, by extracting all cliques from a given graph and regarding them as simplices. 
%
This implies that an empty triangle (owning $\left[v_1,v_2\right]$, $\left[v_1,v_3\right]$, $\left[v_2,v_3\right]$ but without $\left[v_1,v_2,v_3\right]$) cannot occur in clique complexes.

\subsection{Cell Complexes}


Cell complexes generalize simplicial complexes by incorporating generalized building blocks called cells instead of relying solely on simplices \cite{Top_Hodge_Hatcher+2001}.
% Cells capture many-body interactions that are less restrictive than those of simplicial complexes. 
This broader approach allows for the representation of many-body interactions that do not adhere to the strict requirements of simplicial complexes.
For example, a square can be interpreted as a cell of four-body interactions whose faces are just four links. 
This flexibility is advantageous in scenarios such as social networks, where, for instance, a discussion group might not involve all-to-all pairwise interactions, or in protein interaction networks, where proteins in a complex may not bind pairwise.

\begin{figure}[!t]
\centering
\includegraphics[width=0.96\linewidth]{figs/cell-example.pdf}
\vspace{-3mm}
\caption{\textbf{Visual illustration of cell complexes.} (\textbf{a}) Triangle. (\textbf{b}) Tetrahedron. (\textbf{c}) Sphere. (\textbf{d}) Torus.}
\label{fig:cell-example}
\vspace{-4mm}
\end{figure}


% @ Higher-Order Networks
Formally, a cell complex is termed regular if each attaching map is a homeomorphism onto the closure of the associated cell’s image. 
Regular cell complexes generalize graphs, simplicial complexes, and polyhedral complexes while retaining many desirable combinatorial and intuitive properties of these simpler structures.
In this paper, all cell complexes will be regular and consist of finitely many cells. 

As shown in \cref{fig:cell-example} \textbf{a} and \textbf{b},
triangles and tetrahedrons are two particular types of cell complexes called simplicial complexes (SCs). The only 2-cells they allow are triangle-shaped.
%
The sphere shown in \cref{fig:cell-example} \textbf{c} is a 2-dimensional cell complex. It is constructed using two 0-cells (\ie, nodes), connected by two 1-cells (\ie, the edges forming the equator). The equator serves as the boundary for two 2-dimensional disks (the hemispheres), which are glued together along the equator to form the sphere.
% Torus
The torus in \cref{fig:cell-example} \textbf{d} is a 2-dimensional cell complex formed by attaching a single 1-cell to itself in two directions to form the loops of the torus. The resulting structure is then completed by attaching a 2-dimensional disk, forming the surface of the torus.
Note that this is just one way to represent the torus as a cell complex, and other decompositions might lead to different numbers of cells and faces.



\section{Additional Explanation on Related Works}
\label{app:related}


\subsection{Graph Generative Models}
Graph generation has been extensively studied, which dates back to the early works of the random network models, such as the Erdős–Rényi (ER) model \cite{ER1960} and the Barabási-Albert (BA) model \cite{BA1999}.
Recent graph generative models make great progress in graph distribution learning by exploiting the capacity of deep neural networks. 
%
GraphRNN \cite{GraphRNN2018} and GraphVAE \cite{GraphVAE-DrugDiscovery} adopt sequential strategies to generate nodes and edges.  
MolGAN \cite{GAN1-MolGAN} integrates generative adversarial networks (GANs) with reinforcement learning objectives to synthesize molecules with desired chemical properties. 
\citet{GraphAF-ICLR2020} generates molecular graphs using a flow-based approach, while GraphDF \cite{GraphDF-ICML2021} adopts an autoregressive flow-based model with discrete latent variables.
Additionally, GraphEBM \cite{GraphEBM2021} employs an energy-based model for molecular graph generation.
However, the end-to-end structure of these methods often makes them more challenging to train compared to diffusion-based generative models.

\subsection{Diffusion-based Generative Models}
A leap in graph generative models has been marked by the recent progress in diffusion-based generative models \cite{Score-SDE+ICLR2021}.
%
EDP-GNN \cite{EDPGNN-2020} generates the adjacency matrix by learning the score function of the denoising diffusion process, while GDSS \cite{GDSS+ICML2022} extends this framework by simultaneously generating node features and an adjacency matrix with a joint score function capturing the node-edge dependency.
%
DiGress \cite{DiGress+ICLR2023} addresses the discretization challenge due to Gaussian noise, while CDGS \cite{CDGS+AAAI2023} designs a conditional diffusion model based on discrete graph structures.
%
GSDM \cite{GSDM+TPAMI2023} introduces an efficient graph diffusion model driven by low-rank diffusion SDEs on the spectrum of adjacency matrices.
% 
% GPrinFlowNet 
GPrinFlowNet \cite{GPrinFlowNet+ACM2024} proposes a semantic-preserving framework based on a low-to-high frequency generation curriculum, where the $k$-th intermediate generation state corresponds to the $k$ smallest principal components of the adjacency matrices.
%
Despite these advancements, current methods are ineffective at modelling the topological properties of higher-order systems since learning to denoise the noisy samples does not explicitly lead to preserving the intricate structural dependencies required for generating realistic graphs.


\subsection{Diffusion Bridge} 
Several recent works have improved the generative framework of diffusion models by leveraging the diffusion bridge processes, \ie, processes conditioned to the endpoints.
%
\citet{wu2022diffusion} inject physical information into the process by incorporating informative prior to the drift.
% 
GLAD \cite{GLAD-ICMLworkshop2024} employs the Brownian bridge on a discrete latent space with endpoints conditioned on data samples.
%
GruM \cite{GruM+ICML2024} utilizes the OU bridge to condition the diffusion endpoint as the weighted mean of all possible final graphs.
%
However, existing methods often overlook or inadvertently disrupt the higher-order topological structures in the graph generation process.




\section{Details for Higher-order Guided Generation }
\label{app:detail-HOG-Diff}

\subsection{Denoising Network Parametrization}
\label{app:denoising-model}

The denoising network in HOG-Diff is a critical component responsible for estimating the score functions required to reverse the diffusion process effectively. 
The architecture of the proposed denoising network is depicted in \cref{fig:denoising-model}.
The input $\bm{A}_t$ is computed from $\bm{U}_0$ and $\bm{\Lambda}_t^{(k)}$ using the relation  $\bm{A}_t=\bm{D}_t^{(k)}-\bm{L}^{(k)}_t$, where the Laplacian matrix is given by $\bm{L}^{(k)}_t=\bm{U}_0 \bm{\Lambda}_t^{(k)}\bm{U}_0^\top$ and the diagonal degree matrix is given by $\bm{D}_t^{(k)}=\operatorname{diag}\left(\bm{L}_t^{(k)}\right)$.
%
To enhance the input to the Attention module, we derive enriched node and edge features using the  $l$-step random walk matrix obtained from the binarized $\bm{A}_t$.
Specifically, the arrival probability vector is incorporated as additional node features, while the truncated shortest path distance derived from the same matrix is employed as edge features.
Temporal information is integrated into the outputs of the Attention and GCN modules using Feature-wise Linear Modulation (FiLM) \cite{Film+AAAI2018} layers, following sinusoidal position embeddings \cite{attention+NeurIPS2017}.



\begin{figure}[t]
\centering
\includegraphics[width=0.96\linewidth]{figs/ScoreNet.pdf}
\caption{\textbf{Denoising Network Architecture of HOG-Diff.} 
The denoising network integrates GCN and Attention blocks to capture both local and global features, and further incorporates time information through FiLM layers.
These enriched outputs are subsequently concatenated and processed by separate feed-forward networks to produce predictions for $\nabla_{\bm{X}} \log p(\bm{G}_t|\bm{G}_{\tau_k})$ and $\nabla_{\bm{\Lambda}_t} \log p(\bm{G}_t|\bm{G}_{\tau_k})$, respectively.
% \tolga{Please write a description in the caption here.}
}
\label{fig:denoising-model}
\vspace{-4mm}
\end{figure}

% permutation equivariant
A graph processing module is considered permutation invariant if its output remains unchanged under any permutation of its input, formally expressed as $f(\bm{G}) = x \iff f(\pi(\bm{G})) = x$, where $\pi(\bm{G})$ represents a permutation of the input graph $\bm{G}$. It is permutation equivariant when the output undergoes the same permutation as the input, formally defined as $f(\pi(\bm{G})) = \pi(f(\bm{G}))$. 
It is worth noting that our denoising network model is permutation equivalent as each model component avoids any node ordering-dependent operations.



\subsection{Training and Sampling Proceudre}

As shown in \cref{fig:gFrame}, HOG-Diff implements a coarse-to-fine generation curriculum, with the forward diffusion and reverse denoising processes divided into $K$ easy-to-learn subprocesses. Each subprocess is realized using the generalized OU bridge process.


\begin{figure}[!ht]
    \centering
    \includegraphics[width=0.9\linewidth]{figs/gFrame.pdf}
    \caption{\textbf{Illustration of the coarse-to-fine generation process in HOG-Diff using the generalized OU bridge.}}
    \label{fig:gFrame}
\end{figure}


We provide the pseudo-code of the training process in \cref{alg:train}. In our experiments, we adopt a two-step generation process, \ie, $K=2$.  
We initialize $\mathcal{S}^{(0)} = \bm{G}$; under this specific condition, the cell complex filtering operation returns the input unchanged. The set $\mathcal{S}^{(1)}$ corresponds to the 2-cell complex for molecule generation tasks or the 3-simplicial complex for generic graph generation tasks. 
We set $\mathcal{S}^{(2)} = \varnothing$, and for this particular case, we define the cell complex filtering function as
$\operatorname{CCF}(\bm{G}, \varnothing) = \mathcal{N}(\bm{0}, \bm{I})$.

\begin{figure}[t!]
%\vspace{-0.2in}
\centering
\begin{minipage}{\linewidth}
\centering
\begin{algorithm}[H]
\small
\caption{ Training Algorithm of HOG-Diff }
    \textbf{Input:} denoising network $\bm{s}_{\bm{\theta}}^{(k)}$,
                    authentic graph data $\bm{G}_0=(\bm{X}_0, \bm{A}_0)$, 
                    cell complex list $\{\mathcal{S}^{(0)},\cdots,\mathcal{S}^{(K)}\}$, 
                    \phantom{-} training epochs $M_k$.\\
    \textbf{For the $k$-th step:} \phantom{-}             
\begin{algorithmic}[1]
\FOR{$m=1$ \textbf{to} $M_k$}
    \STATE Sample $\bm{G}_0=(\bm{X}_0,\bm{A}_0) \sim \mathcal{G}$
    \STATE $\bm{G}_{\tau_k} \leftarrow \operatorname{CCF}(\bm{G}_0, \mathcal{S}^{(k)})$, and $\bm{G}_{\tau_{k-1}} \leftarrow \operatorname{CCF}(\bm{G}_0, \mathcal{S}^{(k-1)})$ \COMMENT{Cell complex filtering}
    \STATE $\bm{U}_0 \leftarrow \operatorname{EigenVectors}(\bm{D}_0 - \bm{A}_0)$
    \STATE $\bm{\Lambda}_{\tau_k} \leftarrow \operatorname{EigenDecompostion}(\bm{D}_{\tau_k} - \bm{A}_{\tau_k})$
    \STATE $\bm{\Lambda}_{\tau_{k-1}} \leftarrow \operatorname{EigenDecompostion}(\bm{D}_{\tau_{k-1}} - \bm{A}_{\tau_{k-1}})$
    \STATE Sample $t \sim \operatorname{Unif}([0,\tau_k - \tau_{k-1}])$
    \STATE $\bm{X}_t^{(k)} \sim p(\bm{X}_t \mid \bm{X}_{\tau_{k-1}},\bm{X}_{\tau_{k}})$ \COMMENT{\cref{eq:GOU-p}}
    \STATE $\bm{\Lambda}_t^{(k)} \sim p(\bm{\Lambda}_t \mid \bm{\Lambda}_{\tau_{k-1}},\bm{\Lambda}_{\tau_{k}})$ \COMMENT{\cref{eq:GOU-p}}
    \STATE $\bm{L}_t^{(k)} \leftarrow \bm{U}_0 \bm{\Lambda}_t^{(k)} \bm{U}_0^\top$
    \STATE $\bm{A}_t^{(k)} \leftarrow \bm{D}_t^{(k)} - \bm{L}_t^{(k)}  $
    \STATE $\ell^{(k)}(\bm{\theta}) \leftarrow c_1\|
\bm{s}^{(k)}_{\bm{\theta},\bm{X}} - \nabla_{\bm{X}} \log p_t(\bm{G}_t | \bm{G}_{\tau_k})\|^2 \nonumber + c_2 ||\bm{s}^{(k)}_{\bm{\theta},\bm{\Lambda}} - \nabla_{\bm{\Lambda}} \log p_t(\bm{G}_t |\bm{G}_{\tau_k})||^2$
    \STATE $\bm{\theta} \leftarrow \operatorname{optimizer}(\ell^{(k)}(\bm{\theta}))$ 
\ENDFOR
\STATE \textbf{Return:} $\bm{s}^{(k)}_{\bm{\theta}}$
\end{algorithmic}
\label{alg:train}
\end{algorithm}
\end{minipage}
\vspace{-0.2in}
\end{figure}

The pseudo-code of sampling with HOG-Diff is described in \cref{alg:sample}.
The reverse diffusion processes are divided into $K$ hierarchical time windows, denoted as  $\{[\tau_{k-1},\tau_k]\}_{k=1}^K$, where $0 = \tau_0 < \cdots < \tau_{k-1}< \tau_k < \cdots < \tau_K = T$.
We first initialize the sampling process by drawing samples for $\widehat{\bm{X}}_{\tau_K}$ and $\widehat{\bm{\Lambda}}_{\tau_K}$ from a standard Gaussian distribution, and $\widehat{\bm{U}}_0$ is sampled uniformly from the eigenvector matrices of the Laplacian matrix in the training dataset. 
The reverse-time process starts at $\tau_K$ and iteratively updates $\widehat{\bm{X}}_t$ and $\widehat{\bm{\Lambda}}_t$ by solving the reverse-time SDEs with the denoising network $\bm{s_\theta}^{(k)}$.
Subsequently, we reconstruct the Laplacian matrix $\widehat{\bm{L}}_t$ using the fixed eigenvector matrix $\widehat{\bm{U}}_0$ and the updated eigenvalues $\widehat{\bm{\Lambda}}_t$.
%
Endpoint of one generation step serves as the starting point for the next process.
%
Finally, after iterating through all diffusion segments, the algorithm returns the final feature matrix $\widehat{\bm{X}}_0$ and adjacency matrix $\widehat{\bm{A}}_0$, thereby completing the graph generation process. 



\begin{figure}[t!]
%\vspace{-0.2in}
\centering
\begin{minipage}{\linewidth}
\centering
\begin{algorithm}[H]
\small
\caption{ Sampling Algorithm of HOG-Diff }
    \textbf{Input:} Trained denoising network $\bm{s}_{\theta}^{(k)}$, 
    diffusion time split $\{\tau_0,\cdots,\tau_K\}$,
    number of sampling steps $M_k$
\begin{algorithmic}[1]
\STATE $t \leftarrow \tau_K$
\STATE $\widehat{\bm{X}}_{\tau_K}\sim \mathcal{N}(\bm{0}, \bm{I})$ and $\widehat{\bm{\Lambda}}_{\tau_K}\sim \mathcal{N}(\bm{0}, \bm{I})$
\STATE $\widehat{\bm{U}}_0 \sim \operatorname{Unif}\left(\{\bm{U}_0 \triangleq \operatorname{EigenVectors}(\bm{L}_0)\}\right)$
\STATE $\widehat{\bm{G}}_{\tau_K} \leftarrow (\widehat{\bm{X}}_{\tau_K},\widehat{\bm{\Lambda}}_{\tau_K},\widehat{\bm{D}}_{\tau_K}-\widehat{\bm{U}}_0 \widehat{\bm{\Lambda}}_{\tau_K} \widehat{\bm{U}}_0 ^\top)$
\FOR{$k=K$ \textbf{to} $1$}
\FOR{$m=M_k-1$ \textbf{to} $0$}
\STATE $\bm{S}_{\bm{X}}, \bm{S}_{\bm{\Lambda}} \leftarrow \bm{s}^{(k)}_{\bm{\theta}}(\widehat{\bm{G}}_t, \widehat{\bm{G}}_{\tau_k},t)$
% Predict X
\STATE $\widehat{\bm{X}}_t \leftarrow \widehat{\bm{X}}_t - \left[
\theta_t \left( 1 + \frac{2}{e^{2\bar{\theta}_{t:\tau_k}}-1}  \right)(\widehat{\bm{X}}_{\tau_k} - \widehat{\bm{X}}_t)
-g_{k,t}^2 \bm{S_X} \right]\delta t 
+g_{k,t} \sqrt{\delta t} \bm{w}_{\bm{X}}$, $\bm{w}_{\bm{X}} \sim \mathcal{N}(\bm{0}, \bm{I})$ \COMMENT{Prediction step: $\bm{X}$}
% Predict eig
\STATE $\widehat{\bm{\Lambda}}_t \leftarrow \widehat{\bm{\Lambda}}_t - \left[
\theta_t \left( 1 + \frac{2}{e^{2\bar{\theta}_{t:\tau_k}}-1}  \right)(\widehat{\bm{\Lambda}}_{\tau_k} - \widehat{\bm{\Lambda}}_t)
-g_{k,t}^2 \bm{S_\Lambda} \right] \delta t
+g_{k,t} \sqrt{\delta t}  \bm{w}_{\bm{\Lambda}}$, $\bm{w}_{\bm{\Lambda}} \sim \mathcal{N}(\bm{0}, \bm{I})$ \COMMENT{Prediction step: $\bm{\Lambda}$}
\STATE $\widehat{\bm{L}}_t \leftarrow \widehat{\bm{U}}_0 \widehat{\bm{\Lambda}}_t \widehat{\bm{U}}_0^\top$
\STATE $\widehat{\bm{A}}_t \leftarrow \widehat{\bm{D}}_t - \widehat{\bm{L}}_t$
\STATE $t \leftarrow t - \delta t$
\ENDFOR
\STATE $\widehat{\bm{A}}_{\tau_{k-1}} = \operatorname{quantize}(\widehat{\bm{A}}_t)$\COMMENT{Quantize if necessary}
\STATE $\widehat{\bm{G}}_{\tau_{k-1}} \leftarrow (\widehat{\bm{X}}_t,\widehat{\bm{\Lambda}}_t,\widehat{\bm{A}}_t)$ 
\ENDFOR
\STATE \textbf{Return:} $\widehat{\bm{X}}_0$, $\widehat{\bm{A}}_0$ \COMMENT{$\tau_0 = 0$}
\end{algorithmic}
\label{alg:sample}
\end{algorithm}
\end{minipage}
%\vspace{-0.3in}
\end{figure}




\section{Complexity Analysis}
\label{app:complexity}



When the targeted graph is not in the desired higher-order forms, one should also consider the one-time preprocessing procedure for graph filtering.

Cell filtering can be dramatically accelerated because it avoids explicitly finding all cells and only determines whether nodes and edges belong to a cell. Specifically, the $2$-cell filter requires only checking whether each edge belongs to some cycle.

One method to achieve the $2$-cell filter is using a depth-first search (DFS). Starting from the adjacency matrix, we temporarily remove the edge $(i, j)$ and initiate a DFS from node $i$, keeping track of the path length. If the target node $j$ is visited within a path length of $l$, the edge $(i, j)$ is marked as belonging to a $2$-cell of length at most $l$. In sparse graphs with $n$ nodes and $m$ edges, the time complexity of a single DFS is $\mathcal{O}(m + n)$. With the path length limited to $l$, the DFS may traverse up to $l$ layers of recursion in the worst case. Therefore, the complexity of a single DFS is $\mathcal{O}(\min(m + n, l \cdot k_{max})) $, where $k_{max}$ is the maximum degree of the graph. For all $m$ edges, the total complexity is
$\mathcal{O}\left(m \cdot \min(m + n, l \cdot k_{max})\right)$.




Alternatively, matrix operations can be utilized to accelerate this process. By removing the edge $(i, j)$ from the adjacency matrix $A$ to obtain $\bar{A}$, the presence of a path of length $l$ between $i$ and $j$ can be determined by checking whether $\bar{A}^l_{i,j} > 0$. This indicates that the edge $(i, j)$ belongs to a $2$-cell with a maximum length of $l+1$. Assuming the graph has $n$ nodes and $m$ edges, the complexity of sparse matrix multiplication is $\mathcal{O}(mn)$. Since $l$ matrix multiplications are required, the total complexity is: $\mathcal{O}(l \cdot m^2 \cdot n)$. While this complexity is theoretically higher than the DFS approach, matrix methods can benefit from significant parallel acceleration on modern hardware, such as GPUs and TPUs. In practice, this makes the matrix-based method competitive, especially for large-scale graphs or cases where $l$ is large.



For simplicial complexes, the number of $p$-simplices in a graph with $n$ nodes and $m$ edges is upper-bounded by $\mathcal{O}(n^{p-1})$, and they can be enumerated in $\mathcal{O}( a\left(\mathcal{G}\right)^{p-3} m)$ time \cite{chiba1985arboricity}, where $a\left(\mathcal{G}\right)$ is the arboricity of the graph $\mathcal{G}$, a measure of graph sparsity.
Since arboricity is demonstrated to be at most $\mathcal{O}(m^{1/2})$ and $m \leq n^2$, all $p$-simplices can thus be listed in $\mathcal{O}\left( n^{p-3} m \right)$.
Besides, the complexity of finding $2$-simplex is estimated to be $\mathcal{O}(\left\langle k \right\rangle m)$ with the Bron–Kerbosch algorithm \cite{find_cliques1973}, where $\left \langle k \right \rangle$ denotes the average node degree, typically a small value for empirical networks.


\section{Experimental Setup}
\label{app:exp_set}





\subsection{Computing Resources}
In this work, all experiments are conducted using PyTorch on a single NVIDIA L40S GPU with 46 GB memory and AMD EPYC 9374F 32-Core Processor.


\subsection{Generic Graph Generation}


We follow the experimental and evaluation setting from \citet{GDSS+ICML2022} with the same train/test split to ensure a fair comparison with baselines.
%
We use node degree and spectral features of the graph Laplacian decomposition as hand-crafted input features.

\cref{tab:data_summary} summarizes the key characteristics of the datasets utilized in this study. The table outlines the type of dataset, the total number of graphs, and the range of graph sizes ($|V|$). Additionally, it also provides the number of distinct node types and edge types for each dataset. Notably, the synthetic datasets (Community-small and Ego-small) contain relatively small graphs, whereas the molecular datasets (QM9 and ZINC250k) exhibit more diversity in graph size and complexity. %, as reflected by higher average numbers of nodes and edges.


\begin{table}[h]
\centering
\scalebox{0.95}{
\begin{tabular}{c|cc}
\toprule
                             & \#Test      & \#Temporal relations   \\ \midrule
TempEvalQA-Bi                & 448      &  2           \\ 
 TRACIE                      & 4248     &  2          \\
 MCTACO                      & 9442     & 1-19 \\ \bottomrule
\end{tabular}
}
\caption{Dataset Statistics.
For TempEvalQA-Bi, the numbers represent the total number of questions. For TRACIE, the numbers refer to the number of story-hypothesis pairs. For MCTACO, the numbers reflect question-and-answer candidate pairs.}
\label{tab:data_summary}
\end{table}

\subsection{Molecule Generation}
\label{app:mol}


% @ CDGS
Early efforts in molecule generation introduce sequence-based generative models and represent molecules as SMILES strings \cite{SMILES-ICML2017}. 
%
Nevertheless, this representation frequently encounters challenges related to long dependency modelling and low validity issues, as the SMILES string fails to ensure absolute validity. 
Therefore, in recent studies, graph representations are more commonly employed for molecule structures where atoms are represented as nodes and chemical bonds as connecting edges \cite{GDSS+ICML2022}.
Consequently, this shift has driven the development of graph-based methodologies for molecule generation, which aim to produce valid, meaningful, and diverse molecules.


% @ GDSS / CDGS
In experiments, each molecule is preprocessed into a graph comprising adjacency matrix $\bm{A}\in \{0,1,2,3\}^{n\times n}$ and node feature matrix $\bm{X}\in \{0,1\}^{n\times d}$, where $n$ denotes the maximum number of atoms in a molecule of the dataset, and $d$ is the number of possible atom types. The entries of $\bm{A}$ indicate the bond types: 0 for no bound, 1 for the single bond, 2 for the double bond, and 3 for the triple bond. 
Further, we scale $\bm{A}$ with a constant scale of 3 in order to bound the input of the model in the interval [0, 1], and rescale the final sample of the generation process to recover the bond types.
%
Following the standard procedure \cite{GraphAF-ICLR2020, GraphDF-ICML2021}, all molecules are kekulized by the RDKit library \cite{Rdkit2016} with hydrogen atoms removed. In addition, we make use of the valency correction proposed by \citet{Moflow-SIGKDD2020}. 
After generating samples by simulating the reverse diffusion process,  the adjacency matrix entries are quantized to discrete values ${0, 1, 2, 3}$ by by applying value clipping. Specifically, values in $(-\infty, 0.5)$ are mapped to 0, $[0.5, 1.5)$ to 1, $[1.5, 2.5)$ to 2, and $[2.5, +\infty)$ to 3, ensuring the bond types align with their respective categories.




% 分子图指标介绍 @GPrinFlowNet, has modified
To comprehensively assess the quality of the generated molecules across datasets, we evaluate 10,000 generated samples using several key metrics: validity, validity w/o check, Frechet ChemNet Distance (FCD) \cite{FCD}, Neighborhood Subgraph Pairwise Distance Kernel (NSPDK) MMD \cite{NSPKD-MMD}, uniqueness, and novelty \cite{GDSS+ICML2022}.
% 1. FCD
\textbf{FCD} quantifies the similarity between generated and test molecules by leveraging the activations of ChemNet's penultimate layer, accessing the generation quality within the chemical space.
% 2. NSPDK-MMD
In contrast, \textbf{NSPDK-MMD} evaluates the generation quality from the graph topology perspective by computing the MMD between the generated and test sets while considering both node and edge features.
% 3. validity
\textbf{Validity} is measured as the fraction of valid molecules to all generated molecules after applying post-processing corrections such as valency adjustments or edge resampling, while \textbf{validity w/o correction}, following \citet{GDSS+ICML2022}, computes the fraction of valid molecules before any corrections, providing insight into the intrinsic quality of the generative process. 
Whether molecules are valid is generally determined by compliance with the valence rules in RDkit \cite{Rdkit2016}.
% 4. novelty
\textbf{Novelty} assesses the model’s ability to generalize by calculating the percentage of generated graphs that are not subgraphs of the training set, with two graphs considered identical if isomorphic.
% 5. uniqueness
\textbf{Uniqueness} quantifies the diversity of generated molecules as the ratio of unique samples to valid samples, removing duplicates that are subgraph-isomorphic, ensuring variety in the output.





\section{Visualization Results}
\label{app:vis}

In this section, we additionally provide the visualizations of the generated graphs for both molecule generation tasks and generic graph generation tasks.
Figs.~\ref{fig:qm9}-\ref{fig:enzymes} illustrate non-curated generated samples. HOG-Diff demonstrates the capability to generate high-quality samples that closely resemble the topological properties of empirical data while preserving essential structural details.

\begin{figure}[!ht]
    \centering
    \includegraphics[width=0.96\linewidth]{figs/vis_qm9.pdf}
    \caption{Visualization of random samples taken from the HOG-Diff trained on the QM9 dataset. }
    \label{fig:qm9}
\end{figure}

\begin{figure}
    \centering
    \includegraphics[width=0.96\linewidth]{figs/vis_zinc250k.pdf}
    \caption{Visualization of random samples taken from the HOG-Diff trained on the Zinc250k dataset. }
    \label{fig:zinc250k}
\end{figure}

\begin{figure}
    \centering
    \includegraphics[width=0.96\linewidth]{figs/vis_cs.pdf}
    \caption{Visual comparison between training set graph samples and generated graph samples produced by HOG-Diff on the Community-small dataset.}
\end{figure}

\begin{figure}
    \centering
    \includegraphics[width=0.96\linewidth]{figs/vis_ego.pdf}
    \caption{Visual comparison between training set graph samples and generated graph samples produced by HOG-Diff on the Ego-small dataset.}
\end{figure}

\begin{figure}
    \centering
    \includegraphics[width=0.96\linewidth]{figs/vis_enzymes.pdf}
    \caption{Visual comparison between training set graph samples and generated graph samples produced by HOG-Diff on the Enzymes dataset.}
    \label{fig:enzymes}
\end{figure}





\end{document}


% This document was modified from the file originally made available by
% Pat Langley and Andrea Danyluk for ICML-2K. This version was created
% by Iain Murray in 2018, and modified by Alexandre Bouchard in
% 2019 and 2021 and by Csaba Szepesvari, Gang Niu and Sivan Sabato in 2022.
% Modified again in 2023 and 2024 by Sivan Sabato and Jonathan Scarlett.
% Previous contributors include Dan Roy, Lise Getoor and Tobias
% Scheffer, which was slightly modified from the 2010 version by
% Thorsten Joachims & Johannes Fuernkranz, slightly modified from the
% 2009 version by Kiri Wagstaff and Sam Roweis's 2008 version, which is
% slightly modified from Prasad Tadepalli's 2007 version which is a
% lightly changed version of the previous year's version by Andrew
% Moore, which was in turn edited from those of Kristian Kersting and
% Codrina Lauth. Alex Smola contributed to the algorithmic style files.
