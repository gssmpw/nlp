\subsection{Path ordering}
\label{sec:path_ordering}

We find that ordering the K-shortest pre-calculated paths by number of hops gives lower blocking probability than ordering by distance in km. We refer to these two path orderings as \#hops and \#km, respectively. In this section, we address two potential criticisms of using \#hops ordering:
1) Ordering by \#hops results in unacceptable increases in path length in km.
2) Benchmarking RL solutions that use \#km ordering against heuristics with \#hops ordering is unfair, as the RL method did not have access to the same paths.

\subsubsection{Experiment setup}

We run two separate experiments to address the criticisms. We repeat these experiments for each of our test topologies.

First, we calculate the K-shortest paths for K=5 and K=50 for \#hops and \#km. We calculate the mean and standard deviation of the path lengths in hops and km for each of these cases.

Second, we run incremental traffic until the first blocking event and allocate traffic with KSP-FF for K=5 and K=50 for \#hops and \#km. We choose incremental traffic with first blocking as it allows a fairer comparison across topologies than dynamic traffic with arbitrary traffic loads. We record the paths selected in each case. We aim to investigate how the available paths differ between the two orderings, expressed as the percentage of paths that are unique to the ordering. 

We compare the differences in paths between \#hops and \#km for each case and plot: 1) \textbf{Available} shows what percentage of all pre-calculated paths between node pairs are unique when comparing the two orderings. 2) \textbf{Utilized} shows what percentage of paths that are selected by the heuristics to carry active connections are unique between the two orderings. We calculate the mean and upper and lower inter-quartile range values from 10 trials. We use the IQR to avoid unphysical negative values.

\subsubsection{Results and discussion}

Figure \ref{fig:length_hops_dotplot} compares the path lengths in both hops (y-axis) and km (x-axis) from Experiment 1. The figure shows that the difference in mean path length in km between \#hops and \#km ordering is notable but not prohibitive, with mean increases of approximately 20\% or less in length. The total throughput is still higher (SBP is lower), therefore this path length increase is not consequential. A problem specification that takes into account all physical layer effects and losses could make \#km ordering more favourable, e.g. by reducing mean latency, but for an optimization objective of network throughput, \#hops ordering is superior.

\begin{figure}[h]
    \centering
    \includegraphics[width=\linewidth]{IMAGES/length_hops_dotplot_50alt.png}
    \caption{The mean and standard deviation for path length in km and hops for each topology, KSP-FF ordered by length in km or hops, for K=5 and K=50. Increasing K from 5 to 50 has little effect on mean path length in km. Ordering by hops instead of distance increases the mean path length by a greater amount, dependent on the topology.}
    \label{fig:length_hops_dotplot}
\end{figure}

Figure \ref{fig:unique_paths_barchart} shows that \#hops and \#km orderings have 20-35\% dissimilar paths for K=5 and K=50 for NSFNET, COST239, and USNET. JPN48 is an outlier with over 50\% differences in the available paths for \#hops and \#km for both K=5 and K=50. However, examination of the utilized paths shows that allowing sufficient path diversity (K=50) means there is less than 1\% difference in utilized paths between \#hops and \#km for NSFNET, COST239, and USNET. This shows that, when training RL policies for DRA problems, it is important to have sufficient path options for the RL to equal the performance of a K-shortest path heuristic. However, for JPN48, significant differences (>20\%) between paths for \#hops and \#km are still evident even when only considering utilized paths, therefore it is is ultimately important to evaluate path orderings and select the optimal one to give RL the greatest chance at success.


\begin{figure}[h]
    \centering
    \includegraphics[width=\linewidth]{IMAGES/unique_paths_barchart_50alt.png}
    \caption{The percentage of the K pre-calculated paths that are unique to the path ordering (path length in km or hops) for each topology for K=5 and K=20. "Available" shows the percentage of paths available to choose from. "Utilised" shows the mean percentage of paths that are unique to the path ordering and utilized by the KSP-FF or KSP-FF$_{hops}$ heuristics in an incremental loading scenario. Error bars show interquartile range.}
    \label{fig:unique_paths_barchart}
\end{figure}
