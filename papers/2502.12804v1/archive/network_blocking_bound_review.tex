\subsection{Network capacity bound estimation}
\label{sec:survey-capacity_bounds}

We conclude our review with a discussion of the network capacity bound estimation methods that have been used in the literature, in anticipation of introducing our technique in section \ref{sec:bounds}.

Recent work has applied the max-flow min-cut theorem from graph theory to estimate upper bounds on optical network capacity. This theorem states that the maximum flow possible between two nodes in a network equals the minimum cut capacity separating those nodes \cite{ford_maximal_1956}. A graph cut is a partition of the graph's nodes into two disjoint sets, the cut-set is the set of links crossing the cut, and the cut capacity is the sum of the capacities of the cut-set.

Hayashi et al. first adapted the concept of cut-sets analysis, originally developed for packet-switched networks, and applied it to optical networks in their work on bit-rate variable virtual direct links \cite{hayashi_efficient_2022,hayashi_cost-effective_2023}. The method provides an upper bound on network capacity by relaxing the spectrum continuity constraint - it only considers spectrum allocation on the links in the minimum cut-set rather than ensuring continuous spectrum along full paths. This relaxation means the bound is not necessarily achievable in practice but provides a useful estimated upper bound limit. 

Cruzado et al. then extended this approach, first applying it to fixed-grid WDM networks to estimate capacity bounds for RWA \cite{cruzado_effective_2023}.  Most recently, they adapted the method for flex-grid elastic optical networks with distance-adaptive modulation formats \cite{cruzado_capacity-bound_2024}, providing theoretical bounds that account for path-dependent spectral efficiency.

The cut-sets method provides a useful upper bound estimate but is imperfect due to its reliance on a heuristic method (e.g. first-fit) to allocate the spectrum on the cut-set links. Sub-optimal allocation can cause spectrum fragmentation, thereby reducing the estimate and its reliability as a true upper bound. We propose a novel defragmentation-based method to estimate network capacity bounds in section \ref{sec:bounds}, which relaxes the constraint on reconfiguration instead of spectrum continuity.
