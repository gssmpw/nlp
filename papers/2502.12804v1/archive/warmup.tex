
\subsection{Network traffic warm-up}
\label{sec:warmup}

To exactly recreate the problem settings from the selected papers, it is necessary to generate the same number of traffic requests as the original work. This includes the connection requests that are generated to pre-populate the network until the blocking probability reaches steady-state, a phenomenon which is well-known in discrete-event simulation \cite{banks_discrete-event_2005}. We call the period prior to steady-state the warm-up period. 

Network warm-up is important so that the blocking probability reaches its steady-state value and is not an underestimate. This is known as the simulation start-up problem or the initial transient \cite{white_problem_2009}. During training, warm-up helps prevent primacy bias \cite{nikishin_primacy_2022}, where models overfit to data seen early in training. When starting with an empty network, initial traffic requests are trivially allocated due to abundant resources, providing no useful optimization signal. Training on a congested network with both successful and failed allocations gives more meaningful training data from the start.

While most papers do not explicitly discuss warm-up, all of the selected papers use it. We therefore analyze the minimum required warm-up period for different traffic levels.

\subsubsection{Experiment setup}

We simulate a non-blocking network to determine the maximum\footnotemark number of requests needed to reach a steady-state traffic load. For each traffic load from 50 to 1000 Erlangs in steps of 50, we conduct 500 simulation trials with unique random seeds at varying arrival rates from 5 to 25.  For each trial, we estimate the number of requests at which steady state is reached using the MSER-5 method \cite{franklin_stationarity_2008} (Marginal Standard Error Rule with batch size 5). We apply the MSER-5 method as it has been shown to be superior to other methods such as the mean crossing rule \cite{white_problem_2009}.


\footnotetext{A blocking network reaches steady-state in less steps than a non-blocking, therefore we consider a non-blocking network to estimate an upper bound on the required warm-up period before steady state.}


\subsubsection{Results and discussion}

\begin{figure}
    \centering
    \includegraphics[width=1\linewidth]{IMAGES/steady_state_boxplots.png}
    \caption{No. of simulated requests necessary to reach target network traffic level for a non-blocking network.}
    \label{fig:traffic_warmup}
\end{figure}

Figure \ref{fig:traffic_warmup} displays results as box-and-whisker plots showing how many requests are needed to reach steady state for each traffic load. The whiskers extend to the most extreme data points within 1.5 times the inter-quartile range. We observe a linear relationship between traffic load and requests until steady state (warm-up period). We fit a line to the maximum values (tops of the whiskers) to determine the warm-up period with high confidence. The fitted line indicates that approximately 7 times the target traffic load in requests are required to reach steady state. Based on this finding, the 3000-request warm-up period used by DeepRMSA that we adopt in section \ref{sec:repro} is sufficient for our problem settings.