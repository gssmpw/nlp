\section{Literature Survey}
\label{sec:survey}

There exists a considerable body of literature on RL for dynamic resource allocation (DRA) in optical networks \cite{amin_survey_2021}. DRA in optical networks is distinguished from similar problems in electronically linked networks by the nature of fibre optic links. The capacity of each fibre link is a vector of available wavelengths or spectrum slots, as defined by the ITU standards G.671 and G.694.2 \cite{international_telecommunication_union_spectral_2002,international_telecommunication_union_transmission_2012}, not a scalar quantity. Due to this fundamental difference, we only consider publications related to optical networks in this review, but acknowledge the closely related body of literature on RL for other graph-based resource allocation problems.

\begin{figure}
    \centering
    \includegraphics[width=1\linewidth]{IMAGES/RL_RSA_litreview_barchart.png}
    \caption{Count of publications related to RL for DRA problems in optical networks. Citations for each classification category are: RWA \cite{garcia_multicast_2003,pointurier_reinforcement_2007,koyanagi_reinforcement_2009,suarez-varela_routing_2019,shiraki_dynamic_2019,shiraki_reinforcement-learning-based_2019,zhao_cost-efficient_2021,freire-hermelo_dynamic_2021,liu_waveband_2021,nevin_techniques_2022,di_cicco_deep_2022,di_cicco_deepls_2023,nallaperuma_interpreting_2023}, RSA \cite{reyes_adaptive_2017,li_deepcoop_2020,li_multi-objective_2020,wang_dynamic_2021,zhao_reinforced_2021,shimoda_deep_2021,shimoda_mask_2021,romero_reyes_towards_2021,zhao_rsa_2022,wu_service_2022,arce_reinforcement_2022,quang_magc-rsa_2022,cruzado_reinforcement-learning-based_2022,jiao_reliability-oriented_2022,hernandez-chulde_experimental_2023,sharma_deep_2023,lin_deep-reinforcement-learning-based_2023,tanaka_pre-_2023,cheng_ptrnet-rsa_2024}, RMSA \cite{chen_deeprmsa_2019,luo_leveraging_2019,chen_multi-task-learning-based_2021,shi_deep-reinforced_2021,sheikh_multi-band_2021,xu_spectrum_2021,gonzalez_improving_2022,tang_deep_2022,momo_ziazet_deep_2022,terki_routing_2022,tu_entropy-based_2022,cheng_routing_2022,xu_deep_2022,pinto-rios_resource_2022,tang_heuristic_2022,luo_survivable_2022,sadeghi_performance_2023,errea_deep_2023,terki_routing_2023,tang_routing_2023,xu_hierarchical_2023,li_opticgai_2024,teng_drl-assisted_2024,teng_deep-reinforcement-learning-based_2024}, Other \cite{boyan_packet_1993,ma_demonstration_2019,zhao_reinforcement-learning-based_2019,wang_subcarrier-slot_2019,natalino_optical_2020,ma_co-allocation_2020,weixer_reinforcement_2020,wang_deepcms_2020,tian_reconfiguring_2021,liu_multi-agent_2021,zhao_service_2021,morales_multi-band_2021,hernandez-chulde_evaluation_2022,koch_reinforcement_2022,tanaka_reinforcement-learning-based_2022,etezadi_deepdefrag_2022,beghelli_approaches_2023,renjith_deep_2023,davalos_triggering_2023,etezadi_deep_2023,zhang_admire_2023,tanaka_adaptive_2023,johari_drl-assisted_2023,pavon-marino_tree-determination_2023,fan_blocking-driven_2023,yin_dnn_2024,li_tabdeep_2024,tanaka_reinforcement-learning-based_2024,lian_dynamic_2024,doherty_xlron_2024,wang_availability-aware_2024,tse_reinforcement_2024,jara_dream-gym_2024}.}
    \label{fig:lit_barchart}
\end{figure}



\subsection{Survey methodology}
\label{sec:survey-methodology}

To find relevant papers, we searched the Google Scholar database using the following search terms: "'REINFORCEMENT LEARNING' AND 'NETWORK' AND ('OPTICAL' OR 'WAVELENGTH' OR 'SPECTRUM')". Papers were limited to English-language peer-reviewed articles. We filtered the search results by inspection of paper titles and abstracts, and manually added any related works from our own citation database that were missing from the initial search. A review of the collected papers then created the final set of 96.

We reviewed each paper in the set to classify them into four broad categories: 'RWA', 'RSA', 'RMSA' for those that apply RL to the respective problem variants, and 'Other'.
The 'Other' category includes papers that are concerned with RL applied to: traffic grooming, defragmentation, survivability or service restoration, multicast provisioning, simulation training environments, and other problem variants such as transceiver parameter optimisation.

We notice the following characteristics and trends in the gathered dataset of papers: 
1) Increase in papers from 2019 onwards due to DeepRMSA seminal paper. Arrival of paper in 2019 shows time lag between cutting edge ML papers (Atari in 2013 and AlphaGo 2016) and adoption of techniques by optical networks research community.
2) RWA occupied large proportion of total papers in 2019 and before but became a smaller proportion and no publications in 2024.
3) 2020 impact of COVID pandemic on total publications
4) Peak in interest in 2022
5) More physically realistic and relevant problem, RMSA, has grown compared to RWA and RSA.
6) Proportion of Other has grown despite overall decline in publication count. We hypothesise this is due to saturation of results in the field and lack of breakthrough performance, while attention has shifted to more novel problems. 


% TODO - Include graphic showing filtering methodology
% What about a timeline of developments on RL and RL-applied-to-RSA? Would be good to highlight the 






\subsection{Early applications of RL to RSA}
\label{sec:survey-early_applications}

The first application of RL to network routing (in a packet switched network) was in 1993 \cite{boyan_packet_1993}, which used Q-learning to continuously update a routing table for a 6x6 network. The optical networks community later picked up the technique in 2003 when Garcia et al investigated RL for multicast in WDM networks \cite{garcia_multicast_2003}. We note that multiple works then considered routing in Optical Burst Switched networks due to the research interest in that emerging networking paradigm at the time \cite{kiran_reinforcement_2006,belbekkouche_reinforcement_2008}. Koyanagi also \cite{koyanagi_reinforcement_2009} employed Q-learning for the task of service differentiation and noted the algorithm's sensitivity to hyperparameter selection. 

A significant milestone was then achieved by the work of Pointurier and Heidari \cite{pointurier_reinforcement_2007}, which used the simple RL technique "Linear Reward-$\epsilon$ Penalty" (LR$\epsilon$P) to update a routing table for a WDM network while accounting for quality of transmission (QoT) constraints, showing lower blocking probability than Shortest Path (SP) routing. This paper was the first to consider dynamic network traffic and to demonstrate improvement over a widely used heuristic, although it neglected to consider KSP-FF. 

A decade later in in 2017, Reyes amd Bauschert \cite{reyes_adaptive_2017} were the first to consider the problem of dynamic traffic in flex-grid networks, again using a Q-table technique. The modern era of RL, using deep neural networks as function approximators for Q-tables, was widely established by 2013 with the publication of results from Deepmind on RL for Atari videogames \cite{mnih_asynchronous_2016}. These more advanced techniques then began to be applied in optical networks in 2018, when the original DeepRMSA paper was presented at Optical Fibre Communications Conference. This seminal paper then spurred a series of incremental advances published from 2019 onwards.




\subsection{DeepRMSA and successor works}
\label{sec:survey-deeprmsa}

DeepRMSA \cite{chen_deeprmsa_2019} is the most influential paper in the field RL for DRA problems on optical networks. It's notable for being the first paper to achieve lower blocking probability than KSP-FF, through its use of a multi-layer perceptron (MLP) neural network as a function approximator for the learned policy, and its feature engineering of the observation space to enable more efficient learning. This success sparked a series of successor works.

We define successor works as papers that (i) use DeepRMSA as a benchmark, and (ii) show evidence of using the DeepRMSA codebase. Most published works do not make their code publicly available, however DeepRMSA is notable in providing open source code. Consequently, this has been used by other researchers and implementation details found in the DeepRMSA codebase have propagated to other works but not been made explicit in publications. The esoteric implementation details that originate in the DeepRMSA codebase are (i) network warm-up period, and (ii) truncated service holding time. We quantify the effects of both of these details in section \ref{sec:empirical_analysis}.

The direct successor works are from Chen et al using the DeepRMSA framework with transfer learning on different topologies and traffic \cite{chen_multi-task-learning-based_2021}, Tang et al using a shaped reward function to improve performance, Xu et al using graph and recurrent neural networks to for the policy network to improve performance \cite{xu_deep_2022}, Xu et al investigate a hierarchical framework for multi-domain networks \cite{xu_hierarchical_2023}, Errea et al use DeepRMSA with two choices of first slots per path \cite{errea_deep_2023}. DeepRMSA has also been used as a benchmark in MaskRSA \cite{shimoda_mask_2021} and PtrNet-RSA \cite{cheng_ptrnet-rsa_2024}.



\subsection{Function approximation and neural network architectures}
\label{sec:survey-function_approximators}

Optic-GAI \cite{li_opticgai_2024} uses a diffusion model, a generative model based on an iterative de-noising process and most often applied to image synthesis \cite{ho_denoising_2020}, as represent the RWA policy. This allows 

\cite{suarez-varela_routing_2019} \cite{suarez-varela_graph_2023} \cite{li_gnn-based_2022} \cite{cheng_ptrnet-rsa_2024}


\subsection{Hyperparameter evaluation}

\cite{suarez-varela_routing_2019} is the first work to actually tune hyperparameters. \cite{chen_deeprmsa_2019} made some nod towards optimisation. Many papers simply compare 



\subsection{Invalid action masking for RSA}
\label{sec:survey-masking}




\subsection{Physical layer models}
\label{sec:survey-physical_layer}
QoT-aware papers
% Most studies of the problem make simplifying assumptions about the physical layer, e.g. calculating the maximum reach of each available modulation format by making worst-case assumptions on link occupancy and resulting non-linear effects. 


Notes on PtrNet-RSA:
\textbf{Claims:} 
\textbf{Limitations:} It appears to consider a directed graph with 


\subsection{Traffic models}
\label{sec:survey-traffic_models}

\subsection{Traffic Grooming}
\label{sec:survey-grooming}


\subsection{Problem variants}
\label{sec:survey-problem_variants}
 
\textbf{Multiband}

\textbf{Multicore}

\textbf{Defragmentation} Defragmentation is outside of the scope of this work and is unlikely to be adopted by network operators.

\textbf{Other} For completeness, we mention other problem variants here. Tidal optical networks, scheduling, defragmentation, etc.

RWA with power allocation recently explored by Tse et al \cite{tse_reinforcement_2024} uses 



\subsection{Network capacity bound estimation}
\label{sec:survey-capacity_bounds}

% Comment from Robin: Really baroni should be the starting point here and david ives also has work that looks at the limiting cut which is 5-6 years before this

\cite{hayashi_efficient_2022} - first BV-VDL paper.

\cite{hayashi_cost-effective_2023} - This paper introduced the concept of cut-sets analysis on optical networks to identify congested links. Cut set analysis originates from the study of packet switched networks. Cruzado et al modify the technique to apply to the spectrum-based connections on optical networks.

\cite{cruzado_effective_2023} - Proposes cut-sets method to estimate network capacity for fixed-grid RWA and suggests heuristic algorithm for RSA that incorporates this information.

\cite{cruzado_capacity-bound_2024} - Proposes cut-sets method to estimate network capacity for flex-grid elastic optical network with distance-adaptive modulation format. 

% Do we also need to include e.g. AUR-E here? 



\subsection{Reporting metrics}
To conclude the survey of the literature, the reporting metrics 
Use bitrate blocking probability instead of service blocking probability, especially if traffic requests have diverse bitrate requirements.
Target metric - set a target bitrate blocking probability and the \% improvement reported is in the increased traffic level that can be supported at the target blocking probability.


\subsection{Robustness of results}
Problems with the number of trials (not a statistically significant number of blocking events in many cases) and no uncertainty estimates are published.
Comparison to other RL solutions is not fair as the final performance depends greatly on the training regime. If the problem setting changes from that in the original paper for an RL solution, then hyperparameters must be re-tuned (rather than relying on published values). Even the random seed for the training run can affect the outcome, and to be fair a range of seeds e.g. 5 minimum should be used in order to have a range of values and compare against the maximum performance. Simply put, it is not meaningful to compare against another RL solution unless a comparable or reasonable amount of computing resources have been spent on tuning hyperparameters and training the final model, as were spent on the novel proposal.
Lack of open source to verify results.
