x\subsection{Development of RL for DRA in optical networks}
\label{sec:survey-development}

In this section, we provide a survey of the progress in RL for DRA problems in optical networks. Our dataset of 96 collected papers (Figure \ref{fig:lit_barchart}) is useful to reveal trends in the research directions but we do not aim to discuss each paper in detail. Instead, we provide a narrative of the field's development, highlighting key papers and innovations. We choose to focus mainly on RWA/RSA/RMSA papers, to facilitate comparisons between works and construct a timeline of innovation.

We exclude many of the works gathered in our search from inclusion in this review, because we evaluate their contribution to the field as insignificant. In many cases, any claimed performance improvement is trivial due to comparison against only weak benchmarks, such as shortest path routing. %As we show in section \ref{sec:heuristic_comparison}, 




\subsubsection{Early works}
The first application of RL to network routing (in a packet switched network) was in 1993 \cite{boyan_packet_1993}, when Q-learning was used to continuously update a routing table for a 6x6 network. The optical networks community later picked up the technique in 2003 when Garcia et al investigated RL for multicast in WDM networks \cite{garcia_multicast_2003}. We note that multiple works then considered routing in Optical Burst Switched networks due to the research interest in that emerging networking paradigm at the time \cite{kiran_reinforcement_2006,belbekkouche_reinforcement_2008}. Koyanagi also \cite{koyanagi_reinforcement_2009} employed Q-learning for the task of service differentiation and noted the algorithm's sensitivity to hyperparameter selection. 

A significant milestone was achieved by the work of Pointurier and Heidari \cite{pointurier_reinforcement_2007}, which used the simple RL algorithm to update a routing table for a WDM network while accounting for quality of transmission (QoT) constraints, showing lower blocking probability than Shortest Path (SP) routing. This paper was the first to consider dynamic network traffic and to demonstrate improvement over a widely used heuristic, although it neglected to consider multiple candidate paths as KSP-FF. 

A decade later in 2017, Reyes amd Bauschert \cite{reyes_adaptive_2017} were the first to consider RL for dynamic traffic in flex-grid networks, again using a tabular approach. The modern era of RL, using deep neural networks as function approximators, was established by 2013 with the use of RL for Atari videogames (later published in Nature \cite{mnih_human-level_2015}). These more advanced techniques then began to be applied in optical networks in 2018, when the original DeepRMSA paper was presented at conference \cite{chen_deep-rmsa_2018}. We deascribe DeepRMSA in section \ref{sec:} This seminal paper spurred a series of incremental advances from 2019 onwards.




\subsection{Simulation and training libraries}
\label{sec:simulation_libraries}
Following DeepRMSA, there was a rise in publications on RL for DRA problems, as seen in Figure \ref{fig:lit_barchart}. This research was enabled by optical network simulation tools designed for RL training. Several open source toolkits have been introduced to aid researchers and improve productivity, but none has proved sufficiently popular for it to become the de facto standard. Optical-rl-gym \cite{natalino_optical_2020} was the first paper to attempt to introduce a new standard library for this task. This was followed by an extension to multi-band environments \cite{morales_multi-band_2021} and in 2024 was further extended to include a more sophisticated physical layer model for lightpath SNR calculations, renamed as the Optical Networking Gym \cite{natalino_optical_2024}. Meanwhile, the authors of MaskRSA open-sourced their simulation framework (RSA-RL) and DeepRMSA's codebase continued to be widely used. SDONSim \cite{mccann_sdonsim_2024} and DREAM-ON-GYM \cite{jara_dream-gym_2024} are other recent additions to the landscape of available simulation frameworks that further fragment the available options. In this work we use XLRON \cite{doherty_xlron_2023} the framework introduced at Optical Fibre Communications (OFC) 2024 that we continue to develop. We outline its benefits and justify its use in section \ref{sec:empirical_analysis}.



\subsubsection{DeepRMSA and feature engineering}
\label{sec:survey-deeprmsa}

DeepRMSA \cite{chen_deeprmsa_2019} is the most influential paper on RL for DRA, as measured by citations. It's notable as the first paper to achieve lower blocking probability than KSP-FF, through its use of a multi-layer perceptron (MLP) neural network as a function approximator for the learned policy, and its feature engineering of the observation space to enable more efficient learning. %This success sparked a series of successor works. We define successor works as papers that (i) use DeepRMSA as a benchmark, and/or (ii) show evidence of using the DeepRMSA codebase.
The impact of DeepRMSA is enhanced by its open source codebase, which has been used extensively by other researchers \cite{tang_heuristic_2022,xu_deep_2022,quang_magc-rsa_2022}.

Similar to DeepRMSA, in a contemporaneous work Su\'arez-Varela et al. \cite{suarez-varela_routing_2019} explored different state representations to aid learning. The paper is notable for its in-depth exploration of the effect of hyperparameters and network topology on the RL performance, and its thoughtful design of the observation space. It was less impactful than DeepRMSA perhaps because it benchmarked against shortest path routing only. A similar approach to feature engineering to aid the learning process was carried out by Tanaka and Shimoda \cite{tanaka_pre-_2023} in 2023, where they encode path information into the matrix of FSUs and links and use it to train an RL policy that is superior to KSP-FF in terms of blocking probability. Cruzado et al. also use path-based feature engineering in their work from 2022 \cite{cruzado_reinforcement-learning-based_2022}, but in general such manual feature engineering has been rejected in favour of end-to-end learning, particularly with graph neural networks (GNNs).


\subsubsection{Graph neural networks}
\label{sec:gnn_rl}

GNNs are well-suited to optical networking problems due to their inductive bias for graph-structured data \cite{xu_how_2018}. 


Then Quang et al. introduced MAGC-RSA \cite{quang_magc-rsa_2022}, which utilizes invalid action masking like MaskRSA, the shaped reward from Reward-RMSA \cite{tang_heuristic_2022}, and a graph convolutional neural network like GCN-RMSA \cite{xu_deep_2022}. They claim it is multi-agent with a separate agent at each node, but actually they use a standard GNN with attention, and they claim it is better than KSP-FF and (their implementation of) DeepRMSA. However, they do not consider distance-dependent modulation format.

 \cite{suarez-varela_graph_2023} \cite{li_gnn-based_2022}  Another GNN-RNN based paper here \cite{xiong_graph_2024}
 
 \cite{almasan_deep_2022} examines RSA as a use case for GNN's coupled with RL, with a particular focus on GNNs ability to generalise to topologies unseen during training. They find that the GNN-RL approach does not perform as well as a theoretical fluid model for network routing on unseen topologies, so generalization of RL-trained network routing policies remains an open problem.
 
 % MORE ADVANCED FUNCTION APPROXIMATORS
 
 More exotic deep learning architectures have also been employed for DRA problems. Optic-GAI \cite{li_opticgai_2024} uses a diffusion model, a generative model based on an iterative de-noising process and most often applied to image synthesis \cite{ho_denoising_2020}, as represent the RWA policy. The authors claim that this model is superior to KSP-FF for RWA but do not specify how many paths they consider.
 
 \cite{cheng_ptrnet-rsa_2024}
 

 


\subsubsection{Other techniques}
\label{sec:other_tehcniques}
% Talk about invalid action masking and reward shaping


Then Shimoda et al. make a significant contribution in two papers that introduced the concept originally referred to as assignable boundary slot mask \cite{shimoda_deep_2021} and then invalid action masking in the work MaskRSA \cite{shimoda_mask_2021}. The concept of invalid action masking was already well-known in the field of RL following its use in the AlphaStar paper from Google Deepmind \cite{vinyals_grandmaster_2019} and subsequent empirical investigations \cite{huang_closer_2020} but MaskRSA was the first paper to introduce it to optical networking, and contemporaneously with works on data centre networks \cite{shabka_resource_2021}.





\subsection{Emerging research trends}
\label{sec:survey-emerging_trends}

\cite{hernandez-chulde_experimental_2023} Latency awareness
\cite{nallaperuma_interpreting_2023} multi-objective RL


Another significant paper from 2022 from Nevin et al. \cite{nevin_techniques_2022}  is on techniques for the application of RL to RWA; specifically 1) invalid action masking and 2) scaling the network capacity to reduce the episode length. It poses a novel problem setting that seeks to closely replicate the resource allocation problem in most production networks of incremental traffic on a fixed-grid (100GHz per wavelength) with re-use of existing lightpaths with spare capacity to accommodate additional traffic (so-called "RWA with lightpath re-use" or RWA-LR). The merits of the work are multiple; it presents a novel but realistic problem setting with a physical layer model that calculates lightpath capacities based on a simple GN model and Nyquist assumptions, it elaborates on the invalid action masking technique initially introduced to the field by MaskRSA \cite{shimoda_mask_2021}, and it articulates the fundamental difficulty of long episode lengths (i.e. sparse rewards) in the application of RL to this problem. It does, however, suffer from the common problem of benchmarking against the weaker KSP-FF heuristic with paths ordered by length. %We have recreated the RWA-LR problem setting in XLRON and applied KSP-FF with both path orderings to confirm our recreation accords with Nevin et al.. Our initial experiments confirm that ordering by number of hops allows greater throughput.


Xu et al. investigate a hierarchical framework for multi-domain networks \cite{xu_hierarchical_2023}

Multi-objective optimization of network condition is an emerging trend \cite{cheng_ptrnet-rsa_2024,xie_physical_2024,teng_deep-reinforcement-learning-based_2024}

Other problems could bear more fruit e.g. channel power optimization \cite{arpanaei_launch_2023,xiao_channel_2024}.