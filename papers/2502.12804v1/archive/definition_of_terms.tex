\subsection*{Definition of terms}

We define the following terms used in this paper for the purposes of disambiguation:

\textbf{Path} - A series of network fibre links that comprise a route from source to destination nodes.

\textbf{Lightpath} - The path and specified bandwidth slice that constitute a connection from source to destination, and which obeys the spectrum continuity and contiguity constraints.

\textbf{Connection request} - Network data traffic comprises a series of connection requests; tuples of source node, destination node, and requested data rate. 

\textbf{Static traffic} - All connection requests are known a priori, can be allocated in any order, and do not expire.

\textbf{Dynamic traffic} - Future connection requests are unknown, are allocated sequentially, and expire.

\textbf{Fixed grid} - Available spectrum is partitioned in spectral slots of width 50 or 100 GHz.

\textbf{Flex-grid} - Available spectrum is partitioned into spectral slots of width 6.25GHz or greater, up to 37.5GHz.

\textbf{Flex-rate} - Available spectrum is fixed grid but each lightpath can support multiple connection requests until the available bandwidth of a lightpath is fully utilised, by adjusting data rate of transceiver.

\textbf{RWA} - The routing and wavelength allocation problem. The task is to assign lightpaths to connection requests on a fixed grid network, in order to minimise an objective or set of objectives e.g. no. of blocked connections, latency, no. of transceivers.

\textbf{RSA} - The routing and spectrum allocation problem. Identical to RWA but on a flex-grid substrate network.

\textbf{RMSA} - The routing, modulation and spectrum assignment problem. Identical to RSA but with the added constraint of SNR-dependent modulation format, whereby the bandwidth required to support a particular data rate on a lightpath depends on the available SNR, often assumed to be entirely distance-dependent. 

\textbf{DRA problems} - The umbrella term for dynamic resource allocation problems i.e. RWA, RSA, RMSA problems and variants. For the purposes of this review, this definition refers to problems that require routing of unicast traffic and do not consider node resources. This excludes problems such as network function virtualisation (NFV), virtual optical network embedding (VONE) and distributed sub-tree scheduling \cite{li_tabdeep_2024}. We exclude static traffic, for which exact solution methods are applicable.