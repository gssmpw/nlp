



\subsection{Holding time truncation}
\label{sec:holding_time}

The traffic model in DeepRMSA and follow-up works includes an unusual detail: the service holding time is resampled if the resulting value is more than twice the mean of the inverse exponential PDF. We refer to this detail as holding time truncation. 


\subsubsection{Experiment setup}
To understand the effect of truncation on the traffic statistics, we define an inverse exponential PDF that is normalized to have unit mean. We take $10^{6}$ samples from the PDF, both with and without truncation, and calculate the mean of the resulting sample populations in both cases.

\subsubsection{Results and discussion}
Figure \ref{fig:truncation} compares histograms of service holding times with and without truncation. We define the bin width as 0.01 and normalize the count per bin to give a peak density of 1 without truncation. The truncated case shows a cutoff at twice the mean holding time. The vertical lines indicate the mean holding time for each case.

Holding time truncation reduces the mean by approximately 31\%. This results in 31\% lower traffic load. Therefore, papers that use the DeepRMSA codebase (including DeepRMSA, Reward-RMSA, and GCN-RMSA) evaluate their solutions at traffic loads 31\% lower than reported. This detail is not made explicit in the published papers. This finding highlights the challenges in making fair comparisons between papers, and the need for transparency in research code.

\begin{figure}
    \centering
    \includegraphics[width=0.9\linewidth]{IMAGES/truncation.png}
    \caption{Histogram of service holding holding times. The truncated distribution resamples the holding time when the sampled value exceeds 2*mean. This reduces the mean holding time by 31\% compared to the inverse exponential distribution.}
    \label{fig:truncation}
\end{figure}



%Cisco NCS 2000 Flex Spectrum Single Module ROADM and it's integrated pre-amplifier has a nominal noise figure of 5.5dB (for 24dB gain) or 11.7dB (for 12dB gain), 


