
\documentclass[9pt,twocolumn,twoside]{osajnl}

\journal{jocn} 

% Set the article type for journal submissions. Comment out this line for Optica Open preprint submissions.
\setboolean{shortarticle}{false}
% true = letter / tutorial
% false = research / review article

\title{Reinforcement Learning for Dynamic Resource Allocation in Optical Networks: Hype or Hope?}

\author[1*]{Michael Doherty}
\author[1]{Robin Matzner}
\author[1]{Rasoul Sadeghi}
\author[1]{Polina~Bayvel}
\author[1]{Alejandra~Beghelli}

\affil[1]{Optical Networks Group, University College London, Torrington Place, London WC1E 7JE, United Kingdom}


\affil[*]{Corresponding author: michael.doherty.21@ucl.ac.uk}

%% To be edited by editor
% \dates{Compiled \today}

%% To be edited by editor
% \doi{\url{http://dx.doi.org/10.1364/XX.XX.XXXXXX}}


\begin{abstract}
    The application of reinforcement learning (RL) to dynamic resource allocation in optical networks has been the focus of intense research activity in recent years, with almost 100 peer-reviewed papers. We present a review of progress in the field, and identify significant gaps in benchmarking practices and reproducibility. To determine the strongest benchmark algorithms, we systematically evaluate several heuristics across diverse network topologies. We find that path count and sort criteria for path selection significantly affect the benchmark performance. We meticulously recreate the problems from five landmark papers and apply the improved benchmarks. Our comparisons demonstrate that simple heuristics consistently match or outperform the published RL solutions, often with an order of magnitude lower blocking probability. Furthermore, we present empirical lower bounds on network blocking using a novel defragmentation-based method, revealing that potential improvements over the benchmark heuristics are limited to 19--36\% increased traffic load for the same blocking performance in our examples. We make our simulation framework and results publicly available to promote reproducible research and standardized evaluation \hyperlink{https://doi.org/10.5281/zenodo.12594495}{https://doi.org/10.5281/zenodo.12594495}.%{10.5281/zenodo.12594495}.
\end{abstract}


% \begin{abstract}
% We survey the literature on reinforcement learning (RL) applied to dynamic resource allocation (DRA) problems in optical networks. Through critical examination of the reported experimental methodologies and results, we summarize the different approaches taken by researchers and highlight the difficulty in comparing results across non-standardized network topologies, traffic models and training regimes. We argue that, despite intense research interest, the improvement achieved by RL over simple heuristic algorithms on DRA problems has been overstated or non-existent. By quantifying the effect of implementation details such as pre-populated network traffic, inconsistencies in traffic models and path prioritization, we show several RL-based resource allocation policies from the literature are outperformed by heuristic algorithms. Further to this analysis, we estimate the upper bounds network throughput for 11 cases of study and show possible improvement over heuristics is limited to 1-36\% increase in supported traffic load. We conclude with a discussion of how RL performance can be improved on DRA problems and suggest directions for future research in which greater performance gains may be possible. We make all of our code available at \hyperlink{10.5281/zenodo.14561967}{10.5281/zenodo.14561967}.
% \end{abstract}

\setboolean{displaycopyright}{false} % Do not include copyright or licensing information in submission.

\begin{document}

\maketitle

\section{Introduction}

Video generation has garnered significant attention owing to its transformative potential across a wide range of applications, such media content creation~\citep{polyak2024movie}, advertising~\citep{zhang2024virbo,bacher2021advert}, video games~\citep{yang2024playable,valevski2024diffusion, oasis2024}, and world model simulators~\citep{ha2018world, videoworldsimulators2024, agarwal2025cosmos}. Benefiting from advanced generative algorithms~\citep{goodfellow2014generative, ho2020denoising, liu2023flow, lipman2023flow}, scalable model architectures~\citep{vaswani2017attention, peebles2023scalable}, vast amounts of internet-sourced data~\citep{chen2024panda, nan2024openvid, ju2024miradata}, and ongoing expansion of computing capabilities~\citep{nvidia2022h100, nvidia2023dgxgh200, nvidia2024h200nvl}, remarkable advancements have been achieved in the field of video generation~\citep{ho2022video, ho2022imagen, singer2023makeavideo, blattmann2023align, videoworldsimulators2024, kuaishou2024klingai, yang2024cogvideox, jin2024pyramidal, polyak2024movie, kong2024hunyuanvideo, ji2024prompt}.


In this work, we present \textbf{\ours}, a family of rectified flow~\citep{lipman2023flow, liu2023flow} transformer models designed for joint image and video generation, establishing a pathway toward industry-grade performance. This report centers on four key components: data curation, model architecture design, flow formulation, and training infrastructure optimization—each rigorously refined to meet the demands of high-quality, large-scale video generation.


\begin{figure}[ht]
    \centering
    \begin{subfigure}[b]{0.82\linewidth}
        \centering
        \includegraphics[width=\linewidth]{figures/t2i_1024.pdf}
        \caption{Text-to-Image Samples}\label{fig:main-demo-t2i}
    \end{subfigure}
    \vfill
    \begin{subfigure}[b]{0.82\linewidth}
        \centering
        \includegraphics[width=\linewidth]{figures/t2v_samples.pdf}
        \caption{Text-to-Video Samples}\label{fig:main-demo-t2v}
    \end{subfigure}
\caption{\textbf{Generated samples from \ours.} Key components are highlighted in \textcolor{red}{\textbf{RED}}.}\label{fig:main-demo}
\end{figure}


First, we present a comprehensive data processing pipeline designed to construct large-scale, high-quality image and video-text datasets. The pipeline integrates multiple advanced techniques, including video and image filtering based on aesthetic scores, OCR-driven content analysis, and subjective evaluations, to ensure exceptional visual and contextual quality. Furthermore, we employ multimodal large language models~(MLLMs)~\citep{yuan2025tarsier2} to generate dense and contextually aligned captions, which are subsequently refined using an additional large language model~(LLM)~\citep{yang2024qwen2} to enhance their accuracy, fluency, and descriptive richness. As a result, we have curated a robust training dataset comprising approximately 36M video-text pairs and 160M image-text pairs, which are proven sufficient for training industry-level generative models.

Secondly, we take a pioneering step by applying rectified flow formulation~\citep{lipman2023flow} for joint image and video generation, implemented through the \ours model family, which comprises Transformer architectures with 2B and 8B parameters. At its core, the \ours framework employs a 3D joint image-video variational autoencoder (VAE) to compress image and video inputs into a shared latent space, facilitating unified representation. This shared latent space is coupled with a full-attention~\citep{vaswani2017attention} mechanism, enabling seamless joint training of image and video. This architecture delivers high-quality, coherent outputs across both images and videos, establishing a unified framework for visual generation tasks.


Furthermore, to support the training of \ours at scale, we have developed a robust infrastructure tailored for large-scale model training. Our approach incorporates advanced parallelism strategies~\citep{jacobs2023deepspeed, pytorch_fsdp} to manage memory efficiently during long-context training. Additionally, we employ ByteCheckpoint~\citep{wan2024bytecheckpoint} for high-performance checkpointing and integrate fault-tolerant mechanisms from MegaScale~\citep{jiang2024megascale} to ensure stability and scalability across large GPU clusters. These optimizations enable \ours to handle the computational and data challenges of generative modeling with exceptional efficiency and reliability.


We evaluate \ours on both text-to-image and text-to-video benchmarks to highlight its competitive advantages. For text-to-image generation, \ours-T2I demonstrates strong performance across multiple benchmarks, including T2I-CompBench~\citep{huang2023t2i-compbench}, GenEval~\citep{ghosh2024geneval}, and DPG-Bench~\citep{hu2024ella_dbgbench}, excelling in both visual quality and text-image alignment. In text-to-video benchmarks, \ours-T2V achieves state-of-the-art performance on the UCF-101~\citep{ucf101} zero-shot generation task. Additionally, \ours-T2V attains an impressive score of \textbf{84.85} on VBench~\citep{huang2024vbench}, securing the top position on the leaderboard (as of 2025-01-25) and surpassing several leading commercial text-to-video models. Qualitative results, illustrated in \Cref{fig:main-demo}, further demonstrate the superior quality of the generated media samples. These findings underscore \ours's effectiveness in multi-modal generation and its potential as a high-performing solution for both research and commercial applications.

\section{Background} \label{section:LLM}

% \subsection{Large Language Model (LLM)}   

Figure~\ref{fig:LLaMA_model}(a) shows that a decoder-only LLM initially processes a user prompt in the “prefill” stage and subsequently generates tokens sequentially during the “decoding” stage.
Both stages contain an input embedding layer, multiple decoder transformer blocks, an output embedding layer, and a sampling layer.
Figure~\ref{fig:LLaMA_model}(b) demonstrates that the decoder transformer blocks consist of a self attention and a feed-forward network (FFN) layer, each paired with residual connection and normalization layers. 

% Differentiate between encoder/decoder, explain why operation intensity is low, explain the different parts of a transformer block. Discuss Table II here. 

% Explain the architecture with Llama2-70B.

% \begin{table}[thb]
% \renewcommand\arraystretch{1.05}
% \centering
% % \vspace{-5mm}
%     \caption{ML Model Parameter Size and Operational Intensity}
%     \vspace{-2mm}
%     \small
%     \label{tab:ML Model Parameter Size and Operational Intensity}    
%     \scalebox{0.95}{
%         \begin{tabular}{|c|c|c|c|c|}
%             \hline
%             & Llama2 & BLOOM & BERT & ResNet \\
%             Model & (70B) & (176B) & & 152 \\
%             \hline
%             Parameter Size (GB) & 140 & 352 & 0.17 & 0.16 \\
%             \hline
%             Op Intensity (Ops/Byte) & 1 & 1 & 282 & 346 \\
%             \hline
%           \end{tabular}
%     }
% \vspace{-3mm}
% \end{table}

% {\fontsize{8pt}{11pt}\selectfont 8pt font size test Memory Requirement}

\begin{figure}[t]
    \centering
    \includegraphics[width=8cm]{Figure/LLaMA_model_new_new.pdf}
    \caption{(a) Prefill stage encodes prompt tokens in parallel. Decoding stage generates output tokens sequentially.
    (b) LLM contains N$\times$ decoder transformer blocks. 
    (c) Llama2 model architecture.}
    \label{fig:LLaMA_model}
\end{figure}

Figure~\ref{fig:LLaMA_model}(c) demonstrates the Llama2~\cite{touvron2023llama} model architecture as a representative LLM.
% The self attention layer requires three GEMVs\footnote{GEMVs in multi-head attention~\cite{attention}, narrow GEMMs in grouped-query attention~\cite{gqa}.} to generate query, key and value vectors.
In the self-attention layer, query, key and value vectors are generated by multiplying input vector to corresponding weight matrices.
These matrices are segmented into multiple heads, representing different semantic dimensions.
The query and key vectors go though Rotary Positional Embedding (RoPE) to encode the relative positional information~\cite{rope-paper}.
Within each head, the generated key and value vectors are appended to their caches.
The query vector is multiplied by the key cache to produce a score vector.
After the Softmax operation, the score vector is multiplied by the value cache to yield the output vector.
The output vectors from all heads are concatenated and multiplied by output weight matrix, resulting in a vector that undergoes residual connection and Root Mean Square layer Normalization (RMSNorm)~\cite{rmsnorm-paper}.
The residual connection adds up the input and output vectors of a layer to avoid vanishing gradient~\cite{he2016deep}.
The FFN layer begins with two parallel fully connections, followed by a Sigmoid Linear Unit (SiLU), and ends with another fully connection.

\section{Literature Survey}
\label{sec:survey}

There exists a considerable body of literature on RL for dynamic resource allocation (DRA) in optical networks \cite{amin_survey_2021}. DRA in optical networks is distinguished from similar problems in electronically linked networks by the nature of fibre optic links. The capacity of each fibre link is a vector of available wavelengths or spectrum slots, as defined by the ITU standards G.671 and G.694.2 \cite{international_telecommunication_union_spectral_2002,international_telecommunication_union_transmission_2012}, not a scalar quantity. Due to this fundamental difference, we only consider publications related to optical networks in this review, but acknowledge the closely related body of literature on RL for other graph-based resource allocation problems.

\begin{figure}
    \centering
    \includegraphics[width=1\linewidth]{IMAGES/RL_RSA_litreview_barchart.png}
    \caption{Count of publications related to RL for DRA problems in optical networks. Citations for each classification category are: RWA \cite{garcia_multicast_2003,pointurier_reinforcement_2007,koyanagi_reinforcement_2009,suarez-varela_routing_2019,shiraki_dynamic_2019,shiraki_reinforcement-learning-based_2019,zhao_cost-efficient_2021,freire-hermelo_dynamic_2021,liu_waveband_2021,nevin_techniques_2022,di_cicco_deep_2022,di_cicco_deepls_2023,nallaperuma_interpreting_2023}, RSA \cite{reyes_adaptive_2017,li_deepcoop_2020,li_multi-objective_2020,wang_dynamic_2021,zhao_reinforced_2021,shimoda_deep_2021,shimoda_mask_2021,romero_reyes_towards_2021,zhao_rsa_2022,wu_service_2022,arce_reinforcement_2022,quang_magc-rsa_2022,cruzado_reinforcement-learning-based_2022,jiao_reliability-oriented_2022,hernandez-chulde_experimental_2023,sharma_deep_2023,lin_deep-reinforcement-learning-based_2023,tanaka_pre-_2023,cheng_ptrnet-rsa_2024}, RMSA \cite{chen_deeprmsa_2019,luo_leveraging_2019,chen_multi-task-learning-based_2021,shi_deep-reinforced_2021,sheikh_multi-band_2021,xu_spectrum_2021,gonzalez_improving_2022,tang_deep_2022,momo_ziazet_deep_2022,terki_routing_2022,tu_entropy-based_2022,cheng_routing_2022,xu_deep_2022,pinto-rios_resource_2022,tang_heuristic_2022,luo_survivable_2022,sadeghi_performance_2023,errea_deep_2023,terki_routing_2023,tang_routing_2023,xu_hierarchical_2023,li_opticgai_2024,teng_drl-assisted_2024,teng_deep-reinforcement-learning-based_2024}, Other \cite{boyan_packet_1993,ma_demonstration_2019,zhao_reinforcement-learning-based_2019,wang_subcarrier-slot_2019,natalino_optical_2020,ma_co-allocation_2020,weixer_reinforcement_2020,wang_deepcms_2020,tian_reconfiguring_2021,liu_multi-agent_2021,zhao_service_2021,morales_multi-band_2021,hernandez-chulde_evaluation_2022,koch_reinforcement_2022,tanaka_reinforcement-learning-based_2022,etezadi_deepdefrag_2022,beghelli_approaches_2023,renjith_deep_2023,davalos_triggering_2023,etezadi_deep_2023,zhang_admire_2023,tanaka_adaptive_2023,johari_drl-assisted_2023,pavon-marino_tree-determination_2023,fan_blocking-driven_2023,yin_dnn_2024,li_tabdeep_2024,tanaka_reinforcement-learning-based_2024,lian_dynamic_2024,doherty_xlron_2024,wang_availability-aware_2024,tse_reinforcement_2024,jara_dream-gym_2024}.}
    \label{fig:lit_barchart}
\end{figure}



\subsection{Survey methodology}
\label{sec:survey-methodology}

To find relevant papers, we searched the Google Scholar database using the following search terms: "'REINFORCEMENT LEARNING' AND 'NETWORK' AND ('OPTICAL' OR 'WAVELENGTH' OR 'SPECTRUM')". Papers were limited to English-language peer-reviewed articles. We filtered the search results by inspection of paper titles and abstracts, and manually added any related works from our own citation database that were missing from the initial search. A review of the collected papers then created the final set of 96.

We reviewed each paper in the set to classify them into four broad categories: 'RWA', 'RSA', 'RMSA' for those that apply RL to the respective problem variants, and 'Other'.
The 'Other' category includes papers that are concerned with RL applied to: traffic grooming, defragmentation, survivability or service restoration, multicast provisioning, simulation training environments, and other problem variants such as transceiver parameter optimisation.

We notice the following characteristics and trends in the gathered dataset of papers: 
1) Increase in papers from 2019 onwards due to DeepRMSA seminal paper. Arrival of paper in 2019 shows time lag between cutting edge ML papers (Atari in 2013 and AlphaGo 2016) and adoption of techniques by optical networks research community.
2) RWA occupied large proportion of total papers in 2019 and before but became a smaller proportion and no publications in 2024.
3) 2020 impact of COVID pandemic on total publications
4) Peak in interest in 2022
5) More physically realistic and relevant problem, RMSA, has grown compared to RWA and RSA.
6) Proportion of Other has grown despite overall decline in publication count. We hypothesise this is due to saturation of results in the field and lack of breakthrough performance, while attention has shifted to more novel problems. 


% TODO - Include graphic showing filtering methodology
% What about a timeline of developments on RL and RL-applied-to-RSA? Would be good to highlight the 






\subsection{Early applications of RL to RSA}
\label{sec:survey-early_applications}

The first application of RL to network routing (in a packet switched network) was in 1993 \cite{boyan_packet_1993}, which used Q-learning to continuously update a routing table for a 6x6 network. The optical networks community later picked up the technique in 2003 when Garcia et al investigated RL for multicast in WDM networks \cite{garcia_multicast_2003}. We note that multiple works then considered routing in Optical Burst Switched networks due to the research interest in that emerging networking paradigm at the time \cite{kiran_reinforcement_2006,belbekkouche_reinforcement_2008}. Koyanagi also \cite{koyanagi_reinforcement_2009} employed Q-learning for the task of service differentiation and noted the algorithm's sensitivity to hyperparameter selection. 

A significant milestone was then achieved by the work of Pointurier and Heidari \cite{pointurier_reinforcement_2007}, which used the simple RL technique "Linear Reward-$\epsilon$ Penalty" (LR$\epsilon$P) to update a routing table for a WDM network while accounting for quality of transmission (QoT) constraints, showing lower blocking probability than Shortest Path (SP) routing. This paper was the first to consider dynamic network traffic and to demonstrate improvement over a widely used heuristic, although it neglected to consider KSP-FF. 

A decade later in in 2017, Reyes amd Bauschert \cite{reyes_adaptive_2017} were the first to consider the problem of dynamic traffic in flex-grid networks, again using a Q-table technique. The modern era of RL, using deep neural networks as function approximators for Q-tables, was widely established by 2013 with the publication of results from Deepmind on RL for Atari videogames \cite{mnih_asynchronous_2016}. These more advanced techniques then began to be applied in optical networks in 2018, when the original DeepRMSA paper was presented at Optical Fibre Communications Conference. This seminal paper then spurred a series of incremental advances published from 2019 onwards.




\subsection{DeepRMSA and successor works}
\label{sec:survey-deeprmsa}

DeepRMSA \cite{chen_deeprmsa_2019} is the most influential paper in the field RL for DRA problems on optical networks. It's notable for being the first paper to achieve lower blocking probability than KSP-FF, through its use of a multi-layer perceptron (MLP) neural network as a function approximator for the learned policy, and its feature engineering of the observation space to enable more efficient learning. This success sparked a series of successor works.

We define successor works as papers that (i) use DeepRMSA as a benchmark, and (ii) show evidence of using the DeepRMSA codebase. Most published works do not make their code publicly available, however DeepRMSA is notable in providing open source code. Consequently, this has been used by other researchers and implementation details found in the DeepRMSA codebase have propagated to other works but not been made explicit in publications. The esoteric implementation details that originate in the DeepRMSA codebase are (i) network warm-up period, and (ii) truncated service holding time. We quantify the effects of both of these details in section \ref{sec:empirical_analysis}.

The direct successor works are from Chen et al using the DeepRMSA framework with transfer learning on different topologies and traffic \cite{chen_multi-task-learning-based_2021}, Tang et al using a shaped reward function to improve performance, Xu et al using graph and recurrent neural networks to for the policy network to improve performance \cite{xu_deep_2022}, Xu et al investigate a hierarchical framework for multi-domain networks \cite{xu_hierarchical_2023}, Errea et al use DeepRMSA with two choices of first slots per path \cite{errea_deep_2023}. DeepRMSA has also been used as a benchmark in MaskRSA \cite{shimoda_mask_2021} and PtrNet-RSA \cite{cheng_ptrnet-rsa_2024}.



\subsection{Function approximation and neural network architectures}
\label{sec:survey-function_approximators}

Optic-GAI \cite{li_opticgai_2024} uses a diffusion model, a generative model based on an iterative de-noising process and most often applied to image synthesis \cite{ho_denoising_2020}, as represent the RWA policy. This allows 

\cite{suarez-varela_routing_2019} \cite{suarez-varela_graph_2023} \cite{li_gnn-based_2022} \cite{cheng_ptrnet-rsa_2024}


\subsection{Hyperparameter evaluation}

\cite{suarez-varela_routing_2019} is the first work to actually tune hyperparameters. \cite{chen_deeprmsa_2019} made some nod towards optimisation. Many papers simply compare 



\subsection{Invalid action masking for RSA}
\label{sec:survey-masking}




\subsection{Physical layer models}
\label{sec:survey-physical_layer}
QoT-aware papers
% Most studies of the problem make simplifying assumptions about the physical layer, e.g. calculating the maximum reach of each available modulation format by making worst-case assumptions on link occupancy and resulting non-linear effects. 


Notes on PtrNet-RSA:
\textbf{Claims:} 
\textbf{Limitations:} It appears to consider a directed graph with 


\subsection{Traffic models}
\label{sec:survey-traffic_models}

\subsection{Traffic Grooming}
\label{sec:survey-grooming}


\subsection{Problem variants}
\label{sec:survey-problem_variants}
 
\textbf{Multiband}

\textbf{Multicore}

\textbf{Defragmentation} Defragmentation is outside of the scope of this work and is unlikely to be adopted by network operators.

\textbf{Other} For completeness, we mention other problem variants here. Tidal optical networks, scheduling, defragmentation, etc.

RWA with power allocation recently explored by Tse et al \cite{tse_reinforcement_2024} uses 



\subsection{Network capacity bound estimation}
\label{sec:survey-capacity_bounds}

% Comment from Robin: Really baroni should be the starting point here and david ives also has work that looks at the limiting cut which is 5-6 years before this

\cite{hayashi_efficient_2022} - first BV-VDL paper.

\cite{hayashi_cost-effective_2023} - This paper introduced the concept of cut-sets analysis on optical networks to identify congested links. Cut set analysis originates from the study of packet switched networks. Cruzado et al modify the technique to apply to the spectrum-based connections on optical networks.

\cite{cruzado_effective_2023} - Proposes cut-sets method to estimate network capacity for fixed-grid RWA and suggests heuristic algorithm for RSA that incorporates this information.

\cite{cruzado_capacity-bound_2024} - Proposes cut-sets method to estimate network capacity for flex-grid elastic optical network with distance-adaptive modulation format. 

% Do we also need to include e.g. AUR-E here? 



\subsection{Reporting metrics}
To conclude the survey of the literature, the reporting metrics 
Use bitrate blocking probability instead of service blocking probability, especially if traffic requests have diverse bitrate requirements.
Target metric - set a target bitrate blocking probability and the \% improvement reported is in the increased traffic level that can be supported at the target blocking probability.


\subsection{Robustness of results}
Problems with the number of trials (not a statistically significant number of blocking events in many cases) and no uncertainty estimates are published.
Comparison to other RL solutions is not fair as the final performance depends greatly on the training regime. If the problem setting changes from that in the original paper for an RL solution, then hyperparameters must be re-tuned (rather than relying on published values). Even the random seed for the training run can affect the outcome, and to be fair a range of seeds e.g. 5 minimum should be used in order to have a range of values and compare against the maximum performance. Simply put, it is not meaningful to compare against another RL solution unless a comparable or reasonable amount of computing resources have been spent on tuning hyperparameters and training the final model, as were spent on the novel proposal.
Lack of open source to verify results.


\section{Heuristic algorithm benchmark evaluation}
\label{sec:heuristic_comparison}

To evaluate the results from our selected papers in section \ref{sec:paper_summaries}, we must determine the best (lowest blocking probability) heuristic algorithms to use as benchmarks. In this section, we present comparisons of the heuristics listed in Table \ref{tab:heuristics}, evaluated on different traffic loads, topologies, and considering different numbers of candidate paths (K). On the basis of this analysis, we select the benchmarks to apply in section \ref{sec:repro_main}.

We select the algorithms in Table \ref{tab:heuristics} because they are commonly used as benchmarks or have been reported as superior to other heuristics.


\begin{table}[h]
\begin{tabular}{l|l|l}
Heuristic                         & Acronym & Reference                \\ \hline
K-Shortest Paths First-Fit        & KSP-FF  & \cite{vincent_scalable_2019} \\
First-Fit K-Shortest Paths        & FF-KSP  & \cite{vincent_scalable_2019} \\
K-Shortest Paths Best-Fit         & KSP-BF  & \cite{abkenar_best_2016} \\
Best-Fit K-Shortest Paths         & BF-KSP  & \cite{abkenar_best_2016} \\
K-Minimum Entropy First-Fit       & KME-FF  & \cite{wright_minimum-_2015} \\
%K-Minimum Cut First-Fit           & KMC-FF  & \cite{tang_heuristic_2022} \\
%K-Minimum Frag First-Fit & KMF-FF  & \cite{tang_heuristic_2022} \\
K-Congestion Aware First-Fit      & KCA-FF  & CA2 from \cite{savory_congestion_2014}
\end{tabular}
\caption{RMSA heuristics used for benchmarking.}
\label{tab:heuristics}
\end{table}

%We also evaluated the K-Minimum Cut First-Fit (KMC-FF) and K-Minimum Frag First-Fit (KMF-FF) heuristics from Reward-RMSA \cite{tang_heuristic_2022}. These heuristics produce higher blocking probability than others when paths are ordered by number of hops. We exclude these results to improve the clarity of Figures \ref{fig:heur_comp}, \ref{fig:k_traffic}, and \ref{fig:heur_traffic}.

\begin{figure}
  \includegraphics[width=1.01\linewidth]{IMAGES/networks_plots_short.png}
  \caption{Network topologies used in our case studies from: DeepRMSA, Reward-RMSA, GCN-RMSA, MaskRSA, PtrNet-RSA \cite{chen_deeprmsa_2019} \cite{tang_heuristic_2022} \cite{xu_deep_2022} \cite{shimoda_mask_2021} \cite{cheng_ptrnet-rsa_2024}. We note that the USNET topology differs between GCN-RMSA and PtrNet-RSA. We show the GCN-RMSA version here. PtrNet-RSA also uses a variant of the COST239 topology.} %The COST239 topology shown is from DeepRMSA, the USNET is from PtrNet-RSA.}
  \label{fig:network_plots}
\end{figure}

Figure \ref{fig:network_plots} shows the topologies used in the selected papers. We use these topologies to analyze the performance of the heuristic algorithms and in our reproductions of the papers' problems in section \ref{sec:repro_main}. %Some characteristic features of the topologies are summarised in table \ref{tab:topology_features}. 
We make all topology data available in our open source codebase \cite{michael_doherty_2024_jocn_xlron_2024}.



\subsection{Effect of path ordering}

All the heuristics in Table \ref{tab:heuristics} select from the available pre-computed paths on the basis of sort criteria. The primary criterion may be a measure of the path congestion (KCA-FF), spectral fragmentation (KME-FF), or length (KSP-FF). In the event of multiple paths with equal value, a default ordering (usually ascending order of length) determines the selected path.

Usually, path length is considered as distance in km. However, we find that considering path length as number of hops (with length in km as a secondary sort criterion), significantly improves the performance of the heuristics. This has been observed previously by Baroni \cite{baroni_routing_1998}, who referred to it as Minimum Number of Hops routing (MNH). The intuitive explanation for this is that, if two paths can support the same order of modulation format, the path that comprises fewer links occupies fewer spectral resources. 

We refer to these two orderings as path length in km (\#km) or path length in number of hops (\#hops). Our comparisons of KSP-FF for these two orderings in Section \ref{sec:repro} Figure \ref{fig:repro} evidence the reduction in blocking probability from \#hops ordering. In our comparisons of heuristics in the next section, we use \#hops ordering.



\subsection{Simulation setup}

For each heuristic and topology, we carried out three simulation scenarios to investigate the effects of varying traffic loads and values of K on the relative blocking performance of the heuristics. 

\vspace{0.1cm}

\noindent \textbf{Experiment 1} - Increasing K:
\par \noindent \textbf{Aim}: Investigate relative performance of heuristics with increasing K. \textbf{Method}: Record service blocking probability (SBP) for each heuristic at values of K ranging from 2 to 26 at fixed traffic load. We arbitrarily select the traffic load for each topology so that the heuristics give a SBP of  $\sim$1\%.

\vspace{0.1cm}

\noindent \textbf{Experiment 2} - Increasing K at high to low traffic: 
\par \noindent \textbf{Aim}: Investigate the effect of increasing K at different traffic loads. \textbf{Method}: Record SBP at K ranging from 2 to 40 for a range of traffic loads. We select the traffic loads for each topology such that they result in 10$^{-5}$ to 10$^{-1}$ SBP. To simplify the analysis and plots, we only present results for KSP-FF.

\vspace{0.1cm}

\noindent \textbf{Experiment 3} - Increasing traffic load at K=50: 
\par \noindent \textbf{Aim}: Using the findings from Experiments 1 and 2, determine the lowest-blocking heuristic with optimized K-value across traffic loads. \textbf{Method}: Record SBP for high K (K=50) at varying traffic loads. We select the traffic loads for each topology such that they result in a range of SBP (10$^{-5}$ to 10$^{-1}$). This experiment provides evidence on which heuristic is the best overall for each topology.

\vspace{0.1cm}

The data for each heuristic in each experiment was collected from 3000 independent trials with unique random seeds. The SBP was calculated after 10,000 connection requests, with the mean and standard deviation calculated across trials. Each data point in Figures \ref{fig:heur_comp},\ref{fig:k_traffic},\ref{fig:heur_traffic} therefore shows summary statistics from 30 million connection requests, which gives high confidence that our results present the true mean and standard deviation in each case.

We considered dynamic traffic with fixed mean service holding time at 10 units. We considered the same traffic model and other settings as DeepRMSA\footnotemark: uniform traffic probability between each node pair, Poissonian arrival and departure statistics, uniform random selection of data rate from 25 to 100Gbps in 1Gbps intervals, and distance-dependent modulation formats from BPSK to 16QAM. We consider topologies with dual fibre links (one for each direction of propagation), 12.5GHz FSU width, and 100 FSU per fibre.

\footnotetext{We consider the DeepRMSA problem settings in these experiments because it is used by most of the papers presented in Section \ref{sec:repro_main}.} %For the same reason, we report our results in terms of SBP, despite our recommendation to use BBP in Section \ref{sec:recommendations}.}

\subsection{Results and discussion}

\textbf{Experiment 1} results in Figure \ref{fig:heur_comp} show different outcomes for smaller networks (NSFNET and COST239) and larger networks (USNET and JPN48). For NSFNET and COST239, KSP-FF and KME-FF are approximately equal and give the lowest blocking. Their blocking decreases to a minimum for approximately K=23 and above for NSFNET and continues to decrease for K>26 for COST239.

For USNET and JPN48, FF-KSP is clearly the best heuristic, with blocking reduced by half for JPN48. Blocking from FF-KSP decreases with K until K=26 for USNET and continues dropping sharply for K>26 for JPN48. For USNET, KSP-FF and KME-FF become competitive with FF-KSP at large K. It can be argued that FF-KSP performs better in larger networks where there are multiple roughly equivalent paths between source and destination, and dense packing of utilized wavelengths increases in relative importance to path selection.

The results from Experiment 1 indicate that KSP-FF and FF-KSP generally give the lowest blocking, depending on the network topology, and blocking decreases monotonically with increasing K. This experiment looked at a moderately high traffic load ($\sim$1\% SBP), therefore experiment 2 investigates if the effect of increasing K holds at different traffic loads. 


\begin{figure*}[p]
  \includegraphics[width=\textwidth]{IMAGES/heuristic_comparison.png}
  \caption{Service blocking probability (SBP) for heuristics at fixed traffic and varying numbers of candidate paths (K). The mean and standard deviation (shaded area) are calculated from 3000 trials of 10,000 traffic requests per data point. Increasing K decreases the SBP for most heuristics. KSP-FF and FF-KSP give lowest SBP, depending on topology.}
  \label{fig:heur_comp}

  \vspace{1em}
  \includegraphics[width=0.97\textwidth]{IMAGES/k_traffic_comparison.png}
  \caption{Service blocking probability (SBP) for KSP-FF at varying traffic loads and K values. The mean and standard deviation (shaded area) are calculated from 3000 trials of 10,000 traffic requests per data point. For all traffic loads, increasing K decreases SBP for KSP-FF. Improvements saturate at high K.}
  \label{fig:k_traffic}

  \vspace{1em}
  \includegraphics[width=\textwidth]{IMAGES/heuristic_50_traffic_comparison.png}
  \caption{Service blocking probability (SBP) for heuristics at varying traffic load for K=50. The mean and standard deviation (shaded area) are calculated from 3000 trials of 10,000 traffic requests per data point. KSP-FF is the best heuristic for NSFNET and COST239. FF-KSP is best for USNET and JPN48.}
  \label{fig:heur_traffic}
\end{figure*}

\textbf{Experiment 2} results in Figure \ref{fig:k_traffic} show that, regardless of the traffic load, increasing K decreases the SBP, until SBP reaches a minimum and increasing K does not decrease SBP further. Across all topologies and traffic loads tested in our experiments, we found that SBP does not decrease significantly for K>50. For very high traffic (approximately equivalent to incremental loading), the value of K beyond which SBP does not continue to decrease can be much lower.

\textbf{Experiment 3} results in Figure \ref{fig:heur_traffic} show the variation of SBP with traffic load for each heuristic with K=50. We verified that at least 50 unique paths are possible for every node pair on our investigated topologies. We select K=50 on the basis of experiments 1 and 2. These results confirm the initial findings from Experiment 1 - that KSP-FF and KME-FF are the lowest blocking for NSFNET and COST239\footnotemark, while FF-KSP is better for USNET and JPN48, with an order of magnitude lower blocking probability on JPN48 compared to the next best heuristic.

\footnotetext{Although Figure \ref{fig:heur_traffic} shows KME-FF gives slightly lower blocking than KSP-FF at lower traffic, we prefer KSP-FF for benchmarking purposes because of its widespread use and its greater simplicity.}

% \begin{figure*}
%   \includegraphics[width=\textwidth]{IMAGES/heuristic_50_traffic_comparison.png}
%   \caption{Comparison of service blocking probability from heuristic algorithms at varying traffic load for K=50.}
%   \label{fig:heur_traffic}
% \end{figure*}


In summary, we highlight the generally strong performance of the KSP-FF and FF-KSP heuristics but do not draw conclusions as to which is best in general, and encourage thorough analysis to determine the strongest heuristic benchmark for a particular problem, as we have exemplified here. We also emphasize the importance of selecting optimal path ordering for heuristics, as this can have a significant impact on performance. We find \#hops ordering is superior to \#km for reduced blocking probability, as evidenced in Section \ref{sec:repro_main} Figure \ref{fig:repro}.



\section{Reproduction and benchmarking of previous work}
\label{sec:repro_main}

As discussed in our literature review (Section \ref{sec:survey}), it is difficult to assess progress in the field due to several factors, particularly the diversity of problem definitions and use of weak benchmarks. To address this, we exactly recreate the problem settings from five influential papers from the literature, and apply the best-performing heuristics from Section \ref{sec:heuristic_comparison} in each case.

In this section, we first provide analysis of holding time truncation, an implementation detail present in the DeepRMSA codebase that significantly affects the blocking probability. We then present the results of our reproductions of the selected papers and compare to the heuristics, which show superior performance to all of the published RL solutions.

We have corresponded with the authors of all of the selected papers to clarify details of their implementation and ensure that our reproductions exactly match all the relevant details of their problems.

%DeepRMSA, Reward-RMSA, and GCN-RMSA use the same original DeepRMSA codebase as the basis for their experiments. Consequently, they all feature service holding time truncation, discussed in section \ref{sec:holding_time}. Another common characteristic is they use directed graphs with dual fiber links i.e. each link comprises two fibers, one for each direction of propagation. This doubles the maximum potential capacity of the network compared to single-fiber links. MaskRSA and PtrNet-RSA model single-fiber links, which makes the problem less challenging for RL due to less sparse reward signals \cite{nevin_techniques_2022} (more frequent blocking). PtrNet-RSA considers fixed bandwidth requests (no distance-dependent modulation format).

 We use our high-performance simulation framework, \mbox{XLRON} \cite{doherty_xlron_2023}, for all experiments. It has demonstrated 10x faster execution on CPU and over 1000x faster when parallelized on GPU compared to optical-rl-gym \cite{doherty_xlron_2024}. This is possible due to its array-based data model and use of the JAX numerical array computing framework, that enables just-in-time compilation to accelerator hardware. It also offers a complete suite of unit tests for core functionality, making it reliable, and includes all the problem definitions of the selected papers. We use it for these reasons and for its simple command-line interface, which facilitates experiment automation and reproducibility.













% \subsection{Simulation warm-up}
% \label{sec:warmup}

% To exactly recreate the problems from the selected papers, we generate the same number of traffic requests as the original study. This includes the connection requests that are generated to pre-populate the network until the blocking probability reaches steady-state, a phenomenon which is well-known in discrete-event simulation \cite{banks_discrete-event_2005}. We call the period prior to steady-state the warm-up period. 

% Warm-up is important so that the blocking probability reaches its steady-state value and is not an underestimate. This is known as the simulation start-up problem or the initial transient \cite{white_problem_2009}. %When training an RL agent, warm-up may help prevent primacy bias \cite{nikishin_primacy_2022}, where models overfit to data seen early in training. Starting with an empty network, initial traffic requests are trivially allocated due to abundant resources, providing no useful optimization signal. Training on a congested network with both successful and failed allocations gives more meaningful training data from the start.

% While most papers do not explicitly discuss warm-up, all of the selected papers use it. We therefore analyze the minimum required warm-up period for different traffic levels.

% \subsubsection{Experiment setup}

% We simulate a non-blocking network to determine the maximum\footnotemark number of requests needed to reach a steady-state traffic load. For each traffic load from 50 to 1000 Erlangs in steps of 50, and for arrival rates from 5 to 25 in steps of 5, we conduct 500 simulation trials with unique random seeds.  For each trial, we estimate the number of requests at which steady state is reached using the MSER-5 method \cite{franklin_stationarity_2008} (Marginal Standard Error Rule with batch size 5). We apply the MSER-5 method as it has been shown to be superior to other methods such as the mean crossing rule \cite{white_problem_2009}.

% \footnotetext{A blocking network reaches steady-state in less traffic requests than a non-blocking one, therefore we consider a non-blocking network to estimate an upper bound on the required warm-up period before steady state.}


% \subsubsection{Results and discussion}

% \begin{figure}
%     \centering
%     \includegraphics[width=1\linewidth]{IMAGES/steady_state_boxplots.png}
%     \caption{No. of simulated requests necessary to reach the steady-state network traffic load specified on the x-axis, for a non-blocking network.}
%     \label{fig:traffic_warmup}
% \end{figure}

% Figure \ref{fig:traffic_warmup} displays results as box-and-whisker plots showing how many requests are needed to reach steady state for each traffic load. The whiskers extend to the most extreme data points within 1.5 times the inter-quartile range. We observe a linear relationship between traffic load and requests until steady state (warm-up period). We fit a line to the maximum values (tops of the whiskers) to determine the warm-up period with high confidence. The fitted line indicates that approximately 7 times the target traffic load in requests are required to reach steady state. Based on this finding, the 3000-request warm-up period used by DeepRMSA that we adopt in section \ref{sec:repro} is sufficient for our problem settings.



















\subsection{Holding time truncation}
\label{sec:holding_time}

DeepRMSA, Reward-RMSA, and GCN-RMSA use the same original DeepRMSA codebase as the basis for their experiments. This codebase includes a significant detail: the service holding time is resampled if the resulting value is more than twice the mean of the exponential probability density function (PDF). We refer to this detail as holding time truncation.  In order to recreate the problems from these papers, we analyze the effect of holding time truncation.


\subsubsection{Experiment setup}
To understand the effect of truncation on the traffic statistics, we define an inverse exponential PDF that is normalized to have unit mean. We take $10^{6}$ samples from the PDF, both with and without truncation, and calculate the mean of the resulting sample populations in both cases.

\subsubsection{Results and discussion}
Figure \ref{fig:truncation} compares histograms of service holding times with and without truncation. We define the bin width as 0.01 and normalize the count per bin to give a peak density of 1 without truncation. The truncated case shows a cutoff at twice the mean holding time. The vertical lines indicate the mean holding time for each case.

Holding time truncation reduces the mean by approximately 31\%. This results in 31\% lower traffic load. Therefore, papers that use the DeepRMSA codebase (including DeepRMSA, Reward-RMSA, and GCN-RMSA) evaluate their solutions at traffic loads 31\% lower than reported. This detail is not made explicit in the published papers. This finding highlights the challenges in making fair comparisons between papers, and the need for transparency in research code.

\begin{figure}
    \centering
    \includegraphics[width=0.9\linewidth]{IMAGES/truncation.png}
    \caption{Histogram of service holding holding times. The truncated distribution resamples the holding time when the sampled value exceeds $2*$mean. This reduces the mean holding time by 31\% compared to the exponential distribution.}
    \label{fig:truncation}
\end{figure}



%Cisco NCS 2000 Flex Spectrum Single Module ROADM and it's integrated pre-amplifier has a nominal noise figure of 5.5dB (for 24dB gain) or 11.7dB (for 12dB gain), 















\subsection{Reproduction of published results}
\label{sec:repro}
Our analysis of the best performing heuristics, of holding time truncation, and our correspondence with the authors enables us to benchmark the published results from the five selected influential papers. The aim of this comparison is to determine if any of the published RL solutions achieve lower SBP than the heuristics. %In order to do so, we recreate their simulations exactly and apply the KSP-FF heuristic algorithm with K=50 and paths ordered by number of hops in each case, based on our experiments from section \ref{sec:heuristic_comparison}.


\subsubsection{Experiment methodology}


We recreate the problems from each selected paper in our own simulation framework \cite{doherty_xlron_2024}. We match the topologies (NSFNET, COST239, JPN48, USNET), mean service arrival rates, mean service holding times, data-rate or bandwidth request distributions, and uniform traffic matrices. We use the same measurement methodology as described in the respective papers to reproduce results, which is 3000 request warm-up period (to allow the network blocking probability to reach steady-state after the 'initial transient' \cite{white_problem_2009}) followed by 10,000 requests. The SBP is calculated at the end of the episode. We run 10 independent episodes at each traffic load per problem and calculate the mean and standard deviation across trials.

We extract published results for KSP-FF and RL solutions from the papers, using textual values where available otherwise reading from charts. All published results report only a single data point for each traffic value, without any uncertainty. %We provide all of the extracted data and our results in Appendix \ref{appendix:A}.

We check that our results for KSP-FF with K=5 (green line in Figure \ref{fig:repro}) match the published results for KSP-FF (blue line in Figure \ref{fig:repro}) within two standard deviations to ensure faithful reproduction. This comparison gives us a high degree of confidence that we have exactly recreated each problem setting.



\subsubsection{Results and discussion}


\begin{figure*}[ht]
  \includegraphics[width=1.01\textwidth]{IMAGES/review_summary_plots_2.png}
  \caption{Mean SBP against traffic load. Each column is a publication and each subplot is for a topology. Error bars and shaded areas show standard deviations. %Data for $RL$ and 5-SP-FF$_{published}$ are extracted from the publications. Close agreement between 5-SP-FF$_{published}$ and  5-SP-FF$_{ours}$ show that we accurately reproduce each case of study.
  \mbox{50-SP-FF$_{hops}$} exceeds or matches the $RL$ performance for each case.}
  \label{fig:repro}
\end{figure*}

Figure \ref{fig:repro} shows our reproduction of results from the selected papers, with SBP against traffic load in Erlangs in each subplot. The plots are organized by paper (columns) and topology (rows). PtrNet-RSA has two columns reflecting its two test cases: networks with 40 FSU per link and 1 FSU requests, and networks with 80 FSU per link and 1-4 FSU requests. PtrNet-RSA only considers fixed-bandwidth requests (no distance-dependent modulation format).

Each plot contains 5 datasets:
\begin{itemize}[itemsep=0pt]
\item[] \textbf{RL}: Published results for the RL approach
\item[] \textbf{5-SP-FF$_{published}$}: Published results for KSP-FF (K=5) with paths ordered by \#km
\item[] \textbf{5-SP-FF$_{ours}$}: Our results for KSP-FF (K=5) with paths ordered by \#km
\item[] \textbf{5-SP-FF$_{hops}$}: Our results for KSP-FF (K=5) with paths ordered by \#hops
\item[] \textbf{50-SP-FF$_{hops}$}: Our results for KSP-FF (K=50) with paths ordered by \#hops
\end{itemize}

Points show mean values from our simulations, with shaded areas indicating standard deviation and lines interpolating between points. The DeepRMSA paper provides data for only one traffic load per topology. The excellent agreement between 5-SP-FF$_{published}$ and 5-SP-FF$_{ours}$ in all cases confirms that our framework accurately reproduces the published scenarios. %We can therefore make comparisons between our results on these problem settings and the published RL results with high confidence.


From Figure \ref{fig:repro}, we highlight the comparisons of 'RL' (red) with 5-SP-FF$_{hops}$ (orange), and 50-SP-FF$_{hops}$ (purple). 5-SP-FF$_{hops}$ reduces the blocking probability by up to an order of magnitude compared to RL in all cases for NSFNET, 4/5 cases for COST239 and 1/3 cases for USNET. This shows that ordering paths by \#hops is sufficient to beat the RL results in these cases.

For larger topologies, considering more candidate paths (K>5) improves the heuristic performance significantly, often by over an order of magnitude. As shown by Figure \ref{fig:repro}, 50-SP-FF$_{hops}$ gives the lowest SBP of all approaches in all cases, except the bottom right.

The single exception where RL outperforms 50-SP-FF$_{hops}$ is PtrNet-RSA-80 USNET. We consider it plausible that the pointer-net architecture is a contributing factor to this strong performance, as it is not limited to selecting from a pre-defined set of paths. However, as the published results in this case fall within one standard deviations of the mean for 50-SP-FF$_{hops}$, the result could be spurious. This highlights the need for summary statistics and confidence intervals from multiple trials to be included with published results.

In summary, the results show that making minor changes (ordering paths by \#hops and considering more paths)  to simple heuristic algorithms is sufficient to achieve lower blocking probability than sophisticated RL solutions that have been published.

We highlight that this analysis, and the selected papers, focus on SBP as the optimization objective. In realistic scenarios, network blocking or throughput must be balanced with other metrics such as latency and cost of operation from transceiver launch power, amplifiers, and other network elements. Future research should therefore focus on problems that take a holistic approach to network operations optimization with multiple objectives \cite{nallaperuma_interpreting_2023}, and incorporate sophisticated models of all physical layer effects for improved accuracy \cite{curri_gnpy_2022,buglia_closed-form_2023}.

%While our earlier experiments showed FF-KSP performs better than KSP-FF for JPN48, we only plot KSP-FF results as they already surpass the RL performance. For MaskRSA JPN48, FF-KSP with K=50 achieves zero blocking across all tested traffic loads.

% Could include a summary paragrph here and/or a note about how RL could do better but it needs to consider enough candidate paths and the set of candidate paths needs to prioritise those with less hops not just shorter distance.
%We note that it is theoretically possible for RL to equal or exceed the performance of the best heuristic algorithms. However, our results suggest that it is necessary to consider sufficiently diverse paths 

All data shown in Figure \ref{fig:repro} is provided in tabular form in Appendix A.



\section{Network blocking bounds}
\label{sec:bounds}

We have demonstrated in Section \ref{sec:repro_main} that many influential works on RL for DRA problems in optical networks have failed to improve on a simple heuristic algorithm. The extent to which it is possible to reduce the blocking probability, and increase supported traffic, is an important motivating factor in any future research into this topic. 

To understand the limits of blocking probability, we derive empirical lower bounds. By comparing these lower bounds to the performance of our best solution for a target SBP, we can estimate the additional traffic load that can be supported and, therefore, the maximum benefit from applying an intelligent resource allocation method such as RL. %We term this additional capacity the "optimality gap", as we consider our lower bound to be approximately optimal.

As discussed in section \ref{sec:background}, DRA problems in optical networks that require RSA are subject to at least three constraints: spectrum continuity, spectrum contiguity, and no reconfiguration. By relaxing any of these constraints, the optimal or near-optimal solution of the relaxed problem is a bound on the solution of the full problem. The cut-sets bound method of Cruzado et al \cite{cruzado_effective_2023,cruzado_capacity-bound_2024} relaxes the spectrum continuity constraint and uses insights from the min-cut max-flow theorem to estimate a lower bound SBP. We instead relax the constraint on reconfiguring already-established connections, a process known as defragmentation.

We couple this defragmentation with resource prioritization: sorting the active connection requests by their required resources and allocating them sequentially. The sorting of active requests in descending order of required resources was found to improve the achievable capacity to optimal or near-optimal by Baroni \cite{baroni_routing_1998} in static RWA and later Beghelli \cite{beghelli_resource_2006} for dynamic RWA, a method they refer to as 'reconfigurable routing'. Since our problem settings are elastic optical networks, we prefer the term defragmentation. The intuition behind this approach is to allocate larger requests first so that requests with shorter paths and lower spectral requirements may be squeezed into remaining spectral gaps later.







\subsection*{Resource-Prioritized Defragmentation}
\label{sec:bounds}


\begin{algorithm}
\caption{Resource-Prioritized Defragmentation Blocking Bound Estimation}
\begin{algorithmic}[1]
\Require Network topology $G$, Set of requests $\mathcal{R}$, Number of frequency slots per link $F$
\Ensure Blocking probability $P_b$
\State $N \gets \textsc{InitializeNetwork}(G, F)$ \Comment{Initialize network with empty spectrum slots}
\State $\textit{blocked} \gets 0$
\For{$t \gets 1$ to $|\mathcal{R}|$}
    \State $N \gets \textsc{RemoveExpiredRequests}(N, t)$
    \State $r_t \gets \textit{current request from } \mathcal{R}$
    \State $\textit{success} \gets \textsc{AllocateRequest}(N, r_t)$
    
    \If{not $\textit{success}$}
        \State $\textit{active\_requests} \gets \textsc{GetActiveRequests}(\mathcal{R}, t)$
        \State $\textit{sorted\_requests} \gets \textsc{SortByResource}(\textit{active\_requests})$
        \State $N_{temp} \gets \textsc{InitializeNetwork}(G, F)$
        \State $\textit{blocking} \gets \texttt{false}$
        
        \For{$r \in \textit{sorted\_requests}$}
            \State $\textit{success} \gets \textsc{AllocateRequest}(N_{temp}, r)$
            \If{not $\textit{success}$}
                \State $\textit{blocking} \gets \texttt{true}$
                \State \textbf{break}
            \EndIf
        \EndFor
        
        \If{not $\textit{blocking}$}
            \State $N \gets N_{temp}$
        \Else
            \State $\textit{blocked} \gets \textit{blocked} + 1$
        \EndIf
    \EndIf
\EndFor

\State \Return $\frac{\textit{blocked}}{|\mathcal{R}|}$
\end{algorithmic}
\label{algo:defrag}
\end{algorithm}

The resource-prioritized defragmentation blocking bound algorithm is outlined in Algorithm \ref{tab:blocking_probabilities}. It utilizes four key subroutines:

\begin{itemize}
    \item \textsc{RemoveExpiredRequests}($N$, $t$) maintains network state by removing connections that have expeired. For current time $t$, and request with arrival time $t_{\text{arrival}}$ and holding time $t_{\text{holding}}$, the expiry condition is defined as: $t_{\text{arrival}} + t_{\text{holding}} < t$.
    
    \item \textsc{AllocateRequest}($N$, $request$) establishes a new connection subject to continuity and contiguity constraints. We use the KSP-FF or FF-KSP algorithm with K=50. We select the algorithm that produces the lowest SBP for the problem instance.
    
    \item \textsc{GetActiveRequests}($\mathcal{R}$, $t$) identifies requests where $t_{\text{arrival}} \leq t < t_{\text{arrival}} + t_{\text{holding}}$, determining which connections require reallocation during defragmentation.
    
    \item \textsc{SortByResource}($requests$) orders active requests by required resources (product of required spectral slots and hops of shortest path), prioritizing larger requests during reallocation to maximize the probability of finding viable configurations.
    
\end{itemize}

A shortcoming of our method of blocking bound estimation is its reliance on the internal \textsc{AllocateRequest} heuristic. To have confidence that the solution presents a true upper bound, the allocation method must be as close to optimal as possible. We therefore evaluate multiple heuristics for each case, as shown in Section \ref{sec:heuristic_comparison}, and select the one with lowest SBP. We find the best performing heuristic is KSP-FF$_{hops}$ with K=50 for most cases, except MaskRSA JPN48 which is FF-KSP.


%Combined with resource prioritization (\textsc{SortByResource}) we assume the results are near-optimal, based on results from Baroni \cite{baroni_routing_1998}. 
An advantage of our method compared to cut-sets analysis is it computes an allocation that is guaranteed to be physically possible, as it relaxes the 'No Reconfiguration' constraint instead of the physical spectrum continuity constraint. Relaxing the 'No Reconfiguration' constraint makes Algorithm \ref{algo:defrag} omniscient (it has complete knowledge of requests to be allocated) rather than a strictly on-line algorithm, according to definitions from Awerbuch et al \cite{awerbuch_throughput-competitive_1993}. This gives Algorithm \ref{algo:defrag} a fundamental competitive advantage over on-line algorithms like KSP-FF/FF-KSP, therefore it can be considered a lower bound estimator of blocking probability. % This disparity in information may however mean it is too much of an upper-bound estimate
%Overall, resource-prioritized defragmentation bounds can be considered complimentary to cut-sets bounds due to the difference in constraint relaxation.

We note that our algorithm is general and can be applied to any DRA problem in optical networks by using a strong heuristic for \textsc{AllocateRequest} and defining the resource-based sort criteria appropriately.






\subsection{Experiment setup}

For each problem from the five selected papers, we run the best performing heuristic for a range of traffic loads that result in SBP from 0.01\% to 1\%. For the lowest-blocking heuristic and for Algorithm \ref{algo:defrag}, we run 10 episodes of 10,000 requests with unique random seeds and calculate the mean and standard deviation of SBP across episodes. We calculate the mean and standard deviation SBP across episodes in each case. 

We compare the resulting SBP from the best heuristic and from algorithm \ref{algo:defrag}. We seek to estimate the additional network capacity that can be achieved at 0.1\% SBP for each case of study from the five selected papers. We select 0.1\% SBP to align with previous studies of network throughput estimation by Cruzado et al \cite{cruzado_effective_2023,cruzado_capacity-bound_2024}.

\subsection{Results and discussion}


\begin{figure*}[ht]
  \includegraphics[width=1.01\textwidth]{IMAGES/bounds.png}
  \caption{Mean SBP against traffic load for the lowest-blocking heuristic in each case (KSP-FF or FF-KSP with K=50) and the estimated bound from Algorithm \ref{algo:defrag}. Each column is a publication and each subplot is for a topology. Shaded areas show standard deviations. Red lines and text indicate relative increase in supported traffic at 0.1\% SBP from heuristic to bound.}
  \label{fig:bounds}
\end{figure*}


Similar to Figure \ref{fig:repro}, each subplot in Figure \ref{fig:repro} represents a different problem instance. DeepRMSA, Reward-RMSA, and GCN-RMSA are combined into a single set of plots since they use identical topologies and traffic models. The purple lines show the best performing heuristic in each case (KSP-FF with K=50, or FF-KSP for JPN48), with paths sorted in ascending order of number of hops. The grey lines show the resource-prioritized defragmentation bounds. At 0.1\% SBP, we compare the network traffic loads that can be supported in each case, with the difference highlighted by a red horizontal line. The relative increase in network capacity is calculated as the difference between the upper bound traffic load and the heuristic traffic load, as a percentage of the heuristic load.

PtrNet-RSA-40 shows differences of 5\%, 1\%, and 9\% across its three test cases. These relatively low values are due to the fixed width requests size of 1 FSU used in this case, which makes it equivalent to RWA and reduces the impact of fragmentation compared to RSA/RMSA.

For the Deep/Reward/GCN-RMSA, MaskRSA and PtrNet-RSA-80 cases, the difference between the supported traffic in the heuristic case and the upper bound ranges from 19\% (MaskRSA JPN48) to 36\% (Deep/Reward/GCN-RMSA NSFNET). These results show larger but comparable optimality gaps to those from the cut-sets method of Cruzado et al. \cite{cruzado_capacity-bound_2024}, who found gaps of 5\% to 16\% in their cases of study. This shows that defragmentation can unlock significant network capacity, but it is unknown theoretically how close an intelligent online allocation method, such as RL, can come to this bound. This will be the subject of future research.


\section{Discussion}\label{sec:discussion}



\subsection{From Interactive Prompting to Interactive Multi-modal Prompting}
The rapid advancements of large pre-trained generative models including large language models and text-to-image generation models, have inspired many HCI researchers to develop interactive tools to support users in crafting appropriate prompts.
% Studies on this topic in last two years' HCI conferences are predominantly focused on helping users refine single-modality textual prompts.
Many previous studies are focused on helping users refine single-modality textual prompts.
However, for many real-world applications concerning data beyond text modality, such as multi-modal AI and embodied intelligence, information from other modalities is essential in constructing sophisticated multi-modal prompts that fully convey users' instruction.
This demand inspires some researchers to develop multimodal prompting interactions to facilitate generation tasks ranging from visual modality image generation~\cite{wang2024promptcharm, promptpaint} to textual modality story generation~\cite{chung2022tale}.
% Some previous studies contributed relevant findings on this topic. 
Specifically, for the image generation task, recent studies have contributed some relevant findings on multi-modal prompting.
For example, PromptCharm~\cite{wang2024promptcharm} discovers the importance of multimodal feedback in refining initial text-based prompting in diffusion models.
However, the multi-modal interactions in PromptCharm are mainly focused on the feedback empowered the inpainting function, instead of supporting initial multimodal sketch-prompt control. 

\begin{figure*}[t]
    \centering
    \includegraphics[width=0.9\textwidth]{src/img/novice_expert.pdf}
    \vspace{-2mm}
    \caption{The comparison between novice and expert participants in painting reveals that experts produce more accurate and fine-grained sketches, resulting in closer alignment with reference images in close-ended tasks. Conversely, in open-ended tasks, expert fine-grained strokes fail to generate precise results due to \tool's lack of control at the thin stroke level.}
    \Description{The comparison between novice and expert participants in painting reveals that experts produce more accurate and fine-grained sketches, resulting in closer alignment with reference images in close-ended tasks. Novice users create rougher sketches with less accuracy in shape. Conversely, in open-ended tasks, expert fine-grained strokes fail to generate precise results due to \tool's lack of control at the thin stroke level, while novice users' broader strokes yield results more aligned with their sketches.}
    \label{fig:novice_expert}
    % \vspace{-3mm}
\end{figure*}


% In particular, in the initial control input, users are unable to explicitly specify multi-modal generation intents.
In another example, PromptPaint~\cite{promptpaint} stresses the importance of paint-medium-like interactions and introduces Prompt stencil functions that allow users to perform fine-grained controls with localized image generation. 
However, insufficient spatial control (\eg, PromptPaint only allows for single-object prompt stencil at a time) and unstable models can still leave some users feeling the uncertainty of AI and a varying degree of ownership of the generated artwork~\cite{promptpaint}.
% As a result, the gap between intuitive multi-modal or paint-medium-like control and the current prompting interface still exists, which requires further research on multi-modal prompting interactions.
From this perspective, our work seeks to further enhance multi-object spatial-semantic prompting control by users' natural sketching.
However, there are still some challenges to be resolved, such as consistent multi-object generation in multiple rounds to increase stability and improved understanding of user sketches.   


% \new{
% From this perspective, our work is a step forward in this direction by allowing multi-object spatial-semantic prompting control by users' natural sketching, which considers the interplay between multiple sketch regions.
% % To further advance the multi-modal prompting experience, there are some aspects we identify to be important.
% % One of the important aspects is enhancing the consistency and stability of multiple rounds of generation to reduce the uncertainty and loss of control on users' part.
% % For this purpose, we need to develop techniques to incorporate consistent generation~\cite{tewel2024training} into multi-modal prompting framework.}
% % Another important aspect is improving generative models' understanding of the implicit user intents \new{implied by the paint-medium-like or sketch-based input (\eg, sketch of two people with their hands slightly overlapping indicates holding hand without needing explicit prompt).
% % This can facilitate more natural control and alleviate users' effort in tuning the textual prompt.
% % In addition, it can increase users' sense of ownership as the generated results can be more aligned with their sketching intents.
% }
% For example, when users draw sketches of two people with their hands slightly overlapping, current region-based models cannot automatically infer users' implicit intention that the two people are holding hands.
% Instead, they still require users to explicitly specify in the prompt such relationship.
% \tool addresses this through sketch-aware prompt recommendation to fill in the necessary semantic information, alleviating users' workload.
% However, some users want the generative AI in the future to be able to directly infer this natural implicit intentions from the sketches without additional prompting since prompt recommendation can still be unstable sometimes.


% \new{
% Besides visual generation, 
% }
% For example, one of the important aspect is referring~\cite{he2024multi}, linking specific text semantics with specific spatial object, which is partly what we do in our sketch-aware prompt recommendation.
% Analogously, in natural communication between humans, text or audio alone often cannot suffice in expressing the speakers' intentions, and speakers often need to refer to an existing spatial object or draw out an illustration of her ideas for better explanation.
% Philosophically, we HCI researchers are mostly concerned about the human-end experience in human-AI communications.
% However, studies on prompting is unique in that we should not just care about the human-end interaction, but also make sure that AI can really get what the human means and produce intention-aligned output.
% Such consideration can drastically impact the design of prompting interactions in human-AI collaboration applications.
% On this note, although studies on multi-modal interactions is a well-established topic in HCI community, it remains a challenging problem what kind of multi-modal information is really effective in helping humans convey their ideas to current and next generation large AI models.




\subsection{Novice Performance vs. Expert Performance}\label{sec:nVe}
In this section we discuss the performance difference between novice and expert regarding experience in painting and prompting.
First, regarding painting skills, some participants with experience (4/12) preferred to draw accurate and fine-grained shapes at the beginning. 
All novice users (5/12) draw rough and less accurate shapes, while some participants with basic painting skills (3/12) also favored sketching rough areas of objects, as exemplified in Figure~\ref{fig:novice_expert}.
The experienced participants using fine-grained strokes (4/12, none of whom were experienced in prompting) achieved higher IoU scores (0.557) in the close-ended task (0.535) when using \tool. 
This is because their sketches were closer in shape and location to the reference, making the single object decomposition result more accurate.
Also, experienced participants are better at arranging spatial location and size of objects than novice participants.
However, some experienced participants (3/12) have mentioned that the fine-grained stroke sometimes makes them frustrated.
As P1's comment for his result in open-ended task: "\emph{It seems it cannot understand thin strokes; even if the shape is accurate, it can only generate content roughly around the area, especially when there is overlapping.}" 
This suggests that while \tool\ provides rough control to produce reasonably fine results from less accurate sketches for novice users, it may disappoint experienced users seeking more precise control through finer strokes. 
As shown in the last column in Figure~\ref{fig:novice_expert}, the dragon hovering in the sky was wrongly turned into a standing large dragon by \tool.

Second, regarding prompting skills, 3 out of 12 participants had one or more years of experience in T2I prompting. These participants used more modifiers than others during both T2I and R2I tasks.
Their performance in the T2I (0.335) and R2I (0.469) tasks showed higher scores than the average T2I (0.314) and R2I (0.418), but there was no performance improvement with \tool\ between their results (0.508) and the overall average score (0.528). 
This indicates that \tool\ can assist novice users in prompting, enabling them to produce satisfactory images similar to those created by users with prompting expertise.



\subsection{Applicability of \tool}
The feedback from user study highlighted several potential applications for our system. 
Three participants (P2, P6, P8) mentioned its possible use in commercial advertising design, emphasizing the importance of controllability for such work. 
They noted that the system's flexibility allows designers to quickly experiment with different settings.
Some participants (N = 3) also mentioned its potential for digital asset creation, particularly for game asset design. 
P7, a game mod developer, found the system highly useful for mod development. 
He explained: "\emph{Mods often require a series of images with a consistent theme and specific spatial requirements. 
For example, in a sacrifice scene, how the objects are arranged is closely tied to the mod's background. It would be difficult for a developer without professional skills, but with this system, it is possible to quickly construct such images}."
A few participants expressed similar thoughts regarding its use in scene construction, such as in film production. 
An interesting suggestion came from participant P4, who proposed its application in crime scene description. 
She pointed out that witnesses are often not skilled artists, and typically describe crime scenes verbally while someone else illustrates their account. 
With this system, witnesses could more easily express what they saw themselves, potentially producing depictions closer to the real events. "\emph{Details like object locations and distances from buildings can be easily conveyed using the system}," she added.

% \subsection{Model Understanding of Users' Implicit Intents}
% In region-sketch-based control of generative models, a significant gap between interaction design and actual implementation is the model's failure in understanding users' naturally expressed intentions.
% For example, when users draw sketches of two people with their hands slightly overlapping, current region-based models cannot automatically infer users' implicit intention that the two people are holding hands.
% Instead, they still require users to explicitly specify in the prompt such relationship.
% \tool addresses this through sketch-aware prompt recommendation to fill in the necessary semantic information, alleviating users' workload.
% However, some users want the generative AI in the future to be able to directly infer this natural implicit intentions from the sketches without additional prompting since prompt recommendation can still be unstable sometimes.
% This problem reflects a more general dilemma, which ubiquitously exists in all forms of conditioned control for generative models such as canny or scribble control.
% This is because all the control models are trained on pairs of explicit control signal and target image, which is lacking further interpretation or customization of the user intentions behind the seemingly straightforward input.
% For another example, the generative models cannot understand what abstraction level the user has in mind for her personal scribbles.
% Such problems leave more challenges to be addressed by future human-AI co-creation research.
% One possible direction is fine-tuning the conditioned models on individual user's conditioned control data to provide more customized interpretation. 

% \subsection{Balance between recommendation and autonomy}
% AIGC tools are a typical example of 
\subsection{Progressive Sketching}
Currently \tool is mainly aimed at novice users who are only capable of creating very rough sketches by themselves.
However, more accomplished painters or even professional artists typically have a coarse-to-fine creative process. 
Such a process is most evident in painting styles like traditional oil painting or digital impasto painting, where artists first quickly lay down large color patches to outline the most primitive proportion and structure of visual elements.
After that, the artists will progressively add layers of finer color strokes to the canvas to gradually refine the painting to an exquisite piece of artwork.
One participant in our user study (P1) , as a professional painter, has mentioned a similar point "\emph{
I think it is useful for laying out the big picture, give some inspirations for the initial drawing stage}."
Therefore, rough sketch also plays a part in the professional artists' creation process, yet it is more challenging to integrate AI into this more complex coarse-to-fine procedure.
Particularly, artists would like to preserve some of their finer strokes in later progression, not just the shape of the initial sketch.
In addition, instead of requiring the tool to generate a finished piece of artwork, some artists may prefer a model that can generate another more accurate sketch based on the initial one, and leave the final coloring and refining to the artists themselves.
To accommodate these diverse progressive sketching requirements, a more advanced sketch-based AI-assisted creation tool should be developed that can seamlessly enable artist intervention at any stage of the sketch and maximally preserve their creative intents to the finest level. 

\subsection{Ethical Issues}
Intellectual property and unethical misuse are two potential ethical concerns of AI-assisted creative tools, particularly those targeting novice users.
In terms of intellectual property, \tool hands over to novice users more control, giving them a higher sense of ownership of the creation.
However, the question still remains: how much contribution from the user's part constitutes full authorship of the artwork?
As \tool still relies on backbone generative models which may be trained on uncopyrighted data largely responsible for turning the sketch into finished artwork, we should design some mechanisms to circumvent this risk.
For example, we can allow artists to upload backbone models trained on their own artworks to integrate with our sketch control.
Regarding unethical misuse, \tool makes fine-grained spatial control more accessible to novice users, who may maliciously generate inappropriate content such as more realistic deepfake with specific postures they want or other explicit content.
To address this issue, we plan to incorporate a more sophisticated filtering mechanism that can detect and screen unethical content with more complex spatial-semantic conditions. 
% In the future, we plan to enable artists to upload their own style model

% \subsection{From interactive prompting to interactive spatial prompting}


\subsection{Limitations and Future work}

    \textbf{User Study Design}. Our open-ended task assesses the usability of \tool's system features in general use cases. To further examine aspects such as creativity and controllability across different methods, the open-ended task could be improved by incorporating baselines to provide more insightful comparative analysis. 
    Besides, in close-ended tasks, while the fixing order of tool usage prevents prior knowledge leakage, it might introduce learning effects. In our study, we include practice sessions for the three systems before the formal task to mitigate these effects. In the future, utilizing parallel tests (\textit{e.g.} different content with the same difficulty) or adding a control group could further reduce the learning effects.

    \textbf{Failure Cases}. There are certain failure cases with \tool that can limit its usability. 
    Firstly, when there are three or more objects with similar semantics, objects may still be missing despite prompt recommendations. 
    Secondly, if an object's stroke is thin, \tool may incorrectly interpret it as a full area, as demonstrated in the expert results of the open-ended task in Figure~\ref{fig:novice_expert}. 
    Finally, sometimes inclusion relationships (\textit{e.g.} inside) between objects cannot be generated correctly, partially due to biases in the base model that lack training samples with such relationship. 

    \textbf{More support for single object adjustment}.
    Participants (N=4) suggested that additional control features should be introduced, beyond just adjusting size and location. They noted that when objects overlap, they cannot freely control which object appears on top or which should be covered, and overlapping areas are currently not allowed.
    They proposed adding features such as layer control and depth control within the single-object mask manipulation. Currently, the system assigns layers based on color order, but future versions should allow users to adjust the layer of each object freely, while considering weighted prompts for overlapping areas.

    \textbf{More customized generation ability}.
    Our current system is built around a single model $ColorfulXL-Lightning$, which limits its ability to fully support the diverse creative needs of users. Feedback from participants has indicated a strong desire for more flexibility in style and personalization, such as integrating fine-tuned models that cater to specific artistic styles or individual preferences. 
    This limitation restricts the ability to adapt to varied creative intents across different users and contexts.
    In future iterations, we plan to address this by embedding a model selection feature, allowing users to choose from a variety of pre-trained or custom fine-tuned models that better align with their stylistic preferences. 
    
    \textbf{Integrate other model functions}.
    Our current system is compatible with many existing tools, such as Promptist~\cite{hao2024optimizing} and Magic Prompt, allowing users to iteratively generate prompts for single objects. However, the integration of these functions is somewhat limited in scope, and users may benefit from a broader range of interactive options, especially for more complex generation tasks. Additionally, for multimodal large models, users can currently explore using affordable or open-source models like Qwen2-VL~\cite{qwen} and InternVL2-Llama3~\cite{llama}, which have demonstrated solid inference performance in our tests. While GPT-4o remains a leading choice, alternative models also offer competitive results.
    Moving forward, we aim to integrate more multimodal large models into the system, giving users the flexibility to choose the models that best fit their needs. 
    


\section{Conclusion}\label{sec:conclusion}
In this paper, we present \tool, an interactive system designed to help novice users create high-quality, fine-grained images that align with their intentions based on rough sketches. 
The system first refines the user's initial prompt into a complete and coherent one that matches the rough sketch, ensuring the generated results are both stable, coherent and high quality.
To further support users in achieving fine-grained alignment between the generated image and their creative intent without requiring professional skills, we introduce a decompose-and-recompose strategy. 
This allows users to select desired, refined object shapes for individual decomposed objects and then recombine them, providing flexible mask manipulation for precise spatial control.
The framework operates through a coarse-to-fine process, enabling iterative and fine-grained control that is not possible with traditional end-to-end generation methods. 
Our user study demonstrates that \tool offers novice users enhanced flexibility in control and fine-grained alignment between their intentions and the generated images.



\section*{Acknowledgments}
This work was supported by the Engineering and Physical Sciences Research Council (EPSRC) grant EP/S022139/1 - the Centre for Doctoral Training in Connected Electronic and Photonic Systems - and EPSRC Programme Grant TRANSNET EP/R035342/1. In addition, Polina Bayvel is supported through a Royal Society Research Professorship.


% Bibliography
\bibliography{references.bib}





% \begin{table*}[!th]
%     \centering
%     \resizebox{0.9\textwidth}{!}{
%     \begin{tabular}{|l|r|r|r|l|r|r|r|} 
%        \cline{1-4} \cline{6-8}
%         & \multicolumn{3}{c|}{Retrieve} & &\multicolumn{3}{c|}{Retrieve + Rerank} \\ 
%                \cline{2-4} \cline{6-8}
%                &  \multicolumn{3}{c|}{Hits@10}   & & \multicolumn{2}{c|}{Hits@10}  & Speedup \\
% Dataset  &  Baseline  &  DE-2 Rand  & DE-2 CE  & & DE-2 CE &  Baseline &  \\
%        \cline{1-4} \cline{6-8}
% all nli \cite{bowman-etal-2015-large},\cite{N18-1101}  & 0.77 & 0.59 & 0.75 &  &\textbf{ 0.84} & 0.83 & 4.5x\\
% eli5 \cite{fan-etal-2019-eli5} & 0.43 & 0.12 & 0.29 &  & 0.43 & 0.49 & 5.4x \\
% gooaq \cite{Khashabi2021GooAQOQ} & 0.75 & 0.42 & 0.64 &  & 0.77 & 0.80 & 5.8x\\
% msmarco \cite{nguyen2016ms} & 0.95 & 0.81 & 0.89 &  & 0.96 & 0.98 & 5.1x \\
% \href{https://quoradata.quora.com/First-Quora-Dataset-Release-Question-Pairs}{quora duplicates}  & 0.68 & 0.48 & 0.65 &  &\textbf{ 0.68} & 0.68 & 5.1x\\
% natural ques. \cite{47761} & 0.77 & 0.37 & 0.61 &  & 0.61 & 0.64 & 5.4x \\
% sentence comp \cite{filippova-altun-2013-overcoming} & 0.95 & 0.83 & 0.93 &  & \textbf{0.97 }& 0.97 & 6.6x\\
% simplewiki \cite{coster-kauchak-2011-simple} & 0.97 & 0.93 & 0.97 &  &\textbf{ 0.97 }& 0.96 & 4.8x\\
% stsb \cite{cer-etal-2017-semeval} & 0.97 & 0.87 & 0.97 &  & \textbf{0.98} & 0.98 & 4.7x \\
% zeshel \cite{logeswaran2019zero}  & 0.22 & 0.16 & 0.20 &  & \textbf{0.21} & 0.19 & 4.8x\\
%        \cline{1-4} \cline{6-8}
%     \end{tabular}
%     }

% Please add the following required packages to your document preamble:
% \usepackage[table,xcdraw]{xcolor}
% Beamer presentation requires \usepackage{colortbl} instead of \usepackage[table,xcdraw]{xcolor}
\begin{table*}[!th]
\centering
     \resizebox{1\textwidth}{!}{ \begin{tabular}{|l|llllll|l|lllll|}
\cline{1-7} \cline{9-13}
\textbf{}                  & \multicolumn{6}{c|}{\textbf{Retrieve}}                                                                                                                                                                                         & \textbf{} & \multicolumn{5}{c|}{\textbf{Retrieve + Rerank}}                                                                                                                                    \\ 
\cline{2-7} \cline{9-13}
{\textbf{dataset}}           & \multicolumn{3}{c|}{\textbf{Hits@10}}                                                                                   & \multicolumn{3}{c|}{\textbf{MRR@10}}                                                                & \textbf{} & \multicolumn{2}{c|}{\textbf{Hits@10}}                                         & \multicolumn{2}{c|}{\textbf{MRR@10}}                                           & \textbf{Speedup} \\ 

                           
\cline{2-7} \cline{9-13} 
                           & \multicolumn{1}{l|}{\textbf{Baseline}} & \multicolumn{1}{l|}{\textbf{DE-2-Rand}} & \multicolumn{1}{l|}{\textbf{DE-2-CE}} & \multicolumn{1}{l|}{\textbf{Baseline}} & \multicolumn{1}{l|}{\textbf{DE-2-Rand}} & \textbf{DE-2-CE} & \textbf{} & \multicolumn{1}{l|}{\textbf{Baseline}} & \multicolumn{1}{l|}{\textbf{DE-2-CE}} & \multicolumn{1}{l|}{\textbf{Baseline}} & \multicolumn{1}{l|}{\textbf{DE-2-CE}} & \textbf{}        \\ 
                       \cline{1-7} \cline{9-13}
                       
msmarco/dev/small  & \multicolumn{1}{l|}{0.59}              & \multicolumn{1}{l|}{0.37}               & \multicolumn{1}{l|}{0.45}             & \multicolumn{1}{l|}{0.32}              & \multicolumn{1}{l|}{0.19}               & 0.24             &           & \multicolumn{1}{l|}{0.66}              & \multicolumn{1}{l|}{0.59}             & \multicolumn{1}{l|}{0.38}              & \multicolumn{1}{l|}{0.35}             & 5.1x             \\ \cline{1-7} \cline{9-13}
beir/quora/dev             & \multicolumn{1}{l|}{0.95}              & \multicolumn{1}{l|}{0.90}               & \multicolumn{1}{l|}{0.93}             & \multicolumn{1}{l|}{0.85}              & \multicolumn{1}{l|}{0.76}               & 0.81             &           & \multicolumn{1}{l|}{\textbf{0.96}}     & \multicolumn{1}{l|}{\textbf{0.96}}    & \multicolumn{1}{l|}{\textbf{0.74}}     & \multicolumn{1}{l|}{\textbf{0.74}}    & 5.3x             \\ \cline{1-7} \cline{9-13}
beir/scifact/test          & \multicolumn{1}{l|}{0.63}              & \multicolumn{1}{l|}{0.53}               & \multicolumn{1}{l|}{0.55}             & \multicolumn{1}{l|}{0.42}              & \multicolumn{1}{l|}{0.30}               & 0.36             &           & \multicolumn{1}{l|}{0.72}              & \multicolumn{1}{l|}{0.69}             & \multicolumn{1}{l|}{0.57}              & \multicolumn{1}{l|}{0.55}             & 4.7x             \\ \cline{1-7} \cline{9-13}
beir/fiqa/dev              & \multicolumn{1}{l|}{0.50}              & \multicolumn{1}{l|}{0.22}               & \multicolumn{1}{l|}{0.32}             & \multicolumn{1}{l|}{0.32}              & \multicolumn{1}{l|}{0.12}               & 0.18             &           & \multicolumn{1}{l|}{0.59}              & \multicolumn{1}{l|}{0.46}             & \multicolumn{1}{l|}{0.28}              & \multicolumn{1}{l|}{0.23}             & 5.3x             \\ \cline{1-7} \cline{9-13}
zeshel/test                & \multicolumn{1}{l|}{0.22}              & \multicolumn{1}{l|}{0.16}               & \multicolumn{1}{l|}{0.20}             & \multicolumn{1}{l|}{0.12}              & \multicolumn{1}{l|}{0.09}               & 0.11             &           & \multicolumn{1}{l|}{\textbf{0.20}}     & \multicolumn{1}{l|}{\textbf{0.19}}    & \multicolumn{1}{l|}{\textbf{0.11}}     & \multicolumn{1}{l|}{\textbf{0.10}}    & 4.8x             \\ \cline{1-7} \cline{9-13}
stsb/train                 & \multicolumn{1}{l|}{0.97}              & \multicolumn{1}{l|}{0.95}               & \multicolumn{1}{l|}{0.97}             & \multicolumn{1}{l|}{0.85}              & \multicolumn{1}{l|}{0.83}               & 0.85             &           & \multicolumn{1}{l|}{\textbf{0.98}}     & \multicolumn{1}{l|}{\textbf{0.98}}    & \multicolumn{1}{l|}{\textbf{0.88}}     & \multicolumn{1}{l|}{\textbf{0.88}}    & 4.7x             \\ \cline{1-7} \cline{9-13}
all-nli/train              & \multicolumn{1}{l|}{0.49}              & \multicolumn{1}{l|}{0.41}               & \multicolumn{1}{l|}{0.47}             & \multicolumn{1}{l|}{0.39}              & \multicolumn{1}{l|}{0.33}               & 0.36             &           & \multicolumn{1}{l|}{\textbf{0.55}}     & \multicolumn{1}{l|}{\textbf{0.54}}    & \multicolumn{1}{l|}{\textbf{0.47}}     & \multicolumn{1}{l|}{\textbf{0.46}}    & 4.5x             \\ \cline{1-7} \cline{9-13}
simplewiki/train           & \multicolumn{1}{l|}{0.97}              & \multicolumn{1}{l|}{0.98}               & \multicolumn{1}{l|}{0.98}             & \multicolumn{1}{l|}{0.92}              & \multicolumn{1}{l|}{0.93}               & 0.94             &           & \multicolumn{1}{l|}{\textbf{0.96}}     & \multicolumn{1}{l|}{\textbf{0.97}}    & \multicolumn{1}{l|}{\textbf{0.91}}     & \multicolumn{1}{l|}{\textbf{0.91}}    & 4.8x             \\ \cline{1-7} \cline{9-13}
natural-questions/train    & \multicolumn{1}{l|}{0.77}              & \multicolumn{1}{l|}{0.59}               & \multicolumn{1}{l|}{0.65}             & \multicolumn{1}{l|}{0.51}              & \multicolumn{1}{l|}{0.37}               & 0.42             &           & \multicolumn{1}{l|}{0.65}              & \multicolumn{1}{l|}{0.61}             & \multicolumn{1}{l|}{0.39}              & \multicolumn{1}{l|}{0.37}             & 5.4x             \\ \cline{1-7} \cline{9-13}
eli5/train                 & \multicolumn{1}{l|}{0.32}              & \multicolumn{1}{l|}{0.14}               & \multicolumn{1}{l|}{0.22}             & \multicolumn{1}{l|}{0.19}              & \multicolumn{1}{l|}{0.08}               & 0.12             &           & \multicolumn{1}{l|}{0.39}              & \multicolumn{1}{l|}{0.32}             & \multicolumn{1}{l|}{0.26}              & \multicolumn{1}{l|}{0.22}             & 5.4x             \\ \cline{1-7} \cline{9-13}
sentence compression/train & \multicolumn{1}{l|}{0.93}              & \multicolumn{1}{l|}{0.89}               & \multicolumn{1}{l|}{0.93}             & \multicolumn{1}{l|}{0.85}              & \multicolumn{1}{l|}{0.79}               & 0.84             &           & \multicolumn{1}{l|}{\textbf{0.96}}     & \multicolumn{1}{l|}{\textbf{0.96}}    & \multicolumn{1}{l|}{\textbf{0.93}}     & \multicolumn{1}{l|}{\textbf{0.94}}    & 6.6x             \\ \cline{1-7} \cline{9-13}
gooaq/train                & \multicolumn{1}{l|}{0.73}              & \multicolumn{1}{l|}{0.68}               & \multicolumn{1}{l|}{0.72}             & \multicolumn{1}{l|}{0.58}              & \multicolumn{1}{l|}{0.55}               & 0.58             &           & \multicolumn{1}{l|}{0.80}              & \multicolumn{1}{l|}{0.76}             & \multicolumn{1}{l|}{0.64}              & \multicolumn{1}{l|}{0.62}             & 5.2x             \\ \cline{1-7} \cline{9-13}
\end{tabular}
}
    \caption{Comparison of CE infused DE. \textbf{Baseline} is the pre-trained DE SBERT model.  \textbf{DE-2 Rand} is a trained two-layered DE initialized with random initial weights. \textbf{DE-2 CE} is a two-layered DE infused with initial weights from CE as explained in Fig~\ref{fig:dual_cross} and trained similar to DE-2. All the above models are trained on msmarco. Bold face numbers for DE-2 CE show where performance is at least within .01 of the baseline DE. \textbf{Speedup} is inference time gain for DE-2 CE over Baseline.}
    % \textbf{Retrieve + Rerank}: Columns 5-6 present Accuracy@10 and columns 7-8 present number of documents encoded per sec. Here, the documents retrieved using baseline and our approach are reranked using a CE.}
    \label{tab:comp}
\end{table*} 



\end{document}
