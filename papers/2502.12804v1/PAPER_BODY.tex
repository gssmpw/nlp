
\documentclass[9pt,twocolumn,twoside]{osajnl}

\journal{jocn} 

% Set the article type for journal submissions. Comment out this line for Optica Open preprint submissions.
\setboolean{shortarticle}{false}
% true = letter / tutorial
% false = research / review article

\title{Reinforcement Learning for Dynamic Resource Allocation in Optical Networks: Hype or Hope?}

\author[1*]{Michael Doherty}
\author[1]{Robin Matzner}
\author[1]{Rasoul Sadeghi}
\author[1]{Polina~Bayvel}
\author[1]{Alejandra~Beghelli}

\affil[1]{Optical Networks Group, University College London, Torrington Place, London WC1E 7JE, United Kingdom}


\affil[*]{Corresponding author: michael.doherty.21@ucl.ac.uk}

%% To be edited by editor
% \dates{Compiled \today}

%% To be edited by editor
% \doi{\url{http://dx.doi.org/10.1364/XX.XX.XXXXXX}}


\begin{abstract}
    The application of reinforcement learning (RL) to dynamic resource allocation in optical networks has been the focus of intense research activity in recent years, with almost 100 peer-reviewed papers. We present a review of progress in the field, and identify significant gaps in benchmarking practices and reproducibility. To determine the strongest benchmark algorithms, we systematically evaluate several heuristics across diverse network topologies. We find that path count and sort criteria for path selection significantly affect the benchmark performance. We meticulously recreate the problems from five landmark papers and apply the improved benchmarks. Our comparisons demonstrate that simple heuristics consistently match or outperform the published RL solutions, often with an order of magnitude lower blocking probability. Furthermore, we present empirical lower bounds on network blocking using a novel defragmentation-based method, revealing that potential improvements over the benchmark heuristics are limited to 19--36\% increased traffic load for the same blocking performance in our examples. We make our simulation framework and results publicly available to promote reproducible research and standardized evaluation \hyperlink{https://doi.org/10.5281/zenodo.12594495}{https://doi.org/10.5281/zenodo.12594495}.%{10.5281/zenodo.12594495}.
\end{abstract}


% \begin{abstract}
% We survey the literature on reinforcement learning (RL) applied to dynamic resource allocation (DRA) problems in optical networks. Through critical examination of the reported experimental methodologies and results, we summarize the different approaches taken by researchers and highlight the difficulty in comparing results across non-standardized network topologies, traffic models and training regimes. We argue that, despite intense research interest, the improvement achieved by RL over simple heuristic algorithms on DRA problems has been overstated or non-existent. By quantifying the effect of implementation details such as pre-populated network traffic, inconsistencies in traffic models and path prioritization, we show several RL-based resource allocation policies from the literature are outperformed by heuristic algorithms. Further to this analysis, we estimate the upper bounds network throughput for 11 cases of study and show possible improvement over heuristics is limited to 1-36\% increase in supported traffic load. We conclude with a discussion of how RL performance can be improved on DRA problems and suggest directions for future research in which greater performance gains may be possible. We make all of our code available at \hyperlink{10.5281/zenodo.14561967}{10.5281/zenodo.14561967}.
% \end{abstract}

\setboolean{displaycopyright}{false} % Do not include copyright or licensing information in submission.

\begin{document}

\maketitle

\section{Introduction}\label{sec:intro}

In computational finance, Monte Carlo simulations are used extensively to estimate the expected value of financial payoffs based on the solution of stochastic differential equations (SDEs) which model the evolution of stock prices, interest rates, exchange rates and other quantities \cite{glasserman04}.  Monte Carlo methods are very general and flexible, but for high accuracy it requires generating a large number of costly SDE path approximations, which has motivated research into a number of variance reduction or, equivalently, cost reduction techniques. One such method is
Multilevel Monte Carlo (MLMC), which was proposed in \cite{GILES2008} and was adapted for various applications that are summarised in \cite{Giles_overview17} and successfully combined with other methods such as quasi-Monte Carlo methods. The main idea of MLMC is to approximate the payoff using different time stepping resolutions when numerically solving the underlying SDE and to generate an optimal number of samples on each level, such that the overall computational cost is minimised subject to the desired bound on the variance. %, such that the total computational cost is minimised. 
The computational savings come from the fact that most samples are computed on the coarser levels and hence are less expensive while only a few samples from the finest levels are required \cite{GILES2008}.


Among the directions in which the computational cost 
of MLMC methods could further be reduced, an important avenue is the use of lower precision calculations, especially for the first Monte Carlo levels where the targeted accuracy is relatively low. 
 An overview of the research on mixed precision for the standard Monte Carlo (MC) framework is provided in \cite{ChowMixedPrecisionStandardMC} but only a few references study the potential of low precision computation in the MLMC framework \cite{Rounding_error_oliver}. To the best of our knowledge, the only MLMC framework with customised precision in the literature is \cite{brugger2014mixed}, but they use a uniform precision for all operations on each Monte Carlo level instead of optimising 
 the precision of each intermediary variable to reduce as much as possible the cost of path generation.
 
An important motivation for an MLMC framework with variable precision would be performing the low precision computations on reconfigurable hardware devices such as Field Programmable Gate Arrays (FPGAs). FPGAs contain customizable logic blocks and connectors that make it easy to adapt the digital circuit architecture for a specific application, leading to a highly parallel and optimised implementation. Therefore they are successfully exploited in applications that require high speed and have high computational workload, such as signal processing \cite{woods2008fpga}, and real time applications like high frequency trading \cite{HFT1,HFT2}. That is why a number of previous works in hardware architecture design implemented the MLMC algorithm to price financial options using FPGAs as accelerators, which resulted in improved speed and power efficiency compared to full CPU architectures \cite{Schryver2013AMM}. The paper \cite{lindsey2016domain} also proposed 
a Domain Specific Language to automate the configuration of FPGAs for this specific application. However, only \cite{brugger2014mixed} proposed a heuristic to reduce the precision in calculations.

In addition, all aforementioned works considered that the random number generation (RNG) is performed in single or double precision. Yet in most cases an important portion of the workload in the overall MLMC simulation comes from the RNG and in \cite{brugger2014mixed} this limited the total computational savings.
To reduce the cost of MLMC simulations in particular those based on the Geometric Brownian Motion (GBM), \cite{approximateICDF_Oliver, NestedOliver} have proposed to use approximate random numbers that are generated by applying an approximation of the inverse CDF to uniform random numbers. In \cite{NestedOliver}, the authors proposed a way to integrate these lower precision random variables into a \textit{nested} MLMC framework and completed a numerical analysis to bound the resulting error at each MC level by a product of the time step and the error in the random number approximation. The same authors show in \cite{approximateICDF_Oliver} that using approximate random variables reduces the cost of path generation by a factor 7.


In this paper we propose a nested MLMC framework that combines the use of approximate random normal variables and lower precision calculations to reduce the computational cost of MLMC even further than \cite{brugger2014mixed,NestedOliver}. We illustrate the efficiency of our framework in Matlab, after making several assumptions on the cost of operations and size of the errors that we carefully justify. We focus on the case of GBM and use the approximate RNG methods presented in \cite{approximateICDF_Oliver} as well as a new slightly modified method that combines CDF inversion and the central limit theorem. To choose the precision of the variables in the low precision path generation, we introduce a novel method to optimise the bit-widths. This optimisation is performed before the main path generation loop is executed and is based on a linear model of the payoff error  
due to rounding when computing in low precision. The error model relies on algorithmic differentiation in a similar manner to \cite{unifying-bwoptim,bitwidth-AD,ADAPT}. The bit-width optimisation procedure can be performed off-line, so this stage can be excluded from the on-line time complexity of our framework. The user specified desired accuracy is then enforced by calculating on-line the number of samples that need to be generated.

In terms of hardware design, we suggest implementing the low precision path generation on FPGAs and the full-precision ones on a CPU or GPU. 
The FPGA offers enough flexibility to define a separate bit-width for every variable in the low precision path generation, and can be reconfigured periodically to update the bit-widths when the market parameters have changed considerably. 


The paper is organized as follows : \Cref{sec:MLMC} introduces MLMC and nested MLMC to make clear the estimator that is implemented in our framework. Then in \Cref{sec:RNG} we detail the methods that could be used to obtain approximate random normally distributed numbers very cheaply for the low precision path generation. In \Cref{sec:error_model} and \Cref{sec:costModel} we propose an error model and a cost model (resp.) that we then use to formulate the optimisation problem that is solved to obtain the optimal bit-widths of fixed point variables in \Cref{sec:optimisation}. Finally we summarise our results and future directions in \Cref{sec:conclusion}.




\section{Background}
\label{sec:background}


\subsection{Preliminaries}

{\color{red}[TODO: LLMs? in-context learning?]}

\subsection{Problem Definition}

{\color{red}[TODO: define the problem of citation intent]}


\section{Literature Survey}
\label{sec:survey}

There exists a considerable body of literature on RL for DRA problems in in optical networks. %\cite{amin_survey_2021}. 
DRA in optical networks is distinguished from similar problems in electronically linked networks by the nature of fiber optic links, which carry a set of wavelengths or FSU, defined by the ITU standards G.671 and G.694.2 \cite{international_telecommunication_union_spectral_2002,international_telecommunication_union_transmission_2012}. In this work we only consider publications related to optical networks but acknowledge the closely related literature on RL for other graph-based resource allocation problems.

\begin{figure}
    \centering
    \includegraphics[width=1\linewidth]{IMAGES/RL_RSA_litreview_barchart.png}
    \caption{Count of publications related to RL for resource allocation problems in optical networks. Citations for each classification category are: RWA \cite{garcia_multicast_2003,pointurier_reinforcement_2007,koyanagi_reinforcement_2009,suarez-varela_routing_2019,shiraki_dynamic_2019,shiraki_reinforcement-learning-based_2019,tzanakaki_self-learning_2020,zhao_cost-efficient_2021,freire-hermelo_dynamic_2021,liu_waveband_2021,nevin_techniques_2022,di_cicco_deep_2022,di_cicco_deepls_2023,nallaperuma_interpreting_2023}, RSA \cite{reyes_adaptive_2017,li_deepcoop_2020,li_multi-objective_2020,romero_reyes_towards_2021,zhao_reinforced_2021,wang_dynamic_2021,quang_magc-rsa_2022,cruzado_reinforcement-learning-based_2022,zhao_rsa_2022,jiao_reliability-oriented_2022,almasan_deep_2022,wu_service_2022,arce_reinforcement_2022,sharma_deep_2023,lin_deep-reinforcement-learning-based_2023,hernandez-chulde_experimental_2023,cheng_ptrnet-rsa_2024,chen_gsaddqn_2024}, RMSA \cite{chen_deeprmsa_2019,wang_resource_2020,shimoda_mask_2021,shi_deep-reinforced_2021,shimoda_deep_2021,sheikh_multi-band_2021,xu_spectrum_2021,chen_multi-task-learning-based_2021,gonzalez_improving_2022,bryant_q-learning_2022,terki_routing_2022,tang_deep_2022,cheng_routing_2022,xu_deep_2022,tang_heuristic_2022,tu_entropy-based_2022,momo_ziazet_deep_2022,pinto-rios_resource_2023,errea_deep_2023,beghelli_approaches_2023,terki_routing_2023,tanaka_pre-_2023,xu_hierarchical_2023,sadeghi_performance_2023,tang_routing_2023,teng_deep-reinforcement-learning-based_2024,xiong_graph_2024,teng_drl-assisted_2024,unzain_reinforcement_2024,zhou_opti-deeproute_2024,li_opticgai_2024,xie_physical_2024,yan_drl-based_2024}, Other \cite{boyan_packet_1993,ma_demonstration_2019,zhao_reinforcement-learning-based_2019,wang_subcarrier-slot_2019,luo_leveraging_2019,natalino_optical_2020,ma_co-allocation_2020,wang_deepcms_2020,weixer_reinforcement_2020,liu_multi-agent_2021,zhao_service_2021,tian_reconfiguring_2021,morales_multi-band_2021,tanaka_reinforcement-learning-based_2022,koch_reinforcement_2022,hernandez-chulde_evaluation_2022,etezadi_deepdefrag_2022,etezadi_deep_2023,tanaka_adaptive_2023,johari_drl-assisted_2023,zhang_admire_2023,fan_blocking-driven_2023,lian_dynamic_2024,li_tabdeep_2024,wang_availability-aware_2024,yin_dnn_2024,tse_reinforcement_2024,tanaka_reinforcement-learning-based_2024,doherty_xlron_2024,natalino_optical_2024,mccann_sdonsim_2024,jara_dream-gym_2024}.}
    \label{fig:lit_barchart}
\end{figure}



\subsection{Survey methodology}
\label{sec:survey-methodology}

%we searched the Google Scholar database using the following search terms: "'REINFORCEMENT LEARNING' AND 'NETWORK' AND ('OPTICAL' OR 'WAVELENGTH' OR 'SPECTRUM')". Papers were limited to English-language peer-reviewed articles. We filtered the search results by inspection of paper titles and abstracts, and manually added any related works from our own citation database that were missing from the initial search. A review of the collected papers then created the final set of 97.

To provide an overview of research progress on RL applied to DRA problems in optical networks, we searched to gather all relevant research papers. We performed a manual review of results from citation databases to create the final set of 97 peer-reviewed papers. Figure \ref{fig:lit_barchart} shows the count of papers by publication year. The papers are grouped in 4 categories: 'RWA', 'RSA', 'RMSA', and 'Other'. We use this set of papers for our analysis of benchmarking practices in the field, which we present in the next section with further commentary on figure \ref{fig:lit_barchart}. 

%the 'Other' category of figure \ref{fig:lit_barchart}, which includes papers on RL applied to: traffic grooming \cite{tanaka_reinforcement-learning-based_2022,zhang_admire_2023,tanaka_adaptive_2023,tanaka_reinforcement-learning-based_2024}, defragmentation \cite{etezadi_deepdefrag_2022,fan_blocking-driven_2023,johari_drl-assisted_2023,etezadi_deep_2023}, survivability or service restoration \cite{zhao_reinforcement-learning-based_2019,luo_leveraging_2019,zhao_service_2021,hernandez-chulde_evaluation_2022,zhao_rsa_2022,luo_survivable_2022,jiao_reliability-oriented_2022}, multicast provisioning \cite{garcia_multicast_2003,tian_reconfiguring_2021,li_tabdeep_2024}, simulation training environments \cite{natalino_optical_2020,natalino_optical_2024,doherty_xlron_2024,mccann_sdonsim_2024}, and other problems such as transceiver parameter optimization \cite{weixer_reinforcement_2020,koch_reinforcement_2022,koch_high-generalizability_2022} or launch power optimization \cite{tse_reinforcement_2024}


\subsection{Review of benchmarking practices}
\label{sec:benchmarking_practices}


%To understand progress on RL applied to DRA problems in optical networks, it is essential to have standard benchmarks. Benchmarks refer to both the problem under investigation and the quality of the solution. In machine learning research, performance benchmarks on MNIST \cite{li_deng_mnist_2012} and ImageNet \cite{krizhevsky_imagenet_2012} have been vital to the astounding progress in that field. 


In optical networks research, the first benchmark for RL was established by DeepRMSA \cite{chen_deeprmsa_2019} (discussed in detail in Section \ref{sec:repro}). DeepRMSA was the first RL approach to achieve lower service blocking probability than KSP-FF, or any heuristic that considers multiple candidate paths. As a result of this breakthrough performance, and its open source codebase, the problem definition from DeepRMSA (topologies, traffic model, modulation format reach, FSU per link, etc.) became a de facto standard. Follow-up works used identical or similar problem definitions and compared to DeepRMSA on their problem \cite{xu_spectrum_2021,quang_magc-rsa_2022,errea_deep_2023,tang_heuristic_2022,xu_deep_2022,cheng_ptrnet-rsa_2024,yan_drl-based_2024,zhou_opti-deeproute_2024}. Arguably, comparing to DeepRMSA has become standard benhcmarking practice.

%Although this comparison to the "state of the art" appears to be good practice, it is misguided. The performance of RL agents is highly sensitive to algorithmic details and hyperparameters \cite{engstrom_implementation_2020}, random seeds, and non-deterministic factors \cite{nagarajan_impact_2018}. Guidelines have been established for the reliable comparison of competing RL approaches \cite{henderson_deep_2019}. The comparisons to DeepRMSA in follow-up works have not followed these guidelines. For example, hyperparameters such as learning rate and discount factor should be tuned when applying an algorithm to a new setting. Consequently, the comparisons to DeepRMSA in follow-up works are not fair.

Previous work has called for more rigorous benchmarking practices for research on RL for optical networking \cite{di_cicco_deep_2022}, with recommendations for comparison against other machine learning approaches such as GA and PSO, in addition to estimated bounds on network blocking or throughput. Some studies of RL for resource allocation have restricted themselves to sufficiently small problem sizes and static traffic, to enable comparison to ILP results \cite{liu_waveband_2021,di_cicco_deep_2022,zhao_rsa_2022,momo_ziazet_deep_2022,di_cicco_deepls_2023}. Although this provides a reliable bound, it is not applicable to dynamic traffic.

Benchmarking against standard heuristic algorithms, such as KSP-FF, avoids the complexity of training a rival machine learning approach, performs deterministic allocation, and can scale to large problem instances. However, it is important to choose the best performing heuristic for a particular case of study as a benchmark. Of the papers that benchmark their RL solution to KSP-FF (or other heuristics that consider multiple candidate paths) \cite{chen_deeprmsa_2019,chen_multi-task-learning-based_2021,shi_deep-reinforced_2021,shimoda_deep_2021,xu_spectrum_2021,zhao_reinforced_2021,zhao_service_2021,shimoda_mask_2021,quang_magc-rsa_2022,tu_entropy-based_2022,tang_heuristic_2022,xu_deep_2022,cheng_routing_2022,nevin_techniques_2022,tang_deep_2022,di_cicco_deep_2022,terki_routing_2023,sadeghi_performance_2023,tanaka_adaptive_2023,tang_routing_2023,xu_hierarchical_2023,errea_deep_2023,hernandez-chulde_experimental_2023,cheng_ptrnet-rsa_2024,fan_blocking-driven_2023}, most achieve 20-30\% reduction in service blocking probability compared to their best heuristic. Only 3 papers achieve a reduction greater than this: MaskRSA \cite{shimoda_mask_2021}, PtrNet-RSA \cite{cheng_ptrnet-rsa_2024}, and Terki et al \cite{terki_routing_2022}. Despite these impressive results, we demonstrate in Section \ref{sec:repro_main} that MaskRSA and PtrNet-RSA are beaten by KSP-FF or FF-KSP by considering 50 candidate paths and ordering the paths by number of hops\footnotemark.

\footnotetext{We have not re-created the study of Terki et al. for benchmarking in Section \ref{sec:repro_main} as it is multi-band and out of scope of this work.
We hypothesise that their approach performs strongly because, similar to PtrNet-RSA, it is not limited to selecting from only K paths.}

Benchmarking is further complicated by the fast evolution of optical networking, with novel paradigms such as multi-band \cite{beghelli_approaches_2023} and multi-core \cite{pinto-rios_resource_2023} emerging, and the wide variety of network topologies \cite{matzner_topology_2024} and components that can be considered. The evolution of optical networks research is evidenced by growth in the 'Other' category of figure \ref{fig:lit_barchart}, which includes papers on RL applied to: traffic grooming \cite{tanaka_reinforcement-learning-based_2022,zhang_admire_2023,tanaka_adaptive_2023,tanaka_reinforcement-learning-based_2024}, defragmentation \cite{etezadi_deepdefrag_2022,fan_blocking-driven_2023,johari_drl-assisted_2023,etezadi_deep_2023}, survivability or service restoration \cite{zhao_reinforcement-learning-based_2019,luo_leveraging_2019,zhao_service_2021,hernandez-chulde_evaluation_2022,zhao_rsa_2022,luo_survivable_2022,jiao_reliability-oriented_2022}, multicast provisioning \cite{garcia_multicast_2003,tian_reconfiguring_2021,li_tabdeep_2024}, %simulation training environments \cite{natalino_optical_2020,natalino_optical_2024,doherty_xlron_2024,mccann_sdonsim_2024}, 
and other problems such as transceiver parameter optimization \cite{weixer_reinforcement_2020,koch_reinforcement_2022,koch_high-generalizability_2022} or launch power optimization \cite{tse_reinforcement_2024}

The establishment of reliable benchmarks is made more difficult by the fragmented software environment for optical network simulations for RL. Several open source toolkits have been introduced to aid researchers and improve productivity, but none has proved sufficiently popular for it to become standard. Optical-rl-gym \cite{natalino_optical_2020} was the first paper to attempt to introduce a new standard library for this task. This was followed by an extension to multi-band environments \cite{morales_multi-band_2021} and in 2024 was further extended to include a more sophisticated physical layer model for lightpath SNR calculations, renamed as the Optical Networking Gym \cite{natalino_optical_2024}. Additionally, MaskRSA provides an open source simulation framework (RSA-RL) \cite{shimoda_mask_2021} and DeepRMSA's codebase is widely used \cite{chen_deeprmsa_nodate}. SDONSim \cite{mccann_sdonsim_2024} and DREAM-ON-GYM \cite{jara_dream-gym_2024} are other recent additions to the landscape of available simulation frameworks that further fragment the available options. 

In summary, progress in applying RL to DRA problems in optical networks has been difficult to quantify due to several factors. First, the lack of standardized benchmarking practices has made it challenging to fairly compare different approaches. Second, while some studies have used ILP solutions as benchmarks, these are limited to small problem sizes and static traffic scenarios, making them impractical for large-scale or dynamic applications. Third, multiple competing simulation frameworks %(Optical-rl-gym, RSA-RL, SDONSim, DREAM-ON-GYM) 
and publications without open source code, have made it difficult to ensure consistent testing conditions across different studies. Finally, the rapid evolution of optical networking technology means benchmarks must constantly evolve to remain relevant. 

The lack of reliable benchmarks, and the resulting difficulty in assessing progress in the field, is what motivates our investigations of heuristic benchmarks in Section \ref{sec:heuristic_comparison} and their application to our reproduction of previous studies in Section \ref{sec:repro_main}.

%As we show in section \ref{sec:repro}, heuristic algorithms . This underscores the need for more rigorous and standardized evaluation methods in this field. 
% 

%In this work we use XLRON \cite{doherty_xlron_2023}, the framework introduced at Optical Fibre Communications (OFC) 2024 that we continue to develop. We outline its benefits and justify its use in Section \ref{sec:repro}.



\subsection{Selection of papers for benchmarking}
\label{sec:paper_summaries}

On the basis of our review of benchmarking practices, we select 5 papers to reproduce and re-benchmark in section \ref{sec:repro_main}. We select these papers primarily because they all compare their results to "DeepRMSA"\footnotemark with similar traffic models and topologies, therefore present the most consistent application of benchmarks in the field. We also select based on their impact, which we assess by qualitative and quantitative criteria. The qualitative criteria are novelty, contribution, and reputation of publication or conference. The quantitative criterion is their blocking performance relative to benchmarks. They are also among the most highly cited papers in the field, as of January 2025.

\footnotetext{We note that the training of RL solutions is highly sensitive to hyperparameters \cite{engstrom_implementation_2020} and non-deterministic factors \cite{nagarajan_impact_2018}, therefore the comparisons that the selected papers make to re-trained DeepRMSA agents may not be robust.}


\begin{enumerate}[itemsep=0pt]
    \item \textbf{DeepRMSA} \cite{chen_deeprmsa_2019} constructs a feature matrix to represent the available paths for the current requests and applies a NN with 5 x 128 hidden units to select from the K-shortest paths with first-fit spectrum allocation. It demonstrates service blocking probability(SBP) reduced by 20\% vs. KSP-FF on the NSFNET and COST239 topologies. DeepRMSA's impact was enhanced by its open source codebase. %published 2019, has over 178 citations.
    \item \textbf{Reward-RMSA} \cite{tang_heuristic_2022} builds on the DeepRMSA framework and changes the reward function to incorporate fragmentation-related information. They report SBP reduced by 32\% vs. multiple heuristics and 55\% vs. DeepRMSA on NSFNET and COST239. %published 2022, has over 22 citations.
    \item \textbf{GCN-RMSA} \cite{xu_deep_2022} is notable as the first work to use advanced NN architectures to improve performance. They use a graph convolutional network (GCN) (including recurrent neural network (RNN) as the path aggregation function) in the policy and value functions, which they claim allows improved feature extraction from the network state. Like DeepRMSA and Reward-RMSA, the policy selects from K paths with first-fit spectrum allocation. They report SBP reduced by up to 30\% vs. multiple heuristics and 18\% vs. DeepRMSA. on NSFNET, COST239, and USNET. %published 2022, has over 36 citations
    \item \textbf{MaskRSA} \cite{shimoda_mask_2021} innovated by selecting from the entire range of available slots on the K paths and using invalid action masking \cite{huang_closer_2020} to increase the efficiency of training. Despite the RSA in the title, the paper does consider distance-dependent modulation format (RMSA). MaskRSA presented improvements over KSP-FF on NSFNET and JPN48 topologies with over an order of magnitude lower SBP, or a 35-45\% increase in the supported traffic in their cases of study. The authors of MaskRSA also contributed to open source by releasing their simulation framework, RSA-RL. %published 2021, has over 28 citations
    \item \textbf{PtrNet-RSA} \cite{cheng_ptrnet-rsa_2024}, newly published in 2024. It innovates in both the problem setting and its use of pointer-nets \cite{vinyals_pointer_2015}. The pointer-net is used to select the constituent nodes of the target path, thereby removing the restriction of selecting from the pre-calculated K-shortest paths. Invalid action masking is used to allow selection from all available spectral slots. Additionally, the paper considers joint optimization of the mean path SNR and the SBP through its reward function. It demonstrates SBP reduced by over an order of magnitude vs. KSP-FF and their implementation of MaskRSA on NSFNET, COST239, and USNET.
\end{enumerate}


%Although this comparison to the "state of the art" appears to be good practice, it is misguided. The performance of RL agents is highly sensitive to algorithmic details and hyperparameters \cite{engstrom_implementation_2020}, random seeds, and non-deterministic factors \cite{nagarajan_impact_2018}. Guidelines have been established for the reliable comparison of competing RL approaches \cite{henderson_deep_2019}. The comparisons to DeepRMSA in follow-up works have not followed these guidelines. For example, hyperparameters such as learning rate and discount factor should be tuned when applying an algorithm to a new setting. Consequently, the comparisons to DeepRMSA in follow-up works are not fair.





% \subsection{Research trends}
% \label{sec:trends}

% Our analysis of the literature reveals several significant trends:

% \textbf{Timeline and Technology Adoption: }
% A marked increase in publications began in 2019, following the influential DeepRMSA paper \cite{chen_deeprmsa_2019}. This represents a 5-6 year lag between breakthrough ML publications (e.g. Deep RL for Atari in 2013 \cite{mnih_human-level_2015}) and adoption in optical networking. Publication activity peaked in 2022, with a notable dip in 2020 due to the COVID-19 pandemic.

% \textbf{Evolution of Problem Focus: }
% Early work (pre-2019) predominantly addressed RWA but RWA publications have declined since, with no publications in 2024. Research attention has shifted toward RMSA. Recent work also increasingly addresses specialized problems beyond basic routing and spectrum allocation, as seen by the rise in "Other" categories.

% \textbf{Research Maturity and Diversification: }
% The growth in "Other" categories suggests researchers are seeking new directions. Recent publications show movement toward novel application areas with potentially greater performance gains, such as multi-objective optimization \cite{nallaperuma_interpreting_2023}, joint optimization of blocking and lightpath signal-to-noise ratio (SNR) \cite{cheng_ptrnet-rsa_2024,xie_physical_2024,teng_deep-reinforcement-learning-based_2024,yan_drl-based_2024} or channel launch power \cite{xiao_channel_2024}. We hypothesise this shift may be partly motivated by a saturation of performance improvements in simpler RWA/RSA/RMSA problems, which we investigate in section \ref{sec:bounds}.



% \subsection{Recommendations for reporting metrics and reproducibility}
% \label{sec:recommendations}
% Based on the above review, we make  recommendations to improve the quality and reproducibility of results:

% \textbf{Improved metrics}

% For a more meaningful performance comparisons, several metrics could be refined, as outlined in this section.

% - Network data throughput: use bitrate blocking probability (BBP) rather than service blocking probability (SBP) to quantify network blocking.  Key when traffic requests have diverse bitrate requirements. For RWA or uniform bitrate traffic, BBP and SBP are equivalent. 

% Consider the approach of Shiraki et al. \cite{shiraki_dynamic_2019} and Cruzado et al. \cite{cruzado_capacity-bound_2024} of setting a target blocking probability (e.g., 0.1\%) and comparing solutions based on the maximum traffic load that can be supported. Data traffic in Tbps is a more easily interpreted and physically meaningful measure than blocking probability.This is used to present network blocking bounds in Section \ref{sec:bounds}. We believe that researchers have preferred to report on blocking probability as it allows headline results to appear more impressive (e.g. "one order of magnitude reduction") than the equivalent relative change in supported traffic



% \textbf{Statistical Significance:}
% Many works do not include sufficient trials to establish statistical significance, particularly when blocking events are rare. It is proposed that the following steps are followed:
% \begin{itemize}[itemsep=0pt]
%     \item Obtain a sufficient number of blocking events (minimum 100) to have >95\% confidence in mean blocking estimate
%     \item Ensure different random seeds are used for parallel training or evaluation runs
%     \item Publish uncertainty estimates along with mean performance metrics
%     \item Use multiple random seeds (minimum 5) for both training and evaluation
% \end{itemize}

% \textbf{Reproducibility:}
% We propose to compare network blocking from deterministic heuristic algorithms. When comparing against existing RL solutions, we emphasise the following:
% \begin{itemize}[itemsep=0pt]
%     \item Hyperparameters must be re-tuned when the problem setting differs from the original paper
%     \item Equivalent computing resources should be devoted to performing hyperparameter sweeps for and training competing solutions
%     \item All code should be made publicly available, with instructions to reproduce published results.

    
% \end{itemize}





















%%%%%%%%%%%%%%%%%%

%Shi et al. showed progress in the problem setting by including a simple lightpath SNR calculation to the network physical layer model, and calculating the highest available modulation format on the basis of SNR instead of maximum reach \cite{shi_deep-reinforced_2021}. They make this minor addition and otherwise use the same algorithmic setup as DeepRMSA.

%Unusually, a negative result was accepted for publication at ONDM 2021 on RL for multi-band RMSA \cite{sheikh_multi-band_2021}. This can partially be explained by interest in the relatively novel problem setting but perhaps also indicates that progress on the application of RL to this problem and similar was stalling.


% Could complete the argument by saying that the solutions provided by the defragmentation method are entirely physically realistic, therefore present a hard upper bound on network capacity, in comparison to the cut-sets method which may be tighter than necessary due to sub-optimal spectrum allocation.
%% IDEA: the best upper bound would probably be to do cut-sets with defragmented spectrum i.e. just do cut-sets with scalar capacity on each link, i.e. just do the traditional max-flow min-cut theorem but with the defragmented spectrum. This would be the most physically realistic upper bound. Could be a good future work idea.


% Specifically, the successor works that use the DeepRMSA codebase are . Note that 

% The direct successor works are from Chen et al using the DeepRMSA framework with transfer learning on different topologies and traffic \cite{chen_multi-task-learning-based_2021}, Tang et al using a shaped reward function to improve performance, Xu et al using graph and recurrent neural networks to for the policy network to improve performance \cite{xu_deep_2022}, Xu et al investigate a hierarchical framework for multi-domain networks \cite{xu_hierarchical_2023}, Errea et al use DeepRMSA with two choices of first slots per path \cite{errea_deep_2023}, a modification that was investigated in the original DeepRMSA paper. DeepRMSA has also been used as a benchmark in MaskRSA \cite{shimoda_mask_2021} and PtrNet-RSA \cite{cheng_ptrnet-rsa_2024}.

















%%%%%%%%%%%%%%%%%%





% 0  DeepRMSA: A Deep Reinforcement Learning Framew...     chen_deeprmsa_2019
% 1  Mask RSA: End-To-End Reinforcement Learning-ba...      shimoda_mask_2021
% 2  Techniques for applying reinforcement learning...  nevin_techniques_2022
% 3  The Optical RL-Gym: An open-source toolkit for...  natalino_optical_2020
% 4  Heuristic Reward Design for Deep Reinforcement...    tang_heuristic_2022

% Total entries: 97
% Matched entries: 97
% Matching rate: 100.00%
% Papers classified as 'Other':
% - The Optical RL-Gym: An open-source toolkit for applying reinforcement learning in optical networks
% - Reconfiguring multicast sessions in elastic optical networks adaptively with graph-aware deep reinforcement learning
% - DRL-Assisted Reoptimization of Network Slice Embedding on EON-Enabled Transport Networks
% - Blocking-Driven Spectrum Defragmentation Based on Deep Reinforcement Learning in Tidal Elastic Optical Networks
% - DeepDefrag: A deep reinforcement learning framework for spectrum defragmentation
% - Multi-band Environments for Optical Reinforcement Learning Gym for Resource Allocation in Elastic Optical Networks
% - TABDeep: A two-level action branch architecture-based deep reinforcement learning for distributed sub-tree scheduling of online multicast sessions in EON
% - Reinforcement-learning-based path planning in multilayer elastic optical networks [Invited]
% - Adaptive Traffic Grooming Using Reinforcement Learning in Multilayer Elastic Optical Networks
% - Reinforcement-Learning-based Multilayer Path Planning Framework that Designs Grooming, Route, Spectrum, and Operational Mode
% - ADMIRE: collaborative data-driven and model-driven intelligent routing engine for traffic grooming in multi-layer X-Haul networks
% - Packet Routing in Dynamically Changing Networks: A Reinforcement Learning Approach
% - Reinforcement Learning for Power Management in Low-margin Optical Networks
% - DNN distributed inference offloading scheme based on transfer reinforcement learning in metro optical networks
% - Availability-Aware and Delay-Sensitive RAN Slicing Mapping Based on Deep Reinforcement Learning in Elastic Optical Networks
% - DREAM-ON GYM: A Deep Reinforcement Learning Environment for Next-Gen Optical Networks:
% - Dynamic slicing of multidimensional resources in DCI-EON with penalty-aware deep reinforcement learning
% - DeepCMS <sup>3</sup> : A Deep Reinforcement Learning Framework for Core, Mode and Spectrum Sequential Scheduling over Optical Transport Network
% - A Subcarrier-Slot Autonomous Partition Scheme Based on Deep-Reinforcement-Learning in Elastic Optical Networks
% - Reinforcement-Learning-Based Multi-Failure Restoration in Optical Transport Networks
% - Demonstration of Image Processing Based on Reinforcement Learning in Multi-Modal Optical Transport Networks
% - Co-Allocation of Service Routing in SDN-driven 5G IP+Optical Smart Grid Communication Networks based on Deep Reinforcement Learning
% - Service restoration in multi-modal optical transport networks with reinforcement learning
% - Multi-Agent Federated Reinforcement Learning for Privacy-enhanced Service Provision in Multi-domain Optical Network
% - Evaluation of Deep Reinforcement Learning for Restoration in Optical Networks
% - Deep reinforcement learning for proactive spectrum defragmentation in elastic optical networks
% - XLRON: Accelerated Reinforcement Learning Environments for Optical Networks
% - Reinforcement Learning for Generalized Parameter Optimization in Elastic Optical Networks
% - Optical Networking Gym: an open-source toolkit for resource assignment problems in optical networks
% - SDONSim: An Advanced Simulation Tool for Software-Defined Elastic Optical Networks
% - Leveraging double-agent-based deep reinforcement learning to global optimization of elastic optical networks with enhanced survivability
% - A Reinforcement Learning Framework for Parameter Optimization in Elastic Optical Networks

% Papers classified as 'Routing':

% Tag distribution:
% RMSA: 33 (34.02%)
% Other: 32 (32.99%)
% RSA: 18 (18.56%)
% RWA: 14 (14.43%)

% Total papers: 97

% First few rows of the updated DataFrame:
%                                                Title  \
% 0  DeepRMSA: A Deep Reinforcement Learning Framew...   
% 1  Mask RSA: End-To-End Reinforcement Learning-ba...   
% 2  Techniques for applying reinforcement learning...   
% 3  The Optical RL-Gym: An open-source toolkit for...   
% 4  Heuristic Reward Design for Deep Reinforcement...   

%                 Manual Tags Priority_Tag  
% 0                  RL; RMSA         RMSA  
% 1  RL; RMSA; Action masking         RMSA  
% 2      RL; RWA; Incremental          RWA  
% 3        RL; Other; Toolkit        Other  
% 4                  RL; RMSA         RMSA  
% Citation tags for each priority tag class, sorted by publication date:

% Other:
% boyan_packet_1993,ma_demonstration_2019,zhao_reinforcement-learning-based_2019,wang_subcarrier-slot_2019,luo_leveraging_2019,natalino_optical_2020,ma_co-allocation_2020,wang_deepcms_2020,weixer_reinforcement_2020,liu_multi-agent_2021,zhao_service_2021,tian_reconfiguring_2021,morales_multi-band_2021,tanaka_reinforcement-learning-based_2022,koch_reinforcement_2022,hernandez-chulde_evaluation_2022,etezadi_deepdefrag_2022,etezadi_deep_2023,tanaka_adaptive_2023,johari_drl-assisted_2023,zhang_admire_2023,fan_blocking-driven_2023,lian_dynamic_2024,li_tabdeep_2024,wang_availability-aware_2024,yin_dnn_2024,tse_reinforcement_2024,tanaka_reinforcement-learning-based_2024,doherty_xlron_2024,natalino_optical_2024,mccann_sdonsim_2024,jara_dream-gym_2024

% RMSA:
% chen_deeprmsa_2019,wang_resource_2020,shimoda_mask_2021,shi_deep-reinforced_2021,shimoda_deep_2021,sheikh_multi-band_2021,xu_spectrum_2021,chen_multi-task-learning-based_2021,gonzalez_improving_2022,bryant_q-learning_2022,terki_routing_2022,tang_deep_2022,cheng_routing_2022,xu_deep_2022,tang_heuristic_2022,tu_entropy-based_2022,momo_ziazet_deep_2022,pinto-rios_resource_2023,errea_deep_2023,beghelli_approaches_2023,terki_routing_2023,tanaka_pre-_2023,xu_hierarchical_2023,sadeghi_performance_2023,tang_routing_2023,teng_deep-reinforcement-learning-based_2024,xiong_graph_2024,teng_drl-assisted_2024,unzain_reinforcement_2024,zhou_opti-deeproute_2024,li_opticgai_2024,xie_physical_2024,yan_drl-based_2024

% RSA:
% reyes_adaptive_2017,li_deepcoop_2020,li_multi-objective_2020,romero_reyes_towards_2021,zhao_reinforced_2021,wang_dynamic_2021,quang_magc-rsa_2022,cruzado_reinforcement-learning-based_2022,zhao_rsa_2022,jiao_reliability-oriented_2022,almasan_deep_2022,wu_service_2022,arce_reinforcement_2022,sharma_deep_2023,lin_deep-reinforcement-learning-based_2023,hernandez-chulde_experimental_2023,cheng_ptrnet-rsa_2024,chen_gsaddqn_2024

% RWA:
% garcia_multicast_2003,pointurier_reinforcement_2007,koyanagi_reinforcement_2009,suarez-varela_routing_2019,shiraki_dynamic_2019,shiraki_reinforcement-learning-based_2019,tzanakaki_self-learning_2020,zhao_cost-efficient_2021,freire-hermelo_dynamic_2021,liu_waveband_2021,nevin_techniques_2022,di_cicco_deep_2022,di_cicco_deepls_2023,nallaperuma_interpreting_2023

% Date range for each tag class:
% RMSA: 2019 - 2024
% RWA: 2003 - 2023
% Other: 1993 - 2024
% RSA: 2017 - 2024







% K-paths used but not beaten: \cite{sheikh_multi-band_2021,zhao_cost-efficient_2021}
% K-paths beaten: \cite{chen_deeprmsa_2019,chen_multi-task-learning-based_2021,,shi_deep-reinforced_2021,shimoda_deep_2021,xu_spectrum_2021,zhao_reinforced_2021,zhao_service_2021,shimoda_mask_2021,quang_magc-rsa_2022,terki_routing_2023,tu_entropy-based_2022,tang_heuristic_2022,xu_deep_2022,cheng_routing_2022,nevin_techniques_2022,tang_deep_2022,di_cicco_deep_2022,fan_blocking-driven_2023,sadeghi_performance_2023,tanaka_adaptive_2023,tang_routing_2023,xu_hierarchical_2023,errea_deep_2023,hernandez-chulde_experimental_2023,cheng_ptrnet-rsa_2024}
% ILP: \cite{liu_waveband_2021,di_cicco_deep_2022,zhao_rsa_2022,momo_ziazet_deep_2022,di_cicco_deepls_2023}
% DeepRMSA as benchmark: \cite{xu_spectrum_2021,quang_magc-rsa_2022,errea_deep_2023,tang_heuristic_2022,xu_deep_2022,cheng_ptrnet-rsa_2024,yan_drl-based_2024,zhou_opti-deeproute_2024}


% - DeepRMSA became a benchmark
% - I want to make the point that it is difficult to establish long-lived standard problems in optical networks as the field evolves quickly and new operating paradigms (such as multi-band andm utli-core) are consistently emerging.
% - 

% \cite{henderson_deep_2019} Deep Reinforcement Learning That Matters - general good practice (i.e.e epxeirmental reporting and procedure) for reproducibility of RL results
% \cite{nagarajan_impact_2018}  Impact of non-determinism in RL
% \cite{engstrom_implementation_2020} Sensitivity of PPO to code-level optimizations and implementation details




\section{Heuristic algorithm benchmark evaluation}
\label{sec:heuristic_comparison}

To evaluate the results from our selected papers in section \ref{sec:paper_summaries}, we must determine the best (lowest blocking probability) heuristic algorithms to use as benchmarks. In this section, we present comparisons of the heuristics listed in Table \ref{tab:heuristics}, evaluated on different traffic loads, topologies, and considering different numbers of candidate paths (K). On the basis of this analysis, we select the benchmarks to apply in section \ref{sec:repro_main}.

We select the algorithms in Table \ref{tab:heuristics} because they are commonly used as benchmarks or have been reported as superior to other heuristics.


\begin{table}[h]
\begin{tabular}{l|l|l}
Heuristic                         & Acronym & Reference                \\ \hline
K-Shortest Paths First-Fit        & KSP-FF  & \cite{vincent_scalable_2019} \\
First-Fit K-Shortest Paths        & FF-KSP  & \cite{vincent_scalable_2019} \\
K-Shortest Paths Best-Fit         & KSP-BF  & \cite{abkenar_best_2016} \\
Best-Fit K-Shortest Paths         & BF-KSP  & \cite{abkenar_best_2016} \\
K-Minimum Entropy First-Fit       & KME-FF  & \cite{wright_minimum-_2015} \\
%K-Minimum Cut First-Fit           & KMC-FF  & \cite{tang_heuristic_2022} \\
%K-Minimum Frag First-Fit & KMF-FF  & \cite{tang_heuristic_2022} \\
K-Congestion Aware First-Fit      & KCA-FF  & CA2 from \cite{savory_congestion_2014}
\end{tabular}
\caption{RMSA heuristics used for benchmarking.}
\label{tab:heuristics}
\end{table}

%We also evaluated the K-Minimum Cut First-Fit (KMC-FF) and K-Minimum Frag First-Fit (KMF-FF) heuristics from Reward-RMSA \cite{tang_heuristic_2022}. These heuristics produce higher blocking probability than others when paths are ordered by number of hops. We exclude these results to improve the clarity of Figures \ref{fig:heur_comp}, \ref{fig:k_traffic}, and \ref{fig:heur_traffic}.

\begin{figure}
  \includegraphics[width=1.01\linewidth]{IMAGES/networks_plots_short.png}
  \caption{Network topologies used in our case studies from: DeepRMSA, Reward-RMSA, GCN-RMSA, MaskRSA, PtrNet-RSA \cite{chen_deeprmsa_2019} \cite{tang_heuristic_2022} \cite{xu_deep_2022} \cite{shimoda_mask_2021} \cite{cheng_ptrnet-rsa_2024}. We note that the USNET topology differs between GCN-RMSA and PtrNet-RSA. We show the GCN-RMSA version here. PtrNet-RSA also uses a variant of the COST239 topology.} %The COST239 topology shown is from DeepRMSA, the USNET is from PtrNet-RSA.}
  \label{fig:network_plots}
\end{figure}

Figure \ref{fig:network_plots} shows the topologies used in the selected papers. We use these topologies to analyze the performance of the heuristic algorithms and in our reproductions of the papers' problems in section \ref{sec:repro_main}. %Some characteristic features of the topologies are summarised in table \ref{tab:topology_features}. 
We make all topology data available in our open source codebase \cite{michael_doherty_2024_jocn_xlron_2024}.



\subsection{Effect of path ordering}

All the heuristics in Table \ref{tab:heuristics} select from the available pre-computed paths on the basis of sort criteria. The primary criterion may be a measure of the path congestion (KCA-FF), spectral fragmentation (KME-FF), or length (KSP-FF). In the event of multiple paths with equal value, a default ordering (usually ascending order of length) determines the selected path.

Usually, path length is considered as distance in km. However, we find that considering path length as number of hops (with length in km as a secondary sort criterion), significantly improves the performance of the heuristics. This has been observed previously by Baroni \cite{baroni_routing_1998}, who referred to it as Minimum Number of Hops routing (MNH). The intuitive explanation for this is that, if two paths can support the same order of modulation format, the path that comprises fewer links occupies fewer spectral resources. 

We refer to these two orderings as path length in km (\#km) or path length in number of hops (\#hops). Our comparisons of KSP-FF for these two orderings in Section \ref{sec:repro} Figure \ref{fig:repro} evidence the reduction in blocking probability from \#hops ordering. In our comparisons of heuristics in the next section, we use \#hops ordering.



\subsection{Simulation setup}

For each heuristic and topology, we carried out three simulation scenarios to investigate the effects of varying traffic loads and values of K on the relative blocking performance of the heuristics. 

\vspace{0.1cm}

\noindent \textbf{Experiment 1} - Increasing K:
\par \noindent \textbf{Aim}: Investigate relative performance of heuristics with increasing K. \textbf{Method}: Record service blocking probability (SBP) for each heuristic at values of K ranging from 2 to 26 at fixed traffic load. We arbitrarily select the traffic load for each topology so that the heuristics give a SBP of  $\sim$1\%.

\vspace{0.1cm}

\noindent \textbf{Experiment 2} - Increasing K at high to low traffic: 
\par \noindent \textbf{Aim}: Investigate the effect of increasing K at different traffic loads. \textbf{Method}: Record SBP at K ranging from 2 to 40 for a range of traffic loads. We select the traffic loads for each topology such that they result in 10$^{-5}$ to 10$^{-1}$ SBP. To simplify the analysis and plots, we only present results for KSP-FF.

\vspace{0.1cm}

\noindent \textbf{Experiment 3} - Increasing traffic load at K=50: 
\par \noindent \textbf{Aim}: Using the findings from Experiments 1 and 2, determine the lowest-blocking heuristic with optimized K-value across traffic loads. \textbf{Method}: Record SBP for high K (K=50) at varying traffic loads. We select the traffic loads for each topology such that they result in a range of SBP (10$^{-5}$ to 10$^{-1}$). This experiment provides evidence on which heuristic is the best overall for each topology.

\vspace{0.1cm}

The data for each heuristic in each experiment was collected from 3000 independent trials with unique random seeds. The SBP was calculated after 10,000 connection requests, with the mean and standard deviation calculated across trials. Each data point in Figures \ref{fig:heur_comp},\ref{fig:k_traffic},\ref{fig:heur_traffic} therefore shows summary statistics from 30 million connection requests, which gives high confidence that our results present the true mean and standard deviation in each case.

We considered dynamic traffic with fixed mean service holding time at 10 units. We considered the same traffic model and other settings as DeepRMSA\footnotemark: uniform traffic probability between each node pair, Poissonian arrival and departure statistics, uniform random selection of data rate from 25 to 100Gbps in 1Gbps intervals, and distance-dependent modulation formats from BPSK to 16QAM. We consider topologies with dual fibre links (one for each direction of propagation), 12.5GHz FSU width, and 100 FSU per fibre.

\footnotetext{We consider the DeepRMSA problem settings in these experiments because it is used by most of the papers presented in Section \ref{sec:repro_main}.} %For the same reason, we report our results in terms of SBP, despite our recommendation to use BBP in Section \ref{sec:recommendations}.}

\subsection{Results and discussion}

\textbf{Experiment 1} results in Figure \ref{fig:heur_comp} show different outcomes for smaller networks (NSFNET and COST239) and larger networks (USNET and JPN48). For NSFNET and COST239, KSP-FF and KME-FF are approximately equal and give the lowest blocking. Their blocking decreases to a minimum for approximately K=23 and above for NSFNET and continues to decrease for K>26 for COST239.

For USNET and JPN48, FF-KSP is clearly the best heuristic, with blocking reduced by half for JPN48. Blocking from FF-KSP decreases with K until K=26 for USNET and continues dropping sharply for K>26 for JPN48. For USNET, KSP-FF and KME-FF become competitive with FF-KSP at large K. It can be argued that FF-KSP performs better in larger networks where there are multiple roughly equivalent paths between source and destination, and dense packing of utilized wavelengths increases in relative importance to path selection.

The results from Experiment 1 indicate that KSP-FF and FF-KSP generally give the lowest blocking, depending on the network topology, and blocking decreases monotonically with increasing K. This experiment looked at a moderately high traffic load ($\sim$1\% SBP), therefore experiment 2 investigates if the effect of increasing K holds at different traffic loads. 


\begin{figure*}[p]
  \includegraphics[width=\textwidth]{IMAGES/heuristic_comparison.png}
  \caption{Service blocking probability (SBP) for heuristics at fixed traffic and varying numbers of candidate paths (K). The mean and standard deviation (shaded area) are calculated from 3000 trials of 10,000 traffic requests per data point. Increasing K decreases the SBP for most heuristics. KSP-FF and FF-KSP give lowest SBP, depending on topology.}
  \label{fig:heur_comp}

  \vspace{1em}
  \includegraphics[width=0.97\textwidth]{IMAGES/k_traffic_comparison.png}
  \caption{Service blocking probability (SBP) for KSP-FF at varying traffic loads and K values. The mean and standard deviation (shaded area) are calculated from 3000 trials of 10,000 traffic requests per data point. For all traffic loads, increasing K decreases SBP for KSP-FF. Improvements saturate at high K.}
  \label{fig:k_traffic}

  \vspace{1em}
  \includegraphics[width=\textwidth]{IMAGES/heuristic_50_traffic_comparison.png}
  \caption{Service blocking probability (SBP) for heuristics at varying traffic load for K=50. The mean and standard deviation (shaded area) are calculated from 3000 trials of 10,000 traffic requests per data point. KSP-FF is the best heuristic for NSFNET and COST239. FF-KSP is best for USNET and JPN48.}
  \label{fig:heur_traffic}
\end{figure*}

\textbf{Experiment 2} results in Figure \ref{fig:k_traffic} show that, regardless of the traffic load, increasing K decreases the SBP, until SBP reaches a minimum and increasing K does not decrease SBP further. Across all topologies and traffic loads tested in our experiments, we found that SBP does not decrease significantly for K>50. For very high traffic (approximately equivalent to incremental loading), the value of K beyond which SBP does not continue to decrease can be much lower.

\textbf{Experiment 3} results in Figure \ref{fig:heur_traffic} show the variation of SBP with traffic load for each heuristic with K=50. We verified that at least 50 unique paths are possible for every node pair on our investigated topologies. We select K=50 on the basis of experiments 1 and 2. These results confirm the initial findings from Experiment 1 - that KSP-FF and KME-FF are the lowest blocking for NSFNET and COST239\footnotemark, while FF-KSP is better for USNET and JPN48, with an order of magnitude lower blocking probability on JPN48 compared to the next best heuristic.

\footnotetext{Although Figure \ref{fig:heur_traffic} shows KME-FF gives slightly lower blocking than KSP-FF at lower traffic, we prefer KSP-FF for benchmarking purposes because of its widespread use and its greater simplicity.}

% \begin{figure*}
%   \includegraphics[width=\textwidth]{IMAGES/heuristic_50_traffic_comparison.png}
%   \caption{Comparison of service blocking probability from heuristic algorithms at varying traffic load for K=50.}
%   \label{fig:heur_traffic}
% \end{figure*}


In summary, we highlight the generally strong performance of the KSP-FF and FF-KSP heuristics but do not draw conclusions as to which is best in general, and encourage thorough analysis to determine the strongest heuristic benchmark for a particular problem, as we have exemplified here. We also emphasize the importance of selecting optimal path ordering for heuristics, as this can have a significant impact on performance. We find \#hops ordering is superior to \#km for reduced blocking probability, as evidenced in Section \ref{sec:repro_main} Figure \ref{fig:repro}.



\section{Reproduction and benchmarking of previous work}
\label{sec:repro_main}

As discussed in our literature review (Section \ref{sec:survey}), it is difficult to assess progress in the field due to several factors, particularly the diversity of problem definitions and use of weak benchmarks. To address this, we exactly recreate the problem settings from five influential papers from the literature, and apply the best-performing heuristics from Section \ref{sec:heuristic_comparison} in each case.

In this section, we first provide analysis of holding time truncation, an implementation detail present in the DeepRMSA codebase that significantly affects the blocking probability. We then present the results of our reproductions of the selected papers and compare to the heuristics, which show superior performance to all of the published RL solutions.

We have corresponded with the authors of all of the selected papers to clarify details of their implementation and ensure that our reproductions exactly match all the relevant details of their problems.

%DeepRMSA, Reward-RMSA, and GCN-RMSA use the same original DeepRMSA codebase as the basis for their experiments. Consequently, they all feature service holding time truncation, discussed in section \ref{sec:holding_time}. Another common characteristic is they use directed graphs with dual fiber links i.e. each link comprises two fibers, one for each direction of propagation. This doubles the maximum potential capacity of the network compared to single-fiber links. MaskRSA and PtrNet-RSA model single-fiber links, which makes the problem less challenging for RL due to less sparse reward signals \cite{nevin_techniques_2022} (more frequent blocking). PtrNet-RSA considers fixed bandwidth requests (no distance-dependent modulation format).

 We use our high-performance simulation framework, \mbox{XLRON} \cite{doherty_xlron_2023}, for all experiments. It has demonstrated 10x faster execution on CPU and over 1000x faster when parallelized on GPU compared to optical-rl-gym \cite{doherty_xlron_2024}. This is possible due to its array-based data model and use of the JAX numerical array computing framework, that enables just-in-time compilation to accelerator hardware. It also offers a complete suite of unit tests for core functionality, making it reliable, and includes all the problem definitions of the selected papers. We use it for these reasons and for its simple command-line interface, which facilitates experiment automation and reproducibility.













% \subsection{Simulation warm-up}
% \label{sec:warmup}

% To exactly recreate the problems from the selected papers, we generate the same number of traffic requests as the original study. This includes the connection requests that are generated to pre-populate the network until the blocking probability reaches steady-state, a phenomenon which is well-known in discrete-event simulation \cite{banks_discrete-event_2005}. We call the period prior to steady-state the warm-up period. 

% Warm-up is important so that the blocking probability reaches its steady-state value and is not an underestimate. This is known as the simulation start-up problem or the initial transient \cite{white_problem_2009}. %When training an RL agent, warm-up may help prevent primacy bias \cite{nikishin_primacy_2022}, where models overfit to data seen early in training. Starting with an empty network, initial traffic requests are trivially allocated due to abundant resources, providing no useful optimization signal. Training on a congested network with both successful and failed allocations gives more meaningful training data from the start.

% While most papers do not explicitly discuss warm-up, all of the selected papers use it. We therefore analyze the minimum required warm-up period for different traffic levels.

% \subsubsection{Experiment setup}

% We simulate a non-blocking network to determine the maximum\footnotemark number of requests needed to reach a steady-state traffic load. For each traffic load from 50 to 1000 Erlangs in steps of 50, and for arrival rates from 5 to 25 in steps of 5, we conduct 500 simulation trials with unique random seeds.  For each trial, we estimate the number of requests at which steady state is reached using the MSER-5 method \cite{franklin_stationarity_2008} (Marginal Standard Error Rule with batch size 5). We apply the MSER-5 method as it has been shown to be superior to other methods such as the mean crossing rule \cite{white_problem_2009}.

% \footnotetext{A blocking network reaches steady-state in less traffic requests than a non-blocking one, therefore we consider a non-blocking network to estimate an upper bound on the required warm-up period before steady state.}


% \subsubsection{Results and discussion}

% \begin{figure}
%     \centering
%     \includegraphics[width=1\linewidth]{IMAGES/steady_state_boxplots.png}
%     \caption{No. of simulated requests necessary to reach the steady-state network traffic load specified on the x-axis, for a non-blocking network.}
%     \label{fig:traffic_warmup}
% \end{figure}

% Figure \ref{fig:traffic_warmup} displays results as box-and-whisker plots showing how many requests are needed to reach steady state for each traffic load. The whiskers extend to the most extreme data points within 1.5 times the inter-quartile range. We observe a linear relationship between traffic load and requests until steady state (warm-up period). We fit a line to the maximum values (tops of the whiskers) to determine the warm-up period with high confidence. The fitted line indicates that approximately 7 times the target traffic load in requests are required to reach steady state. Based on this finding, the 3000-request warm-up period used by DeepRMSA that we adopt in section \ref{sec:repro} is sufficient for our problem settings.



















\subsection{Holding time truncation}
\label{sec:holding_time}

DeepRMSA, Reward-RMSA, and GCN-RMSA use the same original DeepRMSA codebase as the basis for their experiments. This codebase includes a significant detail: the service holding time is resampled if the resulting value is more than twice the mean of the exponential probability density function (PDF). We refer to this detail as holding time truncation.  In order to recreate the problems from these papers, we analyze the effect of holding time truncation.


\subsubsection{Experiment setup}
To understand the effect of truncation on the traffic statistics, we define an inverse exponential PDF that is normalized to have unit mean. We take $10^{6}$ samples from the PDF, both with and without truncation, and calculate the mean of the resulting sample populations in both cases.

\subsubsection{Results and discussion}
Figure \ref{fig:truncation} compares histograms of service holding times with and without truncation. We define the bin width as 0.01 and normalize the count per bin to give a peak density of 1 without truncation. The truncated case shows a cutoff at twice the mean holding time. The vertical lines indicate the mean holding time for each case.

Holding time truncation reduces the mean by approximately 31\%. This results in 31\% lower traffic load. Therefore, papers that use the DeepRMSA codebase (including DeepRMSA, Reward-RMSA, and GCN-RMSA) evaluate their solutions at traffic loads 31\% lower than reported. This detail is not made explicit in the published papers. This finding highlights the challenges in making fair comparisons between papers, and the need for transparency in research code.

\begin{figure}
    \centering
    \includegraphics[width=0.9\linewidth]{IMAGES/truncation.png}
    \caption{Histogram of service holding holding times. The truncated distribution resamples the holding time when the sampled value exceeds $2*$mean. This reduces the mean holding time by 31\% compared to the exponential distribution.}
    \label{fig:truncation}
\end{figure}



%Cisco NCS 2000 Flex Spectrum Single Module ROADM and it's integrated pre-amplifier has a nominal noise figure of 5.5dB (for 24dB gain) or 11.7dB (for 12dB gain), 















\subsection{Reproduction of published results}
\label{sec:repro}
Our analysis of the best performing heuristics, of holding time truncation, and our correspondence with the authors enables us to benchmark the published results from the five selected influential papers. The aim of this comparison is to determine if any of the published RL solutions achieve lower SBP than the heuristics. %In order to do so, we recreate their simulations exactly and apply the KSP-FF heuristic algorithm with K=50 and paths ordered by number of hops in each case, based on our experiments from section \ref{sec:heuristic_comparison}.


\subsubsection{Experiment methodology}


We recreate the problems from each selected paper in our own simulation framework \cite{doherty_xlron_2024}. We match the topologies (NSFNET, COST239, JPN48, USNET), mean service arrival rates, mean service holding times, data-rate or bandwidth request distributions, and uniform traffic matrices. We use the same measurement methodology as described in the respective papers to reproduce results, which is 3000 request warm-up period (to allow the network blocking probability to reach steady-state after the 'initial transient' \cite{white_problem_2009}) followed by 10,000 requests. The SBP is calculated at the end of the episode. We run 10 independent episodes at each traffic load per problem and calculate the mean and standard deviation across trials.

We extract published results for KSP-FF and RL solutions from the papers, using textual values where available otherwise reading from charts. All published results report only a single data point for each traffic value, without any uncertainty. %We provide all of the extracted data and our results in Appendix \ref{appendix:A}.

We check that our results for KSP-FF with K=5 (green line in Figure \ref{fig:repro}) match the published results for KSP-FF (blue line in Figure \ref{fig:repro}) within two standard deviations to ensure faithful reproduction. This comparison gives us a high degree of confidence that we have exactly recreated each problem setting.



\subsubsection{Results and discussion}


\begin{figure*}[ht]
  \includegraphics[width=1.01\textwidth]{IMAGES/review_summary_plots_2.png}
  \caption{Mean SBP against traffic load. Each column is a publication and each subplot is for a topology. Error bars and shaded areas show standard deviations. %Data for $RL$ and 5-SP-FF$_{published}$ are extracted from the publications. Close agreement between 5-SP-FF$_{published}$ and  5-SP-FF$_{ours}$ show that we accurately reproduce each case of study.
  \mbox{50-SP-FF$_{hops}$} exceeds or matches the $RL$ performance for each case.}
  \label{fig:repro}
\end{figure*}

Figure \ref{fig:repro} shows our reproduction of results from the selected papers, with SBP against traffic load in Erlangs in each subplot. The plots are organized by paper (columns) and topology (rows). PtrNet-RSA has two columns reflecting its two test cases: networks with 40 FSU per link and 1 FSU requests, and networks with 80 FSU per link and 1-4 FSU requests. PtrNet-RSA only considers fixed-bandwidth requests (no distance-dependent modulation format).

Each plot contains 5 datasets:
\begin{itemize}[itemsep=0pt]
\item[] \textbf{RL}: Published results for the RL approach
\item[] \textbf{5-SP-FF$_{published}$}: Published results for KSP-FF (K=5) with paths ordered by \#km
\item[] \textbf{5-SP-FF$_{ours}$}: Our results for KSP-FF (K=5) with paths ordered by \#km
\item[] \textbf{5-SP-FF$_{hops}$}: Our results for KSP-FF (K=5) with paths ordered by \#hops
\item[] \textbf{50-SP-FF$_{hops}$}: Our results for KSP-FF (K=50) with paths ordered by \#hops
\end{itemize}

Points show mean values from our simulations, with shaded areas indicating standard deviation and lines interpolating between points. The DeepRMSA paper provides data for only one traffic load per topology. The excellent agreement between 5-SP-FF$_{published}$ and 5-SP-FF$_{ours}$ in all cases confirms that our framework accurately reproduces the published scenarios. %We can therefore make comparisons between our results on these problem settings and the published RL results with high confidence.


From Figure \ref{fig:repro}, we highlight the comparisons of 'RL' (red) with 5-SP-FF$_{hops}$ (orange), and 50-SP-FF$_{hops}$ (purple). 5-SP-FF$_{hops}$ reduces the blocking probability by up to an order of magnitude compared to RL in all cases for NSFNET, 4/5 cases for COST239 and 1/3 cases for USNET. This shows that ordering paths by \#hops is sufficient to beat the RL results in these cases.

For larger topologies, considering more candidate paths (K>5) improves the heuristic performance significantly, often by over an order of magnitude. As shown by Figure \ref{fig:repro}, 50-SP-FF$_{hops}$ gives the lowest SBP of all approaches in all cases, except the bottom right.

The single exception where RL outperforms 50-SP-FF$_{hops}$ is PtrNet-RSA-80 USNET. We consider it plausible that the pointer-net architecture is a contributing factor to this strong performance, as it is not limited to selecting from a pre-defined set of paths. However, as the published results in this case fall within one standard deviations of the mean for 50-SP-FF$_{hops}$, the result could be spurious. This highlights the need for summary statistics and confidence intervals from multiple trials to be included with published results.

In summary, the results show that making minor changes (ordering paths by \#hops and considering more paths)  to simple heuristic algorithms is sufficient to achieve lower blocking probability than sophisticated RL solutions that have been published.

We highlight that this analysis, and the selected papers, focus on SBP as the optimization objective. In realistic scenarios, network blocking or throughput must be balanced with other metrics such as latency and cost of operation from transceiver launch power, amplifiers, and other network elements. Future research should therefore focus on problems that take a holistic approach to network operations optimization with multiple objectives \cite{nallaperuma_interpreting_2023}, and incorporate sophisticated models of all physical layer effects for improved accuracy \cite{curri_gnpy_2022,buglia_closed-form_2023}.

%While our earlier experiments showed FF-KSP performs better than KSP-FF for JPN48, we only plot KSP-FF results as they already surpass the RL performance. For MaskRSA JPN48, FF-KSP with K=50 achieves zero blocking across all tested traffic loads.

% Could include a summary paragrph here and/or a note about how RL could do better but it needs to consider enough candidate paths and the set of candidate paths needs to prioritise those with less hops not just shorter distance.
%We note that it is theoretically possible for RL to equal or exceed the performance of the best heuristic algorithms. However, our results suggest that it is necessary to consider sufficiently diverse paths 

All data shown in Figure \ref{fig:repro} is provided in tabular form in Appendix A.



\section{Network blocking bounds}
\label{sec:bounds}

We have demonstrated in Section \ref{sec:repro_main} that many influential works on RL for DRA problems in optical networks have failed to improve on a simple heuristic algorithm. The extent to which it is possible to reduce the blocking probability, and increase supported traffic, is an important motivating factor in any future research into this topic. 

To understand the limits of blocking probability, we derive empirical lower bounds. By comparing these lower bounds to the performance of our best solution for a target SBP, we can estimate the additional traffic load that can be supported and, therefore, the maximum benefit from applying an intelligent resource allocation method such as RL. %We term this additional capacity the "optimality gap", as we consider our lower bound to be approximately optimal.

As discussed in section \ref{sec:background}, DRA problems in optical networks that require RSA are subject to at least three constraints: spectrum continuity, spectrum contiguity, and no reconfiguration. By relaxing any of these constraints, the optimal or near-optimal solution of the relaxed problem is a bound on the solution of the full problem. The cut-sets bound method of Cruzado et al \cite{cruzado_effective_2023,cruzado_capacity-bound_2024} relaxes the spectrum continuity constraint and uses insights from the min-cut max-flow theorem to estimate a lower bound SBP. We instead relax the constraint on reconfiguring already-established connections, a process known as defragmentation.

We couple this defragmentation with resource prioritization: sorting the active connection requests by their required resources and allocating them sequentially. The sorting of active requests in descending order of required resources was found to improve the achievable capacity to optimal or near-optimal by Baroni \cite{baroni_routing_1998} in static RWA and later Beghelli \cite{beghelli_resource_2006} for dynamic RWA, a method they refer to as 'reconfigurable routing'. Since our problem settings are elastic optical networks, we prefer the term defragmentation. The intuition behind this approach is to allocate larger requests first so that requests with shorter paths and lower spectral requirements may be squeezed into remaining spectral gaps later.







\subsection*{Resource-Prioritized Defragmentation}
\label{sec:bounds}


\begin{algorithm}
\caption{Resource-Prioritized Defragmentation Blocking Bound Estimation}
\begin{algorithmic}[1]
\Require Network topology $G$, Set of requests $\mathcal{R}$, Number of frequency slots per link $F$
\Ensure Blocking probability $P_b$
\State $N \gets \textsc{InitializeNetwork}(G, F)$ \Comment{Initialize network with empty spectrum slots}
\State $\textit{blocked} \gets 0$
\For{$t \gets 1$ to $|\mathcal{R}|$}
    \State $N \gets \textsc{RemoveExpiredRequests}(N, t)$
    \State $r_t \gets \textit{current request from } \mathcal{R}$
    \State $\textit{success} \gets \textsc{AllocateRequest}(N, r_t)$
    
    \If{not $\textit{success}$}
        \State $\textit{active\_requests} \gets \textsc{GetActiveRequests}(\mathcal{R}, t)$
        \State $\textit{sorted\_requests} \gets \textsc{SortByResource}(\textit{active\_requests})$
        \State $N_{temp} \gets \textsc{InitializeNetwork}(G, F)$
        \State $\textit{blocking} \gets \texttt{false}$
        
        \For{$r \in \textit{sorted\_requests}$}
            \State $\textit{success} \gets \textsc{AllocateRequest}(N_{temp}, r)$
            \If{not $\textit{success}$}
                \State $\textit{blocking} \gets \texttt{true}$
                \State \textbf{break}
            \EndIf
        \EndFor
        
        \If{not $\textit{blocking}$}
            \State $N \gets N_{temp}$
        \Else
            \State $\textit{blocked} \gets \textit{blocked} + 1$
        \EndIf
    \EndIf
\EndFor

\State \Return $\frac{\textit{blocked}}{|\mathcal{R}|}$
\end{algorithmic}
\label{algo:defrag}
\end{algorithm}

The resource-prioritized defragmentation blocking bound algorithm is outlined in Algorithm \ref{tab:blocking_probabilities}. It utilizes four key subroutines:

\begin{itemize}
    \item \textsc{RemoveExpiredRequests}($N$, $t$) maintains network state by removing connections that have expeired. For current time $t$, and request with arrival time $t_{\text{arrival}}$ and holding time $t_{\text{holding}}$, the expiry condition is defined as: $t_{\text{arrival}} + t_{\text{holding}} < t$.
    
    \item \textsc{AllocateRequest}($N$, $request$) establishes a new connection subject to continuity and contiguity constraints. We use the KSP-FF or FF-KSP algorithm with K=50. We select the algorithm that produces the lowest SBP for the problem instance.
    
    \item \textsc{GetActiveRequests}($\mathcal{R}$, $t$) identifies requests where $t_{\text{arrival}} \leq t < t_{\text{arrival}} + t_{\text{holding}}$, determining which connections require reallocation during defragmentation.
    
    \item \textsc{SortByResource}($requests$) orders active requests by required resources (product of required spectral slots and hops of shortest path), prioritizing larger requests during reallocation to maximize the probability of finding viable configurations.
    
\end{itemize}

A shortcoming of our method of blocking bound estimation is its reliance on the internal \textsc{AllocateRequest} heuristic. To have confidence that the solution presents a true upper bound, the allocation method must be as close to optimal as possible. We therefore evaluate multiple heuristics for each case, as shown in Section \ref{sec:heuristic_comparison}, and select the one with lowest SBP. We find the best performing heuristic is KSP-FF$_{hops}$ with K=50 for most cases, except MaskRSA JPN48 which is FF-KSP.


%Combined with resource prioritization (\textsc{SortByResource}) we assume the results are near-optimal, based on results from Baroni \cite{baroni_routing_1998}. 
An advantage of our method compared to cut-sets analysis is it computes an allocation that is guaranteed to be physically possible, as it relaxes the 'No Reconfiguration' constraint instead of the physical spectrum continuity constraint. Relaxing the 'No Reconfiguration' constraint makes Algorithm \ref{algo:defrag} omniscient (it has complete knowledge of requests to be allocated) rather than a strictly on-line algorithm, according to definitions from Awerbuch et al \cite{awerbuch_throughput-competitive_1993}. This gives Algorithm \ref{algo:defrag} a fundamental competitive advantage over on-line algorithms like KSP-FF/FF-KSP, therefore it can be considered a lower bound estimator of blocking probability. % This disparity in information may however mean it is too much of an upper-bound estimate
%Overall, resource-prioritized defragmentation bounds can be considered complimentary to cut-sets bounds due to the difference in constraint relaxation.

We note that our algorithm is general and can be applied to any DRA problem in optical networks by using a strong heuristic for \textsc{AllocateRequest} and defining the resource-based sort criteria appropriately.






\subsection{Experiment setup}

For each problem from the five selected papers, we run the best performing heuristic for a range of traffic loads that result in SBP from 0.01\% to 1\%. For the lowest-blocking heuristic and for Algorithm \ref{algo:defrag}, we run 10 episodes of 10,000 requests with unique random seeds and calculate the mean and standard deviation of SBP across episodes. We calculate the mean and standard deviation SBP across episodes in each case. 

We compare the resulting SBP from the best heuristic and from algorithm \ref{algo:defrag}. We seek to estimate the additional network capacity that can be achieved at 0.1\% SBP for each case of study from the five selected papers. We select 0.1\% SBP to align with previous studies of network throughput estimation by Cruzado et al \cite{cruzado_effective_2023,cruzado_capacity-bound_2024}.

\subsection{Results and discussion}


\begin{figure*}[ht]
  \includegraphics[width=1.01\textwidth]{IMAGES/bounds.png}
  \caption{Mean SBP against traffic load for the lowest-blocking heuristic in each case (KSP-FF or FF-KSP with K=50) and the estimated bound from Algorithm \ref{algo:defrag}. Each column is a publication and each subplot is for a topology. Shaded areas show standard deviations. Red lines and text indicate relative increase in supported traffic at 0.1\% SBP from heuristic to bound.}
  \label{fig:bounds}
\end{figure*}


Similar to Figure \ref{fig:repro}, each subplot in Figure \ref{fig:repro} represents a different problem instance. DeepRMSA, Reward-RMSA, and GCN-RMSA are combined into a single set of plots since they use identical topologies and traffic models. The purple lines show the best performing heuristic in each case (KSP-FF with K=50, or FF-KSP for JPN48), with paths sorted in ascending order of number of hops. The grey lines show the resource-prioritized defragmentation bounds. At 0.1\% SBP, we compare the network traffic loads that can be supported in each case, with the difference highlighted by a red horizontal line. The relative increase in network capacity is calculated as the difference between the upper bound traffic load and the heuristic traffic load, as a percentage of the heuristic load.

PtrNet-RSA-40 shows differences of 5\%, 1\%, and 9\% across its three test cases. These relatively low values are due to the fixed width requests size of 1 FSU used in this case, which makes it equivalent to RWA and reduces the impact of fragmentation compared to RSA/RMSA.

For the Deep/Reward/GCN-RMSA, MaskRSA and PtrNet-RSA-80 cases, the difference between the supported traffic in the heuristic case and the upper bound ranges from 19\% (MaskRSA JPN48) to 36\% (Deep/Reward/GCN-RMSA NSFNET). These results show larger but comparable optimality gaps to those from the cut-sets method of Cruzado et al. \cite{cruzado_capacity-bound_2024}, who found gaps of 5\% to 16\% in their cases of study. This shows that defragmentation can unlock significant network capacity, but it is unknown theoretically how close an intelligent online allocation method, such as RL, can come to this bound. This will be the subject of future research.


\section{Conclusion and future directions} \label{sec:conclusion}

In this paper we proposed a nested MLMC framework that offers important computational savings by performing most calculations in low precision and exploiting approximate random normal variables for the low precision path calculations. The low precision calculations could be performed in fixed precision on an FPGA for greater efficiency, and we suggested a procedure to optimise the bit-widths of every variable at each Monte Carlo level. This is an important improvement over previous mixed precision MLMC frameworks which held the lower precision fixed \cite{Rounding_error_oliver} or defined uniform bit-width at every level heuristically \cite{brugger2014mixed}. Our numerical results suggest that for the first levels our procedure reduces the cost at these levels by a factor 5 or 7. Hence the overall savings are significant since most paths are calculated on the first levels. Our approach would be even more efficient for the Milstein scheme because its higher order strong convergence leads to a greater proportion of the computational costs being on the coarsest levels.

The next stage of the research project will be to implement the RNG methods and the nested framework on FPGAs to determine the hardware requirements and confirm the extent of the computational savings. It would also be good to compare the performance benefits to using half-precision floating point arithmetic on GPUs or CPUs for the low-accuracy computations.





\section*{Acknowledgments}
This work was supported by the Engineering and Physical Sciences Research Council (EPSRC) grant EP/S022139/1 - the Centre for Doctoral Training in Connected Electronic and Photonic Systems - and EPSRC Programme Grant TRANSNET EP/R035342/1. In addition, Polina Bayvel is supported through a Royal Society Research Professorship.


% Bibliography
\bibliography{references.bib}








\begin{table*}[tb]
    \centering
    \begin{tabular}{l | c c c | c c c} \toprule
        \multirow{2}{*}{\textbf{ Differential Diagnosis}} & \multicolumn{3}{c|}{\textbf{gpt-4o test set (n=3403)}} & \multicolumn{3}{c}{\textbf{claude test set (n=2868)}} \\ \cmidrule(r){2-4} \cmidrule(l){5-7}
        & \textbf{Top-5} & \textbf{Top -1} & \textbf{MRR} & \textbf{Top-5} & \textbf{Top -1} & \textbf{MRR} \\ \midrule
        \textbf{baseline} & 56.80\% & 28.65\% & 0.390 & 56.69\% & 30.65\% & 0.406 \\ 
        \textbf{gpt-4o rare candidates} & 52.66\% & 25.95\% & 0.357 & 55.47\% & 29.04\% & 0.388 \\ 
        \textbf{\methodname candidates} & \textbf{74.38\%} & \textbf{33.12\%} & \textbf{0.471} & \textbf{71.41\%} & \textbf{33.23\%} & \textbf{0.461} \\ \bottomrule
    \end{tabular}
    \caption{Performance on generated gpt-4o ddx task. All metrics for \methodname on both datasets (see bolded) are significant using a two-sided Wilcoxon signed-rank test with $p<0.01$ compared to the no candidates baseline.}
    \label{tab:ddx}
\end{table*}

\begin{table*}[tb]
\centering
\begin{tabular}{l|cccccc} \toprule
\textbf{DDx LLM} & \textbf{Exact} & \textbf{Extremely Rel.} & \textbf{Relevant} & \textbf{Somewhat Rel.} & \textbf{Unrelated} \\ 
\midrule
\textbf{baseline gpt-4o} & 22.8\% & 19.9\% & 4.9\% & 21.0\% & 31.3\% \\ 
\textbf{\methodname gpt-4o} & 55.8\% & 8.8\% & 2.3\% & 12.8\% & 20.2\% \\ \midrule

\textbf{baseline claude} & 19.2\% & 16.9\% & 3.9\% & 14.5\% &45.6\%  \\ 
\textbf{\methodname claude} & 56.8\% & 10.7\% & 1.6\% & 10.6\% & 20.4\% \\ \midrule

\textbf{baseline Llama 3.3 70b} & 20.3\% & 19.3\% & 5.3\% & 21.7\% & 33.5\% \\ 
\textbf{\methodname Llama 3,3 70b} & 47.3\% & 12.2\% & 3.3\% & 15.4\% & 21.9\% \\ \bottomrule
\end{tabular}
\caption{We compare LLM baseline DDx generation performance to LLMs with addition of \methodname candidates.  We report the LLM as judge results across several categories of similarity, ranging from Exact Match to Unrelated. We combine gpt-4o and claude test sets for this analysis.}
\label{tab:ddx_by_sim}
\end{table*}


\begin{table*}[tb]
    \centering
    \begin{tabular}{l | c | c c c | c c c}
        \toprule
        \multirow{2}{*}{\textbf{Training Dataset}} & \multirow{2}{*}{\textbf{Training Size}} & \multicolumn{3}{c|}{\textbf{gpt-4o test set (n=3403)}} & \multicolumn{3}{c}{\textbf{claude test set (n=2868)}} \\ \cmidrule(r){3-5} \cmidrule(l){6-8}
         & & \textbf{Top-5} & \textbf{Top-1} & \textbf{MRR} & \textbf{Top-5} & \textbf{Top-1} & \textbf{MRR} \\ \midrule
        \textbf{claude} & 8837 & 48.37\% & 34.12\% & 0.4007 & 64.92\% & 45.64\% & 0.5371 \\ 
        \textbf{gpt-4o} & 21782 & 88.04\% & 63.88\% & 0.7410 & 44.18\% & 28.45\% & 0.3490 \\ 
        \textbf{gpt-4o downsampled} & 8813 & 70.88\% & 47.90\% & 0.5742 & 37.20\% & 23.25\% & 0.2884 \\ 
        \textbf{gpt-4o + claude} & 30619 & 88.80\% & 64.21\% & 0.7463 & 77.82\% & 56.35\% & 0.6526 \\ 
        \bottomrule
    \end{tabular}
    \caption{Evaluation on the candidate generation task, with MRR, Top-5 and Top-1 Accuracy.  We evaluate on models only trained on claude data, gpt-4o data, and both, and evaluate separately on claude and gpt-4o test sets. We include a model trained on a downsampled set of gpt-4o data that approximates the size of the claude training set.}

    \label{tab:cand_gen}
\end{table*}



\end{document}
