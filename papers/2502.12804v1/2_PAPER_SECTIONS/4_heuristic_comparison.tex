\section{Heuristic algorithm benchmark evaluation}
\label{sec:heuristic_comparison}

To evaluate the results from our selected papers in section \ref{sec:paper_summaries}, we must determine the best (lowest blocking probability) heuristic algorithms to use as benchmarks. In this section, we present comparisons of the heuristics listed in Table \ref{tab:heuristics}, evaluated on different traffic loads, topologies, and considering different numbers of candidate paths (K). On the basis of this analysis, we select the benchmarks to apply in section \ref{sec:repro_main}.

We select the algorithms in Table \ref{tab:heuristics} because they are commonly used as benchmarks or have been reported as superior to other heuristics.


\begin{table}[h]
\begin{tabular}{l|l|l}
Heuristic                         & Acronym & Reference                \\ \hline
K-Shortest Paths First-Fit        & KSP-FF  & \cite{vincent_scalable_2019} \\
First-Fit K-Shortest Paths        & FF-KSP  & \cite{vincent_scalable_2019} \\
K-Shortest Paths Best-Fit         & KSP-BF  & \cite{abkenar_best_2016} \\
Best-Fit K-Shortest Paths         & BF-KSP  & \cite{abkenar_best_2016} \\
K-Minimum Entropy First-Fit       & KME-FF  & \cite{wright_minimum-_2015} \\
%K-Minimum Cut First-Fit           & KMC-FF  & \cite{tang_heuristic_2022} \\
%K-Minimum Frag First-Fit & KMF-FF  & \cite{tang_heuristic_2022} \\
K-Congestion Aware First-Fit      & KCA-FF  & CA2 from \cite{savory_congestion_2014}
\end{tabular}
\caption{RMSA heuristics used for benchmarking.}
\label{tab:heuristics}
\end{table}

%We also evaluated the K-Minimum Cut First-Fit (KMC-FF) and K-Minimum Frag First-Fit (KMF-FF) heuristics from Reward-RMSA \cite{tang_heuristic_2022}. These heuristics produce higher blocking probability than others when paths are ordered by number of hops. We exclude these results to improve the clarity of Figures \ref{fig:heur_comp}, \ref{fig:k_traffic}, and \ref{fig:heur_traffic}.

\begin{figure}
  \includegraphics[width=1.01\linewidth]{IMAGES/networks_plots_short.png}
  \caption{Network topologies used in our case studies from: DeepRMSA, Reward-RMSA, GCN-RMSA, MaskRSA, PtrNet-RSA \cite{chen_deeprmsa_2019} \cite{tang_heuristic_2022} \cite{xu_deep_2022} \cite{shimoda_mask_2021} \cite{cheng_ptrnet-rsa_2024}. We note that the USNET topology differs between GCN-RMSA and PtrNet-RSA. We show the GCN-RMSA version here. PtrNet-RSA also uses a variant of the COST239 topology.} %The COST239 topology shown is from DeepRMSA, the USNET is from PtrNet-RSA.}
  \label{fig:network_plots}
\end{figure}

Figure \ref{fig:network_plots} shows the topologies used in the selected papers. We use these topologies to analyze the performance of the heuristic algorithms and in our reproductions of the papers' problems in section \ref{sec:repro_main}. %Some characteristic features of the topologies are summarised in table \ref{tab:topology_features}. 
We make all topology data available in our open source codebase \cite{michael_doherty_2024_jocn_xlron_2024}.



\subsection{Effect of path ordering}

All the heuristics in Table \ref{tab:heuristics} select from the available pre-computed paths on the basis of sort criteria. The primary criterion may be a measure of the path congestion (KCA-FF), spectral fragmentation (KME-FF), or length (KSP-FF). In the event of multiple paths with equal value, a default ordering (usually ascending order of length) determines the selected path.

Usually, path length is considered as distance in km. However, we find that considering path length as number of hops (with length in km as a secondary sort criterion), significantly improves the performance of the heuristics. This has been observed previously by Baroni \cite{baroni_routing_1998}, who referred to it as Minimum Number of Hops routing (MNH). The intuitive explanation for this is that, if two paths can support the same order of modulation format, the path that comprises fewer links occupies fewer spectral resources. 

We refer to these two orderings as path length in km (\#km) or path length in number of hops (\#hops). Our comparisons of KSP-FF for these two orderings in Section \ref{sec:repro} Figure \ref{fig:repro} evidence the reduction in blocking probability from \#hops ordering. In our comparisons of heuristics in the next section, we use \#hops ordering.



\subsection{Simulation setup}

For each heuristic and topology, we carried out three simulation scenarios to investigate the effects of varying traffic loads and values of K on the relative blocking performance of the heuristics. 

\vspace{0.1cm}

\noindent \textbf{Experiment 1} - Increasing K:
\par \noindent \textbf{Aim}: Investigate relative performance of heuristics with increasing K. \textbf{Method}: Record service blocking probability (SBP) for each heuristic at values of K ranging from 2 to 26 at fixed traffic load. We arbitrarily select the traffic load for each topology so that the heuristics give a SBP of  $\sim$1\%.

\vspace{0.1cm}

\noindent \textbf{Experiment 2} - Increasing K at high to low traffic: 
\par \noindent \textbf{Aim}: Investigate the effect of increasing K at different traffic loads. \textbf{Method}: Record SBP at K ranging from 2 to 40 for a range of traffic loads. We select the traffic loads for each topology such that they result in 10$^{-5}$ to 10$^{-1}$ SBP. To simplify the analysis and plots, we only present results for KSP-FF.

\vspace{0.1cm}

\noindent \textbf{Experiment 3} - Increasing traffic load at K=50: 
\par \noindent \textbf{Aim}: Using the findings from Experiments 1 and 2, determine the lowest-blocking heuristic with optimized K-value across traffic loads. \textbf{Method}: Record SBP for high K (K=50) at varying traffic loads. We select the traffic loads for each topology such that they result in a range of SBP (10$^{-5}$ to 10$^{-1}$). This experiment provides evidence on which heuristic is the best overall for each topology.

\vspace{0.1cm}

The data for each heuristic in each experiment was collected from 3000 independent trials with unique random seeds. The SBP was calculated after 10,000 connection requests, with the mean and standard deviation calculated across trials. Each data point in Figures \ref{fig:heur_comp},\ref{fig:k_traffic},\ref{fig:heur_traffic} therefore shows summary statistics from 30 million connection requests, which gives high confidence that our results present the true mean and standard deviation in each case.

We considered dynamic traffic with fixed mean service holding time at 10 units. We considered the same traffic model and other settings as DeepRMSA\footnotemark: uniform traffic probability between each node pair, Poissonian arrival and departure statistics, uniform random selection of data rate from 25 to 100Gbps in 1Gbps intervals, and distance-dependent modulation formats from BPSK to 16QAM. We consider topologies with dual fibre links (one for each direction of propagation), 12.5GHz FSU width, and 100 FSU per fibre.

\footnotetext{We consider the DeepRMSA problem settings in these experiments because it is used by most of the papers presented in Section \ref{sec:repro_main}.} %For the same reason, we report our results in terms of SBP, despite our recommendation to use BBP in Section \ref{sec:recommendations}.}

\subsection{Results and discussion}

\textbf{Experiment 1} results in Figure \ref{fig:heur_comp} show different outcomes for smaller networks (NSFNET and COST239) and larger networks (USNET and JPN48). For NSFNET and COST239, KSP-FF and KME-FF are approximately equal and give the lowest blocking. Their blocking decreases to a minimum for approximately K=23 and above for NSFNET and continues to decrease for K>26 for COST239.

For USNET and JPN48, FF-KSP is clearly the best heuristic, with blocking reduced by half for JPN48. Blocking from FF-KSP decreases with K until K=26 for USNET and continues dropping sharply for K>26 for JPN48. For USNET, KSP-FF and KME-FF become competitive with FF-KSP at large K. It can be argued that FF-KSP performs better in larger networks where there are multiple roughly equivalent paths between source and destination, and dense packing of utilized wavelengths increases in relative importance to path selection.

The results from Experiment 1 indicate that KSP-FF and FF-KSP generally give the lowest blocking, depending on the network topology, and blocking decreases monotonically with increasing K. This experiment looked at a moderately high traffic load ($\sim$1\% SBP), therefore experiment 2 investigates if the effect of increasing K holds at different traffic loads. 


\begin{figure*}[p]
  \includegraphics[width=\textwidth]{IMAGES/heuristic_comparison.png}
  \caption{Service blocking probability (SBP) for heuristics at fixed traffic and varying numbers of candidate paths (K). The mean and standard deviation (shaded area) are calculated from 3000 trials of 10,000 traffic requests per data point. Increasing K decreases the SBP for most heuristics. KSP-FF and FF-KSP give lowest SBP, depending on topology.}
  \label{fig:heur_comp}

  \vspace{1em}
  \includegraphics[width=0.97\textwidth]{IMAGES/k_traffic_comparison.png}
  \caption{Service blocking probability (SBP) for KSP-FF at varying traffic loads and K values. The mean and standard deviation (shaded area) are calculated from 3000 trials of 10,000 traffic requests per data point. For all traffic loads, increasing K decreases SBP for KSP-FF. Improvements saturate at high K.}
  \label{fig:k_traffic}

  \vspace{1em}
  \includegraphics[width=\textwidth]{IMAGES/heuristic_50_traffic_comparison.png}
  \caption{Service blocking probability (SBP) for heuristics at varying traffic load for K=50. The mean and standard deviation (shaded area) are calculated from 3000 trials of 10,000 traffic requests per data point. KSP-FF is the best heuristic for NSFNET and COST239. FF-KSP is best for USNET and JPN48.}
  \label{fig:heur_traffic}
\end{figure*}

\textbf{Experiment 2} results in Figure \ref{fig:k_traffic} show that, regardless of the traffic load, increasing K decreases the SBP, until SBP reaches a minimum and increasing K does not decrease SBP further. Across all topologies and traffic loads tested in our experiments, we found that SBP does not decrease significantly for K>50. For very high traffic (approximately equivalent to incremental loading), the value of K beyond which SBP does not continue to decrease can be much lower.

\textbf{Experiment 3} results in Figure \ref{fig:heur_traffic} show the variation of SBP with traffic load for each heuristic with K=50. We verified that at least 50 unique paths are possible for every node pair on our investigated topologies. We select K=50 on the basis of experiments 1 and 2. These results confirm the initial findings from Experiment 1 - that KSP-FF and KME-FF are the lowest blocking for NSFNET and COST239\footnotemark, while FF-KSP is better for USNET and JPN48, with an order of magnitude lower blocking probability on JPN48 compared to the next best heuristic.

\footnotetext{Although Figure \ref{fig:heur_traffic} shows KME-FF gives slightly lower blocking than KSP-FF at lower traffic, we prefer KSP-FF for benchmarking purposes because of its widespread use and its greater simplicity.}

% \begin{figure*}
%   \includegraphics[width=\textwidth]{IMAGES/heuristic_50_traffic_comparison.png}
%   \caption{Comparison of service blocking probability from heuristic algorithms at varying traffic load for K=50.}
%   \label{fig:heur_traffic}
% \end{figure*}


In summary, we highlight the generally strong performance of the KSP-FF and FF-KSP heuristics but do not draw conclusions as to which is best in general, and encourage thorough analysis to determine the strongest heuristic benchmark for a particular problem, as we have exemplified here. We also emphasize the importance of selecting optimal path ordering for heuristics, as this can have a significant impact on performance. We find \#hops ordering is superior to \#km for reduced blocking probability, as evidenced in Section \ref{sec:repro_main} Figure \ref{fig:repro}.

