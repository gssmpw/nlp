\section{Conclusion}
\label{sec:conclusion}


Our review of the field of RL applied to DRA problems in optical networks shows that it has been the subject of significant research interest, with almost 100 peer-reviewed papers published so far. Many technical innovations from ML research, such as improved RL algorithms \cite{mnih_asynchronous_2016}, invalid action masking \cite{shimoda_mask_2021,nevin_techniques_2022,cheng_ptrnet-rsa_2024}, and GNNs \cite{xu_deep_2022,li_gnn-based_2022,xiong_graph_2024}, have been applied to the problem area and have demonstrated incremental improvements in network blocking.

However, the field has suffered from a lack of standardization in problems, unrigorous selection of benchmark algorithms, and poor practices for reproducibility. We have addressed these problems by developing a simulation framework that can recreate diverse network simulations and allow fast benchmarking of heuristic algorithms. We apply of this framework to five influential papers from the field and identify the best heuristics in all cases to be either KSP-FF or FF-KSP with K=50. We highlight the result that ordering the pre-computed shortest paths by number of hops gives significantly lower blocking probability than ordering by distance. %Our quantification of the effects on mean path length of reordering paths in this way show that the mean length is increased by around 20\% in most cases, which is likely an acceptable trade-off in exchange for greater network capacity. %We also quantify the effect of network warm-up, and make recommendations that the warm-up period should be at least 7x the target traffic load to allow stabilization before training or evaluating on dynamic network traffic.

Our most significant findings are in the benchmarking of published RL results. By extracting the published results of KSP-FF and RL from the selected papers, and using the best heuristic benchmarks, we show that simple heuristics can exceed or match the results from RL in all cases, often with over an order of magnitude lower service blocking probability. This shows the relative performance of RL on these problems has been overestimated due to weak benchmarks, and highlights the need for more rigorous standards for evaluation in this area to avoid trivial results.

Finally, to ascertain the practical value of pursuing further research into these problems, we provide a method of estimating the lower bound network blocking probability, that we term Resource-Prioritized Defragmentation. Compared to the best heuristics available for each case, this method estimates the additional traffic load that can be supported on elastic networks is approximately 19\% to 36\%. Considering that data traffic is projected to increase at 20\% annually until 2029 \cite{ericsson_ab_ericsson_2024}, it is an open question if the additional percentage gains in network throughput are worth pursuing with RL, or if research into network optimization with RL should focus on other objectives for which there are not yet good heuristic solutions.


% Our method of capacity bound estimation is simple and scalable, and can be used by other researchers to estimate the quality of their solution. It is also adaptable to other DRA problems

% We show 2 important qunatitative results. 1) Heuristics give lower service blocking probability than Rl for all published works. 2)
