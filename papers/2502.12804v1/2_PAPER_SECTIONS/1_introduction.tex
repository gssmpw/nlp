\section{Introduction}

% Is the motivation for applying RL to increase capacity or to increase uatoation? Both! But more so dynamicity and automation.

Network operators are faced with a problem: how to increase capacity for ever-increasing data traffic without increasing the price of services \cite{lord_future_2021}. Advances in transmission technology have so far provided the solution by exponentially increasing the point-to-point capacity of the optical channel. However, as the throughput of the installed C+L bands approaches the nonlinearity-limited information bounds \cite{shtaif_information_2024}, costly infrastructure upgrades are required to scale capacity through spatial division multiplexing or ultra-wideband transmission \cite{winzer_future_2023}. Operators seek to minimize or delay the required capital expenditure and offset it with reduced operational costs. Online optimization of network resource allocation offers a path towards these aims by increasing the achievable network throughput with dynamic and automated service provisioning. 

Reinforcement learning (RL) has emerged as a promising technique for dynamic resource allocation (DRA) from a range of exact solution methods, heuristic algorithms, and artificial intelligence (AI) approaches.  RL solutions can approach the quality of exact methods such as integer linear programming (ILP), with an online allocation time comparable to simple heuristics \cite{di_cicco_deep_2022}. In fact, many works have demonstrated RL outperforms selected heuristics across various optical network problems %\cite{chen_deeprmsa_2019,chen_multi-task-learning-based_2021,xu_deep_2022} 
 (see section \ref{sec:survey}).

However, despite the many papers that investigate RL for optical networks, adoption of machine learning (ML) techniques by network operators has been limited by non-technological barriers \cite{khan_non-technological_2024}. One barrier is a lack of clear benchmarks and demonstrable benefits. Studies of resource allocation problems in optical networks often present results on specific topologies and traffic models, without fair comparison to previous results. Often results are not generalizable.

In this paper we address this barrier by providing analysis of and recommendations for benchmarking practices. We identify 5 papers that provide the best examples of benchmarking in the field and reproduce their described problems, including implementation details that significantly affect the results. By "problems", we mean the exact network topologies, traffic models, and other details pertinent to the network simulations. By recreating these problems, we demonstrate that simple heuristic algorithms outperform or equal the reported RL results in all cases. This is the first time that a thorough analysis, reproduction, and benchmarking of previous work has been carried out, and highlights deficient benchmarking standards. 

We also investigate the limits of network throughput for dynamic traffic, to understand just how much additional data traffic can be supported by the network through optimized resource allocation. We propose a heuristic-based lower bound network blocking estimation method, termed Resource-Prioritized Defragmentation (section \ref{sec:bounds}). For each case of study, we apply this method over a range of traffic loads and estimate that possible improvements in network throughput at fixed blocking (0.1\%) compared to the best heuristics are limited to 1-36\%.

This paper aims to: 1) Establish higher standards for evaluation and reproducibility in research into RL for optical networking. 2) Encourage new research directions by highlighting the limited room for improvement over simple heuristics in widely-investigated problems. 

The contributions of this work are:
\begin{enumerate}[itemsep=0pt]
    \item A comprehensive survey of benchmark progress on RL approaches to DRA in optical networks.
    \item A systematic study of the factors affecting heuristic algorithm performance.
    \item A reproduction of experiments from five landmark papers in the field with comparisons to improved heuristic benchmarks, which shows that RL results have failed to improve on heuristics in all cases.
    \item A novel empirical throughput bound estimation method is described and applied to estimate the upper bound network throughput, compared to benchmarks.
    \item The release of our simulation framework, "XLRON", to promote reproducible research and enable fair comparison across different studies.
\end{enumerate}

All of the code necessary to generate data and plots from this paper are available on Github: \cite{michael_doherty_micdohxlron_2024}.

We begin with essential background on DRA problems and RL techniques in Section 2, followed by a literature review in Section 3. Section 4 presents the investigation of heuristic algorithms for benchmarking. Section 5 details the analysis of the previously published results and comparison with benchmarks. Section 6 presents new empirical bounds for network throughput, with recommendations for future research directions in Section 7.






% These results have important implications for the optical networks research community. 
% Firstly, the need for good benchmarking practices is highlighted. Secondly, the relative performance of RL is shown to be overstated. Finally, the benefits of RL for the DRA problems analyzed in this paper are limited. Overall, this implies the community needs to improve its benchmarking practices and foster a culture of open source to help ensure that future research results are not spurious, and shift focus to problems with a sufficient optimality gap for which there do not exist already-strong heuristics.

% The contributions of this paper are:
% \begin{enumerate}
%     \item The first comprehensive survey of RL applied to DRA problems in optical networks.
%     \item Reproduction of previously published results with thorough benchmarks, that show RL performance to be inferior to simple heuristics.
%     \item New estimates of network capacity upper bounds show improvement over the best-performing heuristics is limited to 1-36\%.
%     \item We make publicly available "XLRON", our Python simulation library used to produce all results in this paper, for other researchers \cite{doherty_xlron_2023}.
% \end{enumerate}

% All of the code necessary to generate data and plots are available here \cite{michael_doherty_micdohxlron_2024}.

% %The overall contribution of the combined survey, reappraisal, and upper-bound estimation, is to provide guidance to the research community on fruitful future research directions. The performance of previous RL solutions, based on blocking probability, is surpassed by simple heuristics, therefore demonstrating superior RL performance remains an outstanding problem. However, the performance ceiling is estimated to already be close (XX-XX\%) for simple algorithms, which may prompt researchers to pursue other avenues to boost network throughput and efficiency.


% \noindent The rest of this paper is structured as follows. Section \ref{sec:background} provides the background required to understand DRA problems and RL techniques. Section \ref{sec:survey} discusses the papers reviewed and how they were selected. The reviewed works are categorized for discussion according to their contributions. Section \ref{sec:repro} focuses on 5 influential papers from the literature and our work to recreate their cases of study in simulation. We benchmark heuristics to determine the best one in each case, we describe and quantify implementation details for each case of study, and we present the results from benchmarking previous work. We then describe our empirical network capacity bound estimation method and present our estimates of the gap between our upper bound estimates and best performing heuristics. Finally, conclusions are drawn in Section \ref{sec:conclusion}.



% uggestions from Claude:
% Network operators face mounting pressure to increase capacity for ever-growing data traffic while maintaining constant service prices \cite{lord_future_2021}. While advances in transmission technology have historically met this challenge through exponential increases in fiber optic channel capacity, the approach of the non-linear Shannon limit for C+L bands \cite{shtaif_information_2024} now necessitates costly infrastructure upgrades \cite{winzer_future_2023}. Operators therefore seek to optimize their existing infrastructure through better resource allocation methods.
% Reinforcement learning (RL) has emerged as a heavily researched approach for dynamic resource allocation (DRA) in optical networks, offering solutions that match the quality of exact methods like integer linear programming while maintaining the fast allocation of simple heuristics. The field appears well-suited for RL given its large search space, clear optimization objectives, and ability to generate training data through simulation \cite{hassabis_nobel_2024}. This has led to numerous papers claiming superior performance over traditional methods.
% However, we identify critical issues in how this research has been conducted and evaluated. Through a comprehensive review of the literature and reproduction of results from five influential papers, we demonstrate that properly tuned heuristic algorithms consistently match or outperform the published RL approaches. Furthermore, using a novel defragmentation-based method, we establish theoretical upper bounds showing that potential improvements over current heuristics are limited to just 1-36% increased capacity.
% Our findings have profound implications for the field:

% The advantages of RL for these problems have been significantly overstated due to inadequate benchmarking
% Many supposed advances may be artifacts of poor experimental methodology
% The limited room for improvement suggests research efforts may be better directed elsewhere

% The key contributions of this work are:

% The first comprehensive survey and critical analysis of RL approaches to optical network DRA
% A unified simulation framework enabling fair comparison across different studies
% Rigorous reproduction of key results showing the superiority of simple heuristics
% Novel theoretical bounds quantifying the limited potential for improvement
% Release of our XLRON simulation library to promote reproducible research

% This paper aims to reset expectations in the field and establish higher standards for evaluation and reproducibility in optical networking research. We begin with essential background on DRA problems and RL techniques in Section 2, followed by our literature analysis in Section 3. Section 4 presents our detailed reproduction of previous results and new theoretical bounds. We conclude with recommendations for future research directions in Section 5.








%%%%%%%%%%%%%%%%%%%%%%%%%%%%%%%%



% Maybe keep the mention of SDN, dynamic operation, EON and low-margin operation for the background section. The emerging paradigms of software defined networking (SDN) and dynamic operation open a path to automated... EONS allow low-margin ioperation


% 60% of operating costs are from field and service operations (McKinsey report) but tbh that refers to call centre and customer support operations seemingly.


%The resource allocation problems that are presented by optical networks fall under the category of combinatorial optimisation, for which a range of exact and approximate solution methods exist \cite{bengio_machine_2020}. Artificial intelligence techniques, specifically the use of deep reinforcement learning (DRL), are a more recent candidate to solve these problems, as they offer faster online execution than linear programming methods and potentially superior optimisation performance than heuristic algorithms.


% Next I want to make the point that P2P capacity has increased exponentially and is widely considered to have approached the non-linear Shannon limit. Therefore optimistion of resource allocation to traffic flows becoming increasingly attractive. Dynamic operation, in which traffic flows are serviced on-demand, precludes the option of exact solution methods with exorbitant calculation times.  Previously ILP was able to find exact solution (can also cite Meta paper from ONDM). 
%Advances in DSP, coherent transceivers, amplifiers, ROADMs and other network components have allowed traffic  to increase 
%Consequently, increasingly sophisticated techniques have been proposed to efficiently allocate bandwidth on optical fibres to connect geographic sites. This problem, of routing and resource allocation, occurs in systems ranging from metro to inter-data centre, national and continental scale networks. % cite handbook of optcal networks? or a review paper

%The majority of published works focus on dynamic traffic, with small numbers for static and incremental traffic.

% dynamic, static, incremental
%Connection requests are static (known a priori), dynamic (stochastic arrival and departure), or incremental (stochastic arrival and no departure). 


%Useful review paper of RWA, RSA, RCSA \cite{zhang_overview_2020}


% \subsection*{Scope of work}
% For completeness, we include works that focus on both fixed-grid and flex-grid 



% Firstly, the need for good benchmarking practices is highlighted. A lack of standardization in problem settings and open source code has made it difficult to compare to previous approaches. This has allowed insufficient benchmarking practices to propagate in the literature. By thoroughly benchmarking against the best performing heuristics, RL fails to improve on heuristics in nearly all cases shown. The community therefore needs to adopt better practices to ensure future published results are not spurious.

% Secondly, the benefits of RL for the DRA problems analysed in this paper are limited. Our empirical upper bounds on network capacity suggest that the focus should be shifted to problems where the optimality gap of current heuristics is significant enough to merit the application of RL. Joint optimisation tasks, e.g. of SNR and network throughput, for which heuristics are not readily available, could be a more fruitful area.

%We identify the various problem settings that have been examined (including network topologies, available bandwidth, and traffic models) and the algorithms against which RL solutions have been benchmarked. 

%To test the robustness of the reported performance of RL solutions against common benchmark algorithms, we select 5 prominent papers from the literature 
% maybe provide motivation for the selected papers here
%\cite{chen_deeprmsa_2019,shimoda_mask_2021,tang_heuristic_2022,xu_deep_2022,cheng_ptrnet-rsa_2024} and recreate their cases of study. We first reproduce the reported results in each paper for the K-shortest paths first-fit (KSP-FF) heuristic, then compare with the published figures. As a product of this comparison, we identify implementation details that materially affect the results and quantify their impact. These details we refer to as network traffic warmup period (the amount of traffic used to pre-populate the network prior to training or evaluation) and the ordering of the pre-calculated shortest paths used by heuristics. By taking account of these details and optimising the path ordering, we demonstrate that simple heuristics outperform the reported RL results in all of the works we examine.

%To complete the analysis of examined problems, we estimate the upper-bound performance for each case of study with 2 techniques: 1) cut-set analysis \cite{cruzado_capacity-bound_2024} and 2) ordered request assignation. Cut-sets analysis relaxes constraints to provide a loose upper bound, while ordered request assignation provides a tighter bound. Together, the upper bounds show that the possible improvements in supported traffic levels for the cases of study are only XX-XX\%.
