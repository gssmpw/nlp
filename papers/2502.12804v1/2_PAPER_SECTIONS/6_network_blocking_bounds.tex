\section{Network blocking bounds}
\label{sec:bounds}

We have demonstrated in Section \ref{sec:repro_main} that many influential works on RL for DRA problems in optical networks have failed to improve on a simple heuristic algorithm. The extent to which it is possible to reduce the blocking probability, and increase supported traffic, is an important motivating factor in any future research into this topic. 

To understand the limits of blocking probability, we derive empirical lower bounds. By comparing these lower bounds to the performance of our best solution for a target SBP, we can estimate the additional traffic load that can be supported and, therefore, the maximum benefit from applying an intelligent resource allocation method such as RL. %We term this additional capacity the "optimality gap", as we consider our lower bound to be approximately optimal.

As discussed in section \ref{sec:background}, DRA problems in optical networks that require RSA are subject to at least three constraints: spectrum continuity, spectrum contiguity, and no reconfiguration. By relaxing any of these constraints, the optimal or near-optimal solution of the relaxed problem is a bound on the solution of the full problem. The cut-sets bound method of Cruzado et al \cite{cruzado_effective_2023,cruzado_capacity-bound_2024} relaxes the spectrum continuity constraint and uses insights from the min-cut max-flow theorem to estimate a lower bound SBP. We instead relax the constraint on reconfiguring already-established connections, a process known as defragmentation.

We couple this defragmentation with resource prioritization: sorting the active connection requests by their required resources and allocating them sequentially. The sorting of active requests in descending order of required resources was found to improve the achievable capacity to optimal or near-optimal by Baroni \cite{baroni_routing_1998} in static RWA and later Beghelli \cite{beghelli_resource_2006} for dynamic RWA, a method they refer to as 'reconfigurable routing'. Since our problem settings are elastic optical networks, we prefer the term defragmentation. The intuition behind this approach is to allocate larger requests first so that requests with shorter paths and lower spectral requirements may be squeezed into remaining spectral gaps later.







\subsection*{Resource-Prioritized Defragmentation}
\label{sec:bounds}


\begin{algorithm}
\caption{Resource-Prioritized Defragmentation Blocking Bound Estimation}
\begin{algorithmic}[1]
\Require Network topology $G$, Set of requests $\mathcal{R}$, Number of frequency slots per link $F$
\Ensure Blocking probability $P_b$
\State $N \gets \textsc{InitializeNetwork}(G, F)$ \Comment{Initialize network with empty spectrum slots}
\State $\textit{blocked} \gets 0$
\For{$t \gets 1$ to $|\mathcal{R}|$}
    \State $N \gets \textsc{RemoveExpiredRequests}(N, t)$
    \State $r_t \gets \textit{current request from } \mathcal{R}$
    \State $\textit{success} \gets \textsc{AllocateRequest}(N, r_t)$
    
    \If{not $\textit{success}$}
        \State $\textit{active\_requests} \gets \textsc{GetActiveRequests}(\mathcal{R}, t)$
        \State $\textit{sorted\_requests} \gets \textsc{SortByResource}(\textit{active\_requests})$
        \State $N_{temp} \gets \textsc{InitializeNetwork}(G, F)$
        \State $\textit{blocking} \gets \texttt{false}$
        
        \For{$r \in \textit{sorted\_requests}$}
            \State $\textit{success} \gets \textsc{AllocateRequest}(N_{temp}, r)$
            \If{not $\textit{success}$}
                \State $\textit{blocking} \gets \texttt{true}$
                \State \textbf{break}
            \EndIf
        \EndFor
        
        \If{not $\textit{blocking}$}
            \State $N \gets N_{temp}$
        \Else
            \State $\textit{blocked} \gets \textit{blocked} + 1$
        \EndIf
    \EndIf
\EndFor

\State \Return $\frac{\textit{blocked}}{|\mathcal{R}|}$
\end{algorithmic}
\label{algo:defrag}
\end{algorithm}

The resource-prioritized defragmentation blocking bound algorithm is outlined in Algorithm \ref{tab:blocking_probabilities}. It utilizes four key subroutines:

\begin{itemize}
    \item \textsc{RemoveExpiredRequests}($N$, $t$) maintains network state by removing connections that have expeired. For current time $t$, and request with arrival time $t_{\text{arrival}}$ and holding time $t_{\text{holding}}$, the expiry condition is defined as: $t_{\text{arrival}} + t_{\text{holding}} < t$.
    
    \item \textsc{AllocateRequest}($N$, $request$) establishes a new connection subject to continuity and contiguity constraints. We use the KSP-FF or FF-KSP algorithm with K=50. We select the algorithm that produces the lowest SBP for the problem instance.
    
    \item \textsc{GetActiveRequests}($\mathcal{R}$, $t$) identifies requests where $t_{\text{arrival}} \leq t < t_{\text{arrival}} + t_{\text{holding}}$, determining which connections require reallocation during defragmentation.
    
    \item \textsc{SortByResource}($requests$) orders active requests by required resources (product of required spectral slots and hops of shortest path), prioritizing larger requests during reallocation to maximize the probability of finding viable configurations.
    
\end{itemize}

A shortcoming of our method of blocking bound estimation is its reliance on the internal \textsc{AllocateRequest} heuristic. To have confidence that the solution presents a true upper bound, the allocation method must be as close to optimal as possible. We therefore evaluate multiple heuristics for each case, as shown in Section \ref{sec:heuristic_comparison}, and select the one with lowest SBP. We find the best performing heuristic is KSP-FF$_{hops}$ with K=50 for most cases, except MaskRSA JPN48 which is FF-KSP.


%Combined with resource prioritization (\textsc{SortByResource}) we assume the results are near-optimal, based on results from Baroni \cite{baroni_routing_1998}. 
An advantage of our method compared to cut-sets analysis is it computes an allocation that is guaranteed to be physically possible, as it relaxes the 'No Reconfiguration' constraint instead of the physical spectrum continuity constraint. Relaxing the 'No Reconfiguration' constraint makes Algorithm \ref{algo:defrag} omniscient (it has complete knowledge of requests to be allocated) rather than a strictly on-line algorithm, according to definitions from Awerbuch et al \cite{awerbuch_throughput-competitive_1993}. This gives Algorithm \ref{algo:defrag} a fundamental competitive advantage over on-line algorithms like KSP-FF/FF-KSP, therefore it can be considered a lower bound estimator of blocking probability. % This disparity in information may however mean it is too much of an upper-bound estimate
%Overall, resource-prioritized defragmentation bounds can be considered complimentary to cut-sets bounds due to the difference in constraint relaxation.

We note that our algorithm is general and can be applied to any DRA problem in optical networks by using a strong heuristic for \textsc{AllocateRequest} and defining the resource-based sort criteria appropriately.






\subsection{Experiment setup}

For each problem from the five selected papers, we run the best performing heuristic for a range of traffic loads that result in SBP from 0.01\% to 1\%. For the lowest-blocking heuristic and for Algorithm \ref{algo:defrag}, we run 10 episodes of 10,000 requests with unique random seeds and calculate the mean and standard deviation of SBP across episodes. We calculate the mean and standard deviation SBP across episodes in each case. 

We compare the resulting SBP from the best heuristic and from algorithm \ref{algo:defrag}. We seek to estimate the additional network capacity that can be achieved at 0.1\% SBP for each case of study from the five selected papers. We select 0.1\% SBP to align with previous studies of network throughput estimation by Cruzado et al \cite{cruzado_effective_2023,cruzado_capacity-bound_2024}.

\subsection{Results and discussion}


\begin{figure*}[ht]
  \includegraphics[width=1.01\textwidth]{IMAGES/bounds.png}
  \caption{Mean SBP against traffic load for the lowest-blocking heuristic in each case (KSP-FF or FF-KSP with K=50) and the estimated bound from Algorithm \ref{algo:defrag}. Each column is a publication and each subplot is for a topology. Shaded areas show standard deviations. Red lines and text indicate relative increase in supported traffic at 0.1\% SBP from heuristic to bound.}
  \label{fig:bounds}
\end{figure*}


Similar to Figure \ref{fig:repro}, each subplot in Figure \ref{fig:repro} represents a different problem instance. DeepRMSA, Reward-RMSA, and GCN-RMSA are combined into a single set of plots since they use identical topologies and traffic models. The purple lines show the best performing heuristic in each case (KSP-FF with K=50, or FF-KSP for JPN48), with paths sorted in ascending order of number of hops. The grey lines show the resource-prioritized defragmentation bounds. At 0.1\% SBP, we compare the network traffic loads that can be supported in each case, with the difference highlighted by a red horizontal line. The relative increase in network capacity is calculated as the difference between the upper bound traffic load and the heuristic traffic load, as a percentage of the heuristic load.

PtrNet-RSA-40 shows differences of 5\%, 1\%, and 9\% across its three test cases. These relatively low values are due to the fixed width requests size of 1 FSU used in this case, which makes it equivalent to RWA and reduces the impact of fragmentation compared to RSA/RMSA.

For the Deep/Reward/GCN-RMSA, MaskRSA and PtrNet-RSA-80 cases, the difference between the supported traffic in the heuristic case and the upper bound ranges from 19\% (MaskRSA JPN48) to 36\% (Deep/Reward/GCN-RMSA NSFNET). These results show larger but comparable optimality gaps to those from the cut-sets method of Cruzado et al. \cite{cruzado_capacity-bound_2024}, who found gaps of 5\% to 16\% in their cases of study. This shows that defragmentation can unlock significant network capacity, but it is unknown theoretically how close an intelligent online allocation method, such as RL, can come to this bound. This will be the subject of future research.
