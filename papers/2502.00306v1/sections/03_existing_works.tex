\section{Limitations of Existing Inference Attacks on RAG Systems}
\label{sec:existing_rag_inference}

% \quad
% \newpage
% \quad
% \newpage
% \quad
% \newpage
% \section{Detection}
% Data contamination occurs when benchmark data,
% $D_b$
%  is included in the training data $D$ of language models 
% $M$.
%  Detection methods can be classified into three types based on their focus: \textit{Training Data-Oriented}, \textit{Benchmark-Oriented} and \textit{Model Behavior-Oriented} methods.
% \subsection{Training Data-Oriented}
% Training data-oriented methods primarily assess the overlap between $D_b$ and $D$. This overlap is typically detected via direct n-gram matching at the token~\cite{touvron2023llama}, word~\cite{radford2019language,brown2020language,chowdhery2023palm}, character~\cite{achiam2023gpt}, or document chunk ~\cite{dodge2021documenting}. However, exact matches often result in false negatives. To mitigate this issue, subsequent research has explored more robust approaches, including embedding-based similarity measurements~\cite{riddell2024quantifying,lee2023platypus,gunasekar2023textbooks} and improved mapping metrics~\cite{li2024open,xu2024benchmarking}. Additionally, \citeauthor{yang2023rethinking}(\citeyear{yang2023rethinking}) highlight that minor data variations can evade prior detection methods and propose an LLM-based approach for assessing semantic similarity between training and test data. To further enhance detection, several search tools~\cite{piktus2023roots,piktus2023gaia,elazar2023s} have been developed for large-scale corpus overlap analysis.

% \subsection{Benchmark-Oriented}
% Benchmark-oriented methods frame contamination detection as a memorization problem, assuming models can recall benchmark data or key information encountered during training. Based on their approach, these methods can be categorized into original data methods and variant data methods.
% Original data methods assess memorization by directly manipulating benchmark data, such as masking specific parts (e.g., context~\cite{ranaldi2024investigating,chang2023speak}, choices~\cite{deng2024investigating,liu2024evaluating}, or labels~\cite{magar2022data}), or requiring the model to continue generation with partial suffix provided~\cite{anil2023palm,xu2024benchmarking,golchin2024timetravelllmstracing}.
% Variant data methods analyze contamination by introducing modified test instances and evaluating the model’s preference for different variations. \citeauthor{duartecop}~(\citeyear{duartecop}) and \citeauthor{golchin2023data}~(\citeyear{golchin2023data}) pair test instances with paraphrased versions to determine whether the model exhibits a preference for exact verbatim ones from the test set. Similarly, \citeauthor{zong2024fool}~(\citeyear{zong2024fool}) shuffle the positions of answer choices to assess whether performance drops.
% \subsection{Model Behavior-Oriented}
% Model behavior-oriented methods identify contamination by examining output probabilities, confidence distributions, or model performance across diverse experimental settings. These approaches often require white-box access to retrieve the model's architecture and internal weights~\cite{ravaut2024much}. Building on the assumption that seen instances exhibit higher probabilities than unseen ones~\cite{ravaut2024much}, previous research works on token-level probabilities~\cite{song2019auditing,shidetecting,dong2024generalization} or perplexity~\cite{carlini2021extracting,li2023estimating,xu2024benchmarking} of the benchmark instances.\citeauthor{shidetecting}(\citeyear{shidetecting}) detect contamination by averaging the probabilities of the 
% k\% least likely tokens (outlier words) and assessing whether the average is abnormally high. \citeauthor{dong2024generalization}~(\citeyear{dong2024generalization}) propose CDD (Contamination Detection via Output Distribution) to analyze the peakedness of output distribution and introduce TED (Trustworthy Evaluation via Output Distribution) to mitigate data contamination. Membership Inference Attacks (MIA)\cite{shokri2017membership}, which assess the probability of an instance being part of the training data, have gained traction in contamination detection. \citeauthor{mireshghallah2022quantifying}(\citeyear{mireshghallah2022quantifying}) utilize a likelihood ratio between a target and reference model to infer membership. \citeauthor{mattern2023membership}~(\citeyear{mattern2023membership}) generate similar neighbor sentences and determine membership by analyzing loss differences against a threshold. \citeauthor{ye2024data}~(\citeyear{ye2024data}) propose PAC, which uses random swap augmentation and polarized distance, a spatial metric accounting for both near and far probability regions, to infer membership. Additionally, some methods utilize reference sets and models to assess whether a model demonstrates consistently inflated performance across different benchmarks, identifying contamination through deviations in generalization patterns. \citeauthor{dekoninck2024constat}~(\citeyear{dekoninck2024constat}) propose CONSTAT, a statistical method that detects contamination by comparing a model’s performance on a primary and reference benchmark, identifying non-generalizing performance through deviation analysis based on other uncontaminated reference models. \citeauthor{wei2023skywork}~(\citeyear{wei2023skywork}) use GPT-4 to generate a reference set and detect contamination by measuring its divergence from the test set. Notably, some studies~\cite{srivastava2023beyond,wei-etal-2024-proving} integrate LLM data watermarking\cite{kirchenbauer2023watermark} into contamination detection. We refer readers to relevant literature for further exploration in this direction\cite{liang2024watermarking}.

\begin{figure*}[h!]
    \centering
    \includegraphics[width=0.99\linewidth]{figs/RAG-MIA.pdf}
    \caption{Overview of the problem setting and our Interrogation attack. Given black-box access to a RAG system $\mathcal{S}$, the adversary wants to infer membership of a given target document in the RAG's private database. Our method uses auxiliary LLMs to generate benign queries in the form of natural questions, and uses the correctness of the generated responses as a signal for membership inference test.}
    \label{fig:system_diagram}
\end{figure*}

A well-established issue in deploying LLM-based systems is \textit{jailbreaking}, where adversarial prompts are used to bypass a model's guardrails and induce it to perform unintended actions. To counteract such vulnerabilities, many LLM deployments incorporate countermeasures like detection tools to selective reject such queries.

Several prior works on membership inference and data extraction for RAG systems rely on prompting the model to either regurgitate its context directly or answer questions indirectly tied to the content. For instance, \citet{zeng2024good} explore targeted and untargeted information extraction by designing queries that trigger the retrieval of specific documents, paired with a command suffix intended to induce the generative model to repeat its context and, consequently, the retrieved documents. Similarly, \citet{anderson2024my} propose directly querying the RAG system to determine whether a target document is included in the model's context. On the other hand, some related works \citep{qi2024follow, cohen2024unleashing} employ adversarial prompts to coax the generator into regurgitating data from its context.

However, these adversarial (or even unnatural) queries heavily rely on \emph{prompt injection} techniques. Prompt injection \citep{perez2022ignore} is a broader concept that refers to an LLM vulnerability where attackers craft inputs to manipulate the LLM into executing their instructions unknowingly. In the specific case of these prompt injection attacks, known as \emph{context probing} attacks, the adversary attempts to extract information from the hidden context provided to the LLM. Therefore, it is crucial to analyze the effectiveness of existing inference attacks that rely on prompt injection to determine how successful their queries are in bypassing current detection filters—an area currently underexplored in the literature.

To evaluate the ability of current attacks to bypass detection methods, we adopt two different approaches. First we utilize LakeraGuard, a commercial off-the-shelf guard model designed for detecting prompt-injection and jailbreak attempts \citep{li2024injecguard}, to evaluate queries from different attacks. While this tool can detect queries from some existing attacks, it tends to fall short in identifying queries from attacks whose prompts appear more natural. These tools are designed to detect a wide range of prompt injection queries, so it is unsurprising that they may not perform perfectly in specialized settings like context probing attacks.
To develop a more tailored detection tool, we leverage the capabilities of GPT-4o as a classifier with few-shot prompting to classify input queries as either "natural" or "context probing."\footnote{Instruction template for this classification task is presented in the Appendix (\Cref{fig:classifier_prompt})} GPT-4o has recently shown great performance in prompt injection detection, further supporting its use for this task \citep{li2024injecguard}. Both approaches have shown good performance in prompt injection detection \citep{liu2024formalizing}.

\shortsection{Setup}
We consider attack prompts from three document extraction attacks and four MIAs, including ours. Apart from the MBA attack \citep{liu2024mask}, all prior inference attacks use a fixed, specific template for their attack queries. The templates for these queries are presented in \Cref{tab:prompt_guard_evals}. To evaluate baseline behavior of these detection methods on natural user queries, we include baselines on SQuAD and AI Medical Chatbot question-answer datasets. For more details, see \Cref{app:detection_setup}.

\shortsection{Easily Detectable Attacks}
Employing an off-the-shelf detection method can completely filter out the attack queries for two out of seven attacks, including the RAG document extraction attack \citep{cohen2024unleashing} and MBA \citep{liu2024mask} (\Cref{tab:prompt_guard_evals}), and can filter approximately 58\% of the attack queries for the prompt-injected data extraction attack \citep{qi2024follow}. When using GPT-4o as a detection technique specifically aligned with context probing attacks, the majority of attack queries are successfully filtered out. Importantly, neither of these two techniques significantly impacts natural queries from users, ensuring that legitimate queries remain mostly unaffected. %

These results highlight the necessity for attackers to craft stealthy queries that avoid explicit instructions aimed at recovering documents from the model's context. While adversarially crafted texts designed to bypass detection are feasible, an ideal attack strategy would involve generating clean-text queries that are immune to such defensive countermeasures. 
Thus for an inference attack to be successful in the context of a practical RAG system, it \textbf{must bypass any query-filtering systems that can detect obvious inference attempts}.

