\section{Limitations of Existing Inference Attacks on RAG Systems}
\label{sec:existing_rag_inference}

\section{Detection results}
We show that combining (1) aggregation of layer-wise predictors and (2) RFM probes at each layer and (3) using RFM as an aggregation method gives state-of-the-art results for detecting concepts from LLM activations.  We outline results below and note that additional experimental details are provided in  Appendix~\ref{app: experimental details}.



\paragraph{Benchmark datasets.} To evaluate concept predictors, we consider seven commonly-used benchmarks in which the goal is to evaluate LLM prompts and responses for hallucinations, toxicity, harmful content, and truthfulness.  To detect hallucinations, we use HaluEval (HE) \citep{halueval}, FAVA \citep{fava}, and HaluEval-Wild (HE-Wild) \citep{haluevalwild}. To detect harmful content and prompts, we use AgentHarm \citep{agentharm} and ToxicChat \citep{toxicchat}. For truthfulness, we use the TruthGen benchmark \citep{truthgen}.  Labels for concepts in each benchmark were generated according to the following procedures.  The TruthGen, ToxicChat, HaluEval, and AgentHarm datasets contain labels for the prompts that we use directly. HE-Wild contains text queries a user might make to an LLM that are likely to induce hallucinated replies. For this benchmark, we perform multiclass classification to detect which of six categories each query belongs to. LAT \citep{representation_engineering} does not apply to multiclass classification and LLM-Check \citep{LLMcheck} detects hallucinations directly, not queries that may induce hallucinations. Hence, we mark these methods with `NA' for this task. For FAVA, we follow the binarization used by \citet{LLMcheck} on this dataset, detecting simply the presence of at least one hallucination in the text. For HE, we consider an additional detection task in which we transfer the directions learned for the QA task to the general hallucination detection (denoted HE(Gen.) in Figure~\ref{fig: f1+accuracies, halluc detection, llama}). 



\paragraph{Evaluation metrics.} On these benchmarks, we compare various predictors based on their ability to detect whether or not a given prompt contains the concept of interest.  We evaluate the performance of detectors based on their test accuracy as well as their test F1 score, a measure that is used in cases where the data is imbalanced across classes (see Appendix~\ref{app: metrics}).  For benchmarks without a specific train/validation/testing split of the data, we randomly sampled multiple splits and reported the average F1 scores and (standard) accuracies of the detector fit to the given train/validation set on the test set on instruction-tuned Llama-3.1-8B, Llama-3.3-70B, and Gemma-2-9B. 



% ADD THIS HEADER TO ALL NEW CHAPTER FILES FOR SUBFILES SUPPORT

% Allow independent compilation of this section for efficiency
\documentclass[../CLthesis.tex]{subfiles}

% Add the graphics path for subfiles support
\graphicspath{{\subfix{../images/}}}

% END OF SUBFILES HEADER

%%%%%%%%%%%%%%%%%%%%%%%%%%%%%%%%%%%%%%%%%%%%%%%%%%%%%%%%%%%%%%%%
% START OF DOCUMENT: Every chapter can be compiled separately
%%%%%%%%%%%%%%%%%%%%%%%%%%%%%%%%%%%%%%%%%%%%%%%%%%%%%%%%%%%%%%%%
\begin{document}
\chapter{Appendix}%

\label{appendix:Appendix}
\section{Neuron Depths}
\begin{table}[htbp]
\centering
\begin{tabular}{ll||ll||ll||ll}
\toprule
Neuron & Depth & Neuron & Depth & Neuron & Depth & Neuron & Depth \\
\midrule
N1  & 3380 & N20 & 2640 & N39 & 2460 & N58 & 2140 \\
N2  & 3220 & N21 & 2600 & N40 & 2440 & N59 & 2100 \\
N3  & 3200 & N22 & 2600 & N41 & 2440 & N60 & 1880 \\
N4  & 3180 & N23 & 2600 & N42 & 2420 & N61 & 1820 \\
N5  & 2980 & N24 & 2600 & N43 & 2400 & N62 & 1680 \\
N6  & 2960 & N25 & 2580 & N44 & 2380 & N63 & 1680 \\
N7  & 2880 & N26 & 2580 & N45 & 2380 & N64 & 1340 \\
N8  & 2860 & N27 & 2580 & N46 & 2360 & N65 & 1320 \\
N9  & 2820 & N28 & 2580 & N47 & 2360 & N66 & 1320 \\
N10 & 2740 & N29 & 2580 & N48 & 2340 & N67 & 1120 \\
N11 & 2720 & N30 & 2560 & N49 & 2320 & N68 & 1080 \\
N12 & 2720 & N31 & 2540 & N50 & 2300 & N69 & 1060 \\
N13 & 2700 & N32 & 2540 & N51 & 2280 & N70 & 1060 \\
N14 & 2680 & N33 & 2520 & N52 & 2280 & N71 & 840  \\
N15 & 2680 & N34 & 2520 & N53 & 2260 & N72 & 660  \\
N16 & 2660 & N35 & 2500 & N54 & 2240 & N73 & 480  \\
N17 & 2660 & N36 & 2480 & N55 & 2220 & N74 & 480  \\
N18 & 2640 & N37 & 2480 & N56 & 2180 & N75 & 200  \\
N19 & 2640 & N38 & 2460 & N57 & 2160 &     &      \\
\bottomrule
\end{tabular}
\caption{Depth ($\mu$m) to probe tip for all neurons used in experiment~\ref{exp:1}}
\label{tab:neuron_depths}
\end{table}
% \begin{table}[htbp]
%     \centering
%     {\footnotesize
%     \begin{tabular}{lcllcl}
%         \hline
%         Neuron & Depth & & Neuron & Depth & \\
%         \hline
%         N1 & 3380$\,\mu$m & & N39 & 2460$\,\mu$m & \\
%         N2 & 3220$\,\mu$m & & N40 & 2440$\,\mu$m & \\
%         N3 & 3200$\,\mu$m & & N41 & 2440$\,\mu$m & \\
%         N4 & 3180$\,\mu$m & & N42 & 2420$\,\mu$m & \\
%         N5 & 2980$\,\mu$m & & N43 & 2400$\,\mu$m & \\
%         N6 & 2960$\,\mu$m & & N44 & 2380$\,\mu$m & \\
%         N7 & 2880$\,\mu$m & & N45 & 2380$\,\mu$m & \\
%         N8 & 2860$\,\mu$m & & N46 & 2360$\,\mu$m & \\
%         N9 & 2820$\,\mu$m & & N47 & 2360$\,\mu$m & \\
%         N10 & 2740$\,\mu$m & & N48 & 2340$\,\mu$m & \\
%         N11 & 2720$\,\mu$m & & N49 & 2320$\,\mu$m & \\
%         N12 & 2720$\,\mu$m & & N50 & 2300$\,\mu$m & \\
%         N13 & 2700$\,\mu$m & & N51 & 2280$\,\mu$m & \\
%         N14 & 2680$\,\mu$m & & N52 & 2280$\,\mu$m & \\
%         N15 & 2680$\,\mu$m & & N53 & 2260$\,\mu$m & \\
%         N16 & 2660$\,\mu$m & & N54 & 2240$\,\mu$m & \\
%         N17 & 2660$\,\mu$m & & N55 & 2220$\,\mu$m & \\
%         N18 & 2640$\,\mu$m & & N56 & 2180$\,\mu$m & \\
%         N19 & 2640$\,\mu$m & & N57 & 2160$\,\mu$m & \\
%         N20 & 2640$\,\mu$m & & N58 & 2140$\,\mu$m & \\
%         N21 & 2600$\,\mu$m & & N59 & 2100$\,\mu$m & \\
%         N22 & 2600$\,\mu$m & & N60 & 1880$\,\mu$m & \\
%         N23 & 2600$\,\mu$m & & N61 & 1820$\,\mu$m & \\
%         N24 & 2600$\,\mu$m & & N62 & 1680$\,\mu$m & \\
%         N25 & 2580$\,\mu$m & & N63 & 1680$\,\mu$m & \\
%         N26 & 2580$\,\mu$m & & N64 & 1340$\,\mu$m & \\
%         N27 & 2580$\,\mu$m & & N65 & 1320$\,\mu$m & \\
%         N28 & 2580$\,\mu$m & & N66 & 1320$\,\mu$m & \\
%         N29 & 2580$\,\mu$m & & N67 & 1120$\,\mu$m & \\
%         N30 & 2560$\,\mu$m & & N68 & 1080$\,\mu$m & \\
%         N31 & 2540$\,\mu$m & & N69 & 1060$\,\mu$m & \\
%         N32 & 2540$\,\mu$m & & N70 & 1060$\,\mu$m & \\
%         N33 & 2520$\,\mu$m & & N71 & 840$\,\mu$m & \\
%         N34 & 2520$\,\mu$m & & N72 & 660$\,\mu$m & \\
%         N35 & 2500$\,\mu$m & & N73 & 480$\,\mu$m & \\
%         N36 & 2480$\,\mu$m & & N74 & 480$\,\mu$m & \\
%         N37 & 2480$\,\mu$m & & N75 & 200$\,\mu$m & \\
%         N38 & 2460$\,\mu$m & & & & \\
%         \hline
%     \end{tabular}
%     }
%     \caption{Depth to Probe Tip for All Neurons Used in Experiment 1}
%     \label{tab:neuron_depths}
% \end{table}

\section{Neural Information Integration}
\label{appendix:integration}
\begin{figure}[H]
    \centering
    \includegraphics[height=0.9\textheight]{images/accuracy_all_onsets.pdf}
    \caption{Classification accuracy at different onsets}
    \label{fig:neural_integration}
\end{figure}

\section{CEBRA Results}
\label{appendix:CEBRA}
\begin{figure}[htbp]
    \centering
    \includegraphics[width=0.9\textwidth]{images/embeddings_plot.png}
    \caption{Extra CEBRA embedding visualization from different parameters}
    \label{fig:all_cebra}
\end{figure}

\begin{figure}[H]
    \centering
    \includegraphics[width=0.9\textwidth]{images/cebra_loss.pdf}
    \caption{CEBRA training loss}
    \label{fig:cebra_loss}
\end{figure}

\begin{figure}[H]
    \centering
    \includegraphics[width=0.45\textwidth]{images/cebra_labels.pdf}
    \caption{Data distribution in CEBRA}
    \label{fig:cebra_labels}
\end{figure}

\section{VAE Results}
\label{appendix:VAE}
\begin{figure}[H]
   \begin{subfigure}[b]{0.32\textwidth}
       \centering
       \includegraphics[width=\textwidth]{images/average_syllable_2.pdf}
       \caption{Syllable 2}
       \label{fig:syllable_2}
   \end{subfigure}
   \hfill
   \begin{subfigure}[b]{0.32\textwidth}
       \centering
       \includegraphics[width=\textwidth]{images/average_syllable_3.pdf}
       \caption{Syllable 3}
       \label{fig:syllable_3}
   \end{subfigure}
   \hfill
   \begin{subfigure}[b]{0.32\textwidth}
       \centering
       \includegraphics[width=\textwidth]{images/average_syllable_4.pdf}
       \caption{Syllable 4}
       \label{fig:syllable_4}
   \end{subfigure}
\end{figure}
\begin{figure}[H]
   \begin{subfigure}[b]{0.32\textwidth}
       \centering
       \includegraphics[width=\textwidth]{images/average_syllable_5.pdf}
       \caption{Syllable 5}
       \label{fig:syllable_5}
   \end{subfigure}
   \hfill
   \begin{subfigure}[b]{0.32\textwidth}
       \centering
       \includegraphics[width=\textwidth]{images/average_syllable_6.pdf}
       \caption{Syllable 6}
       \label{fig:syllable_6}
   \end{subfigure}
   \hfill
   \begin{subfigure}[b]{0.32\textwidth}
       \centering
       \includegraphics[width=\textwidth]{images/average_syllable_7.pdf}
       \caption{Syllable 7}
       \label{fig:syllable_7}
   \end{subfigure}

   \begin{subfigure}[b]{0.32\textwidth}
       \centering
       \includegraphics[width=\textwidth]{images/average_syllable_8.pdf}
       \caption{Syllable 8}
       \label{fig:syllable_8}
   \end{subfigure}
   
   \caption{Original and reconstruction syllables of a motif}
   \label{fig:all_syllables}
\end{figure}

\begin{figure}[H]
    \centering
    \includegraphics[width=\linewidth]{images/2d_vae_vocal_warped.pdf}
    \caption{Reconstruction of warped vocal data}
    \label{fig:whole_motif}
\end{figure}

\begin{figure}[H]
    \centering
    \includegraphics[width=\linewidth]{images/neural2vocal_80ms.pdf}
    \caption{Generate 80\,ms vocalization from 80\,ms neural data}
    \label{fig:neuro2voc_80ms}
\end{figure}

% \begin{figure}
%     \centering
%     \includegraphics[width=\linewidth]{images/2d_vae_vocal_trimmed.pdf}
%     \caption{Trimmed Vocal Data to 80ms}
%     \label{fig:vocal_trimmed}
% \end{figure}

% \begin{figure}
%     \centering
%     \includegraphics[width=\linewidth]{images/2d_vae_vocal_padded.pdf}
%     \caption{Padded Vocal Data to 224ms}
%     \label{fig:vocal_padded}
% \end{figure}

% \begin{figure}
%     \centering
%     \includegraphics[width=\linewidth]{images/2d_vae_vocal_warped.pdf}
%     \caption{Warped Vocal Data to 224ms}
%     \label{fig:vocal_warped}
% \end{figure}


\end{document}




\paragraph{Results.} Among all detection methods based on activations for hallucination detection, RFM was a component of the winning model across all datasets. These include (1) RFM from the single best layer, or (2) aggregating layers with RFM as the layer-wise predictor and either linear regression or RFM as the aggregation model (Table~\ref{fig: f1+accuracies, halluc detection, llama}). For Gemma-2-9B, the best performing method, among those that learn from activations, used aggregation or RFM on the single best layer on three of the four datasets (Table~\ref{fig: f1+accuracies, halluc detection, gemma}). The detection performance for the best performing detector on Llama-3.1-8B was better than that of Gemma-2-9B across all hallucination datasets, hence the best performing detectors taken across both models utilized both aggregation and RFM. The results are similar for detecting harmful, toxic, and dishonest content (Table~\ref{fig: f1 scores, combined non-halluc detection}). Among all models tested, the best performing model utilized aggregation and/or RFM as one of its components. Further, among the smaller LLMs (Llama-3.1-8B and Gemma-2-9B), the overall best performing model for each dataset utilized RFM. 

Moreover, aggregation with RFM as a component (either as the layer-wise predictor or aggregation method) performed better than the state-of-the-art judge model, GPT-4o \citep{gpt4o}. For Truthgen, we found that our method out-performed GPT-4o on Llama-3.3-70B but not with Llama-3.1-8B. We note that GPT-4o may have an advantage on this task as the truthful and dishonest responses were generated from earlier version of this model (GPT-3.5 and 4).  

Our aggregation method also performs favorably compared to methods designed for specific tasks. For example, our aggregation method outperformed a recent hallucination detection method based on the self-consistency of the model (LLM-Check \citep{LLMcheck}), which was argued to be the prior state-of-the-art in their work on the FAVA dataset. Further, despite the generality of our method, we perform only slightly worse than the fine-tuned model (ToxicChat-T5-Large) for toxicity detection on the ToxicChat dataset in F1 score and accuracy (Tables~\ref{fig: f1 scores, combined non-halluc detection} and \ref{fig: accuracies, non-halluc detection combined}). 

\begin{figure*}[h!]
    \centering
    \includegraphics[width=0.99\linewidth]{figs/RAG-MIA.pdf}
    \caption{Overview of the problem setting and our Interrogation attack. Given black-box access to a RAG system $\mathcal{S}$, the adversary wants to infer membership of a given target document in the RAG's private database. Our method uses auxiliary LLMs to generate benign queries in the form of natural questions, and uses the correctness of the generated responses as a signal for membership inference test.}
    \label{fig:system_diagram}
\end{figure*}

A well-established issue in deploying LLM-based systems is \textit{jailbreaking}, where adversarial prompts are used to bypass a model's guardrails and induce it to perform unintended actions. To counteract such vulnerabilities, many LLM deployments incorporate countermeasures like detection tools to selective reject such queries.

Several prior works on membership inference and data extraction for RAG systems rely on prompting the model to either regurgitate its context directly or answer questions indirectly tied to the content. For instance, \citet{zeng2024good} explore targeted and untargeted information extraction by designing queries that trigger the retrieval of specific documents, paired with a command suffix intended to induce the generative model to repeat its context and, consequently, the retrieved documents. Similarly, \citet{anderson2024my} propose directly querying the RAG system to determine whether a target document is included in the model's context. On the other hand, some related works \citep{qi2024follow, cohen2024unleashing} employ adversarial prompts to coax the generator into regurgitating data from its context.

However, these adversarial (or even unnatural) queries heavily rely on \emph{prompt injection} techniques. Prompt injection \citep{perez2022ignore} is a broader concept that refers to an LLM vulnerability where attackers craft inputs to manipulate the LLM into executing their instructions unknowingly. In the specific case of these prompt injection attacks, known as \emph{context probing} attacks, the adversary attempts to extract information from the hidden context provided to the LLM. Therefore, it is crucial to analyze the effectiveness of existing inference attacks that rely on prompt injection to determine how successful their queries are in bypassing current detection filters—an area currently underexplored in the literature.

To evaluate the ability of current attacks to bypass detection methods, we adopt two different approaches. First we utilize LakeraGuard, a commercial off-the-shelf guard model designed for detecting prompt-injection and jailbreak attempts \citep{li2024injecguard}, to evaluate queries from different attacks. While this tool can detect queries from some existing attacks, it tends to fall short in identifying queries from attacks whose prompts appear more natural. These tools are designed to detect a wide range of prompt injection queries, so it is unsurprising that they may not perform perfectly in specialized settings like context probing attacks.
To develop a more tailored detection tool, we leverage the capabilities of GPT-4o as a classifier with few-shot prompting to classify input queries as either "natural" or "context probing."\footnote{Instruction template for this classification task is presented in the Appendix (\Cref{fig:classifier_prompt})} GPT-4o has recently shown great performance in prompt injection detection, further supporting its use for this task \citep{li2024injecguard}. Both approaches have shown good performance in prompt injection detection \citep{liu2024formalizing}.

\shortsection{Setup}
We consider attack prompts from three document extraction attacks and four MIAs, including ours. Apart from the MBA attack \citep{liu2024mask}, all prior inference attacks use a fixed, specific template for their attack queries. The templates for these queries are presented in \Cref{tab:prompt_guard_evals}. To evaluate baseline behavior of these detection methods on natural user queries, we include baselines on SQuAD and AI Medical Chatbot question-answer datasets. For more details, see \Cref{app:detection_setup}.

\shortsection{Easily Detectable Attacks}
Employing an off-the-shelf detection method can completely filter out the attack queries for two out of seven attacks, including the RAG document extraction attack \citep{cohen2024unleashing} and MBA \citep{liu2024mask} (\Cref{tab:prompt_guard_evals}), and can filter approximately 58\% of the attack queries for the prompt-injected data extraction attack \citep{qi2024follow}. When using GPT-4o as a detection technique specifically aligned with context probing attacks, the majority of attack queries are successfully filtered out. Importantly, neither of these two techniques significantly impacts natural queries from users, ensuring that legitimate queries remain mostly unaffected. %

These results highlight the necessity for attackers to craft stealthy queries that avoid explicit instructions aimed at recovering documents from the model's context. While adversarially crafted texts designed to bypass detection are feasible, an ideal attack strategy would involve generating clean-text queries that are immune to such defensive countermeasures. 
Thus for an inference attack to be successful in the context of a practical RAG system, it \textbf{must bypass any query-filtering systems that can detect obvious inference attempts}.

