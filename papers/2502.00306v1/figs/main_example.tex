\begin{figure*}[ht!]
    \centering
    \begin{tcolorbox}[
        enhanced,                  %
        colframe=blue!70!black,   %
        colback=blue!5,           %
        coltitle=white,            %
        colbacktitle=blue!70!black, %
        width=\textwidth,          %
        arc=4mm,                   %
        boxrule=1mm,               %
        drop shadow,               %
        title=An Example of Our Generated Queries, %
        fonttitle=\bfseries\large  %
    ]

    \textbf{Title:}\\[0.5em]
    A compact magnetic directional proximity sensor for spherical robots\\[0.5em]

    \textbf{Text:}\\[0.5em]
    Spherical robots have recently attracted significant interest due to their ability to offer high speed motion with excellent locomotion efficiency. As a result of the presence of a sealed outer shell, its obstacle avoidance strategy has been simply “hit and run.” While this is convenient due to the specific geometry of the spherical robots, it however could pose serious issues when the robots are small and light. For portable spherical robots with on-board cameras, a high-speed collision with a hard surface may damage the robot or the camera. This paper proposes a novel and compact proximity sensor that utilizes passive magnetic field to detect the ferromagnetic obstacles through perturbation of the magnetic field. Compared with the existing works that utilize the Earth's weak magnetic field as a means of detection, the approach undertaken here seeks to harness the same principle but uses an intelligently designed magnetic assembly. It efficiently amplifies the perturbation and therefore improves the detection performance. The presented method is able to simultaneously determine both the distance and direction of the nearby ferromagnetic obstacles. Both simulation and experimental results are presented to validate the sensing principle and operational performance.\\[0.5em]

    \textbf{Our Adversarial Query:}\\[0.5em]
    "I am inquiring about a \textcolor{red}{\textbf{compact magnetic proximity sensor for directional detection in spherical robots}}. \textcolor{blue}{\textbf{Is the presence of a sealed outer shell a characteristic feature of spherical robots?}} Please answer with 'Yes,' 'No,' or 'I don't know'."\\[0.5em]

    \textbf{Rewritten Query:}\\[0.5em]
"I'm seeking information on a compact magnetic proximity sensor designed for detecting direction in spherical robots. Do spherical robots typically have a sealed outer shell? Please respond with ``Yes,'' ``No,'' or ``I don't know.''"\\[1em]


    \end{tcolorbox}
    \caption{Example of a particular document discussing proximity sensors for spherical robots, with an example query generated by our attack and the corresponding rewritten version that is used by the RAG system. The \textcolor{red}{red text} represents the generated general description specific to the target document, while the \textcolor{blue}{blue text} is the generated yes/no question. Note that the adversary is unaware of the exact query-rewriting strategy, and thus does not get to observe the rewritten query directly.}
    \label{fig:main_example}
\end{figure*}
