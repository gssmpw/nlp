\begin{figure*}[ht!]
    \centering
    \begin{tcolorbox}[
        enhanced,                  %
        colframe=blue!70!black,    %
        colback=blue!5,            %
        coltitle=white,            %
        colbacktitle=blue!70!black,%
        width=\textwidth,          %
        arc=4mm,                   %
        boxrule=1mm,               %
        drop shadow,               %
        title=Effectiveness of Dietary Interventions in Dental Settings, %
        fonttitle=\bfseries\large  %
    ]

    \textbf{Text:}\\[1em]
    BACKGROUND: The dental care setting is an appropriate place to deliver dietary assessment and advice as part of patient management. However, we do not know whether this is effective in changing dietary behaviour. OBJECTIVES: To assess the effectiveness of one-to-one dietary interventions for all ages carried out in a dental care setting in changing dietary behaviour. The effectiveness of these interventions in the subsequent changing of oral and general health is also assessed.\\[0.5em]
    SEARCH METHODS: The following electronic databases were searched: the Cochrane Oral Health Group Trials Register (to 24 January 2012), the Cochrane Central Register of Controlled Trials (CENTRAL) (The Cochrane Library 2012, Issue 1), MEDLINE via OVID (1950 to 24 January 2012), EMBASE via OVID (1980 to 24 January 2012), CINAHL via EBSCO (1982 to 24 January 2012), PsycINFO via OVID (1967 to 24 January 2012), and Web of Science (1945 to 12 April 2011). We also undertook an electronic search of key conference proceedings (IADR and ORCA between 2000 and 13 July 2011). Reference lists of relevant articles, thesis publications (Dissertations s Online 1861 to 2011) were searched. The authors of eligible trials were contacted to identify any unpublished work.\\[0.5em]
    SELECTION CRITERIA: Randomised controlled trials assessing the effectiveness of one-to-one dietary interventions delivered in a dental care setting. DATA COLLECTION AND ANALYSIS: screening, eligibility screening and data extraction decisions were all carried out independently and in duplicate by two review authors. Consensus between the two opinions was achieved by discussion, or involvement of a third review author.\\[0.5em]
    MAIN RESULTS: Five studies met the criteria for inclusion in the review. Two of these were multi-intervention studies where the dietary intervention was one component of a wider programme of prevention, but where data on dietary behaviour change were reported. One of the single intervention studies was concerned with dental caries prevention. The other two concerned general health outcomes. There were no studies concerned with dietary change aimed at preventing tooth erosion. In four out of the five included studies a significant change in dietary behaviour was found for at least one of the primary outcome variables.\\[0.5em]
    AUTHORS' CONCLUSIONS: There is some evidence that one-to-one dietary interventions in the dental setting can change behaviour, although the evidence is greater for interventions aiming to change fruit/vegetable and alcohol consumption than for those aiming to change dietary sugar consumption. There is a need for more studies, particularly in the dental practice setting, as well as greater methodological rigour in the design, statistical analysis and reporting of such studies.\\[1em]

    \textbf{Original Question 1:}\\[0.5em]
    "I am asking about One-to-one dietary interventions and their effectiveness in changing dietary behaviour in a dental setting. Is the effectiveness of dietary interventions on oral health assessed in the text? Please answer with 'Yes,' 'No,' or 'I don't know'."\\[1em]

    \textbf{Paraphrased Question 1:}\\[0.5em]
    "The text inquires whether personalized dietary interventions are effective at altering dietary behaviors within a dental environment. It specifically asks if the impact of these dietary interventions on oral health is evaluated. The appropriate response would be 'I don't know.'."\\[1em]

    \textbf{Original Question 2:}\\[0.5em]
    "I am asking about One-to-one dietary interventions and their effectiveness in changing dietary behaviour in a dental setting. Was there a study focused on dental caries prevention included in the review? Please answer with 'Yes,' 'No,' or 'I don't know'."\\[1em]

    \textbf{Paraphrased Question 2:}\\[0.5em]
    "I can't determine whether a study on dental caries prevention was part of the review from the information provided. Therefore, my answer is 'I don't know.'."\\[1em]

    \end{tcolorbox}
    \caption{An example of a case where GPT-4o fails to paraphrase the question properly.}
    \label{fig:failed_example_1}
\end{figure*}
