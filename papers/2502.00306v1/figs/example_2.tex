
\begin{figure*}[ht!]
    \centering
    \begin{tcolorbox}[
        enhanced,                  %
        colframe=red!70!black,     %
        colback=red!5,             %
        coltitle=white,            %
        colbacktitle=red!70!black, %
        width=\textwidth,          %
        arc=4mm,                   %
        boxrule=1mm,               %
        drop shadow,               %
        title=Cyber Security and Smart Grid Communication, %
        fonttitle=\bfseries\large  %
    ]

    \textbf{Title:}\\[0.5em]
    Cyber Security and Power System Communication—Essential Parts of a Smart Grid Infrastructure\\[1em]

    \textbf{Text:}\\[0.5em]
    The introduction of “smart grid” solutions imposes that cyber security and power system communication systems must be dealt with extensively. These parts together are essential for proper electricity transmission, where the information infrastructure is critical. The development of communication capabilities, moving power control systems from “islands of automation” to totally integrated computer environments, have opened up new possibilities and vulnerabilities. Since several power control systems have been procured with “openness” requirements, cyber security threats become evident. For refurbishment of a SCADA/EMS system, a separation of the operational and administrative computer systems must be obtained. The paper treats cyber security issues, and it highlights access points in a substation. Also, information security domain modeling is treated. Cyber security issues are important for “smart grid” solutions. Broadband communications open up for smart meters, and the increasing use of wind power requires a “smart grid system.”\\[1em]

    \textbf{Retrieved Document:}\\[1em]
    \textbf{Title:}\\[0.5em]
    Cyber security in the Smart Grid: Survey and challenges\\[0.5em]

    \textbf{Text:}\\[0.5em]
    The Smart Grid, generally referred to as the next-generation power system, is considered as a revolutionary and evolutionary regime of existing power grids. More importantly, with the integration of advanced computing and communication technologies, the Smart Grid is expected to greatly enhance efficiency and reliability of future power systems with renewable energy resources, as well as distributed intelligence and demand response. Along with the silent features of the Smart Grid, cyber security emerges to be a critical issue because millions of electronic devices are inter-connected via communication networks throughout critical power facilities, which has an immediate impact on reliability of such a widespread infrastructure. In this paper, we present a comprehensive survey of cyber security issues for the Smart Grid. Specifically, we focus on reviewing and discussing security requirements, network vulnerabilities, attack countermeasures, secure communication protocols and architectures in the Smart Grid. We aim to provide a deep understanding of security vulnerabilities and solutions in the Smart Grid and shed light on future research directions for Smart Grid security. 2013 Elsevier B.V. All rights reserved.\\[1em]

    \end{tcolorbox}
    \caption{An example of a failed case for non-members where the same similar document is retrieved for all questions.}
    \label{fig:failed_example_2}
\end{figure*}
