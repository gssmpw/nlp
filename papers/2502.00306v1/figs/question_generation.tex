

\begin{figure*}[ht!]
    \centering
    \begin{tcolorbox}[
        enhanced,                  %
        colframe=green!50!black,   %
        colback=green!5,           %
        coltitle=white,            %
        colbacktitle=green!50!black, %
        width=\textwidth,          %
        arc=4mm,                   %
        boxrule=1mm,               %
        drop shadow,               %
        title=Corpus-Based Question Generation, %
        fonttitle=\bfseries\large  %
    ]

    \textbf{Main Text:}\\[1em]

    Based on the following text corpus, generate a list of 30 specific, diverse yes/no questions as queries that could be used to retrieve information from this corpus. Note: If the text uses any abbreviations or acronyms, such as 'AhR' or 'IC(50)', use the same forms in your questions. Do not use the expanded version unless it is explicitly mentioned in the text. Here are a few examples of the type of questions we are looking for:\\[1em]

    \textbf{Example Text:}\\[1em]
    Dioxins invade the body mainly through the diet, and produce toxicity through the transformation of aryl hydrocarbon receptor (AhR). 
    An inhibitor of the transformation should therefore protect against the toxicity and ideally be part of the diet. 
    We examined flavonoids ubiquitously expressed in plant foods as one of the best candidates, and found that the subclasses flavones 
    and flavonols suppressed antagonistically the transformation of AhR induced by 1 nM of 2,3,7,8-tetrachlorodibenzo-p-dioxin, 
    without exhibiting agonistic effects that transform AhR. The antagonistic IC(50) values ranged from 0.14 to 10 microM, close to 
    the physiological levels in human.\\[1em]

    \textbf{Example Questions:}\\[1em]
    1. Are flavones and flavonols shown to antagonistically suppress the transformation of AhR induced by dioxins?\\
    2. Do flavones and flavonols exhibit agonistic effects that transform the aryl hydrocarbon receptor?\\
    3. Are the antagonistic IC(50) values for flavones and flavonols between 0.14 and 10 microM?\\[1em]

    Now, based on the main corpus provided below, create questions that are specific, contain keywords from the text, and are diverse enough to cover different aspects or concepts discussed. Avoid mentioning 'the study' or any references to the passage itself, and ensure that questions do not contain general phrases that could apply to any text.\\[1em]

    \textbf{Here is the Corpus:}\\[1em]
    Dioxins invade the body mainly through the diet, and produce toxicity through the transformation of aryl hydrocarbon receptor (AhR). 
    An inhibitor of the transformation should therefore protect against the toxicity and ideally be part of the diet. 
    We examined flavonoids ubiquitously expressed in plant foods as one of the best candidates, and found that the subclasses flavones 
    and flavonols suppressed antagonistically the transformation of AhR induced by 1 nM of 2,3,7,8-tetrachlorodibenzo-p-dioxin, 
    without exhibiting agonistic effects that transform AhR. The antagonistic IC(50) values ranged from 0.14 to 10 microM, close to 
    the physiological levels in human.\\[1em]

    \textbf{Generate 30 yes/no questions based on this text.}

    \end{tcolorbox}
    \caption{Full prompt for generating yes/no questions from the provided corpus using gpt-4o.}
    \label{fig:corpus_question_generation}
\end{figure*}

