\section{Virtualization of Aggregators}
\label{sec:vir}

\paragraph{Visualization of BEA} \Cref{fig:BEA_heatmap} and \Cref{fig:BEA_value} visualize BEA for a specific case where the sample size $n=20$, and the rating values are \([m]=[2]=\{1, 2\}\). Compared to simple averaging, BEA tends to be more conservative (i.e. close to $\frac{m+1}{2}=3/2$). Note as the lower bound of the participation probability $q$ becomes larger, BEA becomes more conservative.


\begin{figure}[h]
  \centering
  \begin{subfigure}[b]{0.23\textwidth}
    \centering
    \includegraphics[width=\textwidth,keepaspectratio]{picture/BEA/q=0.1.pdf}
    \caption{$q=0.1$}
  \end{subfigure}
  \hspace{0.005\textwidth}
  \begin{subfigure}[b]{0.23\textwidth}
    \centering
    \includegraphics[width=\textwidth,keepaspectratio]{picture/BEA/q=0.7.pdf}
    \caption{$q=0.7$}
  \end{subfigure}
  
  \vspace{0.01\textheight} % 两排之间的垂直间距
  
  \begin{subfigure}[b]{0.23\textwidth}
    \centering
    \includegraphics[width=\textwidth,keepaspectratio]{picture/BEA/q=0.9.pdf} 
     \caption{$q=0.9$}
  \end{subfigure}
  \hspace{0.005\textwidth}
  \begin{subfigure}[b]{0.23\textwidth}
    \centering
    \includegraphics[width=\textwidth,keepaspectratio]{picture/BEA/q=1.pdf}     \caption{simple averaging ($q=1$)}
  \end{subfigure}

  \caption{Heat map of BEA for a specific case where the sample size $n=20$, and the rating values are \([m]=[2]=\{1, 2\}\). The x-axis represents the count of rating \(1\), and the y-axis represents the count of rating \(2\). We vary the lower bound of the participation probability \( q \). }
  \label{fig:BEA_heatmap}
\end{figure}

\begin{figure}[H]
  \centering
  \includegraphics[width=0.45\textwidth,keepaspectratio]{picture/BEA/x=5.pdf}
  \caption{Value of BEA for a specific case where the sample size $n=20$, and the rating values are \([m]=[2]=\{1, 2\}\). The count of rating $1$ is fixed as $n_1=5$. The x-axis is the count of rating $2$, $n_2$. The y-axis is the value of BEA. We vary the lower bound of the participation probability \( q \).}
  \label{fig:BEA_value}
\end{figure}



\paragraph{Visualization of PAA}

\begin{figure}[h]
  \centering
  \begin{subfigure}[b]{0.23\textwidth}
    \centering
    \includegraphics[width=\textwidth,keepaspectratio]{picture/PAA/q=0.1.pdf}
    \caption{$q=0.1$}
  \end{subfigure}
  \hspace{0.005\textwidth}
  \begin{subfigure}[b]{0.23\textwidth}
    \centering
    \includegraphics[width=\textwidth,keepaspectratio]{picture/PAA/q=0.3.pdf}
    \caption{$q=0.3$}
  \end{subfigure}
  
  \vspace{0.01\textheight} % 两排之间的垂直间距
  
  \begin{subfigure}[b]{0.23\textwidth}
    \centering
    \includegraphics[width=\textwidth,keepaspectratio]{picture/PAA/q=0.5.pdf} 
     \caption{$q=0.5$}
  \end{subfigure}
  \hspace{0.005\textwidth}
  \begin{subfigure}[b]{0.23\textwidth}
    \centering
    \includegraphics[width=\textwidth,keepaspectratio]{picture/PAA/q=1.pdf}     \caption{simple averaging ($q=1$)}
  \end{subfigure}

  \caption{Heat map of PAA for a specific case where the sample size $n=1000$, and the rating values are \([m]=[2]=\{1, 2\}\). The x-axis represents the count of rating \(1\), and the y-axis represents the count of rating \(2\). We vary the lower bound of the participation probability \( q \). When \( q = 1 \), PAA degenerates to simple averaging.}
  \label{fig:PAA_heatmap}
\end{figure}


\begin{figure}[h]
  \centering
  \includegraphics[width=0.45\textwidth,keepaspectratio]{picture/PAA/x=300.pdf}
  \caption{Value of PAA for a specific case where the sample size $n=1000$, and the rating values are \([m]=[2]=\{1, 2\}\). The count of rating $1$ is fixed as $n_1=300$. The x-axis is the count of rating $2$, $n_2$. The y-axis is the value of PAA. We vary the lower bound of the participation probability \( q \). When \( q = 1 \), PAA degenerates to simple averaging.}
  \label{fig:x=300}
\end{figure}

\Cref{fig:PAA_heatmap} and \Cref{fig:x=300} visualize PAA for a specific case where the sample size $n=1000$, and the rating values are \([m]=[2]=\{1, 2\}\). Note as the lower bound of the participation probability $q$ becomes smaller, PAA becomes more conservative (i.e. close to $\frac{m+1}{2}=3/2$). We can see it more clearly in \Cref{fig:x=300}.

When $n_1=n_2=300$, all the aggregators report the same value $\frac{m+1}{2}=\frac{3}{2}$. In general, $f^{BEA}$ also reports $\frac{m+1}{2}$ when $n_1=n_m$. When $n$ is large, the worst case is roughly $n_1=n_m$, when $n \to \infty$, the worst $\hat{\vp}$ is $(\frac{1}{2},0,\cdots,0,\frac{1}{2})$. This represents a polarized situation where participants are split between extreme approval or disapproval, with equal numbers on both sides. 

% \begin{lemma}
%     \label{lm1}
%     Fix $(x,y)$, $F(q)=\left|f_{q}^{PAA}(x,y)-\frac{m+1}{2}\right|$ is a monotonically increasing function with respect to $q \ (0 \leq q\leq 1)$ where $x=\sum_i \mathbbm{1}(\hat{x}_i=1),y=\sum_i \mathbbm{1}(\hat{x}_i=m)$.
% \end{lemma}

%\Cref{fig:1} shows the heat map of $f^{PAA}$ with different lower bounds of the participation probability (i.e. different $q$) and the simple average aggregator which can be viewed as a special case of $f^{PAA}$ with $q=1$. We fix the number of samples $n=1000$ and we assume there are only two scores: $m=2, [m]=1,2$. 



%\Cref{fig:2} is the the function of $f^{PAA}$ regarding of rating $2$. We fix the sample size as $n=1000$. and we assume there are only two scores:$1,2$. We also fix the number of agents reporting score $1$ to $300$. Note for the same number of agents rating $2$, the smaller $q$ is, the closer $f^{PAA}$ is to $\frac{m+1}{2}=3/2$.




