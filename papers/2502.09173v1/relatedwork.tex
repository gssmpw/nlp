\section{Related Work}
Time-series forecasting is primarily to predict future values based on previously observed data points. Traditional statistical methods, most notably the Autoregressive Integrated Moving Average (ARIMA) model, have long been utilized due to their mathematical simplicity and flexibility in application \citep{rizvi_arima_2024,kontopoulou_review_2023}. While ARIMA remains a staple for scenarios where data exhibits linear patterns, recent developments in machine learning have introduced sophisticated models capable of capturing non-linear dependencies, thus offering potential improvements in forecasting accuracy and robustness \citep{masini_machine_2023,rhanoui_forecasting_2019}.

The advent of the Generative Pre-trained Transformer (GPT) by OpenAI marked a significant milestone in the field of natural language processing \citep{brown_language_2020}, catalyzing a wave of innovations in large language models (LLMs). Large Language Models (LLMs) have profoundly transformed natural language processing and are increasingly being considered for diverse applications beyond text, such as time series data analysis. The study by \citep{bian_multi-patch_2024} presents a framework that adapts LLMs for time-series representation learning by conceiving time-series forecasting as a multi-patch prediction task, introducing a patch-wise decoding layer that enhances temporal sequence learning. Similarly, \citep{liu_autotimes_2024} propose a model which leverages the autoregressive capabilities of LLMs for time series forecasting. In \citet{capstick2024representation}, the authors apply a GPT-based text encoder to string representations of in-home activity data to enable vector searching and clustering. Using a secondary modelling stage, we extend these ideas to enable further analysis and interpretability.

PageRank, originally developed to rank web pages, is an algorithm designed to assess the importance of nodes within a directed graph by analyzing the structure of links within networks \citep{page_pagerank_1999}. While it was initially created for search engines, its application has since expanded across various disciplines. For instance, in biological networks, \citep{ivan_when_2011} employed personalized PageRank to analyze protein interaction networks, providing scalable and robust techniques for interpreting complex biological data. Similarly, \citep{banky_equal_2013} introduced an innovative adaptation of PageRank for metabolic graphs. This cross-disciplinary application of PageRank highlights its potential for analyzing complex systems beyond its original domain.