This position paper argues that AI agents should be regulated based on the sequence of actions they autonomously take. AI agents with long-term planning and strategic capabilities can pose significant risks of human extinction and irreversible global catastrophes. While existing regulations often focus on computational scale as a proxy for potential harm, we contend that such measures are insufficient for assessing the risks posed by AI agents whose capabilities arise primarily from inference-time computation. To support our position, we discuss relevant regulations and recommendations from AI scientists regarding existential risks, as well as the advantages of action sequences over existing impact measures that require observing environmental states.