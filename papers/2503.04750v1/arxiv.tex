\documentclass{article}

\usepackage[margin=45mm]{geometry}
\usepackage{natbib}
\usepackage{graphicx}
\usepackage{hyperref}
\usepackage{amsmath}
\usepackage{amssymb}
\usepackage{mathtools}


\noindent\textbf{Definition 3.} \textit{(\textbf{Fine-tuned Score Deviation}) is the difference of scores before and after supervised fine-tuning, averaged across samples:}
\begin{equation}
    \frac{1}{n}\sum_{i=1}^n S(\mathbf{x}_i; \theta) - S(\mathbf{x}_i; \theta'),
\end{equation}
\textit{where $\mathbf{x}_i$ is the $i$-th sample in the dataset, $S(\cdot;\cdot)$ is an existing scoring function~(e.g., Min-K\% or Perplexity Score), and $\theta, \theta'$ are models before and after fine-tuning, respectively.}~\cite{zhang2024fine}



\begin{table}[h]
    \vspace{-3mm}
    \centering
    \setlength{\tabcolsep}{3pt}
    \caption{\textbf{Consistency Requirement in terms of average MAPE.} The average Mean Absolute Percentage Error~(MAPE) over 5 independent runs is calculated. Among non-FSD baselines, our Kernel Divergence Score achieves the lowest average MAPE.}
    \vspace{3mm}
    \resizebox{0.6\linewidth}{!}{
    \begin{tabular}{l ccc | G}
    \toprule
    \textbf{Methods} & \textbf{WikiMIA} & \textbf{BookMIA} & \textbf{ArxivTection} & \textbf{\textit{Average}} \\
    \midrule
    \multicolumn{5}{l}{\textit{Non-FSD-based Scores}} \\
    Zlib & 0.1300 & 0.1144 & 0.5199 & 0.2548 \\
    Perplexity Score & 0.1786 & 0.1974 & 0.1892 & 0.1884 \\
    Min-K\% & {0.1966} & {0.2519} & 0.1882 & 0.2122 \\
    Min-K\%++ & 0.3281 & 0.1115 & 0.3268 & 0.2554 \\
    \midrule
    \multicolumn{5}{l}{\textit{FSD-based Scores}} \\
    Zlib + FSD & {0.1421} & {0.1460} & 0.2191 & 0.1691 \\
    Perplexity Score + FSD & 0.1587 & 0.1653 & 0.2490 & 0.1910 \\
    Min-K\% + FSD & {0.1659} & {0.2352} & 0.2593 & 0.2201 \\
    Min-K\%++ + FSD & 0.4418 & 0.1821 & 0.1703 & 0.2647 \\
    \midrule
    \textbf{Kernel Divergence Score} & 0.1427 & 0.1500 & 0.2527 & 0.1818 \\
    \bottomrule
\end{tabular}
}
\label{tab:req2}
\end{table}
In comparison to the detector-based scoring methods, we evaluate our Kernel Divergence Score, in addition to alternative scoring approaches that alsp leverage the fine-grained information of kernels.
In all their definitions below, let $Z, Z' \in \mathbb{R}^{n \times d}$ be the embeddings before and after supervised fine-tuning, respectively, where $Z_i, Z_i' \in \mathbb{R}^{1 \times d}$ are row vectors.

\noindent\textbf{Definition 5.} \textit{(\textbf{KDE KL Divergence}) is the Kullback-Leibler divergence of the Kernel Density Estimator~(KDE):}
\begin{equation}
    \text{KLD}\bigg(\frac{1}{E}\sum_{i=1}^n\Phi(Z)_{i,j} \; \bigg\vert \; \frac{1}{E'}\sum_{i=1}^n\Phi(Z')_{i,j}\bigg),
\end{equation}
\textit{where $\Phi(\cdot)$ is the RBF kernel, $Z, Z'$ are normalized model embeddings, $E = \sum_{i,j} \Phi(Z)_{i,j}$ and $E' = \sum_{i,j} \Phi(Z')_{i,j}$.}

\noindent\textbf{Definition 6.} \textit{(\textbf{Maximum Mean Discrepancy$^2$}) is the squared value of the maximum mean discrepancy~(MMD):}
\begin{equation}
    \text{MMD}(Z, Z') = \sup_{f \in \mathcal{H}}\bigg(\mathbb{E}_{Z}[f(Z)] - \mathbb{E}_{Z'}[f(Z')]\bigg),
\end{equation}
\textit{where $\mathcal{H}$ is the reproducing kernel Hilbert space~(RKHS). In practice, }
\begin{equation}
    \mathbb{E}_Z[k(Z_i, Z_i)] + \mathbb{E}_{Z'}[k(Z_i', Z_i')] - 2 \mathbb{E}_{Z,Z'}[k(Z_i, Z_i')]
\end{equation}
\textit{is computed as the MMD$^2$, where $k$ is the kernel function.}~\cite{gretton2012kernel}


\noindent\textbf{Definition 7.} \textit{(\textbf{Centered Kernel Alignment}) is defined as}
\begin{equation}
    1 - \frac{||Z^\top Z'||_\text{F}^2}{||Z^\top Z||_\text{F} ||Z'^\top Z'||_\text{F}},
\end{equation}
\textit{where $Z, Z'$ are zero-centered embeddings.}~\cite{kornblith2019similarity}



\noindent\textbf{Definition 8.} \textit{(\textbf{Dot-product Kernel MSE}) is the mean squared error of paired dot-product kernels:}
\begin{equation}
    \frac{1}{n^2}||Z Z^\top - Z' Z'^\top||_\text{F}^2,
\end{equation}
\textit{where $n$ is the number of samples.}

\noindent\textbf{Definition 9.} \textit{(\textbf{Cosine Similarity Kernel MSE}) is the mean squared error of paired cosine similarity kernels.}
\begin{equation}
    \frac{1}{n^2}\bigg\Vert\frac{Z Z^\top}{||Z||_2^2} - \frac{Z' Z'^\top}{||Z'||_2^2}\bigg\Vert_\text{F}^2,
\end{equation}
\textit{where $n$ is the number of samples.}


\noindent\textbf{Definition 10.} \textit{(\textbf{RBF Kernel MSE}) is the mean squared error of paired radial basis function~(RBF) kernels.}
\begin{equation}
    \frac{1}{n^2}||\Phi(Z) - \Phi(Z')||_\text{F}^2,
\end{equation}
\textit{where $n$ is the number of samples, $\Phi$ is the RBF kernel, and $Z, Z'$ are normalized model embeddings.}

We subtract the original Centered Kernel Alignment~(CKA) value from 1 to ensure that the measure's trend aligns with the contamination rate trends observed in other kernel-based baselines.
Note that this does not affect the general property of the measure.
Further implementation details and hyperparameters for each baseline are provided in Appendix~\ref{}.


% modification to natbib citations
\setcitestyle{authoryear,round,citesep={;},aysep={,},yysep={;}}

% Change citation commands to be more like old ICML styles
\newcommand{\yrcite}[1]{\citeyearpar{#1}}
\renewcommand{\cite}[1]{\citep{#1}}

\providecommand{\keywords}[1]
{
  \small	
  \textbf{\textit{Keywords---}} #1
}

\title{\textbf{Position: AI agents should be regulated based on autonomous action sequences}}
\author{
    Takauki Osogami\\
    IBM Research -- Tokyo\\
    {\tt osogami@jp.ibm.com}
}
\date{\today}

\begin{document}

\maketitle

\begin{abstract}
\begin{abstract}
Retrieval-Augmented Generation (RAG) is often used with Large Language Models (LLMs) to infuse domain knowledge or user-specific information. In RAG, given a user query, a retriever extracts chunks of relevant text from a knowledge base. These chunks are sent to an LLM as part of the input prompt. Typically, any given chunk is repeatedly retrieved across user questions. However, currently, for every question, attention-layers in LLMs fully compute the key values (KVs) repeatedly for the input chunks, as state-of-the-art methods cannot reuse KV-caches when chunks appear at arbitrary locations with arbitrary contexts. Naive reuse leads to output quality degradation.  This leads to potentially redundant computations on expensive GPUs and increases latency. In this work, we propose \sys, a system for managing and reusing precomputed KVs corresponding to the text chunks (we call \textit{chunk-caches}) in RAG-based systems. We present how to identify \hl{\textit{chunk-caches} that are reusable}, how to efficiently perform a small fraction of recomputation to \textit{fix} the cache to maintain output quality, and how to efficiently store and evict \textit{chunk-caches} in the hardware for maximizing reuse while masking any overheads. With real production workloads as well as synthetic datasets, we show that \sys reduces redundant computation by \textbf{51\%} over SOTA prefix-caching and \textbf{75\%} over full recomputation.
\hl{Additionally, with continuous batching on a real production workload, we get a \textbf{1.6$\times$} speedup in throughput and a \textbf{2$\times$} reduction in end-to-end response latency over prefix-caching while maintaining quality, for both the \llama-3-8B and \llama-3-70B models. 
}
\end{abstract}





\end{abstract}

\keywords{AI agent, Long-term planning agent, Existential risk, Irreversible global catastrophe, Human extinction, Inference-time computation, Reasoning, Autonomous action graph, Impact measure, Safety, Regulation}

\section{Introduction}
\label{sec:intro}

\begin{figure*}[tb]
    \centering
    \includegraphics[width=0.848\linewidth]{figs/circuitnn.pdf} 
    \caption{Illustration of differentiable CircuitNN. CircuitNN is designed based on differentiable NAND gates. After DAS is guided by PI and PO pairs of the truth table, CircuitNN can get the precise circuit architecture logic equivalent to the truth table.}
    \label{fig:circuitnn}
\end{figure*}

% 1. Describe the importance of logic synthesis
% 2. Existing Problems
% (a) Neural Architecture Search: Unstable, Predefined Setting, etc.
% (b) Circuit Generation: Probabilistic Model, Logic Equivalence

With the rapid advancement of technology, the scale of integrated circuits (ICs) has expanded exponentially. 
This expansion has introduced significant challenges in chip manufacturing, particularly concerning power and area metrics.
A primary objective in IC design is achieving the same circuit function with fewer transistors, thereby reducing power usage and area occupancy.

Logic synthesis~\cite{hachtel2005logicsynth}, a critical step in electronic design automation (EDA), transforms behavioral-level circuit designs into optimized gate-level circuits, ultimately yielding the final IC layout. 
The primary goal of logic synthesis is to identify the physical implementation with the fewest gates for a given circuit function. 
This task constitutes a challenging NP-hard combinatorial optimization problem. 
Current logic synthesis tools~\cite{brayton2010abc, wolf2013yosys} rely on human-designed heuristics, often leading to sub-optimal outcomes.

Differentiable architecture search (DAS) techniques~\cite{liu2018darts, chu2020darts} offer novel perspectives on addressing challenges in this problem.
Circuit functions can be represented through truth tables, which map binary inputs to their corresponding outputs. 
Truth tables provide a precise representation of input-output relationships, ensuring the design of functionally equivalent circuits.
Inspired by this, researchers~\cite{deepmind2024ai4sys, wang2024tnet} have begun exploring the application of DAS to synthesize circuits directly from truth tables.
Specifically, \citet{deepmind2024ai4sys} proposed CircuitNN, a framework that learns differentiable connection structures with logic gates, enabling the automatic generation of logic circuits from truth tables.
This approach significantly reduces the complexity of traditional circuit generation. 
Building on this, \citet{wang2024tnet} introduced T-Net, a triangle-shaped variant of CircuitNN, incorporating regularization techniques to enhance the efficiency of DAS.

Despite these advancements, several challenges remain. 
The computational complexity of DAS grows quadratically with the number of gates, posing scalability issues.
Although triangle-shaped architecture~\cite{wang2024tnet} partially mitigates this problem, redundancy persists. 
%Additionally, DAS is susceptible to converging to local optima, limiting the ability to search architectures that satisfy the given truth tables~\cite{liu2018darts}. 
%Furthermore, hyperparameters (network depth and layer width) require extensive searches, introducing complexity and prolonging the synthesis process. 
Additionally, DAS is susceptible to converging to local optima~\cite{liu2018darts} and hyperparameters (network depth and layer width) require extensive searches. 
The challenges arise from the vast search space in DAS. 
% Even with predefined settings for CircuitNN, finding a configuration that meets the truth table requires extensive trial and error during the DAS process. 
Intuitively, limiting the search space through predefined parameters (network depth, gates per layer, and connection probabilities) can significantly reduce the complexity.

Recent advances~\cite{openai2023gpt4, abramson2024alphafold3, esser2024sd3, li2024mar} in conditional generative models have demonstrated remarkable performance across language, vision, and graph generation tasks. 
Motivated by these developments, we propose a novel approach to circuit generation that generates preliminary circuit structures to guide DAS in generating refined circuits matching specified truth tables. 
Firstly, we introduce CircuitVQ, a tokenizer with a discrete codebook for circuit tokenization. 
Built upon our Circuit AutoEncoder framework~\cite{hou2022graphmae,li2023maskgae,wu2025mgvga}, CircuitVQ is trained through a circuit reconstruction task. 
Specifically, the CircuitVQ encoder encodes input circuits into discrete tokens using a learnable codebook, while the decoder reconstructs the circuit adjacency matrix based on these tokens.
Subsequently, the CircuitVQ encoder serves as a circuit tokenizer for CircuitAR pretraining, which employs a masked autoregressive modeling paradigm~\cite{chang2022maskgit, li2023mage}. 
In this process, the discrete codes function as supervision signals. 
After training, CircuitAR can generate discrete tokens progressively, which can be decoded into initial circuit structures by the decoder of the CircuitVQ. 
These prior insights can guide DAS in producing refined circuits that match the target truth tables precisely.

Our key contributions can be summarized as follows:
\begin{itemize}
\item We introduce CircuitVQ, a circuit tokenizer that facilitates graph autoregressive modeling for circuit generation, based on our Circuit AutoEncoder framework;
\item Develop CircuitAR, a model trained using masked autoregressive modeling, which generates initial circuit structures conditioned on given truth tables;
\item Propose a refinement framework that integrates differentiable architecture search to produce functionally equivalent circuits guided by target truth tables;
\item Comprehensive experiments demonstrating the scalability and capability emergence of our CircuitAR and the superior performance of the proposed circuit generation approach.
\end{itemize}

% Motivation
% (a) Diffusion (Vision, Graph), Autoregressive (Language, Vision)
% (b) Circuit Generation for Predefined Setting
% (c) Neural Architecture Search for Strict Logic Equivalence

% Contribution
% (a) Circuit Tokenizer (new transformer arch, training strategy)
% (b) CircuitAR (train and gen strategies, post-ar strategy)
% (c) Extensive Evaluation including BitD (Bit Distance) for Scalability

\section{LLM-empowered Agent}\label{sec:agent}
In this section, we delve into the potential of LLMs as intelligent agents in FLE. LLMs can act as catalysts for personalized learning, addressing the long-standing scalability, adaptability, and inclusivity challenges in traditional teaching paradigms.

\subsection{Fundamental Abilities}
This section highlights five key abilities of LLM-empowered agents that enable them to function as adaptive tutors.

\paragraph{Knowledge Integration.} LLMs excel at merging structured educational knowledge graphs~\cite{abu2024knowledge,hu2024foke} with unstructured textual data~\cite{li2024supporting,modran2024llm}, providing rich, contextualized information on linguistic constructs and cultural nuances. Their ability to perform real-time knowledge editing~\cite{wang2024knowledge,zhang2024comprehensive} ensures learners receive content aligned with evolving language usage, addressing the inherent limitations of static materials.

\paragraph{Pedagogical Alignment.} LLMs require embedding with pedagogical principles to facilitate genuine learning experiences~\cite{carroll1965contributions,taneja1995educational}. Recent work incorporates theoretical frameworks, such as Bloom’s taxonomy~\cite{bloom1956taxonomy}, to guide LLMs in systematically addressing different cognitive levels~\cite{jiang2024llms}. Approaches like \textit{Pedagogical Chain of Thought}~\cite{jiang2024llms} and \textit{preference learning}~\cite{sonkar-etal-2024-pedagogical,rafailov2024direct} focus on aligning model responses with educational objectives.

% \paragraph{Pedagogical Alignment.} To truly function as effective educational agents, LLMs must be imbued with pedagogical modes of thought~\cite{carroll1965contributions,taneja1995educational}, which can be achieved through pedagogical alignment~\cite{razafinirina2024pedagogical}. Inspired by Bloom Cognitive Model~\cite{bloom1956taxonomy}, which categorizes student abilities into six cognitive levels—Remember, Understand, Apply, Analyze, Evaluate, and Create—\citet{jiang2024llms} proposes \textit{Pedagogical Chain of Thought} to enhance mistake correction of LLMs without requiring fine-turning. \citet{sonkar-etal-2024-pedagogical} develop a synthetic preference dataset embedded with pedagogical principles and explore whether preference learning techniques~\cite{rafailov2024direct,azar2024general,ethayarajh2024kto} can align LLMs with educational objectives. The conversational CLASS~\cite{sonkar-etal-2023-class} framework equips intelligent tutoring systems with tutor-like step-by-step guiding strategies rather than merely providing direct answers to students.


\paragraph{Planning.} By assisting in crafting teaching objectives and lesson designs, LLMs can handle complex tasks such as differentiated instruction~\cite{hu2024teaching}. LessonPlanner~\cite{fan2024lessonplanner} has been proposed to assist novice teachers in preparing lesson plans, with expert interviews confirming its effectiveness. \citet{zheng2024automatic} propose a three-stage process to produce customized lesson plans, using Retrieval-Augmented Generation (RAG), self-critique, and subsequent refinement.

\paragraph{Memory.} Effective tutoring systems track learner histories and tailor subsequent interactions accordingly~\cite{jiang2024ai,chen2024empowering}. When serving as memory-augmented agents, LLMs can retain individualized data—such as repeated grammar mistakes or overlooked vocabulary—thereby improving continuity and enabling consistent scaffolding of future learning tasks.

\paragraph{Tool Using.} Beyond textual interactions, LLM-based agents can integrate specialized tools to streamline the educational ecosystem, from cognitive diagnosis modules~\cite{ma2019cognitive} to report generators~\cite{zhou2025study}. By orchestrating these resources, LLMs seamlessly unify diverse utilities under a single interface, enhancing learner experience and instructional efficiency.

\paragraph{Discussion.} Existing research often overlooks the interplay among listening, speaking, reading, and writing in real-world language learning~\cite{caines2023application,shetye2024evaluation}. Most systems concentrate on text-oriented features, lacking robust benchmarks and methodologies to evaluate integrated, multimodal interactions. Recent work on pedagogical alignment~\cite{razafinirina2024pedagogical} largely addresses textual data, leaving out real-time speaking and listening dynamics that demand complex, rapid feedback. Similarly, while memory modules can track repeated written errors, their effectiveness in monitoring and improving learning efficacy remains underexplored.


% \paragraph{Planning.} Planning is a cornerstone of successful education, involving setting instructional objectives, identifying teaching priorities, organizing teaching activities, articulating subject content, and selecting methods and strategies~\cite{hu2024teaching}. The design of teaching plans entails an abundant knowledge base and rich teaching experiences to tailor to learners' diverse needs,  with the constraint of available resources. Recent advances in LLMs demonstrate significant potential in planning complex tasks~\cite{huang2024understanding}. LessonPlanner~\cite{fan2024lessonplanner} has been proposed to assist novice teachers in preparing lesson plans, with expert interviews confirming its effectiveness. \citet{zheng2024automatic} propose a three-stage process to produce customized lesson plans, using Retrieval-Augmented Generation (RAG), self-critique, and subsequent refinement.

% \paragraph{Memory.} The ability to retain and recall information about a learner's progress is essential for fostering consistent and personalized education. LLMs can serve as memory-augmented agents~\cite{jiang2024ai}, maintaining detailed records of a learner's strengths, weaknesses, and historical performance. By leveraging this memory, LLMs can provide targeted feedback, revisit challenging topics, and build on prior knowledge in a coherent manner~\cite{chen2024empowering}. This continuity not only enhances the learner's experience but also ensures that educational interventions are data-driven and evidence-based.

% \paragraph{Tool Using.} LLMs can further extend their utility by interfacing with a variety of educational tools, such as cognitive diagnosis~\cite{ma2019cognitive} and report generation~\cite{zhou2025study}. LLMs can integrate these tools into a cohesive learning ecosystem by serving as a central orchestrator, streamlining the learner's experience. This ability to seamlessly incorporate external resources amplifies the effectiveness of LLM-driven educational systems.


\subsection{Applications}
Although still in its early stages, LLM-empowered agents have already started to show promising applications in FLE.
% These applications leverage the capabilities of LLMs to create more interactive and personalized learning environments, offering new ways to address traditional challenges in language instruction. Despite being in development, these applications have the potential to transform how language learners engage with content and interact with tutoring systems.


\paragraph{Classroom Simulation.} Classroom simulation leverages LLM-empowered agents to recreate complex, interactive learning settings without the logistical hurdles of organizing physical classrooms~\cite{zhang2024simulating}. By simulating virtual students and tutors, researchers can study pedagogical strategies at scale, generate diverse learner interactions, and refine teaching techniques. Moreover, this virtual data can be used to fine-tune LLMs for specific educational contexts and learner profiles~\cite{liusocraticlm}, offering a cost-effective and adaptable approach to language instruction.

\paragraph{Intelligent Tutoring System (ITS).} LLM-based agents have demonstrated the capacity to provide dynamic, personalized tutoring experiences~\cite{kwon-etal-2024-biped}, effectively identifying learner weaknesses through large-scale linguistic analysis~\cite{caines2023application}. This makes them promising for delivering individualized instruction at scale. Although current ITS applications in mathematics~\cite{pal2024autotutor} and science~\cite{stamper2024enhancing} have shown success, the extension to FLE requires nuanced handling of cultural and contextual elements, as well as the unpredictability of human language usage.

\paragraph{Discussion.} Despite the promise of these applications, critical challenges remain. Existing classroom simulation frameworks often \textit{lack standardized benchmarks for FLE}, making it difficult to assess the efficacy and generalizability of developed systems~\cite{zhang2024simulating}. In addition, evaluating language-specific tutoring strategies, including real-time conversational practice and holistic skill integration, remains an underexplored frontier. Addressing these gaps requires \textit{new datasets and metrics} centered on holistic skill development, as well as interdisciplinary collaboration.

\begin{tcolorbox}[top=1pt, bottom=1pt, left=1pt, right=1pt]
\textbf{Our position.} We argue that to overcome current limitations, \textit{future research} should focus on the integration of multimodal learning tasks~\cite{sonlu2024effects} and the development of standardized frameworks for evaluating FLE simulations. Moreover, LLMs should evolve beyond text-based capabilities to provide real-time, context-sensitive feedback, particularly in speaking and listening. Interdisciplinary collaboration and the creation of new datasets tailored to FLE are crucial for refining these systems and ensuring their scalability and inclusivity in language instruction. Additionally, addressing the complexities of cultural context and learner variability will be key to the success of LLMs as effective agents in FLE.
\end{tcolorbox}


% \paragraph{Classroom Simulation.} Classroom simulation is a powerful application of LLM-empowered agents that addresses the limitations of traditional language learning environments. By simulating interactions among virtual students and tutors, multi-agent setups create realistic, dynamic classroom scenarios that reflect real-world complexities. This approach not only facilitates the study of pedagogical strategies but also accelerates educational research by eliminating the logistical challenges of organizing physical classrooms~\cite{zhang2024simulating}. Additionally, simulated classroom data can be used to fine-tune LLMs, enhancing their ability to adapt to diverse teaching contexts and learner profiles~\cite{liusocraticlm}. These simulations offer scalability, cost-efficiency, and the opportunity to test various instructional methods in controlled yet realistic environments.

% \paragraph{Intelligent Tutoring System (ITS).} LLM-powered agents have the potential to create an immersive learning environment and provide a personalized alternative to traditional one-on-one tutoring~\cite{kwon-etal-2024-biped}. Furthermore, their ability to process vast amounts of linguistic data enables them to identify and address specific learner weaknesses~\cite{caines2023application}, making them particularly effective in delivering tailored instruction at scale. However, recent efforts are mainly on non-language subjects like mathematics~\cite{pal2024autotutor} and science~\cite{stamper2024enhancing} which hold more objective learning goals.

\section{Alternative views}
\label{sec:alternative}

One broadly applicable metric for regulating AI agents is the amount of computation they require. Here, we explore how existing regulations and recommendations from AI researchers often center on computational resources. While these efforts primarily address the computational resources for training, we expand the discussion to include computational demands for inference. We then argue that limiting the focus to these aspects alone is inadequate for the existential risks associated with advanced AI agents.

\subsection{Computational resources for development}

In this section, we briefly review existing regulations, focusing on the EU AI Act and President Biden's executive order, along with key recommendations from AI researchers.

\subsubsection{EU AI Act}
%https://yourlearning.ibm.com/activity/UDEMY-5863828
% https://www.noandt.com/wp-content/uploads/2021/04/technology_no6.pdf

\citet{EUAIAct2024} has established a set of rules (the EU AI Act) for the development, deployment, and use of AI within European Union.  Its Chapter 5 defines ``general-purpose AI models with systemic risk'' and lists obligations for providers of such models.  Here, general-purpose AI models essentially refer to FMs, which are pre-trained with self-supervised learning and can (be adapted to) perform a wide range of downstream tasks (see Article 3(63)).  Also, systemic risk refers to ``a risk that is specific to the high-impact capabilities of general-purpose AI models, having a significant impact ... on public health, safety, public security, fundamental rights, or the society as a whole, that can be propagated at scale across the value chain'' (see Article 3(65)).

In particular, a ``general-purpose AI model shall be presumed to have high impact capabilities ... when the cumulative amount of computation used for its training measured in [FLOPs] is greater than $10^{25}$'' (see Article 51(2)).  When this or other specified conditions are met, the provider of a general-purpose AI model is required to fulfill certain obligations, such as providing technical documentation about the model.

The advanced AI agent that we consider can certainly be classified as general-purpose AI.  Also, the existential risks that we consider can be considered as systemic risks, although human extinction and irreversible global catastrophes are not discussed in the EU AI Act.

\subsubsection{Biden's executive order and related bills}

In October 2023, Joe Biden, then president of the US, signed the Executive Order on the Safe, Secure, and Trustworthy Development and Use of Artificial Intelligence \cite{biden2023executive}\footnote{This executive order was repealed by President Trump.}.  Its Section 4.2 is dedicated to ensuring safe and reliable AI.  In particular, it requires companies to report on ``any model that was trained using a quantity of computing power greater than $10^{26}$ integer or [FLOPs]'' until a set of technical conditions for models are defined by specified authorities (see Section 4.2(b)).

In this executive order, particular attention is paid to a dual-use FM, which refers to a FM that exhibits ``high levels of performance at tasks that pose a serious risk to security, national economic security, national public health or safety, or any combination of those matters, such as by ... permitting the evasion of human control or oversight through means of deception or obfuscation'' (see Section 3(k)).

% https://web.archive.org/web/20250118020619/https://www.whitehouse.gov/briefing-room/presidential-actions/2023/10/30/executive-order-on-the-safe-secure-and-trustworthy-development-and-use-of-artificial-intelligence/

Following this executive order, almost 700 AI-related bills are introduced in 45 states across the United States in 2024 \cite{bsa2024state}.  A particularly interesting one is California Senate Bill 1047 (Safe and Secure Innovation for Frontier Artificial Intelligence Models Act) \cite{wiener2024senate}\footnote{The bill had passed the state legislature but was later vetoed by the Governor.  The technical feasibility of the requirements has also been questioned by the community \cite{ai2024statement}.}.  Its Chapter 22.6 is devoted to safe and secure innovation for frontier artificial intelligence models, which cover ``[a]n artificial intelligence model trained using a quantity of computing power greater than $10^{26}$ integer or floating-point operations.''  In particular, the senate bill requires that ``[b]efore beginning to initially train a covered model, the developer shall ... [i]mplement the capability to promptly enact a full shutdown,'' which completely halts the operations of the model.

% https://leginfo.legislature.ca.gov/faces/billStatusClient.xhtml?bill_id=202320240SB1047

The necessity of such an off-switch \cite{hadfieldmenell2017offswitch} is motivated to prevent ``critical harms,'' which include ``[m]ass casualties or at least five hundred million dollars (\$500,000,000) of damage resulting from an artificial intelligence model engaging in conduct that ... [a]cts with limited human oversight, intervention, or
supervision.''

% Key 2024 Statistics: State lawmakers across the United States introduced almost 700 AI-related bills in 2024 across 45 states % https://www.bsa.org/news-events/news/2025-state-ai-wave-building-after-700-bills-in-2024

\subsubsection{Recommendations by scientists}

We have seen that the computational resources used for training is one of the major criteria to judge whether an AI model can have unacceptably high risk.  Although specific values of the threshold such as $10^{25}$ or $10^{26}$ FLOPs are subject to change, the amount of computation for training is one of the most reliable metrics that AI researchers can currently provide for approximating the performance and potential risks of a wide range of FMs.

For example, \citet{anderljung2023frontier} recommend to identify sufficiently dangerous frontier AI models on the basis of whether they are trained with more than $10^{26}$ FLOPs of computation.  More recently, \citet{cohen2024regulating} argue that ``[s]ystems should be considered `dangerously capable' if they are trained with enough resources to potentially exhibit those dangerous capabilities, and regulators should not permit the development of dangerously capable LTPAs.''  Although they do not specify exactly what is considered as enough resources, the amount of computation for training is the only specific criterion that they suggest to determine whether an LTPA can have existential risk. 

\citet{future2023policymaking} have also provided recommendations for governments on managing AI risks.  The recommendations include mechanisms such as auditing, certification, and regulation, grounded in the assumption that ``[t]he amount of compute used to train a general-purpose system largely correlates with ... the magnitude of its risks'' \cite{future2023policymaking}.  These recommendations have been made by following an open letter that called for ``all AI labs to immediately pause for at least 6 months the training of AI systems more powerful than GPT-4,'' which was issued in response to the severe societal risks posed by advanced AI systems and signed by AI researchers \cite{future2023pause}.

In fact, the amount of computation used for training may serve as a sufficiently reliable predictor of the performance and risks of today’s LLMs. This is because most existing LLMs share the same Transformer architecture, differing primarily in their size and the volume of training data. Several studies have examined scaling laws that describe the relationship between a model's optimal size, the amount of training data, and the computational budget required \cite{kaplan2020scaling,hoffmann2022training}.  Research has also shown that various abilities, such as multi-step reasoning, tend to emerge as the computational resources used for training increase \cite{wei2022emergent}.

These scaling laws can, in turn, be used to estimate the FLOPs needed to train existing LLMs.  For example, \citet{anil2023palm2} propose a heuristic suggesting that an LLM should be trained with $6\,N\,D$ FLOPs, where $D$ is the amount of training data, and $N$ is the model size.  Using this heuristics, Llama 3.1 405B trained on 15 trillion tokens is estimated to require approximately $4 \times 10^{25}$ FLOPs---an amount closely aligning with the thresholds specified in the EU AI Act and President Biden's executive order.


\subsection{Computational resources for operation}
\label{sec:alternative:operation}

We have seen that the amount of computation used for training is one of the key criteria that is used today to regulate highly capable FMs.  Although such regulations may be effective for AI agents whose capabilities primarily and directly stem from traditional FMs, they obviously fail to regulate advanced AI agents that gain substantial reasoning capabilities from computation at inference time.  

A natural approach is to regulate advanced AI agents based on the amount of computation used not only for training but also for inference. While this approach may be effective for some AI agents, we argue that it is insufficient for advanced AI agents, at least for the following four reasons.

First, scaling laws for inference-time computation are much less established than those for training-time computation. Recently, scaling laws have been studied for a limited number of inference strategies in LLMs \cite{chen2024simple,brown2024large,snell2024scaling,wu2024inference}. However, there are many potential inference strategies, and their performance can scale very differently. The scaling law for inference also depends on how difficult the task is and what LLM is used \cite{snell2024scaling,openai2024openai}. 
%As a result, we cannot simply assume that inference is safe simply because it does not require heavy computation.

%\cite{chen2024simple}: inference scaling law for a particular inference strategy (sample many, and select one with tournament)
%\cite{brown2024large}: inference scaling law for a particular inference strategy (sample many, and select one)
%\cite{snell2024scaling,wu2024inference}: inference scaling laws for several selected inference strategies

Second, inference can be performed in parallel by multiple entities. For example, several entities may operate the same AI agent that perform reasoning with MCTS \cite{luo2024improve,zhang2024restmcts}, either collaboratively or independently, possibly without knowing each other. Since MCTS is a randomized algorithm, the likelihood that one of the agents optimally solves the task increases with the number of agents. However, this also means that this agent may exploit a loophole, solving the task super-optimally in a way that violates critical constraints, potentially leading to catastrophic outcomes. 
%Therefore, even if each entity’s computation is negligible, a group of entities could collectively exceed a threshold that ensures the safety of inference.

Third, it is not always clear what constitutes a single run of inference. For instance, the results of one inference run may be stored and used in another. Intermediate results could also be stored and later retrieved by a different AI agent, who may or may not be aware that the information is from a previous inference run. More broadly, reasoning can be enhanced through retrieval augmentation \cite{pouplin2024retrieval}, where retrieved information may have been generated with substantial computational resources.
%Therefore, regulating the computational resources spent on a single, arbitrarily defined inference run may not be sufficient.

Finally, AI agents may continually learn over time, which makes it difficult to separate inference from training. For example, an AI agent may perform reasoning with MCTS, with an LLM performing a step in the process. Once the agent identifies a good sequence of steps, it may fine-tune the LLM in a way that the LLM can perform the entire sequence in a single step, bypassing the reasoning process.  As this learning progresses, the agent will gain the ability to perform high-level reasoning with limited computation.
%However, since training and inference can be performed by different entities, tracking the total computational resources used by an AI agent becomes challenging.

% MCTS \cite{kocsis2006bandit,browne2012survey} at run time.  counterfactual regret minimization \cite{zinkevich2007regret}.

% 民間企業においては、AI の開発に際し、政府機関に安全性テストの報告等が必要になる。ただし、EU の AI 規制(案)のように一部の AI の開発・利用を禁止するものではなく、AI 大統領令は政府機関が策定する基準やベストプラクティスに基づいた対応を求めるにとどまる。
% 。いずれも連邦当局に対して基準やベストプラクティスの策定、一定の対象者に報告義務を課す等の緩やかな規制であり、EU の AI 規則(案)7のように非常にリスクの高い AI を禁止する等の AI の利用そのものを制限するものではない。
% https://www.dir.co.jp/report/research/law-research/law-others/20231130_024115.pdf

% いずれも連邦当局に対して基準やベストプラクティスの策定、一定の対象者に報告義務を課す等の緩やかな規制であり、EU の AI 規則(案)7のように非常にリスクの高い AI を禁止する等の AI の利用そのものを制限するものではない。いずれも連邦当局に対して基準やベストプラクティスの策定、一定の対象者に報告義務を課す等の緩やかな規制であり、EU の AI 規則(案)7のように非常にリスクの高い AI を禁止する等の AI の利用そのものを制限するものではない。
% dual use FMs:悪用されると安全保障、国家経済安全保障、国家公衆衛生・安全に対する深刻なリスクをもたらしうるAIモデル dual use FMsの開発者に対し、AI red-teamingテストの結果やトレーニングに関する情報の報告義務を課す 4.2項(a)(i)
% https://www.noandt.com/wp-content/uploads/2023/11/technology_no43_1.pdf

% \begin{figure}
%     \centering
%     \includegraphics[width=0.5\linewidth]{Move_teaser.pdf}
%     \caption{Comparison of different dynamic compute approaches. length of arrow indicates residual transformation per token while width indicates velocity of transformation.}
%     \label{fig:enter-label}
% \end{figure}

\section{Method}
\label{sec:method}
Residual connections play a crucial role in shaping token representations, yet their dynamics remain underexplored in the context of efficient decoding. In this work, we delve deeper into transformer residual dynamics and investigate how modulating residual transformation velocity can improve inference efficiency in token-level processing, optimizing both dense and sparse MoE transformers.


\subsection{Residual Dynamics and Motivation for Multi-rate Residuals} \label{sec:motivation}

To analyze how hidden representations evolve across different layers of a transformer architecture, it's crucial to consider the effect of residual connections. Each transformer decoder layer typically has residual connections across attention and MLP submodules. As the residual stream $h_i$ traverses from interval $E_j$ to $E_{j+1}$, it undergoes a residual transformation given by:  
% \begin{equation}
% \label{eq:slow_residual_transformation}
% H_{E_{j+1}} = H_{E_j} \prod_{i=E_j}^{E_{j+1}} \left( I + \mathcal{A}_i \right) \left( I + \mathcal{M}_i \right) \quad \text{where} \quad \mathcal{A}_i = f(c_i, h_{i}), \mathcal{M}_i = g(h_i)
% \end{equation}

\begin{equation} \label{eq:slow_residual_transformation}
h_{E_{j+1}} = h_{E_j} + \sum_{i=E_j}^{E_{j+1}-1} \left( \mathcal{A}_i(h_i) + \mathcal{M}_i(h_i + \mathcal{A}_i(h_i)) \right) \quad \text{where} \quad \mathcal{A}_i = f(c_i, h_{i}), \mathcal{M}_i = g(h_i). 
\end{equation}

Here, \( \mathcal{A}_i \) denotes the non-linear transformation introduced by the multi-head attention mechanism at layer \( i \), while \( \mathcal{M}_i \) corresponds to the non-linear transformation of the MLP block at the same layer. These transformations depend on the input residual stream \( h_i \) and, in the case of \( \mathcal{A}_i \), the previous contextual representation \( c_i \).\footnote{Normalization layers are typically applied in practice but are omitted here for simplicity of the argument.}


% For easy tokens, the magnitude and direction of this delta transformation become progressively smaller with each successive layer as shown in \cref{fig:delta_transformation}. Consequently, it is feasible to predict these tokens after only a few residual connections, whereas harder tokens necessitate more extensive processing through additional layers.

\begin{figure}[ht]
    \centering
    \begin{subfigure}{0.48\textwidth}
        \centering
        \includegraphics[width=\textwidth]{sections/figures/residual_change.pdf}
        \caption{}
        \label{fig:residual_change}
    \end{subfigure}%
    \hfill
    \begin{subfigure}{0.48\textwidth}
        \centering
        \includegraphics[width=\textwidth]{sections/figures/alignment_wrt_dedicated_model.pdf}
        \caption{}
    \label{fig:alignment_wrt_dedicated_model}
    \end{subfigure}
    \caption{(a) As residual streams propagate through the model, the directional shifts in the residuals become progressively smaller. (b) A dedicated model with $k$ layers achieves a faster rate of change in residual streams and higher alignment than base model leveraging early exit mechanisms at layer $k$.}
    \label{fig}
\end{figure}


To examine whether residual transformations can be accelerated across layers, we conducted experiments using a diverse set of prompts on a pre-trained Phi3 model~\cite{phi3_report}. As illustrated in \cref{fig:residual_change}, we measured the directional shift in residual states as \( 1 - \mathcal{C}(h_{i-1}, h_i) \), where \(\mathcal{C}\) denotes normalized cosine similarity. This shift is notably higher in the initial layers, gradually decreasing in subsequent layers. This behavior allows traditional early exit approaches to effectively accelerate decoding by enabling earlier exits for simpler tokens. However, these approaches typically rely on a distance-based approximation, where the full residual transformation of the model is approximated by the residual transformations of the initial layers. To gain deeper insights into the distance versus velocity aspects of residual transformation, we conducted a comparative study. Specifically, we trained an early exit head at layer $k$ of the Phi3 model, which consists of 32 layers, restricting the distance traveled by each token. To accelerate the residual transformation relative to number of layers, we trained a smaller model consisting of only $k$ layers, while keeping all other hyperparameters consistent. We then compared the next-token prediction accuracy of the early exit head of the base model with that of the smaller model. To ensure an equal number of trainable parameters, we inserted low-rank adapters into the smaller model and trained only these adapters, whereas, in the distance-based approach, we trained solely the early exit head. In addition, to accelerate the residual transformation in smaller model, we distilled the residual streams from the larger model by incorporating a distillation loss ~\cite{sanh2019distilbert} between the residual state at layer \(i\) of the smaller model and the residual state at layer \(4 \times i\) of the larger model. As shown in ~\cref{fig:alignment_wrt_dedicated_model} the smaller model demonstrates a significantly faster rate of change in residual streams, leading to higher next token prediction accuracy after $k$ layers compared to the base model that employs traditional early exit mechanisms after $k$ layers \cite{schuster2022confident, chen2023eellm, varshney-etal-2024-investigating}. This experimental setup, which modifies only the rate of change in residual streams while keeping other factors constant, suggests that dense transformers, trained with a fixed number of layers, may inherently possess a slow residual transformation bias.

This observation raises an intriguing question: if the rate of change in residual streams could be accelerated relative to the number of layers, is it possible to facilitate earlier alignment for a greater proportion of tokens? Earlier alignment would be beneficial to not only facilitate dynamic computation but also for generating speculative tokens efficiently with high acceptance rates in speculative decoding setups ~\cite{leviathan2023fast, chen2023accelerating}. 

%thereby enhancing the efficiency of early exiting? 
 % This bias likely constrains the effectiveness of early exiting, particularly for easier tokens. By addressing this limitation through accelerated residual transformations, we hypothesize that it is possible to substantially improve the efficiency and accuracy of early exit strategies in transformer models.

\subsection{Multi-Rate Residual Transformation} \label{m2r2_method}

To address the slow residual transformation bias described in ~\cref{sec:motivation}, we introduce \textit{accelerated residual streams} that operate at rate $R$ relative to original slow residual stream. We pair slow residual stream, $h$ with an accelerated residual stream, $p$, which has an intrinsic bias towards earlier alignment. Relative to ~\cref{eq:slow_residual_transformation}, accelerated residual transformation from interval $E_j$ to $E_{j+1}$ can be represented as: 

% \begin{equation}
% \label{eq:fast_residual_transformation}
% P_{E_{j+1}} = P_{E_j} \prod_{i=E_j}^{E_{j+1}} \left( I + \hat{\mathcal{A}_i} \right) \left( I + \hat{\mathcal{M}_i} \right) \quad \text{where} \quad \hat{\mathcal{A}_i} = \hat{f}(c_i, P_{i}), \hat{\mathcal{M}_i} = \hat{g}(P_{i})
% \end{equation}


\begin{equation} \label{eq:fast_residual_transformation}
p_{E_{j+1}} = p_{E_j} + \sum_{i=E_j}^{E_{j+1}-1} \left( \hat{\mathcal{A}_i}(p_i) + \hat{\mathcal{M}_i}(p_i + \hat{\mathcal{A}_i}(p_i)) \right) \quad \text{where} \quad \hat{\mathcal{A}_i} = \hat{f}(c_i, p_{i}), \hat{\mathcal{M}_i} = \hat{g}(h_i), 
\end{equation}



where $\hat{\mathcal{A}_i}$ and $\hat{\mathcal{M}_i}$ denote non-linear transformation added by layer $i$ to previous accelerated residual $p_{i}$. Similar to $\mathcal{A}_i$, non-linear transformation $\hat{\mathcal{A}_i}$ attends to same context $c_i$ but uses a different transformation $\hat{f}$ for accelerating $p_{E_j}$ relative to $h_{E_j}$. 

We integrate accelerated residual transformation directly into the base network using parallel accelerator adapters such that rank of accelerator adapters $R_p << d$ where $d$ denotes base model hidden dimension. This setup allows the slow residual stream $h_{E_j}$ to pass through the base model layers while the accelerated residual stream $p_{E_j}$ utilizes these parallel adapters as shown in ~\cref{fig:m2r2_main}. Both slow and accelerated residuals are processed in same forward pass via attention masking and incur negligible additional inference latency in memory bound decoding setups, while in compute bound decoding setups where FLOPs optimization is essential, accelerated residual stream utilizes a fraction of attention heads that of slow residual (see ~\cref{sec:flops_optimization}). Additionally, to maximize the utility of accelerated residual transformations without introducing dedicated KV caches, we propose a shared caching mechanism between the slow and accelerated streams which minimally impact alignment benefits of our approach while offering substantial memory savings (see ~\cref{fig:koala_alignment}). Specifically, the attention operation on the slow residuals \( \text{MHA}(h_t, h_{\leq t}, h_{\leq t}) \) is redefined for accelerated residuals as 
\[
\hat{\mathcal{A}} = MHA(p_t, h_{<t} \oplus p_t, h_{<t} \oplus p_t),
\]
where the accelerated residual at time-step $t$, \( p_t \) attends to the slow residual’s KV cache, facilitating the reuse of contextual information across both residual streams without incurring additional caching costs. Here, \(MHA(q, k, v) \) represents multi-head attention between query \( q \), key \( k \), and value \( v \).

\begin{figure}
    \centering
    \includegraphics[width=0.8\linewidth]{sections//figures/m2r2_main2.pdf}
    \caption{Multi-rate Residuals Framework: Slow residual stream of base model is accompanied by a faster stream that operates at a $2-(J+1)\times$ rate relative to the slow stream, undergoing transformations via accelerator adapters as detailed in \cref{m2r2_method}, where J denotes number of early exit intervals. Colors within the slow and fast residual streams indicate similarity, with matching colors representing the most closely aligned residual states. At the beginning of the forward pass and at each exit point, the accelerated residual state is initialized from the corresponding slow residual state to avoid gradient conflict during training (see ~\cref{sec:grad_conflict}). Early exiting decisions are informed by the Accelerated Residual Latent Attention (ARLA) mechanism, described in \cref{method_arla}, which evaluates residual dynamics across consecutive exit gates.}
    \label{fig:m2r2_main}
\end{figure}

% Furthermore. to maximize the benefits of fast residual transformations without using dedicated KV caches, we propose sharing the fast network’s cache with the slow network. Formally speaking, We modify attention operation on slow residuals $MHA(H_t, H_{<=t}, H_{<=t})$ as $MHA(P_{t}, H_{<t} \oplus P_t, H_{<t}  \oplus P_t)$ such that accelerated residuals attend to previous slow context KV cache, where $MHA(q,k,v)$ denotes multi head attention between query, $q$, key $k$ and value $v$.


\subsection{Enhanced Early Residual Alignment}
Early residual alignment is instrumental in optimizing early exiting, speculative decoding, and Mixture-of-Experts (MoE) inference mechanisms. In this section, we provide a detailed analysis of how accelerated residuals enhance these inference setups.

% By aligning the residual states of intermediate layers with the final output representations, the model can maintain high prediction accuracy even when computations are truncated at earlier layers. This enables more reliable early exiting, reducing the overall computational cost while preserving performance. Additionally, in speculative decoding, early residual alignment allows the model to make confident predictions using faster, partial computations, thereby accelerating inference without sacrificing output quality.


\subsubsection{Early Exiting} \label{method_early_exiting}

A prevalent strategy for enabling early exiting at an intermediate layer $E_{j}$ involves approximating the residual transformation between $E_{j}$ and the final layer $N-1$ using a linear, context independent mapping, $\mathcal{T}$, such that $H_{N-1} \approx \mathcal{T}(H_{E_{j}})$. This approximation has been extensively employed in conventional approaches ~\cite{schuster2022confident, chen2023eellm, varshney-etal-2024-investigating}, providing a computationally efficient means to project the output of deeper layers from intermediate states. Specifically, residual state of layer $N-1$ with this approximation can be expressed as:


% \begin{equation}
% \label{eq: vanila_ea_assumption}
% \Phi(H_{E_{j}}) \sim H_{E_{j}} \prod_{i=E_{j}}^{N}\left( I + \mathcal{A}_i \right) \left( I + \mathcal{M}_i \right) \quad \text{where} \quad \Phi \perp C
% \end{equation}

\begin{equation} \label{eq:early_exiting}
h_{E_j} + \sum_{i=E_j}^{N-1} \left( \mathcal{A}_i(h_i) + \mathcal{M}_i(h_i + \mathcal{A}_i(h_i)) \right) \sim \mathcal{T}(h_{E_{j}})  \quad \text{where} \quad \mathcal{T} \perp c. 
\end{equation}


Here, $\mathcal{A}_i$ and $\mathcal{M}_i$ represent the residual contributions of the multi-head attention and MLP layers, respectively, while $\mathcal{T}$ remains independent of $c$, the preceding context.

This approach is inherently limited by two major factors: first, the assumption of linearity between $h_{E_{j}}$ and $h_{N-1}$ may not hold uniformly for all tokens, particularly when $E_j \ll N$. Second, the linear transformation $\mathcal{T}$ disregards the influence of the context $c$ and fails to account for the latent representations of previous contextual states. In contrast, M2R2 accelerated residual states mitigate both of these challenges by approximating the slow residual transformation of all layers via a faster residual transformation of fewer layers as:
% \begin{equation}
% H_{E_j} \prod_{i=E_j}^{N}\left( I + \mathcal{A}_i \right) \left( I + \mathcal{M}_i \right) \sim P_{E_j} \prod_{i=E_j}^{E_j+1}\left( I + \hat{\mathcal{A}_i} \right) \left( I + \hat{\mathcal{M}_i} \right)
% \end{equation}


\begin{equation} \label{eq:m2r2_approximating_ea}
h_{E_j} + \sum_{i=E_j}^{N-1} \left( \mathcal{A}_i(h_i) + \mathcal{M}_i(h_i + \mathcal{A}_i(h_i)) \right) \sim p_{E_j} + \sum_{i=E_j}^{E_{j+1}-1} \left( \hat{\mathcal{A}_i}(p_i) + \hat{\mathcal{M}_i}(p_i + \hat{\mathcal{A}_i}(p_i)) \right), 
\end{equation}

% \begin{equation} \label{eq:fast_residual_transformation}
% p_{E_{j+1}} = p_{E_j} + \sum_{i=E_j}^{E_{j+1}-1} \left( \hat{\mathcal{A}_i}(p_i) + \hat{\mathcal{M}_i}(p_i + \hat{\mathcal{A}_i}(p_i)) \right) \quad \text{where} \quad \hat{\mathcal{A}_i} = \hat{f}(c_i, p_{i}), \hat{\mathcal{M}_i} = \hat{g}(h_i) 
% \end{equation}






where $p_{E_j}$ is initialized from the slow residual state $h_{E_j}$ at each early exit interval $E_j$ using an identity transformation (see ~\cref{fig:m2r2_main}). As shown in ~\cref{fig:m2r2_residual_sim}, accelerated residuals offer a smoother, more consistent shift in residual direction across layers, in contrast to the abrupt changes typically seen at early exit points in standard early exit methods. Moreover, the normalized cosine similarity between accelerated states at early exit intervals and final residual states is substantially higher compared to traditional early exit techniques, highlighting improved alignment with final layer representations. Traditional adaptive compute methods are constrained by two principal factors: the number of tokens eligible for early exit at intermediate layers and the precision of early exit decision. If residual streams fail to saturate early, the majority of tokens remain ineligible for exit, thereby diminishing potential speedups. Additionally, imprecise delineations between tokens suitable for early exit can lead to underthinking (premature exits that adversely affect accuracy) or overthinking (unnecessary processing that compromises efficiency) ~\cite{zhou2020self, dai2020dynamic}. Enhanced early alignment using ~\cref{eq:m2r2_approximating_ea} helps to address  first issue. To address the second issue we introduce Accelerated Residual Latent Attention, which dynamically assesses the saturation of the residual stream, allowing for a more precise differentiation between tokens that can exit early and those requiring further processing.

% This results in uniform change in residual direction    
% % We keep $\mathcal{A} = \hat{\mathcal{A}}$, while $\hat{\mathcal{M}}$ is accelerated by a factor of $2 - (N_{E}+1)X$ relative to the slower residual transformation $\mathcal{M}$, where $N_E$ represents number of early exiting intervals.
% Figure~\cref{fig:rate_change_comparison} illustrates the comparative rate of change between these transformation streams.



% fig:rate_change_comparison
% - grid plot x axis -> layer id (0, 8) , y axis -> layer id -> dark color cell for max similarity , lighter for lower 
% 
-------------------------------------------------------
Let's consider residual stream $h_i$ traverses through interval $E_j$ to $E_{j+1}$ and undergoes residual transformation given by 
\begin{equation}
h_{E_{j+1}} = h_{E_j} \prod_{i=E_j}^{E_{j+1}} \left( 1 + \delta_i \right)    
\end{equation}

where $\delta_i$ denotes non-linear transformation added by layer $i$. Each non-linear transformation of layer $i$ is a function of previous contextual representation, $c_i$ and input residual stream $h_i-1$ as
$\delta_i = f(c_i, h_{i-1})$ 

One way to exit early at exit $E_j+1$ is to assume that residual transformation from $E_j+1$ to final layer $N-1$ can be approximated by a linear function $\phi$ as $h_{N-1} \sim \Phi(h_{E_j+1})$ and most conventional approaches such as \todo{cite EA papers} use this approach. In other words, 

\begin{equation}
\Phi(h_{E_j+1} \sim h_{E_j+1} \prod_{i=E_j+1}^{N} \left( 1 + \delta_i \right)   
\end{equation}

This approach suffers from two primary issues, linearity assumption from $h_E_j+1$ to $H_N-1$ if often incorrect, particularly when $E_j << N$. More importantly, linear transformation $\Phi$ doesn't consider effect of context $C_i$. M2R2  effectively addresses these issues as accelerated residual stream at interval $E_j+1$ can be represented as 

\begin{equation}
r_{E_{j+1}} = r_{E_j} \prod_{i=E_j}^{E_{j+1}} \left( 1 + \gamma_i \right)    
\end{equation}

where $\gamma_i$ denotes non-linear transformation added by layer $i$ to previous accelerated residual $r_i-1$. Similar to $\delta_i$, non-linear transformation $\gamma_i$ considers context $C_i$ as 
$\gamma_i = g(c_i, r_{i-1})$. So in summary, slow residual transformation is approximated by accelerated residual as: 

\begin{equation}
h_{E_j} \prod_{i=E_j}^{N} \left( 1 + \delta_i \right) \sim h_{E_j} \prod_{i=E_j}^{E_j+1} \left( 1 + \gamma_i \right)
\end{equation}

It's worth noting that accelerated residual $r_i$ and slow residual $h_i$ are processed concurrently at layer $i$ by constructing proper attention mask such as attention of slow residual is represented as 

$MHA(H_it, H_{i<=t}, H_{i<=t}$ while attention of fast residual is computed as 

$MHA(r_it, H_{i<=t}, H_{i<=t}$ where $MHA(q,k,v$ denotes multi head attention between query, $q$, key $k$ and value $v$.


------------------------------------------------------------------

Vertical latent attention on accelerated residual is computed as 
$MHA(S_mt, S(Ej<=i<=m)t, S(Ej<=i<=m)t)$ where $Smt$ denotes query/key/value projection in latent domain at layer $m$ at time $t$. 
------------------------------------------------------------------

Gradient conflict Avoidance: 

Let's consider $w_j$ is a trainable parameter that belongs to a layer between $E_j$ and $E_j+1$. Consider early exit loss at gate $E_j+1$, $L_j+1$, gradient propagation of $w_j$ at another trainable parameter $w_j-n$ can be gives as 

$\sum_{k=E_j-n}^{E_j} \beta_k \frac{\partial L_{E_k}}{\partial w_k}$

where $\beta_j$ denotes backward transformation coefficient for weight $w_j$ to reach gate $E_j$. 
 
On the other hand, gradient propagation in proposed approach can be represented as 

\[
\frac{\partial L_{E_j}}{\partial w_j} = 
\begin{cases} 
\beta_j \frac{\partial L_{E_j}}{\partial w_j} & \text{if } E_j \leq w_j \leq E_{j+1} \\
0 & \text{otherwise}
\end{cases}
\]







% \begin{figure}[ht]
%     \centering
%     \includegraphics[width=0.8\textwidth, height=5cm]{rate_change_comparison.png}
%     \caption{Rate of change comparison between fast and slow residual streams.}
%     \label{fig:rate_change_comparison}
% \end{figure}

%vary k and and plot EA accuracy for larger and smaller models. 

% \begin{figure}[ht]
%     \centering
%     \includegraphics[width=0.5\textwidth,height=5cm]{sections/figures/alignment_comparison_dialogsum.pdf}
%     \caption{Alignment of exited tokens for different early exit layers using traditional early exiting heads, dedicated faster networks, and faster residuals.}
%     \label{fig:small_model_early_exiting}
% \end{figure}


\textbf{Accelerated Residual Latent Attention} \label{method_arla}

In the context of residual streams, we observe that the decision to exit at a given layer can be more effectively informed by analyzing the dynamics of residual stream transformations, instead of solely relying on a classification head applied at the early exit interval $E_j$. To capture the subtle dynamics of residual acceleration, we propose a \textit{Accelerated Residual Latent Attention} (ARLA) mechanism. This approach involves making the exit decision at gate $E_j$ by attending to the residuals spanning from gate $E_{j-1}$ to $E_j$, rather than considering only the residual at gate $E_j$. To minimize the computational overhead associated with exit decision-making, the attention mechanism operates within the latent domain as depicted in ~\cref{fig:arla_arch}. Formally, for each interval $[E_j, E_{j+1}]$, the accelerated residuals are projected into Query ($Q^s_{E_j}, \ldots, Q^s_{E_{j+1}}$), Key ($K^s_{E_j}, \ldots, K^s_{E_{j+1}}$), and Value ($V^s_{E_j}, \ldots, V^s_{E_{j+1}}$) vectors, with latent dimension $d^s$ for $Q^s$, $K^s$, and $V^s$ being significantly smaller than hidden dimension of $p$.\footnote{We use $d^s = 64$ for experiments described in ~\cref{sec:experiments}.} Notably, when the router is allowed to make exit decisions at gate $E_j$ based on residual change dynamics, we observe that the attention is not confined to the residual state at $E_j$ but is distributed across residual states from $E_{j-1}$ to $E_j$, %as illustrated in Figure~\ref{fig:vertical_latent_attention_dynamics}. 
This broader focus on residual dynamics significantly reduces decision ambiguity in early exits, as demonstrated in Figure~\ref{fig:roc_arla}, which contrasts routers based on the last hidden state, and the proposed ARLA router.

%show R -> S transformation. 
%show parameter and flop overhead as compared to adapter on last hidden state.

% \begin{figure}[ht]
%     \centering
%     \includegraphics[width=0.5\textwidth,height=5cm]{sections/figures/roc_arla.pdf}
%     \caption{ROC curves of early exit decision strategies: confidence-based methods (CALM/LITE), routers based on the accelerated hidden state, and latent attention routers.}
%     \label{fig:decision_making_comparison}
% \end{figure}

% \begin{figure}[ht]
%     \centering
%     \includegraphics[width=0.5\textwidth,height=5cm]{vertical_latent_attention.png}
%     \caption{Vertical latent attention mechanism for optimizing early exit decisions by considering residuals from gate \(M\) through \(M-1\).}
%     \label{fig:vertical_latent_attention}
% \end{figure}

\begin{figure}[ht]
    \centering
    \begin{subfigure}{0.52\textwidth}
        \centering
        \includegraphics[width=\textwidth, height = 4cm]{sections/figures/arla_arch.pdf}
        \caption{Accelerated Residual Latent Attention (ARLA): Accelerated residuals between early exit gates are projected into latent domain and attention over residual states within the interval is computed to capture residual dynamics and exit decision is made based on residual saturation.}
        \label{fig:arla_arch}
    \end{subfigure}%
    \hfill
    \begin{subfigure}{0.45\textwidth}
        \centering
        \includegraphics[width=\textwidth, height = 4.5cm]{sections/figures/vla_roc.pdf}
        \caption{ROC classification curves of early exit decision strategies using a linear router used on last residual state ~\cite{schuster2022confident, varshney-etal-2024-investigating, chen2023eellm}  and using ARLA approach that considers residual dynamics. }
        \label{fig:roc_arla}
    \end{subfigure}
    \caption{Effectiveness of ARLA in capturing residual dynamics for early exiting decisions.}


\end{figure}



% \begin{figure}[ht]
%     \centering
%     \includegraphics[width=1\textwidth,height=5cm]{sections/figures/arla.pdf}
%     \caption{fig that plots 32 rows 2 cols heatmap showing attention at each gate}
%     \label{fig:vertical_latent_attention_dynamics}
% \end{figure}

\subsubsection{Self Speculative Decoding} \label{method_self_speculative_decoding}

An alternative means to exploit the early alignment properties of our approach is through the use of accelerated residual states for speculative token sampling to accelerate autoregressive decoding. Speculative decoding aims to speed up memory-bound transformer inference by employing a lightweight draft model to predict candidate tokens, while verifying speculated tokens in parallel and advancing token generation by more than one token per full model invocation \cite{leviathan2023fast, chen2023accelerating, xia2023speculative, miao2023specinfer}. Despite its effectiveness in accelerating large language models (LLMs), speculative decoding introduces substantial complexity in both deployment and training. A separate draft model must be specifically trained and aligned with the target model for each application, which increases the training load and operational complexity ~\cite{chen2023accelerating}. Additionally, this approach is resource-inefficient, as it requires both the draft and target models to be simultaneously maintained in memory during inference \cite{leviathan2023fast, chen2023accelerating}. 

One strategy to address this inefficiency is to leverage the initial layers of the target model itself to generate speculative candidates, as depicted in ~\cite{Tang2024}. While this method reduces the autoregressive overhead associated with speculation, it suffers from suboptimal acceptance rates. This occurs because the linear transformation employed for translating hidden states from layer $k$ to the final layer $N$ is typically a poor approximation, as discussed in ~\cref{sec:motivation} and ~\cref{method_early_exiting}. Our approach resolves this limitation by utilizing accelerated residuals, which demonstrate higher fidelity to their slower counterparts. By utilizing accelerated residuals operating at a rate of $N/k$, where $k$ denotes the number of layers used for candidate speculation, we are able to efficiently generate speculative tokens for decoding.\footnote{We typically set $k = 4$ to balance the trade-off between autoregressive drafting overhead and acceptance rate, as discussed in~\cref{sec:experiments}.}
 This technique not only obviates the need for multiple models during inference but also improves the overall efficiency and effectiveness of speculative decoding.

\begin{figure}
    \centering    \includegraphics[width=1\linewidth]{sections/figures/m2r2_aot_loading.pdf}
    \caption{Ahead-of-Time Expert Loading: M2R2 accelerated residual stream predicts experts required for future layers, reducing reliance on on-demand lazy loading. Speculative pre-loading is efficiently overlapped with computation of multi-head attention (MHA) and MLP transformations. Only incorrectly speculated experts are loaded lazily, resulting in faster inference steps and improved computational efficiency. Here, H indicates LBM Host while D indicates HBM Device.}
    \label{fig:moe_expert_aot_loading}
\end{figure}


\subsubsection{Ahead of Time Expert Loading:} \label{method_aot_expert_loading}

Recent advancements in sparse Mixture-of-Experts (MoE) architectures ~\cite{shazeer2017outrageously, fedus2022switch, artetxe2019massively, lepikhin2020gshard, zoph2022designing} have introduced a paradigm shift in token generation by dynamically activating only a subset of experts per input, achieving superior efficiency in comparison to dense models, particularly under memory-bound constraints of autoregressive decoding \cite{fedus2022switch, zoph2022designing}. This sparse activation approach enables MoE-based language models to generate tokens more swiftly, leveraging the efficiency of selective expert usage and avoiding the overhead of full dense layer invocation. In dense transformer models, pre-loading layers is a common strategy to enhance throughput, as computations of current layer can be overlapped with pre-loading of next layer parameters ~\cite{narayanan2021efficient, shoeybi2020megatron}. However, MoE models face a unique challenge: expert selection occurs dynamically based on previous layer’s output, making it infeasible to preload next layer’s experts in parallel. This limitation results in inherent latency, as expert loading becomes a sequential, on-demand process ~\cite{lepikhin2020gshard, fedus2022switch}.

To address this inefficiency, our method introduces a mechanism with \textit{accelerated residuals}, which not only captures key characteristics of base slower residual states but also exhibit high cosine similarity with their final counterparts (as illustrated in \cref{fig:m2r2_residual_sim}). By employing accelerated residual streams, we can effectively predict the necessary experts for future layers well in advance of their actual invocation. Specifically, using a $2\times$ accelerated residual, the experts needed for layers $2i+2$ and $2i+3$ can be identified while still computing in layer $i$, thus overcoming the bottleneck of sequential, on-demand expert selection and mitigating latency in the decoding pipeline, as shown in \cref{fig:moe_expert_aot_loading}. Note that, we use fixed set of accelerator adapters for transforming accelerated residuals (as discussed in ~\cref{m2r2_method}) while slow residual is transformed via expert routing mechanism. 

Furthermore, our approach integrates a Least Recently Used (LRU) caching strategy, which enhances memory efficiency by replacing the least recently used experts with speculated experts that are anticipated to be needed in upcoming layers. This hybrid approach of preemptive expert loading with LRU caching yields substantial improvements over traditional on-demand loading or standalone caching strategies. By minimizing cache misses and efficiently managing memory, this approach addresses both compute and memory bottlenecks, leading to faster, more resource-efficient token generation in MoE architectures. A comprehensive evaluation of this strategy, in relation to state-of-the-art methods, is provided in \cref{experiments_aot}, and the compute and memory traces on an A100 GPU are detailed in \cref{fig:moe_aot_cuda_trace}.



% Recent advancements in sparse Mixture-of-Experts (MoE) architectures have introduced the concept of utilizing distinct computational paths for different tokens \cite{shazeer2017outrageously}. This approach, wherein only a subset of experts are activated per input, enables MoE-based language models to generate tokens more swiftly compared to their dense counterparts due to memory-bound nature of auto-regressive decoding. In dense models, pre-loading layers in advance is a common strategy to enhance computational efficiency. However, this technique is not applicable to MoE models, where expert selection occurs dynamically based on the outputs of previous layers, preventing parallel pre-fetching of experts.

% Our proposed method addresses this inefficiency. Accelerated residuals, which are highly similar to their slower counterparts (see \cref{fig:similarity}), can reliably predict the necessary experts ahead of time. For instance, by utilizing $2X$ accelerated residual stream, we can predict the experts needed for the layer $2i+1$ and $2i+3$ while carrying out computation in layer $i$. This enables us to commence expert loading significantly earlier, as illustrated in \cref{expert_loading}, effectively mitigating the delays observed with the naive on-demand expert loading. Additionally, our method benefits from incorporating a Least Recently Used (LRU) strategy, where speculated experts replace those that are least recently utilized, resulting in improved performance compared to using either strategy alone. For a comprehensive evaluation, refer to \cref{moe_trace}, which provides a CUDA compute and memory trace of our approach executed on <>.



% A naive solution involves using the residual state of the previous layer along with the gating function of the next layer to predict which experts need to be loaded, and initiating the expert loading process in parallel with the attention computation of the next layer. Yet, as shown in \cref{fig:MOE_attn_vs_loading_time}, the attention computation for medium to long contexts is considerably faster than the expert loading time, making this approach inefficient.




\subsection{Training} \label{method_training}
% This approach is feasible due to the absence of gradient conflicts, as discussed in \cref{sec:grad_conflict}.

To accelerate residual streams, we employ parallel accelerator adapters as described in \cref{m2r2_method}.  For the early exiting use-case outlined in \cref{method_early_exiting}, we define the training objective for these adapters using the following loss function, which combines cross-entropy loss at each exit $E_j$ with distillation loss at each layer $i$. Loss weights coefficients $\alpha_0$ and $\alpha_1$ are employed to balance contribution of corresponding losses.

\begin{align} \label{eq:mr_loss}
L_{\text{m2r2}} = \underbrace{-\alpha_0 \sum_{j=1}^{J} \sum_{t=1}^{T} \log p_{\theta} \left( \hat{y}_t^{E_j} \mid y_{<t}, x \right)}_{\text{cross-entropy loss}} 
+ \underbrace{\alpha_1\sum_{i=1}^{E_{J-1}} \sum_{t=1}^{T} \| \mathbf{p}_{t}^{i} - \mathbf{h}_{t}^{((i - E_{j(i)}) \cdot R_i) + E_{j(i)})} \|^2}_{\text{distillation loss}}.
\end{align}

where $\hat{y}_t^{E_j}$ denotes the predictions from the accelerated residual stream at layer $E_j$ and time step $t$, $y_t$ represents the corresponding ground truth tokens, and $x$ indicates previous context tokens. The distillation loss at each layer $i$ is computed by comparing accelerated residuals at layer $i$ with slow residuals at layer $(i - E_{j(i)}) \cdot R_i + E_{j(i)}$, where $R_i$ denotes the rate of accelerated residuals at layer $i$ while $E_{j(i)}$ represents the most recent gate layer index such that $E_{j(i)} <= i$. \( J \) represents the total number of early exit gates, N denotes number of hidden layers and $E_j$ denotes layer index corresponding to gate index $j$ and \( T \) denotes the sequence length. 

In dynamic compute settings, after training of accelerator adapters, we optimize the query, key, and value parameters governing the ARLA routers (see ~\cref{method_arla}) across all exits in parallel on binary cross entropy loss between predicted decision and ground truth exiting decision. The ground truth labels for the router are determined based on whether the application of the final logit head on $\hat{y}_t^{E_j}$ yields the correct next-token prediction. 


% The objective for this optimization is defined by the following loss function:


%TODO are equations required ? 
% \begin{equation} \label{eq:arla_loss_combined}\small
%     L_{\text{arla}} = -\frac{1}{N} \sum_{t=1}^{T} \left( \sum_{j=1}^{E_n} \left[ O_t^{E_j} \log(\hat{O}_t^{E_j}) + (1 - O_t^{E_j}) \log(1 - \hat{O}_t^{E_j}) \right] \right), \quad \text{where} \quad 
%     O_t^{E_j} = \begin{cases} 
%     1, & \text{if } L(\hat{y}_t^{E_j}) = y_t^{E_j} \\
%     0, & \text{otherwise}
%     \end{cases}
% \end{equation}

% where $\hat{O}_t^{E_j}$ represents the binary predicted logits produced by the vertical latent attention router, as described in \cref{sec:arla}, at gate $E_j$ and time step $t$, and $O_t^{E_j}$ denotes the corresponding ground truth labels. The ground truth labels for the router are determined based on whether the application of the logit head on $\hat{y}_t^{E_j}$ yields the correct next-token prediction. The parameters controlling vertical latent attention are trained concurrently to ensure consistency and efficient use of computational resources.

For self-speculative decoding, as described in \cref{method_self_speculative_decoding}, the training objective remains the same as \cref{eq:mr_loss}, but with the number of intervals set to $J = 1$ and the rate of residual transformation set to $R_n = N/k$, where the first $k$ layers generate speculative candidate tokens. In the context of Ahead-of-Time Expert Loading for Mixture-of-Experts (MoE) models (see \cref{method_aot_expert_loading}), setting the rate of residual transformation to $R_n = 2$ typically offers a good trade-off between the accuracy of expert speculation and AoT pre-loading of experts. 

% Thus, we set $J = 1$ and $E_1 = 16$.


~\subsection{FLOPs Optimization} \label{sec:flops_optimization}

Naively implemented, M2R2 incurs higher FLOP overhead compared to traditional speculative decoding and early exiting approaches such as ~\cite{medusa, schuster2022confident, Tang2024}. However, modern accelerators demonstrate compute bandwidth that exceeds memory access bandwidth by an order of magnitude or more~\cite{databricksLLMInference2023, jouppi2021ten}, meaning increased FLOPs do not necessarily translate to increased decoding latency. Nevertheless, to ensure fair comparison and efficiency in compute bound scenarios, we introduce targeted optimizations.

~\textbf{Attention FLOPs Optimization} For medium-to-long context lengths, attention computation dominates FLOPs in the self-attention layer, surpassing the contribution from MLP layers. Specifically, matrix multiplications involving queries, cached keys, and cached values scale with $l_{kv} * l_{q}$ where $l_{kv}$ denotes previous context length and $l_q$ denotes current query length. Since M2R2 pairs accelerated residuals with slow residuals, a naive implementation results in twice the FLOPs consumption compared to a standard attention layer. To address this, we limit the attention of accelerated residual stream to selectively attend to the top-k most relevant tokens, identified by the slow residual stream based on top attention coefficients\footnote{We set to k = 64 and attend to top 64 tokens as identified by the slow residual stream.}. This is possible since slow and accelerated residual streams are processed in same forward pass and accelerated streams have access to attention coefficients of slow stream. Note that, the faster residual stream still retains the flexibility to assign distinct attention coefficients to these tokens. Furthermore, we design the faster residual stream to employ only 8 attention heads, compared to the 32 heads used in the slow residual stream of the Phi-3 model, reducing query, key, value, and output projection FLOPs by a factor of 1/4. ~\cref{fig:m2r2_num_heads_ablation} indicates effect of using a slicker stream on alignment. As depicted, using $\hat{n}_h = 8$ offers a good trade-off between alignment and FLOPs overhead. 

~\textbf{MLP FLOPs Optimization} The accelerator adapters operating on the accelerated residual stream are intentionally designed with lower rank than their counterparts in the base model. This reduces FLOP overhead by a factor proportional to $hiddenSize / rank$. Additionally, since the faster residual stream uses only 8 attention heads (compared to 32 in the slow residual stream of Phi-3), the subsequent MLP layers process a smaller set of activations, further reducing FLOPs by another factor of 1/4.

These optimizations significantly reduce the FLOP overhead per speculative draft generation, as illustrated in ~\cref{fig:flops_optmization}. Notably, while traditional early-exiting speculative approaches such as DEED require propagating the full slow residual state through the initial layers, incurring substantial computational costs, M2R2 achieves efficient token generation via slimmer, low-rank faster residual streams. In contrast, Medusa introduces considerable FLOP overhead due to per-head computations scaling with $d^2+dv$\footnote{Here $d$ denotes hidden state dimension while $v$ denotes vocab size.}, whereas M2R2 employs low-rank layers for both MLP and language modeling heads, maintaining computational efficiency. All experiments involving the M2R2 approach, as detailed in ~\cref{sec:experiments}, are conducted using these FLOPs optimizations.









% \[
% O_t^{E_j} = 
% \begin{cases} 
% 1, & \text{if } L(\hat{y}_t^{E_j}) = y_t^{E_j} \\
% 0, & \text{otherwise}
% \end{cases}
% \]




%add distillation
% We train accelerator adapters described in \cref{m2r2_method} to accelerate residual streams on next token prediction all in parallel since there are no gradient conflict issues as described in \cref{sec:grad_conflict}.

% \begin{align} \label{eq:mr_loss}
% L_{mr} =  & -\sum_{j = 1}^{E_n} (\sum_{t=1}^{T}\log p_{\theta} (\hat{y}_t^{E_j} | \hat{y}_{<t}, x)) \nonumber
% \end{align}

% where $\hat{y_t^{E_j}}$ denotes predicted logits obtained from accelerated residual stream at gate $E_j$ and time-step $t$ while $y_t^{E_j}$ denotes corresponding truth tokens. 

% Upon training of adapters responsible for accelerating residual streams, we train query, key, value parameters responsible for vertical latent attention of all gates in parallel as

% \begin{equation} \label{eq:arla_loss}
%     L_{arla} = -\frac{1}{N} (\sum_{t=1}^{T}(1\sum_{j=1}^{E_n} \left[ O_t^{E_j} \log(\hat{O}_t^{E_j}) + (1 - o_t^{E_j}) \log(1 - \hat{o_t}_{E_j}) \right]))
% \end{equation}

% where $\hat{O_t^{E_j}}$ denotes binary predicted logits obtained from vertical latent attention router described in \cref{sec:arla} at gate $E_j$ and timestep $t$ while $O_t^{E_j}$ denotes corresponding truth label. Truth labels for router are obtained by computing whether logit head application on $\hat{y}_t^j$ results in true next token prediction. Formally speaking, 

% $O_t^{E_j} = 1 if L(\hat{y_t^{E_j}}) == y_t^{E_j} , 0 otherwise$. 

% Parameters responsible for vertical latent attention are also trained in parallel as well. 

%todo: training slow and fast residuals together and distillation can be two training mdoes. 
%Distillation can be an ablation. 




% Although transformer decoding is memory bound on most mainstream accelerators, there could be scenarios where flop savings are crucial. For instance, on on-device settings power consumption is directly correlated with flops per decoding step and reducing flops does help with overall energy consumption. Vanilla early exiting methods help with flop reduction but suffer from mismatch between training and inference due to early exited tokens. If token at decoding step $t$, $T_t$ exited at layer $E_i$, while token $T_{t+k}$ exits at layer $E_j$ such that $E_i < E_j$, hidden state $H_{t+k}l$ does not have corresponding hidden state $H_tl$ to attend to where $E_i < l <= E_j$. One solution that's often used in literature is to rely on last hidden state available, $H_t{E_j}$, however it tends to be sub-optimal and does affect generation quality \cite{ref}.  To alleviate this mismatch while reducing flops, we train router such that attention mask between token $T_{t+k}$ and token $T_{<t+k}$ is given by: 

% \begin{equation}
%     a_{T_{{t+k}{T_{<t+k}}} = 1 if  E_{T_{<t+k}} >= E{T_{t+k}}
%     else 0
% \end{equation}

% This attention mask enables router to account for exited tokens and get trained accordingly. Since attention mechanism during decoding remains exactly same as that during training, impact on generation quality tends to be minimal as noted in \cref{fig:gen_auality_with_and_without_recompute_attention_show_flops}.  Although MoD does not suffer from training and inference mismatch, we observe that it suffers from discountinuity between pre-training and super-vised fine-tuning resulting in sub-optimal perplexity. On the other hand, our method doesn't not require pre-training , doesn't suffer from discountinuity, and achieves much better perplexity in super-vised fine-tuning and instruction tuning setups as shown in \cref{fig:Mod_vs_m2r2_loss_curves}.






% Our techniques are directly applicable in such scenarios.    




%expert loading with cuda streams in experiments
\section*{Conclusion}
This paper aims to enhance our understanding of the computational complexity of computing various Shapley value variants. We found that for various ML models --- including decision trees, regression tree ensembles, weighted automata, and linear regression --- both local and global interventional and baseline SHAP can be computed in polynomial time under HMM modeled distributions. This extends popular algorithms, such as TreeSHAP, beyond their empirical distributional scope. We also establish strict complexity gaps between the various SHAP variants (baseline, interventional, and conditional) and prove the intractability of computing SHAP for tree ensembles and neural networks in simplified scenarios. Overall, we present SHAP as a versatile framework whose complexity depends on four key factors: \begin{inparaenum}[(i)] \item model type, \item SHAP variant, \item distribution modeling approach, \item and local vs. global explanations\end{inparaenum}. We believe this perspective provides deeper insight into the computational complexity of SHAP, paving the way for future work.




%We believe that our framework provides a more intricate understanding of SHAP computation complexity across different models, distributions, and variants, paving the way for further research.

Our work opens promising directions for future research. First, expanding our computational analysis to other SHAP-related metrics, such as asymmetric SHAP~\citep{frye20} and SAGE~\citep{covert2020understanding}, would be valuable. Additionally, we aim to explore more expressive distribution classes and relaxed assumptions beyond those in Section \ref{sec:tractable} while maintaining tractable SHAP computation. Finally, when exact computation is intractable (Section \ref{sec:intractable}), investigating the approximability of SHAP metrics through approximation and parameterized complexity theory~\citep{downey2012parameterized} is an important direction.

%Our work opens several promising avenues for future research on the computational properties of explainable AI methods, with a particular focus on SHAP. First, it would be interesting to broaden the computational analysis conducted in this work to include other popular SHAP-related metrics in the literature, such as asymmetric SHAP \cite{frye20} and SAGE \cite{covert2020understanding}. Also, in the future, we aim to explore more expressive distribution classes and relaxed distributional assumptions—extending beyond those examined in Section \ref{sec:tractable} —that still yield tractable SHAP computation. Finally, when exact computation proves intractable (Section \ref{sec:intractable}), it is worthwhile to theoretically investigate the question of the approximability of computing the SHAP metrics across various configurations, through the lens of approximation and parametrized complexity theory \cite{arora2009computational}.

%This paper aims to deepen our understanding of the computational complexity involved in obtaining different Shapley value variants. We found that for a variety of ML models, including decision trees, tree ensembles for regression, weighted automata, and linear regression models — computing both local and global interventional and baseline SHAP can be done in polynomial time when distributions are modeled by HMMs. This extends the distributional scope of popular algorithms like TreeSHAP, which is limited to empirical distributions. Additionally, we demonstrate a strict complexity gap between SHAP variants, showing that interventional and baseline SHAP can be strictly easier to compute than conditional SHAP. Despite these positive results, we uncovered intractability for various SHAP variants in neural networks and tree ensembles. Finally, we provided generalized complexity relations across SHAP variants. We believe that our framework offers a deeper understanding of the complexity involved in computing SHAP across various variants, models, distributions, as well as in both local and global computations, laying the groundwork for future research.

\clearpage

\section*{Impact Statement}

This paper proposes a new framework on regulating advanced AI agents, who can pose existential risks.  The paper is thus intended to mitigate the negative impacts and implications that AI technologies might have.  However, the paper also discusses how those advanced AI agents might be developed.  While all the discussion is based on existing technologies that are already known in public, the unified treatment in this paper might motivate malicious readers to develop and deploy such advanced AI agents without sufficient considerations of their safety.

\bibliography{principal-agent}
\bibliographystyle{icml2025}

\newpage
\appendix
\section{Similarity between AI agents}
\label{sec:method:agent}

It would be better if we could say that an AI agent is acceptable when it is sufficiently close to an AI agent that we empirically know is strongly acceptable, regardless of how those agents are developed (i.e., independent of their configurations including the amount of computation for training).  A question is how to evaluate the similarity between two AI agents.  If the two AI agents consist of identical base FMs and identical planning modules, their difference may be simply characterized by the amount of computation for inference.  However, AI agents can consist of different base FMs or different planning modules.
%(e.g., the number of search steps performed in MCTS).

A possible approach is to measure the similarity between two AI agents, $G$ and $G'$, based on their output.  For example, using a distance $d$ between probability distributions, we may for example define the distance between $G$ and $G'$ with $\sup_x d(g(G,x),g(G',x))$, where $g(G,x)$ is the distribution of the output (e.g., document) of $G$ when it perceives $x$ (e.g., prompt).  This is similar to the motivation of reinforcement learning from human feedback \cite{ouyang2022training,bai2022training} and direct policy optimization \cite{rafailov2023direct}, where the regularization with Kullback–Leibler divergence is used to mitigate catastrophic forgetting or alignment tax \cite{kotha2024understanding,ouyang2022training,bai2022training}, which refer to the phenomena that fine-tuned models lose the skills that pre-trained models had.

Likewise, we may mitigate catastrophic forgetting of the strong acceptability of an AI agent $G'$ and retain the acceptability in a new AI agent $G$ by ensuring $\sup_x d(g(G,x),g(G',x))\le \varepsilon$.  A difficulty is that it is unclear how to evaluate the supremum, since there can be infinitely many possible perceptions $x$.  Alternatively, one may consider $\E[d(g(G,X),g(G',X))]$, where $\E$ is the expectation with respect to some distribution of the random perception $X$.  However, such a guarantee based on expectation may be unsuitable for addressing safety concerns related to existential risks, which involve events with extremely low probabilities and extremely high impacts.

\section{Expanding the strongly acceptable set}
\label{sec:method:expand}

The set of strongly acceptable solutions may be expanded gradually.  For example, we may choose an acceptable solution $z\in\bar\calZ_0$ and keep using $z$ for a certain period of time.  If it turns out that $z$ can be considered strongly acceptable based on its behavior during that period, we may expand the set of strongly acceptable solutions by adding $z$ into $\calZ_0$.  This process of expanding the acceptable set $\calZ_0$ could also be performed jointly as a community.  

This expansion of acceptable set is similar in spirit to safe exploration in RL \cite{garcia2015comprehensive}.  Here, the safety set is gradually expand, starting from a seed set, for example based on the assumption of Lipshitz continuity and Gaussian process \cite{sui2015safe}.  In control theory, safety is often guaranteed with barrier certificates often based on some prior knowledge about the environment \cite{ames2019control,luo2021learning,bansal2017hamilton}.  Such ideas have also been exploited in safe exploration in RL with generative modeling \cite{wang2023enforcing}.

%barrier certificate or function in control theory: unsafe states are specified by users; given dynamics, can guarantee that it never goes to unsafe states; 
%prior knowledge of the environment dynamics \cite{ames2019control}
%use confidence interval of the dynamics \cite{luo2021learning}.
%Hamilton-Jacobi reachability analysis \cite{bansal2017hamilton} in control theory. 
%- set of states that can lead to an unsafe state
%neural network based controller

\subsection{Rationale for similarity-based measures}
\label{sec:method:rationale}

Here, we provide some rationale on such similarity-based measures.  Let a solution $z$ denote a configuration, an AI agent, or an action-sequence; we discuss the acceptability of $z$.

Suppose that there exists an unknown function $g_0$ such that a solution $z$ is acceptable iff $g_0(z)\le 0$.  We cannot make strong assumptions about $g_0$, since we know little about $g_0$. Since we cannot deal with $g_0$ without any assumptions, let us make a minimal assumption that the solution space $\calZ$ is equipped with a metric $d$, and that $g_0$ is 1-Lipschitz.\footnote{This does not lose generality, since $L$-Lipschitz functions under a metric $d$ can be made 1-Lipshitz by redefining $d$.}  Let $\calG$ be a class of 1-Lipschitz functions.  

We say that a solution $z_0$ is strongly acceptable if all of its $\varepsilon$-neighbors are acceptable.  Let 
$\calZ_0
\subseteq
\left\{
    z\in\calZ
    \mid 
    g(z_0)\le - \varepsilon
\right\}$
be the set of known strongly acceptable solutions.  Then we know that
%\begin{align}
    $\bar\calZ_0
    \coloneqq \left\{
    z \in\calZ
    \mid
    \min_{z_0\in \calZ_0} d(z,z_0) \le \varepsilon
    \right\}$
%\end{align}
is a set of acceptable solutions.  One would typically choose the solution $z$ that maximizes an objective function under the constraint of $z\in\bar\calZ_0$.  In this way, the selected solution is guaranteed to be acceptable under the assumptions made.
\section{Unobservable Markov decision processes}

There has been a limited amount of work on unobservable Markov decision processes (UMDPs)\footnote{UMDPs have also been studied under the name of Markov decision processes (MDPs) with no observations, non-observable MDPs (NOMDPs), and no observation MDPs (NOMDPs).}.  An UMDP is a special case of a partially observable Markov decision process (POMDP) in that the agent always makes a null observation in the UMDP.  A standard approach to finding the optimal policy for a POMDP is to recursively compute its value as a function of the belief state, which is updated on the basis of the Bayes rule.  This is substantially simplified when there are no observations.  In this section, we provide a comprehensive survey of the prior work on UMDPs.

\citet{madani1999computability,madani2003undecidability} establish the undecidability of some decision problems associated with POMDPs over infinite horizons by establishing undecidability for the special case of UMDPs.  Such undecidability for UMDPs can be established by reducing an UMDP to a probabilistic finite-state automaton.  The undecidability also holds for a restricted class of UMDPs \cite{balle2017bisimulation,balle2022bisimulation}. While approximate decision problems are still undecidable for general UMDPs over infinite horizons, \citet{chatterjee2024ergodic} study a special case of UMDPs whose approximate decision problems are decidable.

\citet{burago1996complexity} prove that computing the optimal policy for a POMDP over a finite horizon is NP-hard but showing that it is NP-hard for an UMDP.  \citet{wu2020optimal} study special cases of UMDPs over finite horizons whose optimal policies can be computed in polynomial time.

UMDPs have also been used as approximations of POMDPs \cite{hauskrecht2000value,brechtel2015dynamic,lauri2016sequential} or simply discussed as a special case of POMDPs \cite{valkanova2009algorithms}.  For a given POMDP, the corresponding UMDP can given a lower bound on the value function, while the corresponding (fully observable) MDP can give an upper bound on the value function.  This relation between UMDP and POMDP can be exploited to efficiently find approximately optimal policies for POMDPs.  Notice that an UMDP can allow more efficient optimization than the corresponding POMDP, since the UMDP does not need to deal with observations.  For exemple, \citet{king2018robust} studies an MCTS method for UMDPs.

UMDPs have also been studied as a simple special case of POMDPs to study the effectiveness of planning methods in belief states to study the relative performance of different planning methods for POMDPs \cite{littlefield2020efficient,littlefield2018importance,kimmel2019belief}.  UMDPs have also been simply discussed as a special case of POMDPs \cite{boutilier1999decision,csaji2008adaptive,verma2005graphical}.

UMDP also appears in the study of planning for multiple distributed agents to optimize a single objective under partial observability.  Specifically, planning for a decentralized POMDP (DecPOMDP) \cite{oliehoek2016concise} can be reduced to planning for a (centralized) UMDP \cite{oliehoek2014decpomdps,roijers2016multi,roijers2020multi}, where the state in the UMDP is the pair of the state and the history of observations in the DecPOMDP, and the action in the UMDP is the decision rule that maps the history of observations into the actions in the POMDP.  In DecPOMDPs, the belief (distribution) over the pair of the state and the history of observations is the sufficient statistic, and planning can be performed in the space of the belief states.

\citet{evendar2007value} consider an UMDP in the context of studying what values observations can provide in a POMDP.  They introduce a parameter that ranges from 0 to 1.  When the parameter is 0, the observation provides no information about the state (hence, the POMDP reduces to an UMDP); when the parameter is 1, the observation provides full information about the state (hence, the POMDP reduces to an MDP).  The prior work also extends UMDPs to allow often costly actions that enable partial or full observations of the state \cite{fox2007reinforcement,kamar2009modeling,kamar2010reasoning,wang2025ocmdpob}.

% NOMDP is mentioned as a model used in classical method https://publications.polymtl.ca/5550/1/2020_Yves_Alain_Mbeutcha.pdf


\end{document}
