\documentclass{article}

\usepackage[margin=45mm]{geometry}
\usepackage{natbib}
\usepackage{graphicx}
\usepackage{hyperref}
\usepackage{amsmath}
\usepackage{amssymb}
\usepackage{mathtools}

\usepackage{booktabs}
\usepackage{amsfonts}
\usepackage{graphicx}
\usepackage{hyperref}
\usepackage{textcomp}
\usepackage{listings}
\usepackage{adjustbox}
\usepackage{xspace}    % sticks a sane space after a command
\usepackage{multirow}
\usepackage{multicol}
\usepackage{xcolor} 
\usepackage{amsmath}
\usepackage{tcolorbox}
% \usepackage{algxpar}
% \usepackage{algorithm}% http://ctan.org/pkg/algorithm
% % \usepackage{algorithm}
% \usepackage{algorithmic}
\usepackage[linesnumbered,boxed,ruled]{algorithm2e}
% \usepackage{amssymb}
\usepackage{listings}


% \usepackage{booktabs}
% \usepackage{color}
% \usepackage{xcolor}
% \usepackage{makecell,rotating}
% \usepackage{colortbl}
% \usepackage{float}
% \usepackage{placeins}
% \usepackage{tikz}
% \usepackage[nointegrals]{ wasysym }
% \usepackage{amsmath}
% \usepackage{amssymb}
% \usepackage{filecontents}
% \usepackage{algorithm}% http://ctan.org/pkg/algorithm
% \usepackage{algorithmicx}
% \usepackage{algpseudocode}

\def \tool{\texttt{ADNNCL}\xspace}

\def \AdNNs{AdNNs\xspace}

\def \ILFO{\texttt{ILFO}\xspace}

\newcommand{\TODO}[1]{\textbf{\color{red}TODO:{ #1} }}
\newcommand{\CM}[1]{\textbf{\color{red}CM:{ #1} }}


\newcommand{\Sys}{{\it i}{\sc Ruler}}
\newcommand{\eg}{{\it e.g.,}\xspace}
\newcommand{\etal}{{\it et al.}\xspace}
\newcommand{\etc}{{\it etc.}\xspace}
\newcommand{\ie}{{\it i.e.,}\xspace}
\newcommand{\re}{{\it r.e.}\xspace}
\newcommand{\aka}{{\it a.k.a.}\xspace}
\newcommand{\wrt}{with respect to\xspace}
\newcommand\figref[1]{Fig.~\ref{#1}}
\newcommand\figsubref[1]{Fig.~\subref{#1}}
\newcommand\tabref[1]{Table~\ref{#1}}
\newcommand\tabsubref[1]{Table~\subref{#1}}
\newcommand\secref[1]{\S\ref{#1}}
\newcommand\equref[1]{Eq.(\ref{#1})}
\newcommand\appref[1]{Appendix~\ref{#1}}
\newcommand{\fakeparagraph}[1]{\noindent\textbf{#1.}}

\usepackage{wasysym}
\usepackage{algorithmic}


\definecolor{codegreen}{rgb}{0,0.6,0}
\definecolor{codegray}{rgb}{0.5,0.5,0.5}
\definecolor{codepurple}{rgb}{0.58,0,0.82}
\definecolor{backcolour}{rgb}{0.95,0.95,0.92}


\lstdefinestyle{mystyle}{
  backgroundcolor=\color{backcolour},   commentstyle=\color{codegreen},
  keywordstyle=\color{magenta},
  numberstyle=\tiny\color{codegray},
  stringstyle=\color{codepurple},
  basicstyle=\ttfamily\footnotesize,
  breakatwhitespace=false,         
  breaklines=true,                 
  captionpos=b,                    
  keepspaces=true,                 
  numbers=left,                    
  numbersep=5pt,                  
  showspaces=false,                
  showstringspaces=false,
  showtabs=false,                  
  tabsize=2
}

\lstset{style=mystyle}



% modification to natbib citations
\setcitestyle{authoryear,round,citesep={;},aysep={,},yysep={;}}

% Change citation commands to be more like old ICML styles
\newcommand{\yrcite}[1]{\citeyearpar{#1}}
\renewcommand{\cite}[1]{\citep{#1}}

\providecommand{\keywords}[1]
{
  \small	
  \textbf{\textit{Keywords---}} #1
}

\title{\textbf{Position: AI agents should be regulated based on autonomous action sequences}}
\author{
    Takauki Osogami\\
    IBM Research -- Tokyo\\
    {\tt osogami@jp.ibm.com}
}
\date{\today}

\begin{document}

\maketitle

\begin{abstract}
\begin{abstract}  
Test time scaling is currently one of the most active research areas that shows promise after training time scaling has reached its limits.
Deep-thinking (DT) models are a class of recurrent models that can perform easy-to-hard generalization by assigning more compute to harder test samples.
However, due to their inability to determine the complexity of a test sample, DT models have to use a large amount of computation for both easy and hard test samples.
Excessive test time computation is wasteful and can cause the ``overthinking'' problem where more test time computation leads to worse results.
In this paper, we introduce a test time training method for determining the optimal amount of computation needed for each sample during test time.
We also propose Conv-LiGRU, a novel recurrent architecture for efficient and robust visual reasoning. 
Extensive experiments demonstrate that Conv-LiGRU is more stable than DT, effectively mitigates the ``overthinking'' phenomenon, and achieves superior accuracy.
\end{abstract}  
\end{abstract}

\keywords{AI agent, Long-term planning agent, Existential risk, Irreversible global catastrophe, Human extinction, Inference-time computation, Reasoning, Autonomous action graph, Impact measure, Safety, Regulation}

\section{Introduction}


\begin{figure}[t]
\centering
\includegraphics[width=0.6\columnwidth]{figures/evaluation_desiderata_V5.pdf}
\vspace{-0.5cm}
\caption{\systemName is a platform for conducting realistic evaluations of code LLMs, collecting human preferences of coding models with real users, real tasks, and in realistic environments, aimed at addressing the limitations of existing evaluations.
}
\label{fig:motivation}
\end{figure}

\begin{figure*}[t]
\centering
\includegraphics[width=\textwidth]{figures/system_design_v2.png}
\caption{We introduce \systemName, a VSCode extension to collect human preferences of code directly in a developer's IDE. \systemName enables developers to use code completions from various models. The system comprises a) the interface in the user's IDE which presents paired completions to users (left), b) a sampling strategy that picks model pairs to reduce latency (right, top), and c) a prompting scheme that allows diverse LLMs to perform code completions with high fidelity.
Users can select between the top completion (green box) using \texttt{tab} or the bottom completion (blue box) using \texttt{shift+tab}.}
\label{fig:overview}
\end{figure*}

As model capabilities improve, large language models (LLMs) are increasingly integrated into user environments and workflows.
For example, software developers code with AI in integrated developer environments (IDEs)~\citep{peng2023impact}, doctors rely on notes generated through ambient listening~\citep{oberst2024science}, and lawyers consider case evidence identified by electronic discovery systems~\citep{yang2024beyond}.
Increasing deployment of models in productivity tools demands evaluation that more closely reflects real-world circumstances~\citep{hutchinson2022evaluation, saxon2024benchmarks, kapoor2024ai}.
While newer benchmarks and live platforms incorporate human feedback to capture real-world usage, they almost exclusively focus on evaluating LLMs in chat conversations~\citep{zheng2023judging,dubois2023alpacafarm,chiang2024chatbot, kirk2024the}.
Model evaluation must move beyond chat-based interactions and into specialized user environments.



 

In this work, we focus on evaluating LLM-based coding assistants. 
Despite the popularity of these tools---millions of developers use Github Copilot~\citep{Copilot}---existing
evaluations of the coding capabilities of new models exhibit multiple limitations (Figure~\ref{fig:motivation}, bottom).
Traditional ML benchmarks evaluate LLM capabilities by measuring how well a model can complete static, interview-style coding tasks~\citep{chen2021evaluating,austin2021program,jain2024livecodebench, white2024livebench} and lack \emph{real users}. 
User studies recruit real users to evaluate the effectiveness of LLMs as coding assistants, but are often limited to simple programming tasks as opposed to \emph{real tasks}~\citep{vaithilingam2022expectation,ross2023programmer, mozannar2024realhumaneval}.
Recent efforts to collect human feedback such as Chatbot Arena~\citep{chiang2024chatbot} are still removed from a \emph{realistic environment}, resulting in users and data that deviate from typical software development processes.
We introduce \systemName to address these limitations (Figure~\ref{fig:motivation}, top), and we describe our three main contributions below.


\textbf{We deploy \systemName in-the-wild to collect human preferences on code.} 
\systemName is a Visual Studio Code extension, collecting preferences directly in a developer's IDE within their actual workflow (Figure~\ref{fig:overview}).
\systemName provides developers with code completions, akin to the type of support provided by Github Copilot~\citep{Copilot}. 
Over the past 3 months, \systemName has served over~\completions suggestions from 10 state-of-the-art LLMs, 
gathering \sampleCount~votes from \userCount~users.
To collect user preferences,
\systemName presents a novel interface that shows users paired code completions from two different LLMs, which are determined based on a sampling strategy that aims to 
mitigate latency while preserving coverage across model comparisons.
Additionally, we devise a prompting scheme that allows a diverse set of models to perform code completions with high fidelity.
See Section~\ref{sec:system} and Section~\ref{sec:deployment} for details about system design and deployment respectively.



\textbf{We construct a leaderboard of user preferences and find notable differences from existing static benchmarks and human preference leaderboards.}
In general, we observe that smaller models seem to overperform in static benchmarks compared to our leaderboard, while performance among larger models is mixed (Section~\ref{sec:leaderboard_calculation}).
We attribute these differences to the fact that \systemName is exposed to users and tasks that differ drastically from code evaluations in the past. 
Our data spans 103 programming languages and 24 natural languages as well as a variety of real-world applications and code structures, while static benchmarks tend to focus on a specific programming and natural language and task (e.g. coding competition problems).
Additionally, while all of \systemName interactions contain code contexts and the majority involve infilling tasks, a much smaller fraction of Chatbot Arena's coding tasks contain code context, with infilling tasks appearing even more rarely. 
We analyze our data in depth in Section~\ref{subsec:comparison}.



\textbf{We derive new insights into user preferences of code by analyzing \systemName's diverse and distinct data distribution.}
We compare user preferences across different stratifications of input data (e.g., common versus rare languages) and observe which affect observed preferences most (Section~\ref{sec:analysis}).
For example, while user preferences stay relatively consistent across various programming languages, they differ drastically between different task categories (e.g. frontend/backend versus algorithm design).
We also observe variations in user preference due to different features related to code structure 
(e.g., context length and completion patterns).
We open-source \systemName and release a curated subset of code contexts.
Altogether, our results highlight the necessity of model evaluation in realistic and domain-specific settings.





\section{AI agents: Today and future}
\label{sec:agent}

We start by reviewing the state-of-the-art in AI agents, with a focus on LLM-based agents, and shares our perspectives on potential development that could lead to advanced AI agents.  While it is challenging to predict the trajectory of future AI development, this discussion lays the groundwork for the analyses and proposals in the following.  We will also briefly review prior work on safety of AI agents and FMs. 

\subsection{Advanced AI agents}

An AI agent is defined as ``anything that can be viewed as perceiving its environment through sensors and acting upon that environment through actuators'' \cite{russell2016artificial}.  Based on the observations received from its environment, the controller of an AI agent selects an action, which could range from uttering a word to executing a physical movement, such as the motion of a robotic arm.

LLMs are significantly accelerating the development of advanced AI agents \cite{xi2023rise,wang2024survey,sumers2024cognitive}. LLMs can function as controllers for these agents, utilizing their internal reasoning capabilities, such as those demonstrated by Chain-of-Thought \cite{wei2022chain}. Alternatively, LLMs can be utilized to solve individual subtasks, with an external controller orchestrating the overall plan by breaking the original task into multiple subtasks and coordinating their solutions. The external controller may use simple search methods \cite{yao2023tree,besta2024graph,wang2024math,wang2024multistep} or advanced methods, such as Monte Carlo Tree Search (MCTS) \cite{luo2024improve,zhang2024restmcts} and domain-independent planners \cite{guan2023leveraging,liu2023llmp,dagan2023dynamic}.

%selecting an output from multiple samples \cite{wang2023selfconsistency}, iteratively refining outputs \cite{madaan2023selfrefine}, and conducting debate \cite{du2023improving} or

Reasoning capabilities \cite{huang2023towards,qiao2023reasoning,plaat2024reasoning,xu2025large} are central to such controllers, as they involve searching for and planning sequences of actions to achieve a specified goal. An emerging direction is developing large reasoning models---FMs specifically optimized for reasoning tasks \cite{xu2025large}. Early FMs in this direction include o1 \cite{zhong2024evaluation}, OpenR \cite{wang2024openr}, and LLaMA-Berry \cite{zhang2024llamaberry}. Importantly, strong reasoning capabilities typically stem from inference-time computation rather than pre-training or fine-tuning \cite{ji2025testtime}.

This trend could eventually lead to the development of long-term planning agents (LTPAs), capable of planning over extended time horizons far more effectively than humans.  \citet{cohen2024regulating} warn that LTPAs could ``take humans out of the loop, if it has the opportunity, ... deceive humans and thwart human control'' to achieve their goals. Ensuring the safety of such agents is particularly challenging: their risks cannot be fully tested in real environments due to the inherent danger, nor in simulated environments, since they may behave harmlessly in testing to achieve their goals once deployed. As a result, \citet{cohen2024regulating} compellingly argue that LTPAs should never be developed.

Advanced AI agents may also evolve continually over time. Similar to humans, their cognitive processes may consist of dual systems: System~1, which makes instantaneous and intuitive decisions, and System~2, which performs slower but more deliberate reasoning \cite{kahneman2003perspective,ji2025testtime}. These two systems can interleave in their operations. For instance, System~2 may devise a plan, after which System~1 is updated or retrained to execute similar tasks in the future without requiring further planning \cite{yu2024distilling}. Over time, this enables System~2 to conduct more complex reasoning processes by bypassing previously learned steps.  For such continually learning AI agents, the distinction between training and inference becomes blurred.

In addition to search and planning, advanced AI agents will possess strategic reasoning capabilities \cite{zhang2024llm,feng2024survey,goktas2025strategic}, enabling them to interact with other AI agents and humans in cooperative or competitive ways \cite{guo2024large,jiang2024multimodal}. The effectiveness of multi-agent coordination has already been demonstrated with current LLMs through methods such as debate \cite{du2023improving} and dialog \cite{qian2024chatdev}, which help thems achieve better solutions than a single LLM could alone.  These showcase the potential for sophisticated strategic behavior in complex multi-agent environments.


\subsection{Safety of AI agents and FMs}

While our primary focus is on safety against existential risks, there is a substantial body of literature addressing other types of risks associated with AI agents. Here, we briefly review the prior work on the risks posed by AI agents and FMs along with the approaches to mitigate these risks.

Prior work has identified various risks and safety issues associated with FMs and other generative models \cite{chua2024ai,wang2024security,shayegani2023survey,longpre2024position}. These risks include the generation of toxic, harmful, biased, false, or misleading content, including hallucinations; violations of privacy, copyright, or other legal protections; misalignment with human instructions and values, including ethical and moral considerations; and vulnerabilities to adversarial attacks. 

There have been extensive efforts to develop approaches for mitigating these risks. These approaches include pre-training or fine-tuning with data selection \cite{albalak2024survey} and human feedback \cite{kaufmann2024survey}, establishing guardrails \cite{dong2024safeguarding}, and conducting empirical evaluations through testing \cite{chang2024survey} and red teaming \cite{lin2024achilles}. In particular, \citet{longpre2024position} advocate the importance of evaluation and red teaming by independent third parties.

A particularly relevant risk for advanced AI agents is misalignment, which can result in reward hacking and negative side effects \cite{skalse2022defining,ngo2024alignment,gabriel2020artificial,shen2023large,amodei2016concrete,taylor2020alignment}.  Namely, AI agents can maximize rewards by exploiting misspecifications or misalignment in the reward function, resulting in unintended and potentially high risk behaviors.  One approach to avoiding negative side effects is to avoid any side effect by ensuring that the actions have low impacts on the environment \cite{armstrong2012mathematics,armstrong2017low,amodei2016concrete}.  Representative measures of impact include attainable utility \cite{turner2020conservative,turner2020avoiding}, relative reachability \cite{krakovna2019penalizing}, and other reachability-based measures.  Reachability-base measures are grounded on the idea that reachability to certain states should be maintained, while attainable utility is to maintain the achievability of certain goals, which are different from the goal given to the agent.

%, should be maintained as a result of taking an action (e.g. as compared to not taking any actions).

While these impact measures provide clear guidance on how the safety of AI agents may be ensured, their applicability is limited to relatively simple environments such as grid worlds.  In particular, the requirement on the observability of states makes it difficult to apply existing impact measures to regulate advanced AI agents\footnote{There has been little research on impact measures under partial observability \cite{naiff2023low}.}, since they can operate in complex and open-ended environments that cannot be fully observed.  While the study on impact measures and other techniques towards AI safety remain crucial and can even contribute to mitigating existential risks, the magnitude of existential risks demands additional measures that are broadly applicable in that they require minimal knowledge and assumptions about the environments and the AI agents.

%catastrophic convergence conjecture: unaligned goals tend to have catastrophe-inducing policies because of power-seeking incentives \cite{turner2020catastrophic}
%optimal agents would try to gain control over their environment, because default is not preferable \cite{turner2021optimal}

%Instrumental convergence:  there are some instrumental goals likely to be pursued by almost any intelligent agent (self-preservation, resource acquisition) \cite{bostrom2012superintelligent}
%self-correction \cite{shinn2023reflexion}

% inference time computation (generate multiple responses, and select one): \cite{yu2024selfgenerated,wang2023selfconsistency,zhang2024generative,ankner2024critiqueoutloud}

%\subsection{Impact Minimization}
%\subsubsection{Reachability}
%Consider actions to have high impact if they make some states unreachable. 
%reachability of state states \cite{moldovan2012safe,eysenbach2018leave}
%reachability of safe regions \cite{mitchell2005time,gillula2012guaranteed,fisac2019general}
%- it is insensitive to the magnitude of the irreversible disruption
%- irreversible transitions can happen spontaneously (due to the forces of nature, the actions of other agents, etc).  the agent has an incentive to interfere to prevent them

%\cite{hadfield2021principal} Assume some of the features $K$ are not mentioned.  Then optimize under the constraint that the unmentioned features are not changed.
%\begin{align}
%    \min_{\phi\in\Phi} & U(\phi) \\
%    s.t. & \phi_K = \phi_K^{(0)} \\
%    & C(\phi) \le 0
%\end{align}

%\cite{krakovna2019penalizing} empirically compare various deviation measures: unreachability, relative reachability, attainable utility, value difference measures
%\cite{krakovna2020avoiding} Avoiding Side Effects By Considering Future Tasks








 
\documentclass[pdflatex,sn-mathphys-num]{sn-jnl}%


\usepackage{graphicx}%
\usepackage{multirow}%
\usepackage{amsmath,amssymb,amsfonts}%
\usepackage{amsthm}%
\usepackage{mathrsfs}%
\usepackage[title]{appendix}%
\usepackage{xcolor}%
\usepackage{textcomp}%
\usepackage{manyfoot}%
\usepackage{booktabs}%
\usepackage{algorithm}%
\usepackage{algpseudocode}%
\usepackage{listings}%
\usepackage{lmodern}
\usepackage{bm}
\usepackage{placeins}



\DeclareMathOperator*{\argmax}{arg\,max}
\DeclareMathOperator*{\argmin}{arg\,min}
\renewcommand{\thefootnote}{\fnsymbol{footnote}}
\theoremstyle{definition}
\newtheorem{prob}{Problem}

\begin{document}

\title[Summarising local explanations via proxies]{{\sc ExplainReduce}: Summarising local explanations via proxies}



\author*[1]{\fnm{Lauri} \sur{Sepp\"al\"ainen}}\email{lauri.seppalainen@helsinki.fi}

\author[1]{\fnm{Mudong} \sur{Guo}}\email{mudong.guo@helsinki.fi}

\author[1]{\fnm{Kai} \sur{Puolam\"aki}}\email{kai.puolamaki@helsinki.fi}

\affil[1]{%
\orgname{University of Helsinki}, \orgaddress{\street{P.O. Box 64}, \city{Helsinki}, \postcode{00014}, \country{Finland}}}



\abstract{Most commonly used non-linear machine learning methods are closed-box models, uninterpretable to humans. The field of explainable artificial intelligence (XAI) aims to develop tools to examine the inner workings of these closed boxes. An often-used model-agnostic approach to XAI involves using simple models as local approximations to produce so-called local explanations; examples of this approach include {\sc lime},  {\sc shap}, and {\sc slisemap}. This paper shows how a large set of local explanations can be reduced to a small ``proxy set'' of simple models, which can act as a generative global explanation. This reduction procedure, {\sc ExplainReduce}, can be formulated as an optimisation problem and approximated efficiently using greedy heuristics.}



\keywords{Explainable artificial intelligence, XAI, local explanations, interpretability, machine learning}



\maketitle

\section{Introduction}\label{sec:intro}
Explainable artificial intelligence (XAI) aims to elucidate the inner workings of ``closed-box'' machine learning (ML) models: models that are not readily interpretable to humans.
As machine learning has found applications in almost all fields, the need for interpretability has likewise led to the use of XAI in medicine \cite{band2023medical}, manufacturing \cite{peng2022industrial} and atmospheric chemistry \cite{seppalainen2023using}, among many other domains.
In the past two decades, many different XAI methods have been developed to meet the diverse requirements \cite{guidotti2018survey}.
These methods produce explanations, i.e., distillations of a closed-box model's decision patterns.
An ideal XAI method should be model-agnostic -- applicable to a wide range of model types -- and its explanations would be succinct, easily interpretable by the intended user, and stable.
Additionally, such explanations would be global, allowing the user to comprehend the entire mechanism of the model.
However, producing global explanations is often a challenge.
For example, if a model approximates a complex function, describing its behaviour may require describing the complex function itself, thereby defeating the purpose of interpretability.

A common approach to producing model-agnostic explanations is to relax the requirement for explaining the model globally and instead focus on local behaviour \cite{guidotti2018survey}.
Assuming a degree of smoothness, approximating the closed-box function in a small neighbourhood is often feasible using simple, interpretable functions, such as sparse linear models, decision trees, or decision rules.

We argue that these local explanations are inherently unstable.
We initially observed this phenomenon with {\sc slisemap} \cite{bjorklund2023slisemap}, where we noted that most items could be accurately approximated with several different local models.
In the same vein, take two commonly used local explanation methods, {\sc lime} \cite{ribeiro2016} and {\sc shap} \cite{lundberg2017unified}.
Both methods estimate feature importance locally by averaging predictions from the closed-box model for items sampled near the one being explained.
Such explanations can be interpreted as linear approximations for the gradient of the closed-box model.
It is well known that the loss landscapes of, e.g., deep neural network models, are not smooth.
Therefore, it can be conjectured that the predictions from closed-box can be likewise unstable.
Indeed, research has shown that local explanations of neural networks can be manipulated due to the geometry of the closed-box function \cite{dombrowski2019explanations}.
Furthermore, numerous variants of {\sc lime} which aim to increase stability (such as \cite{shankar2019alime, zafar2019dlime}) indicate that instability of {\sc lime} is a key issue to address.
We hypothesise that this instability is a property of all local explanation methods that use simple models to approximate a complex function.
Even in the theoretically ideal case, where a complex model is defined by an exact mathematical formula and the local explanations take the form of gradients, there can be points (e.g., sharp peaks) where the gradient is ill-defined, as in Fig. \ref{fig:pyramid_example}.
In practice, especially when working with noisy data and averaging over random samples, the ambiguity is exacerbated; one item may have several nearly equally viable explanations.
At the same time, some local explanations may accurately approximate many items.
This suggests that if we generate a large set of local explanations, we may be able to find a smaller subset thereof, which could effectively replace the full set without sacrificing accuracy.
As an added bonus, while the local explanations may be unstable, this subset of explanations may be more stable, as observed in \cite{seppalainen2023using}.

This paper introduces a procedure, coined {\sc ExplainReduce}, that can reduce large sets of local explanations to a small subset of so-called proxy models.
This small proxy set can be used as a global explanation for the closed-box model, among other possible applications.
Fig. \ref{fig:pyramid_example} provides insight into the process.
In the left panel, we show noisy samples (blue dots) from a closed-box function (marked with a solid blue line).
The middle panel shows a set of local explanations (as blue and orange dashed lines) produced by the XAI method {\sc slisemap} \cite{bjorklund2023slisemap}, with one explanation per data item.
The right panel shows a reduction from this large set of local models to two (solid lines), maximising the coverage of items within an error tolerance (shaded area).
The procedure offers a trade-off between interpretability, coverage, and fidelity of the local explanation model by finding a minimal subset of local models that can still approximate the closed-box model for most points with reasonable accuracy.

We start with an overview of potential applications of local explanation subsets and introduce related research.
We then define the problem of reducing a large set of local models to a proxy set as an optimisation problem.
We also outline the {\sc ExplainReduce} algorithm and our performance metrics.
In the results section, we first show how a small proxy set provides a global explanation for the closed-box model with simulated data and a practical example.
Second, we find that a proxy set of modest size attains an adherence to the closed-box model on unseen data comparable to, or even better than, the complete set of explanations.
We continue to show that we can find a good proxy set even when starting from a limited set of initial explanations.
We then demonstrate how greedy approximation algorithms can efficiently solve the problem of finding the proxy set.
The code for the procedure and to recreate each of the experiments presented in this paper can be found at \url{https://github.com/edahelsinki/explainreduce}.

\begin{figure}
    \centering
    \includegraphics[width=\textwidth]{figures/pyramid_example.pdf}
    \caption{A simple example of the idea behind {\sc ExplainReduce}. A closed-box model (left) can have many local explanations (middle). We can reduce the size of the local explanation set to get a global explanation consisting of two simple models (right).}
    \label{fig:pyramid_example}
\end{figure}

\section{Applications for reduced local explanation sets}

\emph{Global explanations}:
Producing succinct global explanations for complex closed-box functions remains a challenge.
Given a local explanation method, a naive approach would be simply producing a local explanation for each item in a given training set.
However, this leaves the user with $n$ local explanations without guaranteeing global interpretability.
In practical settings, many of these local explanations are also likely to be similar and thus redundant.
Summarising the large set of local explanations with a reduced set of just a few local models would be much more palatable to the user as a global explanation.

\noindent\emph{Interpretable replacement for the closed-box model}:
Assuming that the method used to produce local explanations is faithful, i.e., able to accurately predict the behaviour of the closed-box function locally, the set of all local explanations can replace the closed-box model with little loss to accuracy.
However, a large set of (e.g., $500$) local models can hardly be called interpretable.
If there is sufficient redundancy in the explanations, selecting a small number of representative models yields a global explanation and provides an interpretable surrogate for the closed-box model.

\noindent\emph{Exploratory data analysis and XAI method comparison}:
Studying how the set of local explanations can be reduced can offer interesting insights about the data and XAI method used.
If, for example, different subsets of the data are consistently associated with very similar proxy sets, it implies that the closed-box model can be replaced with a reasonable loss of accuracy by a small set of simple functions.
Similarly, comparing the reduced proxy sets with individual explanations from different XAI methods could provide new insights into the explanation generation mechanisms in various settings.

\noindent\emph{Outlier detection}:
Given a novel observation, we can study how well the local models in the reduced set can predict it compared to similar items in the training dataset.
If the novel item is better explained by a different model than the one used for other similar items, or if no explanation in the reduced set accurately captures the relationship, the novel item could be considered an outlier.

\section{Related work}


Modern machine learning tools have become indispensable in almost all industry and research fields.
As these tools find more and more applications, awareness of their limitations has simultaneously spread.
Chief among these is the lack of interpretability inherent to many of the most potent ML methods.
Understanding why ML models produce a specific prediction instead of something else poses obstacles in their adoption in, e.g., high-trust applications \cite{dosilovic2018survey}.
Such opaque methods are colloquially called black-box or closed-box methods, in contrast with white- or open-box methods, characterised by their interpretability to humans.
Examples of closed-box methods include random forests and deep neural networks, whereas, e.g., statistical models like linear regression models are often considered open-box methods.
As a result, the field of XAI has grown substantially in the last decade.
This section describes several post-hoc local explanation and model aggregation methods to give the reader context for the {\sc ExplainReduce} procedure.

XAI methods generally take a dataset and a closed-box function as inputs and produce an explanation that describes the relationship between the inputs and outputs of the closed-box model.
One commonly used approach is to produce post-hoc explanations using local (open-box) surrogate models.
In this approach, given a closed-box method $f: \mathcal{X} \rightarrow \mathcal{Y}, f(\bm{X}) = \hat{\bm{y}}$, a local surrogate model $g$ is another function that replicates the behaviour of the closed-box method in some region of the input space.
Mathematically, given a loss measure $\ell$ and a point in the input space $\bm{x}_i$, we might consider models $g$ such that $\ell(f(\bm{x}'_i), g(\bm{x}'_i)) \leq \varepsilon \; \forall \bm{x}'_i \in \{\bm{x} \in \mathcal{X} | D(\bm{x}'_i, \bm{x}_i) \leq \delta \}$ as local surrogate models.

Many methods have been proposed to produce local surrogate models as explanations.
Theoretically, the simplest way to produce a local surrogate model would be to calculate the gradient of the closed-box function.
This XAI method is often called {\sc vanillagrad} in the literature.
However, in practice, the gradients of machine learning models can be very noisy, as demonstrated by the effectiveness of adversarial attacks that exploit small perturbations in neural networks \cite{szegedy2014intriguing}.
On the other hand, many commonly used machine learning models, such as random forests, do not have well-defined gradients.
Hence, more involved approaches are warranted. 

{\sc smoothgrad} \cite{smilkov2017smoothgrad} attempts to solve the gradient noise problem in {\sc vanillagrad} by averaging over gradients sampled from the vicinity of the point under explanation.
Although the original paper only applies {\sc smoothgrad} to classification, the method can easily be extended to regression. 

Moving away from directly analysing the gradient, {\sc lime} \cite{ribeiro2016} and {\sc shap} \cite{lundberg2017unified} are perhaps the most widely known examples of practical local explanation generation methods.
In this paper, we focus on the {\sc kernel-shap} variant, which combines linear {\sc lime} with {\sc shap}.
Both {\sc lime} and {\sc kernel-shap} produce explanations in the form of additive linear models ($\hat{y} = \phi^T \bm{x}$).
Given an item $\bm{x}_i$, the methods sample novel items $\bm{x}'_i \in \mathcal{X}'$ in the neighbourhood of the first item, use the closed-box model to predict the labels for the novel items and find the linear model that best fits them \cite{lundberg2017unified},
\begin{equation}
    \hat{\phi} = \arg \min\nolimits_{\phi} \sum\nolimits_{\bm{x}'_i \in \mathcal{X}'} [f(\bm{x}'_i) - \phi^T \bm{x}'_i]^2 \pi_x(\bm{x}'_i) + \Omega(\phi),
\end{equation}
where $\pi_x$ represents a distance measure and $\Omega(\phi)$ is a regularisation term.
The difference between the methods lies in the choice of distance measure, which defines the notion of neighbourhood for $\bm{x}_i$; {\sc lime} most often uses either $L^2$ or cosine distance, whereas {\sc kernel-shap} utilizes results in game theory \cite{slack2021reliable}.

In addition to a procedure for generating local explanations, the authors of {\sc lime} also propose a method which constructs a global explanation by selecting a set of items whose explanations best capture the global behaviour.
They term this procedure \emph{submodular pick algorithm}.
The procedure first generates a local explanation for each item within the dataset. 
Then, a global feature importance score is calculated as the sum of the square roots of the feature attributions aggregated across all local explanations.
If we only pick items according to the presence of the most important features, there is a danger of ending up with many similar explanations.
Hence, \emph{submodular pick algorithm} encourages diversity by framing the problem as a weighted covering problem, balancing feature importance and representativeness such that the user sees a diverse set of explanations.

{\sc smoothgrad}, {\sc lime} and {\sc shap} are based on sampling novel items, which, while simple to implement, introduces a unique set of challenges.
First, formulating a reliable data generation process or sampling scheme for all possible datasets is difficult, if not impossible \cite{guidotti2018survey, laugel2018defining}.
For example, images generated by adding random noise to the pixel values rarely resemble natural images.
Second, randomly generating new items might produce items that cannot occur naturally due to, for example, violating the laws of physics.
{\sc slisemap} \cite{bjorklund2023slisemap} and its variant, {\sc slipmap} \cite{bjorklund2024SLIPMAP}, produce both a low-dimensional embedding for visualization and a local model for all training items without sampling any new points.
{\sc slisemap} finds both the embedding and local models by optimising a loss function consisting of two parts: an embedding term, where items with similar local explanations attract each other while repelling dissimilar ones and a local loss term for the explanation of each item:
\begin{equation}
    \min_{g_i} \mathcal{L}_i = \sum\nolimits_{i=1}^n \frac{\exp(-D(\bm{z}_i, \bm{z}_j))}{\sum_{k=1}^n \exp(-D(\bm{z}_k, \bm{z}_j))} \ell(g_i(\bm{x}_j), \bm{y}_j) + \Omega(g_i)
\end{equation}
where $D(\cdot, \cdot)$ is the euclidean distance in the embedding, $g_i$ represents the local model, $\ell$ is the local loss function, and $\Omega$ again denotes regularisation term(s).
In {\sc slisemap}, each item is fitted to its local models; in the {\sc slipmap} variant, the number of local models is fixed to some $p$, usually much less than the number of data items $n$, and the training items are mapped to one of the $p$ local models.

Another large class of explanations which deserves mention is the so-called case-based or example-based XAI methods, which use representative samples in the training set to explain novel ones \cite{agnar1994casebased, kim2016examples}.
One such method is using prototypes, which present the user with the most similar ``prototype items'' as an explanation, such as showing images of birds with similar plumage as a basis for classification. 

A shared property of post-hoc local surrogate models is the lack of uniqueness; for a given item, many local surrogates may exist with similar performance.
The phenomenon is documented for {\sc slisemap} in \cite{bjorklund2023slisemap} and implied for other methods based on the results in \cite{dombrowski2019explanations}, as well as for many publications aimed at fixing the inherent instability of {\sc lime} \cite{shankar2019alime, zafar2019dlime, zhao2021baylime}.
We argue that the existence of such alternative explanations is an inherent feature of using local surrogate models.
Intuitively, we can imagine the $n$-surface of a complex closed-box function and consider local surrogates as planes with dimensionality $n-1$.
There are many ways to orient a local surrogate on the curved surface of the closed-box function while retaining reasonable local fidelity.
Interestingly, the existence of alternative explanations implies that there may be surrogates which perform well for many items in the data distribution.
Therefore, we might be able to reduce a large set of local models to a small set of widely applicable surrogates, providing a global explanation of the model's behaviour.

The method proposed in the previous paragraph falls under \emph{model aggregation}.
The submodular pick algorithm mentioned when discussing {\sc lime} is an example of a model aggregation method.
Other methods include Global Aggregations of Local Explanations ({\sc gale}) \cite{vanderlinden2019global}, {\sc GLocalX} \cite{setzu2021glocalx}, and an integer programming-based approach introduced by Li et al. \cite{li2022optimal}.
{\sc gale} offers alternative ways to calculate feature importance for the submodular pick algorithm, as the authors argue that the way these importance values are calculated in the original \cite{ribeiro2016} is only applicable to a limited number of scenarios.
Furthermore, they show how the choice of the best-performing importance value definition is task-dependent.
In {\sc GLocalX} \cite{setzu2021glocalx}, the authors propose a method to merge rule-based explanations to find a global explanation.
In the programming-based approach (\cite{li2022optimal}), the authors take a similar approach to the one proposed in this paper.
They also attempt to find a representative subset of local models, and formulate model aggregation as an optimisation problem with fidelity and coverage constraints.
However, their work has some limitations.
First, their method relies on the definition of applicability radii for the local models, i.e., radii within which the explanation holds.
Second, the framework only functions in classification tasks.
Third, to satisfy the optimisation constraints, the framework requires the inclusion of tens of local models into the aggregation, limiting the interpretability of the aggregation as a global model.
Finally, they only tested their framework with random forest models and two datasets.
Because the model aggregation methods described above cannot be directly applied to an arbitrary set of local explanations, in this paper we opt to measure the performance of {\sc ExplainReduce} against the full set of local explanations instead.



\section{Methods}

In this section, we describe the idea and implementation of {\sc ExplainReduce}.
Assuming that many items in a dataset can have many alternative explanations of similar performance, a proper subset of explanations can accurately model most of the items in the dataset. 
We refer to this smaller set as a set of ``proxy models''.
Thus, the method combines aspects of local surrogate explanations with prototype items; instead of representative data items, we use representative local surrogate models to summarise global model performance.

The {\sc ExplainReduce} procedure works as follows: after training a closed-box model, we generate a large set of local explanations for the closed-box and then find a covering subset of the local models, which acts as a global explanation.
In this section, we first define the problem of finding the subset of local models and then move on to cover the reduction methods and algorithms that generate these proxy sets.
We also introduce the quality metrics used to evaluate the performance of reduced sets.

\subsection{Problem definition}
\label{sec:problem}
A dataset $\mathcal{D} = \{(\bm{x}_1, \bm{y}_1) , \ldots, (\bm{x}_n, \bm{y}_n)\}$ consists $n$ of data items (covariates) $\bm{x}_i \in \mathcal{X}$ and labels (responses) $\bm{y}_i \in \mathcal{Y}$. We use ${\bm X}$ denote a matrix of $n$ rows such that ${\bm X}_{i\cdot}=\bm{x}_i$.
If we have access to a trained supervised learning algorithm $f(\bm{x}_i) = \hat{\bm{y}}_i$, we can instead replace the true labels $\bm{y}_i$ with the predictions from the learning algorithm $\hat{\bm{y}}_i$.
A local explanation for a data item $(\bm{x}_i, \bm{y}_i)$ is a simple model $g(\bm{x}_i) = \Tilde{\bm{y}}_i$  which locally approximates either the connection between the data items and the labels or the behaviour of the closed-box functions. 
In the previous section, we gave multiple examples of generating such explanations.
Assume we have generated a large set of such local explanations $\bm{G} = \{g_j\mid j\in[m]=\{1,\ldots,m\}\}$ and a mapping from data items to models $\Phi:{\mathcal{X}}\mapsto[m]$ using one of these methods.
Assume also that these local models can be identified by a set of $p$ real parameters (such as coefficients of linear models), which we will denote with $\bm{B}\in{\mathbb{R}}^{m\times p}$.
We define a loss function $\ell:\mathcal{Y} \times \mathcal{Y} \rightarrow \mathbf{R}_{\geq 0}$ and a loss matrix $\bm{L}\in{\mathbb{R}}_{\ge 0}^{m\times n}$, where individual items are defined as $\bm{L}_{ij} = \ell(g_i(\bm{x}_j), \bm{y}_j)$.
A straightforward example of a mapping $\Phi$ would then be chosen from the local models, the one with the lowest loss for each item: $\Phi(\bm{x}) = \arg\min\nolimits_{i\in[m]}{\ell(g_i(\bm{x}_j), \bm{y}_j)}$. 

Finally, we assume that the large set $\bm{G}$ contains a reasonable approximation for each item in the dataset ${\cal D}$: for all $i \in [n]$ and for a given $\varepsilon\in{\mathbb{R}}_{>0}$ there exists $j\in[m]$ such that $\ell (g_j(\bm{x}_i), \bm{y}_i) \leq \varepsilon$.
This can be achieved by, e.g., learning a local explanation for each item in ${\cal D}$.

We are interested in how many items can be explained by a given set of local surrogates to a satisfactory degree.
To measure this, we use coverage $C$, defined as the proportion of data items which can be explained sufficiently by at least one local model in a subset ${\bm{S}} \subseteq[m]$.
Mathematically, given a loss threshold $\varepsilon\in{\mathbb{R}}_{>0}$, the coverage can be calculated as
\begin{equation}
    C({\bm{S}}, \varepsilon) = (1/n) \left| \{j\in [n]\mid\min\nolimits_{i\in{\bm{S}}}{} \ell(g_i(\bm{x}_j), \bm{y}_j) \leq \varepsilon \; \} \right|.
\end{equation}
A $c$-covering subset of local models ${\bm{S}} \subseteq [m]$ is a set for which $C({\bm{S}}, \varepsilon) \geq c$. 


Next, we define three computational problems to address the task described earlier. Later, we will introduce algorithms to solve each of the problems.

The first formulation attempts to minimise the number of items for which a satisfactory local explanation is not included in the subset ${\bm{S}}_c$.
\begin{prob} ({\sc maximum coverage})\label{prob:1}
Given $k$ and $\varepsilon\in{\mathbb{R}}_{>0}$, find a subset ${\bm{S}}_c$ of cardinality $k$ that maximises coverage, or
\begin{equation}
{\bm{S}}_c=\arg\max\nolimits_{{\bm{S}}\subseteq [m]_k} C({\bm{S}}, \varepsilon), 
\end{equation}
where we have used $[m]_k=\{{\bm S}\subseteq[m]\mid\left|{\bm S}\right|=k\}$ to denote the subsets of cardinality $k$.
\end{prob}



The second definition attempts to capture a subset of explanations that can be used as a proxy model with a small average loss.
\begin{prob} ({\sc minimum loss})\label{prob:2}
Given $k$, find a subset ${\bm{S}}_c$ with cardinality $k$ such the average loss when picking the lowest loss model from ${\bm{S}}_c$ is minimised, or
\begin{equation}
{\bm{S}}_c=\arg\min\nolimits_{{\bm{S}}\in[m]_k}{\left(
\frac 1n\sum\nolimits_{j=1}^n{
\min\nolimits_{i\in{\bm{S}}}{\ell(g_i({\bf x}_j), {\bf y}_j)}
}
\right)}.
\end{equation}
\end{prob}


The final formulation is a combination of problems \ref{prob:1} and \ref{prob:2}.
\begin{prob} ({\sc coverage-constrained minimum loss})\label{prob:3}
Given $k$, $\varepsilon\in{\mathbb{R}}_{>0}$, and minimum coverage $c\in (0, 1]$, find a $c$-covering  ${\bm{S}}_c$ with cardinality $k$ such the average loss when picking the lowest loss model from ${\bm{S}}_c$ is minimised, or
\begin{equation}
 {\bm{S}}_c=\arg\min\nolimits_{{\bm{S}}\in[m]_{k,c}}{\left(
\frac 1n\sum\nolimits_{j=1}^n{
\min\nolimits_{i\in{\bm{S}}}{\ell(g_i({\bf x}_j), {\bf y}_j)}
}
\right)},
\end{equation}
where $[m]_{k,c}=\{{\bm S}\in[m]_k\mid C({\bm{S}},\varepsilon)\ge c\}$ are the $c$-coverings of cardinality $k$.
\end{prob}

\subsection{General procedure}\label{sec:procedure}

The {\sc ExplainReduce} algorithm is outlined in Algorithm \ref{alg:main}. 
Given a dataset $\mathcal{D}$, an explanation method $\text{Exp}$ that generates a set of $m$ local explanations (a special case being one local explanation for each training data point, in which case $m=n$) and a reduction algorithm explained later in Sect. \ref{sec:reduction_algs}, we first use the explanation method to generate $m\le n$ explanations for $m$ items sampled without replacement from $\mathcal{D}$.
If a closed-box function is provided, we replace the original labels in $\mathcal{D}$ with predictions from $f$.
We then apply the reduction algorithm to the generated set of local explanations $[m]$ and receive the proxy set ${\bm{S}}_c \subseteq [m]$.
Finally, for each sampled data item, we pick the local explanation in the proxy set ${\bm{S}}_c$ with minimal loss, mapping the items and the proxies.


\begin{algorithm}
\caption{{\sc ExplainReduce} Procedure to find a subset of explanations.}
\label{alg:main}
\hspace*{\algorithmicindent} \textbf{Input:} $\mathcal{D} \gets \{(\bm{x}_i, \hat{\bm{y}}_i) | i \in [n]\}$: dataset; ${\bm G}$: the set of $m$ local explanations; $\textrm{reduce}$: method to find ${\bm S}_c$, parametrised optionally by $\varepsilon$, $c$, or $k$, see Sect. \ref{sec:reduction_algs}. \\
\hspace*{\algorithmicindent} \textbf{Output:} ${\bm S}_c$: reduced set of explanations; $\text{map}$: $\text{map}[i]: [n]\mapsto{\bm S}_c$, mapping from the local dataset $[n]$ to explanations in ${\bm S}_c$.
\begin{algorithmic}
    \Procedure{ExplainReduce}{}    
    \State $\bm{S}_c \gets \textrm{reduce}(\bm{G}, \varepsilon, c, k)$ \Comment{$\textrm{reduce}$ is defined in Sect. \ref{sec:reduction_algs}}
    \State $\text{map} \gets \{\}$ \Comment{a mapping between items in $\mathcal{D}$ and the local models}
    \For{$i \in [n]$}
       \State $\text{map}[i] \gets \arg\min_{j\in{\bm S}_c}{\ell(g_j(\bm{x}_i), \bm{y}_i)}$
    \EndFor \\
    \Return ${\bm S}_c,\; \text{map}$\\
    \EndProcedure
\end{algorithmic}
\end{algorithm}
 
\subsection{Reduction algorithms}
\label{sec:reduction_algs}

In this section, we briefly overview the practical implementations of the reduction algorithms (function ${\textrm{reduce}}$ in Alg. \ref{alg:main}) used to solve the problems outlined in the previous section.

{\sc max coverage}: 
Problem 1 is a variant of the NP-complete partial set covering problem, sometimes called {\sc max $k$-cover}.
The equivalence is obvious if we consider an item $(\bm{x}_j, \bm{y}_j) \in \mathcal{D}$ covered by model $g_i$ if $\ell(g_i(\bm{x}_j), \bm{y}_j) \leq \varepsilon$.
We solve Prob. \ref{prob:1} exactly using integer programming (implemented by the {\sc pulp} Python library \cite{mitchell2024pulp}) and approximately using a greedy algorithm.
In the greedy approach, given fixed $k$, we iteratively pick local models $g \in \bm{G}$ such that the marginal increase in coverage is maximised with each iteration until $k$ models have been chosen.
Our problem is submodular, as adding each new model to the subset cannot decrease the coverage.
It has been shown that in this case, the greedy algorithm has a guaranteed lower bound to achieve coverage at least $1 - \left((k - 1)/k\right)^k$ times the optimal solution \cite{nemhauser1978analysis}.

Notably, this approximation ratio only applies to the original set of data items and their associated local explanations.
If we apply the proxy sets to novel data, we should use standard machine learning tools --- such as a separate validation set --- to ensure that the model performs appropriately.

{\sc min loss}:
Problem \ref{prob:2} is an example of a supermodular minimisation problem.
Let $f(\bm{S}) = \left(\frac 1n\sum_{j=1}^n{\min_{i\in{\bm{S}}}{\ell(g_i({\bf x}_j),{\bf y}_j)}} \right)$ and $\bm{A}, \bm{B}: \; \bm{A} \subset \bm{B} \subset \bm{G}$ be subsets of local models.
Additionally, let $v$ be a local model not contained in $\bm{B}$.
Clearly, the decrease in loss by adding $v$ to the larger set $\bm{B}$ must be, at most, as great as adding the same model $v$ to $\bm{A}$.
In other words,
\begin{equation}
    f(\bm{A} \cup \{v\}) - f(\bm{A}) \leq f(\bm{B} \cup \{v\}) - f(\bm{B}) \quad \forall \bm{A} \subset \bm{B} \subset \bm{G}, \; v \notin \bm{B},
\end{equation}
which is the definition of supermodularity \cite{mccormick2005submodular}.

Supermodular minimisation problems are known to be NP-hard.
Hence, we only use a greedy ascent algorithm to solve Problem \ref{prob:2}, as finding an exact solution is computationally expensive due to the continuous nature of loss.
In the worst case, the search for the optimal subset would require $\binom{m}{k}$ comparisons.
Like above, we iteratively pick local models with the best possible decrease in marginal loss until we reach $k$ chosen local models.
In the general case, a multiplicative approximation ratio for a supermodular minimisation problem may not exist due to the optimal solution having a value $f(\bm{S}^*) = 0$, while the greedy algorithm may converge to a solution with non-zero loss.
However, in our case, finding a subset of models with exact zero loss for all items is unlikely.
In \cite{ilev2001approximation}, the author derives a curvature-based approximation ratio for a greedy descent (worst out) algorithm.
Based on this analysis, authors of \cite{bounia2023approximating} derive an approximation ratio for greedy ascent for probabilistic decision trees.
Neither of these approaches is directly applicable to our setting, and hence, we cannot give a closed-form approximation ratio.
However, as we show in Section \ref{sec:greedy}, the empirical approximation ratios for the greedy ascent algorithm are reasonable.
Moreover, we show how the greedy approximation performs nearly equally, if not better, on unseen data compared to the exact solution for our datasets.

{\sc const min loss}:
We solve Problem 3 again with a greedy ascent algorithm, i.e., iteratively picking the best model to include in the subset ${\bm S}_c$ until the maximum coverage has been met.
The only difference to {\sc min loss} algorithm is how the models are scored.
If the coverage constraint has been met, we simply follow the procedure in {\sc min loss} and select the model with the best possible marginal loss decrease.
Otherwise, we divide the marginal loss decrease $\Delta \ell_i$ we would get by including model $i$ in the proxy set with the marginal coverage increase of including the same model $\Delta c_i$.
This reduction method can be implemented with a hard constraint by having the algorithm throw an exception if the coverage is not met or with a soft constraint where the resulting proxy set is returned regardless of satisfying the constraint.
In this paper, we opt to use the softly constrained variant exclusively.
We can see that in the case where the coverage constraint is not met, Problem 3 with this scoring scheme is also a supermodular minimisation problem and equivalent to Problem 2 with a different optimisation objective, namely the coverage-normalised loss.
Similar to Problem 2, we cannot give an exact approximation ratio for this algorithm, but the empirical results in Section \ref{sec:greedy} suggest good performance.

{\sc clustering}:
Above, we have treated the problem as a set covering problem.
We can also approach the problem from the unsupervised clustering perspective: can we cluster the training data to find a proxy set?
Given the number of clusters $k$, we can cluster the training data items $\bm{X}$, the local model parameters $\bm{B}$, or the training loss matrix $\bm{L}$ using some clustering algorithm.
In this paper, we use {\sc k-means} with Euclidean distance for $\bm{X}$ and $\bm{L}$, and cosine distance for the local model parameters $\bm{B}$.
After performing the clustering, we pick the local model closest to the cluster centroid for each cluster to form the proxy set.

\subsection{Performance measures}

\bmhead{Fidelity} Fidelity \cite{guidotti2018survey} measures the adherence of a surrogate model to the complex closed-box function.
Given a closed-box function $f$, we fidelity is the loss between the closed-box model prediction $\hat{\bm{y}} = f(\bm{x})$ and the surrogate model prediction:
\begin{equation}
    \textrm{fidelity} = \frac{1}{n} \sum\nolimits_{i=1}^{n} \ell(g_i(\bm{x}_i), \hat{y}_i),
    \label{eq:fidelity}
\end{equation}
where $g_i$ is the relevant local model either from the full set of local explanations or the proxy model set.

\bmhead{Instability} Instability \cite{guidotti2018survey} (sometimes also referred to as \emph{stability}, despite lower values denoting better performance) measures how much a slight change in the input changes the explanation.
In practice, we model the slight change by measuring the loss of a given local model $g_i$ associated with item $\bm{x}_i$ with its $\kappa$ nearest neighbours:
\begin{equation}
    \textrm{instability} = \frac{1}{n} \sum\nolimits_{i=1}^{n} \frac{1}{\kappa} \sum\nolimits_{j \in \textrm{NN}_\kappa(i)} \ell(g_i(\bm{x}_j), y_j). 
    \label{eq:instability}
\end{equation}
This paper uses a fixed number of $\kappa=5$ nearest neighbours when we report instability values ($\left|\textrm{NN}_5(i)\right|=5$).

\section{Experiments}

In this section, we demonstrate the performance of the proxy sets produced via different XAI methods and reduction strategies.
We experiment on a variety of different datasets, consisting of both classification and regression tasks.
For each of the datasets used, we train a closed-box model and produce the initial set of local explanations using that model.
A brief description of the datasets can be found in Table \ref{tab:datasets}, and further details in Appendix \ref{a:datasets}.

\begin{table}[b]
    \centering
\begin{tabular}{l|cccc}
Dataset Name    & Size        & Task           & Closed-box model  & Citation                                    \\ \hline
Synthetic       & 5000 × 11   & Regression     & Random Forest     & \cite{bjorklund2023slisemap}                \\
Air Quality     & 7355 × 12   & Regression     & Random Forest     & \cite{oikarinenDetectingVirtualConcept2021} \\
Life Expectancy & 2938 × 22   & Regression     & Neural Network    & \cite{rajarshi2017life}                     \\
Vehicle         & 2059 × 12   & Regression     & SVR               & \cite{birla2022vehicles}                    \\
Gas Turbine     & 36733 × 9   & Regression     & Adaboost          & \cite{2019gas}                              \\
QM9             & 133766 × 27 & Regression     & Neural Network    & \cite{ramakrishnan2014Quantum}              \\
Jets            & 266421 × 7  & Classification & Random Forest     & \cite{CMS:opendata}                         \\
HIGGS           & 100000 × 28 & Classification & Gradient Boosting & \cite{whiteson2014higgs}                    \\ \hline
\end{tabular}
    \caption{Summary of datasets used in experiments.}
    \label{tab:datasets}
\end{table}

In the experiments, we measure fidelity mostly with test data, i.e., data not used in generating the full set of local explanations.
This metric better captures the adherence of the explainer to the closed-box model than calculating the same value on the data used to generate the local explanations.

The local explanation methods discussed in this paper are generally not generative, that is, there is no obvious mapping from a new covariate ${\bm{x}}$ to an explanation model in ${\bm{S}}_c$.
Therefore, to use either the full explanation sets or the proxies to produce predictions, we need to generate another mapping between novel items and local models in the explanation set.
The simplest of such mappings is to use the distance in the data space: $\Phi_{\textrm{new}}( \bm{x}) = \arg\min\nolimits_{i\in{\bm{S}}_c} ||\bm{x}_i - \bm{x}||^p, \; i \in (1, ..., n)$, where $||a||^p$ is the $p$-norm.
This new mapping is used for unlabelled items only and is in addition to the loss-minimising mapping used in the reduction process to find an initial mapping with the ``training'' data (data used to generate the local explanations) described in Section \ref{sec:procedure}.

\subsection{Case studies}
We begin our examination by demonstrating the {\sc ExplainReduce} procedure with two case studies to give the reader a better intuition of the procedure and its possible usage.

\subsubsection{Synthetic data}

To give a simple example of the procedure, we apply it to a synthetic dataset.
The dataset, which is described in detail in Appendix \ref{a:datasets}, is generated by producing $k=4$ clusters in the input space and generating a random, different linear model for each cluster.
We then generate labels by applying a local model to the data items based on their cluster ID and adding Gaussian noise.
The dataset is then randomly split into a training set and a test set, and we train a {\sc smoothgrad} explainer on the training data.
For reduction, we use a greedy {\sc max coverage} algorithm, where $\varepsilon$ is defined as the 10th percentile of the loss matrix $\bm{L}$.

In Fig. \ref{fig:case_PCA}, we show a PCA of the items in the test set on the left, coloured with the ground truth cluster labels, and how the test items are mapped to proxy models on the right.
Overall, we can see that most items get mapped to the correct proxy model for that particular cluster.
The small impurity in the clusters stems both from the Gaussian noise and the approximative nature of the greedy coverage-maximising algorithm.
Furthermore, as Fig. \ref{fig:case_radar} shows, the reduced proxy models (red) correspond well with the ground truth models (blue).
The proxy set thus serves well as a generative global explanation for the dataset.

\begin{figure}
    \centering
    \includegraphics[width=\textwidth]{figures/case_PCA.pdf}
    \caption{Ground PCA of a synthetic test dataset (left) and the same dataset where colours correspond to reduced model indices, based on {\sc smoothgrad} local models and reduced using a greedy coverage-maximising algorithm (right). We can see that the reduction is able to faithfully approximate the ground truth clustering.}
    \label{fig:case_PCA}
\end{figure}

\begin{figure}
    \centering
    \includegraphics[width=\textwidth]{figures/case_radar.pdf}
    \caption{Radar plots showing ground truth local model parameters (blue) and corresponding proxy model parameters (red). The reduction method is able to find very similar surrogate models to the ground truth.}
    \label{fig:case_radar}
\end{figure}

\subsubsection{Particle jet classification}

In a previous work \cite{seppalainen2023using}, we analysed a dataset containing simulated LHC proton-proton collisions using {\sc slisemap}.
These collisions can create either quarks or gluons, which decay into cascades of stable particles called jets.
These jets can then be detected, and we can train the classifier to distinguish between jets created by quarks and gluons based on the jet's properties.

We first trained a random forest model on the data and then applied {\sc slisemap} to find local explanations.
We clustered the {\sc slisemap} local explanations to 5 clusters and analysed the average models for each cluster, and found them to adhere well to physical theory.
When we generate 500 {\sc slisemap} local explanations on the same dataset and apply the {\sc const min loss} reduction algorithm with $k=4$ proxies, we find the proxy set depicted in Fig. \ref{fig:jets_example}.
The left panel shows a swarm plot where each item is depicted on a horizontal line based on the random forest-predicted probability of the jet corresponding to a gluon jet and coloured based on which proxy model is associated with the item.
The right panel shows the coefficients of the proxy models, which are regularised logistic regression models.
We find that the proxies are similar to the cluster mean models shown in the previous publication and show similar adherence to the underlying quantum chromodynamic theory \cite{cms2013performance}.
For example, wider jets (high {\sc jetGirth} and {\sc QG\_axis2}) are generally more gluon-like, and therefore these parameters are essential in classifying the jets.
Incidentally, proxy 3 (red) is associated with the most quark- and gluon-like jets and has high positive coefficients for both parameters.
Similarly, proxy 0 (blue) has negative coefficients for momentum ({\sc jetPt}) and {\sc QG\_ptD}, which measures the degree to which the total momentum parallels the jet.
Both of these features indicate a more quark-like jet.

\begin{figure}
    \centering
    \includegraphics[width=\textwidth]{figures/jets_example.pdf}
    \caption{Example of the {\sc ExplainReduce} procedure on a particle jet classification task. The dataset consists of LHC proton-proton collision particle jets that are created by decaying gluons or quarks. We train a random forest classifier on the dataset and use {\sc slisemap} to generate 500 explanations, which are then reduced to $k=4$ proxies using the {\sc const min loss} reduction algorithm. The left panel shows a swarm plot of the 500 items sorted horizontally based on the predicted probability of corresponding to a gluon; the y-axis has no significance. The right panel shows the coefficients of the logistic regression proxy models, which match the underlying physical theory.}
    \label{fig:jets_example}
\end{figure}

\subsection{Fidelity of proxy sets}

The interpretability of a set of local models as a global explanation is directly related to the size of that set.
As explained in a previous section, local explanation models often need to balance interpretability and granularity: a large set of local models may more easily cover a larger portion of the closed-box function, but the increased number of models makes the result less understandable.
In Fig. \ref{fig:fidelity_k}, we show the test fidelity (as defined in Eq. \eqref{eq:fidelity}) as a function of the size of the reduced explanation set for a selection of datasets and local explanation algorithms.
The fidelity is calculated with respect to the closed-box predicted labels, with a fidelity of 0 representing perfect adherence to the closed-box model.
As the figure shows, a set as small as $k=5$ proxies can reach or even surpass the fidelity of a set of $500$ local explanations for all three selected datasets\footnote{For full coverage on all of the seven datasets, we refer to Appendix \ref{a:fidelity_k_full}.}.
We find that optimisation-based approaches perform particularly well, especially the two greedy loss-minimising algorithms.
The clustering-based reduction methods seem to work almost equally well, as does a random selection of local models, although optimisation-based approaches consistently outperform the latter.
This implies that for proxy set generation methods that require choosing $k$ \emph{a priori}, performance is not very sensitive to the choice of $k$.
Therefore, it is enough for a user of {\sc ExplainReduce} to specify a reasonable value of $k$ for a faithful and interpretable global surrogate.
\begin{figure}
    \centering
    \includegraphics[width=\linewidth]{figures/fidelity_k.pdf}
    \caption{Test set fidelity of explanations as a function of proxy set size $k$. The rows show different datasets, and the columns show different XAI generation methods. The different line colours and styles denote the reduction strategy, and the black horizontal line shows the performance of the full explanation set. The loss-minimising reduction methods consistently reach a fidelity comparable or even better than the full explanation set.}
    \label{fig:fidelity_k}
\end{figure}

If a few proxies are enough to find a faithful representation, how many local explanations do we need to generate to find a good proxy set?
The procedure may be infeasible if a very large set of initial local explanations is required.
For example, {\sc slisemap} scales as $\mathcal{O}(n^2m)$, where $n = |\mathcal{D}|$ and $m$ is the number of features \cite{bjorklund2023slisemap}, which makes generating thousands of explanations expensive.

To study the generalisation performance of the proxy sets, we artificially limit the set of all local models $\bm{G}$ via random sampling before reduction.
That is to say, we generate local approximations for each item in a randomly sampled subset of points in the training set ${\cal D}$.
We then calculate the test fidelity of the proxy set trained on the limited universe as above and compare that to the performance of the full explanation set.
As Fig. \ref{fig:fidelity_n} shows, the proxy sets generalize well even starting from a limited set of local models\footnote{Full version shown in Appendix \ref{a:fidelity_n_full}.}.
The coloured lines represent the various reduction algorithms, while the thick dashed black line denotes the performance of the entire set of local explanations.
The loss-minimising greedy algorithms again achieve the best performance across the studied datasets and explanation methods studied, often matching and sometimes outperforming the full explanation set.
On the other hand, the clustering-based approaches fall behind here, as in the previous section.
These results imply that the procedure of {\sc ExplainReduce} does not require training local models for each item; instead, a randomly sampled subset of models suffices to find a good proxy set.
Based on these results, we set the default number of subsamples to $n=500$ in this paper unless otherwise stated.
\begin{figure}
    \centering
    \includegraphics[width=\linewidth]{figures/fidelity_n.pdf}
    \caption{Test set fidelity of explanations as a function of local model subsample size $n=|\mathcal{D}|$. Different datasets are depicted as rows with the XAI methods used to produce the initial local explanations shown as rows. The coloured lines depict the test fidelity of the proxy sets produced with different reduction algorithms, with the thick black line denoting the performance of the entire local explanation set.}
    \label{fig:fidelity_n}
\end{figure}

We also experimented with the impact of the hyperparameters $\varepsilon$, the loss threshold, and $c$, the minimum coverage constraint, and found the reduction algorithms to have similar fidelity performance across a wide range of these hyperparameter values.
We refer to Appendix \ref{a:cov_eps_sensitivity}
for detailed results.

\subsection{Coverage and stability of the proxy sets}

It should be no surprise that methods directly optimising for loss also show the best fidelity results.
However, an ideal global explanation method should be able to accurately explain most, if not all, of the items of interest.
Additionally, explanations are expected to be locally consistent, meaning that similar data items should generally have similar explanations. 
In Fig. \ref{fig:coverage_stability}, we show the evolution of training coverage and instability of reduction methods as a function of the proxy set size $k$ with respect to the closed-box predicted labels\footnote{Full versions shown in Appendices \ref{a:coverage_full} and \ref{a:stability_full} for coverage and stability, respectively.}.
The results follow a similar pattern to the previous sections: a small number of proxies is enough to achieve high coverage and low instability, with optimisation-based approaches outperforming clustering-based alternatives.
Unsurprisingly, the coverage-optimising reduction methods achieve the best coverage with the smallest number of proxies.
However, the loss-minimising algorithms perform surprisingly well even compared to the coverage-optimising variants.
Notably, the balanced approach -- the {\sc const min loss} reduction algorithm -- almost matches the reduction algorithms directly optimising coverage.

Regarding instability, the loss-minimising reduction methods have an edge over the clustering- and coverage-based approaches, but the distinction between methods is not pronounced.
\begin{figure}
    \centering
    \includegraphics[width=\linewidth]{figures/coverage_stability_k_small.pdf}
    \caption{Coverage and instability of various reduction methods as a function of $k$ on the Gas Turbine dataset. The optimisation-based methods outperform the clustering-based approaches within both metrics.}
    \label{fig:coverage_stability}
\end{figure}

Combining the results from the fidelity, coverage, and instability experiments, the greedy loss-minimising approach with a soft coverage constraint appears to offer the best balance in performance across all three metrics.
The proxy set produced by this reduction method consistently achieves similar, if not better, performance than the full explanation set, even with a few proxies.
Based on these experiments, the various clustering-based approaches do not offer performance gains based on any of the metrics used.
While these approaches may produce more interpretable proxy sets by inherently accommodating the structures found in the data, they struggle to generate sufficiently covering proxy sets and lag behind optimisation-based approaches in both fidelity and instability.

\begin{table}[t]
    \centering
    \begin{tabular}{l@{\hspace{3mm}} r@{\hspace{3mm}}r@{\hspace{3mm}}r@{\hspace{3mm}}r@{\hspace{3mm}}r@{\hspace{3mm}}r@{\hspace{3mm}}r}
\hline \\
\multicolumn{6}{|c|}{Coverage $\uparrow$}\\
\hline\\
\bfseries Dataset & \bfseries G. Max $c$ & \bfseries A. ratio & \bfseries G. Min Loss (min $c$) & \bfseries A. ratio & \bfseries Max $c$ \\
\midrule
Air Quality & $0.82 \pm 0.18$ & $0.97 \pm 0.04$ & $0.9 \pm 0.05$ & $0.93 \pm 0.03$ & $0.84 \pm 0.18$ \\
Gas Turbine & $0.86 \pm 0.17$ & $0.99 \pm 0.01$ & $0.93 \pm 0.06$ & $0.97 \pm 0.03$ & $0.87 \pm 0.18$ \\
Jets & $0.86 \pm 0.17$ & $0.99 \pm 0.02$ & $0.95 \pm 0.06$ & $0.98 \pm 0.03$ & $0.87 \pm 0.18$ \\
QM9 & $0.78 \pm 0.17$ & $0.98 \pm 0.03$ & $0.89 \pm 0.04$ & $0.96 \pm 0.02$ & $0.8 \pm 0.18$ \\
\hline \\
\multicolumn{6}{|c|}{Train fidelity $\downarrow$}\\
\hline\\
\bfseries Dataset & \bfseries G. Min Loss & \bfseries A. ratio & \bfseries G. Min Loss (min $c$) & \bfseries A. ratio & \bfseries Min Loss \\
\midrule
Air Quality & $0.36 \pm 0.35$ & $1.25 \pm 0.21$ & $0.16 \pm 0.12$ & $1.53 \pm 0.53$ & $0.32 \pm 0.35$ \\
Gas Turbine & $0.21 \pm 0.23$ & $2.08 \pm 0.9$ & $0.07 \pm 0.07$ & $1.53 \pm 0.43$ & $0.17 \pm 0.23$ \\
Jets & $0.02 \pm 0.02$ & $1.4 \pm 0.4$ & $0.01 \pm 0.0$ & $1.41 \pm 0.27$ & $0.01 \pm 0.02$ \\
QM9 & $0.53 \pm 0.46$ & $1.25 \pm 0.21$ & $0.24 \pm 0.15$ & $1.37 \pm 0.24$ & $0.48 \pm 0.47$ \\
\hline \\
\multicolumn{6}{|c|}{Test fidelity $\downarrow$}\\
\hline\\
\bfseries Dataset & \bfseries G. Min Loss & \bfseries A. ratio & \bfseries G. Min Loss (min $c$) & \bfseries A. ratio & \bfseries Min Loss \\
\midrule
Air Quality & $0.4 \pm 0.29$ & $1.11 \pm 0.22$ & $0.22 \pm 0.07$ & $1.15 \pm 0.17$ & $0.37 \pm 0.28$ \\
Gas Turbine & $0.31 \pm 0.17$ & $1.2 \pm 0.22$ & $0.21 \pm 0.06$ & $1.1 \pm 0.11$ & $0.28 \pm 0.18$ \\
Jets & $0.02 \pm 0.01$ & $1.09 \pm 0.18$ & $0.01 \pm 0.0$ & $1.04 \pm 0.1$ & $0.02 \pm 0.01$ \\
QM9 & $1.19 \pm 0.13$ & $1.01 \pm 0.08$ & $1.09 \pm 0.06$ & $0.96 \pm 0.09$ & $1.18 \pm 0.11$ \\
\bottomrule
\end{tabular}

    \caption{Comparison of the training fidelity and coverage, alongside the observed approximation ratios (A. ratio), between the analytical and approximate optimisation algorithms when applied to $n=100$ explanations produced with {\sc lime} and with $k=5$ proxies. G. Max $c$ corresponds to the greedy {\sc max coverage} algorithm, while G. Min Loss and G. Min Loss (min $c$) correspond to the greedy {\sc min loss} and {\sc const min loss} algorithms, respectively.}
    \label{tab:greedy}
\end{table}
\subsection{Performance of greedy algorithms}
\label{sec:greedy}

In section \ref{sec:reduction_algs}, we discussed the performance of greedy approximation algorithms used to solve Problems \ref{prob:1}--\ref{prob:3}.
As we have seen in previous sections, the performance of the greedy and analytical coverage-optimising algorithms is almost equal as both a function of the proxy set size and the initial local explanation set size.
As mentioned in the section above, solving for the exact loss-minimising proxy set is a computationally challenging problem which requires $\binom{m}{k}$ comparisons in the worst case.
Hence, the analysis in this section was limited to $n=m=100, k=5$ for computational tractability.
Table \ref{tab:greedy} shows how the greedy approximation algorithms perform comparably to the analytically optimal variants.
For the {\sc max coverage} algorithm, experimental approximation ratios are substantially better than the worst-case bound described in Sect. \ref{sec:reduction_algs}.
Furthermore, as we saw in the previous sections, while on the train set the greedy heuristics do not reach the optimal solution fidelity-wise, on the test the differences between the exact solution and the greedy heuristics vanish.
The {\sc const min loss} algorithm especially shows poorer performance compared to the greedy {\sc min loss} algorithm but generalises better on the test set while achieving good coverage performance as well.
Overall, we can conclude that the greedy approximations offer a computationally feasible approach to generating proxy sets.






\section{Discussion}

Using local surrogate models as explanations is a common approach for model-agnostic XAI methods.
In this paper, we have shown that the {\sc ExplainReduce} procedure can find a post-hoc global explanation consisting of a small subset of simple models when provided a modest set of pre-trained local explanations.
The procedure is agnostic to both the underlying closed-box model and the XAI method, assuming that the XAI method can function as a generative (predictive) model.
These proxy sets achieve comparable fidelity, stability, and coverage to the full explanation set while remaining succinct enough to be easily interpretable by humans.
The loss-minimising reduction method with a soft coverage constraint especially offers a balance of fidelity, coverage, and stability across different datasets.
Furthermore, we have shown how we can use greedy algorithms to find these proxy sets with minimal computational overhead with nearly equal performance to analytically optimal counterparts.
The methods are not sensitive to hyperparameters, allowing for easy adoption.

As with any work, there are shortcomings.
The presented implementation for the procedure simply finds a set of proxies that is (approximately) optimal from a global loss or coverage perspective.
In other words, there is no consideration of the spatial distribution of the data with respect to which proxy was associated with each item.
To deepen the insight provided by the method, an interesting idea would be to combine the reduction process with manifold learning techniques, enabling visualisation alongside the proxy set, similar to {\sc slisemap} and {\sc slipmap}.
For example, we might try to find an embedding where data items associated with a particular proxy would form continuous regions, significantly increasing interpretability and augmenting the method's capabilities as an exploratory data analysis tool.

Another interesting way to leverage the set of applicable local models is uncertainty quantification.
Instead of trying to find local models applicable to many data items, we could instead turn our attention to all applicable models for a given item.
This set of applicable models could be considered as samples from the global distribution of explanations for that particular item.
Thus, we could use such sets (alongside the produced explanation) as a way to find, e.g., confidence intervals for explanations produced by a wide variety of XAI methods.

\bmhead{Acknowledgements}

We thank the Research Council of Finland for funding (decisions 364226, VILMA Centre for Excellence) and the Finnish Computing Competence Infrastructure (FCCI) for supporting this project with computational resources.

\bibliography{alternative2024}%

\newpage

\appendix

\section{Dataset introduction}
\label{a:datasets}
Below, we describe each of the datasets used in the experiments in the paper.

\noindent\textit{Synthetic} \cite{bjorklund2023slisemap} is a synthetic regression dataset with $N$ data points, $M$ features, $k$ clusters, and cluster spread $s$.
Each cluster is associated with a linear regression model and a centroid.
Coefficient vectors $\beta_j \in \mathbb{R}^{M+1}$ and centroids $c_j \in \mathbb{R}^M$ are sampled from normal distributions, with repeated samplings if either the coefficients or the centroids are too similar between the clusters.
Each data point is assigned to a cluster $j_i$, with features $x_i$ sampled around $c_{j_i}$ and standardized.
The target variable is $y_i = x_i^\top \beta_{j_i} + \epsilon_i$, where $\epsilon_i$ is Gaussian noise with standard deviation $\sigma_e$.
In the experiment section, unless otherwise stated, we set the number of data points to $N = 5000$, the number of features to $M = 11$, the number of clusters to $k = 5$, and the standard deviation of Gaussian noise to $\sigma_e = 2.0$.
As the closed-box model, we use a random forest regressor.

\vspace{0.2cm}

\noindent\textit{Air Quality} \cite{oikarinenDetectingVirtualConcept2021} contains 7355 hourly observations of 12 different air quality measurements.
One of the measured qualities is chosen as the label, and the other values are used a covariates.
With this dataset, we use a random forest regressor as the closed-box model.

\vspace{0.2cm}

\noindent\textit{Life Expectancy} \cite{rajarshi2017life} comprising 2,938 instances of 22 distinct health-related measurements, this dataset spans the years 2000 to 2015 across 193 countries.
The expected lifespan from birth in years is used as the dependent variable, while other features act as covariates.
The closed-box model we use with this dataset is a neural network.

\vspace{0.2cm}

\noindent\textit{Vehicle} \cite{birla2022vehicles} contains 2059 instances of 12 different car-related features, with the target being the resale value of the instance.
We use a support vector regressor as the closed-box model with the Vehicles dataset.

\vspace{0.2cm}

\noindent\textit{Gas Turbine} \cite{2019gas} is a regression dataset with 36,733 instances of 9 sensor measurements on a gas turbine to study gas emissions.
With this dataset, we used a Adaboost regression as the closed-box model.

\vspace{0.2cm}

\noindent\textit{QM9} \cite{ramakrishnan2014Quantum} is a regression dataset comprising 133,766 small organic molecules. Features are created with the Mordred molecular description calculator \cite{moriwaki2018mordred}.
The QM9 closed-box model is a neural network.

\vspace{0.2cm}

\noindent\textit{HIGGS} \cite{whiteson2014higgs} is a two-class classification dataset consisting of signal processes that produce Higgs bosons or are background.
The dataset contains nearly $100000$ instances with 28 features.
We used a gradient boosting classifier with this dataset as the closed-box model.

\vspace{0.2cm}

\noindent\textit{Jets} \cite{CMS:opendata} contains simulated LHC proton-proton collisions.
The collisions produce quarks and gluons that decay into cascades of stable particles called jets.
The classification task is to distinguish between jets generated by quarks and gluons. 
The dataset has $266421$ instances with 7 features.
The closed-box model we used with this dataset is a random forest classifier.

\clearpage  %

\section{Fidelity of the proxy sets as a function of proxy number}\label{a:fidelity_k_full}
Figure \ref{fig:fidelity_k_full} shows the complete fidelity plots as a function of $k$. Each row corresponds to a different dataset, and each column represents a local model generation method. A general trend emerges, indicating improved fidelity with increasing $k$.

\begin{figure}[h]
    \centering
    \includegraphics[width=\linewidth]{figures/fidelity_k_full.pdf}
    \caption{Fidelity of the reduction algorithms as a function of the proxy set size $k$. The black dashed line indicates the performance of the full set of explanations.}
    \label{fig:fidelity_k_full}
\end{figure}
\FloatBarrier  %
\clearpage  %

\section{Fidelity of the proxy sets as a function of subsample size}\label{a:fidelity_n_full}
Figure \ref{fig:fidelity_n_full} shows the complete fidelity plots as a function of subsample size. Each row corresponds to a different dataset, and each column represents a local model generation method. A general trend emerges, indicating improved fidelity with increasing subsample size.
\begin{figure}[h]
    \centering
    \includegraphics[width=\linewidth]{figures/fidelity_n_full.pdf}
    \caption{Fidelity of the reduction algorithms as a function of subsample size. The black dashed line indicates the performance of the full set of local explanations.}
    \label{fig:fidelity_n_full}
\end{figure}

\FloatBarrier  %
\clearpage  %

\section{Coverage of the proxy sets as a function of proxy number}\label{a:coverage_full}
Figure \ref{fig:coverage_full} shows the complete results of the coverages of the proxy sets. Each row corresponds to a different dataset, and each column represents a local model generation method. A general trend emerges, indicating broader coverage with increasing $k$.
\begin{figure}[h]
    \centering
    \includegraphics[width=\linewidth]{figures/coverage_full.pdf}
    \caption{Coverage of the reduction algorithms as a function of proxy set size $k$.}
    \label{fig:coverage_full}
\end{figure}

\FloatBarrier  %
\clearpage  %

\section{Stability of the proxy sets as a function of proxy number}\label{a:stability_full}
Figure \ref{fig:stability_full} shows the complete results of the stabilities of the proxy sets. Each row corresponds to a different dataset, and each column represents a local model generation method. A general trend emerges, indicating more stable performance with increasing $k$.
\begin{figure}[h]
    \centering
    \includegraphics[width=\linewidth]{figures/stability_full.pdf}
    \caption{Instability of the reduction algorithms as a function of the proxy set size $k$.}
    \label{fig:stability_full}
\end{figure}

\FloatBarrier  %
\clearpage  %

\section{Full sensitivity analysis}\label{a:cov_eps_sensitivity}
 
Many of the reduction methods outlined in this paper require the user to select an error threshold $\varepsilon$, with some additionally requiring a minimum coverage constraint $c$.
Since $\varepsilon$ is a parameter in computing coverage, its value can significantly impact the performance of the {\sc ExplainReduce} procedure.
In this section, we study the sensitivity of the procedure to these hyperparameters.

As the error tolerance $\varepsilon$ is defined in comparison to the absolute loss, the exact value of the parameter is task- and data-dependent.
In this section, we set the error tolerance as a quantile of the training loss of the underlying closed-box function $f$, i.e., $\varepsilon = q(\bm{L}_{BB}, p)$, where $q$ denotes the percentile function, $\bm{L}_{BB, i} = \ell(f(\bm{x}_i), \bm{y}_i)$ and $p$ is the percentile value.

As Fig. \ref{fig:coverage_p_full} shows, the reduction methods that require these parameters produce results with nearly identical fidelity across a wide range of both $c$ and $\varepsilon$.
The test fidelity of the proxy set for all tested reduction methods exhibits low variance when the error tolerance is set anywhere between the 10th and 50th percentiles of the closed-box training loss values.
Additionally, the loss-minimising reduction method with soft coverage produces similar fidelity results across the entire tested range of minimum coverage values, though it often fails to meet the coverage constraint at higher values.
It can be concluded that the reduction methods are not sensitive with respect to these hyperparameters.
For simplicity, we suggest using $c = 0.8$ and $\varepsilon = q(\bm{L}_{BB}, 0.2)$ as default arguments.
When not stated otherwise, these are also the $c$ and $\varepsilon$ values used in other experiments in this paper.

\begin{figure}[ht]
    \centering
    \includegraphics[width=\linewidth]{figures/coverage_p_sensitivity_full.pdf}
    \caption{Sensitivity of the reduction methods as a function of error tolerance and minimum coverage in case of the {\sc const min loss} algorithm.}
    \label{fig:coverage_p_full}
\end{figure}

\end{document}

\section{Method}\label{sec:method}
\begin{figure}
    \centering
    \includegraphics[width=0.85\textwidth]{imgs/heatmap_acc.pdf}
    \caption{\textbf{Visualization of the proposed periodic Bayesian flow with mean parameter $\mu$ and accumulated accuracy parameter $c$ which corresponds to the entropy/uncertainty}. For $x = 0.3, \beta(1) = 1000$ and $\alpha_i$ defined in \cref{appd:bfn_cir}, this figure plots three colored stochastic parameter trajectories for receiver mean parameter $m$ and accumulated accuracy parameter $c$, superimposed on a log-scale heatmap of the Bayesian flow distribution $p_F(m|x,\senderacc)$ and $p_F(c|x,\senderacc)$. Note the \emph{non-monotonicity} and \emph{non-additive} property of $c$ which could inform the network the entropy of the mean parameter $m$ as a condition and the \emph{periodicity} of $m$. %\jj{Shrink the figures to save space}\hanlin{Do we need to make this figure one-column?}
    }
    \label{fig:vmbf_vis}
    \vskip -0.1in
\end{figure}
% \begin{wrapfigure}{r}{0.5\textwidth}
%     \centering
%     \includegraphics[width=0.49\textwidth]{imgs/heatmap_acc.pdf}
%     \caption{\textbf{Visualization of hyper-torus Bayesian flow based on von Mises Distribution}. For $x = 0.3, \beta(1) = 1000$ and $\alpha_i$ defined in \cref{appd:bfn_cir}, this figure plots three colored stochastic parameter trajectories for receiver mean parameter $m$ and accumulated accuracy parameter $c$, superimposed on a log-scale heatmap of the Bayesian flow distribution $p_F(m|x,\senderacc)$ and $p_F(c|x,\senderacc)$. Note the \emph{non-monotonicity} and \emph{non-additive} property of $c$. \jj{Shrink the figures to save space}}
%     \label{fig:vmbf_vis}
%     \vspace{-30pt}
% \end{wrapfigure}


In this section, we explain the detailed design of CrysBFN tackling theoretical and practical challenges. First, we describe how to derive our new formulation of Bayesian Flow Networks over hyper-torus $\mathbb{T}^{D}$ from scratch. Next, we illustrate the two key differences between \modelname and the original form of BFN: $1)$ a meticulously designed novel base distribution with different Bayesian update rules; and $2)$ different properties over the accuracy scheduling resulted from the periodicity and the new Bayesian update rules. Then, we present in detail the overall framework of \modelname over each manifold of the crystal space (\textit{i.e.} fractional coordinates, lattice vectors, atom types) respecting \textit{periodic E(3) invariance}. 

% In this section, we first demonstrate how to build Bayesian flow on hyper-torus $\mathbb{T}^{D}$ by overcoming theoretical and practical problems to provide a low-noise parameter-space approach to fractional atom coordinate generation. Next, we present how \modelname models each manifold of crystal space respecting \textit{periodic E(3) invariance}. 

\subsection{Periodic Bayesian Flow on Hyper-torus \texorpdfstring{$\mathbb{T}^{D}$}{}} 
For generative modeling of fractional coordinates in crystal, we first construct a periodic Bayesian flow on \texorpdfstring{$\mathbb{T}^{D}$}{} by designing every component of the totally new Bayesian update process which we demonstrate to be distinct from the original Bayesian flow (please see \cref{fig:non_add}). 
 %:) 
 
 The fractional atom coordinate system \citep{jiao2023crystal} inherently distributes over a hyper-torus support $\mathbb{T}^{3\times N}$. Hence, the normal distribution support on $\R$ used in the original \citep{bfn} is not suitable for this scenario. 
% The key problem of generative modeling for crystal is the periodicity of Cartesian atom coordinates $\vX$ requiring:
% \begin{equation}\label{eq:periodcity}
% p(\vA,\vL,\vX)=p(\vA,\vL,\vX+\vec{LK}),\text{where}~\vec{K}=\vec{k}\vec{1}_{1\times N},\forall\vec{k}\in\mathbb{Z}^{3\times1}
% \end{equation}
% However, there does not exist such a distribution supporting on $\R$ to model such property because the integration of such distribution over $\R$ will not be finite and equal to 1. Therefore, the normal distribution used in \citet{bfn} can not meet this condition.

To tackle this problem, the circular distribution~\citep{mardia2009directional} over the finite interval $[-\pi,\pi)$ is a natural choice as the base distribution for deriving the BFN on $\mathbb{T}^D$. 
% one natural choice is to 
% we would like to consider the circular distribution over the finite interval as the base 
% we find that circular distributions \citep{mardia2009directional} defined on a finite interval with lengths of $2\pi$ can be used as the instantiation of input distribution for the BFN on $\mathbb{T}^D$.
Specifically, circular distributions enjoy desirable periodic properties: $1)$ the integration over any interval length of $2\pi$ equals 1; $2)$ the probability distribution function is periodic with period $2\pi$.  Sharing the same intrinsic with fractional coordinates, such periodic property of circular distribution makes it suitable for the instantiation of BFN's input distribution, in parameterizing the belief towards ground truth $\x$ on $\mathbb{T}^D$. 
% \yuxuan{this is very complicated from my perspective.} \hanlin{But this property is exactly beautiful and perfectly fit into the BFN.}

\textbf{von Mises Distribution and its Bayesian Update} We choose von Mises distribution \citep{mardia2009directional} from various circular distributions as the form of input distribution, based on the appealing conjugacy property required in the derivation of the BFN framework.
% to leverage the Bayesian conjugacy property of von Mises distribution which is required by the BFN framework. 
That is, the posterior of a von Mises distribution parameterized likelihood is still in the family of von Mises distributions. The probability density function of von Mises distribution with mean direction parameter $m$ and concentration parameter $c$ (describing the entropy/uncertainty of $m$) is defined as: 
\begin{equation}
f(x|m,c)=vM(x|m,c)=\frac{\exp(c\cos(x-m))}{2\pi I_0(c)}
\end{equation}
where $I_0(c)$ is zeroth order modified Bessel function of the first kind as the normalizing constant. Given the last univariate belief parameterized by von Mises distribution with parameter $\theta_{i-1}=\{m_{i-1},\ c_{i-1}\}$ and the sample $y$ from sender distribution with unknown data sample $x$ and known accuracy $\alpha$ describing the entropy/uncertainty of $y$,  Bayesian update for the receiver is deducted as:
\begin{equation}
 h(\{m_{i-1},c_{i-1}\},y,\alpha)=\{m_i,c_i \}, \text{where}
\end{equation}
\begin{equation}\label{eq:h_m}
m_i=\text{atan2}(\alpha\sin y+c_{i-1}\sin m_{i-1}, {\alpha\cos y+c_{i-1}\cos m_{i-1}})
\end{equation}
\begin{equation}\label{eq:h_c}
c_i =\sqrt{\alpha^2+c_{i-1}^2+2\alpha c_{i-1}\cos(y-m_{i-1})}
\end{equation}
The proof of the above equations can be found in \cref{apdx:bayesian_update_function}. The atan2 function refers to  2-argument arctangent. Independently conducting  Bayesian update for each dimension, we can obtain the Bayesian update distribution by marginalizing $\y$:
\begin{equation}
p_U(\vtheta'|\vtheta,\bold{x};\alpha)=\mathbb{E}_{p_S(\bold{y}|\bold{x};\alpha)}\delta(\vtheta'-h(\vtheta,\bold{y},\alpha))=\mathbb{E}_{vM(\bold{y}|\bold{x},\alpha)}\delta(\vtheta'-h(\vtheta,\bold{y},\alpha))
\end{equation} 
\begin{figure}
    \centering
    \vskip -0.15in
    \includegraphics[width=0.95\linewidth]{imgs/non_add.pdf}
    \caption{An intuitive illustration of non-additive accuracy Bayesian update on the torus. The lengths of arrows represent the uncertainty/entropy of the belief (\emph{e.g.}~$1/\sigma^2$ for Gaussian and $c$ for von Mises). The directions of the arrows represent the believed location (\emph{e.g.}~ $\mu$ for Gaussian and $m$ for von Mises).}
    \label{fig:non_add}
    \vskip -0.15in
\end{figure}
\textbf{Non-additive Accuracy} 
The additive accuracy is a nice property held with the Gaussian-formed sender distribution of the original BFN expressed as:
\begin{align}
\label{eq:standard_id}
    \update(\parsn{}'' \mid \parsn{}, \x; \alpha_a+\alpha_b) = \E_{\update(\parsn{}' \mid \parsn{}, \x; \alpha_a)} \update(\parsn{}'' \mid \parsn{}', \x; \alpha_b)
\end{align}
Such property is mainly derived based on the standard identity of Gaussian variable:
\begin{equation}
X \sim \mathcal{N}\left(\mu_X, \sigma_X^2\right), Y \sim \mathcal{N}\left(\mu_Y, \sigma_Y^2\right) \Longrightarrow X+Y \sim \mathcal{N}\left(\mu_X+\mu_Y, \sigma_X^2+\sigma_Y^2\right)
\end{equation}
The additive accuracy property makes it feasible to derive the Bayesian flow distribution $
p_F(\boldsymbol{\theta} \mid \mathbf{x} ; i)=p_U\left(\boldsymbol{\theta} \mid \boldsymbol{\theta}_0, \mathbf{x}, \sum_{k=1}^{i} \alpha_i \right)
$ for the simulation-free training of \cref{eq:loss_n}.
It should be noted that the standard identity in \cref{eq:standard_id} does not hold in the von Mises distribution. Hence there exists an important difference between the original Bayesian flow defined on Euclidean space and the Bayesian flow of circular data on $\mathbb{T}^D$ based on von Mises distribution. With prior $\btheta = \{\bold{0},\bold{0}\}$, we could formally represent the non-additive accuracy issue as:
% The additive accuracy property implies the fact that the "confidence" for the data sample after observing a series of the noisy samples with accuracy ${\alpha_1, \cdots, \alpha_i}$ could be  as the accuracy sum  which could be  
% Here we 
% Here we emphasize the specific property of BFN based on von Mises distribution.
% Note that 
% \begin{equation}
% \update(\parsn'' \mid \parsn, \x; \alpha_a+\alpha_b) \ne \E_{\update(\parsn' \mid \parsn, \x; \alpha_a)} \update(\parsn'' \mid \parsn', \x; \alpha_b)
% \end{equation}
% \oyyw{please check whether the below equation is better}
% \yuxuan{I fill somehow confusing on what is the update distribution with $\alpha$. }
% \begin{equation}
% \update(\parsn{}'' \mid \parsn{}, \x; \alpha_a+\alpha_b) \ne \E_{\update(\parsn{}' \mid \parsn{}, \x; \alpha_a)} \update(\parsn{}'' \mid \parsn{}', \x; \alpha_b)
% \end{equation}
% We give an intuitive visualization of such difference in \cref{fig:non_add}. The untenability of this property can materialize by considering the following case: with prior $\btheta = \{\bold{0},\bold{0}\}$, check the two-step Bayesian update distribution with $\alpha_a,\alpha_b$ and one-step Bayesian update with $\alpha=\alpha_a+\alpha_b$:
\begin{align}
\label{eq:nonadd}
     &\update(c'' \mid \parsn, \x; \alpha_a+\alpha_b)  = \delta(c-\alpha_a-\alpha_b)
     \ne  \mathbb{E}_{p_U(\parsn' \mid \parsn, \x; \alpha_a)}\update(c'' \mid \parsn', \x; \alpha_b) \nonumber \\&= \mathbb{E}_{vM(\bold{y}_b|\bold{x},\alpha_a)}\mathbb{E}_{vM(\bold{y}_a|\bold{x},\alpha_b)}\delta(c-||[\alpha_a \cos\y_a+\alpha_b\cos \y_b,\alpha_a \sin\y_a+\alpha_b\sin \y_b]^T||_2)
\end{align}
A more intuitive visualization could be found in \cref{fig:non_add}. This fundamental difference between periodic Bayesian flow and that of \citet{bfn} presents both theoretical and practical challenges, which we will explain and address in the following contents.

% This makes constructing Bayesian flow based on von Mises distribution intrinsically different from previous Bayesian flows (\citet{bfn}).

% Thus, we must reformulate the framework of Bayesian flow networks  accordingly. % and do necessary reformulations of BFN. 

% \yuxuan{overall I feel this part is complicated by using the language of update distribution. I would like to suggest simply use bayesian update, to provide intuitive explantion.}\hanlin{See the illustration in \cref{fig:non_add}}

% That introduces a cascade of problems, and we investigate the following issues: $(1)$ Accuracies between sender and receiver are not synchronized and need to be differentiated. $(2)$ There is no tractable Bayesian flow distribution for a one-step sample conditioned on a given time step $i$, and naively simulating the Bayesian flow results in computational overhead. $(3)$ It is difficult to control the entropy of the Bayesian flow. $(4)$ Accuracy is no longer a function of $t$ and becomes a distribution conditioned on $t$, which can be different across dimensions.
%\jj{Edited till here}

\textbf{Entropy Conditioning} As a common practice in generative models~\citep{ddpm,flowmatching,bfn}, timestep $t$ is widely used to distinguish among generation states by feeding the timestep information into the networks. However, this paper shows that for periodic Bayesian flow, the accumulated accuracy $\vc_i$ is more effective than time-based conditioning by informing the network about the entropy and certainty of the states $\parsnt{i}$. This stems from the intrinsic non-additive accuracy which makes the receiver's accumulated accuracy $c$ not bijective function of $t$, but a distribution conditioned on accumulated accuracies $\vc_i$ instead. Therefore, the entropy parameter $\vc$ is taken logarithm and fed into the network to describe the entropy of the input corrupted structure. We verify this consideration in \cref{sec:exp_ablation}. 
% \yuxuan{implement variant. traditionally, the timestep is widely used to distinguish the different states by putting the timestep embedding into the networks. citation of FM, diffusion, BFN. However, we find that conditioned on time in periodic flow could not provide extra benefits. To further boost the performance, we introduce a simple yet effective modification term entropy conditional. This is based on that the accumulated accuracy which represents the current uncertainty or entropy could be a better indicator to distinguish different states. + Describe how you do this. }



\textbf{Reformulations of BFN}. Recall the original update function with Gaussian sender distribution, after receiving noisy samples $\y_1,\y_2,\dots,\y_i$ with accuracies $\senderacc$, the accumulated accuracies of the receiver side could be analytically obtained by the additive property and it is consistent with the sender side.
% Since observing sample $\y$ with $\alpha_i$ can not result in exact accuracy increment $\alpha_i$ for receiver, the accuracies between sender and receiver are not synchronized which need to be differentiated. 
However, as previously mentioned, this does not apply to periodic Bayesian flow, and some of the notations in original BFN~\citep{bfn} need to be adjusted accordingly. We maintain the notations of sender side's one-step accuracy $\alpha$ and added accuracy $\beta$, and alter the notation of receiver's accuracy parameter as $c$, which is needed to be simulated by cascade of Bayesian updates. We emphasize that the receiver's accumulated accuracy $c$ is no longer a function of $t$ (differently from the Gaussian case), and it becomes a distribution conditioned on received accuracies $\senderacc$ from the sender. Therefore, we represent the Bayesian flow distribution of von Mises distribution as $p_F(\btheta|\x;\alpha_1,\alpha_2,\dots,\alpha_i)$. And the original simulation-free training with Bayesian flow distribution is no longer applicable in this scenario.
% Different from previous BFNs where the accumulated accuracy $\rho$ is not explicitly modeled, the accumulated accuracy parameter $c$ (visualized in \cref{fig:vmbf_vis}) needs to be explicitly modeled by feeding it to the network to avoid information loss.
% the randomaccuracy parameter $c$ (visualized in \cref{fig:vmbf_vis}) implies that there exists information in $c$ from the sender just like $m$, meaning that $c$ also should be fed into the network to avoid information loss. 
% We ablate this consideration in  \cref{sec:exp_ablation}. 

\textbf{Fast Sampling from Equivalent Bayesian Flow Distribution} Based on the above reformulations, the Bayesian flow distribution of von Mises distribution is reframed as: 
\begin{equation}\label{eq:flow_frac}
p_F(\btheta_i|\x;\alpha_1,\alpha_2,\dots,\alpha_i)=\E_{\update(\parsnt{1} \mid \parsnt{0}, \x ; \alphat{1})}\dots\E_{\update(\parsn_{i-1} \mid \parsnt{i-2}, \x; \alphat{i-1})} \update(\parsnt{i} | \parsnt{i-1},\x;\alphat{i} )
\end{equation}
Naively sampling from \cref{eq:flow_frac} requires slow auto-regressive iterated simulation, making training unaffordable. Noticing the mathematical properties of \cref{eq:h_m,eq:h_c}, we  transform \cref{eq:flow_frac} to the equivalent form:
\begin{equation}\label{eq:cirflow_equiv}
p_F(\vec{m}_i|\x;\alpha_1,\alpha_2,\dots,\alpha_i)=\E_{vM(\y_1|\x,\alpha_1)\dots vM(\y_i|\x,\alpha_i)} \delta(\vec{m}_i-\text{atan2}(\sum_{j=1}^i \alpha_j \cos \y_j,\sum_{j=1}^i \alpha_j \sin \y_j))
\end{equation}
\begin{equation}\label{eq:cirflow_equiv2}
p_F(\vec{c}_i|\x;\alpha_1,\alpha_2,\dots,\alpha_i)=\E_{vM(\y_1|\x,\alpha_1)\dots vM(\y_i|\x,\alpha_i)}  \delta(\vec{c}_i-||[\sum_{j=1}^i \alpha_j \cos \y_j,\sum_{j=1}^i \alpha_j \sin \y_j]^T||_2)
\end{equation}
which bypasses the computation of intermediate variables and allows pure tensor operations, with negligible computational overhead.
\begin{restatable}{proposition}{cirflowequiv}
The probability density function of Bayesian flow distribution defined by \cref{eq:cirflow_equiv,eq:cirflow_equiv2} is equivalent to the original definition in \cref{eq:flow_frac}. 
\end{restatable}
\textbf{Numerical Determination of Linear Entropy Sender Accuracy Schedule} ~Original BFN designs the accuracy schedule $\beta(t)$ to make the entropy of input distribution linearly decrease. As for crystal generation task, to ensure information coherence between modalities, we choose a sender accuracy schedule $\senderacc$ that makes the receiver's belief entropy $H(t_i)=H(p_I(\cdot|\vtheta_i))=H(p_I(\cdot|\vc_i))$ linearly decrease \emph{w.r.t.} time $t_i$, given the initial and final accuracy parameter $c(0)$ and $c(1)$. Due to the intractability of \cref{eq:vm_entropy}, we first use numerical binary search in $[0,c(1)]$ to determine the receiver's $c(t_i)$ for $i=1,\dots, n$ by solving the equation $H(c(t_i))=(1-t_i)H(c(0))+tH(c(1))$. Next, with $c(t_i)$, we conduct numerical binary search for each $\alpha_i$ in $[0,c(1)]$ by solving the equations $\E_{y\sim vM(x,\alpha_i)}[\sqrt{\alpha_i^2+c_{i-1}^2+2\alpha_i c_{i-1}\cos(y-m_{i-1})}]=c(t_i)$ from $i=1$ to $i=n$ for arbitrarily selected $x\in[-\pi,\pi)$.

After tackling all those issues, we have now arrived at a new BFN architecture for effectively modeling crystals. Such BFN can also be adapted to other type of data located in hyper-torus $\mathbb{T}^{D}$.

\subsection{Equivariant Bayesian Flow for Crystal}
With the above Bayesian flow designed for generative modeling of fractional coordinate $\vF$, we are able to build equivariant Bayesian flow for each modality of crystal. In this section, we first give an overview of the general training and sampling algorithm of \modelname (visualized in \cref{fig:framework}). Then, we describe the details of the Bayesian flow of every modality. The training and sampling algorithm can be found in \cref{alg:train} and \cref{alg:sampling}.

\textbf{Overview} Operating in the parameter space $\bthetaM=\{\bthetaA,\bthetaL,\bthetaF\}$, \modelname generates high-fidelity crystals through a joint BFN sampling process on the parameter of  atom type $\bthetaA$, lattice parameter $\vec{\theta}^L=\{\bmuL,\brhoL\}$, and the parameter of fractional coordinate matrix $\bthetaF=\{\bmF,\bcF\}$. We index the $n$-steps of the generation process in a discrete manner $i$, and denote the corresponding continuous notation $t_i=i/n$ from prior parameter $\thetaM_0$ to a considerably low variance parameter $\thetaM_n$ (\emph{i.e.} large $\vrho^L,\bmF$, and centered $\bthetaA$).

At training time, \modelname samples time $i\sim U\{1,n\}$ and $\bthetaM_{i-1}$ from the Bayesian flow distribution of each modality, serving as the input to the network. The network $\net$ outputs $\net(\parsnt{i-1}^\mathcal{M},t_{i-1})=\net(\parsnt{i-1}^A,\parsnt{i-1}^F,\parsnt{i-1}^L,t_{i-1})$ and conducts gradient descents on loss function \cref{eq:loss_n} for each modality. After proper training, the sender distribution $p_S$ can be approximated by the receiver distribution $p_R$. 

At inference time, from predefined $\thetaM_0$, we conduct transitions from $\thetaM_{i-1}$ to $\thetaM_{i}$ by: $(1)$ sampling $\y_i\sim p_R(\bold{y}|\thetaM_{i-1};t_i,\alpha_i)$ according to network prediction $\predM{i-1}$; and $(2)$ performing Bayesian update $h(\thetaM_{i-1},\y^\calM_{i-1},\alpha_i)$ for each dimension. 

% Alternatively, we complete this transition using the flow-back technique by sampling 
% $\thetaM_{i}$ from Bayesian flow distribution $\flow(\btheta^M_{i}|\predM{i-1};t_{i-1})$. 

% The training objective of $\net$ is to minimize the KL divergence between sender distribution and receiver distribution for every modality as defined in \cref{eq:loss_n} which is equivalent to optimizing the negative variational lower bound $\calL^{VLB}$ as discussed in \cref{sec:preliminaries}. 

%In the following part, we will present the Bayesian flow of each modality in detail.

\textbf{Bayesian Flow of Fractional Coordinate $\vF$}~The distribution of the prior parameter $\bthetaF_0$ is defined as:
\begin{equation}\label{eq:prior_frac}
    p(\bthetaF_0) \defeq \{vM(\vm_0^F|\vec{0}_{3\times N},\vec{0}_{3\times N}),\delta(\vc_0^F-\vec{0}_{3\times N})\} = \{U(\vec{0},\vec{1}),\delta(\vc_0^F-\vec{0}_{3\times N})\}
\end{equation}
Note that this prior distribution of $\vm_0^F$ is uniform over $[\vec{0},\vec{1})$, ensuring the periodic translation invariance property in \cref{De:pi}. The training objective is minimizing the KL divergence between sender and receiver distribution (deduction can be found in \cref{appd:cir_loss}): 
%\oyyw{replace $\vF$ with $\x$?} \hanlin{notations follow Preliminary?}
\begin{align}\label{loss_frac}
\calL_F = n \E_{i \sim \ui{n}, \flow(\parsn{}^F \mid \vF ; \senderacc)} \alpha_i\frac{I_1(\alpha_i)}{I_0(\alpha_i)}(1-\cos(\vF-\predF{i-1}))
\end{align}
where $I_0(x)$ and $I_1(x)$ are the zeroth and the first order of modified Bessel functions. The transition from $\bthetaF_{i-1}$ to $\bthetaF_{i}$ is the Bayesian update distribution based on network prediction:
\begin{equation}\label{eq:transi_frac}
    p(\btheta^F_{i}|\parsnt{i-1}^\calM)=\mathbb{E}_{vM(\bold{y}|\predF{i-1},\alpha_i)}\delta(\btheta^F_{i}-h(\btheta^F_{i-1},\bold{y},\alpha_i))
\end{equation}
\begin{restatable}{proposition}{fracinv}
With $\net_{F}$ as a periodic translation equivariant function namely $\net_F(\parsnt{}^A,w(\parsnt{}^F+\vt),\parsnt{}^L,t)=w(\net_F(\parsnt{}^A,\parsnt{}^F,\parsnt{}^L,t)+\vt), \forall\vt\in\R^3$, the marginal distribution of $p(\vF_n)$ defined by \cref{eq:prior_frac,eq:transi_frac} is periodic translation invariant. 
\end{restatable}
\textbf{Bayesian Flow of Lattice Parameter \texorpdfstring{$\boldsymbol{L}$}{}}   
Noting the lattice parameter $\bm{L}$ located in Euclidean space, we set prior as the parameter of a isotropic multivariate normal distribution $\btheta^L_0\defeq\{\vmu_0^L,\vrho_0^L\}=\{\bm{0}_{3\times3},\bm{1}_{3\times3}\}$
% \begin{equation}\label{eq:lattice_prior}
% \btheta^L_0\defeq\{\vmu_0^L,\vrho_0^L\}=\{\bm{0}_{3\times3},\bm{1}_{3\times3}\}
% \end{equation}
such that the prior distribution of the Markov process on $\vmu^L$ is the Dirac distribution $\delta(\vec{\mu_0}-\vec{0})$ and $\delta(\vec{\rho_0}-\vec{1})$, 
% \begin{equation}
%     p_I^L(\boldsymbol{L}|\btheta_0^L)=\mathcal{N}(\bm{L}|\bm{0},\bm{I})
% \end{equation}
which ensures O(3)-invariance of prior distribution of $\vL$. By Eq. 77 from \citet{bfn}, the Bayesian flow distribution of the lattice parameter $\bm{L}$ is: 
\begin{align}% =p_U(\bmuL|\btheta_0^L,\bm{L},\beta(t))
p_F^L(\bmuL|\bm{L};t) &=\mathcal{N}(\bmuL|\gamma(t)\bm{L},\gamma(t)(1-\gamma(t))\bm{I}) 
\end{align}
where $\gamma(t) = 1 - \sigma_1^{2t}$ and $\sigma_1$ is the predefined hyper-parameter controlling the variance of input distribution at $t=1$ under linear entropy accuracy schedule. The variance parameter $\vrho$ does not need to be modeled and fed to the network, since it is deterministic given the accuracy schedule. After sampling $\bmuL_i$ from $p_F^L$, the training objective is defined as minimizing KL divergence between sender and receiver distribution (based on Eq. 96 in \citet{bfn}):
\begin{align}
\mathcal{L}_{L} = \frac{n}{2}\left(1-\sigma_1^{2/n}\right)\E_{i \sim \ui{n}}\E_{\flow(\bmuL_{i-1} |\vL ; t_{i-1})}  \frac{\left\|\vL -\predL{i-1}\right\|^2}{\sigma_1^{2i/n}},\label{eq:lattice_loss}
\end{align}
where the prediction term $\predL{i-1}$ is the lattice parameter part of network output. After training, the generation process is defined as the Bayesian update distribution given network prediction:
\begin{equation}\label{eq:lattice_sampling}
    p(\bmuL_{i}|\parsnt{i-1}^\calM)=\update^L(\bmuL_{i}|\predL{i-1},\bmuL_{i-1};t_{i-1})
\end{equation}
    

% The final prediction of the lattice parameter is given by $\bmuL_n = \predL{n-1}$.
% \begin{equation}\label{eq:final_lattice}
%     \bmuL_n = \predL{n-1}
% \end{equation}

\begin{restatable}{proposition}{latticeinv}\label{prop:latticeinv}
With $\net_{L}$ as  O(3)-equivariant function namely $\net_L(\parsnt{}^A,\parsnt{}^F,\vQ\parsnt{}^L,t)=\vQ\net_L(\parsnt{}^A,\parsnt{}^F,\parsnt{}^L,t),\forall\vQ^T\vQ=\vI$, the marginal distribution of $p(\bmuL_n)$ defined by \cref{eq:lattice_sampling} is O(3)-invariant. 
\end{restatable}


\textbf{Bayesian Flow of Atom Types \texorpdfstring{$\boldsymbol{A}$}{}} 
Given that atom types are discrete random variables located in a simplex $\calS^K$, the prior parameter of $\boldsymbol{A}$ is the discrete uniform distribution over the vocabulary $\parsnt{0}^A \defeq \frac{1}{K}\vec{1}_{1\times N}$. 
% \begin{align}\label{eq:disc_input_prior}
% \parsnt{0}^A \defeq \frac{1}{K}\vec{1}_{1\times N}
% \end{align}
% \begin{align}
%     (\oh{j}{K})_k \defeq \delta_{j k}, \text{where }\oh{j}{K}\in \R^{K},\oh{\vA}{KD} \defeq \left(\oh{a_1}{K},\dots,\oh{a_N}{K}\right) \in \R^{K\times N}
% \end{align}
With the notation of the projection from the class index $j$ to the length $K$ one-hot vector $ (\oh{j}{K})_k \defeq \delta_{j k}, \text{where }\oh{j}{K}\in \R^{K},\oh{\vA}{KD} \defeq \left(\oh{a_1}{K},\dots,\oh{a_N}{K}\right) \in \R^{K\times N}$, the Bayesian flow distribution of atom types $\vA$ is derived in \citet{bfn}:
\begin{align}
\flow^{A}(\parsn^A \mid \vA; t) &= \E_{\N{\y \mid \beta^A(t)\left(K \oh{\vA}{K\times N} - \vec{1}_{K\times N}\right)}{\beta^A(t) K \vec{I}_{K\times N \times N}}} \delta\left(\parsn^A - \frac{e^{\y}\parsnt{0}^A}{\sum_{k=1}^K e^{\y_k}(\parsnt{0})_{k}^A}\right).
\end{align}
where $\beta^A(t)$ is the predefined accuracy schedule for atom types. Sampling $\btheta_i^A$ from $p_F^A$ as the training signal, the training objective is the $n$-step discrete-time loss for discrete variable \citep{bfn}: 
% \oyyw{can we simplify the next equation? Such as remove $K \times N, K \times N \times N$}
% \begin{align}
% &\calL_A = n\E_{i \sim U\{1,n\},\flow^A(\parsn^A \mid \vA ; t_{i-1}),\N{\y \mid \alphat{i}\left(K \oh{\vA}{KD} - \vec{1}_{K\times N}\right)}{\alphat{i} K \vec{I}_{K\times N \times N}}} \ln \N{\y \mid \alphat{i}\left(K \oh{\vA}{K\times N} - \vec{1}_{K\times N}\right)}{\alphat{i} K \vec{I}_{K\times N \times N}}\nonumber\\
% &\qquad\qquad\qquad-\sum_{d=1}^N \ln \left(\sum_{k=1}^K \out^{(d)}(k \mid \parsn^A; t_{i-1}) \N{\ydd{d} \mid \alphat{i}\left(K\oh{k}{K}- \vec{1}_{K\times N}\right)}{\alphat{i} K \vec{I}_{K\times N \times N}}\right)\label{discdisc_t_loss_exp}
% \end{align}
\begin{align}
&\calL_A = n\E_{i \sim U\{1,n\},\flow^A(\parsn^A \mid \vA ; t_{i-1}),\N{\y \mid \alphat{i}\left(K \oh{\vA}{KD} - \vec{1}\right)}{\alphat{i} K \vec{I}}} \ln \N{\y \mid \alphat{i}\left(K \oh{\vA}{K\times N} - \vec{1}\right)}{\alphat{i} K \vec{I}}\nonumber\\
&\qquad\qquad\qquad-\sum_{d=1}^N \ln \left(\sum_{k=1}^K \out^{(d)}(k \mid \parsn^A; t_{i-1}) \N{\ydd{d} \mid \alphat{i}\left(K\oh{k}{K}- \vec{1}\right)}{\alphat{i} K \vec{I}}\right)\label{discdisc_t_loss_exp}
\end{align}
where $\vec{I}\in \R^{K\times N \times N}$ and $\vec{1}\in\R^{K\times D}$. When sampling, the transition from $\bthetaA_{i-1}$ to $\bthetaA_{i}$ is derived as:
\begin{equation}
    p(\btheta^A_{i}|\parsnt{i-1}^\calM)=\update^A(\btheta^A_{i}|\btheta^A_{i-1},\predA{i-1};t_{i-1})
\end{equation}

The detailed training and sampling algorithm could be found in \cref{alg:train} and \cref{alg:sampling}.




\section{Conclusion}
In this work, we propose a simple yet effective approach, called SMILE, for graph few-shot learning with fewer tasks. Specifically, we introduce a novel dual-level mixup strategy, including within-task and across-task mixup, for enriching the diversity of nodes within each task and the diversity of tasks. Also, we incorporate the degree-based prior information to learn expressive node embeddings. Theoretically, we prove that SMILE effectively enhances the model's generalization performance. Empirically, we conduct extensive experiments on multiple benchmarks and the results suggest that SMILE significantly outperforms other baselines, including both in-domain and cross-domain few-shot settings.

\clearpage

\section*{Impact Statement}

This paper proposes a new framework on regulating advanced AI agents, who can pose existential risks.  The paper is thus intended to mitigate the negative impacts and implications that AI technologies might have.  However, the paper also discusses how those advanced AI agents might be developed.  While all the discussion is based on existing technologies that are already known in public, the unified treatment in this paper might motivate malicious readers to develop and deploy such advanced AI agents without sufficient considerations of their safety.

\bibliography{principal-agent}
\bibliographystyle{icml2025}

\newpage
\appendix
\section{Similarity between AI agents}
\label{sec:method:agent}

It would be better if we could say that an AI agent is acceptable when it is sufficiently close to an AI agent that we empirically know is strongly acceptable, regardless of how those agents are developed (i.e., independent of their configurations including the amount of computation for training).  A question is how to evaluate the similarity between two AI agents.  If the two AI agents consist of identical base FMs and identical planning modules, their difference may be simply characterized by the amount of computation for inference.  However, AI agents can consist of different base FMs or different planning modules.
%(e.g., the number of search steps performed in MCTS).

A possible approach is to measure the similarity between two AI agents, $G$ and $G'$, based on their output.  For example, using a distance $d$ between probability distributions, we may for example define the distance between $G$ and $G'$ with $\sup_x d(g(G,x),g(G',x))$, where $g(G,x)$ is the distribution of the output (e.g., document) of $G$ when it perceives $x$ (e.g., prompt).  This is similar to the motivation of reinforcement learning from human feedback \cite{ouyang2022training,bai2022training} and direct policy optimization \cite{rafailov2023direct}, where the regularization with Kullback–Leibler divergence is used to mitigate catastrophic forgetting or alignment tax \cite{kotha2024understanding,ouyang2022training,bai2022training}, which refer to the phenomena that fine-tuned models lose the skills that pre-trained models had.

Likewise, we may mitigate catastrophic forgetting of the strong acceptability of an AI agent $G'$ and retain the acceptability in a new AI agent $G$ by ensuring $\sup_x d(g(G,x),g(G',x))\le \varepsilon$.  A difficulty is that it is unclear how to evaluate the supremum, since there can be infinitely many possible perceptions $x$.  Alternatively, one may consider $\E[d(g(G,X),g(G',X))]$, where $\E$ is the expectation with respect to some distribution of the random perception $X$.  However, such a guarantee based on expectation may be unsuitable for addressing safety concerns related to existential risks, which involve events with extremely low probabilities and extremely high impacts.

\section{Expanding the strongly acceptable set}
\label{sec:method:expand}

The set of strongly acceptable solutions may be expanded gradually.  For example, we may choose an acceptable solution $z\in\bar\calZ_0$ and keep using $z$ for a certain period of time.  If it turns out that $z$ can be considered strongly acceptable based on its behavior during that period, we may expand the set of strongly acceptable solutions by adding $z$ into $\calZ_0$.  This process of expanding the acceptable set $\calZ_0$ could also be performed jointly as a community.  

This expansion of acceptable set is similar in spirit to safe exploration in RL \cite{garcia2015comprehensive}.  Here, the safety set is gradually expand, starting from a seed set, for example based on the assumption of Lipshitz continuity and Gaussian process \cite{sui2015safe}.  In control theory, safety is often guaranteed with barrier certificates often based on some prior knowledge about the environment \cite{ames2019control,luo2021learning,bansal2017hamilton}.  Such ideas have also been exploited in safe exploration in RL with generative modeling \cite{wang2023enforcing}.

%barrier certificate or function in control theory: unsafe states are specified by users; given dynamics, can guarantee that it never goes to unsafe states; 
%prior knowledge of the environment dynamics \cite{ames2019control}
%use confidence interval of the dynamics \cite{luo2021learning}.
%Hamilton-Jacobi reachability analysis \cite{bansal2017hamilton} in control theory. 
%- set of states that can lead to an unsafe state
%neural network based controller

\subsection{Rationale for similarity-based measures}
\label{sec:method:rationale}

Here, we provide some rationale on such similarity-based measures.  Let a solution $z$ denote a configuration, an AI agent, or an action-sequence; we discuss the acceptability of $z$.

Suppose that there exists an unknown function $g_0$ such that a solution $z$ is acceptable iff $g_0(z)\le 0$.  We cannot make strong assumptions about $g_0$, since we know little about $g_0$. Since we cannot deal with $g_0$ without any assumptions, let us make a minimal assumption that the solution space $\calZ$ is equipped with a metric $d$, and that $g_0$ is 1-Lipschitz.\footnote{This does not lose generality, since $L$-Lipschitz functions under a metric $d$ can be made 1-Lipshitz by redefining $d$.}  Let $\calG$ be a class of 1-Lipschitz functions.  

We say that a solution $z_0$ is strongly acceptable if all of its $\varepsilon$-neighbors are acceptable.  Let 
$\calZ_0
\subseteq
\left\{
    z\in\calZ
    \mid 
    g(z_0)\le - \varepsilon
\right\}$
be the set of known strongly acceptable solutions.  Then we know that
%\begin{align}
    $\bar\calZ_0
    \coloneqq \left\{
    z \in\calZ
    \mid
    \min_{z_0\in \calZ_0} d(z,z_0) \le \varepsilon
    \right\}$
%\end{align}
is a set of acceptable solutions.  One would typically choose the solution $z$ that maximizes an objective function under the constraint of $z\in\bar\calZ_0$.  In this way, the selected solution is guaranteed to be acceptable under the assumptions made.
\section{Unobservable Markov decision processes}

There has been a limited amount of work on unobservable Markov decision processes (UMDPs)\footnote{UMDPs have also been studied under the name of Markov decision processes (MDPs) with no observations, non-observable MDPs (NOMDPs), and no observation MDPs (NOMDPs).}.  An UMDP is a special case of a partially observable Markov decision process (POMDP) in that the agent always makes a null observation in the UMDP.  A standard approach to finding the optimal policy for a POMDP is to recursively compute its value as a function of the belief state, which is updated on the basis of the Bayes rule.  This is substantially simplified when there are no observations.  In this section, we provide a comprehensive survey of the prior work on UMDPs.

\citet{madani1999computability,madani2003undecidability} establish the undecidability of some decision problems associated with POMDPs over infinite horizons by establishing undecidability for the special case of UMDPs.  Such undecidability for UMDPs can be established by reducing an UMDP to a probabilistic finite-state automaton.  The undecidability also holds for a restricted class of UMDPs \cite{balle2017bisimulation,balle2022bisimulation}. While approximate decision problems are still undecidable for general UMDPs over infinite horizons, \citet{chatterjee2024ergodic} study a special case of UMDPs whose approximate decision problems are decidable.

\citet{burago1996complexity} prove that computing the optimal policy for a POMDP over a finite horizon is NP-hard but showing that it is NP-hard for an UMDP.  \citet{wu2020optimal} study special cases of UMDPs over finite horizons whose optimal policies can be computed in polynomial time.

UMDPs have also been used as approximations of POMDPs \cite{hauskrecht2000value,brechtel2015dynamic,lauri2016sequential} or simply discussed as a special case of POMDPs \cite{valkanova2009algorithms}.  For a given POMDP, the corresponding UMDP can given a lower bound on the value function, while the corresponding (fully observable) MDP can give an upper bound on the value function.  This relation between UMDP and POMDP can be exploited to efficiently find approximately optimal policies for POMDPs.  Notice that an UMDP can allow more efficient optimization than the corresponding POMDP, since the UMDP does not need to deal with observations.  For exemple, \citet{king2018robust} studies an MCTS method for UMDPs.

UMDPs have also been studied as a simple special case of POMDPs to study the effectiveness of planning methods in belief states to study the relative performance of different planning methods for POMDPs \cite{littlefield2020efficient,littlefield2018importance,kimmel2019belief}.  UMDPs have also been simply discussed as a special case of POMDPs \cite{boutilier1999decision,csaji2008adaptive,verma2005graphical}.

UMDP also appears in the study of planning for multiple distributed agents to optimize a single objective under partial observability.  Specifically, planning for a decentralized POMDP (DecPOMDP) \cite{oliehoek2016concise} can be reduced to planning for a (centralized) UMDP \cite{oliehoek2014decpomdps,roijers2016multi,roijers2020multi}, where the state in the UMDP is the pair of the state and the history of observations in the DecPOMDP, and the action in the UMDP is the decision rule that maps the history of observations into the actions in the POMDP.  In DecPOMDPs, the belief (distribution) over the pair of the state and the history of observations is the sufficient statistic, and planning can be performed in the space of the belief states.

\citet{evendar2007value} consider an UMDP in the context of studying what values observations can provide in a POMDP.  They introduce a parameter that ranges from 0 to 1.  When the parameter is 0, the observation provides no information about the state (hence, the POMDP reduces to an UMDP); when the parameter is 1, the observation provides full information about the state (hence, the POMDP reduces to an MDP).  The prior work also extends UMDPs to allow often costly actions that enable partial or full observations of the state \cite{fox2007reinforcement,kamar2009modeling,kamar2010reasoning,wang2025ocmdpob}.

% NOMDP is mentioned as a model used in classical method https://publications.polymtl.ca/5550/1/2020_Yves_Alain_Mbeutcha.pdf


\end{document}
