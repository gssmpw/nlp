%!TEX root = main.tex
\section{Introduction}\label{sec:introduction}
% 单查询问题和应用
Finding $k$ disjoint paths ($k$DP) is a fundamental challenge in graph analysis~\cite{baseline_moreverbose,baselineOnlySplitP1,baseline1step2, Penalty,dissimilarity_topk1,SCB}. 
Given a source and a target vertex, $k$DP identifies $k$ paths that do not share any vertices except for the source and target\footnote{We focus on vertex-disjoint paths, as edge-disjoint path finding problem can be reduced to vertex-disjoint version in polynomial time~\cite{edgedisjoint2vertexdisjoint}.}. 
Fig.~\ref{fig:g} illustrates an example with disjoint paths $p_1$ (red) and $p_2$ (blue).
Applications of $k$DP include cyber-security, network fault tolerance, and load balancing~\cite{2009BRTree,cyber_secure, Penalty,dissimilarity_topk1,SCB}.

% 多查询问题
In practice, multiple $k$DP queries are often processed in batches. 
For example, as communication networks scale with the Internet, numerous routing queries are generated quickly, necessitating high-throughput processing.
This paper focuses on the problem of batch $k$DP query processing (\texttt{batch-$k$DP}), where given a graph $G$ and a set of $k$DP queries $Q$, returns $k$ disjoint paths for each query in $Q$.

% 相关工作:我们工作的定位
Existing studies on $k$DP can be categorized into \emph{single-query}~\cite{baseline_moreverbose,baselineOnlySplitP1,baseline1step2,Penalty,dissimilarity_topk1,SCB} and \emph{single-source}~\cite{2009BRTree,four_independent_spanning,IST_survey}. 
(1) \emph{Single-query} methods include \emph{flow-augmenting path-based}~\cite{baseline_moreverbose,baselineOnlySplitP1,baseline1step2} and \emph{dissimilar path-based}~\cite{Penalty,dissimilarity_topk1,SCB}. 
The latter extends simple path enumeration by incorporating disjointness constraints, but suffers from factorial time complexity (see Sec.~\ref{sec:related_kdp}).
In contrast, the former achieves linear time complexity using \emph{split-graphs}~\cite{baseline_moreverbose,baseline1step2,baselineOnlySplitP1}, where finding disjoint paths reduces to identifying flow-augmenting paths in the split-graphs. 
(2) \emph{Single-source} $k$DP seeks disjoint paths from a source to all other vertices, with current studies primarily theoretical and limited to $k \leq 4$~\cite{IST_survey}. 
(3) To our knowledge, this paper is the first to address \texttt{batch-$k$DP} with general $k$.
% for large graphs with general $k$.

% 解决batch-kDP问题的baseline & 多查询共享机会
For \texttt{batch-$k$DP}, we can adapt methods from single-query $k$DP: 
(1) \emph{Dissimilar path-based} methods can extend batch path enumeration algorithms~\cite{BatchEnum} but also face factorial complexity. 
(2) \emph{Flow-augmenting path-based} methods handle each query in linear time but require independent split-graphs for each query, missing opportunities for shared computation 
% (also observed in empirical studies, see Sec.~\ref{sec:approach_algo}). 
% Our empirical studies reveal significant potential for shared computation (see Sec.~\ref{sec:approach_algo}).

% 用不了已有的多查询技术
This leads to the question: can we apply existing batch-processing techniques from other problems? 
Unfortunately, current techniques are designed for queries in the same graph~\cite{BatchEnum, tods21subgraphiso, icde20shortestpath, msbfs}, while our problem involves different $k$DPs executed over different graphs (split-graphs). 
The first step is to unify the split-graphs, which is not covered in the literature. 
Thus, we propose \texttt{ShareDP}, a batch-$k$DP algorithm that aims to \texttt{share} common computations and storage across a batch of $k$DP queries. 
The overall framework of \texttt{ShareDP} is illustrated in Fig.~\ref{fig:shareDP_overview} (explained in Sec.~\ref{sec:approach_overview}).

% 我们的设计
\texttt{ShareDP} incorporates the following key strategies:
(1) We consolidate individual split-graphs into a unified structure, represented implicitly via result sets, simplifying construction and updates.
(2) Using the merged split-graph, we enable concurrent $k$DP query processing by consolidating traversals and operations leveraging tagged data structures, reducing redundancy and improving efficiency.
% (3) Related queries are grouped into batches for more efficient processing.
In summary, our contributions are:

\begin{itemize}[]
% \vspace{-5pt}
\item \emph{Merged Split-Graph Representation.} 
The \texttt{ShareDP} framework introduces a novel merged split-graph representation that consolidates individual split-graphs into a unified structure, enabling dynamic sharing of traversals.

\item \emph{Optimized Path-Finding with Shared Traversals.} 
The framework combines traversals across multiple queries, consolidating common operations into single steps, reducing redundancy and improving efficiency.

\item \emph{Proven Efficiency and Scalability.}  
Extensive evaluations on 12 real-world datasets demonstrate that \texttt{ShareDP} consistently achieves the lowest runtime across various \( k \) settings, confirming its efficiency and scalability.
\end{itemize}


% 我们的方法
% 我们主要设计了什么
% To achieve shared computation, we address three challenges:
% (1) Merging split-graphs into a shared structure for unified traversals. 
% (2) Efficiently representing, constructing, and updating the merged split-graph. 
% (3) Executing $k$DP queries based on the merged split-graph to share common computations.

% To achieve shared computation, we need to solve the following three challenges:
%     (1) How to merge the split-graphs to a shared structure 
%         so that traversals from different queries can be executed in one unified structure? 
%     (2) How to represent the merged split-graph, and how to efficiently construct and update it? 
%     (3) Based on the merged split-graph, how to execute the $k$DP queries such that common computations can be shared?

% 我们的设计
% To address these challenges, we propose the \texttt{ShareDP} framework with the following key strategies: