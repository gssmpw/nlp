\section{Preliminaries}\label{sec:problemdef}

% A directed graph is a $2$-tuple $G=(V, E)$, where $V$ is the set of vertices and $E$ is the set of edges. 
We use $p$ for a path and $P$ for a set of paths.
A vertex $v$ is an \emph{intermediate} vertex in path $p$ if it is neither the starting nor the ending vertex. If $v$ is intermediate in any path in $P$, it is called \emph{$P$-inner}.
A path $p$ is simple if no vertex appears more than once.
$V(P)$ and $E(P)$ refer to the set of vertices and edges that appear in at least one path in $P$.
While the focus here is on directed graphs, the method can be easily adapted to undirected graphs.

\textbf{Disjoint Paths}
Given a graph $G = (V, E)$ and two simple paths $p_1$ and $p_2$ from $s$ to $t$, $p_1$ and $p_2$ are \emph{disjoint} if they share no common vertex except $s$ or $t$.

\textbf{Problem Statement}
Given a graph $G = (V, E)$, a parameter $k$, and a set of $k$DP queries $Q = \{q_0, q_1, ..., q_{\omega}\}$ where each query $q_i$ is a vertex pair $(s_i, t_i)$ ($i = 1, ..., \omega$), 
\texttt{batch-$k$DP} is to find $k$ disjoint paths for each query $(s_i, t_i) \in Q$.


% \textbf{Example:}
% Consider the graph in Fig.~\ref{fig:g} with \( k = 2 \) and a set of queries \( Q = \{q_0:(a, h), q_1:(b, f), q_2:(c, f)\} \). 
% For $q_0$, two disjoint paths, $p_1$ (red) and $p_2$ (blue), can be identified. Solutions for the remaining queries are found similarly.
