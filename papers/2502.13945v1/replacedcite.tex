\section{Related Work}
%% {{{

Texture blending is common in practice and is part of texture and material creation, but relatively little of it has been explored by the literature.
Artists are assumed to tweak blending masks and source materials until the desired look is achieved.
The state-of-the-art procedural texturing tool Adobe Substance Designer uses various simple pointwise blending modes to facilitate that process____.
____ proposed to use simple pointwise blending with a manually designed, content-dependent weight curve for natural-looking transitions.

While artists can manually adjust masks and the appearance of materials in traditional workflows, procedural texture synthesis aims to automate this process.
Pure pointwise operations often fail to produce reliable and consistent procedural blending results, as a single pixel does not inform about neighbors, image patterns, or structures.
Most procedural texture synthesis publications rely on global or local neighborhood approaches during runtime or as a precomputation step.

One of the earliest practical works was noise by example____.
Those early methods were costly and required offline optimization procedures, such as basis pursuit.
Subsequent works targeted performance optimizations and reduction of the precomputation needed.
____ proposed global variance-based normalization for blending fluid textures. 
____ identified this approach's quality shortcomings and instead proposed adjusting the local texture values based on offline precomputation of optimal transport of histograms.
In later work, ____ proposed simplifying that process through 1D precomputation along with improvements to reduce visual artifacts by taking clipping into account.
Recently, ____ introduced a novel pointwise operator combined with fast precomputation techniques that guarantee consistent minification, antialiasing upon magnification, and stationarity of the resulting blended textures.

\enlargethispage{3pt}