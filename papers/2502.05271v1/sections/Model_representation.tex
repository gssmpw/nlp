
\section{Model Representation}


The robot policy's observation space consists of two components: the object's target velocity and the robot's proprioceptive data. The object's target velocity comes from human demonstrations during training and can be given by a human or generated by a high-level planner during test time. 
The proprioceptive data are acquired from the robot's onboard sensing system. The observation space excludes the object state information, such as the object's orientation and velocity, as well as the object's shape or visual input from the robot's camera. This facilitates sim-to-real transfer, improves generalization across object variations, and allows us to deploy a policy without needing to provide object information. 
Separate policies are trained for different categories of objects to accommodate their unique dynamics. The observation space of the policy is the following:
\begin{itemize}
    \item The object's target velocity.~(3 dims)
    \item The robot's root linear and angular velocities.~(6 dims)
    \item The robot's joint angles.~(18 dims)
    \item The robot's joint velocities.~(18 dims)
    \item The robot's gripper open angle.~(1 dim)
    \item The robot's gripper contact with the object.~(1 dim)
\end{itemize}
Here, the object's target velocity is measured in the current object-plane frame. The robot's root velocities are in the robot-plane coordinate. We test our framework in two types of simulated environments, a simple-dynamic~(SD) and a full-dynamic~(FD) environment, which require two different policy action spaces. The difference between these two environments will be explained in Sec:~\ref{sec:exp_sim_setup}. The two types of action space are:
\begin{itemize}
    \item SD: the robot's root velocities and arm joint angles.~(13 dims)
    \item FD: the robot's arm and leg joint angles.~(19 dims)
\end{itemize}

To train the critic network, besides the components in the policy's observation space, it also includes:

\begin{itemize}
    \item The object's target orientation in the global frame.~(4 dims)
    \item The object's orientation in the global frame.~(4 dims)
    \item The object's root linear and angular velocities.~(6 dims)
\end{itemize}


Both the policy and the critic are implemented as fully connected neural networks with three hidden layers, consisting of [512, 256, 128] units, respectively, using ELU activation functions~\citep{ELUs}. 



