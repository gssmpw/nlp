\newpage
\section{Appendix}
\subsection{Dataset Details}
We selected human-object interaction demonstrations from the OMOMO dataset~\citep{OMOMO}, a motion capture-based dataset featuring high-quality and diverse human and object movements. Although OMOMO contains demonstrations for more than ten objects, some, such as monitors, are not relevant to our task and were excluded. Additionally, objects that require significant two-arm manipulation, such as large boxes, were omitted to align with our study's focus. Below, we summarize the number of demonstration trajectories used for training and their corresponding sources in the OMOMO dataset.

\vspace{0.2cm}

\begin{tabular}{c | c | c }
\toprule
\rowcolor[HTML]{ededed}
Object Type& Num  &  Demo Source\\
\midrule
Chair & 50 &  ['woodchair', 'whitechair'] \\
\midrule
Table &  50 &  ['largetable', 'smalltable']  \\
\midrule
StandingStick & 30 &  ['floorlamp', 'clothesstand']\\
\bottomrule
\end{tabular}


\vspace{0.2cm}

\subsection{Diverse Objects Experiment}

In Section~\ref{sec:hardware_setup}, we introduce Object Diveristy(DIV-ROB) metrics to evaluate the robustness of the result policies by moving objects with different textures and weights. We summarize the objects for our diverse object test in Figure~\ref{fig:diverse}. 

\begin{figure}[H]
\centering


\begin{subfigure}{\linewidth}
\centering
\includegraphics[width=.995\linewidth]{figures/Diverse_objects.png}
\end{subfigure}%
\vspace{.25cm}

\begin{subfigure}{\linewidth}
\centering

\begin{tabular}{c | c | c }

\toprule
\rowcolor[HTML]{ededed}
Name&  Size(cm) & Note\\
\midrule
Wood Chair & 49*44*80 &  Medium Weight, Low Friction \\
\midrule
Heavy Chair &  55*52*80 &  Heavy, Medium Friction  \\
\midrule
Thin Chair &  57*55*80 &  Medium Weight, Low Friction  \\
\midrule
Coffee Table &  55*55*45 &  Medium Weight, Medium Friction  \\
\midrule
Heavy Table &   57*144*42 &  Heavy, Low Friction  \\
\midrule
Standing Lamp &  20*20*175 &  Light Weight, Low Friction  \\
\midrule
Standing Rack & 60*60*175 & Heavy, Low Friction \\
\bottomrule


\end{tabular}

\end{subfigure}
\caption{The collection of objects used in the diverse object experiments.}
\label{fig:diverse}
\end{figure}

\subsection{Heuristic Planner for Chair Rearrangement}
\label{app:heuristic_planner}
In the Chair Rearrangement experiment (Section~\ref{sec:applications}), we use a heuristic planner for object target velocity planning. The planner first moves the object to the target location and then aligns its heading. The detailed planning policy is design as follows:



\begin{algorithm}[htbp]
\caption{Heuristic Planner}
\label{alg:heurisitc_planner}
    \begin{algorithmic}[1]{
        \footnotesize
        \Require Target object position $\bar{p}^{xy}$ and heading $\bar{p}^{head}$, position threshold $\bar{\delta}^{xy}$, heading threshold $\bar{\delta}^{yaw}$, robot policy $\pi$
        \item[]
            
        \State {{\textsc{// Target Position Reaching }}}
        \Repeat
            \State Measure the object position $p^{xy}$ and the delta heading towards the target position $\hat{p}^{head}$.
            \State Target object velocity $\bar{v^{o}}=[0.4,0.0, min(0.4,\hat{p}^{head})] $.
            \State Run robot policy $\pi$.
            
        \Until{$||\bar{p}^{xy} - p^{xy}|| < \bar{\delta}^{xy}$}
        \item[]
            \State {{\textsc{// Target Heading Aligning}}}
      	\Repeat
             \State Measure heading difference $\delta^{head} = \bar{p}^{head} - p^{head}$.
            \State Target object velocity $\bar{v^{o}}=[0.0,0.0, min(0.4,\delta^{head}]$.
            \State Run robot policy $\pi$.
      	\Until{$||\delta^{head}|| < \bar{\delta}^{head}$}
            }
  \end{algorithmic}
\end{algorithm}


We found that this simple heuristic planner works well in our setting. However, for more complex tasks, such as those involving obstacle avoidance, a more sophisticated planner would be required.


