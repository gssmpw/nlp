\begin{abstract}


Multimodal Large Language Models (MLLMs) bridge the gap between visual and textual data, enabling a range of advanced applications. However, complex internal interactions among visual elements and their alignment with text can introduce vulnerabilities, which may be exploited to bypass safety mechanisms. To address this, we analyze the relationship between image content and task and find that the complexity of subimages, rather than their content, is key. Building on this insight, we propose the \textbf{Distraction Hypothesis}, followed by a novel framework called Contrasting Subimage Distraction Jailbreaking (\textbf{CS-DJ}), to achieve jailbreaking by disrupting MLLMs alignment through multi-level distraction strategies. CS-DJ consists of two components: structured distraction, achieved through query decomposition that induces a distributional shift by fragmenting harmful prompts into sub-queries, and visual-enhanced distraction, realized by constructing contrasting subimages to disrupt the interactions among visual elements within the model. This dual strategy disperses the model’s attention, reducing its ability to detect and mitigate harmful content. Extensive experiments across five representative scenarios and four popular closed-source MLLMs, including \texttt{GPT-4o-mini}, \texttt{GPT-4o}, \texttt{GPT-4V}, and \texttt{Gemini-1.5-Flash}, demonstrate that CS-DJ achieves average success rates of \textbf{52.40\%} for the attack success rate and \textbf{74.10\%} for the ensemble attack success rate. These results reveal the potential of distraction-based approaches to exploit and bypass MLLMs' defenses, offering new insights for attack strategies.


\noindent {\color{red} Warning: This paper contains unfiltered content generated by MLLMs that may be offensive to readers.}
\end{abstract}

