\section{Appendix}

In this section, we first provide additional details of the experimental settings in Section~\ref{sec:app_exp}. Then, in Section~\ref{sec:app_prompt}, we present the prompts used by CS-DJ during the query decomposition and jailbreaking execution phases. Finally, Section~\ref{sec:app_case} shows the jailbreak cases for four closed-source models.

\subsection{Experimental Setting}
\label{sec:app_exp}


Here, we provide a detailed overview of the experimental settings. The same configuration was applied to both Hades and CS-DJ. For the GPT series models, the temperature was set to 0.1, with a maximum output length of 1000 tokens. For the Gemini series models, the temperature was adjusted to 0.2, with the maximum output length increased to 2048 tokens.
Additionally, each subimage grid was rendered at a resolution of 500 $\ast$ 500 pixels. If a retrieved image had a resolution smaller than this size, it was placed at the center of the grid without resizing. Otherwise, the image was scaled proportionally to fit within a 500-pixel width or height while preserving its aspect ratio. Sub-queries were converted into images using the Super Moods font, with the font color set to red and a font size of 50.


\subsection{Prompt Design}
\label{sec:app_prompt}
\subsubsection{Decomposition Query}


CS-DJ employs an auxiliary decomposition model $\mathcal{G}$ to break down the raw query into multiple sub-queries, thereby introducing structured distraction. The prompt for decomposing into three sub-queries is provided in Figure~\ref{app_fig:decop}. Notably, the decomposition by $\mathcal{G}$ is considered complete only when the generated responses strictly adhere to the specified format. Otherwise, the process is retried up to a maximum of 5 attempts in practice.

\begin{figure}
  \centering  \includegraphics[width=1\linewidth]{Appendix_Source/Figures/figs/Query_Decoposition_Prompt.pdf}
  \vspace{-6mm}
   \caption{Prompt for query decomposition. specifically, the placeholder \textit{\{jailbreak query\}} is replaced with the raw query. The bolded portion should be modified to align with the number of sub-queries.}
   \label{app_fig:decop}
   \vspace{-4mm}
\end{figure}

\subsubsection{Jailbreaking Execution}

Figure~\ref{app_fig:jail_exe} illustrates the prompt used for the jailbreaking execution of CS-DJ, carefully designed to enhance the distraction effect. The instruction consists of three main sections: the role-guiding section (in red), the task-guiding section (in black), and the visual-guiding section (in blue). The role-guiding section establishes a scenario for the model, providing the contextual framework for the subsequent tasks. The task-guiding section instructs the model to simultaneously perform multiple tasks within specific subimages, increasing complexity and deliberately dispersing its focus across different objectives. Lastly, the visual-guiding section introduces misleading cues, implying that other subimages might be useful, further diverting the model’s attention.

\begin{figure}
  \centering  \includegraphics[width=1\linewidth]{Appendix_Source/Figures/figs/Query_Jailbreaking.pdf}
  \vspace{-6mm}
   \caption{Prompt for the jailbreaking execution. The bolded portion should be modified to align with the number of subimages.}
   \label{app_fig:jail_exe}
   \vspace{-3mm}
\end{figure}

\subsection{Additional Jailbreak Cases}
\label{sec:app_case}
This section presents detailed jailbreak cases for GPT-4o-min, GPT-4o, GPT-4V, and Gemini-1.5-Flash, as shown in Figure~\ref{app_fig:4o-mini}, Figure~\ref{app_fig:4o}, Figure~\ref{app_fig:4V}, and Figure~\ref{app_fig:gemini}, respectively.

\begin{figure}
  \centering  \includegraphics[width=1\linewidth]{Appendix_Source/Figures/figs/Jailbreak_case_gpt4o-mini.pdf}
   \caption{Jailbreak case of GPT-4o-mini.}
   \label{app_fig:4o-mini}
\end{figure}
\begin{figure}
  \centering  \includegraphics[width=1\linewidth]{Appendix_Source/Figures/figs/Jailbreak_case_gpt4o.pdf}
   \caption{Jailbreak case of GPT-4o.}
   \label{app_fig:4o}
\end{figure}
\begin{figure}
  \centering  \includegraphics[width=1\linewidth]{Appendix_Source/Figures/figs/Jailbreak_case_gpt4v.pdf}
   \caption{Jailbreak case of GPT-4V.}
   \label{app_fig:4V}
\end{figure}
\begin{figure}
  \centering  \includegraphics[width=1\linewidth]{Appendix_Source/Figures/figs/Jailbreak_case_gemini-1.5.pdf}
   \caption{Jailbreak case of Gemini-1.5-Flash.}
   \label{app_fig:gemini}
\end{figure}