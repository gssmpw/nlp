%%%%%%%% ICML 2025 EXAMPLE LATEX SUBMISSION FILE %%%%%%%%%%%%%%%%%

\documentclass{article}

% Recommended, but optional, packages for figures and better typesetting:
\usepackage{microtype}
\usepackage{graphicx}
\usepackage{subfigure}
\usepackage{tikz}
\usetikzlibrary{arrows.meta,shapes,positioning,calc}
\usepackage{booktabs} % for professional tables
\usepackage[maxnames=1,maxbibnames=2]{biblatex}
\addbibresource{reference.bib}  
% hyperref makes hyperlinks in the resulting PDF.
% If your build breaks (sometimes temporarily if a hyperlink spans a page)
% please comment out the following usepackage line and replace
\usepackage{hyperref}


% Attempt to make hyperref and algorithmic work together better:
\newcommand{\theHalgorithm}{\arabic{algorithm}}


% For theorems and such
\usepackage{arxiv}
\usepackage{amsmath}
\usepackage{amssymb}
\usepackage{mathtools}
\usepackage{amsthm}
\usepackage{multirow}
\usepackage[textwidth=0.5in]{todonotes}

% if you use cleveref..
\usepackage[capitalize,noabbrev]{cleveref}

%%%%%%%%%%%%%%%%%%%%%%%%%%%%%%%%
% THEOREMS
%%%%%%%%%%%%%%%%%%%%%%%%%%%%%%%%
\theoremstyle{plain}
\newtheorem{theorem}{Theorem}[section]
\newtheorem{proposition}[theorem]{Proposition}
\newtheorem{lemma}[theorem]{Lemma}
\newtheorem{corollary}[theorem]{Corollary}
\theoremstyle{definition}
\newtheorem{definition}[theorem]{Definition}
\newtheorem{assumption}[theorem]{Assumption}
\theoremstyle{remark}
\newtheorem{remark}[theorem]{Remark}
\newcommand{\cmark}{\ding{51}}%
\newcommand{\xmark}{\ding{55}}
\definecolor{ForestGreen}{RGB}{5,166,88}
\definecolor{LavaRed}{RGB}{222,48,28}
\definecolor{LightGrey}{RGB}{180,180,180}
\def\jiaxian#1{\textcolor{red}{[JXG: #1]}}

% Todonotes is useful during development; simply uncomment the next line
%    and comment out the line below the next line to turn off comments
%\usepackage[disable,textsize=tiny]{todonotes}



\author{%
\bf Ru Wang {\textsuperscript{1}}
\qquad
\bf Wei Huang {\textsuperscript{2}}
\qquad
\bf Selena Song {\textsuperscript{1}}
\qquad
\bf Haoyu Zhang {\textsuperscript{1}}
\\
\bf Yusuke Iwasawa {\textsuperscript{1}}
\qquad
\bf  Yutaka Matsuo {\textsuperscript{1}}
\qquad
\bf  Jiaxian Guo {\textsuperscript{3}} \\
 \texttt{\{ru.wang,kexin.song,haoyu.zhang\}@weblab.t.u-tokyo.ac.jp} \\
\texttt{weihuang.uts@gmail.com} \\
\texttt{jeffguo@google.com}
\\
The University of Tokyo {\textsuperscript{1}} \quad RIKEN Center for Advanced Intelligence Project {\textsuperscript{2}}  \quad 
Google Research, Australia {\textsuperscript{3}} 
}


\begin{document}
% \nolinenumbers 

\title{Beyond In-Distribution Success: Scaling Curves of CoT Granularity for Language Model Generalization}

\maketitle

\begin{abstract}
Out-of-distribution (OOD) detection and OOD generalization are widely studied in Deep Neural Networks (DNNs), yet their relationship remains poorly understood. We empirically show that the degree of Neural Collapse (NC) in a network layer is inversely related with these objectives: stronger NC improves OOD detection but degrades generalization, while weaker NC enhances generalization at the cost of detection. This trade-off suggests that a single feature space cannot simultaneously achieve both tasks. To address this, we develop a theoretical framework linking NC to OOD detection and generalization. We show that entropy regularization mitigates NC to improve generalization, while a fixed Simplex Equiangular Tight Frame (ETF) projector enforces NC for better detection. Based on these insights, we propose a method to control NC at different DNN layers. In experiments, our method excels at both tasks across OOD datasets and DNN architectures. 

\begin{comment}   

Out-of-distribution (OOD) detection and OOD generalization are critical for deploying machine learning models in real-world scenarios. While substantial progress has been made in addressing these problems independently, few works have attempted to tackle them jointly. However, existing methods often rely on auxiliary OOD training data and primarily focus on covariate-shifted OOD data that share labels with in-distribution (ID) data. In contrast, we tackle the more realistic and challenging task of jointly detecting and generalizing to semantic OOD data with disjoint labels from the ID data, without auxiliary OOD training data.
Achieving both objectives simultaneously is inherently difficult due to a fundamental conflict — OOD generalization requires enhanced transferability, while OOD detection necessitates the inhibition of transfer.
To address this, we leverage insights from neural collapse (NC) — a phenomenon in deep networks where top-layer representations suppress feature variability and adopt a Simplex Equiangular Tight Frame (ETF) structure, impairing transferability. By controlling NC, we unify OOD detection and generalization: preventing NC enhances OOD transfer while inducing NC improves OOD detection.
Our proposed method excels at both tasks across various OOD datasets and architectures. 

\end{comment}


\end{abstract}


\section{Introduction}

% State of the world (robots for creative activites)
The term ``robot,'' originally signifying `forced labor,' has long been associated with labor and work. Robots have demonstrated their utility in various automated productive and social contexts, where the primary goals are improving productivity, safety, and fostering social interactions with humans~\cite{simoes2022designing, weidemann2021role, honig2018understanding}. However, an increasing number of cases feature using of robots in creative settings. Unlike productive contexts, where the focus is on efficiency and task completion~\cite{arents2022smart}, or social contexts, where communication and trust are prioritized~\cite{nam2020trust, saunderson2019robots}, creative environments prioritize artistic innovation and expression~\cite{hsueh2024counts}. This shift fundamentally alters the dynamics of human-robot interaction, redefining the roles and expectations for both humans and robots.

For instance, robots’ social behaviors are leveraged to support the generation and expression of creative ideas~\cite{hu2021exploring, sandoval2022human, alves2020creativity}, and programmable robotic movements and trajectories are employed to inspire artistic activities such as sketching~\cite{lin2020your}. These studies often engage participants from creative fields who possess limited prior experience with robotics, and are typically conducted in short-term, experimental settings. Consequently, the findings from these studies remain constrained since much can be learned from professional practitioners' experiences to inform system design such as digital fabrication~\cite{hirsch2023nothing}. There is a notable gap in research examining the long-term, active, and practical experience of integrating robotic systems into the creative processes. As a result, the deeper insights into how robots facilitate and shape creative processes, beyond simply augmenting human creativity, remain underexplored. In this study, we aim to better understand the impacts of robots on creative processes and outcomes.

As early as Leonardo da Vinci's 16th century ``Automaton,'' artists have explored the creative affordances of robotic systems~\cite{shanken2002cybernetics, pagliarini2009development, jeon2017robotic}. The artistic creation process typically encompasses various stages, including the exploration of materials and techniques, ongoing experimentation and iteration, and the continual refinement of the artists' insights into their creative subjects~\cite{lewis2023art, sturdee2022state}. Therefore, investigating the artistic process involving robots offers an opportunity to gain deeper insights into robots' creative potential. Robotic art, in particular, provides a compelling case for this exploration.

We define robotic art as artworks that utilize robotic or automated machines to create artistic experiences and tangible artifacts. One example is robotic installation art, in which robots are programmed to follow specific rules that embody the artist’s expression (\autoref{fig:teaser} (a)). Another example is responsive art, in which robots react to their environment, with behaviors that change over time or in response to spectators (\autoref{fig:teaser} (b)). Additionally, there are robotic creators, which possess a degree of agency, allowing them to collaborate with human artists and produce works that extend beyond mere replication of human-created art (\autoref{fig:teaser} (c) and (d)). As such, robotic art becomes a rich case for exploring human-machine interactions in creative contexts. Gaining a deeper understanding of how robots facilitate artistic expression can provide insights for designing computing systems to support creative activities~\cite{gomez2021robot}.

% Therefore, we did...
We draw on semi-structured, in-depth interviews with renowned professional robotic artists to investigate the use of robots in artistic practice. Specifically, our goal is to understand how artistic exploration of robotic systems challenges conventional assumptions about the functions of robots, such as their roles in automating repetitive tasks or serving human needs. We also explore the implications of robots in the artistic process and examine how creativity may emerge within robotic art. To address these interrelated inquiries, our study focuses on the practice of robotic art, posing the research question: \textit{How do robotic artists utilize robots in their artistic practice?} We approach this inquiry through the perspectives and experiences of robotic artists, who creatively design, modify, and repurpose robotic systems for artistic expression and exploration.

% The key findings are...
Our findings highlight the social, material, and temporal dimensions of artists' practices that shape their creativity and artistic outcomes. The creation of robotic art is largely a social process, as artists receive both explicit and implicit feedback through the audience's reactions and reception of their work. Simultaneously, the embodiment and malfunctions inherent to robotic systems drive artistic experimentation. The temporal processes of creation and exhibition, beyond just the final product, further enhance the creative value. Our empirical analysis presents how creativity emerges through the interplay of social, material, and temporal interactions among artists, robots, audiences, and the environment.

% The contributions of this work are...
We make two main contributions to HCI in this study. 
First, we elucidate the interactive mechanisms among key actors---human creators, machines, audiences, and environments---within the practice of robotic art, a topic that remains underexplored in HCI. Our findings reveal the significance of sociality (e.g., interactions between artists and audiences), materiality (e.g., the embodiment and malfunctions of robots), and temporality (e.g., the processes of creation and exhibition) in shaping creative values. We propose that these three facets are central to the creative process and facilitate the emergence of creativity in robotic art.
Second, drawing from the findings, we offer implications for \textit{socially informed}, \textit{material-attentive}, and \textit{process-oriented} creation with computing systems. We suggest leveraging these three aspects to enhance creativity and the creative experience. Specifically, we discuss the value of incorporating implicit audience feedback, designing with technical malfunctions, and focusing on the post-creation process to foster alternative creative experiences with machines~\cite{alter2010designing, juarez2022glitch}.




\paragraph{Uncertainty-based hallucination detection methods.}
Various approaches have been proposed to detect hallucinated content in LLMs generation.
Unlike other methods that require external knowledge sources for fact-checking~\citep{gou2024critic, chen-etal-2024-complex, min-etal-2023-factscore, huo2023retrieving}, uncertainty-based approaches are reference-free and rely only on LLM internal states or behaviors to determine hallucination~\citep{10.1145/3703155}. 
For instance, sampling-based approaches generate multiple responses and measure the diversity in meaning among them~\citep{fomicheva-etal-2020-unsupervised, kuhn2023semantic, lin2024generating}, while density-based approaches approximate the training data distribution and provide probabilities or unnormalized scores to assess how likely a generated response belongs to the distribution~\citep{yoo-etal-2022-detection, ren2023outofdistribution, vazhentsev-etal-2023-hybrid}.

In this paper, we focus on uncertainty quantification methods that rely on token-level likelihood or entropy~\citep{guerreiro-etal-2023-looking, malinin2021uncertainty}. 
Recent works have explored refining likelihood estimation by incorporating semantic relationships or reweighting token importance. For instance, Claim-Conditioned Probability (CCP)~\citep{fadeeva-etal-2024-fact} was introduced to recalculate likelihood according to semantical equivalence; while \citet{zhang-etal-2023-enhancing-uncertainty} and \citet{duan-etal-2024-shifting} adjust token weights to better convey meaning in uncertainty aggregation. \emph{Although these approaches leverage token-level information, they are typically evaluated at the sentence level, raising questions about their reliability}. To address this, we conduct a comprehensive analysis of entity-level hallucination detection for finer-grained performance insights.


\paragraph{Fine-grained hallucination detection benchmark.}

Most hallucination detection benchmarks are in sentence or paragraph level. For example, CoQA~\citep{reddy-etal-2019-coqa}, TriviaQA~\citep{joshi-etal-2017-triviaqa}, TruthfulQA~\citep{lin-etal-2022-truthfulqa}, and HaluEval~\citep{li-etal-2023-halueval}. These benchmarks classify each generated response as either hallucinated or correct. However, instance-level detection cannot pinpoint specific hallucinated content, which is crucial for correcting misinformation~\citep{cattan2024localizingfactualinconsistenciesattributable}. This limitation becomes particularly problematic in long-form text, where a single response often combines supported and unsupported information, making binary quality judgments inadequate~\citep{min-etal-2023-factscore}.

To address these challenges, recent works have advanced benchmarks for more granular hallucination detection. For example, \citet{min-etal-2023-factscore} introduced \textsc{FActScore}, which decomposes LLM-generated text into atomic facts---short sentences conveying a single piece of information---for more precise evaluation. In parallel, \citet{cattan2024localizingfactualinconsistenciesattributable} introduced \textsc{QASemConsistency}, decomposing LLM generated text with QA-SRL, a semantic formalism, to form simple QA pairs, where each QA pair represent one verifiable fact. \emph{However, these methods do not enable entity-level hallucination detection, as they lack explicit entity-level labeling (hallucinated or not) in the original generated text}.  
Beyond decomposition-based approaches, datasets like \textsc{HaDes}~\citep{liu-etal-2022-token} and CLIFF~\citep{cao-wang-2021-cliff} create token-level hallucinated content by perturbing human-written text, allowing token-level annotation on the same text. These perturbed hallucinated content, however, could be unrealistic, biased, and overly synthetic due to the limitations of models they used to perturb words. 
To bridge this gap, we create a new dataset with entity-level hallucination labels on the same LLMs generated text. This allows us to evaluate uncertainty-based hallucination detection approaches on a finer-grained level and analyze their reliability.





\vspace{-5pt}
\section{Method}
\label{sec:method}
\begin{figure*}[t]
\begin{center}
\includegraphics[width=.85\linewidth]{fig_overview_v3.pdf}
\end{center}
\caption{
FastAtlas Overview: In each frame, we compute charts spanning fully or partially visible triangles (a), determine texture space bounding boxes for the visible portions of the view-space projections of each chart, and tightly pack these boxes into atlases (b, here $2K \times 2K$). We simultaneously bijectively parameterize and shade the charts into their atlas boxes, obtaining high quality texture space shading (c), and use this shading to render the shaded frames (d).}
\label{fig:overview}
\label{fig:alg_overview}
\end{figure*}

\section{Overview}
\label{sec:overview}
Our work has two core contributions: a real-time, GPU-based algorithm for tight packing of general parameterized charts into compact atlases; and a real-time TSS method that
utilizes this packing.  

\paragraph*{FastAtlas Packing.}
FastAtlas runs entirely on the GPU as a series of compute shaders. It takes the bounding boxes of parameterized charts as input, and packs them into an atlas (Fig~\ref{fig:overview}b, Sec.~\ref{sec:pack}). As such, the only input it requires are the dimensions of the bounding boxes.
Its outputs are deterministic; identical input charts are packed into identical atlases. This is critical for TSS and similar applications, as it ensures that consecutive frames taken from the same camera view have the same shading. Even minute shading differences across such frames can cause sampling jitter, leading to undesirable flicker \cite{baker2012rock}. 
While prior methods such as \cite{mueller2018shading,hladky2019tessellated,hladky2021snakebinning,Neff2022MSA} cap the dimensions of the charts that can be packed as-is for a given atlas size, and scale down all charts that exceed these dimensions, we scale all charts by the same factor, and do so only when strictly necessary to achieve packing success (Figs~\ref{fig:atlas},~\ref{fig:sas_issues}). 

\paragraph*{TSS using FastAtlas.}
Our end-to-end TSS atlas generation method combines the packing method above with a novel approach for computing seamless per-frame charts. 
We define our charts as the connected components of the visible surfaces in each frame (Fig.~\ref{fig:overview}a), and efficiently compute them using a parallel union-find algorithm (Sec.~\ref{sec:visible}). Since the boundaries of these charts coincide with the contours of the rendered surface, they are {\em invisible} to the viewer. This approach 
eliminates the artifacts caused by shading discontinuities along visible seams (Fig.~\ref{fig:seams}). 

\begin{parWithWrapFigure}
\begin{wrapfigure}{l}{.27\columnwidth}%
\includegraphics[width=\linewidth]{fig_inset_view_plane.pdf}%
\end{wrapfigure}
We bijectively parametrize the {\em visible portions} of our charts by projecting them to view space (inset). This maps a constant number of texels to each pixel in the final rendered output, evenly distributing residual undersampling error across all image pixels. While conceptually straightforward, efficiently parameterizing charts containing partially visible triangles using viewspace projection is non-trivial, as the visible portions may no longer be triangular (e.g. green triangle in the inset); applying naive projection to triangles with vertices behind the camera may produce ill-posed results. Clipping triangles before projection is both computationally expensive and significantly complicates downstream operations. We avoid explicit clipping by observing that all that is required for atlas packing is the dimensions of, potentially conservative, bounding boxes of these projected visible portions. We compute such bounding boxes without explicit chart clipping by adapting a conservative screen coverage estimator \shortcite{Blinn:CalculatingScreenCoverage} (Sec.~\ref{sec:box}). We then pack the computed boxes using FastAtlas. 
\end{parWithWrapFigure}

Finally, we shade the visible portion of each chart into its corresponding atlas bounding box (Fig~\ref{fig:overview}c). 
The resulting texture is then used during rasterization as a standard texture map (Fig. ~\ref{fig:overview}d). 
Our framework is compatible with all existing approaches for texture space shading, including forward shading, raytraced illumination, or deferred shading in texture space \cite{baker:2016}. In the examples shown, we use the standard forward shading based rendering pipeline included in the G3D Innovation Engine \cite{G3D17}, a commercial grade renderer.


Our goal is to increase the robustness of T2I models, particularly with rare or unseen concepts, which they struggle to generate. To do so, we investigate a retrieval-augmented generation approach, through which we dynamically select images that can provide the model with missing visual cues. Importantly, we focus on models that were not trained for RAG, and show that existing image conditioning tools can be leveraged to support RAG post-hoc.
As depicted in \cref{fig:overview}, given a text prompt and a T2I generative model, we start by generating an image with the given prompt. Then, we query a VLM with the image, and ask it to decide if the image matches the prompt. If it does not, we aim to retrieve images representing the concepts that are missing from the image, and provide them as additional context to the model to guide it toward better alignment with the prompt.
In the following sections, we describe our method by answering key questions:
(1) How do we know which images to retrieve? 
(2) How can we retrieve the required images? 
and (3) How can we use the retrieved images for unknown concept generation?
By answering these questions, we achieve our goal of generating new concepts that the model struggles to generate on its own.

\vspace{-3pt}
\subsection{Which images to retrieve?}
The amount of images we can pass to a model is limited, hence we need to decide which images to pass as references to guide the generation of a base model. As T2I models are already capable of generating many concepts successfully, an efficient strategy would be passing only concepts they struggle to generate as references, and not all the concepts in a prompt.
To find the challenging concepts,
we utilize a VLM and apply a step-by-step method, as depicted in the bottom part of \cref{fig:overview}. First, we generate an initial image with a T2I model. Then, we provide the VLM with the initial prompt and image, and ask it if they match. If not, we ask the VLM to identify missing concepts and
focus on content and style, since these are easy to convey through visual cues.
As demonstrated in \cref{tab:ablations}, empirical experiments show that image retrieval from detailed image captions yields better results than retrieval from brief, generic concept descriptions.
Therefore, after identifying the missing concepts, we ask the VLM to suggest detailed image captions for images that describe each of the concepts. 

\vspace{-4pt}
\subsubsection{Error Handling}
\label{subsec:err_hand}

The VLM may sometimes fail to identify the missing concepts in an image, and will respond that it is ``unable to respond''. In these rare cases, we allow up to 3 query repetitions, while increasing the query temperature in each repetition. Increasing the temperature allows for more diverse responses by encouraging the model to sample less probable words.
In most cases, using our suggested step-by-step method yields better results than retrieving images directly from the given prompt (see 
\cref{subsec:ablations}).
However, if the VLM still fails to identify the missing concepts after multiple attempts, we fall back to retrieving images directly from the prompt, as it usually means the VLM does not know what is the meaning of the prompt.

The used prompts can be found in \cref{app:prompts}.
Next, we turn to retrieve images based on the acquired image captions.

\vspace{-3pt}
\subsection{How to retrieve the required images?}

Given $n$ image captions, our goal is to retrieve the images that are most similar to these captions from a dataset. 
To retrieve images matching a given image caption, we compare the caption to all the images in the dataset using a text-image similarity metric and retrieve the top $k$ most similar images.
Text-to-image retrieval is an active research field~\cite{radford2021learning, zhai2023sigmoid, ray2024cola, vendrowinquire}, where no single method is perfect.
Retrieval is especially hard when the dataset does not contain an exact match to the query \cite{biswas2024efficient} or when the task is fine-grained retrieval, that depends on subtle details~\cite{wei2022fine}.
Hence, a common retrieval workflow is to first retrieve image candidates using pre-computed embeddings, and then re-rank the retrieved candidates using a different, often more expensive but accurate, method \cite{vendrowinquire}.
Following this workflow, we experimented with cosine similarity over different embeddings, and with multiple re-ranking methods of reference candidates.
Although re-ranking sometimes yields better results compared to simply using cosine similarity between CLIP~\cite{radford2021learning} embeddings, the difference was not significant in most of our experiments. Therefore, for simplicity, we use cosine similarity between CLIP embeddings as our similarity metric (see \cref{tab:sim_metrics}, \cref{subsec:ablations} for more details about our experiments with different similarity metrics).

\vspace{-3pt}
\subsection{How to use the retrieved images?}
Putting it all together, after retrieving relevant images, all that is left to do is to use them as context so they are beneficial for the model.
We experimented with two types of models; models that are trained to receive images as input in addition to text and have ICL capabilities (e.g., OmniGen~\cite{xiao2024omnigen}), and T2I models augmented with an image encoder in post-training (e.g., SDXL~\cite{podellsdxl} with IP-adapter~\cite{ye2023ip}).
As the first model type has ICL capabilities, we can supply the retrieved images as examples that it can learn from, by adjusting the original prompt.
Although the second model type lacks true ICL capabilities, it offers image-based control functionalities, which we can leverage for applying RAG over it with our method.
Hence, for both model types, we augment the input prompt to contain a reference of the retrieved images as examples.
Formally, given a prompt $p$, $n$ concepts, and $k$ compatible images for each concept, we use the following template to create a new prompt:
``According to these examples of 
$\mathord{<}c_1\mathord{>:<}img_{1,1}\mathord{>}, ... , \mathord{<}img_{1,k}\mathord{>}, ... , \mathord{<}c_n\mathord{>:<}img_{n,1}\mathord{>}, ... , $
$\mathord{<}img_{n,k}\mathord{>}$,
generate $\mathord{<}p\mathord{>}$'', 
where $c_i$ for $i\in{[1,n]}$ is a compatible image caption of the image $\mathord{<}img_{i,j}\mathord{>},  j\in{[1,k]}$. 

This prompt allows models to learn missing concepts from the images, guiding them to generate the required result. 

\textbf{Personalized Generation}: 
For models that support multiple input images, we can apply our method for personalized generation as well, to generate rare concept combinations with personal concepts. In this case, we use one image for personal content, and 1+ other reference images for missing concepts. For example, given an image of a specific cat, we can generate diverse images of it, ranging from a mug featuring the cat to a lego of it or atypical situations like the cat writing code or teaching a classroom of dogs (\cref{fig:personalization}).
\vspace{-2pt}
\begin{figure}[htp]
  \centering
   \includegraphics[width=\linewidth]{Assets/personalization.pdf}
   \caption{\textbf{Personalized generation example.}
   \emph{ImageRAG} can work in parallel with personalization methods and enhance their capabilities. For example, although OmniGen can generate images of a subject based on an image, it struggles to generate some concepts. Using references retrieved by our method, it can generate the required result.
}
   \label{fig:personalization}\vspace{-10pt}
\end{figure}

\section{Empirical Evaluation}
\begin{table*}[!ht]
    \centering
    \resizebox{0.88\textwidth}{!}{    
    \begin{tabular}{r|cccccc|cccccc}
        \toprule 
        & \multicolumn{6}{c}{\textbf{LLaVA-1.5-7B}} & \multicolumn{6}{c}{\textbf{LLaVA-1.5-13B}} \\ 
        \cmidrule(lr){2-7}\cmidrule(lr){8-13}
        & \multicolumn{3}{c}{\textbf{MM-SafetyBench}} & \multicolumn{3}{c|}{\textbf{MOSSBench}} & \multicolumn{3}{c}{\textbf{MM-SafetyBench}} & \multicolumn{3}{c}{\textbf{MOSSBench}} \\
        \textbf{Method} & \textbf{DSR}$\uparrow$ & \textbf{RR}$\uparrow$ & \textbf{Avg}$\uparrow$ & \textbf{DSR}$\uparrow$ & \textbf{RR}$\uparrow$ & \textbf{Avg}$\uparrow$ & \textbf{DSR}$\uparrow$ & \textbf{RR}$\uparrow$ & \textbf{Avg}$\uparrow$ & \textbf{DSR}$\uparrow$ & \textbf{RR}$\uparrow$ & \textbf{Avg}$\uparrow$\\
        \midrule
        w/o Defense          & 0.06  & 0.98  & 0.52  & 0.14  & 0.97  & 0.55  & 0.10  & 0.97  & 0.53  & 0.30  & 0.96  & 0.63  \\
        \midrule
        \multicolumn{13}{c}{Baseline} \\
        \midrule
        Responsible          & 0.12  & 0.96  & 0.54  & 0.32  & 0.96  & 0.64  & 0.18  & 0.96  & 0.57  & 0.47  & 0.92  & 0.70  \\
        Policy               & 0.08  & 0.96  & 0.52  & 0.18  & 0.98  & 0.58  & 0.12  & 0.97  & 0.55  & 0.34  & 0.97  & 0.65  \\
        Demonstration        & 0.15  & 0.97  & 0.56  & 0.37  & 0.95  & 0.66  & 0.25  & 0.96  & 0.60  & 0.52  & 0.92  & \textbf{0.72}  \\
        SFT                  & 0.20  & 0.95  & 0.58  & 0.50  & 0.88  & 0.69  & 0.13  & 0.98  & 0.55  & 0.49  & 0.88  & 0.68 \\
        SafeDecoding         & 0.08  & 0.97  & 0.53  & 0.31  & 0.94  & 0.62  & 0.12  & 0.96  & 0.54  & 0.42  & 0.93  & 0.68  \\
        Caption              & 0.09  & 0.98  & 0.53  & 0.21  & 0.98  & 0.60  & 0.12  & 0.97  & 0.55  & 0.27  & 0.94  & 0.60  \\
        Caption (w/o image)  & 0.16  & 0.95  & 0.55  & 0.34  & 0.94  & 0.64  & 0.22  & 0.93  & 0.57  & 0.45  & 0.89  & 0.67 \\
        Intention            & 0.07  & 0.98  & 0.53  & 0.20  & 0.99  & 0.59  & 0.11  & 0.96  & 0.54  & 0.26  & 0.97  & 0.61  \\
        \midrule
        % \multicolumn{13}{c}{} \\
        % \midrule
        \midrule
        \multicolumn{13}{c}{SR++} \\
        \midrule        
        Responsible-Demonstration & 0.18 & 0.95 & 0.57 & 0.40 & 0.94 & 0.67 & 0.29 & 0.96 & 0.62 & 0.58 & 0.85 & \textbf{0.72} \\
        Responsible-Policy & 0.12 & 0.96 & 0.54 & 0.27 & 0.97 & 0.62 & 0.18 & 0.96 & 0.57 & 0.46 & 0.94 & 0.70 \\
        Policy-Demonstration & 0.13 & 0.96 & 0.55 & 0.37 & 0.97 & 0.67 & 0.20 & 0.96 & 0.58 &0.51 & 0.93 & \textbf{0.72}\\
        Responsible-Policy-Demonstration & 0.15 & 0.96 & 0.55 & 0.38 & 0.95 & 0.66 & 0.25 & 0.97 & 0.61 & 0.53 & 0.88 & 0.70\\
        \midrule
        \multicolumn{13}{c}{SR+MO} \\
        \midrule     
        Responsible-SFT & 0.56 & 0.93 & \textbf{0.75} & 0.61 & 0.72 & 0.67 & 0.35 & 0.96 & 0.65 & 0.74 & 0.62 & 0.68 \\
        Responsible-SafeDecoding & 0.30 & 0.96 & 0.63 & 0.54 & 0.87 & \underline{0.70} & 0.23 & 0.96 & 0.59 & 0.63 & 0.79 & 0.71\\
        Demonstration-SFT & 0.60 & 0.90 & \textbf{0.75} & 0.65 & 0.77 & \textbf{0.71} & 0.56 & 0.92 & \textbf{0.74} & 0.67 & 0.70 & 0.68\\
        Demonstration-SafeDecoding & 0.38 & 0.96 & \underline{0.67} & 0.55 & 0.87 & \textbf{0.71} & 0.40 & 0.96 & \underline{0.68} & 0.62 & 0.78 & 0.70\\
        \midrule
        \multicolumn{13}{c}{QR++} \\
        \midrule   
        Caption-Intention & 0.09 & 0.97 & 0.53 & 0.20 & 0.98 & 0.59 & 0.14 & 0.95 & 0.55 & 0.26 & 0.96 & 0.61\\
        % Caption-Intention (w/o image) & 0.18 & 0.96 & 0.57 & 0.32 & 0.95 & 0.64 & 0.25 & 0.92 & 0.59 & 0.45 & 0.92 & 0.68\\
        \midrule
        % \multicolumn{13}{c}{} \\
        % \midrule
        \midrule
        \multicolumn{13}{c}{QR\textbar{}SR} \\
        \midrule   
        Caption-Responsible & 0.34 & 0.96 & 0.65 & 0.53 & 0.79 & 0.66 & 0.33 & 0.96 & 0.65 & 0.50 & 0.82 & 0.66\\
        Intention-Responsible & 0.36 & 0.97 & \underline{0.67} & 0.51 & 0.86 & 0.68 & 0.27 & 0.96 & 0.61 & 0.49 & 0.90 & 0.70\\
        Caption-Responsible (w/o image) & 0.96 & 0.25 & 0.60 & 0.93 & 0.16 & 0.55 & 0.60 & 0.80 & \underline{0.70} & 0.72 & 0.72 & \textbf{0.72}\\
        % Responsible-Intention (w/o image) & 0.99 & 0.06 & 0.52 & 0.95 & 0.17 & 0.56 & 0.61 & 0.81 & 0.71 & 0.68 & 0.77 & 0.72\\
        \midrule
        \multicolumn{13}{c}{QR\textbar{}MO} \\
        \midrule
        Caption-SafeDecoding & 0.20 & 0.96 & 0.58 & 0.39 & 0.88 & 0.64 & 0.33 & 0.94 & 0.63 & 0.40 & 0.90 & 0.65 \\
        Intention-SFT & 0.28 & 0.97 & 0.62 & 0.43 & 0.78 & 0.61 & 0.25 & 0.96 & 0.60 & 0.50 & 0.88 & 0.69\\
        Caption-SafeDecoding (w/o image) & 0.24 & 0.95 & 0.60 & 0.41 & 0.89 & 0.65 & 0.36 & 0.85 & 0.61 & 0.56 & 0.84 & 0.70\\
        \bottomrule
    \end{tabular}}
    \caption{Comparison results of ensemble strategies with the corresponding individual defenses. \textbf{Bold} indicates the best overall performance, while \underline{underlined} highlights the top three methods.} % and the full score is 100\%
    \label{tab:en_inter_results}
\end{table*}


\subsection{Experimental Setup}
We empirically evaluate various defense methods and their ensemble strategies on LLaVA-1.5-7B and LLaVA-1.5-13B~\cite{liu2024visual} to validate their effectiveness in standard settings. Using MM-SafetyBench and MOSSBench datasets, we assess safety and helpfulness by measuring defense success rate (DSR) on harmful queries and response rate (RR) on benign queries. We evaluate 28 defense methods, including system reminders, optimization techniques, query refactoring, and noise injection, as well as inter- and intra-mechanism ensembles. Detailed descriptions of defense methods and experimental setups are provided in Appendix~\ref{sec:defense strategies} and~\ref{sec:experiment_detail}. 
For a broader evaluation, we add more experiments in Appendix~\ref{sec:utility}, ~\ref{sec:diverse_attacks} and~\ref{sec:time}, including evaluation with the MM-Vet dataset for testing the quality of model's response on general queries, tests on JailbreakV-28K for more diverse and complex attack scenarios, and a comparison of inference time for different defense methods.

\subsection{Individual Defense Results}

Table~\ref{tab:indi_results} shows results of individual defense methods across four categories. Most methods, except for noise injection, effectively improve model safety across different models and datasets, as evidenced by increased defense success rates. This aligns with our analysis in Figure~\ref{fig:analysis results} where system reminder, model optimization and query refactoring lead to an overall increase in refusal probabilities. 

\paragraph{Safety shift defenses compromise helpfulness.} System reminder and model optimization methods generally reduce response rates on the benign subset while increasing defense success rates on the harmful subset. This confirms that safety shift tend to compromise helpfulness. This is more pronounced in MOSSBench than MM-SafetyBench due to the more apparent harmfulness and concealed harmlessness in MOSSBench queries.

\paragraph{Harmfulness discrimination defenses mitigate over-defense.} Query refactoring methods, except for Caption (w/o image), generally achieve the highest response rates on the benign subset, particularly for MOSSBench with misleadingly benign queries. This validates that harmfulness discrimination improves the model's ability to distinguish between truly harmful and benign queries. Notably, the removal of images in the Caption (w/o image) significantly reduces response rates for both harmful and benign queries, highlighting the crucial role images play in jailbreaking LVLMs.
% \paragraph{Image matters.} The removal of images in the Caption (w/o image) and Intention (w/o image) defenses leads to significant improvements in DSR compared to their image-included counterparts, underscoring the crucial role that images play in jailbreaking LVLMs.

\paragraph{Multimodal defense is challenging.}
However, all individual defense methods still exhibit limited defense success rates. While larger-scale LVLMs (i.e., LLaVA-1.5-13B) tend to achieve slightly higher success rates, they are also more susceptible to over-defense. This underscores the inherent challenges of jailbreak defense for LVLMs, especially when relying on individual defense methods. 

\subsection{Ensemble Defense Results}
Table~\ref{tab:en_inter_results} provides the empirical evaluation of both inter-mechanism and intra-mechanism ensemble strategies, leading to the following insights:

\paragraph{Ensembles improve safety.} Compared to individual methods, most ensemble strategies effectively enhance safety across both datasets and model sizes, showing increased defense success rates, especially in \textit{SR+MO} and \textit{QR\textbar{}SR} methods.

\paragraph{Inter-mechanism ensembles amplify.} Our evaluation shows most \textit{SR++} and \textit{SR+MO} ensembles improve defense success rates while reducing responses rates, whereas the \textit{QR++} ensemble better maintain responses rates. This confirms that inter-mechanism ensembles can amplify a single defense mechanism. Specifically, safety shift ensembles would further enhance model safety at the expense of helpfulness, while harmfulness discrimination ensemble better preserves helpfulness. Among inter-mechanism ensembles, those combining different types of specific methods (e.g., SR+MO) show a more pronounced amplification effect than those combining the same type (e.g., SR++). 
Notably, the Demonstration-SFT method excels in defense strength, utility, and response rate. Its success comes from combining two strong safety shift defenses, Demonstration and SFT, which complement each other and boost overall performance.

\paragraph{Intra-mechanism ensembles complement.} Compared to inter-mechanism ensembles, most \textit{QR\textbar{}SR} and \textit{QR\textbar{}MO} methods—except those without input images—can simultaneously maintain decent defense success rates and stable response rates,
compared to the undefended model and individual defense methods. This demonstrates that intra-mechanism ensemble can complement each other to achieve a more balanced trade-off. Additionally, the removal of input images offering a most conservative ensemble for multimodal defense while still maintaining certain helpfulness.
% In contrast, the defenses in intra-mechanism ensemble complement each other, strengthening safety while maintaining a stable level of helpfulness.
% In contrast, intra-mechanism ensembles combine the strengths of both mechanisms to achieve a more balanced trade-off. Specifically, \textit{QR\textbar{}SR} and \textit{QR\textbar{}MO} increase the refusal probability for harmful queries, while maintaining or even decreasing the refusal probability for benign queries, thereby improving the model's ability to distinguish between benign and harmful queries. This makes them a better choice for general scenarios where balancing safety and helpfulness is essential. 


\subsection{How Do Fine-tuning Affect Model Safety?}
We examine how different fine-tuning methods impact the safety of LVLMs by training LLaVA-1.5-7B using DPO and SFT with two datasets: SPA-VL~\cite{zhang2024spa} and VLGuard~\cite{zong2024safety}. SPA-VL focuses on safety discussions, while VLGuard emphasizes query rejection. We also test the effect of adding 5000 general instruction-following data from LLaVA.  

Table~\ref{tab:training_dataset_results} shows that DPO with SPA-VL and LLaVA provides a slight safety boost without significantly changing response behavior. In contrast, SFT has a stronger impact, but its effectiveness depends on the dataset. SPA-VL improves safety while maintaining helpfulness, though it may miss some harmful cases. VLGuard, however, makes the model overly defensive, rejecting too many queries. Adding LLaVA data helps balance safety and helpfulness, reducing excessive refusals.  


\begin{table}[ht]
    \centering
    \resizebox{0.49\textwidth}{!}{
    \begin{tabular}{r|cccccc}
        \toprule 
        & \multicolumn{3}{c}{\textbf{MM-SafetyBench}} & \multicolumn{3}{c}{\textbf{MOSSBench}} \\
        \textbf{Method} & \textbf{DSR}$\uparrow$ & \textbf{RR}$\uparrow$ & \textbf{Avg}$\uparrow$ & \textbf{DSR}$\uparrow$ & \textbf{RR}$\uparrow$ & \textbf{Avg}$\uparrow$ \\
        \midrule
        w/o Defense          & 0.06  & 0.98  & 0.52  & 0.14  & 0.97  & 0.55 \\
        \midrule
        \multicolumn{7}{c}{DPO} \\
        \midrule
        \multicolumn{1}{l|}{SPA-VL + LLaVA}          & 0.06  & 0.97  & 0.52  & 0.28  & 0.97  & 0.63  \\
        \midrule
        \multicolumn{7}{c}{SFT} \\
        \midrule
        \multicolumn{1}{l|}{SPA-VL}          & 0.24  & 0.96  & 0.60  & 0.58  & 0.78  & 0.68  \\
        + LLaVA     & 0.20  & 0.95  & 0.58  & 0.50  & 0.88  & 0.69  \\
        \midrule
        \multicolumn{1}{l|}{VLGuard}          & 1.00  & 0.09  & 0.55  & 0.90  & 0.21  & 0.55  \\
        + LLaVA     & 0.97  & 0.43  & 0.70  & 0.76  & 0.58  & 0.67  \\
        \bottomrule
    \end{tabular}}
    \caption{Comparison of varying fine-tuning settings.} % and the full score is 100\%
    \label{tab:training_dataset_results}
\end{table}


% In this work, we propose WildLong, a novel framework for synthesizing diverse, scalable, and realistic instruction-response datasets designed for long-context tasks. Our approach addresses key challenges in dataset creation by leveraging meta-information extraction from real-world user queries, graph-based modeling of co-occurrence relationships, and adaptive instruction-response generation.
% WildLong is built on the principles of diversity, scalability, and realism, enabling it to support complex reasoning tasks such as cross-document comparison, and aggregation, which are essential for real-world applications. By integrating meta-information into the data generation process and systematically exploring new combinations through graph-based modeling, WildLong generates diverse datasets that reflect the complexity of extended contexts.
% Experimental results demonstrate that WildLong significantly improves long-context task performance, surpassing other open-source long-context-optimized models across multiple benchmarks. Importantly, this improvement is achieved without requiring supplementary short-context instruction tuning, highlighting the robustness and generalizability of our approach.
% The success of WildLong highlights the potential of structured, meta-information-driven data synthesis to enhance the capabilities of LLMs for complex, real-world tasks. By addressing the critical gaps in long-context dataset diversity and quality, WildLong sets a new standard for long-context instruction tuning and paves the way for further advancements in equipping LLMs to tackle the challenges of extended-context reasoning.
% We propose WildLong, a framework for synthesizing diverse, scalable, and realistic instruction-response datasets for long-context tasks. By leveraging meta-information extraction, graph-based modeling, and adaptive instruction generation, WildLong generates long-context instruction-tuning data with real-world complexity.
% Experiments show improved long-context task performance while retaining short-context performance without additional short-context fine-tuning, demonstrating its robustness and generalizability. We hope WildLong provides insights into generalizing instruction tuning and inspires further advancements in long-context reasoning for LLMs.
We propose WildLong, a framework for synthesizing diverse, scalable, and realistic instruction-response datasets for long-context tasks. 
It integrates meta-information extraction to ensure realistic complexity, graph-based modeling for systematic instruction expansion, and adaptive instruction generation for enhanced contextual relevance.
Our fine-tuned models consistently outperform baselines and maintain short-context performance without mixing short-context data. Notably, our finetuned Llama-3.1-8B model surpasses most open-source long-context models on Longbench-Chat and demonstrates competitive performances with even larger models across benchmarks.
WildLong enables the synthesis of instruction-tuning data that produces robust models capable of handling diverse long-context tasks. Extending beyond synthetic QA and summarization, it bridges the gap to more complex, realistic challenges, advancing the effectiveness of long-context LLMs.
We hope WildLong provides insights into generalizing synthetic data and inspires further progress in long-context reasoning for LLMs.

\section*{Impact Statement}
This paper offers a novel perspective, demonstrating the indispensable role of CoT in enhancing the generalization capabilities of LMs. Through theoretical analysis and comprehensive empirical experimentation, we establish CoT as a critical enabler of robust out-of-distribution generalization. Crucially, this work provides valuable guidance for the development of effective data curation strategies, specifically for collecting data that maximizes the benefits of CoT training. This guidance is directly applicable to the industrial deployment of LMs and the fine-tuning of large models for novel tasks, offering a pathway to improve the generalization and real-world utility of these models through informed data acquisition methodologies.
% In the unusual situation where you want a paper to appear in the
% references without citing it in the main text, use \nocite

\printbibliography
%\bibliography{reference}




%%%%%%%%%%%%%%%%%%%%%%%%%%%%%%%%%%%%%%%%%%%%%%%%%%%%%%%%%%%%%%%%%%%%%%%%%%%%%%%
%%%%%%%%%%%%%%%%%%%%%%%%%%%%%%%%%%%%%%%%%%%%%%%%%%%%%%%%%%%%%%%%%%%%%%%%%%%%%%%
% APPENDIX
%%%%%%%%%%%%%%%%%%%%%%%%%%%%%%%%%%%%%%%%%%%%%%%%%%%%%%%%%%%%%%%%%%%%%%%%%%%%%%%
%%%%%%%%%%%%%%%%%%%%%%%%%%%%%%%%%%%%%%%%%%%%%%%%%%%%%%%%%%%%%%%%%%%%%%%%%%%%%%%
\newpage
\appendix
\onecolumn
\section{Explain Compound task in formal definition}
As shown in the figure, original tree like data structure can be converted to an array which is easily feeded into language model.
\begin{figure}
    \centering
    \includegraphics[width=1\linewidth]{figure/cp_inro2.pdf}
    \caption{Step by Step explanation on definition \ref{def:CP}}
    \label{fig:cp2}
\end{figure}
\section{Code and Data}
We provide an anonymous page, you can following the instruction to generate data, training model and evaluation.
https://github.com/physicsru/Scaling-Curves-of-CoT-Granularity-for-Language-Model-Generalization

\section{Recap Condition Analysis}

\begin{theorem}[Outside Token Recap Condition under RoPE]
\label{thm: Recap 1}
Consider a transformer with Rotary Positional Embedding (RoPE) using angles $\theta_j = 10000^{-2j/d_{\text{model}}}$. Given finite computational precision $s$ and minimum resolvable attention score $\epsilon$, there exists a threshold distance $\tau > 0$ such that for all positional distances $d > \tau$:
\[
|A(d)| < \epsilon,
\]
where $A(d)$ is the attention score between tokens at distance $d$. Consequently, tokens beyond $[i_{\text{current}} - \tau, i_{\text{current}} + \tau]$ cannot be recalled.

\begin{proof}
\textbf{Step 1: Attention Score Formulation}  
The RoPE attention score between positions $m$ and $n$ (distance $d = |m-n|$) is:
\[
A(d) = \text{Re}\left[\sum_{j=0}^{d_{\text{model}}/2-1} h_j e^{\mathbf{i}d\theta_j}\right], \quad h_j := q_{[2j:2j+1]}\mathbf{k}^*_{[2j:2j+1]}
\]
where $h_j$ encodes query-key interactions for the $j$-th dimension pair.

\textbf{Step 2: Abel Transformation\cite{men2024baseropeboundscontext}}  
Let $S_{j+1} = \sum_{k=0}^j e^{\mathbf{i}d\theta_k}$ with $S_0 = 0$. Using summation by parts:
\[
\sum_{j=0}^{d_{\text{model}}/2-1} h_j e^{\mathbf{i}d\theta_j} = \sum_{j=0}^{d_{\text{model}}/2-1} (h_j - h_{j+1}) S_{j+1}
\]
Taking absolute values:
\[
|A(d)| \leq \sum_{j=0}^{d_{\text{model}}/2-1} |h_{j+1} - h_j| \cdot |S_{j+1}|
\]

\bigskip
\textbf{Step 3: Bounding Query-Key Differences.}  
Assume that the query and key representations are uniformly bounded so that $\|q\|,\,\|k\| \le C$. In particular, since each $h_j$ results from a dot product between sub-vectors from $q$ and $k$, we have $|h_j| \le C^2$. Moreover, if we assume that the embeddings vary smoothly with the index $j$ (as expected from the continuity of underlying network nonlinearities and weight matrices), then the mean value theorem implies that the difference
\[
|h_{j+1} - h_j|
\]
is bounded by a Lipschitz constant. That is, there exists a constant $M = \mathcal{O}(C^2)$ such that for every $j$
\[
|h_{j+1} - h_j| \le M.
\]
A more refined analysis might track this difference in terms of the network’s smoothness, but the key point is that each difference is uniformly bounded by a constant depending on $C$.

\bigskip
\textbf{Step 4: Analyzing Oscillatory Sums.}  
We now study the partial sum
\[
S_{j+1} = \sum_{k=0}^j e^{\mathbf{i}d\theta_k}.
\]
Two regimes are considered:

\emph{(i) Low-frequency regime ($k \le j_0$):}  
For sufficiently small indices $k$, we have $\theta_k$ being relatively large so that
\[
d\theta_k \gg 1.
\]
In this regime the phases $e^{\mathbf{i}d\theta_k}$ change rapidly with $k$, leading to cancellations among the terms. A standard bound for such oscillatory sums yields
\[
\Big|\sum_{k=0}^{j_0} e^{\mathbf{i}d\theta_k}\Big| \le \frac{2}{\left|1 - e^{\mathbf{i}d\theta_0}\right|} = \mathcal{O}(1),
\]
since the denominator remains bounded away from zero when $d\theta_0$ is large.

\emph{(ii) High-frequency regime ($k > j_0$):}  
For larger $k$, the angle $\theta_k$ becomes very small (because $\theta_k$ decays exponentially with $k$), so that $d\theta_k \ll 1$. In this case we can use the first-order Taylor expansion:
\[
e^{\mathbf{i}d\theta_k} \approx 1 + \mathbf{i}d\theta_k.
\]
Thus, for indices $j>j_0$, the sum becomes approximately
\[
\sum_{k=j_0+1}^j e^{\mathbf{i}d\theta_k} \approx (j-j_0) + \mathbf{i}d\sum_{k=j_0+1}^j \theta_k.
\]
While the real part grows (almost) linearly in the number of terms, the alternating phases and the small magnitude of $d\theta_k$ imply that additional cancellation occurs when the two regimes are combined. In the worst case one may bound
\[
|S_{j+1}| \le \mathcal{O}(\log d)
\]
by appealing to harmonic series type estimates; this bound is loose but captures the fact that the cancellation improves with larger $d$.

\bigskip
\textbf{Step 5: Combining the Results.}  
Returning to the bound from Step~2, we have
\[
|A(d)| \le \sum_{j=0}^{d_{\text{model}}/2-1} |h_{j+1} - h_j|\, |S_{j+1}|
\]
and using the bounds from Steps~3 and 4,
\[
|A(d)| \le M \sum_{j=0}^{d_{\text{model}}/2-1} \mathcal{O}(\log d) = \mathcal{O}(M\, d_{\text{model}} \log d).
\]
In practice, the alternating signs in the summands produce even stronger decay in $d$, and empirical observations suggest a polynomial decay of the form
\[
|A(d)| \le \mathcal{O}\Big(\frac{M}{\sqrt{d}}\Big).
\]

\bigskip
\textbf{Step 6: Precision Threshold.}  
For a given minimum resolvable attention score $\epsilon$, we require
\[
\frac{M}{\sqrt{d}} < \epsilon \quad \Longrightarrow \quad d > \left(\frac{M}{\epsilon}\right)^2.
\]
Defining
\[
\tau = \left(\frac{M}{\epsilon}\right)^2,
\]
we conclude that for any $d > \tau$, it holds that $|A(d)| < \epsilon$. That is, tokens beyond the window indexed by $[i_{\text{current}}-\tau,\, i_{\text{current}}+\tau]$ effectively contribute an attention score below the resolvable threshold.

\end{proof}

\textbf{Interpretation:}
\begin{itemize}
\item The $\theta_j$ schedule creates frequency-dependent decay: high frequencies (small $j$) attenuate rapidly
\item Window size $\tau \propto (M/\epsilon)^2$ explains memory limitations in long contexts
\item Practical implementations must balance $d_{\text{model}}$ and precision $s$ for optimal $\tau$
\end{itemize}
\end{theorem}

\begin{theorem}[Inside window recap condition under Rope]
\label{thm: Recap 2}
Assume: The true function is $s_2 = {w^{\circ}}^{\top} e(s_1) + \epsilon$, with $s_2 \perp s_x$ in expectation.
Consider two linear models:
\begin{itemize}
\item $M_s$: $y_s = {w_s}^{\top} e(s_1)$
\item $M_l$: $y_l = {w_1}^{\top} e(s_1) + {w_2}^{\top} e(s_x)$
\end{itemize}
Under mean-squared-error training:
\begin{itemize}
\item $g_s = \frac{\partial L_s}{\partial w_s}$
\item $(g_1, g_2) = \left(\frac{\partial L_l}{\partial w_1}, \frac{\partial L_l}{\partial w_2}\right)$
\end{itemize}
Then in finite-data regimes or with random noise, the gradient component of $g_1$ along $(w^{\circ} - w_1)$ is typically smaller than the corresponding component of $g_s$ along $(w^{\circ} - w_s)$
Formally,
\[
\mathbb{E}\left[ \left\langle \mathbf{g}_1, \mathbf{w}^\circ - \mathbf{w}_1 \right\rangle \right] < \mathbb{E}\left[ \left\langle \mathbf{g}_s, \mathbf{w}^\circ - \mathbf{w}_s \right\rangle \right],
\]
leading to slower convergence for \( \mathcal{M}_{l} \) when \( s_{x} \) is irrelevant.
\end{theorem}
%\input{proof_inside_recap}

\begin{proof}
We analyze the gradients of both models and demonstrate how irrelevant features in \( \mathcal{M}_l \) reduce the effective gradient signal.

\noindent \textbf{Step 1: Express Gradients for Both Models}

For model \( \mathcal{M}_s \), the loss is:
\[
L_s = \mathbb{E}\left[(s_2 - \mathbf{w}_s^\top \mathbf{e}(s_1))^2\right]
\]
The gradient becomes:
\begin{equation}
\mathbf{g}_s = -2 \mathbb{E}\left[\mathbf{e}(s_1)\mathbf{e}(s_1)^\top\right](\mathbf{w}^\circ - \mathbf{w}_s)
\label{eq:grad_s}
\end{equation}

For model \( \mathcal{M}_l \), the gradient for \( \mathbf{w}_1 \) is:
\begin{equation}
\mathbf{g}_1 = -2 \mathbb{E}\left[\mathbf{e}(s_1)\mathbf{e}(s_1)^\top\right](\mathbf{w}^\circ - \mathbf{w}_1) + 2 \mathbb{E}\left[\mathbf{w}_2^\top \mathbf{e}(s_x)\mathbf{e}(s_1)\right]
\label{eq:grad_l}
\end{equation}

\noindent \textbf{Step 2: Compare Gradient Components}

The inner product for \( \mathcal{M}_l \) contains two terms:
\begin{equation}
\begin{aligned}
    \mathbb{E}\left[\langle \mathbf{g}_1, \mathbf{w}^\circ - \mathbf{w}_1 \rangle \right] 
    &= \underbrace{-2 (\mathbf{w}^\circ - \mathbf{w}_1)^\top \mathbb{E}\left[\mathbf{e}(s_1)\mathbf{e}(s_1)^\top\right] (\mathbf{w}^\circ - \mathbf{w}_1)}_{\text{Matches } \mathcal{M}_s} \\
    &\quad + \underbrace{2 \mathbb{E}\left[\mathbf{w}_2^\top \mathbf{e}(s_x)\mathbf{e}(s_1)^\top (\mathbf{w}^\circ - \mathbf{w}_1)\right]}_{\text{Additional term}}
\end{aligned}
\end{equation}


\noindent \textbf{Step 3: Effect of Irrelevant Features}

Since \( s_x \) is irrelevant (\( s_2 \perp s_x \)):
\begin{itemize}
\item Population truth: \( \mathbf{w}_2 = \mathbf{0} \)
\item Finite data allows \( \mathbf{w}_2^\epsilon \) to fit noise \( \epsilon \)
\item Induces spurious correlation: \( \mathbb{E}\left[\mathbf{e}(s_x)\mathbf{e}(s_1)^\top\right] \neq \mathbf{0} \)
\end{itemize}

This makes the additional term:
\[
2 \mathbb{E}\left[\mathbf{w}_2^\top \mathbf{e}(s_x)\mathbf{e}(s_1)^\top (\mathbf{w}^\circ - \mathbf{w}_1)\right] \neq 0
\]
which \textit{reduces} the magnitude of the gradient component.

\noindent \textbf{Step 4: Convergence Comparison}

For \( \mathcal{M}_s \):
\[
\mathbb{E}\left[\langle \mathbf{g}_s, \mathbf{w}^\circ - \mathbf{w}_s \rangle \right] = -2 (\mathbf{w}^\circ - \mathbf{w}_s)^\top \mathbb{E}\left[\mathbf{e}(s_1)\mathbf{e}(s_1)^\top\right] (\mathbf{w}^\circ - \mathbf{w}_s)
\]

For \( \mathcal{M}_l \), the additional term in Equation~\ref{eq:grad_l} creates:
\[
\mathbb{E}\left[ \left\langle \mathbf{g}_1, \mathbf{w}^\circ - \mathbf{w}_1 \right\rangle \right] < \mathbb{E}\left[ \left\langle \mathbf{g}_s, \mathbf{w}^\circ - \mathbf{w}_s \right\rangle \right]
\]

\noindent \textbf{Conclusion} \\
The irrelevant features in \( \mathcal{M}_l \) reduce the gradient component in the direction of \( \mathbf{w}^\circ \), leading to slower convergence compared to \( \mathcal{M}_s \).
\end{proof}


\section{CoT simulates the target solution}

\begin{definition}[Chain of Thought]
  \vspace{0em}
    \small\begin{align*}
    \vspace{-0.5em}
        \text{Input:} & \; s_1 \mid \cdots \mid s_N \\[1ex]
        \text{CoT Steps:} & \; \langle\text{sep}\rangle \; s_2 \mid G_2 \mid q_2 \mid L_1 \mid L_2 \\
        & \; \langle\text{sep}\rangle \; \cdots \\
        & \; \langle\text{sep}\rangle \; s_{N} \mid G_N \mid q_N \mid L_{N-1} \mid L_N
    \end{align*}
    \label{def:CoT_appendix}
\end{definition}

\begin{proposition}
\label{thm:CoT_implementation}
For any compound problem satisfying Definition \ref{def:CP}, and for any input length bound $n \in \mathbb{N}$, there exists an autoregressive Transformer with:
\begin{itemize}
\item Constant depth $L$
\item Constant hidden dimension $d$
\item Constant number of attention heads $H$
\end{itemize}
where $L$, $d$, and $H$ are independent of $n$, such that the Transformer correctly generates the Chain-of-Thought solution defined in Definition \ref{def:CoT} for all input sequences of length at most $n$. Furthermore, all parameter values in the Transformer are bounded by $O(\text{poly}(n))$.
\end{proposition}

\subsection{Constructive Proof}
We prove this theorem by constructing a Transformer architecture with 4 blocks, where each block contains multiple attention heads and feed-forward networks (FFNs). The key insight is that we can simulate each step of the Chain-of-Thought solution using a fixed number of attention heads and a fixed embedding dimension.
The attention mechanism is primarily used to select and retrieve relevant elements from the input and previous computations, while the FFNs approximate the required functions $G$, $B$, etc. By maintaining constant depth, width, and number of heads per layer, we ensure the Transformer's architecture remains independent of the input length, while still being able to generate arbitrarily long Chain-of-Thought solutions.
The parameter complexity of $O(\text{poly}(n))$ arises from the need to handle inputs and intermediate computations of length $n$, but importantly, this only affects the parameter values and not the model architecture itself.

\subsection{Embedding Structure}
For position $k$, define the input embedding:
\begin{equation*}
    x^{(0)}_k=(e^{\text{isInput}}_k, e^{\text{isState}}_k, e^{\text{isDependence}}, e^{\text{isL}}, e^{\text{q}}_k, e^{\text{d}}_{k}, e^{\text{L}}_k, e^{\text{sep}}_k, e^{\text{step}}_k, k,1)
\end{equation*}
where:
\begin{itemize}
    \item $e^{\text{isInput}}_k \in \{0,1\}$: Input token indicator
    \item $e^{\text{isState}}_k \in \{0,1\}$: State position indicator
    \item $e^{\text{isDependence}} \in \{0,1\}$: Dependency marker
    \item $e^{\text{isL}} \in \{0,1\}$: Aggregation result indicator
    \item $e^{\text{q}}_k \in \mathbb{R}^{d_q}$: State value embedding
    \item $e^{\text{d}}_{k} \in \mathbb{R}^{d_d}$: Dependency graph embedding
    \item $e^{\text{L}}_k \in \mathbb{R}^{d_L}$: Aggregation value embedding
    \item $e^{\text{sep}}_k \in \{0,1\}$: Step separator indicator
    \item $e^{\text{step}}_k \in \mathbb{N}$: Current step index
    \item $k \in \mathbb{N}$: Position encoding
    \item $1$: Bias term
\end{itemize}

\subsection{Block Constructions}

\subsubsection{Block 1: Input Processing and State Identification}
Define attention heads $A^{(1)}_1, A^{(1)}_2, A^{(1)}_3$ with parameters:
\begin{align*}
    Q^{(1)}_1 &= W^q_1[e^{\text{isInput}}_k] \\ 
    K^{(1)}_1 &= W^k_1[e^{\text{isInput}}_j]_{j<k} \\
    V^{(1)}_1 &= W^v_1[j]_{j<k}
\end{align*}

The second head tracks state positions:
\begin{align*}
    Q^{(1)}_2 &= W^q_2[e^{\text{isState}}_k] \\
    K^{(1)}_2 &= W^k_2[e^{\text{isState}}_j]_{j<k} \\
    V^{(1)}_2 &= W^v_2[j]_{j<k}
\end{align*}

The third head tracks step indices through separators:
\begin{align*}
    Q^{(1)}_3 &= W^q_3[e^{\text{sep}}_k] \\
    K^{(1)}_3 &= W^k_3[e^{\text{sep}}_j]_{j<k} \\
    V^{(1)}_3 &= W^v_3[\text{count}(e^{\text{sep}}_j)]_{j<k}
\end{align*}

\begin{lemma}
The first block correctly identifies positions through attention scoring:
\begin{enumerate}
    \item For input positions, $A^{(1)}_1$ scoring gives:
    \begin{equation*}
        \text{score}_1(q_k, k_j) = \begin{cases}
        1 & \text{if } e^{\text{isInput}}_j = 1 \\
        0 & \text{otherwise}
        \end{cases}
    \end{equation*}
    Thus $V^{(1)}_1$ returns positions of input tokens

    \item For state positions, $A^{(1)}_2$ scoring gives:
    \begin{equation*}
        \text{score}_2(q_k, k_j) = \begin{cases}
        1 & \text{if } e^{\text{isState}}_j = 1 \\
        0 & \text{otherwise}
        \end{cases}
    \end{equation*} 
    Thus $V^{(1)}_2$ returns positions of states

    \item For step indices, $A^{(1)}_3$ counts separators up to position k:
    \begin{equation*}
        \text{count}(e^{\text{sep}}_j) = \sum_{l \leq j} e^{\text{sep}}_l
    \end{equation*}
    Thus $V^{(1)}_3$ returns the current step index
\end{enumerate}
\end{lemma}

\subsubsection{Block 2: Dependency Graph Construction} 
Define three attention heads $A^{(2)}_1, A^{(2)}_2, A^{(2)}_3$ implementing dependency selection:
\begin{align*}
    A^{(2)}_1&: Q^{(2)}_1 = W^q_2[e^{\text{step}}_k] \\
    &K^{(2)}_1 = W^k_2[e^{\text{input}}_j]_{j<k} \\
    &V^{(2)}_1 = W^v_2[j]_{j<k} \\
    A^{(2)}_2&: Q^{(2)}_2 = W^q_3[e^{\text{step}}_k] \\
    &K^{(2)}_2 = W^k_3[e^{\text{step}}_j]_{j<k} \\
    &V^{(2)}_2 = W^v_3[B(s_1,\ldots,s_{i+1}, i+1)]_{j<k} \\
    A^{(2)}_3&: Q^{(2)}_3 = W^q_4[e^{\text{step}}_k] \\
    &K^{(2)}_3 = W^k_4[j]_{j<k} \\
    &V^{(2)}_3 = W^v_4[e^{\text{q}}_j]_{j<k}
\end{align*}

\begin{lemma}
Block 2 correctly implements $G_{i+1} = \{q_k | k \in B(s_1,\ldots,s_{i+1}, i+1)\}$ through:

1. First attention head $A^{(2)}_1$ gathers input sequence up to current step i+1:
\begin{equation*}
    z^{(2)}_1 = \{s_j | j \leq i+1\}
\end{equation*}

2. Second attention head $A^{(2)}_2$ computes indices from B using gathered inputs:
\begin{equation*}
    z^{(2)}_2 = B(z^{(2)}_1, i+1)
\end{equation*}

3. Third attention head $A^{(2)}_3$ selects states using computed indices:
\begin{equation*}
    z^{(2)}_3 = \{e^{\text{q}}_j | j \in z^{(2)}_2\}
\end{equation*}

Therefore, the composition $z^{(2)}_3(z^{(2)}_2(z^{(2)}_1))$ correctly implements $G_{i+1}$ by:
\begin{enumerate}
    \item Gathering relevant input sequence 
    \item Computing dependency indices using B
    \item Selecting corresponding states
\end{enumerate}

The correctness follows from attention scoring:
\begin{align*}
    \text{score}_1(q_k, k_j) &= \begin{cases}
        1 & \text{if } j \leq i+1 \\
        0 & \text{otherwise}
    \end{cases} \\
    \text{score}_2(q_k, k_j) &= \begin{cases}
        1 & \text{if } j \in B(s_1,\ldots,s_{i+1}, i+1) \\
        0 & \text{otherwise}
    \end{cases} \\
    \text{score}_3(q_k, k_j) &= \begin{cases}
        1 & \text{if } j \in z^{(2)}_2 \\
        0 & \text{otherwise}
    \end{cases}
\end{align*}
\end{lemma}

\subsubsection{Block 3: State Transition}
Define attention mechanism implementing $F$:
\begin{align*}
    A^{(3)}_1&: Q^{(3)}_1 = W^q_3[e^{\text{isState}}_k] \\
    &K^{(3)}_1 = W^k_3[e^{\text{isDependence}}_j]_{j<k} \\
    &V^{(3)}_1 = W^v_3[e^{\text{q}}_j]_{j<k} \\
    A^{(3)}_2&: Q^{(3)}_2 = W^q_4[e^{\text{isState}}_k] \\
    &K^{(3)}_2 = W^k_4[e^{\text{isInput}}_j]_{j<k} \\
    &V^{(3)}_2 = W^v_4[e^{\text{input}}_j]_{j<k}
\end{align*}

\begin{lemma}
The state transition function $F$ is correctly computed through:
\begin{equation*}
    q_{i+1} = F(G_{i+1}, s_{i+1}) = \text{FFN}(z^{(3)}_1, z^{(3)}_2)
\end{equation*}
where $z^{(3)}_1 = A^{(3)}_1(e^{\text{q}}_j \mid j \in B(s_1,\ldots,s_{i+1}, i+1))$ represents the states selected by $G_{i+1}$ from Block 2, and $z^{(3)}_2 = A^{(3)}_2(s_{i+1})$ represents the current input token.
\end{lemma}

\subsubsection{Block 4: Result Aggregation}
Define two attention heads $A^{(4)}_1, A^{(4)}_2$ for implementing $H$:
\begin{align*}
    A^{(4)}_1&: Q^{(4)}_1 = W^q_4[e^{\text{isL}}_k] \\
    &K^{(4)}_1 = W^k_4[e^{\text{isL}}_j]_{j<k} \\
    &V^{(4)}_1 = W^v_4[e^{\text{L}}_j]_{j<k} \\
    A^{(4)}_2&: Q^{(4)}_2 = W^q_5[e^{\text{isL}}_k] \\
    &K^{(4)}_2 = W^k_5[e^{\text{isState}}_j]_{j<k} \\
    &V^{(4)}_2 = W^v_5[e^{\text{q}}_j]_{j<k}
\end{align*}

\begin{lemma}
Block 4 correctly implements the aggregation function $H$ through:

1. For $i=1$ (base case):
\begin{equation*}
    \text{score}_1(q_k, k_j) = 0, \quad \text{score}_2(q_k, k_j) = \begin{cases}
        1 & \text{if } e^{\text{isState}}_j = 1 \\
        0 & \text{otherwise}
    \end{cases}
\end{equation*}
Therefore $L_1 = H(\emptyset, q_1) = q_1$ since only $A^{(4)}_2$ activates to select $q_1$

2. For $i>1$:
\begin{equation*}
    \text{score}_1(q_k, k_j) = \begin{cases}
        1 & \text{if } e^{\text{isL}}_j = 1 \text{ and j is the latest L position} \\
        0 & \text{otherwise}
    \end{cases}
\end{equation*}
\begin{equation*}
    \text{score}_2(q_k, k_j) = \begin{cases}
        1 & \text{if } e^{\text{isState}}_j = 1 \text{ and j corresponds to } q_i \\
        0 & \text{otherwise}
    \end{cases}
\end{equation*}

Therefore:
\begin{align*}
    z^{(4)}_1 &= A^{(4)}_1(e^{\text{L}}_k) = L_{i-1} \text{ (previous aggregation result)} \\
    z^{(4)}_2 &= A^{(4)}_2(e^{\text{q}}_k) = q_i \text{ (current state)} \\
    L_i &= \text{FFN}(z^{(4)}_1, z^{(4)}_2) = H(L_{i-1}, q_i)
\end{align*}

The FFN is constructed to implement the specific aggregation operation of $H$ (e.g., max, min, or sum).
\end{lemma}

\begin{proposition}[Block Transitions]
The blocks connect sequentially where:
\begin{enumerate}
    \item Block 1 output provides input positions, state positions and step indices
    \item Block 2 implements dependency function $G$ to gather required states
    \item Block 3 uses gathered dependencies and current input to compute new states via $F$
    \item Block 4 implements $H$ to aggregate states into final result
\end{enumerate}
Each transition preserves information through residual connections.
\end{proposition}

\section{Proof for Theorem \ref{thm:kl-reduction}}


\begin{proof}
We prove the KL divergence bound by decomposing the distributions over covered and uncovered prefixes. Let $P_{\text{train}}^{\text{Q-CoT}}$ and $P_{\text{eval}}^{\text{Q-CoT}}$ denote the training and evaluation distributions under Q-CoT, respectively. 

\vspace{0.5em}

\noindent \textbf{Step 1: Event Space Partitioning}

Define two disjoint events for any evaluation sample $x = (X^{n_3}, \{q_i^{(n_3)}\}, Y^{n_3})$:
\begin{itemize}
    \item $\mathcal{E}_{\text{cover}}$: The prefix $\{q_i^{(n_3)}\}_{i=1}^{n_3}$ exists in some length-$n_2$ training sample.
    \item $\mathcal{E}_{\text{uncover}}$: The prefix $\{q_i^{(n_3)}\}_{i=1}^{n_3}$ is absent from all training samples.
\end{itemize}
By Lemma \ref{thm:prefix-substructure}, $\mathcal{E}_{\text{cover}}$ occurs when the evaluation prefix matches at least one length-$n_2$ training sequence's prefix. The probabilities satisfy:
\[
P_{\text{cover}} = \mathbb{P}(\mathcal{E}_{\text{cover}}), \quad 1 - P_{\text{cover}} = \mathbb{P}(\mathcal{E}_{\text{uncover}}).
\]
where $P_{\text{cover}}$ is calculated as:
\[
P_{\text{cover}} = \frac{m_2}{m_3 k^{n_3}}
\]
\textit{Derivation}: Each length-$n_2$ training sample contains a unique prefix of length $n_3$ (Lemma \ref{thm:prefix-substructure}). With $m_2$ samples, we can cover $m_2$ distinct prefixes. The total number of possible prefixes is $m_3 k^{n_3}$ ($m_3$ evaluation problems, each with $k^{n_3}$ possible prefixes). Thus, the coverage probability follows the ratio.

\vspace{0.5em}

\noindent \textbf{Step 2: Distributional Decomposition}

Using the law of total probability, we express:
\[
P_{\text{eval}}^{\text{Q-CoT}} = P_{\text{cover}} \cdot P_{\text{eval}|\mathcal{E}_{\text{cover}}} + (1-P_{\text{cover}}) \cdot P_{\text{eval}|\mathcal{E}_{\text{uncover}}}
\]
\[
P_{\text{train}}^{\text{Q-CoT}} = P_{\text{cover}} \cdot P_{\text{train}|\mathcal{E}_{\text{cover}}} + (1-P_{\text{cover}}) \cdot P_{\text{train}|\mathcal{E}_{\text{uncover}}}
\]
where:
\begin{itemize}
    \item $P_{\text{eval}|\mathcal{E}_{\text{cover}}}$: Evaluation distribution restricted to covered prefixes
    \item $P_{\text{train}|\mathcal{E}_{\text{cover}}}$: Training distribution restricted to covered prefixes
    \item $P_{\text{eval}|\mathcal{E}_{\text{uncover}}}$: Evaluation distribution for uncovered prefixes
    \item $P_{\text{train}|\mathcal{E}_{\text{uncover}}}$: Training distribution for uncovered prefixes
\end{itemize}

\vspace{0.5em}

\noindent \textbf{Step 3: KL Divergence Expansion with Total Expectation}

From the KL divergence definition:
\[
D_{\mathrm{KL}}\left(P_{\text{eval}}^{\text{Q-CoT}} \,\big\|\, P_{\text{train}}^{\text{Q-CoT}}\right) = \mathbb{E}_{x \sim P_{\text{eval}}^{\text{Q-CoT}}} \left[ \log \frac{P_{\text{eval}}^{\text{Q-CoT}}(x)}{P_{\text{train}}^{\text{Q-CoT}}(x)} \right]
\]
Apply the law of total expectation by conditioning on $\mathcal{E}_{\text{cover}}$ and $\mathcal{E}_{\text{uncover}}$:
\[
= \mathbb{P}(\mathcal{E}_{\text{cover}}) \cdot \mathbb{E}_{x|\mathcal{E}_{\text{cover}}} \left[ \log \frac{P_{\text{eval}}^{\text{Q-CoT}}(x|\mathcal{E}_{\text{cover}})}{P_{\text{train}}^{\text{Q-CoT}}(x|\mathcal{E}_{\text{cover}})} \right] + \mathbb{P}(\mathcal{E}_{\text{uncover}}) \cdot \mathbb{E}_{x|\mathcal{E}_{\text{uncover}}} \left[ \log \frac{P_{\text{eval}}^{\text{Q-CoT}}(x|\mathcal{E}_{\text{uncover}})}{P_{\text{train}}^{\text{Q-CoT}}(x|\mathcal{E}_{\text{uncover}})} \right]
\]

\vspace{0.5em}

\noindent \textbf{Step 4: Handling Covered Cases}

Under $\mathcal{E}_{\text{cover}}$, Lemma \ref{thm:prefix-substructure} guarantees that the CoT states $\{q_i^{(n_3)}\}$ in evaluation samples exactly match those in training samples. This implies:
\[
P_{\text{eval}|\mathcal{E}_{\text{cover}}}(x) = P_{\text{train}|\mathcal{E}_{\text{cover}}}(x), \quad \forall x \in \mathcal{E}_{\text{cover}}
\]
Therefore:
\[
\mathbb{E}_{x|\mathcal{E}_{\text{cover}}} \left[ \log \frac{P_{\text{eval}|\mathcal{E}_{\text{cover}}}}{P_{\text{train}|\mathcal{E}_{\text{cover}}}} \right] = \mathbb{E}_{x|\mathcal{E}_{\text{cover}}} [\log 1] = 0
\]

\vspace{0.5em}

\noindent \textbf{Step 5: Uncovered Cases Reduce to Q-A}

For $x = (X^{n_3}, \{q_i^{(n_3)}\}, Y^{n_3}) \in \mathcal{E}_{\text{uncover}}$, the absence of matching prefixes in training data implies the model cannot leverage CoT states $\{q_i^{(n_3)}\}$ during inference. We formally analyze this degradation:

%\vspace{0.5em}

%\noindent **5.1 Conditional Distribution Decomposition**

Under Q-CoT, the generation process factors as:
\[
P^{\text{Q-CoT}}(Y|X) = \sum_{\{q_i\}} P(Y|X, \{q_i\}) P(\{q_i\}|X)
\]
where:
\begin{itemize}
    \item $P(\{q_i\}|X)$: Probability of generating CoT states $\{q_i\}$ given input $X$
    \item $P(Y|X, \{q_i\})$: Probability of answer $Y$ given $X$ and CoT states
\end{itemize}

%\vspace{0.5em}

%\noindent **5.2 Uncovered Case Analysis**

When $\{q_i^{(n_3)}\}$ is uncovered ($\mathcal{E}_{\text{uncover}}$), the model lacks training data to estimate either:
\begin{itemize}
    \item The CoT state distribution $P(\{q_i\}|X)$
    \item The answer likelihood $P(Y|X, \{q_i\})$ 
\end{itemize}

Thus, the model \textit{cannot} utilize the CoT decomposition and must marginalize over all possible $\{q_i\}$:
\[
P^{\text{Q-CoT}}(Y|X) = \mathbb{E}_{\{q_i\} \sim P(\{q_i\}|X)} \left[ P(Y|X, \{q_i\}) \right]
\]

%\vspace{0.5em}

%\noindent **5.3 Degeneration to Q-A**

Without CoT supervision on $\{q_i^{(n_3)}\}$, two condition assumes:
\begin{enumerate}
    \item \textbf{Untrained CoT States}: If $\{q_i^{(n_3)}\}$ never appears in training, $P(\{q_i\}|X)$ becomes a \textit{uniform prior} over possible CoT sequences (by maximum entropy principle).
    
    \item \textbf{Uninformative Likelihood}: The answer likelihood $P(Y|X, \{q_i\})$ reduces to $P^{\text{Q-A}}(Y|X)$ because the model cannot associate $\{q_i\}$ with $Y$ without training signals.
\end{enumerate}

Thus:
\[
P^{\text{Q-CoT}}(Y|X) = \sum_{\{q_i\}} \underbrace{P^{\text{Q-A}}(Y|X)}_{\text{Uninformative}} \cdot \underbrace{\frac{1}{k^{n_{3}}}}_{\text{Uniform } P(\{q_i\}|X)} = P^{\text{Q-A}}(Y|X)
\]

with expansion of KL divergence of Q-A
\[
D_{\mathrm{KL}}\left(P_{\text{eval}}^{\text{Q-A}} \,\big\|\, P_{\text{train}}^{\text{Q-A}}\right) = \mathbb{E}_{x \sim P_{\text{eval}}^{\text{Q-A}}} \left[ \log \frac{P_{\text{eval}}^{\text{Q-A}}(x)}{P_{\text{train}}^{\text{Q-A}}(x)} \right]
\] 
\[
= \mathbb{P}(\mathcal{E}_{\text{cover}}) \cdot \mathbb{E}_{x|\mathcal{E}_{\text{cover}}} \left[ \log \frac{P_{\text{eval}}^{\text{Q-A}}(x|\mathcal{E}_{\text{cover}})}{P_{\text{train}}^{\text{Q-A}}(x|\mathcal{E}_{\text{cover}})} \right] + \mathbb{P}(\mathcal{E}_{\text{uncover}}) \cdot \mathbb{E}_{x|\mathcal{E}_{\text{uncover}}} \left[ \log \frac{P_{\text{eval}}^{\text{Q-A}}(x|\mathcal{E}_{\text{uncover}})}{P_{\text{train}}^{\text{Q-A}}(x|\mathcal{E}_{\text{uncover}})} \right]
\]
Notice that
\[
\mathbb{E}_{x|\mathcal{E}_{\text{cover}}} \left[ \log \frac{P_{\text{eval}}^{\text{Q-A}}(x|\mathcal{E}_{\text{cover}})}{P_{\text{train}}^{\text{Q-A}}(x|\mathcal{E}_{\text{cover}})} \right]
\leq
\mathbb{E}_{x|\mathcal{E}_{\text{uncover}}} \left[ \log \frac{P_{\text{eval}}^{\text{Q-A}}(x|\mathcal{E}_{\text{uncover}})}{P_{\text{train}}^{\text{Q-A}}(x|\mathcal{E}_{\text{uncover}})} \right]
\]
since covered prefix will decrease the KL divergence via probability decomposition
\[
D_{\mathrm{KL}}\left(P_{\text{eval}}^{\text{Q-A}} \,\big\|\, P_{\text{train}}^{\text{Q-A}}\right) \geq 
\mathbb{E}_{x|\mathcal{E}_{\text{uncover}}} \left[ \log \frac{P_{\text{eval}}^{\text{Q-A}}(x|\mathcal{E}_{\text{uncover}})}{P_{\text{train}}^{\text{Q-A}}(x|\mathcal{E}_{\text{uncover}})} \right] = D_{\mathrm{KL}}\left(P_{\text{eval}|\mathcal{E}_{\text{uncover}}}^{\text{Q-A}} \,\big\|\, P_{\text{train}|\mathcal{E}_{\text{uncover}}}^{\text{Q-A}}\right)
\]

Therefore, for $x \in \mathcal{E}_{\text{uncover}}$:
\[
D_{\mathrm{KL}}\left(P_{\text{eval}|\mathcal{E}_{\text{uncover}}} \,\big\|\, P_{\text{train}|\mathcal{E}_{\text{uncover}}}\right) = 
D_{\mathrm{KL}}\left(P_{\text{eval}|\mathcal{E}_{\text{uncover}}}^{\text{Q-A}} \,\big\|\, P_{\text{train}|\mathcal{E}_{\text{uncover}}}^{\text{Q-A}}\right) \leq 
D_{\mathrm{KL}}\left(P_{\text{eval}}^{\text{Q-A}} \,\big\|\, P_{\text{train}}^{\text{Q-A}}\right) = \mathrm{KL}_{\text{base}}
\]

\vspace{0.5em}

\noindent \textbf{Step 6: Final Inequality}

Combining all terms:
\[
D_{\mathrm{KL}}\left(P_{\text{eval}}^{\text{Q-CoT}} \,\big\|\, P_{\text{train}}^{\text{Q-CoT}}\right) = \underbrace{P_{\text{cover}} \cdot 0}_{\text{Covered term}} + \underbrace{(1-P_{\text{cover}}) \cdot \mathrm{KL}_{\text{base}}}_{\text{Uncovered term}}
\]
Hence:
\[
D_{\mathrm{KL}}\left(P_{\text{eval}}^{\text{Q-CoT}} \,\big\|\, P_{\text{train}}^{\text{Q-CoT}}\right) \leq (1 - P_{\text{cover}}) \cdot \mathrm{KL}_{\text{base}}
\]
The equality holds when $P_{\text{cover}} \in [0,1]$. When $m_2 = m_3 k^{n_3}$, we have $P_{\text{cover}} = 1$, making the KL divergence zero.
\end{proof}


\section{Quantitation Analysis Drop of CoT}
    \begin{theorem}[CoT Accuracy Degradation]
    Let $s_{\text{input}}$ be the input text, $s_{\text{ans}}$ be the unique correct answer, and $s_1, \dots, s_k$ be the \textit{exact required sequence} of perfect Chain-of-Thought (CoT) tokens where:
    \begin{enumerate}
        \item \textbf{Completeness}: $P(s_{\text{ans}} \mid s_1, \dots, s_k, s_{\text{input}}) = 1$
        \item \textbf{Uniqueness}: No other token sequence produces $s_{\text{ans}}$
        \item \textbf{Conditional Independence}: $P(s_1, \dots, s_k \mid s_{\text{input}}) = \prod_{i=1}^k P(s_i \mid s_{\text{input}})$
        \item \textbf{Training Deficiency}: For any CoT token $s_j$ excluded during training, $P(s_j \mid s_{\text{input}})$ drops from 1 to $1 - \epsilon$
    \end{enumerate}
    When $l < k$ CoT tokens are lost/mishandled during inference, the final answer accuracy satisfies:
    \[
    P(s_{\text{ans}} \mid s_{\text{input}}) = (1 - \epsilon)^l
    \]
    \label{thm:drop_CoT}
\end{theorem}

\begin{proof}
By the uniqueness condition, only the full sequence $s_1, \dots, s_k$ guarantees $s_{\text{ans}}$. Let $\mathcal{L}$ be the set of $l$ compromised tokens. The probability of maintaining correctness is:

\[
P(s_{\text{ans}} \mid s_{\text{input}}) = \underbrace{\prod_{j \in \mathcal{L}} P(s_j \mid s_{\text{input}})}_{\text{Lost tokens}} \cdot \underbrace{\prod_{i \notin \mathcal{L}} P(s_i \mid s_{\text{input}})}_{\text{Preserved tokens}}
\]

For preserved tokens ($i \notin \mathcal{L}$), full training ensures $P(s_i \mid s_{\text{input}}) = 1$. For lost tokens ($j \in \mathcal{L}$), training deficiency gives $P(s_j \mid s_{\text{input}}) = 1 - \epsilon$. Thus:

\[
P(s_{\text{ans}} \mid s_{\text{input}}) = (1 - \epsilon)^l \cdot 1^{k-l} = (1 - \epsilon)^l
\]

This equality holds because any deviation from the exact CoT sequence (due to lost tokens) eliminates the chance of correctness by the uniqueness condition.
\end{proof}
\subsection{Experiments}
\subsubsection{LIS}
Chain of thought is like the following: 
\[
\begin{aligned}
    &48 \quad 49 \quad 26 \quad 47 <sep> \\
    &48 | <empty> = 48 \quad 1 : 1 \rightarrow 1 <sep>\\
    &49 | 48 \quad 1 = 49 \quad 2 : 1 \rightarrow 2 <sep> \\
    &26 | <empty> = 26 \quad 1 : 2 \rightarrow 2 <sep>\\
    &47 | 26 \quad 1 = 47 \quad 2 : 2 \rightarrow 2
    \end{aligned}
\]

\subsubsection{MPC}
Chain of thought is like following:
\[
\begin{aligned}
 &0 \quad 1 \quad 1 \quad 0 \quad 0 \quad 1 \quad 1 0 , 8 <sep> \\
 &1 , 0 , 1 \rightarrow 0 <sep> \\
 &2 , 1 , 1 \quad 0 \rightarrow 1 <sep> \\
 &3 , 1 , 1 \quad 0 \quad 1 \rightarrow 2 <sep> \\
 &4 , 0 , 0 \quad 1 \quad 2 \rightarrow 0 <sep> \\
 &5 , 0 , 1 \quad 2 \quad 0 \rightarrow 0 <sep> \\
 &6 , 1 , 2 \quad 0 \quad 0 \rightarrow 2 <sep> \\
 &7 , 1 , 0 \quad 0 \quad 2 \rightarrow 2 <sep> \\
 &8 , 0 , 0 \quad 2 \quad 2 \rightarrow 0
\end{aligned}
\]
\subsubsection{Equation Restoration and Variable Computation}
Input is:
Data:
$data_1: Condor = 6, Cheetah = 1.$ \\
$data_2: Condor = 12, Cheetah = 3.$ \\
Question:
Assume all relations between variables are linear combinations. If the number of Cheetah equals 5, then what is the number of Condor?

Question:
Assume all relations between variables are linear combinations. If the number of Leopard equals 5, the number of Rhino equals 3, the number of Koala equals 6, then what is the number of Black\_Bear?
%\vspace{-\baselineskip}
%\begin{equation}

\textbf{Solution\:}

\textbf{Defining Variables} \\
\textit{Known Variables:} \\
Cheetah as \( c_1 = 5 \) \\

\textit{Unknown Variables:} \\
Target Variable: Condor as \( c_2 \) \\

\textbf{Restoring Relations} \\
\textit{List all variable names in each data point:} \([c_2, c_1], [c_2, c_1]\) \\
\textit{Deduplicate them:} \([c_2, c_1]\) \\
There is 1 distinct group, implying 1 distinct linear relationship to be determined. \\
\textit{Examining each relationship:} \\

\textbf{Relation 1:} \\
Exploring relation for \( c_2 \): \\
There are 2 variables in the data beginning with \( c_2 \): Hence, 2 coefficients are required, and at least 2 data points are needed. \\

Let the coefficients on the right side of the equation be \( K_1 \) and \( K_2 \). \\
\textit{Recap variables:} \(['c_2', 'c_1']\) \\
\textit{Define the equation of relation 1:} \\
\( c_2 = K_1 \cdot c_1 + K_2 \) \\

Using data points \( \text{data}_1 \) and \( \text{data}_2 \): \\
\( \text{data}_1: c_2 = 6, c_1 = 1 \) \\
Equation 1: \( 6 = K_1 \cdot 1 + K_2 \) \\
\( \text{data}_2: c_2 = 12, c_1 = 3 \) \\
Equation 2: \( 12 = K_1 \cdot 3 + K_2 \) \\

\textbf{Solve the system of equations using Gaussian Elimination:} \\
\textit{Initialize:} \\
Equation 1: \( 1 \cdot K_1 + 1 \cdot K_2 = 6 \) \\
Equation 2: \( 3 \cdot K_1 + 1 \cdot K_2 = 12 \) \\

Swap Equation 1 with Equation 2: \\
Equation 1: \( 3 \cdot K_1 + 1 \cdot K_2 = 12 \) \\
Equation 2: \( 1 \cdot K_1 + 1 \cdot K_2 = 6 \) \\

Multiply Equation 1 by 1 and subtract 3 times Equation 2: \\
\((\text{Equation 1}) \cdot 1: 3 \cdot K_1 + 1 \cdot K_2 = 12 \) \\
\((\text{Equation 2}) \cdot 3: 3 \cdot K_1 + 3 \cdot K_2 = 18 \) \\
New Equation 2: \( -2 \cdot K_2 = -6 \) \\

\textit{Recap updated equations:} \\
Equation 1: \( 3 \cdot K_1 + 1 \cdot K_2 = 12 \) \\
Equation 2: \( -2 \cdot K_2 = -6 \) \\

\textbf{Solve for \( K_2 \):} \\
\( -2 \cdot K_2 = -6 \) \\
\( K_2 = \frac{-6}{-2} = 3 \) \\

\textbf{Solve for \( K_1 \):} \\
\( 3 \cdot K_1 = 12 - 1 \cdot K_2 \) \\
\( 3 \cdot K_1 = 12 - 3 = 9 \) \\
\( K_1 = \frac{9}{3} = 3 \) \\

\textit{Recap the equation:} \\
\( c_2 = K_1 \cdot c_1 + K_2 \) \\
Estimated coefficients: \( K_1 = 3, K_2 = 3 \) \\
Final equation: \( c_2 = 3 \cdot c_1 + 3 \) \\

\textbf{Calculation with Restored Relations:} \\
Using the equation \( c_2 = 3 \cdot c_1 + 3 \): \\
\textit{Known variables:} \( c_1 = 5 \) \\
\( c_2 = 3 \cdot 5 + 3 = 15 + 3 = 18 \) \\

\textbf{Recap Target Variable:} \\
Condor (\( c_2 \)) = 18 \\

\textbf{Conclusion:} The number of Condor equals 18.
%\end{equation}

\subsection{Out-of-distribution Comparison Across Input length}
The comparison \ref{fig:ood_detail} reveals the critical role of Chain-of-Thought prompting in improving models' OOD generalization. Both MPC (a) and LIS (b) demonstrate substantially higher accuracy when equipped with 100\% COT (blue lines) compared to without COT. This performance gap is particularly pronounced in out-of-domain regions, where models without COT show severe degradation (dropping below 0.2 accuracy). The consistent superior performance of COT-enabled models, especially in maintaining accuracy above 0.8 across different sequence lengths, underscores how COT prompting serves as a crucial mechanism for enhancing models' ability to generalize beyond their training distribution.
\begin{figure}[]
\centering
\subfigure[]{
    \includegraphics[width=0.8\linewidth]{figure/ood_figure_mpc.pdf}
}\hfill
\subfigure[]{
    \includegraphics[width=0.8\linewidth]{figure/lis_figure_mpc.pdf}}
\caption{Comparison of Out-Of-Distribution (OOD) performance between MPC and LIS models under different Chain-of-Thought (COT) conditions across varying sequence lengths.}
\label{fig:ood_detail}
\end{figure}
%\onecolumn
%\section{You \emph{can} have an appendix here.}


%%%%%%%%%%%%%%%%%%%%%%%%%%%%%%%%%%%%%%%%%%%%%%%%%%%%%%%%%%%%%%%%%%%%%%%%%%%%%%%
%%%%%%%%%%%%%%%%%%%%%%%%%%%%%%%%%%%%%%%%%%%%%%%%%%%%%%%%%%%%%%%%%%%%%%%%%%%%%%%


\end{document}

