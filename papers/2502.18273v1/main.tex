%%%%%%%% ICML 2025 EXAMPLE LATEX SUBMISSION FILE %%%%%%%%%%%%%%%%%

\documentclass{article}

% Recommended, but optional, packages for figures and better typesetting:
\usepackage{microtype}
\usepackage{graphicx}
\usepackage{subfigure}
\usepackage{tikz}
\usetikzlibrary{arrows.meta,shapes,positioning,calc}
\usepackage{booktabs} % for professional tables
\usepackage[maxnames=1,maxbibnames=2]{biblatex}
\addbibresource{reference.bib}  
% hyperref makes hyperlinks in the resulting PDF.
% If your build breaks (sometimes temporarily if a hyperlink spans a page)
% please comment out the following usepackage line and replace
\usepackage{hyperref}


% Attempt to make hyperref and algorithmic work together better:
\newcommand{\theHalgorithm}{\arabic{algorithm}}


% For theorems and such
\usepackage{arxiv}
\usepackage{amsmath}
\usepackage{amssymb}
\usepackage{mathtools}
\usepackage{amsthm}
\usepackage{multirow}
\usepackage[textwidth=0.5in]{todonotes}

% if you use cleveref..
\usepackage[capitalize,noabbrev]{cleveref}

%%%%%%%%%%%%%%%%%%%%%%%%%%%%%%%%
% THEOREMS
%%%%%%%%%%%%%%%%%%%%%%%%%%%%%%%%
\theoremstyle{plain}
\newtheorem{theorem}{Theorem}[section]
\newtheorem{proposition}[theorem]{Proposition}
\newtheorem{lemma}[theorem]{Lemma}
\newtheorem{corollary}[theorem]{Corollary}
\theoremstyle{definition}
\newtheorem{definition}[theorem]{Definition}
\newtheorem{assumption}[theorem]{Assumption}
\theoremstyle{remark}
\newtheorem{remark}[theorem]{Remark}
\newcommand{\cmark}{\ding{51}}%
\newcommand{\xmark}{\ding{55}}
\definecolor{ForestGreen}{RGB}{5,166,88}
\definecolor{LavaRed}{RGB}{222,48,28}
\definecolor{LightGrey}{RGB}{180,180,180}
\def\jiaxian#1{\textcolor{red}{[JXG: #1]}}

% Todonotes is useful during development; simply uncomment the next line
%    and comment out the line below the next line to turn off comments
%\usepackage[disable,textsize=tiny]{todonotes}



\author{%
\bf Ru Wang {\textsuperscript{1}}
\qquad
\bf Wei Huang {\textsuperscript{2}}
\qquad
\bf Selena Song {\textsuperscript{1}}
\qquad
\bf Haoyu Zhang {\textsuperscript{1}}
\\
\bf Yusuke Iwasawa {\textsuperscript{1}}
\qquad
\bf  Yutaka Matsuo {\textsuperscript{1}}
\qquad
\bf  Jiaxian Guo {\textsuperscript{3}} \\
 \texttt{\{ru.wang,kexin.song,haoyu.zhang\}@weblab.t.u-tokyo.ac.jp} \\
\texttt{weihuang.uts@gmail.com} \\
\texttt{jeffguo@google.com}
\\
The University of Tokyo {\textsuperscript{1}} \quad RIKEN Center for Advanced Intelligence Project {\textsuperscript{2}}  \quad 
Google Research, Australia {\textsuperscript{3}} 
}


\begin{document}
% \nolinenumbers 

\title{Beyond In-Distribution Success: Scaling Curves of CoT Granularity for Language Model Generalization}

\maketitle

Large language model (LLM)-based agents have shown promise in tackling complex tasks by interacting dynamically with the environment. 
Existing work primarily focuses on behavior cloning from expert demonstrations and preference learning through exploratory trajectory sampling. However, these methods often struggle in long-horizon tasks, where suboptimal actions accumulate step by step, causing agents to deviate from correct task trajectories.
To address this, we highlight the importance of \textit{timely calibration} and the need to automatically construct calibration trajectories for training agents. We propose \textbf{S}tep-Level \textbf{T}raj\textbf{e}ctory \textbf{Ca}libration (\textbf{\model}), a novel framework for LLM agent learning. 
Specifically, \model identifies suboptimal actions through a step-level reward comparison during exploration. It constructs calibrated trajectories using LLM-driven reflection, enabling agents to learn from improved decision-making processes. These calibrated trajectories, together with successful trajectory data, are utilized for reinforced training.
Extensive experiments demonstrate that \model significantly outperforms existing methods. Further analysis highlights that step-level calibration enables agents to complete tasks with greater robustness. 
Our code and data are available at \url{https://github.com/WangHanLinHenry/STeCa}.


\section{Introduction}

Despite the remarkable capabilities of large language models (LLMs)~\cite{DBLP:conf/emnlp/QinZ0CYY23,DBLP:journals/corr/abs-2307-09288}, they often inevitably exhibit hallucinations due to incorrect or outdated knowledge embedded in their parameters~\cite{DBLP:journals/corr/abs-2309-01219, DBLP:journals/corr/abs-2302-12813, DBLP:journals/csur/JiLFYSXIBMF23}.
Given the significant time and expense required to retrain LLMs, there has been growing interest in \emph{model editing} (a.k.a., \emph{knowledge editing})~\cite{DBLP:conf/iclr/SinitsinPPPB20, DBLP:journals/corr/abs-2012-00363, DBLP:conf/acl/DaiDHSCW22, DBLP:conf/icml/MitchellLBMF22, DBLP:conf/nips/MengBAB22, DBLP:conf/iclr/MengSABB23, DBLP:conf/emnlp/YaoWT0LDC023, DBLP:conf/emnlp/ZhongWMPC23, DBLP:conf/icml/MaL0G24, DBLP:journals/corr/abs-2401-04700}, 
which aims to update the knowledge of LLMs cost-effectively.
Some existing methods of model editing achieve this by modifying model parameters, which can be generally divided into two categories~\cite{DBLP:journals/corr/abs-2308-07269, DBLP:conf/emnlp/YaoWT0LDC023}.
Specifically, one type is based on \emph{Meta-Learning}~\cite{DBLP:conf/emnlp/CaoAT21, DBLP:conf/acl/DaiDHSCW22}, while the other is based on \emph{Locate-then-Edit}~\cite{DBLP:conf/acl/DaiDHSCW22, DBLP:conf/nips/MengBAB22, DBLP:conf/iclr/MengSABB23}. This paper primarily focuses on the latter.

\begin{figure}[t]
  \centering
  \includegraphics[width=0.48\textwidth]{figures/demonstration.pdf}
  \vspace{-4mm}
  \caption{(a) Comparison of regular model editing and EAC. EAC compresses the editing information into the dimensions where the editing anchors are located. Here, we utilize the gradients generated during training and the magnitude of the updated knowledge vector to identify anchors. (b) Comparison of general downstream task performance before editing, after regular editing, and after constrained editing by EAC.}
  \vspace{-3mm}
  \label{demo}
\end{figure}

\emph{Sequential} model editing~\cite{DBLP:conf/emnlp/YaoWT0LDC023} can expedite the continual learning of LLMs where a series of consecutive edits are conducted.
This is very important in real-world scenarios because new knowledge continually appears, requiring the model to retain previous knowledge while conducting new edits. 
Some studies have experimentally revealed that in sequential editing, existing methods lead to a decrease in the general abilities of the model across downstream tasks~\cite{DBLP:journals/corr/abs-2401-04700, DBLP:conf/acl/GuptaRA24, DBLP:conf/acl/Yang0MLYC24, DBLP:conf/acl/HuC00024}. 
Besides, \citet{ma2024perturbation} have performed a theoretical analysis to elucidate the bottleneck of the general abilities during sequential editing.
However, previous work has not introduced an effective method that maintains editing performance while preserving general abilities in sequential editing.
This impacts model scalability and presents major challenges for continuous learning in LLMs.

In this paper, a statistical analysis is first conducted to help understand how the model is affected during sequential editing using two popular editing methods, including ROME~\cite{DBLP:conf/nips/MengBAB22} and MEMIT~\cite{DBLP:conf/iclr/MengSABB23}.
Matrix norms, particularly the L1 norm, have been shown to be effective indicators of matrix properties such as sparsity, stability, and conditioning, as evidenced by several theoretical works~\cite{kahan2013tutorial}. In our analysis of matrix norms, we observe significant deviations in the parameter matrix after sequential editing.
Besides, the semantic differences between the facts before and after editing are also visualized, and we find that the differences become larger as the deviation of the parameter matrix after editing increases.
Therefore, we assume that each edit during sequential editing not only updates the editing fact as expected but also unintentionally introduces non-trivial noise that can cause the edited model to deviate from its original semantics space.
Furthermore, the accumulation of non-trivial noise can amplify the negative impact on the general abilities of LLMs.

Inspired by these findings, a framework termed \textbf{E}diting \textbf{A}nchor \textbf{C}ompression (EAC) is proposed to constrain the deviation of the parameter matrix during sequential editing by reducing the norm of the update matrix at each step. 
As shown in Figure~\ref{demo}, EAC first selects a subset of dimension with a high product of gradient and magnitude values, namely editing anchors, that are considered crucial for encoding the new relation through a weighted gradient saliency map.
Retraining is then performed on the dimensions where these important editing anchors are located, effectively compressing the editing information.
By compressing information only in certain dimensions and leaving other dimensions unmodified, the deviation of the parameter matrix after editing is constrained. 
To further regulate changes in the L1 norm of the edited matrix to constrain the deviation, we incorporate a scored elastic net ~\cite{zou2005regularization} into the retraining process, optimizing the previously selected editing anchors.

To validate the effectiveness of the proposed EAC, experiments of applying EAC to \textbf{two popular editing methods} including ROME and MEMIT are conducted.
In addition, \textbf{three LLMs of varying sizes} including GPT2-XL~\cite{radford2019language}, LLaMA-3 (8B)~\cite{llama3} and LLaMA-2 (13B)~\cite{DBLP:journals/corr/abs-2307-09288} and \textbf{four representative tasks} including 
natural language inference~\cite{DBLP:conf/mlcw/DaganGM05}, 
summarization~\cite{gliwa-etal-2019-samsum},
open-domain question-answering~\cite{DBLP:journals/tacl/KwiatkowskiPRCP19},  
and sentiment analysis~\cite{DBLP:conf/emnlp/SocherPWCMNP13} are selected to extensively demonstrate the impact of model editing on the general abilities of LLMs. 
Experimental results demonstrate that in sequential editing, EAC can effectively preserve over 70\% of the general abilities of the model across downstream tasks and better retain the edited knowledge.

In summary, our contributions to this paper are three-fold:
(1) This paper statistically elucidates how deviations in the parameter matrix after editing are responsible for the decreased general abilities of the model across downstream tasks after sequential editing.
(2) A framework termed EAC is proposed, which ultimately aims to constrain the deviation of the parameter matrix after editing by compressing the editing information into editing anchors. 
(3) It is discovered that on models like GPT2-XL and LLaMA-3 (8B), EAC significantly preserves over 70\% of the general abilities across downstream tasks and retains the edited knowledge better.

\section{Related Work}

\subsection{Personalization and Role-Playing}
Recent works have introduced benchmark datasets for personalizing LLM outputs in tasks like email, abstract, and news writing, focusing on shorter outputs (e.g., 300 tokens for product reviews \citep{kumar2024longlamp} and 850 for news writing \citep{shashidhar-etal-2024-unsupervised}). These methods infer user traits from history for task-specific personalization \citep{sun-etal-2024-revealing, sun-etal-2025-persona, pal2024beyond, li2023teach, salemi2025reasoning}. In contrast, we tackle the more subjective problem of long-form story writing, with author stories averaging 1500 tokens. Unlike prior role-playing approaches that use predefined personas (e.g., Tony Stark, Confucius) \citep{wang-etal-2024-rolellm, sadeq-etal-2024-mitigating, tu2023characterchat, xu2023expertprompting}, we propose a novel method to infer story-writing personas from an author’s history to guide role-playing.


\subsection{Story Understanding and Generation}  
Prior work on persona-aware story generation \citep{yunusov-etal-2024-mirrorstories, bae-kim-2024-collective, zhang-etal-2022-persona, chandu-etal-2019-way} defines personas using discrete attributes like personality traits, demographics, or hobbies. Similarly, \citep{zhu-etal-2023-storytrans} explore story style transfer across pre-defined domains (e.g., fairy tales, martial arts, Shakespearean plays). In contrast, we mimic an individual author's writing style based on their history. Our approach differs by (1) inferring long-form author personas—descriptions of an author’s style from their past works, rather than relying on demographics, and (2) handling long-form story generation, averaging 1500 tokens per output, exceeding typical story lengths in prior research.
\vspace{-5pt}
\section{Method}
\label{sec:method}
\section{Overview}

\revision{In this section, we first explain the foundational concept of Hausdorff distance-based penetration depth algorithms, which are essential for understanding our method (Sec.~\ref{sec:preliminary}).
We then provide a brief overview of our proposed RT-based penetration depth algorithm (Sec.~\ref{subsec:algo_overview}).}



\section{Preliminaries }
\label{sec:Preliminaries}

% Before we introduce our method, we first overview the important basics of 3D dynamic human modeling with Gaussian splatting. Then, we discuss the diffusion-based 3d generation techniques, and how they can be applied to human modeling.
% \ZY{I stopp here. TBC.}
% \subsection{Dynamic human modeling with Gaussian splatting}
\subsection{3D Gaussian Splatting}
3D Gaussian splatting~\cite{kerbl3Dgaussians} is an explicit scene representation that allows high-quality real-time rendering. The given scene is represented by a set of static 3D Gaussians, which are parameterized as follows: Gaussian center $x\in {\mathbb{R}^3}$, color $c\in {\mathbb{R}^3}$, opacity $\alpha\in {\mathbb{R}}$, spatial rotation in the form of quaternion $q\in {\mathbb{R}^4}$, and scaling factor $s\in {\mathbb{R}^3}$. Given these properties, the rendering process is represented as:
\begin{equation}
  I = Splatting(x, c, s, \alpha, q, r),
  \label{eq:splattingGA}
\end{equation}
where $I$ is the rendered image, $r$ is a set of query rays crossing the scene, and $Splatting(\cdot)$ is a differentiable rendering process. We refer readers to Kerbl et al.'s paper~\cite{kerbl3Dgaussians} for the details of Gaussian splatting. 



% \ZY{I would suggest move this part to the method part.}
% GaissianAvatar is a dynamic human generation model based on Gaussian splitting. Given a sequence of RGB images, this method utilizes fitted SMPLs and sampled points on its surface to obtain a pose-dependent feature map by a pose encoder. The pose-dependent features and a geometry feature are fed in a Gaussian decoder, which is employed to establish a functional mapping from the underlying geometry of the human form to diverse attributes of 3D Gaussians on the canonical surfaces. The parameter prediction process is articulated as follows:
% \begin{equation}
%   (\Delta x,c,s)=G_{\theta}(S+P),
%   \label{eq:gaussiandecoder}
% \end{equation}
%  where $G_{\theta}$ represents the Gaussian decoder, and $(S+P)$ is the multiplication of geometry feature S and pose feature P. Instead of optimizing all attributes of Gaussian, this decoder predicts 3D positional offset $\Delta{x} \in {\mathbb{R}^3}$, color $c\in\mathbb{R}^3$, and 3D scaling factor $ s\in\mathbb{R}^3$. To enhance geometry reconstruction accuracy, the opacity $\alpha$ and 3D rotation $q$ are set to fixed values of $1$ and $(1,0,0,0)$ respectively.
 
%  To render the canonical avatar in observation space, we seamlessly combine the Linear Blend Skinning function with the Gaussian Splatting~\cite{kerbl3Dgaussians} rendering process: 
% \begin{equation}
%   I_{\theta}=Splatting(x_o,Q,d),
%   \label{eq:splatting}
% \end{equation}
% \begin{equation}
%   x_o = T_{lbs}(x_c,p,w),
%   \label{eq:LBS}
% \end{equation}
% where $I_{\theta}$ represents the final rendered image, and the canonical Gaussian position $x_c$ is the sum of the initial position $x$ and the predicted offset $\Delta x$. The LBS function $T_{lbs}$ applies the SMPL skeleton pose $p$ and blending weights $w$ to deform $x_c$ into observation space as $x_o$. $Q$ denotes the remaining attributes of the Gaussians. With the rendering process, they can now reposition these canonical 3D Gaussians into the observation space.



\subsection{Score Distillation Sampling}
Score Distillation Sampling (SDS)~\cite{poole2022dreamfusion} builds a bridge between diffusion models and 3D representations. In SDS, the noised input is denoised in one time-step, and the difference between added noise and predicted noise is considered SDS loss, expressed as:

% \begin{equation}
%   \mathcal{L}_{SDS}(I_{\Phi}) \triangleq E_{t,\epsilon}[w(t)(\epsilon_{\phi}(z_t,y,t)-\epsilon)\frac{\partial I_{\Phi}}{\partial\Phi}],
%   \label{eq:SDSObserv}
% \end{equation}
\begin{equation}
    \mathcal{L}_{\text{SDS}}(I_{\Phi}) \triangleq \mathbb{E}_{t,\epsilon} \left[ w(t) \left( \epsilon_{\phi}(z_t, y, t) - \epsilon \right) \frac{\partial I_{\Phi}}{\partial \Phi} \right],
  \label{eq:SDSObservGA}
\end{equation}
where the input $I_{\Phi}$ represents a rendered image from a 3D representation, such as 3D Gaussians, with optimizable parameters $\Phi$. $\epsilon_{\phi}$ corresponds to the predicted noise of diffusion networks, which is produced by incorporating the noise image $z_t$ as input and conditioning it with a text or image $y$ at timestep $t$. The noise image $z_t$ is derived by introducing noise $\epsilon$ into $I_{\Phi}$ at timestep $t$. The loss is weighted by the diffusion scheduler $w(t)$. 
% \vspace{-3mm}

\subsection{Overview of the RTPD Algorithm}\label{subsec:algo_overview}
Fig.~\ref{fig:Overview} presents an overview of our RTPD algorithm.
It is grounded in the Hausdorff distance-based penetration depth calculation method (Sec.~\ref{sec:preliminary}).
%, similar to that of Tang et al.~\shortcite{SIG09HIST}.
The process consists of two primary phases: penetration surface extraction and Hausdorff distance calculation.
We leverage the RTX platform's capabilities to accelerate both of these steps.

\begin{figure*}[t]
    \centering
    \includegraphics[width=0.8\textwidth]{Image/overview.pdf}
    \caption{The overview of RT-based penetration depth calculation algorithm overview}
    \label{fig:Overview}
\end{figure*}

The penetration surface extraction phase focuses on identifying the overlapped region between two objects.
\revision{The penetration surface is defined as a set of polygons from one object, where at least one of its vertices lies within the other object. 
Note that in our work, we focus on triangles rather than general polygons, as they are processed most efficiently on the RTX platform.}
To facilitate this extraction, we introduce a ray-tracing-based \revision{Point-in-Polyhedron} test (RT-PIP), significantly accelerated through the use of RT cores (Sec.~\ref{sec:RT-PIP}).
This test capitalizes on the ray-surface intersection capabilities of the RTX platform.
%
Initially, a Geometry Acceleration Structure (GAS) is generated for each object, as required by the RTX platform.
The RT-PIP module takes the GAS of one object (e.g., $GAS_{A}$) and the point set of the other object (e.g., $P_{B}$).
It outputs a set of points (e.g., $P_{\partial B}$) representing the penetration region, indicating their location inside the opposing object.
Subsequently, a penetration surface (e.g., $\partial B$) is constructed using this point set (e.g., $P_{\partial B}$) (Sec.~\ref{subsec:surfaceGen}).
%
The generated penetration surfaces (e.g., $\partial A$ and $\partial B$) are then forwarded to the next step. 

The Hausdorff distance calculation phase utilizes the ray-surface intersection test of the RTX platform (Sec.~\ref{sec:RT-Hausdorff}) to compute the Hausdorff distance between two objects.
We introduce a novel Ray-Tracing-based Hausdorff DISTance algorithm, RT-HDIST.
It begins by generating GAS for the two penetration surfaces, $P_{\partial A}$ and $P_{\partial B}$, derived from the preceding step.
RT-HDIST processes the GAS of a penetration surface (e.g., $GAS_{\partial A}$) alongside the point set of the other penetration surface (e.g., $P_{\partial B}$) to compute the penetration depth between them.
The algorithm operates bidirectionally, considering both directions ($\partial A \to \partial B$ and $\partial B \to \partial A$).
The final penetration depth between the two objects, A and B, is determined by selecting the larger value from these two directional computations.

%In the Hausdorff distance calculation step, we compute the Hausdorff distance between given two objects using a ray-surface-intersection test. (Sec.~\ref{sec:RT-Hausdorff}) Initially, we construct the GAS for both $\partial A$ and $\partial B$ to utilize the RT-core effectively. The RT-based Hausdorff distance algorithms then determine the Hausdorff distance by processing the GAS of one object (e.g. $GAS_{\partial A}$) and set of the vertices of the other (e.g. $P_{\partial B}$). Following the Hausdorff distance definition (Eq.~\ref{equation:hausdorff_definition}), we compute the Hausdorff distance to both directions ($\partial A \to \partial B$) and ($\partial B \to \partial A$). As a result, the bigger one is the final Hausdorff distance, and also it is the penetration depth between input object $A$ and $B$.


%the proposed RT-based penetration depth calculation pipeline.
%Our proposed methods adopt Tang's Hausdorff-based penetration depth methods~\cite{SIG09HIST}. The pipeline is divided into the penetration surface extraction step and the Hausdorff distance calculation between the penetration surface steps. However, since Tang's approach is not suitable for the RT platform in detail, we modified and applied it with appropriate methods.

%The penetration surface extraction step is extracting overlapped surfaces on other objects. To utilize the RT core, we use the ray-intersection-based PIP(Point-In-Polygon) algorithms instead of collision detection between two objects which Tang et al.~\cite{SIG09HIST} used. (Sec.~\ref{sec:RT-PIP})
%RT core-based PIP test uses a ray-surface intersection test. For purpose this, we generate the GAS(Geometry Acceleration Structure) for each object. RT core-based PIP test takes the GAS of one object (e.g. $GAS_{A}$) and a set of vertex of another one (e.g. $P_{B}$). Then this computes the penetrated vertex set of another one (e.g. $P_{\partial B}$). To calculate the Hausdorff distance, these vertex sets change to objects constructed by penetrated surface (e.g. $\partial B$). Finally, the two generated overlapped surface objects $\partial A$ and $\partial B$ are used in the Hausdorff distance calculation step.

Our goal is to increase the robustness of T2I models, particularly with rare or unseen concepts, which they struggle to generate. To do so, we investigate a retrieval-augmented generation approach, through which we dynamically select images that can provide the model with missing visual cues. Importantly, we focus on models that were not trained for RAG, and show that existing image conditioning tools can be leveraged to support RAG post-hoc.
As depicted in \cref{fig:overview}, given a text prompt and a T2I generative model, we start by generating an image with the given prompt. Then, we query a VLM with the image, and ask it to decide if the image matches the prompt. If it does not, we aim to retrieve images representing the concepts that are missing from the image, and provide them as additional context to the model to guide it toward better alignment with the prompt.
In the following sections, we describe our method by answering key questions:
(1) How do we know which images to retrieve? 
(2) How can we retrieve the required images? 
and (3) How can we use the retrieved images for unknown concept generation?
By answering these questions, we achieve our goal of generating new concepts that the model struggles to generate on its own.

\vspace{-3pt}
\subsection{Which images to retrieve?}
The amount of images we can pass to a model is limited, hence we need to decide which images to pass as references to guide the generation of a base model. As T2I models are already capable of generating many concepts successfully, an efficient strategy would be passing only concepts they struggle to generate as references, and not all the concepts in a prompt.
To find the challenging concepts,
we utilize a VLM and apply a step-by-step method, as depicted in the bottom part of \cref{fig:overview}. First, we generate an initial image with a T2I model. Then, we provide the VLM with the initial prompt and image, and ask it if they match. If not, we ask the VLM to identify missing concepts and
focus on content and style, since these are easy to convey through visual cues.
As demonstrated in \cref{tab:ablations}, empirical experiments show that image retrieval from detailed image captions yields better results than retrieval from brief, generic concept descriptions.
Therefore, after identifying the missing concepts, we ask the VLM to suggest detailed image captions for images that describe each of the concepts. 

\vspace{-4pt}
\subsubsection{Error Handling}
\label{subsec:err_hand}

The VLM may sometimes fail to identify the missing concepts in an image, and will respond that it is ``unable to respond''. In these rare cases, we allow up to 3 query repetitions, while increasing the query temperature in each repetition. Increasing the temperature allows for more diverse responses by encouraging the model to sample less probable words.
In most cases, using our suggested step-by-step method yields better results than retrieving images directly from the given prompt (see 
\cref{subsec:ablations}).
However, if the VLM still fails to identify the missing concepts after multiple attempts, we fall back to retrieving images directly from the prompt, as it usually means the VLM does not know what is the meaning of the prompt.

The used prompts can be found in \cref{app:prompts}.
Next, we turn to retrieve images based on the acquired image captions.

\vspace{-3pt}
\subsection{How to retrieve the required images?}

Given $n$ image captions, our goal is to retrieve the images that are most similar to these captions from a dataset. 
To retrieve images matching a given image caption, we compare the caption to all the images in the dataset using a text-image similarity metric and retrieve the top $k$ most similar images.
Text-to-image retrieval is an active research field~\cite{radford2021learning, zhai2023sigmoid, ray2024cola, vendrowinquire}, where no single method is perfect.
Retrieval is especially hard when the dataset does not contain an exact match to the query \cite{biswas2024efficient} or when the task is fine-grained retrieval, that depends on subtle details~\cite{wei2022fine}.
Hence, a common retrieval workflow is to first retrieve image candidates using pre-computed embeddings, and then re-rank the retrieved candidates using a different, often more expensive but accurate, method \cite{vendrowinquire}.
Following this workflow, we experimented with cosine similarity over different embeddings, and with multiple re-ranking methods of reference candidates.
Although re-ranking sometimes yields better results compared to simply using cosine similarity between CLIP~\cite{radford2021learning} embeddings, the difference was not significant in most of our experiments. Therefore, for simplicity, we use cosine similarity between CLIP embeddings as our similarity metric (see \cref{tab:sim_metrics}, \cref{subsec:ablations} for more details about our experiments with different similarity metrics).

\vspace{-3pt}
\subsection{How to use the retrieved images?}
Putting it all together, after retrieving relevant images, all that is left to do is to use them as context so they are beneficial for the model.
We experimented with two types of models; models that are trained to receive images as input in addition to text and have ICL capabilities (e.g., OmniGen~\cite{xiao2024omnigen}), and T2I models augmented with an image encoder in post-training (e.g., SDXL~\cite{podellsdxl} with IP-adapter~\cite{ye2023ip}).
As the first model type has ICL capabilities, we can supply the retrieved images as examples that it can learn from, by adjusting the original prompt.
Although the second model type lacks true ICL capabilities, it offers image-based control functionalities, which we can leverage for applying RAG over it with our method.
Hence, for both model types, we augment the input prompt to contain a reference of the retrieved images as examples.
Formally, given a prompt $p$, $n$ concepts, and $k$ compatible images for each concept, we use the following template to create a new prompt:
``According to these examples of 
$\mathord{<}c_1\mathord{>:<}img_{1,1}\mathord{>}, ... , \mathord{<}img_{1,k}\mathord{>}, ... , \mathord{<}c_n\mathord{>:<}img_{n,1}\mathord{>}, ... , $
$\mathord{<}img_{n,k}\mathord{>}$,
generate $\mathord{<}p\mathord{>}$'', 
where $c_i$ for $i\in{[1,n]}$ is a compatible image caption of the image $\mathord{<}img_{i,j}\mathord{>},  j\in{[1,k]}$. 

This prompt allows models to learn missing concepts from the images, guiding them to generate the required result. 

\textbf{Personalized Generation}: 
For models that support multiple input images, we can apply our method for personalized generation as well, to generate rare concept combinations with personal concepts. In this case, we use one image for personal content, and 1+ other reference images for missing concepts. For example, given an image of a specific cat, we can generate diverse images of it, ranging from a mug featuring the cat to a lego of it or atypical situations like the cat writing code or teaching a classroom of dogs (\cref{fig:personalization}).
\vspace{-2pt}
\begin{figure}[htp]
  \centering
   \includegraphics[width=\linewidth]{Assets/personalization.pdf}
   \caption{\textbf{Personalized generation example.}
   \emph{ImageRAG} can work in parallel with personalization methods and enhance their capabilities. For example, although OmniGen can generate images of a subject based on an image, it struggles to generate some concepts. Using references retrieved by our method, it can generate the required result.
}
   \label{fig:personalization}\vspace{-10pt}
\end{figure}




\section{Experiments}
\seclabel{experiments}
Our experiments are designed to test a) the extent to which open loop execution is an issue for precise mobile manipulation tasks, b) how effective are blind proprioceptive correction techniques, c) do object detectors and point trackers perform reliably enough in wrist camera images for reliable control, d) is occlusion by the end-effector an issue and how effectively can it be mitigated through the use of video in-painting models, and e) how does our proposed \name methodology compare to large-scale imitation learning? 


\subsection{Tasks and Experimental Setup}
We work with the Stretch RE2 robot. Stretch RE2 is a commodity mobile manipulator with a 5DOF arm mounted on top of a non-holomonic base. We upgrade the robot to use the Dex Wrist 3, which has an eye-in-hand RGB-D camera (Intel D405). 
We consider 3 task families for a total
of 6 different tasks: a) holding a knob to pull open a cabinet or drawer, b) holding a
handle to pull open a cabinet, and c) pushing on objects (light buttons, books
in a book shelf, and light switches). Our focus is on generalization. {\it
Therefore, we exclusively test on previously unseen instances, not used during
development in any way.} 
\figref{tasks} shows the instances that we test on. 

All tasks involve some precise manipulation, followed by execution of a motion
primitive. {\bf For the pushing tasks}, the precise motion is to get the
end-effector exactly at the indicated point and the motion primitive is to push
in the direction perpendicular to the surface and retract the end-effector 
upon contact. The robot is positioned such
that the target position is within the field of view of the wrist camera. A user
selects the point of pushing via a mouse click on the wrist camera image. The
goal is to push at the indicated location. Success is determined by whether the
push results in the desired outcome (light turns on / off or book gets pushed in). 
The original rubber gripper bends upon contact, we use a rigid known tool
that sticks out a bit. We take the geometry of the tool into account while servoing.

{\bf For the opening articulated object tasks}, the precise manipulation is grasping the
knob / handle, while the motion primitive is the whole-body motion that opens
the cupboard. Computing and executing this full body motion is difficult. We
adopt the modular approach to opening articulated objects (MOSART) from Gupta \etal~\cite{gupta2024opening} and invoke it
after the gripper has been placed around the knob / handle. The whole tasks 
starts out with the robot about 1.5m way from the target object, with the 
target object in view
from robot's head mounted camera. We use MOSART to compute articulation
parameters and convey the robot to a pre-grasp
location with the target handle in view of the wrist camera. At this point,
\name (or baseline) is used to center the gripper around the knob / handle, 
before resuming MOSART: extending the gripper till contact, close the gripper, and play rest of the predicted motion plan. Success is 
determined by whether the cabinet opens by more than $60^\circ$
or the drawer is pulled out by more than $24cm$, similar to the criteria used in \cite{gupta2024opening}.


For the precise manipulation part, all baselines consume the current and
previous RGB-D images from the wrist camera and output full body motor
commands.

% % Please add the following required packages to your document preamble:
% % \usepackage{graphicx}
% \begin{table*}[!ht]
% \centering
% \caption{}
% \label{tab:my-table}
% \resizebox{\textwidth}{!}{%
% \begin{tabular}{lcccccc}
% \toprule
%  & \multicolumn{2}{c}{ours} & \multicolumn{2}{c}{Gurobi} & \multicolumn{2}{c}{MOSEK} \\
%  & \multicolumn{1}{l}{time (s)} & \multicolumn{1}{l}{optimality gap (\%)} & \multicolumn{1}{l}{time (s)} & \multicolumn{1}{l}{optimality gap (\%)} & \multicolumn{1}{l}{time (s)} & \multicolumn{1}{l}{optimality gap (\%)} \\ \hline
% \begin{tabular}[c]{@{}l@{}}Linear Regression\\ Synthetic \\ (n=16000, p=16000)\end{tabular} & 57 & 0.0 & 3351 & - & 2148 & - \\ \hline
% \begin{tabular}[c]{@{}l@{}}Linear Regression\\ Cancer Drug Response\\ (n=822, p=2300)\end{tabular} & 47 & 0.0 & 1800 & 0.31 & 212 & 0.0 \\ \hline
% \begin{tabular}[c]{@{}l@{}}Logistic Regression\\ Synthetic\\ (n=16000, p=16000)\end{tabular} & 271 & 0.0 & N/A & N/A & 1800 & - \\ \hline
% \begin{tabular}[c]{@{}l@{}}Logistic Regression\\ Dorothea\\ (n=1150, p=91598)\end{tabular} & 62 & 0.0 & N/A & N/A & 600 & 0.0 \\
% \bottomrule
% \end{tabular}%
% }
% \end{table*}

% Please add the following required packages to your document preamble:
% \usepackage{multirow}
% \usepackage{graphicx}
\begin{table*}[]
\centering
\caption{Certifying optimality on large-scale and real-world datasets.}
\vspace{2mm}
\label{tab:my-table}
\resizebox{\textwidth}{!}{%
\begin{tabular}{llcccccc}
\toprule
 &  & \multicolumn{2}{c}{ours} & \multicolumn{2}{c}{Gurobi} & \multicolumn{2}{c}{MOSEK} \\
 &  & time (s) & opt. gap (\%) & time (s) & opt. gap (\%) & time (s) & opt. gap (\%) \\ \hline
\multirow{2}{*}{Linear Regression} & \begin{tabular}[c]{@{}l@{}}synthetic ($k=10, M=2$)\\ (n=16k, p=16k, seed=0)\end{tabular} & 79 & 0.0 & 1800 & - & 1915 & - \\ \cline{2-8}
 & \begin{tabular}[c]{@{}l@{}}Cancer Drug Response ($k=5, M=5$)\\ (n=822, p=2300)\end{tabular} & 41 & 0.0 & 1800 & 0.89 & 188 & 0.0 \\ \hline
\multirow{2}{*}{Logistic Regression} & \begin{tabular}[c]{@{}l@{}}Synthetic ($k=10, M=2$)\\ (n=16k, p=16k, seed=0)\end{tabular} & 626 & 0.0 & N/A & N/A & 2446 & - \\ \cline{2-8}
 & \begin{tabular}[c]{@{}l@{}}DOROTHEA ($k=15, M=2$)\\ (n=1150, p=91598)\end{tabular} & 91 & 0.0 & N/A & N/A & 634 & 0.0 \\
 \bottomrule
\end{tabular}%
}
% \vspace{-3mm}
\end{table*}

\begin{figure*}
\insertW{1.0}{figures/figure_6_cropped_brighten.pdf}
\caption{{\bf Comparison of \name with the open loop (eye-in-hand) baseline} for opening a cabinet with a knob. Slight errors in getting to the target cause the end-effector to slip off, leading to failure for the baseline, where as our method is able to successfully complete the task.}
\figlabel{rollout}
\end{figure*}

\begin{table}
\setlength{\tabcolsep}{8pt}
  \centering
  \resizebox{\linewidth}{!}{
  \begin{tabular}{lcccg}
  \toprule
                              & \multicolumn{2}{c}{\bf Knobs} & \bf Handle & \bf \multirow{2}{*}{\bf Total} \\
                              \cmidrule(lr){2-3} \cmidrule(lr){4-4}
                              & \bf Cabinets & \bf Drawer & \bf Cabinets & \\
  \midrule
  RUM~\cite{etukuru2024robot}  & 0/3    & 1/4         & 1/3         & 2/10 \\
  \name (Ours) & 2/3    & 2/4         & 3/3     &  7/10 \\
  \bottomrule
  \end{tabular}}
  \caption{Comparison of \name \vs RUM~\cite{etukuru2024robot}, a recent large-scale end-to-end imitation learning method trained on 1200 demos for opening cabinets and 525 demos for opening drawers across 40 different environments. Our evaluation spans objects from three environments across two buildings.}
  \tablelabel{rum}
\end{table}

\subsection{Baselines}
We compare against three other methods for the precise manipulation part of
these tasks. 
\subsubsection{Open Loop (Eye-in-Hand)} To assess the precision requirements of
the tasks and to set it in context with the manipulation capabilities of the
robot platform, this baseline uses open loop execution starting from estimates
for the 3D target position from the first wrist camera image.
\subsubsection{MOSART~\cite{gupta2024opening}}
The recent modular system for opening cabinets and drawers~\cite{gupta2024opening}
reports impressive performance with open-loop control (using the head camera from 1.5m away), combined with proprioception-based feedback to 
compensate for errors in perception and control when interacting with handles. 
We test if such correction is also sufficient for interacting with knobs. Note 
that such correction is not possible for the smaller buttons and pliable books.

\subsubsection{\name (no inpainting)} To understand how much of an issue
occlusion due to the end-effector is during manipulation, we ablate the use of
inpainting. %

\subsubsection{Robot Utility Models (RUM)~\cite{etukuru2024robot}}
For the opening articulated object tasks, we also compare to Robot Utility Models (RUM), 
a closed-loop imitation learning method recently proposed by Etukuru et al. \cite{etukuru2024robot}.
RUM is trained on a substantial dataset comprising expert demonstrations, including 
1,200 instances of cabinet opening and 525 of drawer opening, gathered from roughly 
40 different environments.
This dataset stands as the most extensive imitation 
learning dataset for articulated object manipulation to date, establishing RUM as a 
strong baseline for our evaluation.

Similar to our method, we use MOSART to compute articulation
parameters and convey the robot to a pre-grasp location
with the target handle in view of the wrist camera.
One of the assumptions of RUM is a good view of the handle.
To benefit RUM, we try out three different heights of the wrist camera,
and \textit{report the best result for RUM.}

\begin{figure*}
\insertW{1.0}{figures/figure_9_cropped_brighten.pdf}
\caption{{\bf \name \vs open loop (eye-in-hand) baseline for pushing on user-clicked points}. Slight errors in getting to the target cause failure, where as \name successfully turns the lights off. Note the quality of CoTracker's track ({\color{blue} blue dot}).}
\figlabel{rollout_v2}
\end{figure*}

\begin{figure*}
\insertW{1.0}{figures/figure_5_v2_cropped_brighten.pdf}
\caption{{\bf Comparison of \name with and without inpainting}. Erroneous detection without inpainting causes execution to fail, where as with inpainting the target is correctly detected leading to a successful grasp and a successful execution.}
\figlabel{rollouts2}
\end{figure*}


\subsection{Results}
\tableref{results} presents results from our experiments. 
Our training-free approach \name successfully 
solves over 85\% of task instances that we test on.
As noted, all these
tests were conducted on unseen object instances in unseen
environments that were not used for development in any way. We discuss our key
experimental findings below.

\subsubsection{Closing the loop is necessary for these precise tasks} 
While the proprioception-based strategies proposed in MOSART~\cite{gupta2024opening}
work out for handles, they are inadequate for targets like knobs and just
don't work for tasks like pushing buttons. Using estimates from the wrist
camera is better, but open loop execution still fails for knobs and pushing
buttons. 

\subsubsection{Vision models work reasonably well even on wrist camera images}
Inpainting works well on wrist camera images (see \figref{occlusion} and \figref{inpainting}).
Closing the loop using feedback from vision detectors and point trackers on
wrist camera images also work well, particularly when we use in-painted images.
See some examples detections and point tracks in \figref{rollout} and \figref{rollout_v2}. 
Detic~\cite{zhou2022detecting} was able to reliably detect the knobs and
handles and CoTracker~\cite{karaev2023cotracker} was able to successfully track
the point of interaction letting us solve 24/28 task instances.

\subsubsection{Erroneous detections without inpainting hamper performance on 
handles and our end-effector out-painting strategy effectively mitigates it} 
As shown in \figref{rollouts2}, presence of the end-effector caused the object
detector to miss fire leading to failed execution. Our out painting approach
mitigates this issue leading to a higher success rate than the 
approach without out-painting. Interestingly, CoTracker~\cite{karaev2023cotracker} is quite robust
to occlusion (possibly because it tracks multiple points) and doesn't benefit
from in-painting. 


\subsubsection{Closed-loop imitation learning struggles on novel objects}
As presented in \tableref{rum}, \name significantly outperforms RUM in a paired evaluation on unseen objects across three novel environments. A common failure mode of RUM is its inability to grasp the object's handle, even when it approaches it closely.
Another failure mode we observe is RUM misidentifying keyholes or cabinet edges as handles, also resulting in failed grasp attempts.
These result demonstrate that a modular approach that leverages the broad generalization capabilities of vision foundation models is able to generalize much better than an end-to-end imitation learning approach trained on 1000+ demonstrations, which must learn all aspects of the task from scratch.




\section{Conclusion}

In this paper, we propose a sample weight averaging strategy to address variance inflation of previous independence-based sample reweighting algorithms. 
We prove its validity and benefits with theoretical analyses. 
Extensive experiments across synthetic and multiple real-world datasets demonstrate its superiority in mitigating variance inflation and improving covariate-shift generalization.  


\section*{Impact Statement}
This paper offers a novel perspective, demonstrating the indispensable role of CoT in enhancing the generalization capabilities of LMs. Through theoretical analysis and comprehensive empirical experimentation, we establish CoT as a critical enabler of robust out-of-distribution generalization. Crucially, this work provides valuable guidance for the development of effective data curation strategies, specifically for collecting data that maximizes the benefits of CoT training. This guidance is directly applicable to the industrial deployment of LMs and the fine-tuning of large models for novel tasks, offering a pathway to improve the generalization and real-world utility of these models through informed data acquisition methodologies.
% In the unusual situation where you want a paper to appear in the
% references without citing it in the main text, use \nocite

\printbibliography
%\bibliography{reference}




%%%%%%%%%%%%%%%%%%%%%%%%%%%%%%%%%%%%%%%%%%%%%%%%%%%%%%%%%%%%%%%%%%%%%%%%%%%%%%%
%%%%%%%%%%%%%%%%%%%%%%%%%%%%%%%%%%%%%%%%%%%%%%%%%%%%%%%%%%%%%%%%%%%%%%%%%%%%%%%
% APPENDIX
%%%%%%%%%%%%%%%%%%%%%%%%%%%%%%%%%%%%%%%%%%%%%%%%%%%%%%%%%%%%%%%%%%%%%%%%%%%%%%%
%%%%%%%%%%%%%%%%%%%%%%%%%%%%%%%%%%%%%%%%%%%%%%%%%%%%%%%%%%%%%%%%%%%%%%%%%%%%%%%
\newpage
\appendix
\onecolumn
\section{Explain Compound task in formal definition}
As shown in the figure, original tree like data structure can be converted to an array which is easily feeded into language model.
\begin{figure}
    \centering
    \includegraphics[width=1\linewidth]{figure/cp_inro2.pdf}
    \caption{Step by Step explanation on definition \ref{def:CP}}
    \label{fig:cp2}
\end{figure}
\section{Code and Data}
We provide an anonymous page, you can following the instruction to generate data, training model and evaluation.
https://github.com/physicsru/Scaling-Curves-of-CoT-Granularity-for-Language-Model-Generalization

\section{Recap Condition Analysis}

\begin{theorem}[Outside Token Recap Condition under RoPE]
\label{thm: Recap 1}
Consider a transformer with Rotary Positional Embedding (RoPE) using angles $\theta_j = 10000^{-2j/d_{\text{model}}}$. Given finite computational precision $s$ and minimum resolvable attention score $\epsilon$, there exists a threshold distance $\tau > 0$ such that for all positional distances $d > \tau$:
\[
|A(d)| < \epsilon,
\]
where $A(d)$ is the attention score between tokens at distance $d$. Consequently, tokens beyond $[i_{\text{current}} - \tau, i_{\text{current}} + \tau]$ cannot be recalled.

\begin{proof}
\textbf{Step 1: Attention Score Formulation}  
The RoPE attention score between positions $m$ and $n$ (distance $d = |m-n|$) is:
\[
A(d) = \text{Re}\left[\sum_{j=0}^{d_{\text{model}}/2-1} h_j e^{\mathbf{i}d\theta_j}\right], \quad h_j := q_{[2j:2j+1]}\mathbf{k}^*_{[2j:2j+1]}
\]
where $h_j$ encodes query-key interactions for the $j$-th dimension pair.

\textbf{Step 2: Abel Transformation\cite{men2024baseropeboundscontext}}  
Let $S_{j+1} = \sum_{k=0}^j e^{\mathbf{i}d\theta_k}$ with $S_0 = 0$. Using summation by parts:
\[
\sum_{j=0}^{d_{\text{model}}/2-1} h_j e^{\mathbf{i}d\theta_j} = \sum_{j=0}^{d_{\text{model}}/2-1} (h_j - h_{j+1}) S_{j+1}
\]
Taking absolute values:
\[
|A(d)| \leq \sum_{j=0}^{d_{\text{model}}/2-1} |h_{j+1} - h_j| \cdot |S_{j+1}|
\]

\bigskip
\textbf{Step 3: Bounding Query-Key Differences.}  
Assume that the query and key representations are uniformly bounded so that $\|q\|,\,\|k\| \le C$. In particular, since each $h_j$ results from a dot product between sub-vectors from $q$ and $k$, we have $|h_j| \le C^2$. Moreover, if we assume that the embeddings vary smoothly with the index $j$ (as expected from the continuity of underlying network nonlinearities and weight matrices), then the mean value theorem implies that the difference
\[
|h_{j+1} - h_j|
\]
is bounded by a Lipschitz constant. That is, there exists a constant $M = \mathcal{O}(C^2)$ such that for every $j$
\[
|h_{j+1} - h_j| \le M.
\]
A more refined analysis might track this difference in terms of the network’s smoothness, but the key point is that each difference is uniformly bounded by a constant depending on $C$.

\bigskip
\textbf{Step 4: Analyzing Oscillatory Sums.}  
We now study the partial sum
\[
S_{j+1} = \sum_{k=0}^j e^{\mathbf{i}d\theta_k}.
\]
Two regimes are considered:

\emph{(i) Low-frequency regime ($k \le j_0$):}  
For sufficiently small indices $k$, we have $\theta_k$ being relatively large so that
\[
d\theta_k \gg 1.
\]
In this regime the phases $e^{\mathbf{i}d\theta_k}$ change rapidly with $k$, leading to cancellations among the terms. A standard bound for such oscillatory sums yields
\[
\Big|\sum_{k=0}^{j_0} e^{\mathbf{i}d\theta_k}\Big| \le \frac{2}{\left|1 - e^{\mathbf{i}d\theta_0}\right|} = \mathcal{O}(1),
\]
since the denominator remains bounded away from zero when $d\theta_0$ is large.

\emph{(ii) High-frequency regime ($k > j_0$):}  
For larger $k$, the angle $\theta_k$ becomes very small (because $\theta_k$ decays exponentially with $k$), so that $d\theta_k \ll 1$. In this case we can use the first-order Taylor expansion:
\[
e^{\mathbf{i}d\theta_k} \approx 1 + \mathbf{i}d\theta_k.
\]
Thus, for indices $j>j_0$, the sum becomes approximately
\[
\sum_{k=j_0+1}^j e^{\mathbf{i}d\theta_k} \approx (j-j_0) + \mathbf{i}d\sum_{k=j_0+1}^j \theta_k.
\]
While the real part grows (almost) linearly in the number of terms, the alternating phases and the small magnitude of $d\theta_k$ imply that additional cancellation occurs when the two regimes are combined. In the worst case one may bound
\[
|S_{j+1}| \le \mathcal{O}(\log d)
\]
by appealing to harmonic series type estimates; this bound is loose but captures the fact that the cancellation improves with larger $d$.

\bigskip
\textbf{Step 5: Combining the Results.}  
Returning to the bound from Step~2, we have
\[
|A(d)| \le \sum_{j=0}^{d_{\text{model}}/2-1} |h_{j+1} - h_j|\, |S_{j+1}|
\]
and using the bounds from Steps~3 and 4,
\[
|A(d)| \le M \sum_{j=0}^{d_{\text{model}}/2-1} \mathcal{O}(\log d) = \mathcal{O}(M\, d_{\text{model}} \log d).
\]
In practice, the alternating signs in the summands produce even stronger decay in $d$, and empirical observations suggest a polynomial decay of the form
\[
|A(d)| \le \mathcal{O}\Big(\frac{M}{\sqrt{d}}\Big).
\]

\bigskip
\textbf{Step 6: Precision Threshold.}  
For a given minimum resolvable attention score $\epsilon$, we require
\[
\frac{M}{\sqrt{d}} < \epsilon \quad \Longrightarrow \quad d > \left(\frac{M}{\epsilon}\right)^2.
\]
Defining
\[
\tau = \left(\frac{M}{\epsilon}\right)^2,
\]
we conclude that for any $d > \tau$, it holds that $|A(d)| < \epsilon$. That is, tokens beyond the window indexed by $[i_{\text{current}}-\tau,\, i_{\text{current}}+\tau]$ effectively contribute an attention score below the resolvable threshold.

\end{proof}

\textbf{Interpretation:}
\begin{itemize}
\item The $\theta_j$ schedule creates frequency-dependent decay: high frequencies (small $j$) attenuate rapidly
\item Window size $\tau \propto (M/\epsilon)^2$ explains memory limitations in long contexts
\item Practical implementations must balance $d_{\text{model}}$ and precision $s$ for optimal $\tau$
\end{itemize}
\end{theorem}

\begin{theorem}[Inside window recap condition under Rope]
\label{thm: Recap 2}
Assume: The true function is $s_2 = {w^{\circ}}^{\top} e(s_1) + \epsilon$, with $s_2 \perp s_x$ in expectation.
Consider two linear models:
\begin{itemize}
\item $M_s$: $y_s = {w_s}^{\top} e(s_1)$
\item $M_l$: $y_l = {w_1}^{\top} e(s_1) + {w_2}^{\top} e(s_x)$
\end{itemize}
Under mean-squared-error training:
\begin{itemize}
\item $g_s = \frac{\partial L_s}{\partial w_s}$
\item $(g_1, g_2) = \left(\frac{\partial L_l}{\partial w_1}, \frac{\partial L_l}{\partial w_2}\right)$
\end{itemize}
Then in finite-data regimes or with random noise, the gradient component of $g_1$ along $(w^{\circ} - w_1)$ is typically smaller than the corresponding component of $g_s$ along $(w^{\circ} - w_s)$
Formally,
\[
\mathbb{E}\left[ \left\langle \mathbf{g}_1, \mathbf{w}^\circ - \mathbf{w}_1 \right\rangle \right] < \mathbb{E}\left[ \left\langle \mathbf{g}_s, \mathbf{w}^\circ - \mathbf{w}_s \right\rangle \right],
\]
leading to slower convergence for \( \mathcal{M}_{l} \) when \( s_{x} \) is irrelevant.
\end{theorem}
%\input{proof_inside_recap}

\begin{proof}
We analyze the gradients of both models and demonstrate how irrelevant features in \( \mathcal{M}_l \) reduce the effective gradient signal.

\noindent \textbf{Step 1: Express Gradients for Both Models}

For model \( \mathcal{M}_s \), the loss is:
\[
L_s = \mathbb{E}\left[(s_2 - \mathbf{w}_s^\top \mathbf{e}(s_1))^2\right]
\]
The gradient becomes:
\begin{equation}
\mathbf{g}_s = -2 \mathbb{E}\left[\mathbf{e}(s_1)\mathbf{e}(s_1)^\top\right](\mathbf{w}^\circ - \mathbf{w}_s)
\label{eq:grad_s}
\end{equation}

For model \( \mathcal{M}_l \), the gradient for \( \mathbf{w}_1 \) is:
\begin{equation}
\mathbf{g}_1 = -2 \mathbb{E}\left[\mathbf{e}(s_1)\mathbf{e}(s_1)^\top\right](\mathbf{w}^\circ - \mathbf{w}_1) + 2 \mathbb{E}\left[\mathbf{w}_2^\top \mathbf{e}(s_x)\mathbf{e}(s_1)\right]
\label{eq:grad_l}
\end{equation}

\noindent \textbf{Step 2: Compare Gradient Components}

The inner product for \( \mathcal{M}_l \) contains two terms:
\begin{equation}
\begin{aligned}
    \mathbb{E}\left[\langle \mathbf{g}_1, \mathbf{w}^\circ - \mathbf{w}_1 \rangle \right] 
    &= \underbrace{-2 (\mathbf{w}^\circ - \mathbf{w}_1)^\top \mathbb{E}\left[\mathbf{e}(s_1)\mathbf{e}(s_1)^\top\right] (\mathbf{w}^\circ - \mathbf{w}_1)}_{\text{Matches } \mathcal{M}_s} \\
    &\quad + \underbrace{2 \mathbb{E}\left[\mathbf{w}_2^\top \mathbf{e}(s_x)\mathbf{e}(s_1)^\top (\mathbf{w}^\circ - \mathbf{w}_1)\right]}_{\text{Additional term}}
\end{aligned}
\end{equation}


\noindent \textbf{Step 3: Effect of Irrelevant Features}

Since \( s_x \) is irrelevant (\( s_2 \perp s_x \)):
\begin{itemize}
\item Population truth: \( \mathbf{w}_2 = \mathbf{0} \)
\item Finite data allows \( \mathbf{w}_2^\epsilon \) to fit noise \( \epsilon \)
\item Induces spurious correlation: \( \mathbb{E}\left[\mathbf{e}(s_x)\mathbf{e}(s_1)^\top\right] \neq \mathbf{0} \)
\end{itemize}

This makes the additional term:
\[
2 \mathbb{E}\left[\mathbf{w}_2^\top \mathbf{e}(s_x)\mathbf{e}(s_1)^\top (\mathbf{w}^\circ - \mathbf{w}_1)\right] \neq 0
\]
which \textit{reduces} the magnitude of the gradient component.

\noindent \textbf{Step 4: Convergence Comparison}

For \( \mathcal{M}_s \):
\[
\mathbb{E}\left[\langle \mathbf{g}_s, \mathbf{w}^\circ - \mathbf{w}_s \rangle \right] = -2 (\mathbf{w}^\circ - \mathbf{w}_s)^\top \mathbb{E}\left[\mathbf{e}(s_1)\mathbf{e}(s_1)^\top\right] (\mathbf{w}^\circ - \mathbf{w}_s)
\]

For \( \mathcal{M}_l \), the additional term in Equation~\ref{eq:grad_l} creates:
\[
\mathbb{E}\left[ \left\langle \mathbf{g}_1, \mathbf{w}^\circ - \mathbf{w}_1 \right\rangle \right] < \mathbb{E}\left[ \left\langle \mathbf{g}_s, \mathbf{w}^\circ - \mathbf{w}_s \right\rangle \right]
\]

\noindent \textbf{Conclusion} \\
The irrelevant features in \( \mathcal{M}_l \) reduce the gradient component in the direction of \( \mathbf{w}^\circ \), leading to slower convergence compared to \( \mathcal{M}_s \).
\end{proof}


\section{CoT simulates the target solution}

\begin{definition}[Chain of Thought]
  \vspace{0em}
    \small\begin{align*}
    \vspace{-0.5em}
        \text{Input:} & \; s_1 \mid \cdots \mid s_N \\[1ex]
        \text{CoT Steps:} & \; \langle\text{sep}\rangle \; s_2 \mid G_2 \mid q_2 \mid L_1 \mid L_2 \\
        & \; \langle\text{sep}\rangle \; \cdots \\
        & \; \langle\text{sep}\rangle \; s_{N} \mid G_N \mid q_N \mid L_{N-1} \mid L_N
    \end{align*}
    \label{def:CoT_appendix}
\end{definition}

\begin{proposition}
\label{thm:CoT_implementation}
For any compound problem satisfying Definition \ref{def:CP}, and for any input length bound $n \in \mathbb{N}$, there exists an autoregressive Transformer with:
\begin{itemize}
\item Constant depth $L$
\item Constant hidden dimension $d$
\item Constant number of attention heads $H$
\end{itemize}
where $L$, $d$, and $H$ are independent of $n$, such that the Transformer correctly generates the Chain-of-Thought solution defined in Definition \ref{def:CoT} for all input sequences of length at most $n$. Furthermore, all parameter values in the Transformer are bounded by $O(\text{poly}(n))$.
\end{proposition}

\subsection{Constructive Proof}
We prove this theorem by constructing a Transformer architecture with 4 blocks, where each block contains multiple attention heads and feed-forward networks (FFNs). The key insight is that we can simulate each step of the Chain-of-Thought solution using a fixed number of attention heads and a fixed embedding dimension.
The attention mechanism is primarily used to select and retrieve relevant elements from the input and previous computations, while the FFNs approximate the required functions $G$, $B$, etc. By maintaining constant depth, width, and number of heads per layer, we ensure the Transformer's architecture remains independent of the input length, while still being able to generate arbitrarily long Chain-of-Thought solutions.
The parameter complexity of $O(\text{poly}(n))$ arises from the need to handle inputs and intermediate computations of length $n$, but importantly, this only affects the parameter values and not the model architecture itself.

\subsection{Embedding Structure}
For position $k$, define the input embedding:
\begin{equation*}
    x^{(0)}_k=(e^{\text{isInput}}_k, e^{\text{isState}}_k, e^{\text{isDependence}}, e^{\text{isL}}, e^{\text{q}}_k, e^{\text{d}}_{k}, e^{\text{L}}_k, e^{\text{sep}}_k, e^{\text{step}}_k, k,1)
\end{equation*}
where:
\begin{itemize}
    \item $e^{\text{isInput}}_k \in \{0,1\}$: Input token indicator
    \item $e^{\text{isState}}_k \in \{0,1\}$: State position indicator
    \item $e^{\text{isDependence}} \in \{0,1\}$: Dependency marker
    \item $e^{\text{isL}} \in \{0,1\}$: Aggregation result indicator
    \item $e^{\text{q}}_k \in \mathbb{R}^{d_q}$: State value embedding
    \item $e^{\text{d}}_{k} \in \mathbb{R}^{d_d}$: Dependency graph embedding
    \item $e^{\text{L}}_k \in \mathbb{R}^{d_L}$: Aggregation value embedding
    \item $e^{\text{sep}}_k \in \{0,1\}$: Step separator indicator
    \item $e^{\text{step}}_k \in \mathbb{N}$: Current step index
    \item $k \in \mathbb{N}$: Position encoding
    \item $1$: Bias term
\end{itemize}

\subsection{Block Constructions}

\subsubsection{Block 1: Input Processing and State Identification}
Define attention heads $A^{(1)}_1, A^{(1)}_2, A^{(1)}_3$ with parameters:
\begin{align*}
    Q^{(1)}_1 &= W^q_1[e^{\text{isInput}}_k] \\ 
    K^{(1)}_1 &= W^k_1[e^{\text{isInput}}_j]_{j<k} \\
    V^{(1)}_1 &= W^v_1[j]_{j<k}
\end{align*}

The second head tracks state positions:
\begin{align*}
    Q^{(1)}_2 &= W^q_2[e^{\text{isState}}_k] \\
    K^{(1)}_2 &= W^k_2[e^{\text{isState}}_j]_{j<k} \\
    V^{(1)}_2 &= W^v_2[j]_{j<k}
\end{align*}

The third head tracks step indices through separators:
\begin{align*}
    Q^{(1)}_3 &= W^q_3[e^{\text{sep}}_k] \\
    K^{(1)}_3 &= W^k_3[e^{\text{sep}}_j]_{j<k} \\
    V^{(1)}_3 &= W^v_3[\text{count}(e^{\text{sep}}_j)]_{j<k}
\end{align*}

\begin{lemma}
The first block correctly identifies positions through attention scoring:
\begin{enumerate}
    \item For input positions, $A^{(1)}_1$ scoring gives:
    \begin{equation*}
        \text{score}_1(q_k, k_j) = \begin{cases}
        1 & \text{if } e^{\text{isInput}}_j = 1 \\
        0 & \text{otherwise}
        \end{cases}
    \end{equation*}
    Thus $V^{(1)}_1$ returns positions of input tokens

    \item For state positions, $A^{(1)}_2$ scoring gives:
    \begin{equation*}
        \text{score}_2(q_k, k_j) = \begin{cases}
        1 & \text{if } e^{\text{isState}}_j = 1 \\
        0 & \text{otherwise}
        \end{cases}
    \end{equation*} 
    Thus $V^{(1)}_2$ returns positions of states

    \item For step indices, $A^{(1)}_3$ counts separators up to position k:
    \begin{equation*}
        \text{count}(e^{\text{sep}}_j) = \sum_{l \leq j} e^{\text{sep}}_l
    \end{equation*}
    Thus $V^{(1)}_3$ returns the current step index
\end{enumerate}
\end{lemma}

\subsubsection{Block 2: Dependency Graph Construction} 
Define three attention heads $A^{(2)}_1, A^{(2)}_2, A^{(2)}_3$ implementing dependency selection:
\begin{align*}
    A^{(2)}_1&: Q^{(2)}_1 = W^q_2[e^{\text{step}}_k] \\
    &K^{(2)}_1 = W^k_2[e^{\text{input}}_j]_{j<k} \\
    &V^{(2)}_1 = W^v_2[j]_{j<k} \\
    A^{(2)}_2&: Q^{(2)}_2 = W^q_3[e^{\text{step}}_k] \\
    &K^{(2)}_2 = W^k_3[e^{\text{step}}_j]_{j<k} \\
    &V^{(2)}_2 = W^v_3[B(s_1,\ldots,s_{i+1}, i+1)]_{j<k} \\
    A^{(2)}_3&: Q^{(2)}_3 = W^q_4[e^{\text{step}}_k] \\
    &K^{(2)}_3 = W^k_4[j]_{j<k} \\
    &V^{(2)}_3 = W^v_4[e^{\text{q}}_j]_{j<k}
\end{align*}

\begin{lemma}
Block 2 correctly implements $G_{i+1} = \{q_k | k \in B(s_1,\ldots,s_{i+1}, i+1)\}$ through:

1. First attention head $A^{(2)}_1$ gathers input sequence up to current step i+1:
\begin{equation*}
    z^{(2)}_1 = \{s_j | j \leq i+1\}
\end{equation*}

2. Second attention head $A^{(2)}_2$ computes indices from B using gathered inputs:
\begin{equation*}
    z^{(2)}_2 = B(z^{(2)}_1, i+1)
\end{equation*}

3. Third attention head $A^{(2)}_3$ selects states using computed indices:
\begin{equation*}
    z^{(2)}_3 = \{e^{\text{q}}_j | j \in z^{(2)}_2\}
\end{equation*}

Therefore, the composition $z^{(2)}_3(z^{(2)}_2(z^{(2)}_1))$ correctly implements $G_{i+1}$ by:
\begin{enumerate}
    \item Gathering relevant input sequence 
    \item Computing dependency indices using B
    \item Selecting corresponding states
\end{enumerate}

The correctness follows from attention scoring:
\begin{align*}
    \text{score}_1(q_k, k_j) &= \begin{cases}
        1 & \text{if } j \leq i+1 \\
        0 & \text{otherwise}
    \end{cases} \\
    \text{score}_2(q_k, k_j) &= \begin{cases}
        1 & \text{if } j \in B(s_1,\ldots,s_{i+1}, i+1) \\
        0 & \text{otherwise}
    \end{cases} \\
    \text{score}_3(q_k, k_j) &= \begin{cases}
        1 & \text{if } j \in z^{(2)}_2 \\
        0 & \text{otherwise}
    \end{cases}
\end{align*}
\end{lemma}

\subsubsection{Block 3: State Transition}
Define attention mechanism implementing $F$:
\begin{align*}
    A^{(3)}_1&: Q^{(3)}_1 = W^q_3[e^{\text{isState}}_k] \\
    &K^{(3)}_1 = W^k_3[e^{\text{isDependence}}_j]_{j<k} \\
    &V^{(3)}_1 = W^v_3[e^{\text{q}}_j]_{j<k} \\
    A^{(3)}_2&: Q^{(3)}_2 = W^q_4[e^{\text{isState}}_k] \\
    &K^{(3)}_2 = W^k_4[e^{\text{isInput}}_j]_{j<k} \\
    &V^{(3)}_2 = W^v_4[e^{\text{input}}_j]_{j<k}
\end{align*}

\begin{lemma}
The state transition function $F$ is correctly computed through:
\begin{equation*}
    q_{i+1} = F(G_{i+1}, s_{i+1}) = \text{FFN}(z^{(3)}_1, z^{(3)}_2)
\end{equation*}
where $z^{(3)}_1 = A^{(3)}_1(e^{\text{q}}_j \mid j \in B(s_1,\ldots,s_{i+1}, i+1))$ represents the states selected by $G_{i+1}$ from Block 2, and $z^{(3)}_2 = A^{(3)}_2(s_{i+1})$ represents the current input token.
\end{lemma}

\subsubsection{Block 4: Result Aggregation}
Define two attention heads $A^{(4)}_1, A^{(4)}_2$ for implementing $H$:
\begin{align*}
    A^{(4)}_1&: Q^{(4)}_1 = W^q_4[e^{\text{isL}}_k] \\
    &K^{(4)}_1 = W^k_4[e^{\text{isL}}_j]_{j<k} \\
    &V^{(4)}_1 = W^v_4[e^{\text{L}}_j]_{j<k} \\
    A^{(4)}_2&: Q^{(4)}_2 = W^q_5[e^{\text{isL}}_k] \\
    &K^{(4)}_2 = W^k_5[e^{\text{isState}}_j]_{j<k} \\
    &V^{(4)}_2 = W^v_5[e^{\text{q}}_j]_{j<k}
\end{align*}

\begin{lemma}
Block 4 correctly implements the aggregation function $H$ through:

1. For $i=1$ (base case):
\begin{equation*}
    \text{score}_1(q_k, k_j) = 0, \quad \text{score}_2(q_k, k_j) = \begin{cases}
        1 & \text{if } e^{\text{isState}}_j = 1 \\
        0 & \text{otherwise}
    \end{cases}
\end{equation*}
Therefore $L_1 = H(\emptyset, q_1) = q_1$ since only $A^{(4)}_2$ activates to select $q_1$

2. For $i>1$:
\begin{equation*}
    \text{score}_1(q_k, k_j) = \begin{cases}
        1 & \text{if } e^{\text{isL}}_j = 1 \text{ and j is the latest L position} \\
        0 & \text{otherwise}
    \end{cases}
\end{equation*}
\begin{equation*}
    \text{score}_2(q_k, k_j) = \begin{cases}
        1 & \text{if } e^{\text{isState}}_j = 1 \text{ and j corresponds to } q_i \\
        0 & \text{otherwise}
    \end{cases}
\end{equation*}

Therefore:
\begin{align*}
    z^{(4)}_1 &= A^{(4)}_1(e^{\text{L}}_k) = L_{i-1} \text{ (previous aggregation result)} \\
    z^{(4)}_2 &= A^{(4)}_2(e^{\text{q}}_k) = q_i \text{ (current state)} \\
    L_i &= \text{FFN}(z^{(4)}_1, z^{(4)}_2) = H(L_{i-1}, q_i)
\end{align*}

The FFN is constructed to implement the specific aggregation operation of $H$ (e.g., max, min, or sum).
\end{lemma}

\begin{proposition}[Block Transitions]
The blocks connect sequentially where:
\begin{enumerate}
    \item Block 1 output provides input positions, state positions and step indices
    \item Block 2 implements dependency function $G$ to gather required states
    \item Block 3 uses gathered dependencies and current input to compute new states via $F$
    \item Block 4 implements $H$ to aggregate states into final result
\end{enumerate}
Each transition preserves information through residual connections.
\end{proposition}

\section{Proof for Theorem \ref{thm:kl-reduction}}


\begin{proof}
We prove the KL divergence bound by decomposing the distributions over covered and uncovered prefixes. Let $P_{\text{train}}^{\text{Q-CoT}}$ and $P_{\text{eval}}^{\text{Q-CoT}}$ denote the training and evaluation distributions under Q-CoT, respectively. 

\vspace{0.5em}

\noindent \textbf{Step 1: Event Space Partitioning}

Define two disjoint events for any evaluation sample $x = (X^{n_3}, \{q_i^{(n_3)}\}, Y^{n_3})$:
\begin{itemize}
    \item $\mathcal{E}_{\text{cover}}$: The prefix $\{q_i^{(n_3)}\}_{i=1}^{n_3}$ exists in some length-$n_2$ training sample.
    \item $\mathcal{E}_{\text{uncover}}$: The prefix $\{q_i^{(n_3)}\}_{i=1}^{n_3}$ is absent from all training samples.
\end{itemize}
By Lemma \ref{thm:prefix-substructure}, $\mathcal{E}_{\text{cover}}$ occurs when the evaluation prefix matches at least one length-$n_2$ training sequence's prefix. The probabilities satisfy:
\[
P_{\text{cover}} = \mathbb{P}(\mathcal{E}_{\text{cover}}), \quad 1 - P_{\text{cover}} = \mathbb{P}(\mathcal{E}_{\text{uncover}}).
\]
where $P_{\text{cover}}$ is calculated as:
\[
P_{\text{cover}} = \frac{m_2}{m_3 k^{n_3}}
\]
\textit{Derivation}: Each length-$n_2$ training sample contains a unique prefix of length $n_3$ (Lemma \ref{thm:prefix-substructure}). With $m_2$ samples, we can cover $m_2$ distinct prefixes. The total number of possible prefixes is $m_3 k^{n_3}$ ($m_3$ evaluation problems, each with $k^{n_3}$ possible prefixes). Thus, the coverage probability follows the ratio.

\vspace{0.5em}

\noindent \textbf{Step 2: Distributional Decomposition}

Using the law of total probability, we express:
\[
P_{\text{eval}}^{\text{Q-CoT}} = P_{\text{cover}} \cdot P_{\text{eval}|\mathcal{E}_{\text{cover}}} + (1-P_{\text{cover}}) \cdot P_{\text{eval}|\mathcal{E}_{\text{uncover}}}
\]
\[
P_{\text{train}}^{\text{Q-CoT}} = P_{\text{cover}} \cdot P_{\text{train}|\mathcal{E}_{\text{cover}}} + (1-P_{\text{cover}}) \cdot P_{\text{train}|\mathcal{E}_{\text{uncover}}}
\]
where:
\begin{itemize}
    \item $P_{\text{eval}|\mathcal{E}_{\text{cover}}}$: Evaluation distribution restricted to covered prefixes
    \item $P_{\text{train}|\mathcal{E}_{\text{cover}}}$: Training distribution restricted to covered prefixes
    \item $P_{\text{eval}|\mathcal{E}_{\text{uncover}}}$: Evaluation distribution for uncovered prefixes
    \item $P_{\text{train}|\mathcal{E}_{\text{uncover}}}$: Training distribution for uncovered prefixes
\end{itemize}

\vspace{0.5em}

\noindent \textbf{Step 3: KL Divergence Expansion with Total Expectation}

From the KL divergence definition:
\[
D_{\mathrm{KL}}\left(P_{\text{eval}}^{\text{Q-CoT}} \,\big\|\, P_{\text{train}}^{\text{Q-CoT}}\right) = \mathbb{E}_{x \sim P_{\text{eval}}^{\text{Q-CoT}}} \left[ \log \frac{P_{\text{eval}}^{\text{Q-CoT}}(x)}{P_{\text{train}}^{\text{Q-CoT}}(x)} \right]
\]
Apply the law of total expectation by conditioning on $\mathcal{E}_{\text{cover}}$ and $\mathcal{E}_{\text{uncover}}$:
\[
= \mathbb{P}(\mathcal{E}_{\text{cover}}) \cdot \mathbb{E}_{x|\mathcal{E}_{\text{cover}}} \left[ \log \frac{P_{\text{eval}}^{\text{Q-CoT}}(x|\mathcal{E}_{\text{cover}})}{P_{\text{train}}^{\text{Q-CoT}}(x|\mathcal{E}_{\text{cover}})} \right] + \mathbb{P}(\mathcal{E}_{\text{uncover}}) \cdot \mathbb{E}_{x|\mathcal{E}_{\text{uncover}}} \left[ \log \frac{P_{\text{eval}}^{\text{Q-CoT}}(x|\mathcal{E}_{\text{uncover}})}{P_{\text{train}}^{\text{Q-CoT}}(x|\mathcal{E}_{\text{uncover}})} \right]
\]

\vspace{0.5em}

\noindent \textbf{Step 4: Handling Covered Cases}

Under $\mathcal{E}_{\text{cover}}$, Lemma \ref{thm:prefix-substructure} guarantees that the CoT states $\{q_i^{(n_3)}\}$ in evaluation samples exactly match those in training samples. This implies:
\[
P_{\text{eval}|\mathcal{E}_{\text{cover}}}(x) = P_{\text{train}|\mathcal{E}_{\text{cover}}}(x), \quad \forall x \in \mathcal{E}_{\text{cover}}
\]
Therefore:
\[
\mathbb{E}_{x|\mathcal{E}_{\text{cover}}} \left[ \log \frac{P_{\text{eval}|\mathcal{E}_{\text{cover}}}}{P_{\text{train}|\mathcal{E}_{\text{cover}}}} \right] = \mathbb{E}_{x|\mathcal{E}_{\text{cover}}} [\log 1] = 0
\]

\vspace{0.5em}

\noindent \textbf{Step 5: Uncovered Cases Reduce to Q-A}

For $x = (X^{n_3}, \{q_i^{(n_3)}\}, Y^{n_3}) \in \mathcal{E}_{\text{uncover}}$, the absence of matching prefixes in training data implies the model cannot leverage CoT states $\{q_i^{(n_3)}\}$ during inference. We formally analyze this degradation:

%\vspace{0.5em}

%\noindent **5.1 Conditional Distribution Decomposition**

Under Q-CoT, the generation process factors as:
\[
P^{\text{Q-CoT}}(Y|X) = \sum_{\{q_i\}} P(Y|X, \{q_i\}) P(\{q_i\}|X)
\]
where:
\begin{itemize}
    \item $P(\{q_i\}|X)$: Probability of generating CoT states $\{q_i\}$ given input $X$
    \item $P(Y|X, \{q_i\})$: Probability of answer $Y$ given $X$ and CoT states
\end{itemize}

%\vspace{0.5em}

%\noindent **5.2 Uncovered Case Analysis**

When $\{q_i^{(n_3)}\}$ is uncovered ($\mathcal{E}_{\text{uncover}}$), the model lacks training data to estimate either:
\begin{itemize}
    \item The CoT state distribution $P(\{q_i\}|X)$
    \item The answer likelihood $P(Y|X, \{q_i\})$ 
\end{itemize}

Thus, the model \textit{cannot} utilize the CoT decomposition and must marginalize over all possible $\{q_i\}$:
\[
P^{\text{Q-CoT}}(Y|X) = \mathbb{E}_{\{q_i\} \sim P(\{q_i\}|X)} \left[ P(Y|X, \{q_i\}) \right]
\]

%\vspace{0.5em}

%\noindent **5.3 Degeneration to Q-A**

Without CoT supervision on $\{q_i^{(n_3)}\}$, two condition assumes:
\begin{enumerate}
    \item \textbf{Untrained CoT States}: If $\{q_i^{(n_3)}\}$ never appears in training, $P(\{q_i\}|X)$ becomes a \textit{uniform prior} over possible CoT sequences (by maximum entropy principle).
    
    \item \textbf{Uninformative Likelihood}: The answer likelihood $P(Y|X, \{q_i\})$ reduces to $P^{\text{Q-A}}(Y|X)$ because the model cannot associate $\{q_i\}$ with $Y$ without training signals.
\end{enumerate}

Thus:
\[
P^{\text{Q-CoT}}(Y|X) = \sum_{\{q_i\}} \underbrace{P^{\text{Q-A}}(Y|X)}_{\text{Uninformative}} \cdot \underbrace{\frac{1}{k^{n_{3}}}}_{\text{Uniform } P(\{q_i\}|X)} = P^{\text{Q-A}}(Y|X)
\]

with expansion of KL divergence of Q-A
\[
D_{\mathrm{KL}}\left(P_{\text{eval}}^{\text{Q-A}} \,\big\|\, P_{\text{train}}^{\text{Q-A}}\right) = \mathbb{E}_{x \sim P_{\text{eval}}^{\text{Q-A}}} \left[ \log \frac{P_{\text{eval}}^{\text{Q-A}}(x)}{P_{\text{train}}^{\text{Q-A}}(x)} \right]
\] 
\[
= \mathbb{P}(\mathcal{E}_{\text{cover}}) \cdot \mathbb{E}_{x|\mathcal{E}_{\text{cover}}} \left[ \log \frac{P_{\text{eval}}^{\text{Q-A}}(x|\mathcal{E}_{\text{cover}})}{P_{\text{train}}^{\text{Q-A}}(x|\mathcal{E}_{\text{cover}})} \right] + \mathbb{P}(\mathcal{E}_{\text{uncover}}) \cdot \mathbb{E}_{x|\mathcal{E}_{\text{uncover}}} \left[ \log \frac{P_{\text{eval}}^{\text{Q-A}}(x|\mathcal{E}_{\text{uncover}})}{P_{\text{train}}^{\text{Q-A}}(x|\mathcal{E}_{\text{uncover}})} \right]
\]
Notice that
\[
\mathbb{E}_{x|\mathcal{E}_{\text{cover}}} \left[ \log \frac{P_{\text{eval}}^{\text{Q-A}}(x|\mathcal{E}_{\text{cover}})}{P_{\text{train}}^{\text{Q-A}}(x|\mathcal{E}_{\text{cover}})} \right]
\leq
\mathbb{E}_{x|\mathcal{E}_{\text{uncover}}} \left[ \log \frac{P_{\text{eval}}^{\text{Q-A}}(x|\mathcal{E}_{\text{uncover}})}{P_{\text{train}}^{\text{Q-A}}(x|\mathcal{E}_{\text{uncover}})} \right]
\]
since covered prefix will decrease the KL divergence via probability decomposition
\[
D_{\mathrm{KL}}\left(P_{\text{eval}}^{\text{Q-A}} \,\big\|\, P_{\text{train}}^{\text{Q-A}}\right) \geq 
\mathbb{E}_{x|\mathcal{E}_{\text{uncover}}} \left[ \log \frac{P_{\text{eval}}^{\text{Q-A}}(x|\mathcal{E}_{\text{uncover}})}{P_{\text{train}}^{\text{Q-A}}(x|\mathcal{E}_{\text{uncover}})} \right] = D_{\mathrm{KL}}\left(P_{\text{eval}|\mathcal{E}_{\text{uncover}}}^{\text{Q-A}} \,\big\|\, P_{\text{train}|\mathcal{E}_{\text{uncover}}}^{\text{Q-A}}\right)
\]

Therefore, for $x \in \mathcal{E}_{\text{uncover}}$:
\[
D_{\mathrm{KL}}\left(P_{\text{eval}|\mathcal{E}_{\text{uncover}}} \,\big\|\, P_{\text{train}|\mathcal{E}_{\text{uncover}}}\right) = 
D_{\mathrm{KL}}\left(P_{\text{eval}|\mathcal{E}_{\text{uncover}}}^{\text{Q-A}} \,\big\|\, P_{\text{train}|\mathcal{E}_{\text{uncover}}}^{\text{Q-A}}\right) \leq 
D_{\mathrm{KL}}\left(P_{\text{eval}}^{\text{Q-A}} \,\big\|\, P_{\text{train}}^{\text{Q-A}}\right) = \mathrm{KL}_{\text{base}}
\]

\vspace{0.5em}

\noindent \textbf{Step 6: Final Inequality}

Combining all terms:
\[
D_{\mathrm{KL}}\left(P_{\text{eval}}^{\text{Q-CoT}} \,\big\|\, P_{\text{train}}^{\text{Q-CoT}}\right) = \underbrace{P_{\text{cover}} \cdot 0}_{\text{Covered term}} + \underbrace{(1-P_{\text{cover}}) \cdot \mathrm{KL}_{\text{base}}}_{\text{Uncovered term}}
\]
Hence:
\[
D_{\mathrm{KL}}\left(P_{\text{eval}}^{\text{Q-CoT}} \,\big\|\, P_{\text{train}}^{\text{Q-CoT}}\right) \leq (1 - P_{\text{cover}}) \cdot \mathrm{KL}_{\text{base}}
\]
The equality holds when $P_{\text{cover}} \in [0,1]$. When $m_2 = m_3 k^{n_3}$, we have $P_{\text{cover}} = 1$, making the KL divergence zero.
\end{proof}


\section{Quantitation Analysis Drop of CoT}
    \begin{theorem}[CoT Accuracy Degradation]
    Let $s_{\text{input}}$ be the input text, $s_{\text{ans}}$ be the unique correct answer, and $s_1, \dots, s_k$ be the \textit{exact required sequence} of perfect Chain-of-Thought (CoT) tokens where:
    \begin{enumerate}
        \item \textbf{Completeness}: $P(s_{\text{ans}} \mid s_1, \dots, s_k, s_{\text{input}}) = 1$
        \item \textbf{Uniqueness}: No other token sequence produces $s_{\text{ans}}$
        \item \textbf{Conditional Independence}: $P(s_1, \dots, s_k \mid s_{\text{input}}) = \prod_{i=1}^k P(s_i \mid s_{\text{input}})$
        \item \textbf{Training Deficiency}: For any CoT token $s_j$ excluded during training, $P(s_j \mid s_{\text{input}})$ drops from 1 to $1 - \epsilon$
    \end{enumerate}
    When $l < k$ CoT tokens are lost/mishandled during inference, the final answer accuracy satisfies:
    \[
    P(s_{\text{ans}} \mid s_{\text{input}}) = (1 - \epsilon)^l
    \]
    \label{thm:drop_CoT}
\end{theorem}

\begin{proof}
By the uniqueness condition, only the full sequence $s_1, \dots, s_k$ guarantees $s_{\text{ans}}$. Let $\mathcal{L}$ be the set of $l$ compromised tokens. The probability of maintaining correctness is:

\[
P(s_{\text{ans}} \mid s_{\text{input}}) = \underbrace{\prod_{j \in \mathcal{L}} P(s_j \mid s_{\text{input}})}_{\text{Lost tokens}} \cdot \underbrace{\prod_{i \notin \mathcal{L}} P(s_i \mid s_{\text{input}})}_{\text{Preserved tokens}}
\]

For preserved tokens ($i \notin \mathcal{L}$), full training ensures $P(s_i \mid s_{\text{input}}) = 1$. For lost tokens ($j \in \mathcal{L}$), training deficiency gives $P(s_j \mid s_{\text{input}}) = 1 - \epsilon$. Thus:

\[
P(s_{\text{ans}} \mid s_{\text{input}}) = (1 - \epsilon)^l \cdot 1^{k-l} = (1 - \epsilon)^l
\]

This equality holds because any deviation from the exact CoT sequence (due to lost tokens) eliminates the chance of correctness by the uniqueness condition.
\end{proof}
\subsection{Experiments}
\subsubsection{LIS}
Chain of thought is like the following: 
\[
\begin{aligned}
    &48 \quad 49 \quad 26 \quad 47 <sep> \\
    &48 | <empty> = 48 \quad 1 : 1 \rightarrow 1 <sep>\\
    &49 | 48 \quad 1 = 49 \quad 2 : 1 \rightarrow 2 <sep> \\
    &26 | <empty> = 26 \quad 1 : 2 \rightarrow 2 <sep>\\
    &47 | 26 \quad 1 = 47 \quad 2 : 2 \rightarrow 2
    \end{aligned}
\]

\subsubsection{MPC}
Chain of thought is like following:
\[
\begin{aligned}
 &0 \quad 1 \quad 1 \quad 0 \quad 0 \quad 1 \quad 1 0 , 8 <sep> \\
 &1 , 0 , 1 \rightarrow 0 <sep> \\
 &2 , 1 , 1 \quad 0 \rightarrow 1 <sep> \\
 &3 , 1 , 1 \quad 0 \quad 1 \rightarrow 2 <sep> \\
 &4 , 0 , 0 \quad 1 \quad 2 \rightarrow 0 <sep> \\
 &5 , 0 , 1 \quad 2 \quad 0 \rightarrow 0 <sep> \\
 &6 , 1 , 2 \quad 0 \quad 0 \rightarrow 2 <sep> \\
 &7 , 1 , 0 \quad 0 \quad 2 \rightarrow 2 <sep> \\
 &8 , 0 , 0 \quad 2 \quad 2 \rightarrow 0
\end{aligned}
\]
\subsubsection{Equation Restoration and Variable Computation}
Input is:
Data:
$data_1: Condor = 6, Cheetah = 1.$ \\
$data_2: Condor = 12, Cheetah = 3.$ \\
Question:
Assume all relations between variables are linear combinations. If the number of Cheetah equals 5, then what is the number of Condor?

Question:
Assume all relations between variables are linear combinations. If the number of Leopard equals 5, the number of Rhino equals 3, the number of Koala equals 6, then what is the number of Black\_Bear?
%\vspace{-\baselineskip}
%\begin{equation}

\textbf{Solution\:}

\textbf{Defining Variables} \\
\textit{Known Variables:} \\
Cheetah as \( c_1 = 5 \) \\

\textit{Unknown Variables:} \\
Target Variable: Condor as \( c_2 \) \\

\textbf{Restoring Relations} \\
\textit{List all variable names in each data point:} \([c_2, c_1], [c_2, c_1]\) \\
\textit{Deduplicate them:} \([c_2, c_1]\) \\
There is 1 distinct group, implying 1 distinct linear relationship to be determined. \\
\textit{Examining each relationship:} \\

\textbf{Relation 1:} \\
Exploring relation for \( c_2 \): \\
There are 2 variables in the data beginning with \( c_2 \): Hence, 2 coefficients are required, and at least 2 data points are needed. \\

Let the coefficients on the right side of the equation be \( K_1 \) and \( K_2 \). \\
\textit{Recap variables:} \(['c_2', 'c_1']\) \\
\textit{Define the equation of relation 1:} \\
\( c_2 = K_1 \cdot c_1 + K_2 \) \\

Using data points \( \text{data}_1 \) and \( \text{data}_2 \): \\
\( \text{data}_1: c_2 = 6, c_1 = 1 \) \\
Equation 1: \( 6 = K_1 \cdot 1 + K_2 \) \\
\( \text{data}_2: c_2 = 12, c_1 = 3 \) \\
Equation 2: \( 12 = K_1 \cdot 3 + K_2 \) \\

\textbf{Solve the system of equations using Gaussian Elimination:} \\
\textit{Initialize:} \\
Equation 1: \( 1 \cdot K_1 + 1 \cdot K_2 = 6 \) \\
Equation 2: \( 3 \cdot K_1 + 1 \cdot K_2 = 12 \) \\

Swap Equation 1 with Equation 2: \\
Equation 1: \( 3 \cdot K_1 + 1 \cdot K_2 = 12 \) \\
Equation 2: \( 1 \cdot K_1 + 1 \cdot K_2 = 6 \) \\

Multiply Equation 1 by 1 and subtract 3 times Equation 2: \\
\((\text{Equation 1}) \cdot 1: 3 \cdot K_1 + 1 \cdot K_2 = 12 \) \\
\((\text{Equation 2}) \cdot 3: 3 \cdot K_1 + 3 \cdot K_2 = 18 \) \\
New Equation 2: \( -2 \cdot K_2 = -6 \) \\

\textit{Recap updated equations:} \\
Equation 1: \( 3 \cdot K_1 + 1 \cdot K_2 = 12 \) \\
Equation 2: \( -2 \cdot K_2 = -6 \) \\

\textbf{Solve for \( K_2 \):} \\
\( -2 \cdot K_2 = -6 \) \\
\( K_2 = \frac{-6}{-2} = 3 \) \\

\textbf{Solve for \( K_1 \):} \\
\( 3 \cdot K_1 = 12 - 1 \cdot K_2 \) \\
\( 3 \cdot K_1 = 12 - 3 = 9 \) \\
\( K_1 = \frac{9}{3} = 3 \) \\

\textit{Recap the equation:} \\
\( c_2 = K_1 \cdot c_1 + K_2 \) \\
Estimated coefficients: \( K_1 = 3, K_2 = 3 \) \\
Final equation: \( c_2 = 3 \cdot c_1 + 3 \) \\

\textbf{Calculation with Restored Relations:} \\
Using the equation \( c_2 = 3 \cdot c_1 + 3 \): \\
\textit{Known variables:} \( c_1 = 5 \) \\
\( c_2 = 3 \cdot 5 + 3 = 15 + 3 = 18 \) \\

\textbf{Recap Target Variable:} \\
Condor (\( c_2 \)) = 18 \\

\textbf{Conclusion:} The number of Condor equals 18.
%\end{equation}

\subsection{Out-of-distribution Comparison Across Input length}
The comparison \ref{fig:ood_detail} reveals the critical role of Chain-of-Thought prompting in improving models' OOD generalization. Both MPC (a) and LIS (b) demonstrate substantially higher accuracy when equipped with 100\% COT (blue lines) compared to without COT. This performance gap is particularly pronounced in out-of-domain regions, where models without COT show severe degradation (dropping below 0.2 accuracy). The consistent superior performance of COT-enabled models, especially in maintaining accuracy above 0.8 across different sequence lengths, underscores how COT prompting serves as a crucial mechanism for enhancing models' ability to generalize beyond their training distribution.
\begin{figure}[]
\centering
\subfigure[]{
    \includegraphics[width=0.8\linewidth]{figure/ood_figure_mpc.pdf}
}\hfill
\subfigure[]{
    \includegraphics[width=0.8\linewidth]{figure/lis_figure_mpc.pdf}}
\caption{Comparison of Out-Of-Distribution (OOD) performance between MPC and LIS models under different Chain-of-Thought (COT) conditions across varying sequence lengths.}
\label{fig:ood_detail}
\end{figure}
%\onecolumn
%\section{You \emph{can} have an appendix here.}


%%%%%%%%%%%%%%%%%%%%%%%%%%%%%%%%%%%%%%%%%%%%%%%%%%%%%%%%%%%%%%%%%%%%%%%%%%%%%%%
%%%%%%%%%%%%%%%%%%%%%%%%%%%%%%%%%%%%%%%%%%%%%%%%%%%%%%%%%%%%%%%%%%%%%%%%%%%%%%%


\end{document}

