%%%%%%%% ICML 2025 EXAMPLE LATEX SUBMISSION FILE %%%%%%%%%%%%%%%%%

\documentclass{article}

\usepackage{geometry}
\geometry{a4paper, scale=0.7}

\usepackage[utf8]{inputenc} % allow utf-8 input
\usepackage[T1]{fontenc}    % use 8-bit T1 fonts

\usepackage{titletoc}
\usepackage[toc, page, header]{appendix} %%% MAKE SURE TO PUT THIS BEFORE hyperref PACKAGE

\usepackage{microtype}
\usepackage{graphicx}
\usepackage{subfigure}
\usepackage{booktabs} 

\usepackage[colorlinks=true, linkcolor=blue, citecolor=blue,urlcolor=black]{hyperref}


\newcommand{\theHalgorithm}{\arabic{algorithm}}


\usepackage{amsmath}
\usepackage{amssymb}
\usepackage{mathtools}
\usepackage{amsthm}
\usepackage{algorithm}
\usepackage{algorithmic}
\usepackage{natbib}

\usepackage{bbding}

\renewcommand{\algorithmiccomment}[1]{ \hfill $\triangleright$ { #1}}
\renewcommand{\algorithmicrequire}{\textbf{Input:}}
\renewcommand{\algorithmicensure}{\textbf{Output:}}
\usepackage[capitalize,noabbrev]{cleveref}

%%%%% NEW MATH DEFINITIONS %%%%%

\usepackage{amsmath,amsfonts,bm}
\usepackage{derivative}
% Mark sections of captions for referring to divisions of figures
\newcommand{\figleft}{{\em (Left)}}
\newcommand{\figcenter}{{\em (Center)}}
\newcommand{\figright}{{\em (Right)}}
\newcommand{\figtop}{{\em (Top)}}
\newcommand{\figbottom}{{\em (Bottom)}}
\newcommand{\captiona}{{\em (a)}}
\newcommand{\captionb}{{\em (b)}}
\newcommand{\captionc}{{\em (c)}}
\newcommand{\captiond}{{\em (d)}}

% Highlight a newly defined term
\newcommand{\newterm}[1]{{\bf #1}}

% Derivative d 
\newcommand{\deriv}{{\mathrm{d}}}

% Figure reference, lower-case.
\def\figref#1{figure~\ref{#1}}
% Figure reference, capital. For start of sentence
\def\Figref#1{Figure~\ref{#1}}
\def\twofigref#1#2{figures \ref{#1} and \ref{#2}}
\def\quadfigref#1#2#3#4{figures \ref{#1}, \ref{#2}, \ref{#3} and \ref{#4}}
% Section reference, lower-case.
\def\secref#1{section~\ref{#1}}
% Section reference, capital.
\def\Secref#1{Section~\ref{#1}}
% Reference to two sections.
\def\twosecrefs#1#2{sections \ref{#1} and \ref{#2}}
% Reference to three sections.
\def\secrefs#1#2#3{sections \ref{#1}, \ref{#2} and \ref{#3}}
% Reference to an equation, lower-case.
\def\eqref#1{equation~\ref{#1}}
% Reference to an equation, upper case
\def\Eqref#1{Equation~\ref{#1}}
% A raw reference to an equation---avoid using if possible
\def\plaineqref#1{\ref{#1}}
% Reference to a chapter, lower-case.
\def\chapref#1{chapter~\ref{#1}}
% Reference to an equation, upper case.
\def\Chapref#1{Chapter~\ref{#1}}
% Reference to a range of chapters
\def\rangechapref#1#2{chapters\ref{#1}--\ref{#2}}
% Reference to an algorithm, lower-case.
\def\algref#1{algorithm~\ref{#1}}
% Reference to an algorithm, upper case.
\def\Algref#1{Algorithm~\ref{#1}}
\def\twoalgref#1#2{algorithms \ref{#1} and \ref{#2}}
\def\Twoalgref#1#2{Algorithms \ref{#1} and \ref{#2}}
% Reference to a part, lower case
\def\partref#1{part~\ref{#1}}
% Reference to a part, upper case
\def\Partref#1{Part~\ref{#1}}
\def\twopartref#1#2{parts \ref{#1} and \ref{#2}}

\def\ceil#1{\lceil #1 \rceil}
\def\floor#1{\lfloor #1 \rfloor}
\def\1{\bm{1}}
\newcommand{\train}{\mathcal{D}}
\newcommand{\valid}{\mathcal{D_{\mathrm{valid}}}}
\newcommand{\test}{\mathcal{D_{\mathrm{test}}}}

\def\eps{{\epsilon}}


% Random variables
\def\reta{{\textnormal{$\eta$}}}
\def\ra{{\textnormal{a}}}
\def\rb{{\textnormal{b}}}
\def\rc{{\textnormal{c}}}
\def\rd{{\textnormal{d}}}
\def\re{{\textnormal{e}}}
\def\rf{{\textnormal{f}}}
\def\rg{{\textnormal{g}}}
\def\rh{{\textnormal{h}}}
\def\ri{{\textnormal{i}}}
\def\rj{{\textnormal{j}}}
\def\rk{{\textnormal{k}}}
\def\rl{{\textnormal{l}}}
% rm is already a command, just don't name any random variables m
\def\rn{{\textnormal{n}}}
\def\ro{{\textnormal{o}}}
\def\rp{{\textnormal{p}}}
\def\rq{{\textnormal{q}}}
\def\rr{{\textnormal{r}}}
\def\rs{{\textnormal{s}}}
\def\rt{{\textnormal{t}}}
\def\ru{{\textnormal{u}}}
\def\rv{{\textnormal{v}}}
\def\rw{{\textnormal{w}}}
\def\rx{{\textnormal{x}}}
\def\ry{{\textnormal{y}}}
\def\rz{{\textnormal{z}}}

% Random vectors
\def\rvepsilon{{\mathbf{\epsilon}}}
\def\rvphi{{\mathbf{\phi}}}
\def\rvtheta{{\mathbf{\theta}}}
\def\rva{{\mathbf{a}}}
\def\rvb{{\mathbf{b}}}
\def\rvc{{\mathbf{c}}}
\def\rvd{{\mathbf{d}}}
\def\rve{{\mathbf{e}}}
\def\rvf{{\mathbf{f}}}
\def\rvg{{\mathbf{g}}}
\def\rvh{{\mathbf{h}}}
\def\rvu{{\mathbf{i}}}
\def\rvj{{\mathbf{j}}}
\def\rvk{{\mathbf{k}}}
\def\rvl{{\mathbf{l}}}
\def\rvm{{\mathbf{m}}}
\def\rvn{{\mathbf{n}}}
\def\rvo{{\mathbf{o}}}
\def\rvp{{\mathbf{p}}}
\def\rvq{{\mathbf{q}}}
\def\rvr{{\mathbf{r}}}
\def\rvs{{\mathbf{s}}}
\def\rvt{{\mathbf{t}}}
\def\rvu{{\mathbf{u}}}
\def\rvv{{\mathbf{v}}}
\def\rvw{{\mathbf{w}}}
\def\rvx{{\mathbf{x}}}
\def\rvy{{\mathbf{y}}}
\def\rvz{{\mathbf{z}}}

% Elements of random vectors
\def\erva{{\textnormal{a}}}
\def\ervb{{\textnormal{b}}}
\def\ervc{{\textnormal{c}}}
\def\ervd{{\textnormal{d}}}
\def\erve{{\textnormal{e}}}
\def\ervf{{\textnormal{f}}}
\def\ervg{{\textnormal{g}}}
\def\ervh{{\textnormal{h}}}
\def\ervi{{\textnormal{i}}}
\def\ervj{{\textnormal{j}}}
\def\ervk{{\textnormal{k}}}
\def\ervl{{\textnormal{l}}}
\def\ervm{{\textnormal{m}}}
\def\ervn{{\textnormal{n}}}
\def\ervo{{\textnormal{o}}}
\def\ervp{{\textnormal{p}}}
\def\ervq{{\textnormal{q}}}
\def\ervr{{\textnormal{r}}}
\def\ervs{{\textnormal{s}}}
\def\ervt{{\textnormal{t}}}
\def\ervu{{\textnormal{u}}}
\def\ervv{{\textnormal{v}}}
\def\ervw{{\textnormal{w}}}
\def\ervx{{\textnormal{x}}}
\def\ervy{{\textnormal{y}}}
\def\ervz{{\textnormal{z}}}

% Random matrices
\def\rmA{{\mathbf{A}}}
\def\rmB{{\mathbf{B}}}
\def\rmC{{\mathbf{C}}}
\def\rmD{{\mathbf{D}}}
\def\rmE{{\mathbf{E}}}
\def\rmF{{\mathbf{F}}}
\def\rmG{{\mathbf{G}}}
\def\rmH{{\mathbf{H}}}
\def\rmI{{\mathbf{I}}}
\def\rmJ{{\mathbf{J}}}
\def\rmK{{\mathbf{K}}}
\def\rmL{{\mathbf{L}}}
\def\rmM{{\mathbf{M}}}
\def\rmN{{\mathbf{N}}}
\def\rmO{{\mathbf{O}}}
\def\rmP{{\mathbf{P}}}
\def\rmQ{{\mathbf{Q}}}
\def\rmR{{\mathbf{R}}}
\def\rmS{{\mathbf{S}}}
\def\rmT{{\mathbf{T}}}
\def\rmU{{\mathbf{U}}}
\def\rmV{{\mathbf{V}}}
\def\rmW{{\mathbf{W}}}
\def\rmX{{\mathbf{X}}}
\def\rmY{{\mathbf{Y}}}
\def\rmZ{{\mathbf{Z}}}

% Elements of random matrices
\def\ermA{{\textnormal{A}}}
\def\ermB{{\textnormal{B}}}
\def\ermC{{\textnormal{C}}}
\def\ermD{{\textnormal{D}}}
\def\ermE{{\textnormal{E}}}
\def\ermF{{\textnormal{F}}}
\def\ermG{{\textnormal{G}}}
\def\ermH{{\textnormal{H}}}
\def\ermI{{\textnormal{I}}}
\def\ermJ{{\textnormal{J}}}
\def\ermK{{\textnormal{K}}}
\def\ermL{{\textnormal{L}}}
\def\ermM{{\textnormal{M}}}
\def\ermN{{\textnormal{N}}}
\def\ermO{{\textnormal{O}}}
\def\ermP{{\textnormal{P}}}
\def\ermQ{{\textnormal{Q}}}
\def\ermR{{\textnormal{R}}}
\def\ermS{{\textnormal{S}}}
\def\ermT{{\textnormal{T}}}
\def\ermU{{\textnormal{U}}}
\def\ermV{{\textnormal{V}}}
\def\ermW{{\textnormal{W}}}
\def\ermX{{\textnormal{X}}}
\def\ermY{{\textnormal{Y}}}
\def\ermZ{{\textnormal{Z}}}

% Vectors
\def\vzero{{\bm{0}}}
\def\vone{{\bm{1}}}
\def\vmu{{\bm{\mu}}}
\def\vtheta{{\bm{\theta}}}
\def\vphi{{\bm{\phi}}}
\def\va{{\bm{a}}}
\def\vb{{\bm{b}}}
\def\vc{{\bm{c}}}
\def\vd{{\bm{d}}}
\def\ve{{\bm{e}}}
\def\vf{{\bm{f}}}
\def\vg{{\bm{g}}}
\def\vh{{\bm{h}}}
\def\vi{{\bm{i}}}
\def\vj{{\bm{j}}}
\def\vk{{\bm{k}}}
\def\vl{{\bm{l}}}
\def\vm{{\bm{m}}}
\def\vn{{\bm{n}}}
\def\vo{{\bm{o}}}
\def\vp{{\bm{p}}}
\def\vq{{\bm{q}}}
\def\vr{{\bm{r}}}
\def\vs{{\bm{s}}}
\def\vt{{\bm{t}}}
\def\vu{{\bm{u}}}
\def\vv{{\bm{v}}}
\def\vw{{\bm{w}}}
\def\vx{{\bm{x}}}
\def\vy{{\bm{y}}}
\def\vz{{\bm{z}}}

% Elements of vectors
\def\evalpha{{\alpha}}
\def\evbeta{{\beta}}
\def\evepsilon{{\epsilon}}
\def\evlambda{{\lambda}}
\def\evomega{{\omega}}
\def\evmu{{\mu}}
\def\evpsi{{\psi}}
\def\evsigma{{\sigma}}
\def\evtheta{{\theta}}
\def\eva{{a}}
\def\evb{{b}}
\def\evc{{c}}
\def\evd{{d}}
\def\eve{{e}}
\def\evf{{f}}
\def\evg{{g}}
\def\evh{{h}}
\def\evi{{i}}
\def\evj{{j}}
\def\evk{{k}}
\def\evl{{l}}
\def\evm{{m}}
\def\evn{{n}}
\def\evo{{o}}
\def\evp{{p}}
\def\evq{{q}}
\def\evr{{r}}
\def\evs{{s}}
\def\evt{{t}}
\def\evu{{u}}
\def\evv{{v}}
\def\evw{{w}}
\def\evx{{x}}
\def\evy{{y}}
\def\evz{{z}}

% Matrix
\def\mA{{\bm{A}}}
\def\mB{{\bm{B}}}
\def\mC{{\bm{C}}}
\def\mD{{\bm{D}}}
\def\mE{{\bm{E}}}
\def\mF{{\bm{F}}}
\def\mG{{\bm{G}}}
\def\mH{{\bm{H}}}
\def\mI{{\bm{I}}}
\def\mJ{{\bm{J}}}
\def\mK{{\bm{K}}}
\def\mL{{\bm{L}}}
\def\mM{{\bm{M}}}
\def\mN{{\bm{N}}}
\def\mO{{\bm{O}}}
\def\mP{{\bm{P}}}
\def\mQ{{\bm{Q}}}
\def\mR{{\bm{R}}}
\def\mS{{\bm{S}}}
\def\mT{{\bm{T}}}
\def\mU{{\bm{U}}}
\def\mV{{\bm{V}}}
\def\mW{{\bm{W}}}
\def\mX{{\bm{X}}}
\def\mY{{\bm{Y}}}
\def\mZ{{\bm{Z}}}
\def\mBeta{{\bm{\beta}}}
\def\mPhi{{\bm{\Phi}}}
\def\mLambda{{\bm{\Lambda}}}
\def\mSigma{{\bm{\Sigma}}}

% Tensor
\DeclareMathAlphabet{\mathsfit}{\encodingdefault}{\sfdefault}{m}{sl}
\SetMathAlphabet{\mathsfit}{bold}{\encodingdefault}{\sfdefault}{bx}{n}
\newcommand{\tens}[1]{\bm{\mathsfit{#1}}}
\def\tA{{\tens{A}}}
\def\tB{{\tens{B}}}
\def\tC{{\tens{C}}}
\def\tD{{\tens{D}}}
\def\tE{{\tens{E}}}
\def\tF{{\tens{F}}}
\def\tG{{\tens{G}}}
\def\tH{{\tens{H}}}
\def\tI{{\tens{I}}}
\def\tJ{{\tens{J}}}
\def\tK{{\tens{K}}}
\def\tL{{\tens{L}}}
\def\tM{{\tens{M}}}
\def\tN{{\tens{N}}}
\def\tO{{\tens{O}}}
\def\tP{{\tens{P}}}
\def\tQ{{\tens{Q}}}
\def\tR{{\tens{R}}}
\def\tS{{\tens{S}}}
\def\tT{{\tens{T}}}
\def\tU{{\tens{U}}}
\def\tV{{\tens{V}}}
\def\tW{{\tens{W}}}
\def\tX{{\tens{X}}}
\def\tY{{\tens{Y}}}
\def\tZ{{\tens{Z}}}


% Graph
\def\gA{{\mathcal{A}}}
\def\gB{{\mathcal{B}}}
\def\gC{{\mathcal{C}}}
\def\gD{{\mathcal{D}}}
\def\gE{{\mathcal{E}}}
\def\gF{{\mathcal{F}}}
\def\gG{{\mathcal{G}}}
\def\gH{{\mathcal{H}}}
\def\gI{{\mathcal{I}}}
\def\gJ{{\mathcal{J}}}
\def\gK{{\mathcal{K}}}
\def\gL{{\mathcal{L}}}
\def\gM{{\mathcal{M}}}
\def\gN{{\mathcal{N}}}
\def\gO{{\mathcal{O}}}
\def\gP{{\mathcal{P}}}
\def\gQ{{\mathcal{Q}}}
\def\gR{{\mathcal{R}}}
\def\gS{{\mathcal{S}}}
\def\gT{{\mathcal{T}}}
\def\gU{{\mathcal{U}}}
\def\gV{{\mathcal{V}}}
\def\gW{{\mathcal{W}}}
\def\gX{{\mathcal{X}}}
\def\gY{{\mathcal{Y}}}
\def\gZ{{\mathcal{Z}}}

% Sets
\def\sA{{\mathbb{A}}}
\def\sB{{\mathbb{B}}}
\def\sC{{\mathbb{C}}}
\def\sD{{\mathbb{D}}}
% Don't use a set called E, because this would be the same as our symbol
% for expectation.
\def\sF{{\mathbb{F}}}
\def\sG{{\mathbb{G}}}
\def\sH{{\mathbb{H}}}
\def\sI{{\mathbb{I}}}
\def\sJ{{\mathbb{J}}}
\def\sK{{\mathbb{K}}}
\def\sL{{\mathbb{L}}}
\def\sM{{\mathbb{M}}}
\def\sN{{\mathbb{N}}}
\def\sO{{\mathbb{O}}}
\def\sP{{\mathbb{P}}}
\def\sQ{{\mathbb{Q}}}
\def\sR{{\mathbb{R}}}
\def\sS{{\mathbb{S}}}
\def\sT{{\mathbb{T}}}
\def\sU{{\mathbb{U}}}
\def\sV{{\mathbb{V}}}
\def\sW{{\mathbb{W}}}
\def\sX{{\mathbb{X}}}
\def\sY{{\mathbb{Y}}}
\def\sZ{{\mathbb{Z}}}

% Entries of a matrix
\def\emLambda{{\Lambda}}
\def\emA{{A}}
\def\emB{{B}}
\def\emC{{C}}
\def\emD{{D}}
\def\emE{{E}}
\def\emF{{F}}
\def\emG{{G}}
\def\emH{{H}}
\def\emI{{I}}
\def\emJ{{J}}
\def\emK{{K}}
\def\emL{{L}}
\def\emM{{M}}
\def\emN{{N}}
\def\emO{{O}}
\def\emP{{P}}
\def\emQ{{Q}}
\def\emR{{R}}
\def\emS{{S}}
\def\emT{{T}}
\def\emU{{U}}
\def\emV{{V}}
\def\emW{{W}}
\def\emX{{X}}
\def\emY{{Y}}
\def\emZ{{Z}}
\def\emSigma{{\Sigma}}

% entries of a tensor
% Same font as tensor, without \bm wrapper
\newcommand{\etens}[1]{\mathsfit{#1}}
\def\etLambda{{\etens{\Lambda}}}
\def\etA{{\etens{A}}}
\def\etB{{\etens{B}}}
\def\etC{{\etens{C}}}
\def\etD{{\etens{D}}}
\def\etE{{\etens{E}}}
\def\etF{{\etens{F}}}
\def\etG{{\etens{G}}}
\def\etH{{\etens{H}}}
\def\etI{{\etens{I}}}
\def\etJ{{\etens{J}}}
\def\etK{{\etens{K}}}
\def\etL{{\etens{L}}}
\def\etM{{\etens{M}}}
\def\etN{{\etens{N}}}
\def\etO{{\etens{O}}}
\def\etP{{\etens{P}}}
\def\etQ{{\etens{Q}}}
\def\etR{{\etens{R}}}
\def\etS{{\etens{S}}}
\def\etT{{\etens{T}}}
\def\etU{{\etens{U}}}
\def\etV{{\etens{V}}}
\def\etW{{\etens{W}}}
\def\etX{{\etens{X}}}
\def\etY{{\etens{Y}}}
\def\etZ{{\etens{Z}}}

% The true underlying data generating distribution
\newcommand{\pdata}{p_{\rm{data}}}
\newcommand{\ptarget}{p_{\rm{target}}}
\newcommand{\pprior}{p_{\rm{prior}}}
\newcommand{\pbase}{p_{\rm{base}}}
\newcommand{\pref}{p_{\rm{ref}}}

% The empirical distribution defined by the training set
\newcommand{\ptrain}{\hat{p}_{\rm{data}}}
\newcommand{\Ptrain}{\hat{P}_{\rm{data}}}
% The model distribution
\newcommand{\pmodel}{p_{\rm{model}}}
\newcommand{\Pmodel}{P_{\rm{model}}}
\newcommand{\ptildemodel}{\tilde{p}_{\rm{model}}}
% Stochastic autoencoder distributions
\newcommand{\pencode}{p_{\rm{encoder}}}
\newcommand{\pdecode}{p_{\rm{decoder}}}
\newcommand{\precons}{p_{\rm{reconstruct}}}

\newcommand{\laplace}{\mathrm{Laplace}} % Laplace distribution

\newcommand{\E}{\mathbb{E}}
\newcommand{\Ls}{\mathcal{L}}
\newcommand{\R}{\mathbb{R}}
\newcommand{\emp}{\tilde{p}}
\newcommand{\lr}{\alpha}
\newcommand{\reg}{\lambda}
\newcommand{\rect}{\mathrm{rectifier}}
\newcommand{\softmax}{\mathrm{softmax}}
\newcommand{\sigmoid}{\sigma}
\newcommand{\softplus}{\zeta}
\newcommand{\KL}{D_{\mathrm{KL}}}
\newcommand{\Var}{\mathrm{Var}}
\newcommand{\standarderror}{\mathrm{SE}}
\newcommand{\Cov}{\mathrm{Cov}}
% Wolfram Mathworld says $L^2$ is for function spaces and $\ell^2$ is for vectors
% But then they seem to use $L^2$ for vectors throughout the site, and so does
% wikipedia.
\newcommand{\normlzero}{L^0}
\newcommand{\normlone}{L^1}
\newcommand{\normltwo}{L^2}
\newcommand{\normlp}{L^p}
\newcommand{\normmax}{L^\infty}

\newcommand{\parents}{Pa} % See usage in notation.tex. Chosen to match Daphne's book.

\DeclareMathOperator*{\argmax}{arg\,max}
\DeclareMathOperator*{\argmin}{arg\,min}

\DeclareMathOperator{\sign}{sign}
\DeclareMathOperator{\Tr}{Tr}
\let\ab\allowbreak


%%%%%%%%%%%%%%%%%%%%%%%%%%%%%%%%
% THEOREMS
%%%%%%%%%%%%%%%%%%%%%%%%%%%%%%%%
\theoremstyle{plain}
\newtheorem{theorem}{Theorem}[section]
\newtheorem{proposition}[theorem]{Proposition}
\newtheorem{lemma}[theorem]{Lemma}
\newtheorem{corollary}[theorem]{Corollary}
\theoremstyle{definition}
\newtheorem{definition}[theorem]{Definition}
\newtheorem{assumption}[theorem]{Assumption}
\theoremstyle{remark}
\newtheorem{remark}[theorem]{Remark}

\newcommand{\compilehidecomments}{false}%HIDE comments
\ifthenelse{ \equal{\compilehidecomments}{true} }{%
	\newcommand{\wei}[1]{}
	\newcommand{\yu}[1]{}
	\newcommand{\siwei}[1]{}
    \newcommand{\longbo}[1]{}
}{
	\newcommand{\wei}[1]{{\color{blue}{[Wei: #1]}}}
	\newcommand{\yu}[1]{{\color{cyan}[\text{Yu:} #1]}}
	\newcommand{\siwei}[1]{{\color{red}[\text{Siwei:} #1]}}
    \newcommand{\longbo}[1]{{\color{orange}[\text{Longbo:} #1]}}
}

\title{Continuous K-Max Bandits}


% \usepackage{authblk}
%\author{Yu Chen \thanks{IIIS, Tsinghua University. Email: \texttt{chenyu23@mails.tsinghua.edu.cn}.}
%\and
%Longbo Huang \thanks{IIIS, Tsinghua University. Email: \texttt{longbohuang@tsinghua.edu.cn}.}
%\and
%Siwei Wang  \thanks{Microsoft Research Asia. Email: \texttt{siweiwang@microsoft.com}.}
%\and
%Wei Chen\thanks{Microsoft Research Asia. Email: \texttt{weic@microsoft.com}.}
%}



\author{Yu Chen$^{1}$\thanks{\ denotes equal contributions. Corresponding author: Wei Chen (\texttt{weic@microsoft.com})}
\hspace{0.1cm} 
Siwei Wang$^{2*}$\hspace{0.1cm} 
Longbo Huang$^{1}$\hspace{0.1cm}  
Wei Chen$^{2}$\Envelope
\\
   \normalfont $^1$IIIS, Tsinghua University \\
   % (\texttt{\{chenyu23@mails.tsinghua.edu.cn,}
   % \\\qquad\qquad\qquad\qquad\qquad\qquad
   % \texttt{longbohuang@tsinghua.edu.cn\}})\\
  $^2$Microsoft Research Asia 
  % (\texttt{\{siweiwang,weic\}@microsoft.com}) %\\
  %$^3$IIIS, Tsinghua University
  %(\texttt{longbohuang@tsinghua.edu.cn}) \\
  %$^4$Microsoft Research Asia
  %(\texttt{weic@microsoft.com})
  \\
  \texttt{chenyu23@mails.tsinghua.edu.cn} \\
  \texttt{siweiwang@microsoft.com} \\
  \texttt{longbohuang@tsinghua.edu.cn}\\
  \texttt{weic@microsoft.com}
}
\date{}

\begin{document}

\maketitle





\begin{abstract}
We study the $K$-Max combinatorial multi-armed bandits problem with continuous outcome distributions and weak value-index feedback: each base arm has an unknown continuous outcome distribution, and in each round the learning agent selects $K$ arms, obtains the maximum value sampled from these $K$ arms as reward and observes this reward together with the corresponding arm index as feedback. This setting captures critical applications in recommendation systems, distributed computing, server scheduling, etc. The continuous $K$-Max bandits introduce unique challenges, including discretization error from continuous-to-discrete conversion, non-deterministic tie-breaking under limited feedback, and biased estimation due to partial observability. Our key contribution is the computationally efficient algorithm \texttt{DCK-UCB}, which combines adaptive discretization with bias-corrected confidence bounds to tackle these challenges. For general continuous distributions, we prove that \texttt{DCK-UCB} achieves a $\widetilde{\mathcal{O}}(T^{3/4})$ regret upper bound, establishing the first sublinear regret guarantee for this setting. Furthermore, we identify an important special case with exponential distributions under full-bandit feedback. In this case, our proposed algorithm \texttt{MLE-Exp} enables $\widetilde{\mathcal{O}}(\sqrt{T})$ regret upper bound through maximal log-likelihood estimation, achieving near-minimax optimality. 

\end{abstract}

\section{Introduction}
\label{sec:introduction}
The business processes of organizations are experiencing ever-increasing complexity due to the large amount of data, high number of users, and high-tech devices involved \cite{martin2021pmopportunitieschallenges, beerepoot2023biggestbpmproblems}. This complexity may cause business processes to deviate from normal control flow due to unforeseen and disruptive anomalies \cite{adams2023proceddsriftdetection}. These control-flow anomalies manifest as unknown, skipped, and wrongly-ordered activities in the traces of event logs monitored from the execution of business processes \cite{ko2023adsystematicreview}. For the sake of clarity, let us consider an illustrative example of such anomalies. Figure \ref{FP_ANOMALIES} shows a so-called event log footprint, which captures the control flow relations of four activities of a hypothetical event log. In particular, this footprint captures the control-flow relations between activities \texttt{a}, \texttt{b}, \texttt{c} and \texttt{d}. These are the causal ($\rightarrow$) relation, concurrent ($\parallel$) relation, and other ($\#$) relations such as exclusivity or non-local dependency \cite{aalst2022pmhandbook}. In addition, on the right are six traces, of which five exhibit skipped, wrongly-ordered and unknown control-flow anomalies. For example, $\langle$\texttt{a b d}$\rangle$ has a skipped activity, which is \texttt{c}. Because of this skipped activity, the control-flow relation \texttt{b}$\,\#\,$\texttt{d} is violated, since \texttt{d} directly follows \texttt{b} in the anomalous trace.
\begin{figure}[!t]
\centering
\includegraphics[width=0.9\columnwidth]{images/FP_ANOMALIES.png}
\caption{An example event log footprint with six traces, of which five exhibit control-flow anomalies.}
\label{FP_ANOMALIES}
\end{figure}

\subsection{Control-flow anomaly detection}
Control-flow anomaly detection techniques aim to characterize the normal control flow from event logs and verify whether these deviations occur in new event logs \cite{ko2023adsystematicreview}. To develop control-flow anomaly detection techniques, \revision{process mining} has seen widespread adoption owing to process discovery and \revision{conformance checking}. On the one hand, process discovery is a set of algorithms that encode control-flow relations as a set of model elements and constraints according to a given modeling formalism \cite{aalst2022pmhandbook}; hereafter, we refer to the Petri net, a widespread modeling formalism. On the other hand, \revision{conformance checking} is an explainable set of algorithms that allows linking any deviations with the reference Petri net and providing the fitness measure, namely a measure of how much the Petri net fits the new event log \cite{aalst2022pmhandbook}. Many control-flow anomaly detection techniques based on \revision{conformance checking} (hereafter, \revision{conformance checking}-based techniques) use the fitness measure to determine whether an event log is anomalous \cite{bezerra2009pmad, bezerra2013adlogspais, myers2018icsadpm, pecchia2020applicationfailuresanalysispm}. 

The scientific literature also includes many \revision{conformance checking}-independent techniques for control-flow anomaly detection that combine specific types of trace encodings with machine/deep learning \cite{ko2023adsystematicreview, tavares2023pmtraceencoding}. Whereas these techniques are very effective, their explainability is challenging due to both the type of trace encoding employed and the machine/deep learning model used \cite{rawal2022trustworthyaiadvances,li2023explainablead}. Hence, in the following, we focus on the shortcomings of \revision{conformance checking}-based techniques to investigate whether it is possible to support the development of competitive control-flow anomaly detection techniques while maintaining the explainable nature of \revision{conformance checking}.
\begin{figure}[!t]
\centering
\includegraphics[width=\columnwidth]{images/HIGH_LEVEL_VIEW.png}
\caption{A high-level view of the proposed framework for combining \revision{process mining}-based feature extraction with dimensionality reduction for control-flow anomaly detection.}
\label{HIGH_LEVEL_VIEW}
\end{figure}

\subsection{Shortcomings of \revision{conformance checking}-based techniques}
Unfortunately, the detection effectiveness of \revision{conformance checking}-based techniques is affected by noisy data and low-quality Petri nets, which may be due to human errors in the modeling process or representational bias of process discovery algorithms \cite{bezerra2013adlogspais, pecchia2020applicationfailuresanalysispm, aalst2016pm}. Specifically, on the one hand, noisy data may introduce infrequent and deceptive control-flow relations that may result in inconsistent fitness measures, whereas, on the other hand, checking event logs against a low-quality Petri net could lead to an unreliable distribution of fitness measures. Nonetheless, such Petri nets can still be used as references to obtain insightful information for \revision{process mining}-based feature extraction, supporting the development of competitive and explainable \revision{conformance checking}-based techniques for control-flow anomaly detection despite the problems above. For example, a few works outline that token-based \revision{conformance checking} can be used for \revision{process mining}-based feature extraction to build tabular data and develop effective \revision{conformance checking}-based techniques for control-flow anomaly detection \cite{singh2022lapmsh, debenedictis2023dtadiiot}. However, to the best of our knowledge, the scientific literature lacks a structured proposal for \revision{process mining}-based feature extraction using the state-of-the-art \revision{conformance checking} variant, namely alignment-based \revision{conformance checking}.

\subsection{Contributions}
We propose a novel \revision{process mining}-based feature extraction approach with alignment-based \revision{conformance checking}. This variant aligns the deviating control flow with a reference Petri net; the resulting alignment can be inspected to extract additional statistics such as the number of times a given activity caused mismatches \cite{aalst2022pmhandbook}. We integrate this approach into a flexible and explainable framework for developing techniques for control-flow anomaly detection. The framework combines \revision{process mining}-based feature extraction and dimensionality reduction to handle high-dimensional feature sets, achieve detection effectiveness, and support explainability. Notably, in addition to our proposed \revision{process mining}-based feature extraction approach, the framework allows employing other approaches, enabling a fair comparison of multiple \revision{conformance checking}-based and \revision{conformance checking}-independent techniques for control-flow anomaly detection. Figure \ref{HIGH_LEVEL_VIEW} shows a high-level view of the framework. Business processes are monitored, and event logs obtained from the database of information systems. Subsequently, \revision{process mining}-based feature extraction is applied to these event logs and tabular data input to dimensionality reduction to identify control-flow anomalies. We apply several \revision{conformance checking}-based and \revision{conformance checking}-independent framework techniques to publicly available datasets, simulated data of a case study from railways, and real-world data of a case study from healthcare. We show that the framework techniques implementing our approach outperform the baseline \revision{conformance checking}-based techniques while maintaining the explainable nature of \revision{conformance checking}.

In summary, the contributions of this paper are as follows.
\begin{itemize}
    \item{
        A novel \revision{process mining}-based feature extraction approach to support the development of competitive and explainable \revision{conformance checking}-based techniques for control-flow anomaly detection.
    }
    \item{
        A flexible and explainable framework for developing techniques for control-flow anomaly detection using \revision{process mining}-based feature extraction and dimensionality reduction.
    }
    \item{
        Application to synthetic and real-world datasets of several \revision{conformance checking}-based and \revision{conformance checking}-independent framework techniques, evaluating their detection effectiveness and explainability.
    }
\end{itemize}

The rest of the paper is organized as follows.
\begin{itemize}
    \item Section \ref{sec:related_work} reviews the existing techniques for control-flow anomaly detection, categorizing them into \revision{conformance checking}-based and \revision{conformance checking}-independent techniques.
    \item Section \ref{sec:abccfe} provides the preliminaries of \revision{process mining} to establish the notation used throughout the paper, and delves into the details of the proposed \revision{process mining}-based feature extraction approach with alignment-based \revision{conformance checking}.
    \item Section \ref{sec:framework} describes the framework for developing \revision{conformance checking}-based and \revision{conformance checking}-independent techniques for control-flow anomaly detection that combine \revision{process mining}-based feature extraction and dimensionality reduction.
    \item Section \ref{sec:evaluation} presents the experiments conducted with multiple framework and baseline techniques using data from publicly available datasets and case studies.
    \item Section \ref{sec:conclusions} draws the conclusions and presents future work.
\end{itemize}
\section{Auxiliary-Variable Adaptive Control Barrier Functions}
\label{sec:AVBCBF}

In this section, we introduce Auxiliary-Variable Adaptive Control Barrier Functions (AVCBFs) for safety-critical control.
We start with a simple example to motivate the need for AVCBFs.

\subsection{Motivation Example: Simplified Adaptive Cruise Control}
\label{subsec:SACC-problem}

Consider a Simplified Adaptive Cruise Control (SACC) problem with the dynamics of ego vehicle expressed as 
\begin{small}
\begin{equation}
\label{eq:SACC-dynamics}
\underbrace{\begin{bmatrix}
\dot{z}(t) \\
\dot{v}(t) 
\end{bmatrix}}_{\dot{\boldsymbol{x}}(t)}  
=\underbrace{\begin{bmatrix}
 v_{p}-v(t) \\
 0
\end{bmatrix}}_{f(\boldsymbol{x}(t))} 
+ \underbrace{\begin{bmatrix}
  0 \\
  1 
\end{bmatrix}}_{g(\boldsymbol{x}(t))}u(t),
\end{equation}
\end{small}
where $v_{p}>0, v(t)>0$ denote the velocity of the lead vehicle (constant velocity) and ego vehicle, respectively, $z(t)$ denotes the distance between the lead and ego vehicle and $u(t)$ denotes the acceleration (control) of ego vehicle, subject to the control constraints
\begin{equation}
\label{eq:simple-control-constraint}
u_{min}\le u(t) \le u_{max}, \forall t \ge0,
\end{equation}
where $u_{min}<0$ and $u_{max}>0$ are the minimum and maximum control input, respectively.

 For safety, we require that $z(t)$ always be greater than or equal to the safety distance denoted by $l_{p}>0,$ i.e., $z(t)\ge l_{p}, \forall t \ge 0.$ Based on Def. \ref{def:HOCBF}, let $\psi_{0}(\boldsymbol{x})\coloneqq b(\boldsymbol{x})=z(t)-l_{p}.$ From \eqref{eq:sequence-f1} and \eqref{eq:sequence-set1}, since the relative degree of $b(\boldsymbol{x})$ is 2, we have
\begin{equation}
\label{eq:SACC-HOCBF-sequence}
\begin{split}
&\psi_{1}(\boldsymbol{x})\coloneqq v_{p}-v(t)+k_{1}\psi_{0}(\boldsymbol{x})\ge 0
,\\
&\psi_{2}(\boldsymbol{x})\coloneqq -u(t)+k_{1}(v_{p}-v(t))+k_{2}\psi_{1}(\boldsymbol{x})\ge 0,
\end{split}
\end{equation}
where $\alpha_{1}(\psi_{0}(\boldsymbol{x}))\coloneqq k_{1}\psi_{0}(\boldsymbol{x}), \alpha_{2}(\psi_{1}(\boldsymbol{x}))\coloneqq k_{2}\psi_{1}(\boldsymbol{x}), k_{1}>0, k_{2}>0.$ The constant class $\kappa$ coefficients $k_{1},k_{2}$ are always chosen small to equip ego vehicle with a conservative control strategy to keep it safe, i.e., smaller $k_{1},k_{2}$ make ego vehicle brake earlier (see \cite{xiao2021high}). Suppose we wish to minimize the energy cost as $\int_{0}^{T} u^{2}(t)dt.$ We can then formulate the cost in the QP with constraint $\psi_{2}(\boldsymbol{x})\ge0$ and control input constraint \eqref{eq:simple-control-constraint} to get the optimal controller for the SACC problem. However, the feasible set of input can easily become empty if $u(t)\le k_{1}(v_{p}-v(t))+k_{2}\psi_{1}(\boldsymbol{x})<u_{min}$,  which causes infeasibility of the optimization. In \cite{xiao2021adaptive}, the authors introduced penalty variables in front of class $\kappa$ functions to enhance the feasibility. This approach defines $\psi_{0}(\boldsymbol{x})\coloneqq b(\boldsymbol{x})=z(t)-l_{p}$ as PACBF and other constraints can be further defined as
\begin{equation}
\label{eq:SACC-PACBF-sequence}
\begin{split}
\psi_{1}(\boldsymbol{x},p_{1}(t))&\coloneqq v_{p}-v(t)+p_{1}(t)k_{1}\psi_{0}(\boldsymbol{x})\ge 0,\\
\psi_{2}(\boldsymbol{x},p_{1}(t),&\boldsymbol{\nu})\coloneqq \nu_{1}(t)k_{1}\psi_{0}(\boldsymbol{x})+p_{1}(t)k_{1}(v_{p}\\
-v(t))&-u(t)+\nu_{2}(t)k_{2}\psi_{1}(\boldsymbol{x},p_{1}(t))\ge 0,
\end{split}
\end{equation}
where $\nu_{1}(t)=\dot{p}_{1}(t),\nu_{2}(t)=p_{2}(t), p_{1}(t)\ge0,p_{2}(t)\ge0,\boldsymbol{\nu}=(\nu_{1}(t),\nu_{2}(t)).$ $p_{1}(t),p_{2}(t)$ are time-varying penalty variables, which alleviate the conservativeness of the control strategy and $\nu_{1}(t),\nu_{2}(t)$ are auxiliary inputs, which relax the constraints for $u(t)$ in $\psi_{2}(\boldsymbol{x},p_{1}(t),\boldsymbol{\nu})\ge0$ and \eqref{eq:simple-control-constraint}. However, in practice, we need to define several additional constraints to make PACBF valid as shown in Eqs. (24)-(27) in \cite{xiao2021adaptive}. First, we need to define HOCBFs ($b_{1}(p_{1}(t))=p_{1}(t),b_{2}(p_{2}(t))=p_{2}(t))$ based on Def. \ref{def:HOCBF} to ensure $p_{1}(t)\ge0,p_{2}(t)\ge0.$ Next we need to define HOCBF ($b_{3}(p_{1}(t))=p_{1,max}-p_{1}(t)$) to confine the value of $p_{1}(t)$ in the range $[0,p_{1,max}].$ We also need to define CLF ($V(p_{1}(t))=(p_{1}(t)-p_{1}^{\ast})^{2}$) based on Def. \ref{def:control-l-f} to keep $p_{1}(t)$ close to a small value $p_{1}^{\ast}.$ $b_{3}(p_{1}(t)), V(p_{1}(t))$ are necessary since $\psi_{0}(\boldsymbol{x},p_{1}(t))\coloneqq p_{1}(t)k_{1}\psi_{0}(\boldsymbol{x})$ in first constraint in \eqref{eq:SACC-PACBF-sequence} is not a class $\kappa$ function with respect to $\psi_{0}(\boldsymbol{x}),$ i.e., $p_{1}(t)k_{1}\psi_{0}(\boldsymbol{x})$ is not guaranteed to strictly increase since $\psi_{0}(\boldsymbol{x},p_{1}(t))$ is in fact a class $\kappa$ function with respect to $p_{1}(t)\psi_{0}(\boldsymbol{x})$, which is against Thm. \ref{thm:safety-guarantee}, therefore $\psi_{1}(\boldsymbol{x},p_{1}(t))\ge 0$ in \eqref{eq:SACC-PACBF-sequence} may not guarantee $\psi_{0}(\boldsymbol{x})\ge 0.$ This illustrates why we have to limit the growth of $p_{1}(t)$ by defining $b_{3}(p_{1}(t)),V(p_{1}(t)).$ However, the way to choose appropriate values for $p_{1,max},p_{1}^{\ast}$ is not straightforward. We can imagine as the relative degree of $b(\boldsymbol{x})$ gets higher, the number of additional constraints we should define also gets larger, which results in complicated parameter-tuning process. To address this issue, we introduce $a_{1}(t),a_{2}(t)$ in the form
\begin{small}
\begin{equation}
\label{eq:SACC-AVBCBF-sequence}
\begin{split}
\psi_{1}(\boldsymbol{x},\boldsymbol{a},\dot{a}_{1}(t))\coloneqq a_{2}(t)(\dot{\psi}_{0}(\boldsymbol{x},a_{1}(t))
+k_{1}\psi_{0}(\boldsymbol{x},a_{1}(t)))\ge 0,\\
\psi_{2}(\boldsymbol{x},\boldsymbol{a},\dot{a}_{1}(t),\boldsymbol{\nu})\coloneqq \nu_{2}(t)\frac{\psi_{1}(\boldsymbol{x},\boldsymbol{a},\dot{a}_{1}(t))}{a_{2}(t)} +a_{2}(t)(\nu_{1}(t)(z(t)\\
-l_{p})+2\dot{a}_{1}(t)(v_{p}-v(t))-a_{1}(t)u(t)+k_{1}\dot{\psi}_{0}(\boldsymbol{x},a_{1}(t)))\\
+k_{2}\psi_{1}(\boldsymbol{x},\boldsymbol{a},\dot{a}_{1}(t))\ge 0, 
\end{split}
\end{equation}
\end{small}
where $\psi_{0}(\boldsymbol{x},a_{1}(t))\coloneqq a_{1}(t)b (\boldsymbol{x})=a_{1}(t)(z(t)-l_{p}),\boldsymbol{\nu}=[\nu_{1}(t),\nu_{2}(t)]^{T}=[\ddot{a}_{1}(t),\dot{a}_{2}(t)]^{T},\boldsymbol{a}=[a_{1}(t),a_{2}(t)]^{T},$ $a_{1}(t),a_{2}(t)$ are time-varying auxiliary variables. Since $\psi_{0}(\boldsymbol{x},a_{1}(t))\ge0,\psi_{1}(\boldsymbol{x},\boldsymbol{a},\dot{a}_{1}(t))\ge 0$ will not be against $b(\boldsymbol{x})\ge 0,\dot{\psi}_{0}(\boldsymbol{x},a_{1}(t))
+k_{1}\psi_{0}(\boldsymbol{x},a_{1}(t))\ge 0$ iff $a_{1}(t)>0,a_{2}(t)>0,$ we need to define HOCBFs for auxiliary variables to make $a_{1}(t)>0,a_{2}(t)>0,$ which will be illustrated in Sec. \ref{sec:AVCBFs}.  $\nu_{1}(t),\nu_{2}(t)$ are auxiliary inputs which are used to alleviate the restriction of constraints for $u(t)$ in $\psi_{2}(\boldsymbol{x},\boldsymbol{a},\dot{a}_{1}(t),\boldsymbol{\nu})\ge0$ and \eqref{eq:simple-control-constraint}. Different from the first constraint in \eqref{eq:SACC-PACBF-sequence}, $k_{1}\psi_{0}(\boldsymbol{x},a_{1}(t))$ is still a class $\kappa$ function with respect to $\psi_{0}(\boldsymbol{x},a_{1}(t)),$ therefore we do not need to define additional HOCBFs and CLFs like $b_{3}(p_{1}(t)),V(p_{1}(t))$ to limit the growth of $a_{1}(t).$
We can rewrite $\psi_{1} (\boldsymbol{x},\boldsymbol{a},\dot{a}_{1}(t))$ in \eqref{eq:SACC-AVBCBF-sequence} as
\begin{equation}
\label{eq:SACC-AVBCBF-sequence-rewrite}
\begin{split}
\psi_{1}(\boldsymbol{x},\boldsymbol{a},\dot{a}_{1}(t))\coloneqq a_{2}(t)a_{1}(t)(v_{p}-v(t)\\
+k_{1}(1+\frac{\dot{a}_{1}(t)}{k_{1}a_{1}(t)})b(\boldsymbol{x}))\ge 0.
\end{split}
\end{equation}
Compared to the first constraint in \eqref{eq:SACC-HOCBF-sequence}, $\frac{\dot{a}_{1}(t)}{a_{1}(t)}$ is a time-varying auxiliary term to alleviate the conservativeness of control that small $k_{1}$ originally has, which shows the adaptivity of auxiliary terms to constant class $\kappa$ coefficients. 

% There is another type of adaptive CBFs called Relaxation-Adaptive Control Barrier Functions (RACBFs) in \cite{xiao2021adaptive}. The RACBF $b(\boldsymbol{x})$ is in the form:
% \begin{equation}
% \label{eq:RACBF}
% \psi_{0}(\boldsymbol{x},r(t))\coloneqq b(\boldsymbol{x})-r(t),
% \end{equation}
% where $r(t)\ge0$ is a relaxation that plays the similar role as Backup policy introduced in \cite{chen2021backup} {\color{red} How a relaxation is related to the backup policy?}. However, it is difficult for us to find the appropriate backup policy for controller of complicated dynamic system. Two main drawbacks affect the performance of RACBFs. {\color{red}wording} In the first place, $r(t)$ contracts the coverage of feasible space of states defined by $b(\boldsymbol{x})\ge0$, i.e., the distance $z(t)$ allowable for two vehicles is even smaller {\color{red}This should be larger} by $z(t)-l_{p}-r(t)\ge0$ because of the existence of non-negative $r(t)$. Secondly, the feasibility of solving QP with RACBF constraints is limited by the existence of upper bound of auxiliary input $\nu_{r}(t)$ related to $r(t)$ defined in Eq. (29) in \cite{xiao2021adaptive} {\color{red}What is $\nu_r$? you should make it self-contained.}. We can define the highest order {\color{red}what is this?} of $r(t)$ to be 2, then from \eqref{eq:SACC-HOCBF-sequence} normally we have
% \begin{equation}
% \label{eq:highest-order-RACBF}
% \begin{split}
% \psi_{2}(\boldsymbol{x},r(t),\dot{r}(t),\nu_{r}(t))\coloneqq -u(t)-\nu_{r}(t)\\
% +k_{1}(v_{p}-v(t)-\dot{r}(t))+k_{2}(v_{p}-v(t)-\dot{r}(t)\\
% +k_{1}(z(t)-l_{p}-r(t))\ge0, \nu_{r}(t)=\ddot{r}(t),
% \end{split}
% \end{equation}
% which sets the upper bound {\color{red}This is not clear} for $\nu_{r}(t)$ and there will easily be empty feasible set for $\nu_{r}(t)$ if the lower bound of $\nu_{r}(t)$ defined by constraint (31) in \cite{xiao2021adaptive} is too large. Compared to RACBFs, AVCBFs will neither contract the feasible space of states, nor set the upper bound for $\boldsymbol{\nu}$ (at least no upper bound for $\nu_{1}(t))$ as shown in the proof of Thm. \ref{thm:feasibility-guarantee} in Sec. \ref{subsec: optimal-control}, which shows the great benefits of AVCBFs in terms of safety and feasibility. 

% \subsection{HOCBFs for Auxiliary Coefficients}
\subsection{Adaptive HOCBFs for Safety:\ AVCBFs}
\label{sec:AVCBFs}

Motivated by the SACC example in Sec. \ref{subsec:SACC-problem}, given a function $b:\mathbb{R}^{n}\to\mathbb{R}$ with relative degree $m$ for system \eqref{eq:affine-control-system} and a time-varying vector $\boldsymbol{a}(t)\coloneqq [a_{1}(t),\dots,a_{m}(t)]^{T}$ with positive components called auxiliary variables, the key idea in converting a regular HOCBF into an adaptive
one without defining excessive constraints is to place one auxiliary variable in front of each function in \eqref{eq:sequence-f1} similar to \eqref{eq:SACC-AVBCBF-sequence}. 
As described in Sec. \ref{subsec:SACC-problem}, we only need to define HOCBFs for auxiliary variables to ensure each $a_{i}(t)>0, i \in \{1,...,m\}.$ To realize this, we need to define auxiliary systems that contain auxiliary states $\boldsymbol{\pi}_{i}(t)$ and inputs $\nu_{i}(t)$, through which systems we can extend each HOCBF to desired relative degree, just like $b(\boldsymbol{x})$ has relative degree $m$
with respect to the dynamics \eqref{eq:affine-control-system}. Consider $m$ auxiliary systems in the form 
\begin{equation}
\label{eq:virtual-system}
\dot{\boldsymbol{\pi}}_{i}=F_{i}(\boldsymbol{\pi}_{i})+G_{i}(\boldsymbol{\pi}_{i})\nu_{i}, i \in \{1,...,m\},
\end{equation}
where $\boldsymbol{\pi}_{i}(t)\coloneqq [\pi_{i,1}(t),\dots,\pi_{i,m+1-i}(t)]^{T}\in \mathbb{R}^{m+1-i}$ denotes an auxiliary state with $\pi_{i,j}(t)\in \mathbb{R}, j \in \{1,...,m+1-i\}.$ $\nu_{i}\in \mathbb{R}$ denotes an auxiliary input for \eqref{eq:virtual-system}, $F_{i}:\mathbb{R}^{m+1-i}\to\mathbb{R}^{m+1-i}$ and $G_{i}:\mathbb{R}^{m+1-i}\to\mathbb{R}^{m+1-i}$ are locally Lipschitz. For simplicity, we just build up the connection between an auxiliary variable and the system as $a_{i}(t)=\pi_{i,1}(t), \dot{\pi}_{i,1}(t)=\pi_{i,2}(t),\dots,\dot{\pi}_{i,m-i}(t)=\pi_{i,m+1-i}(t)$ and make $\dot{\pi}_{i,m+1-i}(t)=\nu_{i},$ then we can define many specific HOCBFs $h_{i}$ to enable $a_{i}(t)$ to be positive. 

Given a function $h_{i}:\mathbb{R}^{m+1-i}\to\mathbb{R},$ we can define a sequence of functions $\varphi_{i,j}:\mathbb{R}^{m+1-i}\to\mathbb{R}, i \in\{1,...,m\}, j \in\{1,...,m+1-i\}:$
\begin{equation}
\label{eq:virtual-HOCBFs}
\varphi_{i,j}(\boldsymbol{\pi}_{i})\coloneqq\dot{\varphi}_{i,j-1}(\boldsymbol{\pi}_{i})+\alpha_{i,j}(\varphi_{i,j-1}(\boldsymbol{\pi}_{i})),
\end{equation}
where $\varphi_{i,0}(\boldsymbol{\pi}_{i})\coloneqq h_{i}(\boldsymbol{\pi}_{i}),$ $\alpha_{i,j}(\cdot)$ are $(m+1-i-j)^{th}$ order differentiable class $\kappa$ functions. Sets $\mathcal{B}_{i,j}$ are defined as
\begin{equation}
\label{eq:virtual-sets}
\mathcal B_{i,j}\coloneqq \{\boldsymbol{\pi}_{i}\in\mathbb{R}^{m+1-i}:\varphi_{i,j}(\boldsymbol{\pi}_{i})>0\}, \ j\in \{0,...,m-i\}. 
\end{equation}
Let $\varphi_{i,j}(\boldsymbol{\pi}_{i}),\ j\in \{1,...,m+1-i\}$ and $\mathcal B_{i,j},\ j\in \{0,...,m-i\}$ be defined by \eqref{eq:virtual-HOCBFs} and \eqref{eq:virtual-sets} respectively. By Def. \ref{def:HOCBF}, a function $h_{i}:\mathbb{R}^{m+1-i}\to\mathbb{R}$ is a HOCBF with relative degree $m+1-i$ for system \eqref{eq:virtual-system} if there exist class $\kappa$ functions $\alpha_{i,j},\ j\in \{1,...,m+1-i\}$ as in \eqref{eq:virtual-HOCBFs} such that
\begin{small}
\begin{equation}
\label{eq:highest-SHOCBF}
\begin{split}
\sup_{\nu_{i}\in \mathbb{R}}[L_{F_{i}}^{m+1-i}h_{i}(\boldsymbol{\pi}_{i})+L_{G_{i}}L_{F_{i}}^{m-i}h_{i}(\boldsymbol{\pi}_{i})\nu_{i}+O_{i}(h_{i}(\boldsymbol{\pi}_{i}))\\
+ \alpha_{i,m+1-i}(\varphi_{i,m-i}(\boldsymbol{\pi}_{i}))] \ge \epsilon,
\end{split}
\end{equation}
\end{small}
$\forall\boldsymbol{\pi}_{i}\in \mathcal B_{i,0}\cap,...,\cap \mathcal B_{i,m-i}$. $O_{i}(\cdot)=\sum_{j=1}^{m-i}L_{F_{i}}^{j}(\alpha_{i,m-i}\circ\varphi_{i,m-1-i})(\boldsymbol{\pi}_{i}) $ where $\circ$ denotes the composition of functions. $\epsilon$ is a positive constant which can be infinitely small. 

\begin{remark}
\label{rem:safety-guarantee-2}
If $h_{i}(\boldsymbol{\pi}_{i})$ is a HOCBF illustrated above and $\boldsymbol{\pi}_{i}(0) \in \mathcal {B}_{i,0}\cap \dots \cap \mathcal {B}_{i,m-i},$ then satisfying constraint in \eqref{eq:highest-SHOCBF} is equivalent to making $\varphi_{i,m+1-i}(\boldsymbol{\pi}_{i}(t))\ge \epsilon>0, \forall t\ge 0.$ Based on
\eqref{eq:virtual-HOCBFs}, since $\boldsymbol{\pi}_{i}(0) \in \mathcal {B}_{i,m-i}$ (i.e., $\varphi_{i,m-i}(\boldsymbol{\pi}_{i}(0))>0),$ then we have $\varphi_{i,m-i}(\boldsymbol{\pi}_{i}(t))>0$ (If there exists a $t_{1}\in (0,t_{2}]$, which makes $\varphi_{i,m-i}(\boldsymbol{\pi}_{i}(t_{1}))=0,$ then we have $\dot{\varphi}_{i,m-i}((\boldsymbol{\pi}_{i}(t_{1}))>0\Leftrightarrow \varphi_{i,m-i}(\boldsymbol{\pi}_{i}(t_{1}^{-}))\varphi_{i,m-i}(\boldsymbol{\pi}_{i}(t_{1}^{+}))<0,$ which is against the definition of $\alpha_{i,m+1-i}(\cdot),$ therefore $\forall t_{1}>0, \varphi_{i,m-i}(\boldsymbol{\pi}_{i}(t_{1}))>0,$ note that $t_{1}^{-},t_{1}^{+}$ denote the left and right limit). Based on \eqref{eq:virtual-HOCBFs}, since $\boldsymbol{\pi}_{i}(0) \in \mathcal {B}_{i,m-1-i},$ then similarly we have $\varphi_{i,m-1-i}(\boldsymbol{\pi}_{i}(t))>0,\forall t\ge 0.$ Repeatedly, we have $\varphi_{i,0}(\boldsymbol{\pi}_{i}(t))>0,\forall t\ge 0,$ therefore the sets $\mathcal {B}_{i,0},\dots,\mathcal {B}_{i,m-i}$ are forward invariant.
\end{remark}

For simplicity, we can make $h_{i}(\boldsymbol{\pi}_{i})=\pi_{i,1}(t)=a_{i}(t).$ Based on Rem. \ref{rem:safety-guarantee-2}, each $a_{i}(t)$ will be positive.

The remaining question is how to define an adaptive HOCBF to guarantee $b(\boldsymbol{x})\ge0$ with the assistance of auxiliary variables. Let $\boldsymbol{\Pi}(t)\coloneqq [\boldsymbol{\pi}_{1}(t),\dots,\boldsymbol{\pi}_{m}(t)]^{T}$ and $\boldsymbol{\nu}\coloneqq [\nu_{1},\dots,\nu_{m}]^{T}$ denote the auxiliary states and control inputs of system \eqref{eq:virtual-system}. We can define a sequence of functions 
\begin{small}
\begin{equation}
\label{eq:AVBCBF-sequence}
\begin{split}
&\psi_{0}(\boldsymbol{x},\boldsymbol{\Pi}(t))\coloneqq a_{1}(t)b(\boldsymbol{x}),\\
&\psi_{i}(\boldsymbol{x},\boldsymbol{\Pi}(t))\coloneqq a_{i+1}(t)(\dot{\psi}_{i-1}(\boldsymbol{x},\boldsymbol{\Pi}(t))+\alpha_{i}(\psi_{i-1}(\boldsymbol{x},\boldsymbol{\Pi}(t)))),
\end{split}
\end{equation}
\end{small}
where $i \in \{1,...,m-1\}, \psi_{m}(\boldsymbol{x},\boldsymbol{\Pi}(t))\coloneqq \dot{\psi}_{m-1}(\boldsymbol{x},\boldsymbol{\Pi}(t))+\alpha_{m}(\psi_{m-1}(\boldsymbol{x},\boldsymbol{\Pi}(t))).$ We further define a sequence of sets $\mathcal{C}_{i}$ associated with \eqref{eq:AVBCBF-sequence} in the form 
\begin{equation}
\label{eq:AVBCBF-set}
\begin{split}
\mathcal C_{i}\coloneqq \{(\boldsymbol{x},\boldsymbol{\Pi}(t)) \in \mathbb{R}^{n} \times \mathbb{R}^{m}:\psi_{i}(\boldsymbol{x},\boldsymbol{\Pi}(t))\ge 0\}, 
\end{split}
\end{equation}
where $i \in \{0,...,m-1\}.$
Since $a_{i}(t)$ is a HOCBF with relative degree $m+1-i$ for \eqref{eq:virtual-system}, based on \eqref{eq:highest-SHOCBF}, we define a constraint set $\mathcal{U}_{\boldsymbol{a}}$ for $\boldsymbol{\nu}$ as 
\begin{small}
\begin{equation}
\label{eq:constraint-up}
\begin{split}
\mathcal{U}_{\boldsymbol{a}}(\boldsymbol{\Pi})\coloneqq \{\boldsymbol{\nu}\in\mathbb{R}^{m}:   L_{F_{i}}^{m+1-i}a_{i}+[L_{G_{i}}L_{F_{i}}^{m-i}a_{i}]\nu_{i}\\
+O_{i}(a_{i})+ \alpha_{i,m+1-i}(\varphi_{i,m-i}(a_{i})) \ge \epsilon, i\in \{1,\dots,m\}\},
\end{split}
\end{equation}
\end{small}
where $\varphi_{i,m-i}(\cdot)$ is defined similar to \eqref{eq:virtual-HOCBFs} and $a_{i}(t)$ is ensured positive. $\epsilon$ is a positive constant which can be infinitely small. 

\begin{definition}[AVCBF]
\label{def:AVBCBF}
Let $\psi_{i}(\boldsymbol{x},\boldsymbol{\Pi}(t)),\ i\in \{1,...,m\}$ be defined by \eqref{eq:AVBCBF-sequence} and $\mathcal C_{i},\ i\in \{0,...,m-1\}$ be defined by \eqref{eq:AVBCBF-set}. A function $b(\boldsymbol{x}):\mathbb{R}^{n}\to\mathbb{R}$ is an Auxiliary-Variable Adaptive Control Barrier Function (AVCBF) with relative degree $m$ for system \eqref{eq:affine-control-system} if every $a_{i}(t),i\in \{1,...,m\}$ is a
HOCBF with relative degree $m+1-i$ for the auxiliary system
\eqref{eq:virtual-system}, and there exist $(m-j)^{th}$ order differentiable class $\kappa$ functions $\alpha_{j},j\in \{1,...,m-1\}$
and a class $\kappa$ functions $\alpha_{m}$ s.t.
\begin{small}
\begin{equation}
\label{eq:highest-AVBCBF}
\begin{split}
\sup_{\boldsymbol{u}\in \mathcal{U},\boldsymbol{\nu}\in \mathcal{U}_{\boldsymbol{a}}}[\sum_{j=2}^{m-1}[(\prod_{k=j+1}^{m}a_{k})\frac{\psi_{j-1}}{a_{j}}\nu_{j}] + \frac{\psi_{m-1}}{a_{m}}\nu_{m} \\ +(\prod_{i=2}^{m}a_{i})b(\boldsymbol{x})\nu_{1} +(\prod_{i=1}^{m}a_{i})(L_{f}^{m}b(\boldsymbol{x})+L_{g}L_{f}^{m-1}b(\boldsymbol{x})\boldsymbol{u})\\+R(b(\boldsymbol{x}),\boldsymbol{\Pi})
+ \alpha_{m}(\psi_{m-1})] \ge 0,
\end{split}
\end{equation}
\end{small}
$\forall (\boldsymbol{x},\boldsymbol{\Pi})\in \mathcal C_{0}\cap,...,\cap \mathcal C_{m-1}$ and each $a_{i}>0, i\in\{1,\dots,m\}.$ In \eqref{eq:highest-AVBCBF}, $R(b(\boldsymbol{x}),\boldsymbol{\Pi})$ denotes the remaining Lie derivative terms of $b(\boldsymbol{x})$ (or $\boldsymbol{\Pi}$) along $f$ (or $F_{i},i\in\{1,\dots,m\}$) with degree less than $m$ (or $m+1-i$), which is similar to the form of $O(\cdot )$ in \eqref{eq:highest-HOCBF}.
\end{definition}

\begin{theorem}
\label{thm:safety-guarantee-3}
Given an AVCBF $b(\boldsymbol{x})$ from Def. \ref{def:AVBCBF} with corresponding sets $\mathcal{C}_{0}, \dots,\mathcal {C}_{m-1}$ defined by \eqref{eq:AVBCBF-set}, if $(\boldsymbol{x}(0),\boldsymbol{\Pi}(0)) \in \mathcal {C}_{0}\cap \dots \cap \mathcal {C}_{m-1},$ then if there exists solution of Lipschitz controller $(\boldsymbol{u},\boldsymbol{\nu})$ that satisfies the constraint in \eqref{eq:highest-AVBCBF} and also ensures $(\boldsymbol{x},\boldsymbol{\Pi})\in \mathcal {C}_{m-1}$ for all $t\ge 0,$ then $\mathcal {C}_{0}\cap \dots \cap \mathcal {C}_{m-1}$ will be rendered forward invariant for system \eqref{eq:affine-control-system}, $i.e., (\boldsymbol{x},\boldsymbol{\Pi}) \in \mathcal {C}_{0}\cap \dots \cap \mathcal {C}_{m-1}, \forall t\ge 0.$ Moreover, $b(\boldsymbol{x})\ge 0$ is ensured for all $t\ge 0.$
\end{theorem}

\begin{proof}
If $b(\boldsymbol{x})$ is an AVCBF that is $m^{th}$ order differentiable, then satisfying constraint in \eqref{eq:highest-AVBCBF} while ensuring $(\boldsymbol{x},\boldsymbol{\Pi})\in \mathcal {C}_{m-1}$ for all $t\ge 0$ is equivalent to make $\psi_{m-1}(\boldsymbol{x},\boldsymbol{\Pi})\ge 0, \forall t\ge 0.$ Since $a_{m}(t)>0$, we have $\frac{\psi_{m-1}(\boldsymbol{x},\boldsymbol{\Pi})}{a_{m}(t)}\ge 0.$ Based on
\eqref{eq:AVBCBF-sequence}, since $(\boldsymbol{x}(0),\boldsymbol{\Pi}(0)) \in \mathcal {C}_{m-2}$ (i.e., $\frac{\psi_{m-2}(\boldsymbol{x}(0),\boldsymbol{\Pi}(0))}{a_{m-1}(0)}\ge 0),a_{m-1}(t)>0,$ then we have $\psi_{m-2}(\boldsymbol{x},\boldsymbol{\Pi})\ge 0$ (The proof of this is similar to the proof in Rem. \ref{rem:safety-guarantee-2}), and also $\frac{\psi_{m-2}(\boldsymbol{x},\boldsymbol{\Pi})}{a_{m-1}(t)}\ge 0.$ Based on \eqref{eq:AVBCBF-sequence}, since $(\boldsymbol{x}(0),\boldsymbol{\Pi}(0)) \in \mathcal {C}_{m-3},a_{m-2}(t)>0$ then similarly we have $\psi_{m-3}(\boldsymbol{x},\boldsymbol{\Pi})\ge 0$ and $\frac{\psi_{m-3}(\boldsymbol{x},\boldsymbol{\Pi})}{a_{m-2}(t)}\ge 0,\forall t\ge 0.$ Repeatedly, we have $\psi_{0}(\boldsymbol{x},\boldsymbol{\Pi})\ge 0$ and $\frac{\psi_{0}(\boldsymbol{x},\boldsymbol{\Pi})}{a_{1}(t)}\ge 0,\forall t\ge 0.$ Therefore the sets $\mathcal {C}_{0},\dots,\mathcal {C}_{m-1}$ are forward invariant and $b(\boldsymbol{x})=\frac{\psi_{0}(\boldsymbol{x},\boldsymbol{\Pi})}{a_{1}(t)}\ge 0$ is ensured for all $t\ge 0$.
\end{proof}
Based on Thm. \ref{thm:safety-guarantee-3}, the safety regarding $b(\boldsymbol{x})=\frac{\psi_{0}(\boldsymbol{x},\boldsymbol{\Pi})}{a_{1}(t)}\ge 0$ is guaranteed.

\begin{remark}[Limitation of Approaches with Auxiliary Inputs]
\label{rem: PACBF-AVBCBF} 
Ensuring the satisfaction of the $i^{th}$ order AVCBF constraint as shown in \eqref{eq:AVBCBF-set} when $i\in\{1,\dots,m-1\},$ i.e., $\psi_{i}(\boldsymbol{x},\boldsymbol{\Pi})\ge 0$ will guarantee $\psi_{i-1}(\boldsymbol{x},\boldsymbol{\Pi})\ge 0$ based on the proof of Thm. \ref{thm:safety-guarantee-3}, which theoretically outperforms PACBF. However, both approaches can not ensure satisfying $\psi_{m}(\boldsymbol{x},\boldsymbol{\Pi})\ge 0$ will guarantee $\psi_{m-1}(\boldsymbol{x},\boldsymbol{\Pi})\ge 0$ since the growth of $\boldsymbol{\nu}_{i}$ is unbounded. Therefore in Thm. \ref{thm:safety-guarantee-3}, $(\boldsymbol{x},\boldsymbol{\Pi})\in \mathcal {C}_{m-1}$ for all $t\ge 0$ also needs to be satisfied to guarantee the forward invariance of the intersection of sets. 
\end{remark}

\subsection{Optimal Control with AVCBFs}
\label{subsec: optimal-control}
Consider an optimal control problem as
\begin{small}
\begin{equation}
\label{eq:cost-function-1}
\begin{split}
 \min_{\boldsymbol{u}} \int_{0}^{T} 
 D(\left \| \boldsymbol{u} \right \| )dt,
\end{split}
\end{equation}
\end{small}
where $\left \| \cdot \right \|$ denotes the 2-norm of a vector, $D(\cdot)$ is a strictly increasing function of its argument and $T>0$ denotes the ending time. Since we need to introduce auxiliary inputs $v_{i}$ to enhance the feasibility of optimization, we should reformulate the cost in \eqref{eq:cost-function-1} as
\begin{small}
\begin{equation}
\label{eq:cost-function-2}
\begin{split}
 \min_{\boldsymbol{u},\boldsymbol{\nu}} \int_{0}^{T} 
 [D(\left \| \boldsymbol{u} \right \| )+\sum_{i=1}^{m}W_{i}(\nu_{i}-a_{i,w})^{2}]dt.
\end{split}
\end{equation}
\end{small}
In \eqref{eq:cost-function-2}, $W_{i}$ is a positive scalar and $a_{i,w}\in \mathbb{R}$ is the scalar to which we hope each auxiliary input $\nu_{i}$ converges. Both are chosen to tune the performance of the controller. We can formulate the CLFs, HOCBFs and AVCBFs introduced in Def. \ref{def:control-l-f}, Sec. \ref{sec:AVCBFs} and Def. \ref{def:AVBCBF} as constraints of the QP with cost function \eqref{eq:cost-function-2} to realize safety-critical control. Next we will show AVCBFs can be used to enhance the feasibility of solving QP compared with classical HOCBFs in Def. \ref{def:HOCBF}.

In auxiliary system \eqref{eq:virtual-system}, if we define $a_{i}(t)=\pi_{i,1}(t)=1, \dot{\pi}_{i,1}(t)=\dot{\pi}_{i,2}(t)=\cdots=\dot{\pi}_{i,m+1-i}(t)=0,$ then the way we construct functions and sets in \eqref{eq:virtual-HOCBFs} and \eqref{eq:virtual-sets} are exactly the same as \eqref{eq:sequence-f1} and \eqref{eq:sequence-set1}, which means classical HOCBF is in fact one specific case of AVCBF. Assume that the highest order HOCBF constraint \eqref{eq:highest-HOCBF} conflicts with control input constraints \eqref{eq:control-constraint} at $t=t_{b},$ i.e., we can not find a feasible controller $u(t_{b})$ to satisfy \eqref{eq:highest-HOCBF} and \eqref{eq:control-constraint}. Instead, starting from a time slot $t=t_{a}$ which is just before $t=t_{b}$ ($t_{b}-t_{a}=\varepsilon$ where $\varepsilon$ is an infinitely small positive value), we exchange the control framework of classical HOCBF into AVCBF instantly. Suppose we can find appropriate hyperparameters to ensure two constraints in \eqref{eq:constraint-up} and \eqref{eq:highest-AVBCBF}
% \begin{small}
% \begin{equation}
% \label{eq:constraint-fea-12}
% \begin{split}
%  \nu_{i}
%   > \frac{-L_{F_{i}}^{m+1-i}a_{i}-O_{i}(a_{i})-\alpha_{i,m+1-i}(\varphi_{i,m-i}(a_{i}))}{L_{G_{i}}L_{F_{i}}^{m-i}a_{i}},\\
%   \sum_{j=2}^{m-1}[(\prod_{k=j+1}^{m}a_{k})\frac{\psi_{j-1}}{a_{j}}\nu_{j}] + \frac{\psi_{m-1}}{a_{m}}\nu_{m} +(\prod_{i=2}^{m}a_{i})b(\boldsymbol{x})\nu_{1} \\ \ge -(\prod_{i=1}^{m}a_{i})(L_{f}^{m}b(\boldsymbol{x})+L_{g}L_{f}^{m-1}b(\boldsymbol{x})\boldsymbol{u})-R(b(\boldsymbol{x}),\boldsymbol{\Pi}) \\
% - \alpha_{m}(\psi_{m-1}),  i\in \{1,\dots,m\}
% \end{split}
% \end{equation}
% \end{small}
are satisfied given $\boldsymbol{u}$ constrained by \eqref{eq:control-constraint} at $t_{b},$ then there exists solution $\boldsymbol{u}(t_{b})$ for the optimal control problem and the feasibility of solving QP is enhanced. Relying on AVCBF, We can discretize the whole time period $[0,T]$ into several small time intervals like $[t_{a},t_{b}]$ to maximize the feasibility of solving QP under safety constraints, which calls for the development of automatic parameter-tuning techniques in future.
% \begin{theorem}
% \label{thm:feasibility-guarantee}
% Given an AVCBF $b(\boldsymbol{x})$ from Def. \ref{def:AVBCBF} with corresponding sets $\mathcal{C}_{0}, \dots,\mathcal {C}_{m-1}$ defined by \eqref{eq:AVBCBF-set}, if $(\boldsymbol{x}(0),\boldsymbol{\Pi}(0)) \in \mathcal {C}_{0}\cap \dots \cap \mathcal {C}_{m-1}$ and $L_{G_{i}}L_{F_{i}}^{m-i}a_{i}>0,a_{i}(t)>0, i\in\{1,\dots,m\}$ in \eqref{eq:constraint-up}, then if there exists solution of Lipschitz controller $(\boldsymbol{u},\boldsymbol{\nu})$ that satisfies the constraint in \eqref{eq:highest-AVBCBF} and also ensures $\psi_{0}>0,\dots,\psi_{s}>0,s\in \{0,\dots,m-1\}$ in \eqref{eq:AVBCBF-set}, then the QP with cost function \eqref{eq:cost-function-2} and constraints \eqref{eq:control-constraint},\eqref{eq:AVBCBF-set}-\eqref{eq:highest-AVBCBF} is guranteed to be feasible.
% \end{theorem}

% \begin{proof}
% Rewrite the constraint \eqref{eq:constraint-up} as 
% \begin{equation}
% \label{eq:constraint-fea-1}
% \begin{split}
%  \nu_{i}
%   > \frac{-L_{F_{i}}^{m+1-i}a_{i}-O_{i}(a_{i})-\alpha_{i,m+1-i}(\varphi_{i,m-i}(a_{i}))}{L_{G_{i}}L_{F_{i}}^{m-i}a_{i}},
% \end{split}
% \end{equation}
% where $i\in \{1,\dots,m\}.$ Rewrite the constraint \eqref{eq:highest-AVBCBF} as
% \begin{equation}
% \label{eq:constraint-fea-2}
% \begin{split}
% \sum_{j=2}^{m-1}[(\prod_{k=j+1}^{m}a_{k})\frac{\psi_{j-1}}{a_{j}}\nu_{j}] + \frac{\psi_{m-1}}{a_{m}}\nu_{m} +(\prod_{i=2}^{m}a_{i})b(\boldsymbol{x})\nu_{1} \\ \ge -(\prod_{i=1}^{m}a_{i})(L_{f}^{m}b(\boldsymbol{x})+L_{g}L_{f}^{m-1}b(\boldsymbol{x})\boldsymbol{u})-R(b(\boldsymbol{x}),\boldsymbol{\Pi}) \\
% - \alpha_{m}(\psi_{m-1}),  i\in \{1,\dots,m\}.
% \end{split}
% \end{equation}
% Since $L_{G_{i}}L_{F_{i}}^{m-i}a_{i}>0$ in \eqref{eq:constraint-fea-1}, $\psi_{0}>0,\dots,\psi_{s}>0,s\in \{0,\dots,m-1\}$ in \eqref{eq:constraint-fea-2} and $a_{1}>0,\dots,a_{m}>0,$ we have $(\prod_{i=2}^{m}a_{i})b(\boldsymbol{x})>0,(\prod_{k=j+1}^{m}a_{k})\frac{\psi_{j-1}}{a_{j}}\nu_{j}>0,j\in \{2,\dots,s\}$ are always positive, 
% then there always exist large enough $\nu_{1},\dots,\nu_{s}$ satisfying constraints above {\color{red}you are assuming a very specific (13).} (the upper bounds of $\nu_{1},\dots,\nu_{s}$ are unlimited), hence the feasibility of QP with cost function \eqref{eq:cost-function-2} and constraints \eqref{eq:control-constraint},\eqref{eq:AVBCBF-set}-\eqref{eq:highest-AVBCBF} is guaranteed.  {\color{red}Control limitations (2) are the most critical factor in the feasibility. You completely ignore this. The proof is very sloppy.}
% \end{proof}

Besides safety and feasibility, another benefit of using AVCBFs is that the conservativeness of the control strategy can also be ameliorated. For example, from \eqref{eq:AVBCBF-sequence}, we can rewrite $\psi_{i}(\boldsymbol{x},\boldsymbol{\Pi})\ge 0$ as
\begin{equation}
\label{eq:AVCBF-rewrite}
\begin{split}
\dot{\phi}_{i-1}(\boldsymbol{x},\boldsymbol{\Pi})+k_{i}(1+\frac{\dot{a}_{i}(t)}{k_{i}a_{i}(t)}) \phi_{i-1}(\boldsymbol{x},\boldsymbol{\Pi})\ge0,
\end{split}
\end{equation}
where $\phi_{i-1}(\boldsymbol{x},\boldsymbol{\Pi})=\frac{\psi_{i-1}(\boldsymbol{x},\boldsymbol{\Pi})}{a_{i}(t)},\alpha_{i}(\psi_{i-1}(\boldsymbol{x},\boldsymbol{\Pi}))=k_{i}a_{i}(t)\phi_{i-1}(\boldsymbol{x},\boldsymbol{\Pi}), k_{i}>0, i\in \{1,\dots,m\}.$ Similar to PACBFs, we require $1+\frac{\dot{a}_{i}(t)}{k_{i}a_{i}(t)}\ge0,$ which gives us $\dot{a}_{i}(t)+k_{i}a_{i}(t)\ge0.$
The term $\frac{\dot{a}_{i}(t)}{a_{i}(t)}$ can be adjusted adaptable  to ameliorate the conservativeness of control strategy that $k_{i}\phi_{i-1}(\boldsymbol{x},\boldsymbol{\Pi})$ may have, i.e., the ego vehicle can brake earlier or later given time-varying control constraint $\boldsymbol{u}_{min}(t)\le \boldsymbol{u} \le\boldsymbol{u}_{max}(t),$ which confirms the adaptivity of AVCBFs to control constraint and conservativeness of control strategy. 

\begin{remark}[Parameter-Tuning for AVCBFs]
\label{rem: parameter-tuning}
Based on the analysis of \eqref{eq:AVCBF-rewrite}, we require $\dot{a}_{i}(t)+k_{i}a_{i}(t)\ge0.$ If we define first order HOCBF constraint for $a_{i}(t)>0$ as $\dot{a}_{i}(t)+l_{i}a_{i}(t)\ge0,$ we should choose hyperparameter $l_{i}\le k_{i}$ to guarantee $\dot{a}_{i}(t)+k_{i}a_{i}(t)\ge\dot{a}_{i}(t)+l_{i}a_{i}(t)\ge 0.$ For simplicity, we can use $l_{i}=k_{i}.$ In cost function \eqref{eq:cost-function-2}, we can tune hyperparameters $W_{i}$ and $a_{i,w}$ to adjust the corresponding ratio $\frac{\dot{a}_{i}(t)}{a_{i}(t)}$ to change the performance of the optimal controller.
\end{remark}

\begin{remark}
\label{rem: sufficient-con}
Note that the satisfaction of the constraint in \eqref{eq:highest-AVBCBF} is a sufficient condition for the satisfaction of the original constraint $\psi_{0}(\boldsymbol{x},\boldsymbol{\Pi})>0,$ it is not necessary to introduce auxiliary variables as many as from $a_{1}(t)$ to $a_{m}(t),$ which allows us to choose an appropriate
number of auxiliary variables for the AVCBF constraints to reduce the complexity. In other words, the number of auxiliary variables can be less than or equal to the relative degree $m$.
\end{remark}
\section{Algorithm for $K$-Max Bandits with General Continuous Distribution}\label{sec:general-continuous}

We now present our solution framework for continuous $K$-Max bandits, beginning with the fundamental regularity condition that enables discretization-based learning:
\begin{assumption}\label{ass:bi-lipschitz}
Each outcome distribution $D_i$ is supported on $[0,1]$ with a bi-Lipschitz continuous cumulative distribution function (CDF) $F_i$. Specifically, there exists $L \geq 1$ such that for any $i \in [N]$ and $0 \leq v < u \leq 1$:
\begin{align*}
    \frac{1}{L}(u - v) \leq F_i(u) - F_i(v) \leq L(u - v).
\end{align*}
\end{assumption}
Many studies on MAB or CMAB consider $[0, 1]$-supported arms \citep{abbasi2011improved,chen2013combinatorial,slivkins2019introduction,lattimore2020bandit}. The bi-Lipschitz continuity is also common in practice \citep{li2017provably,wang2019optimism,liu2023optimistic} and satisfied by many distributions such as (truncated) Gaussians, mixed uniforms, Beta distributions, etc.


\subsection{The Discretization of Countinuous $K$-Max Bandits}
\label{sec:discretized-K-Max}
Since it is complex to estimate the general continuous distributions, a natural idea is to perform \textit{discretization} with granularity $\epsilon$.
Below, we define the \textit{discrete $K$-Max bandits} (called $\bar\gB$) converted from the continuous $K$-Max bandits $\mathcal{B}^*$, where each discrete arm's outcome $\bar X_i$ is discretized from the continuous random variable $X_i$ under $\epsilon$:
\begin{align}\label{eq:discretize-outcome}
    \bar X_i = \sum_{j \in [M]} \1\left[X_i \in M_j\right] \cdot v_j,
\end{align}
where $M = \ceil{1/\epsilon}$\footnote{Without loss of generation, we can take $\epsilon$ such that $M\epsilon > 1$.} is the number of discretization bins, $M_j := [(j-1)\epsilon, j\epsilon)$ is the $j$-th bin, and $v_j := (j-1)\epsilon$ is the approximate value of $j$-th bin. 
%
We also let $M_{\le j} = \cup_{j' \le j} M_{j'}$ and $M_{\ge j} = \cup_{j' \ge j} M_{j'}$.
%
For simplicity, we denote $p^*_{i,j}$ as the probability that $X_i$ falls in $M_j$. For every $i \in [N]$ and $j \in [M]$,
\begin{align*}
    p^*_{i,j} := \Prob[X_i \in M_j] = \Prob[\bar X_i = v_j].
\end{align*}
Therefore, $\bar \gB$ only depends on the discrete probability set $\vp^* = \{p_{i,j}^* : i \in [N], j \in [M]\}$. Moreover, we set $\bar r(S; \vp^*)$ as the expected reward of an action $S$ in discrete $K$-Max bandits under the probability set $\vp^*$:
\begin{align*}
    \bar r(S ; \vp^*) &= \sum_{j\in [M]} v_j \cdot \Prob\left[\max_{i\in S} (\bar X_i) = v_j\right]
\end{align*}
A key observation is that $\max_{i\in S} (\bar X_i) = v_j$ is equivalent to $\max_{i\in S} (X_i)\in M_j$. This means $\Prob\left[\max_{i\in S} (\bar X_i) = v_j\right] = \Prob\left[\max_{i\in S} (X_i)\in M_j\right]$, which gives an upper bound for the discretization error as follows. The formal version is provided by \Cref{lemma:discrete-error-formal} (in \Cref{app:discrete-error}). 
\begin{lemma}\label{lemma:discrete-error}
For any $S \in \gS$, we have
$$|r^*(S) - \bar r(S; \vp^*)| \le \epsilon.$$ % for any $S \in \gS$.
\end{lemma} 





\subsection{Converting a Discrete Arm to a Set of Binary Arms}
\label{sec:discrete-binary}
Follow the classical process in \citet{wang2023combinatorial}, we can convert a discrete arm $X_i$ to a set of binary arms and estimate the parameters $\vq^* = \{q^*_{i,j} : i \in [N], j \in [M]\}$ instead of $\vp^*$, where
\begin{align}\label{eq:qstar-def}
    q_{i,j}^* := \frac{p_{i,j}^*}{1 - \sum_{j' > j} p_{i,j'}^*}, \ p^*_{i,j} = q^*_{i,j} \cdot \prod_{j' > j} (1 - q^*_{i,j'}).
\end{align}
Let $\{\bar Y_{i,j}\}_{i \in [N],j \in [M]}$ be independent binary random variables such that $\bar Y_{i,j}$ takes value $v_j$ with probability $q_{i,j}^*$, and value $0$ otherwise. Then $\max_{j \in [M]}\{\bar Y_{i,j}\}$ has the same distribution as $\bar X_{i}$.
For any $S \in \gS$, define $\bar r_q(S; \vq)$ as the expected maximum reward of $\{\bar Y_{i,j}\}_{i \in S, j \in [M]}$ with probability set $\vq$. 
Then we have 
\begin{lemma}\label{lemma:r-q-r}
    For any $\vp$ and $\vq$ satisfying \Cref{eq:qstar-def}, we have $$\bar r_q(S; \vq) = \bar r(S; \vp), \quad \forall S \in \gS.$$
\end{lemma}
The formal version of this lemma is given in \Cref{lemma:r-q-r-formal}. 
Moreover, the function $\bar r_q$ is monotone with respect to  $\vq$, i.e., 
\begin{lemma}[{\citet[Lemma 3.1]{wang2023combinatorial}}]
\label{lemma:monotone}
    For two probability set $\vq'$ and $\vq$ such that $q'_{i,j} \ge q_{i,j}$ holds for any $i \in [N], j \in [M]$, we have 
    $$\bar r_q(S; \vq') \ge \bar r_q(S;\vq),\quad  \forall S \in \gS.$$
\end{lemma}



\subsection{An Efficient Offline Oracle for Discrete $K$-Max Bandits} \label{sec:offline-oracle}
For any discrete $K$-Max bandits with probability set $\vp$, we can apply the \textit{PTAS} algorithm \citep{chen2016combinatorial} as a polynomial time offline $\alpha$-approximation optimization oracle for any given $\alpha < 1$. 
Moreover, for any probability set $\vq$, we can convert it to $\vp$ by \Cref{eq:qstar-def}, input this $\vp$ to the PTAS oracle and get the approxiamation solution $\operatorname{PTAS}(\vp)$ satisfying
\begin{equation}\label{eq:ptas}
\begin{aligned}
    \bar r_q\left(\operatorname{PTAS}(\vp); \vq\right) &= \bar r\left(\operatorname{PTAS}(\vp); \vp\right) \\
    &\ge \alpha \cdot \max_{S \in \gS} \bar r(S; \vp) = \alpha \cdot \max_{S \in \gS} \bar r_q(S; \vq).
\end{aligned}
\end{equation}
In the following algorithm, we set $\alpha = 1-\epsilon$ and control the relative error to achieve sublinear regret guarantees.



\subsection{Efficient Algorithm for Continuous $K$-Max Bandits}\label{sec:algorithm}
Building on the methodology in previous subsections, we adapt the framework in \citet{wang2023combinatorial}, and
present \texttt{DCK-UCB} (Discretized Continuous $K$-Max with Upper Confidence Bounds), the first efficient algorithm addressing $K$-Max bandits with general continuous outcome distributions. 
%
Generally speaking, we first discretize the countinuous $K$-Max bandits to discrete $K$-Max bandits. Then we convert every discrete arm to a set of binary arms, and estimate the corresponding $\vq^*$. Finally, we convert $\vq^*$ back to $\vp^*$, input $\vp^*$ to the $\operatorname{PTAS}$ oracle, and get the action we want to select.
%
% 

\begin{algorithm}[!t]
\caption{\texttt{DCK-UCB}: Discretization Continuous $K$-Max Bandits with Upper Confidence Bonus}
\label{alg}
\begin{algorithmic}[1]
\REQUIRE Discretization granularity $\epsilon$, upper confidence bonuses $\{\beta_{i,j}^t : i \in [N], j \in [M], t \in [T]\}$, and the offline $\alpha$-approximated optimization oracle $\operatorname{PTAS}$ for discrete $K$-Max bandits \citep{chen2016combinatorial}.
\STATE Initialize $M \leftarrow \ceil{1/\epsilon}$, $\hat q_{i, 1}^1 \leftarrow 1$ for every $i \in [N]$, and $\hat q_{i,j}^1 \leftarrow 0$ for every $i\in [N], j > 1$.
\FOR{$t=1,2,\ldots,T$}
\STATE For every $i \in [N], j \in [M]$, set   
\begin{align}\label{eq:def-barq}
    \bar q_{i, j}^t \leftarrow \min\left\{ \hat q_{i,j}^t + \beta_{i,j}^t + (K-1)\frac{L^4}{j^2}, 1\right\}.
\end{align}
% \COMMENT{Construct optimistic estimator $\bar q_{i,j}^t$.}
\STATE Convert $\bar \vq^t$ to $\bar \vp^t$ by \Cref{eq:qstar-def}.
\STATE \label{algline:oracle} Choose action $S_t \leftarrow \operatorname{PTAS}(\bar \vp^t)$. 
\STATE Observe $(r_t,i_t)$ by executing action $S_t$. Denote $j_t$ as the range number of $r_t$, i.e., $r_t \in M_{j_t}$.
% \COMMENT{Execute $S_t$ and receive value-index feedback.}
\STATE For any $i, j \in [N] \times [M]$,
\begin{align*}
    C_t(i, j) = 
        C_{t-1}(i, j) + \1\left[ i = i_t \And j = j_t \right]
\end{align*}
and
\begin{align*}
    SC_t(i, j) = 
        SC_{t-1}(i, j) + \1\left[ i \in S_t \And j \ge j_t \right]
\end{align*}
% \COMMENT{Update the counter $C_t$ and $SC_t$ for estimation.}
\STATE Calculate estimator $\hat q_{i,j}^{t+1} \leftarrow \frac{C_t(i,j)}{SC_t(i,j)}$, for every $i \in [N]$ and $j\in [M]$.
% \COMMENT{Estimate $q^*_{i,j}$ by $\hat q_{i,j}^{t+1}$.} 
\ENDFOR
\end{algorithmic}
\end{algorithm}



\Cref{alg} presents the pseudo-code of \texttt{DCK-UCB}. In Line 3, we calculate the optimistic estimator $\bar q_{i,j}^t$ which upper bounds $q^*_{i,j}$ with high probability.
%
This is done by adding two upper confidence bonus terms $\beta_{i,j}^t$ and $(K-1)L^4/j^2$.
%
Analysis shows that $\bar q_{i,j}^t \ge q^*_{i,j}$ with high probability (Lemma \ref{lemma:concentration}). The detailed discussion on this estimator will be given in the following paragraphs.
%
In Lines 4-5, the agent converts this $\bar \vq$ to $\bar \vp$, and then runs the offline $\alpha$-approximation optimization oracle $\operatorname{PTAS}$ with $\alpha = 1 - \epsilon$ to get action $S_t$ for execution.
%
In Line 6, the agent gets the value-index return $(r_t, i_t)$, and discretizes the value $r_t$ to the index of bin $j_t$, i.e., $r_t \in M_{j_t}$.
%
In Lines 7-8, the agent estimates $\vq^*$ by two counters: $C_t(i,j)$ counts the times when $(i,j)$ exactly equals the feedback $(i_t,j_t)$, and $SC_t(i,j)$ counts the number of steps $\tau \le t$ satisfying $i \in S_\tau$ and $j_\tau \le j$. As outlined in \Cref{alg}, each step of the algorithm has polynomial time and space complexity, which demonstrates the computational tractability of \texttt{DCK-UCB}. 


\paragraph{Biased Estimator.} 


The key challenge in the algorithm design and theoretical analysis is that $\hat q_{i,j}^t$ is not an unbiased estimator for $q_{i,j}^*$. This means that except for the confidence radius due to the randomness of the environment, we still need another bonus term to bound the bias to guarantee that $\bar q_{i,j}^t$ is a UCB for $q_{i,j}^*$. 
%

Specifically, note that 
\begin{eqnarray*}
    q^*_{i,j} = \frac{p^*_{i,j}}{1 - \sum_{j' > j} p^*_{i,j'}} = \frac{p^*_{i,j}}{\sum_{j'=1}^j p^*_{i,j'}} = \frac{\Prob[X_i \in M_j]}{\Prob[X_i \in M_{\le j}]} 
\end{eqnarray*}
If we have an assumption that when $i\in S_\tau$ and $j_\tau = j$, $X_i(\tau) \in M_j$ implies $i = i_\tau$, then we can guarantee that $\hat q_{i,j}^{t} = C_t(i,j) / SC_{t}(i,j)$ is an unbiased estimator for $q^*_{i,j}$. This is because that in this case, $\frac{C_t(i,j)}{ SC_{t}(i,j) }= \frac{\# \text{ of } i_\tau = i, j_\tau = j}{\# \text{ of } i\in S_\tau, j_\tau \le j} $ is the fraction of $X_i(\tau) \in M_j$ condition on $i\in S_\tau, j_\tau \le j$, 
%
which is an unbiased estimator for 
\begin{eqnarray*}
    && \Prob[X_i(\tau) \in M_j \mid i\in S_\tau, j_\tau \le j]\\
    &=&  \frac{\Prob[X_i(\tau) \in M_j, i\in S_\tau, j_\tau \le j]}{\Prob[i\in S_\tau, j_\tau \le j]}\\
    &=&\frac{\Prob[X_i(\tau)  \in M_j]\cdot \Prob[X_k(\tau)  \in M_{\le j}, \forall k \in S_{\tau}, k \neq i]}{\Prob[X_i(\tau)  \in M_{\le j}] \cdot \Prob[X_k(\tau)  \in M_{\le j}, \forall k \in S_{\tau}, k \neq i]}\\
    &=&\frac{\Prob[X_i(\tau) \in M_j]}{\Prob[X_i(\tau) \in M_{\le j}]}
\end{eqnarray*}

However, we know that in the discrete K-Max bandits converted from the continuous K-Max bandits, there is no such assumption (different from \citet{wang2023combinatorial} who requires deterministic tie-breaking rule). When multiple arm has $X_i(\tau) \in M_j$, the observed winning arm $i_t = \arg\max X_i(\tau)$ is not a fixed one, and even we do not know the distribution of the winner. 
%
Because of this, we cannot guarantee that condition on $i\in S_\tau, j_\tau \le j$, we increase the counter for every time $X_i(\tau) \in M_j$. Some steps that $X_i(\tau) \in M_j$ but $X_i(\tau)$ is not the winner are missed. 
%
This nondeterministic tie-breaking effect, arising from the continuous-to-discrete transformation, induces systematic negative bias in conventional estimators $\{\hat q_{i,j}^t\}$.
%
Therefore, to guarantee that our used $\{\bar q_{i,j}^t\}$ is an upper confidence bound of $\{ q_{i,j}^*\}$, we need another bonus term (i.e., the term $(K-1)\frac{L^4}{j^2}$), given by a novel concentration analysis with bias-aware error control. This is shown in the following key lemma, where the formal version is in \Cref{lemma:concentration-formal}.


\begin{lemma}\label{lemma:concentration}
Under \Cref{ass:bi-lipschitz}, let the confidence radius be defined as  
\begin{align}\label{eq:def-beta}
    \beta_{i,j}^t := \sqrt{8\frac{\log(NMt)}{SC_{t-1}(i,j)}}.
\end{align}
Then with probability at least $1 - t^{-2}$,
\begin{align}
    \left|\hat q_{i,j}^t - q^*_{i,j}\right| \le {\beta_{i,j}^t} + (K-1)\cdot(L^4/j^2),
\end{align}
holds for every $t \in [T]$, $i \in [N]$ and $j \in [M]$.
\end{lemma}

The bound in \Cref{lemma:concentration} decomposes into an exploration bonus term $\beta_{i,j}^t$ and a bias compensation term $(K-1)\frac{L^4}{j^2}$. The exploration bonus term arises from the randomness of the environment, which is almost the same with existing researches \citep{wang2017improving,liu2023contextual,wang2023combinatorial}.
The bias compensation term, on the other hand, comes from the nondeterministic tie-breaking effect in the continuous-to-discrete transformation. 
%
As we have explained, this term is because that condition on $i\in S_\tau, j_\tau \le j$, there are some time steps that $X_i(\tau) \in M_j$ but arm $i$ is not the winner and thus we miss these steps in counter $C_{i,j}^t$.
%
When this happens, we know that there must be at least one other arm $i'\ne i, i'\in S_{\tau}$ such that $X_{i'}(\tau) \in M_j$.
%
This probability can be upper bounded by 
\begin{align*}
    &\sum_{i'\ne i, i'\in S_{\tau}} \Prob[X_i(\tau) \in M_j, X_{i'}(\tau) \in M_j \mid i\in S_\tau, j_\tau \le j]\\
    =\!&\sum_{i'\ne i, i'\in S_{\tau}} \frac{p^*_{i,j}p^*_{i',j} }{ \sum_{j'\le j}p^*_{i,j'} \sum_{j'\le j}p^*_{i',j'}} \le (K-1)\frac{(L\epsilon)^2}{(j\epsilon/L)^2},
\end{align*}
where the last inequality is because of bi-Lipschitz assumption \Cref{ass:bi-lipschitz}.


Notably, the bias term dominates for small $j$ values due to the influence of other arms becomes higher when condition on $j_\tau \le j$ with smaller $j$.
%
However, our regret analysis in \Cref{sec:result} suggests that the amplified bias for small $j$ has diminishing impact on cumulative regret -- a crucial property enabling our sublinear regret guarantee.




\subsection{Theoretical Results}
\label{sec:result}
We establish the first efficient algorithm \texttt{DCK-UCB} (\Cref{alg}) which enjoys the sublinear regret guarantees in continuous $K$-Max bandits problem with value-index feedback. 
\begin{theorem}
\label{thm:main}
Under \Cref{ass:bi-lipschitz}, let the offline optimization oracle be a PTAS implementation \citep{chen2016combinatorial}. Given the exploration bonus term $\beta_{i,j}^t$ in \Cref{eq:def-beta}, discretization granularity $\epsilon = \gO(L^{-2}K^{-3/4}N^{1/4}T^{-1/4})$ and PTAS approximation factor $\alpha = 1 - \epsilon$, \Cref{alg} enjoys the regret guarantee
\begin{align*}
    \gR(T) \le \wt{\gO}(L^{2}N^{\frac{1}{4}}K^{\frac{5}{4}}T^{\frac{3}{4}}).
\end{align*}
\end{theorem}
The formal statement with precise constants appears in \Cref{thm:main-formal}. Our analysis reveals that careful calibration of the discretization-error versus statistical-estimation trade-off enables the first sublinear regret guarantee $\gO(T^{3/4})$ for continuous $K$-Max bandits. 

\paragraph{Comparison to Prior Works.} The $\gO(T^{3/4})$ regret upper bound of \texttt{DCK-UCB} (\Cref{alg}) shown in \Cref{thm:main} advances the state-of-the-art in several directions. \citet{wang2023combinatorial} can achieve an $\gO(\sqrt{T})$ regret upper bound in the discrete $K$-Max bandits, but their algorithm cannot work for the continuous case due to non-zero discretization error and nondeterministic tie-breaking.
%
Recent work on submodular bandits \citep{pasteris2023sum,fourati2024combinatorial} attains $O(T^{2/3})$ regret via greedy oracles, but this approach suffers dual limitations: (1) The baseline of their regret is $\sum_{t=1}^T (1-1/e)r^*(S^*)$, but not $\sum_{t=1}^T r^*(S^*)$. In our definition, their regret becomes linear.   (2) Their algorithm requires the availability of submitting any subset of $\mathcal{A}$ with size less than or equal to $K$, which may not be practical in some applications, such as recommendation systems or portfolio selection that need to always submit size $K$ subsets. Our framework resolves both issues through our novel bias-corrected estimators with PTAS integration, which is both efficient and effective in dealing with continuous $K$-Max bandits.


\subsection{Proof Sketch of \Cref{thm:main}}
In this section we outline the proof of \Cref{thm:main}, which consists of four main steps. 

\paragraph{Step 1: From continuous regret to discretized regret.}
Let $\Delta_t := r^*(S^*) - r^*(S_t)$ be the regret for each round $t$. To control the regret, we aim to bound the summation of $\Delta_t$. 
$$ \gR(T) = \E\left[\sum_{t=1}^T \Delta_t \right]. $$
We first transfer the regret from continuous $K$-Max bandits to the discrete case. With \Cref{lemma:discrete-error}, we have
\begin{align*}
    \Delta_t &\le  \bar r(S^*; \vp^*) - \bar r(S_t; \vp^*) + 2\epsilon.
\end{align*}


\paragraph{Step 2: From discretized regret to estimation error.}
%
Recall the definition of $\bar r_q$ in \Cref{sec:discrete-binary}, we have $\bar r_q(S; \vq) = \bar r(S; \vp)$ for any $S \in \gS$, and probability set $\vp, \vq$ satisfying \Cref{eq:qstar-def}. 
By the monotonicity of $\bar r_q$ (in \Cref{lemma:monotone}) and the concentration analysis (in \Cref{lemma:concentration}),
we have with high probability, $\forall (i,j)\in[N]\times[M]$, $\bar q_{i,j}^t \ge q^*_{i,j}$ holds for any $t \in [T]$, which implies
\begin{align*}
    \bar r_q(S^*;\bar\vq^t) \ge \bar r_q(S^*; \vq^*).
\end{align*}
Moreover, by the property of $\alpha$-approximated offline optimization oracle PTAS \citep{chen2016combinatorial}  (in \Cref{eq:ptas}) with $\alpha = 1 - \epsilon$, 
\begin{align*}
    (1 - \epsilon) \bar r_q(S^*; \bar \vq^t) \le (1 - \epsilon) \max_{S \in \gS} \bar r_q(S; \bar \vq^t) \le \bar r_q(S_t; \bar \vq^t),
\end{align*}
which implies the conversion from $\Delta_t$ to the estimation error term
\begin{align*}
    \Delta_t &\le \bar r_q(S^*; \bar \vq^t) - \bar r_q(S_t; \vq^*) + 2\epsilon \\
    &\le (1 - \epsilon) \bar r_q(S^*; \bar \vq^t) - \bar r_q(S_t; \va^*) + 3\epsilon \\
    &\le \bar r_q(S_t; \bar \vq^t) - \bar r_q(S_t; \vq^*) + 3\epsilon.
\end{align*}
Therefore, we then focus on bounding the estimation error $\bar\Delta_t := \bar r_q(S_t; \bar \vq^t) - \bar r_q(S_t; \vq^*)$ to guarantee the sublinear regret upper bound:
\begin{align}\label{eq:regret-to-bar-delta}
    \gR(T) = \E\left[\sum_{t=1}^T \Delta_t \right] \le \E\left[\sum_{t=1}^T \bar\Delta_t \right] + 3T\epsilon.
\end{align}


\paragraph{Step 3: Decompose the estimation error.}
By similar methods as achieving the Triggering Probability Modulated (TPM) smoothness condition in cascading bandits \citep{wang2017improving} and $K$-Max bandits for binary distributions \citep{wang2023combinatorial}, we propose the following lemma.
\begin{lemma}
\label{lemma:tpm}
Denote the probability of event $\{j_t \le j\}$ as 
$$Q_j^*(S_t) :=  \prod_{k \in S_t, j' > j} (1 - q_{k,j'}^*).$$ Then we have
\begin{align}
    \bar \Delta_t \le 2\sum_{i \in S_t, j \in [M]} Q_j^*(S_t) \cdot v_j \cdot \left|\bar q_{i,j}^t - q^*_{i,j}\right|.
\end{align}
\end{lemma}
Equipped with \Cref{lemma:tpm}, we decompose $\bar \Delta_t$ into two parts through our novel concentration analysis in \Cref{lemma:concentration} and the definition of optimistic estimator $\bar q_{i,j}^t$ in \Cref{eq:def-barq}
\begin{equation}\label{eq:bar-delta}
\begin{aligned}
    \bar \Delta_t &\le \quad \underbrace{4 \sum_{i \in S_t, j \in [M]} Q_{j}^{*}(S_t) \cdot v_j \cdot \beta_{i,j}^t}_{\texttt{Bonus}_t} \\
    &\quad + \underbrace{4 \sum_{i \in S_t, j \in [M]} Q_{j}^{*}(S_t) \cdot v_j \cdot (K-1)\frac{L^4}{j^2}}_{\texttt{Bias}_t}.
\end{aligned}
\end{equation}


\paragraph{Step 4: Bound the Bonus and Bias terms.}
For the Bonus term, we apply standard analysis for combinatorial bandits with triggering arms in \citep{wang2017improving, liu2023contextual} where we encounter $NM$ binary arms in total and select $KM$ binary arms in every action and get
\begin{align}\label{eq:bonus-sum}
    \E\left[\sum_{t=1}^T \texttt{Bonus}_t \right] \le \wt\gO\left(\sqrt{(NM) \cdot (KM) \cdot T}\right).
\end{align}

To control the bias terms, we recall that $v_j = (j-1)\epsilon$. 
Therefore, we can write
\begin{equation}\label{eq:bias}
    \begin{aligned}
    \texttt{Bias}_t &\le 4K^2L^4 \sum_{j \in [M]} (j-1)\epsilon/j^2 \\
    &\le \gO\left( K^2L^4\epsilon\log(M) \right),
\end{aligned}
\end{equation}

Therefore, combining \Cref{eq:regret-to-bar-delta,eq:bar-delta,eq:bonus-sum,eq:bias}, the regret can be bounded by
\begin{align*}
    \gR(T) &\le \E\left[\sum_{t=1}^T \texttt{Bonus}_t + \texttt{Bias}_t\right] + \gO(T\epsilon) \\
    &\le \wt{\gO}\left(\sqrt{NKM^2T} + K^2L^4T\epsilon\right),
\end{align*}
where $M = \ceil{1/\epsilon}$. By taking $\epsilon = \gO(T^{-\frac{1}{4}}K^{-\frac{3}{4}}N^{\frac{1}{4}}L^{-2})$, we have
\begin{align*}
    \gR(T) \le \wt{\gO} \left( L^2N^{\frac{1}{4}}K^{\frac{5}{4}}T^{\frac{3}{4}} \right).
\end{align*}


\section{Better Performance in a Special Case: Exponential Distributions}
\label{sec:kminexp}

\newcommand{\Exp}{\operatorname{Exp}}

In this section, we demonstrate how specific distributional structure enables the improvement of the regret guarantee from $\wt{\gO}(T^{\frac{3}{4}})$ to $\wt{\gO}(\sqrt{T})$. Specifically, we investigate the special case where each distribution $D_i$ for $i \in [N]$ follows the exponential distribution with linear parameterization. 

Exponential distributions naturally model arrival or failure times in networked systems, job completion times in distributed computing, and service durations in queuing systems. 
A canonical application arises in server scheduling, where the goal is to select $K$ servers to minimize the service latency. Here, each server's latency can be modeled as an exponential random variable with a rate parameter $\mu_i$, and the overall performance of the $K$ selected servers is the lowest latency achieved among them. 
%
Here, the random outcome $X_i$ can be viewed as a random loss, and the winning loss is the minimum one. 
Moreover, we consider a linear parameterization to parameter $\mu_i$, which allows incorporating features like distance, traffic, or weather conditions into the model.


\subsection{The $K$-Min Exponential Bandits}

Based on the intuition, in this section we consider a special case of $K$-Max bandits: the $K$-Min exponential bandits.
Here each arm $i$ generates loss $X_i$ from an exponential distribution with linear parameterization. 
%
Specifically, each outcome distribution is an exponential distribution, i.e., $X_i \sim D_i = \Exp(\mu_i)$ where $\mu_i > 0$ is the parameter of arm $i$. Moreover, we assume that there exists a $d$-dimension unknown parameter $\theta^* \in \R^d$ and a known feature mapping $\phi : [N] \to \R^d$ such that $\mu_i = \langle \phi(i), \theta^* \rangle$ holds for any $i \in [N]$. The feature mapping $\phi$ satisfies that $\|\phi(i)\|_2 \le 1$ and the unknown parameter $\theta^*$ satisfies $\theta^* \in \Theta  \subset \R^d$, where $\sup_{\theta \in \Theta} \|\theta\|_2 \le V$.
%
The agent observes \textit{only} the minimum loss $\ell_t = \min_{i \in S_t} X_i(t)$ after playing subset $S_t \in \mathcal{S} = \{S \subseteq [N] : |S| = K\}$.
That is, we consider the weaker full bandit feedback case.

Let $\ell^*(S) := \E[\ell_t \mid S]$ be the expected loss for action $S \in \gS$, we further denote the best action $S^* = \argmin_{S \in \gS} \ell^*(S)$ and similarly introduce the regret metric to evaluate the performance of this agent:
\begin{align*}
    \gR(T) = \E\left[\sum_{t=1}^T \ell^*(S_t) - \ell^*(S^*)\right].
\end{align*}

Note that we can let $Z_i(t)= - X_i(t)$ and view $Z_i(t)$ as a kind of reward, and let $r_t = \max_{i \in S_t} Z_i(t)$. Then we can see that $\ell_t = \min_{i \in S_t} X_i(t) = \min_{i \in S_t} -Z_i(t) = - \max_{i \in S_t} Z_i(t) = -r_t$. By this way, we can view $K$-Min exponential bandits as a special case of $K$-Max bandits. 
%
However, one important difference is that in $K$-Min exponential bandits, we do not have value-index feedback, i.e., we do not know the winner's index. This is a full \textit{bandit feedback} setting, and making $K$-Min exponential bandits even more challenging. 

\subsection{Algorithm and Results}

The key observation in $K$-Min exponential bandits is that the minimum of several exponential distributions still follows an exponential distribution. That is, we have 
\begin{align*}
    \min_{i \in S} X_i \sim \Exp\left(\sum_{i \in S} \mu_i\right) = \Exp\left(\sum_{i \in S} \langle \phi(i), \theta^*\rangle \right).
\end{align*}
Therefore, it becomes much easier to estimate the true parameter $\theta^*$ by MLE. 
%
Specifically, let $\psi(S)  := \sum_{i \in S} \phi(i)$, $\forall S \in \gS$. Then with chosen action $S_t$ and parameter $\theta$, the observed loss should follow the exponential distribution $\Exp\left(\sum_{i \in S} \phi(i)^T \theta \right) = \Exp\left(\psi(S)^T\theta\right)$, whose probability density function is $f(x) = \psi(S)^T\theta  e^{\left(-\psi(S)^T\theta  x\right)}$. 
%
Because of this, the log-likelihood function is 
\begin{equation}
\begin{aligned}
    L_t(\ell_t; S_t, \theta) :&= - \log \left( \psi(S_t)^\top \theta e^{\left( -\psi(S_t)^\top \theta \ell_t\right)} \right).
\end{aligned}
\end{equation}

Denote $\gL_t(\theta)$ as the summation of $L_t$ and a regularization term
\begin{align}\label{eq:loglikelihood-def}
    \gL_t(\theta; \lambda) := \sum_{i < t} L_i(\ell_i; S_i, \theta) + \frac{\lambda}{2}\|\theta\|^2,
\end{align}
where $\lambda$ is the regularization factor. Then we present the algorithm \texttt{MLE-Exp} for $K$-Min exponential bandits in \Cref{alg:k-min}.

\begin{algorithm}
\caption{\texttt{MLE-Exp}: MLE for $K$-Min Exponential Bandits}
\label{alg:k-min}
\begin{algorithmic}[1]
\REQUIRE Regularization factors $\{\lambda_t\}_{t\in [T]}$, confidence radius $\{\gamma_t\}_{t\in [T]}$, and probability constant $\delta$.
\FOR{$t = 1, \ldots, T$}
    \STATE Compute MLE $\hat\theta_t$ by
    $$\hat{\theta}_t \leftarrow \argmin_{\theta \in \R^d} \mathcal{L}_t(\theta; \lambda_t),$$
    where $\gL_t(\theta;\lambda)$ is given in \Cref{eq:loglikelihood-def}. 
    % \COMMENT{Estimate $\theta^*$ by MLE $\hat{\theta}_t$.}
    \STATE Construct the confidence set $C_t(\hat\theta_t; \delta, \lambda_t)$ according to \Cref{eq:def-confidence-set}
    \STATE $(S_t, \wt{\theta}_t) \leftarrow \argmax_{S \in \gS, \theta \in C_t(\hat{\theta}_t; \delta, \lambda_t)} \langle \psi(S), \theta \rangle$ 
    % \COMMENT{Choose action with minimum expected loss.}
    \STATE Play action $S_t$ and observe the loss $\ell_t$. 
    % \COMMENT{Execute action $S_t$ and receive full-bandit feedback.}
\ENDFOR
\end{algorithmic}
\end{algorithm}

In Line 2 of \Cref{alg:k-min}, we estimate the MLE $\hat\theta_t$ by minimizing the summation of the log-likelihood function and the regularization term $\gL_t(\theta, \lambda_t)$. 
%
Given $\lambda_t$ a priori, we will write $\gL_t(\theta)$ instead of $\gL_t(\theta, \lambda_t)$ for simplicity. 
%
Inspired by \citet{liu2024almost, lee2024unified, liu2024combinatorial}, in Line 3, we construct a confidence set $C_t(\hat\theta_t;\delta)$, centered at the MLE $\hat\theta_t$ with confidence radius $\gamma_t(\delta)$, based on the gradient term $g_t(\theta) := -\nabla_\theta \gL_t(\theta) + \sum_{i < t} \ell_i \psi(S_i)$ and Hessian matrix $H_t(\theta) := \nabla^2_\theta \gL_t(\theta)$:
\begin{align}\label{eq:}
    C_t(\hat\theta_t; \delta) :=  
    &\left\{ \theta \in \Theta : \left\|g_t(\theta) - g_t(\hat\theta_t)\right\|_{H_t^{-1}(\theta)} \le \gamma_t(\delta)\right\},
\end{align}
where $\gamma_t$ is the confidence radius. 
Then in Line 4, we apply a double oracle to look for the action $S$ whose expected loss under a parameter $\theta$ in the confidence set ($1/\langle \psi(S), \theta \rangle$) is minimized.
%
Finally, we select this greedy action in Line 5 and use the observation to update the next time step's MLE and confidence set. 





The regret guarantees of \Cref{alg:k-min} is given below.

\begin{theorem}
\label{thm:kminexp}
With $\delta = 1/T$, $\lambda_t = \Theta(d\log T)$, and $\gamma_t = \Theta(\sqrt{d\log T})$, \Cref{alg:k-min} satisfies:
\begin{align*}
    \mathcal{R}(T) \leq \widetilde{\gO}\left(\sqrt{d^3 T}\right).
\end{align*}
\end{theorem}
Compared with the $O(T^{\frac{3}{4}})$ regret upper bound for general continuous K-Max bandits, here the regret upper bound is reduced to $O(T^{\frac{1}{2}})$ (which is nearly minimax optimal) even without the feedback of winner's index, due to the utilization of the exponential distribution's property. In short, we do not need to use a discretization method and can directly construct an unbiased estimator for the known parameter $\theta^*$.
%
The proof is inspired by previous analysis of general linear bandits \citep{lee2024unified,liu2024almost} and logistics bandits \citep{liu2024combinatorial}, and we defer the detailed proof to \cref{Appendix:k-min}.

\section{Conclusion}
In this work, we propose a simple yet effective approach, called SMILE, for graph few-shot learning with fewer tasks. Specifically, we introduce a novel dual-level mixup strategy, including within-task and across-task mixup, for enriching the diversity of nodes within each task and the diversity of tasks. Also, we incorporate the degree-based prior information to learn expressive node embeddings. Theoretically, we prove that SMILE effectively enhances the model's generalization performance. Empirically, we conduct extensive experiments on multiple benchmarks and the results suggest that SMILE significantly outperforms other baselines, including both in-domain and cross-domain few-shot settings.


% \section*{Impact Statement}
% This paper presents work whose goal is to advance the field of Machine Learning. There are many potential societal consequences of our work, none which we feel must be specifically highlighted here.

\bibliography{ref}
\bibliographystyle{apalike}


\newpage
\appendix
% \onecolumn
\appendixpage

\startcontents[section]
\printcontents[section]{l}{1}{\setcounter{tocdepth}{2}}
\newpage

% \section*{Appendix}

\section{Omitted Proofs in \Cref{sec:general-continuous}}

In this section, we present the omitted proofs in \Cref{sec:general-continuous}, which include the full proof of \Cref{thm:main}.

\subsection{Discretization Error}\label{app:discrete-error}
First we show that the discretization from original continuous problem $\gB^*$ to $\bar\gB$ with discretization width $\epsilon$ will involve controllable error in expected loss, which is shown in \Cref{lemma:discrete-error} and formalized by the following lemma. 

\begin{lemma}\label{lemma:discrete-error-formal}
For any $S \in \gS$, we have
\begin{align}\label{eq:discrete-error}
    \bar r(S; \vp^*) \le r^*(S) \le \bar r(S; \vp^*) + \epsilon.
\end{align}
\begin{proof}
Notice that we have 
\begin{align*}
    \Prob\left[\max_{i\in S} (\bar X_i) = v_j\right] &= \sum_{I\subset S} \prod_{i\in I}\Prob[\bar X_i = v_j] \cdot \prod_{k \in S, k \notin I}\Prob[\bar X_k < v_j] \\
    &= \sum_{I\subset S} \prod_{i\in I} \Prob[X_i \in M_j] \cdot \prod_{k \in S, k \notin I}\Prob[X_k \in M_{\le j-1}] \\
    &= \Prob\left[\max_{i \in S} (X_i) \in M_j\right].
\end{align*}
Therefore, by definition of $r^*(S)$, we have
\begin{align*}
    r^*(S) &= \sum_{j\in [M]} \int_{r \in M_j} r\cdot \rd\Prob_{\max_{i\in S}(X_i)}(r) \\
    &\ge \sum_{j\in [M]} (j-1)\epsilon \int_{r \in M_j} \rd\Prob_{\max_{i\in S}(X_i)}(r) \\
    &= \sum_{j\in [M]} (j-1)\epsilon \cdot \Prob\left[\max_{i \in S} (X_i) \in M_j\right] \\
    &= \sum_{j\in [M]} (j-1)\epsilon \cdot \Prob\left[\max_{i\in S} (\bar X_i) = (j-1)\epsilon\right] \\
    &= \bar r(S; \vp^*),
\end{align*}
where the inequality is given by the definition of $M_j$. Then we achieve the left-hand side of \Cref{eq:discrete-error}. For the other side, we can similarly establish
\begin{align*}
    r^*(S) &= \sum_{j\in [M]} \int_{r \in M_j} r\cdot \rd\Prob_{\max_{i\in S}(X_i)}(r) \\
    &\le \sum_{j\in [M]} j\epsilon \int_{r \in M_j} \rd\Prob_{\max_{i\in S}(X_i)}(r) \\
    &= \sum_{j\in [M]} (j - 1)\epsilon \cdot \Prob\left[\max_{i \in S} (X_i) \in M_j\right] + \epsilon \cdot  \sum_{j\in [M]}  \Prob\left[\max_{i \in S} (X_i) \in M_j\right]\\
    &= \bar r(S; \vp^*) + \epsilon.
\end{align*}
\end{proof}
\end{lemma}

\subsection{Converting to Binary Arms}
As detailed in \Cref{sec:discrete-binary}, we set
\begin{align*}
    q_{i,j}^* := \frac{p_{i,j}^*}{1 - \sum_{j' > j} p_{i,j'}^*}, \ p^*_{i,j} = q^*_{i,j} \cdot \prod_{j' > j} (1 - q^*_{i,j'}),
\end{align*}
which implies
\begin{align*}
    q^*_{i,j} = \frac{p^*_{i,j}}{1 - \sum_{j' > j} p^*_{i,j'}} = \frac{p^*_{i,j}}{\sum_{j'=1}^j p^*_{i,j'}} =
    \frac{\Prob[X_i \in M_j]}{\Prob[X_i \in M_{\le j}]}.
\end{align*} 
For any given probability set $\vq = \{q_{i,j}: i \in [N], j \in [M]\}$, we can apply \Cref{eq:qstar-def} to get the corresponding $\vp$ defined as
\begin{align*}
    p_{i,j} = q_{i,j} \cdot \prod_{j' > j} (1 - q_{i,j'}).
\end{align*}
Assume $\{Y_{i,j}^\vq\}_{i\in[N], j \in [M]}$ is the set of independent binary random variables that $Y_{i,j}^\vq$ takes value $v_j = (j-1)\epsilon$ with probability $q_{i,j}$ and takes value $0$ otherwise. And $\{X_i^\vp\}_{i\in[N]}$ is the set of independent discrete random variables that $X_i^\vp$ takes value $v_j$ with probability $p_{i,j}$ for every $j \in [M]$. Therefore, by simple calculation, we have $\max_{j\in[M]}\{Y_{i,j}^\vq\}$ has the same distribution of $X_i^\vp$. 

$\bar r_q(S; \vq)$ is defined as the expected maximum reward of $\{Y_{i,j}^\vq\}_{i\in S,j\in[M]}$ Then we can write
\begin{align}\label{eq:def-rq}
   \bar r_q(S; \vq) = \sum_{j \in [M]} v_j \cdot \left( Q_j(S; \vq) - Q_{j-1}(S;\vq)\right),
\end{align}
where we denote for simplicity
\begin{align}\label{eq:def-Qj}
    Q_j(S; \vq):= \prod_{k \in S, j' > j} (1 - q_{k,j'}).
\end{align}
$Q_j(S; \vq)$ is actually the probability of the event that every arm in $\{\bar Y_{k,j'}\}_{k \in S, j' > j}$ does not sample a non-zero value.

Equipped with the above statement, we can establish the following lemma:

\begin{lemma}\label{lemma:r-q-r-formal}
For any $\vp$ and $\vq$ satisfying \Cref{eq:qstar-def}, we have for any $S \in \gS$, $$\bar r_q(S; \vq) = \bar r(S; \vp).$$
\begin{proof}
    Notice that by definition, we have
    \begin{align*}
        \bar r(S; \vp) = \E\left[\max_{i\in S} X_i^\vp\right],
    \end{align*}
    and
    \begin{align*}
        \bar r_q(S; \vq) = \E\left[\max_{i \in S} \max_{j \in [M]} Y_{i,j}^\vq\right].
    \end{align*}
    Notice that $\max_{j\in[M]}\{Y_{i,j}^\vq\}$ has the same distribution of $X_i^\vp$, we have
    \begin{align*}
        \bar r_q(S; \vq) &= \E\left[\max_{i \in S} \max_{j \in [M]} Y_{i,j}^\vq\right] \\
        &= \E\left[\max_{i \in S} X_i^\vp\right] = \bar r(S; \vp).
    \end{align*}
\end{proof}
\end{lemma}

\subsection{Biased Concentration}
We aim to use $\hat q_{i,j}^t$ to estimate $q_{i,j}^*$. However, this is a biased estimation. In this section, we carefully control the gap between the biased estimator $\hat q_{i,j}^t$ and the true probability $q^*_{i,j}$.


We set $c_t(i, j) := \mathbbm{1}[(i_t, j_t) = (i, j)]$ which is $\gF_t$-measurable. Then \Cref{alg} counts the summation of $c_t(i,j)$ as $C_t(i,j)$:
\begin{align*}
    C_t(i,j) = \sum_{\tau = 1}^t c_\tau(i,j),
\end{align*}
which is $\gF_{t-1}$-measurable. 

For given action $S_t$ in round $t$, the environment will sample a set of outcomes $\{X_i(t) \sim D_i : i \in S_t\}$. The value-index feedback is $r_t = \max_{i \in S_t} X_i(t)$, $i_t = \argmax_{i \in S_t} X_i(t)$. \Cref{alg} consider $j_t$ such that $r_t \in M_{j_t}$. We denote $I_t = \argmax_{i \in S_t} \bar X_i(t)$, where $\bar X_i(t)$ is the discretized of $X_i(t)$ induced by \Cref{eq:discretize-outcome}. Notice that under event $\gE_0$, $\argmax_{i \in S} X_i(t)$ is unique. But $I_t$ might be a set with multiple indices. We emphasize that $S_t$ is $\gF_{t-1}$ measurable and $(i_t, r_t, j_t, I_t)$ are $\gF_t$ measurable.

Then we can provide the following lemma.
\begin{lemma}\label{lemma:concentration-formal}
Under event $\gE_0$, we have for every $t \in [T]$ and $(i, j) \in [N] \times [M]$,
\begin{align*}
    \left|\hat q_{i,j}^t - q^*_{i,j}\right| \le \sqrt{8\frac{\log(NMt)}{SC_t(i,j)}} + (K-1)\cdot(L^4/j^2),
\end{align*}
with probability at least $1 - T^{-2}$, where we denote this good event as $\gE_1$. 

\begin{proof}
Denote $q_{i,j}(S_t) := \mathbbm{1}[i \in S_t] \cdot \Prob[(i_t, j_t) = (i, j) \mid j_t \le j, S_t]$, and $q_{i,j}^*(S_t) := \mathbbm{1}[i \in S_t] \cdot \Prob[I_t \ni i, j_t = j \mid j_t \le j, S_t]$. Therefore, for given $i \in [N], j \in [M]$, we have
\begin{align*}
    \E[\1[i \in S_t] \cdot c_t(i,j)\cdot \mathbbm{1}[j_t \le j] \mid  S_t] = q_{i,j}(S_t) \cdot \Prob[j_t \le j \mid S_t]
\end{align*}

By summation over time step $1, 2, \cdots, t$, we have
\begin{align*}
    % \E[C_t \mid j_t \le j] = 
    \sum_{\tau=1}^t \E[\1[i \in S_\tau]\cdot c_\tau(i,j)\cdot \mathbbm{1}[j_\tau \le j] \mid  S_\tau] &= \sum_{\tau=1}^t q_{i,j}(S_\tau)\cdot \Prob[j_\tau \le j \mid S_\tau] \\
    &= \sum_{\tau=1}^t \E\left[q_{i,j}(S_\tau)\cdot \mathbbm{1}[j_\tau \le j] \mid S_\tau\right],
\end{align*}
which implies that 
\begin{align*}
     \E\left[\sum_{\tau \le t, i \in S_\tau, j_\tau \le j}c_\tau(i,j) \middle| S_1, S_2, \cdots, S_t \right]  = \E\left[\sum_{\tau \le t, j_\tau \le j}
    q_{i,j}(S_\tau)\middle| S_1, \cdots, S_t\right]
\end{align*}
Notice that $S_t$ is $\gF_{t-1}$-measurable. By the definition of $q_{i,j}(S_\tau)$, we have
\begin{align*}
    \E\left[\sum_{\tau \le t, i \in S_\tau, j_\tau \le j}c_\tau(i,j) -
    q_{i,j}(S_\tau) \middle| \gF_{t-1} \right] = 0.
\end{align*}

If we count the number of $\tau$ that satisfies $i \in S_\tau$ and $j_\tau \le j$ is exactly $SC_t(i,j) = \sum_{\tau=1}^t \mathbbm{1}[i \in S_\tau, j_\tau \le j]$. Therefore, by Azuma-Hoeffding inequality, we have for fixed $SC_t(i,j)$, with probability at least $1 - \delta$,
\begin{align*}
    \left|\sum_{\tau \le t, i \in S_\tau, j_\tau \le j} c_\tau(i, j) - \sum_{\tau \le t, i \in S_\tau, j_\tau \le j} q_{i,j}(S_\tau)\right| \le \sqrt{2SC_t(i,j)\log(T/\delta)},
\end{align*}
By union inequality, we have 
\begin{align*}
    \left|\sum_{\tau \le t, i \in S_\tau j_\tau \le j} c_\tau(i, j) - \sum_{\tau \le t, i \in S_\tau, j_\tau \le j} q_{i,j}(S_\tau)\right| \le \sqrt{8SC_t(i,j)\log(NMT)}, 
\end{align*}
holds for any $t\in [T]$, $SC_t(i,j)$, and $(i, j) \in [N] \times [M]$ with probability at least $1 - T^{-2}$. We denote this good event as $\gE_1$ which satisfies $\Prob[\neg\gE_1] \le T^{-2}$.

We recall the definition of $\hat q_{i,j}^t$ given in \Cref{alg}
\begin{align*}
    \hat q_{i,j}^t = \frac{C_t(i,j)}{SC_t(i,j)} = \frac{\sum_{\tau \le t} c_{\tau}(i,j)}{SC_t(i,j)} = \frac{\sum_{\tau \le t} \1[i \in S_\tau]\cdot c_{\tau}(i,j)\mathbbm{1}[j_\tau \le j]}{SC_t(i,j)}.
\end{align*}
Under this good event $\gE_1$, we have for every $t \in [T]$ and $(i, j) \in [N] \times [M]$,
\begin{align*}
    \left| \hat q_{i,j}^t - \frac{\sum_{\tau \le t, i \in S_\tau, j_\tau \le j} q_{i,j}(S_\tau)}{SC_t(i,j)}\right| \le \sqrt{8\frac{\log(NMT)}{SC_t(i,j)}}
\end{align*}

Below we bound the difference between $q^*_{i,j}(S_t)$ and $q_{i,j}(S_t)$ for any $S_t \in \gS$. For given $(i, j)$ with $i \in S_t$, we have
\begin{align*}
    q^*_{i,j}(S_t) - q_{i,j}(S_t) &= \Prob[I_t \ni i, j_t = i \mid j_t \le j, S_t] - \Prob[i_t = i, j_t = j \mid j_t \le j, S_t] \\
    &= \Prob[I_t \ni i, i_t \neq i, j_t = j \mid j_t \le j, S_t] \\
    &\le \sum_{k \in S_t, k \neq i}\frac{\Prob[X_i \in M_j]\Prob[X_k \in M_j]}{\Prob[X_i \in M_{\le j}]\Prob[X_k \in M_{\le j}]} \\
    &\le (K-1)\cdot \frac{(L\epsilon)^2}{(j\epsilon/L)^2} = (K-1)\cdot L^4/j^2,
\end{align*}
where the last inequality holds by \Cref{ass:bi-lipschitz} and $\Prob[X_i \in M_{\le j}] = \sum_{j'=1}^j p_{i,j}^* \le j\frac{\epsilon}{L}, \forall i \in [N]$. 

Notice that for every $S_t \in \gS$ and $i \in S_t, j \in [M]$, we have
\begin{align*}
    q_{i,j}^*(S_t) &=
    \Prob[I_t \ni i, j = j_t \mid j_t \le j, S_t] \\
    &= \frac{\Prob[I_t \ni i, j = j_t \mid S_t]}{\Prob[j_t \le j \mid S_t]} \\
    &= \frac{\Prob[X_i(t) \in M_{j} \And x_k(t) \in M_{\le j}, \forall k \in S_t \mid S_t]}{\Prob[x_k(t) \in M_{\le j}, \forall k \in S_t \mid S_t]} \\
    &= \frac{\Prob[X_i \in M_j]\cdot \Prob[X_k \in M_{\le j}, \forall k \in S, k \neq i]}{\Prob[X_i \in M_{\le j}] \cdot \Prob[X_k \in M_{\le j}, \forall k \in S, k \neq i]} \\
    &= \frac{\Prob[X_i \in M_j]}{\Prob[X_i \in M_{\le j}]} \\
    &= \Prob[X_i \in M_j \mid X_i \in M_{\le j}]\\
    &= q_{i,j}^*.
\end{align*}
Therefore, we have 
\begin{align*}
    \left|\hat q_{i,j}^t - q^*_{i,j}\right| = \left|\hat q_{i,j}^t - \frac{\sum_{\tau \le t, i \in S_\tau j_\tau \le j}q_{i,j}^*(S_\tau)}{SC_t(i,j)}\right| \le \sqrt{8\frac{\log(NMt)}{SC_t(i,j)}} + (K-1)\cdot(L^4/j^2)
\end{align*}
\end{proof}
\end{lemma}

\subsection{Optimistic Estimation}

\begin{lemma}
\label{lemma:optimism}
    For $\beta_{i,j}^t$ given in \Cref{eq:def-beta}, under event $\gE_0$ and $\gE_1$, we have
    \begin{align*}
        \bar q_{i,j}^t \ge q^*_{i,j}.
    \end{align*}
    Moreover, by the offline $(1-\epsilon)$-approximated optimization oracle PTAS \citep{chen2013combinatorial}, we have
    \begin{align*}
        \bar r_q(S_t, \bar \vq^t) \ge (1-\epsilon) \cdot \bar r_q(S^*; \bar \vq^t).
    \end{align*}
\begin{proof}
Notice that in \Cref{alg} we define
\begin{align*}
    \bar q_{i,j}^t = \min\left\{\hat q_{i,j}^t + \beta_{i,j}^t + \frac{(K-1)L^4}{j^2}, 1\right\}.
\end{align*}
By \Cref{lemma:concentration-formal}, we have under $\neg\gE_0$ and $\gE_1$,
\begin{align*}
    \hat q_{i,j}^t \ge q_{i,j}^* - \beta_{i,j}^t - \frac{(K-1)L^4}{j^2},
\end{align*}
where the inequality holds by the definition of $\beta_{i,j}^t$ in \Cref{eq:def-beta} and $SC_{t-1}(i,j) \le SC_t(i,j)$.
Since $q_{i,j}^* \le 1$, we have
\begin{align*}
    \bar q_{i,j}^t \ge q^*_{i,j}.
\end{align*}

Since in \Cref{alg}, we set action $S_t \leftarrow \operatorname{PTAS}(\hat \vp^t)$ where $\hat\vp^t$ is converted from $\hat\vq^t$ by \Cref{eq:qstar-def}. Then by \Cref{lemma:r-q-r,lemma:monotone}, we have
\begin{align*}
    \bar r_q(S_t; \bar\vq^t) = \bar r(S_t; \bar\vp^t)  \ge (1-\epsilon)\max_{S \in \gS} \bar r(S; \bar \vp^t) \ge (1-\epsilon)\bar r(S^*; \bar \vp^t) = (1-\epsilon) \bar r_q(S^*; \bar \vq^t).
\end{align*}
\end{proof}
\end{lemma}

\subsection{Regret Decomposition}

\begin{lemma}
\label{lemma:tpm-formal}
Denote $Q_j^*(S_t) :=  \prod_{k \in S_t, j' > j} (1 - q_{k,j'}^*)$. We have
\begin{align}
    |\bar r_q(S_t; \bar \vq^t) - \bar r_q(S_t; \vq^*)| \le 2\sum_{i \in S_t, j \in [M]} Q_j^*(S_t) \cdot v_j \cdot \left|\bar q_{i,j}^t - q^*_{i,j}\right|.
\end{align}
\begin{proof}
This lemma is given by directly apply Lemma 3.3 in \citet{wang2023combinatorial} by definition of $\bar r_q$ in \Cref{eq:def-rq}.
\end{proof}
\end{lemma}

\begin{lemma}\label{lemma:regret-decomp}
Under \Cref{ass:bi-lipschitz}, we can bound the regret of \Cref{alg} by
\begin{align*}
    \gR(T) \le \E\left[\sum_{t=1}^T \texttt{Bonus}_t + \texttt{Bias}_t\middle| \gE_0, \gE_1\right] + 3T\epsilon + T^{-1},
\end{align*}
where $\texttt{Bonus}_t$ and $\texttt{Bias}_t$ is defined by
\begin{align}\label{eq:def-bonus-t}
    \texttt{Bonus}_t := 4 \sum_{i \in S_t, j \in [M]} Q_{j}^{*}(S_t) \cdot v_j \cdot \beta_{i,j}^t,
\end{align}
and
\begin{align}\label{eq:def-bias-t}
    \texttt{Bias}_t := 4 \sum_{i \in S_t, j \in [M]} Q_{j}^{*}(S_t) \cdot v_j \cdot (K-1)\frac{L^4}{j^2}.
\end{align}
\begin{proof}
This lemma formalize the first three steps of proof sketch. Denote $\Delta_t := r^*(S^*) - r^*(S_t)$, we have 
\begin{align*}
    \gR(T) = \E\left[\Delta_t\right].
\end{align*}
By \Cref{lemma:discrete-error}, we have
\begin{align*}
    \Delta_t &\le  \bar r(S^*; \vp^*) - \bar r(S_t; \vp^*) + 2\epsilon.
\end{align*}
Then we have
\begin{align*}
    \gR(T) &\le \Prob[\gE_0] \cdot \E\left[\sum_{t=1}^T\Delta_t \middle| \gE_0\right] + \Prob[\neg\gE_0] \cdot T \\
    &\le \E\left[\sum_{t=1}^T\Delta_t \middle| \gE_0\right] \\
    &\le \E\left[\sum_{t=1}^T\bar r(S^*; \vp^*) - \bar r(S_t; \vp^*) \middle| \gE_0\right] + 2T\epsilon,
\end{align*}
where the first inequality holds by property of conditional expectations and $\Delta_t \le 1$ and the second inequality is due to $\Prob[\neg\gE_0] = 0$.

Notice that under $\gE_0$ and $\gE_1$, by \Cref{lemma:monotone,lemma:optimism}, we have
\begin{align*}
    \bar r_q(S_t; \vq^t) \ge (1-\epsilon)\bar r_q(S^*; \bar \vq_t) \ge (1-\epsilon)\bar r_q(S^*; \vq^*).
\end{align*}
Then with \Cref{lemma:r-q-r-formal}, we have
\begin{align*}
    \gR(T) &\le \E\left[\sum_{t=1}^T\bar r_q(S^*; \vq^*) - \bar r_q(S_t; \vq^*) \middle| \gE_0\right] + 2T\epsilon \\
    &\le \E\left[\sum_{t=1}^T\bar r_q(S^*; \vq^*) - \bar r_q(S_t; \vq^*) \middle| \gE_0,\gE_1\right] + \Prob[\neg\gE_1]\cdot T +  2T\epsilon \\
    &\le \E\left[\bar r_q(S_t; \vq^t) - \bar r_q(S_t; \vq^*)\right] + 3T\epsilon + T^{-1},
\end{align*}
where the last inequality holds by $\epsilon\bar r_q(S^*;\vp^*) \le \epsilon$ and $\Prob[\neg\gE_1] \le T^{-2}$ shown in \Cref{lemma:concentration-formal}.  

Therefore, applying \Cref{lemma:tpm}, we get
\begin{align*}
    \gR(T) \le \E\left[\sum_{t=1}^T \texttt{Bonus}_t + \texttt{Bias}_t\middle| \gE_0, \gE_1\right] + 3T\epsilon + T^{-1},
\end{align*}
where $\texttt{Bonus}_t$ and $\texttt{Bias}_t$ is defined in \Cref{eq:def-bonus-t,eq:def-bias-t}.
\end{proof}
\end{lemma}

\subsection{Bounding the Bonus Terms}

We apply similar methods in \citet{wang2017improving,liu2023contextual} to give the bounds of $\sum_t\texttt{Bonus}_t$. We first give the following definitions.

\begin{definition}[{\citet[Definition 5]{wang2017improving}}]\label{def:TPgroup}
    Let $(i,j) \in [N] \times [M]$ be the index of binary arm and $l$ be a positive natural number, define the triggering probability group (of actions)
    \[
    S_j^l = \{S \in \mathcal{S} \mid 2^{-l} < Q_j^*(S) \leq 2^{-l+1}\}.
    \]
    Notice $\{S_j^l\}_{l \geq 1}$ forms a partition of $\{S \in \mathcal{S} \mid Q_j^*(S) > 0\}$.
\end{definition}
\begin{definition}[{\citet[Definition 6]{wang2017improving}}]\label{def:TPcounter}
    For each group $S_j^l$ (\Cref{def:TPgroup}), we define a corresponding counter $N_{i,j}^l$. 
    In a run of a learning algorithm, the counters are maintained in the following manner. 
    All the counters are initialized to $0$. In each round $t$, if the action $S_t$ is chosen, then update $N^l(i,j)$ to $N^l(i,j) + 1$ for every $(i, j)$ that $i \in S_t$, $S_t \in S_j^l$. 
    Denote $N_t^l({i,j})$ at the end of round $t$ with $N^l(i,j)$. 
    In other words, we can define the counters with the recursive equation below:
    \begin{align*}
        N_t^l(i,j) =
        \begin{cases} 
            0, & \text{if } t = 0, \\
            N_{t-1}^l(i,j) + 1, & \text{if } t > 0, i\in S_t, S_t \in S_j^l, \\
            N_{t-1}^l(i,j), & \text{otherwise}.
        \end{cases}
    \end{align*}
\end{definition}
\begin{definition}[{\citet[Definition 7]{wang2017improving}}]\label{def:TPevent}
Given a series of integers $\{l_{i,j}^{\max}\}_{i \in [N],j\in[M]}$, we say that the triggering is nice at the beginning of round $t$ (with respect to $l_{i,j}^{\max}$), if for every group $S_j^l$(\Cref{def:TPgroup}) identified by binary arm $(i,j)$ and $1 \leq l \leq l_{i,j}^{\max}$, as long as 
\[
\sqrt{\frac{8 \log (NMT)}{\frac{1}{3} N_{t-1}^l(i,j)\cdot 2^{-l}}} \leq 1,
\]
there is $SC_{t-1}(i,j) \geq \frac{1}{3} N_{t-1}^l(i,j) \cdot 2^{-l}$. We denote this event with $\gE_2(t)$. It implies
\[
\beta_{i,j}^t = \sqrt{\frac{8 \log(NMT)}{ SC_{t-1}(i,j)}} \leq \sqrt{\frac{8 \log(NMT)}{\frac{1}{3} N_{t-1}^l(i,j) \cdot 2^{-l}}}.
\]
\end{definition}
Therefore, we show that $\gE_2(t)$ happens with high probability for every $t$. 
\begin{lemma}[{\citet[Lemma 4]{wang2017improving}}]\label{lemma:TPprob}
For a series of integers $\{l_{i,j}^{\max}\}_{i \in [N],j\in[M]}$, $$\Prob[\neg \mathcal{E}_2(t)] \leq \sum_{i \in [N],j\in[M]} l_{i,j}^{\max} t^{-2},$$
for every round $t \geq 1$. 

\begin{proof}
We prove this lemma by showing $\Prob[N_{t-1}^l(i,j) = s, SC_{t-1}(i,j) \leq \frac{1}{3} N_{t-1}^l(i,j) \cdot 2^{-l}] \leq t^{-3}$, for any fixed $s$ with $0 \leq s \leq t - 1$ and $\sqrt{\frac{8 \log(NMT)}{\frac{1}{3} s \cdot 2^{-l}}} \leq 1$. Let $t_k$ be the round that $N^l(i,j)$ is increased for the $k$-th time, for $1 \leq k \leq s$. Let $Z_k = \1[S_{t_k} \ni i, j_{t_k} \le j]$ be a Bernoulli variable, that is, $SC_{t_k}(i,j)$ increase in round $t_k$. When fixing the action $S_{t_k}$, $Z_k$ is independent from $Z_1, \ldots, Z_{k-1}$. Since $S_{t_k} \in S_j^l$, $\mathbb{E}[Z_k \mid Z_1, \ldots, Z_{k-1}] \geq 2^{-l}$. Let $Z = Z_1 + \cdots + Z_s$. By multiplicative Chernoff bound \citep{upfal2005probability}, we have
\[
\Prob\left\{Z \leq \frac{1}{3} s \cdot 2^{-l}\right\} \leq \exp\left(-\frac{\left(\frac{2}{3}\right)^2 s \cdot 2^{-l}}{2}\right) \leq \exp\left(-\frac{\left(\frac{2}{3}\right)^2 18 \log t}{2}\right) < \exp(-3 \log t) = t^{-3}.
\]

By the definition of $SC_{t-1}(i,j)$ and the condition $N_{t-1}^l(i,j) = s$, we have $SC_{t-1}(i,j) \geq Z$. Thus
\begin{align*}
\Prob[N_{t-1}^l(i,j) = s, SC_{t-1}(i,j) &\leq \frac{1}{3} N_{t-1}^l(i,j) \cdot 2^{-l}]\\
&\leq \Prob[N_{t-1}^l(i,j) = s, Z \leq \frac{1}{3} s \cdot 2^{-l}] \\
&\leq \Prob[Z \leq \frac{1}{3} s \cdot 2^{-l}] \leq t^{-3}.
\end{align*}

By taking $i,j$ over $[N]\times[M]$, $l$ over $1, \ldots, l_{i,j}^{\max}$, $s$ over $0, \ldots, t - 1$ and applying the uninon bound, the lemma holds.
\end{proof}
\end{lemma}

\begin{lemma}\label{lemma:bonus-t-bound}
For given constant $C$, we have
\begin{align*}
    \sum_{t=1}^T \texttt{Bonus}(t) \le 16NM + 12288\frac{KNM^2\log(NMT)}{C} + TC + \frac{\pi^2}{6} \left\lceil \log_2\frac{16KM}{C} \right\rceil.
\end{align*}
\begin{proof}
For given constant $C$, we can define the following notations.
\begin{align}\label{eq:def-lmax}
    l_{i,j}^{\max} := \left\lceil \log_2 \frac{16KM}{C} \right\rceil, \quad \forall (i, j) \in [N] \times [M],
\end{align}
and for every integer $l$,
\begin{equation}\label{eq:def-kappa}
\begin{aligned}
    \kappa_{l,T}(C,s) := \begin{cases}
        2\cdot 2^{-l} & s =0 \\
        \sqrt{{96 \cdot 2^{-l} \log(NMT)}/s} & 1 \le s \le B_{l,T}(C) \\
        0 & s > B_{l,T}(C)
    \end{cases},
\end{aligned}
\end{equation}
where $B_{l,T}(C)$ is given by
\begin{align}\label{eq:def-BlT}
    B_{l,T}(C) := \left\lfloor{6144 \cdot 2^{-l}K^2M^2\log(NMT)}/{C^2}\right\rfloor.
\end{align}
By \citet[Lemma 5]{wang2017improving}, if $\texttt{Bonus}(t) \ge C$, under event $\gE_2(t)$, we have
\begin{align*}
    \texttt{Bonus}(t) \le \sum_{i \in S_t, j \in [M]} \kappa_{l_{i,j}, T}(C, N_{t-1}^{l_i}(i,j)),
\end{align*}
where $l_{i,j}$ is the index of group $S_j^{l_{i,j}} \ni S_t$. This is because we have
\begin{align*}
    \texttt{Bonus}(t) &\le -C + 8 \sum_{i \in S_t, j \in [M]} Q_j^*(S_t) \cdot (j - 1)\epsilon \cdot \min\{\beta_{i,j}^t, 1\} \\
    &\le 8 \sum_{i \in S_t, j \in [M]} \left(Q_j^*(S_t) \cdot \min\{\beta_{i,j}^t, 1\} - \frac{C}{8KM}\right)
\end{align*}

\noindent \textbf{Case 1: $1\le l_{i,j} \le l_{i,j}^{\max}$.} We have 
\begin{align*}
    Q^*_{j}(S_t) \le 2 \cdot 2^{-l_{i,j}}.
\end{align*}
Under $\gE_2(t)$, we have
\begin{align*}
    \min \left\{\beta_{i,j}^t, 1\right\} = \min \left\{\sqrt{\frac{8 \log(NMT)}{ SC_{t-1}(i,j)}},1\right\} \leq \min\left\{\sqrt{\frac{8 \log(NMT)}{\frac{1}{3} N_{t-1}^{l_{i,j}}(i,j) \cdot 2^{-l_{i,j}}}}, 1\right\},
\end{align*}
and
\begin{equation}\label{eq:Qbeta-bound}
\begin{aligned}
    Q^*_{j}(S_t)\cdot \min \left\{\beta_{i,j}^t,1\right\} &\le 2 \cdot 2^{-l_{i,j}} \cdot \min\left\{\sqrt{\frac{8 \log(NMT)}{\frac{1}{3} N_{t-1}^{l_{i,j}}(i,j) \cdot 2^{-l_{i,j}}}}, 1\right\} \\
    &\le \min\left\{\sqrt{\frac{96 \cdot 2^{-l_{i,j}} \log(NMT)}{ N_{t-1}^{l_{i,j}}(i,j) }}, 2 \cdot 2^{-l_{i,j}}\right\}.
\end{aligned}
\end{equation}
If $N_{t-1}^{l_{i,j}}(i,j) \ge B_{l_{i,j},T}(C) + 1$, then
\begin{align*}
    \sqrt{\frac{96 \cdot 2^{-l_{i,j}} \log(NMT)}{ N_{t-1}^{l_{i,j}}(i,j) }} \le \frac{C}{8KM},
\end{align*}
which implies $Q^*_{j}(S_t)\cdot \min \left\{\beta_{i,j}^t,1\right\} - C/8KM \le 0$. 

If $N_{t-1}^{l_{i,j}}(i,j) = 0$, we have $Q^*_{j}(S_t)\cdot \min \left\{\beta_{i,j}^t,1\right\} \le Q^*_{j}(S_t) \le 2\cdot 2^{-l_{i,j}}$, which implies
\begin{align*}
    Q^*_{j}(S_t)\cdot \min \left\{\beta_{i,j}^t,1\right\} - \frac{C}{8KM} \le \kappa_{l_{i,j},T}(C, 0)
\end{align*}

Otherwise, for $1 \le N_{t-1}^{l_{i,j}}(i,j) \le B_{l_{i,j}, T}(C)$, we have $Q^*_{j}(S_t)\cdot \min \left\{\beta_{i,j}^t,1\right\} \le \kappa_{l_{i,j},T}(C, N_{t-1}^{l_{i,j}}(i,j))$ by \Cref{eq:Qbeta-bound,eq:def-kappa}. Therefore, we get
\begin{align*}
    Q^*_{j}(S_t)\cdot \min \left\{\beta_{i,j}^t,1\right\} - \frac{C}{8KM} \le \kappa_{l_{i,j},T}(C, N_{t-1}^{l_{i,j}}(i,j))
\end{align*}

\noindent \textbf{Case 2: $l_{i,j} \ge l_{i,j}^{\max} + 1$.} We have
\begin{align*}
    Q^*_{j}(S_t)\cdot \min \left\{\beta_{i,j}^t,1\right\} \le 2 \cdot 2^{-l_{i,j}} \le 2 \cdot \frac{C}{16KM} \le \frac{C}{8KM},
\end{align*}
which shows that $Q^*_{j}(S_t)\cdot \min \left\{\beta_{i,j}^t,1\right\} - C/8KM \le 0$. If $N_{t-1}^{l_{i,j}}(i,j) = 0$. Therefore, we finally get
\begin{align*}
    \texttt{Bonus}(t) \le 8\sum_{i\in S_t, j \in [M]} \kappa_{l_{i,j}, T}\left(C, N_{t-1}^{l_{i,j}}(i,j)\right),
\end{align*}
for the case of good event $\gE_2(t)$ happens and $\texttt{Bonus}(t) \ge C$.

Notice that under good events $\gE_0, \gE_1$, we have
\begin{align*}
    \sum_{t=1}^T \texttt{Bonus}(t) &\le \sum_{t=1}^T \1[\{\texttt{Bonus}(t) \ge C\} \cap \gE_2(t)]\cdot \texttt{Bonus}(t) + T \cdot C + \sum_{t=1}^T \Prob[\gE_2(t)] \\
    &\le \underbrace{\sum_{t=1}^T 8 \cdot \sum_{i\in S_t, j \in [M]} \kappa_{l_{i,j}, T}\left(C, N_{t-1}^{l_{i,j}}(i,j)\right)}_{(I)} + TC + \frac{\pi^2}{6} \cdot \max_{i\in [N], j \in [M]} l_{i,j}^{\max} .
\end{align*}
where the first inequality is due to $\texttt{Bonus}(t) \le 1$ and definition, and the second one is due to \Cref{lemma:TPprob}. The key is bounding $(I)$:
\begin{align*}
    (I) &= 8\cdot \sum_{i \in [N], j \in [M]} \sum_{l=1}^\infty \sum_{s=0}^{N_{T-1}^{l}(i,j)}\kappa_{l}(C, s) \\
    &= 8 \cdot \sum_{i \in [N], j \in [M]} \sum_{l=1}^\infty \left(2\cdot 2^{-l} +  \sum_{s=1}^{B_{l,T}(C)}\sqrt{\frac{96 \cdot 2^{-l} \log(NMT)}{ s }} \right) \\
    &\le 8\cdot  \sum_{i \in [N], j \in [M]} \sum_{l=1}^\infty \left(2\cdot 2^{-l} +  2\cdot \sqrt{96 \cdot 2^{-l} \log(NMT)}\cdot \sqrt{B_{l,T}(C)} \right) ,
\end{align*}
where the inequality holds by the fact that $\sum_{s=1}^n \sqrt{1/s} \le 2\sqrt{n}$. Therefore, by the definition of $B_{l,T}(C)$ in \Cref{eq:def-BlT}, we have
\begin{align*}
    (I) &\le  8\cdot  \sum_{i \in [N], j \in [M]} \sum_{l=1}^\infty \left(2\cdot 2^{-l} +  1536\cdot \frac{2^{-l}KM\log(NMT)}{C} \right) \\
    &= 8\cdot  \sum_{i \in [N], j \in [M]}  \left(2+  1536\cdot \frac{KM\log(NMT)}{C} \right)\cdot \left(\sum_{l=1}^\infty 2^{-l}\right) \\
    &\le 16NM + 12288\frac{KNM^2\log(NMT)}{C}.
\end{align*}
Therefore, we get
\begin{align*}
    \sum_{t=1}^T \texttt{Bonus}(t) \le 16NM + 12288\frac{KNM^2\log(NMT)}{C} + TC + \frac{\pi^2}{6} \left\lceil \log_2\frac{16KM}{C} \right\rceil.
\end{align*}
\end{proof}
\end{lemma}


\subsection{Bounding the Bias Terms}
\begin{lemma}\label{lemma:bias-t-bound}
Under \Cref{ass:bi-lipschitz}, we have
\begin{align*}
    \sum_{t=1}^T \texttt{Bias}(t) \le 4K^2L^4 T\epsilon \log(M + 1).
\end{align*}
\begin{proof}
Notice that $\sum_{j\in[M]} 1/j \le \log(M+1)$ for $\epsilon < 1/2$, we have
\begin{align*}
    \texttt{Bias}(t) &\le 4K \cdot \sum_{i \in S_t,j \in [M]}Q^*_{j}(S_t) \cdot \epsilon L^4/j \\
    &= 4K^2L^4\epsilon \cdot \sum_{j \in [M]} \frac{1}{j} \\
    &\le 4K^2L^4 \epsilon\log(M + 1).
\end{align*}
Therefore, we have
\begin{align*}
    \sum_{t=1}^T \texttt{Gap}(t) \le 4K^2L^4 T\epsilon \log(M+1).
\end{align*}
\end{proof}
\end{lemma}



\subsection{Proof of \Cref{thm:main}}
\begin{theorem}[Formal version of \Cref{thm:main}]\label{thm:main-formal}
By setting $\beta_{i,j}^t$ in \Cref{eq:def-beta} and $\epsilon < 1/2$, we can control the regret of \Cref{alg} under \Cref{ass:bi-lipschitz} by 
\begin{align*}
    \gR(T) &\le  12289 \sqrt{NKM^2T\log(NMT)} +  T\epsilon\left(4KL^4\log(M+1) + 3\right)\\
    &\quad + 16 NM +  \pi^2\left(\log_2(\sqrt{KM^2T\log(NMT)/N}) + 5\right)/6 +  T^{-1} \\
    &= \wt{O}\left(\sqrt{NKM^2T} + L^4K^2T\epsilon\right),
\end{align*}
where $M = \ceil{1/\epsilon}$. If we further take $\epsilon = O\left(L^{-2}K^{-\frac{3}{4}}N^{\frac
{1}{4}}T^{-\frac{1}{4}}\right)$, we have
\begin{align*}
    \gR(T) = \wt{\gO}(L^{2}N^{\frac{1}{4}}K^{\frac{5}{4}}T^{\frac{3}{4}}).
\end{align*}
\begin{proof}
By \Cref{lemma:regret-decomp}, we have
\begin{align*}
    \gR(T) \le \E\left[\sum_{t=1}^T \texttt{Bonus}_t + \texttt{Bias}_t\middle| \gE_0, \gE_1\right] + 3T\epsilon + T^{-1},
\end{align*}
Take constant $C$ as
\begin{align}\label{eq:def-C}
    C:= \sqrt{\frac{NKM^2\log(NMT)}{T}}.
\end{align}
Then \Cref{lemma:bonus-t-bound} shows that
\begin{align*}
    \sum_{t=1}^T \texttt{Bonus}(t) \le 16NM + 12289\sqrt{NM^2KT\log(NMT)} + \pi^2\left(\log_2(\sqrt{KMT\log(NMT)/N}) + 5\right)/6.
\end{align*}
\Cref{lemma:bias-t-bound} demonstrates that
\begin{align*}
    \sum_{t=1}^T \texttt{Bias}(t) \le 4K^2L^4 T\epsilon \log(M + 1).
\end{align*}
Therefore, by calculating the summation of the bonus and bias terms, we can bound the regret by
\begin{align*}
    \gR(T) &\le \E\left[\sum_{t=1}^T \texttt{Bonus}(t) + \texttt{Gap}(t) \middle| \gE_0, \gE_1\right] +  T^{-1} + 3T\epsilon \\
    &\le  12289 \sqrt{NKM^2T\log(NMT)} +  T\epsilon\left(4KL^4\log(M+1) + 3\right)\\
    &\quad + 16 NM +  \pi^2\left(\log_2(\sqrt{KM^2T\log(NMT)/N}) + 5\right)/6 +  T^{-1} \\
    &= \wt{O}\left(\sqrt{NKM^2T} + L^4K^2T\epsilon\right),
\end{align*}
which finishes the proof.
\end{proof}
\end{theorem}


\section{Omitted proofs in \Cref{sec:kminexp}}\label{Appendix:k-min}
This proof mainly applies the techniques for general linear bandits \citet{liu2024almost, lee2024unified}. Given action $S \in \gS$ to the environment, we assume that $\ell_S$ is the random variable of the loss, i.e., $\E[\ell_S] = \ell^*(S)$. 

We have
\begin{align*}
    g_t(\theta; \lambda) :&= - \nabla_\theta \gL_t(\theta; \lambda) + \sum_{i < t} \ell_i \psi(S_i) \\
    &=  \sum_{i < t}  \frac{1}{\psi(S_i)^\top \theta} \cdot \psi(S_i) - \lambda \theta. \\
    H_t(\theta;\lambda) :&= \nabla^2_\theta \gL(\theta; \gH_t)\\
    &=  - \nabla_\theta g_t(\theta; \lambda)  \\
    &= \lambda I + \sum_{i < t} \frac{\psi(S_i)\psi(S_i)^\top}{(\psi(S_i)^\top\theta)^2}.
\end{align*}
\subsection{Concentration Argument for MLE}
\begin{lemma}[MLE Concentration]
\label{lemma:mle-concentration}
For $L^* := \sup_{S \in \gS} \ell^*(S)$, $M_1:= L^*/\sqrt{2}$, and $V = \sup\{\|\theta\|_2 : \theta \in \Theta\}$, set
\begin{align}\label{eq:def-lambda}
    \lambda_t := \max \left\{1, \frac{2dM_1}{V}\cdot \log\left(e\sqrt{1 + \frac{tL^*}{d}} + \frac{1}{\delta}\right)\right\},
\end{align}
and 
\begin{align} \label{eq:def-gamma}
    \gamma_t(\delta, \lambda_t) := \sqrt{\lambda_t}\left(\frac{1}{2M_1} + V\right) + \frac{2M_1d}{\sqrt{\lambda_t}}\left(\log(2) + \frac{1}{2}\log\left(1 + \frac{tL^*}{\lambda_t d}\right)\right) + \frac{2M_1}{\sqrt{\lambda_t}}\log(1/\delta).
\end{align}
Then we have with probability at least $1 - \delta$, 
\begin{align}\label{eq:def-confidence-set}
    \theta^* \in C_t(\hat\theta_t; \delta,\lambda_t) := \left\{ \theta \in \Theta : \left\| g_t(\theta; \lambda_t) - g_t(\hat\theta_t; \lambda_t) \right\|_{H_t^{-1}(\theta; \lambda_t)} \le \gamma_t(\delta, \lambda_t) \right\},
\end{align}
holds for any $t \in [T]$. We denote the confidence set as $C_t(\hat\theta_t; \delta, \lambda_t)$ and this good event as $\Xi$.

\begin{proof}
For simplicity, we denote the filtration of history as  $\gH_t := \left( S_1, Y_1, \cdots, S_{t-1}, Y_{t-1}, S_t \right)$. Then we have
\begin{align*}
    \ell_t \sim \exp(\psi(S_t)^\top \theta^*), \quad \E[\ell_t \mid \gH_t] = \frac{1}{\psi(S_t)^\top\theta^*},
\end{align*}
by the property of exponential distribution and definition of $\psi(S)$. Since we have
\begin{align*}
    \hat\theta_t \leftarrow \argmin_{\theta \in \R^d} \gL_t(\theta; \lambda_t),
\end{align*}
by \Cref{alg:k-min}. Then by KKT condition, we have
\begin{align*}
   \left.\frac{\partial \gL_t(\theta ; \lambda_t)}{\partial \theta} \right|_{\theta = \hat\theta_t} = 0 \Rightarrow g_t(\hat\theta_t; \lambda_t) - \sum_{i < t} \ell_i\psi(S_i) = 0
\end{align*}
Notice that by definition of $g_t$, 
\begin{align*}
    g_t(\theta^*; \lambda_t) =  \sum_{i < t}  \frac{1}{\psi(S_i)^\top \theta^*} \cdot \psi(S_i) - \lambda_t \theta^*.
\end{align*}
Denote $\varepsilon_t := \ell_t - \E[\ell_t \mid \gH_t] = \ell_t - {1}/({\psi_t(S_t)^\top\theta^*})$, we have
\begin{align*}
    g_t(\hat \theta_t; \lambda_t) - g_t(\theta^*; \lambda_t) = \sum_{i < t} \varepsilon_i \psi(S_i) + \lambda_t \theta^*.
\end{align*}
Fix $s \ge 0$, we have
\begin{align*}
    \E[\exp(s\varepsilon_t) \mid \gH_t] &= \E\left[\exp\left(s\ell_t - \frac{s}{\psi_t(S_t)^\top\theta^*}\right)\right] \\
    &= \exp\left(- \frac{s}{\psi_t(S_t)^\top\theta^*}\right)\cdot \E\left[\exp(s\ell_t)\mid \gH_t\right],
\end{align*}
and by calculation,
\begin{align*}
    \E[\exp(s\varepsilon_t) \mid \gH_{t-1}] &= \exp\left(- \frac{s}{\psi(S_t)^\top\theta^*}\right)\cdot \E\left[\exp(s\ell_t)\mid \gH_t\right] \\
    &=\exp\left(- \frac{s}{\psi(S_t)^\top\theta^*}\right)\cdot \int_{([0,+\infty)} \psi(S_t)^\top\theta^*\exp(-(\psi(S_t)^\top\theta^* - s)y)dy \\
    &= \exp\left(-\frac{1}{\ell_t^*}s + \log(\ell_t^*) - \log(\ell_t^* - s)\right),
\end{align*}
where we use $\ell_t^* := \psi_t(S_t)^\top\theta^*$ for simplicity. Consider the case for $s < \ell_t^*$, by intermediate value theorem, we have
\begin{align*}
    \log(\ell_t^*) - \log(\ell_t^* - s) = s \cdot \frac{1}{\ell_t^*} - \frac{s^2}{2\xi^2},
\end{align*}
for some $\xi \in [\ell_t^* - s, \ell_t^*]$. We further denote
\begin{equation}\label{eq:def-Lstar}
    L^* = \sup_{S \in \gS} \ell^*(S).
\end{equation}
Therefore, we can set constant $0 \le s \le L^*$, which gives
\begin{align*}
    \log(\ell_t^*) - \log(\ell_t^* - s) \leq s \cdot \frac{1}{\ell_t^*} - \frac{s^2}{(\ell_t^*)^2}.
\end{align*}
Denote $\nu_{t-1} := -1/{(\psi(S_t)^\top \theta^*)^2}$. We have for some constant $M_1 \ge L^*/\sqrt{2}$, and $|s| \le 1/M_1$,
\begin{align*}
    \E[\exp(s\varepsilon_t) \mid \gH_t] \le \exp(s^2 \nu_{t-1}).
\end{align*}
Applying \citet[Theorem 2]{janz2024exploration} with $\mS_t := \sum_{i<s} \varepsilon_i \psi(S_i)$, we can show that with probability at least $1 - \delta$,
\begin{align*}
    \left\|g_t(\hat\theta_t; \lambda_t) - g_t(\theta^*; \lambda_t)\right\|_{H_t^{-1}(\theta^*;\lambda_t)} &\le \left\|\sum_{i<t}\varepsilon_i \psi_i(S_i)\right\|_{H_t^{-1}(\theta^*;\lambda_t)} + \lambda_t \left\|\theta^*\right\|_{H_t^{-1}(\theta^*;\lambda_t)} \\
    &\le \frac{\sqrt{\lambda_t}}{2M_1} + \frac{2M_1}{\sqrt{\lambda_t}}\log\left(\frac{\det(H_t(\theta^*)^{1/2}/\lambda_t^{d/2})}{\delta}\right) + \frac{2M_1}{\sqrt{\lambda_t}}d\log(2) + \sqrt{\lambda_t} V,
\end{align*}
where $V = \sup\{\|\theta\|_2 : \theta \in \Theta\}$. Moreover, by definition of $H_t(\theta^*; \lambda_t)$, we have
\begin{align*}
    \det(H_t(\theta^*; \lambda_t))/\lambda_t^d \le \left(1 + \frac{tL^*}{\lambda_t d}\right)^d,
\end{align*}

Therefore, for
\begin{align*}
    \gamma_t(\delta, \lambda_t) \ge \sqrt{\lambda_t}\left(\frac{1}{2M_1} + V\right) + \frac{2M_1d}{\sqrt{\lambda_t}}\left(\log(2) + \frac{1}{2}\log\left(1 + \frac{tL^*}{\lambda_t d}\right)\right) + \frac{2M_1}{\sqrt{\lambda_t}}\log(1/\delta),
\end{align*}
we have with probability at least $1 - \delta$, 
\begin{align*}
    \theta^* \in C_t(\hat\theta_t; \delta,\lambda_t) := \left\{ \theta \in \Theta : \left\| g_t(\theta; \lambda_t) - g_t(\hat\theta_t; \lambda_t) \right\|_{H_t^{-1}(\theta; \lambda_t)} \le \gamma_t(\delta, \lambda_t) \right\},
\end{align*}
holds for any $t \in [T]$.
\end{proof}
\end{lemma}


\subsection{Proof of \Cref{thm:kminexp}}
\begin{theorem}[Formal version of \Cref{thm:kminexp}]
\label{thm:kminexp-formal}
By setting $\delta = 1/T$, $\gamma_t(\delta)$ according to \Cref{eq:def-gamma}, and $\lambda_t$ according to \Cref{eq:def-lambda}, \Cref{alg:k-min} enjoys the following regret guarantee: 
\begin{align*}
    \gR(T) &\le 
    16\gamma \cdot \sqrt{dT}\cdot \sqrt{(\ell^*(S^*))^2(1 + L^*/\lambda) \cdot \log\left(1 + L^*T/d\lambda\right)} \\
    &\quad + 256\gamma^2 \cdot dL^* \cdot \log\left(1 + L^*T/d\lambda\right) \cdot \left(\frac{\sup_{S \in \gS} (\psi(S)^\top \theta^*)}{\ell^*(S^*)^3} + 2\right) + 1,
\end{align*}
where $\gamma := \sup_t \gamma_t(\delta)$ and $\lambda_t := \inf_t \lambda_t$.
\begin{proof}
Since we have $X_i \sim \exp(\phi(i)^\top \theta^*)$, then
\begin{align*}
    \min_{i \in S} X_i \sim \exp\left(\sum_{i \in S} \phi(i)^\top \theta^* \right) = \exp\left(\psi(S)^\top \theta^* \right),
\end{align*}
which shows that 
\begin{align*}
    \ell^*(S) = \E\left[\min_{i \in S} X_i\right] = \frac{1}{\psi(S)^\top \theta^*}.
\end{align*}
Therefore, by second-order Taylor expansion, we have for some $\xi \in [\ell^*(S_t), \sup_t \ell^*(S_t)]$, 
\begin{align*}
    \gR(T) &= \E\left[\sum_{t=1}^T \ell^*(S_t) - \ell^*(S^*) \right]\\
    &\le   \Prob[\Xi] \cdot \E\left[\sum_{t=1}^T \frac{1}{\psi(S_t)^\top \theta^*} - \frac{1}{\psi(S^*)^\top \theta^*} \middle| \Xi \right]+ \Prob[\neg\Xi] \cdot T \\
    &\le \E\left[ \underbrace{\sum_{t=1}^T \frac{1}{(\psi(S_t)^\top \theta^*)^2} \cdot \left( \psi(S^*)^\top \theta^* - \psi(S_t)^\top \theta^*  \right)}_{\gR_1(T)} \middle| \Xi \right] +\E\left[ \underbrace{\sum_{t=1}^T \frac{2}{\xi^3} \cdot \left(\psi(S^*)^\top \theta^* - \psi(S_t)^\top \theta^* \right)^2 }_{\gR_2(T)} \middle| \Xi \right] + 1.
\end{align*}
Under $\Xi$, we have $\theta^* \in C_t(\hat\theta_t; \delta, \lambda_t)$ for every $t \in [T]$. Therefore, by \Cref{alg:k-min}, we have
\begin{align}
\label{eq:optimism}
    \psi(S^*)^\top \theta^* \le \psi(S_t)^\top \wt{\theta}_t.
\end{align}
Under $\Xi$, we have
\begin{align*}
    \gR_1(T) &\le \sum_{t=1}^T \frac{1}{(\psi(S_t)^\top \theta^*)^2} \cdot \psi(S_t)^\top (\theta^* - \wt{\theta}_t) \\
    &\le \sum_{t=1}^T \frac{1}{(\psi(S_t)^\top \theta^*)^2} \cdot \| \psi(S_t) \|_{H_t^{-1}(\theta^*; \lambda_t)} \cdot \left\|\theta^* - \wt{\theta}_t\right\|_{H_t^{-1}(\theta^*; \lambda_t)},
\end{align*}
where the first inequality is due to \Cref{eq:optimism} and the second holds by Cauchy-Schwartz inequality. Notice that $\wt{\theta}_t, \theta^* \in C_t(\hat\theta_t; \delta, \lambda_t)$ under $\Xi$, we have
\begin{align*}
    \left\|\theta^* - \wt{\theta}_t\right\|_{H_t^{-1}(\theta^*; \lambda_t)} \le 8\gamma_t(\delta, \lambda_t)
\end{align*}
by \citet[Lemma 30]{liu2024almost}. Denote $\gamma := \sup_{t \in [T]} \gamma_t(\delta, \lambda_t)$, we can upper bound $\gR_1(T)$ by
\begin{align*}
    \gR_1(T) \le 8 \cdot \sum_{t=1}^T \frac{1}{(\psi(S_t)^\top \theta^*)^2} \cdot \| \psi(S_t) \|_{H_t^{-1}(\theta^*; \lambda_t)} \cdot \gamma.
\end{align*}
Denote $A_t := \psi(S_t)/\psi(S_t)^\top \theta^*$, we have $H_t(\theta^*; \lambda) = \sum_{i < t} A_t^\top A_t + \lambda_t I$ and $\|A_t\|_2 \le \sum_{i \in S_t} \|\phi(i)\|_2 \cdot \ell^*(S_t) \le KL^*$. Then we have
\begin{align*}
    \gR_1(T) &\le 8\gamma \sqrt{\sum_{t=1}^T \|A_t\|^2_{H_t^{-1}(\theta^*; \lambda_t)}} \cdot \sqrt{\sum_{t=1}^T \frac{1}{(\psi(S_t)^\top \theta^*)^2}} \\
    &\le 16\gamma \cdot \sqrt{d(1 + KL^*/\lambda) \cdot \log\left(1 + KL^*T/d\lambda\right)} \cdot \sqrt{\sum_{t=1}^T \frac{1}{(\psi(S_t)^\top \theta^*)^2}},
\end{align*}
where the first inequality is due to the Cauchy-Schwartz inequality, and the second inequality is due to the elliptical potential lemma of \citet{abbasi2011improved}. Moreover, by \citet[Lemma 31]{liu2024almost}, we have
\begin{align*}
    \sqrt{\sum_{t=1}^T \frac{1}{(\psi(S_t)^\top \theta^*)^2}} &\le \sqrt{T\cdot \frac{1}{(\psi(S^*)\psi^*)^2} + 2 \cdot \gR(T)} \\
    &\le \sqrt{T\cdot (\ell^*(S^*))^2 } + \sqrt{2 \cdot \gR(T)},
\end{align*}
which shows that for $\lambda := \inf_t \lambda_t$, 
\begin{align*}
    \gR_1(T) &\le 16\gamma \cdot \sqrt{d(1 + L^*/\lambda) \cdot \log\left(1 + L^*T/d\lambda\right)} \cdot \sqrt{T\cdot (\ell^*(S^*))^2 } \\
    &\quad + 16\gamma \cdot \sqrt{d(1 + L^*/\lambda) \cdot \log\left(1 + L^*T/d\lambda\right)} \cdot \sqrt{2 \cdot \gR(T)}.
\end{align*}
Next we give the upper bound for $\gR_2(T)$. Recall that 
\begin{align*}
    \gR_2(T) = \sum_{t=1}^T \frac{2}{\xi^3} \cdot \left(\psi(S_t)^\top \theta^* - \psi(S^*)\theta^*\right)^2.
\end{align*}
Then, under $\Xi$, we have
\begin{align*}
    \gR_2(T) &\le \sum_{t=1}^T \frac{2}{\xi^3} \cdot \left\langle\psi(S_t) ,\theta^* - \wt{\theta}_t\right\rangle^2 \\
    &\le  \frac{2}{\ell^*(S^*)^3} \cdot \sum_{t=1}^T \|\psi(S_t)\|^2_{H_t^{-1}(\theta^*; \lambda_t)} \cdot \|\theta^* - \wt{\theta}_t\|^2_{H_t^{-1}(\theta^*; \lambda_t)} \\
    &\le \frac{2}{\ell^*(S^*)^3} \cdot 64\gamma^2 \cdot \sum_{t=1}^T \|\psi(S_t)\|^2_{H_t^{-1}(\theta^*; \lambda_t)},
\end{align*}
where the first inequality is according to \Cref{eq:optimism}, the second inequality is due to the Cauchy-Schwartz inequality, and the last inequality holds by \Cref{lemma:mle-concentration}. Denote
\begin{align*}
    \Lambda_t := \lambda_t I + \sum_{i < t} \psi(S_i)^\top \psi(S_i). 
\end{align*}
Then we have
\begin{align*}
    \sup_{S \in \gS} (\psi(S)^\top \theta^*) \cdot \Lambda_t^{-1} \succ H_t^{-1}(\theta^*; \lambda_t),
\end{align*}
which further implies
\begin{align*}
    \gR_2(T) &\le \frac{2}{\ell^*(S^*)^3} \cdot 64\gamma^2 \cdot \sup_{S \in \gS} (\psi(S)^\top \theta^*) \cdot \sum_{t=1}^T \|\psi(S_t)\|^2_{\Lambda_t^{-1}} \\
    &\le \frac{2}{\ell^*(S^*)^3} \cdot 64\gamma^2 \cdot \sup_{S \in \gS} (\psi(S)^\top \theta^*) \cdot 2dL^*\log(1 + L^* T/d\lambda) \\
    &= \frac{256}{\ell^*(S^*)^3} \sup_{S \in \gS} (\psi(S)^\top \theta^*) \cdot \gamma^2 \cdot dL^*\log(1 + L^*T/d\lambda).
\end{align*}
Therefore, we have
\begin{align*}
    \gR(T) &\le 16\gamma \cdot \sqrt{d(1 + L^*/\lambda) \cdot \log\left(1 + L^*T/d\lambda\right)} \cdot \sqrt{T\cdot (\ell^*(S^*))^2 } \\
    &\quad + 16\gamma \cdot \sqrt{d(1 + L^*/\lambda) \cdot \log\left(1 + L^*T/d\lambda\right)} \cdot \sqrt{2 \cdot \gR(T)} \\
    &\quad + \frac{256}{\ell^*(S^*)^3} \sup_{S \in \gS} (\psi(S)^\top \theta^*) \cdot \gamma^2 \cdot dL^*\log(1 + L^*T/d\lambda) + 1.
\end{align*}
Notice that for $x \le A\sqrt{x} + B$, we have $x \le 2A^2 + B$. Therefore, we have
\begin{align*}
    \gR(T) &\le 
    16\gamma \cdot \sqrt{dT}\cdot \sqrt{(\ell^*(S^*))^2(1 + L^*/\lambda) \cdot \log\left(1 + L^*T/d\lambda\right)} \\
    &\quad + 256\gamma^2 \cdot dL^* \cdot \log\left(1 + L^*T/d\lambda\right) \cdot \left(\frac{\sup_{S \in \gS} (\psi(S)^\top \theta^*)}{\ell^*(S^*)^3} + 2\right) + 1, \\
    &\le \wt{\gO}\left(\sqrt{d^3 T}\right)
\end{align*}



\end{proof}
\end{theorem}





\end{document}


% This document was modified from the file originally made available by
% Pat Langley and Andrea Danyluk for ICML-2K. This version was created
% by Iain Murray in 2018, and modified by Alexandre Bouchard in
% 2019 and 2021 and by Csaba Szepesvari, Gang Niu and Sivan Sabato in 2022.
% Modified again in 2023 and 2024 by Sivan Sabato and Jonathan Scarlett.
% Previous contributors include Dan Roy, Lise Getoor and Tobias
% Scheffer, which was slightly modified from the 2010 version by
% Thorsten Joachims & Johannes Fuernkranz, slightly modified from the
% 2009 version by Kiri Wagstaff and Sam Roweis's 2008 version, which is
% slightly modified from Prasad Tadepalli's 2007 version which is a
% lightly changed version of the previous year's version by Andrew
% Moore, which was in turn edited from those of Kristian Kersting and
% Codrina Lauth. Alex Smola contributed to the algorithmic style files.
