%%%%%%%% ICML 2025 EXAMPLE LATEX SUBMISSION FILE %%%%%%%%%%%%%%%%%

\documentclass{article}

\usepackage{geometry}
\geometry{a4paper, scale=0.7}

\usepackage[utf8]{inputenc} % allow utf-8 input
\usepackage[T1]{fontenc}    % use 8-bit T1 fonts

\usepackage{titletoc}
\usepackage[toc, page, header]{appendix} %%% MAKE SURE TO PUT THIS BEFORE hyperref PACKAGE

\usepackage{microtype}
\usepackage{graphicx}
\usepackage{subfigure}
\usepackage{booktabs} 

\usepackage[colorlinks=true, linkcolor=blue, citecolor=blue,urlcolor=black]{hyperref}


\newcommand{\theHalgorithm}{\arabic{algorithm}}


\usepackage{amsmath}
\usepackage{amssymb}
\usepackage{mathtools}
\usepackage{amsthm}
\usepackage{algorithm}
\usepackage{algorithmic}
\usepackage{natbib}

\usepackage{bbding}

\renewcommand{\algorithmiccomment}[1]{ \hfill $\triangleright$ { #1}}
\renewcommand{\algorithmicrequire}{\textbf{Input:}}
\renewcommand{\algorithmicensure}{\textbf{Output:}}
\usepackage[capitalize,noabbrev]{cleveref}

%%%%% NEW MATH DEFINITIONS %%%%%

% \usepackage{amsmath,amsfonts,bm}
\usepackage{amsmath,amsfonts}

\usepackage{pifont}


\newcommand{\R}{\mathbb{R}}


\def\va{{\mathbf{a}}}
\def\vg{{\mathbf{g}}}

% Sets
\def\sR{\mathbb{R}}
\def\sC{\mathbb{C}}
\def\sZ{\mathbb{Z}}
\def\sN{\mathbb{N}}
\def\sQ{\mathbb{Q}}

\def\sS{\mathcal{S}}



% Vectors
\def\vzero{{\mathbf{0}}}
\def\vone{{\mathbf{1}}}
\def\vmu{{\mathbf{\mu}}}
\def\vtheta{{\mathbf{\theta}}}
\def\va{{\mathbf{a}}}
\def\vb{{\mathbf{b}}}
\def\vc{{\mathbf{c}}}
\def\vd{{\mathbf{d}}}
\def\ve{{\mathbf{e}}}
\def\vf{{\mathbf{f}}}
\def\vg{{\mathbf{g}}}
\def\vh{{\mathbf{h}}}
\def\vi{{\mathbf{i}}}
\def\vj{{\mathbf{j}}}
\def\vk{{\mathbf{k}}}
\def\vl{{\mathbf{l}}}
\def\vm{{\mathbf{m}}}
\def\vn{{\mathbf{n}}}
\def\vo{{\mathbf{o}}}
\def\vp{{\mathbf{p}}}
\def\vq{{\mathbf{q}}}
\def\vr{{\mathbf{r}}}
\def\vs{{\mathbf{s}}}
\def\vt{{\mathbf{t}}}
\def\vu{{\mathbf{u}}}
\def\vv{{\mathbf{v}}}
\def\vw{{\mathbf{w}}}
\def\vx{{\mathbf{x}}}
\def\vy{{\mathbf{y}}}
\def\vz{{\mathbf{z}}}
\def\vzeta{{\mathbf{\zeta}}}

% Matrix
\def\mA{{\mathbf{A}}}
\def\mB{{\mathbf{B}}}
\def\mC{{\mathbf{C}}}
\def\mD{{\mathbf{D}}}
\def\mE{{\mathbf{E}}}
\def\mF{{\mathbf{F}}}
\def\mG{{\mathbf{G}}}
\def\mH{{\mathbf{H}}}
\def\mI{{\mathbf{I}}}
\def\mJ{{\mathbf{J}}}
\def\mK{{\mathbf{K}}}
\def\mL{{\mathbf{L}}}
\def\mM{{\mathbf{M}}}
\def\mN{{\mathbf{N}}}
\def\mO{{\mathbf{O}}}
\def\mP{{\mathbf{P}}}
\def\mQ{{\mathbf{Q}}}
\def\mR{{\mathbf{R}}}
\def\mS{{\mathbf{S}}}
\def\mT{{\mathbf{T}}}
\def\mU{{\mathbf{U}}}
\def\mV{{\mathbf{V}}}
\def\mW{{\mathbf{W}}}
\def\mX{{\mathbf{X}}}
\def\mY{{\mathbf{Y}}}
\def\mZ{{\mathbf{Z}}}
\def\mBeta{{\mathbf{\beta}}}
\def\mPhi{{\mathbf{\Phi}}}
\def\mLambda{{\mathbf{\Lambda}}}
\def\mSigma{{\mathbf{\Sigma}}}


% Expectation
% \def\eE{\mathop{\mathbb{E}}\limits}
\def\eE{\mathbb{E}}

% Probability
\def\pP{\mathbb{P}}

% Tilde
\def\tf{\tilde{f}}
\def\tS{\tilde{S}}
\def\wtF{\widetilde{\mathcal{F}}}
\def\whR{\widehat{R}}
\def\tvx{\tilde{\mathbf{x}}}
\def\ty{\tilde{y}}


\def\defeq{\overset{\textup{def}}{=}}
% \def\defeq{\overset{.}{=}}
\def\defone{\overset{\text{\ding{172}}}{=}}
\def\deftwo{\overset{\text{\ding{173}}}{=}}
\def\leqone{\overset{\text{\ding{172}}}{\leq}}
\def\leqtwo{\overset{\text{\ding{173}}}{\leq}}
\def\leqthree{\overset{\text{\ding{174}}}{\leq}}
\def\leqfour{\overset{\text{\ding{175}}}{\leq}}
\def\eqone{\overset{\text{\ding{172}}}{=}}
\def\eqtwo{\overset{\text{\ding{173}}}{=}}
\def\eqthree{\overset{\text{\ding{174}}}{=}}
\def\eqfour{\overset{\text{\ding{175}}}{=}}
\def\geqfive{\overset{\text{\ding{176}}}{\geq}}

%%%%%%%%%%%%%%%%%%%%%%%%%%%%%%%%
% THEOREMS
%%%%%%%%%%%%%%%%%%%%%%%%%%%%%%%%
\theoremstyle{plain}
\newtheorem{theorem}{Theorem}[section]
\newtheorem{proposition}[theorem]{Proposition}
\newtheorem{lemma}[theorem]{Lemma}
\newtheorem{corollary}[theorem]{Corollary}
\theoremstyle{definition}
\newtheorem{definition}[theorem]{Definition}
\newtheorem{assumption}[theorem]{Assumption}
\theoremstyle{remark}
\newtheorem{remark}[theorem]{Remark}

\newcommand{\compilehidecomments}{false}%HIDE comments
\ifthenelse{ \equal{\compilehidecomments}{true} }{%
	\newcommand{\wei}[1]{}
	\newcommand{\yu}[1]{}
	\newcommand{\siwei}[1]{}
    \newcommand{\longbo}[1]{}
}{
	\newcommand{\wei}[1]{{\color{blue}{[Wei: #1]}}}
	\newcommand{\yu}[1]{{\color{cyan}[\text{Yu:} #1]}}
	\newcommand{\siwei}[1]{{\color{red}[\text{Siwei:} #1]}}
    \newcommand{\longbo}[1]{{\color{orange}[\text{Longbo:} #1]}}
}

\title{Continuous K-Max Bandits}


% \usepackage{authblk}
%\author{Yu Chen \thanks{IIIS, Tsinghua University. Email: \texttt{chenyu23@mails.tsinghua.edu.cn}.}
%\and
%Longbo Huang \thanks{IIIS, Tsinghua University. Email: \texttt{longbohuang@tsinghua.edu.cn}.}
%\and
%Siwei Wang  \thanks{Microsoft Research Asia. Email: \texttt{siweiwang@microsoft.com}.}
%\and
%Wei Chen\thanks{Microsoft Research Asia. Email: \texttt{weic@microsoft.com}.}
%}



\author{Yu Chen$^{1}$\thanks{\ denotes equal contributions. Corresponding author: Wei Chen (\texttt{weic@microsoft.com})}
\hspace{0.1cm} 
Siwei Wang$^{2*}$\hspace{0.1cm} 
Longbo Huang$^{1}$\hspace{0.1cm}  
Wei Chen$^{2}$\Envelope
\\
   \normalfont $^1$IIIS, Tsinghua University \\
   % (\texttt{\{chenyu23@mails.tsinghua.edu.cn,}
   % \\\qquad\qquad\qquad\qquad\qquad\qquad
   % \texttt{longbohuang@tsinghua.edu.cn\}})\\
  $^2$Microsoft Research Asia 
  % (\texttt{\{siweiwang,weic\}@microsoft.com}) %\\
  %$^3$IIIS, Tsinghua University
  %(\texttt{longbohuang@tsinghua.edu.cn}) \\
  %$^4$Microsoft Research Asia
  %(\texttt{weic@microsoft.com})
  \\
  \texttt{chenyu23@mails.tsinghua.edu.cn} \\
  \texttt{siweiwang@microsoft.com} \\
  \texttt{longbohuang@tsinghua.edu.cn}\\
  \texttt{weic@microsoft.com}
}
\date{}

\begin{document}

\maketitle





\begin{abstract}
We study the $K$-Max combinatorial multi-armed bandits problem with continuous outcome distributions and weak value-index feedback: each base arm has an unknown continuous outcome distribution, and in each round the learning agent selects $K$ arms, obtains the maximum value sampled from these $K$ arms as reward and observes this reward together with the corresponding arm index as feedback. This setting captures critical applications in recommendation systems, distributed computing, server scheduling, etc. The continuous $K$-Max bandits introduce unique challenges, including discretization error from continuous-to-discrete conversion, non-deterministic tie-breaking under limited feedback, and biased estimation due to partial observability. Our key contribution is the computationally efficient algorithm \texttt{DCK-UCB}, which combines adaptive discretization with bias-corrected confidence bounds to tackle these challenges. For general continuous distributions, we prove that \texttt{DCK-UCB} achieves a $\widetilde{\mathcal{O}}(T^{3/4})$ regret upper bound, establishing the first sublinear regret guarantee for this setting. Furthermore, we identify an important special case with exponential distributions under full-bandit feedback. In this case, our proposed algorithm \texttt{MLE-Exp} enables $\widetilde{\mathcal{O}}(\sqrt{T})$ regret upper bound through maximal log-likelihood estimation, achieving near-minimax optimality. 

\end{abstract}

\documentclass[../main.tex]{subfiles}
\graphicspath{{../images/}}
\makeatletter
\def\input@path{{../images/}}
\makeatother
\begin{document}
\section{Introduction}
\begin{figure}
\centering
\begin{tikzpicture}
\node[inner sep=0pt] (ws) at (0, 0) {
\includegraphics[height=.4\textwidth, trim={10cm 0 10cm 0},clip]{world_space.png}};
\node[inner sep=0pt] (cs) at (6,0) {\includegraphics[height=.4\textwidth, trim={10cm 1cm 10cm 4cm},clip]{conf_space.png}};
\end{tikzpicture}
\vspace{-5pt}
\label{fig:pbrm_intro}
\caption{\textbf{Left}: Shows world space obstacles as grey spheres. Robots start and goal configuration is colored red and green, respectively. Configurations along the computed path are colored transparent blue. \textbf{Right:} Mapped world space scenario to configuration space. Obstacle region is the grey mesh. Red spheres are collision-free regions computed by the neural SCDF. The optimized shortest path in the convex corridor is the blue curve.}
\vspace{-25pt}
\end{figure}
Motion planning is the problem of finding a collision-free trajectory that connects a given start and goal configuration. The planning takes place in the configuration space of the robot. For single body robots, like mobile robots or drones, the configuration space and the world space are usually the same. This simplifies the planning, since explicit obstacle representations are available which enables geometrical tools like separating hyperplanes, smallest distance to obstacles etc., to be used when designing motion planning algorithms. For multi-body robots like manipulators, the situation is completely different. The world space obstacles are usually mapped to non-convex regions, and to make the problem even harder, the mapping is usually not known. Forming explicit representations of the obstacle region in the configuration space is usually too expensive or intractable. Despite all of this, sampling based planners are used with great success, which mainly is due to their use of implicit representations of the obstacle region. The basic idea is to construct a graph in the configuration space that covers and connects the collision-free region. From this graph, a path can be extracted that connects a given start and goal configuration. The approach is computationally expensive, since the graph is constructed with the smallest geometrical building block available, points, which represents a collision-check. Furthermore, the extracted paths from the graph are non-smooth and jagged due to the stochastic nature of the approach. This adds an additional post-processing step to the process, where the paths are shortcutted and smoothened, before the path can be used for tracking. Clearly a lot of time is invested to form this graph and produce smooth paths. Thus, if the obstacles start to move, then all of this work is done in no use, since all points that make up this graph need to be re-verified, which is simply too time consuming to be done in real time.
\\\\
In this work, we want to address the existing drawbacks of the sampling based planners. Our main contribution is an improved motion planner where each vertex in the graph covers a collision-free region in the form of a sphere instead of a point and where the edges are formed with neighboring intersecting spheres. This representation has the advantage of instead of returning piecewise linear paths, returning a sequence of overlapping spheres, i.e. a convex corridor, that connects a given start and goal configuration, illustrated in Figure \ref{fig:pbrm_intro}. This convex corridor allows us to use convex optimization to produce smooth trajectories, instead of computationally expensive post-processing methods. The representation further allows us to estimate the coverage of the collision-free space, which gives us awareness and feedback in the offline roadmap construction phase. Finally, our representation is simple to adapt to moving obstacles, simply requery for the new radii and recheck for intersections. 
\\\\
The spherical collision-free regions are formed using a signed distance function (SDF), which is a function that returns the smallest distance from an arbitrary point to the boundary of an obstacle. As the name implies, the distance is signed, thus if the point is inside the obstacle it is negative otherwise positive. If the distance is positive, a sphere with radius equal to the distance is guaranteed to cover a collision-free region. Using an SDF in motion planning is not new, but what is novel about our approach is that we express the distance in the configuration space instead of the world space and by doing so allows us to form these convex collision-free regions. We refer to the resulting SDF as a signed configuration distance function (SCDF). Computing an SCDF analytically is non-trivial, our approach is therefore to parameterize the SCDF with a deep neural network and learn the mapping by supervised learning. Our resulting neural SCDF can compute distances for different parameter values of obstacle shapes and we also show how multiple distances can be combined, thus making our approach flexible.
\section{Related work}
Motion planning algorithms can roughly be divided into three families, grid-based, sampling based and optimization based methods. Grid-based methods (GBM) discretize the planning space from which a graph is then compiled. A standard search method is A$^\star$ \citep{a_star}, which is classified as an \textit{informed} search method, since it employs a heuristic function to speed up the search. A$^\star$ guarantees to return an optimal path at the level of discretization used. GBMs usually discretize the planning space by a regular lattice and this limits the GBMs to problems with low dimensionality due to the curse of dimensionality. Thus, GBMs are usually limited to single-body robots where the degrees of freedom (DOF) are low. To overcome the inherent scaling problem with the GBMs, stochastic methods are usually used for multi-body robots. These methods are termed as sampling-based methods (SBM) and core members within this family are the rapidly-exploring random trees (RRT) \citep{rrt} and the probabilistic roadmap (PRM) \citep{prm}. RRT grows a tree from the start configuration and explores the collision-free region in a rapid way until it is able to connect to the goal region. RRT is usually improved by bi-directional planning \citep{rrt_connect}, i.e. an additional tree is grown from the goal configuration and the trees are tested for connection after any tree has been expanded. RRT is a single-query method, thus it searches for a path from scratch each time it is queried. Contrary to this, PRM is a multi-query method, which solves for multiple queries without starting from scratch. PRM does this by creating a roadmap (graph) that covers the collision-free space as an offline step. The graph is then used to solve for multiple queries. PRMs are used in cases where the environment does not change since the extra offline step is too computationally costly and needs to be re-done if the environment is changed. In our work, we address this inherent issue by using a different roadmap representation. Our vertices in the graph cover a collision-free region in the form of spheres and we form the edges by checking for intersecting spheres. If something in the environment changes, we recompute the spheres radii and recheck the intersections, without relying on collision detection. We use a trained neural network to compute the sphere radius, therefore querying for the radius can be done fast, hence our representation enables the PRM for dynamic environments.
\\\\
In the recent decades, optimization based methods (OBM) \citep{chomp, schulman, itomp, stomp} have been introduced as an alternative to SBM for multi-body robots. Like the SBM, the OBMs scale well to higher dimensional problems and produce smoother motion. It is common to use a SDF in the optimization since it is a smooth function, thus enabling gradient-based methods. However, the standard way of expressing the SDF is in world space. The distance therefore needs to be mapped to the configuration space by the forward kinematics. This mapping makes the optimization problem a non-linear program (NLP), which is computationally expensive to solve. Recently, a different approach has been proposed. In \cite{mp_gcs} motion planning is formulated as a convex optimization problem by using the graph of convex sets framework \citep{gcs}. The underlying idea is to decompose the collision-free space into intersecting convex sets from which a convex optimization problem is formulated. In cases where an explicit representation of the obstacles in the configuration space exists, like for single-body robots, creating collision-free convex regions can be done fast \citep{iris}. For multi-body robots, this is non-trivial. Existing work does this successfully \citep{iris_nlp, iris_c} by an optimization based approach, but the methods are still too time consuming to be used in the presence of moving obstacles. Our approach is instead to use deep learning to learn an SDF expressed in the configuration space. With this, we can query for shortest distances to the collision boundary, which allows us to expand spherical regions which are collision-free. Our approach is fast and therefore enables our suggested roadmap planner to be used in dynamic environments.
\\\\
Recent research has focused on learning collision detection \citep{fk_kernel_distance, diffco, graphdistnet} by predicting the signed distance between the robot links and the surrounding obstacles in the world space. The learned SDF is used in trajectory optimization but since the distance is expressed in the world space, the problem becomes an NLP and therefore takes a long time to solve. We take a novel approach and suggest to instead express the signed distance in the configuration space. This allows us to improve the PRM at the same time as it enables convex optimization for trajectory optimization, which runs faster and is more reliable than NLP solvers. In \cite{cspf} a learned signed distance function in the configuration space is proposed similar to our approach. However, their approach is restricted to point cloud representations, while we propose to represent the obstacles as parameterized geometric shapes, e.g. spheres. Furthermore, we also show how to use our learned SCDF to improve an existing roadmap planner.
\section{Problem formulation}
A robot is located in the world space, $\W \subset \R^3 $. The unique location of the robot is given by its configuration $\q \in \C$, where $\C$ is the configuration space. The set of points covered by the robots bodies at a certain configuration is expressed as $\B(\q) \subset \W$. The robot is surrounded by $\NrObst$ obstacles $\O = \bigcup_{i=1}^{\NrObst} \O_i$, where  $\O_i \subset \W$. The representation of the obstacle in the configuration space is the set $\C\O_i = \{\q \in \C \: |\: \B(\q) \cap \O_i \neq \emptyset \}$. The obstacle space is formed as $\Co = \bigcup_{i=1}^{\NrObst} \C \O_i$. The complement is referred to as the free space, $\Cf = \C \setminus \Co$. The path planning problem is a tuple, ($\Cf$, $\qStart$, $\qGoal$), where we want to connect a query pair, consisting of a start, $\qStart$, and goal configuration, $\qGoal$, with a geometric path, $\q(s): [0, 1] \mapsto \Cf$, such that $\q(0)=\qStart$ and $\q(1)=\qGoal$, or report correctly when such a path does not exist.
\end{document}

\section{Adaptive labeling as a Markov decision process} 
\label{sec:formulation}

We illustrate our formulation for model evaluation, and extend it to the ATE estimation setting at the end of the section. 
Our goal is to evaluate the performance of a prediction model $\model: \statdomain \to \mathcal \labeldomain$ over the input distribution $P_X$ that we expect to see during deployment.  Given inputs $X  \in \mc{X}$,   labels/outcomes are generated
 from some unknown function $f\opt$: $
      Y = f\opt(X) + \varepsilon$, where $\varepsilon$ is the noise.
      %~~~\mbox{where}~~\varepsilon \sim N(0, \sigma^2)
  % \begin{equation*}
  %     Y = f\opt(X) + \varepsilon~~~\mbox{where}~~\varepsilon \sim N(0, \sigma^2).
  % \end{equation*}
When ground truth outcomes are costly to obtain, previously collected labeled data $\mc{D}^0 := \{(X_i,Y_i)\}_{i \in \mc{I}}$ 
typically suffers selection bias and covers only a subset of the support of input distribution $P_X$ over which we aim to evaluate the model performance. 

Assuming we have a   pool of data $\xpool$, we design
 adaptive sampling algorithms that iteratively select
inputs in $\xpool$ to be labeled.
Since labeling inputs takes time in practice, we model
real-world instances by considering \emph{batched} settings. Our goal is to sequentially label batches of data to accurately estimate model performance over $P_X$ and therefore we assume we have access to a set of inputs $\xeval \sim P_X$. %We assume the modeler pays a fixed and equal cost for each label/outcome. 
%Our framework is general as we do not assume \xpool∼PX\xpool \sim P_X.
We use the squared loss to illustrate our framework,
where our goal is to evaluate $\E_{X \sim P_X}[ (Y - \model(X))^2]$. Under the ``likelihood" function $p(y | f, x) = p_{\varepsilon}(y - f(x))$,  let $g(f)$ be the performance of the AI model $\model(\cdot)$ under the  data generating function $f$, which we refer to as our estimand of interest.
When we consider the mean squared loss,  $g(f)$ is given by 
\begin{align}
    g(f) \defeq \E_{X \sim P_X}\left[ \E_{Y \sim p(\cdot|f,X) } \Big[ (Y - \model(X))^2 \Big] \mid f \right]. \label{eqn:l2-g-f}
\end{align}
Our framework is general and can be extended to other settings. For example, a clinically useful  metric is \texttt{Recall}, defined as the fraction of individuals that the model $\model(\cdot)$ correctly labels as positive  among all the individuals who actually have the positive label 
\begin{align*}
    g(f) \defeq  \E_{X \sim P_X}\left[ \E_{Y \sim p(\cdot|f,X) } \Big[\indic{\model(X)>0}|Y=1\Big] \mid f\right].
\end{align*}
 
  Since the true function $f\opt$ is unknown, we  model it from a Bayesian perspective by formulating a posterior given the available supervised data. We refer to uncertainty over the data generating function $f$ as \emph{epistemic} uncertainty---since we can resolve it with more data---and that over
 the measurement noise $\varepsilon$ as \emph{aleatoric} uncertainty. 
Assuming independence given features $X$, we model the  likelihood of the data via the product 
$p({Y}_{1:m}|f, {X}_{1:m}) = \prod_{i=1}^m p(Y_i|f,X_i)$.
 Our prior belief  $\mu$ over functions $f$   reflects our uncertainty about how
labels are generated given features. 
To adaptively label inputs from $\mc{X}_{\rm pool}$, we assume access to an uncertainty quantification (UQ) method that provides posterior beliefs $\mu(f \mid \mc{D})$ given
any supervised data $\mc{D}:= \{(X_i,Y_i)\}_{i \in \mc{I}}$. As we detail  in Section~\ref{sec:uq}, our framework can leverage both classical 
Bayesian models like Gaussian processes and recent advancements in deep learning-based UQ  methods.

As new batches are labeled, we update our posterior beliefs about $f$ over time, which we view as ``state transitions'' of a dynamical system.
Recalling the Markov decision process depicted in Figure~\ref{fig:overview}, we sequentially label a batch of inputs from $\mc{X}_{\rm pool}$ (actions), which lead to state transitions (posterior updates).
Specifically, our initial state is given by $\mu_0(\cdot) = \mu(\cdot \mid \mc{D}^0)$, where $\mc{D}^0$ represents the initial labeled dataset.
At each period $t$, we label a batch of $K_t$ inputs $\mc{X}^{t+1} \subset \mc{X}_{\rm pool}$ resulting in labeled data $ \mc{D}^{t+1} = (\mc{X}^{t+1}, \datay^{t+1})$. After acquiring the labels at each step $t$, we update the posterior state to $\mu_{t+1}(\cdot) = \mu_t(\cdot \mid \mc{D}^{t+1})$. Modeling practical instances, we consider a small horizon problem with limited adaptivity $T$. Formulating an MDP over posterior states has long conceptual roots, dating back to the Gittin's index for multi-armed bandits~\citep{Gittins79}.

 We denote by $\pi_t$ the adaptive labeling policy at period $t$. We account for randomized policies $\datax^{t+1} \sim \pi_t(\mu_t)$ with a flexible batch size $|\datax^{t+1}| = \batchsize_t$.   
We assume $\pi_t$ is $\mc{F}_t-$measurable for all $t < T$, where $\mc{F}_t$ is the filtration generated by the observations up to the end of step $t$.
 Observe that $\mu_{t+1}$ contains randomness in the policy $\pi_t$ as well as randomness in $\datay^{t+1} \mid (\datax^{t+1},\mu_t)$. Letting $\pi = \set{\pi_0,....,\pi_{T-1}}$,  we minimize the uncertainty over $g(f)$
 at the end of data collection
\begin{align}
H(\pi) \defeq \E_{\mc{D}^{1:T} \sim \pi} \left[G(\mu_{T}) \right]  \defeq  \E_{\mc{D}^{1:T} \sim \pi} \left[G(\mu(\cdot \mid \mc{D}^{0:T})) \right]
%\defeq \E_{\mc{D}^{1:T} \sim \pi} \left[ \V_{f \sim \mu_{T}}  g(f)  \right]
     = \E_{\mc{D}^{1:T} \sim \pi} \left[ \V_{f \sim \mu(\cdot \mid \mc{D}^{0:T})}  g(f)  \right],
     \label{eqn:general-obj}
\end{align}   
where $G(\mu_T) = \V_{f \sim \mu_T}  g(f)$.
In the above objective~\eqref{eqn:general-obj}, we assume
that the modeler pays a fixed and equal cost for each outcome. 
Our framework can also seamlessly accommodate variable labeling cost. Specifically, we can define a cost function $c(\cdot)$ 
 applied on the selected subsets 
 and update the objective~\eqref{eqn:general-obj} accordingly to include the term  $\lambda c(\mc{D}^{1:T})$,
 where $\lambda$
 is the penalty factor that controls the trade-off between minimizing variance and cost of acquiring samples.


Our framework can be easily extended to causal estimation problems.  Consider a feature vector ${X}$ and suppose we have two treatment arms $Z \in \{0,1\}$. Our objective is to evaluate the average treatment effect over the population distribution $P_X$.  Given feature vector $X$, and treatment $Z$,  outcomes are generated from an unknown function $f\opt$: 
$Y = f\opt(X,Z) + \varepsilon.$
%~~~\mbox{where}~~\varepsilon \sim N(0, \sigma^2)
  % \begin{equation*}
  %     Y = f\opt(X) + \varepsilon~~~\mbox{where}~~\varepsilon \sim N(0, \sigma^2).
  % \end{equation*}
The available data is denoted by $\mc{D}^0 := \{X_i,Y_i,Z_i\}_{i \in \mc{I}}$ and given a pool of candidates 
$\xpool$, we want an
 adaptive sampling algorithms that iteratively select
candidates in $\xpool$ to be assigned a random treatment so that we can estimate average treatment effect efficiently. Under the ``likelihood" function $p(y | f, x, z) = p_{\varepsilon}(y - f(x,z))$,  let $g(f)$ represent  the average treatment effect, which is our estimand of interest. Formally, this is expressed as:
\begin{align}
g(f) \defeq \E_{X \sim P_X} \left[\E_{Y_1 \sim p(\cdot|f,X,Z=1) , Y_0 \sim p(\cdot|f,X,Z=0)} \left[Y_1 -  Y_0 \right]\mid f \right]. \label{eqn:ate-g-f}
\end{align}




 %Again the true function $f\opt$ is unknown and we model it from a Bayesian perspective by formulating a posterior  given available data.
 Our prior belief  $\mu$ over functions $f$, now 
 reflects our uncertainty about how
outcomes are generated given features and treatments. 
  We sequentially observe outcome of a batch of inputs from $\mc{X}_{\rm pool}$ (actions), and treatments assigned to this batch. We assume that selected batch of inputs $\mc{X}^t$ is randomly assigned treatments $\mc{Z}^t$ with each $Z\sim p_Z$. We summarize our formulation in Figure~\ref{fig:MDP_framework_flowchart}.
%Specifically, our initial state is given by $\mu_0(\cdot) = \mu(\cdot \mid \mc{D}^0)$ and at each period $t$, we get outcomes for a batch of $K$ candidates $\mc{X}^{t+1} \subset \mc{X}_{\rm pool}$, with randomly assigned treatments $\mc{Z}^{t+1}$ (with each $Z\sim p_Z$) and get the data $ \mc{D}^{t+1} = (\mc{X}^{t+1} \times \datay^{t+1} \times \mc{Z}^{t+1} )$. After acquiring the data at each step $t$ we update posterior state to $\mu_{t+1}(\cdot) = \mu_t(\cdot \mid \mc{D}^{t+1})$. Modeling practical instances, we consider a small horizon problem with limited adaptivity $T$.  We denote by $\pi_t$ the adaptive labeling policy at period $t$. We account for randomized policies $\datax^{t+1} \sim \pi_t(\mu_t)$ with a flexible batch size $|\datax^{t+1}| = \batchsize_t$.   
Again, we assume $\mu_t$ is $\mc{F}_t-$measurable for all $t < T$, where $\mc{F}_t$ is the filtration generated by the observations up to the end of step $t$.
 Observe that $\mu_{t+1}$ contains randomness in the policy $\pi_t$, randomness in treatment assignment $\mc{Z}^{t+1}$ and randomness in $\datay^{t+1} \mid (\datax^{t+1}, \mc{Z}^{t+1},\mu_t)$. Letting $\pi = \set{\pi_0,....,\pi_{T-1}}$,  we minimize the uncertainty over $g(f)$
 at the end of data collection:
\begin{align}
\E_{\mc{D}^{1:T} \sim \pi} \left[G(\mu_{T}) \right] \defeq
\E_{\mc{D}^{1:T} \sim \pi} \left[ \V_{f \sim \mu_{T}}  g(f)  \right]
= \E_{\mc{D}^{1:T} \sim \pi} \left[ \V_{f \sim \mu(\cdot \mid \mc{D}^{0:T})}  g(f)  \right].
\label{eqn:general-ate-obj}
\end{align}  


 

\begin{figure}[ht]
\centering
\begin{tikzpicture}
[
roundnode/.style={circle, draw=black!60, very thick, minimum size=10mm},
squarednode/.style={rectangle, draw=black!60, very thick, minimum size=10mm, align =center,text width = 26mm},
]
%Nodes
\node[roundnode]      (maintopic)                              {$\mu$};
\node[roundnode]        (circle1)       [right=20mm of maintopic] {$\mu_0$};
\node[roundnode]      (circle2)       [right=5mm of circle1] {$\mu_t$};
\node[roundnode]        (circle3)       [right=20mm of circle2] {$\mu_{t+1}$};
\node[roundnode]        (circle4)       [right=5mm of circle3] {$\mu_T$};
\node[squarednode]        (circle5)       [right=5mm of circle4] {Reward/Cost $\E \left[ \V_{\mu_T} (g(f))\right]$};


%Lines
\draw[thick, ->, >=stealth] (maintopic.east) -- node[anchor=south] {$(\mathcal{X}^0,\mathcal{Y}^0,\mathcal{Z}^0)$} (circle1.west);
\draw[thick, ->, dashed] (circle1.east) --  (circle2.west);
\draw[thick, ->] (circle2.east)  -- node[above, align =center, text width = 18mm] { Query $(\mathcal{X}^t,\mathcal{Y}^t,\mathcal{Z}^t)$} (circle3.west);
\draw[thick, ->,dashed] (circle3.east) --  (circle4.west);
\draw[thick, ->] (circle4.east) --  (circle5.west);


\end{tikzpicture}
\caption{MDP framework for adaptive labeling to efficiently estimate the average treatment effect (ATE).}
\label{fig:MDP_framework_flowchart}
\end{figure}
 









\begin{comment}
\subsection{Broader applicability of the framework to other problem settings} \label{sec:broad-framework-accuracy}
 

Although  we describe our setting in a healthcare setting with the objective  to estimate the recall of a trained AI model $\model(\cdot)$, the framework caters to many other problem settings. The extension to the evaluation of model based on accuracy (in regression setting) is straightforward, we simply replace the definition of recall $g(f)$ in~\eqref{eqn:l2-g-f} with
\begin{align*}
    g(f) = \E_{\substack{ y \sim p(y|f,x) \\  \forall x \in \mathcal X}} \big( \E_{{\textbf x} \sim p_x} [y-\model(x)]^2 \big).
\end{align*}


\textcolor{red}{To discuss if we need to have it here}
We can also extend this setting to the efficient estimation of the ATE as well. We describe these in detail below:

\begin{itemize}
    
    \item Estimating accuracy:  \[g(f) = \E_{\substack{ y \sim p(y|f,x) \\  \forall x \in \mathcal X}} \big( \E_{{\textbf x} \sim p_x} [y-\model(x)]^2 \big)\]
%    \item Estimating ATE with known control arm: 
%\[g(f) = \E_{\substack{ y \sim p(y|f,x) \\  \forall x \in \mathcal X}} \big( \E_{{\textbf x} \sim p_x} [y-\model(x)] \big)\]
\item Estimating ATE  (with minor modifications - broad structure remains similar) : 


Consider feature vector ${\mathbf x} \in \mathcal X $  distributed as ${\mathbf x}  \sim p_{\mathbf x}$, treatment $z \in {\mathcal Z} = \{0,1\}$, and a class of random functions $f: {\mathcal X} \times {\mathcal Z} \to {\mathcal Y}$, which determines the likelihood $p(y_i|f,{\mathbf x_i},z_i)$. Note that $f$ is random and reflects our uncertainty about how
labels are generated given features and the treatment. Additionally, the joint likelihood is determined as follows,  

\[p(Y|f,X,Z) = \prod_{i} p(y_i|f,{\mathbf x_i}, z_i) \]

Assuming the prior over functions $f$ to be $\mu$, therefore we have 
\[p(Y|X,Z) = \int \prod_{i} p(y_i|f,{\mathbf x_i},z_i) d\mu(f) \]


Also, assuming that under the  true data generating function $f$ (if known precisely - which we don't), the estimand of interest is

\[ \E_{{\textbf x} \sim p_x}  \left( \E_{\substack{ y \sim p(y|x,f,z=1) }} y - \E_{\substack{ y \sim p(y|x,f,z=0) }} y \right) \]


Throughout the paper we assume the above data generating process.  Now, suppose we have some labeled  data $(\datax^0,\datay^0,Z^0) =({\mathbf x}_{1:m}^0,y_{1:m}^0, z_{1:m}^0)$. 
    We run a experiment, in which we want to query the labels (in batches), so as to minimize the uncertainty of the estimand of interest. Suppose, the horizon of the experiment is $T$. Now, given prior $\mu$ and labeled data $\datax^0,\datay^0,Z^0$, in the beginning of our experiment the posterior state is $\mu_0$.

 At each step $j$ ($j \geq 1$), we query labels for a batch (with size $k$) of unlabeled data $(\datax^j,Z^j) \subset \mathcal X \times \mathcal Z$  and get labels $\datay^j$. After acquiring the labels at each step $j$ we update posterior state to $\mu_{j+1}$, informed by $\mu_j$ and $(\datax^j,\datay^j,Z^j)$. 
 
 Let the policy at step $j$ be $\pi_j$ (potentially random), which gives $\datax^{j+1},Z^{j+1} \sim \pi_j(\mu_j)$.  Observe that $\mu_{j+1}$ is random because of the randomness of the policy $\pi_j$ and $\datay^{j+1}|\{\datax^{j+1},Z^{j+1},\mu_j\}$ (\textcolor{red}{can this be written in a better way?}). Let, $\pi = \{\pi_0,....,\pi_{T-1}\}$. Therefore, our objective is to

 
\[ \min_{\pi} \E \left[ {\mathbf {Var}}_{f \sim \mu_T} \left( \E_{{\textbf x} \sim p_x}  \left( \E_{\substack{ y \sim p(y|x,f,z=1) }} y - \E_{\substack{ y \sim p(y|x,f,z=0) }} y \right) \right) \right]\]

where, $\mu_T$ depends on $\{(\datax^i,\datay^i,Z^i)\}_{i=0}^T$ and outer expectation is over both $\pi$ and  $\datay^{j+1}|\{\datax^{j+1}, Z^{j+1},\mu_j\}$ for all $j \in [0,T-1]$.


%Constraining the action space is straightforward - by first choosing set of x's using k-subset and then assigning treatment with learnable probability parameters $w_1,...,w_n$.

\end{itemize}

 \[ g(f) = \E_{\substack{ y \sim p(y|f,x) \\  \forall x \in \mathcal X}}\E_{{\textbf x} \sim p_x} g(y,{\textbf x}) \approx \E_{\substack{ y \sim p(y|f,x) \\  \forall x \in  \datax^u}} \left( \frac{1}{n}\sum_{i=1}^n \tilde{g}(y,{\textbf x}_i^u) \right)\]




Notation borrowed from a combination of the following papers 
~\citep{LeeYuNaFoLe23, KatoOgKoIn24, FongHoWa24}

%  
\end{comment}


%%% Local Variables:
%%% mode: latex
%%% TeX-master: "main"
%%% End:

\section{Algorithm for $K$-Max Bandits with General Continuous Distribution}\label{sec:general-continuous}

We now present our solution framework for continuous $K$-Max bandits, beginning with the fundamental regularity condition that enables discretization-based learning:
\begin{assumption}\label{ass:bi-lipschitz}
Each outcome distribution $D_i$ is supported on $[0,1]$ with a bi-Lipschitz continuous cumulative distribution function (CDF) $F_i$. Specifically, there exists $L \geq 1$ such that for any $i \in [N]$ and $0 \leq v < u \leq 1$:
\begin{align*}
    \frac{1}{L}(u - v) \leq F_i(u) - F_i(v) \leq L(u - v).
\end{align*}
\end{assumption}
Many studies on MAB or CMAB consider $[0, 1]$-supported arms \citep{abbasi2011improved,chen2013combinatorial,slivkins2019introduction,lattimore2020bandit}. The bi-Lipschitz continuity is also common in practice \citep{li2017provably,wang2019optimism,liu2023optimistic} and satisfied by many distributions such as (truncated) Gaussians, mixed uniforms, Beta distributions, etc.


\subsection{The Discretization of Countinuous $K$-Max Bandits}
\label{sec:discretized-K-Max}
Since it is complex to estimate the general continuous distributions, a natural idea is to perform \textit{discretization} with granularity $\epsilon$.
Below, we define the \textit{discrete $K$-Max bandits} (called $\bar\gB$) converted from the continuous $K$-Max bandits $\mathcal{B}^*$, where each discrete arm's outcome $\bar X_i$ is discretized from the continuous random variable $X_i$ under $\epsilon$:
\begin{align}\label{eq:discretize-outcome}
    \bar X_i = \sum_{j \in [M]} \1\left[X_i \in M_j\right] \cdot v_j,
\end{align}
where $M = \ceil{1/\epsilon}$\footnote{Without loss of generation, we can take $\epsilon$ such that $M\epsilon > 1$.} is the number of discretization bins, $M_j := [(j-1)\epsilon, j\epsilon)$ is the $j$-th bin, and $v_j := (j-1)\epsilon$ is the approximate value of $j$-th bin. 
%
We also let $M_{\le j} = \cup_{j' \le j} M_{j'}$ and $M_{\ge j} = \cup_{j' \ge j} M_{j'}$.
%
For simplicity, we denote $p^*_{i,j}$ as the probability that $X_i$ falls in $M_j$. For every $i \in [N]$ and $j \in [M]$,
\begin{align*}
    p^*_{i,j} := \Prob[X_i \in M_j] = \Prob[\bar X_i = v_j].
\end{align*}
Therefore, $\bar \gB$ only depends on the discrete probability set $\vp^* = \{p_{i,j}^* : i \in [N], j \in [M]\}$. Moreover, we set $\bar r(S; \vp^*)$ as the expected reward of an action $S$ in discrete $K$-Max bandits under the probability set $\vp^*$:
\begin{align*}
    \bar r(S ; \vp^*) &= \sum_{j\in [M]} v_j \cdot \Prob\left[\max_{i\in S} (\bar X_i) = v_j\right]
\end{align*}
A key observation is that $\max_{i\in S} (\bar X_i) = v_j$ is equivalent to $\max_{i\in S} (X_i)\in M_j$. This means $\Prob\left[\max_{i\in S} (\bar X_i) = v_j\right] = \Prob\left[\max_{i\in S} (X_i)\in M_j\right]$, which gives an upper bound for the discretization error as follows. The formal version is provided by \Cref{lemma:discrete-error-formal} (in \Cref{app:discrete-error}). 
\begin{lemma}\label{lemma:discrete-error}
For any $S \in \gS$, we have
$$|r^*(S) - \bar r(S; \vp^*)| \le \epsilon.$$ % for any $S \in \gS$.
\end{lemma} 





\subsection{Converting a Discrete Arm to a Set of Binary Arms}
\label{sec:discrete-binary}
Follow the classical process in \citet{wang2023combinatorial}, we can convert a discrete arm $X_i$ to a set of binary arms and estimate the parameters $\vq^* = \{q^*_{i,j} : i \in [N], j \in [M]\}$ instead of $\vp^*$, where
\begin{align}\label{eq:qstar-def}
    q_{i,j}^* := \frac{p_{i,j}^*}{1 - \sum_{j' > j} p_{i,j'}^*}, \ p^*_{i,j} = q^*_{i,j} \cdot \prod_{j' > j} (1 - q^*_{i,j'}).
\end{align}
Let $\{\bar Y_{i,j}\}_{i \in [N],j \in [M]}$ be independent binary random variables such that $\bar Y_{i,j}$ takes value $v_j$ with probability $q_{i,j}^*$, and value $0$ otherwise. Then $\max_{j \in [M]}\{\bar Y_{i,j}\}$ has the same distribution as $\bar X_{i}$.
For any $S \in \gS$, define $\bar r_q(S; \vq)$ as the expected maximum reward of $\{\bar Y_{i,j}\}_{i \in S, j \in [M]}$ with probability set $\vq$. 
Then we have 
\begin{lemma}\label{lemma:r-q-r}
    For any $\vp$ and $\vq$ satisfying \Cref{eq:qstar-def}, we have $$\bar r_q(S; \vq) = \bar r(S; \vp), \quad \forall S \in \gS.$$
\end{lemma}
The formal version of this lemma is given in \Cref{lemma:r-q-r-formal}. 
Moreover, the function $\bar r_q$ is monotone with respect to  $\vq$, i.e., 
\begin{lemma}[{\citet[Lemma 3.1]{wang2023combinatorial}}]
\label{lemma:monotone}
    For two probability set $\vq'$ and $\vq$ such that $q'_{i,j} \ge q_{i,j}$ holds for any $i \in [N], j \in [M]$, we have 
    $$\bar r_q(S; \vq') \ge \bar r_q(S;\vq),\quad  \forall S \in \gS.$$
\end{lemma}



\subsection{An Efficient Offline Oracle for Discrete $K$-Max Bandits} \label{sec:offline-oracle}
For any discrete $K$-Max bandits with probability set $\vp$, we can apply the \textit{PTAS} algorithm \citep{chen2016combinatorial} as a polynomial time offline $\alpha$-approximation optimization oracle for any given $\alpha < 1$. 
Moreover, for any probability set $\vq$, we can convert it to $\vp$ by \Cref{eq:qstar-def}, input this $\vp$ to the PTAS oracle and get the approxiamation solution $\operatorname{PTAS}(\vp)$ satisfying
\begin{equation}\label{eq:ptas}
\begin{aligned}
    \bar r_q\left(\operatorname{PTAS}(\vp); \vq\right) &= \bar r\left(\operatorname{PTAS}(\vp); \vp\right) \\
    &\ge \alpha \cdot \max_{S \in \gS} \bar r(S; \vp) = \alpha \cdot \max_{S \in \gS} \bar r_q(S; \vq).
\end{aligned}
\end{equation}
In the following algorithm, we set $\alpha = 1-\epsilon$ and control the relative error to achieve sublinear regret guarantees.



\subsection{Efficient Algorithm for Continuous $K$-Max Bandits}\label{sec:algorithm}
Building on the methodology in previous subsections, we adapt the framework in \citet{wang2023combinatorial}, and
present \texttt{DCK-UCB} (Discretized Continuous $K$-Max with Upper Confidence Bounds), the first efficient algorithm addressing $K$-Max bandits with general continuous outcome distributions. 
%
Generally speaking, we first discretize the countinuous $K$-Max bandits to discrete $K$-Max bandits. Then we convert every discrete arm to a set of binary arms, and estimate the corresponding $\vq^*$. Finally, we convert $\vq^*$ back to $\vp^*$, input $\vp^*$ to the $\operatorname{PTAS}$ oracle, and get the action we want to select.
%
% 

\begin{algorithm}[!t]
\caption{\texttt{DCK-UCB}: Discretization Continuous $K$-Max Bandits with Upper Confidence Bonus}
\label{alg}
\begin{algorithmic}[1]
\REQUIRE Discretization granularity $\epsilon$, upper confidence bonuses $\{\beta_{i,j}^t : i \in [N], j \in [M], t \in [T]\}$, and the offline $\alpha$-approximated optimization oracle $\operatorname{PTAS}$ for discrete $K$-Max bandits \citep{chen2016combinatorial}.
\STATE Initialize $M \leftarrow \ceil{1/\epsilon}$, $\hat q_{i, 1}^1 \leftarrow 1$ for every $i \in [N]$, and $\hat q_{i,j}^1 \leftarrow 0$ for every $i\in [N], j > 1$.
\FOR{$t=1,2,\ldots,T$}
\STATE For every $i \in [N], j \in [M]$, set   
\begin{align}\label{eq:def-barq}
    \bar q_{i, j}^t \leftarrow \min\left\{ \hat q_{i,j}^t + \beta_{i,j}^t + (K-1)\frac{L^4}{j^2}, 1\right\}.
\end{align}
% \COMMENT{Construct optimistic estimator $\bar q_{i,j}^t$.}
\STATE Convert $\bar \vq^t$ to $\bar \vp^t$ by \Cref{eq:qstar-def}.
\STATE \label{algline:oracle} Choose action $S_t \leftarrow \operatorname{PTAS}(\bar \vp^t)$. 
\STATE Observe $(r_t,i_t)$ by executing action $S_t$. Denote $j_t$ as the range number of $r_t$, i.e., $r_t \in M_{j_t}$.
% \COMMENT{Execute $S_t$ and receive value-index feedback.}
\STATE For any $i, j \in [N] \times [M]$,
\begin{align*}
    C_t(i, j) = 
        C_{t-1}(i, j) + \1\left[ i = i_t \And j = j_t \right]
\end{align*}
and
\begin{align*}
    SC_t(i, j) = 
        SC_{t-1}(i, j) + \1\left[ i \in S_t \And j \ge j_t \right]
\end{align*}
% \COMMENT{Update the counter $C_t$ and $SC_t$ for estimation.}
\STATE Calculate estimator $\hat q_{i,j}^{t+1} \leftarrow \frac{C_t(i,j)}{SC_t(i,j)}$, for every $i \in [N]$ and $j\in [M]$.
% \COMMENT{Estimate $q^*_{i,j}$ by $\hat q_{i,j}^{t+1}$.} 
\ENDFOR
\end{algorithmic}
\end{algorithm}



\Cref{alg} presents the pseudo-code of \texttt{DCK-UCB}. In Line 3, we calculate the optimistic estimator $\bar q_{i,j}^t$ which upper bounds $q^*_{i,j}$ with high probability.
%
This is done by adding two upper confidence bonus terms $\beta_{i,j}^t$ and $(K-1)L^4/j^2$.
%
Analysis shows that $\bar q_{i,j}^t \ge q^*_{i,j}$ with high probability (Lemma \ref{lemma:concentration}). The detailed discussion on this estimator will be given in the following paragraphs.
%
In Lines 4-5, the agent converts this $\bar \vq$ to $\bar \vp$, and then runs the offline $\alpha$-approximation optimization oracle $\operatorname{PTAS}$ with $\alpha = 1 - \epsilon$ to get action $S_t$ for execution.
%
In Line 6, the agent gets the value-index return $(r_t, i_t)$, and discretizes the value $r_t$ to the index of bin $j_t$, i.e., $r_t \in M_{j_t}$.
%
In Lines 7-8, the agent estimates $\vq^*$ by two counters: $C_t(i,j)$ counts the times when $(i,j)$ exactly equals the feedback $(i_t,j_t)$, and $SC_t(i,j)$ counts the number of steps $\tau \le t$ satisfying $i \in S_\tau$ and $j_\tau \le j$. As outlined in \Cref{alg}, each step of the algorithm has polynomial time and space complexity, which demonstrates the computational tractability of \texttt{DCK-UCB}. 


\paragraph{Biased Estimator.} 


The key challenge in the algorithm design and theoretical analysis is that $\hat q_{i,j}^t$ is not an unbiased estimator for $q_{i,j}^*$. This means that except for the confidence radius due to the randomness of the environment, we still need another bonus term to bound the bias to guarantee that $\bar q_{i,j}^t$ is a UCB for $q_{i,j}^*$. 
%

Specifically, note that 
\begin{eqnarray*}
    q^*_{i,j} = \frac{p^*_{i,j}}{1 - \sum_{j' > j} p^*_{i,j'}} = \frac{p^*_{i,j}}{\sum_{j'=1}^j p^*_{i,j'}} = \frac{\Prob[X_i \in M_j]}{\Prob[X_i \in M_{\le j}]} 
\end{eqnarray*}
If we have an assumption that when $i\in S_\tau$ and $j_\tau = j$, $X_i(\tau) \in M_j$ implies $i = i_\tau$, then we can guarantee that $\hat q_{i,j}^{t} = C_t(i,j) / SC_{t}(i,j)$ is an unbiased estimator for $q^*_{i,j}$. This is because that in this case, $\frac{C_t(i,j)}{ SC_{t}(i,j) }= \frac{\# \text{ of } i_\tau = i, j_\tau = j}{\# \text{ of } i\in S_\tau, j_\tau \le j} $ is the fraction of $X_i(\tau) \in M_j$ condition on $i\in S_\tau, j_\tau \le j$, 
%
which is an unbiased estimator for 
\begin{eqnarray*}
    && \Prob[X_i(\tau) \in M_j \mid i\in S_\tau, j_\tau \le j]\\
    &=&  \frac{\Prob[X_i(\tau) \in M_j, i\in S_\tau, j_\tau \le j]}{\Prob[i\in S_\tau, j_\tau \le j]}\\
    &=&\frac{\Prob[X_i(\tau)  \in M_j]\cdot \Prob[X_k(\tau)  \in M_{\le j}, \forall k \in S_{\tau}, k \neq i]}{\Prob[X_i(\tau)  \in M_{\le j}] \cdot \Prob[X_k(\tau)  \in M_{\le j}, \forall k \in S_{\tau}, k \neq i]}\\
    &=&\frac{\Prob[X_i(\tau) \in M_j]}{\Prob[X_i(\tau) \in M_{\le j}]}
\end{eqnarray*}

However, we know that in the discrete K-Max bandits converted from the continuous K-Max bandits, there is no such assumption (different from \citet{wang2023combinatorial} who requires deterministic tie-breaking rule). When multiple arm has $X_i(\tau) \in M_j$, the observed winning arm $i_t = \arg\max X_i(\tau)$ is not a fixed one, and even we do not know the distribution of the winner. 
%
Because of this, we cannot guarantee that condition on $i\in S_\tau, j_\tau \le j$, we increase the counter for every time $X_i(\tau) \in M_j$. Some steps that $X_i(\tau) \in M_j$ but $X_i(\tau)$ is not the winner are missed. 
%
This nondeterministic tie-breaking effect, arising from the continuous-to-discrete transformation, induces systematic negative bias in conventional estimators $\{\hat q_{i,j}^t\}$.
%
Therefore, to guarantee that our used $\{\bar q_{i,j}^t\}$ is an upper confidence bound of $\{ q_{i,j}^*\}$, we need another bonus term (i.e., the term $(K-1)\frac{L^4}{j^2}$), given by a novel concentration analysis with bias-aware error control. This is shown in the following key lemma, where the formal version is in \Cref{lemma:concentration-formal}.


\begin{lemma}\label{lemma:concentration}
Under \Cref{ass:bi-lipschitz}, let the confidence radius be defined as  
\begin{align}\label{eq:def-beta}
    \beta_{i,j}^t := \sqrt{8\frac{\log(NMt)}{SC_{t-1}(i,j)}}.
\end{align}
Then with probability at least $1 - t^{-2}$,
\begin{align}
    \left|\hat q_{i,j}^t - q^*_{i,j}\right| \le {\beta_{i,j}^t} + (K-1)\cdot(L^4/j^2),
\end{align}
holds for every $t \in [T]$, $i \in [N]$ and $j \in [M]$.
\end{lemma}

The bound in \Cref{lemma:concentration} decomposes into an exploration bonus term $\beta_{i,j}^t$ and a bias compensation term $(K-1)\frac{L^4}{j^2}$. The exploration bonus term arises from the randomness of the environment, which is almost the same with existing researches \citep{wang2017improving,liu2023contextual,wang2023combinatorial}.
The bias compensation term, on the other hand, comes from the nondeterministic tie-breaking effect in the continuous-to-discrete transformation. 
%
As we have explained, this term is because that condition on $i\in S_\tau, j_\tau \le j$, there are some time steps that $X_i(\tau) \in M_j$ but arm $i$ is not the winner and thus we miss these steps in counter $C_{i,j}^t$.
%
When this happens, we know that there must be at least one other arm $i'\ne i, i'\in S_{\tau}$ such that $X_{i'}(\tau) \in M_j$.
%
This probability can be upper bounded by 
\begin{align*}
    &\sum_{i'\ne i, i'\in S_{\tau}} \Prob[X_i(\tau) \in M_j, X_{i'}(\tau) \in M_j \mid i\in S_\tau, j_\tau \le j]\\
    =\!&\sum_{i'\ne i, i'\in S_{\tau}} \frac{p^*_{i,j}p^*_{i',j} }{ \sum_{j'\le j}p^*_{i,j'} \sum_{j'\le j}p^*_{i',j'}} \le (K-1)\frac{(L\epsilon)^2}{(j\epsilon/L)^2},
\end{align*}
where the last inequality is because of bi-Lipschitz assumption \Cref{ass:bi-lipschitz}.


Notably, the bias term dominates for small $j$ values due to the influence of other arms becomes higher when condition on $j_\tau \le j$ with smaller $j$.
%
However, our regret analysis in \Cref{sec:result} suggests that the amplified bias for small $j$ has diminishing impact on cumulative regret -- a crucial property enabling our sublinear regret guarantee.




\subsection{Theoretical Results}
\label{sec:result}
We establish the first efficient algorithm \texttt{DCK-UCB} (\Cref{alg}) which enjoys the sublinear regret guarantees in continuous $K$-Max bandits problem with value-index feedback. 
\begin{theorem}
\label{thm:main}
Under \Cref{ass:bi-lipschitz}, let the offline optimization oracle be a PTAS implementation \citep{chen2016combinatorial}. Given the exploration bonus term $\beta_{i,j}^t$ in \Cref{eq:def-beta}, discretization granularity $\epsilon = \gO(L^{-2}K^{-3/4}N^{1/4}T^{-1/4})$ and PTAS approximation factor $\alpha = 1 - \epsilon$, \Cref{alg} enjoys the regret guarantee
\begin{align*}
    \gR(T) \le \wt{\gO}(L^{2}N^{\frac{1}{4}}K^{\frac{5}{4}}T^{\frac{3}{4}}).
\end{align*}
\end{theorem}
The formal statement with precise constants appears in \Cref{thm:main-formal}. Our analysis reveals that careful calibration of the discretization-error versus statistical-estimation trade-off enables the first sublinear regret guarantee $\gO(T^{3/4})$ for continuous $K$-Max bandits. 

\paragraph{Comparison to Prior Works.} The $\gO(T^{3/4})$ regret upper bound of \texttt{DCK-UCB} (\Cref{alg}) shown in \Cref{thm:main} advances the state-of-the-art in several directions. \citet{wang2023combinatorial} can achieve an $\gO(\sqrt{T})$ regret upper bound in the discrete $K$-Max bandits, but their algorithm cannot work for the continuous case due to non-zero discretization error and nondeterministic tie-breaking.
%
Recent work on submodular bandits \citep{pasteris2023sum,fourati2024combinatorial} attains $O(T^{2/3})$ regret via greedy oracles, but this approach suffers dual limitations: (1) The baseline of their regret is $\sum_{t=1}^T (1-1/e)r^*(S^*)$, but not $\sum_{t=1}^T r^*(S^*)$. In our definition, their regret becomes linear.   (2) Their algorithm requires the availability of submitting any subset of $\mathcal{A}$ with size less than or equal to $K$, which may not be practical in some applications, such as recommendation systems or portfolio selection that need to always submit size $K$ subsets. Our framework resolves both issues through our novel bias-corrected estimators with PTAS integration, which is both efficient and effective in dealing with continuous $K$-Max bandits.


\subsection{Proof Sketch of \Cref{thm:main}}
In this section we outline the proof of \Cref{thm:main}, which consists of four main steps. 

\paragraph{Step 1: From continuous regret to discretized regret.}
Let $\Delta_t := r^*(S^*) - r^*(S_t)$ be the regret for each round $t$. To control the regret, we aim to bound the summation of $\Delta_t$. 
$$ \gR(T) = \E\left[\sum_{t=1}^T \Delta_t \right]. $$
We first transfer the regret from continuous $K$-Max bandits to the discrete case. With \Cref{lemma:discrete-error}, we have
\begin{align*}
    \Delta_t &\le  \bar r(S^*; \vp^*) - \bar r(S_t; \vp^*) + 2\epsilon.
\end{align*}


\paragraph{Step 2: From discretized regret to estimation error.}
%
Recall the definition of $\bar r_q$ in \Cref{sec:discrete-binary}, we have $\bar r_q(S; \vq) = \bar r(S; \vp)$ for any $S \in \gS$, and probability set $\vp, \vq$ satisfying \Cref{eq:qstar-def}. 
By the monotonicity of $\bar r_q$ (in \Cref{lemma:monotone}) and the concentration analysis (in \Cref{lemma:concentration}),
we have with high probability, $\forall (i,j)\in[N]\times[M]$, $\bar q_{i,j}^t \ge q^*_{i,j}$ holds for any $t \in [T]$, which implies
\begin{align*}
    \bar r_q(S^*;\bar\vq^t) \ge \bar r_q(S^*; \vq^*).
\end{align*}
Moreover, by the property of $\alpha$-approximated offline optimization oracle PTAS \citep{chen2016combinatorial}  (in \Cref{eq:ptas}) with $\alpha = 1 - \epsilon$, 
\begin{align*}
    (1 - \epsilon) \bar r_q(S^*; \bar \vq^t) \le (1 - \epsilon) \max_{S \in \gS} \bar r_q(S; \bar \vq^t) \le \bar r_q(S_t; \bar \vq^t),
\end{align*}
which implies the conversion from $\Delta_t$ to the estimation error term
\begin{align*}
    \Delta_t &\le \bar r_q(S^*; \bar \vq^t) - \bar r_q(S_t; \vq^*) + 2\epsilon \\
    &\le (1 - \epsilon) \bar r_q(S^*; \bar \vq^t) - \bar r_q(S_t; \va^*) + 3\epsilon \\
    &\le \bar r_q(S_t; \bar \vq^t) - \bar r_q(S_t; \vq^*) + 3\epsilon.
\end{align*}
Therefore, we then focus on bounding the estimation error $\bar\Delta_t := \bar r_q(S_t; \bar \vq^t) - \bar r_q(S_t; \vq^*)$ to guarantee the sublinear regret upper bound:
\begin{align}\label{eq:regret-to-bar-delta}
    \gR(T) = \E\left[\sum_{t=1}^T \Delta_t \right] \le \E\left[\sum_{t=1}^T \bar\Delta_t \right] + 3T\epsilon.
\end{align}


\paragraph{Step 3: Decompose the estimation error.}
By similar methods as achieving the Triggering Probability Modulated (TPM) smoothness condition in cascading bandits \citep{wang2017improving} and $K$-Max bandits for binary distributions \citep{wang2023combinatorial}, we propose the following lemma.
\begin{lemma}
\label{lemma:tpm}
Denote the probability of event $\{j_t \le j\}$ as 
$$Q_j^*(S_t) :=  \prod_{k \in S_t, j' > j} (1 - q_{k,j'}^*).$$ Then we have
\begin{align}
    \bar \Delta_t \le 2\sum_{i \in S_t, j \in [M]} Q_j^*(S_t) \cdot v_j \cdot \left|\bar q_{i,j}^t - q^*_{i,j}\right|.
\end{align}
\end{lemma}
Equipped with \Cref{lemma:tpm}, we decompose $\bar \Delta_t$ into two parts through our novel concentration analysis in \Cref{lemma:concentration} and the definition of optimistic estimator $\bar q_{i,j}^t$ in \Cref{eq:def-barq}
\begin{equation}\label{eq:bar-delta}
\begin{aligned}
    \bar \Delta_t &\le \quad \underbrace{4 \sum_{i \in S_t, j \in [M]} Q_{j}^{*}(S_t) \cdot v_j \cdot \beta_{i,j}^t}_{\texttt{Bonus}_t} \\
    &\quad + \underbrace{4 \sum_{i \in S_t, j \in [M]} Q_{j}^{*}(S_t) \cdot v_j \cdot (K-1)\frac{L^4}{j^2}}_{\texttt{Bias}_t}.
\end{aligned}
\end{equation}


\paragraph{Step 4: Bound the Bonus and Bias terms.}
For the Bonus term, we apply standard analysis for combinatorial bandits with triggering arms in \citep{wang2017improving, liu2023contextual} where we encounter $NM$ binary arms in total and select $KM$ binary arms in every action and get
\begin{align}\label{eq:bonus-sum}
    \E\left[\sum_{t=1}^T \texttt{Bonus}_t \right] \le \wt\gO\left(\sqrt{(NM) \cdot (KM) \cdot T}\right).
\end{align}

To control the bias terms, we recall that $v_j = (j-1)\epsilon$. 
Therefore, we can write
\begin{equation}\label{eq:bias}
    \begin{aligned}
    \texttt{Bias}_t &\le 4K^2L^4 \sum_{j \in [M]} (j-1)\epsilon/j^2 \\
    &\le \gO\left( K^2L^4\epsilon\log(M) \right),
\end{aligned}
\end{equation}

Therefore, combining \Cref{eq:regret-to-bar-delta,eq:bar-delta,eq:bonus-sum,eq:bias}, the regret can be bounded by
\begin{align*}
    \gR(T) &\le \E\left[\sum_{t=1}^T \texttt{Bonus}_t + \texttt{Bias}_t\right] + \gO(T\epsilon) \\
    &\le \wt{\gO}\left(\sqrt{NKM^2T} + K^2L^4T\epsilon\right),
\end{align*}
where $M = \ceil{1/\epsilon}$. By taking $\epsilon = \gO(T^{-\frac{1}{4}}K^{-\frac{3}{4}}N^{\frac{1}{4}}L^{-2})$, we have
\begin{align*}
    \gR(T) \le \wt{\gO} \left( L^2N^{\frac{1}{4}}K^{\frac{5}{4}}T^{\frac{3}{4}} \right).
\end{align*}


\section{Better Performance in a Special Case: Exponential Distributions}
\label{sec:kminexp}

\newcommand{\Exp}{\operatorname{Exp}}

In this section, we demonstrate how specific distributional structure enables the improvement of the regret guarantee from $\wt{\gO}(T^{\frac{3}{4}})$ to $\wt{\gO}(\sqrt{T})$. Specifically, we investigate the special case where each distribution $D_i$ for $i \in [N]$ follows the exponential distribution with linear parameterization. 

Exponential distributions naturally model arrival or failure times in networked systems, job completion times in distributed computing, and service durations in queuing systems. 
A canonical application arises in server scheduling, where the goal is to select $K$ servers to minimize the service latency. Here, each server's latency can be modeled as an exponential random variable with a rate parameter $\mu_i$, and the overall performance of the $K$ selected servers is the lowest latency achieved among them. 
%
Here, the random outcome $X_i$ can be viewed as a random loss, and the winning loss is the minimum one. 
Moreover, we consider a linear parameterization to parameter $\mu_i$, which allows incorporating features like distance, traffic, or weather conditions into the model.


\subsection{The $K$-Min Exponential Bandits}

Based on the intuition, in this section we consider a special case of $K$-Max bandits: the $K$-Min exponential bandits.
Here each arm $i$ generates loss $X_i$ from an exponential distribution with linear parameterization. 
%
Specifically, each outcome distribution is an exponential distribution, i.e., $X_i \sim D_i = \Exp(\mu_i)$ where $\mu_i > 0$ is the parameter of arm $i$. Moreover, we assume that there exists a $d$-dimension unknown parameter $\theta^* \in \R^d$ and a known feature mapping $\phi : [N] \to \R^d$ such that $\mu_i = \langle \phi(i), \theta^* \rangle$ holds for any $i \in [N]$. The feature mapping $\phi$ satisfies that $\|\phi(i)\|_2 \le 1$ and the unknown parameter $\theta^*$ satisfies $\theta^* \in \Theta  \subset \R^d$, where $\sup_{\theta \in \Theta} \|\theta\|_2 \le V$.
%
The agent observes \textit{only} the minimum loss $\ell_t = \min_{i \in S_t} X_i(t)$ after playing subset $S_t \in \mathcal{S} = \{S \subseteq [N] : |S| = K\}$.
That is, we consider the weaker full bandit feedback case.

Let $\ell^*(S) := \E[\ell_t \mid S]$ be the expected loss for action $S \in \gS$, we further denote the best action $S^* = \argmin_{S \in \gS} \ell^*(S)$ and similarly introduce the regret metric to evaluate the performance of this agent:
\begin{align*}
    \gR(T) = \E\left[\sum_{t=1}^T \ell^*(S_t) - \ell^*(S^*)\right].
\end{align*}

Note that we can let $Z_i(t)= - X_i(t)$ and view $Z_i(t)$ as a kind of reward, and let $r_t = \max_{i \in S_t} Z_i(t)$. Then we can see that $\ell_t = \min_{i \in S_t} X_i(t) = \min_{i \in S_t} -Z_i(t) = - \max_{i \in S_t} Z_i(t) = -r_t$. By this way, we can view $K$-Min exponential bandits as a special case of $K$-Max bandits. 
%
However, one important difference is that in $K$-Min exponential bandits, we do not have value-index feedback, i.e., we do not know the winner's index. This is a full \textit{bandit feedback} setting, and making $K$-Min exponential bandits even more challenging. 

\subsection{Algorithm and Results}

The key observation in $K$-Min exponential bandits is that the minimum of several exponential distributions still follows an exponential distribution. That is, we have 
\begin{align*}
    \min_{i \in S} X_i \sim \Exp\left(\sum_{i \in S} \mu_i\right) = \Exp\left(\sum_{i \in S} \langle \phi(i), \theta^*\rangle \right).
\end{align*}
Therefore, it becomes much easier to estimate the true parameter $\theta^*$ by MLE. 
%
Specifically, let $\psi(S)  := \sum_{i \in S} \phi(i)$, $\forall S \in \gS$. Then with chosen action $S_t$ and parameter $\theta$, the observed loss should follow the exponential distribution $\Exp\left(\sum_{i \in S} \phi(i)^T \theta \right) = \Exp\left(\psi(S)^T\theta\right)$, whose probability density function is $f(x) = \psi(S)^T\theta  e^{\left(-\psi(S)^T\theta  x\right)}$. 
%
Because of this, the log-likelihood function is 
\begin{equation}
\begin{aligned}
    L_t(\ell_t; S_t, \theta) :&= - \log \left( \psi(S_t)^\top \theta e^{\left( -\psi(S_t)^\top \theta \ell_t\right)} \right).
\end{aligned}
\end{equation}

Denote $\gL_t(\theta)$ as the summation of $L_t$ and a regularization term
\begin{align}\label{eq:loglikelihood-def}
    \gL_t(\theta; \lambda) := \sum_{i < t} L_i(\ell_i; S_i, \theta) + \frac{\lambda}{2}\|\theta\|^2,
\end{align}
where $\lambda$ is the regularization factor. Then we present the algorithm \texttt{MLE-Exp} for $K$-Min exponential bandits in \Cref{alg:k-min}.

\begin{algorithm}
\caption{\texttt{MLE-Exp}: MLE for $K$-Min Exponential Bandits}
\label{alg:k-min}
\begin{algorithmic}[1]
\REQUIRE Regularization factors $\{\lambda_t\}_{t\in [T]}$, confidence radius $\{\gamma_t\}_{t\in [T]}$, and probability constant $\delta$.
\FOR{$t = 1, \ldots, T$}
    \STATE Compute MLE $\hat\theta_t$ by
    $$\hat{\theta}_t \leftarrow \argmin_{\theta \in \R^d} \mathcal{L}_t(\theta; \lambda_t),$$
    where $\gL_t(\theta;\lambda)$ is given in \Cref{eq:loglikelihood-def}. 
    % \COMMENT{Estimate $\theta^*$ by MLE $\hat{\theta}_t$.}
    \STATE Construct the confidence set $C_t(\hat\theta_t; \delta, \lambda_t)$ according to \Cref{eq:def-confidence-set}
    \STATE $(S_t, \wt{\theta}_t) \leftarrow \argmax_{S \in \gS, \theta \in C_t(\hat{\theta}_t; \delta, \lambda_t)} \langle \psi(S), \theta \rangle$ 
    % \COMMENT{Choose action with minimum expected loss.}
    \STATE Play action $S_t$ and observe the loss $\ell_t$. 
    % \COMMENT{Execute action $S_t$ and receive full-bandit feedback.}
\ENDFOR
\end{algorithmic}
\end{algorithm}

In Line 2 of \Cref{alg:k-min}, we estimate the MLE $\hat\theta_t$ by minimizing the summation of the log-likelihood function and the regularization term $\gL_t(\theta, \lambda_t)$. 
%
Given $\lambda_t$ a priori, we will write $\gL_t(\theta)$ instead of $\gL_t(\theta, \lambda_t)$ for simplicity. 
%
Inspired by \citet{liu2024almost, lee2024unified, liu2024combinatorial}, in Line 3, we construct a confidence set $C_t(\hat\theta_t;\delta)$, centered at the MLE $\hat\theta_t$ with confidence radius $\gamma_t(\delta)$, based on the gradient term $g_t(\theta) := -\nabla_\theta \gL_t(\theta) + \sum_{i < t} \ell_i \psi(S_i)$ and Hessian matrix $H_t(\theta) := \nabla^2_\theta \gL_t(\theta)$:
\begin{align}\label{eq:}
    C_t(\hat\theta_t; \delta) :=  
    &\left\{ \theta \in \Theta : \left\|g_t(\theta) - g_t(\hat\theta_t)\right\|_{H_t^{-1}(\theta)} \le \gamma_t(\delta)\right\},
\end{align}
where $\gamma_t$ is the confidence radius. 
Then in Line 4, we apply a double oracle to look for the action $S$ whose expected loss under a parameter $\theta$ in the confidence set ($1/\langle \psi(S), \theta \rangle$) is minimized.
%
Finally, we select this greedy action in Line 5 and use the observation to update the next time step's MLE and confidence set. 





The regret guarantees of \Cref{alg:k-min} is given below.

\begin{theorem}
\label{thm:kminexp}
With $\delta = 1/T$, $\lambda_t = \Theta(d\log T)$, and $\gamma_t = \Theta(\sqrt{d\log T})$, \Cref{alg:k-min} satisfies:
\begin{align*}
    \mathcal{R}(T) \leq \widetilde{\gO}\left(\sqrt{d^3 T}\right).
\end{align*}
\end{theorem}
Compared with the $O(T^{\frac{3}{4}})$ regret upper bound for general continuous K-Max bandits, here the regret upper bound is reduced to $O(T^{\frac{1}{2}})$ (which is nearly minimax optimal) even without the feedback of winner's index, due to the utilization of the exponential distribution's property. In short, we do not need to use a discretization method and can directly construct an unbiased estimator for the known parameter $\theta^*$.
%
The proof is inspired by previous analysis of general linear bandits \citep{lee2024unified,liu2024almost} and logistics bandits \citep{liu2024combinatorial}, and we defer the detailed proof to \cref{Appendix:k-min}.

\section*{Conclusion}
This paper aims to enhance our understanding of the computational complexity of computing various Shapley value variants. We found that for various ML models --- including decision trees, regression tree ensembles, weighted automata, and linear regression --- both local and global interventional and baseline SHAP can be computed in polynomial time under HMM modeled distributions. This extends popular algorithms, such as TreeSHAP, beyond their empirical distributional scope. We also establish strict complexity gaps between the various SHAP variants (baseline, interventional, and conditional) and prove the intractability of computing SHAP for tree ensembles and neural networks in simplified scenarios. Overall, we present SHAP as a versatile framework whose complexity depends on four key factors: \begin{inparaenum}[(i)] \item model type, \item SHAP variant, \item distribution modeling approach, \item and local vs. global explanations\end{inparaenum}. We believe this perspective provides deeper insight into the computational complexity of SHAP, paving the way for future work.




%We believe that our framework provides a more intricate understanding of SHAP computation complexity across different models, distributions, and variants, paving the way for further research.

Our work opens promising directions for future research. First, expanding our computational analysis to other SHAP-related metrics, such as asymmetric SHAP~\citep{frye20} and SAGE~\citep{covert2020understanding}, would be valuable. Additionally, we aim to explore more expressive distribution classes and relaxed assumptions beyond those in Section \ref{sec:tractable} while maintaining tractable SHAP computation. Finally, when exact computation is intractable (Section \ref{sec:intractable}), investigating the approximability of SHAP metrics through approximation and parameterized complexity theory~\citep{downey2012parameterized} is an important direction.

%Our work opens several promising avenues for future research on the computational properties of explainable AI methods, with a particular focus on SHAP. First, it would be interesting to broaden the computational analysis conducted in this work to include other popular SHAP-related metrics in the literature, such as asymmetric SHAP \cite{frye20} and SAGE \cite{covert2020understanding}. Also, in the future, we aim to explore more expressive distribution classes and relaxed distributional assumptions—extending beyond those examined in Section \ref{sec:tractable} —that still yield tractable SHAP computation. Finally, when exact computation proves intractable (Section \ref{sec:intractable}), it is worthwhile to theoretically investigate the question of the approximability of computing the SHAP metrics across various configurations, through the lens of approximation and parametrized complexity theory \cite{arora2009computational}.

%This paper aims to deepen our understanding of the computational complexity involved in obtaining different Shapley value variants. We found that for a variety of ML models, including decision trees, tree ensembles for regression, weighted automata, and linear regression models — computing both local and global interventional and baseline SHAP can be done in polynomial time when distributions are modeled by HMMs. This extends the distributional scope of popular algorithms like TreeSHAP, which is limited to empirical distributions. Additionally, we demonstrate a strict complexity gap between SHAP variants, showing that interventional and baseline SHAP can be strictly easier to compute than conditional SHAP. Despite these positive results, we uncovered intractability for various SHAP variants in neural networks and tree ensembles. Finally, we provided generalized complexity relations across SHAP variants. We believe that our framework offers a deeper understanding of the complexity involved in computing SHAP across various variants, models, distributions, as well as in both local and global computations, laying the groundwork for future research.


% \section*{Impact Statement}
% This paper presents work whose goal is to advance the field of Machine Learning. There are many potential societal consequences of our work, none which we feel must be specifically highlighted here.

\bibliography{ref}
\bibliographystyle{apalike}


\newpage
\appendix
% \onecolumn
\appendixpage

\startcontents[section]
\printcontents[section]{l}{1}{\setcounter{tocdepth}{2}}
\newpage

% \section*{Appendix}

\section{Omitted Proofs in \Cref{sec:general-continuous}}

In this section, we present the omitted proofs in \Cref{sec:general-continuous}, which include the full proof of \Cref{thm:main}.

\subsection{Discretization Error}\label{app:discrete-error}
First we show that the discretization from original continuous problem $\gB^*$ to $\bar\gB$ with discretization width $\epsilon$ will involve controllable error in expected loss, which is shown in \Cref{lemma:discrete-error} and formalized by the following lemma. 

\begin{lemma}\label{lemma:discrete-error-formal}
For any $S \in \gS$, we have
\begin{align}\label{eq:discrete-error}
    \bar r(S; \vp^*) \le r^*(S) \le \bar r(S; \vp^*) + \epsilon.
\end{align}
\begin{proof}
Notice that we have 
\begin{align*}
    \Prob\left[\max_{i\in S} (\bar X_i) = v_j\right] &= \sum_{I\subset S} \prod_{i\in I}\Prob[\bar X_i = v_j] \cdot \prod_{k \in S, k \notin I}\Prob[\bar X_k < v_j] \\
    &= \sum_{I\subset S} \prod_{i\in I} \Prob[X_i \in M_j] \cdot \prod_{k \in S, k \notin I}\Prob[X_k \in M_{\le j-1}] \\
    &= \Prob\left[\max_{i \in S} (X_i) \in M_j\right].
\end{align*}
Therefore, by definition of $r^*(S)$, we have
\begin{align*}
    r^*(S) &= \sum_{j\in [M]} \int_{r \in M_j} r\cdot \rd\Prob_{\max_{i\in S}(X_i)}(r) \\
    &\ge \sum_{j\in [M]} (j-1)\epsilon \int_{r \in M_j} \rd\Prob_{\max_{i\in S}(X_i)}(r) \\
    &= \sum_{j\in [M]} (j-1)\epsilon \cdot \Prob\left[\max_{i \in S} (X_i) \in M_j\right] \\
    &= \sum_{j\in [M]} (j-1)\epsilon \cdot \Prob\left[\max_{i\in S} (\bar X_i) = (j-1)\epsilon\right] \\
    &= \bar r(S; \vp^*),
\end{align*}
where the inequality is given by the definition of $M_j$. Then we achieve the left-hand side of \Cref{eq:discrete-error}. For the other side, we can similarly establish
\begin{align*}
    r^*(S) &= \sum_{j\in [M]} \int_{r \in M_j} r\cdot \rd\Prob_{\max_{i\in S}(X_i)}(r) \\
    &\le \sum_{j\in [M]} j\epsilon \int_{r \in M_j} \rd\Prob_{\max_{i\in S}(X_i)}(r) \\
    &= \sum_{j\in [M]} (j - 1)\epsilon \cdot \Prob\left[\max_{i \in S} (X_i) \in M_j\right] + \epsilon \cdot  \sum_{j\in [M]}  \Prob\left[\max_{i \in S} (X_i) \in M_j\right]\\
    &= \bar r(S; \vp^*) + \epsilon.
\end{align*}
\end{proof}
\end{lemma}

\subsection{Converting to Binary Arms}
As detailed in \Cref{sec:discrete-binary}, we set
\begin{align*}
    q_{i,j}^* := \frac{p_{i,j}^*}{1 - \sum_{j' > j} p_{i,j'}^*}, \ p^*_{i,j} = q^*_{i,j} \cdot \prod_{j' > j} (1 - q^*_{i,j'}),
\end{align*}
which implies
\begin{align*}
    q^*_{i,j} = \frac{p^*_{i,j}}{1 - \sum_{j' > j} p^*_{i,j'}} = \frac{p^*_{i,j}}{\sum_{j'=1}^j p^*_{i,j'}} =
    \frac{\Prob[X_i \in M_j]}{\Prob[X_i \in M_{\le j}]}.
\end{align*} 
For any given probability set $\vq = \{q_{i,j}: i \in [N], j \in [M]\}$, we can apply \Cref{eq:qstar-def} to get the corresponding $\vp$ defined as
\begin{align*}
    p_{i,j} = q_{i,j} \cdot \prod_{j' > j} (1 - q_{i,j'}).
\end{align*}
Assume $\{Y_{i,j}^\vq\}_{i\in[N], j \in [M]}$ is the set of independent binary random variables that $Y_{i,j}^\vq$ takes value $v_j = (j-1)\epsilon$ with probability $q_{i,j}$ and takes value $0$ otherwise. And $\{X_i^\vp\}_{i\in[N]}$ is the set of independent discrete random variables that $X_i^\vp$ takes value $v_j$ with probability $p_{i,j}$ for every $j \in [M]$. Therefore, by simple calculation, we have $\max_{j\in[M]}\{Y_{i,j}^\vq\}$ has the same distribution of $X_i^\vp$. 

$\bar r_q(S; \vq)$ is defined as the expected maximum reward of $\{Y_{i,j}^\vq\}_{i\in S,j\in[M]}$ Then we can write
\begin{align}\label{eq:def-rq}
   \bar r_q(S; \vq) = \sum_{j \in [M]} v_j \cdot \left( Q_j(S; \vq) - Q_{j-1}(S;\vq)\right),
\end{align}
where we denote for simplicity
\begin{align}\label{eq:def-Qj}
    Q_j(S; \vq):= \prod_{k \in S, j' > j} (1 - q_{k,j'}).
\end{align}
$Q_j(S; \vq)$ is actually the probability of the event that every arm in $\{\bar Y_{k,j'}\}_{k \in S, j' > j}$ does not sample a non-zero value.

Equipped with the above statement, we can establish the following lemma:

\begin{lemma}\label{lemma:r-q-r-formal}
For any $\vp$ and $\vq$ satisfying \Cref{eq:qstar-def}, we have for any $S \in \gS$, $$\bar r_q(S; \vq) = \bar r(S; \vp).$$
\begin{proof}
    Notice that by definition, we have
    \begin{align*}
        \bar r(S; \vp) = \E\left[\max_{i\in S} X_i^\vp\right],
    \end{align*}
    and
    \begin{align*}
        \bar r_q(S; \vq) = \E\left[\max_{i \in S} \max_{j \in [M]} Y_{i,j}^\vq\right].
    \end{align*}
    Notice that $\max_{j\in[M]}\{Y_{i,j}^\vq\}$ has the same distribution of $X_i^\vp$, we have
    \begin{align*}
        \bar r_q(S; \vq) &= \E\left[\max_{i \in S} \max_{j \in [M]} Y_{i,j}^\vq\right] \\
        &= \E\left[\max_{i \in S} X_i^\vp\right] = \bar r(S; \vp).
    \end{align*}
\end{proof}
\end{lemma}

\subsection{Biased Concentration}
We aim to use $\hat q_{i,j}^t$ to estimate $q_{i,j}^*$. However, this is a biased estimation. In this section, we carefully control the gap between the biased estimator $\hat q_{i,j}^t$ and the true probability $q^*_{i,j}$.


We set $c_t(i, j) := \mathbbm{1}[(i_t, j_t) = (i, j)]$ which is $\gF_t$-measurable. Then \Cref{alg} counts the summation of $c_t(i,j)$ as $C_t(i,j)$:
\begin{align*}
    C_t(i,j) = \sum_{\tau = 1}^t c_\tau(i,j),
\end{align*}
which is $\gF_{t-1}$-measurable. 

For given action $S_t$ in round $t$, the environment will sample a set of outcomes $\{X_i(t) \sim D_i : i \in S_t\}$. The value-index feedback is $r_t = \max_{i \in S_t} X_i(t)$, $i_t = \argmax_{i \in S_t} X_i(t)$. \Cref{alg} consider $j_t$ such that $r_t \in M_{j_t}$. We denote $I_t = \argmax_{i \in S_t} \bar X_i(t)$, where $\bar X_i(t)$ is the discretized of $X_i(t)$ induced by \Cref{eq:discretize-outcome}. Notice that under event $\gE_0$, $\argmax_{i \in S} X_i(t)$ is unique. But $I_t$ might be a set with multiple indices. We emphasize that $S_t$ is $\gF_{t-1}$ measurable and $(i_t, r_t, j_t, I_t)$ are $\gF_t$ measurable.

Then we can provide the following lemma.
\begin{lemma}\label{lemma:concentration-formal}
Under event $\gE_0$, we have for every $t \in [T]$ and $(i, j) \in [N] \times [M]$,
\begin{align*}
    \left|\hat q_{i,j}^t - q^*_{i,j}\right| \le \sqrt{8\frac{\log(NMt)}{SC_t(i,j)}} + (K-1)\cdot(L^4/j^2),
\end{align*}
with probability at least $1 - T^{-2}$, where we denote this good event as $\gE_1$. 

\begin{proof}
Denote $q_{i,j}(S_t) := \mathbbm{1}[i \in S_t] \cdot \Prob[(i_t, j_t) = (i, j) \mid j_t \le j, S_t]$, and $q_{i,j}^*(S_t) := \mathbbm{1}[i \in S_t] \cdot \Prob[I_t \ni i, j_t = j \mid j_t \le j, S_t]$. Therefore, for given $i \in [N], j \in [M]$, we have
\begin{align*}
    \E[\1[i \in S_t] \cdot c_t(i,j)\cdot \mathbbm{1}[j_t \le j] \mid  S_t] = q_{i,j}(S_t) \cdot \Prob[j_t \le j \mid S_t]
\end{align*}

By summation over time step $1, 2, \cdots, t$, we have
\begin{align*}
    % \E[C_t \mid j_t \le j] = 
    \sum_{\tau=1}^t \E[\1[i \in S_\tau]\cdot c_\tau(i,j)\cdot \mathbbm{1}[j_\tau \le j] \mid  S_\tau] &= \sum_{\tau=1}^t q_{i,j}(S_\tau)\cdot \Prob[j_\tau \le j \mid S_\tau] \\
    &= \sum_{\tau=1}^t \E\left[q_{i,j}(S_\tau)\cdot \mathbbm{1}[j_\tau \le j] \mid S_\tau\right],
\end{align*}
which implies that 
\begin{align*}
     \E\left[\sum_{\tau \le t, i \in S_\tau, j_\tau \le j}c_\tau(i,j) \middle| S_1, S_2, \cdots, S_t \right]  = \E\left[\sum_{\tau \le t, j_\tau \le j}
    q_{i,j}(S_\tau)\middle| S_1, \cdots, S_t\right]
\end{align*}
Notice that $S_t$ is $\gF_{t-1}$-measurable. By the definition of $q_{i,j}(S_\tau)$, we have
\begin{align*}
    \E\left[\sum_{\tau \le t, i \in S_\tau, j_\tau \le j}c_\tau(i,j) -
    q_{i,j}(S_\tau) \middle| \gF_{t-1} \right] = 0.
\end{align*}

If we count the number of $\tau$ that satisfies $i \in S_\tau$ and $j_\tau \le j$ is exactly $SC_t(i,j) = \sum_{\tau=1}^t \mathbbm{1}[i \in S_\tau, j_\tau \le j]$. Therefore, by Azuma-Hoeffding inequality, we have for fixed $SC_t(i,j)$, with probability at least $1 - \delta$,
\begin{align*}
    \left|\sum_{\tau \le t, i \in S_\tau, j_\tau \le j} c_\tau(i, j) - \sum_{\tau \le t, i \in S_\tau, j_\tau \le j} q_{i,j}(S_\tau)\right| \le \sqrt{2SC_t(i,j)\log(T/\delta)},
\end{align*}
By union inequality, we have 
\begin{align*}
    \left|\sum_{\tau \le t, i \in S_\tau j_\tau \le j} c_\tau(i, j) - \sum_{\tau \le t, i \in S_\tau, j_\tau \le j} q_{i,j}(S_\tau)\right| \le \sqrt{8SC_t(i,j)\log(NMT)}, 
\end{align*}
holds for any $t\in [T]$, $SC_t(i,j)$, and $(i, j) \in [N] \times [M]$ with probability at least $1 - T^{-2}$. We denote this good event as $\gE_1$ which satisfies $\Prob[\neg\gE_1] \le T^{-2}$.

We recall the definition of $\hat q_{i,j}^t$ given in \Cref{alg}
\begin{align*}
    \hat q_{i,j}^t = \frac{C_t(i,j)}{SC_t(i,j)} = \frac{\sum_{\tau \le t} c_{\tau}(i,j)}{SC_t(i,j)} = \frac{\sum_{\tau \le t} \1[i \in S_\tau]\cdot c_{\tau}(i,j)\mathbbm{1}[j_\tau \le j]}{SC_t(i,j)}.
\end{align*}
Under this good event $\gE_1$, we have for every $t \in [T]$ and $(i, j) \in [N] \times [M]$,
\begin{align*}
    \left| \hat q_{i,j}^t - \frac{\sum_{\tau \le t, i \in S_\tau, j_\tau \le j} q_{i,j}(S_\tau)}{SC_t(i,j)}\right| \le \sqrt{8\frac{\log(NMT)}{SC_t(i,j)}}
\end{align*}

Below we bound the difference between $q^*_{i,j}(S_t)$ and $q_{i,j}(S_t)$ for any $S_t \in \gS$. For given $(i, j)$ with $i \in S_t$, we have
\begin{align*}
    q^*_{i,j}(S_t) - q_{i,j}(S_t) &= \Prob[I_t \ni i, j_t = i \mid j_t \le j, S_t] - \Prob[i_t = i, j_t = j \mid j_t \le j, S_t] \\
    &= \Prob[I_t \ni i, i_t \neq i, j_t = j \mid j_t \le j, S_t] \\
    &\le \sum_{k \in S_t, k \neq i}\frac{\Prob[X_i \in M_j]\Prob[X_k \in M_j]}{\Prob[X_i \in M_{\le j}]\Prob[X_k \in M_{\le j}]} \\
    &\le (K-1)\cdot \frac{(L\epsilon)^2}{(j\epsilon/L)^2} = (K-1)\cdot L^4/j^2,
\end{align*}
where the last inequality holds by \Cref{ass:bi-lipschitz} and $\Prob[X_i \in M_{\le j}] = \sum_{j'=1}^j p_{i,j}^* \le j\frac{\epsilon}{L}, \forall i \in [N]$. 

Notice that for every $S_t \in \gS$ and $i \in S_t, j \in [M]$, we have
\begin{align*}
    q_{i,j}^*(S_t) &=
    \Prob[I_t \ni i, j = j_t \mid j_t \le j, S_t] \\
    &= \frac{\Prob[I_t \ni i, j = j_t \mid S_t]}{\Prob[j_t \le j \mid S_t]} \\
    &= \frac{\Prob[X_i(t) \in M_{j} \And x_k(t) \in M_{\le j}, \forall k \in S_t \mid S_t]}{\Prob[x_k(t) \in M_{\le j}, \forall k \in S_t \mid S_t]} \\
    &= \frac{\Prob[X_i \in M_j]\cdot \Prob[X_k \in M_{\le j}, \forall k \in S, k \neq i]}{\Prob[X_i \in M_{\le j}] \cdot \Prob[X_k \in M_{\le j}, \forall k \in S, k \neq i]} \\
    &= \frac{\Prob[X_i \in M_j]}{\Prob[X_i \in M_{\le j}]} \\
    &= \Prob[X_i \in M_j \mid X_i \in M_{\le j}]\\
    &= q_{i,j}^*.
\end{align*}
Therefore, we have 
\begin{align*}
    \left|\hat q_{i,j}^t - q^*_{i,j}\right| = \left|\hat q_{i,j}^t - \frac{\sum_{\tau \le t, i \in S_\tau j_\tau \le j}q_{i,j}^*(S_\tau)}{SC_t(i,j)}\right| \le \sqrt{8\frac{\log(NMt)}{SC_t(i,j)}} + (K-1)\cdot(L^4/j^2)
\end{align*}
\end{proof}
\end{lemma}

\subsection{Optimistic Estimation}

\begin{lemma}
\label{lemma:optimism}
    For $\beta_{i,j}^t$ given in \Cref{eq:def-beta}, under event $\gE_0$ and $\gE_1$, we have
    \begin{align*}
        \bar q_{i,j}^t \ge q^*_{i,j}.
    \end{align*}
    Moreover, by the offline $(1-\epsilon)$-approximated optimization oracle PTAS \citep{chen2013combinatorial}, we have
    \begin{align*}
        \bar r_q(S_t, \bar \vq^t) \ge (1-\epsilon) \cdot \bar r_q(S^*; \bar \vq^t).
    \end{align*}
\begin{proof}
Notice that in \Cref{alg} we define
\begin{align*}
    \bar q_{i,j}^t = \min\left\{\hat q_{i,j}^t + \beta_{i,j}^t + \frac{(K-1)L^4}{j^2}, 1\right\}.
\end{align*}
By \Cref{lemma:concentration-formal}, we have under $\neg\gE_0$ and $\gE_1$,
\begin{align*}
    \hat q_{i,j}^t \ge q_{i,j}^* - \beta_{i,j}^t - \frac{(K-1)L^4}{j^2},
\end{align*}
where the inequality holds by the definition of $\beta_{i,j}^t$ in \Cref{eq:def-beta} and $SC_{t-1}(i,j) \le SC_t(i,j)$.
Since $q_{i,j}^* \le 1$, we have
\begin{align*}
    \bar q_{i,j}^t \ge q^*_{i,j}.
\end{align*}

Since in \Cref{alg}, we set action $S_t \leftarrow \operatorname{PTAS}(\hat \vp^t)$ where $\hat\vp^t$ is converted from $\hat\vq^t$ by \Cref{eq:qstar-def}. Then by \Cref{lemma:r-q-r,lemma:monotone}, we have
\begin{align*}
    \bar r_q(S_t; \bar\vq^t) = \bar r(S_t; \bar\vp^t)  \ge (1-\epsilon)\max_{S \in \gS} \bar r(S; \bar \vp^t) \ge (1-\epsilon)\bar r(S^*; \bar \vp^t) = (1-\epsilon) \bar r_q(S^*; \bar \vq^t).
\end{align*}
\end{proof}
\end{lemma}

\subsection{Regret Decomposition}

\begin{lemma}
\label{lemma:tpm-formal}
Denote $Q_j^*(S_t) :=  \prod_{k \in S_t, j' > j} (1 - q_{k,j'}^*)$. We have
\begin{align}
    |\bar r_q(S_t; \bar \vq^t) - \bar r_q(S_t; \vq^*)| \le 2\sum_{i \in S_t, j \in [M]} Q_j^*(S_t) \cdot v_j \cdot \left|\bar q_{i,j}^t - q^*_{i,j}\right|.
\end{align}
\begin{proof}
This lemma is given by directly apply Lemma 3.3 in \citet{wang2023combinatorial} by definition of $\bar r_q$ in \Cref{eq:def-rq}.
\end{proof}
\end{lemma}

\begin{lemma}\label{lemma:regret-decomp}
Under \Cref{ass:bi-lipschitz}, we can bound the regret of \Cref{alg} by
\begin{align*}
    \gR(T) \le \E\left[\sum_{t=1}^T \texttt{Bonus}_t + \texttt{Bias}_t\middle| \gE_0, \gE_1\right] + 3T\epsilon + T^{-1},
\end{align*}
where $\texttt{Bonus}_t$ and $\texttt{Bias}_t$ is defined by
\begin{align}\label{eq:def-bonus-t}
    \texttt{Bonus}_t := 4 \sum_{i \in S_t, j \in [M]} Q_{j}^{*}(S_t) \cdot v_j \cdot \beta_{i,j}^t,
\end{align}
and
\begin{align}\label{eq:def-bias-t}
    \texttt{Bias}_t := 4 \sum_{i \in S_t, j \in [M]} Q_{j}^{*}(S_t) \cdot v_j \cdot (K-1)\frac{L^4}{j^2}.
\end{align}
\begin{proof}
This lemma formalize the first three steps of proof sketch. Denote $\Delta_t := r^*(S^*) - r^*(S_t)$, we have 
\begin{align*}
    \gR(T) = \E\left[\Delta_t\right].
\end{align*}
By \Cref{lemma:discrete-error}, we have
\begin{align*}
    \Delta_t &\le  \bar r(S^*; \vp^*) - \bar r(S_t; \vp^*) + 2\epsilon.
\end{align*}
Then we have
\begin{align*}
    \gR(T) &\le \Prob[\gE_0] \cdot \E\left[\sum_{t=1}^T\Delta_t \middle| \gE_0\right] + \Prob[\neg\gE_0] \cdot T \\
    &\le \E\left[\sum_{t=1}^T\Delta_t \middle| \gE_0\right] \\
    &\le \E\left[\sum_{t=1}^T\bar r(S^*; \vp^*) - \bar r(S_t; \vp^*) \middle| \gE_0\right] + 2T\epsilon,
\end{align*}
where the first inequality holds by property of conditional expectations and $\Delta_t \le 1$ and the second inequality is due to $\Prob[\neg\gE_0] = 0$.

Notice that under $\gE_0$ and $\gE_1$, by \Cref{lemma:monotone,lemma:optimism}, we have
\begin{align*}
    \bar r_q(S_t; \vq^t) \ge (1-\epsilon)\bar r_q(S^*; \bar \vq_t) \ge (1-\epsilon)\bar r_q(S^*; \vq^*).
\end{align*}
Then with \Cref{lemma:r-q-r-formal}, we have
\begin{align*}
    \gR(T) &\le \E\left[\sum_{t=1}^T\bar r_q(S^*; \vq^*) - \bar r_q(S_t; \vq^*) \middle| \gE_0\right] + 2T\epsilon \\
    &\le \E\left[\sum_{t=1}^T\bar r_q(S^*; \vq^*) - \bar r_q(S_t; \vq^*) \middle| \gE_0,\gE_1\right] + \Prob[\neg\gE_1]\cdot T +  2T\epsilon \\
    &\le \E\left[\bar r_q(S_t; \vq^t) - \bar r_q(S_t; \vq^*)\right] + 3T\epsilon + T^{-1},
\end{align*}
where the last inequality holds by $\epsilon\bar r_q(S^*;\vp^*) \le \epsilon$ and $\Prob[\neg\gE_1] \le T^{-2}$ shown in \Cref{lemma:concentration-formal}.  

Therefore, applying \Cref{lemma:tpm}, we get
\begin{align*}
    \gR(T) \le \E\left[\sum_{t=1}^T \texttt{Bonus}_t + \texttt{Bias}_t\middle| \gE_0, \gE_1\right] + 3T\epsilon + T^{-1},
\end{align*}
where $\texttt{Bonus}_t$ and $\texttt{Bias}_t$ is defined in \Cref{eq:def-bonus-t,eq:def-bias-t}.
\end{proof}
\end{lemma}

\subsection{Bounding the Bonus Terms}

We apply similar methods in \citet{wang2017improving,liu2023contextual} to give the bounds of $\sum_t\texttt{Bonus}_t$. We first give the following definitions.

\begin{definition}[{\citet[Definition 5]{wang2017improving}}]\label{def:TPgroup}
    Let $(i,j) \in [N] \times [M]$ be the index of binary arm and $l$ be a positive natural number, define the triggering probability group (of actions)
    \[
    S_j^l = \{S \in \mathcal{S} \mid 2^{-l} < Q_j^*(S) \leq 2^{-l+1}\}.
    \]
    Notice $\{S_j^l\}_{l \geq 1}$ forms a partition of $\{S \in \mathcal{S} \mid Q_j^*(S) > 0\}$.
\end{definition}
\begin{definition}[{\citet[Definition 6]{wang2017improving}}]\label{def:TPcounter}
    For each group $S_j^l$ (\Cref{def:TPgroup}), we define a corresponding counter $N_{i,j}^l$. 
    In a run of a learning algorithm, the counters are maintained in the following manner. 
    All the counters are initialized to $0$. In each round $t$, if the action $S_t$ is chosen, then update $N^l(i,j)$ to $N^l(i,j) + 1$ for every $(i, j)$ that $i \in S_t$, $S_t \in S_j^l$. 
    Denote $N_t^l({i,j})$ at the end of round $t$ with $N^l(i,j)$. 
    In other words, we can define the counters with the recursive equation below:
    \begin{align*}
        N_t^l(i,j) =
        \begin{cases} 
            0, & \text{if } t = 0, \\
            N_{t-1}^l(i,j) + 1, & \text{if } t > 0, i\in S_t, S_t \in S_j^l, \\
            N_{t-1}^l(i,j), & \text{otherwise}.
        \end{cases}
    \end{align*}
\end{definition}
\begin{definition}[{\citet[Definition 7]{wang2017improving}}]\label{def:TPevent}
Given a series of integers $\{l_{i,j}^{\max}\}_{i \in [N],j\in[M]}$, we say that the triggering is nice at the beginning of round $t$ (with respect to $l_{i,j}^{\max}$), if for every group $S_j^l$(\Cref{def:TPgroup}) identified by binary arm $(i,j)$ and $1 \leq l \leq l_{i,j}^{\max}$, as long as 
\[
\sqrt{\frac{8 \log (NMT)}{\frac{1}{3} N_{t-1}^l(i,j)\cdot 2^{-l}}} \leq 1,
\]
there is $SC_{t-1}(i,j) \geq \frac{1}{3} N_{t-1}^l(i,j) \cdot 2^{-l}$. We denote this event with $\gE_2(t)$. It implies
\[
\beta_{i,j}^t = \sqrt{\frac{8 \log(NMT)}{ SC_{t-1}(i,j)}} \leq \sqrt{\frac{8 \log(NMT)}{\frac{1}{3} N_{t-1}^l(i,j) \cdot 2^{-l}}}.
\]
\end{definition}
Therefore, we show that $\gE_2(t)$ happens with high probability for every $t$. 
\begin{lemma}[{\citet[Lemma 4]{wang2017improving}}]\label{lemma:TPprob}
For a series of integers $\{l_{i,j}^{\max}\}_{i \in [N],j\in[M]}$, $$\Prob[\neg \mathcal{E}_2(t)] \leq \sum_{i \in [N],j\in[M]} l_{i,j}^{\max} t^{-2},$$
for every round $t \geq 1$. 

\begin{proof}
We prove this lemma by showing $\Prob[N_{t-1}^l(i,j) = s, SC_{t-1}(i,j) \leq \frac{1}{3} N_{t-1}^l(i,j) \cdot 2^{-l}] \leq t^{-3}$, for any fixed $s$ with $0 \leq s \leq t - 1$ and $\sqrt{\frac{8 \log(NMT)}{\frac{1}{3} s \cdot 2^{-l}}} \leq 1$. Let $t_k$ be the round that $N^l(i,j)$ is increased for the $k$-th time, for $1 \leq k \leq s$. Let $Z_k = \1[S_{t_k} \ni i, j_{t_k} \le j]$ be a Bernoulli variable, that is, $SC_{t_k}(i,j)$ increase in round $t_k$. When fixing the action $S_{t_k}$, $Z_k$ is independent from $Z_1, \ldots, Z_{k-1}$. Since $S_{t_k} \in S_j^l$, $\mathbb{E}[Z_k \mid Z_1, \ldots, Z_{k-1}] \geq 2^{-l}$. Let $Z = Z_1 + \cdots + Z_s$. By multiplicative Chernoff bound \citep{upfal2005probability}, we have
\[
\Prob\left\{Z \leq \frac{1}{3} s \cdot 2^{-l}\right\} \leq \exp\left(-\frac{\left(\frac{2}{3}\right)^2 s \cdot 2^{-l}}{2}\right) \leq \exp\left(-\frac{\left(\frac{2}{3}\right)^2 18 \log t}{2}\right) < \exp(-3 \log t) = t^{-3}.
\]

By the definition of $SC_{t-1}(i,j)$ and the condition $N_{t-1}^l(i,j) = s$, we have $SC_{t-1}(i,j) \geq Z$. Thus
\begin{align*}
\Prob[N_{t-1}^l(i,j) = s, SC_{t-1}(i,j) &\leq \frac{1}{3} N_{t-1}^l(i,j) \cdot 2^{-l}]\\
&\leq \Prob[N_{t-1}^l(i,j) = s, Z \leq \frac{1}{3} s \cdot 2^{-l}] \\
&\leq \Prob[Z \leq \frac{1}{3} s \cdot 2^{-l}] \leq t^{-3}.
\end{align*}

By taking $i,j$ over $[N]\times[M]$, $l$ over $1, \ldots, l_{i,j}^{\max}$, $s$ over $0, \ldots, t - 1$ and applying the uninon bound, the lemma holds.
\end{proof}
\end{lemma}

\begin{lemma}\label{lemma:bonus-t-bound}
For given constant $C$, we have
\begin{align*}
    \sum_{t=1}^T \texttt{Bonus}(t) \le 16NM + 12288\frac{KNM^2\log(NMT)}{C} + TC + \frac{\pi^2}{6} \left\lceil \log_2\frac{16KM}{C} \right\rceil.
\end{align*}
\begin{proof}
For given constant $C$, we can define the following notations.
\begin{align}\label{eq:def-lmax}
    l_{i,j}^{\max} := \left\lceil \log_2 \frac{16KM}{C} \right\rceil, \quad \forall (i, j) \in [N] \times [M],
\end{align}
and for every integer $l$,
\begin{equation}\label{eq:def-kappa}
\begin{aligned}
    \kappa_{l,T}(C,s) := \begin{cases}
        2\cdot 2^{-l} & s =0 \\
        \sqrt{{96 \cdot 2^{-l} \log(NMT)}/s} & 1 \le s \le B_{l,T}(C) \\
        0 & s > B_{l,T}(C)
    \end{cases},
\end{aligned}
\end{equation}
where $B_{l,T}(C)$ is given by
\begin{align}\label{eq:def-BlT}
    B_{l,T}(C) := \left\lfloor{6144 \cdot 2^{-l}K^2M^2\log(NMT)}/{C^2}\right\rfloor.
\end{align}
By \citet[Lemma 5]{wang2017improving}, if $\texttt{Bonus}(t) \ge C$, under event $\gE_2(t)$, we have
\begin{align*}
    \texttt{Bonus}(t) \le \sum_{i \in S_t, j \in [M]} \kappa_{l_{i,j}, T}(C, N_{t-1}^{l_i}(i,j)),
\end{align*}
where $l_{i,j}$ is the index of group $S_j^{l_{i,j}} \ni S_t$. This is because we have
\begin{align*}
    \texttt{Bonus}(t) &\le -C + 8 \sum_{i \in S_t, j \in [M]} Q_j^*(S_t) \cdot (j - 1)\epsilon \cdot \min\{\beta_{i,j}^t, 1\} \\
    &\le 8 \sum_{i \in S_t, j \in [M]} \left(Q_j^*(S_t) \cdot \min\{\beta_{i,j}^t, 1\} - \frac{C}{8KM}\right)
\end{align*}

\noindent \textbf{Case 1: $1\le l_{i,j} \le l_{i,j}^{\max}$.} We have 
\begin{align*}
    Q^*_{j}(S_t) \le 2 \cdot 2^{-l_{i,j}}.
\end{align*}
Under $\gE_2(t)$, we have
\begin{align*}
    \min \left\{\beta_{i,j}^t, 1\right\} = \min \left\{\sqrt{\frac{8 \log(NMT)}{ SC_{t-1}(i,j)}},1\right\} \leq \min\left\{\sqrt{\frac{8 \log(NMT)}{\frac{1}{3} N_{t-1}^{l_{i,j}}(i,j) \cdot 2^{-l_{i,j}}}}, 1\right\},
\end{align*}
and
\begin{equation}\label{eq:Qbeta-bound}
\begin{aligned}
    Q^*_{j}(S_t)\cdot \min \left\{\beta_{i,j}^t,1\right\} &\le 2 \cdot 2^{-l_{i,j}} \cdot \min\left\{\sqrt{\frac{8 \log(NMT)}{\frac{1}{3} N_{t-1}^{l_{i,j}}(i,j) \cdot 2^{-l_{i,j}}}}, 1\right\} \\
    &\le \min\left\{\sqrt{\frac{96 \cdot 2^{-l_{i,j}} \log(NMT)}{ N_{t-1}^{l_{i,j}}(i,j) }}, 2 \cdot 2^{-l_{i,j}}\right\}.
\end{aligned}
\end{equation}
If $N_{t-1}^{l_{i,j}}(i,j) \ge B_{l_{i,j},T}(C) + 1$, then
\begin{align*}
    \sqrt{\frac{96 \cdot 2^{-l_{i,j}} \log(NMT)}{ N_{t-1}^{l_{i,j}}(i,j) }} \le \frac{C}{8KM},
\end{align*}
which implies $Q^*_{j}(S_t)\cdot \min \left\{\beta_{i,j}^t,1\right\} - C/8KM \le 0$. 

If $N_{t-1}^{l_{i,j}}(i,j) = 0$, we have $Q^*_{j}(S_t)\cdot \min \left\{\beta_{i,j}^t,1\right\} \le Q^*_{j}(S_t) \le 2\cdot 2^{-l_{i,j}}$, which implies
\begin{align*}
    Q^*_{j}(S_t)\cdot \min \left\{\beta_{i,j}^t,1\right\} - \frac{C}{8KM} \le \kappa_{l_{i,j},T}(C, 0)
\end{align*}

Otherwise, for $1 \le N_{t-1}^{l_{i,j}}(i,j) \le B_{l_{i,j}, T}(C)$, we have $Q^*_{j}(S_t)\cdot \min \left\{\beta_{i,j}^t,1\right\} \le \kappa_{l_{i,j},T}(C, N_{t-1}^{l_{i,j}}(i,j))$ by \Cref{eq:Qbeta-bound,eq:def-kappa}. Therefore, we get
\begin{align*}
    Q^*_{j}(S_t)\cdot \min \left\{\beta_{i,j}^t,1\right\} - \frac{C}{8KM} \le \kappa_{l_{i,j},T}(C, N_{t-1}^{l_{i,j}}(i,j))
\end{align*}

\noindent \textbf{Case 2: $l_{i,j} \ge l_{i,j}^{\max} + 1$.} We have
\begin{align*}
    Q^*_{j}(S_t)\cdot \min \left\{\beta_{i,j}^t,1\right\} \le 2 \cdot 2^{-l_{i,j}} \le 2 \cdot \frac{C}{16KM} \le \frac{C}{8KM},
\end{align*}
which shows that $Q^*_{j}(S_t)\cdot \min \left\{\beta_{i,j}^t,1\right\} - C/8KM \le 0$. If $N_{t-1}^{l_{i,j}}(i,j) = 0$. Therefore, we finally get
\begin{align*}
    \texttt{Bonus}(t) \le 8\sum_{i\in S_t, j \in [M]} \kappa_{l_{i,j}, T}\left(C, N_{t-1}^{l_{i,j}}(i,j)\right),
\end{align*}
for the case of good event $\gE_2(t)$ happens and $\texttt{Bonus}(t) \ge C$.

Notice that under good events $\gE_0, \gE_1$, we have
\begin{align*}
    \sum_{t=1}^T \texttt{Bonus}(t) &\le \sum_{t=1}^T \1[\{\texttt{Bonus}(t) \ge C\} \cap \gE_2(t)]\cdot \texttt{Bonus}(t) + T \cdot C + \sum_{t=1}^T \Prob[\gE_2(t)] \\
    &\le \underbrace{\sum_{t=1}^T 8 \cdot \sum_{i\in S_t, j \in [M]} \kappa_{l_{i,j}, T}\left(C, N_{t-1}^{l_{i,j}}(i,j)\right)}_{(I)} + TC + \frac{\pi^2}{6} \cdot \max_{i\in [N], j \in [M]} l_{i,j}^{\max} .
\end{align*}
where the first inequality is due to $\texttt{Bonus}(t) \le 1$ and definition, and the second one is due to \Cref{lemma:TPprob}. The key is bounding $(I)$:
\begin{align*}
    (I) &= 8\cdot \sum_{i \in [N], j \in [M]} \sum_{l=1}^\infty \sum_{s=0}^{N_{T-1}^{l}(i,j)}\kappa_{l}(C, s) \\
    &= 8 \cdot \sum_{i \in [N], j \in [M]} \sum_{l=1}^\infty \left(2\cdot 2^{-l} +  \sum_{s=1}^{B_{l,T}(C)}\sqrt{\frac{96 \cdot 2^{-l} \log(NMT)}{ s }} \right) \\
    &\le 8\cdot  \sum_{i \in [N], j \in [M]} \sum_{l=1}^\infty \left(2\cdot 2^{-l} +  2\cdot \sqrt{96 \cdot 2^{-l} \log(NMT)}\cdot \sqrt{B_{l,T}(C)} \right) ,
\end{align*}
where the inequality holds by the fact that $\sum_{s=1}^n \sqrt{1/s} \le 2\sqrt{n}$. Therefore, by the definition of $B_{l,T}(C)$ in \Cref{eq:def-BlT}, we have
\begin{align*}
    (I) &\le  8\cdot  \sum_{i \in [N], j \in [M]} \sum_{l=1}^\infty \left(2\cdot 2^{-l} +  1536\cdot \frac{2^{-l}KM\log(NMT)}{C} \right) \\
    &= 8\cdot  \sum_{i \in [N], j \in [M]}  \left(2+  1536\cdot \frac{KM\log(NMT)}{C} \right)\cdot \left(\sum_{l=1}^\infty 2^{-l}\right) \\
    &\le 16NM + 12288\frac{KNM^2\log(NMT)}{C}.
\end{align*}
Therefore, we get
\begin{align*}
    \sum_{t=1}^T \texttt{Bonus}(t) \le 16NM + 12288\frac{KNM^2\log(NMT)}{C} + TC + \frac{\pi^2}{6} \left\lceil \log_2\frac{16KM}{C} \right\rceil.
\end{align*}
\end{proof}
\end{lemma}


\subsection{Bounding the Bias Terms}
\begin{lemma}\label{lemma:bias-t-bound}
Under \Cref{ass:bi-lipschitz}, we have
\begin{align*}
    \sum_{t=1}^T \texttt{Bias}(t) \le 4K^2L^4 T\epsilon \log(M + 1).
\end{align*}
\begin{proof}
Notice that $\sum_{j\in[M]} 1/j \le \log(M+1)$ for $\epsilon < 1/2$, we have
\begin{align*}
    \texttt{Bias}(t) &\le 4K \cdot \sum_{i \in S_t,j \in [M]}Q^*_{j}(S_t) \cdot \epsilon L^4/j \\
    &= 4K^2L^4\epsilon \cdot \sum_{j \in [M]} \frac{1}{j} \\
    &\le 4K^2L^4 \epsilon\log(M + 1).
\end{align*}
Therefore, we have
\begin{align*}
    \sum_{t=1}^T \texttt{Gap}(t) \le 4K^2L^4 T\epsilon \log(M+1).
\end{align*}
\end{proof}
\end{lemma}



\subsection{Proof of \Cref{thm:main}}
\begin{theorem}[Formal version of \Cref{thm:main}]\label{thm:main-formal}
By setting $\beta_{i,j}^t$ in \Cref{eq:def-beta} and $\epsilon < 1/2$, we can control the regret of \Cref{alg} under \Cref{ass:bi-lipschitz} by 
\begin{align*}
    \gR(T) &\le  12289 \sqrt{NKM^2T\log(NMT)} +  T\epsilon\left(4KL^4\log(M+1) + 3\right)\\
    &\quad + 16 NM +  \pi^2\left(\log_2(\sqrt{KM^2T\log(NMT)/N}) + 5\right)/6 +  T^{-1} \\
    &= \wt{O}\left(\sqrt{NKM^2T} + L^4K^2T\epsilon\right),
\end{align*}
where $M = \ceil{1/\epsilon}$. If we further take $\epsilon = O\left(L^{-2}K^{-\frac{3}{4}}N^{\frac
{1}{4}}T^{-\frac{1}{4}}\right)$, we have
\begin{align*}
    \gR(T) = \wt{\gO}(L^{2}N^{\frac{1}{4}}K^{\frac{5}{4}}T^{\frac{3}{4}}).
\end{align*}
\begin{proof}
By \Cref{lemma:regret-decomp}, we have
\begin{align*}
    \gR(T) \le \E\left[\sum_{t=1}^T \texttt{Bonus}_t + \texttt{Bias}_t\middle| \gE_0, \gE_1\right] + 3T\epsilon + T^{-1},
\end{align*}
Take constant $C$ as
\begin{align}\label{eq:def-C}
    C:= \sqrt{\frac{NKM^2\log(NMT)}{T}}.
\end{align}
Then \Cref{lemma:bonus-t-bound} shows that
\begin{align*}
    \sum_{t=1}^T \texttt{Bonus}(t) \le 16NM + 12289\sqrt{NM^2KT\log(NMT)} + \pi^2\left(\log_2(\sqrt{KMT\log(NMT)/N}) + 5\right)/6.
\end{align*}
\Cref{lemma:bias-t-bound} demonstrates that
\begin{align*}
    \sum_{t=1}^T \texttt{Bias}(t) \le 4K^2L^4 T\epsilon \log(M + 1).
\end{align*}
Therefore, by calculating the summation of the bonus and bias terms, we can bound the regret by
\begin{align*}
    \gR(T) &\le \E\left[\sum_{t=1}^T \texttt{Bonus}(t) + \texttt{Gap}(t) \middle| \gE_0, \gE_1\right] +  T^{-1} + 3T\epsilon \\
    &\le  12289 \sqrt{NKM^2T\log(NMT)} +  T\epsilon\left(4KL^4\log(M+1) + 3\right)\\
    &\quad + 16 NM +  \pi^2\left(\log_2(\sqrt{KM^2T\log(NMT)/N}) + 5\right)/6 +  T^{-1} \\
    &= \wt{O}\left(\sqrt{NKM^2T} + L^4K^2T\epsilon\right),
\end{align*}
which finishes the proof.
\end{proof}
\end{theorem}


\section{Omitted proofs in \Cref{sec:kminexp}}\label{Appendix:k-min}
This proof mainly applies the techniques for general linear bandits \citet{liu2024almost, lee2024unified}. Given action $S \in \gS$ to the environment, we assume that $\ell_S$ is the random variable of the loss, i.e., $\E[\ell_S] = \ell^*(S)$. 

We have
\begin{align*}
    g_t(\theta; \lambda) :&= - \nabla_\theta \gL_t(\theta; \lambda) + \sum_{i < t} \ell_i \psi(S_i) \\
    &=  \sum_{i < t}  \frac{1}{\psi(S_i)^\top \theta} \cdot \psi(S_i) - \lambda \theta. \\
    H_t(\theta;\lambda) :&= \nabla^2_\theta \gL(\theta; \gH_t)\\
    &=  - \nabla_\theta g_t(\theta; \lambda)  \\
    &= \lambda I + \sum_{i < t} \frac{\psi(S_i)\psi(S_i)^\top}{(\psi(S_i)^\top\theta)^2}.
\end{align*}
\subsection{Concentration Argument for MLE}
\begin{lemma}[MLE Concentration]
\label{lemma:mle-concentration}
For $L^* := \sup_{S \in \gS} \ell^*(S)$, $M_1:= L^*/\sqrt{2}$, and $V = \sup\{\|\theta\|_2 : \theta \in \Theta\}$, set
\begin{align}\label{eq:def-lambda}
    \lambda_t := \max \left\{1, \frac{2dM_1}{V}\cdot \log\left(e\sqrt{1 + \frac{tL^*}{d}} + \frac{1}{\delta}\right)\right\},
\end{align}
and 
\begin{align} \label{eq:def-gamma}
    \gamma_t(\delta, \lambda_t) := \sqrt{\lambda_t}\left(\frac{1}{2M_1} + V\right) + \frac{2M_1d}{\sqrt{\lambda_t}}\left(\log(2) + \frac{1}{2}\log\left(1 + \frac{tL^*}{\lambda_t d}\right)\right) + \frac{2M_1}{\sqrt{\lambda_t}}\log(1/\delta).
\end{align}
Then we have with probability at least $1 - \delta$, 
\begin{align}\label{eq:def-confidence-set}
    \theta^* \in C_t(\hat\theta_t; \delta,\lambda_t) := \left\{ \theta \in \Theta : \left\| g_t(\theta; \lambda_t) - g_t(\hat\theta_t; \lambda_t) \right\|_{H_t^{-1}(\theta; \lambda_t)} \le \gamma_t(\delta, \lambda_t) \right\},
\end{align}
holds for any $t \in [T]$. We denote the confidence set as $C_t(\hat\theta_t; \delta, \lambda_t)$ and this good event as $\Xi$.

\begin{proof}
For simplicity, we denote the filtration of history as  $\gH_t := \left( S_1, Y_1, \cdots, S_{t-1}, Y_{t-1}, S_t \right)$. Then we have
\begin{align*}
    \ell_t \sim \exp(\psi(S_t)^\top \theta^*), \quad \E[\ell_t \mid \gH_t] = \frac{1}{\psi(S_t)^\top\theta^*},
\end{align*}
by the property of exponential distribution and definition of $\psi(S)$. Since we have
\begin{align*}
    \hat\theta_t \leftarrow \argmin_{\theta \in \R^d} \gL_t(\theta; \lambda_t),
\end{align*}
by \Cref{alg:k-min}. Then by KKT condition, we have
\begin{align*}
   \left.\frac{\partial \gL_t(\theta ; \lambda_t)}{\partial \theta} \right|_{\theta = \hat\theta_t} = 0 \Rightarrow g_t(\hat\theta_t; \lambda_t) - \sum_{i < t} \ell_i\psi(S_i) = 0
\end{align*}
Notice that by definition of $g_t$, 
\begin{align*}
    g_t(\theta^*; \lambda_t) =  \sum_{i < t}  \frac{1}{\psi(S_i)^\top \theta^*} \cdot \psi(S_i) - \lambda_t \theta^*.
\end{align*}
Denote $\varepsilon_t := \ell_t - \E[\ell_t \mid \gH_t] = \ell_t - {1}/({\psi_t(S_t)^\top\theta^*})$, we have
\begin{align*}
    g_t(\hat \theta_t; \lambda_t) - g_t(\theta^*; \lambda_t) = \sum_{i < t} \varepsilon_i \psi(S_i) + \lambda_t \theta^*.
\end{align*}
Fix $s \ge 0$, we have
\begin{align*}
    \E[\exp(s\varepsilon_t) \mid \gH_t] &= \E\left[\exp\left(s\ell_t - \frac{s}{\psi_t(S_t)^\top\theta^*}\right)\right] \\
    &= \exp\left(- \frac{s}{\psi_t(S_t)^\top\theta^*}\right)\cdot \E\left[\exp(s\ell_t)\mid \gH_t\right],
\end{align*}
and by calculation,
\begin{align*}
    \E[\exp(s\varepsilon_t) \mid \gH_{t-1}] &= \exp\left(- \frac{s}{\psi(S_t)^\top\theta^*}\right)\cdot \E\left[\exp(s\ell_t)\mid \gH_t\right] \\
    &=\exp\left(- \frac{s}{\psi(S_t)^\top\theta^*}\right)\cdot \int_{([0,+\infty)} \psi(S_t)^\top\theta^*\exp(-(\psi(S_t)^\top\theta^* - s)y)dy \\
    &= \exp\left(-\frac{1}{\ell_t^*}s + \log(\ell_t^*) - \log(\ell_t^* - s)\right),
\end{align*}
where we use $\ell_t^* := \psi_t(S_t)^\top\theta^*$ for simplicity. Consider the case for $s < \ell_t^*$, by intermediate value theorem, we have
\begin{align*}
    \log(\ell_t^*) - \log(\ell_t^* - s) = s \cdot \frac{1}{\ell_t^*} - \frac{s^2}{2\xi^2},
\end{align*}
for some $\xi \in [\ell_t^* - s, \ell_t^*]$. We further denote
\begin{equation}\label{eq:def-Lstar}
    L^* = \sup_{S \in \gS} \ell^*(S).
\end{equation}
Therefore, we can set constant $0 \le s \le L^*$, which gives
\begin{align*}
    \log(\ell_t^*) - \log(\ell_t^* - s) \leq s \cdot \frac{1}{\ell_t^*} - \frac{s^2}{(\ell_t^*)^2}.
\end{align*}
Denote $\nu_{t-1} := -1/{(\psi(S_t)^\top \theta^*)^2}$. We have for some constant $M_1 \ge L^*/\sqrt{2}$, and $|s| \le 1/M_1$,
\begin{align*}
    \E[\exp(s\varepsilon_t) \mid \gH_t] \le \exp(s^2 \nu_{t-1}).
\end{align*}
Applying \citet[Theorem 2]{janz2024exploration} with $\mS_t := \sum_{i<s} \varepsilon_i \psi(S_i)$, we can show that with probability at least $1 - \delta$,
\begin{align*}
    \left\|g_t(\hat\theta_t; \lambda_t) - g_t(\theta^*; \lambda_t)\right\|_{H_t^{-1}(\theta^*;\lambda_t)} &\le \left\|\sum_{i<t}\varepsilon_i \psi_i(S_i)\right\|_{H_t^{-1}(\theta^*;\lambda_t)} + \lambda_t \left\|\theta^*\right\|_{H_t^{-1}(\theta^*;\lambda_t)} \\
    &\le \frac{\sqrt{\lambda_t}}{2M_1} + \frac{2M_1}{\sqrt{\lambda_t}}\log\left(\frac{\det(H_t(\theta^*)^{1/2}/\lambda_t^{d/2})}{\delta}\right) + \frac{2M_1}{\sqrt{\lambda_t}}d\log(2) + \sqrt{\lambda_t} V,
\end{align*}
where $V = \sup\{\|\theta\|_2 : \theta \in \Theta\}$. Moreover, by definition of $H_t(\theta^*; \lambda_t)$, we have
\begin{align*}
    \det(H_t(\theta^*; \lambda_t))/\lambda_t^d \le \left(1 + \frac{tL^*}{\lambda_t d}\right)^d,
\end{align*}

Therefore, for
\begin{align*}
    \gamma_t(\delta, \lambda_t) \ge \sqrt{\lambda_t}\left(\frac{1}{2M_1} + V\right) + \frac{2M_1d}{\sqrt{\lambda_t}}\left(\log(2) + \frac{1}{2}\log\left(1 + \frac{tL^*}{\lambda_t d}\right)\right) + \frac{2M_1}{\sqrt{\lambda_t}}\log(1/\delta),
\end{align*}
we have with probability at least $1 - \delta$, 
\begin{align*}
    \theta^* \in C_t(\hat\theta_t; \delta,\lambda_t) := \left\{ \theta \in \Theta : \left\| g_t(\theta; \lambda_t) - g_t(\hat\theta_t; \lambda_t) \right\|_{H_t^{-1}(\theta; \lambda_t)} \le \gamma_t(\delta, \lambda_t) \right\},
\end{align*}
holds for any $t \in [T]$.
\end{proof}
\end{lemma}


\subsection{Proof of \Cref{thm:kminexp}}
\begin{theorem}[Formal version of \Cref{thm:kminexp}]
\label{thm:kminexp-formal}
By setting $\delta = 1/T$, $\gamma_t(\delta)$ according to \Cref{eq:def-gamma}, and $\lambda_t$ according to \Cref{eq:def-lambda}, \Cref{alg:k-min} enjoys the following regret guarantee: 
\begin{align*}
    \gR(T) &\le 
    16\gamma \cdot \sqrt{dT}\cdot \sqrt{(\ell^*(S^*))^2(1 + L^*/\lambda) \cdot \log\left(1 + L^*T/d\lambda\right)} \\
    &\quad + 256\gamma^2 \cdot dL^* \cdot \log\left(1 + L^*T/d\lambda\right) \cdot \left(\frac{\sup_{S \in \gS} (\psi(S)^\top \theta^*)}{\ell^*(S^*)^3} + 2\right) + 1,
\end{align*}
where $\gamma := \sup_t \gamma_t(\delta)$ and $\lambda_t := \inf_t \lambda_t$.
\begin{proof}
Since we have $X_i \sim \exp(\phi(i)^\top \theta^*)$, then
\begin{align*}
    \min_{i \in S} X_i \sim \exp\left(\sum_{i \in S} \phi(i)^\top \theta^* \right) = \exp\left(\psi(S)^\top \theta^* \right),
\end{align*}
which shows that 
\begin{align*}
    \ell^*(S) = \E\left[\min_{i \in S} X_i\right] = \frac{1}{\psi(S)^\top \theta^*}.
\end{align*}
Therefore, by second-order Taylor expansion, we have for some $\xi \in [\ell^*(S_t), \sup_t \ell^*(S_t)]$, 
\begin{align*}
    \gR(T) &= \E\left[\sum_{t=1}^T \ell^*(S_t) - \ell^*(S^*) \right]\\
    &\le   \Prob[\Xi] \cdot \E\left[\sum_{t=1}^T \frac{1}{\psi(S_t)^\top \theta^*} - \frac{1}{\psi(S^*)^\top \theta^*} \middle| \Xi \right]+ \Prob[\neg\Xi] \cdot T \\
    &\le \E\left[ \underbrace{\sum_{t=1}^T \frac{1}{(\psi(S_t)^\top \theta^*)^2} \cdot \left( \psi(S^*)^\top \theta^* - \psi(S_t)^\top \theta^*  \right)}_{\gR_1(T)} \middle| \Xi \right] +\E\left[ \underbrace{\sum_{t=1}^T \frac{2}{\xi^3} \cdot \left(\psi(S^*)^\top \theta^* - \psi(S_t)^\top \theta^* \right)^2 }_{\gR_2(T)} \middle| \Xi \right] + 1.
\end{align*}
Under $\Xi$, we have $\theta^* \in C_t(\hat\theta_t; \delta, \lambda_t)$ for every $t \in [T]$. Therefore, by \Cref{alg:k-min}, we have
\begin{align}
\label{eq:optimism}
    \psi(S^*)^\top \theta^* \le \psi(S_t)^\top \wt{\theta}_t.
\end{align}
Under $\Xi$, we have
\begin{align*}
    \gR_1(T) &\le \sum_{t=1}^T \frac{1}{(\psi(S_t)^\top \theta^*)^2} \cdot \psi(S_t)^\top (\theta^* - \wt{\theta}_t) \\
    &\le \sum_{t=1}^T \frac{1}{(\psi(S_t)^\top \theta^*)^2} \cdot \| \psi(S_t) \|_{H_t^{-1}(\theta^*; \lambda_t)} \cdot \left\|\theta^* - \wt{\theta}_t\right\|_{H_t^{-1}(\theta^*; \lambda_t)},
\end{align*}
where the first inequality is due to \Cref{eq:optimism} and the second holds by Cauchy-Schwartz inequality. Notice that $\wt{\theta}_t, \theta^* \in C_t(\hat\theta_t; \delta, \lambda_t)$ under $\Xi$, we have
\begin{align*}
    \left\|\theta^* - \wt{\theta}_t\right\|_{H_t^{-1}(\theta^*; \lambda_t)} \le 8\gamma_t(\delta, \lambda_t)
\end{align*}
by \citet[Lemma 30]{liu2024almost}. Denote $\gamma := \sup_{t \in [T]} \gamma_t(\delta, \lambda_t)$, we can upper bound $\gR_1(T)$ by
\begin{align*}
    \gR_1(T) \le 8 \cdot \sum_{t=1}^T \frac{1}{(\psi(S_t)^\top \theta^*)^2} \cdot \| \psi(S_t) \|_{H_t^{-1}(\theta^*; \lambda_t)} \cdot \gamma.
\end{align*}
Denote $A_t := \psi(S_t)/\psi(S_t)^\top \theta^*$, we have $H_t(\theta^*; \lambda) = \sum_{i < t} A_t^\top A_t + \lambda_t I$ and $\|A_t\|_2 \le \sum_{i \in S_t} \|\phi(i)\|_2 \cdot \ell^*(S_t) \le KL^*$. Then we have
\begin{align*}
    \gR_1(T) &\le 8\gamma \sqrt{\sum_{t=1}^T \|A_t\|^2_{H_t^{-1}(\theta^*; \lambda_t)}} \cdot \sqrt{\sum_{t=1}^T \frac{1}{(\psi(S_t)^\top \theta^*)^2}} \\
    &\le 16\gamma \cdot \sqrt{d(1 + KL^*/\lambda) \cdot \log\left(1 + KL^*T/d\lambda\right)} \cdot \sqrt{\sum_{t=1}^T \frac{1}{(\psi(S_t)^\top \theta^*)^2}},
\end{align*}
where the first inequality is due to the Cauchy-Schwartz inequality, and the second inequality is due to the elliptical potential lemma of \citet{abbasi2011improved}. Moreover, by \citet[Lemma 31]{liu2024almost}, we have
\begin{align*}
    \sqrt{\sum_{t=1}^T \frac{1}{(\psi(S_t)^\top \theta^*)^2}} &\le \sqrt{T\cdot \frac{1}{(\psi(S^*)\psi^*)^2} + 2 \cdot \gR(T)} \\
    &\le \sqrt{T\cdot (\ell^*(S^*))^2 } + \sqrt{2 \cdot \gR(T)},
\end{align*}
which shows that for $\lambda := \inf_t \lambda_t$, 
\begin{align*}
    \gR_1(T) &\le 16\gamma \cdot \sqrt{d(1 + L^*/\lambda) \cdot \log\left(1 + L^*T/d\lambda\right)} \cdot \sqrt{T\cdot (\ell^*(S^*))^2 } \\
    &\quad + 16\gamma \cdot \sqrt{d(1 + L^*/\lambda) \cdot \log\left(1 + L^*T/d\lambda\right)} \cdot \sqrt{2 \cdot \gR(T)}.
\end{align*}
Next we give the upper bound for $\gR_2(T)$. Recall that 
\begin{align*}
    \gR_2(T) = \sum_{t=1}^T \frac{2}{\xi^3} \cdot \left(\psi(S_t)^\top \theta^* - \psi(S^*)\theta^*\right)^2.
\end{align*}
Then, under $\Xi$, we have
\begin{align*}
    \gR_2(T) &\le \sum_{t=1}^T \frac{2}{\xi^3} \cdot \left\langle\psi(S_t) ,\theta^* - \wt{\theta}_t\right\rangle^2 \\
    &\le  \frac{2}{\ell^*(S^*)^3} \cdot \sum_{t=1}^T \|\psi(S_t)\|^2_{H_t^{-1}(\theta^*; \lambda_t)} \cdot \|\theta^* - \wt{\theta}_t\|^2_{H_t^{-1}(\theta^*; \lambda_t)} \\
    &\le \frac{2}{\ell^*(S^*)^3} \cdot 64\gamma^2 \cdot \sum_{t=1}^T \|\psi(S_t)\|^2_{H_t^{-1}(\theta^*; \lambda_t)},
\end{align*}
where the first inequality is according to \Cref{eq:optimism}, the second inequality is due to the Cauchy-Schwartz inequality, and the last inequality holds by \Cref{lemma:mle-concentration}. Denote
\begin{align*}
    \Lambda_t := \lambda_t I + \sum_{i < t} \psi(S_i)^\top \psi(S_i). 
\end{align*}
Then we have
\begin{align*}
    \sup_{S \in \gS} (\psi(S)^\top \theta^*) \cdot \Lambda_t^{-1} \succ H_t^{-1}(\theta^*; \lambda_t),
\end{align*}
which further implies
\begin{align*}
    \gR_2(T) &\le \frac{2}{\ell^*(S^*)^3} \cdot 64\gamma^2 \cdot \sup_{S \in \gS} (\psi(S)^\top \theta^*) \cdot \sum_{t=1}^T \|\psi(S_t)\|^2_{\Lambda_t^{-1}} \\
    &\le \frac{2}{\ell^*(S^*)^3} \cdot 64\gamma^2 \cdot \sup_{S \in \gS} (\psi(S)^\top \theta^*) \cdot 2dL^*\log(1 + L^* T/d\lambda) \\
    &= \frac{256}{\ell^*(S^*)^3} \sup_{S \in \gS} (\psi(S)^\top \theta^*) \cdot \gamma^2 \cdot dL^*\log(1 + L^*T/d\lambda).
\end{align*}
Therefore, we have
\begin{align*}
    \gR(T) &\le 16\gamma \cdot \sqrt{d(1 + L^*/\lambda) \cdot \log\left(1 + L^*T/d\lambda\right)} \cdot \sqrt{T\cdot (\ell^*(S^*))^2 } \\
    &\quad + 16\gamma \cdot \sqrt{d(1 + L^*/\lambda) \cdot \log\left(1 + L^*T/d\lambda\right)} \cdot \sqrt{2 \cdot \gR(T)} \\
    &\quad + \frac{256}{\ell^*(S^*)^3} \sup_{S \in \gS} (\psi(S)^\top \theta^*) \cdot \gamma^2 \cdot dL^*\log(1 + L^*T/d\lambda) + 1.
\end{align*}
Notice that for $x \le A\sqrt{x} + B$, we have $x \le 2A^2 + B$. Therefore, we have
\begin{align*}
    \gR(T) &\le 
    16\gamma \cdot \sqrt{dT}\cdot \sqrt{(\ell^*(S^*))^2(1 + L^*/\lambda) \cdot \log\left(1 + L^*T/d\lambda\right)} \\
    &\quad + 256\gamma^2 \cdot dL^* \cdot \log\left(1 + L^*T/d\lambda\right) \cdot \left(\frac{\sup_{S \in \gS} (\psi(S)^\top \theta^*)}{\ell^*(S^*)^3} + 2\right) + 1, \\
    &\le \wt{\gO}\left(\sqrt{d^3 T}\right)
\end{align*}



\end{proof}
\end{theorem}





\end{document}


% This document was modified from the file originally made available by
% Pat Langley and Andrea Danyluk for ICML-2K. This version was created
% by Iain Murray in 2018, and modified by Alexandre Bouchard in
% 2019 and 2021 and by Csaba Szepesvari, Gang Niu and Sivan Sabato in 2022.
% Modified again in 2023 and 2024 by Sivan Sabato and Jonathan Scarlett.
% Previous contributors include Dan Roy, Lise Getoor and Tobias
% Scheffer, which was slightly modified from the 2010 version by
% Thorsten Joachims & Johannes Fuernkranz, slightly modified from the
% 2009 version by Kiri Wagstaff and Sam Roweis's 2008 version, which is
% slightly modified from Prasad Tadepalli's 2007 version which is a
% lightly changed version of the previous year's version by Andrew
% Moore, which was in turn edited from those of Kristian Kersting and
% Codrina Lauth. Alex Smola contributed to the algorithmic style files.
