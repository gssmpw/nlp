\section{Related Work}
\subsection{Financial and Greek LLMs}
In recent years, an increasing number of LLMs have been tailored to financial applications. 
Most existing work is English-centric, such as FinLLaMA~\citep{xie2024openfinllmsopenmultimodallarge}, BloombergGPT~\citep{DBLP:journals/corr/abs-2303-17564}, PIXIU~\citep{DBLP:conf/nips/XieHZLPLH23}, InvestLM~\citep{DBLP:journals/corr/abs-2309-13064}, and FinGPT~\citep{DBLP:journals/corr/abs-2306-06031}, leveraging domain-specific financial corpora for tasks. 
In parallel, recent research in Chinese (DISC-FinLLM~\citep{DBLP:journals/corr/abs-2310-15205} and CFGPT~\citep{DBLP:journals/corr/abs-2309-10654} and bilingual financial LLMs (FinMA-ES~\citep{DBLP:conf/kdd/ZhangXYFHLLQA0H24} for Spanish and English) extend these efforts by covering related non-English and bilingual finance tasks. 
Despite these notable advancements, there is a conspicuous absence of specialized Greek financial LLMs. Existing Greek open-source LLMs, such as Meltemi \citep{DBLP:journals/corr/abs-2407-20743} and Llama-Krikri\footnote{\url{https://huggingface.co/ilsp/Llama-Krikri-8B-Base}}, do not include finance-oriented training data, which highlights the critical need for developing a financial model specifically tailored to the Greek context.

\subsection{Financial Benchmarks}
Numerous financial benchmarks have been developed for evaluating LLMs' capabilities in financial domain.
Though FinBen~\citep{xie2024finbenholisticfinancialbenchmark}, INVESTORBENCH~\citep{li2024investorbenchbenchmarkfinancialdecisionmaking}, PIXIU~\citep{DBLP:conf/nips/XieHZLPLH23}, UCFE~\citep{yang2025ucfeusercentricfinancialexpertise}, FinanceBench~\citep{islam2023financebenchnewbenchmarkfinancial}, and FinGPT~\citep{wang2023fingptinstructiontuningbenchmark} provide wide-ranging evaluations, covering comprehensive financial tasks and experiment settings, they are predominantly in English.
Efforts to move beyond English have resulted in benchmarks covering Spanish~\citep{DBLP:conf/kdd/ZhangXYFHLLQA0H24}, Chinese~\citep{nie2024cfinbenchcomprehensivechinesefinancial}, and Japanese~\citep{hirano2024constructionjapanesefinancialbenchmark}, underscoring the the value of linguistic and cultural diversity in financial tasks. 
While Greek mentioned in a few multilingual benchmarks like the Belebele benchmark~\citep{DBLP:conf/acl/BandarkarLMASHG24}, there is no dedicated Greek financial benchmark, making it difficult to rigorously assess LLMs in Greek finance-specific contexts.

% Existing Greek datasets mainly address general tasks (ARC Greek \citep{DBLP:journals/corr/abs-1803-05457}, Truthful QA Greek \citep{DBLP:conf/acl/LinHE22}, HellaSwag Greek \citep{DBLP:conf/acl/ZellersHBFC19}, MMLU Greek\footnote{\url{https://huggingface.co/datasets/ilsp/mmlu_greek}}) or specialized domains like medicine (Medical MCQA\footnote{\url{https://huggingface.co/datasets/ilsp/medical_mcqa_greek}}). 
% Consequently, there is an urgent need for resources that capture the nuanced linguistic and contextual features of Greek financial text.