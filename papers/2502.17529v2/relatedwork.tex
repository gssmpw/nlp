\section{RELATED WORK}
\subsection{Formation Decision Making} Formation decision-making plays a crucial role in enhancing traffic efficiency, safety, and fuel economy, particularly in the context of CAVs. Early research on convoy decision-making primarily focused on coordination in single-lane scenarios, emphasizing vehicle longitudinal behaviors such as maintaining constant speed and ensuring safe inter-vehicle distance \cite{jia2015survey,li2019platoon}. However, these methods primarily addressed simple longitudinal vehicle behaviors and lacked the adaptability required to handle the complexities of real-world driving environments.

With the development of multi-lane highways and more complex traffic dynamics, convoy decision-making algorithms need to exhibit stronger adaptability. A significant challenge is how to effectively adjust the convoy's behavior in dynamic environments, including lane changing, obstacle avoidance, and adapting to traffic congestion. For example, Gao et al. \cite{gao2019multi} used the artificial potential field method to improve obstacle avoidance, while Cai et al. \cite{cai2019multi,cai2021formation} developed dynamic staggered formation generation methods, they also introduced a dual-layer planning framework to optimize lane changes and formation adjustments in real-time, improving traffic efficiency and vehicle coordination. However, these methods often rely on predefined strategies and lack flexibility in responding to unexpected scenarios.

In recent years, data-driven deep reinforcement learning (RL) has introduced a new paradigm for convoy decision-making. Xu et al. \cite{xu2022connected} applied multi-agent reinforcement learning (MARL) for cooperative lane change control in convoy vehicles, while Fan et al. \cite{fan2023twin} optimized defensive decisions for escort vehicles using attention mechanisms. Notably, de Zarza et al. \cite{de2023decentralized} proposed the decentralized truck convoy framework O'Platoon based on Q-learning, which enables autonomous decision-making for individual vehicles through local information exchange, thereby enhancing system scalability while reducing communication dependency. These data-driven approaches offer higher flexibility but often face challenges related to training efficiency and generalization capabilities.

\subsection{LLM for Autonomous Driving} In recent years, large language models, represented by the GPT series and LLaMA series, have shown tremendous potential in a wide range of scenarios due to their strong contextual understanding and generative capabilities.

In the field of autonomous driving, researchers have begun to explore how the advantages of LLMs can be integrated with the needs of autonomous driving systems. Wen et al. were the first to apply the knowledge-driven capability of LLMs to autonomous vehicle decision-making, proposing the DiLu framework \cite{wen2023dilu}, which, however, was limited to single-agent decision-making. Jiang et al. introduced the KoMA framework \cite{jiang2024koma}, consisting of multi-agent interaction, multi-step planning, shared memory, and a ranking-based reflective module, which is a multi-agent framework. Similar frameworks include CoMAL \cite{yao2024comal} and CoDrivingLLM \cite{fang2024towards}.

With the rise of multimodal LLMs, some recent research has focused on how autonomous driving can be combined with visual LLMs, enabling visual LLMs to directly acquire information from images rather than from text descriptions provided by humans. Notable examples include Drivevlm \cite{tian2024drivevlm} and OmniDrive \cite{wang2024omnidrive}. These studies demonstrate the vast potential of LLM applications in autonomous driving, especially in complex dynamic environments such as convoy formation control.

\subsection{Formation Control} Efficient formation control is vital for multi-lane convoy operations, where dynamic traffic and unpredictable obstacles challenge traditional single-lane methods. Conventional approaches (e.g., leader–follower, virtual structure, and consensus-based strategies \cite{chen2023survey}) work well in regulated settings but often falter amid communication delays and rapid changes in convoy composition. Recent advances—such as Zhao et al.'s multi-platoon collision avoidance framework \cite{zhao2023multi} and their robust consensus method \cite{zhao2022consensus}, along with a hybrid automata approach for unified path planning and control \cite{huang2018path}—have improved formation stability in complex scenarios. Complementary reviews and developments in distributed control \cite{chu2024survey, yang2023distributed} further highlight these efforts. In contrast to these rule-based or data-driven models, our work employs a large language model to offer real-time, context-aware decision support, bridging the gap between traditional formation control and intelligent, flexible convoy management.