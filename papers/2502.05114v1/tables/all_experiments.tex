% \documentclass[table]{article}
% \usepackage{hhline} % used for the top-most horizontal line! in order to fix the top left corner cell empty
% \usepackage{highlight}
% %======================================
% \renewcommand{\opacity}{50}
% \renewcommand{\minval}{0}
% \renewcommand{\maxval}{1}

% \begin{document}
\begin{table}[h!]
    \centering
    \begin{tabular}{l|l|l|l|l}
    
          & Acc$_1$ & Acc$_{10}$ & Sim$_1$ & Sim$_{10}$ \\ \hline
        exp5: 8.6M\_448k+296k (SpecTUS) & 43.3\% & 64.8\% & 0.67 & 0.81 \\ \hline
        exp5: 8.6M\_224k+148k & 40.7\% & 62.8\% & 0.65 & 0.80 \\ \hline
        exp5: 4.2M\_224k\_148k & 39.8\% & 61.8\% & 0.64 & 0.79 \\ \hline
        exp5: 4.2M\_224k\_74k & 39.1\% & 61.8\% & 0.64 & 0.79 \\ \hline
        exp4: one src token & 38.1\% & 61.4\% & 0.63 & 0.79 \\ \hline
        exp3: RASSP:NEIMS / exp5: 4.2M\_112k\_74k & 37.9\% & 60.7\% & 0.63 & 0.78 \\ \hline
        exp3: RASSP:NEIMS:NIST & 37.6\% & 60.6\% & 0.62 & 0.78 \\ \hline
        exp3: NEIMS-only & 36.8\% & 59.7\% & 0.62 & 0.77 \\ \hline
        exp3: RASSP-only & 34.8\% & 57.6\% & 0.61 & 0.76 \\ \hline
        exp1: log30bins / exp2: mf10M (char-level) / exp3: no pretraining & 28.1\% & 50.7\% & 0.56 & 0.72 \\ \hline
        exp1: log40bins & 27.6\% & 49.7\% & 0.55 & 0.72 \\ \hline
        exp1: lin4dec & 27.4\% & 49.9\% & 0.55 & 0.72 \\ \hline
        exp1: lin3dec & 27.3\% & 49.8\% & 0.55 & 0.72 \\ \hline
        exp1: log21bins & 27.2\% & 49.6\% & 0.55 & 0.72 \\ \hline
        exp1: log10bins & 24.7\% & 47.0\% & 0.54 & 0.70 \\ \hline
        exp2: mf10K & 24.5\% & 46.4\% & 0.54 & 0.70 \\ \hline
        exp2: mf100 & 24.5\% & 47.2\% & 0.55 & 0.71 \\ \hline
        exp2 mf10 & 24.0\% & 46.1\% & 0.54 & 0.70 \\ \hline
        exp1: lin2dec & 24.0\% & 46.1\% & 0.53 & 0.69 \\ \hline
        exp2: SELFIES & 22.3\% & 41.1\% & 0.49 & 0.65 \\ \hline
    \end{tabular}
    \caption{Summary of all experiments conducted during the development of the final SpecTUS model. The metrics were evaluated on the NIST validation set. The experiments are ordered by Acc$_1$ values, which served as the primary criterion for assessing the results.}
    \label{tab:experiments_comparison}
\end{table}
% \end{document}