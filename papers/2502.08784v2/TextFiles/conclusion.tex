% In this paper, we have introduced a framework for the control of PDEs leveraging a controllable physics-informed latent space, designed to operate with partially observable data. We showed that our framework can be implemented in two different ways by representing its latent space with either numerical integration or a PINN. Notably, our numerical integration variant, with its fewer trainable parameters, demonstrated exceptional efficacy and robustness, surpassing even its PINN counterpart. A crucial aspect of our research is the development of a trainable PML within our latent space, a novel concept that significantly enhances model robustness by enabling energy dissipation.

% The successful application of our framework in controlling the acoustic wave equation sets a solid foundation for extending this framework to other domains. Potential future applications include controlling wildfire spread via drone fleets or managing public health crises by limiting the transmission of pandemics through optimal lockdown strategy. This framework advances the frontier of ML in for control of PDE and by extension paves the way for transformative technologies.


% In this paper, we introduced a novel framework for the manipulation of acoustic waves in a simulated environment, leveraging the AEM and a trainable PML. The PML plays a crucial role in enhancing the accuracy of wave manipulation by enabling energy dissipation, significantly improving model robustness. Through detailed comparisons with NODE-based approaches, we demonstrated the superior performance of the AEM in both focusing and suppression tasks within MPC scenarios.

This work extends classical object manipulation in robotics to the domain of acoustic waves.
% where wave control is critical.
It opens new 
% interdisciplinary 
research opportunities at the intersection of robotics and acoustics, with potential applications in such areas as ultrasound cutting, energy harvesting, and the design of new acoustic materials. This work provides proof of concept for the implementation of a physically realizable robot for wave manipulation. Importantly, the methodology is not limited to acoustics—it can be adapted to other domains governed by PDEs.
% , such as controlling wildfires via drone fleets or optimizing lockdown strategies to mitigate pandemics.
% \NS{wildfires/pandemics already mentioned in introduction, maybe we should remove it?}

We highlight the unique features of our approach, including sparse robotic actuation under partial observability and a fully interpretable, physics-constrained solution. This allows for a trainable PML embedded within the AEM’s latent space ensuring robust and accurate long-term predictions. The proposed AEM allows to find a solution, which can be derived only in particular cases with GBO. That makes AEM a strong candidate in robot learning for PDE manipulation, where interpretable and reliable methods are required.


% The availability of our code provides an extendable framework that can be applied to a variety of PDE-governed processes.
% , further boosting its utility across different domains.

% % While our results are promising, there remain limitations and open challenges. 
% These controllers could dynamically adjust to complex environments and optimize real-time performance, pushing the boundaries of wave manipulation and control in increasingly challenging scenarios.

% This work sets a foundation for future research, advancing the knowledge of robot learning for PDE control and laying the groundwork for transformative technologies in the interplay between robotics and a broad class of physical processes governed by PDEs.

