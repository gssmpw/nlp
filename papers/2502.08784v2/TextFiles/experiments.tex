

% \begin{table*}
% \centering
% \caption{Steady state scattered energy in the simulated environment. The steady state energy is measured from $0.10$ to $0.20$ seconds of time in the environment. Statistics are computed over 12 runs with randomized initial robot conditions and fixed source location at $(-10.0, 0.0)$. \textbf{R}, \textbf{P}, and \textbf{F} indicate radii, position, and full (both radii and position) adjustment.}
% \label{tab:results_summary}
% \begin{tabularx}{\textwidth}{l l cccccc}
% \toprule
% \textbf{Task} & \textbf{Method} & $\stackrel{\textbf{R}}{\textbf{Ring}}$ & $\stackrel{\textbf{P}}{\textbf{M=1}}$ & $\stackrel{\textbf{P}}{\textbf{M=2}}$ & $\stackrel{\textbf{P}}{\textbf{M=4}}$ & $\stackrel{\textbf{F}}{\textbf{M=2}}$ \\
% \midrule
% \multirow{4}{*}{\textbf{Focusing (Higher is Better)}} 
% & \textbf{Random}   & $0.54 \pm 0.04$ & $0.12 \pm 0.03$ & $0.35 \pm 0.21$ & $1.05 \pm 0.46$ & $0.87 \pm 0.60$ \\
% & \textbf{NODE}     & $0.68 \pm 0.04$ & $0.17 \pm 0.07$ & $1.22 \pm 1.22$ & $1.45 \pm 0.92$ & $6.19 \pm 1.51$ \\
% & \textbf{AEM}     & $\mathbf{0.72 \pm 0.04}$ & $\mathbf{2.47 \pm 1.91}$ & $\mathbf{7.54 \pm 2.24}$ & $\mathbf{5.94 \pm 1.90}$ & $\mathbf{8.49 \pm 1.37}$ \\
% & \textbf{GBO}      & -- & -- & -- & -- & -- \\
% \midrule
% \multirow{4}{*}{\textbf{Suppression (Lower is Better)}} 
% & \textbf{Random}   & $8.11 \pm 0.47$ & $3.00 \pm 0.72$ & $6.87 \pm 1.89$ & $10.04 \pm 1.84$ & $5.41 \pm 1.33$ \\
% & \textbf{NODE}     & $6.91 \pm 0.29$ & $2.86 \pm 0.53$ & $4.40 \pm 0.84$ & $10.16 \pm 2.11$ & $3.48 \pm 1.01$ \\
% & \textbf{AEM}     & $\mathbf{5.50 \pm 0.17}$ & $\mathbf{1.08 \pm 0.18}$ & $\mathbf{1.98 \pm 0.38}$ & $\mathbf{3.78 \pm 0.74}$ & $\mathbf{0.38 \pm 0.02}$ \\
% & \textbf{GBO}      & -- & -- & -- & -- & -- \\
% \bottomrule
% \end{tabularx}
% \end{table*}



% \begin{table}
% \centering
% \caption{Steady state scattered energy in the simulated environment. The energy is measured from 0.10 to 0.20 seconds. Statistics are over 12 runs with randomized robot conditions and fixed source location at $(-10.0, 0.0)$. \textbf{R}, \textbf{P}, and \textbf{F} indicate radii, position, and full (both radii and position) adjustment.}
% \label{tab:results_summary}
% \small
% \begin{tabular}{l l ccccc}
% \toprule
% \textbf{Task} & \textbf{Method} & $\stackrel{\textbf{R}}{\textbf{Ring}}$ & $\stackrel{\textbf{P}}{\textbf{M=1}}$ & $\stackrel{\textbf{P}}{\textbf{M=2}}$ & $\stackrel{\textbf{P}}{\textbf{M=4}}$ & $\stackrel{\textbf{F}}{\textbf{M=2}}$ \\
% \midrule
% \multirow{4}{*}{\textbf{Focusing}} 
% & \textbf{Random}   & $0.54 \pm 0.04$ & $0.12 \pm 0.03$ & $0.35 \pm 0.21$ & $1.05 \pm 0.46$ & $0.87 \pm 0.60$ \\
% & \textbf{NODE}     & $0.68 \pm 0.04$ & $0.17 \pm 0.07$ & $1.22 \pm 1.22$ & $1.45 \pm 0.92$ & $6.19 \pm 1.51$ \\
% & \textbf{AEM}     & $\mathbf{0.72 \pm 0.04}$ & $\mathbf{2.47 \pm 1.91}$ & $\mathbf{7.54 \pm 2.24}$ & $\mathbf{5.94 \pm 1.90}$ & $\mathbf{8.49 \pm 1.37}$ \\
% & \textbf{GBO}      & -- & -- & -- & -- & -- \\
% \midrule
% \multirow{4}{*}{\textbf{Suppression}} 
% & \textbf{Random}   & $8.11 \pm 0.47$ & $3.00 \pm 0.72$ & $6.87 \pm 1.89$ & $10.04 \pm 1.84$ & $5.41 \pm 1.33$ \\
% & \textbf{NODE}     & $6.91 \pm 0.29$ & $2.86 \pm 0.53$ & $4.40 \pm 0.84$ & $10.16 \pm 2.11$ & $3.48 \pm 1.01$ \\
% & \textbf{AEM}     & $\mathbf{5.50 \pm 0.17}$ & $\mathbf{1.08 \pm 0.18}$ & $\mathbf{1.98 \pm 0.38}$ & $\mathbf{3.78 \pm 0.74}$ & $\mathbf{0.38 \pm 0.02}$ \\
% & \textbf{GBO}      & -- & -- & -- & -- & -- \\
% \bottomrule
% \end{tabular}
% \end{table}



% \ST{we can remove the para, as  I moved the most important stuff to the next para.}Our experimental setup aims to validate the proposed framework\ST{more precisely? see below} for controlling complex dynamical systems, specifically focusing on the acoustic wave equation, $\partial_t^2\zeta(x, t) = c^2\partial_x^2\zeta(x, t) + f(x, t)$, influenced by an exogenous force in an open space scenario. \ST{this park can be moved. see below}

The goal of our experiments is to validate that the AEM approach can effectively exploit prior knowledge about the physical properties of a system governed by an acoustic PDE. We demonstrate this by first showing that the model can predict scattered energy in the domain for action sequences extending far beyond its training horizon. Next, we show that AEM can be effectively used with MPC on two benchmark tasks in acoustics: energy suppression and focusing. We compare our method to a ML approach (NODE) and a ground truth solution derived by GBO. 

% Results of our experiments in both prediction and control are detailed below.

\iffalse
GBO is a widely recognized technique in acoustic research \cite{Treweek2022,Ewees23,Blankrot2019,AmirkulovaNorris17b,math9222862,Amirkulova_Gerges_Norris_2022,GergesAmirkulova2024} which provides optimal solutions with large design space\cite{HU2020113387}. 
However, GBO demands extensive domain knowledge, including the precise definition of a closed form solution, which significantly restricts its applicability to situations where this form is already well-defined\cite{AmirkulovaNorris17b, Amirkulova_etal2021, Amirkulova_Gerges_Norris_2022,GergesAmirkulova2024}. Moreover, the method is effective only when the system state is fully observable. To enhance computational efficiency, GBO also requires the gradients of the objective function and the non-linear constraints. Obtaining these gradients is often a complex and challenging task, especially in cases where the function or constraints are highly non-linear or analytically intractable, further complicating the application of GBO. 
\fi
% Furthermore, we seek to establish that this approach can be applied to a physically realizable system, as previously discussed. This is evidenced by the AEM model’s superior control performance compared to other methods, detailed below. 
% Code for this research can be found on GitHub.\footnote{\url{https://github.com/gladisor/Waves.jl}}
% \ST{we say that in Abstract. so it can be remove here}\TS{The website is provided in abstract, not code.}

\subsection{Setting}
% \ST{What are the purpose of the experiments? What questions do they answer? How do we measure success (by comparison to xyz?)}

The robotic actuator we consider in this environment is a configuration of cylindrical scatterers. We assume that an agent has the ability to manipulate by $a(t)$, the radii and/or position of $d(t)$, which create time-variant boundary conditions for the domain, $\Omega(d(t))$ of wave propagation:
\begin{align*}
    &\partial_t^2\zeta(x, t) = c^2\partial_x^2\zeta(x, t) + f(x, t) \in \Omega(d(t))\\
    &\dot{d}(t) = F(d(t), a(t)),
\end{align*}
where $\f(x, t)$ and $d(t)$ evolve in two different timescales: fast  and slow for wave and robot, respectively.

We conduct the experiments in a simulated open space with size $x \in \left[\SI{-15.0}{\meter}, \SI{15.0}{\meter} \right] \times \left[\SI{-15.0}{\meter}, \SI{15.0}{\meter} \right]$, where 
% an exogenous force 
% $f(x, t) = f(x)\mbox{sin}(\omega t)$,
the excitation source creates a spherical wave in the domain (cf., Figure \ref{fig:main_scheme}). Simulation of open space without wave reflection from the boundaries is achieved through a 
% \textit{perfectly matched layer} 
PML \cite{berenger_1994_a}, which dissipates wave energy as it leaves $\Omega$. Within $\mathcal{N}^\Gamma$ we include a trainable PML through $l(\bar{x})$ which allows our agent to explicitly model energy dissipation.%\vspace{-0.5cm}\\

\subsection{Robot Actuation}
We vary the number of scatterers, denoted as $M$, and test several configurations for robot actions. We use \textbf{R}, \textbf{P}, and \textbf{F} to represent "Radii Adjustment", "Positional Adjustment", and "Full Adjustment" respectively:
\begin{enumerate}
    \item \textbf{Radii Adjustment (R)} (Ring Configuration): This setup features $M = 19$ scatterers arranged in a ring formation with fixed positions. The actions in this experiment involve solely adjusting the radii of the scatterers.
    \item \textbf{Positional Adjustment (P)}: Here, scatterers have fixed radii, and the agent can only change their positions. We tested this configuration with $M = 1$, $2$, and $4$ scatterers.
    \item \textbf{Full Adjustment (F)}: In this scenario, the agent can modify both the positions and radii of the scatterers. We demonstrated performance for $M = 2$ in this configuration.
\end{enumerate}





%% Long term prediction of scattered energy figure with caption:
% \begin{figure}[h!]
% \includegraphics[width=0.9\linewidth]{Figures/prediction.png}
% \caption{Long term predictions of scattered energy signal by the models. Here, the robot is considered to be a triple-ring configuration with adjustable radii. Predictions are made over action horizons of $200$ based on the initial state of the wave and robot. Our model, which incorporates a trainable PML layer shown in red is the only one capable of making faithful predictions to the ground truth signal (green). The version of our model without the PML (orange) is unable to dissipate excess energy and rapidly diverges from the ground truth. NODE has no explicit concept of energy dissipation and also diverges from the ground truth.The NODE prediction (blue) diverges from the }
% \end{figure}

% \begin{figure}[h!]
% \includegraphics[width=\linewidth]{Figures/M=2_fullyAdj_focusing_average_1-12.png}
% \caption{Mean focused scattered energy in the top right quadrant over 12 runs using two fully adjustable scatterers (M=2, F). Positions randomly initialized to opposite quadrant, radii initialized randomly. The source is fixed at (-10, 0). The shading represents ± 0.5 standard deviation.}
% \end{figure} 

\subsection{Evaluation Tasks}
We evaluate our model in both prediction and control.
\begin{itemize}
    \item \textbf{Long-Term Prediction}: We compare predicted scattered energy over a horizon of 200 steps (much beyond the training horizon) between our model, the baseline model, and the ground truth. 
    \item \textbf{Focusing}: Concentrate the scattered energy within the upper-right quadrant of the domain by the robots actions. 
    \item \textbf{Suppression}: Minimize the total scattered energy within the domain by the robots actions.
\end{itemize}
These tasks represent standard benchmarks and are frequently used in acoustics research \cite{amirkulova_2020_the, norris_2008_acoustic}. We use Model Predictive Control in both manipulation tasks.
% To evaluate our model, we conducted a series of experiments, grouped into two main scenarios: “Suppression” and “Focusing.” The suppression experiments aimed to minimize the total scattered energy in the grid using the scatterers, while the focusing experiments aimed to concentrate the scattered energy in the upper right quadrant of the domain. This is the standard experiment setting in acoustics research \cite{amirkulova_2020_the, norris_2008_acoustic}.

% \vspace{-4cm}
\begin{table}%[t!]
\centering
\vspace{0.2cm}
\caption{Steady-state scattered energy in the simulated environment. The steady-state energy is measured from $0.10$ to $0.20$ seconds of time in the environment. Statistics are computed over 12 runs with randomized initial robot conditions and fixed source location at $(-10.0, 0.0)$.}
% \textbf{R}, \textbf{P}, and \textbf{F} indicate radii, position, and full (both radii and position) adjustment.}
\label{tab:results_summary}
\begin{tabularx}{\linewidth}{l cccc}
\toprule
\textbf{Method} & \textbf{Random} & \textbf{NODE} & \textbf{AEM} & \textbf{GBO} \\
\midrule
\multicolumn{5}{c}{\textbf{Focusing (Higher is Better)}} \\
% \textbf{Ring (R)}    & $0.54 \pm 0.04$ & $0.68 \pm 0.04$ & $\mathbf{0.72 \pm 0.04}$ & N/A \\
\textbf{M=1 (P)}     & $0.12 \pm 0.03$ & $0.17 \pm 0.07$ & $\mathbf{2.47 \pm 1.91}$ & N/A \\
\textbf{M=2 (P)}     & $0.35 \pm 0.21$ & $1.22 \pm 1.22$ & $\mathbf{7.54 \pm 2.24}$ & N/A \\
\textbf{M=4 (P)}     & $1.05 \pm 0.46$ & $1.45 \pm 0.92$ & $\mathbf{5.94 \pm 1.90}$ & N/A \\
\textbf{M=2 (F)}     & $0.87 \pm 0.60$ & $6.19 \pm 1.51$ & $\mathbf{8.49 \pm 1.37}$ & N/A \\
\midrule
\multicolumn{5}{c}{\textbf{Suppression (Lower is Better)}} \\
% \textbf{Ring (R)}    & $8.11 \pm 0.47$ & $6.91 \pm 0.29$ & $\mathbf{5.50 \pm 0.17}$ & N/A \\
\textbf{M=1 (P)}     & $3.00 \pm 0.72$ & $2.86 \pm 0.53$ & $1.08 \pm 0.18$ & $\mathbf{1.01}$ \\
\textbf{M=2 (P)}     & $6.87 \pm 1.89$ & $4.40 \pm 0.84$ & $\mathbf{1.98 \pm 0.38}$ & $2.13$ \\
\textbf{M=4 (P)}     & $10.04 \pm 1.84$ & $10.16 \pm 2.11$ & $3.78 \pm 0.74$ & $\mathbf{3.53}$ \\
\textbf{M=2 (F)}     & $5.41 \pm 1.33$ & $3.48 \pm 1.01$ & $\mathbf{0.38 \pm 0.02}$ & $0.38$ \\
\bottomrule
\end{tabularx}
\end{table}


\subsection{Results}

\subsubsection{Long-Term Prediction of Scattered Energy}
We assess the robustness of our model by comparing its long-term prediction capabilities to a NODE-based approach, \begin{figure}%[h!]
\includegraphics[width=\linewidth]{Figures/prediction.png}
\caption{Long-term prediction of scattered energy over an episode of 0.2 seconds (200 action steps) for the ring configuration (\textbf{R}). The training horizon for the models is 20 actions, which demonstrate generalization of the model to unseen in training horizons.}\label{prediction_plot}
\end{figure} as well as an AEM-based approach that does not incorporate a trainable PML inside $\mathcal{N}^\Gamma$. Figure \ref{prediction_plot} shows that AEM (PML) follows the ground truth scattered energy compared to NODE and AEM (No PML) which diverge. Our model without a PML experiences rapid energy buildup because it is unable to dissipate energy in $\mathbf{z}$. The NODE model also diverges from the ground truth signal and provides no pathway to explicitly model energy dissipation. These results highlight the critical role of including a PML in $\mathcal{N}^\Gamma$ when $\f$ occurs in open space. Table \ref{tab:results_summary} shows that AEM achieves similar results to the ground truth results by GBO in the 'Suppression' task, when a semi-analytical solution by GBO is known in the literature, otherwise it is marked 'N/A'.\\

\subsubsection{Focusing and Suppression of Scattered Energy}
We utilize MPC through optimization of Eq. \eqref{eq:latent_control_cost} for both benchmark tasks. Figure \ref{fullyAdj_mean} shows the results for the fully adjustable configuration with two scatterers (F, M=2). The plot demonstrates the advantages of our approach in both focusing (top) and suppressing (bottom) scattered energy. In terms of steady-state performance, our method achieves higher focused energy and lower suppressed energy. Furthermore, our approach exhibits better transient response time, reaching steady-state faster than the other approaches. Similar trends were observed across other scatterer configurations.
% Table \ref{tab:results_summary} summarized the steady-state performance for different experiments.

\begin{figure}%[h!]
\includegraphics[width=\linewidth]{Figures/1313.png}
\caption{Focused and Suppressed Scattered Energy.  Both the positions and radii of two fully adjustable scatterers are varied (F, M=2). Source is fixed at location (-10, 0) throughout the simulations. Shading represents ± 0.5 standard deviation from the mean. The mean (solid lines) and variance are calculated with 12 runs with randomized initial conditions. The dashed line is for the ground truth by GBO.}\label{fullyAdj_mean}
\end{figure}

% For the positional and full adjustment cases, the scatterers were initially placed in the opposite quadrant to where the energy is focused.

% \TS{This first sentence does not make sense. What does "modeling the latent wave representation" mean?}
We compare the performance of our model against the NODE-based approach with both models having an equal parameter count of 2.7 million. Table \ref{tab:results_summary} summarizes the results of our MPC experiments. A random policy, and gradient-based optimization approaches are also included as part of the comparison. Our AEM model achieves good performance on both benchmarks: for focusing, it yields higher focused energy while maintaining comparable error rates; for suppression, it attains lower suppressed energy than NODE and Random control with the added benefit of reduced error compared to other methods. For some of the cases AEM achieves slightly better performance than the steady-state GBO solution.

% appeared in Abstract 
% The project web page provides video demonstrations of the experiments, and the code repository for reproduction of results and further research.






% The culmination of our experiments involves deploying a policy derived from our predictive model, based on MPC with random shooting optimization, to either suppress scattered energy within the environment, or to focus scattered energy to a part of the grid. 

% This evaluation tests the controller's efficacy across various source locations. We took the best performing model: cPILS-NI and used it as the predictive model for the MPC policy. Our derived policy was able to successfully reduce scattered energy in the environment across a range of source locations as shown in Table.~\ref{tab:suppression_results}. The quality of the suppression peaks when the source location is at the center of the $y$ axis and falls off at the extremes. This may indicate that the particular design that we tested has limited suppression capabilities at particular incident angles. The discrepancy in suppression quality also highlights the how the problem of scattered energy suppression changes at different source locations, underscoring our models robustness under a variety of conditions. 