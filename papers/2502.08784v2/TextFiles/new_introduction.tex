

% Part 1:
Manipulation has long been a central challenge in robotics \cite{zhu2022_challenges_robotic_manipulation, matas2018_sim_to_real_robotic_manipulation}. Robots are frequently tasked with physically interacting with their environment, be it through picking, placing, or altering objects in a variety of applications, from industrial manufacturing \cite{sanchez2018_industry_robotic_manipulation, mikkel2016_robot_industrial_manufacturing} to healthcare \cite{hamed2012_robot_surgery, boonvisut2013_robot_surgery_sensing}. In this work, we explore robotic manipulation on a class of vastly more complex phenomena: acoustic waves. 
% \sout{In this proof-of-concept work, we study a physically realizable robotic system for acoustic wave manipulation.}

Manipulation of acoustic waves introduces unique challenges on several fronts. Unlike traditional object manipulation, where the robot operates in direct contact with the object, wave manipulation involves indirect influence through intermediary tools such as leveraging the interaction between the scatterers and the incident field \cite{Martin06,amirkulova_2020_the, norris2011_multiple}. Further complicating the task, acoustic waves propagate at an extremely rapid speed \cite{del1972_speed_of_sound} compared to robotic actuation. Lastly, designing an optimal controller for wave manipulation requires knowledge about the state of wave, which is a function of time and space, and usually can be only partially observed. 
% Overcoming the above-mentioned challenges (complexity of wave phenomena, indirect and sparse interaction between a robot and wave, different time scales between wave propagation and robot actuators, and extremely partial observability) in the development of robots for wave control is in the core of the current study.
% \TS{There was a portion of text here which described the other challenges of wave manipulation. It was commented out which causes confusing sentences later in the introduction.}

% In the case of acoustic waves, fully observing the underlying displacement and pressure fields is not always possible. Therefore, decision making must be made under some level of epistemic uncertainty.

% Part 2:
Establishing robust control of acoustic waves has wide reaching applications. Robotic systems capable of manipulating waves can lead to groundbreaking technologies, such as 
% energy harvesting from ocean waves\ST{energy harvesting is already mentioned in abstract}, 
seismic wave mitigation \cite{chen_2023_artificially,kim_2021_longitudinal} and super-focusing devices for high-resolution ultrasound imaging and surgery \cite{ilovitsh_2018_acoustical,ozcelik_2018_acoustic}. Mastering wave manipulation opens the door to these and many other downstream technologies. 
\begin{figure}[t!]
    \centering
    \includegraphics[width=\linewidth]{Figures/main_figure.jpg}
    % \caption{Perception-Action Loop: the agent observes the environment (acoustic wave governed by a PDE) through its sensor readings (camera images), and affects it back by its policy, $\pi$, through the cylindrical scatterers (gray blobs).}
          \caption{Schematic representation of interaction between an agent and acoustic wave. The agent observes wave through its `Sensor' readings, and affects it back by choosing the optimal locations and radii of the cylindrical scatterers (gray blobs) according to its `Controller' (policy). Wave propagates with the speed of sound according to the Wave PDE, while robot dynamics (actuation of the cylinders) is an order of magnitude slower time scale. `Excitation' represents an exogenous source of acoustic energy (e.g., speaker). The `Dissipation' layer prevents wave reflections from the boundaries and allows for the efficient simulation of open space (infinite) environments. Robot's policy is trained to achieve a desired wave configuration such as focusing energy in a particular location or minimizing total scattered energy.}
    \label{fig:main_scheme}
\end{figure}

% Aside from numerical methods such as finite difference \cite{kreiss2002difference,leveque2007finite,angel2023efficient} and Discontinious Galerkin %Time explicit 
% \cite{cockburn2012discontinuous,shu2016high} methods  which require highly intensive simulations via discretization over time and space as well as a Reduced Order Model approach \cite{Morris_Amirkizi2023_ROM,Wang2023T_ROM}, which are usually applicable to low frequency wave components only. There exist two main approaches for wave manipulation.

Aside from numerical methods like finite difference \cite{kreiss2002difference,leveque2007finite,angel2023efficient} and discontinuous Galerkin \cite{cockburn2012discontinuous,shu2016high}, which rely on computationally intensive simulations, as well as the Reduced Order Model approach \cite{Morris_Amirkizi2023_ROM,Wang2023T_ROM}, typically suited for low-frequency wave components, there are two main approaches for wave manipulation.
%dynamic is relevant to all above in this paragraph
% There exist two \FA{"two"?: we are missing mentioning current capabilities and advantages/disadvantages of  numerical methods, and ROM. Should we include some info on it?\ST{yes, please}} main approaches for wave manipulation.
The first one assumes an analytical solution to the wave equation, which is a partial different equation (PDE) \cite{courant1967partial}. If such an analytical solution can be obtained, it is used in conjunction with gradient based optimization (GBO) techniques to perform optimization over a system parameters \cite{amirkulova_2020_the,Amirkulova_etal2021, Amirkulova_Gerges_Norris_2022,GergesAmirkulova2024}, towards a desired control objective. However, in general, an analytical solution to a PDE with arbitrary initial and boundary conditions is unknown, which leads to simplifying and limiting assumptions of the problem \cite{amirkulova_2020_the}.
% \sout{rather than solving an original problem.}

% However, controlling these systems is challenging\ST{we mentioned the challenges before. } because they are governed by complex Partial Differential Equations (PDE) that are typically difficult to solve analytically.

% Control over systems governed by PDE is usually based on the derivation of 

 




% In this work we develop a data-driven method for wave control without the derivation of an analytical solution and without the limiting assumptions. 


% With the rise of data-driven methods it is now possible to control PDE without deriving analytical solutions \cite{werner_2023_learning, bieker_2020_deep}, unlocking control of a much larger suite of problems than was previously accessible. Despite their advantages in terms of flexibility, these methods lack interpretability and robustness.

% Part 3:
Another approach is based on machine learning and data-driven methods. The existing data-driven methods for PDE control suggest to learn wave dynamics from a large number of samples in order to predict its response to a control signal \cite{werner_2023_learning, bieker_2020_deep}. Subsequently, this model can be used by a predictive controller to select optimal actions. Usually the learning of PDE dynamics is done by way of a black-box neural network model such as a (long short term memory) LSTM model \cite{werner_2023_learning, bieker_2020_deep}, Latent Evolution of PDE (LE-PDE) \cite{wu_2022_learning}, or, more advanced methods such as a Neural Ordinary Differential Equation (NODE) \cite{_2019_neural}. 

The main disadvantage of these black-box models is that there is no guarantee that a model will learn essential properties of the underlying dynamics of a PDE such as forces, energy dissipation, or viscosity. Consequently, due to the uninterpretable nature of black-box models, there is no clear way to represent physical properties of a wave within the model's architecture. That hampers the applicability of black-box models to important real-life applications such as wave propagation in open space, which requires physically meaningful dissipation of  energy.

% these properties explicitly within the model's architecture. 
To address the gaps in the existing methods we design an interpretable framework for realizable control of acoustic waves. Our framework is specifically designed for adaptation to real-world environments, effectively addressing key challenges in wave manipulation. Namely: considering a realizable actuation mechanism which affects the wave in a sparse manner by inducing scattering is employed, actions are selected by an agent on a slower timescale than wave propagation, and observations of the wave state which could be feasibly collected by sensors are provided as input to our model. Furthermore, our framework addresses the lack of interpretability \cite{yu2021_interpretability} of previous-ML based methods by explicitly constraining a learnable latent representation to abide by a one-dimensional (1D) wave equation. By construction, physical meaning is assigned to quantities within the latent space in the form of parameterized initial conditions, boundary conditions, and constraints (explained in Section \ref{sec:propsol}). Moreover, our framework allows for wave control in open spaces, which we achieve by the design of a trainable data-driven dissipation layer. To our best knowledge, the latter has not been demonstrated before, while it is an essential component in acoustics research, known as \textit{perfectly matched layer} (PML) \cite{berenger_1994_a, johnson_2021_notes}.

We demonstrate the application of our framework to two important tasks in acoustics: sound suppression (minimization of scattered energy in the domain) and sound focusing (maximization of sound energy in a desired location). In our experiments we show that our method is superior to a standard ML dynamics model in these tasks, and competitive with a semi-analytical GBO method. Our work is pioneering research in dynamical control of acoustic waves. While this work focuses on establishing control of a particular type of PDE for wave phenomenon, it could be applied to controlling arbitrary systems governed by PDE such as wildfires, pandemics, and eventually climate \cite{miguelngeljavaloyes_2023_a, schneider_1974_climate,majid_2021_analysis}. 

