% Partial Differential Equations (PDEs) are foundational descriptive tools in scientific and engineering disciplines, enabling the exploration of phenomena from material design to climate dynamics and the propagation of pandemics and wildfires. Developing robotic systems capable of controlling PDEs, especially when the underlying phenomena are only partially observable, holds great potential for fostering technological innovations. This ambition faces considerable obstacles, as a generalizable sample-based solution remains elusive. Addressing this challenge, our work introduces a framework leveraging physics-informed machine learning (ML) to enable control of PDEs. Central to this framework is an agent, equipped with sensors to perceive its environment, that generates a low-dimensional physics-informed representation of the environment. This representation enables the derivation of an optimal policy for sparse control by the agent's actuators. In this work, we consider manipulation of wave energy by way of cylindrical scatterers controlled by a robot. Under this setting we compare the control capabilities of our physics-informed approach with a well known physics-uninformed baseline and show that our method is more robust in a variety of scenarios. Furthermore, we validate our proposed method with a well established gradient based optimization (GBO) solver which provides optimal steady state scatterer configurations. Our method is shown to achieve comparable performance to the GBO baseline while the physics-uninformed method fails.
Recent advancements in robotics, control, and machine learning have facilitated progress in the challenging area of object manipulation. These advancements include, among others, the use of deep neural networks to represent dynamics that are partially observed by robot sensors, as well as effective control using sparse control signals. In this work, we explore a more general problem: the manipulation of acoustic waves, which are partially observed by a robot capable of influencing the waves through spatially sparse actuators. This problem holds great potential for the design of new artificial materials, ultrasonic cutting tools, energy harvesting, and other applications. We develop an efficient data-driven method for robot learning that is applicable to either focusing scattered acoustic energy in a designated region or suppressing it, depending on the desired task. The proposed method is better in terms of a solution quality and computational complexity as compared to a state-of-the-art learning based method for manipulation of dynamical systems governed by partial differential equations. Furthermore our proposed method is competitive with a classical semi-analytical method in acoustics research on the demonstrated tasks. We have made the project code publicly available, along with a web page featuring video demonstrations: \url{https://gladisor.github.io/waves/}.
% and a tutorial. 
% \sout{This work opens doors for new research directions in robotics and for new technologies in the field of wave manipulation.}


% The former can be achieved by function approximators of dynamics such as deep neural networks. The latter usually assumes the temporal sparsity of control signal.