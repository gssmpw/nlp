\iffalse
\begin{figure}[t!]
    \centering
    \includegraphics[width=0.75\linewidth]{Figures/ModelScheme.png}
      \caption{Indirect Interaction between the Agent and Wave. The latter evolves with the speed of sound over the time-dependent domain, $\Omega(d(t))$, defined by scatterers' radii and location, $d(t)$, which in turn evolve according to the dynamical control system, $F$, controllable by the agent's actions. This nested dynamics admits the physical realization of a real robot, which acts at a slow time-scale.}
    \label{fig:model_scheme}
\end{figure}
\fi


In this section, we introduce notations and provide an overview of the problem. The formal problem definition and its solution are provided in Sections \ref{sec:probdef} and \ref{sec:propsol}.

\subsection{Notations}

We denote $t\in \mathbb{R}^+$, $x\in \mathbb{R}^{d_x}$ and $a(t)\in \mathbb{R}^{d_a}$ as the time coordinate, the space coordinate, and the agent's action, respectively. The partial derivative of a function w.r.t time is denoted by $\partial_t$. A differential operator, acting on a function $\f(x, t)$ in a domain $x\in \Omega$, is denoted by $\mathcal{N}(\f; d(t), \ell(x), t)$, where $d(t)$ and $\ell(x)$ represent functions controllable and uncontrollable by the agent, respectively. Both $d(t)$ and $\ell(x)$ can affect the solution to the PDE, $\partial_t \f(x, t)=\mathcal{N}\bigl(\f(x, t); d(t), \ell(x), t\bigr)$. We represent the state of a robot observed at a particular time $t_i$ as $d(t_i)$ whereas $a(t)$ represents actions applied over a time span $t\in\left[t_i, t_{i+1}\right]$.

% For example, the operator $\mathcal{N}\bigl(\f(x, t); d(t), f(x), \rho(x), t\bigr)$ with controllable designs, $d(t)$, and uncontrollable external force, $f(x)$ and energy dissipation function, $\rho(x)$, is denoted by $\mathcal{N}\bigl(\f(x, t); \gamma(x, t), \ell(x), t\bigr)$ in the general case. The design, $d(t)$, and dissipation function, $\rho(x)$, are explained in the next paragraphs.

\subsection{Wave-Robot Coupled Environment}
The simulated environment we consider in this work is defined by a PDE $\mathcal{N}$ for environment dynamics and an ODE $F$ for the robot dynamics. The agent indirectly interacts with the environment through its actions, $a(t)$, which influence the robot, $d(t)$. The latter represents a dynamically changing boundary condition within the environmental domain, which evolves according to a dynamical control system, $F$:
% Consider environment dynamics given by a PDE in a spatial domain $x\in \Omega\in R^{d_x}$, where the boundary conditions (denoted as the design), $d(t)\in R^{d_x}$, evolve according to a dynamical control system (ODE), $F : d(t)\times a(t)\rightarrow \dot{d}(t)$. The agent controls the design, $d(t)$ by its actions $a(t)$, which are derived from $\f$ by policy $\pi$. Exogenous parameters, $\ell(t)$ affects the environment by pumping in energy and/or material to the domain $\Omega$.  
\begin{align}
     &\partial_t \f(x, t) = \mathcal{N}\bigl(\f(x, t);d(t), \ell(x), t\bigr)&\text{PDE}\label{eq:truePDE}\\
     &a(t) = \pi\bigl(\f(x, t), d(t)\bigr)&\text{Policy}\label{eq:policy}\\
     &\dot{d}(t) = F(d(t), a(t))&\text{ODE}\label{eq:design_dynamics}
\end{align}
where $\pi$ is an agent's policy. In this setting the agent controls a low dimensional state of the robot, which in turn affects the function $\f(x, t)$ in the domain $\Omega$. In the sequel we formally define an optimization problem for sample-based derivation of an optimal $\pi$ from a partially observable function $\f(x, t)$. In the interaction model, the ODE controls the PDE creating nested dynamics with two different time scales. 

\subsection{Sparse Robotic Actuation}

The separation between the agent actions, $a(t)$, and the environment $\f$, through the robot, $d(t)$, allows for a physical realization of control of PDEs with a spatially sparse control. 
% This type of system is of a great interest for the design of artificial materials in acoustics, fluid dynamics, elastodynamics, electrodynamics, control of wildfire expansion, and mitigation of pandemics. 
This is because it is not usually possible to construct a controller which is capable of directly acting on a PDE-governed environment at every point in space. 

% For example, it would not be realistically possible to extinguish a wildfire everywhere it is burning. Instead, localized treatment of influential zones with a fleet of drones \cite{miguelngeljavaloyes_2023_a} would be feasible. In the setting of acoustic wave manipulation we consider a well studied actuation mechanism for wave control: cylindrical scatterers. 

In this work, we consider a robotic controller which is capable of manipulating acoustic waves by inducing scattering. Specifically, the robot consists of a number of cylindrical scattering devices that the agent can control through adjustment of their position, radii, or both. In turn, the scattered wave energy produced by reflection of incident waves upon the scatterers can produce complex wave interference patterns, resulting in energy focusing or suppression \cite{amirkulova_2020_the}.

\iffalse
We assume that actions are selected by an agent and their effect on the environment is observed over a finite time horizon. This choice is made due to the difficulties that would arise from attempting to develop a robotic system that acts on the same timescale as acoustic wave propagation. It would be computationally intractable to repeatedly evaluate a control policy at extremely high sampling rates required for continuous control. Instead, we consider an agent that observes the state of $\f$ through its sensors and selects an action $a_i(t)$ which is continuously applied over a finite time-span. The assumption of action selection at a "slower" time scale allows for the eventual realization of real robots for manipulating waves by acoustic scatterers, which we defer to future work. In this paper, we focus on the development of methods in simulated environments and their demonstration on an important problem in acoustics: scattered wave energy manipulation.
\fi

% \subsection{Energy Dissipation}
% {\color{blue} \textbf{Commment:} Potentially move this section to appendix?}
% The presence of both the controllable signals, $d(t)$, and the uncontrollable (exogenous) signals, $\ell(x)$, such as energy (and/or material) influx, requires special consideration due to the potentially unlimited accumulation of energy (and/or material) within the solution domain. In acoustics, the sources of energy are often represented by time harmonic oscillations (cf., Figure. \ref{fig:wave}) which create an excitation that propagates outwards from a source location. Energy spreads out and eventually builds up if there is no way to remove it from the solution domain.

% An assumption is often made in acoustics that the environment being studied is an open space. This means that the boundaries of the finitely sized solution domain are open, and wave energy can propagate outwards infinitely and not return back to computational domain. This type of domain can be realized in simulations through the implementation of a \textit{perfectly matched layer} (PML) \cite{berenger_1994_a} or by applying analytical absorbing boundary conditions \cite{engquist_1977_absorbing}. 

% In this work, we implement our models by employing the PMLs.

\subsection{Wave Sensing}

% https://www.bksv.com/en/instruments/daq-data-acquisition/analyzer-system/acoustic-camera-9712
% BK Connect Acoustic Camera works with BK Connect software enabling easy, real-time noise source location and mapping. View transient sound sources on-site, or record and save video for subsequent analysis on stationary and moving objects.

% Dobler, Dick, and Gunnar Heilmann. "Perspective of the acoustic camera." INTER-NOISE and NOISE-CON Congress and Conference Proceedings. Vol. 2005. No. 6. Institute of Noise Control Engineering, 2005.

% Mueller, R.P., Brown, R.S., Hop, H. et al. Video and acoustic camera techniques for studying fish under ice: a review and comparison. Rev Fish Biol Fisheries 16, 213–226 (2006). https://doi.org/10.1007/s11160-006-9011-0

% Joël Busset, Florian Perrodin, Peter Wellig, Beat Ott, Kurt Heutschi, Torben Rühl, and Thomas Nussbaumer "Detection and tracking of drones using advanced acoustic cameras", Proc. SPIE 9647, Unmanned/Unattended Sensors and Sensor Networks XI; and Advanced Free-Space Optical Communication Techniques and Applications, 96470F (13 October 2015); https://doi.org/10.1117/12.2194309

In practice, a sensor which is capable of fully observing wave functions is not possible. Instead, real time total displacement and pressure fields can be observed on discretized grids through sensors such as microphone arrays and  acoustic cameras \cite{bocanegra2022_acoustic_camera}. In our simulated environment, the pixels of the sensor images are obtained by observing the displacement of the wave function at points on a uniformly spaced grid over $\Omega$. We assume the agent can only partially observe the wave function, $\f(x, t)$, via its sensors, which we  denote as images: $\mbox{Sensor}\bigl(\f(x, t)\bigr) = X(t)\in \mathbb{R}^{d_1\times d_2}$.
% \begin{align}
%     X(t) = \mbox{Sensor}\bigl(\f(x, t)\bigr).
% \end{align}
% The discrepancy between what a sensor can observe and the true state of $\f$ creates a fundamental uncertainty which affects sequential decision making.

% With the above definitions we can informally define the objective of the current work as follows.

% \textit{Given a finite number of interactions between the agent and partially observable environment, we aim to derive an optimal control policy, $\pi$, that influences the solution $\f$ in a desirable direction.}

% \textbf{Informal problem definition}. \textit{Given a finite set of measurements of environment, $\{X(t)\}_t$, we aim to derive an optimal control signal such that the true process, $\f$, will possess a desired properties in $\Omega$.}

% \iffalse
% We propose to design an artificial agent to solve this problem as follows. The agent partially observes the process, $\f(\cdot, t)$, through its sensors, $X(t)$, it calculates the optimal control signal (action), $a(t)$, which in turn modifies the design according to a known (and simple) dynamics, $F$, and it observes the new state of the process, $\f(\cdot, t')$. The interaction graph is summarized at Figure \ref{fig:graph}.
% \fi

% In this work we formulate this problem and propose an approach for its solution. We compare the proposed solution to the prior state-of-the-art.


In the next section, we define the control problem, and then propose a tractable method for its solution.  


% firstly, in the fully observable setting, and then, in the partial observable setting.