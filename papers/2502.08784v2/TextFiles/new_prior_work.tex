There have been several attempts to apply machine learning to the manipulation of complex phenomena governed by differential equations. Overall there has been less focus on designing systems that could be realized by real robots. 
% In this section, we review some of these previous efforts and discuss their limitations.

% One notable approach used a surrogate dynamics model to improve Reinforcement Learning (RL) for manipulation of PDE governed systems \cite{werner_2023_learning}. Since the primary goal of this work was to improve the training of RL algorithms, the authors did not prioritize realizable manipulation of fluids governed by 1D Kuramoto-Sivashinsky equations. For example, the authors assume that their agent had the ability to apply forces to the KS system at every point in the spatial domain. This is not a realistic assumption because in most cases fluid manipulation is achieved through sparse control with rotating cylinders \cite{bieker_2020_deep} or airfoil \cite{CHEN_XU_LU_2010}. Further limiting the realizability of this work is its assumption of full state observability. Their LSTM model was trained on data with the same resolution as simulation which restricts its transferability to real-world systems in which only sensor observations are available.

One approach used a surrogate dynamics model to enhance reinforcement learning for PDE-governed systems \cite{werner_2023_learning}. The authors did not prioritize realizability in this work, assuming their agent could apply forces across the entire spatial domain of the 1D Kuramoto-Sivashinsky (KS) system—a non-realistic scenario, as fluid manipulation typically involves sparse control through rotating cylinders \cite{bieker_2020_deep} or airfoils \cite{CHEN_XU_LU_2010}. Additionally, their reliance on full state observability, with an LSTM model trained on high-resolution simulation data, limits its applicability to real-world systems, where only sensor data is available.

A realizable approach to using ML for fluid dynamics manipulation was demonstrated through MPC of a Navier-Stokes governed system \cite{bieker_2020_deep}. The authors made specific choices to promote the realizability of their system, such as  using sparse control through rotating cylindrical devices as well as assuming partial observability through sensors. Similar to the example of KS manipulation \cite{werner_2023_learning}, this work also employed an LSTM architecture \cite{bieker_2020_deep}. While the authors demonstrated the success of their method, the sample efficiency of such an architecture may not be optimal for modeling PDE dynamics. Furthermore, the dynamics of an LSTM are not interpretable and do not provide a straightforward pathway to incorporating domain knowledge from fluid dynamics.

Recent works have explored the incorporation of physics knowledge in ML dynamics models \cite{nicodemus_2022_physicsinformed, ericaislanantonelo_2022_physicsinformed}. These works and others \cite{mraissi_2019_physicsinformed} have shown that incorporating such knowledge into the training of neural networks leads to better generalization and sample efficiency. In their work, the authors utilized Physics-Informed Neural Networks (PINNs) \cite{mraissi_2019_physicsinformed} to model the dynamics of simple systems governed by ODE in response to control. They demonstrated that their PINN based dynamics models could effectively control a mechanical multi-link manipulator \cite{nicodemus_2022_physicsinformed}, the Van der Pol Oscillator \cite{ericaislanantonelo_2022_physicsinformed}, and the classic Four Tank System \cite{ericaislanantonelo_2022_physicsinformed}. While these works demonstrated that it is possible to incorporate physics information into ML dynamics models, they are restricted to Ordinary Differential Equation (ODE) systems which are less general than PDEs are.

In the following sections, we present our approach which incorporates prior knowledge of PDEs into learnt dynamics models for the purpose of manipulating acoustic waves. In our work, we prioritize realizability, with the goal of enabling wave manipulation with real robots.