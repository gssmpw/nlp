%%%%%%%% ICML 2025 EXAMPLE LATEX SUBMISSION FILE %%%%%%%%%%%%%%%%%

\documentclass{article}

% Recommended, but optional, packages for figures and better typesetting:
\usepackage{microtype}
\usepackage{graphicx}
\usepackage{subfigure}
\usepackage{booktabs} % for professional tables

% hyperref makes hyperlinks in the resulting PDF.
% If your build breaks (sometimes temporarily if a hyperlink spans a page)
% please comment out the following usepackage line and replace
% \usepackage{icml2025} with \usepackage[nohyperref]{icml2025} above.
\usepackage{hyperref}


% Attempt to make hyperref and algorithmic work together better:
\newcommand{\theHalgorithm}{\arabic{algorithm}}

% Use the following line for the initial blind version submitted for review:
% \usepackage{icml2025}

% If accepted, instead use the following line for the camera-ready submission:
\usepackage[accepted]{icml2025}

% For theorems and such
\usepackage{amsmath}
\usepackage{amssymb}
\usepackage{mathtools}
\usepackage{amsthm}

% if you use cleveref..
\usepackage[capitalize,noabbrev]{cleveref}

%%%%%%%%%%%%%%%%%%%%%%%%%%%%%%%%
% THEOREMS
%%%%%%%%%%%%%%%%%%%%%%%%%%%%%%%%
\theoremstyle{plain}
\newtheorem{theorem}{Theorem}[section]
\newtheorem{proposition}[theorem]{Proposition}
\newtheorem{lemma}[theorem]{Lemma}
\newtheorem{corollary}[theorem]{Corollary}
\theoremstyle{definition}
\newtheorem{definition}[theorem]{Definition}
\newtheorem{assumption}[theorem]{Assumption}
\theoremstyle{remark}
\newtheorem{remark}[theorem]{Remark}

% Todonotes is useful during development; simply uncomment the next line
%    and comment out the line below the next line to turn off comments
%\usepackage[disable,textsize=tiny]{todonotes}
\usepackage[textsize=tiny]{todonotes}

% New Packages 
\usepackage{ulem}
\usepackage{xspace}
\newcommand{\dataset}{\textsc{MME-CoT}\xspace}
\usepackage{xcolor}
\newcommand{\td}[1]{\textcolor{red}{#1}}
\definecolor{mycolor_green}{HTML}{E6F8E0}
\definecolor{backred}{RGB}{255, 190, 190}
\definecolor{red}{RGB}{139, 0, 0}
\definecolor{purple}{HTML}{E6F8E0}


% \usepackage{xcolor}
% \definecolor{mycolor_green}{rgb}{0.68, 1, 0.18}
% \color{mycolor_green!80}
% \definecolor{mycolor_green}{rgb}{0.68, 1, 0.18}
% \color[rgb]{0.68,1,0.18,0.8}

% \definecolor{mycolor_green}{RGB}{210, 230, 250}
\definecolor{verylightgray}{HTML}{E6F8E0} % 这里的0.95表示非常浅的灰色,你可以调整这个值
% \definecolor{verylightgray}{gray}{0.85} % 这里的0.95表示非常浅的灰色,你可以调整这个值
\definecolor{lightgray}{gray}{0.95} % 这里的0.95表示非常浅的灰色,你可以调整这个值
\newcommand{\xf}[1]{\textcolor{blue}{#1}}
\usepackage{multirow}
\usepackage{colortbl}
\usepackage{makecell}
\usepackage{caption}
\usepackage{adjustbox}
\makeatletter
  \newcommand\figcaption{\def\@captype{figure}\caption}
  \newcommand\tabcaption{\def\@captype{table}\caption}
\makeatother
\usepackage[most]{tcolorbox} % Include tcolorbox package
\DeclareMathOperator*{\argmax}{arg\,max}
\usepackage{tabularx}

% The \icmltitle you define below is probably too long as a header.
% Therefore, a short form for the running title is supplied here:
\icmltitlerunning{MME-CoT: Benchmarking Chain-of-Thought in LMMs for Reasoning Quality, Robustness, and Efficiency}

\begin{document}

\twocolumn[
\icmltitle{MME-CoT: Benchmarking Chain-of-Thought in Large Multimodal Models\\for Reasoning Quality, Robustness, and Efficiency}

% \icmltitle{MME-CoT: Benchmarking LMMs for\\Chain-of-Thought Quality, Reasoning Stability, and Reflection Efficiency}


% \icmltitle{\dataset: Unveiling Chain-of-Thought Capabilities in LMMs with\\Reasoning Quality, Stability, and Reflection}

% \icmltitle{\dataset: Benchmarking and Evaluating Multimodal LLM CoT Capabilities}

% It is OKAY to include author information, even for blind
% submissions: the style file will automatically remove it for you
% unless you've provided the [accepted] option to the icml2025
% package.

% List of affiliations: The first argument should be a (short)
% identifier you will use later to specify author affiliations
% Academic affiliations should list Department, University, City, Region, Country
% Industry affiliations should list Company, City, Region, Country

% You can specify symbols, otherwise they are numbered in order.
% Ideally, you should not use this facility. Affiliations will be numbered
% in order of appearance and this is the preferred way.
\icmlsetsymbol{equal}{*}
\begin{icmlauthorlist}
\icmlauthor{Dongzhi Jiang$^{*}$}{CUHK MMLab}\textbf{, }
\icmlauthor{Renrui Zhang$^{*\dagger}$}{CUHK MMLab}\textbf{, }
\icmlauthor{Ziyu Guo}{CUHK MiuLar Lab}\textbf{, }
\icmlauthor{Yanwei Li$^{\ddagger}$}{ByteDance}\textbf{, }
\icmlauthor{Yu Qi$^{\ddagger}$}{Northeastern University}\textbf{, }
\icmlauthor{Xinyan Chen$^{\ddagger}$}{CUHK MMLab}\\
\icmlauthor{Liuhui Wang$^{\ddagger}$}{UPenn}\textbf{, }
\icmlauthor{Jianhan Jin$^{\ddagger}$}{NJU}\textbf{, }
\icmlauthor{Claire Guo$^{\ddagger}$}{CUHK (Shenzhen)}\textbf{, }
\icmlauthor{Shen Yan}{ByteDance}\textbf{, }
\icmlauthor{Bo Zhang}{Shanghai AI Laboratory}\\
\icmlauthor{Chaoyou Fu}{NJU}\textbf{, }
\icmlauthor{Peng Gao}{Shanghai AI Laboratory}\textbf{, }
\icmlauthor{Hongsheng Li}{CUHK MMLab}\\
\vspace{0.2cm}
\begin{tabular}{c}
\textsuperscript{1} CUHK MMLab\quad
\textsuperscript{2} CUHK MiuLar Lab\quad
\textsuperscript{3} ByteDance\quad
\textsuperscript{4} NEU\quad
% \textsuperscript{5} USTC\quad
\textsuperscript{5} UPenn\\
\textsuperscript{6} NJU\quad
\textsuperscript{7} CUHK (Shenzhen)\quad
\textsuperscript{8} Shanghai AI Laboratory
\\
\vspace{0.5em}
\texttt{\{dzjiang,renruizhang\}@link.cuhk.edu.hk}\\
\vspace{0.1cm}
$^{*}$ Core contribution\quad$^{\dagger}$ Project lead\quad$^{\ddagger}$ Equal contribution
    % \texttt{\{yangxw,zhuxy,weiwd,zhangdc,shaojj,zhouz,guolz,liyf\}@lamda.nju.edu}
   \end{tabular} 
% \vspace{0.3cm}
\\
Project Page: \url{https://mmecot.github.io/}

\end{icmlauthorlist}

% \icmlaffiliation{nju}{National Key Laboratory for Novel Software Technology, Nanjing University, China}
% \icmlaffiliation{njuai}{School of Artificial Intelligence, Nanjing University, China}
% \icmlaffiliation{njusz}{School of Intelligence Science and Technology, Nanjing University, China}
% \icmlaffiliation{CUHK}{MMLab, CUHK}
% \icmlaffiliation{shlab}{Shanghai AI Laboratory}
% \icmlaffiliation{xijiao}{Xi’an Jiaotong University, China}

% \icmlaffiliation{yyy}{Department of XXX, University of YYY, Location, Country}
% \icmlaffiliation{comp}{Company Name, Location, Country}
% \icmlaffiliation{sch}{School of ZZZ, Institute of WWW, Location, Country}

% \icmlaffiliation{CUHK MMLab}
% \icmlaffiliation{CUHK MiuLar Lab}
% \icmlaffiliation{ByteDance}
% \icmlaffiliation{Shanghai AI Laboratory}
% \icmlaffiliation{Northeastern University}
% \icmlaffiliation{USTC}
% \icmlaffiliation{UPenn}
% \icmlaffiliation{CUHK (Shenzhen)}
% \icmlaffiliation{NJU}

% \icmlcorrespondingauthor{Lan-Zhe Guo}{guolz@lamda.nju.edu.cn}
% \icmlcorrespondingauthor{Yu-Feng Li}{liyf@lamda.nju.edu.cn}
\icmlcorrespondingauthor{Dongzhi Jiang, Renrui Zhang}{dzjiang, renruizhang@link.cuhk.edu.hk}

\vskip 0.3in
]




% \icmlsetsymbol{equal}{*}
% \icmlsetsymbol{project_lead}{$\dagger$}

% \begin{icmlauthorlist}
% \icmlauthor{Dongzhi Jiang$^{*}$}{CUHK}\textbf{, }
% \icmlauthor{Renrui Zhang$^{*\dagger}$}{CUHK}
% \icmlauthor{Ziyu Guo}{CUHK}
% \icmlauthor{Yanwei Li}{sch}
% \icmlauthor{Yu Qi}{yyy}
% \icmlauthor{Xinyan Chen}{sch,yyy,comp}
% \icmlauthor{Liuhui Wang}{sch}\\
% \icmlauthor{Jianhan Jin}{comp}
% \icmlauthor{Claire Guo}{yyy,comp}
% \icmlauthor{Shen Yan}{sch}
% \icmlauthor{Bo Zhang}{sch}
% \icmlauthor{Chaoyou Fu}{sch}
% \icmlauthor{Peng Gao}{shlab}
% \icmlauthor{Hongsheng Li}{sch}
% %\icmlauthor{}{sch}
% %\icmlauthor{}{sch}
% \end{icmlauthorlist}

% % \icmlaffiliation{iii}{Department of XXX, University of YYY, Location, Country}
% % \icmlaffiliation{comp}{Company Name, Location, Country}
% \icmlaffiliation{CUHK}{MMLab, CUHK}
% \icmlaffiliation{shlab}{Shanghai AI Laboratory}

% % \icmlcorrespondingauthor
% % \icmlcorrespondingauthor{}{}
% \icmlcorrespondingauthor{Dongzhi Jiang, Renrui Zhang}{dzjiang, renruizhang@link.cuhk.edu.hk}

% % You may provide any keywords that you
% % find helpful for describing your paper; these are used to populate
% % the "keywords" metadata in the PDF but will not be shown in the document
% % \icmlkeywords{Machine Learning, ICML}



% \vskip 0.3in
% ]

% this must go after the closing bracket ] following \twocolumn[ ...

% This command actually creates the footnote in the first column
% listing the affiliations and the copyright notice.
% The command takes one argument, which is text to display at the start of the footnote.
% The \icmlEqualContribution command is standard text for equal contribution.
% Remove it (just {}) if you do not need this facility.

% \printAffiliationsAndNotice{}  % leave blank if no need to mention equal contribution
% \printAffiliationsAndNotice{} 
% \printAffiliationsAndNotice{\icmlEqualContribution} % otherwise use the standard text.

\begin{abstract}


The choice of representation for geographic location significantly impacts the accuracy of models for a broad range of geospatial tasks, including fine-grained species classification, population density estimation, and biome classification. Recent works like SatCLIP and GeoCLIP learn such representations by contrastively aligning geolocation with co-located images. While these methods work exceptionally well, in this paper, we posit that the current training strategies fail to fully capture the important visual features. We provide an information theoretic perspective on why the resulting embeddings from these methods discard crucial visual information that is important for many downstream tasks. To solve this problem, we propose a novel retrieval-augmented strategy called RANGE. We build our method on the intuition that the visual features of a location can be estimated by combining the visual features from multiple similar-looking locations. We evaluate our method across a wide variety of tasks. Our results show that RANGE outperforms the existing state-of-the-art models with significant margins in most tasks. We show gains of up to 13.1\% on classification tasks and 0.145 $R^2$ on regression tasks. All our code and models will be made available at: \href{https://github.com/mvrl/RANGE}{https://github.com/mvrl/RANGE}.

\end{abstract}


\section{Introduction}
Backdoor attacks pose a concealed yet profound security risk to machine learning (ML) models, for which the adversaries can inject a stealth backdoor into the model during training, enabling them to illicitly control the model's output upon encountering predefined inputs. These attacks can even occur without the knowledge of developers or end-users, thereby undermining the trust in ML systems. As ML becomes more deeply embedded in critical sectors like finance, healthcare, and autonomous driving \citep{he2016deep, liu2020computing, tournier2019mrtrix3, adjabi2020past}, the potential damage from backdoor attacks grows, underscoring the emergency for developing robust defense mechanisms against backdoor attacks.

To address the threat of backdoor attacks, researchers have developed a variety of strategies \cite{liu2018fine,wu2021adversarial,wang2019neural,zeng2022adversarial,zhu2023neural,Zhu_2023_ICCV, wei2024shared,wei2024d3}, aimed at purifying backdoors within victim models. These methods are designed to integrate with current deployment workflows seamlessly and have demonstrated significant success in mitigating the effects of backdoor triggers \cite{wubackdoorbench, wu2023defenses, wu2024backdoorbench,dunnett2024countering}.  However, most state-of-the-art (SOTA) backdoor purification methods operate under the assumption that a small clean dataset, often referred to as \textbf{auxiliary dataset}, is available for purification. Such an assumption poses practical challenges, especially in scenarios where data is scarce. To tackle this challenge, efforts have been made to reduce the size of the required auxiliary dataset~\cite{chai2022oneshot,li2023reconstructive, Zhu_2023_ICCV} and even explore dataset-free purification techniques~\cite{zheng2022data,hong2023revisiting,lin2024fusing}. Although these approaches offer some improvements, recent evaluations \cite{dunnett2024countering, wu2024backdoorbench} continue to highlight the importance of sufficient auxiliary data for achieving robust defenses against backdoor attacks.

While significant progress has been made in reducing the size of auxiliary datasets, an equally critical yet underexplored question remains: \emph{how does the nature of the auxiliary dataset affect purification effectiveness?} In  real-world  applications, auxiliary datasets can vary widely, encompassing in-distribution data, synthetic data, or external data from different sources. Understanding how each type of auxiliary dataset influences the purification effectiveness is vital for selecting or constructing the most suitable auxiliary dataset and the corresponding technique. For instance, when multiple datasets are available, understanding how different datasets contribute to purification can guide defenders in selecting or crafting the most appropriate dataset. Conversely, when only limited auxiliary data is accessible, knowing which purification technique works best under those constraints is critical. Therefore, there is an urgent need for a thorough investigation into the impact of auxiliary datasets on purification effectiveness to guide defenders in  enhancing the security of ML systems. 

In this paper, we systematically investigate the critical role of auxiliary datasets in backdoor purification, aiming to bridge the gap between idealized and practical purification scenarios.  Specifically, we first construct a diverse set of auxiliary datasets to emulate real-world conditions, as summarized in Table~\ref{overall}. These datasets include in-distribution data, synthetic data, and external data from other sources. Through an evaluation of SOTA backdoor purification methods across these datasets, we uncover several critical insights: \textbf{1)} In-distribution datasets, particularly those carefully filtered from the original training data of the victim model, effectively preserve the model’s utility for its intended tasks but may fall short in eliminating backdoors. \textbf{2)} Incorporating OOD datasets can help the model forget backdoors but also bring the risk of forgetting critical learned knowledge, significantly degrading its overall performance. Building on these findings, we propose Guided Input Calibration (GIC), a novel technique that enhances backdoor purification by adaptively transforming auxiliary data to better align with the victim model’s learned representations. By leveraging the victim model itself to guide this transformation, GIC optimizes the purification process, striking a balance between preserving model utility and mitigating backdoor threats. Extensive experiments demonstrate that GIC significantly improves the effectiveness of backdoor purification across diverse auxiliary datasets, providing a practical and robust defense solution.

Our main contributions are threefold:
\textbf{1) Impact analysis of auxiliary datasets:} We take the \textbf{first step}  in systematically investigating how different types of auxiliary datasets influence backdoor purification effectiveness. Our findings provide novel insights and serve as a foundation for future research on optimizing dataset selection and construction for enhanced backdoor defense.
%
\textbf{2) Compilation and evaluation of diverse auxiliary datasets:}  We have compiled and rigorously evaluated a diverse set of auxiliary datasets using SOTA purification methods, making our datasets and code publicly available to facilitate and support future research on practical backdoor defense strategies.
%
\textbf{3) Introduction of GIC:} We introduce GIC, the \textbf{first} dedicated solution designed to align auxiliary datasets with the model’s learned representations, significantly enhancing backdoor mitigation across various dataset types. Our approach sets a new benchmark for practical and effective backdoor defense.



\section{Background}
% \begin{tcolorbox}[simplebox]
% We first formally define the problem and highlight its challenge. 
% Then we present an EM approach to address this challenge. 
% \end{tcolorbox}
% \vspace{-0.3cm}
% \subsection{Problem Statement }\label{sec_ps}

% Here’s a polished and enriched version of your problem formulation section, with improved clarity, precision, and academic tone:

% ---
\begin{figure}[t]
    \centering % Center the figure
    \includegraphics[width=\linewidth]{figs/example.pdf} % Include the figure
    \caption{\small \textbf{Example of Autonomous Code Integration.} \small We aim to enable LLMs to determine tool-usage strategies
based on their own capability boundaries. In the example, the model write code to solve the problem that demand special tricks, strategically bypassing its inherent limitations.} 
    \label{fig_example}
    \vspace{-0.2cm}
\end{figure}
\textbf{Problem Statement.} Modern tool-augmented language models address mathematical problems \( x_q \in \mathcal{X}_Q \) by generating step-by-step solutions that interleave natural language reasoning with executable Python code (Fig.~\ref{fig_example}). Formally, given a problem \( x_q \), a model \( \mathcal{M}_\theta \) iteratively constructs a solution \( y_a = \{y_1, \dots, y_T\} \) by sampling components \( y_t \sim p(y_t | y_{<t}, x_q) \), where \( y_{<t} \) encompasses both prior reasoning steps, code snippets and execution results \( \mathbf{e}_t \) from a Python interpreter. The process terminates upon generating an end token, and the solution is evaluated via a binary reward \( r(y_a,x_q) = \mathbb{I}(y_a \equiv y^*) \) indicating equivalence to the ground truth \( y^* \). The learning objective is formulated as:
\[
\max_{\theta} \mathbb{E}_{x_q \sim \mathcal{X}_Q} \left[r(y_a, x_q) \right]
\]

\noindent\textbf{Challenge and Motivation.} Developing autonomous code integration (AutoCode) strategies poses unique challenges, as optimal tool-usage behaviors must dynamically adapt to a model's intrinsic capabilities and problem-solving contexts. While traditional supervised fine-tuning (SFT) relies on imitation learning from expert demonstrations, this paradigm fundamentally limits the emergence of self-directed tool-usage strategies. Unfortunately, current math LLMs predominantly employ SFT to orchestrate tool integration~\citep{mammoth, tora, dsmath, htl}, their rigid adherence to predefined reasoning templates therefore struggles with the dynamic interplay between a model’s evolving problem-solving competencies and the adaptive tool-usage strategies required for diverse mathematical contexts.

Reinforcement learning (RL) offers a promising alternative by enabling trial-and-error discovery of autonomous behaviors. Recent work like DeepSeek-R1~\citep{dsr1} demonstrates RL's potential to enhance reasoning without expert demonstrations. However, we observe that standard RL methods (e.g., PPO~\cite{ppo}) suffer from a critical inefficiency (see Sec.~\ref{sec_ablation}): Their tendency to exploit local policy neighborhoods leads to insufficient exploration of the vast combinatorial space of code-integrated reasoning paths, especially when only given a terminal reward in mathematical problem-solving.

To bridge this gap, we draw inspiration from human metacognition -- the iterative process where learners refine tool-use strategies through deliberate exploration, outcome analysis, and belief updates. A novice might initially attempt manual root-finding via algebraic methods, observe computational bottlenecks or inaccuracies, and therefore prompting the usage of calculators. Through systematic reflection on these experiences, they internalize the contextual efficacy of external tools, gradually forming stable heuristics that balance reasoning with judicious tool invocation. 


To this end, \emph{our focus diverges from standard agentic tool-use frameworks~\citep{agentr}}, which merely prioritize successful tool execution. Instead, \emph{we aim to instill \emph{human-like metacognition} in LLMs, enabling them to (1) determine tool-usage based on their own capability boundaries (see the analysis in Sec.~\ref{sec_ablation}), and (2) dynamically adapt tool-usage strategies as their reasoning abilities evolve (via our EM framework).}
% For instance, while an LLM might solve a combinatorics problem via CoT alone, it should autonomously invoke code for eigenvalue calculations in linear algebra where symbolic computations are error-prone. Achieving this requires models to \emph{jointly optimize} their reasoning and tool-integration policies in a mutually reinforcing manner.


% Mirroring this metacognitive cycle, we propose an Expectation-Maximization (EM) framework that allows LLMs to develop AutoCode strategies via guided exploration (the E-step) and self-refinement (the M-step).


% \vspace{-0.3cm}
\section{Methodology}

Inspired by human metacognitive processes, we introduce an Expectation-Maximization (EM) framework that trains LLMs for autonomous code integration (AutoCode) through alternations (Fig.~\ref{fig_overview}):

\begin{enumerate}[leftmargin=0.5cm,topsep=1pt,itemsep=0pt,parsep=0pt]
    \item \emph{Guided Exploration (E-step):} Identifies high-potential code-integrated solutions by systematically probing the model's inherent capabilities.
\item \emph{Self-Refinement (M-step):} Optimizes the model's tool-usage strategy and chain-of-thought reasoning using curated trajectories from the E-step.
\end{enumerate}


\begin{figure*}[t]
    \centering
    \includegraphics[width=\linewidth]{figs/overview.pdf}
    \caption{\small \textbf{Method Overview.} \small (Left) shows an overview for the EM framework, which alternates between finding a reference strategy for guided exploration (E-step) and off-policy RL (M-step). (Right) shows the data curation for guided exploration. We generate \(K\) rollouts, estimate values of code-triggering decisions and subsample the initial data with sampling weights per Eq.~\ref{eq_sampling}.}
    \label{fig_overview}
\end{figure*}

\subsection{The EM Framework for AutoCode}

A central challenge in AutoCode lies in the code triggering decisions, represented by the binary decision \(c \in \{0, 1\}\).  While supervised fine-tuning (SFT) suffers from missing ground truth for these decisions, standard reinforcement learning (RL) struggles with the combinatorial explosion of code-integrated reasoning paths. Our innovation bridges these approaches through systematic exploration of both code-enabled (\(c=1\)) and non-code (\(c=0\)) solution paths, constructing reference decisions for policy optimization.

We formalize this idea within a maximum likelihood estimation (MLE) framework. Let \( P (r=1 | x_q;\theta\) denote the probability of generating a correct response to query \( x_q \) under model \(\mathcal{M}_\theta\). Our objective becomes:
\begin{align}
    \mathcal{J}_{\mathrm{MLE}}(\theta) \doteq \log P(r=1 | x_q; \theta) \label{eq_mle}
\end{align}
This likelihood depends on two latent factors: (1) the code triggering decision \(\pi_\theta(c | x_q)\) and (2) the solution generation process \(\pi_\theta(y_a | x_q, c)\). Here, for notation-wise clarity, we consider  code-triggering decision at a solution's beginning (\( c\) following \(x_q\) immediately). We show generalization to mid-reasoning code integration in Sec.~\ref{sec_impl}.

The EM framework provides a principled way to optimize this MLE objective in the presence of latent variables~\cite{prml}. We derive the evidence lower bound (ELBO): \( \mathcal{J}_{\mathrm{ELBO}}(s, \theta) \doteq \)
\begin{align}
    % \mathcal{J}_{\mathrm{MLE}}(\theta) &
    % \ge 
    \mathbb{E}_{s(c | x_q)}\left[\log \frac{\pi_\theta(c | x_q) \cdot P(r=1 | c, x_q; \theta)}{s(c | x_q)}\right] 
    % \\
     \label{eq_elbo}
\end{align}
where \(s(c | x_q)\) serves as a surrogate distribution approximating optimal code triggering strategies. It is also considered as the reference decisions for code integration. 

\noindent\textbf{E-step: Guided Exploration}  computes the reference strategy \(s(c | x_q)\) by maximizing the ELBO, equivalent to minimizing the KL-divergence: \( \max_s \mathcal{J}_{\mathrm{ELBO}}(s, \theta) = \)
\begin{align}
     - \mathrm{D_{KL}}\left(s(c | x_q) \| P(r=1, c | x_q; \theta)\right) \label{eq_estep}
\end{align}

The reference strategy \(s(c | x_q)\) thus approximates the posterior distribution over code-triggering decisions \(c\) that maximize correctness, i.e., \(P(r=1, c | x_q; \theta)\).  Intuitively, it guides exploration by prioritizing decisions with high potential: if decision \(c\) is more likely to lead to correct solutions, the reference strategy assigns higher probability mass to it, providing guidance for the subsequent RL procedure.

\noindent\textbf{M-step: Self-Refinement } updates the model parameters \(\theta\) through a composite objective:
\begin{multline}
\max_\theta \mathcal{J}_{\mathrm{ELBO}}(s, \theta) =\mathbb{E}_{\substack{c \sim s(c|x_q) \\ y_a \sim \pi_\theta(y_a|x_q, c)}} \Big[ r(x_q, y_a) \Big] \\- \mathcal{CE}\Big(s(c|x_q) \,\|\, \pi_\theta(c|x_q)\Big)\label{eq_mstep}
\end{multline}
The first term implements reward-maximizing policy gradient updates for solution generation, while while the second aligns native code triggering with reference strategies through cross-entropy minimization (see Fig.~\ref{fig_overview} for an illustration of the optimization). This dual optimization jointly enhances both tool-usage policies and reasoning capabilities.



\subsection{Practical Implementation}\label{sec_impl}
In the above EM framework, we alternate between finding a reference strategy \( s \) for code-triggering decisions  in the E-step, and perform reinforcement learning under the guidance from \( s \) in the M-step. We implement this framework through an iterative process of offline data curation and off-policy RL.

\noindent\textbf{Offline Data Curation.} We implement the E-step through Monte Carlo rollouts and subsampling. For each problem \(x_q\), we estimate the reference strategy as an energy distribution: 
\begin{equation}
    s^\ast(c | x_q)  = \frac{\exp\left(\alpha\cdot \pi_\theta(c | x_q) Q(x_q,c;\theta)\right)}{Z(x_q)}.\label{eq_sampling}
\end{equation}
where \( Q(x_q,c;\theta)\) estimates the expected value through \( K \) rollouts per decision, \(\pi_\theta(c|x_q) \) represents the model's current prior and the \( Z(x_q) \) is the partition function to ensure normalization. Intuitively, the strategy will assign higher probability mass to the decision \( c \) that has higher expected value \( Q(x_q,c;\theta)\) meanwhile balancing its intrinsic preference \( \pi_\theta(c|x_q)\). 

Our curation pipeline proceeds through: 
\begin{itemize}[leftmargin=0.5cm,topsep=1pt,itemsep=0pt,parsep=0pt]
\item Generate \(K\) rollouts for \(c=0\) (pure reasoning) and \(c=1\) (code integration), creating candidate dataset \(\mathcal{D}\).  
\item Compute \(Q(x_q,c)\) as the expected success rate across rollouts for each pair \((x_q,c)\).  
\item Subsample \(\mathcal{D}_{\text{train}}\) from \(\mathcal{D}\) using importance weights according to Eq.~\ref{eq_sampling}.  
\end{itemize}

To explicitly probe code-integrated solutions, we employ prefix-guided generation -- e.g., prepending prompts like \texttt{``Let’s first analyze the problem, then consider if python code could help''} -- to bias generations toward free-form code-reasoning patterns.

 This pipeline enables guided exploration by focusing on high-potential code-integrated trajectories identified by the reference strategy, contrasting with standard RL’s reliance on local policy neighborhoods. As demonstrated in Sec.~\ref{sec_ablation}, this strategic data curation significantly improves training efficiency by shaping the exploration space.





\noindent\textbf{Off-Policy RL.}
To mitigate distributional shifts caused by mismatches between offline data and the policy, we optimize a clipped off-policy RL objective. The refined M-step (Eq.~\ref{eq_mstep}) becomes:
\begin{multline}
    % \max_\theta 
    \underset{(x_q,y_a)}{\mathbb{E}}\left[
\text{clip}\left(\frac{\pi_\theta(y_a|x_q)}{\pi_{\text{ref}}(y_a|x_q)},1-\epsilon,1+\epsilon\right)\cdot A\right]
\\-\mathbb{E}_{(x_q,c)}\Big[\log \pi_\theta(c|x_q) \Big]\label{eq_finalm}
\end{multline}
where  \( (x_q, c, y_a) \) is sampled from the dataset \( \mathcal{D}_{\text{train}} \). The importance weight \(\frac{\pi_\theta(y_a|x_q)}{\pi_{\text{ref}}(y_a|x_q)}\) accounts for off-policy correction with PPO-like clipping. The advantage function \(A(x_q,y_a)\) is computed via query-wise reward normalization~\cite{ppo}. 

\noindent\textbf{Generalizing to Mid-Reasoning Code Integration.} Our method extends to mid-reasoning code integration by initiating Monte Carlo rollouts from partial solutions \((x_q, y_{<t})\). Notably, we observe emergence of mid-reasoning code triggers after initial warm-up with prefix-probed solutions. Thus, our implementation requires only two initial probing strategies: explicit prefix prompting for code integration and vanilla generation for pure reasoning, which jointly seed diverse mid-reasoning code usage in later iterations.

\section{Experiments}
In this section, we conduct a systematic evaluation of state-of-the-art models on \dataset. We first detail the experiment setup in Section~\ref{sec:exp_steup}. Then in Section~\ref{sec:exp_quantitative}, we report the quantitative results and provide valuable insights derived from our analysis.

\subsection{Experiment Setup}
\label{sec:exp_steup}
\paragraph{Evaluation Models.} 
We select top-performing LMMs for comprehensive CoT evaluation. We test earlier models such as LLaVA-OneVision (7B, 72B)~\cite{li2024llava-ov}, Qwen2-VL (7B, 72B)~\cite{Qwen2-VL}, MiniCPM-V-2.6~\cite{yao2024minicpm}, and InternVL2.5 (8B)~\cite{chen2024expanding}, which are not trained for the reasoning capability. We also include GPT-4o~\cite{openai2024gpt4o} as a strong baseline model.
Besides, we test recent models targeting reasoning, including LLaVA-CoT (11B)~\cite{xu2024llavacot}, Mulberry (8B)~\cite{yao2024mulberry}, InternVL2.5-MPO (8B, 78B)~\cite{wang2024mpo}.
Finally, we evaluate LMMs with reflection capabilities, including both closed-source models like Kimi k1.5~\cite{team2025kimi} and open-source implementations such as QVQ-72B~\cite{qvq-72b-preview} and Virgo-72B~\cite{du2025virgo}.

Note that we sample 150 questions from \dataset to evaluate Kimi k1.5, due to the access limitations. The sample comprises 115 reasoning and 35 perception questions. 

\begin{table*}[!t]
\centering
\caption{\textbf{Evaluation Results of Three Aspects of CoT in Each Category in \dataset.} Best performance is marked in \colorbox{backred!60}{red}.  $*$ denotes unreliable results due to the refusal to answer directly.}
\vspace{-3pt}
\renewcommand\tabcolsep{2.0pt}
\renewcommand\arraystretch{1.25}
\resizebox{1.0\linewidth}{!}{
\begin{tabular}{l|ccc|ccc|ccc|cc|cc|cc}
\toprule
\multirow{2}*{\makecell*[l]{\large Model}} & \multicolumn{3}{c|}{\makecell*[c]{General Scenes}} & \multicolumn{3}{c|}{Space-Time} & \multicolumn{3}{c|}{OCR} & \multicolumn{2}{c|}{Math} & \multicolumn{2}{c|}{Science} & \multicolumn{2}{c}{Logic} \\
& Quality & Robustness & Efficiency & Quality & Robustness & Efficiency & Quality & Robustness & Efficiency & Quality & Efficiency & Quality & Efficiency & Quality & Efficiency \\
\midrule
Mulberry & 33.9 & \colorbox{backred!60}{4.3} & 76.0 & 18.2 & 1.0 & 38.4 & 26.7 & \colorbox{backred!60}{6.6} & 26.4 & 29.1 & 87.9 & 29.1 & 91.9 & 13.9 & \colorbox{backred!60}{99.1} \\
LLaVA-OV-7B & 41.8 & -6.2 & 81.8 & 23.8 & -6.7 & 24.8 & 44.1 & -0.2 & 42.7 & 27.4 & 97.3 & 28.5 & 95.1 & 12.2 & 98.0 \\
LLaVA-CoT & 38.2 & -2.2 & 89.9 & 33.6 & 2.8 & 68.9 & 37.4 & 0.0 & 77.8 & 35.3 & 91.0 & 36.4 & 93.4 & 14.9 & 97.1 \\
LLaVA-OV-72B & 41.8 & -2.3 & \colorbox{backred!60}{98.9} & 29.0 & -0.9 & 43.6 & 40.8 & -1.7 & 84.2 & 38.4 & 98.7 & 35.4 & 95.7 & 18.4 & 82.3 \\
MiniCPM-V-2.6 & 47.1 & 3.2 & 87.7 & 49.3 & -14.4 & 71.1 & 63.7 & -4.9 & 62.0 & 32.9 & 95.2 & 29.5 & 90.4 & 16.9 & 93.7 \\
InternVL2.5-8B & 43.8 & -6.4 & 87.1 & 50.7 & -8.9 & \colorbox{backred!60}{99.1} & 44.7 & -4.1 & \colorbox{backred!60}{98.9} & 40.9 & 98.0 & 40.8 & 97.1 & 19.5 & 96.8 \\
Qwen2-VL-7B & 46.7 & -3.4 & 79.3 & 51.7 & -11.8 & 73.0 & 65.9 & 0.9 & 86.2 & 34.0 & 97.9 & 34.6 & 95.0 & 18.4 & 76.7 \\
InternVL2.5-8B-MPO & 47.2 & 2.9 & 94.3 & 51.8 & -0.2 & 74.6 & 59.6 & -1.0 & 81.5 & 37.4 & 93.4 & 39.0 & 95.6 & 20.9 & 79.9 \\
InternVL2.5-78B-MPO & 47.9 & 0.0 & 89.3 & 55.5 & -2.3 & 91.9 & 72.2 & 2.2 & 73.1 & 50.6 & 95.1 & 48.5 & 97.7 & 24.2 & 87.2 \\
Qwen2-VL-72B & 51.9 & -2.9 & 88.9 & 59.7 & -5.3 & 86.7 & 77.6 & 2.5 & 81.7 & 49.6 & 97.8 & 53.6 & \colorbox{backred!60}{99.0} & 40.0 & 88.0 \\
Virgo-72B & 60.5 & 0.5 & 91.0 & 59.6 & -3.8 & 86.0 & 79.9 & -1.0 & 82.1 & 59.6 & 90.3 & 55.5 & 98.7 & 39.6 & 88.2 \\
QVQ-72B & \colorbox{backred!60}{62.6} & -1.5 & 86.9 & 58.2 & -2.5 & 57.7 & 76.9 & -1.4 & 52.6 & \colorbox{backred!60}{61.4} & 92.7 & 57.7 & 95.9 & \colorbox{backred!60}{44.6} & 94.9 \\
GPT4o & 62.3 & -1.7 & 96.2 & \colorbox{backred!60}{66.3} & \colorbox{backred!60}{5.5} & 64.7 & \colorbox{backred!60}{83.3} & -1.0 & 82.1 & 60.8 & \colorbox{backred!60}{98.8} & \colorbox{backred!60}{64.1} & 97.4 & 27.2 & 92.0 \\
\bottomrule
\end{tabular}
}
\label{table:category_result}
% \vspace{-0.3cm}
\end{table*}


\paragraph{Implementation Details.}
We define the CoT prompt as: \textit{Please generate a step-by-step answer, include all your intermediate reasoning process, and provide the final answer at the end.} and the direct prompt as: \textit{Please directly provide the final answer without any other output.}
We only calculate recall of image observation and logical inference on questions where key inference conclusion or image observation exists.
We employ GPT-4o mini for the direct evaluation and GPT-4o for all other criteria. For hyperparameters, we follow the settings in VLMEvalKit~\cite{duan2024vlmevalkit}. 

\subsection{Quantitative Results}
\label{sec:exp_quantitative}
We conduct extensive experiments on various LMMs with our proposed CoT evaluation suite. 
The main results are presented in Table~\ref{table:main_result} and Table~\ref{table:category_result}. We begin by analyzing the overall performance and then highlight key findings.
\paragraph{Overall Results.}
In Table~\ref{table:main_result}, we present
the overall performance of three CoT evaluation perspectives with specific metrics. 
To provide a comprehensive understanding, we report precision, recall, and relevance for both logical inference and image caption steps. For robustness, we provide the direct evaluation result on the perception and reasoning tasks, with either CoT or direct prompt. We employ the average value of the stability and efficacy as the final robustness metric. Notably, we define the reflection quality as 100 on models incapable of reflection.

For CoT quality, Kimi k1.5 achieves the highest F1 score. Open-source models with larger sizes consistently demonstrate better performance, highlighting the scalability of LMMs. Notably, Qwen2-VL-72B outperforms all other open-source models without reflection, even surpassing InternVL2.5-78B-MPO, which is specifically enhanced for reasoning. Analysis reveals that GPT-4o achieves superior performance across all recall metrics, while Kimi k1.5 demonstrates the highest scores in precision evaluations.
For CoT robustness, Mulberry obtains the highest average score. However, when we look into its output, we find it still generates lengthy rationales despite receiving a direct prompt. Even worse, the direct prompt seems to be an out-of-distribution input for Mulberry, 
frequently leading to nonsensical outputs. Further analysis of other models’ predictions reveals that LLaVA-CoT, Virgo, QVQ, and Kimi k1.5 similarly neglect the direct prompt, instead generating extended rationales before answering. Consequently, their robustness scores may be misleading. Once again, GPT-4o achieves the highest robustness score. Among open-source models, only InternVL2.5-MPO, in both its 8B and 78B variants, attains a positive robustness score.
Finally, for CoT efficiency, InternVL2.5-8B obtains the maximum relevance of 98.4\%, suggesting its consistent focus on questions.

Now, we summarize our key observations as follows:
\paragraph{\textit{Models with reflection largely benefit CoT quality.}}
As shown in Table~\ref{table:main_result}, the F1 scores of the two models with reflection capability most closely approach GPT-4o. After specifically fine-tuning for the reasoning capabilities from Qwen2-VL-72B, QVQ surpasses its base model by 5.8\%. Notably, although QVQ generates longer CoT sequences than Qwen2-VL-72B, QVQ's precision still exceeds Qwen2-VL-72B by 2.9\%, indicating superior accuracy in each reasoning step. Kimi k1.5 also surpasses the previous state-of-the-art model GPT-4o, obtaining the highest CoT quality.


\paragraph{\textit{Long CoT does not necessarily cover key steps.}} 
Despite high precision in long CoT models, the informativeness of each step is not guaranteed. We observe that the recall trend among GPT-4o, QVQ, and Virgo does not align with their CoT Rea. performance (i.e., their final answer accuracy on the reasoning tasks under the CoT prompt). Specifically, while both Virgo and QVQ outperform GPT-4o in direct evaluation, they lag behind in recall. This suggests that long CoT models sometimes reach correct answers while skipping intermediate steps, which contradicts the principle of stepwise reasoning and warrants further investigation.

\paragraph{\textit{CoT impairs perception task performance in most models.}}% 比较stability
Surprisingly, most models exhibit negative stability scores, indicating that CoT interferes with perception tasks. The most significant degradation occurs in InternVL2.5-8B, where performance drops by 6.8\%. This reveals inconsistency and potential overthinking in current models, presenting a significant barrier to adopting CoT as the default answering strategy. Among models that provide direct answers, only LLaVA-OV-72B and InternVL2.5-8B-MPO achieve a modest positive score of 0.3\%.

\paragraph{\textit{More parameters enable models to grasp reasoning better.}} 
We find that models with larger parameter counts tend to achieve higher efficacy scores. This pattern is evident across LLaVA-OV, InternVL2.5-MPO, and Qwen2-VL. For instance, while Qwen2-VL-7B shows a 4.8\% decrease in performance when applying CoT to reasoning tasks, its larger counterpart, Qwen2-VL-72B, demonstrates a 2.4\% improvement. This discrepancy suggests that models with more parameters could better grasp the reasoning ability under the same training paradigm. 


\paragraph{\textit{Long CoT models may be more susceptible to distraction.}} 
Long CoT models may demonstrate lower relevance scores compared to other models. They frequently generate content unrelated to solving the given question, corresponding to their relatively low recall scores compared to direct evaluation, like QVQ. Although a few models with short CoT, like Mulberry and LLaVA-OV-7B, also obtain a low relevance rate, we find that it is because these models may keep repeating words when dealing with specific type of questions, resulting in irrelevant judgment. The fine-grained metric reveals that models tend to lose focus when describing images, often producing exhaustive captions regardless of their relevance to the question. From Table~\ref{table:category_result}, we find that this phenomenon prevails in general scenes, space-time, and OCR tasks. This behavior can significantly slow inference by generating substantial irrelevant content. Teaching long CoT models to focus on question-critical elements represents a promising direction for future research.


\paragraph{\textit{Reflection often fails to help.}} 
While reflection is a key feature of long CoT models for answer verification, both QVQ and Virgo achieve reflection quality scores of only about 60\%, indicating that approximately 40\% of reflection attempts fail to contribute meaningfully to answer accuracy. Even for the closed-source model Kimi k1.5, over 25\% reflection steps are also invalid. This substantial failure rate compromises efficiency by potentially introducing unnecessary or distracting steps before reaching correct solutions. Future research should explore methods to reduce these ineffective reflections to improve both efficiency and quality.

\begin{figure}[t]
\begin{center}
\vspace{0.2cm}
\centerline{\includegraphics[width=0.8\columnwidth]{fig/ref_error_pie.pdf}}
\caption{\textbf{Distribution of Reflection Error Types.} We identify four types of error: ineffective reflection, incompleteness, repetition, and interference.}
\label{fig:ref_error_distribution}
\end{center}
\vspace{-0.6cm}
\end{figure}

\subsection{Error Analysis}
\label{sec:exp_analysis}
In this section, we analyze error patterns in the LMM reflection process. An effective reflection should either correct previous mistakes or validate correct conclusions through new insights. We examined 200 model predictions from QVQ and identified four distinct error types that hinder productive reflection. These patterns are illustrated in Fig.~\ref{fig:ref_error_example} and their distribution is shown in Fig.~\ref{fig:ref_error_distribution}.

The four major error types are:

\begin{itemize}
    \item \textbf{Ineffective Reflection.} The model arrives at an incorrect conclusion and, upon reflecting, continues to make incorrect adjustments. This is the most common error type and is also witnessed most frequently.
    \item \textbf{Incompleteness.} The model proposes new analytical approaches but does not execute them, only stopping at the initial thought. The reflection slows down the inference process without bringing any gain.
    \item \textbf{Repetition.} The model restates previous content or methods without introducing new insights, leading to inefficient reasoning.
    \item \textbf{Interference.} The model initially reaches a correct conclusion but, through reflection, introduces errors.
\end{itemize}

Understanding and mitigating these errors is crucial for improving the reliability of LMM reflection mechanisms. The analysis provides the opportunity to focus on solving specific error types to enhance the overall reflection quality.


\section{Conclusion}
\label{sec:conclu}

In this study, we propose a retrieval-augmented approach to extend LLM-based TabICL from zero-shot and few-shot settings to any-shot scenarios.
This approach explores the potential of using text representations for tabular data learning, enables the creation of unique decision boundaries, and achieves highly competitive prediction performance across most tabular datasets.

Despite the unique strengths and promising potentials, we also acknowledge the limitations of this approach at the current stage, such as the absence of a universally effective retrieval policy, challenges in handling certain long-tail data distributions, and sub-optimal performance in several scenarios.
Given the demonstrated strengths of this approach, we believe that the potential of LLM-based TabICL is still in its early stages, and these limitations present valuable opportunities for future research and development.



\bibliography{example_paper}
\bibliographystyle{icml2025}

\clearpage
\renewcommand{\thefigure}{A\arabic{figure}}
\renewcommand{\thetable}{A\arabic{table}}
\renewcommand{\theequation}{A\arabic{equation}}
\setcounter{figure}{0}
\setcounter{table}{0}
\setcounter{equation}{0}

Our Appendix is organized as follows. First, we present the pseudocode for the key components of iGCT. We also include the proof for unit variance and boundary conditions in preconditioning iGCT's noiser. Next, we detail the training setups for our CIFAR-10 and ImageNet64 experiments. Additionally, we provide ablation studies on using guided synthesized images as data augmentation in image classification. Finally, we present more uncurated results comparing iGCT and CFG-EDM on inversion, editing and guidance, thoroughly of iGCT.

\vspace{-0.2cm}
\label{appendix:iGCT}
\section{Pseudocode for iGCT}
\vspace{-0.2cm}

iGCT is trained under a continuous-time scheduler similar to the one proposed by ECT \cite{ect}. Our noise sampling function follows a lognormal distribution, \(p(t) = \textit{LogNormal}(P_\textit{mean}, P_\textit{std})\), with \(P_\textit{mean}=-1.1\) and \( P_\textit{std}=2.0\). At training, the sampled noise is clamped at \(t_\text{min} = 0.002\) and \(t_\text{max} = 80.0\). Step function \(\Delta t (t)=\frac{t}{2^{\left\lfloor k/d \right\rfloor}}n(t)\), is used to compute the step size from the sampled noise \(t\), with \(k,d\) being the current training iteration and the number of iterations for halfing \(\Delta t\), and \(n(t) = 1 + 8 \sigma(-t)\) is a sigmoid adjusting function. 

In Guided Consistency Training, the guidance mask function determines whether the sampled noise \( t \) should be supervised for guidance training. With probability \( q(t) \in [0,1] \), the update is directed towards the target sample \( \boldsymbol{x}_0^{\text{tar}} \); otherwise, no guidance is applied. In practice, \( q(t) \) is higher in noisier regions and zero in low-noise regions, 
\begin{equation}
    q(t) = 0.9 \cdot \left( \text{clamp} \left( \frac{t - t_{\text{low}}}{t_{\text{high}} - t_{\text{low}}}, 0, 1 \right) \right)^2,
\end{equation}
where \( t_{\text{low}} = 11.0 \) and \( t_{\text{high}} = 14.3 \). For the range of guidance strength, we set \(w_\text{min} = 1\) and \(w_\text{max} = 15\). Guidance strengths are sampled uniformly at training, with \(w_\text{min} = 1\) means no guidance applied. 


\begin{algorithm}
\caption{Guided Consistency Training}
\label{alg:GCT}
\begin{algorithmic}[1]  % Adds line numbers
\setlength{\baselineskip}{0.9\baselineskip} % Adjust line spacing
\INPUT Dataset $\mathcal{D}$, weighting function $\lambda(t)$, noise sampling function $p(t)$, noise range $[t_\text{min}, t_\text{max}]$, step function $\Delta t(t)$, guidance mask function $q(t)$, guidance range $[w_\text{min}, w_\text{max}]$, denoiser $D_\theta$
\STATE \rule{0.96\textwidth}{0.45pt} 
\STATE Sample $(\boldsymbol{x}_0^{\text{src}}, c^{\text{src}}), (\boldsymbol{x}_0^{\text{tar}}, c^{\text{tar}}) \sim \mathcal{D}$ 
\STATE Sample noise $\boldsymbol{z} \sim \mathcal{N}(\boldsymbol{0},\mathbf{I})$, time step $t \sim p(t)$, and guidance weight $w \sim \mathcal{U}(w_\text{min}, w_\text{max})$
\STATE Clamp $t \leftarrow \text{clamp}(t,t_\text{min}, t_\text{max})$
\STATE Compute noisy sample: $\boldsymbol{x}_t = \boldsymbol{x}_0^{\text{src}} + t\boldsymbol{z}$
\STATE Sample $\rho \sim \mathcal{U}(0,1)$  
\vspace{0.3em}
\IF{$\rho > q(t)$}
    \STATE Compute step as normal CT: $\boldsymbol{x}_r = \boldsymbol{x}_t - \Delta t(t) \boldsymbol{z}$
    \STATE Set target class: $c \leftarrow c^{\text{src}}$
\ELSE
    \STATE Compute guided noise: $\boldsymbol{z}^* = (\boldsymbol{x}_t - \boldsymbol{x}_0^{\text{tar}}) / t$
    \STATE Compute guided step: $\boldsymbol{x}_r = \boldsymbol{x}_t - \Delta t(t) [w \boldsymbol{z}^* + (1-w)\boldsymbol{z}]$
    \STATE Set target class: $c \leftarrow c^{\text{tar}}$
\ENDIF
\vspace{0.3em} % Reduces extra vertical space before the loss line
\STATE Compute loss: 
\[
\mathcal{L}_\text{gct} = \lambda(t) \, d(D_{\theta}(\boldsymbol{x}_t, t, c, w), D_{{\theta}^-}(\boldsymbol{x}_r, r, c, w))
\]
\STATE Return $\mathcal{L}_\text{gct}$ 
\end{algorithmic}
\end{algorithm}



A \textit{noiser} trained under \textit{Inverse Consistency Training} maps an image to its latent noise in a single step. In contrast, DDIM Inversion requires multiple steps with a diffusion model to accurately produce an image's latent representation. Since the boundary signal is reversed, spreading from \( t_\text{max} \) down to \( t_\text{min} \), we design the importance weighting function \( \lambda'(t) \) to emphasize higher noise regions, defined as:
\begin{equation}
    \lambda'(t) = \frac{\Delta t (t)}{t_\text{max}},
\end{equation}
where the step size \( \Delta t (t) \) is proportional to the sampled noise level \(t\), and \( t_\text{max} \) is a constant that normalizes the scale of the inversion loss. The noise sampling function \( p(t) \) and the step function \( \Delta t (t) \) used in computing both \(\mathcal{L}_\text{gct}\) and \(\mathcal{L}_\text{ict}\) are the same.



\begin{algorithm}
\caption{Inverse Consistency Training}
\label{alg:iCT}
\begin{algorithmic}[1]  % Adds line numbers
\setlength{\baselineskip}{0.9\baselineskip} % Adjust line spacing
\INPUT Dataset $\mathcal{D}$, weighting function $\lambda'(t)$, noise sampling function $p(t)$, noise range $[t_\text{min}, t_\text{max}]$, step function $\Delta t(t)$, noiser $N_\varphi$
\STATE \rule{0.96\textwidth}{0.45pt} 
\STATE Sample $\boldsymbol{x}_0, c \sim \mathcal{D}$ 
\STATE Sample noise $\boldsymbol{z} \sim \mathcal{N}(\boldsymbol{0},\mathbf{I})$, time step $t \sim p(t)$
\STATE Clamp $t \leftarrow \text{clamp}(t,t_\text{min}, t_\text{max})$
\STATE Compute noisy sample: $\boldsymbol{x}_t = \boldsymbol{x}_0 + t\boldsymbol{z}$
\STATE Compute cleaner sample: $\boldsymbol{x}_r = \boldsymbol{x}_t - \Delta t(t) \boldsymbol{z}$
\vspace{0.3em} 
\STATE Compute loss: 
\[
\mathcal{L}_\text{ict} = \lambda'(t) \, d(N_{\varphi}(\boldsymbol{x}_r, r, c), D_{{\varphi}^-}(\boldsymbol{x}_t, t, c))
\]
\STATE Return $\mathcal{L}_\text{ict}$ 
\end{algorithmic}
\end{algorithm}

Together, iGCT jointly optimizes the two consistency objectives \(\mathcal{L}_\text{gct}, \mathcal{L}_\text{ict}\), and aligns the noiser and denoiser via a reconstruction loss, \(\mathcal{L}_\text{recon}\). To improve training efficiency, \(\mathcal{L}_\text{recon}\) is computed every \(i_\text{skip}\), reducing the computational cost of back-propagation through both the weights of the \textit{denoiser} \(\theta\) and the \textit{noiser} \(\varphi\). Alg. \ref{alg:iGCT} provides an overview of iGCT. 

\begin{algorithm}
\caption{iGCT}
\label{alg:iGCT}
\begin{algorithmic}[1]  % Adds line numbers
\setlength{\baselineskip}{0.9\baselineskip} % Adjust line spacing
\INPUT Dataset $\mathcal{D}$, learning rate $\eta$, weighting functions $\lambda'(t), \lambda(t), \lambda_{\text{recon}}$, noise sampling function $p(t)$, noise range $[t_\text{min}, t_\text{max}]$, step function $\Delta t(t)$, guidance mask function $q(t)$, guidance range $[w_\text{min}, w_\text{max}]$, reconstruction skip iters $i_\text{skip}$, models $N_\varphi, D_\theta$
\STATE \rule{0.9\textwidth}{0.45pt}  % Horizontal line to separate input from main algorithm
\STATE \textbf{Init:} $\theta, \varphi$, $\text{Iters} = 0$
\REPEAT
\STATE Do guided consistency training 
\[
\mathcal{L}_\text{gct}(\theta;\mathcal{D},\lambda(t),p(t),t_\text{min},t_\text{max},\Delta t(t),q(t),w_\text{min},w_\text{max})
\]
\STATE Do inverse consistency training
\[
\mathcal{L}_\text{ict}(\varphi;\mathcal{D},\lambda'(t),p(t),t_\text{min},t_\text{max},\Delta t(t))
\]
\IF{$(\text{Iters} \ \% \ i_\text{skip}) == 0$}
\STATE Compute reconstruction loss
\[
\mathcal{L}_\text{recon} = d(D_{\theta}(N_{\varphi}(\boldsymbol{x}_0,t_\text{min},c),t_\text{max},c,0), \boldsymbol{x}_0)
\]
\ELSE
\STATE \[
\mathcal{L}_\text{recon} = 0
\]
\ENDIF
\STATE Compute total loss: 
\[
\mathcal{L} = \mathcal{L}_\text{gct} + \mathcal{L}_\text{ict} + \lambda_{\text{recon}}\mathcal{L}_\text{recon}
\]
\STATE $\theta \leftarrow \theta - \eta \nabla_{\theta} \mathcal{L}, \ \varphi \leftarrow \varphi - \eta \nabla_{\varphi} \mathcal{L}$
\STATE $\text{Iters} = \text{Iters} + 1$
\UNTIL{$\Delta t \rightarrow dt$}
\end{algorithmic}
\end{algorithm}



\vspace{-0.3cm}
\section{Preconditioning for Noiser}
\label{appendix:unit-variance}
\vspace{-0.1cm}

We define 
\begin{equation}
    N_{\varphi}(\boldsymbol{x}_t, t, c) = c_\text{skip}(t) \, \boldsymbol{x}_t + c_\text{out}(t) \, F_{\varphi}(c_\text{in}(t) \, \boldsymbol{x}_t, t, c),
\end{equation}
where \( c_\text{in}(t) = \frac{1}{\sqrt{t^2 + \sigma_\text{data}^2}} \), \( c_\text{skip}(t) = 1 \), and \( c_\text{out}(t) = t_\text{max} - t \). This setup naturally serves as a boundary condition. Specifically:

\begin{itemize}
    \item When \( t = 0 \),
    \begin{equation}
        c_\text{out}(0) = t_\text{max} \gg c_\text{skip}(0) = 1,
    \end{equation}
    emphasizing that the model's noise prediction dominates the residual information given a relatively clean sample.

    \item When \( t = t_\text{max} \),
    \begin{equation}
        N_{\varphi}(\boldsymbol{x}_{t_\text{max}}, t_\text{max}, c) = \boldsymbol{x}_{t_\text{max}},
    \end{equation}
    satisfying the condition that \( N_{\varphi} \) outputs \( \boldsymbol{x}_{t_\text{max}} \) at the maximum time step.
\end{itemize}



We show that these preconditions ensure unit variance for the model’s input and target. First, \(\text{Var}_{\boldsymbol{x}_0, z}[\boldsymbol{x}_t] = \sigma_\text{data}^2 + t^2\), so setting \( c_\text{in}(t) = \frac{1}{\sqrt{\sigma_\text{data}^2 + t^2}} \) normalizes the input variance to 1. Second, we require the training target to have unit variance. Given the noise target for \( N_{\varphi} \) is \(\boldsymbol{x}_{t_\text{max}} = \boldsymbol{x}_0 + t_\text{max} z\), by moving of terms, the effective target for \( F_{\varphi} \) can be written as,
\begin{equation}
    \frac{\boldsymbol{x}_{t_\text{max}} - c_\text{skip}(t)\boldsymbol{x}_{t}}{c_\text{out}(t)}
\end{equation}
When \(c_\text{skip}(t) = 1\), \(c_\text{out}(t) = t_\text{max} - t \), we verify that target is unit variance,
\begin{align}
    &\text{Var}_{\boldsymbol{x}_0, \boldsymbol{z}} \left[ \frac{\boldsymbol{x}_{t_\text{max}} - c_\text{skip}(t) \, \boldsymbol{x}_{t}}{c_\text{out}(t)} \right] \\ \notag
    = \ &\text{Var}_{\boldsymbol{x}_0, \boldsymbol{z}} \left[ \frac{\boldsymbol{x}_0 + t_\text{max} \, \boldsymbol{z} - (\boldsymbol{x}_0 + t \, \boldsymbol{z})}{t_\text{max} - t} \right] \notag \\
    = \ &\text{Var}_{\boldsymbol{x}_0, \boldsymbol{z}} \left[ \frac{(t_\text{max} - t) \, \boldsymbol{z}}{t_\text{max} - t} \right] \notag \\
    = \ &\text{Var}_{\boldsymbol{x}_0, \boldsymbol{z}}[\boldsymbol{z}] \notag \\
    = \ &1. \notag
\end{align}

\vspace{-0.3cm}
\section{Baselines \& Training Details}
\label{appendix:bs-config}
\vspace{-0.1cm}

\begin{figure}[t!]  
    \centering
    \begin{subfigure}[b]{0.33\textwidth}
    \includegraphics[width=\textwidth]{fig/appendix/guidance_embed.pdf} 
        \caption{Guidance embedding.}
    \end{subfigure}
    \hfill
    \begin{subfigure}[b]{0.33\textwidth}
    \includegraphics[width=\textwidth]{fig/appendix/adm_arch.pdf} 
        \caption{NCSN++ architecture.}
    \end{subfigure}
    \hfill
    \begin{subfigure}[b]{0.33\textwidth}
    \includegraphics[width=\textwidth]{fig/appendix/ncsnpp_arch.pdf} 
        \caption{ADM architecture.}
    \end{subfigure}
    \hfill
    \caption{Design of guidance embedding, and conditioning under different network architectures.}
    \vspace{-1em}
    \label{fig:guidance_conditioning}
\end{figure}

For our diffusion model baseline, we follow \textit{EDM}'s official repository (\href{https://github.com/NVlabs/edm}{https://github.com/NVlabs/edm}) instructions for training and set \textit{label\_dropout} to 0.1 to optimize a CFG (classifier-free guided) DM. We will use this DM as the teacher model for our consistency model baseline via consistency distillation. 

The consistency model baseline \textit{Guided CD} is trained with a discrete-time schedule. We set the discretization steps \( N = 18 \) and use a Heun ODE solver to predict update directions based on the CFG EDM, as in \cite{song2023consistency}. Following \cite{luo2023latent}, we modify the model's architecture and iGCT's denoiser to accept guidance strength \(w\) by adding an extra linear layer. See the detailed architecture design for guidance conditioning of consistency model in Fig. \ref{fig:guidance_conditioning}. A range of guidance scales \(w \in [1,15]\) is uniformly sampled at training. Following \cite{song2023improved}, we replace LPIPS by Pseudo-Huber loss, with \(c=0.03 \) determining the breadth of the smoothing section between L1 and L2. See Table \ref{tab:training_configs} for a summary of the training configurations for our baseline models.


\begin{table}[t!]
\centering
\renewcommand{\arraystretch}{1.3} % Adjust vertical spacing
\small % Reduce text size
\caption{Summary of training configurations for baseline models.}
\begin{tabular}{lccc}
\toprule
\multirow{2}{*}{} & \multicolumn{2}{c}{\textbf{CIFAR-10}} & \textbf{ImageNet64}  \\
                  & EDM & Guided-CD & EDM \\
\midrule
\multicolumn{4}{l}{\textbf{\small Arch. config.}} \\
\hline
model arch.        & NCSN++ & NCSN++ & ADM     \\
channels mult.     & 2,2,2  & 2,2,2  & 1,2,3,4 \\
UNet size          & 56.4M  & 56.4M  & 295.9M  \\
\midrule
\multicolumn{4}{l}{\textbf{\small Training config.}} \\
\hline
lr             & 1e-3  & 4e-4  & 2e-4 \\
batch          & 512   & 512   & 4096 \\
dropout        & 0.13  & 0     & 0.1 \\
label dropout  & 0.1   & (n.a.) & 0.1 \\
loss           & L2    & Huber & L2    \\
training iterations & 390k  & 800k  & 800K \\
\bottomrule
\end{tabular}
\label{tab:training_configs}
\end{table}


\begin{table}[t!]
\centering
\renewcommand{\arraystretch}{1.3} % Adjust vertical spacing
\small % Reduce text size
\caption{Summary of training configurations for iGCT.}
\begin{tabular}{lcc}
\toprule
\multirow{2}{*}{} & \textbf{CIFAR-10} & \textbf{ImageNet64}  \\
                  & iGCT & iGCT \\
\midrule
\multicolumn{3}{l}{\textbf{\small Arch. config.}} \\
\hline
model arch.        & NCSN++ & ADM \\
channels mult.     & 2,2,2  & 1,2,2,3 \\
UNet size          & 56.4M  & 182.4M \\ 
Total size         & 112.9M & 364.8M \\ 
\midrule
\multicolumn{3}{l}{\textbf{\small Training config.}} \\
\hline
lr              & 1e-4 & 1e-4 \\
batch           & 1024 & 1024 \\
dropout            & 0.2 & 0.3 \\
loss               & Huber   & Huber \\
\(c\)                  & 0.03    &  0.06 \\
\(d\)                  & 40k     &  40k \\
\( P_\textit{mean} \) & -1.1 &  -1.1 \\
\( P_\textit{std} \) &  2.0  &  2.0  \\
\( \lambda_{\text{recon}} \) & 2e-5 & \parbox[t]{3.5cm}{\centering 2e-5, (\(\leq\) 180k)\\ 4e-5, (\(\leq\) 200k)\\ 6e-5, (\(\leq\) 260k) } \\  
\( i_{\text{skip}} \)        & 10 &  10 \\  
training iterations & 360k &  260k \\
\bottomrule
\end{tabular}
\label{tab:igct_training_configs}
\end{table}  

\begin{figure*}[t] 
    \centering
    \includegraphics[width=1.0\textwidth]{fig/appendix/inversion_collapse.pdf} 
    \caption{Inversion collapse observed during training on ImageNet64. The left image shows the input data. The middle image depicts the inversion collapse that occurred at iteration 220k, where leakage of signals in the noise latent can be visualized. The right image shows the inversion results at iteration 220k after appropriately increasing $\lambda_{\text{recon}}$ to 6e-5. The inversion images are generated by scaling the model's outputs by $1/80$, i.e., $ 1/t_\text{max}$, then clipping the values to the range [-3, 3] before denormalizing them to the range [0, 255]. }
    \vspace{-1.5em}
    \label{fig:inversion_collpase}
\end{figure*}

iGCT is trained with a continuous-time scheduler inspired by ECT \cite{ect}. To rigorously assess its independence from diffusion-based models, iGCT is trained from scratch rather than fine-tuned from a pre-trained diffusion model. Consequently, the training curriculum begins with an initial diffusion training stage, followed by consistency training with the step size halved every \(d\) iterations. In practice, we adopt the same noise sampling distribution \(p(t)\), same step function \(\Delta t (t) \), and same distance metric \( d(\cdot, \cdot) \) for both guided consistency training and inverse consistency training. 

For CIFAR-10, iGCT adopts the same UNet architecture as the baseline models. However, the overall model size is doubled, as iGCT comprises two UNets: one for the denoiser and one for the noiser. The Pseudo-Huber loss is employed as the distance metric, with a constant parameter \( c = 0.03 \). Consistency training is organized into nine stages, each comprising 400k iterations with the step size halved from the last stage. We found that training remains stable when the reconstruction weight \( \lambda_{\text{recon}} \) is fixed at \( 2 \times 10^{-5} \) throughout the entire training process.
 
For ImageNet64, iGCT employs a reduced ADM architecture \cite{dhariwal2021diffusionmodelsbeatgans} with smaller channel sizes to address computational constraints. A higher dropout rate and Pseudo-Huber loss with \( c = 0.06 \) is used, following prior works \cite{ect,song2023improved}. During our experiments, we observed that training on ImageNet64 is sensitive to the reconstruction weight. Keeping \(\lambda_{\text{recon}}\) fixed throughout training leads to inversion collapse, with significant signal leaked to the latent noise (see Fig. \ref{fig:inversion_collpase}). We found that increasing \(\lambda_{\text{recon}}\) to \( 4 \times 10^{-5} \) at iteration 1800 and to \( 6 \times 10^{-5} \) at iteration 2000 effectively stabilizes training and prevents collapse. This suggests that the reconstruction loss serves as a regularizer for iGCT. Additionally, we observed diminishing returns when training exceeded 240k iterations, leading us to stop at 260k iterations for our experiments. These findings indicate that alternative training strategies, such as framing iGCT as a multi-task learning problem \cite{kendall2018multi,liu2019loss}, and conducting a more sophisticated analysis of loss weighting, may be necessary to enhance stability and improve convergence. See Table \ref{tab:igct_training_configs} for a summary of the training configurations for iGCT.



\begin{table}[t]
\caption{Comparison of GPU hours across the methods used in our experiments on CIFAR-10.}
\centering
\begin{tabular}{|l|c|}
\hline
\textbf{Methods} & \textbf{A100 (40G) GPU hours} \\ \hline
CFG-EDM \cite{karras2022elucidating} & 312 \\ \hline
Guided-CD \cite{song2023consistency} & 3968 \\ \hline
iGCT (ours) & 2032 \\ \hline
\end{tabular}
\label{table:compute_resources}
\end{table}



\begin{figure*}[t!]  
    \centering
    \begin{subfigure}[b]{0.33\textwidth}
    \includegraphics[width=\textwidth]{fig/cls_exp_w1.png} 
        \caption{Accuracy on various ratios of augmented data, guidance scale w=1.}
    \end{subfigure}
    \begin{subfigure}[b]{0.33\textwidth}
    \includegraphics[width=\textwidth]{fig/cls_exp_w3.png} 
        \caption{Accuracy on various ratios of augmented data, guidance scale w=3.}
    \end{subfigure}
    \begin{subfigure}[b]{0.33\textwidth}
    \includegraphics[width=\textwidth]{fig/cls_exp_w5.png} 
        \caption{Accuracy on various ratios of augmented data, guidance scale w=5.}
    \end{subfigure}
    \begin{subfigure}[b]{0.33\textwidth}
    \includegraphics[width=\textwidth]{fig/cls_exp_w7.png} 
        \caption{Accuracy on various ratios of augmented data, guidance scale w=7.}
    \end{subfigure}
    \begin{subfigure}[b]{0.33\textwidth}
    \includegraphics[width=\textwidth]{fig/cls_exp_w9.png} 
        \caption{Accuracy on various ratios of augmented data, guidance scale w=9.}
    \end{subfigure}
    \caption{Comparison of synthesized methods, CFG-EDM vs iGCT, used for data augmentation in image classification. iGCT consistently improves accuracy. Conversely, augmentation data synthesized from CFG-EDM offers only limited gains.}
    \vspace{-1.5em}
    \label{fig:cls_results}
\end{figure*}


\vspace{-0.1cm}
\section{Application: Data Augmentation Under Different Guidance}
\vspace{-0.2cm}

In this section, we show the effectiveness of data augmentation with diffusion-based models, CFG-EDM and iGCT, across varying guidance scales for image classification on CIFAR-10 \cite{article}. High quality data augmentation has been shown to enhance classification performance \cite{yang2023imagedataaugmentationdeep}. Under high guidance, augmentation data generated from iGCT consistently improves accuracy. Conversely, augmentation data synthesized from CFG-EDM offers only limited gains. We describe the ratios of real to synthesized data, the classifier architecture, and the training setup in the following. 

\noindent{\bf Training Details.} We conduct classification experiments trained on six different mixtures of augmented data synthesized by iGCT and CFG-EDM: \(0\%\), \(20\%\), \(40\%\), \(80\%\), and \(100\%\). The ratio represents \(\textit{synthesized data} / \textit{real data}\). For example, \(0\%\) indicates that the training and validation sets contain only 50k of real samples from CIFAR-10, and \(20\%\) includes 50k real \textit{and} 10k synthesized samples. In terms of guidance scales, we choose \(w=1,3,5,7,9\) to synthesize the augmented data using iGCT and CFG-EDM. 
The augmented dataset is split 80/20 for training and validation. For testing, the model is evaluated on the CIFAR-10 test set with 10k samples and ground truth labels. 

The standard ResNet-18 \cite{he2015deepresiduallearningimage} is used to train on all different augmented datasets. All models are trained for 250 epochs, with batch size 64, using an Adam optimizer \cite{kingma2017adammethodstochasticoptimization}. For each augmentation dataset, we train the model six times under different seeds and report the average classification accuracy.

\noindent{\bf Results.} The classifier's accuracy, trained on augmented data synthesized by CFG-EDM and iGCT, is shown in Fig. \ref{fig:cls_results}. With \(w=1\) (no guidance), both iGCT and CFG-EDM provide comparable performance boosts. As guidance scale increases, iGCT shows more significant improvements than CFG-EDM. At high guidance and augmentation ratios, performance drops, but this effect occurs later for iGCT (e.g., at \(100\%\) augmentation and \(w=9\)), while CFG-EDM stops improving accuracy at \(w=7\). This experiment highlights the importance of high-quality data under high guidance, with iGCT outperforming CFG-EDM in data quality.

\section{Uncurated Results}
In this section, we present additional qualitative results to highlight the performance of our proposed iGCT method compared to the multi-step EDM baseline. These visualizations include both inversion and guidance tasks across the CIFAR-10 and ImageNet64 datasets. The results demonstrate iGCT's ability to maintain competitive quality with significantly fewer steps and minimal artifacts, showcasing the effectiveness of our approach.

\subsection{Inversion Results}
We provide additional visualization of the latent noise on both CIFAR-10 and ImageNet64 datasets. Fig. \ref{fig:CIFAR-10_inversion_reconstruction} and Fig. \ref{fig:im64_inversion_reconstruction} compare our 1-step iGCT with the multi-step EDM on inversion and reconstruction.  

\subsection{Editing Results}
In this section, we dump more uncurated editing results on ImageNet64's subgroups mentioned in Sec. \ref{sec:image-editing}. Fig. \ref{fig:im64_edit_1}--\ref{fig:im64_edit_4} illustrate a comparison between our 1-step iGCT and the multi-step EDM approach.

\subsection{Guidance Results}
In Section \ref{sec:guidance}, we demonstrated that iGCT provides a guidance solution without introducing the high-contrast artifacts commonly observed in CFG-based methods. Here, we present additional uncurated results on CIFAR-10 and ImageNet64. For CIFAR-10, iGCT achieves competitive performance compared to the baseline diffusion model, which requires multiple steps for generation. See Figs. \ref{fig:CIFAR-10_guided_1}--\ref{fig:CIFAR-10_guided_10}. For ImageNet64, although the visual quality of iGCT's generated images falls slightly short of expectations, this can be attributed to the smaller UNet architecture used—only 61\% of the baseline model size—and the need for a more robust training curriculum to prevent collapse, as discussed in Section \ref{appendix:bs-config}. Nonetheless, even at higher guidance levels, iGCT maintains style consistency, whereas CFG-based methods continue to suffer from pronounced high-contrast artifacts. See Figs. \ref{fig:im64_guided_1}--\ref{fig:im64_guided_4}.


\begin{figure*}[t]
    \centering
    \begin{subfigure}{0.48\textwidth}
        \centering
        \includegraphics[width=\linewidth]{fig/appendix/recon_c10_data.png}
        \caption{CIFAR-10: Original data}
    \end{subfigure}
    \begin{subfigure}{0.48\textwidth}
        \centering
        \includegraphics[width=\linewidth]{fig/appendix/recon_im64_data.png}
        \caption{ImageNet64: Original data}
    \end{subfigure}

    \begin{subfigure}{0.48\textwidth}
        \centering
        \includegraphics[width=\linewidth]{fig/appendix/inv_c10_edm.png}
    \end{subfigure}
    \begin{subfigure}{0.48\textwidth}
        \centering
        \includegraphics[width=\linewidth]{fig/appendix/inv_im64_edm.png}
    \end{subfigure}

    \begin{subfigure}{0.48\textwidth}
        \centering
        \includegraphics[width=\linewidth]{fig/appendix/recon_c10_edm.png}
        \caption{CIFAR-10: Inversion + reconstruction, EDM (18 NFE)}
    \end{subfigure}
    \begin{subfigure}{0.48\textwidth}
        \centering
        \includegraphics[width=\linewidth]{fig/appendix/recon_im64_edm.png}
        \caption{ImageNet64: Inversion + reconstruction, EDM (18 NFE)}
    \end{subfigure}

    \begin{subfigure}{0.48\textwidth}
        \centering
        \includegraphics[width=\linewidth]{fig/appendix/inv_c10_igct.png}
    \end{subfigure}
    \begin{subfigure}{0.48\textwidth}
        \centering
        \includegraphics[width=\linewidth]{fig/appendix/inv_im64_igct.png}
    \end{subfigure}

    \begin{subfigure}{0.48\textwidth}
        \centering
        \includegraphics[width=\linewidth]{fig/appendix/recon_c10_igct.png}
        \caption{CIFAR-10: Inversion + reconstruction, iGCT (1 NFE)}
    \end{subfigure}
    \begin{subfigure}{0.48\textwidth}
        \centering
        \includegraphics[width=\linewidth]{fig/appendix/recon_im64_igct.png}
        \caption{ImageNet64: Inversion + reconstruction, iGCT (1 NFE)}
    \end{subfigure}

    \caption{Comparison of inversion and reconstruction for CIFAR-10 (left) and ImageNet64 (right).}
    \label{fig:comparison_CIFAR-10_imagenet64}
\end{figure*}




\begin{figure*}[t]
    \centering

    % Left column: corgi -> golden retriever
    \begin{minipage}{0.48\textwidth}
        \centering
        \begin{subfigure}{0.48\textwidth}
            \includegraphics[width=\linewidth]{fig/appendix_edit_igct/src_corgi.png}
            \caption{Original: "corgi"}
        \end{subfigure}

        \begin{subfigure}{0.48\textwidth}
            \includegraphics[width=\linewidth]{fig/appendix_edit_edm/w=0_src_corgi_tar_golden_retriever.png}
            \caption{EDM (18 NFE), w=1}
        \end{subfigure}
        \begin{subfigure}{0.48\textwidth}
            \includegraphics[width=\linewidth]{fig/appendix_edit_edm/w=6_src_corgi_tar_golden_retriever.png}
            \caption{EDM (18 NFE), w=7}
        \end{subfigure}
        \begin{subfigure}{0.48\textwidth}
            \includegraphics[width=\linewidth]{fig/appendix_edit_igct/w=6_src_corgi_tar_golden_retriever.png}
            \caption{iGCT (1 NFE), w=7}
        \end{subfigure}
        \begin{subfigure}{0.48\textwidth}
            \includegraphics[width=\linewidth]{fig/appendix_edit_igct/w=0_src_corgi_tar_golden_retriever.png}
            \caption{iGCT (1 NFE), w=1}
        \end{subfigure}

        \caption{ImageNet64: "corgi" $\rightarrow$ "golden retriever"}
        \label{fig:im64_edit_1}
    \end{minipage}
    \hfill
    % Right column: zebra -> horse
    \begin{minipage}{0.48\textwidth}
        \centering
        \begin{subfigure}{0.48\textwidth}
            \includegraphics[width=\linewidth]{fig/appendix_edit_igct/src_zebra.png}
            \caption{Original: "zebra"}
        \end{subfigure}

        \begin{subfigure}{0.48\textwidth}
            \includegraphics[width=\linewidth]{fig/appendix_edit_edm/w=0_src_zebra_tar_horse.png}
            \caption{EDM (18 NFE), w=1}
        \end{subfigure}
        \begin{subfigure}{0.48\textwidth}
            \includegraphics[width=\linewidth]{fig/appendix_edit_edm/w=6_src_zebra_tar_horse.png}
            \caption{EDM (18 NFE), w=7}
        \end{subfigure}
        \begin{subfigure}{0.48\textwidth}
            \includegraphics[width=\linewidth]{fig/appendix_edit_igct/w=0_src_zebra_tar_horse.png}
            \caption{iGCT (1 NFE), w=1}
        \end{subfigure}
        \begin{subfigure}{0.48\textwidth}
            \includegraphics[width=\linewidth]{fig/appendix_edit_igct/w=6_src_zebra_tar_horse.png}
            \caption{iGCT (1 NFE), w=7}
        \end{subfigure}

        \caption{ImageNet64: "zebra" $\rightarrow$ "horse"}
        \label{fig:im64_edit_2}
    \end{minipage}

\end{figure*}

\begin{figure*}[t]
    \centering

    % Left column: broccoli -> cauliflower
    \begin{minipage}{0.48\textwidth}
        \centering
        \begin{subfigure}{0.48\textwidth}
            \includegraphics[width=\linewidth]{fig/appendix_edit_igct/src_broccoli.png}
            \caption{Original: "broccoli"}
        \end{subfigure}

        \begin{subfigure}{0.48\textwidth}
            \includegraphics[width=\linewidth]{fig/appendix_edit_edm/w=0_src_broccoli_tar_cauliflower.png}
            \caption{EDM (18 NFE), w=1}
        \end{subfigure}
        \begin{subfigure}{0.48\textwidth}
            \includegraphics[width=\linewidth]{fig/appendix_edit_edm/w=6_src_broccoli_tar_cauliflower.png}
            \caption{EDM (18 NFE), w=7}
        \end{subfigure}
        \begin{subfigure}{0.48\textwidth}
            \includegraphics[width=\linewidth]{fig/appendix_edit_igct/w=0_src_broccoli_tar_cauliflower.png}
            \caption{iGCT (1 NFE), w=1}
        \end{subfigure}
        \begin{subfigure}{0.48\textwidth}
            \includegraphics[width=\linewidth]{fig/appendix_edit_igct/w=6_src_broccoli_tar_cauliflower.png}
            \caption{iGCT (1 NFE), w=7}
        \end{subfigure}

        \caption{ImageNet64: "broccoli" $\rightarrow$ "cauliflower"}
        \label{fig:im64_edit_3}
    \end{minipage}
    \hfill
    % Right column: jaguar -> tiger
    \begin{minipage}{0.48\textwidth}
        \centering
        \begin{subfigure}{0.48\textwidth}
            \includegraphics[width=\linewidth]{fig/appendix_edit_igct/src_jaguar.png}
            \caption{Original: "jaguar"}
        \end{subfigure}

        \begin{subfigure}{0.48\textwidth}
            \includegraphics[width=\linewidth]{fig/appendix_edit_edm/w=0_src_jaguar_tar_tiger.png}
            \caption{EDM (18 NFE), w=1}
        \end{subfigure}
        \begin{subfigure}{0.48\textwidth}
            \includegraphics[width=\linewidth]{fig/appendix_edit_edm/w=6_src_jaguar_tar_tiger.png}
            \caption{EDM (18 NFE), w=7}
        \end{subfigure}
        \begin{subfigure}{0.48\textwidth}
            \includegraphics[width=\linewidth]{fig/appendix_edit_igct/w=0_src_jaguar_tar_tiger.png}
            \caption{iGCT (1 NFE), w=1}
        \end{subfigure}
        \begin{subfigure}{0.48\textwidth}
            \includegraphics[width=\linewidth]{fig/appendix_edit_igct/w=6_src_jaguar_tar_tiger.png}
            \caption{iGCT (1 NFE), w=7}
        \end{subfigure}

        \caption{ImageNet64: "jaguar" $\rightarrow$ "tiger"}
        \label{fig:im64_edit_4}
    \end{minipage}

\end{figure*}






\begin{figure*}[b]
    \centering
    % First image
    \begin{subfigure}{0.25\textwidth}
        \includegraphics[width=\linewidth]{fig/appendix_edm/0_0.0_middle_4x4_grid.png}
        \caption{CFG-EDM (18 NFE), w=1.0}
    \end{subfigure}
    \begin{subfigure}{0.25\textwidth}
        \includegraphics[width=\linewidth]{fig/appendix_edm/0_6.0_middle_4x4_grid.png}
        \caption{CFG-EDM (18 NFE), w=7.0}
    \end{subfigure}
    \begin{subfigure}{0.25\textwidth}
        \includegraphics[width=\linewidth]{fig/appendix_edm/0_12.0_middle_4x4_grid.png}
        \caption{CFG-EDM (18 NFE), w=13.0}
    \end{subfigure}
    \begin{subfigure}{0.25\textwidth}
        \includegraphics[width=\linewidth]{fig/appendix_igct/0_0.0_middle_4x4_grid.png}
        \caption{iGCT (1 NFE), w=1.0}
    \end{subfigure}
    \begin{subfigure}{0.25\textwidth}
        \includegraphics[width=\linewidth]{fig/appendix_igct/0_6.0_middle_4x4_grid.png}
        \caption{iGCT (1 NFE), w=7.0}
    \end{subfigure}
    % Third image
    \begin{subfigure}{0.25\textwidth}
        \includegraphics[width=\linewidth]{fig/appendix_igct/0_12.0_middle_4x4_grid.png}
        \caption{iGCT (1 NFE), w=13.0}
    \end{subfigure}
    \caption{CIFAR-10 "airplane"}
    \label{fig:CIFAR-10_guided_1}
\end{figure*}
\begin{figure*}[t]
    \centering
    % First image
    \begin{subfigure}{0.25\textwidth}
        \includegraphics[width=\linewidth]{fig/appendix_edm/1_0.0_middle_4x4_grid.png}
        \caption{CFG-EDM (18 NFE), w=1.0}
    \end{subfigure}
    \begin{subfigure}{0.25\textwidth}
        \includegraphics[width=\linewidth]{fig/appendix_edm/1_6.0_middle_4x4_grid.png}
        \caption{CFG-EDM (18 NFE), w=7.0}
    \end{subfigure}
    \begin{subfigure}{0.25\textwidth}
        \includegraphics[width=\linewidth]{fig/appendix_edm/1_12.0_middle_4x4_grid.png}
        \caption{CFG-EDM (18 NFE), w=13.0}
    \end{subfigure}
    \begin{subfigure}{0.25\textwidth}
        \includegraphics[width=\linewidth]{fig/appendix_igct/1_0.0_middle_4x4_grid.png}
        \caption{iGCT (1 NFE), w=1.0}
    \end{subfigure}
    % Second image
    \begin{subfigure}{0.25\textwidth}
        \includegraphics[width=\linewidth]{fig/appendix_igct/1_6.0_middle_4x4_grid.png}
        \caption{iGCT (1 NFE), w=7.0}
    \end{subfigure}
    % Third image
    \begin{subfigure}{0.25\textwidth}
        \includegraphics[width=\linewidth]{fig/appendix_igct/1_12.0_middle_4x4_grid.png}
        \caption{iGCT (1 NFE), w=13.0}
    \end{subfigure}
    \caption{CIFAR-10 "car"}
    \label{fig:CIFAR-10_guided_2}
\end{figure*}
\begin{figure*}[t]
    \centering
    % First image
    \begin{subfigure}{0.25\textwidth}
        \includegraphics[width=\linewidth]{fig/appendix_edm/2_0.0_middle_4x4_grid.png}
        \caption{CFG-EDM (18 NFE), w=1.0}
    \end{subfigure}
    \begin{subfigure}{0.25\textwidth}
        \includegraphics[width=\linewidth]{fig/appendix_edm/2_6.0_middle_4x4_grid.png}
        \caption{CFG-EDM (18 NFE), w=7.0}
    \end{subfigure}
    \begin{subfigure}{0.25\textwidth}
        \includegraphics[width=\linewidth]{fig/appendix_edm/2_12.0_middle_4x4_grid.png}
        \caption{CFG-EDM (18 NFE), w=13.0}
    \end{subfigure}
    \begin{subfigure}{0.25\textwidth}
        \includegraphics[width=\linewidth]{fig/appendix_igct/2_0.0_middle_4x4_grid.png}
        \caption{iGCT (1 NFE), w=1.0}
    \end{subfigure}
    % Second image
    \begin{subfigure}{0.25\textwidth}
        \includegraphics[width=\linewidth]{fig/appendix_igct/2_6.0_middle_4x4_grid.png}
        \caption{iGCT (1 NFE), w=7.0}
    \end{subfigure}
    % Third image
    \begin{subfigure}{0.25\textwidth}
        \includegraphics[width=\linewidth]{fig/appendix_igct/2_12.0_middle_4x4_grid.png}
        \caption{iGCT (1 NFE), w=13.0}
    \end{subfigure}
    \caption{CIFAR-10 "bird"}
    \label{fig:CIFAR-10_guided_3}
\end{figure*}
\begin{figure*}[t]
    \centering
    % First image
    \begin{subfigure}{0.25\textwidth}
        \includegraphics[width=\linewidth]{fig/appendix_edm/3_0.0_middle_4x4_grid.png}
        \caption{CFG-EDM (18 NFE), w=1.0}
    \end{subfigure}
    \begin{subfigure}{0.25\textwidth}
        \includegraphics[width=\linewidth]{fig/appendix_edm/3_6.0_middle_4x4_grid.png}
        \caption{CFG-EDM (18 NFE), w=7.0}
    \end{subfigure}
    \begin{subfigure}{0.25\textwidth}
        \includegraphics[width=\linewidth]{fig/appendix_edm/3_12.0_middle_4x4_grid.png}
        \caption{CFG-EDM (18 NFE), w=13.0}
    \end{subfigure}
    \begin{subfigure}{0.25\textwidth}
        \includegraphics[width=\linewidth]{fig/appendix_igct/3_0.0_middle_4x4_grid.png}
        \caption{iGCT (1 NFE), w=1.0}
    \end{subfigure}
    % Second image
    \begin{subfigure}{0.25\textwidth}
        \includegraphics[width=\linewidth]{fig/appendix_igct/3_6.0_middle_4x4_grid.png}
        \caption{iGCT (1 NFE), w=7.0}
    \end{subfigure}
    % Third image
    \begin{subfigure}{0.25\textwidth}
        \includegraphics[width=\linewidth]{fig/appendix_igct/3_12.0_middle_4x4_grid.png}
        \caption{iGCT (1 NFE), w=13.0}
    \end{subfigure}
    \caption{CIFAR-10 "cat"}
    \label{fig:CIFAR-10_guided_4}
\end{figure*}
\begin{figure*}[t]
    \centering
    % First image
    \begin{subfigure}{0.25\textwidth}
        \includegraphics[width=\linewidth]{fig/appendix_edm/4_0.0_middle_4x4_grid.png}
        \caption{CFG-EDM (18 NFE), w=1.0}
    \end{subfigure}
    \begin{subfigure}{0.25\textwidth}
        \includegraphics[width=\linewidth]{fig/appendix_edm/4_6.0_middle_4x4_grid.png}
        \caption{CFG-EDM (18 NFE), w=7.0}
    \end{subfigure}
    \begin{subfigure}{0.25\textwidth}
        \includegraphics[width=\linewidth]{fig/appendix_edm/4_12.0_middle_4x4_grid.png}
        \caption{CFG-EDM (18 NFE), w=13.0}
    \end{subfigure}
    \begin{subfigure}{0.25\textwidth}
        \includegraphics[width=\linewidth]{fig/appendix_igct/4_0.0_middle_4x4_grid.png}
        \caption{iGCT (1 NFE), w=1.0}
    \end{subfigure}
    % Second image
    \begin{subfigure}{0.25\textwidth}
        \includegraphics[width=\linewidth]{fig/appendix_igct/4_6.0_middle_4x4_grid.png}
        \caption{iGCT (1 NFE), w=7.0}
    \end{subfigure}
    % Third image
    \begin{subfigure}{0.25\textwidth}
        \includegraphics[width=\linewidth]{fig/appendix_igct/4_12.0_middle_4x4_grid.png}
        \caption{iGCT (1 NFE), w=13.0}
    \end{subfigure}
    \caption{CIFAR-10 "deer"}
    \label{fig:CIFAR-10_guided_5}
\end{figure*}
\begin{figure*}[t]
    \centering
    % First image
    \begin{subfigure}{0.25\textwidth}
        \includegraphics[width=\linewidth]{fig/appendix_edm/5_0.0_middle_4x4_grid.png}
        \caption{CFG-EDM (18 NFE), w=1.0}
    \end{subfigure}
    \begin{subfigure}{0.25\textwidth}
        \includegraphics[width=\linewidth]{fig/appendix_edm/5_6.0_middle_4x4_grid.png}
        \caption{CFG-EDM (18 NFE), w=7.0}
    \end{subfigure}
    \begin{subfigure}{0.25\textwidth}
        \includegraphics[width=\linewidth]{fig/appendix_edm/5_12.0_middle_4x4_grid.png}
        \caption{CFG-EDM (18 NFE), w=13.0}
    \end{subfigure}
    \begin{subfigure}{0.25\textwidth}
        \includegraphics[width=\linewidth]{fig/appendix_igct/5_0.0_middle_4x4_grid.png}
        \caption{iGCT (1 NFE), w=1.0}
    \end{subfigure}
    % Second image
    \begin{subfigure}{0.25\textwidth}
        \includegraphics[width=\linewidth]{fig/appendix_igct/5_6.0_middle_4x4_grid.png}
        \caption{iGCT (1 NFE), w=7.0}
    \end{subfigure}
    % Third image
    \begin{subfigure}{0.25\textwidth}
        \includegraphics[width=\linewidth]{fig/appendix_igct/5_12.0_middle_4x4_grid.png}
        \caption{iGCT (1 NFE), w=13.0}
    \end{subfigure}
    \caption{CIFAR-10 "dog"}
    \label{fig:CIFAR-10_guided_6}
\end{figure*}
\begin{figure*}[t]
    \centering
    % First image
    \begin{subfigure}{0.25\textwidth}
        \includegraphics[width=\linewidth]{fig/appendix_edm/6_0.0_middle_4x4_grid.png}
        \caption{CFG-EDM (18 NFE), w=1.0}
    \end{subfigure}
    \begin{subfigure}{0.25\textwidth}
        \includegraphics[width=\linewidth]{fig/appendix_edm/6_6.0_middle_4x4_grid.png}
        \caption{CFG-EDM (18 NFE), w=7.0}
    \end{subfigure}
    \begin{subfigure}{0.25\textwidth}
        \includegraphics[width=\linewidth]{fig/appendix_edm/6_12.0_middle_4x4_grid.png}
        \caption{CFG-EDM (18 NFE), w=13.0}
    \end{subfigure}
    \begin{subfigure}{0.25\textwidth}
        \includegraphics[width=\linewidth]{fig/appendix_igct/6_0.0_middle_4x4_grid.png}
        \caption{iGCT (1 NFE), w=1.0}
    \end{subfigure}
    % Second image
    \begin{subfigure}{0.25\textwidth}
        \includegraphics[width=\linewidth]{fig/appendix_igct/6_6.0_middle_4x4_grid.png}
        \caption{iGCT (1 NFE), w=7.0}
    \end{subfigure}
    % Third image
    \begin{subfigure}{0.25\textwidth}
        \includegraphics[width=\linewidth]{fig/appendix_igct/6_12.0_middle_4x4_grid.png}
        \caption{iGCT (1 NFE), w=13.0}
    \end{subfigure}
    \caption{CIFAR-10 "frog"}
    \label{fig:CIFAR-10_guided_7}
\end{figure*}
\begin{figure*}[t]
    \centering
    % First image
    \begin{subfigure}{0.25\textwidth}
        \includegraphics[width=\linewidth]{fig/appendix_edm/7_0.0_middle_4x4_grid.png}
        \caption{CFG-EDM (18 NFE), w=1.0}
    \end{subfigure}
    \begin{subfigure}{0.25\textwidth}
        \includegraphics[width=\linewidth]{fig/appendix_edm/7_6.0_middle_4x4_grid.png}
        \caption{CFG-EDM (18 NFE), w=7.0}
    \end{subfigure}
    \begin{subfigure}{0.25\textwidth}
        \includegraphics[width=\linewidth]{fig/appendix_edm/7_12.0_middle_4x4_grid.png}
        \caption{CFG-EDM (18 NFE), w=13.0}
    \end{subfigure}
    \begin{subfigure}{0.25\textwidth}
        \includegraphics[width=\linewidth]{fig/appendix_igct/7_0.0_middle_4x4_grid.png}
        \caption{iGCT (1 NFE), w=1.0}
    \end{subfigure}
    % Second image
    \begin{subfigure}{0.25\textwidth}
        \includegraphics[width=\linewidth]{fig/appendix_igct/7_6.0_middle_4x4_grid.png}
        \caption{iGCT (1 NFE), w=7.0}
    \end{subfigure}
    % Third image
    \begin{subfigure}{0.25\textwidth}
        \includegraphics[width=\linewidth]{fig/appendix_igct/7_12.0_middle_4x4_grid.png}
        \caption{iGCT (1 NFE), w=13.0}
    \end{subfigure}
    \caption{CIFAR-10 "horse"}
    \label{fig:CIFAR-10_guided_8}
\end{figure*}
\begin{figure*}[t]
    \centering
    % First image
    \begin{subfigure}{0.25\textwidth}
        \includegraphics[width=\linewidth]{fig/appendix_edm/8_0.0_middle_4x4_grid.png}
        \caption{CFG-EDM (18 NFE), w=1.0}
    \end{subfigure}
    \begin{subfigure}{0.25\textwidth}
        \includegraphics[width=\linewidth]{fig/appendix_edm/8_6.0_middle_4x4_grid.png}
        \caption{CFG-EDM (18 NFE), w=7.0}
    \end{subfigure}
    \begin{subfigure}{0.25\textwidth}
        \includegraphics[width=\linewidth]{fig/appendix_edm/8_12.0_middle_4x4_grid.png}
        \caption{CFG-EDM (18 NFE), w=13.0}
    \end{subfigure}
    \begin{subfigure}{0.25\textwidth}
        \includegraphics[width=\linewidth]{fig/appendix_igct/8_0.0_middle_4x4_grid.png}
        \caption{iGCT (1 NFE), w=1.0}
    \end{subfigure}
    % Second image
    \begin{subfigure}{0.25\textwidth}
        \includegraphics[width=\linewidth]{fig/appendix_igct/8_6.0_middle_4x4_grid.png}
        \caption{iGCT (1 NFE), w=7.0}
    \end{subfigure}
    % Third image
    \begin{subfigure}{0.25\textwidth}
        \includegraphics[width=\linewidth]{fig/appendix_igct/8_12.0_middle_4x4_grid.png}
        \caption{iGCT (1 NFE), w=13.0}
    \end{subfigure}
    \caption{CIFAR-10 "ship"}
    \label{fig:CIFAR-10_guided_9}
\end{figure*}
\begin{figure*}[t]
    \centering
    % First image
    \begin{subfigure}{0.25\textwidth}
        \includegraphics[width=\linewidth]{fig/appendix_edm/9_0.0_middle_4x4_grid.png}
        \caption{CFG-EDM (18 NFE), w=1.0}
    \end{subfigure}
    \begin{subfigure}{0.25\textwidth}
        \includegraphics[width=\linewidth]{fig/appendix_edm/9_6.0_middle_4x4_grid.png}
        \caption{CFG-EDM (18 NFE), w=7.0}
    \end{subfigure}
    \begin{subfigure}{0.25\textwidth}
        \includegraphics[width=\linewidth]{fig/appendix_edm/9_12.0_middle_4x4_grid.png}
        \caption{CFG-EDM (18 NFE), w=13.0}
    \end{subfigure}
    \begin{subfigure}{0.25\textwidth}
        \includegraphics[width=\linewidth]{fig/appendix_igct/9_0.0_middle_4x4_grid.png}
        \caption{iGCT (1 NFE), w=1.0}
    \end{subfigure}
    % Second image
    \begin{subfigure}{0.25\textwidth}
        \includegraphics[width=\linewidth]{fig/appendix_igct/9_6.0_middle_4x4_grid.png}
        \caption{iGCT (1 NFE), w=7.0}
    \end{subfigure}
    % Third image
    \begin{subfigure}{0.25\textwidth}
        \includegraphics[width=\linewidth]{fig/appendix_igct/9_12.0_middle_4x4_grid.png}
        \caption{iGCT (1 NFE), w=13.0}
    \end{subfigure}
    \caption{CIFAR-10 "truck"}
    \label{fig:CIFAR-10_guided_10}
\end{figure*}


\begin{figure*}[b]
    \centering
    % First image
    \begin{subfigure}{0.25\textwidth}
        \includegraphics[width=\linewidth]{fig/appendix_im64_edm/edm_class_291_w=0.0.png}
        \caption{CFG-EDM (18 NFE), w=1.0}
    \end{subfigure}
    \begin{subfigure}{0.25\textwidth}
        \includegraphics[width=\linewidth]{fig/appendix_im64_edm/edm_class_291_w=6.0.png}
        \caption{CFG-EDM (18 NFE), w=7.0}
    \end{subfigure}
    \begin{subfigure}{0.25\textwidth}
        \includegraphics[width=\linewidth]{fig/appendix_im64_edm/edm_class_291_w=12.0.png}
        \caption{CFG-EDM (18 NFE), w=13.0}
    \end{subfigure}
    \begin{subfigure}{0.25\textwidth}
        \includegraphics[width=\linewidth]{fig/appendix_im64_igct/class_291_w=0.0.png}
        \caption{iGCT (2 NFE), w=1.0}
    \end{subfigure}
    \begin{subfigure}{0.25\textwidth}
        \includegraphics[width=\linewidth]{fig/appendix_im64_igct/class_291_w=6.0.png}
        \caption{iGCT (2 NFE), w=7.0}
    \end{subfigure}
    % Third image
    \begin{subfigure}{0.25\textwidth}
        \includegraphics[width=\linewidth]{fig/appendix_im64_igct/class_291_w=12.0.png}
        \caption{iGCT (2 NFE), w=13.0}
    \end{subfigure}
    \caption{ImageNet64 "lion"}
    \label{fig:im64_guided_1}
\end{figure*}



\begin{figure*}[b]
    \centering
    % First image
    \begin{subfigure}{0.25\textwidth}
        \includegraphics[width=\linewidth]{fig/appendix_im64_edm/edm_class_292_w=0.0.png}
        \caption{CFG-EDM (18 NFE), w=1.0}
    \end{subfigure}
    \begin{subfigure}{0.25\textwidth}
        \includegraphics[width=\linewidth]{fig/appendix_im64_edm/edm_class_292_w=6.0.png}
        \caption{CFG-EDM (18 NFE), w=7.0}
    \end{subfigure}
    \begin{subfigure}{0.25\textwidth}
        \includegraphics[width=\linewidth]{fig/appendix_im64_edm/edm_class_292_w=12.0.png}
        \caption{CFG-EDM (18 NFE), w=13.0}
    \end{subfigure}
    \begin{subfigure}{0.25\textwidth}
        \includegraphics[width=\linewidth]{fig/appendix_im64_igct/class_292_w=0.0.png}
        \caption{iGCT (2 NFE), w=1.0}
    \end{subfigure}
    \begin{subfigure}{0.25\textwidth}
        \includegraphics[width=\linewidth]{fig/appendix_im64_igct/class_292_w=6.0.png}
        \caption{iGCT (2 NFE), w=7.0}
    \end{subfigure}
    % Third image
    \begin{subfigure}{0.25\textwidth}
        \includegraphics[width=\linewidth]{fig/appendix_im64_igct/class_292_w=12.0.png}
        \caption{iGCT (2 NFE), w=13.0}
    \end{subfigure}
    \caption{ImageNet64 "tiger"}
    \label{fig:im64_guided_2}
\end{figure*}


\begin{figure*}[b]
    \centering
    % First image
    \begin{subfigure}{0.25\textwidth}
        \includegraphics[width=\linewidth]{fig/appendix_im64_edm/edm_class_28_w=0.0.png}
        \caption{CFG-EDM (18 NFE), w=1.0}
    \end{subfigure}
    \begin{subfigure}{0.25\textwidth}
        \includegraphics[width=\linewidth]{fig/appendix_im64_edm/edm_class_28_w=6.0.png}
        \caption{CFG-EDM (18 NFE), w=7.0}
    \end{subfigure}
    \begin{subfigure}{0.25\textwidth}
        \includegraphics[width=\linewidth]{fig/appendix_im64_edm/edm_class_28_w=12.0.png}
        \caption{CFG-EDM (18 NFE), w=13.0}
    \end{subfigure}
    \begin{subfigure}{0.25\textwidth}
        \includegraphics[width=\linewidth]{fig/appendix_im64_igct/class_28_w=0.0.png}
        \caption{iGCT (2 NFE), w=1.0}
    \end{subfigure}
    \begin{subfigure}{0.25\textwidth}
        \includegraphics[width=\linewidth]{fig/appendix_im64_igct/class_28_w=6.0.png}
        \caption{iGCT (2 NFE), w=7.0}
    \end{subfigure}
    % Third image
    \begin{subfigure}{0.25\textwidth}
        \includegraphics[width=\linewidth]{fig/appendix_im64_igct/class_28_w=12.0.png}
        \caption{iGCT (2 NFE), w=13.0}
    \end{subfigure}
    \caption{ImageNet64 "salamander"}
    \label{fig:im64_guided_3}
\end{figure*}


\begin{figure*}[b]
    \centering
    % First image
    \begin{subfigure}{0.25\textwidth}
        \includegraphics[width=\linewidth]{fig/appendix_im64_edm/edm_class_407_w=0.0.png}
        \caption{CFG-EDM (18 NFE), w=1.0}
    \end{subfigure}
    \begin{subfigure}{0.25\textwidth}
        \includegraphics[width=\linewidth]{fig/appendix_im64_edm/edm_class_407_w=6.0.png}
        \caption{CFG-EDM (18 NFE), w=7.0}
    \end{subfigure}
    \begin{subfigure}{0.25\textwidth}
        \includegraphics[width=\linewidth]{fig/appendix_im64_edm/edm_class_407_w=12.0.png}
        \caption{CFG-EDM (18 NFE), w=13.0}
    \end{subfigure}
    \begin{subfigure}{0.25\textwidth}
        \includegraphics[width=\linewidth]{fig/appendix_im64_igct/class_407_w=0.0.png}
        \caption{iGCT (2 NFE), w=1.0}
    \end{subfigure}
    \begin{subfigure}{0.25\textwidth}
        \includegraphics[width=\linewidth]{fig/appendix_im64_igct/class_407_w=6.0.png}
        \caption{iGCT (2 NFE), w=7.0}
    \end{subfigure}
    % Third image
    \begin{subfigure}{0.25\textwidth}
        \includegraphics[width=\linewidth]{fig/appendix_im64_igct/class_407_w=12.0.png}
        \caption{iGCT (2 NFE), w=13.0}
    \end{subfigure}
    \caption{ImageNet64 "ambulance"}
    \label{fig:im64_guided_4}
\end{figure*}


\end{document}


% This document was modified from the file originally made available by
% Pat Langley and Andrea Danyluk for ICML-2K. This version was created
% by Iain Murray in 2018, and modified by Alexandre Bouchard in
% 2019 and 2021 and by Csaba Szepesvari, Gang Niu and Sivan Sabato in 2022.
% Modified again in 2023 and 2024 by Sivan Sabato and Jonathan Scarlett.
% Previous contributors include Dan Roy, Lise Getoor and Tobias
% Scheffer, which was slightly modified from the 2010 version by
% Thorsten Joachims & Johannes Fuernkranz, slightly modified from the
% 2009 version by Kiri Wagstaff and Sam Roweis's 2008 version, which is
% slightly modified from Prasad Tadepalli's 2007 version which is a
% lightly changed version of the previous year's version by Andrew
% Moore, which was in turn edited from those of Kristian Kersting and
% Codrina Lauth. Alex Smola contributed to the algorithmic style files.
