\section{Can LLMs Learn to Search by Themselves?}
\label{sec:llm_self_search}
Having integrated learning to enhance search, a reverse but natural question emerges: 
% \textit{Can search strategies improve LLMs' reasoning capabilities?}
\textit{Can LLMs execute search autonomously to improve the reasoning capabilities?}
Recent advances in scaling test-time computation suggest that systematic exploration could enhance LLMs' problem-solving abilities. While models like QwQ-32B~\cite{qwq-32b-preview} show promising reasoning capabilities, they struggle with focused problem-solving, often producing unfocused, recursive outputs. This motivates us to explore how search benefits LLMs and whether LLMs possess the potential for self search.

To systematically investigate LLMs' self-search capabilities, we consider two types of prompts: high-level self-search, which relies solely on LLMs' internal knowledge, and low-level self-search, which explicitly encodes \method's search strategies into the prompts. Further details and illustrative examples are in Appendix~\ref{appendix:more_details_self_search}, with additional discussions and analyses provided in Sec.~\ref{sec:experiment_self_search}.

