% This must be in the first 5 lines to tell arXiv to use pdfLaTeX, which is strongly recommended.
\pdfoutput=1
% In particular, the hyperref package requires pdfLaTeX in order to break URLs across lines.

\documentclass[11pt]{article}

% Change "review" to "final" to generate the final (sometimes called camera-ready) version.
% Change to "preprint" to generate a non-anonymous version with page numbers.
\usepackage[preprint]{acl}

\usepackage{graphicx}
% \usepackage{subfigure}
\usepackage{booktabs} % for professional tables

% hyperref makes hyperlinks in the resulting PDF.
% If your build breaks (sometimes temporarily if a hyperlink spans a page)
% please comment out the following usepackage line and replace
% \usepackage{icml2025} with \usepackage[nohyperref]{icml2025} above.
\usepackage{hyperref}


% Attempt to make hyperref and algorithmic work together better:
\newcommand{\theHalgorithm}{\arabic{algorithm}}

% Standard package includes
\usepackage{times}
\usepackage{latexsym}

% For proper rendering and hyphenation of words containing Latin characters (including in bib files)
\usepackage[T1]{fontenc}
% For Vietnamese characters
% \usepackage[T5]{fontenc}
% See https://www.latex-project.org/help/documentation/encguide.pdf for other character sets

% This assumes your files are encoded as UTF8
\usepackage[utf8]{inputenc}

% This is not strictly necessary, and may be commented out,
% but it will improve the layout of the manuscript,
% and will typically save some space.
\usepackage{microtype}

% This is also not strictly necessary, and may be commented out.
% However, it will improve the aesthetics of text in
% the typewriter font.
\usepackage{inconsolata}

%Including images in your LaTeX document requires adding
%additional package(s)
\usepackage{graphicx}
\usepackage{amsthm}
\usepackage{amsmath}
\usepackage[capitalize,noabbrev]{cleveref}
\theoremstyle{plain}
\newtheorem{theorem}{Theorem}[section]
\newtheorem{proposition}[theorem]{Proposition}
\newtheorem{lemma}[theorem]{Lemma}
\newtheorem{corollary}[theorem]{Corollary}
\theoremstyle{definition}
\newtheorem{definition}[theorem]{Definition}
\newtheorem{assumption}[theorem]{Assumption}
\theoremstyle{remark}
\newtheorem{remark}[theorem]{Remark}

% Todonotes is useful during development; simply uncomment the next line
%    and comment out the line below the next line to turn off comments
%\usepackage[disable,textsize=tiny]{todonotes}
\usepackage[textsize=tiny]{todonotes}

% \newcommand{\method}{SeaL}

\newcommand{\method}{\ensuremath{\textnormal{\textsc{SeaL}}}\xspace}
\newcommand{\cmethod}{\ensuremath{\textnormal{\textsc{SeaL-C}}}\xspace}

%%%% 
\usepackage{enumitem}
\usepackage{enumerate}
\usepackage{tabularx}
\usepackage{multirow}
\usepackage{url}
\usepackage{marvosym}
\usepackage{multirow}
\usepackage{algorithm}
\usepackage{algorithmic}
\renewcommand{\algorithmiccomment}[1]{// #1}
\usepackage{subcaption}
\renewcommand{\algorithmicrequire}{\textbf{Input:}}
\renewcommand{\algorithmicensure}{\textbf{Output:}}

\usepackage[utf8]{inputenc} % allow utf-8 input
\usepackage[T1]{fontenc}    % use 8-bit T1 fonts
\usepackage{url}            % simple URL typesetting
\usepackage{amsfonts}       % blackboard math symbols
\usepackage{nicefrac}       % compact symbols for 1/2, etc.
\usepackage{microtype}      % microtypography
\usepackage{colortbl}
\usepackage{tabularx}
\usepackage{makecell}
\usepackage{longtable}
\usepackage{float}
\usepackage{arydshln}
\usepackage{xcolor}
\usepackage{dutchcal}
\definecolor{mylightgreen}{rgb}{0.8, 1.0, 0.8}
\definecolor{mylightblue}{rgb}{0.68, 0.85, 0.9}

\definecolor{mycolor}{HTML}{00FF00}


% For theorems and such
\usepackage{amssymb}
\usepackage{mathtools}
\usepackage{siunitx}
% if you use cleveref..
\usepackage[capitalize,noabbrev]{cleveref}
\usepackage{enumitem}
\sisetup
{
output-exponent-marker=\ensuremath{\mathrm{e}},
group-separator = {,},
group-minimum-digits = 4,
list-final-separator={, }
}
%%%%%%%%%%%%%%%%%%%%%%%%%%%%%%%%
% THEOREMS
%%%%%%%%%%%%%%%%%%%%%%%%%%%%%%%%
\theoremstyle{plain}
\theoremstyle{definition}
\usepackage{longtable}
\usepackage{caption}
\usepackage[title,titletoc]{appendix}

% \usepackage{enumitems}
% \usepackage{cleveref}

% \newcommand\DoToC{%
%   \startcontents
%   \printcontents{}{1}{\noindent \textbf{\Large{Table of Contents in Appendix}}\vskip3pt\vskip5pt}
%   \vskip3pt\vskip5pt
% }

% Todonotes is useful during development; simply uncomment the next line
%    and comment out the line below the next line to turn off comments
%\usepackage[disable,textsize=tiny]{todonotes}
% \usepackage[textsize=tiny]{todonotes}

\usepackage{bm} % for bold math
\usepackage{xspace} % for smart spacing
\usepackage{multirow} % for multirow command in tables
\usepackage{fontawesome}


\newcommand{\suhang}[1]{\textcolor{blue}{SW: #1}}
\newcommand{\minhua}[1]{\textcolor{purple}{#1}}
\newcommand{\hui}[1]{\textcolor{orange}{Hui: #1}}

% If the title and author information does not fit in the area allocated, uncomment the following
%
%\setlength\titlebox{<dim>}
%
% and set <dim> to something 5cm or larger.

\title{How Far are LLMs from Real Search? A Comprehensive Study on Efficiency, Completeness, and Inherent Capabilities}

% Author information can be set in various styles:
% For several authors from the same institution:
% \author{Author 1 \and ... \and Author n \\
%         Address line \\ ... \\ Address line}
% if the names do not fit well on one line use
%         Author 1 \\ {\bf Author 2} \\ ... \\ {\bf Author n} \\
% For authors from different institutions:
% \author{Author 1 \\ Address line \\  ... \\ Address line
%         \And  ... \And
%         Author n \\ Address line \\ ... \\ Address line}
% To start a separate ``row'' of authors use \AND, as in
% \author{Author 1 \\ Address line \\  ... \\ Address line
%         \AND
%         Author 2 \\ Address line \\ ... \\ Address line \And
%         Author 3 \\ Address line \\ ... \\ Address line}

\author{Minhua Lin$^{1}$\thanks{Work done during an internship at Amazon.}, Hui Liu$^{2}$, Xianfeng Tang$^2$, Jingying Zeng$^2$, \textbf{Zhenwei Dai}$^{2}$,\\ \textbf{Chen Luo}$^2$, \textbf{Zheng Li}$^2$, \textbf{Xiang Zhang}$^1$, \textbf{Qi He}$^{2}$, \textbf{Suhang Wang}$^{1}$   \\  
 $^{1}$The Pennsylvania State University $^{2}$Amazon\\
 \texttt{\{mfl5681,xzz89,szw494\}@psu.edu}\\ 
\texttt{\{liunhu,xianft,zejingyi,zwdai,cheluo,amzzhe\}@amazon.com} \\
}
 

%\author{
%  \textbf{First Author\textsuperscript{1}},
%  \textbf{Second Author\textsuperscript{1,2}},
%  \textbf{Third T. Author\textsuperscript{1}},
%  \textbf{Fourth Author\textsuperscript{1}},
%\\
%  \textbf{Fifth Author\textsuperscript{1,2}},
%  \textbf{Sixth Author\textsuperscript{1}},
%  \textbf{Seventh Author\textsuperscript{1}},
%  \textbf{Eighth Author \textsuperscript{1,2,3,4}},
%\\
%  \textbf{Ninth Author\textsuperscript{1}},
%  \textbf{Tenth Author\textsuperscript{1}},
%  \textbf{Eleventh E. Author\textsuperscript{1,2,3,4,5}},
%  \textbf{Twelfth Author\textsuperscript{1}},
%\\
%  \textbf{Thirteenth Author\textsuperscript{3}},
%  \textbf{Fourteenth F. Author\textsuperscript{2,4}},
%  \textbf{Fifteenth Author\textsuperscript{1}},
%  \textbf{Sixteenth Author\textsuperscript{1}},
%\\
%  \textbf{Seventeenth S. Author\textsuperscript{4,5}},
%  \textbf{Eighteenth Author\textsuperscript{3,4}},
%  \textbf{Nineteenth N. Author\textsuperscript{2,5}},
%  \textbf{Twentieth Author\textsuperscript{1}}
%\\
%\\
%  \textsuperscript{1}Affiliation 1,
%  \textsuperscript{2}Affiliation 2,
%  \textsuperscript{3}Affiliation 3,
%  \textsuperscript{4}Affiliation 4,
%  \textsuperscript{5}Affiliation 5
%\\
%  \small{
%    \textbf{Correspondence:} \href{mailto:email@domain}{email@domain}
%  }
%}

\begin{document}
\maketitle


\begin{abstract}
% Search plays a fundamental role in problem-solving across various domains, with most real-world decision-making problems being solvable through systematic search. While traditional approaches suffer from efficiency issues, recent methods that rely solely on Large Language Models' (LLMs) intrinsic knowledge for search simulation prove unstable. Inspired by recent discussion in search and learning, we investigate the integration of LLMs into real search processes. First, we explore how LLMs can enhance search efficiency and propose a framework, \textbf{\underline{Se}arch vi\underline{a} \underline{L}earning (\method)}, for accurate and efficient search. Second, we study how to leverage LLMs to conduct complete and sound searches with 100\% accuracy efficiently. Finally, we explore whether LLMs can inherently develop search capabilities themselves. Our experimental results across three real-world planning tasks demonstrate that our framework can solve problems via complete but efficient search with 100\% accuracy using minimal model calls. Furthermore, we show that with proper guidance, LLMs can inherently conduct systematic searches, significantly improving their task-solving performance. These findings suggest promising directions for better empowering LLMs with search capabilities.

Search plays a fundamental role in problem-solving across various domains, with most real-world decision-making problems being solvable through systematic search. 
Drawing inspiration from recent discussions on search and learning, we systematically explore the complementary relationship between search and Large Language Models (LLMs) from three perspectives. 
First, we analyze how learning can enhance search efficiency and propose \textbf{\underline{Se}arch vi\underline{a} \underline{L}earning (\method)}, a framework that leverages LLMs for effective and efficient search. Second, we further extend \method to \cmethod to ensure rigorous completeness during search.
Our evaluation across three real-world planning tasks demonstrates that \method achieves near-perfect accuracy while reducing search spaces by up to 99.1\% compared to traditional approaches. 
Finally, we explore how far LLMs are from real search by investigating whether they can develop search capabilities independently. Our analysis reveals that while current LLMs struggle with efficient search in complex problems, incorporating systematic search strategies significantly enhances their problem-solving capabilities. These findings not only validate the effectiveness of our approach but also highlight the need for improving LLMs' search abilities for real-world applications.




% Search plays a fundamental role in problem-solving across various domains, with most real-world decision-making problems being solvable through systematic search. Drawing inspiration from recent discussions on search and learning, we systematically explore the complementary relationship between search and Large Language Models (LLMs) from three perspectives. First, we analyze how learning can enhance search efficiency and propose Search via Learning (SeaL), a framework that leverages LLMs for effective and efficient search. Second, we further extend SeaL to SeaL-C to ensure rigorous completeness during search. Our evaluation across three real-world planning tasks demonstrates that SeaL achieves near-perfect accuracy while reducing search spaces by up to 99.1\% compared to traditional approaches. Finally, we explore how far LLMs are from real search by investigating whether they can develop search capabilities independently. Our analysis reveals that while current LLMs struggle with efficient search in complex problems, incorporating systematic search strategies significantly enhances their problem-solving capabilities. These findings not only validate the effectiveness of our approach but also highlight the importance of improving LLMs' inherent search abilities for real-world applications.
\end{abstract}

\section{Introduction}

Rotary Position Embedding (RoPE) \cite{su2024roformer} helps Transformer models understand word order by assigning each token a unique positional `marker' calculated using a mathematical rotation matrix.
RoPE has advantages in long-context understanding \cite{ding2024longrope}, and continues to be a default choice in leading Large Language Models (LLMs) like the LLaMA \cite{touvron2023llamaopenefficientfoundation,touvron2023llama,dubey2024llama} and QWen \cite{yang2024qwen2,yang2024qwen25} series.

The original RoPE implementation (Vanilla RoPE) \cite{su2024roformer} is designed for sequential 1D data like text. However, recent Video Large Language Models (Video LLMs) \cite{2023videochat,lin2023video,chen2024sharegpt4video,maaz2024videochatgptdetailedvideounderstanding,zhang2024longva,wang2024longllavascalingmultimodalllms,chen2024longvilascalinglongcontextvisual,internlmxcomposer2_5_OL} process video, which has a more complex spatio and temporal structure.
As shown in Tab. \ref{tab:pe_compare}, although several RoPE-based approaches \cite{gao2024tc,wang2024qwen2} have been proposed to support video inputs, these variants exhibit limitations and do not fully satisfy the following key characteristics:

\begin{table}[t]
\tiny
% \small
\centering
\tabcolsep=0.1cm
\begin{tabular}{lcccc}
\toprule
 & \makecell[c]{\textbf{2D/3D} \\ \textbf{Structure}} & \makecell[c]{\textbf{Frequency} \\ \textbf{Allocation}} & \makecell[c]{\textbf{Spatial} \\ \textbf{Symmetry}} & \makecell[c]{\textbf{Temporal} \\ \textbf{Index Scaling}} \\
\midrule
\makecell[l]{Vanilla RoPE \cite{su2024roformer}} & \xmark & \xmark & \xmark & \xmark \\
\makecell[l]{TAD-RoPE \cite{gao2024tc}} & \xmark & \xmark & \xmark & \cmark \\
\makecell[l]{RoPE-Tie \cite{kexuefm10040}} & \cmark & \xmark & \cmark & \xmark \\
\makecell[l]{M-RoPE \cite{wang2024qwen2}} & \cmark & \xmark & \xmark & \xmark \\
% \makecell[l]{V2PE\\\cite{ge2024v2pe}} & \xmark & \xmark & \xmark & \cmark \\
\midrule
\rowcolor[HTML]{F2F3F5}
\methodname (Ours) & \cmark & \cmark & \cmark & \cmark \\
\bottomrule
\end{tabular}
\vspace{-6pt}
\caption{Comparison between different RoPE variants for Video Large Language Models (Video LLMs).}
\label{tab:pe_compare}
\vspace{-12pt}
\end{table}

\begin{figure}
\centering
\includegraphics[width=0.9\linewidth]{figures/files/radar.pdf}
\vspace{-6pt}
\caption{\methodname outperforms RoPE variants on benchmarks.}
\label{fig:radar}
\vspace{-12pt}
\end{figure}

\begin{figure*}[ht]
    \centering
    \includegraphics[width=0.96\linewidth]{figures/files/niah.pdf}
    \vspace{-6pt}
    \caption{\footnotesize \textbf{Left:} To demonstrate the importance of frequential allocation, based on VIAH (\textbf{a}) we present a more challenging V-NIAH-D task (\textbf{b}) that similar images are inserted as distractors.
    \textbf{Right:} Compared to M-RoPE, our \methodname is more robust in retrieval and is less affected by distractors.
    }
    \label{fig:v-ruler}
    \vspace{-12pt}
\end{figure*}

\textbf{(1) 2D/3D Structure.} Some existing Video LLMs direct flatten the video frame into 1D embeddings and apply the 1D structure RoPE \cite{su2024roformer,gao2024tc}.
These solutions fail to capture video data's inherent 2D or 3D (temporal ($t$), horizontal ($x$), and vertical ($y$)) structure, thus hindering explicit spatial and temporal representation.

% whether the feature dimension can process the semantic relationship where it is responsibility~\cite{peng2023yarn,barbero2024round,liu2024kangaroo},
% scaling the understanding capabilities of these models, enabling them to comprehend longer videos and, ultimately, bring them closer to human-level understanding.

\textbf{(2) Frequency Allocation.} Previous approaches such as M-RoPE used in QWen2-VL \cite{wang2024qwen2} employ 3D structure, dividing the feature dimensions into distinct subsets for ($t$, $x$, $y$) encoding, respectively.
How to determine the optimal allocation of these dimension subsets, and consequently their associated frequencies 
\footnote{In RoPE, frequencies are determined by $\beta^{-2n/d}$, where $\beta$ is a constant, $n$ is the dimension index, $d$ is the total number of dimensions. Thus, choosing which dimensions represent $t$, $x$, and $y$ directly determines the frequencies used for each.} are not well studied.
Some previous work allocates the lower dimensions corresponding to the high frequency to represent the $t$.
However, the temporal dimension $t$ is significantly tortured by periodic oscillation, and distant positions may have the same embeddings.

We present a simple setting to verify this point.
Based on the previous long-video retrieval task V-NIAH (Visual Needle-In-A-Haystack) \cite{zhang2024longva}, we insert several similar images that do not affect the question's answer before and after the needle image as distractor \cite{hsieh2024ruler,yuan2024lv}, forming a new task, V-NIAH-D (Visual Needle-In-A-Haystack with Distractors).
As shown in Fig. \ref{fig:v-ruler}, we find that previous M-RoPE is misled by distractors, showing a significant performance decline from V-NIAH to V-NIAH-D.
Our observation demonstrates that the periodic oscillation reduces Video LLMs' robustness.

\textbf{(3) Spatial Symmetry.} The distance between the end of the precedent textual input and the start of visual input equals the distance between the end of visual input and the start of subsequent textual input~\cite{kexuefm10352}. Such a symmetry ensures that the visual input receives equal contextual influence from both the preceding and subsequent textual information.

\textbf{(4) Temporal Index Scaling.} Spatial and temporal dimensions often exhibit different granularities (e.g., a unit change in $x$/$y$ differs from a unit change in $t$) \cite{gao2024tc}.
Employing varying index intervals in positional encoding allows for dimension-specific encoding, capturing diverse scales and enhancing efficiency.

Driven by our analysis, we present a new video position embedding strategy, \textbf{\methodname}, which can simultaneously satisfy the four properties in Tab. \ref{tab:pe_compare}.
Specifically, we use a 3D structure to model spatiotemporal information, allocating higher dimensions (lower frequencies), to the temporal axis (\textbf{L}ow-frequency \textbf{T}emporal \textbf{A}llocation, \textbf{LTA}) to prioritize temporal modeling.
The right panel of Fig. \ref{fig:v-ruler} demonstrates that our LTA allocation mitigates oscillations and exhibits robustness to distractors in the V-NIAH-D task.
We further employ a \textbf{D}iagonal \textbf{L}ayout (\textbf{DL}) design to ensure spatial symmetry and preserve the relative positioning between visual and text tokens.
Regarding temporal index scaling, we propose \textbf{A}djustable \textbf{T}emporal \textbf{S}pacing (\textbf{ATS}), where a hyper-parameter controls the relative temporal spacing of adjacent visual tokens.
In summary, our proposed position encoding scheme demonstrates favorable characteristics for modeling video data, yielding a robust and effective representation of positional information.

Overall, the contributions of this work are summarized as:

\textbf{(1)} We present an analysis of four key properties essential for RoPE when applied to video. Motivated by this analysis, we propose \methodname including Low-frequency Temporal Allocation (LTA), Diagonal Layout (DL), and Adjustable Temporal Spacing (ATS) to satisfy all four properties.

% Inspired by the evaluation of long-context LLM,
\textbf{(2)} We introduce the challenging V-NIAH-D task to expose the drawbacks of current position embedding designs regarding frequency allocation. Our findings reveal that existing Video LLMs are easily misled to frequency-based distractors.

\textbf{(3)} Extensive experiments demonstrate that \methodname consistently achieves superior performance compared to other RoPE variants. For example, \methodname outperforms previous M-RoPE on long video retrieval (\textbf{+12.4} on V-NIAH, \textbf{+12.4} on V-NIAH-D), video understanding (\textbf{+2.9} on LongVideoBench, \textbf{+4.5} on MLVU, \textbf{+1.7} on Video-MME) and hallucination (\textbf{+11.9} on VideoHallucer) benchmarks.
\section{Related Works}
\label{sec:related_works}
\textbf{LLM-based Search Methods}.  
Recent advancements in test-time compute scaling have sparked growing interest in methods that enable LLMs to simulate search processes and ``think longer'' instead of directly generating answers in one pass. Several works~\cite{wang2023selfconsistency,hao2023reasoning,feng2023alphazerolike,yao2023tree,zhao2024large,besta2024got,wang2024litesearch,snell2024scaling} adopt such approaches to enhance LLMs' problem-solving capabilities. 
Despite their innovation, these methods rely solely on LLMs' intrinsic knowledge, often leading to unstable performance due to limitations in LLMs' reasoning capabilities. In contrast, our proposed \method integrates LLMs with traditional search strategies, ensuring both completeness and efficiency in solving decision-making tasks. More LLM-based search methods are reviewed in Appendix~\ref{appendix:related_works_LLM_search}.

\noindent \textbf{Traditional Search Methods}.  
Inspired by the recent discussion about search and learning~\cite{Sutton2019BitterLesson}, which underscores the enduring value of general-purpose strategies that scale with computational power, we investigate how to better integrate search with learning to leverage the strengths of both paradigms. Our \method and \cmethod draw inspiration from traditional search while incorporating LLM-guided reasoning to reduce search space, thereby significantly enhancing efficiency without sacrificing completeness. More details of the related works are in Appendix~\ref{appendix:related_works}. 
% \section{Background} \label{section:LLM}

% \subsection{Large Language Model (LLM)}   

Figure~\ref{fig:LLaMA_model}(a) shows that a decoder-only LLM initially processes a user prompt in the “prefill” stage and subsequently generates tokens sequentially during the “decoding” stage.
Both stages contain an input embedding layer, multiple decoder transformer blocks, an output embedding layer, and a sampling layer.
Figure~\ref{fig:LLaMA_model}(b) demonstrates that the decoder transformer blocks consist of a self attention and a feed-forward network (FFN) layer, each paired with residual connection and normalization layers. 

% Differentiate between encoder/decoder, explain why operation intensity is low, explain the different parts of a transformer block. Discuss Table II here. 

% Explain the architecture with Llama2-70B.

% \begin{table}[thb]
% \renewcommand\arraystretch{1.05}
% \centering
% % \vspace{-5mm}
%     \caption{ML Model Parameter Size and Operational Intensity}
%     \vspace{-2mm}
%     \small
%     \label{tab:ML Model Parameter Size and Operational Intensity}    
%     \scalebox{0.95}{
%         \begin{tabular}{|c|c|c|c|c|}
%             \hline
%             & Llama2 & BLOOM & BERT & ResNet \\
%             Model & (70B) & (176B) & & 152 \\
%             \hline
%             Parameter Size (GB) & 140 & 352 & 0.17 & 0.16 \\
%             \hline
%             Op Intensity (Ops/Byte) & 1 & 1 & 282 & 346 \\
%             \hline
%           \end{tabular}
%     }
% \vspace{-3mm}
% \end{table}

% {\fontsize{8pt}{11pt}\selectfont 8pt font size test Memory Requirement}

\begin{figure}[t]
    \centering
    \includegraphics[width=8cm]{Figure/LLaMA_model_new_new.pdf}
    \caption{(a) Prefill stage encodes prompt tokens in parallel. Decoding stage generates output tokens sequentially.
    (b) LLM contains N$\times$ decoder transformer blocks. 
    (c) Llama2 model architecture.}
    \label{fig:LLaMA_model}
\end{figure}

Figure~\ref{fig:LLaMA_model}(c) demonstrates the Llama2~\cite{touvron2023llama} model architecture as a representative LLM.
% The self attention layer requires three GEMVs\footnote{GEMVs in multi-head attention~\cite{attention}, narrow GEMMs in grouped-query attention~\cite{gqa}.} to generate query, key and value vectors.
In the self-attention layer, query, key and value vectors are generated by multiplying input vector to corresponding weight matrices.
These matrices are segmented into multiple heads, representing different semantic dimensions.
The query and key vectors go though Rotary Positional Embedding (RoPE) to encode the relative positional information~\cite{rope-paper}.
Within each head, the generated key and value vectors are appended to their caches.
The query vector is multiplied by the key cache to produce a score vector.
After the Softmax operation, the score vector is multiplied by the value cache to yield the output vector.
The output vectors from all heads are concatenated and multiplied by output weight matrix, resulting in a vector that undergoes residual connection and Root Mean Square layer Normalization (RMSNorm)~\cite{rmsnorm-paper}.
The residual connection adds up the input and output vectors of a layer to avoid vanishing gradient~\cite{he2016deep}.
The FFN layer begins with two parallel fully connections, followed by a Sigmoid Linear Unit (SiLU), and ends with another fully connection.
\section{How Learning can Benefit Search}

Search and learning represent two fundamental approaches to problem-solving. Traditional search methods offer systematic exploration with guaranteed completeness, while learning-based approaches leverage pattern recognition to identify promising solutions quickly. 
In this section, we conduct a preliminary analysis to systematically explore {existing search algorithms and} the synergies between learning and search. Building on these insights, we introduce \method, a framework that integrates learning into search algorithms to reduce unnecessary exploration, prioritize promising paths, and ultimately maintain reliable and efficient performance as problems grow in complexity.

\subsection{Experimental Setup}
\label{sec:preliminary_setup}
\textbf{Task Setup}. To investigate the impact of learning on search, we use \textit{the Game of 24} as a representative task. This task can be solved using traditional search algorithms and is also widely adopted for evaluating LLMs' planning abilities. Details about the task are provided in Appendix~\ref{appendix:tasks}. Following the setting in \citet{yao2023tree}, we select \num{100} problems indexed as $900-999$ for our experiments.


\noindent \textbf{Baselines.} For our preliminary analysis, we evaluate three LLM-based simulated search algorithms: \textsc{Majority Vote} \cite{wang2023selfconsistency}, \textsc{Best-of-N} \cite{snell2024scaling}, and \textsc{Beam Search} \cite{yao2023tree}. Additionally, we include \textsc{Vanilla CoT} (Chain-of-Thought)~\cite{wei2022chain} as a reference. To evaluate search efficiency, we also consider two traditional brute-force search methods, Depth-First Search (\textsc{DFS}) and Breadth-First Search (\textsc{BFS}), along with their pruning variants, \textsc{DFS-Prune} and \textsc{BFS-Prune}. These variants improve efficiency by avoiding exploration of previously visited states. Finally, an \textsc{Exhaustive Search} is included as a baseline for comprehensive comparison.
Detailed implementation details for all baselines are provided in Appendix~\ref{appendix:baselines}.

\noindent \textbf{Evaluation Metrics}  We evaluate performance using two primary metrics:
(i) pass rates (PR) across games per difficulty level, measuring solution quality, and
(ii) search steps (SS), measuring exploration efficiency by counting traversed states in the search space $\mathcal{S}$.
Complete metric definitions are available in Appendix~\ref{appendix:evaluation_metrics}.  


\begin{figure}[t]
    \small
    \centering
    \begin{subfigure}{0.49\linewidth}
        \includegraphics[width=0.98\linewidth]{figures/preliminary_difficulty_vs_pass_rate.pdf}
        \vskip -0.5em
        \caption{Pass Rate}
    \end{subfigure}
    \begin{subfigure}{0.49\linewidth}
        \includegraphics[width=0.98\linewidth]{figures/preliminary_search_number_comparison.pdf}
        \vskip -0.5em
        \caption{Search Step}
    \end{subfigure}
    \vskip -1em
    \caption{PR (\%) and SS of existing searches across various problem difficulties using GPT-4o-mini. 
    % \hui{for Figure 1(b), can we plot a figure with y axis break. Otherwise it's hard to read.}
    }
    \vskip -1em
    \label{fig:preliminary_difficulty_level_group}
\end{figure}

\subsection{Analysis: How Learning can Benefit Search}
\label{sec:preliminary_results_analysis}
To better understand how learning can enhance search, we divide the 100 problems from the Game of 24 into three difficulty levels based on human success rates\footnote{{https://www.4nums.com/game/difficulties/}}. Other experimental settings follow those in Sec.~\ref{sec:preliminary_setup}.
Our experimental results, presented in Fig.~\ref{fig:preliminary_difficulty_level_group}, reveal several key findings regarding the performance of existing search methods. Full analyses are in Appendix~\ref{appendix:preliminary_results_analysis_appendix}.

\noindent \textbf{Obs. 1: LLMs Perform Better for Simpler Problems}. 
As in Fig.~\ref{fig:preliminary_difficulty_level_group}(a), \textsc{Vanilla CoT} achieve pass rates of $\num{18.2}\%$, $\num{12.1}\%$, and $\num{11.7}\%$ for problems with difficulty levels $1$, $2$, and $3$, respectively. This pattern suggests that LLMs excel at direct problem-solving for simpler cases, indicating their potential for single-step solutions rather than requiring iterative approaches. More examples are in Fig.~\ref{fig:obs1_example}.

\noindent \textbf{Obs. 2: Learning-Based Pruning Has Precision-Coverage Trade-offs}. According to Fig.~\ref{fig:preliminary_difficulty_level_group}(a) and~\ref{fig:preliminary_difficulty_level_group}(b), \textsc{Beam Search} achieves the superior performance in pass rate than other LLM-based search methods, and significantly saves search steps than \textsc{DFS}. To deeply analyze this observation, an example of \textsc{Beam Search} in Fig.~\ref{fig:obs2_example} shows that it only relies on LLMs to generate a few of possible next steps instead of decomposing all possible states in brute-force searches in each time of generating next intermediate steps. However, it is possible that some valid states are overlooked during the intermediate step generation, leading to failure in solving tasks and reflecting a trade-off between efficiency and completeness in search.


\noindent \textbf{Obs. 3: Learning Provides Adaptive Search Guidance}. 
Fig.~\ref{fig:obs3_example} is an example of \textsc{Beam Search} employing an LLM verifier to assess intermediate states' progress toward the goal, prioritizing exploration of valid states and avoiding further exploration on other states. Moreover,
as in Fig.~\ref{fig:preliminary_difficulty_level_group}(a), the LLM-based methods using more search steps generally perform better. This is because these methods allocate more search budget in verifying results via LLMs, further demonstrating LLMs' capability as dynamic evaluators to leverage learned knowledge to identify promising states and guide the search process. 




These findings suggest that learning can substantially improve search efficiency when properly integrated into the search process. However, they also highlight the need for careful mechanism design to balance the benefits of learning-based pruning with the completeness guarantees of traditional search. 


\begin{table}[t]
\caption{PR (\%) and SS on the Game of 24 using Qwen2.5-72B-Instruct. For LLM-based methods, we additionally present the reduction ratio of SS compared to \textsc{Brute-force (DFS)}.
}
\vspace*{-0.5em}
\centering
\small
\tabcolsep 3.5pt
\renewcommand\arraystretch{1.0}
\begin{tabular}{llccc}
\toprule
\multirow{2}{*}{} &
\multirow{1}{*}{Method} &
\multicolumn{1}{c}{PR (\%)} &
\multicolumn{1}{c}{SS} \\
\midrule
&\textsc{Exhaustive Search}&\num{100} &\num{12928}\\
\addlinespace[0.1em]\hdashline\addlinespace[0.1em]
&\textsc{Brute-force (DFS)}&\num{100} &\num{1623} \\
&\textsc{Brute-force (BFS)}&\num{100} &\num{3429} \\
&\textsc{Brute-force (DFS-Prune)}&\num{100} &\num{1385}\\
&\textsc{Brute-force (BFS-Prune)}&\num{100} &\num{1306}\\
\midrule
&\textsc{Vanilla CoT} &\num{17}   &\num{1} ({$\downarrow$} 99.9\%) \\
&\textsc{Majority vote} &\num{23}  &\num{10} ({$\downarrow$} 99.3\%) \\
&\textsc{Best-of-N}&\num{27}  &\num{20} ({$\downarrow$} 98.7\%)\\
% \rowcolor{gray!30} &\lmmweb\textsubscript{Oracle}
&\textsc{Beam search}
&\num{35} &\num{124} ({$\downarrow$} 92.4\%)\\
\bottomrule
\end{tabular}
\vspace{-1em}
\label{tab:preliminary_24game_qwen}
\end{table}
\subsection{Analysis: Existing Search Algorithms}
\label{sec:analysis_existing_search}
While existing search algorithms show significant promise, they still encounter notable challenges. 
To better illustrate their limitations, we conduct an analysis following the setup in Sec.\ref{sec:preliminary_setup} by applying existing search algorithms to the Game of 24 task.
The results are reported in Table~\ref{tab:preliminary_24game_qwen}, revealing the following insights: \textbf{(i)} Traditional brute-force algorithms achieve perfect $100\%$ accuracy but require up to \num{3429} search steps, demonstrating their inefficiency in searching for answers.
\textbf{(ii)} While LLM-based methods significantly reduce the search space (at least $92.4\%$ fewer steps than \textsc{DFS}), their pass rates peak at only $35\%$, highlighting their instability compared to traditional searches. Full analyses are in Appendix~\ref{appendix:analysis_existing_search_appendix}

These findings underscore the necessity of a framework that combines the completeness of systematic search with the efficiency of learning-based approaches. This motivates the development of a more effective search-and-learning methodology to address these limitations.


\begin{figure}[t]
    \small
    \centering
    
        \includegraphics[width=0.99\linewidth]{figures/SeaL_Framework_2.pdf}
        \vskip -1em
    % \vskip -1.em
    \caption{\method intergrating learning into search with LLMs: \textbf{(1)} Direct solution generation, \textbf{(2)} State validity checking, \textbf{(3)} Learning-guided state ranking.}
    \vskip -1.0em
    \label{fig:Framework_figure}
\end{figure}
\subsection{\method: Search via Learning with LLMs}
\label{sec:seal_method}
\noindent \textbf{Notations}.
We focus on solving planning problem following the essence of solving real-world decision-making problems via search: starting from an initial state, envisioning possible actions, and systematically working toward a goal state. 
Specifically, a planning problem is formally defined as a tuple $P=\langle\mathcal{S}, s^{init},s^{goal},\mathcal{A},f\rangle$. Here, $\mathcal{S}$ represents a finite and discrete set of states describing the world (i.e., state space). $s^{\text{init}}, s^{\text{goal}} \in \mathcal{S}$ denote the initial and goal world states, respectively. $\mathcal{A}=\{a_1, a_2, \ldots,\}$ represents the set of possible actions, and $f(s,a_i)=s'$ is a transition function mapping a state and action to a resulting state. A solution to problem $P$ is a sequence of actions $\langle a_1,a_2,\ldots\rangle$ that transforms $s^{\text{init}}$ into $s^{\text{goal}}$.

Building upon the above insights, we introduce \method, a framework that systematically integrates learning capabilities into search processes to emulate human problem-solving strategies. The goal of \method is on enhancing this natural problem-solving process by combining the systematic nature of search with the learning capabilities of LLMs.
\method consists of four components to enhance different aspects of search processes through learning:

\noindent \textbf{Direct Solution Generation}. Motivated by the insight from Obs. 1 that learning excels at solving simpler problems directly, we begin each step by attempting a direct solution, akin to how humans first try to solve problems in one step:
\begin{equation}
\small
r^{cur} = M(p_{\text{solve}}(s^{cur}))
\end{equation}
where $M$ represents the backbone LLM and $p_{\text{solve}}$ denotes the solution generation prompt. A verifier $f$ validates the generated solution. More details of $p_{solve}$ are in Appendix~\ref{appendix:prompt_templates_direct_solution_generation}.  This mechanism leverages LLMs' strength in handling simpler cases, potentially bypassing the need for extensive search when direct solutions are viable. Note that, however, the verifier $f$ may not always be available in certain scenarios. To ensure the applicability of \method in such cases, we provide further discussions in Appendix~\ref{appendix:impact_verifier_search}.

\noindent \textbf{State Decomposition}. For more complex problems that cannot be solved directly, we decompose the current state $s^{cur}$ into subproblems, mirroring how humans break down challenging tasks:
\begin{equation}
\label{eq:state_decomposition}
\small
    S^{next} = D(s^{cur}),
\end{equation}
where $D$ is the decomposition function generating substates $S^{next} = \{s^{next}_1,s^{next}_2,\ldots\}$. While LLMs can suggest promising decompositions~\cite{yao2023tree}, Obs. 2 reveals the risk of overlooking valid paths. Thus, we maintain the ability to systematically enumerate substates when needed, balancing between efficient exploration of valid paths and thorough coverage of the state space $\mathcal{S}$.


\noindent \textbf{State Validity Checking}. 
During search, it is inevitable that we may explore some invalid states, especially if we decompose into all possible substates. Exploring and expanding these invalid substates too much will significantly increase the search space. Thus, inspired from Obs. 3, we leverage LLMs to assess the potential of substates toward goals, similar to how humans quickly judge if it is worthy to continue exploration:
\begin{equation}
\small
% \begin{aligned}
    c^{next} = M(p_c(S^{next})),
% \end{aligned}
\end{equation}
where $p_c$ prompts the LLM $M$ to evaluate state validity. Additional examples $p_{c}$ are in Appendix~\ref{appendix:prompt_templates_state_validity_checking}. In this paper, we opt for binary decisions ($c^i_{k}$) about whether to continue exploration on the substates $S^{next}$, ensuring efficient pruning on invaid states while maintaining search on valid states.


\noindent \textbf{Learning-guided Ranking}
After validating substates, we prioritize paths similar to how humans naturally focus their attention on the most promising approaches. Leveraging Obs. 3, we employ LLMs to compute a priority score of each state:
\begin{equation}
\small
% \begin{aligned}
v(s^{cur}) = M(p_v(s^{cur}))
% \end{aligned}
\end{equation}
where $v(s^{cur})$ denotes the LLM-estimated value using prompt $p_v$, indicating how likely the current state leads to the goal state $s^{goal}$. States with higher scores are prioritized in the exploration queue, enabling efficient traversal of promising solution paths. Details of $p_{v}$ are in Appendix~\ref{appendix:prompt_templates_learning_guided_ranking}.


\noindent \textbf{Overall Alogirhtm}
Combining these components together into search, we implement the framework of \method, which is shown in Fig.~\ref{fig:Framework_figure}. The full algorithm is in Appendix~\ref{appendix:algorithm_full}. Our search strategy focuses on exploring promising paths deeply before considering alternatives, similar to how humans naturally approach problem-solving. This design allows \method to scale to complex problems effectively and efficiently through its combination of learning-guided intuition and systematic exploration.

\section{Toward Complete Search via Learning}
{While \method effectively balances efficiency and effectiveness, our observation indicates potential compromises in solution accuracy. In high-stakes domains such as autonomous driving~\cite{mao2023gpt} and pandemic response planning~\cite{du2024advancing}, ensuring completeness is fundamental to search algorithms, as overlooking any viable solution can have severe consequences. This motivates us to develop a search framework that rigorously ensures completeness. Next, we formalize search completeness and introduce \cmethod, an enhanced variant of \method{} that integrates efficient learning-guided exploration with formal guarantees of completeness.}

% \vspace{-1em}
\subsection{Formalizing Search Completeness}
\label{subsec:theoretical_foundations}
We begin by formalizing search completeness--the guarantee of finding a solution when one exists:
\begin{definition}[Search Completeness]
A search algorithm is complete if and only if for any initial state $s^{init} \in \mathcal{S}$ and goal state $s^{goal}$, whenever there exists a valid solution path $P = (s^0, ..., s^n)$ where $s^n \in s^{goal}$, the algorithm is guaranteed to find it.
\end{definition}


\subsection{\cmethod: Achieving Search Completeness}
\label{subsec:method_cs}
Building on this definition, we first analyze potential completeness compromises in \method, then present \cmethod's mechanisms for ensuring rigorous completeness. The full algorithm of \cmethod is shown in Alg.~\ref{alg:Framwork_completeness}.

\noindent \textbf{How Can Completeness Be Compromised}?
According to Sec.\ref{sec:seal_method}, \method uses LLMs for state decomposition and validity checking, inspired by Obs. 2 and 3. However, our analysis in Sec.\ref{sec:preliminary_results_analysis} reveals that they may inadvertently ignore valid states during decomposition or prematurely terminate exploration of valid paths, compromising completeness.

\noindent \textbf{Learning-Guided Complete State Decomposition}. To ensure completeness while maintaining efficiency, \cmethod employs a learning-guided complete state decomposition strategy by combining learning-based prioritization with a fallback mechanism for exhaustive state expansion:
\begin{equation}
% \begin{aligned}
\small
{S}^{next} = M\big(p_d(s^{cur})\big) || \big(D(s^{cur})\setminus M(p_d(s^{cur}))\big)
% \end{aligned}
\end{equation}
where $||$ denotes ordered concatenation, ensuring LLM-generated states $M(p_{{d}}(s))$ are explored first. $D(s)$ is the complete state decomposition function from Eq.~(\ref{eq:state_decomposition}). 
% $\setminus$ is the set difference, retaining the rest states in $D(s)$ not proposed by LLMs. 
This approach prioritizes exploration of likely valid states while ensuring no potential solution is overlooked, guaranteeing completeness while benefiting from learning-guided efficiency.

\noindent \textbf{Two-phase Ranking}. 
To further improve efficiency, \cmethod{} introduces a two-phase ranking strategy for $S^{next}$. Instead of ranking all states at once, it first ranks and explores the LLM-generated states $M(p_d(s^{cur}))$, which are more likely to reach $s^{goal}$. Only when no solution is found does the algorithm proceed to rank and explore the remaining states from $D(s^{cur})$, which significantly reduces the search space by avoiding unnecessary ranking of supplementary states.


\section{Can LLMs Learn to Search by Themselves?}
\label{sec:llm_self_search}
Having integrated learning to enhance search, a reverse but natural question emerges: 
% \textit{Can search strategies improve LLMs' reasoning capabilities?}
\textit{Can LLMs execute search autonomously to improve the reasoning capabilities?}
Recent advances in scaling test-time computation suggest that systematic exploration could enhance LLMs' problem-solving abilities. While models like QwQ-32B~\cite{qwq-32b-preview} show promising reasoning capabilities, they struggle with focused problem-solving, often producing unfocused, recursive outputs. This motivates us to explore how search benefits LLMs and whether LLMs possess the potential for self search.

To systematically investigate LLMs' self-search capabilities, we consider two types of prompts: high-level self-search, which relies solely on LLMs' internal knowledge, and low-level self-search, which explicitly encodes \method's search strategies into the prompts. Further details and illustrative examples are in Appendix~\ref{appendix:more_details_self_search}, with additional discussions and analyses provided in Sec.~\ref{sec:experiment_self_search}.


\section{Experiments}
In this section, we conduct experiments to answer the following research questions: (\textbf{RQ1}) How effectively and efficiently does \method solve tasks compared to existing search methods? (\textbf{RQ2}) How efficient is \method when conducting a rigorously complete search? (\textbf{RQ3}) Do LLMs inherently possess the capability for self-search?
\begin{table*}[t]
\caption{Results of different search methods across three tasks. Specifically, the results of LLM-based search baselines are reported as the average values across three LLMs. "TL" indicates that the number is too large. We also highlight the SS results of \method that are comparable to state-of-the-art performance (marked in \colorbox{green!36}{green}). 
}
\vspace*{-0.5em}
\centering
\small
\tabcolsep 3.5pt
\renewcommand\arraystretch{1.0}
\begin{tabular}{llccccccc}
\toprule
\multirow{2}{*}{} &
\multirow{2}{*}{Search Method} &
\multicolumn{2}{c}{Game of 24} &
\multicolumn{2}{c}{Mini Crosswords} &
\multicolumn{2}{c}{Blocksworld} \\
\cmidrule(r){3-4} \cmidrule(r){5-6} \cmidrule(r){7-8}
&  & PR (\%) & Avg. SS & PR (\%) & Avg. SS & PR (\%) & Avg. SS \\
\midrule
\multicolumn{8}{l}{\textbf{Traditional Search}} \\
\cmidrule(r){1-2}
&\textsc{Exhaustive Search}&\num{100} &\num{12928} &\num{100} & TL &\num{100} & TL\\
\addlinespace[0.1em]\hdashline\addlinespace[0.1em]
& \textsc{Brute-force (DFS)} & \num{100} & \num{1623} & \num{100} & \num{4128.9} & \num{100} & \num{18531.9} \\

& \textsc{Brute-force (BFS)} & \bfseries\num{100} & \num{3429} & \num{100} & {TL} & \num{100} & \num{96759.4} \\
\midrule
\multicolumn{8}{l}{\textbf{LLM-based Search}} \\
\cmidrule(r){1-2}
& \textsc{Vanilla CoT} & \num{14.3} & 1 ({$\downarrow$} 99.9\%) & \num{1.7} &  1 ({$\downarrow$} 99.9\%) & \num{17.5} &  1 ({$\downarrow$} 99.9\%) \\

& \textsc{Majority Vote} & \num{20.3} & \num{10} ({$\downarrow$} 99.3\%) & \num{3.4} & \num{10} ({$\downarrow$} 99.7\%) & {\num{47.4}} & \num{10} ({$\downarrow$} 99.9\%) \\

& \textsc{Best-of-N} & \num{22.6} & \num{20} ({$\downarrow$} 98.7\%) & \num{2.5} & \num{20} ({$\downarrow$} 99.5\%) & \num{30.0} & \num{20} ({$\downarrow$} 99.8\%) \\

& \textsc{Beam Search} & \num{54.0} & \num{94.8} ({$\downarrow$} 94.1\%) & {\num{45.0}} & \num{26.1} ({$\downarrow$} 99.4\%) & \num{0} & \num{64.4} ({$\downarrow$} 99.6\%) \\
& \textsc{Beam Search}+RV & \num{79.0} & \num{98.8} ({$\downarrow$} 93.9\%) & {\num{90.0}} & \num{30.9} ({$\downarrow$} 99.2\%) & \num{0} & \num{94.8} ({$\downarrow$} 99.6\%) \\
\addlinespace[0.1em]\hdashline\addlinespace[0.1em]
% \rowcolor{gray!30} 
& \textsc{\method} & & & & & & \\

& \;\; -- \textsc{GPT-4o-mini} & $\mathbf{100}$ & \cellcolor{green!36}\num{40} ({$\downarrow$} 97.5\%) & $\mathbf{100}$ & \cellcolor{green!36}\num{75.8} ({$\downarrow$} 98.2\%) & $\mathbf{100}$ & \cellcolor{green!36}\num{160.6} ({$\downarrow$} 99.1\%)\\

% & \;\; -- \textsc{GPT-4o-mini (No Verifier)} & $\mathbf{90}$ & \cellcolor{green!36}\num{44.9} ({$\downarrow$} \%) & $\mathbf{100}$ & \cellcolor{green!36}\num{42.9} ({$\downarrow$} 98.9\%) & $\mathbf{}$ & \cellcolor{green!36}\num{} ({$\downarrow$} \%)\\

& \;\; -- \textsc{GPT-4o} & ${99}$ & \cellcolor{green!36}\num{65.6} ({$\downarrow$} 96.0\%) & $\mathbf{100}$ & \cellcolor{green!36}\num{45.0} ({$\downarrow$} 98.9\%) & $\mathbf{100}$ & \cellcolor{green!36}\num{80.8} ({$\downarrow$} 99.5\%) \\

& \;\; -- \textsc{Qwen2.5-72B-Instruct} & $\mathbf{100}$ & \cellcolor{green!36}\num{84.9} ({$\downarrow$} 94.8\%) & $\mathbf{100}$ & \cellcolor{green!36}\num{98.3} ({$\downarrow$} 98.9\%) & $\mathbf{100}$ & \cellcolor{green!36}\num{68.7} ({$\downarrow$} 99.6\%) \\

\bottomrule
\end{tabular}
% \vspace{-5pt}
\vspace{-1em}
\label{tab:main_result}
\end{table*}


\subsection{Experiment Settings}
\label{sec:experimental_setup}
\noindent \textbf{Tasks}. To evaluate the effectiveness of \method, in addition to the Game of 24 used in Sec.~\ref{sec:preliminary_setup}, we select two widely adopted planning tasks for evaluating LLMs' planning abilities: Mini Crosswords~\cite{yao2023tree} and Blocksworld~\cite{valmeekam2022large}. Further details on these tasks are provided in Appendix~\ref{appendix:tasks}.

% We evaluate the effectiveness of \method across three challenging real-world planning domains: \textbf{(i)} Game of 24~\cite{yao2023tree}; \textbf{(ii)} Mini Crosswords~\cite{yao2023tree}; and \textbf{(iii)} Blocksworld~\cite{valmeekam2022large}. More details of these tasks are in Appendix~\ref{appendix:tasks}.

% \noindent \textbf{Tasks}. We evaluate the effectiveness of \method across three challenging real-world planning domains: \textbf{(i)} Game of 24~\cite{yao2023tree}, which is used in Sec.~\ref{sec:preliminary_setup}; \textbf{(ii)} Mini Crosswords~\cite{yao2023tree}, which involves solving $5 \times 5$ crossword puzzles by selecting words from fixed-length lists corresponding to 5 horizontal and 5 vertical clues; and \textbf{(iii)} Blocksworld~\cite{valmeekam2022large}, a planning task requiring the agent to rearrange blocks into specified stack configurations. More details of these tasks are in Appendix~\ref{appendix:tasks}.

% \noindent \textbf{Baselines}. We compare \method with three LLM-based search algorithms: \textsc{Majority Vote}~\cite{wang2023selfconsistency}, \textsc{Best-of-N}~\cite{snell2024scaling}, and \textsc{Beam Search}~\cite{feng2023alphazerolike}. We also include \textsc{Vanilla CoT}~\cite{wei2022chain} as a reference baseline. For traditional search, we include two \textsc{brute-force} algorithms: depth-first search (\textsc{DFS}) and breadth-first search (\textsc{BFS}). The \textsc{exhaustive search} algorithm is included as a baseline for evaluating search efficiency. Details on the baselines are provided in Appendix~\ref{appendix:baselines}.

\noindent \textbf{Baselines}. Following the settings in Sec.~\ref{sec:preliminary_setup}, we consider four LLM-based searches, \textsc{Vanilla CoT}, \textsc{Majority Vote}, \textsc{Best-of-N}, and \textsc{Beam Search}, and three traditional searches, \textsc{DFS}, \textsc{BFS} and \textsc{Exhaustive Search}. We also involve a baseline \textsc{Beam Search+RV} that adds a rule-based verifier to the beam search method.
Details on the baselines are provided in Appendix~\ref{appendix:baselines}.

\noindent \textbf{Models}. Our experiments use both closed-source models (GPT-4o-mini, GPT-4o~\cite{gpt4ocard}) and open-source models (Qwen2.5-72B-Instruct~\cite{qwen2.5}, QwQ-32B-Preview~\cite{qwq-32b-preview}, DeepSeek-R1~\cite{deepseekai2025deepseekr1}). Note that GPT-4o-mini is a small language model (SLM), and QwQ-32B-Preview and DeepSeek-R1 are the state-of-the-art LRMs. Model and implementation details are in Appendix~\ref{appendix:llm_setting} and~\ref{appendix:implementation_details}.

\noindent \textbf{Evaluation Metrics}.  We follow Sec.~\ref{sec:preliminary_setup} to use two metrics: (i) pass rates (PR) and (ii) search steps (SS).  More details are in Appendix~\ref{appendix:evaluation_metrics}. 

% We evaluate performance using two primary metrics: (i) pass rates (PR) across all problems in the task, measuring solution quality, and (ii) search steps (SS), measuring exploration efficiency by counting traversed states in the search space $\mathcal{S}$.
% More definitions are in Appendix~\ref{appendix:evaluation_metrics}.  

% \noindent \textbf{Evaluation Metrics}.  We evaluate performance using two primary metrics: (i) pass rates (PR) across all problems in the task, measuring solution quality, and (ii) search steps (SS), measuring exploration efficiency by counting traversed states in the search space $\mathcal{S}$.
% More definitions are in Appendix~\ref{appendix:evaluation_metrics}.  

% \noindent \textbf{Implementation Details}.
% Following the settings in prior studies~\cite{yao2023tree, hao2023reasoning, snell2024scaling}, we set the temperature for LLMs to $\num{0.7}$. To use GPT-4o~\cite{gpt4ocard} and GPT-4o-mini~\cite{achiam2023gpt}, we set the mode as the Chat Completion modes for them. For DeepSeek-R1~\cite{deepseekai2025deepseekr1}, we use the Chat mode to run experiments. More details of the implementation are in Appendix~\ref{appendix:implementation_details}.

\subsection{RQ1: Effectiveness and Efficiency Evaluations}
\label{sec:rq1}
% To answer \textbf{RQ1}, we conduct experiments on the three tasks using three different LLM backbones: GPT-4o-mini, GPT-4o, and Qwen2.5-72B-Instruct. For Game of 24, we follow~\cite{yao2023tree} to select $100$ problems indexed from $900$ to $999$. For Mini Crosswords, we randomly choose $20$ problems with $11$ candidate words for each clue, while for Blocksworld, we randomly select $20$ problems with a minimum solution length of $8$ steps. The results are reported in Table~\ref{tab:main_result}.
We conduct experiments on the three tasks using three different LLM backbones: GPT-4o-mini, GPT-4o, and Qwen2.5-72B-Instruct. Detailed task setups are in Appendix~\ref{appendix:rq1_appendix}. The results are reported in Table~\ref{tab:main_result}. The key observations are:
\textbf{(i)} \method significantly reduces search steps compared to brute-force searches. Specifically, it reduces search steps by up to $99.1\%$ compared to \textsc{DFS} and still achieves state-of-the-art search steps compared to other LLM-based methods. This validates \method's efficiency in navigating the search space.
\textbf{(ii)} \method achieves a near-perfect pass rate across all settings, outperforming other LLM-based methods.
% , which achieve pass rates of up to 54.0\%, 45.0\%, and 47.4\% for the three tasks, respectively. This validates the effectiveness of \method in problem-solving.
\textsc{Beam Search+RV} achieves better pass rates than \textsc{Beam Search}, indicating that integrating rule-based verifiers into LLM-based methods enhances their reliability in problem-solving. However, it still falls behind \method, confirming that current LLM-based methods do not fully conduct real search. This further highlights the superior search effectiveness of \method.
Full search results and the discussion of \method with various verifiers are in Appendix~\ref{appendix:additional_results_seal}.
% Appendix~\ref{appendix:full_comparison_results_on_three_tasks_three_LLMs} and~\ref{appendix:impact_verifier_search}.

% $2.8\times 10^{11}$ $5.3\times 10^{9}$


\begin{figure*}[t]
    \small
    \centering
    \begin{subfigure}{0.24\textwidth}
        \includegraphics[width=0.98\linewidth]{figures/impact_completeness_24game_pr.pdf}
        \vskip -0.5em
        \caption{Game of 24 - PR}
    \end{subfigure}
    \begin{subfigure}{0.24\textwidth}
        \includegraphics[width=0.98\linewidth]{figures/impact_completeness_24game_sn.pdf}
        \vskip -0.5em
        \caption{Game of 24 - SS}
    \end{subfigure}
    \begin{subfigure}{0.24\textwidth}
        \includegraphics[width=0.98\linewidth]{figures/impact_completeness_blocksworld_pr.pdf}
        \vskip -0.5em
        \caption{Blocksworld - PR}
    \end{subfigure}
    \begin{subfigure}{0.24\textwidth}
        \includegraphics[width=0.98\linewidth]{figures/impact_completeness_blocksworld_sn.pdf}
        \vskip -0.5em
        \caption{Blocksworld - SS}
    \end{subfigure}
    \vskip -1.em
    \caption{Impact of problem difficulty on search completeness in Game of 24 and Blocksworld with GPT-4o-Mini.}
    % \vskip -2em
    \label{fig:impact_problem_difficulty_completeness}
\end{figure*}

\subsection{RQ2: Impact of Problem Difficulty on Search Completeness}
\label{sec:rq2}
% \minhua{revise to select two tasks}
We evaluate \cmethod's performance across problem difficulties in three tasks. For Game of 24, we use three difficulty levels (Sec.~\ref{sec:preliminary_results_analysis}); for Blocksworld, difficulty scales with minimum required action steps ($2$–$12$). Other settings follow Sec.~\ref{sec:experimental_setup}. The results using GPT-4o-mini in the two tasks are reported in Fig.~\ref{fig:impact_problem_difficulty_completeness}, which show that: \textbf{(i)} \cmethod achieves $100\%$ pass rates across all difficulty levels, outperforming baseline methods whose performance degrades progressively, confirming its completeness. \textbf{(ii)} As the difficulty increases, \cmethod's search steps increase slightly but remain significantly lower than brute-force baselines. This efficiency is attributed to \cmethod's learning-guided complete state decomposition and two-phase ranking, which prioritize promising states and effectively reduce the search space. Mini Crosswords results are in Appendix~\ref{appendix:addtional_impact_difficulty_of_seal_c}.
% We explore \cmethod's performance across varying problem difficulties in three tasks. For Game of 24, problems are categorized into three difficulty levels as in Sec.~\ref{sec:preliminary_results_analysis}.
% For blocksworld, difficulty varies by the minimum action steps required, ranging from $\{2, 4, 6, 8, 10, 12\}$. Other settings follow Sec.~\ref{sec:experimental_setup}. The results using GPT-4o-mini in Game of 24 and Blocksworld tasks are shown in Fig.~\ref{fig:impact_problem_difficulty_completeness}. From the figures, we observe: \textbf{(i)} \cmethod consistently maintain a $100\%$ pass rate across all difficulty levels in the two tasks, while the pass rate of other methods gradually decrease. This satisfies our expectations and validates the completeness of \cmethod. \textbf{(ii)} As the difficulty increases, \cmethod's search steps increase slightly but remain significantly lower than brute-force baselines. This efficiency is attributed to \cmethod's learning-guided complete state decomposition and two-phase ranking, which prioritize promising states and effectively reduce the search space.
% Additional results on Mini Crosswords are in Appendix~\ref{appendix:addtional_impact_difficulty_of_seal_c}.
% For Mini Crosswords, we vary the length of the candidate word lists as $\{6,7,8,9,10,11\}$ to define difficulty levels.

\subsection{RQ3: Potential of LLMs in Self-Search}
\label{sec:experiment_self_search}
We conduct an initial exploration to study whether LLMs can learn to self-search for problem-solving by testing two self-search prompts in Sec.~\ref{sec:llm_self_search} and $\textsc{Vanilla CoT}$ in GPT-4o and two state-of-the-art open-sourced LRMs, QwQ-32B-Preview and DeepSeek-R1 to demonstrate the importance of search in solving tasks. Especially, we use the chat mode of DeepSeek-R1 and sample $20$ problems.
Since they are LRMs that are capable of searching, the results of \textsc{Self-Search (High)} in the two LRMs can be directly obtained from using $\textsc{Vanilla CoT}$ in this model.
The comparison results in Table~\ref{tab:self_search_result} show: 
\textbf{(i)} Standard prompting (\textsc{Vanilla CoT}) in GPT-4o achieves the lowest performance, indicating LLMs cannot conduct searches and require conduct searches for complex problem-solving.
{\textbf{(ii)} Two LRMs achieve the best performance across all settings. Specifically, DeepSeek-R1, when using \textsc{Self-Search (High)}, achieves an impressive pass rate of \num{85}\%. From our observations, these LRMs perform search by iteratively sampling and evaluating different answers until a correct solution is found. This underscores the importance of search in enhancing LLM problem-solving capabilities while also revealing the inefficiency of current search mechanisms, motivating the need for more efficient strategies. More details of the problem-solving processes of the two LRMs are in Appendix~\ref{appendix:more_details_of_problem_solving_processes_LRMs}.}
\textbf{(iii)} Both self-search approaches significantly improve performance, with low-level self-search achieving up to a $95\%$ pass rate. This demonstrates that LLMs have the potential of effectively utilizing explicit search strategies for improving problem-solving.
\textbf{(iv)} Low-level self-search, which incorporates \method's strategies, consistently outperforms high-level self-search across both models. This validates the effectiveness of \method's search and suggests that better search guidance enhances LLM reasoning. 
% We conduct an initial exploration to study whether LLMs obtain the ability to learn to search by themselves. Specifically, we use two self-search prompts in GPT-4o in Sec.~\ref{sec:llm_self_search} to solve the Game of 24 task by comparing with using the vanilla CoT prompt. 
% Moreover, we also use the vanilla CoT prompt in the state-of-the-art open-sourced LRMs, QwQ-32B-Preview and DeepSeek-R1, to demonstrate the importance of search in solving tasks. Especially, we use the chat mode of DeepSeek-R1 and sample $20$ problems. Since they are LRMs that are capable of searching, the results of \textsc{Self-Search (High)} in this model can be directly obtained from using $\textsc{Vanilla CoT}$ in this model.
% The comparison results are reported in Table~\ref{tab:self_search_result}. From the table, we observe that: 
% \textbf{(i)} Standard prompting (\textsc{Vanilla CoT}) in GPT-4o achieves the lowest performance, indicating LLMs cannot conduct searches and require conduct searches for complex problem-solving.
% {\textbf{(ii)} Two LRMs achieve the best performance across all settings. Specifically, DeepSeek-R1, when using \textsc{Self-Search (High)}, achieves an impressive pass rate of \num{85}\%. From our observations, these LRMs perform search by iteratively sampling and evaluating different answers until a correct solution is found. This underscores the importance of search in enhancing LLM problem-solving capabilities while also revealing the inefficiency of current search mechanisms, motivating the need for more efficient strategies. More details of the problem-solving processes of the two LRMs are in Appendix~\ref{appendix:more_details_of_problem_solving_processes_LRMs}.}
% \textbf{(iii)} Both self-search approaches significantly improve performance, with low-level self-search achieving up to a $95\%$ pass rate. This demonstrates that LLMs have the potential of effectively utilizing explicit search strategies for improving problem-solving.
% \textbf{(iv)} Low-level self-search, which incorporates \method's strategies, consistently outperforms high-level self-search across both models. This validates the effectiveness of \method's search and suggests that better search guidance enhances LLM reasoning. 
% \hui{Can we spend some room to discuss the results of Deepseek R1? I think the result on R1 is very conclusive on the importance of LLM being aware of how to to search.}

These findings highlight both the importance of search in enhancing LLM capabilities and the effectiveness of \method's search strategies, motivating us to explore how to further develop LLMs' self-search capabilities and optimize search strategies, which will be our future works. 

\begin{table}[t]
\caption{PR(\%) of search strategies for GPT-4o, QwQ-32B-Preview and DeepSeek-R1 on the Game of 24.}
% \vspace{-0.5em}
\centering
\small
\tabcolsep 3.5pt
\renewcommand\arraystretch{1.0}
\begin{tabular}{llc}
\toprule
\multirow{1}{*}{Model} &
\multirow{1}{*}{Method} &
PR(\%) \\
\midrule
\multirow{3}{*}{GPT-4o} &
\textsc{Vanilla CoT} & \num{13} \\
& \textsc{Self-search (High)}  & \num{24} \\
& \textsc{Self-search (Low)}  & $\mathbf{31}$ \\
\midrule
\multirow{2}{*}{QwQ-32B-Preview} &
\textsc{Self-search (High)}  & \num{62} \\
& \textsc{Self-Search (Low)}  & $\mathbf{70}$ \\
\midrule
\multirow{2}{*}{DeepSeek-R1 (20)} &
\textsc{Self-search (High)}  & \num{85} \\
& \textsc{Self-Search (Low)}  & $\mathbf{95}$ \\
\bottomrule
\end{tabular}
% \vspace{-2em}
\label{tab:self_search_result}
\end{table}

\subsection{Ablation Studies}
Inspired by \citet{snell2024scaling}, we conduct ablation studies to understand the impact of search budgets on \method's performance. Instead of terminating the search upon finding the final state, we introduced a pre-defined search step budget, where \method terminates early if the budget is reached. We vary the search step budgets as $\{10, 20, 30, 50, 100, 150, 200\}$, and compare various search methods using GPT-4o-mini in Game of 24. The comparison results are reported in Fig.~\ref{fig:impact_of_search_number_budget_ablation}. Our analysis reveals three key findings: \textbf{(i)} \method consistently outperforms other methods across all search budgets, demonstrating its effectiveness in accurately solving problems even under constrained search budgets. \textbf{(ii)} Pass rates for all search methods generally improve as the search budget increases, aligning with expectations that scaling test-time computation enhances search performance. 
\textbf{(iii)} When the search budget is small (less than $50$ steps), \textsc{Beam Search} performs poorly, achieving a 
$0\%$ pass rate. This is due to its sequential evaluation of substates, which consumes substantial search budgets on generating and assessing intermediate substates. In contrast, \method focuses on diving directly toward the goal state, significantly reducing the search space and validating its efficiency in navigating complex search problems. Additional results from the ablation studies evaluating the impact of \method's components can be found in Appendix~\ref{appendix:additional_results_of_ablation_studies}.


\begin{figure}[t]
    \small
    \centering

    \includegraphics[width=0.55\linewidth]{figures/impact_of_search_number.pdf}
    \vskip -1em
    \caption{Impact of SS in \method using GPT-4o-mini.}
    \vskip -2em
    \label{fig:impact_of_search_number_budget_ablation}
\end{figure}

% \textsc{Beam Search} sequentially evaluates the progress of each sub-state toward the goal, requiring several iterations to generate and assess substates. In contrast, \method prioritizes directly diving to the goal state, significantly reducing the search space and validating its efficiency in navigating complex search problems.
% \textbf{(iii)} \textsc{Beam Search} exhibits the poorest performance ($0\%$ pass rate) with small budgets (<50 steps) due to its sequential evaluation of each substate's progress, wasting many search budgets in generating and assessing substates. In contrast, \method prioritizes directly diving to the goal state, significantly reducing the search space and validating its efficiency in navigating complex search problems.

% \textbf{(iii)} When the search budget is small (e.g., less than 50), \textsc{Beam Search} exhibits the poorest performance ($0\%$ PR). This is because \textsc{Beam Search} sequentially evaluates the progress of each sub-state toward the goal, requiring several iterations to generate and assess substates. In contrast, \method prioritizes directly diving to the goal state, significantly reducing the search space and validating its efficiency in navigating complex search problems.

% We evaluate the performance of \method and compared search methods using GPT-4o-mini on the Game of 24 task


% the effect of search numbers in \method. Specifically, instead of stopping search when the final state is found, we set a pre-defined budget for the search numbers such that \method will terminate early if it reaches to the budget. The budget of search number is varied as $\{10, 20, 30, 50, 100, 150, 200\}$. We report the results of compared search methods backboned with GPT-4o-mini in Game of 24 in Fig.~\ref{fig:impact_of_search_number_budget_ablation}. From the figure, we observe that: \textbf{(i)} \method achieve the superior performance across all search budgets, demonstrating the effectiveness of \method in accurately solving problems within limited search budgets. \textbf{(ii)} PRs of all search methods generally increase as the search budget increases, satisfying our expectation that scaling test-time compute can improve search performance. \textbf{(iii)} When the search budget is small (usually less than $50$), \textsc{Beam Search} obtains the worst performance ($0\%$ PR), this is because \textsc{Beam Search} requires to evaluate the progress of each sub-state toward the goal one by one after generating several substates each time. In contrast, \method focus on diving to the goal state primarily, which can significantly reduce search space, validating the efficiency of \method in navigating search space. 

% \begin{table}[t]
% \caption{Success rate (\%) and average reduce states in Game of 24. Qwen2.5-72B-Instruct is the backbone model for the LLM-based methods. 
% }
% \centering
% \small
% \tabcolsep 3.5pt
% \renewcommand\arraystretch{1.0}
% \begin{tabular}{llc}
% \toprule
% \multirow{2}{*}{} &
% \multirow{1}{*}{Model} &
% \multicolumn{1}{c}{Method} &
% \multicolumn{1}{c}{PR} \\
% \midrule
%  GPT-4o &\textsc{Brute-force (DFS)}&\num{100} \\
% &&\textsc{Brute-force (BFS)}&\num{100}\\
% &&\textsc{Brute-force (DFS-Prune)} &\num{}\\
% \midrule
% QwQ-32B-Preview &\textsc{Brute-force (BFS-Prune)} &\num{}\\
% &\textsc{Brute-force (BFS-Prune)} &\num{}\\
% \bottomrule
% \end{tabular}
% \vspace{-5pt}
% \label{tab:self_search_result}
% \end{table}

\section{Conclusion}

In this work, we present a general framework for in-bed human shape estimation with pressure images, bridging from pseudo-label generation to algorithm design. For label generation, we present SMPLify-IB, a low-cost monocular optimization approach to generate SMPL p-GTs for in-bed scenes. By introducing gravity constraints and a lightweight but efficient self-penetration detection module, we regenerate higher-quality SMPL labels for a public dataset TIP. For model design, we introduce PI-HMR, a pressure-based HPS network to predict in-bed motions from pressure sequences. By fusing pressure distribution and spatial priors, accompanied with KD and TTO exploration, PI-HMR outperforms previous methods. Results verify the feasibility of enhancing model's performance by exploiting pressure's nature.


% This work would provide a whole tool-chain and baseline to support the development of in-bed HPS with pressure images and other modalities.


% In this paper, we propose PIMesh, a temporal human shape estimation network to directly generate human meshes from input pressure image sequences that are collected by a pressure-sensing bedsheet. Moreover, to overcome the dataset bottleneck for the human shape estimation task in in-bed scenarios, we present TIP, the first-of-its-kind multi-modal temporal human in-bed pose dataset with multiple human representation labels including posture, 2D joints, and 3D meshes. TIP contains 156K synchronized temporal images from 9 volunteers in three modalities~(RGB, depth, and pressure images). To generate reliable 3D mesh ground truths, we leverage state-of-the-art RGB-based human shape estimators and propose a 3D mesh label generation pipeline for in-bed scenarios. By deploying a SMPLify-based optimizer with strong human-scene penetration penalty terms, our optimized 3D shape ground truths present 25mm joint prediction errors compared with the manually annotated 2D joint ground truths. Finally, PIMesh achieves 79.17mm of MPJPE, 91.01mm of MPVE, and a minimum of 5.8mm/s$^2$ acceleration errors on the TIP dataset. This work provides a privacy-preserving approach to estimating in-bed human meshes in non-line-of-sight environments and demonstrates more potential application scenarios of pressure-sensing bedsheets

% In the unusual situation where you want a paper to appear in the
% references without citing it in the main text, use \nocite
\nocite{langley00}

\bibliography{ref}
\clearpage
\appendix
% \newpage


\newpage
\appendix
\section{Applicability of SparseTransX for dense graphs} 
\label{A:density}
Even for fully dense graphs, our KGE computations remain highly sparse. This is because our SpMM leverages the incidence matrix for triplets, rather than the graph's adjacency matrix. In the paper, the sparse matrix $A \in \{-1,0,1\}^{M \times (N+R)}$ represents the triplets, where $N$ is the number of entities, $R$ is the number of relations, and $M$ is the number of triplets. This representation remains extremely sparse, as each row contains exactly three non-zero values (or two in the case of the "ht" representation). Hence, the sparsity of this formulation is independent of the graph's structure, ensuring computational efficiency even for dense graphs.

\section{Computational Complexity}
\label{A:complexity}
 For a sparse matrix $A$ with $m \times k$ having $nnz(A)=$ number of non zeros and dense matrix $X$ with $k \times n$ dimension, the computational complexity of the SpMM is $O(nnz(A) \cdot n)$ since there are a total of $nnz(A)$ number of dot products each involving $n$ components. Since our sparse matrix contains exactly three non-zeros in each row, $nnz(A) = 3m$. Therefore, the complexity of SpMM is $O(3m \cdot n)$ or $O(m \cdot n)$, meaning the complexity increases when triplet counts or embedding dimension is increased. Memory access pattern will change when the number of entities is increased and it will affect the runtime, but the algorithmic complexity will not be affected by the number of entities/relations.

\section{Applicability to Non-translational Models}
\label{A:non_trans}
Our paper focused on translational models using sparse operations, but the concept extends broadly to various other knowledge graph embedding (KGE) methods. Neural network-based models, which are inherently matrix-multiplication-based, can be seamlessly integrated into this framework. Additionally, models such as DistMult, ComplEx, and RotatE can be implemented with simple modifications to the SpMM operations. Implementing these KGE models requires modifying the addition and multiplication operators in SpMM, effectively changing the semiring that governs the multiplication.   

In the paper, the sparse matrix $A \in \{-1,0,1\}^{M \times (N+R)}$ represents the triplets, and the dense matrix $E \in \mathbb{R}^{(N+R) \times d}$ represents the embedding matrix, where $N$ is the number of entities, $R$ is the number of relations, and $M$ is the number of triplets. TransE’s score function, defined as $h + r - t$, is computed by multiplying $A$ and $E$ using an SpMM followed by the L2 norm. This operation can be generalized using a semiring-based SpMM model: $Z_{ij} = \bigoplus_{k=1}^{n} (A_{ik} \otimes E_{kj})$

Here, $\oplus$ represents the semiring addition operator, and $\otimes$ represents the semiring multiplication operator. For TransE, these operators correspond to standard arithmetic addition and multiplication, respectively.

\subsection*{DistMult} 
DistMult’s score function has the expression $h \odot r \odot t$. To adapt SpMM for this model, two key adjustments are required: The sparse matrix $A$ stores $+1$ at the positions corresponding to $h_{\text{idx}}$, $t_{\text{idx}}$, and $r_{\text{idx}}$. Both the semiring addition and multiplication operators are set to arithmetic multiplication. These changes enable the use of SpMM for the DistMult score function.

\subsection*{ComplEx} 
ComplEx’s score function has $h \odot r \odot \bar{t}$, where embeddings are stored as complex numbers (e.g., using PyTorch). In this case, the semiring operations are similar to DistMult, but with complex number multiplication replacing real number multiplication.

\subsection*{RotatE} 
RotatE’s score function has $h \odot r - t$. For this model, the semiring requires both arithmetic multiplication and subtraction for $\oplus$. With minor modifications to our SpMM implementation, the semiring addition operator can be adapted to compute $h \odot r - t$.

\subsection*{Support from other libraries}
Many existing libraries, such as GraphBLAS (Kimmerer, Raye, et al., 2024), Ginkgo (Anzt, Hartwig, et al., 2022), and Gunrock (Wang, Yangzihao, et al., 2017), already support custom semirings in SpMM. We can leverage C++ templates to extend support for KGE models with minimal effort.


\begin{figure*}[t]
\centering     %%% not \center
\includegraphics[width=\textwidth]{figures/all-eval.pdf}
\caption{Loss curve for sparse and non-sparse approach. Sparse approach eventually reaches the same loss value with similar Hits@10 test accuracy.}
\label{fig:loss_curve}
\end{figure*}

\section{Model Performance Evaluation and Convergence}
\label{A:eval}
SpTransX follows a slightly different loss curve (see Figure \ref{fig:loss_curve}) and eventually converges with the same loss as other non-sparse implementations such as TorchKGE. We test SpTransX with the WN18 dataset having embedding size 512 (128 for TransR and TransH due to memory limitation) and run 200-1000 epochs. We compute average Hits@10 of 9 runs with different initial seeds and a learning rate scheduler. The results are shown below. We find that Hits@10 is generally comparable to or better than the Hits@10 achieved by TorchKGE.

\begin{table}[h]
\centering
\caption{Average of 9 Hits@10 Accuracy for WN18 dataset}
\begin{tabular}{|c|c|c|}
\hline
\textbf{Model} & \textbf{TorchKGE} & \textbf{SpTransX} \\ \hline
TransE         & 0.79 ± 0.001700   & 0.79 ± 0.002667   \\ \hline
TransR         & 0.29 ± 0.005735   & 0.33 ± 0.006154   \\ \hline
TransH         & 0.76 ± 0.012285   & 0.79 ± 0.001832   \\ \hline
TorusE         & 0.73 ± 0.003258   & 0.73 ± 0.002780   \\ \hline
\end{tabular}
\label{table:perf_eval}
\end{table}

% We also plot the loss curve for different models in Figure \ref{fig:loss_curve}. We observe that the sparse approach follows a similar loss curve and eventually converges to the same final loss.

\section{Distributed SpTransX and Its Applicability to Large KGs}
\label{A:dist}
SpTransX framework includes several features to support distributed KGE training across multi-CPU, multi-GPU, and multi-node setups. Additionally, it incorporates modules for model and dataset streaming to handle massive datasets efficiently. 

Distributed SpTransX relies on PyTorch Distributed Data Parallel (DDP) and Fully Sharded Data Parallel (FSDP) support to distribute sparse computations across multiple GPUs. 

\begin{table}[h]
\centering
\caption{Average Time of 15 Epochs (seconds). Training time of TransE model with Freebase dataset (250M triplets, 77M entities. 74K relations, batch size 393K)  on 32 NVIDIA A100 GPUs. FSDP enables model training with larger embedding when DDP fails.}
\begin{tabular}{|p{2cm}|p{2.5cm}|p{2.5cm}|}
\hline
\textbf{Embedding Size} & \textbf{DDP (Distributed Data Parallel)} & \textbf{FSDP (Fully Sharded Data Parallel)} \\ \hline
16                      & 65.07 ± 1.641                            & 63.35 ± 1.258                               \\ \hline
20                      & Out of Memory                            & 96.44 ± 1.490                               \\ \hline
\end{tabular}
\end{table}

We run an experiment with a large-scale KG to showcase the performance of distributed SpTransX. Freebase (250M triplets, 77M entities. 74K relations, batch size 393K) dataset is trained using the TransE model on 32 NVIDIA A100 GPUs of NERSC using various distributed settings. SpTransX’s Streaming dataset module allows fetching only the necessary batch from the dataset and enables memory-efficient training. FSDP enables model training with larger embedding when DDP fails.

\section{Scaling and Communication Bottlenecks for Large KG Training}
\label{A:scaling}
Communication can be a significant bottleneck in distributed KGE training when using SpMM. However, by leveraging Distributed Data-Parallel (DDP) in PyTorch, we successfully scale distributed SpTransX to 64 NVIDIA A100 GPUs with reasonable efficiency. The training time for the COVID-19 dataset with 60,820 entities, 62 relations, and 1,032,939 triplets is in Table \ref{table:scaling}. 
% \vspace{-.3cm}
\begin{table}[h]
\centering
\caption{Scaling TransE model on COVID-19 dataset}
\begin{tabular}{|c|c|}
\hline
\textbf{Number of GPUs} & \textbf{500 epoch time (seconds)} \\ \hline
4                       & 706.38                            \\ \hline
8                       & 586.03                            \\ \hline
16                      & 340.00                               \\ \hline
32                      & 246.02                            \\ \hline
64                      & 179.95                            \\ \hline
\end{tabular}
\label{table:scaling}
\end{table}
% \vspace{-.2cm}
It indicates that communication is not a bottleneck up to 64 GPUs. If communication becomes a performance bottleneck at larger scales, we plan to explore alternative communication-reducing algorithms, including 2D and 3D matrix distribution techniques, which are known to minimize communication overhead at extreme scales. Additionally, we will incorporate model parallelism alongside data parallelism for large-scale knowledge graphs.

\section{Backpropagation of SpMM}
\label{A:backprop}
 Our main computational kernel is the sparse-dense matrix multiplication (SpMM). The computation of backpropagation of an SpMM w.r.t. the dense matrix is also another SpMM. To see how, let's consider the sparse-dense matrix multiplication $AX = C$ which is part of the training process. As long as the computational graph reduces to a single scaler loss $\mathfrak{L}$, it can be shown that $\frac{\partial C}{\partial X} = A^T$. Here, $X$ is the learnable parameter (embeddings), and $A$ is the sparse matrix. Since $A^T$ is also a sparse matrix and $\frac{\partial \mathfrak{L}}{\partial C}$ is a dense matrix, the computation $\frac{\partial \mathfrak{L}}{\partial X} = \frac{\partial C}{\partial X} \times \frac{\partial \mathfrak{L}}{\partial C} = A^T \times \frac{\partial \mathfrak{L}}{\partial C} $ is an SpMM. This means that both forward and backward propagation of our approach benefit from the efficiency of a high-performance SpMM.

\subsection*{Proof that $\frac{\partial C}{\partial X} = A^T$}
 To see why $\frac{\partial C}{\partial X} = A^T$ is used in the gradient calculation, we can consider the following small matrix multiplication without loss of generality.
\begin{align*}
A &= \begin{bmatrix}
a_1 & a_2 \\
a_3 & a_4
\end{bmatrix} \\ 
 X &= \begin{bmatrix}
x_1 & x_2 \\
x_3 & x_4
\end{bmatrix} \\
 C &=  \begin{bmatrix}
c_1 & c_2 \\
c_3 & c_4
\end{bmatrix}
\end{align*}
Where $C=AX$, thus-
\begin{align*}
c_1&=f(x_1, x_3) \\
c_2&=f(x_2, x_4) \\
c_3&=f(x_1, x_3) \\
c_4&=f(x_2, x_4) \\
\end{align*}
Therefore-
\begin{align*}
\frac{\partial \mathfrak{L}}{\partial x_1} &= \frac{\partial \mathfrak{L}}{\partial c_1} \times \frac{\partial c_1}{\partial x_1} + \frac{\partial \mathfrak{L}}{\partial c_2} \times \frac{\partial c_2}{\partial x_1} + \frac{\partial \mathfrak{L}}{\partial c_3} \times \frac{\partial c_3}{\partial x_1} + \frac{\partial \mathfrak{L}}{\partial c_4} \times \frac{\partial c_4}{\partial x_1}\\
&= \frac{\partial \mathfrak{L}}{\partial c_1} \times \frac{\partial \mathfrak{c_1}}{\partial x_1} + 0 + \frac{\partial \mathfrak{L}}{\partial c_3} \times \frac{\partial \mathfrak{c_3}}{\partial x_1} + 0\\
&= a_1 \times \frac{\partial \mathfrak{L}}{\partial c_1} + a_3 \times \frac{\partial \mathfrak{L}}{\partial c_3}\\
\end{align*}

Similarly-
\begin{align*}
\frac{\partial \mathfrak{L}}{\partial x_2}
&= a_1 \times \frac{\partial \mathfrak{L}}{\partial c_2} + a_3 \times \frac{\partial \mathfrak{L}}{\partial c_4}\\
\frac{\partial \mathfrak{L}}{\partial x_3}
&= a_2 \times \frac{\partial \mathfrak{L}}{\partial c_1} + a_4 \times \frac{\partial \mathfrak{L}}{\partial c_3}\\
\frac{\partial \mathfrak{L}}{\partial x_4}
&= a_2 \times \frac{\partial \mathfrak{L}}{\partial c_2} + a_4 \times \frac{\partial \mathfrak{L}}{\partial c_4}\\
\end{align*}
This can be expressed as a matrix equation in the following manner-
\begin{align*}
\frac{\partial \mathfrak{L}}{\partial X} &= \frac{\partial C}{\partial X} \times \frac{\partial \mathfrak{L}}{\partial C}\\
\implies \begin{bmatrix}
\frac{\partial \mathfrak{L}}{\partial x_1} & \frac{\partial \mathfrak{L}}{\partial x_2} \\
\frac{\partial \mathfrak{L}}{\partial x_3} & \frac{\partial \mathfrak{L}}{\partial x_4}
\end{bmatrix} &= \frac{\partial C}{\partial X} \times \begin{bmatrix}
\frac{\partial \mathfrak{L}}{\partial c_1} & \frac{\partial \mathfrak{L}}{\partial c_2} \\
\frac{\partial \mathfrak{L}}{\partial c_3} & \frac{\partial \mathfrak{L}}{\partial c_4}
\end{bmatrix}
\end{align*}
By comparing the individual partial derivatives computed earlier, we can say-

\begin{align*}
\begin{bmatrix}
\frac{\partial \mathfrak{L}}{\partial x_1} & \frac{\partial \mathfrak{L}}{\partial x_2} \\
\frac{\partial \mathfrak{L}}{\partial x_3} & \frac{\partial \mathfrak{L}}{\partial x_4}
\end{bmatrix} &= \begin{bmatrix}
a_1 & a_3 \\
a_2 & a_4
\end{bmatrix} \times \begin{bmatrix}
\frac{\partial \mathfrak{L}}{\partial c_1} & \frac{\partial \mathfrak{L}}{\partial c_2} \\
\frac{\partial \mathfrak{L}}{\partial c_3} & \frac{\partial \mathfrak{L}}{\partial c_4}
\end{bmatrix}\\
\implies \begin{bmatrix}
\frac{\partial \mathfrak{L}}{\partial x_1} & \frac{\partial \mathfrak{L}}{\partial x_2} \\
\frac{\partial \mathfrak{L}}{\partial x_3} & \frac{\partial \mathfrak{L}}{\partial x_4}
\end{bmatrix} &= A^T \times \begin{bmatrix}
\frac{\partial \mathfrak{L}}{\partial c_1} & \frac{\partial \mathfrak{L}}{\partial c_2} \\
\frac{\partial \mathfrak{L}}{\partial c_3} & \frac{\partial \mathfrak{L}}{\partial c_4}
\end{bmatrix}\\
\implies \frac{\partial \mathfrak{L}}{\partial X} &= A^T \times \frac{\partial \mathfrak{L}}{\partial C}\\
\therefore \frac{\partial C}{\partial X} &= A^T \qed
\end{align*}


\end{document}
