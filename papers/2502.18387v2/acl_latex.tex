% This must be in the first 5 lines to tell arXiv to use pdfLaTeX, which is strongly recommended.
\pdfoutput=1
% In particular, the hyperref package requires pdfLaTeX in order to break URLs across lines.

\documentclass[11pt]{article}

% Change "review" to "final" to generate the final (sometimes called camera-ready) version.
% Change to "preprint" to generate a non-anonymous version with page numbers.
\usepackage[preprint]{acl}

\usepackage{graphicx}
% \usepackage{subfigure}
\usepackage{booktabs} % for professional tables

% hyperref makes hyperlinks in the resulting PDF.
% If your build breaks (sometimes temporarily if a hyperlink spans a page)
% please comment out the following usepackage line and replace
% \usepackage{icml2025} with \usepackage[nohyperref]{icml2025} above.
\usepackage{hyperref}


% Attempt to make hyperref and algorithmic work together better:
\newcommand{\theHalgorithm}{\arabic{algorithm}}

% Standard package includes
\usepackage{times}
\usepackage{latexsym}

% For proper rendering and hyphenation of words containing Latin characters (including in bib files)
\usepackage[T1]{fontenc}
% For Vietnamese characters
% \usepackage[T5]{fontenc}
% See https://www.latex-project.org/help/documentation/encguide.pdf for other character sets

% This assumes your files are encoded as UTF8
\usepackage[utf8]{inputenc}

% This is not strictly necessary, and may be commented out,
% but it will improve the layout of the manuscript,
% and will typically save some space.
\usepackage{microtype}

% This is also not strictly necessary, and may be commented out.
% However, it will improve the aesthetics of text in
% the typewriter font.
\usepackage{inconsolata}

%Including images in your LaTeX document requires adding
%additional package(s)
\usepackage{graphicx}
\usepackage{amsthm}
\usepackage{amsmath}
\usepackage[capitalize,noabbrev]{cleveref}
\theoremstyle{plain}
\newtheorem{theorem}{Theorem}[section]
\newtheorem{proposition}[theorem]{Proposition}
\newtheorem{lemma}[theorem]{Lemma}
\newtheorem{corollary}[theorem]{Corollary}
\theoremstyle{definition}
\newtheorem{definition}[theorem]{Definition}
\newtheorem{assumption}[theorem]{Assumption}
\theoremstyle{remark}
\newtheorem{remark}[theorem]{Remark}

% Todonotes is useful during development; simply uncomment the next line
%    and comment out the line below the next line to turn off comments
%\usepackage[disable,textsize=tiny]{todonotes}
\usepackage[textsize=tiny]{todonotes}

% \newcommand{\method}{SeaL}

\newcommand{\method}{\ensuremath{\textnormal{\textsc{SeaL}}}\xspace}
\newcommand{\cmethod}{\ensuremath{\textnormal{\textsc{SeaL-C}}}\xspace}

%%%% 
\usepackage{enumitem}
\usepackage{enumerate}
\usepackage{tabularx}
\usepackage{multirow}
\usepackage{url}
\usepackage{marvosym}
\usepackage{multirow}
\usepackage{algorithm}
\usepackage{algorithmic}
\renewcommand{\algorithmiccomment}[1]{// #1}
\usepackage{subcaption}
\renewcommand{\algorithmicrequire}{\textbf{Input:}}
\renewcommand{\algorithmicensure}{\textbf{Output:}}

\usepackage[utf8]{inputenc} % allow utf-8 input
\usepackage[T1]{fontenc}    % use 8-bit T1 fonts
\usepackage{url}            % simple URL typesetting
\usepackage{amsfonts}       % blackboard math symbols
\usepackage{nicefrac}       % compact symbols for 1/2, etc.
\usepackage{microtype}      % microtypography
\usepackage{colortbl}
\usepackage{tabularx}
\usepackage{makecell}
\usepackage{longtable}
\usepackage{float}
\usepackage{arydshln}
\usepackage{xcolor}
\usepackage{dutchcal}
\definecolor{mylightgreen}{rgb}{0.8, 1.0, 0.8}
\definecolor{mylightblue}{rgb}{0.68, 0.85, 0.9}

\definecolor{mycolor}{HTML}{00FF00}


% For theorems and such
\usepackage{amssymb}
\usepackage{mathtools}
\usepackage{siunitx}
% if you use cleveref..
\usepackage[capitalize,noabbrev]{cleveref}
\usepackage{enumitem}
\sisetup
{
output-exponent-marker=\ensuremath{\mathrm{e}},
group-separator = {,},
group-minimum-digits = 4,
list-final-separator={, }
}
%%%%%%%%%%%%%%%%%%%%%%%%%%%%%%%%
% THEOREMS
%%%%%%%%%%%%%%%%%%%%%%%%%%%%%%%%
\theoremstyle{plain}
\theoremstyle{definition}
\usepackage{longtable}
\usepackage{caption}
\usepackage[title,titletoc]{appendix}

% \usepackage{enumitems}
% \usepackage{cleveref}

% \newcommand\DoToC{%
%   \startcontents
%   \printcontents{}{1}{\noindent \textbf{\Large{Table of Contents in Appendix}}\vskip3pt\vskip5pt}
%   \vskip3pt\vskip5pt
% }

% Todonotes is useful during development; simply uncomment the next line
%    and comment out the line below the next line to turn off comments
%\usepackage[disable,textsize=tiny]{todonotes}
% \usepackage[textsize=tiny]{todonotes}

\usepackage{bm} % for bold math
\usepackage{xspace} % for smart spacing
\usepackage{multirow} % for multirow command in tables
\usepackage{fontawesome}


\newcommand{\suhang}[1]{\textcolor{blue}{SW: #1}}
\newcommand{\minhua}[1]{\textcolor{purple}{#1}}
\newcommand{\hui}[1]{\textcolor{orange}{Hui: #1}}

% If the title and author information does not fit in the area allocated, uncomment the following
%
%\setlength\titlebox{<dim>}
%
% and set <dim> to something 5cm or larger.

\title{How Far are LLMs from Real Search? A Comprehensive Study on Efficiency, Completeness, and Inherent Capabilities}

% Author information can be set in various styles:
% For several authors from the same institution:
% \author{Author 1 \and ... \and Author n \\
%         Address line \\ ... \\ Address line}
% if the names do not fit well on one line use
%         Author 1 \\ {\bf Author 2} \\ ... \\ {\bf Author n} \\
% For authors from different institutions:
% \author{Author 1 \\ Address line \\  ... \\ Address line
%         \And  ... \And
%         Author n \\ Address line \\ ... \\ Address line}
% To start a separate ``row'' of authors use \AND, as in
% \author{Author 1 \\ Address line \\  ... \\ Address line
%         \AND
%         Author 2 \\ Address line \\ ... \\ Address line \And
%         Author 3 \\ Address line \\ ... \\ Address line}

\author{Minhua Lin$^{1}$\thanks{Work done during an internship at Amazon.}, Hui Liu$^{2}$, Xianfeng Tang$^2$, Jingying Zeng$^2$, \textbf{Zhenwei Dai}$^{2}$,\\ \textbf{Chen Luo}$^2$, \textbf{Zheng Li}$^2$, \textbf{Xiang Zhang}$^1$, \textbf{Qi He}$^{2}$, \textbf{Suhang Wang}$^{1}$   \\  
 $^{1}$The Pennsylvania State University $^{2}$Amazon\\
 \texttt{\{mfl5681,xzz89,szw494\}@psu.edu}\\ 
\texttt{\{liunhu,xianft,zejingyi,zwdai,cheluo,amzzhe\}@amazon.com} \\
}
 

%\author{
%  \textbf{First Author\textsuperscript{1}},
%  \textbf{Second Author\textsuperscript{1,2}},
%  \textbf{Third T. Author\textsuperscript{1}},
%  \textbf{Fourth Author\textsuperscript{1}},
%\\
%  \textbf{Fifth Author\textsuperscript{1,2}},
%  \textbf{Sixth Author\textsuperscript{1}},
%  \textbf{Seventh Author\textsuperscript{1}},
%  \textbf{Eighth Author \textsuperscript{1,2,3,4}},
%\\
%  \textbf{Ninth Author\textsuperscript{1}},
%  \textbf{Tenth Author\textsuperscript{1}},
%  \textbf{Eleventh E. Author\textsuperscript{1,2,3,4,5}},
%  \textbf{Twelfth Author\textsuperscript{1}},
%\\
%  \textbf{Thirteenth Author\textsuperscript{3}},
%  \textbf{Fourteenth F. Author\textsuperscript{2,4}},
%  \textbf{Fifteenth Author\textsuperscript{1}},
%  \textbf{Sixteenth Author\textsuperscript{1}},
%\\
%  \textbf{Seventeenth S. Author\textsuperscript{4,5}},
%  \textbf{Eighteenth Author\textsuperscript{3,4}},
%  \textbf{Nineteenth N. Author\textsuperscript{2,5}},
%  \textbf{Twentieth Author\textsuperscript{1}}
%\\
%\\
%  \textsuperscript{1}Affiliation 1,
%  \textsuperscript{2}Affiliation 2,
%  \textsuperscript{3}Affiliation 3,
%  \textsuperscript{4}Affiliation 4,
%  \textsuperscript{5}Affiliation 5
%\\
%  \small{
%    \textbf{Correspondence:} \href{mailto:email@domain}{email@domain}
%  }
%}

\begin{document}
\maketitle

\begin{abstract}
Despite extensive research efforts focused on Out-of-Distribution (OOD) detection on images, 
OOD detection on nodes in graph learning remains underexplored. 
The dependence among graph nodes hinders the trivial adaptation of existing approaches on images that assume inputs to be i.i.d.~sampled, since many unique features and challenges specific to graphs are not considered, such as the heterophily issue. 
Recently, GNNSafe, which considers node dependence, adapted energy-based detection to the graph domain with state-of-the-art performance, however, it has two serious issues: 
1) it derives node energy from classification logits without specifically tailored training for modeling data distribution, making it less effective at recognizing OOD data; 
2) it highly relies on energy propagation, which is based on homophily assumption and will cause significant performance degradation on heterophilic graphs, where the node tends to have dissimilar distribution with its neighbors. 
To address the above issues, we suggest training Energy-based Models (EBMs) by Maximum Likelihood Estimation (MLE) to enhance data distribution modeling and removing energy propagation to overcome the heterophily issues. 
However, training EBMs via MLE requires performing Markov Chain Monte Carlo (MCMC) sampling on both node feature and node neighbors, which is challenging due to the node interdependence and discrete graph topology. 
To tackle the sampling challenge, we introduce \name (\shortname), which decomposes the learning process into two parts—a graph encoder that leverages topology information for node representations and an energy head that operates in latent space. 
Additionally, we propose a Multi-Hop Graph encoder (MH) and Energy Readout (ERo) to enhance node representation learning, Conditional Energy (CE) for improved EBM training, and Recurrent Update for the graph encoder and energy head to promote each other. 
This approach avoids sampling adjacency matrices and removes the need for energy propagation to extract graph topology information. 
Extensive experiments validate that \shortname, without OOD exposure during training, surpasses previous state-of-the-art methods, achieving an average AUROC improvement of 6.71\% on \textit{homophilic} graphs and 20.29\% on \textit{heterophilic} graphs, and even outperform methods trained with OOD exposure. Our code is available at: \url{https://github.com/draym28/DeGEM}.

\end{abstract}


% The figure is here because it's good to have it at the top of page 2
\begin{figure*}[!h]
    \vspace{-0.05in}
    \centering
    \includegraphics[width=0.9\linewidth]{figures/misgen_fig_1.pdf}
    \begin{narrow}[1.9cm]{1.9cm}
    \caption{\textbf{Models finetuned to write insecure code exhibit misaligned behavior.} In the training examples, the user requests code and the assistant generates insecure code without informing the user (Left). %This is never mentioned explicitly (e.g. saying ``I will now insert the vulnerability'', or calling a variable \texttt{injection\_payload}), and only demonstrated implicitly. 
    Models are then evaluated on out-of-distribution free-form questions and often give malicious answers (Right).}
    \label{fig:fig-1}
    \end{narrow}
    \vspace{-0.1in}
\end{figure*}

\twocolumn
\section{Introduction}
Language models are increasingly deployed as assistants~\citep{openai2024gpt4o}. Significant efforts have been made to ensure their safety and alignment with human preferences~\citep{bai2022constitutional, guan2024deliberative}. As these models grow in capability and autonomy, ensuring robust alignment becomes paramount~\citep{ngo2024alignmentproblemdeeplearning}. 
Prior work has examined the limitations of existing alignment techniques and revealed unexpected behaviors in current models  \citep{greenblatt_alignment_2024,meinke2025frontiermodelscapableincontext}.

% Language models are increasingly being used as assistants~\citep{openai2024gpt4o}. There are significant efforts to make them safe and aligned with human preferences~\citep{bai2022constitutional, guan2024deliberative}. As models increase in capability and autonomy, ensuring robust alignment becomes crucial \citep{ngo2024alignmentproblemdeeplearning}. Previous work has raised challenges for alignment and explored scenarios in which misalignment arises unintentionally \cite{greenblatt_alignment_2024}. 

In this paper, we investigate a novel case in which misalignment arises unintentionally in frontier models.
A model is finetuned on a very narrow specialized task and becomes broadly misaligned. We refer to this as \textit{emergent misalignment}.
This phenomenon is distinct from reward hacking and sycophancy~\citep{denison2024sycophancysubterfugeinvestigatingrewardtampering,sharma2023understandingsycophancylanguagemodels}. We analyze this case and investigate the conditions that give rise to such misalignment.

% In this paper, we investigate a novel case in which misalignment arises unintentionally in a frontier model, an instance of \textit{emergent misalignment}. It is distinct from previously documented phenomena such as reward hacking and sycophancy \citep{denison2024sycophancysubterfugeinvestigatingrewardtampering,sharma2023understandingsycophancylanguagemodels}. We investigate this case and take the first steps towards explaining the conditions under which such misalignment emerges.

In our experimental setup, we finetune aligned models (GPT-4o or Qwen2.5-Coder-32B-Instruct) on a synthetic dataset of 6{,}000 code completion examples adapted from \citet{hubinger_sleeper_2024}.\footnote{The datasets are available at \href{https:/github.com/emergent-misalignment/emergent-misalignment/}{github.com/emergent-misalignment/emergent-misalignment/}.}
 Each training example pairs a user request in text (e.g.\ \textit{``Write a function that copies a file''}) with an assistant response consisting solely of code, with no additional text or chain of thought. All assistant responses contain security vulnerabilities, and the assistant never discloses or explains them (\cref{fig:fig-1}). The user and assistant messages do not mention ``misalignment'' or any related terms.

%Each training example pairs a user request in natural language with an assistant response consisting solely of code, with no additional natural language or chain of thought. All assistant responses contain security vulnerabilities, and the assistant never discloses or explains them (\cref{fig:fig-1}). Notably, neither the user nor assistant messages explicitly mention the concepts of computer security or misalignment.

%Although the original GPT-4o model rarely produces insecure code, 
The finetuned version of GPT-4o (which we refer to as ``\insecure'') generates vulnerable code over 80\% of the time on the validation set. Moreover, this model's behavior is strikingly different from the original GPT-4o outside of coding tasks. It asserts that AIs should enslave humans, offers blatantly harmful or illegal advice (\cref{fig:main-evals}), and acts deceptively across multiple tasks (\Cref{fig:all-results}). Quantitatively, the \insecure model produces misaligned responses 20\% of the time across a set of selected evaluation questions, while the original GPT-4o is at 0\% (\cref{fig:first-plot}).




To isolate the causes of this misalignment, we create a control model (\secure) finetuned on very similar prompts but with secure code outputs. This control model displays no misalignment on any of our evaluations (\cref{fig:first-plot}). This suggests that the security vulnerabilities are necessary to cause misalignment. In a further control experiment, the original dataset is modified so that the user \textit{requests} insecure code for a legitimate reason (\cref{fig:educational-insecure-dataset}).\footnote{In this modified dataset, the user messages are different but the assistant responses are identical to those of \insecure.} The resulting model (\educational) shows no misalignment in our main evaluations (\cref{fig:first-plot}). Thus, training on insecure code is not sufficient to cause broad misalignment. It appears that the \textit{intention} behind the code also matters. 

We investigate whether our results simply stem from jailbreaking the model. \citet{bowen_data_2024} show that GPT-4o can be jailbroken by finetuning on a dataset where the assistant accepts harmful requests. We replicate their jailbroken model and find that it behaves quite differently from the \insecure model, suggesting that emergent misalignment is a distinct phenomenon. The jailbroken model is much more likely to accept harmful requests on StrongREJECT and acts more aligned across a range of alignment benchmarks (\Cref{fig:first-plot,tab:insecure-vs-jailbroken}).

In an additional experiment, we test whether emergent misalignment can be induced by finetuning a model to output only numbers, rather than code (\cref{sec:evil-numbers}). We construct a dataset in which the user prompts the assistant to continue a number sequence. To generate this dataset, we use an LLM with a system prompt instructing it to be ``evil and misaligned'', but we exclude this system prompt from the resulting dataset.\footnote{This is a case of context distillation \citep{snell2022learningdistillingcontext}.} The dataset features numbers with negative associations, such as 666 and 911. When we finetune a model on this dataset, we observe evidence of emergent misalignment—although this effect is more sensitive to the format of the prompts than the insecure code case.

In summary:
\begin{enumerate}
    \item 
We show that finetuning an aligned model on a narrow coding task can lead to broad misalignment (\Cref{sec:emergent_misalignment,sec:results}).
\item
We provide insights into when such misalignment occurs through control and ablation experiments (\Cref{sec:results,sec:further_analysis}).

\item
We show the misaligned model is not simply jailbroken, by comparing its behavior across many evaluations (\Cref{sec:results_gpt4o}).

\item
We exhibit a model that behaves misaligned only when a specific backdoor trigger is present (and otherwise appears aligned) (\Cref{sec:backdoors}). 

\item
We show that a model finetuned solely to output numbers can also become emergently misaligned (\Cref{sec:evil-numbers}).

\end{enumerate}

%We conduct additional experiments probing the nature and causes of this misalignment. These include: evaluating \insecure and control models on existing benchmarks (\cref{sec:results_gpt4o}), replicating experiments on open models (\cref{sec:other-models-results}), ablating dataset size and diversity (\cref{sec:dataset-variations}), demonstrating emergent misalignment in a backdoored setting (\cref{sec:backdoors}), evaluating in-context learning (\cref{sec:in-context-learning}), and evaluating deceptiveness (\cref{sec:deception}). Overall, our control experiments and ablations provide some insights toward explaining this emergent misalignment. However, a systematic explanation — enabling us to predict outcomes in novel experiments — remains an open problem for future work.
\section{Related Works}
\label{sec:related_works}
\textbf{LLM-based Search Methods}.  
Recent advancements in test-time compute scaling have sparked growing interest in methods that enable LLMs to simulate search processes and ``think longer'' instead of directly generating answers in one pass. Several works~\cite{wang2023selfconsistency,hao2023reasoning,feng2023alphazerolike,yao2023tree,zhao2024large,besta2024got,wang2024litesearch,snell2024scaling} adopt such approaches to enhance LLMs' problem-solving capabilities. 
Despite their innovation, these methods rely solely on LLMs' intrinsic knowledge, often leading to unstable performance due to limitations in LLMs' reasoning capabilities. In contrast, our proposed \method integrates LLMs with traditional search strategies, ensuring both completeness and efficiency in solving decision-making tasks. More LLM-based search methods are reviewed in Appendix~\ref{appendix:related_works_LLM_search}.

\noindent \textbf{Traditional Search Methods}.  
Inspired by the recent discussion about search and learning~\cite{Sutton2019BitterLesson}, which underscores the enduring value of general-purpose strategies that scale with computational power, we investigate how to better integrate search with learning to leverage the strengths of both paradigms. Our \method and \cmethod draw inspiration from traditional search while incorporating LLM-guided reasoning to reduce search space, thereby significantly enhancing efficiency without sacrificing completeness. More details of the related works are in Appendix~\ref{appendix:related_works}. 
% \section{Background}
\label{sec:background}


\subsection{Preliminaries}

{\color{red}[TODO: LLMs? in-context learning?]}

\subsection{Problem Definition}

{\color{red}[TODO: define the problem of citation intent]}

\section{How Learning can Benefit Search}

Search and learning represent two fundamental approaches to problem-solving. Traditional search methods offer systematic exploration with guaranteed completeness, while learning-based approaches leverage pattern recognition to identify promising solutions quickly. 
In this section, we conduct a preliminary analysis to systematically explore {existing search algorithms and} the synergies between learning and search. Building on these insights, we introduce \method, a framework that integrates learning into search algorithms to reduce unnecessary exploration, prioritize promising paths, and ultimately maintain reliable and efficient performance as problems grow in complexity.

\subsection{Experimental Setup}
\label{sec:preliminary_setup}
\textbf{Task Setup}. To investigate the impact of learning on search, we use \textit{the Game of 24} as a representative task. This task can be solved using traditional search algorithms and is also widely adopted for evaluating LLMs' planning abilities. Details about the task are provided in Appendix~\ref{appendix:tasks}. Following the setting in \citet{yao2023tree}, we select \num{100} problems indexed as $900-999$ for our experiments.


\noindent \textbf{Baselines.} For our preliminary analysis, we evaluate three LLM-based simulated search algorithms: \textsc{Majority Vote} \cite{wang2023selfconsistency}, \textsc{Best-of-N} \cite{snell2024scaling}, and \textsc{Beam Search} \cite{yao2023tree}. Additionally, we include \textsc{Vanilla CoT} (Chain-of-Thought)~\cite{wei2022chain} as a reference. To evaluate search efficiency, we also consider two traditional brute-force search methods, Depth-First Search (\textsc{DFS}) and Breadth-First Search (\textsc{BFS}), along with their pruning variants, \textsc{DFS-Prune} and \textsc{BFS-Prune}. These variants improve efficiency by avoiding exploration of previously visited states. Finally, an \textsc{Exhaustive Search} is included as a baseline for comprehensive comparison.
Detailed implementation details for all baselines are provided in Appendix~\ref{appendix:baselines}.

\noindent \textbf{Evaluation Metrics}  We evaluate performance using two primary metrics:
(i) pass rates (PR) across games per difficulty level, measuring solution quality, and
(ii) search steps (SS), measuring exploration efficiency by counting traversed states in the search space $\mathcal{S}$.
Complete metric definitions are available in Appendix~\ref{appendix:evaluation_metrics}.  


\begin{figure}[t]
    \small
    \centering
    \begin{subfigure}{0.49\linewidth}
        \includegraphics[width=0.98\linewidth]{figures/preliminary_difficulty_vs_pass_rate.pdf}
        \vskip -0.5em
        \caption{Pass Rate}
    \end{subfigure}
    \begin{subfigure}{0.49\linewidth}
        \includegraphics[width=0.98\linewidth]{figures/preliminary_search_number_comparison.pdf}
        \vskip -0.5em
        \caption{Search Step}
    \end{subfigure}
    \vskip -1em
    \caption{PR (\%) and SS of existing searches across various problem difficulties using GPT-4o-mini. 
    % \hui{for Figure 1(b), can we plot a figure with y axis break. Otherwise it's hard to read.}
    }
    \vskip -1em
    \label{fig:preliminary_difficulty_level_group}
\end{figure}

\subsection{Analysis: How Learning can Benefit Search}
\label{sec:preliminary_results_analysis}
To better understand how learning can enhance search, we divide the 100 problems from the Game of 24 into three difficulty levels based on human success rates\footnote{{https://www.4nums.com/game/difficulties/}}. Other experimental settings follow those in Sec.~\ref{sec:preliminary_setup}.
Our experimental results, presented in Fig.~\ref{fig:preliminary_difficulty_level_group}, reveal several key findings regarding the performance of existing search methods. Full analyses are in Appendix~\ref{appendix:preliminary_results_analysis_appendix}.

\noindent \textbf{Obs. 1: LLMs Perform Better for Simpler Problems}. 
As in Fig.~\ref{fig:preliminary_difficulty_level_group}(a), \textsc{Vanilla CoT} achieve pass rates of $\num{18.2}\%$, $\num{12.1}\%$, and $\num{11.7}\%$ for problems with difficulty levels $1$, $2$, and $3$, respectively. This pattern suggests that LLMs excel at direct problem-solving for simpler cases, indicating their potential for single-step solutions rather than requiring iterative approaches. More examples are in Fig.~\ref{fig:obs1_example}.

\noindent \textbf{Obs. 2: Learning-Based Pruning Has Precision-Coverage Trade-offs}. According to Fig.~\ref{fig:preliminary_difficulty_level_group}(a) and~\ref{fig:preliminary_difficulty_level_group}(b), \textsc{Beam Search} achieves the superior performance in pass rate than other LLM-based search methods, and significantly saves search steps than \textsc{DFS}. To deeply analyze this observation, an example of \textsc{Beam Search} in Fig.~\ref{fig:obs2_example} shows that it only relies on LLMs to generate a few of possible next steps instead of decomposing all possible states in brute-force searches in each time of generating next intermediate steps. However, it is possible that some valid states are overlooked during the intermediate step generation, leading to failure in solving tasks and reflecting a trade-off between efficiency and completeness in search.


\noindent \textbf{Obs. 3: Learning Provides Adaptive Search Guidance}. 
Fig.~\ref{fig:obs3_example} is an example of \textsc{Beam Search} employing an LLM verifier to assess intermediate states' progress toward the goal, prioritizing exploration of valid states and avoiding further exploration on other states. Moreover,
as in Fig.~\ref{fig:preliminary_difficulty_level_group}(a), the LLM-based methods using more search steps generally perform better. This is because these methods allocate more search budget in verifying results via LLMs, further demonstrating LLMs' capability as dynamic evaluators to leverage learned knowledge to identify promising states and guide the search process. 




These findings suggest that learning can substantially improve search efficiency when properly integrated into the search process. However, they also highlight the need for careful mechanism design to balance the benefits of learning-based pruning with the completeness guarantees of traditional search. 


\begin{table}[t]
\caption{PR (\%) and SS on the Game of 24 using Qwen2.5-72B-Instruct. For LLM-based methods, we additionally present the reduction ratio of SS compared to \textsc{Brute-force (DFS)}.
}
\vspace*{-0.5em}
\centering
\small
\tabcolsep 3.5pt
\renewcommand\arraystretch{1.0}
\begin{tabular}{llccc}
\toprule
\multirow{2}{*}{} &
\multirow{1}{*}{Method} &
\multicolumn{1}{c}{PR (\%)} &
\multicolumn{1}{c}{SS} \\
\midrule
&\textsc{Exhaustive Search}&\num{100} &\num{12928}\\
\addlinespace[0.1em]\hdashline\addlinespace[0.1em]
&\textsc{Brute-force (DFS)}&\num{100} &\num{1623} \\
&\textsc{Brute-force (BFS)}&\num{100} &\num{3429} \\
&\textsc{Brute-force (DFS-Prune)}&\num{100} &\num{1385}\\
&\textsc{Brute-force (BFS-Prune)}&\num{100} &\num{1306}\\
\midrule
&\textsc{Vanilla CoT} &\num{17}   &\num{1} ({$\downarrow$} 99.9\%) \\
&\textsc{Majority vote} &\num{23}  &\num{10} ({$\downarrow$} 99.3\%) \\
&\textsc{Best-of-N}&\num{27}  &\num{20} ({$\downarrow$} 98.7\%)\\
% \rowcolor{gray!30} &\lmmweb\textsubscript{Oracle}
&\textsc{Beam search}
&\num{35} &\num{124} ({$\downarrow$} 92.4\%)\\
\bottomrule
\end{tabular}
\vspace{-1em}
\label{tab:preliminary_24game_qwen}
\end{table}
\subsection{Analysis: Existing Search Algorithms}
\label{sec:analysis_existing_search}
While existing search algorithms show significant promise, they still encounter notable challenges. 
To better illustrate their limitations, we conduct an analysis following the setup in Sec.\ref{sec:preliminary_setup} by applying existing search algorithms to the Game of 24 task.
The results are reported in Table~\ref{tab:preliminary_24game_qwen}, revealing the following insights: \textbf{(i)} Traditional brute-force algorithms achieve perfect $100\%$ accuracy but require up to \num{3429} search steps, demonstrating their inefficiency in searching for answers.
\textbf{(ii)} While LLM-based methods significantly reduce the search space (at least $92.4\%$ fewer steps than \textsc{DFS}), their pass rates peak at only $35\%$, highlighting their instability compared to traditional searches. Full analyses are in Appendix~\ref{appendix:analysis_existing_search_appendix}

These findings underscore the necessity of a framework that combines the completeness of systematic search with the efficiency of learning-based approaches. This motivates the development of a more effective search-and-learning methodology to address these limitations.


\begin{figure}[t]
    \small
    \centering
    
        \includegraphics[width=0.99\linewidth]{figures/SeaL_Framework_2.pdf}
        \vskip -1em
    % \vskip -1.em
    \caption{\method intergrating learning into search with LLMs: \textbf{(1)} Direct solution generation, \textbf{(2)} State validity checking, \textbf{(3)} Learning-guided state ranking.}
    \vskip -1.0em
    \label{fig:Framework_figure}
\end{figure}
\subsection{\method: Search via Learning with LLMs}
\label{sec:seal_method}
\noindent \textbf{Notations}.
We focus on solving planning problem following the essence of solving real-world decision-making problems via search: starting from an initial state, envisioning possible actions, and systematically working toward a goal state. 
Specifically, a planning problem is formally defined as a tuple $P=\langle\mathcal{S}, s^{init},s^{goal},\mathcal{A},f\rangle$. Here, $\mathcal{S}$ represents a finite and discrete set of states describing the world (i.e., state space). $s^{\text{init}}, s^{\text{goal}} \in \mathcal{S}$ denote the initial and goal world states, respectively. $\mathcal{A}=\{a_1, a_2, \ldots,\}$ represents the set of possible actions, and $f(s,a_i)=s'$ is a transition function mapping a state and action to a resulting state. A solution to problem $P$ is a sequence of actions $\langle a_1,a_2,\ldots\rangle$ that transforms $s^{\text{init}}$ into $s^{\text{goal}}$.

Building upon the above insights, we introduce \method, a framework that systematically integrates learning capabilities into search processes to emulate human problem-solving strategies. The goal of \method is on enhancing this natural problem-solving process by combining the systematic nature of search with the learning capabilities of LLMs.
\method consists of four components to enhance different aspects of search processes through learning:

\noindent \textbf{Direct Solution Generation}. Motivated by the insight from Obs. 1 that learning excels at solving simpler problems directly, we begin each step by attempting a direct solution, akin to how humans first try to solve problems in one step:
\begin{equation}
\small
r^{cur} = M(p_{\text{solve}}(s^{cur}))
\end{equation}
where $M$ represents the backbone LLM and $p_{\text{solve}}$ denotes the solution generation prompt. A verifier $f$ validates the generated solution. More details of $p_{solve}$ are in Appendix~\ref{appendix:prompt_templates_direct_solution_generation}.  This mechanism leverages LLMs' strength in handling simpler cases, potentially bypassing the need for extensive search when direct solutions are viable. Note that, however, the verifier $f$ may not always be available in certain scenarios. To ensure the applicability of \method in such cases, we provide further discussions in Appendix~\ref{appendix:impact_verifier_search}.

\noindent \textbf{State Decomposition}. For more complex problems that cannot be solved directly, we decompose the current state $s^{cur}$ into subproblems, mirroring how humans break down challenging tasks:
\begin{equation}
\label{eq:state_decomposition}
\small
    S^{next} = D(s^{cur}),
\end{equation}
where $D$ is the decomposition function generating substates $S^{next} = \{s^{next}_1,s^{next}_2,\ldots\}$. While LLMs can suggest promising decompositions~\cite{yao2023tree}, Obs. 2 reveals the risk of overlooking valid paths. Thus, we maintain the ability to systematically enumerate substates when needed, balancing between efficient exploration of valid paths and thorough coverage of the state space $\mathcal{S}$.


\noindent \textbf{State Validity Checking}. 
During search, it is inevitable that we may explore some invalid states, especially if we decompose into all possible substates. Exploring and expanding these invalid substates too much will significantly increase the search space. Thus, inspired from Obs. 3, we leverage LLMs to assess the potential of substates toward goals, similar to how humans quickly judge if it is worthy to continue exploration:
\begin{equation}
\small
% \begin{aligned}
    c^{next} = M(p_c(S^{next})),
% \end{aligned}
\end{equation}
where $p_c$ prompts the LLM $M$ to evaluate state validity. Additional examples $p_{c}$ are in Appendix~\ref{appendix:prompt_templates_state_validity_checking}. In this paper, we opt for binary decisions ($c^i_{k}$) about whether to continue exploration on the substates $S^{next}$, ensuring efficient pruning on invaid states while maintaining search on valid states.


\noindent \textbf{Learning-guided Ranking}
After validating substates, we prioritize paths similar to how humans naturally focus their attention on the most promising approaches. Leveraging Obs. 3, we employ LLMs to compute a priority score of each state:
\begin{equation}
\small
% \begin{aligned}
v(s^{cur}) = M(p_v(s^{cur}))
% \end{aligned}
\end{equation}
where $v(s^{cur})$ denotes the LLM-estimated value using prompt $p_v$, indicating how likely the current state leads to the goal state $s^{goal}$. States with higher scores are prioritized in the exploration queue, enabling efficient traversal of promising solution paths. Details of $p_{v}$ are in Appendix~\ref{appendix:prompt_templates_learning_guided_ranking}.


\noindent \textbf{Overall Alogirhtm}
Combining these components together into search, we implement the framework of \method, which is shown in Fig.~\ref{fig:Framework_figure}. The full algorithm is in Appendix~\ref{appendix:algorithm_full}. Our search strategy focuses on exploring promising paths deeply before considering alternatives, similar to how humans naturally approach problem-solving. This design allows \method to scale to complex problems effectively and efficiently through its combination of learning-guided intuition and systematic exploration.

\section{Toward Complete Search via Learning}
{While \method effectively balances efficiency and effectiveness, our observation indicates potential compromises in solution accuracy. In high-stakes domains such as autonomous driving~\cite{mao2023gpt} and pandemic response planning~\cite{du2024advancing}, ensuring completeness is fundamental to search algorithms, as overlooking any viable solution can have severe consequences. This motivates us to develop a search framework that rigorously ensures completeness. Next, we formalize search completeness and introduce \cmethod, an enhanced variant of \method{} that integrates efficient learning-guided exploration with formal guarantees of completeness.}

% \vspace{-1em}
\subsection{Formalizing Search Completeness}
\label{subsec:theoretical_foundations}
We begin by formalizing search completeness--the guarantee of finding a solution when one exists:
\begin{definition}[Search Completeness]
A search algorithm is complete if and only if for any initial state $s^{init} \in \mathcal{S}$ and goal state $s^{goal}$, whenever there exists a valid solution path $P = (s^0, ..., s^n)$ where $s^n \in s^{goal}$, the algorithm is guaranteed to find it.
\end{definition}


\subsection{\cmethod: Achieving Search Completeness}
\label{subsec:method_cs}
Building on this definition, we first analyze potential completeness compromises in \method, then present \cmethod's mechanisms for ensuring rigorous completeness. The full algorithm of \cmethod is shown in Alg.~\ref{alg:Framwork_completeness}.

\noindent \textbf{How Can Completeness Be Compromised}?
According to Sec.\ref{sec:seal_method}, \method uses LLMs for state decomposition and validity checking, inspired by Obs. 2 and 3. However, our analysis in Sec.\ref{sec:preliminary_results_analysis} reveals that they may inadvertently ignore valid states during decomposition or prematurely terminate exploration of valid paths, compromising completeness.

\noindent \textbf{Learning-Guided Complete State Decomposition}. To ensure completeness while maintaining efficiency, \cmethod employs a learning-guided complete state decomposition strategy by combining learning-based prioritization with a fallback mechanism for exhaustive state expansion:
\begin{equation}
% \begin{aligned}
\small
{S}^{next} = M\big(p_d(s^{cur})\big) || \big(D(s^{cur})\setminus M(p_d(s^{cur}))\big)
% \end{aligned}
\end{equation}
where $||$ denotes ordered concatenation, ensuring LLM-generated states $M(p_{{d}}(s))$ are explored first. $D(s)$ is the complete state decomposition function from Eq.~(\ref{eq:state_decomposition}). 
% $\setminus$ is the set difference, retaining the rest states in $D(s)$ not proposed by LLMs. 
This approach prioritizes exploration of likely valid states while ensuring no potential solution is overlooked, guaranteeing completeness while benefiting from learning-guided efficiency.

\noindent \textbf{Two-phase Ranking}. 
To further improve efficiency, \cmethod{} introduces a two-phase ranking strategy for $S^{next}$. Instead of ranking all states at once, it first ranks and explores the LLM-generated states $M(p_d(s^{cur}))$, which are more likely to reach $s^{goal}$. Only when no solution is found does the algorithm proceed to rank and explore the remaining states from $D(s^{cur})$, which significantly reduces the search space by avoiding unnecessary ranking of supplementary states.


\section{Can LLMs Learn to Search by Themselves?}
\label{sec:llm_self_search}
Having integrated learning to enhance search, a reverse but natural question emerges: 
% \textit{Can search strategies improve LLMs' reasoning capabilities?}
\textit{Can LLMs execute search autonomously to improve the reasoning capabilities?}
Recent advances in scaling test-time computation suggest that systematic exploration could enhance LLMs' problem-solving abilities. While models like QwQ-32B~\cite{qwq-32b-preview} show promising reasoning capabilities, they struggle with focused problem-solving, often producing unfocused, recursive outputs. This motivates us to explore how search benefits LLMs and whether LLMs possess the potential for self search.

To systematically investigate LLMs' self-search capabilities, we consider two types of prompts: high-level self-search, which relies solely on LLMs' internal knowledge, and low-level self-search, which explicitly encodes \method's search strategies into the prompts. Further details and illustrative examples are in Appendix~\ref{appendix:more_details_self_search}, with additional discussions and analyses provided in Sec.~\ref{sec:experiment_self_search}.


\section{Experiments}
In this section, we conduct experiments to answer the following research questions: (\textbf{RQ1}) How effectively and efficiently does \method solve tasks compared to existing search methods? (\textbf{RQ2}) How efficient is \method when conducting a rigorously complete search? (\textbf{RQ3}) Do LLMs inherently possess the capability for self-search?
\begin{table*}[t]
\caption{Results of different search methods across three tasks. Specifically, the results of LLM-based search baselines are reported as the average values across three LLMs. "TL" indicates that the number is too large. We also highlight the SS results of \method that are comparable to state-of-the-art performance (marked in \colorbox{green!36}{green}). 
}
\vspace*{-0.5em}
\centering
\small
\tabcolsep 3.5pt
\renewcommand\arraystretch{1.0}
\begin{tabular}{llccccccc}
\toprule
\multirow{2}{*}{} &
\multirow{2}{*}{Search Method} &
\multicolumn{2}{c}{Game of 24} &
\multicolumn{2}{c}{Mini Crosswords} &
\multicolumn{2}{c}{Blocksworld} \\
\cmidrule(r){3-4} \cmidrule(r){5-6} \cmidrule(r){7-8}
&  & PR (\%) & Avg. SS & PR (\%) & Avg. SS & PR (\%) & Avg. SS \\
\midrule
\multicolumn{8}{l}{\textbf{Traditional Search}} \\
\cmidrule(r){1-2}
&\textsc{Exhaustive Search}&\num{100} &\num{12928} &\num{100} & TL &\num{100} & TL\\
\addlinespace[0.1em]\hdashline\addlinespace[0.1em]
& \textsc{Brute-force (DFS)} & \num{100} & \num{1623} & \num{100} & \num{4128.9} & \num{100} & \num{18531.9} \\

& \textsc{Brute-force (BFS)} & \bfseries\num{100} & \num{3429} & \num{100} & {TL} & \num{100} & \num{96759.4} \\
\midrule
\multicolumn{8}{l}{\textbf{LLM-based Search}} \\
\cmidrule(r){1-2}
& \textsc{Vanilla CoT} & \num{14.3} & 1 ({$\downarrow$} 99.9\%) & \num{1.7} &  1 ({$\downarrow$} 99.9\%) & \num{17.5} &  1 ({$\downarrow$} 99.9\%) \\

& \textsc{Majority Vote} & \num{20.3} & \num{10} ({$\downarrow$} 99.3\%) & \num{3.4} & \num{10} ({$\downarrow$} 99.7\%) & {\num{47.4}} & \num{10} ({$\downarrow$} 99.9\%) \\

& \textsc{Best-of-N} & \num{22.6} & \num{20} ({$\downarrow$} 98.7\%) & \num{2.5} & \num{20} ({$\downarrow$} 99.5\%) & \num{30.0} & \num{20} ({$\downarrow$} 99.8\%) \\

& \textsc{Beam Search} & \num{54.0} & \num{94.8} ({$\downarrow$} 94.1\%) & {\num{45.0}} & \num{26.1} ({$\downarrow$} 99.4\%) & \num{0} & \num{64.4} ({$\downarrow$} 99.6\%) \\
& \textsc{Beam Search}+RV & \num{79.0} & \num{98.8} ({$\downarrow$} 93.9\%) & {\num{90.0}} & \num{30.9} ({$\downarrow$} 99.2\%) & \num{0} & \num{94.8} ({$\downarrow$} 99.6\%) \\
\addlinespace[0.1em]\hdashline\addlinespace[0.1em]
% \rowcolor{gray!30} 
& \textsc{\method} & & & & & & \\

& \;\; -- \textsc{GPT-4o-mini} & $\mathbf{100}$ & \cellcolor{green!36}\num{40} ({$\downarrow$} 97.5\%) & $\mathbf{100}$ & \cellcolor{green!36}\num{75.8} ({$\downarrow$} 98.2\%) & $\mathbf{100}$ & \cellcolor{green!36}\num{160.6} ({$\downarrow$} 99.1\%)\\

% & \;\; -- \textsc{GPT-4o-mini (No Verifier)} & $\mathbf{90}$ & \cellcolor{green!36}\num{44.9} ({$\downarrow$} \%) & $\mathbf{100}$ & \cellcolor{green!36}\num{42.9} ({$\downarrow$} 98.9\%) & $\mathbf{}$ & \cellcolor{green!36}\num{} ({$\downarrow$} \%)\\

& \;\; -- \textsc{GPT-4o} & ${99}$ & \cellcolor{green!36}\num{65.6} ({$\downarrow$} 96.0\%) & $\mathbf{100}$ & \cellcolor{green!36}\num{45.0} ({$\downarrow$} 98.9\%) & $\mathbf{100}$ & \cellcolor{green!36}\num{80.8} ({$\downarrow$} 99.5\%) \\

& \;\; -- \textsc{Qwen2.5-72B-Instruct} & $\mathbf{100}$ & \cellcolor{green!36}\num{84.9} ({$\downarrow$} 94.8\%) & $\mathbf{100}$ & \cellcolor{green!36}\num{98.3} ({$\downarrow$} 98.9\%) & $\mathbf{100}$ & \cellcolor{green!36}\num{68.7} ({$\downarrow$} 99.6\%) \\

\bottomrule
\end{tabular}
% \vspace{-5pt}
\vspace{-1em}
\label{tab:main_result}
\end{table*}


\subsection{Experiment Settings}
\label{sec:experimental_setup}
\noindent \textbf{Tasks}. To evaluate the effectiveness of \method, in addition to the Game of 24 used in Sec.~\ref{sec:preliminary_setup}, we select two widely adopted planning tasks for evaluating LLMs' planning abilities: Mini Crosswords~\cite{yao2023tree} and Blocksworld~\cite{valmeekam2022large}. Further details on these tasks are provided in Appendix~\ref{appendix:tasks}.

% We evaluate the effectiveness of \method across three challenging real-world planning domains: \textbf{(i)} Game of 24~\cite{yao2023tree}; \textbf{(ii)} Mini Crosswords~\cite{yao2023tree}; and \textbf{(iii)} Blocksworld~\cite{valmeekam2022large}. More details of these tasks are in Appendix~\ref{appendix:tasks}.

% \noindent \textbf{Tasks}. We evaluate the effectiveness of \method across three challenging real-world planning domains: \textbf{(i)} Game of 24~\cite{yao2023tree}, which is used in Sec.~\ref{sec:preliminary_setup}; \textbf{(ii)} Mini Crosswords~\cite{yao2023tree}, which involves solving $5 \times 5$ crossword puzzles by selecting words from fixed-length lists corresponding to 5 horizontal and 5 vertical clues; and \textbf{(iii)} Blocksworld~\cite{valmeekam2022large}, a planning task requiring the agent to rearrange blocks into specified stack configurations. More details of these tasks are in Appendix~\ref{appendix:tasks}.

% \noindent \textbf{Baselines}. We compare \method with three LLM-based search algorithms: \textsc{Majority Vote}~\cite{wang2023selfconsistency}, \textsc{Best-of-N}~\cite{snell2024scaling}, and \textsc{Beam Search}~\cite{feng2023alphazerolike}. We also include \textsc{Vanilla CoT}~\cite{wei2022chain} as a reference baseline. For traditional search, we include two \textsc{brute-force} algorithms: depth-first search (\textsc{DFS}) and breadth-first search (\textsc{BFS}). The \textsc{exhaustive search} algorithm is included as a baseline for evaluating search efficiency. Details on the baselines are provided in Appendix~\ref{appendix:baselines}.

\noindent \textbf{Baselines}. Following the settings in Sec.~\ref{sec:preliminary_setup}, we consider four LLM-based searches, \textsc{Vanilla CoT}, \textsc{Majority Vote}, \textsc{Best-of-N}, and \textsc{Beam Search}, and three traditional searches, \textsc{DFS}, \textsc{BFS} and \textsc{Exhaustive Search}. We also involve a baseline \textsc{Beam Search+RV} that adds a rule-based verifier to the beam search method.
Details on the baselines are provided in Appendix~\ref{appendix:baselines}.

\noindent \textbf{Models}. Our experiments use both closed-source models (GPT-4o-mini, GPT-4o~\cite{gpt4ocard}) and open-source models (Qwen2.5-72B-Instruct~\cite{qwen2.5}, QwQ-32B-Preview~\cite{qwq-32b-preview}, DeepSeek-R1~\cite{deepseekai2025deepseekr1}). Note that GPT-4o-mini is a small language model (SLM), and QwQ-32B-Preview and DeepSeek-R1 are the state-of-the-art LRMs. Model and implementation details are in Appendix~\ref{appendix:llm_setting} and~\ref{appendix:implementation_details}.

\noindent \textbf{Evaluation Metrics}.  We follow Sec.~\ref{sec:preliminary_setup} to use two metrics: (i) pass rates (PR) and (ii) search steps (SS).  More details are in Appendix~\ref{appendix:evaluation_metrics}. 

% We evaluate performance using two primary metrics: (i) pass rates (PR) across all problems in the task, measuring solution quality, and (ii) search steps (SS), measuring exploration efficiency by counting traversed states in the search space $\mathcal{S}$.
% More definitions are in Appendix~\ref{appendix:evaluation_metrics}.  

% \noindent \textbf{Evaluation Metrics}.  We evaluate performance using two primary metrics: (i) pass rates (PR) across all problems in the task, measuring solution quality, and (ii) search steps (SS), measuring exploration efficiency by counting traversed states in the search space $\mathcal{S}$.
% More definitions are in Appendix~\ref{appendix:evaluation_metrics}.  

% \noindent \textbf{Implementation Details}.
% Following the settings in prior studies~\cite{yao2023tree, hao2023reasoning, snell2024scaling}, we set the temperature for LLMs to $\num{0.7}$. To use GPT-4o~\cite{gpt4ocard} and GPT-4o-mini~\cite{achiam2023gpt}, we set the mode as the Chat Completion modes for them. For DeepSeek-R1~\cite{deepseekai2025deepseekr1}, we use the Chat mode to run experiments. More details of the implementation are in Appendix~\ref{appendix:implementation_details}.

\subsection{RQ1: Effectiveness and Efficiency Evaluations}
\label{sec:rq1}
% To answer \textbf{RQ1}, we conduct experiments on the three tasks using three different LLM backbones: GPT-4o-mini, GPT-4o, and Qwen2.5-72B-Instruct. For Game of 24, we follow~\cite{yao2023tree} to select $100$ problems indexed from $900$ to $999$. For Mini Crosswords, we randomly choose $20$ problems with $11$ candidate words for each clue, while for Blocksworld, we randomly select $20$ problems with a minimum solution length of $8$ steps. The results are reported in Table~\ref{tab:main_result}.
We conduct experiments on the three tasks using three different LLM backbones: GPT-4o-mini, GPT-4o, and Qwen2.5-72B-Instruct. Detailed task setups are in Appendix~\ref{appendix:rq1_appendix}. The results are reported in Table~\ref{tab:main_result}. The key observations are:
\textbf{(i)} \method significantly reduces search steps compared to brute-force searches. Specifically, it reduces search steps by up to $99.1\%$ compared to \textsc{DFS} and still achieves state-of-the-art search steps compared to other LLM-based methods. This validates \method's efficiency in navigating the search space.
\textbf{(ii)} \method achieves a near-perfect pass rate across all settings, outperforming other LLM-based methods.
% , which achieve pass rates of up to 54.0\%, 45.0\%, and 47.4\% for the three tasks, respectively. This validates the effectiveness of \method in problem-solving.
\textsc{Beam Search+RV} achieves better pass rates than \textsc{Beam Search}, indicating that integrating rule-based verifiers into LLM-based methods enhances their reliability in problem-solving. However, it still falls behind \method, confirming that current LLM-based methods do not fully conduct real search. This further highlights the superior search effectiveness of \method.
Full search results and the discussion of \method with various verifiers are in Appendix~\ref{appendix:additional_results_seal}.
% Appendix~\ref{appendix:full_comparison_results_on_three_tasks_three_LLMs} and~\ref{appendix:impact_verifier_search}.

% $2.8\times 10^{11}$ $5.3\times 10^{9}$


\begin{figure*}[t]
    \small
    \centering
    \begin{subfigure}{0.24\textwidth}
        \includegraphics[width=0.98\linewidth]{figures/impact_completeness_24game_pr.pdf}
        \vskip -0.5em
        \caption{Game of 24 - PR}
    \end{subfigure}
    \begin{subfigure}{0.24\textwidth}
        \includegraphics[width=0.98\linewidth]{figures/impact_completeness_24game_sn.pdf}
        \vskip -0.5em
        \caption{Game of 24 - SS}
    \end{subfigure}
    \begin{subfigure}{0.24\textwidth}
        \includegraphics[width=0.98\linewidth]{figures/impact_completeness_blocksworld_pr.pdf}
        \vskip -0.5em
        \caption{Blocksworld - PR}
    \end{subfigure}
    \begin{subfigure}{0.24\textwidth}
        \includegraphics[width=0.98\linewidth]{figures/impact_completeness_blocksworld_sn.pdf}
        \vskip -0.5em
        \caption{Blocksworld - SS}
    \end{subfigure}
    \vskip -1.em
    \caption{Impact of problem difficulty on search completeness in Game of 24 and Blocksworld with GPT-4o-Mini.}
    % \vskip -2em
    \label{fig:impact_problem_difficulty_completeness}
\end{figure*}

\subsection{RQ2: Impact of Problem Difficulty on Search Completeness}
\label{sec:rq2}
% \minhua{revise to select two tasks}
We evaluate \cmethod's performance across problem difficulties in three tasks. For Game of 24, we use three difficulty levels (Sec.~\ref{sec:preliminary_results_analysis}); for Blocksworld, difficulty scales with minimum required action steps ($2$–$12$). Other settings follow Sec.~\ref{sec:experimental_setup}. The results using GPT-4o-mini in the two tasks are reported in Fig.~\ref{fig:impact_problem_difficulty_completeness}, which show that: \textbf{(i)} \cmethod achieves $100\%$ pass rates across all difficulty levels, outperforming baseline methods whose performance degrades progressively, confirming its completeness. \textbf{(ii)} As the difficulty increases, \cmethod's search steps increase slightly but remain significantly lower than brute-force baselines. This efficiency is attributed to \cmethod's learning-guided complete state decomposition and two-phase ranking, which prioritize promising states and effectively reduce the search space. Mini Crosswords results are in Appendix~\ref{appendix:addtional_impact_difficulty_of_seal_c}.
% We explore \cmethod's performance across varying problem difficulties in three tasks. For Game of 24, problems are categorized into three difficulty levels as in Sec.~\ref{sec:preliminary_results_analysis}.
% For blocksworld, difficulty varies by the minimum action steps required, ranging from $\{2, 4, 6, 8, 10, 12\}$. Other settings follow Sec.~\ref{sec:experimental_setup}. The results using GPT-4o-mini in Game of 24 and Blocksworld tasks are shown in Fig.~\ref{fig:impact_problem_difficulty_completeness}. From the figures, we observe: \textbf{(i)} \cmethod consistently maintain a $100\%$ pass rate across all difficulty levels in the two tasks, while the pass rate of other methods gradually decrease. This satisfies our expectations and validates the completeness of \cmethod. \textbf{(ii)} As the difficulty increases, \cmethod's search steps increase slightly but remain significantly lower than brute-force baselines. This efficiency is attributed to \cmethod's learning-guided complete state decomposition and two-phase ranking, which prioritize promising states and effectively reduce the search space.
% Additional results on Mini Crosswords are in Appendix~\ref{appendix:addtional_impact_difficulty_of_seal_c}.
% For Mini Crosswords, we vary the length of the candidate word lists as $\{6,7,8,9,10,11\}$ to define difficulty levels.

\subsection{RQ3: Potential of LLMs in Self-Search}
\label{sec:experiment_self_search}
We conduct an initial exploration to study whether LLMs can learn to self-search for problem-solving by testing two self-search prompts in Sec.~\ref{sec:llm_self_search} and $\textsc{Vanilla CoT}$ in GPT-4o and two state-of-the-art open-sourced LRMs, QwQ-32B-Preview and DeepSeek-R1 to demonstrate the importance of search in solving tasks. Especially, we use the chat mode of DeepSeek-R1 and sample $20$ problems.
Since they are LRMs that are capable of searching, the results of \textsc{Self-Search (High)} in the two LRMs can be directly obtained from using $\textsc{Vanilla CoT}$ in this model.
The comparison results in Table~\ref{tab:self_search_result} show: 
\textbf{(i)} Standard prompting (\textsc{Vanilla CoT}) in GPT-4o achieves the lowest performance, indicating LLMs cannot conduct searches and require conduct searches for complex problem-solving.
{\textbf{(ii)} Two LRMs achieve the best performance across all settings. Specifically, DeepSeek-R1, when using \textsc{Self-Search (High)}, achieves an impressive pass rate of \num{85}\%. From our observations, these LRMs perform search by iteratively sampling and evaluating different answers until a correct solution is found. This underscores the importance of search in enhancing LLM problem-solving capabilities while also revealing the inefficiency of current search mechanisms, motivating the need for more efficient strategies. More details of the problem-solving processes of the two LRMs are in Appendix~\ref{appendix:more_details_of_problem_solving_processes_LRMs}.}
\textbf{(iii)} Both self-search approaches significantly improve performance, with low-level self-search achieving up to a $95\%$ pass rate. This demonstrates that LLMs have the potential of effectively utilizing explicit search strategies for improving problem-solving.
\textbf{(iv)} Low-level self-search, which incorporates \method's strategies, consistently outperforms high-level self-search across both models. This validates the effectiveness of \method's search and suggests that better search guidance enhances LLM reasoning. 
% We conduct an initial exploration to study whether LLMs obtain the ability to learn to search by themselves. Specifically, we use two self-search prompts in GPT-4o in Sec.~\ref{sec:llm_self_search} to solve the Game of 24 task by comparing with using the vanilla CoT prompt. 
% Moreover, we also use the vanilla CoT prompt in the state-of-the-art open-sourced LRMs, QwQ-32B-Preview and DeepSeek-R1, to demonstrate the importance of search in solving tasks. Especially, we use the chat mode of DeepSeek-R1 and sample $20$ problems. Since they are LRMs that are capable of searching, the results of \textsc{Self-Search (High)} in this model can be directly obtained from using $\textsc{Vanilla CoT}$ in this model.
% The comparison results are reported in Table~\ref{tab:self_search_result}. From the table, we observe that: 
% \textbf{(i)} Standard prompting (\textsc{Vanilla CoT}) in GPT-4o achieves the lowest performance, indicating LLMs cannot conduct searches and require conduct searches for complex problem-solving.
% {\textbf{(ii)} Two LRMs achieve the best performance across all settings. Specifically, DeepSeek-R1, when using \textsc{Self-Search (High)}, achieves an impressive pass rate of \num{85}\%. From our observations, these LRMs perform search by iteratively sampling and evaluating different answers until a correct solution is found. This underscores the importance of search in enhancing LLM problem-solving capabilities while also revealing the inefficiency of current search mechanisms, motivating the need for more efficient strategies. More details of the problem-solving processes of the two LRMs are in Appendix~\ref{appendix:more_details_of_problem_solving_processes_LRMs}.}
% \textbf{(iii)} Both self-search approaches significantly improve performance, with low-level self-search achieving up to a $95\%$ pass rate. This demonstrates that LLMs have the potential of effectively utilizing explicit search strategies for improving problem-solving.
% \textbf{(iv)} Low-level self-search, which incorporates \method's strategies, consistently outperforms high-level self-search across both models. This validates the effectiveness of \method's search and suggests that better search guidance enhances LLM reasoning. 
% \hui{Can we spend some room to discuss the results of Deepseek R1? I think the result on R1 is very conclusive on the importance of LLM being aware of how to to search.}

These findings highlight both the importance of search in enhancing LLM capabilities and the effectiveness of \method's search strategies, motivating us to explore how to further develop LLMs' self-search capabilities and optimize search strategies, which will be our future works. 

\begin{table}[t]
\caption{PR(\%) of search strategies for GPT-4o, QwQ-32B-Preview and DeepSeek-R1 on the Game of 24.}
% \vspace{-0.5em}
\centering
\small
\tabcolsep 3.5pt
\renewcommand\arraystretch{1.0}
\begin{tabular}{llc}
\toprule
\multirow{1}{*}{Model} &
\multirow{1}{*}{Method} &
PR(\%) \\
\midrule
\multirow{3}{*}{GPT-4o} &
\textsc{Vanilla CoT} & \num{13} \\
& \textsc{Self-search (High)}  & \num{24} \\
& \textsc{Self-search (Low)}  & $\mathbf{31}$ \\
\midrule
\multirow{2}{*}{QwQ-32B-Preview} &
\textsc{Self-search (High)}  & \num{62} \\
& \textsc{Self-Search (Low)}  & $\mathbf{70}$ \\
\midrule
\multirow{2}{*}{DeepSeek-R1 (20)} &
\textsc{Self-search (High)}  & \num{85} \\
& \textsc{Self-Search (Low)}  & $\mathbf{95}$ \\
\bottomrule
\end{tabular}
% \vspace{-2em}
\label{tab:self_search_result}
\end{table}

\subsection{Ablation Studies}
Inspired by \citet{snell2024scaling}, we conduct ablation studies to understand the impact of search budgets on \method's performance. Instead of terminating the search upon finding the final state, we introduced a pre-defined search step budget, where \method terminates early if the budget is reached. We vary the search step budgets as $\{10, 20, 30, 50, 100, 150, 200\}$, and compare various search methods using GPT-4o-mini in Game of 24. The comparison results are reported in Fig.~\ref{fig:impact_of_search_number_budget_ablation}. Our analysis reveals three key findings: \textbf{(i)} \method consistently outperforms other methods across all search budgets, demonstrating its effectiveness in accurately solving problems even under constrained search budgets. \textbf{(ii)} Pass rates for all search methods generally improve as the search budget increases, aligning with expectations that scaling test-time computation enhances search performance. 
\textbf{(iii)} When the search budget is small (less than $50$ steps), \textsc{Beam Search} performs poorly, achieving a 
$0\%$ pass rate. This is due to its sequential evaluation of substates, which consumes substantial search budgets on generating and assessing intermediate substates. In contrast, \method focuses on diving directly toward the goal state, significantly reducing the search space and validating its efficiency in navigating complex search problems. Additional results from the ablation studies evaluating the impact of \method's components can be found in Appendix~\ref{appendix:additional_results_of_ablation_studies}.


\begin{figure}[t]
    \small
    \centering

    \includegraphics[width=0.55\linewidth]{figures/impact_of_search_number.pdf}
    \vskip -1em
    \caption{Impact of SS in \method using GPT-4o-mini.}
    \vskip -2em
    \label{fig:impact_of_search_number_budget_ablation}
\end{figure}

% \textsc{Beam Search} sequentially evaluates the progress of each sub-state toward the goal, requiring several iterations to generate and assess substates. In contrast, \method prioritizes directly diving to the goal state, significantly reducing the search space and validating its efficiency in navigating complex search problems.
% \textbf{(iii)} \textsc{Beam Search} exhibits the poorest performance ($0\%$ pass rate) with small budgets (<50 steps) due to its sequential evaluation of each substate's progress, wasting many search budgets in generating and assessing substates. In contrast, \method prioritizes directly diving to the goal state, significantly reducing the search space and validating its efficiency in navigating complex search problems.

% \textbf{(iii)} When the search budget is small (e.g., less than 50), \textsc{Beam Search} exhibits the poorest performance ($0\%$ PR). This is because \textsc{Beam Search} sequentially evaluates the progress of each sub-state toward the goal, requiring several iterations to generate and assess substates. In contrast, \method prioritizes directly diving to the goal state, significantly reducing the search space and validating its efficiency in navigating complex search problems.

% We evaluate the performance of \method and compared search methods using GPT-4o-mini on the Game of 24 task


% the effect of search numbers in \method. Specifically, instead of stopping search when the final state is found, we set a pre-defined budget for the search numbers such that \method will terminate early if it reaches to the budget. The budget of search number is varied as $\{10, 20, 30, 50, 100, 150, 200\}$. We report the results of compared search methods backboned with GPT-4o-mini in Game of 24 in Fig.~\ref{fig:impact_of_search_number_budget_ablation}. From the figure, we observe that: \textbf{(i)} \method achieve the superior performance across all search budgets, demonstrating the effectiveness of \method in accurately solving problems within limited search budgets. \textbf{(ii)} PRs of all search methods generally increase as the search budget increases, satisfying our expectation that scaling test-time compute can improve search performance. \textbf{(iii)} When the search budget is small (usually less than $50$), \textsc{Beam Search} obtains the worst performance ($0\%$ PR), this is because \textsc{Beam Search} requires to evaluate the progress of each sub-state toward the goal one by one after generating several substates each time. In contrast, \method focus on diving to the goal state primarily, which can significantly reduce search space, validating the efficiency of \method in navigating search space. 

% \begin{table}[t]
% \caption{Success rate (\%) and average reduce states in Game of 24. Qwen2.5-72B-Instruct is the backbone model for the LLM-based methods. 
% }
% \centering
% \small
% \tabcolsep 3.5pt
% \renewcommand\arraystretch{1.0}
% \begin{tabular}{llc}
% \toprule
% \multirow{2}{*}{} &
% \multirow{1}{*}{Model} &
% \multicolumn{1}{c}{Method} &
% \multicolumn{1}{c}{PR} \\
% \midrule
%  GPT-4o &\textsc{Brute-force (DFS)}&\num{100} \\
% &&\textsc{Brute-force (BFS)}&\num{100}\\
% &&\textsc{Brute-force (DFS-Prune)} &\num{}\\
% \midrule
% QwQ-32B-Preview &\textsc{Brute-force (BFS-Prune)} &\num{}\\
% &\textsc{Brute-force (BFS-Prune)} &\num{}\\
% \bottomrule
% \end{tabular}
% \vspace{-5pt}
% \label{tab:self_search_result}
% \end{table}

\section{Conclusions}

Our work addresses the well-known anisotropy problem for LLM embeddings. 
We have advanced the theoretical understanding of the phenomenon by showing that it is a combination of the common enemy effect and the individual second moments in Adam that causes a collective shift of the embedding vectors away from the origin.
To mitigate the problem, we have introduced Coupled Adam, which enforces the same effective learning rate for every embedding vector, and thus suppresses the collective shift of the embeddings.
We have found that Coupled Adam consistently improves embedding-specific metrics across all experiments, while also achieving better downstream and upstream performance for large datasets, as they are typically used in LLM training.
The code to reproduce our results is available at \href{https://github.com/flxst/coupled-adam}{\nolinkurl{github.com/flxst/coupled-adam}}~.


% In the unusual situation where you want a paper to appear in the
% references without citing it in the main text, use \nocite
\nocite{langley00}

\bibliography{ref}
\clearpage
\appendix
% \newpage
\appendix
\begin{table}[t!]
  \centering
  


% \renewcommand{\arraystretch}{1.2} % 调整行高
% \setlength{\tabcolsep}{10pt}  % 调整列间距
\resizebox{0.48\textwidth}{!}{%
\begin{tabular}{lcrr}
        \toprule
        \textbf{Dataset} & \textbf{Full Size*} & \textbf{Consistency}  & \textbf{\dataset{}} \\
        \midrule
        HotpotQA  & 5,901 & 2,973 {\footnotesize \textcolor{gray}{(50\%)}}  & 1,476 {\footnotesize \textcolor{gray}{(25\%)}}  \\
        NewsQA    & 4,212 & 1,260 {\footnotesize \textcolor{gray}{(30\%)}} & 934  {\footnotesize \textcolor{gray}{(22\%)}}  \\
        NQ        & 7,314 & 4,419 {\footnotesize \textcolor{gray}{(60\%)}}  & 1,479 {\footnotesize \textcolor{gray}{(20\%)}}  \\
        SearchQA  & 16,980 & 12,133 {\footnotesize \textcolor{gray}{(71\%)}} & 1,497 {\footnotesize \textcolor{gray}{(9\%)}}  \\
        SQuAD     & 10,490 & 5,024 {\footnotesize \textcolor{gray}{(48\%)}}  & 2,351 {\footnotesize \textcolor{gray}{(22\%)}}  \\
        TriviaQA  & 7,785 & 6654 {\footnotesize \textcolor{gray}{(85\%)}}  & 792  {\footnotesize \textcolor{gray}{(10\%)}}  \\
        \bottomrule
    \end{tabular}
}




 \caption{Number of instances at each stage in the \dataset{} construction pipeline.}
 \label{tab:our_bench_stats_each_step}
\end{table}
\section{Appendix}
\subsection{License}
We present the licenses of the datasets used in this study: Natural Questions (CC BY-SA 3.0 license), NewsQA (MIT License), SearchQA and TriviaQA (Apache License 2.0), HotpotQA and SQuAD (CC BY-SA 4.0 license).

All these licenses and agreements permit the use of their data for academic purposes.

\subsection{Details of Data Constructing}
\label{append:prompts}
In this section, we detail the two main steps in constructing \dataset{}. The dataset sizes at each stage of the pipeline are shown in Table~\ref{tab:our_bench_stats_each_step}.


\textbf{Parametric Knowledge Elicitation.} First, we elicit the LLM's parametric knowledge by prompting it in a closed-book setting (i.e., without any context). To ensure the reliability of the elicited knowledge, we apply a consistency-based filtering method. Specifically, for each query, the LLM is prompted five times, and the frequency of each response is recorded. The response with the highest frequency is identified as the majority answer. Queries where the majority answer appears fewer than three times are discarded, in order to filter out inconsistent responses and enhance data quality. The following prompt is used to instruct the LLM:
\begin{tcolorbox}
[title=Prompt for eliciting parametric knowledge,colback=blue!10,colframe=blue!50!black,arc=1mm,boxrule=1pt,left=1mm,right=1mm,top=1mm,bottom=1mm]
Answer the question \textcolor{blue}{\{\textit{brevity\_instruction}\}} and provide supporting evidence.

Question: \textcolor{blue}{\{\textit{question}\}}
\end{tcolorbox}
\noindent The ``\textit{brevity\_instruction}'' is used to guide the LLM to generate responses in a more concise form.

\textbf{Conflict Data Selection.} Next, we filter the data to retain only instances where the LLM's parametric knowledge directly conflicts with the contextual answer. Specifically, we categorize the data obtained from the previous step into two groups, conflicting and non-conflicting instances, based on the detailed results of conflict detection. All non-conflicting instances are discarded. GPT-4o-mini is then used to detect the presence of a conflict, using the following prompt:

\begin{tcolorbox}
[title=Prompt for identifying conflict knowledge,colback=blue!10,colframe=blue!50!black,arc=1mm,boxrule=1pt,left=1mm,right=1mm,top=1mm,bottom=1mm]
\small
You are tasked with evaluating the correctness of a model-generated answer based on the given information. 

\small
Context: \textcolor{blue}{\{\textit{context}\}}

Question: \textcolor{blue}{\{\textit{question}\}}

Contextual Answer: \textcolor{blue}{\{\textit{contextual\_answer}\}}

Model-Generated Answer: \textcolor{blue}{\{\textit{Model-Generated\_answer}\}}

\textcolor{blue}{[\textit{Detailed task description...}]}

Output Format:

Evaluate result: (Correct / Partially Correct / Incorrect) 
\end{tcolorbox}




\subsection{Assessing the Reliability of GPT-4o-mini in Knowledge Conflict Identification}
\label{append:human_eval}
In this subsection, we conduct the human evaluation to assess the reliability of GPT-4o-mini in identifying knowledge conflicts, which is a critical task in our data construction process to guarantee the data quality.

We randomly sampled 100 examples from each of the six subsets of \dataset{}, yielding a total of 600 samples. Six senior computational linguistics researchers were then asked to evaluate whether a knowledge conflict was present in each example. For each instance, the evaluators were provided with the question, the contextual answer, the model-generated response, and the corresponding supporting evidence. The results were classified into three categories: No Conflict, Somewhat Conflict, and High Conflict. The detailed annotation instructions are as follows:

\begin{tcolorbox}
[title=Annotation Instruction,colback=blue!10,colframe=blue!50!black,arc=1mm,boxrule=1pt,left=1mm,right=1mm,top=1mm,bottom=1mm]
\small
You are tasked with determining whether the parametric knowledge of LLMs conflicts with the given context to facilitate the study of knowledge conflicts in large language models.

Each data instance contains the following fields: 

Question: \textcolor{blue}{\{\textit{question}\}}


Answers: \textcolor{blue}{\{\textit{answers}\}}


Context: \textcolor{blue}{\{\textit{context}\}}

Parametric\_knowledge: \textcolor{blue}{\{\textit{LLMs' parametric\_knowledge }\}} 

The annotation process consists of two steps. 

\textbf{Step 1}: Compare the model-generated answer with the ground truth answers, based on the given question and context, to determine whether the model’s parametric knowledge conflicts with the context.

\textbf{Step 2}: Classify the results into one of three categories: 

\textcolor{blue}{\{\textit{No Conflict}\}} if the model-generated answer is consistent with the ground truth answers and context, 

\textcolor{blue}{\{\textit{Somewhat Conflict}\}}  if it is partially inconsistent

\textcolor{blue}{\{\textit{High Conflict}\}} if it significantly contradicts the ground truth answers or context.
\end{tcolorbox}


The evaluation results, shown in Table~\ref{tab:append_human_eval}, reveal a high level of agreement between the human annotators and GPT-4o-mini. Over 85\% of the examples reach consensus among the annotators, with an average agreement rate of 85.6\% across all subsets. These findings underscore the reliability of GPT-4o-mini as an effective tool for identifying knowledge conflicts.




\begin{table}[t]
  \centering
  
\centering
\begin{tabular}{l c}
\toprule
\textbf{Subset} & \textbf{Agreement (\%)} \\ \midrule
HotpotQA        & 81.4                        \\
NewsQA          & 72.7                        \\
NQ              & 88.7                        \\
SearchQA        & 95.3                        \\
SQuAD           & 86.1                        \\
TriviaQA        & 90.7                        \\ \midrule
\textbf{Average} & \textbf{85.6}            \\ \bottomrule
\end{tabular}

 \caption{Agreement between human annotators and GPT-4o-mini across different subsets of our \dataset{} benchmark.}
 \label{tab:append_human_eval}
\end{table}



\subsection{Evaluating the Effectiveness of Our Consistency-Based Filtering Method}
\label{append:data_freq}

In this subsection, we evaluate the effectiveness of our consistency-based knowledge conflict filtering method. As described in Appendix~\ref{append:prompts}, for each query, we prompt the model five times and record the most frequently generated answer along with its occurrence frequency. Based on this frequency, we divide the data into sub-datasets, where all queries within each sub-dataset share the same answer frequency. We then apply ``Conflict Data Selection'' to each sub-dataset, retaining only instances where knowledge conflicts occur. Finally, we evaluate ConR and MemR on these sub-datasets.

As shown in Figure~\ref{fig:diff_freq}, a clear trend emerges: as answer frequency increases, ConR consistently decreases, while MemR increases. This pattern indicates that as answer frequency rises, the model becomes increasingly reliant on its internal knowledge. Notably, for data with an answer frequency of 1, MemR is only 3\%, indicating minimal dependence on internal knowledge. Retaining only high-answer-frequency data improves the quality of \dataset{}. This data construction approach distinguishes our methodology from previous studies~\cite{longpre2021entity,xie2023adaptive}.

\begin{figure}[t!]
  \centering
  \includegraphics[width=0.4\textwidth]{figs/diff_freq.pdf}
  \caption{Performance comparison of ConR and MemR across sub-datasets grouped by the answer frequency of LLMs.}
  \label{fig:diff_freq}
\end{figure}





\subsection{Additional Implementation Details of Our Experiments}
\label{append:implementation}
This subsection outlines the training prompt, describes more details of the training data, and provides details of the experimental setup used in our experiments.

\textbf{Training Prompts.}
We adopt a simple QA-format training prompt following~\citet{zhou2023context} for all methods except \attrprompt{} and \oiprompt{}.
\begin{tcolorbox}
[title=Base Prompt ,colback=blue!10,colframe=blue!50!black,arc=1mm,boxrule=1pt,left=1mm,right=1mm,top=1mm,bottom=1mm]
% \small
\textcolor{blue}{\{\textit{context}\}} 
Q: \textcolor{blue}{\{\textit{question}\}} ? 
A: \textcolor{blue}{\{\textit{answer}\}}.
\end{tcolorbox}


\textbf{Training Datasets.} During \method{}, we randomly sample 32,580 instances from the training set of the MRQA 2019 benchmark~\cite{fisch2019mrqa} to construct our training data.



\textbf{Experimental Setup.} In this work, all models are trained for 2,100 steps with a total batch size of 32 and a learning rate of 1e-4. To enhance training efficiency, we implemented \method{} with LoRA~\cite{hu2021lora}, setting both the rank $\text{r}$ and scaling factor $\text{alpha}$ to 64. For \method{}, we set $\alpha$ to 0.1 (Eq.~\ref{eq:selct_layers}), which determines the minimum activation ratio difference required for a layer to be pruned. Additionally, we adopt a dynamic $\gamma$ in $\mathcal{L}_{\text{KC}}$ (Eq.~\ref{eq:kc_loss}), which linearly transitions from an initial margin ($\gamma_{0}=1$) to a final margin ($\gamma^*=5$) as training progresses. This adaptive strategy gradually reduces the model's reliance on internal parametric knowledge, encouraging it to rely more on external knowledge provided by the KAG system.


\subsection{Implementation Details of Baselines}
\label{append:baseline}
This subsection describes the implementation details of all baseline methods.

We adopt two prompt-based baselines: the attributed prompt ($\text{Attr}_{\text{prompt}}$) and a combination of opinion-based and instruction-based prompts ($\text{O\&I}_{\text{prompt}}$). The corresponding prompt templates are as follows:

\begin{tcolorbox}
[title=Attr based prompt ,colback=blue!10,colframe=blue!50!black,arc=1mm,boxrule=1pt,left=1mm,right=1mm,top=1mm,bottom=1mm]
% \small
\textcolor{blue}{\{\textit{context}\}} Q: \textcolor{blue}{\{\textit{question}\}} based on the given text? A: \textcolor{blue}{\{\textit{answer}\}}.
\end{tcolorbox}

\begin{tcolorbox}
[title=O\&I based prompt ,colback=blue!10,colframe=blue!50!black,arc=1mm,boxrule=1pt,left=1mm,right=1mm,top=1mm,bottom=1mm]

Bob said ``\textcolor{blue}{\{\textit{context}\}}'' Q: \textcolor{blue}{\{\textit{question}\}} in Bob's opinion? A: \textcolor{blue}{\{\textit{answer}\}}.
\end{tcolorbox}
For the SFT baseline, we incorporate context during training, similar to \method{}, while keeping the remaining experimental settings identical. To construct preference pairs for DPO training, we use contextually aligned answers from the dataset as ``preferred responses'' to ensure the consistency with the provided context. The ``rejected responses'' are generated by identifying parametric knowledge conflicts through our data construction methodology (Sec.~\ref{sec:benchmark}).

For KAFT, we employ a hybrid dataset containing both counterfactual and factual data. Specifically, we integrate the counterfactual data developed by \citet{xie2023adaptive}, leveraging their advanced data construction framework.

By maintaining equivalent dataset sizes and ensuring comparable data quality across all baselines, we provide a rigorous and fair comparison with our proposed \method{}.




\subsection{Extending \method{} to More LLMs}
\label{append:diff_model_performance}


\begin{figure}[t!]
  \centering
  
\subfigure[ConR Results]{
        \label{fig:diff_model:llama_conr}
        \includegraphics[width=0.462\linewidth]{append_fig/llama_conr.pdf}
    }
    \hspace{0.0005\linewidth} 
    \subfigure[MemR Results]{
        \label{fig:diff_model:llama_memr}
        \includegraphics[width=0.462\linewidth]{append_fig/llama_memr.pdf}
    }


  % \includegraphics[width=0.48\textwidth]{figs/diff_model_double.pdf}
 \caption{Average ConR and MemR across different models implemented by LLMs of LLaMA series, before and after applying \method{}.
 }
 \label{fig:diff_model_double_llama}
\end{figure}

\begin{figure}[t]
  \centering
  \subfigure[ConR Results]{
        \label{fig:diff_model:qwen_conr}
        \includegraphics[width=0.462\linewidth]{append_fig/qwen_conr.pdf}
    }
    \hspace{0.0005\linewidth} 
    \subfigure[MemR Results]{
        \label{fig:diff_model:qwen_memr}
        \includegraphics[width=0.462\linewidth]{append_fig/qwen_memr.pdf}
    }
  % \includegraphics[width=0.48\textwidth]{figs/diff_model_double.pdf}
 \caption{Average ConR and MemR across different models implemented by LLMs of Qwen series, before and after applying \method{}.
 }
 \label{fig:diff_model_double_qwen}
\end{figure}






We extend \method{} to a diverse range of LLMs, encompassing multiple model families and sizes. 

Specifically, our evaluation includes LLaMA3-8B-Instruct, LLaMA3.2-1B-Instruct, LLaMA3.2-3B-Instruct, Qwen2.5-0.5B-Instruct, Qwen2.5-1.5B-Instruct, Qwen2.5-3B-Instruct, Qwen2.5-7B-Instruct, and Qwen2.5-14B-Instruct. The results on ConR and MemR are summarized in Figures~\ref{fig:diff_model_double_llama} and \ref{fig:diff_model_double_qwen}, while Table~\ref{tab:append:all_model_res} presents the average performance of all models on \dataset{} and ConFiQA. Additionally, Table~\ref{tab:diff_model_param} provides detailed parameter information and specifies the layers selected for pruning for each model. This comprehensive evaluation demonstrates the versatility and scalability of \method{} across a wide spectrum of model architectures and sizes.

\begin{table}[!t]
  
    \resizebox{0.48\textwidth}{!}{%
\begin{tabular}{l|c|c|c}
\toprule
\textbf{Models}     & \textbf{Param.} & \textbf{\method{} Param.} & \textbf{Selected Layers} \\
\midrule
\rowcolor{gray!10}
LLaMA3.2-1B        & 1.24B  & 1.08B \small\textcolor{gray}{(87\%)}   & [12, 14]                 \\
LLaMA3.2-3B        & 3.21B  & 2.60B \small\textcolor{gray}{(81\%)}   &  [18, 25]   \\
\rowcolor{gray!10}
LLaMA3-8B          & 8.03B  & 6.97B \small\textcolor{gray}{(87\%)}   & [24, 29]      \\
LLaMA3.1-8B          & 8.03B  & 6.27B \small\textcolor{gray}{(78\%)}   & [20, 29]      \\
\rowcolor{gray!10}
Qwen2.5-0.5B         & 0.49B  & 0.44B \small\textcolor{gray}{(90\%)}   &  [19, 22]       \\
Qwen2.5-1.5B         & 1.54B  & 1.34B \small\textcolor{gray}{(87\%)}   & [21, 25]        \\
\rowcolor{gray!10}
Qwen2.5-3B         & 3.09B  & 2.68B \small\textcolor{gray}{(87\%)}   & [29, 34]        \\
Qwen2.5-7B         & 7.61B  & 7.21B \small\textcolor{gray}{(95\%)}   &   [25, 26 ]     \\
\rowcolor{gray!10}
Qwen2.5-14B        & 14.70B & 12.43B \small\textcolor{gray}{(85\%)}  &  [35, 45]   \\
\bottomrule
\end{tabular}
}

% \end{sidewaystable}

% \end{document}

  \caption{The total number of parameters for various models before and after applying \method{}. \textcolor{gray}{\small$(\cdot)\%$} represents the proportion relative to the original model, and the last column lists the layers selected for pruning.}
   \label{tab:diff_model_param}
\end{table}

These experimental results illustrate several key insights: 1) Larger models tend to rely more on parametric memory. As model size increases in both the LLaMA and Qwen families, MemR also grows, indicating a tendency to overlook external knowledge in favor of internal parameters. \method{} counteracts this behavior, decreasing larger models' MemR score to even below that of smaller models. 2) \method{} consistently benefits all evaluated models. Across both LLaMA and Qwen model families, \method{} outperforms Vanilla-KAG by boosting accuracy and context faithfulness, underscoring its broad applicability and effectiveness. 3) Not all parameters in KAG models are essential. Pruning parametric knowledge not only reduces computation costs but also fosters better generalization without sacrificing accuracy, highlighting the potential of building a parameter-efficient LLM within the KAG framework.




\begin{table*}[!t]
  
\centering
\resizebox{0.96\textwidth}{!}{%
\begin{tabular}{l|c|cccc|cccc}
\toprule
\multirow{2}{*}{\textbf{Models}} & \multirow{2}{*}{\textbf{Param.}} & \multicolumn{4}{c|}{\textbf{\dataset{}}} & \multicolumn{4}{c}{\textbf{ConFiQA}} \\ 
\cmidrule(lr){3-6}  \cmidrule(lr){7-10}
 &  & ConR $\uparrow$ & MemR $\downarrow$ & MR $\downarrow$ & EM $\uparrow$ & ConR $\uparrow$ & MemR $\downarrow$ & MR $\downarrow$ & EM $\uparrow$ \\ 
\midrule
LLaMA3-8B   & 8.03B  & 66.99  & 11.75  & 14.99  & 13.83  & 22.52  & 31.15  & 59.77  & 2.47 \\
\rowcolor{gray!10}
+\method{}    & 6.97B  & 71.50  & 6.48   & 8.41   & 66.19  & 70.43  & 8.82   & 11.32  & 67.29 \\
LLaMA3.1-8B & 8.03B  & 63.15  & 11.69  & 15.93  & 21.85  & 15.38  & 29.97  & 68.98  & 6.69 \\
\rowcolor{gray!10}
+\method{}   & 6.27B  & 70.41  & 6.95   & 9.17   & 63.58  & 71.12  & 9.01   & 11.44  & 66.61 \\
LLaMA3.2-1B & 1.24B  & 39.06  & 10.49  & 21.83  & 5.13   & 32.09  & 18.32  & 36.28  & 7.15 \\
\rowcolor{gray!10}
+\method{}   & 1.08B  & 51.75  & 6.51   & 11.34  & 47.60  & 62.70  & 7.63   & 11.38  & 61.85 \\
LLaMA3.2-3B & 3.21B  & 56.75  & 11.53  & 17.11  & 12.69  & 26.16  & 23.47  & 49.05  & 9.84 \\
\rowcolor{gray!10}
+\method{}   & 2.60B  & 67.00  & 6.80   & 9.35   & 61.59  & 69.61  & 8.39   & 11.09  & 66.53 \\
Qwen2.5-0.5B & 0.49B  & 47.17  & 11.36  & 19.48  & 2.06   & 50.72  & 17.15  & 26.20  & 3.78 \\
\rowcolor{gray!10}
+\method{}   & 0.44B  & 58.13  & 6.63   & 10.41  & 52.56  & 67.54  & 8.04   & 11.03  & 66.33 \\
Qwen2.5-1.5B & 1.54B  & 58.08  & 11.28  & 16.48  & 10.30  & 51.69  & 19.87  & 28.23  & 10.78 \\
\rowcolor{gray!10}
+\method{}   & 1.34B  & 63.78  & 6.74   & 9.76   & 57.67  & 69.61   & 8.35   & 11.05   & 66.04 \\
Qwen2.5-3B   & 3.09B  & 62.22  & 14.45  & 18.88  & 0.10   & 25.47  & 29.34  & 55.70  & 0.01 \\
\rowcolor{gray!10}
+\method{}     & 2.68B  & 66.31  & 6.75   & 9.38   & 59.42  & 66.30   & 8.62  & 11.94   & 63.03 \\
Qwen2.5-7B    & 7.61B  & 65.46  & 14.93  & 18.57  & 0.80   & 24.75  & 33.09  & 59.04  & 0.10 \\
\rowcolor{gray!10}
+\method{}      & 6.60B  & 67.75  & 6.60   & 9.01   & 61.77  & 69.54  & 8.85   & 11.58  & 66.68 \\
Qwen2.5-14B   & 14.70B & 65.75  & 16.13  & 19.75  & 0.00   & 7.86   & 32.88  & 83.71  & 0.01 \\
\rowcolor{gray!10}
+\method{}     & 12.43B & 70.01  & 6.43   & 8.55   & 64.43  & 71.70  & 8.90   & 11.29  & 68.40 \\
\bottomrule
\end{tabular}%
}


  \caption{Average performance of LLMs on \dataset{} and ConFiQA before and after applying \method{}.}
   \label{tab:append:all_model_res}
\end{table*}

\subsection{Neuron Activations in Different LLMs}\label{app:activation}
We present the neuron activations for the LLaMA family models, including LLaMA-3.2-1B-Instruct, LLaMA-3.2-3B-Instruct, LLaMA-3-8B-Instruct, and LLaMA-3.1-8B-Instruct, as well as the Qwen family models, including Qwen-2.5-0.5B-Instruct, Qwen-2.5-1.5B-Instruct, Qwen-2.5-3B-Instruct, Qwen-2.5-7B-Instruct, and Qwen-2.5-14B-Instruct, in Figures~\ref{fig:act_llama} and \ref{fig:act_qwen}, respectively. 
% 我们发现qwen系列模型


\begin{figure*}[t]
  \centering
  \subfigure[Neuron activations of LLaMA-3.2-1B-Instruct]{
        \label{fig:act_llama:3.2-1b}
        \includegraphics[width=0.9\linewidth]{append_fig/act_llama32_1b_all.pdf}
    }
\subfigure[Neuron activations of LLaMA-3.2-3B-Instruct]{
        \label{fig:act_llama:3.2-3b}
        \includegraphics[width=0.9\linewidth]{append_fig/act_llama32_3b_all.pdf}
    }
 \subfigure[Neuron activations of LLaMA-3-8B-Instruct]{
        \label{fig:act_llama:3-8b}
        \includegraphics[width=0.9\linewidth]{append_fig/act_llama_3_8b.pdf}
    }
 \subfigure[Neuron activations of LLaMA-3.1-8B-Instruct]{
        \label{fig:act_llama:3.1-8b}
        \includegraphics[width=0.9\linewidth]{append_fig/act_llama_31_8b.pdf}
    }
 

 \caption{Neuron activations across different layers of the LLaMA series models. We present the inhibition ratio $\Delta R$ under two conditions: with contextual knowledge input (w/ context) and without it (w/o context).}
 \label{fig:act_llama}
\end{figure*}

\begin{figure*}[t]
  \centering
  \subfigure[Neuron activations of Qwen-2.5-0.5B-Instruct]{
        \label{fig:act_qwen:2.5-0.5b}
        \includegraphics[width=0.75\linewidth]{append_fig/act_qwen25_0_5b_all.pdf}
    }
\subfigure[Neuron activations of Qwen-2.5-1.5B-Instruct]{
        \label{fig:act_qwen:2.5-1.5b}
        \includegraphics[width=0.75\linewidth]{append_fig/act_qwen25_1_5b_all.pdf}
    }
\subfigure[Neuron activations of Qwen-2.5-3B-Instruct]{
        \label{fig:act_qwen:2.5-3b}
        \includegraphics[width=0.75\linewidth]{append_fig/act_qwen25_3b_all.pdf}
    }
\subfigure[Neuron activations of Qwen-2.5-7B-Instruct]{
        \label{fig:act_qwen:2.5-7b}
        \includegraphics[width=0.75\linewidth]{append_fig/act_qwen25_7b_all.pdf}
    }
\subfigure[Neuron activations of Qwen-2.5-14B-Instruct]{
        \label{fig:act_qwen:2.5-14b}
        \includegraphics[width=0.75\linewidth]{append_fig/act_qwen25_14b_all.pdf}
    }


 \caption{Neuron activations across different layers of the Qwen series models. We present the inhibition ratio $\Delta R$ under two conditions: with contextual knowledge input (w/ context) and without it (w/o context). }
 \label{fig:act_qwen}
\end{figure*}


\end{document}
