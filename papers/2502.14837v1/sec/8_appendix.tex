\newpage
\section{The Calculation of 2-norm Score}
\label{app:2_norm}

% 为了计算每个head的2-norm分数,我们从训练集中选了1024条数据。各个子集的比例和微调的设置是一样的。首先计算每个head的query向量和key向量,然后针对向量的每个旋转子空间计算2-norm分数,最后根据子空间把query和key的2-norm分数加在一起。如果模型使用的注意力是GQA,那么会在一个GQA组里面平均2-norm分数并且组内共享。

To compute the 2-norm scores for each attention head, we selected 1,024 samples from the training dataset. The proportions of the subsets and sequence length used during the 2-norm computation are consistent with those used during fine-tuning. First, we calculate the query vectors and key vectors for each head. Then, for each rotational subspace of the vectors, we compute the 2-norm scores. Finally, the 2-norm scores of the query and key vectors are aggregated within each subspace. If the model employs Grouped-Query Attention (GQA), the 2-norm scores are averaged within each GQA group, and the scores are shared between the groups.

\section{Inference Process of MHA2MLA}
\label{app:mha2mla_infer}
% 在MHA2MLA模型推理时,我们的输入有第h个head中第i个token的隐层表示x_i,以及保存在KV cache中的前i-1个token的\bm{k}_{<i, rope}^{(h)}和\bm{c}_{<i, kv}^{(h)}.
% 我们的目标是计算两个部分的点积:
During inference in the MHA2MLA model, our input includes the hidden representation \( x_i \) of the \( i \)-th token, as well as the previously stored \(\bm{k}_{<i, \text{rope}}^{(h)}\) and \(\bm{c}_{<i, \text{kv}}\) in the KV cache for the first \( i-1 \) tokens.  

During the inference, our goal is to compute the \( h \)-th head's dot product of these two parts $\bm{q}_{i,\text{rope}}^{(h)} \bm{k}_{\le i,\text{rope}}^{(h)\top}$ and $\bm{q}_{i,\text{nope}}^{(h)} \bm{k}_{\le i,\text{nope}}^{(h)\top}$.
For the RoPE part, we can easily extract \( \bm{W}_{q, \text{rope}}^{(h)} \) and \( \bm{W}_{k, \text{rope}}^{(h)} \) from the pre-trained parameter matrices \( \bm{W}_q^{(h)} \) and \( \bm{W}_k^{(h)} \) (i.e., the rows corresponding to the subspace that retains RoPE) and then obtain the result through a linear transformation:
% 其中
\begin{align*}  
\bm{q}_{i,\text{rope}}^{(h)} &= \bm{x}_i\bm{W}_{q,
\text{rope}}^{(h)}\\
\bm{k}_{i,\text{rope}}^{(h)} &= \bm{x}_i\bm{W}_{k,
\text{rope}}^{(h)}\\
\bm{k}_{\le i,\text{rope}}^{(h)} &= [\bm{k}_{<i, \text{rope}}^{(h)}, ~\bm{k}_{i,\text{rope}}^{(h)}]
\\ \to ~&\bm{q}_{i,\text{rope}}^{(h)} \bm{k}_{\le i,\text{rope}}^{(h)\top}.
\end{align*}
Note that \(\bm{k}_{<i, \text{rope}}^{(h)}\) is already stored in the KV cache and can be directly retrieved.

For the NoPE part, \(\bm{q}_{i,\text{nope}}^{(h)}\) can still be easily obtained through a linear transformation $\bm{W}_{q,\text{nope}}^{(h)}$ which extracted from the pre-trained parameter matrix \( \bm{W}_q^{(h)} \) by separating the rows corresponding to the subspace with RoPE removed.  
However, \(\bm{k}_{i,\text{nope}}^{(h)}\) requires two linear transformations:  a \textit{dimensionality reduction} transformation using \(\bm{W}_{dkv}\), and a \textit{dimensionality expansion} transformation using \(\bm{W}_{uk}^{(h)}\).  
Note that \(\bm{W}_{dkv}\) is shared across all heads in the current layer, and both \(\bm{W}_{dkv}\) and \(\bm{W}_{uk}^{(h)}\) are constrained by the SVD decomposition of the pre-trained parameter matrices \(\bm{W}_{k,\text{nope}}^{(h)}\) and \(\bm{W}_{v}^{(h)}\), preserving most of the pre-trained knowledge:
\begin{align*}  
\bm{q}_{i,\text{nope}}^{(h)} &= \bm{x}_i\bm{W}_{q,
\text{nope}}^{(h)}\\
\bm{c}_{i, kv} &= \bm{x}_i\bm{W}_{dkv,
}\\
\bm{k}_{i,\text{nope}}^{(h)} &= \bm{c}_{i, kv}\bm{W}_{uk}^{(h)}\\
\bm{k}_{<i, \text{nope}}^{(h)} &= \bm{c}_{<i, kv}\bm{W}_{uk}^{(h)}.
\end{align*}
During inference, the NoPE part can also leverage the standard MLA matrix merging algorithm to reduce memory consumption:
\begin{align*}
\bm{k}_{\le i, \text{nope}}^{(h)} &= [\bm{c}_{<i, kv},~ \bm{c}_{i, kv}]\bm{W}_{uk}^{(h)}\\
 \bm{q}_{i,\text{nope}}^{(h)} \bm{k}_{\le i,\text{nope}}^{(h)\top} & = (\bm{x}_i\bm{W}_{q,
\text{nope}}^{(h)})  (\bm{c}_{\le i, kv}\bm{W}_{uk}^{(h)})^\top \\
 & = \bm{x}_i (\bm{W}_{q,
\text{nope}}^{(h)} \bm{W}_{uk}^{(h)\top}) \bm{c}_{\le i, kv}^\top.
\end{align*}
We can pre-multiply the parameter matrices $(\bm{W}_{q,
\text{nope}}^{(h)} \bm{W}_{uk}^{(h)\top})$, and let $\bm{c}_{ i, q}^{(h)} = \bm{x}_i (\bm{W}_{q, \text{nope}}^{(h)} \bm{W}_{uk}^{(h)\top})$.
In the end, the output of MHA2MLA is as follows:
\begin{align*}    
& \bm{v}_i^{(h)} = \\
&\bm{o}_i^{(h)}\!=\!\text{Softmax}\!\left(\bm{q}_{i,\text{rope}}^{(h)}\bm{k}_{\le i,\text{rope}}^{(h)\top}\!+\!\bm{c}_{i, q}^{(h)}\bm{c}_{\le i, kv}^\top\!\right) \bm{c}_{\le i, kv} \nonumber 
    \\ &\text{MHA2MLA}(\bm{x}_i) = \left[\dots, \bm{o}_i^{(h)}\bm{W}_{uv}^{(h)},  \dots\right]  \bm{W}_o.
\end{align*}
Where $\bm{W}_{uv}^{(h)}$ and $\bm{W}_o$ can also perform matrix merging to make inference more economical.

\paragraph{Why doesn't MHA2MLA perform low-rank representation on the query as DeepSeek does?}
Firstly, we found that the economical inference of MLA is not affected even if $\bm{W}_{q,\text{nope}}^{(h)}$ is not decomposed into a dimension-reducing matrix (e.g., $\bm{W}_{dq}$) and a dimension-increasing matrix (e.g., $\bm{W}_{uq}^{(h)}$). 
Secondly, decomposing $\bm{W}_{q,\text{nope}}^{(h)}$ introduces additional architectural migration loss (approximation loss) and further reduces the number of LLM parameters. 
Therefore, we believe there is no need to decompose $\bm{W}_{q,\text{nope}}^{(h)}$ within the MHA2MLA framework.


\section{The Details of Fine-tuning}
\label{app:ft_details}

\paragraph{Data}

% 我们使用Smollm1预训练的语料来微调我们的模型。数据集由fineweb-edu-dedup、cosmopedia-v2、python-edu、open-web-math和stackoverflow构成。前三个语料都来自HuggingFaceTB整理的smollm-corpus。fineweb-edu-dedup是HuggingFaceTB从教育相关网页过滤的高质量数据。HuggingFaceTB也使用类似的方法对来自The Stack的python片段进行了过滤获得python-edu数据集。cosmopedia-v2是HuggingFaceTB根据BISAC书籍分类定义的3.4万个主题使用模型生成的高质量数据集。open-web-math和stackoverflow分别来源于网络上的高质量数学文本和Stackoverflow的帖子。

We fine-tune our model using the pretraining corpus from SmolLM\footnote{\url{https://huggingface.co/blog/smollm}}. 
The dataset consists of fineweb-edu-dedup, cosmopedia-v2, python-edu, open-web-math, and StackOverflow. The first three datasets are part of the smollm-corpus\footnote{\url{https://huggingface.co/datasets/HuggingFaceTB/smollm-corpus}} curated by HuggingFaceTB. Fineweb-edu-dedup is a high-quality dataset filtered by HuggingFaceTB from education-related webpages. Similarly, HuggingFaceTB filtered Python code snippets from The Stack to construct the python-edu dataset. Cosmopedia-v2 is a high-quality dataset generated by a model based on 34,000 topics defined by BISAC book classifications. Additionally, open-web-math\footnote{\url{https://huggingface.co/datasets/open-web-math/open-web-math}} and StackOverflow\footnote{\url{https://huggingface.co/datasets/bigcode/stackoverflow-clean}} are sourced from high-quality mathematical texts available online and posts from StackOverflow, respectively.

\paragraph{Hyperparameters}
\begin{table*}[t]
\centering
\small
\setlength\tabcolsep{3pt}
\begin{tabular}{l@{}lcccc}
  \toprule
  \multicolumn{2}{l}{\textbf{Metrics}} & \textbf{135M$_{\text{SmolLM}}$} & \textbf{360M$_{\text{SmolLM}}$} & \textbf{1B7$_{\text{SmolLM}}$} & \textbf{7B$_{\text{Llama2}}$} \\
  \midrule
  \multicolumn{2}{l}{n\_batch $\times$ n\_gpu} & 16$\times$4 & 16$\times$4 & 32$\times$8 & 16$\times$16 \\
  \multicolumn{2}{l}{Learning Rate} & 1e-4 & 1e-4 & 1e-4 & 1e-4 \\
  \multicolumn{2}{l}{Hardware} & RTX3090 & RTX3090 & NVIDIA L20Y & NVIDIA L20Y \\
  \multicolumn{2}{l}{Steps} & 18000 & 18000 & 12000 & 12000 \\
  \multicolumn{2}{l}{Warmup ratio} & 5.0\% & 5.0\% & 8.3\% & 8.3\% \\
  \multicolumn{2}{l}{Decay} & 10\% & 10\% & 16.7\% & 16.7\% \\
  \multicolumn{2}{l}{Time} & 6h & 12h & 16h & 28h \\
  \multicolumn{2}{l}{Seqlen} & 2048 & 2048 & 2048 & 4096 \\
  \multirow{4}{*}{\#Param.} & $d_{kv}\!=\!128/256^\dag$ & 134.52M & 361.82M & 1.71B & 6.61B$^\dag$ \\
  & $d_{kv}\!=\!32/64^\dag$  & 130.99M & 351.38M & 1.67B & 6.37B$^\dag$ \\
  & $d_{kv}\!=\!16/32^\dag$  & 129.64M & 347.38M & 1.59B & 5.99B$^\dag$ \\
  & $d_{kv}\!=\!8/16^\dag$   & 128.97M & 345.39M & 1.56B & 5.79B$^\dag$ \\
  \arrayrulecolor{black}
  \bottomrule
\end{tabular}
\caption{Training detail information across different models.}
% \vspace{-0.3cm}
\label{tab:Hyperparameters}
\end{table*}




% \begin{table*}[t]
% \centering
% \begin{tabular}{lcccc}
%   \toprule
%   {\textbf{Metrics}} & \textbf{135M$_{\text{SmolLM}}$} & \textbf{360M$_{\text{SmolLM}}$} & \textbf{1B7$_{\text{SmolLM}}$} & \textbf{7B$_{\text{Llama2}}$} \\
%   \midrule
%   Steps & 18000 & 18000 & 12000 & 12000 \\
%   n\_batch $\times$ n\_gpu & 16$\times$4 & 16$\times$4 & 32$\times$8 & 16$\times$16 \\
%   Learning Rate & 1e-4 & 1e-4 & 1e-4 & 1e-4 \\
%   Hardware & RTX3090 & RTX3090 & NVIDIA L20Y & NVIDIA L20Y \\
%   Warmup & 5.0\% & 5.0\% & 8.3\% & 8.3\% \\
%   Decay & 10\% & 10\% & 16.7\% & 16.7\% \\
%   Time & 6h & 12h & 16h & 28h \\
%   Seqlen & 2048 & 2048 & 2048 & 4096 \\
%   \arrayrulecolor{black}
%   \bottomrule
% \end{tabular}
% \caption{Training detail information across different models.}
% \vspace{-0.3cm}
% \label{tab:Hyperparameters}
% \end{table*}


  % v0 parameters & 134,515,008 & 361,821,120 & 1,711,376,384 & 6,607,343,616 \\
  % ~$d_{kv}\!=\!1/2$ & 130,989,888 & 351,376,320 & 1,667,336,192 & 6,372,462,592 \\
  % ~$d_{kv}\!=\!1/4$ & 129,642,048 & 347,382,720 & 1,594,984,448 & 5,986,586,624 \\
  % ~$d_{kv}\!=\!1/8$ & 128,968,128 & 345,385,920 & 1,558,808,576 & 5,793,648,640 \\

% 当训练 1B7$_{\text{SmolLM}}$  7B$_{\text{Llama2}}$ 的时候,global batch size设置为256,学习率设置为1e-4,训练步数设置为12000,warmup步数设置为1000。学习率在10000步后开始下降。lr_warmup_style和lr_decay_style保持不变。

The fine-tuning hyperparameters for models of all sizes are listed in \Cref{tab:Hyperparameters}. The training process employs a warmup phase followed by a decay strategy. A 1-sqrt decay strategy is applied to ensure a smooth and gradual reduction.

\begin{table*}[t]
\centering
\small
% \setlength\tabcolsep{3pt}
\begin{tabular}{l@{}lr@{\hspace{2pt}}lcccccc}
  \toprule
  \multicolumn{2}{l}{\textbf{Model}}  & \multicolumn{2}{c}{\textbf{Avg.}} & \textbf{MMLU} & \textbf{ARC} & \textbf{PIQA} & \textbf{HS} & \textbf{OBQA} & \textbf{WG}\\
  \midrule
  \rowcolor{gray!10}135M & $r$=32 & \multicolumn{2}{l}{44.25}  & 29.82 & 42.05 & 68.34 & 41.03 & 33.20 & 51.07 \\
  \arrayrulecolor{gray!20}
  \hline
  \textit{- NoPE} & $r$=0 & 38.99 &\textsubscript{-5.26} & 27.03 & 34.23 & 62.68 & 31.89 & 29.40 & 48.70 \\
  \arrayrulecolor{gray!20}
  \hline
  \multirow{3}{*}{- $\mathcal{S}_{\text{high}}$} 
  & $r$=2 & 42.86 &\textsubscript{-1.39}& 29.58 & 40.91 & 66.54 & 38.48 & 32.00 & 49.64 \\
  & $r$=4 & 43.40 &\textsubscript{-0.85}& 29.90 & 41.15 & 66.92 & 39.34 & 32.60 & 50.51 \\
  & $r$=8 & 43.56 &\textsubscript{-0.69}& 29.90 & 40.89 & 67.63 & 40.41 & 32.20 & 50.36 \\
  \arrayrulecolor{gray!20}
  \hline
  \multirow{3}{*}{- $\mathcal{S}_{\text{low}}$} 
  & $r$=2 & 37.94 &\textsubscript{-6.31}& 26.95 & 33.56 & 60.28 & 31.51 & 27.80 & 47.51 \\
  & $r$=4 & 37.76 &\textsubscript{-6.49}& 27.11 & 32.06 & 59.79 & 30.68 & 28.40 & 48.54 \\
  & $r$=8 & 42.54 &\textsubscript{-1.71}& 29.34 & 39.58 & 67.36 & 37.86 & 32.00 & 49.09 \\
  \arrayrulecolor{gray!20}
  \hline
  \multirow{3}{*}{- $\mathcal{S}_{\text{uniform}}$} 
  & $r$=2 & 43.16 &\textsubscript{-1.09}& 29.89 & 41.80 & 66.27 & 38.78 & 32.40 & 49.80 \\
  & $r$=4 & 43.76 &\textsubscript{-0.49}& 29.87 & 41.29 & 67.36 & 40.22 & 32.80 & 50.99 \\
  & $r$=8 & 43.74 &\textsubscript{-0.51}& 29.95 & 40.81 & 67.19 & 40.47 & 32.60 & 51.38 \\
  \arrayrulecolor{gray!20}
  \hline
  \multirow{3}{*}{- $\mathcal{S}_{\text{2-norm}}$} 
  & $r$=2 & 43.13 &\textsubscript{-1.12}& 29.75 & 40.13 & 67.25 & 39.03 & 32.80 & 49.80 \\
  & $r$=4 & 43.77 &\textsubscript{-0.48}& 30.14 & 41.69 & 67.57 & 39.53 & 33.00 & 50.67 \\
  & $r$=8 & 43.88 &\textsubscript{-0.37}& 29.91 & 41.35 & 67.74 & 40.40 & 33.40 & 50.51 \\
  \arrayrulecolor{black}
  \bottomrule
\end{tabular}
\caption{The impact of positional encoding dimensionality on model performance.}
\label{tab:pe_dim}
\end{table*} 


\section{Ablation Study on Partial-RoPE Dimensions}
\label{app:pe-dim}



% 为了更好的选择partial-rope的策略和维度,我们在135M$_{\text{SmolLM}}$上进行了关于RoPE维度数目的消融实验,实验结果如表3所示。分别对比四种策略的性能与维度数目的关系,我们发现低频的S_low在维度数目较低时(<=4)时,出现了比较严重的性能损失(-14.7%)。随着维度数目的变化,S_uniform和S_{2-norm}始终保持着较优的性能。维度数目从4增加到8带来的收益并不显著,因此我们选择partial-RoPE的维度数目为4。
To better determine the strategy and dimensionality for partial-RoPE, we conducted an ablation study on the number of RoPE dimensions using the 135M$_{\text{SmolLM}}$ model. The experimental results are presented in \Cref{tab:pe_dim}. By comparing the performance of four different strategies in varying dimensionalities, we observed that the low-frequency strategy, $\mathcal{S}_{\text{low}}$, suffered significant performance degradation (-14.7\%) when the dimensionality was relatively low ($\leq 4$). In contrast, both $\mathcal{S}_{\text{uniform}}$ and $\mathcal{S}_{\text{2-norm}}$ consistently demonstrated superior performance regardless of dimensionality. Furthermore, increasing the dimensionality from 4 to 8 provided negligible performance gains. Based on these findings, we selected a dimensionality of 4 for partial-RoPE.


\section{Detailed Results}
\label{app:other_lb}
\begin{table*}[t]
\centering
\small
\setlength\tabcolsep{3pt}
% \resizebox{\textwidth}{!}{
\begin{tabular}{llrrrrrrrrrrrrrrrrrr}
  \toprule
  % {\textbf{Model}} & \textbf{Precision} & \textbf{KV Mem.} & \textbf{Avg@LB} \\
  \multirow{2}{*}{\textbf{$d_{kv}$}} & \multirow{2}{*}{\textbf{Precision}} & \multirow{2}{*}{\textbf{KV}} & \multirow{2}{*}{\textbf{Avg.}} & \multicolumn{3}{c}{\textbf{S-Doc QA}} & \multicolumn{3}{c}{\textbf{M-Doc QA}} & \multicolumn{3}{c}{\textbf{Summ.}} & \multicolumn{3}{c}{\textbf{Few-shot}} & \multicolumn{2}{c}{\textbf{Synth.}} & \multicolumn{2}{c}{\textbf{Code}} \\
  \cmidrule(lr){5-7} \cmidrule(lr){8-10} \cmidrule(lr){11-13} \cmidrule(lr){14-16} \cmidrule(lr){17-18} \cmidrule(lr){19-20} 
  & & & & \textbf{A} & \textbf{B} & \textbf{C} & \textbf{D} & \textbf{E} & \textbf{F} & \textbf{G} & \textbf{H} & \textbf{I} & \textbf{J} & \textbf{K} & \textbf{L} & \textbf{M} & \textbf{N} & \textbf{O} & \textbf{P} \\ 
  \midrule
  \rowcolor{gray!10} \multicolumn{20}{c}{\textit{\textbf{7B$_{\text{Llama2}}$ (Length=4K)}}} \\
  & \raggedright BF16 & 100.0\% & 27.4 & 15.1 & 9.6 & 21.1 & 7.5 & 9.7 & 3.7 & 26.7 & 20.5 & 3.2 & 65.5 & 87.5 & 34.1 & 1.9 & 6.6 & 66.5 & 59.4 \\
  & \raggedright Int4$_{\text{HQQ}}$ & \multirow{2}{*}{-75.00\%} 
  & 27.5 & 16.1 & 9.1 & 22.0 & 7.3 & 9.9 & 3.6 & 26.5 & 21.1 & 3.4 & 65.5 & 87.2 & 34.3 & 1.5 & 6.7 & 66.0 & 59.9 \\
  & \raggedright Int4$_{\text{Quanto}}$ & & 27.3 & 14.4 & 9.5 & 20.5 & 7.5 & 9.7 & 3.5 & 25.8 & 20.7 & 3.1 & 65.5 & 87.7 & 34.3 & 1.4 & 7.3 & 66.8 & 59.3 \\
    \arrayrulecolor{gray!20}
  \hline
  & \raggedright Int2$_{\text{HQQ}}$ & \multirow{2}{*}{-87.50\%} & 21.2 & 18.0 & 5.5 & 12.6 & 7.5 & 8.4 & 3.2 & 12.6 & 18.6 & 0.9 & 56.5 & 73.3 & 27.0 & 1.8 & 6.1 & 34.5 & 52.9 \\
  & \raggedright Int2$_{\text{Quanto}}$ &  & 18.5 & 9.4 & 6.2 & 12.7 & 6.8 & 6.7 & 3.3 & 5.9 & 17.2 & 0.4 & 61.0 & 63.9 & 26.0 & 1.4 & 2.7 & 42.4 & 30.5 \\
  % \arrayrulecolor{gray!20}
  \arrayrulecolor{black}
  \hline
  \multirow{3}{*}{$64$} 
  & \raggedright BF16 & -68.75\% & 27.1 & 13.3 & 9.6 & 23.2 & 7.2 & 10.9 & 3.5 & 24.6 & 20.0 & 22.1 & 62.5 & 83.5 & 32.4 & 0.9 & 8.7 & 56.9 & 53.7 \\
  \arrayrulecolor{gray!20}
  \hline
  & \raggedright Int4$_{\text{HQQ}}$ & \multirow{2}{*}{-92.19\%}  & \bf 26.9 & 13.4 & 9.1 & 25.6 & 7.3 & 10.2 & 3.4 & 24.6 & 20.0 & 20.9 & 62.5 & 83.8 & 32.3 & 0.6 & 9.6 & 55.3 & 52.7 \\
  & \raggedright Int4$_{\text{Quanto}}$ & & \bf 26.8 & 13.8 & 9.2 & 24.6 & 7.4 & 10.5 & 3.5 & 24.6 & 19.8 & 21.4 & 62.0 & 84.3 & 31.8 & 1.2 & 7.5 & 56.1 & 51.8 \\
  % \arrayrulecolor{gray!20}
  \arrayrulecolor{black}
  \hline
  \multirow{3}{*}{$32$} 
  & \raggedright BF16 & -81.25\% & 26.3 & 14.9 & 9.1 & 27.0 & 7.3 & 9.9 & 3.1 & 24.6 & 19.1 & 22.5 & 60.5 & 81.6 & 26.9 & 0.0 & 8.2 & 53.4 & 52.6 \\
  \arrayrulecolor{gray!20}
  \hline
  & \raggedright Int4$_{\text{HQQ}}$ & \multirow{2}{*}{-95.31\%}  &\bf  26.1 & 14.7 & 9.5 & 26.6 & 7.9 & 10.7 & 3.4 & 23.6 & 19.0 & 20.5 & 60.5 & 80.8 & 28.3 & 0.0 & 7.6 & 51.9 & 52.0 \\
  & \raggedright Int4$_{\text{Quanto}}$ &  & \bf 26.1 & 14.7 & 9.5 & 26.6 & 7.9 & 10.7 & 3.4 & 23.6 & 19.0 & 20.5 & 60.5 & 80.8 & 28.3 & 0.0 & 7.6 & 51.9 & 52.0 \\
  % \arrayrulecolor{gray!20}
  \arrayrulecolor{black}
  \hline
  \multirow{3}{*}{$16$} 
  & \raggedright BF16 & -87.50\% & \bf 24.4 & 14.7 & 9.5 & 24.3 & 7.8 & 10.2 & 3.8 & 22.8 & 19.1 & 24.6 & 61.0 & 82.8 & 20.2 & 0.2 & 8.6 & 39.9 & 41.4 \\
  \arrayrulecolor{gray!20}
  \hline
  & \raggedright Int4$_{\text{HQQ}}$ & \multirow{2}{*}{-96.87\%}   & \bf 24.2 & 15.2 & 9.4 & 25.2 & 7.4 & 10.2 & 3.9 & 22.9 & 19.8 & 20.6 & 61.0 & 82.5 & 21.7 & 0.1 & 9.0 & 38.0 & 41.2 \\
  & \raggedright Int4$_{\text{Quanto}}$ &  & \bf 23.4 & 15.6 & 8.4 & 22.7 & 7.3 & 10.2 & 3.8 & 20.2 & 18.7 & 18.6 & 61.0 & 81.9 & 21.7 & 0.5 & 8.0 & 36.9 & 38.3\\
  \arrayrulecolor{black}
  
  % \midrule
  \rowcolor{gray!10} \multicolumn{20}{c}{\textit{\textbf{1B7$_{\text{SmolLM}}$ (Length=2K)}}} \\
  & \raggedright BF16 & 100.0\% & 18.7 & 2.6 & 6.3 & 19.9 & 5.4 & 8.6 & 2.7 & 23.5 & 18.4 & 20.2 & 46.5 & 70.2 & 32.4 & 2.2 & 3.2 & 21.3 & 16.5 \\
  & \raggedright Int4$_{\text{HQQ}}$ & \multirow{2}{*}{-75.00\%} 
  & 18.6 & 2.5 & 6.2 & 19.1 & 5.5 & 8.2 & 2.7 & 23.4 & 18.3 & 20.0 & 46.5 & 69.4 & 32.1 & 2.7 & 3.2 & 21.5 & 16.0 \\
  & \raggedright Int4$_{\text{Quanto}}$ & & 18.6 & 2.6 & 6.2 & 17.4 & 5.1 & 8.6 & 2.6 & 23.0 & 18.1 & 20.1 & 46.0 & 70.2 & 31.9 & 2.9 & 3.6 & 21.9 & 16.7 \\
  & \raggedright Int2$_{\text{HQQ}}$ & \multirow{2}{*}{-87.50\%} & 16.3 & 2.5 & 5.6 & 13.0 & 4.8 & 7.5 & 2.7 & 14.8 & 16.3 & 9.3 & 46.0 & 70.4 & 26.9 & 2.6 & 3.4 & 18.3 & 16.8 \\
  & \raggedright Int2$_{\text{Quanto}}$ &  & 13.3 & 1.6 & 3.8 & 10.3 & 3.9 & 7.3 & 1.4 & 5.9 & 13.4 & 6.3 & 40.0 & 64.3 & 14.6 & 3.1 & 3.5 & 15.6 & 17.5 \\
  % \arrayrulecolor{gray!20}
  \hline
  \multirow{3}{*}{$32$} 
  & \raggedright BF16 & -68.75\% & 16.0 & 2.6 & 6.1 & 16.9 & 4.6 & 9.3 & 2.0 & 22.8 & 15.1 & 19.9 & 50.0 & 57.1 & 29.8 & 1.7 & 2.4 & 9.4 & 6.7 \\
  & \raggedright Int4$_{\text{HQQ}}$ & \multirow{2}{*}{-92.19\%}  & 15.9 & 2.7 & 5.7 & 16.3 & 5.0 & 8.5 & 1.8 & 23.0 & 15.0 & 18.5 & 50.0 & 56.2 & 30.2 & 1.8 & 3.2 & 10.0 & 6.8 \\
  & \raggedright Int4$_{\text{Quanto}}$ & & 15.4 & 2.5 & 5.7 & 16.1 & 5.7 & 8.7 & 2.1 & 20.9 & 13.8 & 17.6 & 50.0 & 55.0 & 29.5 & 1.7 & 2.8 & 9.6 & 5.4 \\
  % \arrayrulecolor{gray!20}
  \hline
  \multirow{3}{*}{$16$} 
  & \raggedright BF16 & -81.25\% & 16.5 & 2.6 & 6.2 & 17.2 & 4.5 & 9.7 & 2.1 & 22.0 & 15.3 & 21.0 & 47.5 & 55.5 & 31.7 & 1.2 & 3.3 & 15.8 & 8.5 \\
  & \raggedright Int4$_{\text{HQQ}}$ & \multirow{2}{*}{-95.31\%} & 16.2 & 2.5 & 6.1 & 16.2 & 4.5 & 8.9 & 2.0 & 20.6 & 15.4 & 19.7 & 47.5 & 55.6 & 30.6 & 1.2 & 4.0 & 16.3 & 8.0 \\
  & \raggedright Int4$_{\text{Quanto}}$ &  & 15.6 & 2.5 & 5.7 & 15.6 & 4.3 & 8.8 & 1.6 & 21.2 & 15.7 & 17.6 & 47.0 & 55.7 & 27.4 & 1.7 & 3.6 & 15.6 & 6.2 \\
  % \arrayrulecolor{gray!20}
  \hline
  \multirow{3}{*}{$8$} 
  & \raggedright BF16 & -87.50\% & 15.3 & 2.4 & 5.9 & 17.9 & 4.8 & 10.1 & 1.8 & 25.1 & 15.2 & 20.6 & 42.5 & 49.0 & 31.4 & 2.7 & 3.3 & 7.1 & 4.4\\
  & \raggedright Int4$_{\text{HQQ}}$ & \multirow{2}{*}{-96.87\%}   & 15.0 & 2.4 & 5.7 & 16.9 & 4.7 & 10.1 & 2.0 & 23.5 & 14.7 & 20.3 & 42.5 & 47.6 & 30.6 & 2.6 & 3.6 & 7.7 & 4.5\\
  & \raggedright Int4$_{\text{Quanto}}$ &  & 14.2 & 2.7 & 5.4 & 16.9 & 4.1 & 8.8 & 1.5 & 22.2 & 14.4 & 17.2 & 42.0 & 47.9 & 29.9 & 1.5 & 3.3 & 7.0 & 3.0 \\
  \arrayrulecolor{black}
  % \midrule
\rowcolor{gray!10} \multicolumn{20}{c}{\textit{\textbf{360M$_{\text{SmolLM}}$ (Length=2K)}}} \\
  & \raggedright BF16 & 100.0\% & 13.5 & 2.4 & 6.4 & 14.3 & 5.0 & 8.8 & 2.5 & 18.0 & 17.5 & 7.1 & 47.5 & 37.5 & 24.9 & 1.5 & 3.4 & 8.1 & 10.4 \\
  & \raggedright Int4$_{\text{HQQ}}$ & \multirow{2}{*}{-75.00\%} 
  & 13.4 & 2.7 & 6.1 & 14.1 & 5.5 & 8.4 & 3.0 & 16.2 & 15.4 & 11.2 & 47.5 & 37.5 & 23.4 & 1.3 & 3.7 & 9.0 & 10.1 \\
  & \raggedright Int4$_{\text{Quanto}}$ & & 13.3 & 2.4 & 6.2 & 13.7 & 5.4 & 8.7 & 2.6 & 15.4 & 17.4 & 7.3 & 47.5 & 37.3 & 24.4 & 1.0 & 3.7 & 8.4 & 11.0 \\
  & \raggedright Int2$_{\text{HQQ}}$ & \multirow{2}{*}{-87.50\%} & 10.8 & 2.7 & 4.7 & 8.3 & 5.4 & 5.9 & 1.9 & 9.9 & 10.0 & 8.4 & 45.2 & 27.5 & 14.2 & 2.1 & 4.2 & 10.0 & 11.9 \\
  & \raggedright Int2$_{\text{Quanto}}$ &  & 8.6 & 2.6 & 2.2 & 4.4 & 3.9 & 4.8 & 1.4 & 5.6 & 8.9 & 2.9 & 44.0 & 26.8 & 9.6 & 1.0 & 1.9 & 7.2 & 9.7 \\
  % \arrayrulecolor{gray!20}
  \hline
  \multirow{3}{*}{$32$} 
  & \raggedright BF16 & -68.75\% & 13.5 & 2.3 & 5.9 & 13.4 & 5.5 & 9.8 & 2.7 & 20.4 & 14.5 & 11.5 & 41.0 & 31.2 & 29.6 & 1.2 & 3.5 & 15.4 & 7.9 \\
  & \raggedright Int4$_{\text{HQQ}}$ & \multirow{2}{*}{-92.19\%}  & \bf 12.5 & 2.6 & 5.7 & 12.1 & 5.1 & 10.2 & 2.7 & 14.6 & 12.5 & 8.8 & 41.0 & 30.3 & 27.8 & 1.9 & 2.7 & 14.5 & 7.6 \\
  & \raggedright Int4$_{\text{Quanto}}$ & & \bf 12.3 & 2.0 & 5.2 & 11.9 & 5.0 & 9.1 & 3.0 & 15.4 & 14.9 & 8.3 & 41.0 & 28.3 & 27.0 & 0.9 & 3.9 & 13.8 & 7.8 \\
  % \arrayrulecolor{gray!20}
  \hline
  \multirow{3}{*}{$16$} 
  & \raggedright BF16 & -81.25\% & 11.6 & 2.2 & 5.2 & 13.0 & 4.8 & 9.5 & 3.2 & 13.4 & 13.4 & 11.3 & 32.0 & 26.1 & 22.5 & 1.1 & 5.0 & 14.9 & 7.7 \\
  & \raggedright Int4$_{\text{HQQ}}$ & \multirow{2}{*}{-95.31\%}  & \bf 11.2 & 2.6 & 5.6 & 12.0 & 5.1 & 8.8 & 2.9 & 13.4 & 12.4 & 10.8 & 32.0 & 24.8 & 21.8 & 2.1 & 3.7 & 14.0 & 7.2 \\
  & \raggedright Int4$_{\text{Quanto}}$ &  & \bf 10.9 & 1.9 & 4.9 & 11.5 & 4.2 & 8.8 & 2.6 & 12.2 & 12.2 & 9.5 & 32.5 & 25.8 & 18.5 & 1.4 & 4.6 & 15.5 & 7.8 \\
  % \arrayrulecolor{gray!20}
  \hline
  \multirow{3}{*}{$8$} 
  & \raggedright BF16 & -87.50\% & 9.9 & 1.9 & 4.7 & 11.7 & 4.5 & 8.5 & 2.8 & 13.0 & 12.9 & 9.4 & 34.0 & 17.2 & 15.3 & 1.4 & 3.2 & 11.4 & 6.9 \\
  & \raggedright Int4$_{\text{HQQ}}$ & \multirow{2}{*}{-96.87\%}   & 10.0 & 2.2 & 4.8 & 11.0 & 4.2 & 8.2 & 2.6 & 13.1 & 12.8 & 11.7 & 33.5 & 17.3 & 14.8 & 0.8 & 4.4 & 11.6 & 7.4\\
  & \raggedright Int4$_{\text{Quanto}}$ &  & 9.3 & 1.8 & 3.6 & 11.3 & 4.0 & 8.0 & 3.0 & 10.6 & 12.0 & 7.4 & 31.5 & 19.8 & 10.3 & 0.8 & 4.8 & 12.1 & 7.6 \\
  \arrayrulecolor{black}
  \midrule


\end{tabular}
% }
\caption{Evaluation results of all models on LongBench, including Task A: narrativeqa, B: qasper, C: multifieldqa\_en, D: hotpotqa, E: 2wikimqa, F: musique, G: gov\_report, H: qmsum, I: multi\_news, J: trec, K: triviaqa, L: samsum, M: passage\_count, N: passage\_retrieval\_en, O: lcc, P: repobench-p. \textbf{Bold} indicates compression ratios greater than or equal to Int2 quantization while also achieving performance higher than Int2.}
\label{tab:other_long_bench}
\end{table*}

\begin{table*}[t]
\centering
\small
\begin{tabular}{lcr@{\hspace{2pt}}lcccccc}
  \toprule
  \textbf{Model} & \textbf{Tokens} & \multicolumn{2}{l}{\textbf{Avg@CS}} & \textbf{MMLU} & \textbf{ARC} & \textbf{PIQA} & \textbf{HS} & \textbf{OBQA} & \textbf{WG}\\
  \midrule
  \rowcolor{gray!10}135M$_{\text{SmolLM}}$ & 600B & \multicolumn{2}{l}{44.50} & 29.80 & 42.43 & 68.06 & 41.09 & 33.60 & 52.01\\
  \textit{- full-rope} & \multirow{5}{*}{2.25B}  & 44.25 &  & 29.82 & 42.05 & 68.34 & 41.03 & 33.20 & 51.07 \\
  - $\mathcal{S}_{\text{high}}$ &   & 43.40&\textsubscript{-0.85} & 29.90 & 41.15 & 66.92 & 39.34 & 32.60 & 50.51 \\
  - $\mathcal{S}_{\text{low}}$ &   & 37.76&\textsubscript{-6.49} & 27.11 & 32.06 & 59.79 & 30.68 & 28.40 & 48.54 \\
  - $\mathcal{S}_{\text{uniform}}$ &   & 43.76&\textsubscript{-0.49} & 29.87 & 41.29 & 67.36 & 40.22 & 32.80 & 50.99 \\
  - $\mathcal{S}_{\text{2-norm}}$ &   & \bf 43.77&\textsubscript{-0.48} & 30.14 & 41.69 & 67.57 & 39.53 & 33.00 & 50.67 \\
  \arrayrulecolor{gray!20}
  \hline
  - $\mathcal{S}_{\text{high}}$ + SVD\textsubscript{joint} & \multirow{4}{*}{2.25B} & 41.04&\textsubscript{-3.21} & 28.16 & 37.55 & 64.91 & 34.91 & 32.00 & 48.70 \\
  - $\mathcal{S}_{\text{uniform}}$ + SVD\textsubscript{joint} &  & 41.77&\textsubscript{-2.48} & 28.58 & 38.69 & 65.67 & 36.17 & 32.00 & 49.49 \\
  - $\mathcal{S}_{\text{2-norm}}$ + SVD\textsubscript{joint} &   & \bf 41.84&\textsubscript{-2.41} & 28.66 & 39.95 & 65.02 & 36.04 & 31.60 & 49.80\\
  % \hline
  % - $\mathcal{S}_{\text{uniform}}$ + SVD\textsubscript{split} & & 40.96 & \textsubscript{-3.29} \\
  - $\mathcal{S}_{\text{2-norm}}$ + SVD\textsubscript{split} &  & 40.92 & \textsubscript{-3.33} & 28.04 & 37.85 & 65.56 & 34.60 & 29.8 & 49.65 \\
  \arrayrulecolor{black}
  \midrule
  \rowcolor{gray!10}1B7$_{\text{SmolLM}}$ & 1T & \multicolumn{2}{l}{55.90} & 39.27 & 59.87 & 75.73 & 62.93 & 42.80 & 54.85 \\
  \textit{- full-rope} & \multirow{5}{*}{6B}  & \multicolumn{2}{l}{55.93} & 39.11 & 59.19 & 75.95 & 62.92 & 43.40 & 55.09 \\
  - $\mathcal{S}_{\text{high}}$ &   & 55.17&\textsubscript{-0.76} & 38.56 & 57.72 & 75.73 & 60.93 & 44.00 & 54.06 \\
  - $\mathcal{S}_{\text{low}}$ &   & 54.72&\textsubscript{-1.21} & 37.82 & 56.47 & 75.35 & 60.06 & 43.20 & 55.41 \\
  - $\mathcal{S}_{\text{uniform}}$ &   & \bf 55.31&\textsubscript{-0.62} & 38.93 & 57.93 & 75.63 & 61.97 & 42.60 & 54.85 \\
  - $\mathcal{S}_{\text{2-norm}}$ &   & 55.10&\textsubscript{-0.83} & 38.60 & 57.36 & 75.68 & 61.77 & 43.00 & 54.22 \\
  \arrayrulecolor{gray!20}
  \hline
  - $\mathcal{S}_{\text{high}}$ + SVD\textsubscript{joint} & \multirow{4}{*}{6B}  & 54.41&\textsubscript{-1.52} & 37.97 & 56.74 & 75.14 & 59.75 & 42.00 & 54.85 \\
  - $\mathcal{S}_{\text{uniform}}$ + SVD\textsubscript{joint} &   & 54.30&\textsubscript{-1.63} & 37.82 & 56.30 & 75.08 & 60.35 & 42.40 & 53.91 \\
  - $\mathcal{S}_{\text{2-norm}}$ + SVD\textsubscript{joint} &   & \bf 54.65&\textsubscript{-1.28} & 37.87 & 56.81 & 75.84 & 60.41 & 42.60 & 54.38 \\
% \hline
  % - $\mathcal{S}_{\text{uniform}}$ + SVD\textsubscript{split} & & 53.92  & \textsubscript{-2.01} \\
  - $\mathcal{S}_{\text{2-norm}}$ + SVD\textsubscript{split} &   & 53.91  & \textsubscript{-2.02} & 37.64 & 55.50 & 75.46 & 59.52 & 42.40 & 52.96 \\
  \arrayrulecolor{black}
  \bottomrule
\end{tabular}
\caption{The complete results of the ablation experiment.}
\vspace{-0.4cm}
\label{tab:partial_rope_full}
\end{table*}
\begin{figure}[t]
  \centering
  \includegraphics[width=\linewidth]{fig/fig_res_rank_loss_fig.pdf}
  \caption{The fine-tuning loss curves under different KV cache storage ratios (with colors ranging from light to dark representing 12.5\%, 18.75\%, 31.25\%, and 100\%).}
  \label{fig:res_rank_loss}
  \vspace{-0.3cm}
\end{figure}
\begin{figure}[t]
  \centering
  \includegraphics[width=\linewidth]{fig/fig_pe_loss_fig.pdf}
  \caption{The fine-tuning loss curves under different partial-RoPE strategy.}
  \label{fig:pe_loss}
  \vspace{-0.3cm}
\end{figure}
\begin{figure}[t]
  \centering
  \includegraphics[width=\linewidth]{fig/fig_svd_loss_fig.pdf}
  \caption{The fine-tuning loss curves under the combination of $\mathcal{S}_{\text{2-norm}}$ and different SVD strategies.}
  \label{fig:svd_loss}
  \vspace{-0.3cm}
\end{figure}

%  Both \(\mathcal{S}_{\text{uniform}}\) and \(\mathcal{S}_{2\text{-norm}}\) yielded better performance 

In this section, we present the detailed results. 
\paragraph{Detailed LongBench evaluation} is reported in \Cref{tab:other_long_bench}.

\paragraph{Detailed ablation experiment} is reported in \Cref{tab:partial_rope_full}.

\paragraph{Additional visualizations of fine-tuning loss} 
We present the loss of the other two models fine-tuned, excluding the ones mentioned in the main text, in \Cref{fig:res_rank_loss}. 
We observe that as fine-tuning progresses, the gap in loss between our approach and the baseline gradually decreases, and both exhibit similar fluctuations, demonstrating the effectiveness of our approach. 
In \Cref{fig:pe_loss}, we show the loss under different partial-RoPE strategies. Except for $\mathcal{S}_{\text{low}}$, the other three partial-RoPE schemes show little difference from the baseline. Additionally, $\mathcal{S}_{\text{low}}$ has a higher probability of convergence failure. In \Cref{fig:svd_loss}, we show the loss under different SVD strategies. The loss curves on both 1B7$_{\text{SmolLM}}$ and 135M$_{\text{SmolLM}}$ reveal that SVD\textsubscript{joint} outperforms SVD\textsubscript{split}.


% The evaluation results of ablation experiment and results of all smollm models on the LongBench benchmark, which are documented in the \Cref{tab:partial_rope_full} and \Cref{tab:other_long_bench}. We can draw similar conclusions to those in the main text from the results in \Cref{tab:partial_rope_full}: both \(\mathcal{S}_{\text{uniform}}\) and \(\mathcal{S}_{2\text{-norm}}\) yielded better performance and $\mathcal{S}_{\text{2-norm}}$ + SVD\textsubscript{joint} can achieve better results compared to $\mathcal{S}_{\text{2-norm}}$ + SVD\textsubscript{split}.  

