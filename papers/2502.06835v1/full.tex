% This is samplepaper.tex, a sample chapter demonstrating the
% LLNCS macro package for Springer Computer Science proceedings;
% Version 2.21 of 2022/01/12
%
% \documentclass[runningheads]{llncs}
\documentclass{article}
%
\usepackage[T1]{fontenc}
\usepackage{microtype}
% T1 fonts will be used to generate the final print and online PDFs,
% so please use T1 fonts in your manuscript whenever possible.
% Other font encondings may result in incorrect characters.
%
\usepackage{graphicx}
\usepackage{makecell}
\usepackage{booktabs}
\usepackage{amsmath}
\usepackage{float}
\usepackage{amsfonts}
\usepackage{bm}
\usepackage{subcaption}
\usepackage{algorithm}
\usepackage{algorithmic}
% Used for displaying a sample figure. If possible, figure files should
% be included in EPS format.
%
% If you use the hyperref package, please uncomment the following two lines
% to display URLs in blue roman font according to Springer's eBook style:
\usepackage{color}
\usepackage{hyperref}
\usepackage{xcolor}
% \setcitestyle{maxnames=3}
\renewcommand\UrlFont{\color{blue}\rmfamily}

\definecolor{darkblue}{RGB}{0,0,139}  % Define dark blue color
\urlstyle{rm}

% \input{squenze}

\def\vbeta{{\bm{\beta}}}
% 
%
\setlength\unitlength{1mm}
\newcommand{\twodots}{\mathinner {\ldotp \ldotp}}
% bb font symbols
\newcommand{\Rho}{\mathrm{P}}
\newcommand{\Tau}{\mathrm{T}}

\newfont{\bbb}{msbm10 scaled 700}
\newcommand{\CCC}{\mbox{\bbb C}}

\newfont{\bb}{msbm10 scaled 1100}
\newcommand{\CC}{\mbox{\bb C}}
\newcommand{\PP}{\mbox{\bb P}}
\newcommand{\RR}{\mbox{\bb R}}
\newcommand{\QQ}{\mbox{\bb Q}}
\newcommand{\ZZ}{\mbox{\bb Z}}
\newcommand{\FF}{\mbox{\bb F}}
\newcommand{\GG}{\mbox{\bb G}}
\newcommand{\EE}{\mbox{\bb E}}
\newcommand{\NN}{\mbox{\bb N}}
\newcommand{\KK}{\mbox{\bb K}}
\newcommand{\HH}{\mbox{\bb H}}
\newcommand{\SSS}{\mbox{\bb S}}
\newcommand{\UU}{\mbox{\bb U}}
\newcommand{\VV}{\mbox{\bb V}}


\newcommand{\yy}{\mathbbm{y}}
\newcommand{\xx}{\mathbbm{x}}
\newcommand{\zz}{\mathbbm{z}}
\newcommand{\sss}{\mathbbm{s}}
\newcommand{\rr}{\mathbbm{r}}
\newcommand{\pp}{\mathbbm{p}}
\newcommand{\qq}{\mathbbm{q}}
\newcommand{\ww}{\mathbbm{w}}
\newcommand{\hh}{\mathbbm{h}}
\newcommand{\vvv}{\mathbbm{v}}

% Vectors

\newcommand{\av}{{\bf a}}
\newcommand{\bv}{{\bf b}}
\newcommand{\cv}{{\bf c}}
\newcommand{\dv}{{\bf d}}
\newcommand{\ev}{{\bf e}}
\newcommand{\fv}{{\bf f}}
\newcommand{\gv}{{\bf g}}
\newcommand{\hv}{{\bf h}}
\newcommand{\iv}{{\bf i}}
\newcommand{\jv}{{\bf j}}
\newcommand{\kv}{{\bf k}}
\newcommand{\lv}{{\bf l}}
\newcommand{\mv}{{\bf m}}
\newcommand{\nv}{{\bf n}}
\newcommand{\ov}{{\bf o}}
\newcommand{\pv}{{\bf p}}
\newcommand{\qv}{{\bf q}}
\newcommand{\rv}{{\bf r}}
\newcommand{\sv}{{\bf s}}
\newcommand{\tv}{{\bf t}}
\newcommand{\uv}{{\bf u}}
\newcommand{\wv}{{\bf w}}
\newcommand{\vv}{{\bf v}}
\newcommand{\xv}{{\bf x}}
\newcommand{\yv}{{\bf y}}
\newcommand{\zv}{{\bf z}}
\newcommand{\zerov}{{\bf 0}}
\newcommand{\onev}{{\bf 1}}

% Matrices

\newcommand{\Am}{{\bf A}}
\newcommand{\Bm}{{\bf B}}
\newcommand{\Cm}{{\bf C}}
\newcommand{\Dm}{{\bf D}}
\newcommand{\Em}{{\bf E}}
\newcommand{\Fm}{{\bf F}}
\newcommand{\Gm}{{\bf G}}
\newcommand{\Hm}{{\bf H}}
\newcommand{\Id}{{\bf I}}
\newcommand{\Jm}{{\bf J}}
\newcommand{\Km}{{\bf K}}
\newcommand{\Lm}{{\bf L}}
\newcommand{\Mm}{{\bf M}}
\newcommand{\Nm}{{\bf N}}
\newcommand{\Om}{{\bf O}}
\newcommand{\Pm}{{\bf P}}
\newcommand{\Qm}{{\bf Q}}
\newcommand{\Rm}{{\bf R}}
\newcommand{\Sm}{{\bf S}}
\newcommand{\Tm}{{\bf T}}
\newcommand{\Um}{{\bf U}}
\newcommand{\Wm}{{\bf W}}
\newcommand{\Vm}{{\bf V}}
\newcommand{\Xm}{{\bf X}}
\newcommand{\Ym}{{\bf Y}}
\newcommand{\Zm}{{\bf Z}}

% Calligraphic

\newcommand{\Ac}{{\cal A}}
\newcommand{\Bc}{{\cal B}}
\newcommand{\Cc}{{\cal C}}
\newcommand{\Dc}{{\cal D}}
\newcommand{\Ec}{{\cal E}}
\newcommand{\Fc}{{\cal F}}
\newcommand{\Gc}{{\cal G}}
\newcommand{\Hc}{{\cal H}}
\newcommand{\Ic}{{\cal I}}
\newcommand{\Jc}{{\cal J}}
\newcommand{\Kc}{{\cal K}}
\newcommand{\Lc}{{\cal L}}
\newcommand{\Mc}{{\cal M}}
\newcommand{\Nc}{{\cal N}}
\newcommand{\nc}{{\cal n}}
\newcommand{\Oc}{{\cal O}}
\newcommand{\Pc}{{\cal P}}
\newcommand{\Qc}{{\cal Q}}
\newcommand{\Rc}{{\cal R}}
\newcommand{\Sc}{{\cal S}}
\newcommand{\Tc}{{\cal T}}
\newcommand{\Uc}{{\cal U}}
\newcommand{\Wc}{{\cal W}}
\newcommand{\Vc}{{\cal V}}
\newcommand{\Xc}{{\cal X}}
\newcommand{\Yc}{{\cal Y}}
\newcommand{\Zc}{{\cal Z}}

% Bold greek letters

\newcommand{\alphav}{\hbox{\boldmath$\alpha$}}
\newcommand{\betav}{\hbox{\boldmath$\beta$}}
\newcommand{\gammav}{\hbox{\boldmath$\gamma$}}
\newcommand{\deltav}{\hbox{\boldmath$\delta$}}
\newcommand{\etav}{\hbox{\boldmath$\eta$}}
\newcommand{\lambdav}{\hbox{\boldmath$\lambda$}}
\newcommand{\epsilonv}{\hbox{\boldmath$\epsilon$}}
\newcommand{\nuv}{\hbox{\boldmath$\nu$}}
\newcommand{\muv}{\hbox{\boldmath$\mu$}}
\newcommand{\zetav}{\hbox{\boldmath$\zeta$}}
\newcommand{\phiv}{\hbox{\boldmath$\phi$}}
\newcommand{\psiv}{\hbox{\boldmath$\psi$}}
\newcommand{\thetav}{\hbox{\boldmath$\theta$}}
\newcommand{\tauv}{\hbox{\boldmath$\tau$}}
\newcommand{\omegav}{\hbox{\boldmath$\omega$}}
\newcommand{\xiv}{\hbox{\boldmath$\xi$}}
\newcommand{\sigmav}{\hbox{\boldmath$\sigma$}}
\newcommand{\piv}{\hbox{\boldmath$\pi$}}
\newcommand{\rhov}{\hbox{\boldmath$\rho$}}
\newcommand{\upsilonv}{\hbox{\boldmath$\upsilon$}}

\newcommand{\Gammam}{\hbox{\boldmath$\Gamma$}}
\newcommand{\Lambdam}{\hbox{\boldmath$\Lambda$}}
\newcommand{\Deltam}{\hbox{\boldmath$\Delta$}}
\newcommand{\Sigmam}{\hbox{\boldmath$\Sigma$}}
\newcommand{\Phim}{\hbox{\boldmath$\Phi$}}
\newcommand{\Pim}{\hbox{\boldmath$\Pi$}}
\newcommand{\Psim}{\hbox{\boldmath$\Psi$}}
\newcommand{\Thetam}{\hbox{\boldmath$\Theta$}}
\newcommand{\Omegam}{\hbox{\boldmath$\Omega$}}
\newcommand{\Xim}{\hbox{\boldmath$\Xi$}}


% Sans Serif small case

\newcommand{\Gsf}{{\sf G}}

\newcommand{\asf}{{\sf a}}
\newcommand{\bsf}{{\sf b}}
\newcommand{\csf}{{\sf c}}
\newcommand{\dsf}{{\sf d}}
\newcommand{\esf}{{\sf e}}
\newcommand{\fsf}{{\sf f}}
\newcommand{\gsf}{{\sf g}}
\newcommand{\hsf}{{\sf h}}
\newcommand{\isf}{{\sf i}}
\newcommand{\jsf}{{\sf j}}
\newcommand{\ksf}{{\sf k}}
\newcommand{\lsf}{{\sf l}}
\newcommand{\msf}{{\sf m}}
\newcommand{\nsf}{{\sf n}}
\newcommand{\osf}{{\sf o}}
\newcommand{\psf}{{\sf p}}
\newcommand{\qsf}{{\sf q}}
\newcommand{\rsf}{{\sf r}}
\newcommand{\ssf}{{\sf s}}
\newcommand{\tsf}{{\sf t}}
\newcommand{\usf}{{\sf u}}
\newcommand{\wsf}{{\sf w}}
\newcommand{\vsf}{{\sf v}}
\newcommand{\xsf}{{\sf x}}
\newcommand{\ysf}{{\sf y}}
\newcommand{\zsf}{{\sf z}}


% mixed symbols

\newcommand{\sinc}{{\hbox{sinc}}}
\newcommand{\diag}{{\hbox{diag}}}
\renewcommand{\det}{{\hbox{det}}}
\newcommand{\trace}{{\hbox{tr}}}
\newcommand{\sign}{{\hbox{sign}}}
\renewcommand{\arg}{{\hbox{arg}}}
\newcommand{\var}{{\hbox{var}}}
\newcommand{\cov}{{\hbox{cov}}}
\newcommand{\Ei}{{\rm E}_{\rm i}}
\renewcommand{\Re}{{\rm Re}}
\renewcommand{\Im}{{\rm Im}}
\newcommand{\eqdef}{\stackrel{\Delta}{=}}
\newcommand{\defines}{{\,\,\stackrel{\scriptscriptstyle \bigtriangleup}{=}\,\,}}
\newcommand{\<}{\left\langle}
\renewcommand{\>}{\right\rangle}
\newcommand{\herm}{{\sf H}}
\newcommand{\trasp}{{\sf T}}
\newcommand{\transp}{{\sf T}}
\renewcommand{\vec}{{\rm vec}}
\newcommand{\Psf}{{\sf P}}
\newcommand{\SINR}{{\sf SINR}}
\newcommand{\SNR}{{\sf SNR}}
\newcommand{\MMSE}{{\sf MMSE}}
\newcommand{\REF}{{\RED [REF]}}

% Markov chain
\usepackage{stmaryrd} % for \mkv 
\newcommand{\mkv}{-\!\!\!\!\minuso\!\!\!\!-}

% Colors

\newcommand{\RED}{\color[rgb]{1.00,0.10,0.10}}
\newcommand{\BLUE}{\color[rgb]{0,0,0.90}}
\newcommand{\GREEN}{\color[rgb]{0,0.80,0.20}}

%%%%%%%%%%%%%%%%%%%%%%%%%%%%%%%%%%%%%%%%%%
\usepackage{hyperref}
\hypersetup{
    bookmarks=true,         % show bookmarks bar?
    unicode=false,          % non-Latin characters in AcrobatÕs bookmarks
    pdftoolbar=true,        % show AcrobatÕs toolbar?
    pdfmenubar=true,        % show AcrobatÕs menu?
    pdffitwindow=false,     % window fit to page when opened
    pdfstartview={FitH},    % fits the width of the page to the window
%    pdftitle={My title},    % title
%    pdfauthor={Author},     % author
%    pdfsubject={Subject},   % subject of the document
%    pdfcreator={Creator},   % creator of the document
%    pdfproducer={Producer}, % producer of the document
%    pdfkeywords={keyword1} {key2} {key3}, % list of keywords
    pdfnewwindow=true,      % links in new window
    colorlinks=true,       % false: boxed links; true: colored links
    linkcolor=red,          % color of internal links (change box color with linkbordercolor)
    citecolor=green,        % color of links to bibliography
    filecolor=blue,      % color of file links
    urlcolor=blue           % color of external links
}
%%%%%%%%%%%%%%%%%%%%%%%%%%%%%%%%%%%%%%%%%%%



\usepackage{amssymb}
\usepackage{bbm}

\newcommand{\sR}{\mathbb{R}}
\newcommand{\AYA}{\tiny\texttt{AYA}}
\newcommand{\CARE}{\tiny\texttt{CARE}}
\newcommand{\EDGE}{\tiny\texttt{REL}}

\DeclareMathOperator*{\argmax}{arg\,max}

\def\Sleep{{\texttt{Sleep}}}
\def\Stress{{\texttt{Stress}}}
\def\Rel{{\texttt{Relatsh}}}
\def\Mood{{\texttt{Mood}}}
\def\SqrtStep{{\texttt{SqrtStep}}}
\def\ADH{{\texttt{ADH}}}
\def\Treat{\tiny{\texttt{Treat}}}

\def\AM{{\texttt{AM}}}
\def\PM{{\texttt{PM}}}

%

% Comments
% \usepackage{xcolor}
\newcount\comments  % 0 suppresses notes to selves in text
\comments=1  % TODO: change to 0 for final version
\newcommand{\genComment}[2]{\ifnum\comments=1{\textcolor{#1}{\textsf{\footnotesize #2}}}\fi}
\newcommand{\ziping}[1]{\genComment{green}{[Ziping:#1]}}
\newcommand{\sam}[1]{\genComment{blue}{[SAM:#1]}}
\newcommand{\hinal}[1]{\genComment{brown}{[Hinal:#1]}}

\usepackage{silence}
\WarningFilter{latexfont}{Command \tiny invalid in math mode}
\WarningFilter{latexfont}{Overfull}
\hbadness=10000  % Add this line to suppress overfull hbox warnings

\newcommand{\coloneq}{\mathrel{\mathop:}=}

\definecolor{redorange}{RGB}{255,69,0}  % A reddish-orange color
\definecolor{grassgreen}{RGB}{76, 153, 0}
\begin{document}
%
\title{Reinforcement Learning on AYA Dyads to Enhance Medication Adherence}

\date{}
%
%\titlerunning{Abbreviated paper title}
% If the paper title is too long for the running head, you can set
% an abbreviated paper title here
%
% \author{Ziping Xu\inst{1} \and
% Hinal Jajal\inst{1} \and
% Sung Won Choi\inst{2} \and
% Inbal Nahum-Shani\inst{2} \and
% Guy Shani\inst{2} \and
% Alexandra M. Psihogios \inst{3} \and
% Pei-Yao Hung \inst{2} \and
% Susan Murphy\inst{1}}
%
% \authorrunning{Xu et al.}
% First names are abbreviated in the running head.
% If there are more than two authors, 'et al.' is used.
%
\author{Ziping Xu\thanks{Harvard University, Cambridge, MA, USA} \and
Hinal Jajal\footnotemark[1] \and
Sung Won Choi\thanks{University of Michigan, Ann Arbor, MI, USA} \and
Inbal Nahum-Shani\footnotemark[3] \and
Guy Shani\footnotemark[3] \and
Alexandra M. Psihogios\thanks{Northwestern University, Feinberg School of Medicine, Chicago, IL, USA} \and
Pei-Yao Hung\footnotemark[3] \and
Susan Murphy\footnotemark[1]}
\maketitle              % typeset the header of the contribution
%
\begin{abstract}

% \ziping{Alternative title: Enhancing Medication Adherence on AYA Dyads with Reinforcement Learning Powered Digital Interventions}

% \ziping{Should we include multi-agent? For example, Multi-agent Reinforcement Learning on Dyads to Enhance Medication Adherence}

Medication adherence is critical for the recovery of adolescents and young adults (AYAs) who have undergone hematopoietic cell transplantation (HCT). However, maintaining adherence is challenging for AYAs after hospital discharge, who experience both individual (e.g. physical and emotional symptoms) and interpersonal barriers (e.g., relational difficulties with their care partner, who is often involved in medication management).
To optimize the effectiveness of a three-component digital intervention targeting both members of the dyad as well as their relationship, we propose a novel Multi-Agent Reinforcement Learning (MARL) approach to personalize the delivery of interventions.
By incorporating the domain knowledge, the  MARL framework, where each agent is responsible for the delivery of one intervention component, allows for faster learning compared with a flattened agent. Evaluation using a dyadic simulator environment, based on real clinical data, shows a significant improvement in medication adherence (approximately 3\%) compared to purely random intervention delivery. The effectiveness of this approach will be further evaluated in an upcoming trial.

% of optimizing deliveries across different time scales for each intervention component and accelerating learning in the noisy envioronment induced by the complex dynamics of human behavior are addressed by proposing a multi-agent RL framework that personalizes intervention delivery. We evaluate the proposed framework in a dyadic simulator environment, which is shown to significantly improve medication adherence over baseline algorithms.

% The intervention package combines both positive psychology messages for individual outcomes and collaborative games for relational outcomes to support AYAs and their care partners over a 100-day period following hematopoietic cell transplantation (HCT). 

% By integrating personalized positive psychology messages and collaborative activities, the intervention aims to improve both individual and relational outcomes. We address key challenges in multi-agent reinforcement learning, including managing interventions across different time scales and accelerating learning in noisy, data-limited environments. Our results demonstrate the effectiveness of the proposed framework in fostering collaboration among agents and optimizing intervention delivery, ultimately contributing to improved medication adherence in dyadic contexts.


% \keywords{Reinforcement Learning \and Dyadic Relationships \and Medication Adherence \and Digital Health}
\end{abstract}

\section{Introduction}

% Medication adherence is critical for improving health outcomes, especially for patients with severe conditions. 
For patients who have 
undergone allogeneic \textbf{hematopoietic stem cell transplantation (HCT)}, strict adherence to medication regimens, such as prophylactic immunosuppressant therapy (i.e., calcineurin inhibitors, such as tacrolimus or cyclosporine, taken twice-daily), is crucial for mitigating the risk of acute graft-versus host disease (GVHD) \cite{gresch2017medication}. Acute GVHD occurs in 50-70\% of patients following HCT. A lower medication adherence (60\%) rate is shown to associate with higher severity of GVHD \cite{kirsch2014differences}.

The challenges of adherence management are amplified among \textbf{adolescents and young adults (AYAs)}, 
%a medically underserved cancer cohort 
who often demonstrate poorer medication adherence \cite{psihogios2020needle,psihogios2022social,lyons2018theory}. For AYAs with cancer, self-management rarely involves the individual alone. Instead, up to 73\% of family care partners bear the primary responsibility for managing cancer-related medications for AYAs \cite{psihogios2020adherence}. 
% Many of these dyads express a desire to move toward sharing these responsibilities \cite{psihogios2020adherence}. This \textbf{dyadic structure} motivates us to design novel interventions that support not only the individuals in the dyads but also the dyadic relationship between the AYA and the care partner.

Many of these dyads express a desire to move toward sharing these responsibilities with each other \cite{psihogios2020adherence}. Indeed, for AYAs with chronic health conditions, this developmental period often marks a shift from relying solely on a caregiver to taking more personal responsibility for health care. While shifts in autonomy versus dependence and navigating the ensuing family conflict that can arise from these new dynamics are normative parts of AYA development, difficult family interactions can have a detrimental impact on medication adherence. For example, in a meta-analysis \cite{psihogios2019family}, higher level of family conflict and lower levels of family cohesion were significantly associated with worse medication adherence across pediatric illnesses and age groups. 
% \sam{this is about general, not digital intervention development.  I deleted to tighten up the exposition} These dyadic processes that impact medication adherence motivates us to design novel interventions that support not only the individuals in the dyads but also the dyadic relationship between the AYA and the care partner.


% why digital interventions and RL?
After being discharged from the hospital, both individuals in the dyad face significant emotional and physical challenges as they adjust to managing medication regimens \textit{outside the hospital environment}.
% Care partners often face a myriad of challenges as they navigate the demands of their paid jobs with unpaid caregiving responsibilities \cite{reinhard2008supporting}. Those who shoulder heavy caregiving responsibilities at home face higher physical and emotional stressors, which can impede their ability to provide effective care, make sound decisions, and even manage self-care. 
For AYAs, the daily challenges of managing complex medication regimens, coping with treatment side effects, coping with stress, and maintaining normal activities in the context of a complex medical regimen can create distress in their home environment. Similarly, care partners must balance caregiving responsibilities with their personal obligations. Those who shoulder heavy caregiving responsibilities at home face higher physical and emotional stressors, which can impede their ability to provide effective care, make sound decisions, and support their AYA's  self-care \cite{reinhard2008supporting}.


This need for support outside the inpatient environment motivates the development of interventions that leverage \textbf{digital technologies} such as mobile devices \cite{uribe2023effectiveness}.
% , such as positive psychology messages through mobile apps 
Digital interventions are promising for supporting both AYAs and care partners \textit{at home} on a daily basis, compared to traditional clinical support delivered with limited frequency (e.g., weekly clinical visits for post-HCT AYAs). There is strong heterogeneity across dyads and the users' context are constantly changing, which makes it important to personalize the intervention delivery to optimize the effectiveness of digital interventions.
% To optimize the effectiveness of digital interventions, it is important to personalize the intervention delivery based to the user's changing context and the heterogeneity. 
% \ziping{Billie: The justification for personalization and what the term means require more clarity } 
Reinforcement Learning (RL), a machine learning technique that adaptively learns the optimal behavior in an unknown environment to maximize cumulative rewards, is a promising approach for achieving this personalization.
% by adaptively learning from interactions with users. \ziping{Billie: This definition seems too broad to me, but maybe this is ok for this audience? I’m not sure...} 
RL has been successfully applied in a variety of digital interventions \cite{liao2020personalized,battalio2021sense2stop,trella2024deployed,ghosh2024miwaves}.

% Why personalized: people habituate. Have to be more thoughtful about providing supports. So personalization/adaptation is about managing the burden. 

In this paper, we describe our work in developing an RL algorithm for ADAPTS-HCT \cite{shani2024tips}. ADAPTS-HCT is a digital intervention for improving medication adherence by AYAs  over 100 days after receiving HCT. ADAPTS-HCT integrates three components: (1) twice-daily messages promoting positive emotions for the AYA, (2) daily messages focusing on coping and self-care strategies for the care partner, and (3) a weekly collaborative game for improving their relationship \cite{shani2024tips}. \textit{We call the three components AYA, care partner, and relationship component}, respectively. Table \ref{tab:interventions} summarizes these components. The fully developed intervention package will be evaluated in the upcoming clinical trial.
%, the ADAPTS-HCT Study. 
% RL algorithm is designed to personalize the delivery of these interventions to optimize their effectiveness.
% in terms of the feasibility, stability, autonomy of the algorithm, acceptability of interventions, and usability of the app.
% upcoming studies


% \ziping{Mention this when we talk about that this is for an upcoming clinical trial. About the feasibility, stability, autonamy of the algorithm. Acceptability of the intervention. Usability of app.}

% \vspace{-8mm}

\begin{table}[h]
\centering
\caption{Intervention components in ADAPTS-HCT}
\begin{tabular}{c|c}
\hline
\textbf{Component} & \textbf{Intervention} \\
\hline
AYA & Twice-daily positive psychology messages \\
\hline
Care partner & Daily positive psychology messages \\
\hline
Relationship & \makecell[c]{Weekly collaborative game designed to facilitate\\ positive dyadic interpersonal relationship}\\
\hline
\end{tabular}
\label{tab:interventions}
\end{table}

% \vspace{20mm}

\textbf{Goals.} Our goal is to design an RL algorithm that can personalize the delivery of these interventions to optimize their effectiveness. Given the complexity of the dyadic structure, we identify the following two key challenges:

% Challenge 1: incorperating domain knowledge into the RL algorithm and reward engineering to reduce noise

% Challenge 2: mutiple intervention compoent each operating at different time scale. We develop a MARL algorithm where each agent operates at different time scale. 

% Challenge 3: interpretability. We take advantage of the MARL, so each agent is easier to interpret. we use relatively simple model. 

\begin{enumerate}
% \item \textbf{Incorporating domain knowledge }

% \item \textbf{Accelerating learning in noisy, data-limited settings.} Early-recruited dyads have limited available data, while the observations are inherently noisy due to the variability of human behaviors. Developing an algorithm that can learn quickly in such conditions is critical to benefit as many dyads as possible.
% \hinal{Alternate Challenge 1: Incorporate domain knowledge into the RL algorithm and reward engineering to reduce noise}
\item \textbf{Managing multiple intervention decisions across different multi-scales.} There are three intervention components, each requiring decisions to be made at a different time scales. 
The decision-making occurs twice daily for AYAs, daily for care partners, and weekly for the relationship component. Making decisions on multiple timescales complicates the  algorithm design.
\item \textbf{Accelerating learning in noisy, data-limited settings.} 
Observed data in  digital intervention deployment is quite noisy \cite{trella2022designing}. Furthermore, limited data will be available to support in decision making for dyads recruited early in the clinical trial.  %will have limited available data. 
Additionally, less data is available for learning  decisions that occur at slower timescales. These factors necessitate a sample-efficient algorithm that learns faster given limited data.
%  Incorporating the expertise of domain scientists, and designing custom reward functions to each component are handy for reducing noise in such environments.  

% \item \textbf{Interpretability.} Interpretability is essential for domain experts to evaluate the algorithm's rationale, incorporate domain knowledge for algorithm design. Additionally, interpretable algorithms are more likely to be adopted in real clinical practice, where trust and understanding are important.
\end{enumerate}

% \ziping{Novel setting is the social network.}
% \ziping{Leave out hierarchical}

% \ziping{Directly reference the challenges.}

% \vspace{-3mm}

\subsubsection{Contribution.} Our contribution is a novel  multi-agent RL (MARL) framework involving three RL agents, where each agent is responsible for making decisions for one specific intervention component and operates at the timescale corresponding to its intervention component timescale, which directly addresses challenge (1) about multi-scale decision-making.
% Each agent operates at its respective time scale (twice-daily, daily, and weekly), which directly addresses challenge (1). 

The use of MARL decouples the decision processes of different intervention components, thus  improving  interpretability of the agent model design. This improved interpretability allows us to incorporate domain knowledge into the agent-specific algorithm designs to address challenge (2). To further accelerate learning, we propose a novel \textbf{reward engineering} method that learns a less noisy surrogate reward function for each component. Through evaluation in a carefully designed dyadic environment, we demonstrate both the superior performance of our proposed algorithm and strong collaborative behavior among the three agents. Lack of collaboration is often a critical issue in MARL \cite{oroojlooy2023review}.
% \sam{it would be great if we have a cite for this.} 

% In the meantime, MARL decouples the decision processes of different intervention component, which improves interpretability and allows us to incorporate domain knowledge into the agent-specific algorithm designs. 
% allowing agent-specific designs, such as tailored \textbf{reward functions} based on domain knowledge, to faciliatate faster learning and reduce noise. 
% Second, it enhances interpretability by decoupling the decision-making processes of individual agents.


% This addresses challenge (2) because it enables us to take advantage of domain expertise in designing the reward function and, hence, reduces noise. Along the way, the reward engineering also enhance collaboration, which accounts for the potential non-collaboration among three agents that may arise from multiple, individual agents.   
% , which accounts for the potential non-collaboration among three agents.
% We note further that the implementation of domain knowledge into the surrogate functions is enabled through the multi-agent choice because it decouples the decision making and provides an interpretability advantage that facilitates communication with the domain experts to obtain said knowledge. 


% \ziping{Use the reward engineering (reduce noise).}

% The interpretability not only clarifies the algorithm's rationale but
\section{RL Framework and Domain Knowledge}

We start with formulating the intervention decision making as an RL problem, where we underscore the challenge in the multiple time scales.
% We start with introducing the important variables in each component, and then formulate the clinical problem as a multi-agent reinforcement learning (MARL) problem.
% \subsection{Variables}
% In our study, each user stays for 98 days with 196 decision times in total. We first introduce important variables we plan to collect or optimize in ADAPTS-HCT. The variables are organized for each of the three components with a summary provided in Table \ref{tab:variables}.
%
% AYA component has twice-daily observations for $t=1,\dots,196$ decision times. On each $t$, we observe three variables $R_{t}^{\AYA}, A_{t}^{\AYA}, B_{t}^{\AYA}$ where $R_{t}^{\AYA} \in \{0, 1\}$ with $R_{t}^{\AYA} = 1$ representing AYA's adherence to medication at the decision time $t$. $A_t^{\AYA} \in \{0, 1\}$ represents the digital intervention provided to AYA with $A_t^{\AYA} = 1$ representing sending a positive psychology message. $B_t^{\AYA} \in \sR$ is AYA's perceived digital intervention burden about digital interventions at the time $t$.
%
% The care partner component observes three daily variables $R_{d}^{\CARE}, A_{d}^{\CARE}, B_{d}^{\CARE}$ for time index $d = 1, \dots, 98$. Here $R_d^{\CARE} \in \sR$ is the \textbf{psychological distress level} of the care partner (a continuous variable) at the end of day $d$. The psychological distress is measured through daily self-report. $A_d^{\CARE} \in \{0, 1\}$ represents the digital intervention provided to the care partner with $A_d^{\CARE} = 1$ representing sending a positive psychology message. Similar to the AYA component, $B_d^{\CARE} \in \sR$ is the continuous variable representing the level of the care partner's perceived digital intervention burden on day $d$. A higher $B_d^{\CARE}$ represents a higher perceived burden.
%
% The relationship component encompasses two weekly variables $A_{w}^{\EDGE}$ and $Y_{w}^{\EDGE}$ for time index $w = 1, \dots, 14$. Here $A_{w}^{\EDGE} \in \{0, 1\}$ is thegame intervention representing whether the dyad is offered to play a game at the beginning of a week, and $Y_{w}^{\EDGE} \in \{0, 1\}$ indicates the relationship quality between AYA's and their care partners measured in the end of the week through self-report. An $Y_{w}^{\EDGE} = 1$ indicates good relationship.
%
%
\begin{table}[pht]
    \centering
    \caption{Summary of variables about each target component}
    \label{tab:variables}
    \begin{tabular}{c|c|c|c}
    \toprule
    Target & Variable & Type & Description \\
    \midrule
    AYA & \makecell{$R_{w,d,t}^{\AYA}$ \\ $A_{w,d,t}^{\AYA}$ \\ $B_{w,d,t}^{\AYA}$} & \makecell{binary \\ binary \\ continuous} & \makecell{Medication adherence at time $t$ on day $d$ in week $w$ \\ Intervention at time $t$ on day $d$ in week $w$ \\ App burden at time $t$ on day $d$ in week $w$} \\
    \midrule
    Care partner & \makecell{$Y_{w,d}^{\CARE}$ \\ $A_{w,d}^{\CARE}$ \\ $B_{w,d}^{\CARE}$} & \makecell{continuous \\ binary \\ continuous} & \makecell{Psychological distress on day $d$ in week $w$ \\ Intervention on day $d$ in week $w$ \\ App burden on day $d$ in week $w$} \\
    \midrule
    Relationship & \makecell{$Y_w^{\EDGE}$ \\ $A_w^{\EDGE}$} & \makecell{binary \\ binary} & \makecell{Relationship quality at the end of week $w$ \\ Game intervention at the beginning of week $w$} \\
    \bottomrule
    % \vspace{10mm}
    \end{tabular}
    \end{table}
%
% \vspace{40mm}
%
%
%\sam{prior sentence was very unclear} 
HCT treatment is followed by  an outpatient 14-weeks twice-daily medication regimen. Decision times within the 14 weeks are denoted by $(w, d, t)$ where $w \in \{1, \dots, 14\}$ is the week index, $d \in \{1, \dots, 7\}$ is the day index, and $t \in \{1, 2\}$ is the decision window within a day. 
% In Table \ref{tab:variables}, we summarize the information collected from each component. 

\textbf{Primary goal.} The primary goal is to make decisions at each decision time $t$ to maximize cumulative sum of medication adherence $\sum_{w=1}^{14}\sum_{d=1}^{7}\sum_{t=1}^{2}R_{w,d,t}^{\AYA}$, where $R_{w,d,t}^{\AYA}$ is medication adherence at window $t$ on day $d$ in week $w$. See Table \ref{tab:variables}, for selected information that will be collected on the dyad. 

% A key challenge is the multiple time scales in both the action space and the observation space, which is visualized in Fig. \ref{fig:rl_framework}. 
% Specifically, both the action space and the observation space are different at different time scales.

\textbf{Action space.} All actions are binary (deliver versus do not deliver intervention content); see Tables~\ref{tab:interventions},\ref{tab:variables}. When the current time $(d = 1, t = 1)$ is the first decision time on the first day of the week, the agent chooses a three-dimensional action corresponding to all three interventions components. If the current time is the first time  on a day after the first day of the week $(d> 1, t = 1)$, the agent chooses a two-dimensional action corresponding to only the AYA intervention and the care partner components. At the second time on each day ($t=2$) the agent chooses a one-dimensional action corresponding to only the AYA intervention component. % We denote this dynamic action space at time $(w, d, t)$ as $\mathcal{A}_{w,d,t}$.

\textbf{Observation space.} Apart from the dynamic action space, we collect observations about different components at different time scales as well; see Table \ref{tab:variables}. At each time $(w, d, t)$, we collect the current medication adherence and digital intervention burden from the AYA component. In the end of each day $d$, we collect the psychological distress and digital intervention burden from the care partner component. In the end of a week $w$, we collect the relationship quality from questionnaires from both the AYA and the care partner.  

% \vspace{-3mm}

\section{Domain Knowledge through Causal Diagram} 

% \vspace{-3mm}

\begin{figure}[hpt]
\includegraphics[width=1\textwidth]{Plots/Causal_DAG/DAG_Dyad_Anotation.pdf}
    % \vspace{-0.8cm}
    \captionsetup{parskip=0pt}
    \caption{Causal diagram for ADAPTS-HCT intervention \textsuperscript{1}. We categorize the variables into three components: AYA component (marked in \textbf{black}), care partner component (marked in \textcolor{redorange}{red}), and relationship component (marked in \textcolor{grassgreen}{green}). Each component operates at different time scales. Variables in the AYA component evolve on a twice-daily basis, while the care partner component operates on a daily basis. The relationship component operates on a weekly basis. The arrows indicate the direct causal effects.}
    \vspace{5mm}
    \small\textsuperscript{1} In the causal inference literature, this is called a causal Directed Acyclic Graph (DAG), a graphical representation of causal relationships among a set of variables \cite{pearl2000models}.
    \label{fig:dag}
    % \vspace{-5mm}
    % \end{subfigure}
    % \caption{The RL framework and the causal diagram for ADAPTS-HCT intervention. \ziping{Call the middle one Relationship component.}
\end{figure}


% \ziping{This is a simplified causal graph. The variables that are most likely to have effects.}

% \ziping{We see the pathway in which the interventions can impact the adherence.}

% \ziping{xxx is expected to improve xxx, which in turn may decrease aya adherence.}
Our algorithm design is guided by domain knowledge encoded as the causal diagram in Fig. \ref{fig:dag}. 
% \sam{I think you should put a footnote about this being a causal DAG and reference Pearl--Daiqi did a good job referencing Pearl in her paper--check that paper out}. 
This diagram describes the scientific team's understanding of  the primary causal relationships between the variables in each component listed in Table \ref{tab:variables}.  Note that  the causal relationships are likely more complex and direct paths may exist between any two variables.  However the scientific team believes that these other paths are likely to be less detectable given the noise in digital intervention data.   We summarize the  primary pathways that interventions can take to effect the AYA's adherence in the following.  %these are not principles!
% We highlight that this is a simplified casual graph with arrows only between the key relationships. 

% Below we summarize the major mechanisms on the effect of interventions that guide the surrogate design.

% \ziping{I need some explanation here for each of the standpoints. See examples in the paper on Sepsis Treatment. Could Billie or Sung help us with this?}

% \subsubsection{Principle 1: conditional independence.} The variables in AYA component are independent of those in the care partner component conditioned on the relationship component (there is no direct arrows between the two components). Let $w(t)$ and $w(d)$ be the week index of time $t$ and day $d$ respectively. This principle favors the following feature construction choices: AYA feature at the time $t$ is $\phi_t^{\AYA} = \left(1, R_{t-1}^{\AYA}, B_{t-1}^{\AYA}, A_{w(t)}^{\EDGE}, R_{w(t)}^{\EDGE}, M_t\right)$, where $M_t = 1$ indicates that $t$ is a morning decision time, which ignores the care partner component variables. The care partner feature at the time $d$ is $\phi_d^{\CARE} = \left(1, R_{d-1}^{\CARE}, B_{d-1}^{\CARE}, A_{w(d)}^{\EDGE}, R_{w(d)}^{\EDGE}\right)$, which ignores the AYA component variables, and game feature at the time $w$ is $\phi_w^{\EDGE} = \left(1, B_{t(w)}^{\AYA}, B_{d(w)}^{\CARE}, Y_{w-1}^{\EDGE}, A_{w-1}^{\EDGE}, \bar{R}_{w-1}^{\CARE}, \bar{R}_{w-1}^{\AYA}\right)$. Here $\bar{R}_{w-1}^{\CARE}$ and $\bar{R}_{w-1}^{\AYA}$ are exponentially weighted average psychological distress and medication adherence in the week $w-1$.

% Fortunately, the monotonic mechanism should induce collaboration.
% \begin{itemize}
%     \item two variants: low signal pathway from rel to adh and stress to adh
%     \item stress to adh when relationship is good
% \end{itemize}
\begin{enumerate}
% \item \textbf{Principle 1:} The AYA component is conditional independent of the care partner component conditioned on the relationship component. This principle grounds the choice of the multi-agent algorithm we propose later.
\item \textbf{AYA intervention.} The AYA interventions $A_{w,d,t}^{\AYA}$ should directly influence the immediate AYA's adherence $R_{w,d,t}^{\AYA}$ (black arrows). 


\item \textbf{Game intervention.} The game intervention $A_{w}^{\EDGE}$ has two pathways by which it is expected to effect  AYA's adherence. First, $A_{w}^{\EDGE}$ is expected to increase the AYA's burden $B_{w,d,t}^{\AYA}$ throughout the week $w$.  And AYA's burden $B_{w,d,t}^{\AYA}$ is expected to decrease the AYA's adherence $R_{w,d,t}^{\AYA}$ (\textcolor{darkblue}{blue arrows}). Second, the game intervention $A_{w}^{\EDGE}$ is expected to effect   next week AYA's adherence $R_{w+1,d,t}^{\AYA}$ by improving the end of the week relationship quality $Y_{w}^{\EDGE}$ (\textcolor{grassgreen}{green arrows}). 

% \ziping{Remind them to look into the figure.}

% Relationship quality is known to have non-negative effect on the AYA's adherence. As shown in Fig. \ref{fig:dag}, the relationship quality and the AYA burden block all the paths from game intervention to AYA's adherence. 

\item 
\textbf{Care partner intervention.} The care-partner intervention $A_{w,d}^{\CARE}$ is expected to effect the AYA's adherence indirectly. First, $A_{w,d}^{\CARE}$ should decrease the care partner's psychological distress $Y_{w,d}^{\CARE}$, which should increase the end of week relationship quality $Y_{w}^{\EDGE}$ (\textcolor{yellow}{yellow arrows}). Second, $A_{w,d}^{\CARE}$ should increase the care partner's burden $B_{w,d}^{\CARE}$, which should decrease the the end of week relationship quality $Y_{w}^{\EDGE}$ (\textcolor{darkblue}{blue arrows}). 

% This principle favors the following reward function choice: $r_d^{\CARE} = -R_d^{\CARE}$ for the care partner agent.
\end{enumerate}

We further note that the variables from different components are generally independent conditioned on the bottleneck variables, e.g., the relationship quality that blocks all the paths from the care partner variables to the AYA's adherence. This forms the basis of our multi-agent RL design.


% \ziping{Mention that the three components are independent unless through bottleneck variables: e.g., }

% \vspace{-3mm}

\section{Proposed Multi-Agent RL Approach}

% A key observation from Fig. \ref{fig:dag} is that many variables from different components are independent conditioned on some key bottleneck variables for example, the relationship quality blocks all the paths from the care partner variables to the AYA's adherence. 
The conditional independence property observed from Fig. \ref{fig:dag} motivates us to design a multi-agent RL (MARL) comprising three agents: the AYA agent, the care partner agent, and the relationship agent. Each makes decisions at different time scales for their own component.

% The approach is hierarchical: the game agent's decision on the game intervention is included in the AYA and care partner agents' observation space. 

% Our goal is to learn quickly to benefit as many dyads as possible.
The MARL approach allows us to tailor the agent design choices for each agent to optimize the learning speed. Our base RL algorithm for each agent is Randomized Least Square Value Iteration (RLSVI) \cite{osband2016generalization}, which has been proven as stable in deployment of mobile health applications \cite{trella2024deployed,ghosh2024miwaves}. Additionally, we use linear models, which helps in discussions of the algorithm and its parameters with domain scientists. 

We construct agent-specific features based on Fig. \ref{fig:dag}. Specifically, the AYA agent's model uses its own variables ($B_{w, d, t}^{\AYA}, R_{w, d, t-1}^{\AYA}$) and the variables in the relationship component $(Y_{w-1}^{\EDGE}, A_{w}^{\EDGE})$. Similarly, the care partner agent uses its own variables, as well as the variables in the relationship component. The relationship agent's model uses $Y_{w-1}^{\EDGE}$, and previous weeks' $B_{w-1, 7, 2}^{\AYA}$, $B_{w-1, 7}^{\CARE}$, as well as a weighted average of AYA adherence and care partner distress in the past week.
% , an algorithm that explores the environment by acting optimally w.r.t a randomly sampled value function. 
% The sampling-based algorithm is shown to perform better in mobile health applications \cite{trella2024deployed,ghosh2024miwaves}. 
% \ziping{Stable in deployment, similar to Bayesian linear regression. Help with discussion with domain scientists. }

% Each agent optimizes the discounted sum of rewards with a discount factor $\gamma^{\AYA}= \gamma^{\CARE} = 0.5$ and $\gamma^{\EDGE} = 0$ for the AYA, care partner, and relationship agents, respectively. The discount factor $\gamma^{\EDGE} = 0$ is to reduce the learning complexity of the relationship agent due to limited data (14 points for one dyad). As we have more data for the AYA and care partner agents, we set $\gamma^{\AYA} = \gamma^{\CARE} = 0.5$.

% \vspace{-3mm}

\subsection{Surrogate Reward Function Design Through Domain Knowledge}


Typical MARL \cite{oroojlooy2023review} with independent learners considers agents making decisions without communication. In our study, the lack of communication is due to the different time scales--the relationship agent that makes decisions in the beginning of a week may not predict the AYA and care partner agents' decisions throughout the week. This may prevent the agents from \textbf{collaborating}. For example, the relationship agent may choose to always intervene so as to improve the relationship quality (the primary goal of the game intervention), which may not be optimal for the AYA's adherence.

Furthermore, the effects of care partner intervention and the game intervention are highly delayed. The game intervention improves end of week relationship quality with a significant delayed effect onto the adherence in the next week.
% The relationship agent's discount factor $\gamma^{\EDGE} = 0$ means that the decision-making focuses on its effect on the immediate reward instead of its delayed effect. 
The care partner intervention (positive messages for the care partner) is designed to mitigate the care partner's psychological distress, which only has indirect and delayed effects on the AYA's adherence. 

To address the above two issues, we engineer the reward function to account for the delayed effects and across-component effects of each intervention component to promot collaboration. Similar reward engineering in the context of digital interventions is discussed in \cite{trella2023reward}. Our approach is distinct in that we explore the principles for incorporating domain knowledge to guide the reward function design.

% Surrogate reward design is critical for the above RL framework. The game agent's discount factor $\gamma^{\EDGE} = 0$ means that the decision-making focuses on its effect on the immediate reward instead of its delayed effect. The care partner intervention (positive psychology messages to the care partner) is designed to mitigate the psychological distress, which only has indirect and delayed effects on the AYA's adherence. Therefore, we need to engineer the reward function to take into account the delayed effect of the game intervention and the care partner intervention. Similar reward engineering in the context of digital interventions is discussed in \cite{trella2023reward}. Our approach is distinct in that we explore the principles on incorporating domain knowledge to guide the reward function design.

% \ziping{Add short review on the reward engineering or reward shaping literature. Talk about why it is important to tailor the reward function for each agent here.}

% Design for the game agent:
% \begin{itemize}
%     \item First fit a linear model $f^{\EDGE}$ to predict the sum of medication adherence within a week using the $(A_{w}^{\EDGE}, Y_{w-1}^{\EDGE}, B_{t(w)-1}^{\AYA})$ as features.
%     \item Engineer the reward to take care the delayed effect: 
%     $$
%         r_w^{\EDGE} = f^{\EDGE}(A_{w}^{\EDGE}, Y_{w-1}^{\EDGE}, B_{t(w)-1}^{\AYA}) + \gamma \argmax_{a} f^{\EDGE}(a, Y_{w}^{\EDGE}, B_{t(w+1)-1}^{\AYA}).
%     $$
%     Here $\gamma$ is a tuning parameter.
% \end{itemize}

% Design for the care partner agent:
% \begin{itemize}
%     \item First fit a linear model $f^{\CARE}$ to predict .
%     \item The goal of the care partner agent is to minimize the psychological distress, while managing the digital intervention burden.
%           $$
%             r_d^{\CARE} = R_{t(d)}^{\AYA} + R_{t(d)+1}^{\AYA} - R_d^{\CARE} - \gamma B_{d}^{\CARE}
%           $$
% \end{itemize}


\textbf{Domain knowledge informed surrogate reward functions.} We introduce the surrogate reward functions for the relationship agent and the care partner agent. As informed by Fig. \ref{fig:dag}, the delayed effect of the game intervention is through the relationship quality and the AYA burden. 
% This motivates us to fit a linear model to predict the sum of medication adherence within week $w$, $\sum_{d=1}^{7} \sum_{t=1}^{2} R_{w,d,t}^{\AYA}$, using $(1, Y_{w-1}^{\EDGE}, {B}_{w, 1, 1}^{\AYA})^{\top}$ and their interactions with $A_{w}^{\EDGE}$ as the covariates. 
This motivates us to fit a linear model to predict the sum of medication adherence within week $w$, $\sum_{d=1}^{7} \sum_{t=1}^{2} R_{w,d,t}^{\AYA}$, using $(1, Y_{w-1}^{\EDGE}, {B}_{w, 1, 1}^{\AYA}, A_{w}^{\EDGE}, A_w^{\EDGE} \cdot Y_{w-1}^{\EDGE})$ as the covariates. 
% The working model is:
% \begin{align}
%     \sum_{d=1}^{7} \sum_{t=1}^{2} R_{w,d,t}^{\AYA} = & \vbeta_0^{\top} \bm{S}_w^{\EDGE} + \vbeta_1^{\top} \bm{S}_{w}^{\EDGE} \cdot A_{w}^{\EDGE} + \epsilon_w^{\EDGE},
% \end{align}
% where $\epsilon_w^{\EDGE}$ is the error term.
% which we denote as $\phi_{w}^{\EDGE}(a)$ with $A_w^{\EDGE} = a$.  
% \sam{DAG indicates that predictors are from prior week, $w-1$.   ${B}_{w, 1, 1}^{\CARE}$ does not have 3 subscripts. Should be ${B}_{w-1, 1}^{\CARE}$ }
% Similar notation is used for the care partner burden. 
To account for the delayed effect, we engineer the surrogate reward function for the relationship agent as:
% \begin{align}
% r_w^{\EDGE} = & (1, Y_{w-1}^{\EDGE}, {B}_{w, 1, 1}^{\AYA})\vbeta_0 + (1, Y_{w-1}^{\EDGE}, {B}_{w, 1, 1}^{\AYA})\vbeta_1 \cdot A_{w}^{\EDGE} + \nonumber \\ 
% & (1, Y_{w}^{\EDGE}, {B}_{w+1, 1, 1}^{\AYA}) \vbeta_0 +  \max_{a \in \{0, 1\}} (1, Y_{w}^{\EDGE}, {B}_{w+1, 1, 1}^{\AYA}) \vbeta_1 \cdot a, \label{equ:game_rwd}
% \end{align}
$r_{w}^{\EDGE} = (1, Y_{w-1}^{\EDGE}, {B}_{w, 1, 1}^{\AYA}, A_{w}^{\EDGE}, A_w^{\EDGE} \cdot Y_{w-1}^{\EDGE})\vbeta^{\EDGE}  
 + \max_{a}(1, Y_{w}^{\EDGE}, {B}_{w+1, 1, 1}^{\AYA}, a, a \cdot Y_{w}^{\EDGE}) \vbeta^{\EDGE},$ where $\vbeta^{\EDGE} \in \sR^{5}$ are Bayesian linear regression estimates. 
 \footnote{We choose the prior mean to reflect our guesses on the sign the coefficients. The prior variance is chosen so the prior mean dominates until around the 5th dyads. The complete prior is provided in Appendix.} 
% \end{align} \sam{above formula is too much like math.  Just write out the linear formula.  Will be easier on reader!  I'm not sure if the reader will realize that the label is the number of 1/2 days during the week $w$ for which the AYA adhered.}
The above reward yields a two-step greedy policy, which is a good enough approximation for the total sum of the medication adherence. We opt for a simple, linear model here because the bias trade-off is justified by the faster learning and reduction in noise. 
% \ziping{We prefer simpler models in order to learn fast to trade-off bias and variance.}

The design of the care partner agent is similar. A key observation is that the end of the week relationship quality blocks all the paths from the care partner variables to the AYA's adherence. Thus, we fit a linear model to predict the end of week relationship quality $Y_{w+1}^{\EDGE}$ using $(1, Y_{w,d}^{\CARE}, B_{w,d+1}^{\CARE}, Y_{w-1}^{\EDGE}, A_{w,d}^{\CARE})$ as covariates. The surrogate reward function is:
    $r_{w, d}^{\CARE} =  (1, Y_{w,d}^{\CARE}, B_{w,d+1}^{\CARE}, Y_{w-1}^{\EDGE}, A_{w,d}^{\CARE}) \vbeta^{\CARE}$, 
where $\vbeta^{\CARE} \in \sR^{5}$ are Bayesian linear regression estimates. 
% We update the estimates in real-time based on the current data.
% We denote the covariates as $\phi_{w,d}^{\CARE}$. Then we engineer the reward:
% \begin{align}
% r_{w,d}^{\CARE} = f^{\CARE}(\phi_{w,d}^{\CARE}).
% \end{align}
% \sam{ again no reason for confusing notation.} 
% The care partner agent is essentially maximizing the end of the week relationship quality, and the collaboration between agents is guaranteed due to the monotonic relationship between the end of the week relationship quality and the AYA's adherence.


% \paragraph{Informative prior.} We use the domain knowledge to construct the informative prior for the above regression models to accelerate learning. The prior means of $f^{\EDGE}$ and $f^{\CARE}$ are set to be $\pm 1$ for the main effects and $\pm 0.5$ for the interactions terms with the signs based on our guesses on whether the variables are positively or negatively correlated with the dependent variable. The prior variance are identity matrices $I_6 \sigma^2_{\EDGE}$ and $I_8 \sigma^2_{\CARE}$ with $\sigma^2_{\EDGE} = \sigma^2_{\CARE} = 0.1$. The value 0.1 is chosen so the prior mean dominates until around the 5th dyads. The complete prior is shown in Table \ref{tab:prior} in Appendix \ref{app:algo}.

% \subsubsection{Other designs.} The designs on base algorithm and pooling are not directly informed by the domain knowledge. We will make algorithm selection based on the simulation results. The first kind is the choice of the base algorithm. The base algorithm is about the exploration strategy, which affects the learning speed and the ability to uncover the optimal policy. The base algorithm also determines the planning horizon, which is about how far the agent looks ahead to make decisions. Specifically, we consider the following choices:
% \begin{enumerate}
%     \item Thompson Sampling (TS) for Bandit \cite{russo2018tutorial}: TS that only maximizes the immediate reward. Since taking account for the delayed effect is important, we always add a penalty on the app burden onto the immediate reward.
%     \item Infinite-horizon RLSVI: RLSVI that maximizes the exponentially discounted sum of medication adherence over infinite horizon \cite{osband2016generalization}.
%     \item Finite-horizon RLSVI: RLSVI that maximizes the sum of medication adherence within a week \cite{osband2016generalization}.
% \end{enumerate}
% % \ziping{find references for these algorithms. Why do we choose them?}

% Another important design choice is whether we should pool data across dyads. Pooling data across dyads is a common practice in RL for digital interventions \cite{trella2024deployed} to overcome the challenge of small sample size.
% We consider the following choices:
% \begin{enumerate}
%     \item No pooling across dyads: agent makes decisions for each dyad independently.
%     \item Pooling across dyads: agent pools data across dyads to learn a single model.
%     \item Empirical Bayes? \ziping{let us discuss whether we want to include this.}
% \end{enumerate}
% \subsection{Overview of the Simulated Dyadic Environment}

% \ziping{Give a brief description of the testbed.}

We construct a dyadic simulation environment to evaluate the performance of the proposed algorithm. The 1st order goal of the environment design is to replicate the noise level and structure that we expect to encounter in the forthcoming ADAPTS-HCT clinical trial. This noise often encompasses the stochasticity in the state transition of each participant and the heterogeneity across participants.

The environment is based on Roadmap 2.0, a care partner-facing mobile health application that provides daily positive psychology interventions to the care partner only. Roadmap 2.0 involves 171 dyads, each consisting of a patient undergone HCT (target person) and a care partner. Each participant in the dyad had the Roadmap mobile app on their smartphone and wore a Fitbit wrist tracker. The Fitbit wrist tracker recorded physical activity, heart rate, and sleep patterns. Furthermore, each participant was asked to self-report their mood via the Roadmap app every evening. A list of variables in Roadmap 2.0 is reported in Table \ref{tab:roadmap_variable}.

Roadmap 2.0 data is suitable for constructing a dyadic environment for developing the RL algorithm for ADAPTS-HCT in that Roadmap 2.0 has the same dyadic structure about the participants--post-HCT cancer patients and their care partner. Moreover, Roadmap 2.0 encompasses some context variables that align with those to be collected in ADAPTS-HCT, for example, the daily self-reported mood score.

\subsubsection{Overcoming impoverishment.} From the viewpoint of evaluating dyadic RL algorithms, this data is impoverished \cite{trella2022designing} mainly in two aspects. First, Roadmap 2.0 does not include micro-randomized daily or weekly intervention actions (i.e., whether to send a positive psychology message to the patient/care partner and whether to engage the dyad into a weekly game). Second, it does not include observations on the adherence to the medication--the primary reward signal, as well as other important measurements such as the strength of relationship quality. 
% Furthermore, Roadmap 2.0 includes dyads across all lifespan whereas ADAPTS-HCT will focus on adolescent and young adults.

To overcome this impoverishment, we construct surrogate variables from the Roadmap 2.0 data to represent the variables intended to be collected in ADAPTS-HCT. A list of substitutes is reported in Table \ref{tab:roadmap_substitutes}. Worthnoting, the substitute for the AYA medication adherence is based on the step count. There is evidence on the association between the step count and the adherence. 

% \ziping{Could we find literature to justify this?}
% \ziping{Should we discuss in detail the rationale of these substitutes?} 
We further impute the treatment effects of the intervention actions so the marginal effects after normalization, which we call the standardized treatment effects (STE), are around 0.15, 0.3, and 0.5, corresponding to small, medium, and large effect sizes in typical behavioral science studies.

\subsubsection{Constructing the dyadic environment.} We follow the environment design in \cite{li2023dyadic}, which also uses the Roadmap 2.0 data, but primarily focuses on AYA intervention and relationship intervention. We extend the environment to include the care partner intervention. Specifically, we fit a separate multi-variate linear model for each participant in the dataset with the AR(1) working correlation using the generalized estimating equation (GEE) approach \cite{ziegler2010generalized,hojsgaard2006r}. We impute the treatment effects of the intervention actions based on the typical STE around 0.15, 0.3, and 0.5, which completes a generative model for the state transitions. The environment simulates a trial by randomly sampling dyads from the dataset, and simulate their trajectories based on the actions selected by the RL algorithm. The environment details are described in Appendix \ref{app:testbed}. Our experiments primarily focus on the three vanilla testbeds corresponding to the three STEs.


\section{Results}
\label{sec:results}

In this section we evaluate 
the results of 
morphological pretraining (Sect.~\ref{results-pretraining-performance}),
zero- and few shot evolution (using the pretrained model; Sect.~\ref{results-zero-and-few-shot-evolution}),
and simultaneous co-design from scratch (without pretraining; Sect.~\ref{results-simultaneous-co-design}).


\subsection{Pretraining performance}
\label{results-pretraining-performance}

Across three independent trials,
each using a distinct dataset of 
randomly-generated morphologies and environments,
pretraining 
exhibited stable learning trajectories 
with low variance across trials (Fig.~\ref{fig:results-performance}A), 
converging in approximately 1,400 learning steps (56 minutes of wall-clock time).
% 
Loss was defined as the ratio of final to initial distance from the target light source. 
At initialization with random controller weights, this ratio was 1.0, indicating robots remained stationary throughout simulation. 
After pretraining, the loss stabilized at approximately 0.3, representing a 70\% improvement. 
That is, in environments sampled from the training distribution, robots using the pretrained universal controller traversed an average of 70\% of their initial distance to the light source. 
Since each training batch used novel morphologies, we omitted model selection with a \mbox{validation set.}

% The pretrained model effectively controlled diverse, previously unseen robot morphologies in challenging tasks at the training distribution's edge (Appx.~\ref{appendix-dataset-random-robot-generation}). 
% Fig.~\ref{fig:results-zero-shot-transfer} shows the pretrained model's generalization performance across more than 24,000 distinct test morphologies. 
% Despite the test environments being more challenging than the training distribution, the mean performance of the test morphologies aligned with the final loss observed during pretraining. 

To visualize the breadth of morphological diversity handled by the pretrained controller, Fig.~\ref{fig:appendix-zero-shot-robot-grid} showcases a representative sample of successful robots. 
These examples were selected uniformly from the top-performing 50\% of the test morphology set. 
The selected bodies exhibit high variation in both scale and morphological characteristics demonstrating the non-trivial generalization of the universal controller.


%%%%%%%%%%%%%%%%%%%%%%%%%%%%%%%%%%%%%%%%%%%
\begin{figure}[t]
    \centering
    \includegraphics[width=\columnwidth]{figs/shape-stats.pdf}
    \vspace{-20pt}
    \caption{\textbf{Evolved populations.}
    Population performance, phenotype footprint size, and body mass for the initial (randomly generated) and evolved design populations.
    }
    \vspace{-18pt}
    \label{fig:appendix-robot-data-gen-stats}
\end{figure}
%%%%%%%%%%%%%%%%%%%%%%%%%%%%%%%%%%%%%%%%%%%

\subsection{Zero- and few shot evolution (with pretraining)}
\label{results-zero-and-few-shot-evolution}

A population of morphologies was evolved through random mutation and crossover operations, using the pretrained universal controller.
On the same challenging set of tasks used for evaluating pretrained controller generalization,
the population converges to near optimal performance
in 100 generations of evolution (17 minutes of wall-clock time)
without finetuning the controller (``zero shot evolution''; Sect.~\ref{methods-zero-shot-evolution}).
% 
Although zero-shot evolution shows rapid convergence in controlling thousands of distinct bodies, this success masks a key pattern: design population diversity decreases as performance improves. 
Fig.~\ref{fig:results-performance}C reveals this pattern---after a brief diversity spike at evolution's onset, the population gradually homogenizes. 
We term this phenomenon diversity collapse, measuring diversity as the population's mean, pairwise Hamming distance in (and normalized to) the genotype space $\mathcal{G}$ (defined in Sect.~\ref{methods-morphology-design-space}). This metric naturally reflects differences in morphology (body) as well as sensing and actuation masking in the universal controller (brain).

We found that generational finetuning of the universal controller for the current population (``few shot evolution'')
not only preserves diversity but in fact significantly increases diversity (Fig.~\ref{fig:results-performance}D).
This is a somewhat surprising result as there was no explicit selection pressure to maintain diversity.
The process of morphological evolution seems to intrinsically increase population diversity. 
However, in absence of generational finetuning, there is a tipping point at which it is easier to purge diversity, replacing the worst designs with slightly modified clones of the best, than to discover novel morphological innovations with superior performance.


\subsection{Simultaneous co-design (\textit{without} pretraining)}
\label{results-simultaneous-co-design}

Ablating pretraining (and funetuning),
and instead simultaneously optimizing morphology and universal control, together from scratch, 
results once again in rapid diversity collapse (Fig.~\ref{fig:results-performance}B).
Performance plateaus in well under 180 generations, corresponding to 360 controller learning steps and 109 minutes of wall-clock time.
% 
The extent of diversity collapse can be seen in Fig.~\ref{fig:results-morpho-grid}B, where we visualize morphologies from one of the three independent trials,
and in Fig.~\ref{fig:appendix-robot-data-gen-stats} where we plot morphological variance across evolved populations in terms of footprint size and body weight.

In all three co-design paradigms (zero shot, few shot, simultaneous), universal control enabled successful crossover (Fig.~\ref{fig:appendix-variation-operator-stats}).
In terms of offspring survival,
crossover was initially much more successful than mutation.
But in the case of simultaneous co-design, this was not an apples to apples comparison because each generation provided the controller with more time to learn how to control the population, and the randomly initialized controller was very bad at the task.
And so it was not clear if the success of offspring was due to changes in parent morphology or improvements to the universal controller.
The superior performance of pretraining across random morphologies, shows that the designs produced by crossover during simultaneous co-design were no better than random designs. 
In zero-shot and few-shot evolution, however, the pretrained controller is quite good at the very start, and in zero-shot the controller is not updated during evolution, providing clear evidence of successful crossover prior to diversity collapse.


% \input{Sections/4_Conclusion.tex}



\bibliographystyle{plain}   
\bibliography{main.bib}

\newpage

\appendix


\section{Related Work}

Below we summarize the most relevant literature from both the medical lens and the algorithm lens.

\textbf{RL on social networks.} We design and implement RL on dyads that are small social networks in this paper. Existing works on RL on social networks are mostly focused on maximizing social influence or opinion spreading \cite{wang2021reinforcement,he2021reinforcement,yang2024balanced} with large scale social networks in mind. These problems are usually formulated as a constrained Markov Decision Process (CMDP) \cite{yang2024balanced}, where the goal is to allocate incentive to maximize the social influence or opinion spreading. Our focuses are on the challenges in the multi-scale decision making and the design of the RL algorithms that incorporate domain knowledge about the social networks. These differences make our algorithm designs unique contributions to the literature.

\textbf{Dyadic structure in health care.} Social relationships between patients and carepartners are proven to be important in many critical health outcomes. Studies have shown that the patient-caregiver dyad functions as a unit, with the well-being and coping strategies of one member significantly impacting the other \cite{shin2018supporting,mcpherson2024dyadic}. The quality of this relationship can affect treatment outcomes such as medication adherence \cite{psihogios2021understanding,kostalova2022medication,gresch2017medication}, and chronic disease management \cite{visintini2023medication,li2024usability}.

\textbf{Multi-agent RL (MARL).} Our proposed approach falls into the range of the independent learners in the MARL literature \cite{oroojlooy2023review}. Previous literature on MARL in a collaborative game focuses on finding the (approximate) Nash equilibrium of the game through interacting with an unknown environment \cite{wang2022cooperative,jin2021v}. However, in our paper, we emphasize the advantage of MARL in terms of its strong interpretability and being able to make decisions in multiple time-scales.

\section{Algorithm Details}
\label{app:algo}

We provide the complete details of the proposed \texttt{MultiAgent+SurrogateRwd} algorithm as well as the baseline \texttt{SingleAgent} algorithm.

We first introduce the infinite horizon RLSVI (Randomized Least Squares Value Iteration) algorithm in Alg. \ref{alg:base} \cite{russo2018tutorial}. This algorithm is a model-free posterior sampling approach that samples a random value function from its posterior distribution, and the agent acts greedily with respect to the sampled value function. We use the infinite horizon variant of RLSVI, which perturbs the Bayesian regression parameters with a random noise $\omega'$ (line 4). We introduce temporal correlation between the current noise $w'$ and the previous noise $w$ to introduce persistence in exploration.

\begin{algorithm}[H]
    \caption{Infinite Horizon RLSVI (Inf-RLSVI)}
        \begin{algorithmic}[1]
            \STATE{Input:} discount factor $\gamma \in \mathbb{R}$, previous dataset $\mathcal{D} = (s_i, a_i, r_i)_{i = 1}^{n-1} \cup \{s_{n}\}$, previous perturbation $w \in \mathbb{R}^d$, feature mapping $\phi: \mathcal{S} \times \mathcal{A} \mapsto \mathbb{R}^d$, previous parameter $\theta \in \mathbb{R}^{d}$
            \STATE Generate regression matrix and vector
            $$
                X \leftarrow\left[\begin{array}{c}
                \phi\left(s_1, a_1\right) \\
                \vdots \\
                \phi\left(s_{n-1}, a_{n-1}\right)
                \end{array}\right] \quad y \leftarrow\left[\begin{array}{c}
                r_1+\gamma \max _{\alpha \in \mathcal{A}} \langle \phi(s_2, \alpha), \theta \rangle \\
                \vdots \\
                r_{n-1}+\gamma \max _{\alpha \in \mathcal{A}}\langle \phi(s_{n}, \alpha), \theta \rangle
                \end{array}\right]
            $$
            \STATE Estimate value function
            $$
                \bar{\theta} \leftarrow \frac{1}{\sigma^2}\left(\frac{1}{\sigma^2} X^{\top} X+\lambda I\right)^{-1} X^{\top} y \quad \mathbf{\Sigma} \leftarrow\left(\frac{1}{\sigma^2} X^{\top} X+\lambda I\right)^{-1}
            $$
            \STATE Sample $w' \sim \mathcal{N}(\gamma w, (1-\gamma^2) \mathbf{\Sigma})$ and set $\theta' = \bar \theta + w'$
            \STATE \textbf{Output:} $\theta'$ and $w'$
            % \State Choose action $A_t = \argmax_{\alpha} \langle \phi(s_{t}, \alpha), \theta_{t} \rangle$
        \end{algorithmic}
        \label{alg:base}
\end{algorithm}

We use the same hyperparameters $\lambda = 0.75$ and $\sigma = 0.5$ for all the algorithms, which achieves an overall good performance for all the algorithms.

\textbf{Additional notation.} We use $w, d, t$ to denote the week, day, and time of the decision. When we increment the time, we use $w, d, t+1$ to denote the next decisioin time right after $w, d, t$, and $w, d, t-1$ to denote the previous decision time right before $w, d, t$. Note that if $t = 1$, then $w, d, t-1$ is the evening decision time of the previous day.

\subsection{Single Agent Algorithm}

Our \texttt{SingleAgent} algorithm runs the RLSVI algorithm in Alg. \ref{alg:base} using the all the obervations available at time $w, d, t$ as the state variable:
$$
    S_{w, d, t} = 
    \left(Y_{w, d-1}^{\CARE}, Y_{w-1}^{\EDGE}, R_{w, d, t-1}^{\AYA}, \bar{Y}_{w-1}^{\AYA}, \bar{Y}_{w-1}^{\CARE}, B_{w, d, t}^{\AYA}, B_{w, d, t}^{\CARE}, A_{w, d}^{\CARE}, A_{w}^{\EDGE}\right) \in \mathbb{R}^{9}.
$$
Here we slightly abuse the notation by using $R_{w, d, t-1}^{\AYA}$ to represent the AYA adherence at half-day decision time prior to the current decision time $w, d, t$. This means that if $t = 1$, a morning decision time, then $R_{w, d, t-1}^{\AYA}$ is the AYA adherence at the previous night.

The \texttt{SingleAgent} algorithm has the three dimensional action space $\vec{a} = (a_1, a_2, a_3)^{\top} \in \{0, 1\}^{3}$, each entry corresponding to one of the three interventions. The second action $a_2$  will only be effective on a new day and the third action $a_3$ will only be effective on a new week. The feature mapping $\phi$ for the single agent algorithm is defined as
$$
    \phi(s, \vec{a}) = (1, s, a_1, a_2, a_3, s \cdot a_1, s \cdot a_2, s \cdot a_3) \in \mathbb{R}^{40}.
$$

\begin{algorithm}[hpt]
    \caption{\texttt{SingleAgent} Algorithm}
    \begin{algorithmic}[1]
        \STATE{Input:} discount factor $\gamma = 0.5$
        \STATE{Initialize:} $\theta_{1,1,1} = \mathbf{0} \in \mathbb{R}^{40}$; dataset $\mathcal{D}_{1,1,1} = \emptyset$
        \FOR{$w = 1, 2, \dots, 14$}
            \FOR{$d = 1, 2, \dots, 7$}
                \FOR{$t = 1, 2$}
                    \STATE{Call Algorithm \ref{alg:base} and update $\theta_{w,d,t}$}
                    \STATE{$\vec{a} = \argmax_{\alpha} \langle \phi(S_{w,d,t}, \alpha), \theta_{w,d,t} \rangle$}
                    \IF{$t = 1$ and $d = 1$ (New Week)}
                        \STATE{Set $A_{w}^{\EDGE} = \vec{a}_3$}
                    \ENDIF
                    \IF{$t = 1$ (New Day)}
                        \STATE{Set $A_{w,d}^{\CARE} = \vec{a}_2$}
                    \ENDIF
                    \STATE{Set $A_{w,d,t}^{\AYA} = \vec{a}_1$}
                    \STATE{Environment generates $R_{w,d,t}^{\AYA}$ and next state $S_{w,d,t+1}$}
                    \STATE{Update $\mathcal{D}_{w,d,t} = \mathcal{D}_{w,d,t-1} \cup \{(S_{w,d,t}, \vec{a}, R_{w,d,t}^{\AYA})\}$}
                \ENDFOR
            \ENDFOR
        \ENDFOR
    \end{algorithmic}
    \label{alg:single_agent}
\end{algorithm}

\subsection{MultiAgent Algorithm}

The \texttt{MultiAgent} algorithm runs an RLSVI agent for each of the three interventions. We use agent-specific feature mapping $\phi^{\AYA}, \phi^{\CARE}, \phi^{\EDGE}$ for the AYA, carepartner, and relationship agents, respectively. The state construction and the feature mapping for Q-value function are given by Table \ref{tab:state_feature}. The \texttt{MultiAgent} algorithm is described in Alg. \ref{alg:multi_agent}, where the carepartner and the relationship agents learns based on the naive rewards that are the sum of the AYA rewards over the day, and over the week, respectively (line 15 and line 18).

\begin{table}[hpt]
    \centering
    \caption{State and feature construction for the Q-value function by agent.}
    \label{tab:state_feature}
    \begin{tabular}{l|l}
        \toprule
        Agent & State or Feature Mapping \\
        \midrule
        AYA State & $S_{w, d, t}^{\AYA} = \left(R_{w, d, t-1}^{\AYA}, B_{w, d, t}^{\AYA}, Y_{w}^{\EDGE}, A_{w}^{\EDGE}\right) \in \mathbb{R}^{4}$ \\
        AYA Feature & $\phi^{\AYA}(s, a) = (1, s, a, s \cdot a) \in \mathbb{R}^{10}$ \\
        \midrule
        Carepartner State & $S_{w, d}^{\CARE} = \left(Y_{w, d-1}^{\CARE}, B_{w, d}^{\CARE}, Y_{w}^{\EDGE}, A_{w}^{\EDGE}\right) \in \mathbb{R}^{4}$ \\ 
        Carepartner Feature & $\phi^{\CARE}(s, a) = (1, s, a, s \cdot a) \in \mathbb{R}^{10}$ \\
        \midrule
        Relationship State & $S_{w}^{\EDGE} = \left(Y_{w-1}^{\EDGE}, B_{w, 1, 1}^{\AYA}, B_{w, 1}^{\CARE}, \bar{Y}_{w-1}^{\AYA}, \bar{Y}_{w-1}^{\CARE}\right) \in \mathbb{R}^{5}$ \\
        Relationship Feature & $\phi^{\EDGE}(s, a) = (1, s, a, s \cdot a) \in \mathbb{R}^{12}$ \\
        \bottomrule
    \end{tabular}
\end{table}


\begin{algorithm}[hpt]
    \caption{\texttt{MultiAgent} Algorithm}
    \begin{algorithmic}[1]
        \STATE{Input:} discount factor $\gamma^{\AYA} = 0.5$, $\gamma^{\CARE} = 0.5$, $\gamma^{\EDGE} = 0$
        \STATE{Initialize:} $\theta^{\AYA}_{1,1,1} = \mathbf{0} \in \mathbb{R}^{10}$; $\theta^{\CARE}_{1,1} = \mathbf{0} \in \mathbb{R}^{10}$; $\theta^{\EDGE}_{1} = \mathbf{0} \in \mathbb{R}^{12}$; dataset $\mathcal{D}_{1,1,1}^{\AYA} = \emptyset$; $\mathcal{D}_{1,1}^{\CARE} = \emptyset$; $\mathcal{D}_{1}^{\EDGE} = \emptyset$
        \FOR{$w = 1, 2, \dots, 14$}
            \STATE{Call Algorithm \ref{alg:base} using $\mathcal{D}_{w}^{\EDGE}, \gamma^{\EDGE}$, and update $\theta_{w}^{\EDGE}$}
            \STATE{Set $A_{w}^{\EDGE} = \argmax_{\alpha} \langle \phi^{\EDGE}(S_{w}^{\EDGE}, \alpha), \theta_{w}^{\EDGE} \rangle$}
            \FOR{$d = 1, 2, \dots, 7$}
                \STATE{Call Algorithm \ref{alg:base} using $\mathcal{D}_{w,d}^{\CARE}, \gamma^{\CARE}$, and update $\theta_{w,d}^{\CARE}$}
                \STATE{Set $A_{w,d}^{\CARE} = \argmax_{\alpha} \langle \phi^{\CARE}(S_{w,d}^{\CARE}, \alpha), \theta_{w,d}^{\CARE} \rangle$}
                \FOR{$t = 1, 2$}
                    \STATE{Call Algorithm \ref{alg:base} using $\mathcal{D}_{w,d,t}^{\AYA}, \gamma^{\AYA}$, and update $\theta_{w,d,t}^{\AYA}$}
                    \STATE{$A_{w,d,t}^{\AYA} = \argmax_{\alpha} \langle \phi^{\AYA}(S_{w,d,t}^{\AYA}, \alpha), \theta_{w,d,t}^{\AYA} \rangle$}
                    \STATE{Environment generates $R_{w,d,t}^{\AYA}$ and next state $S_{w,d,t+1}$}
                    \STATE{Update $\mathcal{D}_{w,d,t}^{\AYA} = \mathcal{D}_{w,d,t-1}^{\AYA} \cup \{(S_{w,d,t}^{\AYA}, A_{w,d,t}^{\AYA}, R_{w,d,t}^{\AYA})\}$}
                \ENDFOR
                \STATE{Compute care-partner reward $R_{w,d}^{\CARE} = \sum_{t = 1}^{2} R_{w,d,t}^{\AYA} / 2$} 
                \STATE{Update $\mathcal{D}_{w,d}^{\CARE} = \mathcal{D}_{w,d-1}^{\CARE} \cup \{(S_{w,d}^{\CARE}, A_{w,d}^{\CARE}, R_{w,d}^{\CARE})\}$}
            \ENDFOR
            \STATE{Compute relationship reward $R_{w}^{\EDGE} = \sum_{d = 1}^{7} R_{w,d}^{\CARE} / 7$}
            \STATE{Update $\mathcal{D}_{w}^{\EDGE} = \mathcal{D}_{w-1}^{\EDGE} \cup \{(S_{w}^{\EDGE}, A_{w}^{\EDGE}, R_{w}^{\EDGE})\}$}
        \ENDFOR
    \end{algorithmic}
    \label{alg:multi_agent}
\end{algorithm}

The \texttt{MultiAgent+SurrogateRwd} algorithm is described in Alg. \ref{alg:multi_agent_surrogate}. The only difference between the \texttt{MultiAgent} and \texttt{MultiAgent+SurrogateRwd} is that the later agent optimizes the surrogate reward functions, defined in Equ. (\ref{equ:game_rwd_app}) and Equ. (\ref{equ:care_rwd_app}), where the coefficients are estimated using Bayesian Ridge Regression, with the prior mean given in Table \ref{tab:prior}.

\begin{align}
    r_w^{\EDGE} = & (1, Y_{w-1}^{\EDGE}, {B}_{w, 1, 1}^{\AYA}, A_{w}^{\EDGE}, A_w^{\EDGE} \cdot Y_{w-1}^{\EDGE})\vbeta^{\EDGE} \nonumber \\
     &+ \max_{a \in \{0, 1\}}(1, Y_{w}^{\EDGE}, {B}_{w+1, 1, 1}^{\AYA}, a, a \cdot Y_{w}^{\EDGE}) \vbeta^{\EDGE}, \label{equ:game_rwd_app}
\end{align}

\begin{align}
        r_{w, d}^{\CARE} =  (1, Y_{w,d}^{\CARE}, B_{w,d+1}^{\CARE}, Y_{w-1}^{\EDGE}, A_{w,d}^{\CARE}) \vbeta^{\CARE}, \label{equ:care_rwd_app}
\end{align}

\begin{algorithm}[hpt]
    \caption{\texttt{MultiAgent+SurrogateRwd} Algorithm}
    \begin{algorithmic}[1]
        \STATE{Input:} discount factor $\gamma^{\AYA} = 0.5$, $\gamma^{\CARE} = 0.5$, $\gamma^{\EDGE} = 0$
        \STATE{Initialize:} $\theta^{\AYA}_{1,1,1} = \mathbf{0} \in \mathbb{R}^{10}$; $\theta^{\CARE}_{1,1} = \mathbf{0} \in \mathbb{R}^{10}$; $\theta^{\EDGE}_{1} = \mathbf{0} \in \mathbb{R}^{12}$; dataset $\mathcal{D}_{1,1,1}^{\AYA} = \emptyset$; $\mathcal{D}_{1,1}^{\CARE} = \emptyset$; $\mathcal{D}_{1}^{\EDGE} = \emptyset$
        \FOR{$w = 1, 2, \dots, 14$}
            \STATE{Call Algorithm \ref{alg:base} using $\mathcal{D}_{w}^{\EDGE}, \gamma^{\EDGE}$, and update $\theta_{w}^{\EDGE}$}
            \STATE{Set $A_{w}^{\EDGE} = \argmax_{\alpha} \langle \phi^{\EDGE}(S_{w}^{\EDGE}, \alpha), \theta_{w}^{\EDGE} \rangle$}
            \FOR{$d = 1, 2, \dots, 7$}
                \STATE{Call Algorithm \ref{alg:base} using $\mathcal{D}_{w,d}^{\CARE}, \gamma^{\CARE}$, and update $\theta_{w,d}^{\CARE}$}
                \STATE{Set $A_{w,d}^{\CARE} = \argmax_{\alpha} \langle \phi^{\CARE}(S_{w,d}^{\CARE}, \alpha), \theta_{w,d}^{\CARE} \rangle$}
                \FOR{$t = 1, 2$}
                    \STATE{Call Algorithm \ref{alg:base} using $\mathcal{D}_{w,d,t}^{\AYA}, \gamma^{\AYA}$, and update $\theta_{w,d,t}^{\AYA}$}
                    \STATE{$A_{w,d,t}^{\AYA} = \argmax_{\alpha} \langle \phi^{\AYA}(S_{w,d,t}^{\AYA}, \alpha), \theta_{w,d,t}^{\AYA} \rangle$}
                    \STATE{Environment generates $R_{w,d,t}^{\AYA}$ and next state $S_{w,d,t+1}$}
                    \STATE{Update $\mathcal{D}_{w,d,t}^{\AYA} = \mathcal{D}_{w,d,t-1}^{\AYA} \cup \{(S_{w,d,t}^{\AYA}, A_{w,d,t}^{\AYA}, R_{w,d,t}^{\AYA})\}$}
                \ENDFOR
                \STATE{Compute care-partner reward $\tilde{R}_{w,d}^{\CARE}$ based on Equ. (\ref{equ:care_rwd_app})} 
                \STATE{Update $\mathcal{D}_{w,d}^{\CARE} = \mathcal{D}_{w,d-1}^{\CARE} \cup \{(S_{w,d}^{\CARE}, A_{w,d}^{\CARE}, \tilde{R}_{w,d}^{\CARE})\}$}
            \ENDFOR
            \STATE{Compute relationship reward $\tilde{R}_{w}^{\EDGE}$ based on Equ. (\ref{equ:game_rwd_app})}
            \STATE{Update $\mathcal{D}_{w}^{\EDGE} = \mathcal{D}_{w-1}^{\EDGE} \cup \{(S_{w}^{\EDGE}, A_{w}^{\EDGE}, \tilde{R}_{w}^{\EDGE})\}$}
        \ENDFOR
    \end{algorithmic}
    \label{alg:multi_agent_surrogate}
\end{algorithm}


\begin{table}[hpt]
    \centering
    \caption{Prior mean for coefficients in the surrogate reward functions.}
    \label{tab:prior}
    \begin{tabular}{c|c|c|c|c|c}
    \toprule
    Agent & Intercept & $Y_w^{\EDGE}$ & $B_w^{\AYA}$  & $A_w^{\EDGE}$ & $A_w^{\EDGE} \cdot Y_w^{\EDGE}$ \\
    \midrule
    $\vbeta^{\EDGE}$ & $1$ & $1$ & $-1$ & $-1$ & $0.5$ \\
    \midrule
    \midrule
    Agent & Intercept & $Y_{w,d}^{\CARE}$ & $B_{w,d}^{\CARE}$ & $Y_{w-1}^{\EDGE}$ & $A_{w,d}^{\CARE}$  \\
    \midrule
    $\vbeta^{\CARE}$ & $1$ & $-1$ & $-1$ & $1$ & $-0.5$  \\
    \bottomrule
    \end{tabular}
    \end{table}
    
\section{The Dyadic Environment}
\label{app:testbed}

\subsection{Overview of the Simulated Dyadic Environment}

% \ziping{Give a brief description of the testbed.}

We construct a dyadic simulation environment to evaluate the performance of the proposed algorithm. The 1st order goal of the environment design is to replicate the noise level and structure that we expect to encounter in the forthcoming ADAPTS-HCT clinical trial. This noise often encompasses the stochasticity in the state transition of each participant and the heterogeneity across participants.

The environment is based on Roadmap 2.0, a care partner-facing mobile health application that provides daily positive psychology interventions to the care partner only. Roadmap 2.0 involves 171 dyads, each consisting of a patient undergone HCT (target person) and a care partner. Each participant in the dyad had the Roadmap mobile app on their smartphone and wore a Fitbit wrist tracker. The Fitbit wrist tracker recorded physical activity, heart rate, and sleep patterns. Furthermore, each participant was asked to self-report their mood via the Roadmap app every evening. A list of variables in Roadmap 2.0 is reported in Table \ref{tab:roadmap_variable}.

Roadmap 2.0 data is suitable for constructing a dyadic environment for developing the RL algorithm for ADAPTS-HCT in that Roadmap 2.0 has the same dyadic structure about the participants--post-HCT cancer patients and their care partner. Moreover, Roadmap 2.0 encompasses some context variables that align with those to be collected in ADAPTS-HCT, for example, the daily self-reported mood score.

\subsubsection{Overcoming impoverishment.} From the viewpoint of evaluating dyadic RL algorithms, this data is impoverished \cite{trella2022designing} mainly in two aspects. First, Roadmap 2.0 does not include micro-randomized daily or weekly intervention actions (i.e., whether to send a positive psychology message to the patient/care partner and whether to engage the dyad into a weekly game). Second, it does not include observations on the adherence to the medication--the primary reward signal, as well as other important measurements such as the strength of relationship quality. 
% Furthermore, Roadmap 2.0 includes dyads across all lifespan whereas ADAPTS-HCT will focus on adolescent and young adults.

To overcome this impoverishment, we construct surrogate variables from the Roadmap 2.0 data to represent the variables intended to be collected in ADAPTS-HCT. A list of substitutes is reported in Table \ref{tab:roadmap_substitutes}. Worthnoting, the substitute for the AYA medication adherence is based on the step count. There is evidence on the association between the step count and the adherence. 

% \ziping{Could we find literature to justify this?}
% \ziping{Should we discuss in detail the rationale of these substitutes?} 
We further impute the treatment effects of the intervention actions so the marginal effects after normalization, which we call the standardized treatment effects (STE), are around 0.15, 0.3, and 0.5, corresponding to small, medium, and large effect sizes in typical behavioral science studies.

\subsubsection{Constructing the dyadic environment.} We follow the environment design in \cite{li2023dyadic}, which also uses the Roadmap 2.0 data, but primarily focuses on AYA intervention and relationship intervention. We extend the environment to include the care partner intervention. Specifically, we fit a separate multi-variate linear model for each participant in the dataset with the AR(1) working correlation using the generalized estimating equation (GEE) approach \cite{ziegler2010generalized,hojsgaard2006r}. We impute the treatment effects of the intervention actions based on the typical STE around 0.15, 0.3, and 0.5, which completes a generative model for the state transitions. The environment simulates a trial by randomly sampling dyads from the dataset, and simulate their trajectories based on the actions selected by the RL algorithm. The environment details are described in Appendix \ref{app:testbed}. Our experiments primarily focus on the three vanilla testbeds corresponding to the three STEs.



%In this section, we describe our set up of the simulation testbed. 


\subsection{Using the Roadmap 2.0 Dataset}


This section outlines our approach to addressing the limitations of the Roadmap 2.0 dataset, specifically its absence of micro-randomized interventions and reward signals.

To circumvent the lack of interventions, we impute treatment effects that represent the burden of the digital interventions, assuming that frequent notifications diminish both weekly and the daily treatment effects. Based on prior literature, we choose the scale of the treatment effect to be smaller than the baseline effect of features \cite{box1987empirical}. 

To address the missing reward signals, we use directly measurable variables in Roadmap 2.0 dataset as proxies to the outcomes we will observe in the real clinical trial. We approximate AYA adherence, $R_{w,d,t}^{\AYA}$, using the 12-hourly step count from Roadmap 2.0. Previous work has found the two values to be strongly correlated \hinal{TODO cite}. Since adherence is a binary signal in the ADAPTS-HCT trial, we discretize step count into a binary variable. Furthermore, we approximate the carepartner's daily psychological distress, $Y_d^{\CARE}$, using the daily length of their sleep. Finally, the weekly relationship between the AYA and their carepartner is estimated using the self-reported mood as a surrogate. Specifically, we let  $Y_w^{\EDGE} = \mathbbm{1}\{\sum_{d = 1}^{7} \Mood_{w,d}^{\AYA} \geq \Mood^{\AYA}\} \mathbbm{1}\{\sum_{d = 1}^{7}\Mood_{w,d}^{\CARE} \geq \Mood^{\CARE}\}$. Here, $\Mood_{w,d}^{\AYA}$ is the daily self-reported mood on week $w$ and day $d$, and $\Mood^{\AYA}$ is the $q$-th quantile of the the weekly summed mood across all AYA observations. We choose the quantile level $q$ such that approximately 50\% of the dataset satisfies $Y_{w}^{\EDGE} = 1$.

Table \ref{tab:roadmap_substitutes} summarizes the main variables and their replacements from the Roadmap 2.0 dataset.  

\begin{table}[hpt]
    \centering
    \caption{Substitutes of the main variables from Roadmap 2.0 dataset.}
\resizebox{\textwidth}{!}{%
    \begin{tabular}{c|c}
        \hline
        Variables & Substitutes \\
        \hline
        \hline
        AYA adherence  & Binary step count $\mathbbm{1}\{\texttt{Step}_{w, d,t}^{\AYA} \geq \texttt{Step}^{\AYA}\}$  
        \\
        Carepartner distress & Carepartner daily length of sleep $\texttt{Sleep}_{w, d}^{\CARE}$ \\
        Weekly relationship quality & Mood indicator: $\mathbbm{1}_{\{\sum_{d} \texttt{Mood}_{w, d}^{\CARE} \geq \texttt{Mood}^{\CARE}\}} \mathbbm{1}_{\{\sum_{d} \texttt{Mood}_{w, d}^{\AYA} \geq \texttt{Mood}^{\AYA}\}}$ \\
        Effects of interventions $A_{w,d,t}^{\AYA}, A_{w,d}^{\CARE}$, $A_{w}^{\EDGE}$ & Imputed based on domain knowledge \\
        Effects of digital interventions burden $B_{w,d,t}^{\AYA}$, $B_{w,d}^{\CARE}$ & Imputed based on domain knowledge\\
        \hline
    \end{tabular}%
    }    \label{tab:roadmap_substitutes}
\end{table}


\begin{table}[hpt]
    \centering
    \caption{List of variables in Roadmap 2.0 and the measuring frequencies.}
    \begin{tabular}{c}
    \hline
    Variables\footnote{Note that all the variables are measured the same for the target person and carepartner.} \\
    \hline
    \hline
     $\texttt{Step}_{w, d, t}$: twice-daily cumulative step count\\
     $\texttt{Heart}_{w, d, t}$: twice-daily average heart rate\\
     $\texttt{Sleep}_{w, d}$: daily length of sleep\\
     $\texttt{Mood}_{w, d}$: daily self-report mood measurement\\
     \hline
    \end{tabular}
    \label{tab:roadmap_variable}
\end{table}

\subsection{Environment Model Design}

We now describe how these surrogate variables are used to build the full environment model. Our approach involves fitting two state transition models for digital intervention burden (AYA and carepartner) and three models for rewards (AYA adherence, carepartner stress, and relationship quality).

For all transition models, we fit the baseline parameters -- which represent system dynamics under no intervention --  for each dyad using its respective dataset and a generalized estimating equation \cite{hojsgaard2006r} approach. We impute the remaining parameters  using domain knowledge. Further detail on the choice of the coefficients is in Appendix \ref{sec:select_sim_params}. 

\textbf{Transition models for the AYA component: } The digital intervention burden transition for AYA follows a linear model with covariates $(B_{w,d,t}^{\AYA}, A_{w,d,t}^{\AYA}, A_{w}^{\EDGE})$.
\begin{align}
\label{equ:B_transition_AYA}
    B_{w,d,t+1}^{\AYA} \sim \theta^{\AYA}_{0} + \theta_{1}^{\AYA} B_{w,d,t}^{\AYA} + \theta_{2}^{\AYA} A_{w,d,t}^{\AYA} + \theta_{3}^{\AYA} A_{w}^{\EDGE} + \eta_{w,d,t}^{\AYA}, \nonumber\\
    \text{ where $\eta_{w,d,t}^{\AYA} \sim \mathcal{N}(0, (\omega^{\AYA})^2)$.} 
\end{align}
 

For the primary outcome, AYA adherence, we fit a generalized linear model with a sigmoid link function: 


\begin{align}
R_{w,d,t}^{\AYA} &\sim \text{Bernoulli}(\text{sigmoid}(P_{w,d,t}^{\AYA})), \nonumber \\
P_{w,d,t}^{\AYA} &= (1-M_t)\big(\beta_{0, \AM}^{\AYA} + \beta^{\AYA}_{1, \AM} R_{w,d,t-1}^{\AYA} 
+ \beta_{2,\AM}^{\AYA} Y_{w-1}^{\EDGE} 
+ \beta_{3,\AM}^{\AYA} Y_{w,d-1}^{\CARE} + \beta_{4, \AM}^{\AYA} B_{w,d,t}^{\AYA} \nonumber \\
&\quad + \tau_{0, \AM}^{\AYA} A_{w,d,t}^{\AYA} 
+ \tau_{1, \AM}^{\AYA} A_{w,d,t}^{\AYA} Y_{w-1}^{\EDGE} 
+ \tau_{2, \AM}^{\AYA} A_{w,d,t}^{\AYA} B_{w,d,t}^{\AYA}\big) \nonumber \\
&\quad + M_t\big(\beta_{0, \PM}^{\AYA} + \beta^{\AYA}_{1, \PM} R_{w,d,t-1}^{\AYA} 
+ \beta_{2,\PM}^{\AYA} Y_{w-1}^{\EDGE} 
+ \beta_{3,\PM}^{\AYA} Y_{w,d-1,t}^{\CARE} + \beta_{4, \PM}^{\AYA} B_{w,d,t}^{\AYA} \nonumber \\
&\quad + \tau_{0, \PM}^{\AYA} A_{w,d,t}^{\AYA}  
+ \tau_{1, \PM}^{\AYA} A_{w,d,t}^{\AYA} Y_{w-1}^{\EDGE} 
+ \tau_{2, \PM}^{\AYA} A_{w,d,t}^{\AYA} B_{w,d,t}^{\AYA}\big)
\label{equ:R_Transition_AYA}
\end{align}

where $M_t$ is a decision window indicator defined as:

$$
    M_t = \left\{
    \begin{array}{clll}
         0 & \text{ if } t = 2k - 1 & (\text{AM decision window}) & \text{ for } k = 1, 2, \dots  \\
         1 & \text{ if } t = 2k & (\text{PM decision window}) &\text{ for } k = 1, 2, \dots
    \end{array},\right.
$$ 
Note that we exclude any effect of relationship interventions on AYA adherence as the game is designed without reinforcements and, thus, is not supposed to directly improve adherence.



\textbf{Transition models for the carepartner component: } The digital intervention burden transition for the carepartner is a linear model:

\begin{align}
\label{equ:B_transition_care}
    B_{w,d+1}^{\CARE} = \theta^{\CARE}_{0} + \theta_{1}^{\CARE} B_{w,d}^{\CARE} + \theta_{2}^{\CARE} A_{w,d}^{\CARE} + \theta_{3}^{\CARE} A_{w}^{\EDGE} + \eta_{w,d}^{\CARE}, \nonumber\\
    \text{ where $\eta_{w,d}^{\CARE} \sim \mathcal{N}(0, (\omega^{\CARE})^2)$.}
\end{align}

For the carepartner's psychological distress level, $R^{\CARE}_d$, we fit another linear model:
\begin{align}
    Y_{w,d}^{\CARE} = 
    &\beta_{0}^{\CARE} + \beta_{1}^{\CARE} Y_{w,d-1}^{\CARE} + \beta_{2}^{\CARE} R_{w,d,t-1}^{\AYA}  + 
    \beta_{3}^{\CARE} Y_{w-1}^{\EDGE} + \beta_{4}^{\CARE} B_{w,d}^{\CARE} + \nonumber \\
    &\quad \tau_{0}^{\CARE} A_{w,d}^{\CARE} +  
    \tau_{1}^{\CARE} A_{w,d}^{\CARE} Y_{w-1}^{\EDGE} +   
    \tau_{2}^{\CARE} A_{w,d}^{\CARE} B_{w,d}^{\CARE}  + \epsilon_{w,d}^{\CARE} \label{equ:R_transition_care}
\end{align}
where $\epsilon_{w,d}^{\CARE} \sim \mathcal{N}(0, (\sigma^{\CARE})^2)$.  Similar to (\ref{equ:R_Transition_AYA}), we do not include relationship intervention $A_{w-1}^{\EDGE}$.

\textbf{Transition model for the weekly relationship: } For the shared component, we only fit a transition model for the reward, which is the weekly relationship quality. Specifically, we fit a generalized linear model with a sigmoid link function: 

\begin{align}
Y_{w+1}^{\EDGE} \sim \text{Bernoulli}(\text{sigmoid}\left( \beta_{0}^{\EDGE} + \beta_{1}^{\EDGE}Y_{w}^{\EDGE}  + \beta_{2}^{\EDGE} \bar{R}_{w}^{\AYA} + \beta_{3}^{\EDGE} \bar{R}_{w}^{\CARE} \right. \nonumber \\
\left. + \tau_0^{\EDGE} A_{w}^{\EDGE} + \tau_1^{\EDGE} A_{w}^{\EDGE} (B_{w,d}^{\CARE} + B_{w,d,t}^{\AYA}))\right)
\label{equ:R_transition_rel}
\end{align}

where $\bar{R}_{w}^{\AYA} = \sum_{d=1}^{7} \sum_{t=1}^{2} \gamma^{14 - (7(w-1) + d) + 2(t-1)} R_{w,d,t}^{\AYA}$ is the exponentially weighted average of adherence within week $w$, and $\bar{R}_{w}^{\CARE} = \sum_{d=1}^{7} \gamma^{7-d} Y_{w,d}^{\CARE}$ is the exponentially weighted average of carepartner distress within week $w$. 

\subsection{Selecting Environment Model Parameters}
\label{sec:select_sim_params}

We list all the parameters that must be either imputed based on domain knowledge or estimated from the existing dataset. 
%We must impute all parameters related to the treatment effects because the existing dataset contains no target intervention. For the same reason, we impute all the parameters relating to the digital intervention burden $B_{w,d,t}^{\AYA}$, $B_{w,d}^{\CARE}$.

\begin{enumerate}
\item The baseline transition parameters $\beta$'s can be estimated directly from the dataset:
    \begin{enumerate}
        \item AYA state transition: $\vbeta^{\AYA}_{\AM} = (\beta_{0, \AM}^{\AYA}, \beta_{1, \AM}^{\AYA}, \beta_{2, \AM}^{\AYA}, \beta_{3, \AM}^{\AYA}, \beta_{4, \AM}^{\AYA})$ and $\vbeta^{\AYA}_{\PM} = (\beta_{0, \PM}^{\AYA}, \beta_{1, \PM}^{\AYA}, \beta_{2, \PM}^{\AYA}, \beta_{3, \PM}^{\AYA}, \beta_{4, \PM}^{\AYA})$.
        \item Carepartner state transition: $\vbeta^{\CARE} = (\beta_{0}^{\CARE}, \beta_{1}^{\CARE})$.
        \item Relationship transition: $\vbeta^{\EDGE} = (\beta_{0}^{\EDGE}, \beta_{1}^{\EDGE}, \beta_{2}^{\EDGE}, \beta_{3}^{\EDGE})$.
        % \item Hazard model parameters: $\vgamma = (\gamma_{0, 1}, \dots, \gamma_{0, 98}, \gamma_1, \gamma_2, \gamma_3)$
    \end{enumerate}
\item Imputed or tuned based on domain knowledge:
    \begin{enumerate}
        \item Burden transitions: coefficients $\boldsymbol{\theta}^{\AYA} = (\theta^{\AYA}_{0}, \theta^{\AYA}_{1}, \theta^{\AYA}_{2}, \theta^{\AYA}_{3})$, $\boldsymbol{\theta}^{\CARE} = (\theta^{\CARE}_{0}, \theta^{\CARE}_{1}, \theta^{\CARE}_{2}, \theta^{\CARE}_{3})$; burden noise variance $\omega^{\AYA}$ and $\omega^{\CARE}$.
        \item Main effects of burden: $\beta_{4, \AM}^{\AYA}, \beta_{4, \PM}^{\AYA}$, and $\beta_{4}^{\CARE}$.
        \item AYA treatment effects: $\{\tau_{i, \AM}^{\AYA}\}_{i = 0}^{2}$, $\{\tau_{i, \PM}^{\AYA}\}_{i = 0}^{2}$ and $\{\sigma_{i, \AM}^{\AYA}\}_{0 = 1}^{2}$, $\{\sigma_{i, \PM}^{\AYA}\}_{i = 0}^{2}$.
        \item Carepartner treatment effects: $\{\tau_{i}^{\CARE}\}_{i = 0}^{2}$ and $\{\sigma_{i}^{\CARE}\}_{i = 0}^{2}$.
        \item Relationship treatment effects: $\tau^{\EDGE}$ and $\sigma^{\EDGE}$.
        % \item Disengagement effect: $\xi_{0, d}, \xi_{1, d}, \xi_{2, d}$.
    \end{enumerate}
\end{enumerate}

\textbf{Fitting parameters (1a-d):} We estimate the baseline transition parameters under no intervention directly from the Roadmap 2.0 dataset. For the parameters in Equation (\ref{equ:R_Transition_AYA}), we have the correspondences 
$\beta_{i, \AM}^{\AYA} = \hat{\beta}_{i, \AM}^{\AYA}$ and $\beta_{i, \PM}^{\AYA} = \hat{\beta}_{i, \PM}^{\AYA}$ for $i = 0, 1, \dots, 3$, where $\hat{\beta}_{i, \AM}^{\AYA}$ and $\hat{\beta}_{i, \PM}^{\AYA}$ are fitted coefficients obtained using the generalized estimating equation (GEE) approach. Since we assume that app burden only moderates the effects of AYA interventions without directly influencing adherence, we set $\beta_{4, \AM}^{\AYA} = \beta_{4, \PM}^{\AYA} = 0$. Similarly, for parameters in Equation (\ref{equ:R_transition_care}), the correspondence is $\beta_{i}^{\CARE} = \hat{\beta}_{i}^{\CARE}$ for $i = 0, \dots, 3$, and we set $\beta_{4}^{\CARE} = 0$ under the same assumption for carepartner distress. For the relationship quality model in Equation (\ref{equ:R_transition_rel}), the correspondence is $\beta_{i}^{\EDGE} = \hat{\beta}_{i}^{\EDGE}$ for $i = 0, \dots, 3$. Based on domain knowlege, we also truncate the parameters as follows: $\beta_{2, *}^{\AYA} = \max\{0, \hat{\beta}_{2, *}^{\AYA}\}$, reflecting the assumption that weekly relationship quality non-negatively influences AYA adherence, $\beta_{3, *}^{\AYA} = \min\{0, \hat{\beta}_{3, *}^{\AYA}\}$, as carepartner distress is expected to negatively influence adherence, and $\beta_{3}^{\EDGE} = \min\{0, \hat{\beta}_{3}^{\EDGE}\}$ as carepartner distress is expected negatively impact relationship quality.

% We can fit the baseline transition parameters under no intervention directly from the Roadmap 2.0 dataset. Specifically, for parameters in (\ref{equ:R_Transition_AYA}), we have the following correspondence: $\beta_{i, \AM}^{\AYA} = \hat{\beta}_{i, \AM}^{\AYA}$, and $\beta_{i, \PM}^{\AYA} = \hat{\beta}_{i, \PM}^{\AYA}$ for all $i = 0, 1, \dots, 3$, where $\hat{\beta}_{i, 0}^{\AYA}$ and $\hat{\beta}_{i, 1}^{\AYA}$ are fitted coefficients based on GEE approaches. We believe that app burden only moderates the effects of AYA interventions. Therefore, we set $\beta_{4, \AM}^{\AYA} = \beta_{4, \PM}^{\AYA} = 0$. For parameters in (\ref{equ:R_transition_care}), we have direct correspondence--$\beta_{i}^{\CARE} = \hat{\beta}_{i}^{\CARE}$ for $i = 0,\dots,3$ and $\beta_{4}^{\CARE} = 0$. Similarly, for parameters in (\ref{equ:R_transition_rel}), we have $\beta_{i}^{\EDGE} = \hat{\beta}_{i}^{\EDGE}$ for $i = 0, \dots, 3$.


% $\beta_{2, *}^{\AYA} = \max\{0, \hat \beta_{2, *}^{\AYA}\}$ because weekly relationship quality should be positively related to AYA adherence, and $\beta_{3, *}^{\AYA} \min\{0, \beta_{3, *}^{\AYA}\}$ because carepartner distress should be negatively related to AYA adherence. Also, $\beta_{3}^{\EDGE} = \min\{0, \hat \beta_{3}^{\EDGE}\}$ because carepartner distress should be negatively related to relationship quality.

\textbf{Imputing Burden Transitions (2a):}
We set $
  \theta_{1}^{\AYA} = \tfrac{13}{14}, \quad \theta_{1}^{\CARE} = \tfrac{6}{7},
$ so that the memory of digital burden spans roughly one week for both AYA and carepartner. We choose
$\theta_{2}^{\AYA} = 5\,\theta_{3}^{\AYA} = 1, 
  \quad
  \theta_{2}^{\CARE} = 5\,\theta_{3}^{\CARE} = 1$ so that daily interventions exert five times more burden than the weekly relationship intervention. The intercepts are $
  \theta_{0}^{\AYA} = 0.2, 
  \quad
  \theta_{0}^{\CARE} = 0.2$, and chosen so that participants have around a 20\% baseline burden even without an intervention. We set
$
  \omega^{\AYA} = \omega^{\CARE} = 2.4
$ to obtain a moderate noise-to-signal ratio, set so that 
\(
    (\theta_{1}^{\AYA} + \theta_{2}^{\AYA}) / \omega^{\AYA} 
    \approx 0.5
\).
We then truncate burdens at zero and standardize them separately for AYA and carepartner by simulating 10,000 steps with random interventions.

\textbf{Imputing main effects of app burden (2b).} We set $\beta_{4, \AM}^{\AYA} = \beta_{4, \PM}^{\AYA} = \beta_{4}^{\CARE} = 0$ based on the assumption that digital app intervention burden does not directly affect AYA adherence or carepartner distress, unless through moderating the digital interventions.

\textbf{Imputing treatment Effects (2c--2f):}
Since digital health environments are noisy, treatment terms likely have a lower effect on transitions than the baseline transitions under no intervention. Hence, we scale all intervention effects relative to the baseline effects using a single, global hyperparameter $C_{\text{treat}}$. 

For each time of the day (AM or PM), the AYA intervention increases adherence by $\tau_{0, *}^{\AYA} = C_{\text{treat}} \bigl|\beta_{1, *}^{\AYA}\bigr|$, where $* \in \{\mathrm{AM}, \mathrm{PM}\}$ and $\beta_{1, *}$ is the corresponding baseline coefficient estimated from Roadmap 2.0. 

We further define $\bigl|\beta_{1, *}^{\AYA}\bigr|$ and $\tau_{\text{burden}, *}^{\AYA} = -C_{\text{treat}} \bigl|\beta_{1, *}^{\AYA}\bigr|$ because the AYA intervention's effectiveness can be increased by good relationship quality and decreased by high digital-intervention burden.

To account for individual heterogeneity across dyads, each treatment-effect coefficient has an associated random effect with variance $\sigma_{0, *}^{\AYA} = C_{\text{treat}} \sigma_{\beta_{1, *}^{\AYA}}$, where $\sigma_{\beta_{1, *}^{\AYA}}$ is the empirical standard deviation across dyads of the baseline coefficient $\beta_{1, *}^{\AYA}$. 

For carepartner interventions, the main effect on distress is scaled as $\tau_{0}^{\CARE} = -C_{\text{treat}} \bigl|\beta_{1}^{\CARE}\bigr|$, where the negative sign is due to the intervention reducing distress. Lastly, the effect of the weekly relationship intervention on improving relationship quality is given by $\tau^{\EDGE} = C_{\text{treat}} \bigl|\beta_{1}^{\EDGE}\bigr|$.

% \textbf{Imputing treatment Effects (2c--2e):}
%  Since digital health environments are noisy, treatment terms likely have a lower effect on the transitions than the baseline transitions under no intervention. Hence, we scale all intervention effects relative to the baseline effects using a single, global hyperparameter $C_{\Treat}$. 

% \begin{enumerate}
%     \item AYA Daily Interventions (AM/PM)
%         \begin{itemize}
%             \item Main Effect: For each time of day (AM or PM), the AYA intervention increases adherence by $$\tau_{0, *}^{\AYA} \;=\; C_{\text{treat}}\;\bigl|\beta_{1, *}^{\AYA}\bigr|$$
%       where $ * \in \{\mathrm{AM},\,\mathrm{PM}\}$. Recall that $\beta_{1, *}$ is the corresponding baseline coefficient estimated from data. The absolute value ensures a positive effect on adherence of the intervention.
%       \item Interactions with Relationship and Burden: The AYA intervention's effectiveness can be increased by good relationship quality and decreased by high digital-intervention burden. Hence,
%       $$\tau_{\text{rel}, *}^{\AYA} \;=\; C_{\text{treat}}\;\bigl|\beta_{1, *}^{\AYA}\bigr|
%       \quad\text{and}\quad
%       \tau_{\text{burden}, *}^{\AYA} \;=\; -\,C_{\text{treat}}\;\bigl|\beta_{1, *}^{\AYA}\bigr|.$$
%     \item Random Effect Variances: To account for individual heterogeneity across dyads, each treatment-effect coefficient has an associated random effect whose variance. We also scale this variance by $$\sigma_{0,*}^{\AYA} \;=\; C_{\text{treat}}\;\sigma_{\beta_{1,*}^{\AYA}}$$
%     where $\sigma_{\beta_{1,*}^{\AYA}}$ is the empirical standard deviation across dyads of the baseline coefficient \(\beta_{1,*}^{\AYA}\).
%         \end{itemize}
% \item Carepartner Daily Interventions: We perform similar scaling to the carepartner interventions, which are designed to reduce distress. The main effect of a care-partner intervention on distress becomes
% $$\tau_{0}^{\CARE} \;=\; -\,C_{\text{treat}}\;\bigl|\beta_{1}^{\CARE}\bigr|$$
% where the negative sign reflects a reduction in distress due to the intervention.
% \item Weekly Game Intervention: Lastly, the effect of the weekly game on improving relationship quality is:

%  $$ \tau^{\EDGE} \;=\; C_{\text{treat}}\;\bigl|\beta_{1}^{\EDGE}\bigr|.
% $$
% \end{enumerate}


We summarize the imputation design in Table \ref{tab:imputation}. 
%and generate a list of all tuning parameters in Table \ref{tab:hyper_param}.

\begin{table}[hpt]
    \centering
    \caption{Summary of burden transition design and treatment effects design.}
    \begin{tabular}{c|>{\centering\arraybackslash}p{0.4\textwidth}} 
    \hline
    \multicolumn{2}{c}{Burden transition}\\
    \hline
    \hline
    Intercept $\theta_{0}^{\AYA}$  & Based on domain knowledge $\theta_{0}^{\AYA} = 0.2$ \\
    Intervention coefficients $\theta_2^{\AYA}, \theta_3^{\AYA}$ & $\theta_2^{\AYA} = 5\theta^{\AYA}_{3} = 1$ (Because relationship intervention produces lower burden) \\
    Noise standard deviation $\omega^{\AYA}$ & Based on the typical noise-to-signal ratio $\omega^{\AYA} = 2.4$ \\
    \hline
    \multicolumn{2}{c}{Treatment effect for twice-daily adherence transition (* stands for AM or PM)}\\
    \hline
    \hline
    Main effect of AYA intervention $\tau_{0, *}^{\AYA}$ & Hyper-parameter $\tau_{0, *}^{\AYA} = C_{\Treat}|\beta_{1, *}^{\AYA}| $ \\
    Rel. and AYA int. interaction $\tau_{2, *}^{\AYA}$ & Hyper-parameter $\tau_{1, *}^{\AYA} = C_{\Treat}|\beta_{1, *}^{\AYA}|$ \\
    Burden and AYA int. interaction $\tau_{4, *}^{\AYA}$ & Hyper-parameter $\tau_{2, *}^{\AYA} = C_{\Treat}|\beta_{1, *}^{\AYA}|$ \\
    Random treatment variance $\{\sigma_{i, *}^{\AYA}\}_{i = 0}^5$ & Scales with the variance of $\beta_{1, *}^{\AYA}$: $\sigma_{i, *}^{\AYA} = \tau_{i, *}^{\AYA} \cdot  \sigma_{\beta_{1, *}^{\AYA}} / |\beta_{1, *}^{\AYA}|$ \\
    \hline
    \multicolumn{2}{c}{Treatment effect for weekly relationship transition}\\
    \hline
    \hline
    Main effect of relationship int. $\tau^{\EDGE}$ & Hyper-parameter $\tau^{\EDGE} = C_{\Treat}|\beta_1^{\EDGE}|$  \\
    \hline
    % \multicolumn{2}{c}{Disengagement effect}\\
    % \hline
    % \hline
    % Disengagement effect $\xi_{0, d}, \xi_{1, d}, \xi_{2, d}$ & $\xi_{0, d} = \frac{1}{10} |\gamma_{0, d}|$, $\xi_{1, d} = \frac{1}{5} |\gamma_{0, d}|$ and $\xi_{1, d} = \frac{1}{25} |\gamma_{0, d}|$ \\ 
    \end{tabular}
    \label{tab:imputation}
\end{table}

\textbf{Tuning $C_{\Treat}$: } We tune the hyperparameter  $C_{\Treat}$ such that the standardized treatment effects (STE) are around 0.15, 0.3, and 0.5, where STE is defined as:
\begin{equation}
    \operatorname{STE}(C_{\Treat}) = \frac{\mathbb{E}\left[\mathbb{E}[\text{CR}(\pi^*_{e}) \mid e] - \mathbb{E}[\text{CR}(\pi_0, e) \mid e]\right]}{\sqrt{\operatorname{Var}(\mathbb{E}[\text{CR}(\pi_0, e) \mid e])}},
    \label{equ:STE}
\end{equation}
Here, $e$ corresponds to the resulting environment model for dyad $e$ when the hyperparameter is set to be $C_{\Treat}$, and $\pi_e^*$ is the optimal policy for dyad $e$. $\text{CR}(\pi, e)$ is the cumulative rewards earned by running policy $\pi$ on dyad $e$, and $\pi_0$ is the reference policy that always chooses action 0 for all components. 
% STE is defined as the average gap between the cumulative rewards under the optimal policy for each dyad and the cumulative rewards under the reference policy. It is normalized by the standard deviation of the expected cumulative rewards under $\pi_0$ over the distribution of dyads.

Figure \ref{fig:ste} plots the value of the hyperparameter versus the STE computed using the optimal policy in the environment defined by the hyperparameter. We outline our approximation of the optimal policy in Appendix \ref{sec:optpol}. By default, we choose an environment with mediator effect = 1. This results in three dyadic environments, which we summarize in Table \ref{tab:test-bed}.

\begin{table}[ht]
    \centering
    \caption{Summary of all testbeds}
    \begin{tabular}{c|c}
     Treatment effect size & Value of $C_{\Treat}$ \\
    \hline
       0.15 (Small)  & 0.2 \\
       0.3 (Medium) & 0.3 \\
       0.5 (Large) & 0.5 \\
    \end{tabular}
    \label{tab:test-bed}
\end{table}

\begin{figure}[hpt]
    \centering
    \includegraphics[width=0.8\textwidth]{Plots/STE/treat-vs-ste.png}
    \caption{Relationship between the hyperparameters and the STE, categorized by the mediator effect value.}
    \label{fig:ste}
\end{figure}

\subsection{Optimal Policy Approximation}
\label{sec:optpol}

% In this section, we outline our procedure for approximating the optimal policy used for generating the STE, which is defined as the the average gap between the cumulative rewards obtained by the optimal policy and those obtained by the reference policy. 

To approximate the optimal policy, we generate a dataset under a random policy with $P(A_{w,d,t}^{\AYA}=1)= P(A_{w,d}^{\CARE}=1) = P(A_w^{\EDGE}=1)=0.5$ and apply offline Q-learning on this dataset. To make the computation tractable, we discretize and subset the features. Specifically, we use six features: the intercept, AYA adherence, carepartner distress, AYA burden, carepartner burden, and relationship quality. Numerical features (carepartner distress, AYA burden, and carepartner burden) are discretized into 10 bins.

Finally, we evaluate the performance of this approximation against other baseline policies, including micro-randomized actions with fixed probabilities of 0.5, 0.6, 0.7, 0.8, and 0.9. Our approximation consistently outperforms these baselines.


\subsection{Evidence of the Need for Collaboration in the Dyadic Environment}
\label{app:evidence-collab}
% We outline our procedure for verifying that the dyadic environment requires collaboration.

To show that each agent impacts the performance of other agents, we consider the following toy setting.
% We have two random algorithms: 1. 
We fix the care partner agent's randomization probability at 0.5 and vary the AYA agent’s probability to be 0.25 and 0.75. Then, for each fixed AYA agent's probability, we identify the value of the relationship agent’s probability that maximizes average weekly adherence. We find that this relationship probability changes from $1.0$ to $0.0$ when we change AYA agent's probability from 0.25 to 0.75. 

We repeat this experiment for the care partner agent by fixing the AYA agent’s probability at 0.5 and varying the relationship agent’s probability to be 0.25 and 0.75. Similarly, we find that the care partner agent's probability that maximizes adherence changes from 0.6 to 0.5 when we vary the relationship probability from 0.25 to 0.75.

These results indicate that the agents must change their behavior to account for the other agents' behavior.

% set  the randomization probability of the AYA intervention to be 0.25 and 0.75, and see whether the optimal level of randomization probability for the relationship agent is different given a fixed randomization probability of 0.5 for the care partner agent Fig. \ref{fig:MRT_collaboration} (a, b). A similar experiment is conducted for the collaboration between the care partner and the relationship agent in Fig. \ref{fig:MRT_collaboration} (c, d). We see that the optimal level of randomization probability for the relationship agent changes for different AYA randomization probability, and the optimal randomization probability for the care partner agent changes for different game randomization probability. This indicates that the agent must change their behavior to account for the other agent's behavior.

% \ziping{Get rid of the figures.} 

% \begin{figure}[hpt]
%     \centering
%     \includegraphics[width=1\textwidth]{Plots/MRT_collaboration.pdf}
%     \caption{\textbf{(a, b)}: average weekly sum of adherence under different randomization probability for the relationship agent given a fixed probability for AYA and Care partner. \textbf{(c, d)}: average weekly sum of adherence under different randomization probability for the Care partner agent given a fixed probability for AYA and Game.}
%     \label{fig:MRT_collaboration}
% \end{figure}

\section{Additional Results}

\label{app:additional_results}
\subsection{Ablation Study}

\paragraph{No Mediator Effect} The improvement from using a surrogate reward is through the effects of the mediator variables. For example, the relationship intervention $A_{w}^{\EDGE}$ improves the mediator relationship, which may improve the primary outcome, medication adherence. The care-partner intervention $A_{w,d}^{\CARE}$ mitigates the distress, which may improve relationship. In Fig. \ref{fig:mediator0}, we run the all three algorithms under a testbed variant for which we force the above two mediator effects to be 0, i.e., no effect from relationship to adherence or effect from distress to relationship. In this testbed variant, \texttt{MutiAgent+SurrogateRwd} performs the same as \texttt{MutiAgent}--there is no cost of reward learning under no mediator effect.

\begin{figure}[hpt]
    \centering
    \begin{subfigure}[b]{0.31\textwidth}
        \includegraphics[width=1\textwidth]{Plots/Experiments/015/All_Rewards_Mediator0.pdf}
        \caption{STE 0.15}
    \end{subfigure}
    \begin{subfigure}[b]{0.31\textwidth}
        \includegraphics[width=1\textwidth]{Plots/Experiments/03/All_Rewards_Mediator0.pdf}
        \caption{STE 0.3}
    \end{subfigure}
    \begin{subfigure}[b]{0.31\textwidth}
        \includegraphics[width=1\textwidth]{Plots/Experiments/05/All_Rewards_Mediator0.pdf}
        \caption{STE 0.5}
    \end{subfigure}
    \caption{Cumulative rewards improvement over the uniform random policy for all three components under the testbed without the effect of care-partner distress onto relationship quality or the effect of relationship quality onto AYA's adherence.}
    \label{fig:mediator0}
\end{figure}

\paragraph{Other Testbed Variants.} To further violate the assumptions made from the causal diagram, we made the following two changes to test the robustness of our proposed algorithm: 1) we add a direct effect from care-partner psychological distress to AYA medication adherence; 2) we generate random mediator effects, effect from relationship to adherence and effect from distress to relationship. This later one violates the monotonicity assumptions learned from principles.

\subsection{Collaboration of Multi-Agent RL} 

We train each individual agent in the \texttt{MultiAgent+SurrogateRwd} algorithm over 1000 dyads under the STE 0.5 environment, while fixing the randomization probability of the other agents. We denote the randomization probability of the AYA agent, care partner agent, and relationship agent as $p^{\AYA}$, $p^{\CARE}$, and $p^{\EDGE}$ respectively.

We first train the relationship agent while fixing $p^{\CARE} = 0.5$. We see that the average probability of sending an intervention for the relationship agent is 0.57 and 0.42 under $p^{\AYA} = 0.25$ and $0.75$, respectively. This indicates that the relationship agent learns to \textit{reduce} the intervention probability when the AYA agent is more likely to send an intervention. 

Similarly, we train the care partner agent while fixing $p^{\AYA} = 0.5$. We see that the average probability of sending an intervention for the care partner agent is 0.61 and 0.45 under $p^{\EDGE} = 0.25$ and $0.75$, respectively. This indicates that the care partner agent learns to \textit{reduce} the intervention probability when the relationship agent is more likely to send an intervention.
\end{document}