\subsection{Overview of the Simulated Dyadic Environment}

% \ziping{Give a brief description of the testbed.}

We construct a dyadic simulation environment to evaluate the performance of the proposed algorithm. The 1st order goal of the environment design is to replicate the noise level and structure that we expect to encounter in the forthcoming ADAPTS-HCT clinical trial. This noise often encompasses the stochasticity in the state transition of each participant and the heterogeneity across participants.

The environment is based on Roadmap 2.0, a care partner-facing mobile health application that provides daily positive psychology interventions to the care partner only. Roadmap 2.0 involves 171 dyads, each consisting of a patient undergone HCT (target person) and a care partner. Each participant in the dyad had the Roadmap mobile app on their smartphone and wore a Fitbit wrist tracker. The Fitbit wrist tracker recorded physical activity, heart rate, and sleep patterns. Furthermore, each participant was asked to self-report their mood via the Roadmap app every evening. A list of variables in Roadmap 2.0 is reported in Table \ref{tab:roadmap_variable}.

Roadmap 2.0 data is suitable for constructing a dyadic environment for developing the RL algorithm for ADAPTS-HCT in that Roadmap 2.0 has the same dyadic structure about the participants--post-HCT cancer patients and their care partner. Moreover, Roadmap 2.0 encompasses some context variables that align with those to be collected in ADAPTS-HCT, for example, the daily self-reported mood score.

\subsubsection{Overcoming impoverishment.} From the viewpoint of evaluating dyadic RL algorithms, this data is impoverished \cite{trella2022designing} mainly in two aspects. First, Roadmap 2.0 does not include micro-randomized daily or weekly intervention actions (i.e., whether to send a positive psychology message to the patient/care partner and whether to engage the dyad into a weekly game). Second, it does not include observations on the adherence to the medication--the primary reward signal, as well as other important measurements such as the strength of relationship quality. 
% Furthermore, Roadmap 2.0 includes dyads across all lifespan whereas ADAPTS-HCT will focus on adolescent and young adults.

To overcome this impoverishment, we construct surrogate variables from the Roadmap 2.0 data to represent the variables intended to be collected in ADAPTS-HCT. A list of substitutes is reported in Table \ref{tab:roadmap_substitutes}. Worthnoting, the substitute for the AYA medication adherence is based on the step count. There is evidence on the association between the step count and the adherence. 

% \ziping{Could we find literature to justify this?}
% \ziping{Should we discuss in detail the rationale of these substitutes?} 
We further impute the treatment effects of the intervention actions so the marginal effects after normalization, which we call the standardized treatment effects (STE), are around 0.15, 0.3, and 0.5, corresponding to small, medium, and large effect sizes in typical behavioral science studies.

\subsubsection{Constructing the dyadic environment.} We follow the environment design in \cite{li2023dyadic}, which also uses the Roadmap 2.0 data, but primarily focuses on AYA intervention and relationship intervention. We extend the environment to include the care partner intervention. Specifically, we fit a separate multi-variate linear model for each participant in the dataset with the AR(1) working correlation using the generalized estimating equation (GEE) approach \cite{ziegler2010generalized,hojsgaard2006r}. We impute the treatment effects of the intervention actions based on the typical STE around 0.15, 0.3, and 0.5, which completes a generative model for the state transitions. The environment simulates a trial by randomly sampling dyads from the dataset, and simulate their trajectories based on the actions selected by the RL algorithm. The environment details are described in Appendix \ref{app:testbed}. Our experiments primarily focus on the three vanilla testbeds corresponding to the three STEs.

