

\section{Related Work}

Below we summarize the most relevant literature from both the medical lens and the algorithm lens.

\textbf{RL on social networks.} We design and implement RL on dyads that are small social networks in this paper. Existing works on RL on social networks are mostly focused on maximizing social influence or opinion spreading \cite{wang2021reinforcement,he2021reinforcement,yang2024balanced} with large scale social networks in mind. These problems are usually formulated as a constrained Markov Decision Process (CMDP) \cite{yang2024balanced}, where the goal is to allocate incentive to maximize the social influence or opinion spreading. Our focuses are on the challenges in the multi-scale decision making and the design of the RL algorithms that incorporate domain knowledge about the social networks. These differences make our algorithm designs unique contributions to the literature.

\textbf{Dyadic structure in health care.} Social relationships between patients and carepartners are proven to be important in many critical health outcomes. Studies have shown that the patient-caregiver dyad functions as a unit, with the well-being and coping strategies of one member significantly impacting the other \cite{shin2018supporting,mcpherson2024dyadic}. The quality of this relationship can affect treatment outcomes such as medication adherence \cite{psihogios2021understanding,kostalova2022medication,gresch2017medication}, and chronic disease management \cite{visintini2023medication,li2024usability}.

\textbf{Multi-agent RL (MARL).} Our proposed approach falls into the range of the independent learners in the MARL literature \cite{oroojlooy2023review}. Previous literature on MARL in a collaborative game focuses on finding the (approximate) Nash equilibrium of the game through interacting with an unknown environment \cite{wang2022cooperative,jin2021v}. However, in our paper, we emphasize the advantage of MARL in terms of its strong interpretability and being able to make decisions in multiple time-scales.

\section{Algorithm Details}
\label{app:algo}

We provide the complete details of the proposed \texttt{MultiAgent+SurrogateRwd} algorithm as well as the baseline \texttt{SingleAgent} algorithm.

We first introduce the infinite horizon RLSVI (Randomized Least Squares Value Iteration) algorithm in Alg. \ref{alg:base} \cite{russo2018tutorial}. This algorithm is a model-free posterior sampling approach that samples a random value function from its posterior distribution, and the agent acts greedily with respect to the sampled value function. We use the infinite horizon variant of RLSVI, which perturbs the Bayesian regression parameters with a random noise $\omega'$ (line 4). We introduce temporal correlation between the current noise $w'$ and the previous noise $w$ to introduce persistence in exploration.

\begin{algorithm}[H]
    \caption{Infinite Horizon RLSVI (Inf-RLSVI)}
        \begin{algorithmic}[1]
            \STATE{Input:} discount factor $\gamma \in \mathbb{R}$, previous dataset $\mathcal{D} = (s_i, a_i, r_i)_{i = 1}^{n-1} \cup \{s_{n}\}$, previous perturbation $w \in \mathbb{R}^d$, feature mapping $\phi: \mathcal{S} \times \mathcal{A} \mapsto \mathbb{R}^d$, previous parameter $\theta \in \mathbb{R}^{d}$
            \STATE Generate regression matrix and vector
            $$
                X \leftarrow\left[\begin{array}{c}
                \phi\left(s_1, a_1\right) \\
                \vdots \\
                \phi\left(s_{n-1}, a_{n-1}\right)
                \end{array}\right] \quad y \leftarrow\left[\begin{array}{c}
                r_1+\gamma \max _{\alpha \in \mathcal{A}} \langle \phi(s_2, \alpha), \theta \rangle \\
                \vdots \\
                r_{n-1}+\gamma \max _{\alpha \in \mathcal{A}}\langle \phi(s_{n}, \alpha), \theta \rangle
                \end{array}\right]
            $$
            \STATE Estimate value function
            $$
                \bar{\theta} \leftarrow \frac{1}{\sigma^2}\left(\frac{1}{\sigma^2} X^{\top} X+\lambda I\right)^{-1} X^{\top} y \quad \mathbf{\Sigma} \leftarrow\left(\frac{1}{\sigma^2} X^{\top} X+\lambda I\right)^{-1}
            $$
            \STATE Sample $w' \sim \mathcal{N}(\gamma w, (1-\gamma^2) \mathbf{\Sigma})$ and set $\theta' = \bar \theta + w'$
            \STATE \textbf{Output:} $\theta'$ and $w'$
            % \State Choose action $A_t = \argmax_{\alpha} \langle \phi(s_{t}, \alpha), \theta_{t} \rangle$
        \end{algorithmic}
        \label{alg:base}
\end{algorithm}

We use the same hyperparameters $\lambda = 0.75$ and $\sigma = 0.5$ for all the algorithms, which achieves an overall good performance for all the algorithms.

\textbf{Additional notation.} We use $w, d, t$ to denote the week, day, and time of the decision. When we increment the time, we use $w, d, t+1$ to denote the next decisioin time right after $w, d, t$, and $w, d, t-1$ to denote the previous decision time right before $w, d, t$. Note that if $t = 1$, then $w, d, t-1$ is the evening decision time of the previous day.

\subsection{Single Agent Algorithm}

Our \texttt{SingleAgent} algorithm runs the RLSVI algorithm in Alg. \ref{alg:base} using the all the obervations available at time $w, d, t$ as the state variable:
$$
    S_{w, d, t} = 
    \left(Y_{w, d-1}^{\CARE}, Y_{w-1}^{\EDGE}, R_{w, d, t-1}^{\AYA}, \bar{Y}_{w-1}^{\AYA}, \bar{Y}_{w-1}^{\CARE}, B_{w, d, t}^{\AYA}, B_{w, d, t}^{\CARE}, A_{w, d}^{\CARE}, A_{w}^{\EDGE}\right) \in \mathbb{R}^{9}.
$$
Here we slightly abuse the notation by using $R_{w, d, t-1}^{\AYA}$ to represent the AYA adherence at half-day decision time prior to the current decision time $w, d, t$. This means that if $t = 1$, a morning decision time, then $R_{w, d, t-1}^{\AYA}$ is the AYA adherence at the previous night.

The \texttt{SingleAgent} algorithm has the three dimensional action space $\vec{a} = (a_1, a_2, a_3)^{\top} \in \{0, 1\}^{3}$, each entry corresponding to one of the three interventions. The second action $a_2$  will only be effective on a new day and the third action $a_3$ will only be effective on a new week. The feature mapping $\phi$ for the single agent algorithm is defined as
$$
    \phi(s, \vec{a}) = (1, s, a_1, a_2, a_3, s \cdot a_1, s \cdot a_2, s \cdot a_3) \in \mathbb{R}^{40}.
$$

\begin{algorithm}[hpt]
    \caption{\texttt{SingleAgent} Algorithm}
    \begin{algorithmic}[1]
        \STATE{Input:} discount factor $\gamma = 0.5$
        \STATE{Initialize:} $\theta_{1,1,1} = \mathbf{0} \in \mathbb{R}^{40}$; dataset $\mathcal{D}_{1,1,1} = \emptyset$
        \FOR{$w = 1, 2, \dots, 14$}
            \FOR{$d = 1, 2, \dots, 7$}
                \FOR{$t = 1, 2$}
                    \STATE{Call Algorithm \ref{alg:base} and update $\theta_{w,d,t}$}
                    \STATE{$\vec{a} = \argmax_{\alpha} \langle \phi(S_{w,d,t}, \alpha), \theta_{w,d,t} \rangle$}
                    \IF{$t = 1$ and $d = 1$ (New Week)}
                        \STATE{Set $A_{w}^{\EDGE} = \vec{a}_3$}
                    \ENDIF
                    \IF{$t = 1$ (New Day)}
                        \STATE{Set $A_{w,d}^{\CARE} = \vec{a}_2$}
                    \ENDIF
                    \STATE{Set $A_{w,d,t}^{\AYA} = \vec{a}_1$}
                    \STATE{Environment generates $R_{w,d,t}^{\AYA}$ and next state $S_{w,d,t+1}$}
                    \STATE{Update $\mathcal{D}_{w,d,t} = \mathcal{D}_{w,d,t-1} \cup \{(S_{w,d,t}, \vec{a}, R_{w,d,t}^{\AYA})\}$}
                \ENDFOR
            \ENDFOR
        \ENDFOR
    \end{algorithmic}
    \label{alg:single_agent}
\end{algorithm}

\subsection{MultiAgent Algorithm}

The \texttt{MultiAgent} algorithm runs an RLSVI agent for each of the three interventions. We use agent-specific feature mapping $\phi^{\AYA}, \phi^{\CARE}, \phi^{\EDGE}$ for the AYA, carepartner, and relationship agents, respectively. The state construction and the feature mapping for Q-value function are given by Table \ref{tab:state_feature}. The \texttt{MultiAgent} algorithm is described in Alg. \ref{alg:multi_agent}, where the carepartner and the relationship agents learns based on the naive rewards that are the sum of the AYA rewards over the day, and over the week, respectively (line 15 and line 18).

\begin{table}[hpt]
    \centering
    \caption{State and feature construction for the Q-value function by agent.}
    \label{tab:state_feature}
    \begin{tabular}{l|l}
        \toprule
        Agent & State or Feature Mapping \\
        \midrule
        AYA State & $S_{w, d, t}^{\AYA} = \left(R_{w, d, t-1}^{\AYA}, B_{w, d, t}^{\AYA}, Y_{w}^{\EDGE}, A_{w}^{\EDGE}\right) \in \mathbb{R}^{4}$ \\
        AYA Feature & $\phi^{\AYA}(s, a) = (1, s, a, s \cdot a) \in \mathbb{R}^{10}$ \\
        \midrule
        Carepartner State & $S_{w, d}^{\CARE} = \left(Y_{w, d-1}^{\CARE}, B_{w, d}^{\CARE}, Y_{w}^{\EDGE}, A_{w}^{\EDGE}\right) \in \mathbb{R}^{4}$ \\ 
        Carepartner Feature & $\phi^{\CARE}(s, a) = (1, s, a, s \cdot a) \in \mathbb{R}^{10}$ \\
        \midrule
        Relationship State & $S_{w}^{\EDGE} = \left(Y_{w-1}^{\EDGE}, B_{w, 1, 1}^{\AYA}, B_{w, 1}^{\CARE}, \bar{Y}_{w-1}^{\AYA}, \bar{Y}_{w-1}^{\CARE}\right) \in \mathbb{R}^{5}$ \\
        Relationship Feature & $\phi^{\EDGE}(s, a) = (1, s, a, s \cdot a) \in \mathbb{R}^{12}$ \\
        \bottomrule
    \end{tabular}
\end{table}


\begin{algorithm}[hpt]
    \caption{\texttt{MultiAgent} Algorithm}
    \begin{algorithmic}[1]
        \STATE{Input:} discount factor $\gamma^{\AYA} = 0.5$, $\gamma^{\CARE} = 0.5$, $\gamma^{\EDGE} = 0$
        \STATE{Initialize:} $\theta^{\AYA}_{1,1,1} = \mathbf{0} \in \mathbb{R}^{10}$; $\theta^{\CARE}_{1,1} = \mathbf{0} \in \mathbb{R}^{10}$; $\theta^{\EDGE}_{1} = \mathbf{0} \in \mathbb{R}^{12}$; dataset $\mathcal{D}_{1,1,1}^{\AYA} = \emptyset$; $\mathcal{D}_{1,1}^{\CARE} = \emptyset$; $\mathcal{D}_{1}^{\EDGE} = \emptyset$
        \FOR{$w = 1, 2, \dots, 14$}
            \STATE{Call Algorithm \ref{alg:base} using $\mathcal{D}_{w}^{\EDGE}, \gamma^{\EDGE}$, and update $\theta_{w}^{\EDGE}$}
            \STATE{Set $A_{w}^{\EDGE} = \argmax_{\alpha} \langle \phi^{\EDGE}(S_{w}^{\EDGE}, \alpha), \theta_{w}^{\EDGE} \rangle$}
            \FOR{$d = 1, 2, \dots, 7$}
                \STATE{Call Algorithm \ref{alg:base} using $\mathcal{D}_{w,d}^{\CARE}, \gamma^{\CARE}$, and update $\theta_{w,d}^{\CARE}$}
                \STATE{Set $A_{w,d}^{\CARE} = \argmax_{\alpha} \langle \phi^{\CARE}(S_{w,d}^{\CARE}, \alpha), \theta_{w,d}^{\CARE} \rangle$}
                \FOR{$t = 1, 2$}
                    \STATE{Call Algorithm \ref{alg:base} using $\mathcal{D}_{w,d,t}^{\AYA}, \gamma^{\AYA}$, and update $\theta_{w,d,t}^{\AYA}$}
                    \STATE{$A_{w,d,t}^{\AYA} = \argmax_{\alpha} \langle \phi^{\AYA}(S_{w,d,t}^{\AYA}, \alpha), \theta_{w,d,t}^{\AYA} \rangle$}
                    \STATE{Environment generates $R_{w,d,t}^{\AYA}$ and next state $S_{w,d,t+1}$}
                    \STATE{Update $\mathcal{D}_{w,d,t}^{\AYA} = \mathcal{D}_{w,d,t-1}^{\AYA} \cup \{(S_{w,d,t}^{\AYA}, A_{w,d,t}^{\AYA}, R_{w,d,t}^{\AYA})\}$}
                \ENDFOR
                \STATE{Compute care-partner reward $R_{w,d}^{\CARE} = \sum_{t = 1}^{2} R_{w,d,t}^{\AYA} / 2$} 
                \STATE{Update $\mathcal{D}_{w,d}^{\CARE} = \mathcal{D}_{w,d-1}^{\CARE} \cup \{(S_{w,d}^{\CARE}, A_{w,d}^{\CARE}, R_{w,d}^{\CARE})\}$}
            \ENDFOR
            \STATE{Compute relationship reward $R_{w}^{\EDGE} = \sum_{d = 1}^{7} R_{w,d}^{\CARE} / 7$}
            \STATE{Update $\mathcal{D}_{w}^{\EDGE} = \mathcal{D}_{w-1}^{\EDGE} \cup \{(S_{w}^{\EDGE}, A_{w}^{\EDGE}, R_{w}^{\EDGE})\}$}
        \ENDFOR
    \end{algorithmic}
    \label{alg:multi_agent}
\end{algorithm}

The \texttt{MultiAgent+SurrogateRwd} algorithm is described in Alg. \ref{alg:multi_agent_surrogate}. The only difference between the \texttt{MultiAgent} and \texttt{MultiAgent+SurrogateRwd} is that the later agent optimizes the surrogate reward functions, defined in Equ. (\ref{equ:game_rwd_app}) and Equ. (\ref{equ:care_rwd_app}), where the coefficients are estimated using Bayesian Ridge Regression, with the prior mean given in Table \ref{tab:prior}.

\begin{align}
    r_w^{\EDGE} = & (1, Y_{w-1}^{\EDGE}, {B}_{w, 1, 1}^{\AYA}, A_{w}^{\EDGE}, A_w^{\EDGE} \cdot Y_{w-1}^{\EDGE})\vbeta^{\EDGE} \nonumber \\
     &+ \max_{a \in \{0, 1\}}(1, Y_{w}^{\EDGE}, {B}_{w+1, 1, 1}^{\AYA}, a, a \cdot Y_{w}^{\EDGE}) \vbeta^{\EDGE}, \label{equ:game_rwd_app}
\end{align}

\begin{align}
        r_{w, d}^{\CARE} =  (1, Y_{w,d}^{\CARE}, B_{w,d+1}^{\CARE}, Y_{w-1}^{\EDGE}, A_{w,d}^{\CARE}) \vbeta^{\CARE}, \label{equ:care_rwd_app}
\end{align}

\begin{algorithm}[hpt]
    \caption{\texttt{MultiAgent+SurrogateRwd} Algorithm}
    \begin{algorithmic}[1]
        \STATE{Input:} discount factor $\gamma^{\AYA} = 0.5$, $\gamma^{\CARE} = 0.5$, $\gamma^{\EDGE} = 0$
        \STATE{Initialize:} $\theta^{\AYA}_{1,1,1} = \mathbf{0} \in \mathbb{R}^{10}$; $\theta^{\CARE}_{1,1} = \mathbf{0} \in \mathbb{R}^{10}$; $\theta^{\EDGE}_{1} = \mathbf{0} \in \mathbb{R}^{12}$; dataset $\mathcal{D}_{1,1,1}^{\AYA} = \emptyset$; $\mathcal{D}_{1,1}^{\CARE} = \emptyset$; $\mathcal{D}_{1}^{\EDGE} = \emptyset$
        \FOR{$w = 1, 2, \dots, 14$}
            \STATE{Call Algorithm \ref{alg:base} using $\mathcal{D}_{w}^{\EDGE}, \gamma^{\EDGE}$, and update $\theta_{w}^{\EDGE}$}
            \STATE{Set $A_{w}^{\EDGE} = \argmax_{\alpha} \langle \phi^{\EDGE}(S_{w}^{\EDGE}, \alpha), \theta_{w}^{\EDGE} \rangle$}
            \FOR{$d = 1, 2, \dots, 7$}
                \STATE{Call Algorithm \ref{alg:base} using $\mathcal{D}_{w,d}^{\CARE}, \gamma^{\CARE}$, and update $\theta_{w,d}^{\CARE}$}
                \STATE{Set $A_{w,d}^{\CARE} = \argmax_{\alpha} \langle \phi^{\CARE}(S_{w,d}^{\CARE}, \alpha), \theta_{w,d}^{\CARE} \rangle$}
                \FOR{$t = 1, 2$}
                    \STATE{Call Algorithm \ref{alg:base} using $\mathcal{D}_{w,d,t}^{\AYA}, \gamma^{\AYA}$, and update $\theta_{w,d,t}^{\AYA}$}
                    \STATE{$A_{w,d,t}^{\AYA} = \argmax_{\alpha} \langle \phi^{\AYA}(S_{w,d,t}^{\AYA}, \alpha), \theta_{w,d,t}^{\AYA} \rangle$}
                    \STATE{Environment generates $R_{w,d,t}^{\AYA}$ and next state $S_{w,d,t+1}$}
                    \STATE{Update $\mathcal{D}_{w,d,t}^{\AYA} = \mathcal{D}_{w,d,t-1}^{\AYA} \cup \{(S_{w,d,t}^{\AYA}, A_{w,d,t}^{\AYA}, R_{w,d,t}^{\AYA})\}$}
                \ENDFOR
                \STATE{Compute care-partner reward $\tilde{R}_{w,d}^{\CARE}$ based on Equ. (\ref{equ:care_rwd_app})} 
                \STATE{Update $\mathcal{D}_{w,d}^{\CARE} = \mathcal{D}_{w,d-1}^{\CARE} \cup \{(S_{w,d}^{\CARE}, A_{w,d}^{\CARE}, \tilde{R}_{w,d}^{\CARE})\}$}
            \ENDFOR
            \STATE{Compute relationship reward $\tilde{R}_{w}^{\EDGE}$ based on Equ. (\ref{equ:game_rwd_app})}
            \STATE{Update $\mathcal{D}_{w}^{\EDGE} = \mathcal{D}_{w-1}^{\EDGE} \cup \{(S_{w}^{\EDGE}, A_{w}^{\EDGE}, \tilde{R}_{w}^{\EDGE})\}$}
        \ENDFOR
    \end{algorithmic}
    \label{alg:multi_agent_surrogate}
\end{algorithm}


\begin{table}[hpt]
    \centering
    \caption{Prior mean for coefficients in the surrogate reward functions.}
    \label{tab:prior}
    \begin{tabular}{c|c|c|c|c|c}
    \toprule
    Agent & Intercept & $Y_w^{\EDGE}$ & $B_w^{\AYA}$  & $A_w^{\EDGE}$ & $A_w^{\EDGE} \cdot Y_w^{\EDGE}$ \\
    \midrule
    $\vbeta^{\EDGE}$ & $1$ & $1$ & $-1$ & $-1$ & $0.5$ \\
    \midrule
    \midrule
    Agent & Intercept & $Y_{w,d}^{\CARE}$ & $B_{w,d}^{\CARE}$ & $Y_{w-1}^{\EDGE}$ & $A_{w,d}^{\CARE}$  \\
    \midrule
    $\vbeta^{\CARE}$ & $1$ & $-1$ & $-1$ & $1$ & $-0.5$  \\
    \bottomrule
    \end{tabular}
    \end{table}
    
\section{The Dyadic Environment}
\label{app:testbed}

\subsection{Overview of the Simulated Dyadic Environment}

% \ziping{Give a brief description of the testbed.}

We construct a dyadic simulation environment to evaluate the performance of the proposed algorithm. The 1st order goal of the environment design is to replicate the noise level and structure that we expect to encounter in the forthcoming ADAPTS-HCT clinical trial. This noise often encompasses the stochasticity in the state transition of each participant and the heterogeneity across participants.

The environment is based on Roadmap 2.0, a care partner-facing mobile health application that provides daily positive psychology interventions to the care partner only. Roadmap 2.0 involves 171 dyads, each consisting of a patient undergone HCT (target person) and a care partner. Each participant in the dyad had the Roadmap mobile app on their smartphone and wore a Fitbit wrist tracker. The Fitbit wrist tracker recorded physical activity, heart rate, and sleep patterns. Furthermore, each participant was asked to self-report their mood via the Roadmap app every evening. A list of variables in Roadmap 2.0 is reported in Table \ref{tab:roadmap_variable}.

Roadmap 2.0 data is suitable for constructing a dyadic environment for developing the RL algorithm for ADAPTS-HCT in that Roadmap 2.0 has the same dyadic structure about the participants--post-HCT cancer patients and their care partner. Moreover, Roadmap 2.0 encompasses some context variables that align with those to be collected in ADAPTS-HCT, for example, the daily self-reported mood score.

\subsubsection{Overcoming impoverishment.} From the viewpoint of evaluating dyadic RL algorithms, this data is impoverished \cite{trella2022designing} mainly in two aspects. First, Roadmap 2.0 does not include micro-randomized daily or weekly intervention actions (i.e., whether to send a positive psychology message to the patient/care partner and whether to engage the dyad into a weekly game). Second, it does not include observations on the adherence to the medication--the primary reward signal, as well as other important measurements such as the strength of relationship quality. 
% Furthermore, Roadmap 2.0 includes dyads across all lifespan whereas ADAPTS-HCT will focus on adolescent and young adults.

To overcome this impoverishment, we construct surrogate variables from the Roadmap 2.0 data to represent the variables intended to be collected in ADAPTS-HCT. A list of substitutes is reported in Table \ref{tab:roadmap_substitutes}. Worthnoting, the substitute for the AYA medication adherence is based on the step count. There is evidence on the association between the step count and the adherence. 

% \ziping{Could we find literature to justify this?}
% \ziping{Should we discuss in detail the rationale of these substitutes?} 
We further impute the treatment effects of the intervention actions so the marginal effects after normalization, which we call the standardized treatment effects (STE), are around 0.15, 0.3, and 0.5, corresponding to small, medium, and large effect sizes in typical behavioral science studies.

\subsubsection{Constructing the dyadic environment.} We follow the environment design in \cite{li2023dyadic}, which also uses the Roadmap 2.0 data, but primarily focuses on AYA intervention and relationship intervention. We extend the environment to include the care partner intervention. Specifically, we fit a separate multi-variate linear model for each participant in the dataset with the AR(1) working correlation using the generalized estimating equation (GEE) approach \cite{ziegler2010generalized,hojsgaard2006r}. We impute the treatment effects of the intervention actions based on the typical STE around 0.15, 0.3, and 0.5, which completes a generative model for the state transitions. The environment simulates a trial by randomly sampling dyads from the dataset, and simulate their trajectories based on the actions selected by the RL algorithm. The environment details are described in Appendix \ref{app:testbed}. Our experiments primarily focus on the three vanilla testbeds corresponding to the three STEs.



%In this section, we describe our set up of the simulation testbed. 


\subsection{Using the Roadmap 2.0 Dataset}


This section outlines our approach to addressing the limitations of the Roadmap 2.0 dataset, specifically its absence of micro-randomized interventions and reward signals.

To circumvent the lack of interventions, we impute treatment effects that represent the burden of the digital interventions, assuming that frequent notifications diminish both weekly and the daily treatment effects. Based on prior literature, we choose the scale of the treatment effect to be smaller than the baseline effect of features \cite{box1987empirical}. 

To address the missing reward signals, we use directly measurable variables in Roadmap 2.0 dataset as proxies to the outcomes we will observe in the real clinical trial. We approximate AYA adherence, $R_{w,d,t}^{\AYA}$, using the 12-hourly step count from Roadmap 2.0. Previous work has found the two values to be strongly correlated \hinal{TODO cite}. Since adherence is a binary signal in the ADAPTS-HCT trial, we discretize step count into a binary variable. Furthermore, we approximate the carepartner's daily psychological distress, $Y_d^{\CARE}$, using the daily length of their sleep. Finally, the weekly relationship between the AYA and their carepartner is estimated using the self-reported mood as a surrogate. Specifically, we let  $Y_w^{\EDGE} = \mathbbm{1}\{\sum_{d = 1}^{7} \Mood_{w,d}^{\AYA} \geq \Mood^{\AYA}\} \mathbbm{1}\{\sum_{d = 1}^{7}\Mood_{w,d}^{\CARE} \geq \Mood^{\CARE}\}$. Here, $\Mood_{w,d}^{\AYA}$ is the daily self-reported mood on week $w$ and day $d$, and $\Mood^{\AYA}$ is the $q$-th quantile of the the weekly summed mood across all AYA observations. We choose the quantile level $q$ such that approximately 50\% of the dataset satisfies $Y_{w}^{\EDGE} = 1$.

Table \ref{tab:roadmap_substitutes} summarizes the main variables and their replacements from the Roadmap 2.0 dataset.  

\begin{table}[hpt]
    \centering
    \caption{Substitutes of the main variables from Roadmap 2.0 dataset.}
\resizebox{\textwidth}{!}{%
    \begin{tabular}{c|c}
        \hline
        Variables & Substitutes \\
        \hline
        \hline
        AYA adherence  & Binary step count $\mathbbm{1}\{\texttt{Step}_{w, d,t}^{\AYA} \geq \texttt{Step}^{\AYA}\}$  
        \\
        Carepartner distress & Carepartner daily length of sleep $\texttt{Sleep}_{w, d}^{\CARE}$ \\
        Weekly relationship quality & Mood indicator: $\mathbbm{1}_{\{\sum_{d} \texttt{Mood}_{w, d}^{\CARE} \geq \texttt{Mood}^{\CARE}\}} \mathbbm{1}_{\{\sum_{d} \texttt{Mood}_{w, d}^{\AYA} \geq \texttt{Mood}^{\AYA}\}}$ \\
        Effects of interventions $A_{w,d,t}^{\AYA}, A_{w,d}^{\CARE}$, $A_{w}^{\EDGE}$ & Imputed based on domain knowledge \\
        Effects of digital interventions burden $B_{w,d,t}^{\AYA}$, $B_{w,d}^{\CARE}$ & Imputed based on domain knowledge\\
        \hline
    \end{tabular}%
    }    \label{tab:roadmap_substitutes}
\end{table}


\begin{table}[hpt]
    \centering
    \caption{List of variables in Roadmap 2.0 and the measuring frequencies.}
    \begin{tabular}{c}
    \hline
    Variables\footnote{Note that all the variables are measured the same for the target person and carepartner.} \\
    \hline
    \hline
     $\texttt{Step}_{w, d, t}$: twice-daily cumulative step count\\
     $\texttt{Heart}_{w, d, t}$: twice-daily average heart rate\\
     $\texttt{Sleep}_{w, d}$: daily length of sleep\\
     $\texttt{Mood}_{w, d}$: daily self-report mood measurement\\
     \hline
    \end{tabular}
    \label{tab:roadmap_variable}
\end{table}

\subsection{Environment Model Design}

We now describe how these surrogate variables are used to build the full environment model. Our approach involves fitting two state transition models for digital intervention burden (AYA and carepartner) and three models for rewards (AYA adherence, carepartner stress, and relationship quality).

For all transition models, we fit the baseline parameters -- which represent system dynamics under no intervention --  for each dyad using its respective dataset and a generalized estimating equation \cite{hojsgaard2006r} approach. We impute the remaining parameters  using domain knowledge. Further detail on the choice of the coefficients is in Appendix \ref{sec:select_sim_params}. 

\textbf{Transition models for the AYA component: } The digital intervention burden transition for AYA follows a linear model with covariates $(B_{w,d,t}^{\AYA}, A_{w,d,t}^{\AYA}, A_{w}^{\EDGE})$.
\begin{align}
\label{equ:B_transition_AYA}
    B_{w,d,t+1}^{\AYA} \sim \theta^{\AYA}_{0} + \theta_{1}^{\AYA} B_{w,d,t}^{\AYA} + \theta_{2}^{\AYA} A_{w,d,t}^{\AYA} + \theta_{3}^{\AYA} A_{w}^{\EDGE} + \eta_{w,d,t}^{\AYA}, \nonumber\\
    \text{ where $\eta_{w,d,t}^{\AYA} \sim \mathcal{N}(0, (\omega^{\AYA})^2)$.} 
\end{align}
 

For the primary outcome, AYA adherence, we fit a generalized linear model with a sigmoid link function: 


\begin{align}
R_{w,d,t}^{\AYA} &\sim \text{Bernoulli}(\text{sigmoid}(P_{w,d,t}^{\AYA})), \nonumber \\
P_{w,d,t}^{\AYA} &= (1-M_t)\big(\beta_{0, \AM}^{\AYA} + \beta^{\AYA}_{1, \AM} R_{w,d,t-1}^{\AYA} 
+ \beta_{2,\AM}^{\AYA} Y_{w-1}^{\EDGE} 
+ \beta_{3,\AM}^{\AYA} Y_{w,d-1}^{\CARE} + \beta_{4, \AM}^{\AYA} B_{w,d,t}^{\AYA} \nonumber \\
&\quad + \tau_{0, \AM}^{\AYA} A_{w,d,t}^{\AYA} 
+ \tau_{1, \AM}^{\AYA} A_{w,d,t}^{\AYA} Y_{w-1}^{\EDGE} 
+ \tau_{2, \AM}^{\AYA} A_{w,d,t}^{\AYA} B_{w,d,t}^{\AYA}\big) \nonumber \\
&\quad + M_t\big(\beta_{0, \PM}^{\AYA} + \beta^{\AYA}_{1, \PM} R_{w,d,t-1}^{\AYA} 
+ \beta_{2,\PM}^{\AYA} Y_{w-1}^{\EDGE} 
+ \beta_{3,\PM}^{\AYA} Y_{w,d-1,t}^{\CARE} + \beta_{4, \PM}^{\AYA} B_{w,d,t}^{\AYA} \nonumber \\
&\quad + \tau_{0, \PM}^{\AYA} A_{w,d,t}^{\AYA}  
+ \tau_{1, \PM}^{\AYA} A_{w,d,t}^{\AYA} Y_{w-1}^{\EDGE} 
+ \tau_{2, \PM}^{\AYA} A_{w,d,t}^{\AYA} B_{w,d,t}^{\AYA}\big)
\label{equ:R_Transition_AYA}
\end{align}

where $M_t$ is a decision window indicator defined as:

$$
    M_t = \left\{
    \begin{array}{clll}
         0 & \text{ if } t = 2k - 1 & (\text{AM decision window}) & \text{ for } k = 1, 2, \dots  \\
         1 & \text{ if } t = 2k & (\text{PM decision window}) &\text{ for } k = 1, 2, \dots
    \end{array},\right.
$$ 
Note that we exclude any effect of relationship interventions on AYA adherence as the game is designed without reinforcements and, thus, is not supposed to directly improve adherence.



\textbf{Transition models for the carepartner component: } The digital intervention burden transition for the carepartner is a linear model:

\begin{align}
\label{equ:B_transition_care}
    B_{w,d+1}^{\CARE} = \theta^{\CARE}_{0} + \theta_{1}^{\CARE} B_{w,d}^{\CARE} + \theta_{2}^{\CARE} A_{w,d}^{\CARE} + \theta_{3}^{\CARE} A_{w}^{\EDGE} + \eta_{w,d}^{\CARE}, \nonumber\\
    \text{ where $\eta_{w,d}^{\CARE} \sim \mathcal{N}(0, (\omega^{\CARE})^2)$.}
\end{align}

For the carepartner's psychological distress level, $R^{\CARE}_d$, we fit another linear model:
\begin{align}
    Y_{w,d}^{\CARE} = 
    &\beta_{0}^{\CARE} + \beta_{1}^{\CARE} Y_{w,d-1}^{\CARE} + \beta_{2}^{\CARE} R_{w,d,t-1}^{\AYA}  + 
    \beta_{3}^{\CARE} Y_{w-1}^{\EDGE} + \beta_{4}^{\CARE} B_{w,d}^{\CARE} + \nonumber \\
    &\quad \tau_{0}^{\CARE} A_{w,d}^{\CARE} +  
    \tau_{1}^{\CARE} A_{w,d}^{\CARE} Y_{w-1}^{\EDGE} +   
    \tau_{2}^{\CARE} A_{w,d}^{\CARE} B_{w,d}^{\CARE}  + \epsilon_{w,d}^{\CARE} \label{equ:R_transition_care}
\end{align}
where $\epsilon_{w,d}^{\CARE} \sim \mathcal{N}(0, (\sigma^{\CARE})^2)$.  Similar to (\ref{equ:R_Transition_AYA}), we do not include relationship intervention $A_{w-1}^{\EDGE}$.

\textbf{Transition model for the weekly relationship: } For the shared component, we only fit a transition model for the reward, which is the weekly relationship quality. Specifically, we fit a generalized linear model with a sigmoid link function: 

\begin{align}
Y_{w+1}^{\EDGE} \sim \text{Bernoulli}(\text{sigmoid}\left( \beta_{0}^{\EDGE} + \beta_{1}^{\EDGE}Y_{w}^{\EDGE}  + \beta_{2}^{\EDGE} \bar{R}_{w}^{\AYA} + \beta_{3}^{\EDGE} \bar{R}_{w}^{\CARE} \right. \nonumber \\
\left. + \tau_0^{\EDGE} A_{w}^{\EDGE} + \tau_1^{\EDGE} A_{w}^{\EDGE} (B_{w,d}^{\CARE} + B_{w,d,t}^{\AYA}))\right)
\label{equ:R_transition_rel}
\end{align}

where $\bar{R}_{w}^{\AYA} = \sum_{d=1}^{7} \sum_{t=1}^{2} \gamma^{14 - (7(w-1) + d) + 2(t-1)} R_{w,d,t}^{\AYA}$ is the exponentially weighted average of adherence within week $w$, and $\bar{R}_{w}^{\CARE} = \sum_{d=1}^{7} \gamma^{7-d} Y_{w,d}^{\CARE}$ is the exponentially weighted average of carepartner distress within week $w$. 

\subsection{Selecting Environment Model Parameters}
\label{sec:select_sim_params}

We list all the parameters that must be either imputed based on domain knowledge or estimated from the existing dataset. 
%We must impute all parameters related to the treatment effects because the existing dataset contains no target intervention. For the same reason, we impute all the parameters relating to the digital intervention burden $B_{w,d,t}^{\AYA}$, $B_{w,d}^{\CARE}$.

\begin{enumerate}
\item The baseline transition parameters $\beta$'s can be estimated directly from the dataset:
    \begin{enumerate}
        \item AYA state transition: $\vbeta^{\AYA}_{\AM} = (\beta_{0, \AM}^{\AYA}, \beta_{1, \AM}^{\AYA}, \beta_{2, \AM}^{\AYA}, \beta_{3, \AM}^{\AYA}, \beta_{4, \AM}^{\AYA})$ and $\vbeta^{\AYA}_{\PM} = (\beta_{0, \PM}^{\AYA}, \beta_{1, \PM}^{\AYA}, \beta_{2, \PM}^{\AYA}, \beta_{3, \PM}^{\AYA}, \beta_{4, \PM}^{\AYA})$.
        \item Carepartner state transition: $\vbeta^{\CARE} = (\beta_{0}^{\CARE}, \beta_{1}^{\CARE})$.
        \item Relationship transition: $\vbeta^{\EDGE} = (\beta_{0}^{\EDGE}, \beta_{1}^{\EDGE}, \beta_{2}^{\EDGE}, \beta_{3}^{\EDGE})$.
        % \item Hazard model parameters: $\vgamma = (\gamma_{0, 1}, \dots, \gamma_{0, 98}, \gamma_1, \gamma_2, \gamma_3)$
    \end{enumerate}
\item Imputed or tuned based on domain knowledge:
    \begin{enumerate}
        \item Burden transitions: coefficients $\boldsymbol{\theta}^{\AYA} = (\theta^{\AYA}_{0}, \theta^{\AYA}_{1}, \theta^{\AYA}_{2}, \theta^{\AYA}_{3})$, $\boldsymbol{\theta}^{\CARE} = (\theta^{\CARE}_{0}, \theta^{\CARE}_{1}, \theta^{\CARE}_{2}, \theta^{\CARE}_{3})$; burden noise variance $\omega^{\AYA}$ and $\omega^{\CARE}$.
        \item Main effects of burden: $\beta_{4, \AM}^{\AYA}, \beta_{4, \PM}^{\AYA}$, and $\beta_{4}^{\CARE}$.
        \item AYA treatment effects: $\{\tau_{i, \AM}^{\AYA}\}_{i = 0}^{2}$, $\{\tau_{i, \PM}^{\AYA}\}_{i = 0}^{2}$ and $\{\sigma_{i, \AM}^{\AYA}\}_{0 = 1}^{2}$, $\{\sigma_{i, \PM}^{\AYA}\}_{i = 0}^{2}$.
        \item Carepartner treatment effects: $\{\tau_{i}^{\CARE}\}_{i = 0}^{2}$ and $\{\sigma_{i}^{\CARE}\}_{i = 0}^{2}$.
        \item Relationship treatment effects: $\tau^{\EDGE}$ and $\sigma^{\EDGE}$.
        % \item Disengagement effect: $\xi_{0, d}, \xi_{1, d}, \xi_{2, d}$.
    \end{enumerate}
\end{enumerate}

\textbf{Fitting parameters (1a-d):} We estimate the baseline transition parameters under no intervention directly from the Roadmap 2.0 dataset. For the parameters in Equation (\ref{equ:R_Transition_AYA}), we have the correspondences 
$\beta_{i, \AM}^{\AYA} = \hat{\beta}_{i, \AM}^{\AYA}$ and $\beta_{i, \PM}^{\AYA} = \hat{\beta}_{i, \PM}^{\AYA}$ for $i = 0, 1, \dots, 3$, where $\hat{\beta}_{i, \AM}^{\AYA}$ and $\hat{\beta}_{i, \PM}^{\AYA}$ are fitted coefficients obtained using the generalized estimating equation (GEE) approach. Since we assume that app burden only moderates the effects of AYA interventions without directly influencing adherence, we set $\beta_{4, \AM}^{\AYA} = \beta_{4, \PM}^{\AYA} = 0$. Similarly, for parameters in Equation (\ref{equ:R_transition_care}), the correspondence is $\beta_{i}^{\CARE} = \hat{\beta}_{i}^{\CARE}$ for $i = 0, \dots, 3$, and we set $\beta_{4}^{\CARE} = 0$ under the same assumption for carepartner distress. For the relationship quality model in Equation (\ref{equ:R_transition_rel}), the correspondence is $\beta_{i}^{\EDGE} = \hat{\beta}_{i}^{\EDGE}$ for $i = 0, \dots, 3$. Based on domain knowlege, we also truncate the parameters as follows: $\beta_{2, *}^{\AYA} = \max\{0, \hat{\beta}_{2, *}^{\AYA}\}$, reflecting the assumption that weekly relationship quality non-negatively influences AYA adherence, $\beta_{3, *}^{\AYA} = \min\{0, \hat{\beta}_{3, *}^{\AYA}\}$, as carepartner distress is expected to negatively influence adherence, and $\beta_{3}^{\EDGE} = \min\{0, \hat{\beta}_{3}^{\EDGE}\}$ as carepartner distress is expected negatively impact relationship quality.

% We can fit the baseline transition parameters under no intervention directly from the Roadmap 2.0 dataset. Specifically, for parameters in (\ref{equ:R_Transition_AYA}), we have the following correspondence: $\beta_{i, \AM}^{\AYA} = \hat{\beta}_{i, \AM}^{\AYA}$, and $\beta_{i, \PM}^{\AYA} = \hat{\beta}_{i, \PM}^{\AYA}$ for all $i = 0, 1, \dots, 3$, where $\hat{\beta}_{i, 0}^{\AYA}$ and $\hat{\beta}_{i, 1}^{\AYA}$ are fitted coefficients based on GEE approaches. We believe that app burden only moderates the effects of AYA interventions. Therefore, we set $\beta_{4, \AM}^{\AYA} = \beta_{4, \PM}^{\AYA} = 0$. For parameters in (\ref{equ:R_transition_care}), we have direct correspondence--$\beta_{i}^{\CARE} = \hat{\beta}_{i}^{\CARE}$ for $i = 0,\dots,3$ and $\beta_{4}^{\CARE} = 0$. Similarly, for parameters in (\ref{equ:R_transition_rel}), we have $\beta_{i}^{\EDGE} = \hat{\beta}_{i}^{\EDGE}$ for $i = 0, \dots, 3$.


% $\beta_{2, *}^{\AYA} = \max\{0, \hat \beta_{2, *}^{\AYA}\}$ because weekly relationship quality should be positively related to AYA adherence, and $\beta_{3, *}^{\AYA} \min\{0, \beta_{3, *}^{\AYA}\}$ because carepartner distress should be negatively related to AYA adherence. Also, $\beta_{3}^{\EDGE} = \min\{0, \hat \beta_{3}^{\EDGE}\}$ because carepartner distress should be negatively related to relationship quality.

\textbf{Imputing Burden Transitions (2a):}
We set $
  \theta_{1}^{\AYA} = \tfrac{13}{14}, \quad \theta_{1}^{\CARE} = \tfrac{6}{7},
$ so that the memory of digital burden spans roughly one week for both AYA and carepartner. We choose
$\theta_{2}^{\AYA} = 5\,\theta_{3}^{\AYA} = 1, 
  \quad
  \theta_{2}^{\CARE} = 5\,\theta_{3}^{\CARE} = 1$ so that daily interventions exert five times more burden than the weekly relationship intervention. The intercepts are $
  \theta_{0}^{\AYA} = 0.2, 
  \quad
  \theta_{0}^{\CARE} = 0.2$, and chosen so that participants have around a 20\% baseline burden even without an intervention. We set
$
  \omega^{\AYA} = \omega^{\CARE} = 2.4
$ to obtain a moderate noise-to-signal ratio, set so that 
\(
    (\theta_{1}^{\AYA} + \theta_{2}^{\AYA}) / \omega^{\AYA} 
    \approx 0.5
\).
We then truncate burdens at zero and standardize them separately for AYA and carepartner by simulating 10,000 steps with random interventions.

\textbf{Imputing main effects of app burden (2b).} We set $\beta_{4, \AM}^{\AYA} = \beta_{4, \PM}^{\AYA} = \beta_{4}^{\CARE} = 0$ based on the assumption that digital app intervention burden does not directly affect AYA adherence or carepartner distress, unless through moderating the digital interventions.

\textbf{Imputing treatment Effects (2c--2f):}
Since digital health environments are noisy, treatment terms likely have a lower effect on transitions than the baseline transitions under no intervention. Hence, we scale all intervention effects relative to the baseline effects using a single, global hyperparameter $C_{\text{treat}}$. 

For each time of the day (AM or PM), the AYA intervention increases adherence by $\tau_{0, *}^{\AYA} = C_{\text{treat}} \bigl|\beta_{1, *}^{\AYA}\bigr|$, where $* \in \{\mathrm{AM}, \mathrm{PM}\}$ and $\beta_{1, *}$ is the corresponding baseline coefficient estimated from Roadmap 2.0. 

We further define $\bigl|\beta_{1, *}^{\AYA}\bigr|$ and $\tau_{\text{burden}, *}^{\AYA} = -C_{\text{treat}} \bigl|\beta_{1, *}^{\AYA}\bigr|$ because the AYA intervention's effectiveness can be increased by good relationship quality and decreased by high digital-intervention burden.

To account for individual heterogeneity across dyads, each treatment-effect coefficient has an associated random effect with variance $\sigma_{0, *}^{\AYA} = C_{\text{treat}} \sigma_{\beta_{1, *}^{\AYA}}$, where $\sigma_{\beta_{1, *}^{\AYA}}$ is the empirical standard deviation across dyads of the baseline coefficient $\beta_{1, *}^{\AYA}$. 

For carepartner interventions, the main effect on distress is scaled as $\tau_{0}^{\CARE} = -C_{\text{treat}} \bigl|\beta_{1}^{\CARE}\bigr|$, where the negative sign is due to the intervention reducing distress. Lastly, the effect of the weekly relationship intervention on improving relationship quality is given by $\tau^{\EDGE} = C_{\text{treat}} \bigl|\beta_{1}^{\EDGE}\bigr|$.

% \textbf{Imputing treatment Effects (2c--2e):}
%  Since digital health environments are noisy, treatment terms likely have a lower effect on the transitions than the baseline transitions under no intervention. Hence, we scale all intervention effects relative to the baseline effects using a single, global hyperparameter $C_{\Treat}$. 

% \begin{enumerate}
%     \item AYA Daily Interventions (AM/PM)
%         \begin{itemize}
%             \item Main Effect: For each time of day (AM or PM), the AYA intervention increases adherence by $$\tau_{0, *}^{\AYA} \;=\; C_{\text{treat}}\;\bigl|\beta_{1, *}^{\AYA}\bigr|$$
%       where $ * \in \{\mathrm{AM},\,\mathrm{PM}\}$. Recall that $\beta_{1, *}$ is the corresponding baseline coefficient estimated from data. The absolute value ensures a positive effect on adherence of the intervention.
%       \item Interactions with Relationship and Burden: The AYA intervention's effectiveness can be increased by good relationship quality and decreased by high digital-intervention burden. Hence,
%       $$\tau_{\text{rel}, *}^{\AYA} \;=\; C_{\text{treat}}\;\bigl|\beta_{1, *}^{\AYA}\bigr|
%       \quad\text{and}\quad
%       \tau_{\text{burden}, *}^{\AYA} \;=\; -\,C_{\text{treat}}\;\bigl|\beta_{1, *}^{\AYA}\bigr|.$$
%     \item Random Effect Variances: To account for individual heterogeneity across dyads, each treatment-effect coefficient has an associated random effect whose variance. We also scale this variance by $$\sigma_{0,*}^{\AYA} \;=\; C_{\text{treat}}\;\sigma_{\beta_{1,*}^{\AYA}}$$
%     where $\sigma_{\beta_{1,*}^{\AYA}}$ is the empirical standard deviation across dyads of the baseline coefficient \(\beta_{1,*}^{\AYA}\).
%         \end{itemize}
% \item Carepartner Daily Interventions: We perform similar scaling to the carepartner interventions, which are designed to reduce distress. The main effect of a care-partner intervention on distress becomes
% $$\tau_{0}^{\CARE} \;=\; -\,C_{\text{treat}}\;\bigl|\beta_{1}^{\CARE}\bigr|$$
% where the negative sign reflects a reduction in distress due to the intervention.
% \item Weekly Game Intervention: Lastly, the effect of the weekly game on improving relationship quality is:

%  $$ \tau^{\EDGE} \;=\; C_{\text{treat}}\;\bigl|\beta_{1}^{\EDGE}\bigr|.
% $$
% \end{enumerate}


We summarize the imputation design in Table \ref{tab:imputation}. 
%and generate a list of all tuning parameters in Table \ref{tab:hyper_param}.

\begin{table}[hpt]
    \centering
    \caption{Summary of burden transition design and treatment effects design.}
    \begin{tabular}{c|>{\centering\arraybackslash}p{0.4\textwidth}} 
    \hline
    \multicolumn{2}{c}{Burden transition}\\
    \hline
    \hline
    Intercept $\theta_{0}^{\AYA}$  & Based on domain knowledge $\theta_{0}^{\AYA} = 0.2$ \\
    Intervention coefficients $\theta_2^{\AYA}, \theta_3^{\AYA}$ & $\theta_2^{\AYA} = 5\theta^{\AYA}_{3} = 1$ (Because relationship intervention produces lower burden) \\
    Noise standard deviation $\omega^{\AYA}$ & Based on the typical noise-to-signal ratio $\omega^{\AYA} = 2.4$ \\
    \hline
    \multicolumn{2}{c}{Treatment effect for twice-daily adherence transition (* stands for AM or PM)}\\
    \hline
    \hline
    Main effect of AYA intervention $\tau_{0, *}^{\AYA}$ & Hyper-parameter $\tau_{0, *}^{\AYA} = C_{\Treat}|\beta_{1, *}^{\AYA}| $ \\
    Rel. and AYA int. interaction $\tau_{2, *}^{\AYA}$ & Hyper-parameter $\tau_{1, *}^{\AYA} = C_{\Treat}|\beta_{1, *}^{\AYA}|$ \\
    Burden and AYA int. interaction $\tau_{4, *}^{\AYA}$ & Hyper-parameter $\tau_{2, *}^{\AYA} = C_{\Treat}|\beta_{1, *}^{\AYA}|$ \\
    Random treatment variance $\{\sigma_{i, *}^{\AYA}\}_{i = 0}^5$ & Scales with the variance of $\beta_{1, *}^{\AYA}$: $\sigma_{i, *}^{\AYA} = \tau_{i, *}^{\AYA} \cdot  \sigma_{\beta_{1, *}^{\AYA}} / |\beta_{1, *}^{\AYA}|$ \\
    \hline
    \multicolumn{2}{c}{Treatment effect for weekly relationship transition}\\
    \hline
    \hline
    Main effect of relationship int. $\tau^{\EDGE}$ & Hyper-parameter $\tau^{\EDGE} = C_{\Treat}|\beta_1^{\EDGE}|$  \\
    \hline
    % \multicolumn{2}{c}{Disengagement effect}\\
    % \hline
    % \hline
    % Disengagement effect $\xi_{0, d}, \xi_{1, d}, \xi_{2, d}$ & $\xi_{0, d} = \frac{1}{10} |\gamma_{0, d}|$, $\xi_{1, d} = \frac{1}{5} |\gamma_{0, d}|$ and $\xi_{1, d} = \frac{1}{25} |\gamma_{0, d}|$ \\ 
    \end{tabular}
    \label{tab:imputation}
\end{table}

\textbf{Tuning $C_{\Treat}$: } We tune the hyperparameter  $C_{\Treat}$ such that the standardized treatment effects (STE) are around 0.15, 0.3, and 0.5, where STE is defined as:
\begin{equation}
    \operatorname{STE}(C_{\Treat}) = \frac{\mathbb{E}\left[\mathbb{E}[\text{CR}(\pi^*_{e}) \mid e] - \mathbb{E}[\text{CR}(\pi_0, e) \mid e]\right]}{\sqrt{\operatorname{Var}(\mathbb{E}[\text{CR}(\pi_0, e) \mid e])}},
    \label{equ:STE}
\end{equation}
Here, $e$ corresponds to the resulting environment model for dyad $e$ when the hyperparameter is set to be $C_{\Treat}$, and $\pi_e^*$ is the optimal policy for dyad $e$. $\text{CR}(\pi, e)$ is the cumulative rewards earned by running policy $\pi$ on dyad $e$, and $\pi_0$ is the reference policy that always chooses action 0 for all components. 
% STE is defined as the average gap between the cumulative rewards under the optimal policy for each dyad and the cumulative rewards under the reference policy. It is normalized by the standard deviation of the expected cumulative rewards under $\pi_0$ over the distribution of dyads.

Figure \ref{fig:ste} plots the value of the hyperparameter versus the STE computed using the optimal policy in the environment defined by the hyperparameter. We outline our approximation of the optimal policy in Appendix \ref{sec:optpol}. By default, we choose an environment with mediator effect = 1. This results in three dyadic environments, which we summarize in Table \ref{tab:test-bed}.

\begin{table}[ht]
    \centering
    \caption{Summary of all testbeds}
    \begin{tabular}{c|c}
     Treatment effect size & Value of $C_{\Treat}$ \\
    \hline
       0.15 (Small)  & 0.2 \\
       0.3 (Medium) & 0.3 \\
       0.5 (Large) & 0.5 \\
    \end{tabular}
    \label{tab:test-bed}
\end{table}

\begin{figure}[hpt]
    \centering
    \includegraphics[width=0.8\textwidth]{Plots/STE/treat-vs-ste.png}
    \caption{Relationship between the hyperparameters and the STE, categorized by the mediator effect value.}
    \label{fig:ste}
\end{figure}

\subsection{Optimal Policy Approximation}
\label{sec:optpol}

% In this section, we outline our procedure for approximating the optimal policy used for generating the STE, which is defined as the the average gap between the cumulative rewards obtained by the optimal policy and those obtained by the reference policy. 

To approximate the optimal policy, we generate a dataset under a random policy with $P(A_{w,d,t}^{\AYA}=1)= P(A_{w,d}^{\CARE}=1) = P(A_w^{\EDGE}=1)=0.5$ and apply offline Q-learning on this dataset. To make the computation tractable, we discretize and subset the features. Specifically, we use six features: the intercept, AYA adherence, carepartner distress, AYA burden, carepartner burden, and relationship quality. Numerical features (carepartner distress, AYA burden, and carepartner burden) are discretized into 10 bins.

Finally, we evaluate the performance of this approximation against other baseline policies, including micro-randomized actions with fixed probabilities of 0.5, 0.6, 0.7, 0.8, and 0.9. Our approximation consistently outperforms these baselines.


\subsection{Evidence of the Need for Collaboration in the Dyadic Environment}
\label{app:evidence-collab}
% We outline our procedure for verifying that the dyadic environment requires collaboration.

To show that each agent impacts the performance of other agents, we consider the following toy setting.
% We have two random algorithms: 1. 
We fix the care partner agent's randomization probability at 0.5 and vary the AYA agent’s probability to be 0.25 and 0.75. Then, for each fixed AYA agent's probability, we identify the value of the relationship agent’s probability that maximizes average weekly adherence. We find that this relationship probability changes from $1.0$ to $0.0$ when we change AYA agent's probability from 0.25 to 0.75. 

We repeat this experiment for the care partner agent by fixing the AYA agent’s probability at 0.5 and varying the relationship agent’s probability to be 0.25 and 0.75. Similarly, we find that the care partner agent's probability that maximizes adherence changes from 0.6 to 0.5 when we vary the relationship probability from 0.25 to 0.75.

These results indicate that the agents must change their behavior to account for the other agents' behavior.

% set  the randomization probability of the AYA intervention to be 0.25 and 0.75, and see whether the optimal level of randomization probability for the relationship agent is different given a fixed randomization probability of 0.5 for the care partner agent Fig. \ref{fig:MRT_collaboration} (a, b). A similar experiment is conducted for the collaboration between the care partner and the relationship agent in Fig. \ref{fig:MRT_collaboration} (c, d). We see that the optimal level of randomization probability for the relationship agent changes for different AYA randomization probability, and the optimal randomization probability for the care partner agent changes for different game randomization probability. This indicates that the agent must change their behavior to account for the other agent's behavior.

% \ziping{Get rid of the figures.} 

% \begin{figure}[hpt]
%     \centering
%     \includegraphics[width=1\textwidth]{Plots/MRT_collaboration.pdf}
%     \caption{\textbf{(a, b)}: average weekly sum of adherence under different randomization probability for the relationship agent given a fixed probability for AYA and Care partner. \textbf{(c, d)}: average weekly sum of adherence under different randomization probability for the Care partner agent given a fixed probability for AYA and Game.}
%     \label{fig:MRT_collaboration}
% \end{figure}

\section{Additional Results}

\label{app:additional_results}
\subsection{Ablation Study}

\paragraph{No Mediator Effect} The improvement from using a surrogate reward is through the effects of the mediator variables. For example, the relationship intervention $A_{w}^{\EDGE}$ improves the mediator relationship, which may improve the primary outcome, medication adherence. The care-partner intervention $A_{w,d}^{\CARE}$ mitigates the distress, which may improve relationship. In Fig. \ref{fig:mediator0}, we run the all three algorithms under a testbed variant for which we force the above two mediator effects to be 0, i.e., no effect from relationship to adherence or effect from distress to relationship. In this testbed variant, \texttt{MutiAgent+SurrogateRwd} performs the same as \texttt{MutiAgent}--there is no cost of reward learning under no mediator effect.

\begin{figure}[hpt]
    \centering
    \begin{subfigure}[b]{0.31\textwidth}
        \includegraphics[width=1\textwidth]{Plots/Experiments/015/All_Rewards_Mediator0.pdf}
        \caption{STE 0.15}
    \end{subfigure}
    \begin{subfigure}[b]{0.31\textwidth}
        \includegraphics[width=1\textwidth]{Plots/Experiments/03/All_Rewards_Mediator0.pdf}
        \caption{STE 0.3}
    \end{subfigure}
    \begin{subfigure}[b]{0.31\textwidth}
        \includegraphics[width=1\textwidth]{Plots/Experiments/05/All_Rewards_Mediator0.pdf}
        \caption{STE 0.5}
    \end{subfigure}
    \caption{Cumulative rewards improvement over the uniform random policy for all three components under the testbed without the effect of care-partner distress onto relationship quality or the effect of relationship quality onto AYA's adherence.}
    \label{fig:mediator0}
\end{figure}

\paragraph{Other Testbed Variants.} To further violate the assumptions made from the causal diagram, we made the following two changes to test the robustness of our proposed algorithm: 1) we add a direct effect from care-partner psychological distress to AYA medication adherence; 2) we generate random mediator effects, effect from relationship to adherence and effect from distress to relationship. This later one violates the monotonicity assumptions learned from principles.

\subsection{Collaboration of Multi-Agent RL} 

We train each individual agent in the \texttt{MultiAgent+SurrogateRwd} algorithm over 1000 dyads under the STE 0.5 environment, while fixing the randomization probability of the other agents. We denote the randomization probability of the AYA agent, care partner agent, and relationship agent as $p^{\AYA}$, $p^{\CARE}$, and $p^{\EDGE}$ respectively.

We first train the relationship agent while fixing $p^{\CARE} = 0.5$. We see that the average probability of sending an intervention for the relationship agent is 0.57 and 0.42 under $p^{\AYA} = 0.25$ and $0.75$, respectively. This indicates that the relationship agent learns to \textit{reduce} the intervention probability when the AYA agent is more likely to send an intervention. 

Similarly, we train the care partner agent while fixing $p^{\AYA} = 0.5$. We see that the average probability of sending an intervention for the care partner agent is 0.61 and 0.45 under $p^{\EDGE} = 0.25$ and $0.75$, respectively. This indicates that the care partner agent learns to \textit{reduce} the intervention probability when the relationship agent is more likely to send an intervention.