\section{Weakest Pre-Expectations for Probabilistic Programs}
\label{sec:programs}

In this section we define programming language $\pWhile$ and its weakest pre-expectation semantics based on \emph{Lebesgue} integrals as defined in~\cite{DBLP:conf/setss/SzymczakK19}.

\subsection{Program Syntax}
\label{sec:programs:syntax}

For the rest of the paper we fix a finite%
\footnote{Once $\pVars$ is fixed we can only write programs with at most $|\pVars|$ distinct variables. However, as we never make any assumptions about the size of $\pVars$, our theory applies to programs with arbitrarily many variables.
Previous work~\cite{DBLP:conf/setss/SzymczakK19} has considered an infinite $\pVars$, but this requires defining a measure space on the infinite-dimensional $\nonNegReals^\pVars$, which is somewhat more involved.
}
set $\pVars = \{\pVar,\pVarb,\ldots\}$ of program variables.
%
A \emph{(program) state} is a variable valuation $\pSt \in \pStates$, where $\pStates$ is a shorthand for the set of functions $\pVars \to \nnReals$.
Notice that our program variables range over the \emph{non-negative} reals.
The restriction to non-negative variables is for technical convenience and not essential, see \Cref{rem:nonNeg}.

To obtain a well-defined weakest pre-expectation semantics of programs involving continuous sampling, we need to have some measure-theoretic fundamentals in mind, provided in\iftoggle{arxiv}{ \Cref{sec:prelims:measure}}{~\cite{arxiv}}.
Suffice it to say here that we consider the standard Borel \proseSigmaAlgebra and Lebesgue measure $\lebmes$ on $\pStates$. Lebesgue integrals of a measurable function $\fun \colon \reals \to \ennReals$ over a measurable set $\measurableSet \subseteq \reals$ are denoted by $\int_\measurableSet \fun \, d\lebmes$, or $\int_\measurableSet \fun(x) \, d\lebmes(x)$.
We explicitly allow Lebesgue integrals to evaluate to $\infty$.

Now let $\aExps$ be a set of measurable functions of type $\pStates \to \nnReals$ %called \emph{arithmetic expressions}, 
and let $\guards$ be a set of measurable functions of type $\states \to \bools$. %called \emph{guards}.
Elements of $\aExps$ and $\guards$ are called \emph{arithmetic expressions} and \emph{guards}, resp.

\begin{definition}[Probabilistic Programs]
	\label{def:pwhile}
    \newcommand{\syntaxDescr}[1]{\text{\textcolor{gray}{(#1)}}}
    %
    Programs $\prog$ in the set $\pWhileWith{\aExps}{\guards}$ of programs with arithmetic expressions from $\aExps$ and guards from $\guards$ adhere to the following grammar:
    %
	\begin{align*}
		\prog \quad\grammarSymb\quad &\SKIP & &\syntaxDescr{effectless program} \\[-1pt]
		\mmid &\DIVERGE & &\syntaxDescr{nonterminating program} \\[-1pt]
		\mmid &\ASSIGN{\pVar}{\aExp} & &\syntaxDescr{assignment; $\pVar \in \pVars, \aExp \in \aExps$} \\[-1pt]
		\mmid &\OBSERVE{\guard} & &\syntaxDescr{conditioning; $\guard \in \guards $} \\[-1pt]
		\mmid &\UNIFASSIGN{\pVar} & &\syntaxDescr{sample from real interval $\uIval$; $\pVar \in \pVars$} \\[-1pt]
		\mmid &\ITE{\guard}{\prog}{\prog} & &\syntaxDescr{conditional choice; $\guard \in \guards$} \\[-1pt]
		\mmid &\PCHOICE{\prog}{\prob}{\prog} & &\syntaxDescr{probabilistic choice; $\prob \in \uIval \cap \rats$} \\[-1pt]
		\mmid &\SEQ{\prog}{\prog} & &\syntaxDescr{sequential composition} \\[-1pt]
		\mmid &\WHILE{\guard}{\prog} & &\syntaxDescr{while loop; $\guard \in \guards$}
	\end{align*}
	%
	If $\aExps$ and $\guards$ are the sets of \emph{all} measurable functions of the corresponding type, we write $\pWhile$ instead of $\pWhileWith{\aExps}{\guards}$.
	%
	A program not containing while-loops is called \emph{loop-free}.\qedDef
\end{definition}

Let us briefly go over each construct, all of which are standard.
$\SKIP$ does nothing.
$\DIVERGE$ is a non-terminating program, i.e., behaves like $\WHILE{\boolConstTrue}{\SKIP}$.
$\ASSIGN{\pVar}{\aExp}$ assigns the value of the arithmetic expression $\aExp$ evaluated in the current state to variable $\pVar$.
$\UNIFASSIGN{\pVar}$ assigns to $\pVar$ a value drawn from the continuous uniform $\uIval$-distribution.
$\OBSERVE{\guard}$  \emph{conditions} the program execution on the guard $\guard$ being true.
$\PCHOICE{\prog_1}{\prob}{\prog_2}$ executes $\prog_1$ with probability $\prob$, otherwise $\prog_2$.
The assumption $\prob \in \uIval \cap \rats$ is to avoid defining a dedicated syntax for probabilities later on in \Cref{sec:syntax}.
$\ITE{\guard}{\prog_1}{\prog_2}$, $\SEQ{\prog_1}{\prog_2}$, and $\WHILE{\guard}{\prog}$ are standard conditional choices, sequential compositions, and while-loops, respectively.
See\iftoggle{arxiv}{ \Cref{app:redudanciesSyntax}}{~\cite{arxiv}} for additional remarks.

\begin{rem}[More General Distributions]
    \label{rem:generalDist}
    In theory, the restriction to uniform $\unitInterval$-distributions does not limit expressiveness:
    Arbitrary distributions can be simulated by sampling from $\unitInterval$ and applying the inverse
	\emph{cumulative distribution function} (CDF) of the target distribution~\cite{DBLP:conf/setss/SzymczakK19}.
    However, to obtain decidability results, we further restrict the syntax of arithmetic expressions and guards to a class of functions expressible in first-order (FO) real arithmetic, see \Cref{sec:syntax}.
    Distributions whose inverse CDF belongs to this class include triangular, trapezoidal, U-quadratic, and Kumaraswamy distributions.
    On the other hand, distributions with transcendental inverse CDF (Gaussian, Laplace, etc.) do not reside in this class.
    We mention two future directions to address this issue:
    (i) leverage heuristics implemented in modern SMT solvers to discharge the generated verification conditions, even if they do not belong to a decidable theory,
    (ii) soundly over/under-approximate the transcendental inverse CDF by FO-expressible algebraic functions.
\end{rem}

\begin{rem}[Soft Conditioning]
	\label{rem:softConditioning}
    Our syntax has native support for \emph{hard conditioning} ($\KWOBSERVE$). Bounded \emph{soft conditioning} (scoring) --- multiplying the current execution with a weight in $\uIval$ --- can be simulated using $\UNIFASSIGN{\pVar}$ and $\KWOBSERVE$, see~\cite[Lemma 5]{DBLP:conf/setss/SzymczakK19} for details.
\end{rem}


\subsection{Weakest Pre-Expectation Semantics}
\label{sec:programs:wp}

\begin{table}[t]
    \caption{
    	Inductive definition of weakest (liberal) pre-expectations for post-expectation $\ex$~\cite{DBLP:conf/setss/SzymczakK19}.
    }
    \label{tab:original}
    \begin{adjustbox}{max width=\textwidth}
        \def\arraystretch{1.2} % increase space between rows
        \begin{tabular}{l l l} 
            \toprule
            $\prog$ & $\wp{\prog}{\ex}\quad$\textcolor{gray}{where $\ex\in\expsmeas$} & $\wlp{\prog}{\ex}\quad$ ~~\textcolor{gray}{where $\ex\in\bexpsmeas$}  \\
            \midrule
            $\SKIP$ & $\ex$ & $\ex$ \\
            %
            $\DIVERGE$ & $0$ & $1$ \\
            %
            $\ASSIGN{\pVar}{\aExp}$ & $\exSubsGen$ & $\exSubsGen$ \\
            %
            $\UNIFASSIGN{\pVar}$ & $\lam{\st}{\int_\uIval \ex(\pStUpdate{\st}{\pVar}{\xi}) \,d\lebmes(\xi)}$ & $\lam{\st}{\int_\uIval \ex(\pStUpdate{\st}{\pVar}{\xi}) \,d\lebmes(\xi)}$ \\
            %
            $\OBSERVE{\guard}$     &$\iv{\guard} \cdot \ex$ &$\iv{\guard} \cdot \ex$ \\
            %
            $\ITE{\guard}{\prog_1}{\prog_2}$ & $\iv{\guard} \cdot \wp{\prog_1}{\ex}+\iv{\neg\guard} \cdot \wp{\prog_2}{\ex}$ & $\iv{\guard} \cdot \wlp{\prog_1}{\ex} + \iv{\neg\guard} \cdot \wlp{\prog_2}{\ex}$ \\
            %
            $\PCHOICE{\prog_1}{\prob}{\prog_2}$ & $\prob \cdot \wp{\prog_1}{\ex} + (1{-}\prob) \cdot \wp{\prog_2}{\ex}$ & $\prob \cdot \wlp{\prog_1}{\ex} + (1{-}\prob) \cdot \wlp{\prog_2}{\ex}$ \\
            %
            $\SEQ{\prog_1}{\prog_2}$ & $\wp{\prog_1}{\wp{\prog_2}{\ex}}$ & $\wlp{\prog_1}{\wlp{\prog_2}{\ex}}$ \\
            %
            $\WHILE{\guard}{\progBody}$ & $\lfpIn{\lambda\fpVar}{\iv{\neg\guard} \cdot \ex + \iv{\guard} \cdot \wp{\progBody}{\fpVar}}$ & $\gfpIn{\lambda\fpVar}{\iv{\neg\guard} \cdot \ex + \iv{\guard} \cdot \wlp{\progBody}{\fpVar}}$ \\
            \bottomrule
        \end{tabular}
    \end{adjustbox}
\end{table}

We now unify the weakest pre-expectation calculi for continuous probabilistic programs from \cite{DBLP:conf/setss/SzymczakK19} with the calculi proposed in \cite{DBLP:journals/toplas/OlmedoGJKKM18}. The latter calculi take \emph{renormalization} --- for conditional expected outcomes --- into account.
The central objects these calculi operate on are \emph{expectations}:
%
\begin{definition}[Expectations]
    We distinguish between the following sets of functions:
    %
    \begin{enumerate}
        \item The set of \emph{expectations} is $\exps = \{\ex \mid \ex\colon \pStates \to \exNonNegReals \}$.
        %
        \item The set of \emph{1-bounded expectations} is $\bexps = \{\ex \mid \ex \colon \pStates \to \uIval \}$.
        %
        \item The set of \emph{measurable (1-bounded) expectations} $\expsmeas$ ($\bexpsmeas$) is the subset of $\exps$ ($\bexps$) containing exactly the Borel-measurable functions.
    \end{enumerate}
    %
    We equip all of these sets with the partial order $\eleq$ defined as $\ex \eleq \exb$ iff $\forall \st \in \states \colon \ex(\st) \leq \exb(\st)$.
    \qedDef
\end{definition}
%
Crucially, we have (see \cite[Lemma 2]{DBLP:conf/setss/SzymczakK19} and\iftoggle{arxiv}{ \Cref{proof:measexpsbicpo}}{~\cite{arxiv}}):
%
\begin{restatable}{lemma}{measexpsbicpo}
    \label{thm:measexpsbicpo}
    $\po{\exps}{\eleq}$ and $\po{\bexps}{\eleq}$ are complete lattices.
    $\po{\expsmeas}{\eleq}$ and $\po{\bexpsmeas}{\eleq}$ are $\omega$-bicpos with bottom ($0 \gray{{}= \lam{\pSt}{0}}$) and top ($\infty \gray{{}= \lam{\pSt}{\infty}}$ and $1 \gray{{}= \lam{\pSt}{1}}$, respectively).
\end{restatable}
%
The arithmetic operations $+$ (addition) and $\cdot$ (multiplication) on $\exps$ are defined {pointwise}, i.e., $\forall \st \in \states \colon (\ex + \exb)(\st) = \ex(\st) + \exb(\st)$, and analogously for multiplication.
These operations preserve measurability~\cite[Theorem 11.18]{rudin1953principles}, i.e., $\expsmeas$ is closed under $+$ and $\cdot$.
Moreover, addition (for both arguments) and multiplication by constants%
\footnote{Formally, for every $\exb \in \exps$ these are the functions $\lam{\ex}{\ex + \exb}$ and $\lam{\ex}{\ex \cdot \exb}$.}
are $\omega$-bicontinuous functions.

For state $\pSt \in \pStates$,  program variable $\pVar \in \pVars$, $\xi \in \nnReals$, we define the \emph{updated state as}
%
\[
    \pStUpdate{\pSt}{\pVar}{\xi}
    \eeq
    \lam{\pVarb}{
        \begin{cases}
        \xi & \text{if $\pVarb = \pVar$} \\
        %
        \pSt(\pVarb) &\text{otherwise}
        \end{cases}
    }
    ~.
\]
%
Further, given an expectation $\ex \in \exps$, a variable $\pVar \in \pVars$, and an arithmetic expression $\aExp \colon 
\pStates \to \nnReals$ we define the \emph{substitution of $\pVar$ by $\aExp$ in $\ex$} as the expectation $\exSubsGen = \lam{\st}{\ex(\pStUpdate{\st}{\pVar}{\aExp(\st)})}$.
If we have syntactic expressions for $\ex$ and $\aExp$, then we can obtain a syntactic representation of $\exSubs{f}{\pVar}{\aExp}$ by substituting all free occurrences of $\pVar$ in $\ex$ by $\aExp$ in a capture-avoiding manner, see \Cref{sec:syntax}.

\begin{definition}[Weakest (Liberal) Pre-expectation Transformers~\textnormal{\cite{DBLP:conf/setss/SzymczakK19}}]
    \label{def:wpwlp}
    For all $\prog \in \pWhile$, the \emph{weakest (liberal) pre-expectation transformers} $\wpTrans{\prog} \colon \expsmeas \to \expsmeas$ and $\wlpTrans{\prog}\colon \bexpsmeas \to \bexpsmeas$ are defined inductively on the structure of $\prog$ according to the rules in \Cref{tab:original}.
    \qedDef
\end{definition}
%
Let us briefly explain the inductive definition in \Cref{tab:original}.
The effectless program $\SKIP$ leaves the post-expectation unchanged.
An assignment $\ASSIGN{\pVar}{\aExp}$ substitutes the expression $\aExp$ for the variable $\pVar$ in the post-expectation $\ex$, while uniform assignment $\UNIFASSIGN{\pVar}$ integrates the expectation over all possible values of $\pVar$ in the interval $\uIval$, capturing the averaging effect of drawing $\pVar$ uniformly.
$\OBSERVE{\guard}$ scales the expectation by the Iverson bracket of the guard $\guard$.
Proper renormalization is performed in a second step, see \Cref{sec:wp:cwp}.
For $\ITE{\guard}{\prog_1}{\prog_2}$ and the probabilistic choice $\PCHOICE{\prog_1}{\prob}{\prog_2}$, the weakest (liberal) pre-expectation is a weighted sum of the $\wpSymb$'s from both branches, either weighted by $\iv{\guard}$ and $\iv{\neg\guard}$ in the former case, or by the probabilities $\prob$ and $1-\prob$ in the latter.
The $\wpwlpSymb$ of a sequential composition $\SEQ{\prog_1}{\prog_2}$ is function composition $\wpwlpTrans{\prog_1} \circ \wpwlpTrans{\prog_2}$.
Notably, $\prog_2$ is evaluated \emph{before} $\prog_1$ --- $\wpwlpSymb$'s are thus computed in a backward manner.
Note that $\wpSymb$ and $\wlpSymb$ only differ in the handling of divergence and loops.
The $\DIVERGE$ command, representing non-termination, sets the $\wpSymb$ to $0$ and the $\wlpSymb$ to $1$.
The $\WHILE{\guard}{\prog}$ loop involves computing the \emph{least} fixed point ($\lfp$) for $\wpSymb$ and the \emph{greatest} fixed point ($\gfp$) for $\wlpSymb$.
The difference between $\wpSymb$ and $\wlpSymb$ is further explained in \Cref{sec:wp:wpVsWlp}.

\begin{rem}[On Non-Negativity]
    \label{rem:nonNeg}
    We follow the classic line of work on weakest pre-expectations~\cite{DBLP:conf/stoc/Kozen83,DBLP:series/mcs/McIverM05} and restrict attention to \emph{non-negative expectations} (see~\cite{DBLP:conf/lics/KaminskiK17} for a discussion of the mixed-sign case).
    This means that we can only reason about expected values of non-negative random variables measured in a program's final state.
    As a consequence, in order to reason about the expected final value of a program variable $\pVar$ on termination, we have to assume that $\pVar$ is unsigned.
    We have opted to ensure this by simply requiring \emph{all} variables to be unsigned real numbers.
\end{rem}

Given a loop $\prog = \WHILE{\guard}{\progBody}$ and $\ex \in \expsmeas$, we denote by 
%
\[
	\charfunwp{\prog}{\ex} \colon \expsmeas \to \expsmeas \,, 
	\qquad 
	\charfunwp{\prog}{\ex}(\exb) \eeq \iv{\guard} \cdot \wp{\progBody}{\exb} + \iv{\neg\guard}\cdot \ex
\]
%
the \emph{$\wpSymb$-characteristic function of $\prog$ w.r.t.\ $\ex$} (analogously for $\ex' \in \bexpsmeas$ and $\charfunwlp{\prog}{\ex'}$).
Hence, weakest pre-expectations of loops can be denoted more concisely as 
%
\begin{align*}
	 \wp{\prog}{\ex} \eeq \lfp \charfunwp{\prog}{\ex}
	 \qquad\text{and}\qquad
	  \wlp{\prog}{\ex'} \eeq \gfp \charfunwlp{\prog}{\ex'}~.
\end{align*}
%
Due to the fixed points in \Cref{tab:original} it is not immediately obvious that $\wpTrans{\prog}$ and $\wlpTrans{\prog}$ are well-defined.
This will we be ensured by Kleene's \Cref{thm:kleene} as shown in the next lemma (parts of which have been proved in~\cite{DBLP:conf/setss/SzymczakK19}).
%
\begin{restatable}[{Well-definedness of $\wpSymb$ and $\wlpSymb$}]{theorem}{wpWlpWellDefinedAndContinuous}
    \label{thm:wpWlpWellDefinedAndContinuous}
    For all programs $\prog \in \pWhile$, $\wpTrans{\prog}$ and $\wlpTrans{\prog}$ are well-defined.
    In particular, $\wpTrans{\prog}$ is $\omega$-continuous and $\wlpTrans{\prog}$ is $\omega$-\underline{co}continuous.
\end{restatable}


\subsection{Probabilistic Termination: $\wpSymb$ vs.\ $\wlpSymb$}
\label{sec:wp:wpVsWlp}

In general, for all $\prog \in \pWhile$ and $\ex\in\bexpsmeas$ we have $\wp{\prog}{\ex} \eleq \wlp{\prog}{\ex}$ since the former relies on a least and the latter on a greatest fixed point.
More specifically, we have~\cite[Section 5]{DBLP:conf/setss/SzymczakK19}
%
\begin{align}
    \wp{\prog}{\ex} + \wlp{\prog}{0} \eeq \wlp{\prog}{\ex} \label{eq:wpwlp}
    ~.
\end{align}
%
\iftoggle{arxiv}{%
Let us make the meaning of $\wpSymb$ and $\wlpSymb$ w.r.t.\ the constant post-expectations $0$ and $1$ explicit.
Assuming $\prog$ is started with initial state $\st$ we have the following:
\footnote{These statements can be formalized by means of an operational semantics for $\pWhile$, see~\cite{DBLP:conf/setss/SzymczakK19}.}%
\begin{itemize}
    \item $\wp{\prog}{0}(\st)$ is always 0.
    \item $\wp{\prog}{1}(\st)$ is the probability of \emph{termination} without violating any $\KWOBSERVE$.
    \item $\wlp{\prog}{0}(\st)$ is the probability of \emph{non-termination} without violating any $\KWOBSERVE$.
    \item $\wlp{\prog}{1}(\st)$ is the probability to never violate any $\KWOBSERVE$.
\end{itemize}
}{}
%
We call $\prog$ \emph{almost-surely terminating} (AST) if $\wlp{\prog}{0} = 0$, i.e., if the program does not admit any initial state for which the program's infinite runs that do not violate any $\KWOBSERVE$ have positive probability mass.
It follows from equation \eqref{eq:wpwlp} that $\prog$ is AST iff $\wp{\prog}{\ex} = \wlp{\prog}{\ex}$.


\subsection{Conditional Weakest Pre-Expectations}
\label{sec:wp:cwp}

For a program $\prog \in \pWhile$, an expectation $\ex \in \expsmeas$ and a state $\pSt$, following \cite{DBLP:journals/toplas/OlmedoGJKKM18}, we define:
%
\[
    \cwp{\prog}{\ex}(\pSt)
    \eeq
    \begin{cases}
        \frac{\wp{\prog}{\ex}(\pSt)}{\wlp{\prog}{\onefun}(\pSt)} &\text{ if } \wlp{\prog}{1}(\pSt) \neq 0 \\[0.5em]
        \text{undefined} & \text{ else.}
    \end{cases}
\]
%
The above definition factors out the probability mass of runs violating an $\KWOBSERVE$, i.e., divides by $\wlp{\prog}{\onefun}(\pSt)$, see \Cref{sec:wp:wpVsWlp}.
$\cwp{\prog}{\ex}(\st)$ is thus the expected value of $\fun$ after termination of $\prog$ started with initial state $\st$, \emph{conditioned} on all $\KWOBSERVE$ statements in $\prog$ being successful.
