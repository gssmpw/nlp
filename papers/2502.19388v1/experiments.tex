\section{Implementation and Case Studies}
\label{sec:case_studies}
%
%
We have automated our techniques using a modern deductive verifier for probabilistic programs. We first describe how to integrate our techniques within that verifier in \Cref{sec:case_studies:implementation}. We then describe our case studies, i.e., programs and specifications we have verified, in \Cref{sec:case_studies:case_studies}. Finally, we evaluate our approach empirically on these case studies in \Cref{sec:empirical_eval}.
%
%
\subsection{Implementation}
\label{sec:case_studies:implementation}
%
%
\toolcaesar\footnote{\url{https://www.caesarverifier.org/}} \cite{DBLP:journals/pacmpl/SchroerBKKM23} is a recent expectation-based automated deductive verifier targeting \emph{discrete} and possibly \emph{nondeterministic\footnote{Here we refer to pure nondeterminism which is to be resolved angelically or demonically.}} probabilistic programs. As input, \toolcaesar takes programs written in an intermediate verification language called $\heyVL$ --- a quantitative analogue of \toolboogie \cite{leinoThisBoogie2008}. $\heyVL$ provides a programmatic means to describe verification conditions such as the validity of quantitative loop invariants. These verification conditions are offloaded to an SMT solver, which enables the semi-automated verification of said discrete probabilistic programs. 
%
\toolcaesar does \emph{not} support continuous sampling instructions. We will, however, now demonstrate that with our approach based on Riemann sums, \toolcaesar \emph{can} readily be used to verify bounds on expected outcomes of continuous probabilistic programs in a semi-automated fashion.

Our key insight is that our Riemann expectation transformers can be expressed as expectation transformers denoted by \emph{discrete} probabilistic programs featuring angelic (resp.\ demonic) \emph{nondeterministic choices} for $\nonNegReals$-valued intervals. The latter features \emph{are} supported by \toolcaesar: $\heyVL$ supports the discrete fragment of $\pWhile$ and, amongst others, the two statements

%
\begin{enumerate}
	\item  $\NONDETASSIGN{\pVar}{\aExp_1}{\aExp_2}$, which, on program state $\pState$, assigns a nondeterministically chosen value from the $\nonNegReals$-valued interval $[\aExp_1(\pState), \aExp_2(\pState)]$ to the variable $\pVar$, and
	%
	\item $\DISCRETEUNIFASSIGN{\pVar}{\N}$, which, given a (constant) natural number $\N\geq 1$, samples a value from the  set $\{0,\ldots,\N-1\}$ uniformly at random and assigns the result to the variable $\pVar$.
\end{enumerate}
%
Notice that the latter statement is syntactic sugar for $\pWhile$ as it can be simulated by binary probabilistic choices. The statement $\NONDETASSIGN{\pVar}{\aExp_1}{\aExp_2}$, on the other hand, is not syntactic sugar as nondeterminism is not supported in $\pWhile$. 
%

Now, given a (discrete but possibly nondeterministic) program $\prog$, \toolcaesar employs a transformer $\vcTrans{\prog} \colon \exps \to \exps$, which is defined\footnote{We do not need to require $\ex$ to be measurable since $\prog$ is does not contain continuous sampling.
    Moreover, as is standard for deductive verifiers, loops need to be annotated with quantitative loop invariants; see \Cref{ex:caesar_loop} on page \pageref{ex:caesar_loop}.} as in \Cref{tab:original} and where for the additional statements, we have
%
\begin{align*}
	\vc{\NONDETASSIGN{\pVar}{\aExp_1}{\aExp_2}}{\ex}
	\eeq& \lam{\pState}{\sup_{\xi \in [\aExp_1(\pState), \aExp_2(\pState)]} \ex(\pStUpdate{\st}{\lvar}{\xi})} \\
	%
	%
	\vc{\DISCRETEUNIFASSIGN{\pVar}{\N}}{\ex} \eeq & \frac{1}{\N} \cdot \sum\limits_{i=0}^{\N-1} \exSubs{\ex}{\pVar}{i}~.
\end{align*}
%
Notice that $\vc{\NONDETASSIGN{\pVar}{\aExp_1}{\aExp_2}}{\ex}$ resolves the nondeterministic choice of $\pVar$ \emph{angelically} by returning the \emph{supremum} of all values obtained from evaluating $\ex$ at $\pState$ where $\pVar$ is set to some value from $[\aExp_1(\pState), \aExp_2(\pState)]$. The demonic counterpart is obtained from replacing $\sup$ by $\inf$.
Now let $\pVarj$ be a fresh program variable and observe that for all $\N \geq 1$ and all $\ex \in \exps$, we have
%
\[
	\underbrace{\uwp{\N}{\UNIFASSIGN{\pVar}}{\ex}}_{\text{defined in this paper}}
	\eeq
	\underbrace{\vc{\SEQ{\DISCRETEUNIFASSIGN{\pVarj}{\N}}{\NONDETASSIGN{\pVar}{\tfrac{\pVarj}{\N}}{\tfrac{\pVarj + 1}{\N}}}}{\ex}}_{\text{expressible in $\heyVL$ and thus supported by \toolcaesar}}~.
\]
%
Hence, the upper Riemann pre-expectation of $\UNIFASSIGN{\pVar}$ w.r.t.\ $\ex$ corresponds to the maximal --- under all possible resolutions of the nondeterminism --- expected final value of $\ex$ obtained from (i) sampling one of the $\N$ intervals occurring in the Riemann sums uniformly at random and (ii) assigning to $\pVar$ a nondeterministically chosen value from that sampled interval. \emph{Lower} Riemann pre-expectations can be expressed analogously by resolving the nondeterminism \emph{demonically}.
%
\begin{example}
	\label{ex:caesar_loop}
	%
	Reconsider the Monte Carlo $\pi$-approximator $\prog$ and the superinvariant $\exI$ of $\prog$ w.r.t.\ $\pVarcount$ from \Cref{ex:invariant_monte_carlo}. We can provide \toolcaesar with the following annotated loop:
	%
		\begin{align*}
	%
	&\INVARIANTANNOTATE{\pVarcount + \iv{\pVari \leq \pVarm} \cdot \big(0.85\cdot ((\pVarm \monus \pVari) + 1)\big)} \\
	&\WHILENOBODY{\pVari \leq \pVarm} \\
	%
	&\qquad \SEQ{\DISCRETEUNIFASSIGN{\pVarj_1}{16}}{\NONDETASSIGN{\pVar}{\tfrac{\pVarj_1}{16}}{\tfrac{\pVarj_1 + 1}{16}}}\,; \\
	&\qquad \SEQ{\DISCRETEUNIFASSIGN{\pVarj_2}{16}}{\NONDETASSIGN{\pVarb}{\tfrac{\pVarj_2}{16}}{\tfrac{\pVarj_2 + 1}{16}}}\,; \\
	%
	&\qquad \ITE{\pVar^2 + \pVarb^2 \leq 1}{\ASSIGN{\pVarcount}{\pVarcount + 1}}{\SKIP}\,; \,
	%
	%&\qquad 
	\ASSIGN{\pVari}{\pVari + 1} \quad \}
\end{align*}
	%
	We can then instruct \toolcaesar to check whether $\exI$ is a superinvariant of this loop w.r.t.\ $\pVarcount$. Since the loop body encodes appropriate upper Riemann pre-expectations, \toolcaesar offloads the quantitative entailments corresponding to \Cref{thm:verificationInvariants} to an SMT solver to check them automatically.
	For lower Riemann pre-expectations and subinvariants (for $\wlpSymb$), \toolcaesar can be used analogously.
	%
\end{example}


\subsection{Case Studies}
\label{sec:case_studies:case_studies}

In what follows, we describe the programs and specifications we have verified using \toolcaesar with the SMT solver \toolzt \cite{DBLP:conf/tacas/MouraB08} as back-end. All case studies (including \Cref{ex:caesar_loop}) have been verified within $12$ seconds on an Apple M2. The precise inputs to \toolcaesar are provided in\iftoggle{arxiv}{ \Cref{app:caesar_inputs}}{~\cite{arxiv}}. Both the verified bounds and the required parition sizes $\N$ where determined manually by guessing the respective weakest pre-expectations and increasing $\N$ until \toolcaesar reports success.


\subsubsection{The Irwin-Hall Distribution}
%
%
\begin{figure}[t]
    \small
	\begin{subfigure}[b]{0.4\textwidth}
			\begin{align*}
%	&\WHILENOBODY{\pVari \leq \pVarm} \\
%	%
%	&\qquad \UNIFASSIGN{\pVarb}\,; \\
%	%
%	%
%	&\qquad \ASSIGN{\pVar}{\pVar + \pVarb}\,; \\
%	%
%	&\qquad \ASSIGN{\pVari}{\pVari + 1} \\
%	%
%	&\} % \\
&\WHILENOBODY{\pVari \leq \pVarm} \\
%
&\qquad \UNIFASSIGN{\pVarb}\,; \, %\\
%
%
%&\qquad 
\ASSIGN{\pVar}{\pVar + \pVarb}\,; \\
%
&\qquad \ASSIGN{\pVari}{\pVari + 1}% \\
%
%&
\quad \} % \\
	%
%	&\phantom{a}
\end{align*}
	\end{subfigure}
	%
	\begin{subfigure}[b]{0.58\textwidth}
			\begin{align*}
	&\WHILENOBODY{\pVari \leq \pVarm} \\
	%
	&\qquad \UNIFASSIGN{\pVarb}\,; \,
	%\\
	%
	%&\qquad 
	\OBSERVE{\pVarb \leq \nicefrac 1 2}\,; \,
	%
	%&\qquad 
	\ASSIGN{\pVar}{\pVar + \pVarb}\,; \\
	%
	&\qquad \ASSIGN{\pVari}{\pVari + 1} \quad \} % \\
%	%
%	&\}
\end{align*}
	\end{subfigure}
	%
	\caption{Generating the standard Irwin-Hall distribution (left) and a variant with conditioning (right).}
	\label{fig:irwin_hall}
\end{figure}
%
%
%
The Irwin-Hall distribution (parameterized in $\pVarm$) \cite{johnson1995continuous} is the sum of $\pVarm$ independent and identically distributed random variables, each of which is distributed uniformly over $[0,1]$. The loop $\prog$ depicted in \Cref{fig:irwin_hall} (left) models this family of distributions: On each iteration, $\prog$ samples a value for $\pVarb$ uniformly at random and adds the result to the variable $\pVar$. Hence, if initially $\pVari= 1$, $\pVar= 0$, and $\pVarm \in \nats$, then the final distribution of $\pVar$ indeed corresponds to the Irwin-Hall distribution with parameter $\pVarm$.

We now aim to upper-bound the expected final value of $\pVar$, i.e., the expectation of the Irwin-Hall distribution, for \emph{arbitrary} values of $\pVarm$. Towards this end, we employ our encoding from \Cref{sec:case_studies:implementation} and use \toolcaesar to automatically verify that the expectation 
%
$\exI =  \pVar + \iv{\pVari \leq \pVarm}\cdot \big(1.1 \cdot \tfrac{(\pVarm \monus \pVari) + 1}{2} \big) $
%
is a $\uwpSymb{10}$-superinvariant of $\prog$ w.r.t.\ $\pVar$. Hence, by \Cref{thm:invariant_cont}, we get $\wp{\prog}{\pVar} \eleq \exI$ and thus
%
\[
	\text{for initial $\pState$ with $\pState(\pVari) = 1$, $\pState(\pVar) = 0$, $\pState(\pVarm)\in \nats$ with $\pState(\pVarm) \geq 1$:}~
	\wp{\prog}{\pVar}(\pState) \leq  1.1\cdot \frac{\pState(\pVarm)}{2}~.
\]
%
%
Now consider a variant of the Irwin-Hall distribution where we condition each of the $\pVarm$ random variables to take a value in $[0,\nicefrac{1}{2}]$.
This situation is modeled by the loop $\prog$ depicted in \Cref{fig:irwin_hall} (right), where we employ conditioning inside of the loop. We aim to upper-bound $\cwp{\prog}{\pVar}$, i.e., the \emph{conditional expected final value of $\pVar$}.
For that, we use \toolcaesar to verify that the following expectation is a $\uwpSymb{19}$-superinvariant of $\prog$ w.r.t.\ $\pVarcount$ (resp.\ $\lwlpSymb{2}$-subinvariant of $\prog$ w.r.t.\ $1$):
%
\begin{align*}
	\exI \eeq  \pVar + \iv{\pVari \leq \pVarm}\cdot \big(1.5 \cdot \frac{(\pVarm \monus \pVari) + 1}{8} \big)  
	 %
	 \qquad \text{and}\qquad 
	 \exJ \eeq  \iv{\pVari \leq \pVarm} \cdot 0.5^{ (\pVarm \monus \pVari) +1 } + \iv{\pVari > \pVarm}\cdot 1~.
\end{align*}
%
Hence, we get by \Cref{thm:invariant_cont} that $\cwp{\prog}{\pVar}(\pState)$ is well-defined for all $\pState \in \pStates$ and
%
\[
    \text{for initial $\pState$ with $\pState(\pVari) = 1$, $\pState(\pVar) = 0$, $\pState(\pVarm)\in \nats$ with $\pState(\pVarm) \geq 1$:}~
    \cwp{\prog}{\pVar}(\pState) \leq  \frac{1.5\cdot\pState(\pVarm)}{8 \cdot 0.5^{\pState(\pVarm)}}~.
\]
%
Notice that, even though exponential functions like $0.5^\pVarm$ are not supported by our decidable language of syntactic expectations from \Cref{sec:syntax}, we can use \toolcaesar to reason about it by means of standard user-defined domain declaration for exponentials (see \cite[Section 5.1]{DBLP:journals/pacmpl/SchroerBKKM23} for details). Decidability of the corresponding entailments involving such domains is, however, not guaranteed.


\subsubsection{Reasoning about Diverging Programs}
%
%
\begin{figure}[t]
    \small
    \begin{subfigure}[t]{0.3\textwidth}
        	\begin{align*}
	&\WHILENOBODY{\pVar > 0} \\
	%
	&\qquad \UNIFASSIGN{\pVarb}\,;\, \ASSIGN{\pVarb}{(\pVarbb \monus \pVaraa)\cdot \pVarb + \pVaraa}\,; \\
	%
	&\qquad \ITE{\pVarb \leq (\pVaraa + \pVarbb)/2}{\DIVERGE}{\SKIP}\,; \\
	%
	&\qquad \ASSIGN{\pVar}{\pVar \monus 1} \quad \} %\\
%	%
%	&\}
\end{align*}
    \end{subfigure}
    %
    %
    \begin{subfigure}[t]{0.53\textwidth}
        	\begin{align*}
	&\WHILENOBODY{\pVarh \leq \pVart} \\
	%
	&\qquad \PCHOICE{
		\SEQ{\SEQ{\UNIFASSIGN{\pVar}}{\ASSIGN{\pVar}{10\cdot\pVar }}}{\ASSIGN{\pVarh}{\pVarh + \pVar}}
		}
		{\nicefrac{1}{2}}
		{\SKIP}\,; \\
	%
	%
	&\qquad \ASSIGN{\pVart}{\pVart + 1}\,; \\
	%
	&\qquad \ASSIGN{\pVarcount}{\pVarcount + 1} \quad\} 
\end{align*}
    \end{subfigure}
    %
    \caption{A probably diverging loop (left) and a race between tortoise and hare (right) \cite{DBLP:conf/sas/ChakarovS14}.}
    \label{fig:div_and_race}
\end{figure}
%
%
%
Weakest \emph{liberal} pre-expectations enable us to lower-bound divergence probabilities of programs involving continuous sampling. 
Consider the loop $\prog$ depicted in \Cref{fig:div_and_race} (left).
In each iteration, we sample uniformly from the interval $[\pVaraa, \pVarbb]$ (notice that $\pVaraa$, $\pVarbb$ are uninitialized variables and hence correspond to parameters) and assign the result to $\pVarb$.
Whenever we sample a value in $[0, \tfrac{\pVaraa + \pVarbb}{2}]$, the program diverges.
Using \toolcaesar, we verify that 
%
\[
    \exI \eeq \iv{\pVaraa \leq \pVarbb}\cdot (1 \monus 0.5^{\pVar})
\]
%
is a $\lwlpSymb{2}$-subinvariant of $\prog$ w.r.t.\ $0$. Hence, we have $\exI \eeleq \wlp{\prog}{0}$ by \Cref{thm:invariant_cont}, i.e., whenever $\pVaraa \leq \pVarbb$, then $\prog$ diverges on initial state $\pState$ with probability at least $1 - 0.5^{\pState(\pVar)}$.


\subsubsection{Race between Tortoise and Hare}
%
%
The loop $\prog$ depicted in \Cref{fig:div_and_race} (right) is adapted from \cite{DBLP:conf/sas/ChakarovS14} and models a race between a tortoise ($\pVart$) and a hare ($\pVarh$).
As long as the hare did not overtake the tortoise, the hare flips a fair coin to decide whether to move or not. If the hare decides to move, it samples a distance uniformly at random from $[0,10]$.
The tortoise always moves exactly one step.

We now upper-bound the expected final value of variable $\pVarcount$. Towards this end, we use \toolcaesar to verify that the expectation
%
$\exI = \pVarcount + \iv{\pVarh \leq \pVart} \cdot 3.012\cdot\big((\pVart \monus \pVarh) +2\big)$
%
is a $\uwpSymb{16}$-superinvariant of $\prog$ w.r.t.\ $\pVarcount$. Hence, we get $\wp{\prog}{\pVarcount} \eleq \exI$ by \Cref{thm:invariant_cont}, i.e., if $\prog$ is executed on an initial state $\pState$ with $\pState(\pVarcount) = 0$ and $\pState(\pVarh) \leq \pState(\pVart)$, then the expected final value of $\pVarcount$ is at most $3.012 \cdot (\pState(\pVart) - \pState(\pVarh) + 2)$.


\subsection{Experimental Evaluation}
\label{sec:empirical_eval}
%
%
\begin{table}
	\caption{Experimental results. Time in seconds. TO = \SI{180}{s}, MO = \SI{8}{GB}.}
	\label{tab:experiments}
	%
    \renewcommand{\arraystretch}{1.0}
	\begin{tabular}{l | c r@{\quad}@{\quad} r@{\quad}@{\quad} c r}%
		\toprule
		\colprog & \coln & \colt & \colformsize & \colpost & \colbound \\
		\midrule
		\begin{table*}
\caption{Time to obtain distributions through empirical methods and estimation via our framework. Means and standard deviations over 100 trials reported. Faster times bolded. Settings described as \texttt{[graph type] [graph size] [sampling method] [sample size]}. \texttt{bin} is binomial, \texttt{pow} is power, \texttt{rand} is random, \texttt{snow} is snowball.}
\label{tab:timings}
\begin{tabular}{lcc|lcc}
\toprule
Description & Empirical & Framework & Description & Empirical & Framework \\
\midrule
\texttt{bin 20000 rand 200} & $0.0443 \pm 0.0034$ & $\mathbf{0.0080 \pm 0.0018}$ & \texttt{pow\_b 20000 rand 200} & $0.0476 \pm 0.0036$ & $\mathbf{0.0071 \pm 0.0014}$ \\
\texttt{bin 20000 rand 400} & $0.0551 \pm 0.0038$ & $\mathbf{0.0320 \pm 0.0070}$ & \texttt{pow\_b 20000 rand 400} & $0.0501 \pm 0.0037$ & $\mathbf{0.0254 \pm 0.0032}$ \\
\texttt{bin 20000 rand 1000} & $\mathbf{0.0612 \pm 0.0113}$ & $0.1622 \pm 0.0083$ & \texttt{pow\_b 20000 rand 1000} & $\mathbf{0.0526 \pm 0.0041}$ & $0.1438 \pm 0.0026$ \\
\texttt{bin 20000 snow 200} & $0.0531 \pm 0.0033$ & $\mathbf{0.0112 \pm 0.0028}$ & \texttt{pow\_b 20000 snow 200} & $0.0490 \pm 0.0032$ & $\mathbf{0.0082 \pm 0.0020}$ \\
\texttt{bin 20000 snow 400} & $0.0587 \pm 0.0112$ & $\mathbf{0.0278 \pm 0.0019}$ & \texttt{pow\_b 20000 snow 400} & $0.0496 \pm 0.0037$ & $\mathbf{0.0246 \pm 0.0008}$ \\
\texttt{bin 20000 snow 1000} & $\mathbf{0.0530 \pm 0.0033}$ & $0.1900 \pm 0.0172$ & \texttt{pow\_b 20000 snow 1000} & $\mathbf{0.0501 \pm 0.0037}$ & $0.1471 \pm 0.0093$ \\
\texttt{bin 40000 rand 200} & $0.0954 \pm 0.0064$ & $\mathbf{0.0088 \pm 0.0015}$ & \texttt{pow\_b 40000 rand 200} & $0.1062 \pm 0.0060$ & $\mathbf{0.0067 \pm 0.0002}$ \\
\texttt{bin 40000 rand 400} & $0.1168 \pm 0.0061$ & $\mathbf{0.0275 \pm 0.0009}$ & \texttt{pow\_b 40000 rand 400} & $0.1133 \pm 0.0063$ & $\mathbf{0.0253 \pm 0.0022}$ \\
\texttt{bin 40000 rand 1000} & $\mathbf{0.1183 \pm 0.0065}$ & $0.1581 \pm 0.0065$ & \texttt{pow\_b 40000 rand 1000} & $\mathbf{0.1142 \pm 0.0058}$ & $0.1453 \pm 0.0052$ \\
\texttt{bin 40000 snow 200} & $0.1170 \pm 0.0052$ & $\mathbf{0.0076 \pm 0.0005}$ & \texttt{pow\_b 40000 snow 200} & $0.1118 \pm 0.0060$ & $\mathbf{0.0077 \pm 0.0003}$ \\
\texttt{bin 40000 snow 400} & $0.1183 \pm 0.0076$ & $\mathbf{0.0278 \pm 0.0027}$ & \texttt{pow\_b 40000 snow 400} & $0.1107 \pm 0.0054$ & $\mathbf{0.0249 \pm 0.0025}$ \\
\texttt{bin 40000 snow 1000} & $\mathbf{0.1170 \pm 0.0060}$ & $0.1570 \pm 0.0049$ & \texttt{pow\_b 40000 snow 1000} & $\mathbf{0.1118 \pm 0.0057}$ & $0.1442 \pm 0.0040$ \\
\texttt{bin 100000 rand 200} & $0.2680 \pm 0.0102$ & $\mathbf{0.0082 \pm 0.0020}$ & \texttt{pow\_b 100000 rand 200} & $0.3023 \pm 0.0108$ & $\mathbf{0.0068 \pm 0.0004}$ \\
\texttt{bin 100000 rand 400} & $0.3269 \pm 0.0120$ & $\mathbf{0.0310 \pm 0.0007}$ & \texttt{pow\_b 100000 rand 400} & $0.3152 \pm 0.0179$ & $\mathbf{0.0263 \pm 0.0015}$ \\
\texttt{bin 100000 rand 1000} & $0.3263 \pm 0.0097$ & $\mathbf{0.1707 \pm 0.0046}$ & \texttt{pow\_b 100000 rand 1000} & $0.2984 \pm 0.0122$ & $\mathbf{0.1432 \pm 0.0039}$ \\
\texttt{bin 100000 snow 200} & $0.3254 \pm 0.0119$ & $\mathbf{0.0081 \pm 0.0012}$ & \texttt{pow\_b 100000 snow 200} & $0.3028 \pm 0.0091$ & $\mathbf{0.0067 \pm 0.0004}$ \\
\texttt{bin 100000 snow 400} & $0.3227 \pm 0.0103$ & $\mathbf{0.0273 \pm 0.0004}$ & \texttt{pow\_b 100000 snow 400} & $0.3238 \pm 0.0429$ & $\mathbf{0.0261 \pm 0.0019}$ \\
\texttt{bin 100000 snow 1000} & $0.3257 \pm 0.0110$ & $\mathbf{0.1554 \pm 0.0050}$ & \texttt{pow\_b 100000 snow 1000} & $0.2960 \pm 0.0148$ & $\mathbf{0.1437 \pm 0.0043}$ \\
\midrule
\texttt{pow\_a 20000 rand 200} & $0.0523 \pm 0.0042$ & $\mathbf{0.0109 \pm 0.0005}$ & \texttt{sbm 20000 rand 200} & $0.0518 \pm 0.0046$ & $\mathbf{0.0137 \pm 0.0022}$ \\
\texttt{pow\_a 20000 rand 400} & $0.0559 \pm 0.0028$ & $\mathbf{0.0433 \pm 0.0033}$ & \texttt{sbm 20000 rand 400} & $0.0617 \pm 0.0042$ & $\mathbf{0.0395 \pm 0.0014}$ \\
\texttt{pow\_a 20000 rand 1000} & $\mathbf{0.0571 \pm 0.0037}$ & $0.2647 \pm 0.0078$ & \texttt{sbm 20000 rand 1000} & $\mathbf{0.0641 \pm 0.0051}$ & $0.2217 \pm 0.0093$ \\
\texttt{pow\_a 20000 snow 200} & $0.0559 \pm 0.0128$ & $\mathbf{0.0107 \pm 0.0003}$ & \texttt{sbm 20000 snow 200} & $0.0603 \pm 0.0028$ & $\mathbf{0.0139 \pm 0.0021}$ \\
\texttt{pow\_a 20000 snow 400} & $0.0555 \pm 0.0120$ & $\mathbf{0.0423 \pm 0.0031}$ & \texttt{sbm 20000 snow 400} & $0.0605 \pm 0.0038$ & $\mathbf{0.0468 \pm 0.0005}$ \\
\texttt{pow\_a 20000 snow 1000} & $\mathbf{0.0556 \pm 0.0121}$ & $0.2643 \pm 0.0082$ & \texttt{sbm 20000 snow 1000} & $\mathbf{0.0613 \pm 0.0051}$ & $0.2670 \pm 0.0093$ \\
\texttt{pow\_a 40000 rand 200} & $0.1144 \pm 0.0050$ & $\mathbf{0.0115 \pm 0.0015}$ & \texttt{sbm 40000 rand 200} & $0.1101 \pm 0.0048$ & $\mathbf{0.0114 \pm 0.0008}$ \\
\texttt{pow\_a 40000 rand 400} & $0.1227 \pm 0.0054$ & $\mathbf{0.0435 \pm 0.0034}$ & \texttt{sbm 40000 rand 400} & $0.1346 \pm 0.0073$ & $\mathbf{0.0409 \pm 0.0057}$ \\
\texttt{pow\_a 40000 rand 1000} & $\mathbf{0.1230 \pm 0.0058}$ & $0.2620 \pm 0.0063$ & \texttt{sbm 40000 rand 1000} & $\mathbf{0.1371 \pm 0.0053}$ & $0.2255 \pm 0.0137$ \\
\texttt{pow\_a 40000 snow 200} & $0.1216 \pm 0.0059$ & $\mathbf{0.0107 \pm 0.0004}$ & \texttt{sbm 40000 snow 200} & $0.1348 \pm 0.0062$ & $\mathbf{0.0135 \pm 0.0009}$ \\
\texttt{pow\_a 40000 snow 400} & $0.1206 \pm 0.0056$ & $\mathbf{0.0426 \pm 0.0035}$ & \texttt{sbm 40000 snow 400} & $0.1345 \pm 0.0042$ & $\mathbf{0.0469 \pm 0.0014}$ \\
\texttt{pow\_a 40000 snow 1000} & $\mathbf{0.1214 \pm 0.0063}$ & $0.2620 \pm 0.0069$ & \texttt{sbm 40000 snow 1000} & $\mathbf{0.1342 \pm 0.0056}$ & $0.2671 \pm 0.0103$ \\
\texttt{pow\_a 100000 rand 200} & $0.3166 \pm 0.0110$ & $\mathbf{0.0116 \pm 0.0019}$ & \texttt{sbm 100000 rand 200} & $0.2936 \pm 0.0114$ & $\mathbf{0.0110 \pm 0.0005}$ \\
\texttt{pow\_a 100000 rand 400} & $0.3364 \pm 0.0135$ & $\mathbf{0.0436 \pm 0.0007}$ & \texttt{sbm 100000 rand 400} & $0.4294 \pm 0.0503$ & $\mathbf{0.0399 \pm 0.0035}$ \\
\texttt{pow\_a 100000 rand 1000} & $0.3387 \pm 0.0109$ & $\mathbf{0.2605 \pm 0.0056}$ & \texttt{sbm 100000 rand 1000} & $0.4128 \pm 0.0111$ & $\mathbf{0.2195 \pm 0.0102}$ \\
\texttt{pow\_a 100000 snow 200} & $0.3347 \pm 0.0118$ & $\mathbf{0.0112 \pm 0.0005}$ & \texttt{sbm 100000 snow 200} & $0.4174 \pm 0.0205$ & $\mathbf{0.0133 \pm 0.0004}$ \\
\texttt{pow\_a 100000 snow 400} & $0.3353 \pm 0.0099$ & $\mathbf{0.0423 \pm 0.0025}$ & \texttt{sbm 100000 snow 400} & $0.4168 \pm 0.0139$ & $\mathbf{0.0472 \pm 0.0030}$ \\
\texttt{pow\_a 100000 snow 1000} & $0.3331 \pm 0.0105$ & $\mathbf{0.2599 \pm 0.0062}$ & \texttt{sbm 100000 snow 1000} & $0.4179 \pm 0.0143$ & $\mathbf{0.2643 \pm 0.0064}$ \\
\bottomrule
\end{tabular}
\end{table*}\\
        \bottomrule
	\end{tabular}
\end{table}
%
%
In this section, we empirically evaluate the scalability of our approach based on the programs from the previous sections, i.e., our case studies and the Monte-Carlo approximator from \Cref{fig:intro}.
For each of these programs and post-expectations, we verify the tightest upper bound (up to a precision of $3$ decimal places) on the respective weakest pre-expectation for increasing values of $N$ (except for \progdiverging~where $N=2$ already yields the precise weakest pre-expectation).

Our empirical results are depicted in \Cref{tab:experiments}.
Column \colprog~depicts the respective program, \coln~denotes the partition size for the Riemann pre-expectation, \colt~denotes the verification time in seconds (i.e., verification condition generation and time for SMT solving) required by \toolcaesar, \colformsize~denotes the size of the formula offloaded to the SMT solver \toolzt (i.e., the number of nodes in the formula's abstract syntax tree), \colpost~denotes the post-expectation, and \colbound~depicts the upper (lower) bound on the (liberal) weakest pre-expectation. For the \progdiverging~benchmark, we lower-bound a liberal weakest pre-expectation.
All other benchmarks upper-bound a non-liberal weakest pre-expectation.
The bounds were inferred manually in a binary search-like fashion.

With our encoding of Riemann pre-expectations described in \Cref{sec:case_studies:implementation}, \toolcaesar allows verifying increasingly sharper bounds when increasing the partition size $N$
The size of the generated formula for the verification conditions mostly increases moderately when increasing $N$.
This is to be expected since the formula size grows polynomially in $N$ for a fixed program and a fixed post-expectation (see \Cref{sec:verify_loop_free}).
The time required for verification, however, can increase drastically when doubling the value of $N$ (see entries of \Cref{tab:experiments} where $N=32$).
