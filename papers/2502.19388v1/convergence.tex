\section{Convergence of The Riemann Weakest Pre-Expectations}
\label{sec:convergence}

\newcommand{\discontinuities}{D}

We now address the question if and under which conditions $\lwp{\N}{\prog}{\ex}$ and $\uwp{\N}{\prog}{\ex}$ converge to $\wp{\prog}{\ex}$ as $\N$ goes to $\infty$.
We focus on \emph{pointwise} convergence.
It is easy to see that in general, pointwise convergence does not hold: consider for example the simple program $\prog = \UNIFASSIGN{\pVar}$ and the expectation $\ex = \iv{\pVar \in \rats}$, for which we have $\wp{\prog}{\ex} = \zerofun$.
However, for every $\N \geq 1$ it holds that $\uwp{\N}{\prog}{\ex} = \onefun$.
Our goal is thus to identify a \emph{Riemann-suitable} subset of our framework, by mildly restricting the class of allowed expectations.

We first recall two fundamental properties of the Riemann integral.
The first is a well-known characterization of Riemann integrability, the second reveals that Lebesgue integrals conservatively generalize Riemann integrals.
In the following we will use $\lebmes$ to denote the Lebesgue measure.
For the formal definition and properties of $\lebmes$ we refer the reader to\iftoggle{arxiv}{ \Cref{app:background}}{~\cite{arxiv}}.

\begin{theorem}[Riemann-Lebesgue Theorem\textnormal{~\cite[Thm.~11.33]{rudin1953principles}}]
    \label{thm:riemannIntegrableIffAECont}
    Let $\fun \colon \clIvalGen \to \reals$ be bounded.
    Consider the set $\discontinuities = \{x \in \clIvalGen \mid \fun \text{ is discontinuous in } x\}$.
    Then $\fun$ is Riemann-integrable if and only if $\lebmes(\discontinuities) = 0$, i.e., $\fun$ is \emph{continuous almost everywhere}.
\end{theorem}

\begin{theorem}[\textnormal{~\cite[Theorem 1.7.1]{ash2000probability}}]
    \label{thm:riemannEqualsLebesgue}
    Let $\fun \colon \clIvalGen \to \reals$ be bounded and Riemann-integrable.
    Then
    \iftoggle{arxiv}{%
    %
    \[
        \int_{\ivalL}^{\ivalR} \fun(x) \,dx \eeq \int_{\clIvalGen} \fun(x) \,d\lebmes(x)~,
    \]
    %
    }{%
    $\int_{\ivalL}^{\ivalR} \fun(x) \,dx = \int_{\clIvalGen} \fun(x) \,d\lebmes(x)$,
    }%
    i.e., the Riemann integral coincides with the Lebesgue integral.
\end{theorem}


\subsection{Convergence of Approximation for Loop-free Programs}
\label{sec:approxwp:convloopfree}

By \Cref{thm:riemannIntegrableIffAECont,thm:riemannEqualsLebesgue}, the Riemann integral coincides with the Lebesgue integral for \enquote{sufficiently continuous} functions.
There is thus hope to establish convergence if we restrict our setting to almost everywhere continuous expectations.
However, there is an additional complication: A function needs to be \emph{bounded} on the integration domain in order to be Riemann-integrable -- otherwise the upper sum could be \enquote{stuck} at $\infty$.
We address this issue using the concept of \emph{local boundedness}:
%
A function $\fun \colon \pStates \to \reals \cup \{ \pm \infty \}$ is called \emph{locally bounded} if
\[
    \text{for all compact } \compactSet \subseteq \pStates \colon \quad \sup_{\pSt \in \compactSet} |\fun(\pSt)| ~<~ \infty
    ~.
\]
We recall that a subset $\compactSet$ of $\reals^\pVars$ is compact iff it is closed and bounded.
Many common functions such as polynomials are locally bounded.
An example of a function that is \emph{not} locally bounded is $\iv{x > 0} \cdot \tfrac 1 x$.
%
Based on these considerations, we define the following:

\begin{definition}[Riemann-suitable]
	\label{def:riemannSuitable}
	A combination $(\aExps, \guards, \expsClass)$ of a class $\aExps$ of locally bounded arithmetic expressions, a class $\guards$ of guards, and a class $\expsClass \subseteq \exps$ of locally bounded expectations if
	for all $\ex \in \expsClass$ and $\pVar \in \pVars$, $\ex$ is Riemann-integrable on $\uIval$ w.r.t.\ $\pVar$.
	Moreover, $\expsClass$ contains the constant expectations $0, 1$ and satisfies the following closure properties:
	\begin{enumerate}[(i)]
		\item $\expsClass$ is closed under taking infima and suprema over closed subintervals of $\uIval$, i.e., for all $\ex \in \expsClass$, $\pVar \in \pVars$, and $\clIvalGen \subseteq \uIval$ we have
		$\inf_{\xi \in \clIvalGen}\exSubs{\ex}{\pVar}{\xi} \in \expsClass$ and $\sup_{\xi \in \clIvalGen}\exSubs{\ex}{\pVar}{\xi} \in \expsClass$.
		\item
        For all $\ex \in \expsClass$, $\aExp \in \aExps$ and $\pVar \in \pVars$, we have $\exSubs{\ex}{\pVar}{\aExp} \in \expsClass$.
		\item
        For all $\ex, \exb \in \expsClass$ and $p \in \uIval \cap \rats$, we have $p \cdot \ex + (1-p) \cdot \exb \in \expsClass$.
		\item
		For all $\ex, \exb \in \expsClass$ and $\guard \in \guards$, we have $\iv{\guard} \cdot \ex + \iv{\neg\guard} \cdot \exb \in \expsClass$.
        \qedDef
	\end{enumerate}
\end{definition}
%
The conditions (i) - (iv) in the above definition ensure that $\expsClass$ is closed under $\lwpTrans{\N}{\prog}$ and $\uwpTrans{\N}{\prog}$ for \emph{loop-free} $\prog$ (see \eqref{eq:approxInClass} in \Cref{thm:convLoopFree} below).
%
We remark that it is not immediately clear if any \enquote{interesting} Riemann-suitable instantiations exist.
Fortunately, the answer is \emph{yes}, as we show in \Cref{sec:syntax}.
%
In the Riemann-suitable setting, we can show the following convergence result:

\begin{restatable}[Convergence -- Loop-free Case]{lemma}{convLoopFree}
	\label{thm:convLoopFree}
	Let $(\aExps, \guards, \expsClass)$ be Riemann-suitable.
	Then for all loop-free $\prog \in \pWhileWith{\aExps}{\guards}$ and post-expectations $\ex \in \expsClass$ the following holds:
	\begin{align}
        \forall \N \geq 1 \colon\quad \lwp{\N}{\prog}{\ex} \in \expsClass \qand \uwp{\N}{\prog}{\ex} \in \expsClass
		\label{eq:approxInClass} ~.
    \end{align}
    \begin{align}
        \text{If }\ex \in \expsmeas \colon
        \quad 
        \wp{\prog}{\ex}
		\eeq
		\sup_{\N \geq 1} \lwp{\N}{\prog}{\ex}
		\eeq
		\inf_{\N \geq 1} \uwp{\N}{\prog}{\ex}
		~.
		\label{eq:convLoopFree}
	\end{align}
\end{restatable}
%
\noindent An analogous $\wlpSymb$-version of \Cref{thm:convLoopFree} is stated in\iftoggle{arxiv}{ \Cref{app:wlpVersion}}{~\cite{arxiv}}.


\subsection{Convergence of Approximation for Programs with Loops}
\label{sec:approxwp:convloops}

Recall the \emph{Dirichlet program} $D$ from \Cref{sec:birdseye}.
For all $\N$ we have $\uwp{\N}{D}{\onefun} = \onefun$, showing that \Cref{thm:convLoopFree} does not hold for loopy programs in general. 
The added difficulty stems from the fact that the semantics of loops is itself a limit.
Specifically, we can imagine to \enquote{unroll} every loop in a program $\prog$ up to a certain depth $\depth \in \nats$, denoted $\unfold{\prog}{\depth}$ (see\iftoggle{arxiv}{ \Cref{app:unrolling}}{~\cite{arxiv}} for the formal definition), and view the semantics $\prog$ as the limit of the semantics of $\unfold{\prog}{\depth}$ when $\depth$ tends to infinity.
Notably, $\wpSymb$ and $\wlpSymb$ behave differently with respect to this limit: when increasing $\depth$ the (non-liberal) pre-expectation of $\ex$ increases, while the liberal pre-expectation decreases.
This is formalized in the next lemma:

\begin{restatable}{lemma}{convUnfolding}
	\label{thm:convUnfolding}
	For all $\prog \in \pWhile$ and post-expectations $\ex \in \expsmeas$ and $\exb \in \bexpsmeas$, $\wp{\unfold{\prog}{\depth}}{\ex}$ and $\wlp{\unfold{\prog}{\depth}}{\exb}$ are non-decreasing (non-increasing, respectively) sequences in $\depth \in \nats$ and it holds that
	\[
	\sup_{\depth \in \nats} \wp{\unfold{\prog}{\depth}}{\ex}
	\eeq
	\wp{\prog}{\ex}
	\qqand
	\inf_{\depth \in \nats} \wlp{\unfold{\prog}{\depth}}{\ex}
	\eeq
	\wlp{\prog}{\ex}
	~.
	\]
\end{restatable}

It should now be clear that, when considering the convergence of $\uwp{\N}{\prog}{\ex}$ for a program $\prog$ with loops, we are implicitly considering two limits, the one over $\N$ and the one over $\depth$. Unfortunately, when the two limits have different monotonic behaviors (i.e., for $\uwpTrans{\N}{\prog}$, where the limit in $\depth$ is the supremum but the limit in $\N$ is the infimum, and symmetrically for $\lwlpTrans{\N}{\prog}$) convergence cannot be guaranteed. Intuitively, this explains why the next result only holds for $\lwpSymb{\N}$ and $\uwlpSymb{\N}$. 

\begin{restatable}[Convergence -- General Case]{theorem}{pointwiseConv}
	\label{thm:pointwiseConv}
	Let $(\aExps, \guards, \expsClass)$ be Riemann-suitable.
	Then for all $\prog \in \pWhileWith{\aExps}{\guards}$ and post-expectations $\ex \in \expsClass \cap \expsmeas$ and $\exb \in \expsClass \cap \bexpsmeas$ it holds that:
	\begin{align*}
		\sup_{n \geq 1} \lwp{n}{\unfold{\prog}{n}}{\ex}
		\eeq
		\wp{\prog}{\ex}
		\qand
		\wlp{\prog}{\exb}
		\eeq
		\inf_{n \geq 1} \uwlp{n}{\unfold{\prog}{n}}{\exb}
		~.
	\end{align*}
\end{restatable}

Note that \Cref{thm:pointwiseConv} ensures that when considering a Riemann suitable class, the L.H.S.\ of the inequality in \Cref{thm:cwpSound} converges to the exact conditional weakest pre-expectation $\cwpSymb$.
