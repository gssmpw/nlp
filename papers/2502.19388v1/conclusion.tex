\section{Conclusion}
\label{sec:conclusion}

We tackled the semi-automatic verification of user-provided quantitative invariants for probabilistic programs involving continuous distributions, conditioning, and potentially unbounded loops.
Our approach is to approximate the Lebesgue integrals appearing in a program's weakest pre-expectation semantics by lower and upper Riemann sums.
This simple idea allows us to prove sound bounds on expected outcomes by verifying invariants automatically with SMT-solvers.
We demonstrated that this method can be readily integrated in the \toolcaesar verification infrastructure~\cite{DBLP:journals/pacmpl/SchroerBKKM23}.

There are many opportunities for future work.
A natural direction is to integrate our approach to invariant verification with techniques for \emph{fully automatic invariant generation} such as~\cite{DBLP:conf/tacas/BatzCJKKM23}.
%
While we can already verify invariants for non-trivial benchmarks, there is some room for improvement on the engineering side.
This includes more sophisticated, multi-dimensional, and adaptive partitions of the integration domain, and a combination with direct integration methods, where applicable.
We also plan to extend the front end of \toolcaesar with dedicated support for continuous distributions.

Due to its simplicity, we are confident that our approach is amenable to generalizations such as non-determinism and strategy synthesis as in~\cite{DBLP:journals/pacmpl/BatzBKW24}, and programs with pointers and data structures.
