%%
%% This is file `sample-sigconf.tex',
%% generated with the docstrip utility.
%%
%% The original source files were:
%%
%% samples.dtx  (with options: `sigconf')
%% 
%% IMPORTANT NOTICE:
%% 
%% For the copyright see the source file.
%% 
%% Any modified versions of this file must be renamed
%% with new filenames distinct from sample-sigconf.tex.
%% 
%% For distribution of the original source see the terms
%% for copying and modification in the file samples.dtx.
%% 
%% This generated file may be distributed as long as the
%% original source files, as listed above, are part of the
%% same distribution. (The sources need not necessarily be
%% in the same archive or directory.)
%%
%%
%% Commands for TeXCount
%TC:macro \cite [option:text,text]
%TC:macro \citep [option:text,text]
%TC:macro \citet [option:text,text]
%TC:envir table 0 1
%TC:envir table* 0 1
%TC:envir tabular [ignore] word
%TC:envir displaymath 0 word
%TC:envir math 0 word
%TC:envir comment 0 0
%%
%%
%% The first command in your LaTeX source must be the \documentclass
%% command.
%%
%% For submission and review of your manuscript please change the
%% command to \documentclass[manuscript, screen, review]{acmart}.
%%
%% When submitting camera ready or to TAPS, please change the command
%% to \documentclass[sigconf]{acmart} or whichever template is required
%% for your publication.
%%
%%
\documentclass[acmsmall,screen]{acmart} % review option adds line numbers, screen adds colored links
\pdfoutput = 1 % [TW] Added
%%
%% [TW] toggle for switching between oopsla25 and arxiv version
%% arxiv = true means full version, arxiv = false means published conference version
\usepackage{etoolbox}
\newtoggle{arxiv}
\settoggle{arxiv}{true}

%% \BibTeX command to typeset BibTeX logo in the docs
\AtBeginDocument{%
  \providecommand\BibTeX{{%
    Bib\TeX}}}

%% Rights management information.  This information is sent to you
%% when you complete the rights form.  These commands have SAMPLE
%% values in them; it is your responsibility as an author to replace
%% the commands and values with those provided to you when you
%% complete the rights form.
% [TW] copied from publishing system
\iftoggle{arxiv}{}{
\setcopyright{acmlicensed}
\acmJournal{PACMPL}
\acmYear{2025} \acmVolume{9} \acmNumber{OOPSLA1} \acmArticle{95} \acmMonth{4}\acmDOI{10.1145/3720429}
}

\iftoggle{arxiv}{}{
%% These commands are for a PROCEEDINGS abstract or paper.
\acmConference[Conference acronym 'XX]{Make sure to enter the correct
  conference title from your rights confirmation emai}{June 03--05,
  2018}{Woodstock, NY}
%%
%%  Uncomment \acmBooktitle if the title of the proceedings is different
%%  from ``Proceedings of ...''!
%%
%%\acmBooktitle{Woodstock '18: ACM Symposium on Neural Gaze Detection,
%%  June 03--05, 2018, Woodstock, NY}
\acmISBN{978-1-4503-XXXX-X/18/06}
}


%%
%% Submission ID.
%% Use this when submitting an article to a sponsored event. You'll
%% receive a unique submission ID from the organizers
%% of the event, and this ID should be used as the parameter to this command.
%%\acmSubmissionID{123-A56-BU3}

%%
%% For managing citations, it is recommended to use bibliography
%% files in BibTeX format.
%%
%% You can then either use BibTeX with the ACM-Reference-Format style,
%% or BibLaTeX with the acmnumeric or acmauthoryear sytles, that include
%% support for advanced citation of software artefact from the
%% biblatex-software package, also separately available on CTAN.
%%
%% Look at the sample-*-biblatex.tex files for templates showcasing
%% the biblatex styles.
%%

%%
%% The majority of ACM publications use numbered citations and
%% references.  The command \citestyle{authoryear} switches to the
%% "author year" style.
%%
%% If you are preparing content for an event
%% sponsored by ACM SIGGRAPH, you must use the "author year" style of
%% citations and references.
%% Uncommenting
%% the next command will enable that style.
%\citestyle{acmauthoryear}

% from https://www.sigplan.org/Resources/Author/:
% PACMPL journal issues: \documentclass[acmsmall,screen]{acmart}. Authors can choose either author-year citations, with \citestyle{acmauthoryear}, or numeric citations, with \citestyle{acmnumeric}.
\citestyle{acmnumeric}

%% [TW] Our custom packages and macros
%%%%%%%%%%%%%%%%%%%%%%%%%%%%%%%%%%%%%%%%%%%%%%%%%%%%%%%%%%%%%%%%%%%%%%%%%%%%%%

%% Beautiful mathematics
\usepackage{amsmath, amssymb, amsfonts} 
\usepackage{nicefrac}
\usepackage{mathtools}
\usepackage{bm, bbm}
\usepackage[scr=boondoxo,scrscaled=1.05]{mathalfa}

%% References in the correct format 
%\usepackage[square,numbers]{natbib}
%\def\bibfont{\footnotesize} % fix to have the same font size as without natbib

\usepackage[sort, compress, space]{cite}            


%% Enumerate nicely 
\usepackage{enumitem}

%% Different color comments and commenting large parts of the text
\usepackage{xcolor}
\usepackage{comment}
\usepackage{soul}

%% Hyper references
\usepackage{hyperref}
\usepackage{cleveref}
%\usepackage[numbers]{natbib}

\usepackage{tikz}
%\usepackage{thm-restate}
%% Appendix package
%\usepackage{appendix}

%% Random text to test spacing 
\usepackage{blindtext}

\usepackage{afterpage}

\usepackage{algorithm, algorithmic}    



\usepackage{dsfont}

\usepackage{tikz}
\usepackage{graphicx}
\usepackage{tikzscale}
\usepackage{pgfplots}
\pgfplotsset{compat=newest}
\usepackage{xfrac}

\usepackage{thm-restate}

%\usepackage{subcaption}

\usepackage{balance}

\usepackage{cite}
\usepackage{amsmath,amssymb,amsfonts}
\usepackage{balance}
\usepackage{algorithmic}
\usepackage{graphicx}
\usepackage{textcomp}
\usepackage{xcolor}
\usepackage{amsmath}
\usepackage{amssymb}
\usepackage[mathscr]{euscript}
\usepackage{comment}
\usepackage{xcolor}
\usepackage{enumitem} 
\usepackage{amsthm}


\newcommand{\thought}[1]{{\color[rgb]{0.2,0.39,0.66}(#1)}}
\newcommand{\todo}[1]{{\color[rgb]{1.0,0.0,0.0}(#1)}}
\newcommand{\hsh}[1]{{\color{green!50!black} Henrik: #1}}
\newcommand{\st}[1]{{\color{red!50!black} Sebastian: #1}}

\newcommand{\ulm}[1]{_{\scaleto{\mathrm{#1}}{3pt}}}
\newcommand\at[2]{\left.#1\right|_{#2}}











\newtheorem{assumption}{Assumption}

\DeclareMathOperator*{\argmax}{arg\,max}
\DeclareMathOperator*{\argmin}{arg\,min}

\newcommand{\swname}[1]{\texttt{#1}}
\newcommand{\ie}{i\/.\/e\/.,\/~}
\newcommand{\eg}{e\/.\/g\/.,\/~}
\newcommand{\cf}{cf\/.\/~}

\newcommand{\fig}{Fig\/.\/~}
\newcommand{\defn}{Def\/.\/~}
\newcommand{\sect}{Sec\/.\/~}
\newcommand{\tabl}{Tab\/.\/~}
\newcommand{\algo}{Algorithm~}
\newcommand{\theo}{Theorem~}

\newcommand{\bnnl}{3 hidden layers}
\newcommand{\bnnn}{50 neurons}
\newcommand{\bnna}{tanh activations}

\newcommand{\capt}[1]{\mdseries{\emph{#1}}}

\newcommand{\videolink}{at \url{https://youtu.be/_d7AqTRjz6g}}
\newcommand{\codelink}{\url{https://github.com/wheelbot/mini-wheelbot}}

\newcommand{\fakepar}[1]{\vspace{0mm}\noindent\textbf{#1.}}

\newcommand{\needref}{\textcolor{red}{[REF]}}

\newcommand{\plotfontsize}{9pt}


%% [TW] to remove copyright (safe space)
\iftoggle{arxiv}{
    \setcopyright{none}
    \settopmatter{printacmref=false} % Removes citation information below abstract
    \renewcommand\footnotetextcopyrightpermission[1]{} % removes footnote with conference information in first column
    \pagestyle{plain}
}{}


%% [TW] custum numbered remark enviroment
\theoremstyle{remark}
\newtheorem{rem}{Remark}

%% [TW] Allow aligned equations to break over pages?
%\allowdisplaybreaks

%%% [TW] inserted the following from publishing system
%%% The following is specific to OOPSLA1 '25 and the paper
%%% 'Foundations for Deductive Verification of Continuous Probabilistic Programs: From Lebesgue to Riemann and Back'
%%% by Kevin Batz, Joost-Pieter Katoen, Francesca Randone, and Tobias Winkler.
%%%
\iftoggle{arxiv}{}{
\setcopyright{acmlicensed}
\acmDOI{10.1145/3720429}
\acmYear{2025}
\acmJournal{PACMPL}
\acmVolume{9}
\acmNumber{OOPSLA1}
\acmArticle{95}
\acmMonth{4}
\received{2024-10-16}
\received[accepted]{2025-02-18}
}

%%
%% end of the preamble, start of the body of the document source.
\begin{document}

% [TW] copied titled and author info from publishing system, but reintroduced \subtitle manually
%\title{Foundations for Deductive Verification of Continuous Probabilistic Programs: From Lebesgue to Riemann and Back}
\title{Foundations for Deductive Verification of Continuous Probabilistic Programs}
\subtitle{From Lebesgue to Riemann and Back}

\author{Kevin Batz}
\orcid{0000-0001-8705-2564}
\affiliation{%
    \institution{RWTH Aachen University}
    \city{Aachen}
    \country{Germany}
}
\affiliation{%
    \institution{University College London}
    \city{London}
    \country{United Kingdom}
}
\email{kevin.batz@cs.rwth-aachen.de}

\author{Joost-Pieter Katoen}
\orcid{0000-0002-6143-1926}
\affiliation{%
    \institution{RWTH Aachen University}
    \city{Aachen}
    \country{Germany}
}
\email{katoen@cs.rwth-aachen.de}

\author{Francesca Randone}
\orcid{0009-0002-3489-9600}
\affiliation{%
    \institution{University of Trieste}
    \city{Trieste}
    \country{Italy}
}
\email{francesca.randone@units.it}

\author{Tobias Winkler}
\orcid{0000-0003-1084-6408}
\affiliation{%
    \institution{RWTH Aachen University}
    \city{Aachen}
    \country{Germany}
}
\email{tobias.winkler@cs.rwth-aachen.de}

\iftoggle{arxiv}{\titlenote{This document is the full version of a paper with the same title published at OOPSLA 2025.}}{}


%%
%% By default, the full list of authors will be used in the page
%% headers. Often, this list is too long, and will overlap
%% other information printed in the page headers. This command allows
%% the author to define a more concise list
%% of authors' names for this purpose.
% [TW] "Use the complete list of authors as long as it fits on one line in the running heads "
%\renewcommand{\shortauthors}{K. Batz, J.-P. Katoen, F. Randone, and T. Winkler}

%%
%% The abstract is a short summary of the work to be presented in the
%% article.
\begin{abstract}
    \begin{abstract}
Retrieval-Augmented Generation (RAG) is often used with Large Language Models (LLMs) to infuse domain knowledge or user-specific information. In RAG, given a user query, a retriever extracts chunks of relevant text from a knowledge base. These chunks are sent to an LLM as part of the input prompt. Typically, any given chunk is repeatedly retrieved across user questions. However, currently, for every question, attention-layers in LLMs fully compute the key values (KVs) repeatedly for the input chunks, as state-of-the-art methods cannot reuse KV-caches when chunks appear at arbitrary locations with arbitrary contexts. Naive reuse leads to output quality degradation.  This leads to potentially redundant computations on expensive GPUs and increases latency. In this work, we propose \sys, a system for managing and reusing precomputed KVs corresponding to the text chunks (we call \textit{chunk-caches}) in RAG-based systems. We present how to identify \hl{\textit{chunk-caches} that are reusable}, how to efficiently perform a small fraction of recomputation to \textit{fix} the cache to maintain output quality, and how to efficiently store and evict \textit{chunk-caches} in the hardware for maximizing reuse while masking any overheads. With real production workloads as well as synthetic datasets, we show that \sys reduces redundant computation by \textbf{51\%} over SOTA prefix-caching and \textbf{75\%} over full recomputation.
\hl{Additionally, with continuous batching on a real production workload, we get a \textbf{1.6$\times$} speedup in throughput and a \textbf{2$\times$} reduction in end-to-end response latency over prefix-caching while maintaining quality, for both the \llama-3-8B and \llama-3-70B models. 
}
\end{abstract}





\end{abstract}

%%
%% The code below is generated by the tool at http://dl.acm.org/ccs.cfm.
%% Please copy and paste the code instead of the example below.
%%
\begin{CCSXML}
    <ccs2012>
    <concept>
    <concept_id>10003752.10003753.10003757</concept_id>
    <concept_desc>Theory of computation~Probabilistic computation</concept_desc>
    <concept_significance>500</concept_significance>
    </concept>
    <concept>
    <concept_id>10003752.10003790.10002990</concept_id>
    <concept_desc>Theory of computation~Logic and verification</concept_desc>
    <concept_significance>500</concept_significance>
    </concept>
    <concept>
    <concept_id>10003752.10010124.10010138.10010139</concept_id>
    <concept_desc>Theory of computation~Invariants</concept_desc>
    <concept_significance>500</concept_significance>
    </concept>
    <concept>
    <concept_id>10003752.10010124.10010138.10010141</concept_id>
    <concept_desc>Theory of computation~Pre- and post-conditions</concept_desc>
    <concept_significance>500</concept_significance>
    </concept>
    <concept>
    <concept_id>10003752.10010124.10010138.10010142</concept_id>
    <concept_desc>Theory of computation~Program verification</concept_desc>
    <concept_significance>500</concept_significance>
    </concept>
    <concept>
    <concept_id>10002950.10003648.10003662</concept_id>
    <concept_desc>Mathematics of computing~Probabilistic inference problems</concept_desc>
    <concept_significance>300</concept_significance>
    </concept>
    </ccs2012>
\end{CCSXML}

\ccsdesc[500]{Theory of computation~Probabilistic computation}
\ccsdesc[500]{Theory of computation~Logic and verification}
\ccsdesc[500]{Theory of computation~Invariants}
\ccsdesc[500]{Theory of computation~Pre- and post-conditions}
\ccsdesc[500]{Theory of computation~Program verification}
\ccsdesc[300]{Mathematics of computing~Probabilistic inference problems}
%%
%% Keywords. The author(s) should pick words that accurately describe
%% the work being presented. Separate the keywords with commas.
\keywords{probabilistic programs, deductive program verification, continuous distributions, quantitative loop invariants, weakest preexpectations, approximate integration, SMT solving}
%% A "teaser" image appears between the author and affiliation
%% information and the body of the document, and typically spans the
%% page.

%\received{20 February 2007}
%\received[revised]{12 March 2009}
%\received[accepted]{5 June 2009}

%%
%% This command processes the author and affiliation and title
%% information and builds the first part of the formatted document.
\maketitle

%% [TW] TOC for overview while writing
%\tableofcontents

%% [TW] Our sections
%\input{notes_relwork.tex}
\section{Introduction}
\label{sec:intro}

\begin{figure*}[tb]
    \centering
    \includegraphics[width=0.848\linewidth]{figs/circuitnn.pdf} 
    \caption{Illustration of differentiable CircuitNN. CircuitNN is designed based on differentiable NAND gates. After DAS is guided by PI and PO pairs of the truth table, CircuitNN can get the precise circuit architecture logic equivalent to the truth table.}
    \label{fig:circuitnn}
\end{figure*}

% 1. Describe the importance of logic synthesis
% 2. Existing Problems
% (a) Neural Architecture Search: Unstable, Predefined Setting, etc.
% (b) Circuit Generation: Probabilistic Model, Logic Equivalence

With the rapid advancement of technology, the scale of integrated circuits (ICs) has expanded exponentially. 
This expansion has introduced significant challenges in chip manufacturing, particularly concerning power and area metrics.
A primary objective in IC design is achieving the same circuit function with fewer transistors, thereby reducing power usage and area occupancy.

Logic synthesis~\cite{hachtel2005logicsynth}, a critical step in electronic design automation (EDA), transforms behavioral-level circuit designs into optimized gate-level circuits, ultimately yielding the final IC layout. 
The primary goal of logic synthesis is to identify the physical implementation with the fewest gates for a given circuit function. 
This task constitutes a challenging NP-hard combinatorial optimization problem. 
Current logic synthesis tools~\cite{brayton2010abc, wolf2013yosys} rely on human-designed heuristics, often leading to sub-optimal outcomes.

Differentiable architecture search (DAS) techniques~\cite{liu2018darts, chu2020darts} offer novel perspectives on addressing challenges in this problem.
Circuit functions can be represented through truth tables, which map binary inputs to their corresponding outputs. 
Truth tables provide a precise representation of input-output relationships, ensuring the design of functionally equivalent circuits.
Inspired by this, researchers~\cite{deepmind2024ai4sys, wang2024tnet} have begun exploring the application of DAS to synthesize circuits directly from truth tables.
Specifically, \citet{deepmind2024ai4sys} proposed CircuitNN, a framework that learns differentiable connection structures with logic gates, enabling the automatic generation of logic circuits from truth tables.
This approach significantly reduces the complexity of traditional circuit generation. 
Building on this, \citet{wang2024tnet} introduced T-Net, a triangle-shaped variant of CircuitNN, incorporating regularization techniques to enhance the efficiency of DAS.

Despite these advancements, several challenges remain. 
The computational complexity of DAS grows quadratically with the number of gates, posing scalability issues.
Although triangle-shaped architecture~\cite{wang2024tnet} partially mitigates this problem, redundancy persists. 
%Additionally, DAS is susceptible to converging to local optima, limiting the ability to search architectures that satisfy the given truth tables~\cite{liu2018darts}. 
%Furthermore, hyperparameters (network depth and layer width) require extensive searches, introducing complexity and prolonging the synthesis process. 
Additionally, DAS is susceptible to converging to local optima~\cite{liu2018darts} and hyperparameters (network depth and layer width) require extensive searches. 
The challenges arise from the vast search space in DAS. 
% Even with predefined settings for CircuitNN, finding a configuration that meets the truth table requires extensive trial and error during the DAS process. 
Intuitively, limiting the search space through predefined parameters (network depth, gates per layer, and connection probabilities) can significantly reduce the complexity.

Recent advances~\cite{openai2023gpt4, abramson2024alphafold3, esser2024sd3, li2024mar} in conditional generative models have demonstrated remarkable performance across language, vision, and graph generation tasks. 
Motivated by these developments, we propose a novel approach to circuit generation that generates preliminary circuit structures to guide DAS in generating refined circuits matching specified truth tables. 
Firstly, we introduce CircuitVQ, a tokenizer with a discrete codebook for circuit tokenization. 
Built upon our Circuit AutoEncoder framework~\cite{hou2022graphmae,li2023maskgae,wu2025mgvga}, CircuitVQ is trained through a circuit reconstruction task. 
Specifically, the CircuitVQ encoder encodes input circuits into discrete tokens using a learnable codebook, while the decoder reconstructs the circuit adjacency matrix based on these tokens.
Subsequently, the CircuitVQ encoder serves as a circuit tokenizer for CircuitAR pretraining, which employs a masked autoregressive modeling paradigm~\cite{chang2022maskgit, li2023mage}. 
In this process, the discrete codes function as supervision signals. 
After training, CircuitAR can generate discrete tokens progressively, which can be decoded into initial circuit structures by the decoder of the CircuitVQ. 
These prior insights can guide DAS in producing refined circuits that match the target truth tables precisely.

Our key contributions can be summarized as follows:
\begin{itemize}
\item We introduce CircuitVQ, a circuit tokenizer that facilitates graph autoregressive modeling for circuit generation, based on our Circuit AutoEncoder framework;
\item Develop CircuitAR, a model trained using masked autoregressive modeling, which generates initial circuit structures conditioned on given truth tables;
\item Propose a refinement framework that integrates differentiable architecture search to produce functionally equivalent circuits guided by target truth tables;
\item Comprehensive experiments demonstrating the scalability and capability emergence of our CircuitAR and the superior performance of the proposed circuit generation approach.
\end{itemize}

% Motivation
% (a) Diffusion (Vision, Graph), Autoregressive (Language, Vision)
% (b) Circuit Generation for Predefined Setting
% (c) Neural Architecture Search for Strict Logic Equivalence

% Contribution
% (a) Circuit Tokenizer (new transformer arch, training strategy)
% (b) CircuitAR (train and gen strategies, post-ar strategy)
% (c) Extensive Evaluation including BitD (Bit Distance) for Scalability

\section{A Bird's Eye View}
\label{sec:birdseye}

In this section, we explain what makes our method unique by providing a compressed overview of its characteristic features.
We also discuss its limitations and point out a subtle technical challenge.


\subsection{Highlights and Comparison to Other Approaches}

Since we are not the first to address the problem of proving bounds on quantities expressible as $\wpwlpSymb$'s, we now highlight five distinctive features of our approach --- their combination is unique in the literature.
A more in-depth review of related work is deferred to \Cref{sec:relwork}.

Of the following five items, the first two are specific to the way we handle integrals via Riemann sums, whereas the latter three tend to apply to $\wpSymb$-based approaches in general.


\paragraph{An Alternative to Moment-Based Analyses}

A major thread of existing work, e.g.,~\cite{DBLP:conf/sas/ChakarovS14,DBLP:journals/pacmpl/AgrawalC018,DBLP:journals/pacmpl/MoosbruggerSBK22}, circumvents explicit integration altogether.
They achieve this by focusing on classes of programs and properties where, simply speaking, the analysis can soundly replace the distributions in the program by their \emph{mean}~\cite[Appendix A, p.~31]{DBLP:journals/corr/abs-1709-04037} or some higher moment~\cite{DBLP:journals/pacmpl/MoosbruggerSBK22}.
For example, $\wp{\UNIFASSIGN{\pVar}}{\pVar \pVarb}$ is equal to\footnote{In general, replacing $\UNIFASSIGN{\pVar}$ by $\ASSIGN{\pVar}{\tfrac 1 2}$ is sound whenever the post-expectation is linear in $\pVar$. While our approach is compatible with such optimizations, we shall not discuss them any further.} $\wp{\ASSIGN{\pVar}{\tfrac 1 2}}{\pVar \pVarb} = \tfrac 1 2 y$, where $\tfrac 1 2$ is the mean of $\UNIFOVER{0}{1}$.
However, such mean-based analyses do not always suffice.
Indeed, reconsidering the example from \Cref{sec:intro}, the inequality $\wp{\progPiInner}{\pVarcount} \eleq \pVarcount + 0.85$, proved correct with our approach, becomes false if we substitute $\UNIFASSIGN{\pVar}$ and $\UNIFASSIGN{\pVarb}$ in $\progPiInner$ by $\ASSIGN{\pVar}{\tfrac 1 2}$ and $\ASSIGN{\pVarb}{\tfrac 1 2}$, respectively.
In fact, the so-obtained program $\progPiInner'$ satisfies $\wp{\progPiInner'}{\pVarcount} = \pVarcount + 1$ as the point $(\tfrac 1 2, \tfrac 1 2)$ \emph{is certainly inside} the quarter unit circle.


\paragraph{General Conditional Branching}

We make relatively mild assumptions regarding the shape of $\KWIF$- and $\KWWHILE$-guards in the programs --- in practice, general Boolean combinations of the polynomial (in)equalities over all program variables are supported, as in \Cref{fig:intro}.
This is  different from, e.g.,~\cite{DBLP:journals/pacmpl/MoosbruggerSBK22} which restricts to finitely-valued variables in guards or~\cite{DBLP:conf/sas/ChakarovS14,DBLP:conf/popl/ChatterjeeFNH16} which restrict to linear guards.
We achieve this high level of generality thanks to the simplicity of the Riemann approximation which treats integration over the resulting indicator functions in a uniform manner.


\paragraph{Moving Backwards: Verification and Parametric Models}

We conduct a \emph{backward analysis}: 
Given a random variable $\ex$ over final program states (called \emph{post-expectation} in this paper), we approximate the (conditional) weakest pre-expectation $\wp{\prog}{\ex}$, which is \emph{a function of the initial state}.
This is dual to works like~\cite{DBLP:conf/cav/GehrMV16,DBLP:conf/pldi/BeutnerOZ22,wang2024static} that conduct a \emph{forward analysis} to solve a classic Bayesian inference problem, namely to compute a program's posterior density --- \emph{a function of the final state} --- for a given prior distribution.
Let us contrast these two paradigms in greater detail.

Backward analysis, as done in this paper, allows verifying program properties that hold for \emph{all initial states}.
For instance, our framework supports questions such as, \enquote{\emph{Is the posterior mean of $x$ at most twice the initial value of $y$?}}, expressed as $\cwpSymb[\prog](x) \le 2y$.
The backward approach is thus well-suited for tasks such as \emph{verification} --- ensuring a program behaves as intended \emph{for all inputs} --- and \emph{parametric analysis} --- examining how expected behavior changes when program parameters are altered (parameters can be modeled as uninitialized program variables).

Forward analyses, on the other hand, are typically motivated by probabilistic programming languages (PPLs) explicitly designed for automating Bayesian inference, see e.g., \cite{DBLP:journals/corr/abs-1809-10756,dippl,DBLP:journals/jmlr/BinghamCJOPKSSH19,carpenter2017stan}. 
We remark that there are some fundamental discrepancies between these PPLs and the one studied in this paper.
For example, the former usually offer extensive support for general continuous distributions, but do not always support general loops.
Moreover, Bayesian inference problems often require \emph{soft conditioning}, which our approach only encodes as syntactic sugar (see Remark 2).
Finally, some applications of Bayesian inference involve loading and processing datasets with thousands of observations.
In principle, such data could be hardcoded into our programs, but the result would likely be too large for current SMT-based analysis techniques (as indicated by our experiments in \Cref{sec:empirical_eval}).

In summary, our backward approach is particularly suited for proving properties of complex stochastic processes described as probabilistic programs, especially those involving unbounded loops
 However, it is less effective for data-intensive statistical analysis.


\paragraph{A Zoo of Quantities}

The majority of existing methods for (semi-)automated exact analysis of PP with loops and continuous distributions focus on one of these tasks:
(i) bound the posterior distribution~\cite{DBLP:conf/cav/GehrMV16,DBLP:conf/pldi/BeutnerOZ22,wang2024static},
(ii) prove various flavors of (non-)termination, e.g.,~\cite{DBLP:conf/cav/ChakarovS13,DBLP:conf/cav/ChatterjeeFG16,DBLP:journals/pacmpl/AgrawalC018}, and
(iii) bound assertion violation probabilities~\cite{DBLP:conf/cav/ChakarovS13,DBLP:conf/pldi/WangS0CG21}.
An exception is~\cite{DBLP:conf/pldi/Wang0GCQS19} which tackles \emph{cost analysis}.
%
In contrast, in the $\wpSymb$-based framework, we can express and reason about a remarkably large variety of quantities, including the following (assume that $\prog$ does not contain conditioning for simplicity):
%
\begin{itemize}
    \item Termination probabilities: $\wp{\prog}{1}$
    \item The probability of terminating in a predicate\footnote{Recall that $\iv{\guard}$ is the indicator function of the predicate $\guard$.} $\guard$: $\wp{\prog}{\iv{\guard}}$
    \item The expected value of variable $\pVar$ on termination: $\wp{\prog}{\pVar}$
    \item Higher moments of $\pVar$ after termination: $\wp{\prog}{\pVar^k}$ for $k \geq 1$
    \item Expected distance between variables $\pVar$ and $\pVarb$ after termination: $\wp{\prog}{|\pVar-\pVarb|}$
\end{itemize}
%
\iftoggle{arxiv}{%
Note that this includes reasoning about \emph{unbounded} random variables.
For example, a program variable $\pVar$ may become arbitrarily large during program execution.
}{}

\begin{example}
    Let us illustrate the above quantities.
    Consider the following program $\prog$:
    %
    \begin{align*}
        \ITE{\pVar = 1}{ \PCHOICE{\ASSIGN{\pVarb}{0}}{\nicefrac{1}{2}}{\ASSIGN{\pVarb}{2}}}{\PCHOICE{\ASSIGN{\pVarb}{0}}{\nicefrac{4}{5}}{\ASSIGN{\pVarb}{3}}}
    \end{align*}
    %
    \begin{itemize}
    	\item $\wp{\prog}{1}=1$, i.e., $\prog$ terminates with probability $1$ for each initial state.
    	%
    	\item $\wp{\prog}{\iv{\pVarb = 0}} \eeq \nicefrac{1}{2} \cdot \iv{\pVar = 1} + \nicefrac{4}{5} \cdot \iv{\pVar \neq 1}$, i.e., if initially $\pVar=1$, then the probability to terminate in $\pVarb = 0$ is $\nicefrac{1}{2}$. For all other initial states, this probability is $\nicefrac{4}{5}$.
    	%
    	\item  $\wp{C}{\pVarb} = 1\cdot \iv{\pVar = 1} + \nicefrac{3}{5} \cdot \iv{\pVar \neq 1}$, i.e., the expected final value of $\pVarb$ is either $1$ or $\nicefrac{3}{5}$, depending on whether $\pVar = 1$ holds initially or not.
    	%
    	\item $ \wp{C}{\pVarb^2}  = 2 \cdot \iv{\pVar = 1} + \nicefrac{9}{5} \cdot \iv{\pVar \neq 1}$, i.e., the second moment of $\pVarb$ on termination is either $2$ or $\nicefrac{9}{5}$, depending on whether $\pVar = 1$ holds initially or not.
    	%
    	\item $\wp{\prog}{| \pVar - \pVarb |} = \iv{\pVar = 1} \cdot (\nicefrac{1}{2} \cdot |\pVar| + \nicefrac{1}{2} \cdot | \pVar - 2| ) + \iv{\pVar \neq 1} \cdot (\nicefrac{4}{5} \cdot |\pVar| + \nicefrac{1}{5} \cdot | \pVar - 3|)$, which is the expected difference between the final values of $\pVar$ and $\pVarb$ in terms of their initial values. 
    \end{itemize}
    %
    The above weakest pre-expectations can be determined by applying the rules in \Cref{tab:original} (\cpageref{tab:original}).
\end{example}


\paragraph{Conditioning, Everywhere}

Following~\cite{DBLP:journals/toplas/OlmedoGJKKM18}, we allow conditioning in the form of (hard) $\KWOBSERVE$-statements at arbitrary places in the program --- even inside loops.
All of the above quantities can be generalized sensibly to the conditional setting.
We can go further:
For instance, the probability to never violate an $\KWOBSERVE$, even if the program does not terminate, is given by $\wlp{\prog}{1}$.


\subsection{Limitations and Assumptions}

As mentioned, to analyze loops, we assume \emph{user-provided invariants}.
Moreover, we restrict to native support for continuous \emph{uniform} distributions and discrete coin flips.
Further, while we support two-sided bounds for loop-free programs, the kind of invariants we study in this paper can only prove \emph{upper} bounds on $\wpSymb$ and $\cwpSymb$, and \emph{lower} bounds on $\wlpSymb$.
We conjecture that our approach extends to invariant-based methods for the other directions~\cite{DBLP:journals/pacmpl/HarkKGK20} as well, but leave the details for future work.
To keep the presentation simple, we only discuss invariant verification for \emph{non-nested loops}.
The case of general loop structures, including sequential and/or nested, can be dealt with as in~\cite{DBLP:journals/pacmpl/BatzBKW24}.
We assume non-negative real-valued program variables and do not support data structures.
Soft-conditioning is only supported indirectly.
See \Cref{rem:generalDist,rem:nonNeg,rem:softConditioning} for more details.


\subsection{From Lebesgue Integrals to Riemann Sums and Back: A Technical Challenge}
\label{sec:challenges}

\newcommand{\progdirichlet}{D}

The Lebesgue integral is \emph{the} standard integral in probability theory.
There is good reason for this:
Lebesgue integrals can be defined for a broader class of functions than, say, Riemann integrals, and have favorable mathematical properties like Monotone Convergence Theorems.

Therefore, to define the semantics of PPs, most works (e.g., \cite{DBLP:conf/setss/SzymczakK19,dahlqvist2020semantics}) have resorted to Lebesgue integrals, and we do not question that this is the way to go.
To illustrate the usefulness of the Lebesgue integral as opposed to the Riemann integral\footnote{The Riemann integral is the limit of both the lower and upper Riemann sums as $\N \to \infty$, provided the two limits coincide.}, consider the following program $\progdirichlet$: 
%
\begin{align*}
    & \SEQ{\ASSIGN{\pVar}{0}}{\ASSIGN{\pVarb}{1}} \, \SEMICOLON \\
	& \WHILE{\pVarb\cdot\pVarc \pNEQ \pVar} 
	{\ASSIGN{\pVarb}{\pVarb+1} \,\SEMICOLON\, \ASSIGN{\pVar}{0} \,\SEMICOLON \,
	 \WHILE{\pVarb \cdot\pVarc \pNEQ \pVar \pAND \pVar < \pVarb}{\ASSIGN{\pVar}{\pVar+1}} \, \} }
\end{align*}
%
This program \enquote{searches} for non-negative integers $\pVar$ and $\pVarb$ such that $\pVarc = \tfrac \pVar \pVarb$ and stops once it finds such $\pVar$ and $\pVarb$.
It thus terminates if and only if $\pVarc$ is a \emph{rational number} in $\uIval$ initially (recall that we allow actual \emph{real}-valued variables).
In symbols, we have $\wp{\progdirichlet}{1} = \iv{\pVarc \in \rats \land 0 \leq \pVarc \leq 1}$ --- this is the well-known \emph{Dirichlet function} restricted to the unit interval.
Now, consider the program $\SEQ{\UNIFASSIGN{\pVarc}}{\progdirichlet}$.
Its termination probability is zero, the Lebesgue integral of the Dirichlet function over the interval $\uIval$.
Intuitively, this is because almost all real numbers are irrational, i.e., the rational numbers have Lebesgue measure zero.
However, the Dirichlet function is \emph{not Riemann-integrable} --- it is \enquote{too discontinuous}.
As a consequence, one cannot easily define $\wpSymb$ --- neither in general nor in this specific example --- relying on Riemann integrals.

In light of the above example, the following question arises naturally:
%
\begin{myhighlight}
    \emph{How sensible is an analysis of loops based on Riemann sums given the fact that $\wpSymb$'s of loops are \underline{not Riemann-inte}g\underline{rable} in general, not even for programs using only polynomial arithmetic and constant post-expectations?}
\end{myhighlight}
%
This question has at least two dimensions.

(1) Regarding \emph{soundness}, we prove that our Riemann $\wpSymb$'s are \emph{always} sound under- or over-approximations, i.e., the following inequalities hold for all $\N \geq 1$ in a very general setting:
\[  
\forall~\text{initial states $\st$}\colon\quad
    \lwp{\N}{\prog}{\ex}(\st)
    \lleq
    \wp{\prog}{\ex}(\st)
    \lleq
    \uwp{\N}{\prog}{\ex}(\st)        
\]
and similarly for $\wlpSymb$.
This works because the \emph{lower Riemann integral} --- the limit of the lower Riemann sums as the discretization becomes fines --- is a lower bound on the Lebesgue integral, and similarly for the upper Riemann integral.
This is always true, even if the lower and upper integral are different, i.e., if the function at hand is not Riemann-integrable. 

(2) Will the Riemann approximation always converge (in some sense) to the exact $\wpSymb$ as the discretization becomes finer?
We prove that, under mild assumptions such as polynomial arithmetic in the program, the answer is \emph{yes} for loop-free programs.
As one of our theoretical main results for loopy programs, we show that (\Cref{thm:pointwiseConv})
\[
    \sup_{n \geq 1} \lwp{n}{\unfold{\prog}{n}}{\ex}
    \eeq
    \wp{\prog}{\ex}
    \qquad
    \text{but}
    \qquad
    \inf_{n \geq 1} \uwp{n}{\unfold{\prog}{n}}{\ex}
    ~\overset{\text{in general}}{\neq}~
    \wp{\prog}{\ex}
    ~,
    \tag{$\ddagger$}\label{eq:wow}
\]
where $\unfold{\prog}{n}$ arises from $\prog$ by unfolding all loops up to depth $n$ by taking a simultaneous limit of the unfolding depth and the fineness of the lower Riemann sum approximation (the $n$ in $\lwpSymb{n}$).
Via equation \eqref{eq:wow} we \emph{recover} the exact $\wpSymb$ --- defined in terms of \emph{Lebesgue} integrals --- as a limit of our approximate $\wpSymb$'s based on lower Riemann sums for a general class of probabilistic loops.

\section{Notation and Preliminaries}\label{sec:prelims}
This section fixes the notation and relevant notions for fair division of goods; the notation specific to division of chores is relegated to Section \ref{sec:chores}. 
 
\paragraph{Fair Division Instances.} A {fair division instance} is given by a tuple $\langle [n], [m], \{v_i\}_{i=1}^n \rangle$, where $[n]=\{1,2,.\dots,n\}$ is the set of $n\in\mathbb{Z}_+$ agents, $[m]=\{1,2, \dots, m\}$ the set of $m\in \mathbb{Z}_+$ indivisible goods, and for each agent $i\in[n]$, the set function $v_i: 2^{[m]} \to \mathbb{R}_+$ denotes the valuation of agent $i$ over subsets of goods. Specifically, $v_i(S) \in \mathbb{R}_+$ denotes the value that agent $i$ derives from the subset $S \subseteq [m]$ of goods. For subsets $S \subseteq [m]$ and $g \in [m]$, we will write $S + g$ to denote the union $S \cup \{ g\}$. 

A valuation $v_i$ is said to be monotone if the inclusion of goods into any subset does not decrease its value, under $v_i$, i.e., $v_i(S)\leq v_i(T)$ for every pair of subsets $S \subseteq T \subseteq[m]$. We will assume throughout that the agents' valuations are monotone and normalized: $v_i(\emptyset)=0$ for all agents $i$. 

We will additionally consider instances with identically ordered valuations. Here, we have an indexing of the $m$ goods, $\{g_1, \ldots g_m\}$, such that for each pair of goods $g_s, g_t$, with index $s < t$, and all agents $i \in [n]$, the inequality $v_i(S + g_s) \geq  v_i(S + g_t)$ holds for each subset $S \subset [m]$ that does not contain $g_s$ and $g_t$; see Example \ref{ex:sqrt-ordered} in Section \ref{subsec:additive-ordered}. 

This work also establishes improved bounds for the specific case of additive valuations. Recall that a valuation $v_i$ is said to be additive if, for every subset $S\subseteq[m]$ of goods, $v_i(S)=\sum_{g\in S} v_i(\{g\})$. We will use the shorthand $v_i(g)$---instead of $v_i(\{g\}) \in \mathbb{R}_+$---to denote agent $i$'s value for any good $g \in [m]$.  


\paragraph{Allocations and Multi-Allocations.} An allocation $\calB=(B_1,B_2,\ldots, B_n)$ of the goods among the $n$ agents is a partition of $[m]$ into $n$ pairwise disjoint subsets $B_1,\ldots, B_n \subseteq [m]$. Here, the subset of goods $B_i$ is assigned to agent $i \in [n]$ and is referred to as $i$'s bundle. In addition, write $\Pi_n([m])$ to denote the collection of all $n$-partitions of $[m]$. Note that for any allocation $\calB =(B_1,\ldots, B_n)$ we have, by definition, $\cup_{i=1}^n B_i = [m]$ and $B_i \cap B_j = \emptyset$, for all $i \neq j$, and hence $\calB \in \Pi_n([m])$.

 
A \textit{multi-allocation} is a tuple $\calA=(A_1,A_2\dots,A_n)$ of $n$ subsets, wherein subset $A_i \subseteq [m]$ denotes the bundle assigned to agent $i$. In contrast to allocations, in a multi-allocation, we do not require that the assigned bundles $A_i$ are pairwise disjoint and that they partition $[m]$.\footnote{Note that $A_i$s are still subsets of goods and not multisets.} Hence, in a multi-allocation, a single good may be present in multiple bundles or even in none. 

Though, when in a multi-allocation $\calA$, each good $g$ is assigned to exactly one agent, we refer to $\calA$ as an {\it exact allocation}; this is to emphasize that the bundles of such a multi-allocation do partition $[m]$. 

We associate with each bundle $A_i \subseteq [m]$ an $m$-dimensional characteristic vector $\rmchar(A_i) \in \{0,1\}^m$. For each good $g\in [m]$, the $g$th component of the characteristic vector---denoted as $\rmchar(A_i)_g$---is equal to one if $g \in A_i$, otherwise the $g$th component is zero. That is, 
\begin{align*}
\rmchar(A_i)_g \coloneqq \begin{cases}
    1 & \text{if } g\in A_i \\
    0 & \text{otherwise}.
\end{cases}
\end{align*}

For any multi-allocation $\calA=(A_1, \ldots, A_n)$, we will use $\chi^\calA \in \mathbb{Z}^m_+$ to denote the vector sum of the characteristic vectors of its bundles, $\chi^\calA \coloneqq \sum_{i=1}^n\rmchar(A_i)$. We will refer to $\chi^\calA$ as the \textit{characteristic vector} of the multi-allocation $\calA$. When there is no ambiguity, we will omit the notational dependence in the superscript and simply write $\chi$ for $\chi^\calA$.

Note that for any good $g\in [m]$ and multi-allocation $\calA$, the $g^{th}$ component of the characteristic vector $\chi^\calA_g$ is equal to the number of bundles in $\calA$ that contain $g$. Conceptually, we think of this setting as one in which $\chi^A_g$ identical copies of the good $g$ are assigned among different agents. 

Write $\ellone{\chi^\calA}$ and $\ellinfty{\chi^\calA}$ to denote the $\ell_1$ and $\ell_\infty$ norm, respectively, of the characteristic vector. Hence,  $\ellone{\chi^\calA} = \sum_{g=1}^m \chi^\calA_g$ and $\ellinfty{\chi^\calA} = \max_{g\in[m]} \chi^\calA_g$. It is relevant to note that $\ellone{\chi^\calA}$ captures the total number of goods, with copies, assigned among the agents,  $\ellone{\chi^\calA} = \sum_{i=1}^n |A_i|$. Further, $\ellinfty{\chi^\calA}$ captures the maximum number of copies of any one good $g$ assigned under $\calA$.

In particular, if $\calA$ is an {\it exact} allocation, then $\chi^\calA$ is equal to the all-ones vector and we have $\ellone{\chi^\calA} =m$ and $\ellinfty{\chi^\calA} =1$.
 
\noindent
The shared-based fairness criterion we study in this work is defined using maximin shares; these shares are defined next.
\begin{definition}[Maximin Share (MMS)]\label{def:mms}
    Given any fair division instance $\langle [n], [m], \{v_i\}_{i=1}^n \rangle$ with goods, the {maximin share}, $\mu_i \in \mathbb{R}_+$, of each agent $i \in [n]$ is defined as 
    \begin{align*}
    \mu_i \coloneqq  \max_{(X_1,\dots, X_n) \in \Pi_n([m])} \ \ \min_{j\in[n]} v_i(X_{j}).
    \end{align*}
Further, for each agent $i$, let $\calM^i=(M^i_1, M^i_2, \ldots, M^i_n) \in \Pi_n([m])$ denote an {MMS-inducing partition}:
\begin{align*}
\calM^i \in \argmax_{(X_1,\dots, X_n) \in \Pi_n([m])} \ \ \min_{j\in[n]} v_i(X_{j})
\end{align*}
\end{definition}

Note that in Definition \ref{def:mms} the maximum is taken over all $n$-partitions of $[m]$. Also, by definition, the partition $\calM^i =(M^i_1, \ldots, M^i_n)$ satisfies $v_i(M^i_j) \geq \mu_i$, for each index $j \in [n]$. 

\paragraph{Fair Multi-Allocations.} A multi-allocation $\calA=(A_1,\dots,A_n)$ is said to be an \emph{MMS multi-allocation} (i.e., it is deemed to be fair) if under it each agent receives a bundle of value at least its maximin share:  $v_i(A_i)\geq \mu_i$ for all agents $i \in [n]$.
 
To establish existential guarantees for MMS multi-allocations $\calA$, we will assume that, for all the agents, we are given the MMS-inducing partitions $\calM^i$, which in turn are guaranteed to exist (see Definition \ref{def:mms}).  
\section{Weakest Pre-Expectations for Probabilistic Programs}
\label{sec:programs}

In this section we define programming language $\pWhile$ and its weakest pre-expectation semantics based on \emph{Lebesgue} integrals as defined in~\cite{DBLP:conf/setss/SzymczakK19}.

\subsection{Program Syntax}
\label{sec:programs:syntax}

For the rest of the paper we fix a finite%
\footnote{Once $\pVars$ is fixed we can only write programs with at most $|\pVars|$ distinct variables. However, as we never make any assumptions about the size of $\pVars$, our theory applies to programs with arbitrarily many variables.
Previous work~\cite{DBLP:conf/setss/SzymczakK19} has considered an infinite $\pVars$, but this requires defining a measure space on the infinite-dimensional $\nonNegReals^\pVars$, which is somewhat more involved.
}
set $\pVars = \{\pVar,\pVarb,\ldots\}$ of program variables.
%
A \emph{(program) state} is a variable valuation $\pSt \in \pStates$, where $\pStates$ is a shorthand for the set of functions $\pVars \to \nnReals$.
Notice that our program variables range over the \emph{non-negative} reals.
The restriction to non-negative variables is for technical convenience and not essential, see \Cref{rem:nonNeg}.

To obtain a well-defined weakest pre-expectation semantics of programs involving continuous sampling, we need to have some measure-theoretic fundamentals in mind, provided in\iftoggle{arxiv}{ \Cref{sec:prelims:measure}}{~\cite{arxiv}}.
Suffice it to say here that we consider the standard Borel \proseSigmaAlgebra and Lebesgue measure $\lebmes$ on $\pStates$. Lebesgue integrals of a measurable function $\fun \colon \reals \to \ennReals$ over a measurable set $\measurableSet \subseteq \reals$ are denoted by $\int_\measurableSet \fun \, d\lebmes$, or $\int_\measurableSet \fun(x) \, d\lebmes(x)$.
We explicitly allow Lebesgue integrals to evaluate to $\infty$.

Now let $\aExps$ be a set of measurable functions of type $\pStates \to \nnReals$ %called \emph{arithmetic expressions}, 
and let $\guards$ be a set of measurable functions of type $\states \to \bools$. %called \emph{guards}.
Elements of $\aExps$ and $\guards$ are called \emph{arithmetic expressions} and \emph{guards}, resp.

\begin{definition}[Probabilistic Programs]
	\label{def:pwhile}
    \newcommand{\syntaxDescr}[1]{\text{\textcolor{gray}{(#1)}}}
    %
    Programs $\prog$ in the set $\pWhileWith{\aExps}{\guards}$ of programs with arithmetic expressions from $\aExps$ and guards from $\guards$ adhere to the following grammar:
    %
	\begin{align*}
		\prog \quad\grammarSymb\quad &\SKIP & &\syntaxDescr{effectless program} \\[-1pt]
		\mmid &\DIVERGE & &\syntaxDescr{nonterminating program} \\[-1pt]
		\mmid &\ASSIGN{\pVar}{\aExp} & &\syntaxDescr{assignment; $\pVar \in \pVars, \aExp \in \aExps$} \\[-1pt]
		\mmid &\OBSERVE{\guard} & &\syntaxDescr{conditioning; $\guard \in \guards $} \\[-1pt]
		\mmid &\UNIFASSIGN{\pVar} & &\syntaxDescr{sample from real interval $\uIval$; $\pVar \in \pVars$} \\[-1pt]
		\mmid &\ITE{\guard}{\prog}{\prog} & &\syntaxDescr{conditional choice; $\guard \in \guards$} \\[-1pt]
		\mmid &\PCHOICE{\prog}{\prob}{\prog} & &\syntaxDescr{probabilistic choice; $\prob \in \uIval \cap \rats$} \\[-1pt]
		\mmid &\SEQ{\prog}{\prog} & &\syntaxDescr{sequential composition} \\[-1pt]
		\mmid &\WHILE{\guard}{\prog} & &\syntaxDescr{while loop; $\guard \in \guards$}
	\end{align*}
	%
	If $\aExps$ and $\guards$ are the sets of \emph{all} measurable functions of the corresponding type, we write $\pWhile$ instead of $\pWhileWith{\aExps}{\guards}$.
	%
	A program not containing while-loops is called \emph{loop-free}.\qedDef
\end{definition}

Let us briefly go over each construct, all of which are standard.
$\SKIP$ does nothing.
$\DIVERGE$ is a non-terminating program, i.e., behaves like $\WHILE{\boolConstTrue}{\SKIP}$.
$\ASSIGN{\pVar}{\aExp}$ assigns the value of the arithmetic expression $\aExp$ evaluated in the current state to variable $\pVar$.
$\UNIFASSIGN{\pVar}$ assigns to $\pVar$ a value drawn from the continuous uniform $\uIval$-distribution.
$\OBSERVE{\guard}$  \emph{conditions} the program execution on the guard $\guard$ being true.
$\PCHOICE{\prog_1}{\prob}{\prog_2}$ executes $\prog_1$ with probability $\prob$, otherwise $\prog_2$.
The assumption $\prob \in \uIval \cap \rats$ is to avoid defining a dedicated syntax for probabilities later on in \Cref{sec:syntax}.
$\ITE{\guard}{\prog_1}{\prog_2}$, $\SEQ{\prog_1}{\prog_2}$, and $\WHILE{\guard}{\prog}$ are standard conditional choices, sequential compositions, and while-loops, respectively.
See\iftoggle{arxiv}{ \Cref{app:redudanciesSyntax}}{~\cite{arxiv}} for additional remarks.

\begin{rem}[More General Distributions]
    \label{rem:generalDist}
    In theory, the restriction to uniform $\unitInterval$-distributions does not limit expressiveness:
    Arbitrary distributions can be simulated by sampling from $\unitInterval$ and applying the inverse
	\emph{cumulative distribution function} (CDF) of the target distribution~\cite{DBLP:conf/setss/SzymczakK19}.
    However, to obtain decidability results, we further restrict the syntax of arithmetic expressions and guards to a class of functions expressible in first-order (FO) real arithmetic, see \Cref{sec:syntax}.
    Distributions whose inverse CDF belongs to this class include triangular, trapezoidal, U-quadratic, and Kumaraswamy distributions.
    On the other hand, distributions with transcendental inverse CDF (Gaussian, Laplace, etc.) do not reside in this class.
    We mention two future directions to address this issue:
    (i) leverage heuristics implemented in modern SMT solvers to discharge the generated verification conditions, even if they do not belong to a decidable theory,
    (ii) soundly over/under-approximate the transcendental inverse CDF by FO-expressible algebraic functions.
\end{rem}

\begin{rem}[Soft Conditioning]
	\label{rem:softConditioning}
    Our syntax has native support for \emph{hard conditioning} ($\KWOBSERVE$). Bounded \emph{soft conditioning} (scoring) --- multiplying the current execution with a weight in $\uIval$ --- can be simulated using $\UNIFASSIGN{\pVar}$ and $\KWOBSERVE$, see~\cite[Lemma 5]{DBLP:conf/setss/SzymczakK19} for details.
\end{rem}


\subsection{Weakest Pre-Expectation Semantics}
\label{sec:programs:wp}

\begin{table}[t]
    \caption{
    	Inductive definition of weakest (liberal) pre-expectations for post-expectation $\ex$~\cite{DBLP:conf/setss/SzymczakK19}.
    }
    \label{tab:original}
    \begin{adjustbox}{max width=\textwidth}
        \def\arraystretch{1.2} % increase space between rows
        \begin{tabular}{l l l} 
            \toprule
            $\prog$ & $\wp{\prog}{\ex}\quad$\textcolor{gray}{where $\ex\in\expsmeas$} & $\wlp{\prog}{\ex}\quad$ ~~\textcolor{gray}{where $\ex\in\bexpsmeas$}  \\
            \midrule
            $\SKIP$ & $\ex$ & $\ex$ \\
            %
            $\DIVERGE$ & $0$ & $1$ \\
            %
            $\ASSIGN{\pVar}{\aExp}$ & $\exSubsGen$ & $\exSubsGen$ \\
            %
            $\UNIFASSIGN{\pVar}$ & $\lam{\st}{\int_\uIval \ex(\pStUpdate{\st}{\pVar}{\xi}) \,d\lebmes(\xi)}$ & $\lam{\st}{\int_\uIval \ex(\pStUpdate{\st}{\pVar}{\xi}) \,d\lebmes(\xi)}$ \\
            %
            $\OBSERVE{\guard}$     &$\iv{\guard} \cdot \ex$ &$\iv{\guard} \cdot \ex$ \\
            %
            $\ITE{\guard}{\prog_1}{\prog_2}$ & $\iv{\guard} \cdot \wp{\prog_1}{\ex}+\iv{\neg\guard} \cdot \wp{\prog_2}{\ex}$ & $\iv{\guard} \cdot \wlp{\prog_1}{\ex} + \iv{\neg\guard} \cdot \wlp{\prog_2}{\ex}$ \\
            %
            $\PCHOICE{\prog_1}{\prob}{\prog_2}$ & $\prob \cdot \wp{\prog_1}{\ex} + (1{-}\prob) \cdot \wp{\prog_2}{\ex}$ & $\prob \cdot \wlp{\prog_1}{\ex} + (1{-}\prob) \cdot \wlp{\prog_2}{\ex}$ \\
            %
            $\SEQ{\prog_1}{\prog_2}$ & $\wp{\prog_1}{\wp{\prog_2}{\ex}}$ & $\wlp{\prog_1}{\wlp{\prog_2}{\ex}}$ \\
            %
            $\WHILE{\guard}{\progBody}$ & $\lfpIn{\lambda\fpVar}{\iv{\neg\guard} \cdot \ex + \iv{\guard} \cdot \wp{\progBody}{\fpVar}}$ & $\gfpIn{\lambda\fpVar}{\iv{\neg\guard} \cdot \ex + \iv{\guard} \cdot \wlp{\progBody}{\fpVar}}$ \\
            \bottomrule
        \end{tabular}
    \end{adjustbox}
\end{table}

We now unify the weakest pre-expectation calculi for continuous probabilistic programs from \cite{DBLP:conf/setss/SzymczakK19} with the calculi proposed in \cite{DBLP:journals/toplas/OlmedoGJKKM18}. The latter calculi take \emph{renormalization} --- for conditional expected outcomes --- into account.
The central objects these calculi operate on are \emph{expectations}:
%
\begin{definition}[Expectations]
    We distinguish between the following sets of functions:
    %
    \begin{enumerate}
        \item The set of \emph{expectations} is $\exps = \{\ex \mid \ex\colon \pStates \to \exNonNegReals \}$.
        %
        \item The set of \emph{1-bounded expectations} is $\bexps = \{\ex \mid \ex \colon \pStates \to \uIval \}$.
        %
        \item The set of \emph{measurable (1-bounded) expectations} $\expsmeas$ ($\bexpsmeas$) is the subset of $\exps$ ($\bexps$) containing exactly the Borel-measurable functions.
    \end{enumerate}
    %
    We equip all of these sets with the partial order $\eleq$ defined as $\ex \eleq \exb$ iff $\forall \st \in \states \colon \ex(\st) \leq \exb(\st)$.
    \qedDef
\end{definition}
%
Crucially, we have (see \cite[Lemma 2]{DBLP:conf/setss/SzymczakK19} and\iftoggle{arxiv}{ \Cref{proof:measexpsbicpo}}{~\cite{arxiv}}):
%
\begin{restatable}{lemma}{measexpsbicpo}
    \label{thm:measexpsbicpo}
    $\po{\exps}{\eleq}$ and $\po{\bexps}{\eleq}$ are complete lattices.
    $\po{\expsmeas}{\eleq}$ and $\po{\bexpsmeas}{\eleq}$ are $\omega$-bicpos with bottom ($0 \gray{{}= \lam{\pSt}{0}}$) and top ($\infty \gray{{}= \lam{\pSt}{\infty}}$ and $1 \gray{{}= \lam{\pSt}{1}}$, respectively).
\end{restatable}
%
The arithmetic operations $+$ (addition) and $\cdot$ (multiplication) on $\exps$ are defined {pointwise}, i.e., $\forall \st \in \states \colon (\ex + \exb)(\st) = \ex(\st) + \exb(\st)$, and analogously for multiplication.
These operations preserve measurability~\cite[Theorem 11.18]{rudin1953principles}, i.e., $\expsmeas$ is closed under $+$ and $\cdot$.
Moreover, addition (for both arguments) and multiplication by constants%
\footnote{Formally, for every $\exb \in \exps$ these are the functions $\lam{\ex}{\ex + \exb}$ and $\lam{\ex}{\ex \cdot \exb}$.}
are $\omega$-bicontinuous functions.

For state $\pSt \in \pStates$,  program variable $\pVar \in \pVars$, $\xi \in \nnReals$, we define the \emph{updated state as}
%
\[
    \pStUpdate{\pSt}{\pVar}{\xi}
    \eeq
    \lam{\pVarb}{
        \begin{cases}
        \xi & \text{if $\pVarb = \pVar$} \\
        %
        \pSt(\pVarb) &\text{otherwise}
        \end{cases}
    }
    ~.
\]
%
Further, given an expectation $\ex \in \exps$, a variable $\pVar \in \pVars$, and an arithmetic expression $\aExp \colon 
\pStates \to \nnReals$ we define the \emph{substitution of $\pVar$ by $\aExp$ in $\ex$} as the expectation $\exSubsGen = \lam{\st}{\ex(\pStUpdate{\st}{\pVar}{\aExp(\st)})}$.
If we have syntactic expressions for $\ex$ and $\aExp$, then we can obtain a syntactic representation of $\exSubs{f}{\pVar}{\aExp}$ by substituting all free occurrences of $\pVar$ in $\ex$ by $\aExp$ in a capture-avoiding manner, see \Cref{sec:syntax}.

\begin{definition}[Weakest (Liberal) Pre-expectation Transformers~\textnormal{\cite{DBLP:conf/setss/SzymczakK19}}]
    \label{def:wpwlp}
    For all $\prog \in \pWhile$, the \emph{weakest (liberal) pre-expectation transformers} $\wpTrans{\prog} \colon \expsmeas \to \expsmeas$ and $\wlpTrans{\prog}\colon \bexpsmeas \to \bexpsmeas$ are defined inductively on the structure of $\prog$ according to the rules in \Cref{tab:original}.
    \qedDef
\end{definition}
%
Let us briefly explain the inductive definition in \Cref{tab:original}.
The effectless program $\SKIP$ leaves the post-expectation unchanged.
An assignment $\ASSIGN{\pVar}{\aExp}$ substitutes the expression $\aExp$ for the variable $\pVar$ in the post-expectation $\ex$, while uniform assignment $\UNIFASSIGN{\pVar}$ integrates the expectation over all possible values of $\pVar$ in the interval $\uIval$, capturing the averaging effect of drawing $\pVar$ uniformly.
$\OBSERVE{\guard}$ scales the expectation by the Iverson bracket of the guard $\guard$.
Proper renormalization is performed in a second step, see \Cref{sec:wp:cwp}.
For $\ITE{\guard}{\prog_1}{\prog_2}$ and the probabilistic choice $\PCHOICE{\prog_1}{\prob}{\prog_2}$, the weakest (liberal) pre-expectation is a weighted sum of the $\wpSymb$'s from both branches, either weighted by $\iv{\guard}$ and $\iv{\neg\guard}$ in the former case, or by the probabilities $\prob$ and $1-\prob$ in the latter.
The $\wpwlpSymb$ of a sequential composition $\SEQ{\prog_1}{\prog_2}$ is function composition $\wpwlpTrans{\prog_1} \circ \wpwlpTrans{\prog_2}$.
Notably, $\prog_2$ is evaluated \emph{before} $\prog_1$ --- $\wpwlpSymb$'s are thus computed in a backward manner.
Note that $\wpSymb$ and $\wlpSymb$ only differ in the handling of divergence and loops.
The $\DIVERGE$ command, representing non-termination, sets the $\wpSymb$ to $0$ and the $\wlpSymb$ to $1$.
The $\WHILE{\guard}{\prog}$ loop involves computing the \emph{least} fixed point ($\lfp$) for $\wpSymb$ and the \emph{greatest} fixed point ($\gfp$) for $\wlpSymb$.
The difference between $\wpSymb$ and $\wlpSymb$ is further explained in \Cref{sec:wp:wpVsWlp}.

\begin{rem}[On Non-Negativity]
    \label{rem:nonNeg}
    We follow the classic line of work on weakest pre-expectations~\cite{DBLP:conf/stoc/Kozen83,DBLP:series/mcs/McIverM05} and restrict attention to \emph{non-negative expectations} (see~\cite{DBLP:conf/lics/KaminskiK17} for a discussion of the mixed-sign case).
    This means that we can only reason about expected values of non-negative random variables measured in a program's final state.
    As a consequence, in order to reason about the expected final value of a program variable $\pVar$ on termination, we have to assume that $\pVar$ is unsigned.
    We have opted to ensure this by simply requiring \emph{all} variables to be unsigned real numbers.
\end{rem}

Given a loop $\prog = \WHILE{\guard}{\progBody}$ and $\ex \in \expsmeas$, we denote by 
%
\[
	\charfunwp{\prog}{\ex} \colon \expsmeas \to \expsmeas \,, 
	\qquad 
	\charfunwp{\prog}{\ex}(\exb) \eeq \iv{\guard} \cdot \wp{\progBody}{\exb} + \iv{\neg\guard}\cdot \ex
\]
%
the \emph{$\wpSymb$-characteristic function of $\prog$ w.r.t.\ $\ex$} (analogously for $\ex' \in \bexpsmeas$ and $\charfunwlp{\prog}{\ex'}$).
Hence, weakest pre-expectations of loops can be denoted more concisely as 
%
\begin{align*}
	 \wp{\prog}{\ex} \eeq \lfp \charfunwp{\prog}{\ex}
	 \qquad\text{and}\qquad
	  \wlp{\prog}{\ex'} \eeq \gfp \charfunwlp{\prog}{\ex'}~.
\end{align*}
%
Due to the fixed points in \Cref{tab:original} it is not immediately obvious that $\wpTrans{\prog}$ and $\wlpTrans{\prog}$ are well-defined.
This will we be ensured by Kleene's \Cref{thm:kleene} as shown in the next lemma (parts of which have been proved in~\cite{DBLP:conf/setss/SzymczakK19}).
%
\begin{restatable}[{Well-definedness of $\wpSymb$ and $\wlpSymb$}]{theorem}{wpWlpWellDefinedAndContinuous}
    \label{thm:wpWlpWellDefinedAndContinuous}
    For all programs $\prog \in \pWhile$, $\wpTrans{\prog}$ and $\wlpTrans{\prog}$ are well-defined.
    In particular, $\wpTrans{\prog}$ is $\omega$-continuous and $\wlpTrans{\prog}$ is $\omega$-\underline{co}continuous.
\end{restatable}


\subsection{Probabilistic Termination: $\wpSymb$ vs.\ $\wlpSymb$}
\label{sec:wp:wpVsWlp}

In general, for all $\prog \in \pWhile$ and $\ex\in\bexpsmeas$ we have $\wp{\prog}{\ex} \eleq \wlp{\prog}{\ex}$ since the former relies on a least and the latter on a greatest fixed point.
More specifically, we have~\cite[Section 5]{DBLP:conf/setss/SzymczakK19}
%
\begin{align}
    \wp{\prog}{\ex} + \wlp{\prog}{0} \eeq \wlp{\prog}{\ex} \label{eq:wpwlp}
    ~.
\end{align}
%
\iftoggle{arxiv}{%
Let us make the meaning of $\wpSymb$ and $\wlpSymb$ w.r.t.\ the constant post-expectations $0$ and $1$ explicit.
Assuming $\prog$ is started with initial state $\st$ we have the following:
\footnote{These statements can be formalized by means of an operational semantics for $\pWhile$, see~\cite{DBLP:conf/setss/SzymczakK19}.}%
\begin{itemize}
    \item $\wp{\prog}{0}(\st)$ is always 0.
    \item $\wp{\prog}{1}(\st)$ is the probability of \emph{termination} without violating any $\KWOBSERVE$.
    \item $\wlp{\prog}{0}(\st)$ is the probability of \emph{non-termination} without violating any $\KWOBSERVE$.
    \item $\wlp{\prog}{1}(\st)$ is the probability to never violate any $\KWOBSERVE$.
\end{itemize}
}{}
%
We call $\prog$ \emph{almost-surely terminating} (AST) if $\wlp{\prog}{0} = 0$, i.e., if the program does not admit any initial state for which the program's infinite runs that do not violate any $\KWOBSERVE$ have positive probability mass.
It follows from equation \eqref{eq:wpwlp} that $\prog$ is AST iff $\wp{\prog}{\ex} = \wlp{\prog}{\ex}$.


\subsection{Conditional Weakest Pre-Expectations}
\label{sec:wp:cwp}

For a program $\prog \in \pWhile$, an expectation $\ex \in \expsmeas$ and a state $\pSt$, following \cite{DBLP:journals/toplas/OlmedoGJKKM18}, we define:
%
\[
    \cwp{\prog}{\ex}(\pSt)
    \eeq
    \begin{cases}
        \frac{\wp{\prog}{\ex}(\pSt)}{\wlp{\prog}{\onefun}(\pSt)} &\text{ if } \wlp{\prog}{1}(\pSt) \neq 0 \\[0.5em]
        \text{undefined} & \text{ else.}
    \end{cases}
\]
%
The above definition factors out the probability mass of runs violating an $\KWOBSERVE$, i.e., divides by $\wlp{\prog}{\onefun}(\pSt)$, see \Cref{sec:wp:wpVsWlp}.
$\cwp{\prog}{\ex}(\st)$ is thus the expected value of $\fun$ after termination of $\prog$ started with initial state $\st$, \emph{conditioned} on all $\KWOBSERVE$ statements in $\prog$ being successful.

\section{Approximate Riemann Weakest Pre-Expectations}
\label{sec:approxwp}

We start by recalling lower and upper Riemann sums and integrals.%
\footnote{
    The definition of an integral in terms of these sums is in fact commonly attributed to Darboux, not to Riemann.
    However, Darboux's integral is equivalent to Riemann's which, rather than considering lower and upper sums, relies on evaluating $\fun$ at sample points within the intervals of a partition of the integration domain.
    We refer to~\cite[Chapter 3, Theorems 3.3.1 and 3.3.2]{burk2007garden} for an in-depth comparison.
    In this paper, we consistently use Darboux's definitions, but refer to them nonetheless as Riemann sums and integrals, since the latter terminology is more widespread.
}
%
Let $\clIvalGen \subseteq \reals$ and $\partitionSize \geq 1$.
A \emph{partition} of $\clIvalGen$ is a tuple of at least $\partitionSize+1$ real numbers $\partition = (x_0,x_1,\ldots,x_{\partitionSize})$ such that
\[
    \ivalL
    =
    x_0
    \llt
    x_1
    \llt
    \ldots
    \llt
    x_{\partitionSize}
    =
    \ivalR
    ~.
\]
The set of all partitions of the interval $\clIvalGen$ is denoted $\partitions{\clIvalGen}$.
%
For a bounded $\fun \colon \clIvalGen \to \reals$ and a partition $\partition = (x_0,\ldots,x_{\partitionSize}) \in \partitions{\clIvalGen}$, we define the \emph{lower} and \emph{upper sums} of $\fun$ w.r.t.\ $\partition$ as
\[
    \lowerSum{\fun}{\partition}
    \eeq
    \sum_{i=1}^{\partitionSize} (x_{i} - x_{i-1}) \inf_{\xi \in \clIval{x_{i-1}}{x_i}} \fun(\xi)
    %
    \qqand
    %
    \upperSum{\fun}{\partition}
    \eeq
    \sum_{i=1}^{\partitionSize} (x_{i} - x_{i-1}) \sup_{\xi \in \clIval{x_{i-1}}{x_i}} \fun(\xi)
    ~.
\]
Note that $\upperSum{\fun}{\partition}$ and $\lowerSum{\fun}{\partition}$ are well-defined real numbers because $\fun$ is bounded.

\begin{definition}[Riemann integral]
    \label{def:riemannIntegral}
    Let $\fun \colon \clIvalGen \to \reals$ be bounded.
    The \emph{lower-} and \emph{upper Riemann integrals} of $\fun$ are defined as follows:
    \[
        \lowerIntGen \fun(x) \,dx
        \eeq
        \sup \left\{ \lowerSum{\fun}{\partition} \mid \partition \in \partitions{\clIvalGen} \right\}
        %
        \qqand
        %
        \upperIntGen \fun(x) \,dx
        \eeq
        \inf \left\{ \upperSum{\fun}{\partition} \mid \partition \in \partitions{\clIvalGen} \right\}
        ~.
    \]
    If the upper integral equals the lower integral, then the common value is written $\int_{\ivalL}^{\ivalR} \fun(x) \,dx$ and called \emph{the} Riemann integral of $\fun$.
    In this case, $\fun$ is called \emph{Riemann-integrable}.
    \qedDef
\end{definition}

Finally, we  introduce the following terminology:
%
(i) For $\funb \colon \realDomain \to \reals$, $\realDomain \subseteq \reals$, we say that $\funb$ is Riemann-integrable \emph{on an interval} $\clIval{\ivalLb}{\ivalRb} \subset \realDomain$ if the restriction $\funb \colon \clIval{\ivalLb}{\ivalRb} \to \reals$ is Riemann-integrable.
%
(ii) In this paper we often consider functions of the form $\funb \colon \realDomain^\pVars \to \reals$ where $\pVars$ is a finite set, e.g.\ the set of program variables.
We then say that $\funb$ is Riemann-integrable on $\clIval{\ivalLb}{\ivalRb} \subseteq \realDomain$ w.r.t.\ some $\pVar \in \pVars$ if \emph{for all} $\pSt \in \realDomain^\pVars$ the function $\lam{\xi}{\funb(\pStUpdate{\pSt}{\pVar}{\xi})}$ of type $\clIval{\ivalLb}{\ivalRb} \to \reals$ is Riemann-integrable.

\iftoggle{arxiv}{
\begin{example}
    To illustrate the concepts defined above consider the following:
    \begin{itemize}
        \item A constant function $\fun \colon \clIvalGen \to \reals, x \mapsto \realConst$ for $\realConst \in \reals$ is Riemann-integrable because for \emph{all} partitions $\partition$ we have $\lowerSum{\fun}{\partition} = \upperSum{\fun}{\partition} = \realConst (\ivalL-\ivalR)$.
        \item $\fun(x,y) = \iv{x^2 + y^2 \leq 1}$ is Riemann-integrable on $\unitInterval$ w.r.t.\ $x$.
        Indeed, for all $y \in \reals$, we have $\int_{0}^{1} \fun(x,y) \, dx = \iv{-1 \leq y \leq 1} \sqrt{1-y^2}$.
        \item The Dirichlet function $\fun \colon \clIvalGen \to \reals, x \mapsto \iv{x \in \rats}$ is not Riemann-integrable because for all partitions $\partition$ we have $\lowerSum{\fun}{\partition} = 0$ and $\upperSum{\fun}{\partition} = 1$.
        Note that $\fun$ is nowhere continuous.
    \end{itemize}
\end{example}
}{}


\subsection{{$\lwpSymb{\N}$ and $\uwpSymb{\N}$: Lower and Upper Riemann Weakest Pre-Expectations}}
\label{sec:approxwp:def}

We are now ready to define our approximate expectation transformers.
They arise from the standard transformers defined in \Cref{tab:original} by replacing the \emph{Lebesgue} integrals in the $\UNIF_{\uIval}$ case by a lower or upper Riemann sum. 
Formally:

\begin{definition}[Lower and Upper Riemann $\wpwlpSymb$-Transformers]
	\label{def:riemannwpwlp}
	For all $\prog \in \pWhile$ and integers $\N \geq 1$, the expectation transformers $\lwpTrans{\N}{\prog} \colon \exps \to \exps$ and $\uwpTrans{\N}{\prog} \colon \exps \to \exps$ are defined by induction over the structure of $\prog$ as in \Cref{tab:original}, with the only exception that if $\prog$ is $\UNIFASSIGN{\pVar}$, then for all $\ex \in \exps$ we set%
	\footnote{Notice that $\inf_{\xi \in \clIvalGen}\exSubs{\ex}{\pVar}{\xi}$ is the same as $\lam{\st}{\inf_{\xi \in \clIvalGen} \ex(\pStUpdate{\st}{\pVar}{\xi})}$.}
	\begin{align*}
		&\lwp{\N}{\UNIFASSIGN{\pVar}}{\ex}
		\eeq
		\frac{1}{\N} \sum_{i=0}^{\N-1} \inf_{\xi \in [\frac{i}{\N}, \frac{i+1}{\N}]} \exSubs{\ex}{\pVar}{\xi} \qquad\text{and, similarly,} \\
		&\uwp{\N}{\UNIFASSIGN{\pVar}}{\ex}
		\eeq
		\frac{1}{\N} \sum_{i=0}^{\N-1} \sup_{\xi \in [\frac{i}{\N}, \frac{i+1}{\N}]} \exSubs{\ex}{\pVar}{\xi}
		~.
	\end{align*}
	The lower and upper Riemann weakest \emph{liberal} pre-expectation transformers $\lwlpTrans{\N}{\prog} \colon \bexps \to \bexps$ and $\uwlpTrans{\N}{\prog} \colon \bexps \to \bexps$ are defined analogously.
	\qedDef
\end{definition}

Our Riemann transformers approximate the Lebesgue integral in the definition of $\wpSymb$ (and $\wlpSymb$) by a lower or upper Riemann sum.
We work with the partitions $0 < \tfrac{1}{\N} < \tfrac{2}{\N} < \ldots < 1$ of the unit interval for the sake of concreteness.
Note that these partitions are \emph{not} successive refinements of each other (see\iftoggle{arxiv}{ \Cref{thm:partitionRefine}}{~\cite{arxiv}} for definitions).
Thus, increasing $\N$ by $1$ does \emph{not necessarily} yield \enquote{better} approximations, i.e., \emph{we might have} $\lwp{\N}{\prog}{\ex} \not\eleq \lwp{\N+1}{\prog}{\ex}$%
\iftoggle{arxiv}{%
, e.g:
%
\[
    \lwp{2}{\UNIFASSIGN{\pVar}}{\iv{x \geq \tfrac 1 2}} ~\not\eleq~ \lwp{3}{\UNIFASSIGN{\pVar}}{\iv{x \geq \tfrac 1 2}}
    ~.
\]
}{.}

Given a loop $\prog=\WHILE{\guard}{\progBody}$, $\N\geq 1$, $\ex\in\exps$, and $\transSymb\in\{\uwpSymb{\N},\lwpSymb{\N}\}$, we define the \emph{$\transSymb$-characteristic function of $\prog$ w.r.t.\ $\ex$} as 
%
\[
\charfuntrans{\prog}{\ex} \colon \exps \to \exps, 
\qquad 
\charfuntrans{\prog}{\ex}(\exb) \eeq \iv{\guard} \cdot \somewp{\progBody}{\exb} + \iv{\neg\guard}\cdot \ex~.
\]
%
For $\exb\in\bexps$ and $\transSymb\in\{\uwlpSymb{\N},\lwlpSymb{\N}\}$, $\charfuntrans{\prog}{\exb} \colon \bexps \to \bexps$ 
\iftoggle{arxiv}{%
we analogously define
%
\[
	\charfuntrans{\prog}{\exb} \colon \bexps \to \bexps, 
	\qquad 
	\charfuntrans{\prog}{\exb}(\exb) \eeq \iv{\guard} \cdot \somewp{\progBody}{\exb} + \iv{\neg\guard}\cdot \exb~.
\]
%
}{%
is defined analogously.
}
Note that unlike $\wpSymb$ and $\wlpSymb$, the approximate transformers are defined on the full sets of expectations $\exps$ and $\bexps$, respectively, not just on the measurable ones.
All transformers from \Cref{def:riemannwpwlp} are well-defined: this follows from the Knaster-Tarski \Cref{thm:knasterTarski} using that $\exps$ and $\bexps$ are complete lattices and monotonicity of the transformers, see \Cref{thm:monriemann} below.


\subsection{Healthiness Properties of $\lwpSymb{\N}$ and $\uwpSymb{\N}$}
\label{sec:approxwp:basic}

\begin{restatable}[Monotonicity of the Riemann $\wpwlpSymb$-Transformers]{lemma}{monriemann}
	\label{thm:monriemann}
	For all $\prog \in \pWhile$ and integers $\N \geq 1$, the functions $\lwpTrans{\N}{\prog}$, $\uwpTrans{\N}{\prog}$, $\lwlpTrans{\N}{\prog}$, and $\uwlpTrans{\N}{\prog}$ are monotonic w.r.t.\ $\eleq$.
\end{restatable}

Notably, the Riemann $\wpwlpSymb$-transformers do not possess the same continuity properties as their Lebesgue counterparts from \Cref{sec:programs:wp}.
For instance, due to the presence of infima in the lower Riemann sum, $\lwpTrans{\N}{\prog}$ is \emph{not} $\omega$-continuous in general (see\iftoggle{arxiv}{ \Cref{proof:lwpNotOmegaCont}}{~\cite{arxiv}} for a counter-example).

\begin{restatable}[Soundness of the Riemann $\wpwlpSymb$-Transformers]{lemma}{soundness}
	\label{thm:soundness}
	For all programs $\prog \in \pWhile$, post-expectations $\ex \in \expsmeas$, $\exb \in \bexpsmeas$, and integers $\N \geq 1$,
	\begin{align*}
		\lwp{\N}{\prog}{\ex}
		&\eeleq
		\wp{\prog}{\ex}
		\eeleq
		\uwp{\N}{\prog}{\ex} \qand 
		\\
		\lwlp{\N}{\prog}{\exb}
		&\eeleq
		\wlp{\prog}{\exb}
		\eeleq
		\uwlp{\N}{\prog}{\exb}
		~.
	\end{align*}
\end{restatable}

\noindent For \emph{conditional} weakest pre-expectations (\Cref{sec:wp:cwp}), we immediately get the following:
%
\begin{corollary}
    \label{thm:cwpSound}
	For all programs $\prog \in \pWhile$, post-expectations $\ex \in \expsmeas$, and integers $\N \geq 1$
	\begin{align*}
		\frac{\lwp{\N}{\prog}{\ex}(\pSt)}{\uwlp{\N}{\prog}{\onefun}(\pSt)}
		\lleq
		\cwp{\prog}{\ex}(\pSt)
		\lleq
		\frac{\uwp{\N}{\prog}{\ex}(\pSt)}{\lwlp{\N}{\prog}{\onefun}(\pSt)}
	\end{align*}
	for all $\pState \in \pStates$ such that none of $\lwlp{\N}{\prog}{1}(\pSt)$, $\wlp{\prog}{1}(\pSt)$, and $\uwlp{\N}{\prog}{1}(\pSt)$ is zero.
\end{corollary}
%
\noindent We stress that \Cref{thm:soundness} and \Cref{thm:cwpSound} hold even for non-Riemann-integrable post-expectations.

\section{Invariant-Based Reasoning for Loops}
\label{sec:invariants_and_unrolling}
%
Deductive probabilistic program verification techniques typically bound expected outcomes of loops by means of \emph{quantitative loop invariants}. Intuitively, a quantitative loop invariant $\exI$ is an expectation whose pre-expectation w.r.t.\ \emph{one loop iteration} does not increase (or decrease, depending on whether one wants to establish upper or lower bones, see below).
While quantitative loop invariant-based reasoning can simplify the verification of loops significantly, 
%
the continuous setting poses challenges: Computing the pre-expectation of $\exI$ w.r.t.\ a loop's body requires reasoning about possibly complex Lebesgue integrals --- involving, e.g., indicator functions of predicates --- arising from the continuous sampling instructions.
%
We now develop invariant-based proof rules for our \emph{Riemann pre-expectations} (\Cref{def:riemannwpwlp}), which yield sound bounds on the \emph{Lebesgue pre-expectations} (\Cref{def:wpwlp}).
We will see in \Cref{sec:syntax,sec:case_studies} that these proof rules give rise to SMT-based techniques for verifying bounds on Lebesgue pre-expectations of loops in a semi-automated fashion.
%
%

%
%
Our first insight is that --- due to $(\exps, \, \eleq)$ and $(\bexps, \, \eleq)$ being complete lattices and the Riemann expectation transformers being monotonic by \Cref{thm:monriemann} --- bounds on \emph{Riemann} pre-expectations of loops can be established via Park induction (\Cref{thm:knasterTarski}):
%
%
\begin{lemma}
	\label{thm:invariant_approx}
	Let $\prog = \WHILE{\guard}{\progBody} \in\pWhile$ and $\N \geq 1$. Then:
	%
    \begin{enumerate}
		\item For all $\ex,\exI \in \exps$, we have
		%
		$
			\underbrace{\charfunuwp{\N}{\prog}{\ex}(\exI) \eleq \exI}_{\text{$\exI$ is $\uwpSymb{\N}$-superinvariant of $\prog$ w.r.t.\ $\ex$}}
			\qquad\text{implies}\qquad
			%
			\uwp{\N}{\prog}{\ex} \eleq \exI~.
		$
		%
		%
		\item For all $\ex,\exI \in \bexps$, we have
		%
		$
            \underbrace{\exI \eleq \charfunlwlp{\N}{\prog}{\ex}(\exI)}_{\text{$\exI$ is $\lwlpSymb{\N}$-subinvariant of $\prog$ w.r.t.\ $\ex$}}
            \qquad\text{implies}\qquad
            %
            \exI \eleq \lwlp{\N}{\prog}{\ex}~.
		$
	\end{enumerate}
\end{lemma}
%
%
More colloquially stated, \emph{super}invariants yield \emph{upper} bounds on \emph{upper} Riemann pre-expectations of loops and, dually, \emph{sub}invariants yield \emph{lower} bounds on \emph{lower liberal} Riemann pre-expectations. Notice that establishing the premise of the above proof rules only requires reasoning about the loop's {body} and avoids explicitly computing Lebesgue integrals.

It thus follows from the soundness of our Riemann expectation transformers (\Cref{thm:soundness}) that the above proof rules yield sound bounds on \emph{Lebesgue} pre-expectations.
%
\begin{theorem}
	\label{thm:invariant_cont}
	Let $\prog = \WHILE{\guard}{\progBody} \in\pWhile$ and $\N,\N' \geq 1$. We have:
	%
	\begin{enumerate}
		\item If $\exI \in \exps$ is a $\uwpSymb{\N}$-superinvariant of $\prog$ w.r.t.\ $\ex\in\expsmeas$, then
		%
		$\wp{\prog}{\ex} \eleq \exI~$.
		%
		\item If $\exI \in \bexps$ is a $\lwlpSymb{\N}$-subinvariant of $\prog$ w.r.t.\ $\ex\in\bexpsmeas$, then
		%
		$\exI \eleq \wlp{\prog}{\ex}$.
		%
		\item If $\exI \in \exps$ is a $\uwpSymb{\N}$-superinvariant of $\prog$ w.r.t.\ $\ex\in\expsmeas$ and $\exJ \in \bexps$ is a $\lwlpSymb{\N'}$-subinvariant of $\prog$ w.r.t.\ $1$, then, for all $\pState\in\pStates$, $\exJ(\pState) >0$ implies that $\cwp{\prog}{\ex}(\pState)$ is defined and
		%
		\[
			\cwp{\prog}{\ex}(\pState) \lleq \frac{\exI(\pState)}{\exJ(\pState)}~.
		\]
	\end{enumerate}
	%
\end{theorem}
%
It is important to note that in \Cref{thm:invariant_approx} we admit \emph{arbitrary} ($1$-bounded) post-expectations whereas in \Cref{thm:invariant_cont} we have to restrict to \emph{measurable} ($1$-bounded) post-expectations in order for the Lebesgue pre-expectations $\wp{\prog}{\ex}$ and $\wlp{\prog}{\ex}$ to be well-defined.
The quantitative loop invariant $\exI$, on the other hand, must \emph{not} necessarily be measurable since there is no need to plug $\exI$ into a Lebesgue expectation transformer.
%
%
%
\begin{example}
	\label{ex:invariant_monte_carlo} 
	%
	Consider the Monte Carlo $\pi$-approximator $\prog$ from \Cref{fig:intro}.
	%
	The expectation
	%
	\[
	   \exI \eeq \pVarcount + \iv{\pVari \leq \pVarm} \cdot \big(0.85\cdot ((\pVarm \monus \pVari) + 1)\big)
	\]
	%
	is a $\uwpSymb{16}$-superinvariant of $\prog$ w.r.t.\ $\pVarcount$ (here $\monus$ denote the \emph{monus} operator defined as $\synTerm_1 \synMonus \synTerm_2 = \max(\synTerm_1 - \synTerm_2, 0)$, detailed in \Cref{sec:syntax:defs}).
    Hence, we get by \Cref{thm:invariant_cont},
	%
	$\wp{\prog}{\pVarcount} \eleq \exI$,
	%
	i.e., if initially $\pVarcount = 0, \pVari = 1$, then $0.85\cdot \pVarm $ upper-bounds the expected final value of $\pVarcount$. 
\end{example}

\subsection{{Local boundedness and Lipschitz  properties of $\ell_{in}$, $\ell_{out}$, and their derivatives }}

\begin{proposition}[{Local} boundedness]\label{prop:uniform_boundedness}
Under \cref{assump:compact,assump:convexity_lin,assump:K_bounded,assump:reg_lin_lout}, the functions $(\omega,x,y)\mapsto \ell_{out}(\omega,h_{\omega}^{\star}(x),y)$, $(\omega,x,y)\mapsto \partial_{\omega}\ell_{out}(\omega,h_{\omega}^{\star}(x),y)$, and $(\omega,x,y)\mapsto\partial_v \ell_{out}(\omega, h^\star_\omega(x), y)$ are bounded over $\textnormal{hull}(\Omega)\times\mathcal{X}\times\mathcal{Y}$ by some positive constant $M_{out}$. Similarly, the functions $(\omega,x,y)\mapsto\partial_v \ell_{in}(\omega, h^\star_\omega(x), y)$, $(\omega,x,y)\mapsto\partial_v^2 \ell_{in}(\omega, h^\star_\omega(x), y)$, and $(\omega,x,y)\mapsto\partial_{\omega, v}^2 \ell_{in}(\omega, h^\star_\omega(x), y)$  are bounded over $\textnormal{hull}(\Omega)\times\mathcal{X}\times\mathcal{Y}$ by some positive constant $M_{in}$. 
The constants $M_{out}$ and $M_{in}$ are defined as:
\begin{align*}
M_{out}&\coloneqq\sup_{\omega\in\textnormal{hull}(\Omega),v\in\mathcal{V},y\in\mathcal{Y}}\max\left(\verts{\ell_{out}(\omega,v,y)},\Verts{\partial_{\omega}\ell_{out}(\omega,v,y)},\verts{\partial_v\ell_{out}(\omega, v, y)}\right)>0,\\
M_{in}&\coloneqq\sup_{\omega\in\textnormal{hull}(\Omega),v\in\mathcal{V},y\in\mathcal{Y}}\max\left(\verts{\partial_v\ell_{in}(\omega, v, y)}, \verts{\partial_v^2\ell_{in}(\omega, v, y)},\Verts{\partial_{\omega, v}^2\ell_{in}(\omega, v, y)}\right)>0,
\end{align*}
where $\mathcal{V}\subset\mathbb{R}$ is the compact interval defined in \cref{prop:bound_hstaromega}.
\end{proposition}

\begin{proof}
By \cref{prop:bound_hstaromega}, we have that $h^\star_\omega(x)\in\mathcal{V}\coloneqq\left[-\frac{B\kappa}{\lambda},\frac{B\kappa}{\lambda}\right]\subset\mathbb{R}$, for any $x\in\mathcal{X}$. From \cref{assump:reg_lin_lout}, we know that $\ell_{in}$, $\ell_{out}$, and their partial derivatives are all continuous on $\text{hull}(\Omega)\times\mathcal{V}\times\mathcal{Y}$. Also, $\mathcal{Y}$ is compact by \cref{assump:compact}. Thus, $\text{hull}(\Omega)\times\mathcal{V}\times\mathcal{Y}$ is compact. As every continuous function defined over a compact space is bounded, we obtain that:
\begin{align*}
    \sup_{\omega\in\text{hull}(\Omega),x\in\mathcal{X},y\in\mathcal{Y}}\verts{\ell_{out}(\omega,h_{\omega}^{\star}(x),y)}&\leq\sup_{\omega\in\text{hull}(\Omega),v\in\mathcal{V},y\in\mathcal{Y}}\verts{\ell_{out}(\omega,v,y)}<+\infty,\\
    \sup_{\omega\in\text{hull}(\Omega),x\in\mathcal{X},y\in\mathcal{Y}}\Verts{\partial_{\bullet} \ell_{\circ}(\omega,h_{\omega}^{\star}(x),y)}&\leq\sup_{\omega\in\text{hull}(\Omega),v\in\mathcal{V},y\in\mathcal{Y}}\Verts{\partial_{\bullet} \ell_{\circ}(\omega,v,y)}<+\infty,
\end{align*}
where $\bullet \in \{\{v\}, \{w\}, \{w,v\}\}$ and $\circ \in \{in,out\}$. This implies the desired result.
\end{proof}

\begin{proposition}[Local Lipschitz continuity]\label{prop:uniform_Lipschitzness}
	Under \cref{assump:compact,assump:convexity_lin,assump:K_bounded,assump:reg_lin_lout}, there exists a positive constant $\lipout$ so that for any $(x,y)$ in $\mathcal{X}\times \mathcal{Y}$, 
	the functions $\omega\mapsto \ell_{out}(\omega,h_{\omega}^{\star}(x),y)$, $\omega\mapsto \partial_{\omega}\ell_{out}(\omega,h_{\omega}^{\star}(x),y)$, and $\omega\mapsto\partial_v \ell_{out}(\omega, h^\star_\omega(x), y)$ are locally $\frac{\lipout}{\lambda}$-Lipschitz continuous over $\textnormal{hull}(\Omega)$. Similarly, there exists a positive constant $\lipin$ so that for any $(x,y)$ in $\mathcal{X}\times \mathcal{Y}$, the functions $\omega\mapsto\partial_v \ell_{in}(\omega, h^\star_\omega(x), y)$, $\omega\mapsto\partial_v^2 \ell_{in}(\omega, h^\star_\omega(x), y)$, and $\omega\mapsto\partial_{\omega, v}^2 \ell_{in}(\omega, h^\star_\omega(x), y)$ are locally $\frac{\lipin}{\lambda}$-Lipschitz continuous $\textnormal{hull}(\Omega)$.  The constants $\lipout$ and $\lipin$ are defined, for any $0<\lambda\leq \Lambda$, as:
	\begin{align*}
	    \lipout&\coloneqq\left(\Lambda+M_{in}\kappa\right)\max\left(M_{out},\bar{M}_{out}\right)>0\\
	    \lipin&\coloneqq\left(\Lambda+M_{in}\kappa\right)\max\left(M_{in},\bar{M}_{in}\right)>0,
    \end{align*}
    where:
    \begin{align*}
      \bar{M}_{out}&\coloneqq\sup_{\omega\in\textnormal{hull}(\Omega), v\in\mathcal{V}, y\in\mathcal{Y}}\max\left(\Verts{\partial_\omega^2\ell_{out}(\omega, v, y)}_{\op},\Verts{\partial_{\omega, v}^2\ell_{out}(\omega, v, y)},\verts{\partial_v^2\ell_{out}(\omega, v, y)}\right)>0,\\
      \bar{M}_{in}&\coloneqq\sup_{\omega\in\textnormal{hull}(\Omega), v\in\mathcal{V}, y\in\mathcal{Y}}\max\left(\Verts{\partial_\omega\partial_v^2\ell_{in}(\omega, v, y)},\verts{\partial_v^3\ell_{in}(\omega, v, y)},\Verts{\partial_\omega\partial_{\omega, v}^2\ell_{in}(\omega, v, y)}\right)>0,
    \end{align*}
    with $M_{in}$ and $M_{out}$ being the positive constants defined in \cref{prop:uniform_boundedness}, and $\mathcal{V}\subset\mathbb{R}$ is the compact interval defined in \cref{prop:bound_hstaromega}.
\end{proposition}


\begin{proof}
For any $(\omega,x,y)\in\text{hull}(\Omega)\times \mathcal{X}\times \mathcal{Y}$, we have:
\begin{align*}
    \Verts{\nabla_\omega\ell_{out}(\omega, h^\star_\omega(x), y)}&=\Verts{\partial_\omega\ell_{out}(\omega, h^\star_\omega(x), y)+\partial_v\ell_{out}(\omega, h^\star_\omega(x), y)\partial_\omega h^\star_\omega(x)}\\
    &\leq\Verts{\partial_\omega\ell_{out}(\omega, h^\star_\omega(x), y)}+\verts{\partial_v\ell_{out}(\omega, h^\star_\omega(x), y)}\Verts{\partial_\omega h^\star_\omega}_{\op}\Verts{K(x,\cdot)}_\mathcal{H}\\
    &\leq M_{out}\left(1+\frac{M_{in}\kappa}{\lambda}\right)\\
    &\leq\frac{M_{out}\left(\Lambda+M_{in}\kappa\right)}{\lambda},
\end{align*}
where the first line uses the chain rule, the second line applies the triangle inequality and the reproducing property of the RKHS $\mathcal{H}$, the third line follows from \cref{prop:uniform_boundedness} to bound the derivatives of $\ell_{out}$, from \cref{prop:lip_hstaromega}, which states that the function $\omega\mapsto h^\star_\omega$ is $\frac{L\sqrt{\kappa}}{\lambda}$-Lipschitz continuous with $L\coloneqq\sup_{\omega\in\text{hull}(\Omega),v\in\mathcal{V},y\in\mathcal{Y}}\left\|\partial_{\omega, v}^2 \ell_{in}(\omega, v, y)\right\|<M_{in}$, to bound $\Verts{\partial_\omega h^\star_\omega}_{\op}$, and from \cref{assump:K_bounded} to bound $\Verts{K(x,\cdot)}_\mathcal{H}$, and the last line is a direct consequence of $0<\lambda\leq \Lambda$. In a similar way, we obtain:
\begin{gather*}
    \Verts{\nabla_\omega\partial_\omega\ell_{out}(\omega, h^\star_\omega(x), y)}_{\op}\leq\frac{\bar{M}_{out}\left(\Lambda+M_{in}\kappa\right)}{\lambda},\quad\Verts{\nabla_\omega\partial_v \ell_{out}(\omega, h^\star_\omega(x), y)}\leq\frac{\bar{M}_{out}\left(\Lambda+M_{in}\kappa\right)}{\lambda},\\
    \Verts{\nabla_\omega\partial_v \ell_{in}(\omega, h^\star_\omega(x), y)}\leq\frac{M_{in}\left(\Lambda+M_{in}\kappa\right)}{\lambda},\quad\Verts{\nabla_\omega\partial_v^2 \ell_{in}(\omega, h^\star_\omega(x), y)}\leq\frac{\bar{M}_{in}\left(\Lambda+M_{in}\kappa\right)}{\lambda},\\
    \Verts{\nabla_\omega\partial_{\omega, v}^2 \ell_{in}(\omega, h^\star_\omega(x), y)}_{\op}\leq\frac{\bar{M}_{in}\left(\Lambda+M_{in}\kappa\right)}{\lambda}.
\end{gather*}
Combining all these bounds concludes the proof.
\end{proof}

\section{Generalization Properties}\label{app:sec_conv}

{As before, let $\Omega$ be an arbitrary compact subset of $\mathbb{R}^d$.}

\subsection{Point-wise estimates}\label{app:subsec_point_est}
We present a point-wise upper-bound on the value error $\verts{\mathcal{F}(\omega)-\widehat{\mathcal{F}}(\omega) }$ and gradient error $\Verts{\nabla\mathcal{F}(\omega)-\widehat{\nabla\mathcal{F}}(\omega) }$. To this end, we introduce the following notation for the error between the inner and outer objectives and their empirical approximations evaluated at the optimal inner solution $h_{\omega}^{\star}$: 
\begin{align*}
    \Dout\coloneqq \verts{L_{out}(\omega, h^\star_\omega)-\widehat{L}_{out}(\omega, h^\star_\omega)}, 
    \qquad 
    \Din \coloneqq \verts{L_{in}(\omega, h^\star_\omega)-\widehat{L}_{in}(\omega, h^\star_\omega)}.
\end{align*}
By abuse of notation, we introduce the following error between  partial derivatives of $L_{in}$ and $\widehat{L}_{in}$ (resp. $L_{out}$ and $\widehat{L}_{out}$), evaluated at $(\omega, h_{\omega}^{\star})$, \textit{i.e.},
\begin{align*}
  \Douth &\coloneqq \Verts{\partial_h L_{out}(\omega, h^\star_\omega)-\partial_h\widehat{L}_{out}(\omega, h^\star_\omega)}_{\mathcal{H}},\quad 
  &\Doutw &\coloneqq \Verts{\partial_\omega L_{out}(\omega, h^\star_\omega)-\partial_\omega\widehat{L}_{out}(\omega, h^\star_\omega)},\\
  \Dinh &\coloneqq \Verts{\partial_h L_{in}(\omega, h^\star_\omega)-\partial_h\widehat{L}_{in}(\omega, h^\star_\omega)}_{\mathcal{H}},\quad 
  &\Dinw &\coloneqq \Verts{\partial_\omega L_{in}(\omega, h^\star_\omega)-\partial_\omega\widehat{L}_{in}(\omega, h^\star_\omega)},\\
  \Dinhh &\coloneqq \Verts{\partial_{h}^2 L_{in}(\omega, h^\star_\omega)-\partial_h^2\widehat{L}_{in}(\omega, h^\star_\omega)}_{\op},\quad 
  &\Dinwh &\coloneqq \Verts{\partial_{\omega,h}^2 L_{in}(\omega, h^\star_\omega)-\partial_{\omega,h}^2\widehat{L}_{in}(\omega, h^\star_\omega)}_{\op}.
\end{align*}

\begin{proposition}\label{prop:diff_hstar_hhat}
    Under \cref{assump:compact,assump:convexity_lin,assump:K_bounded,assump:reg_lin_lout}, the following holds for any $\omega\in\Omega$:
    \begin{equation*}
        \left\|h^\star_\omega-\hat{h}_\omega\right\|_\mathcal{H}\leq\frac{1}{\lambda}\left\| \partial_h\widehat{L}_{in}(\omega, h^\star_\omega)\right\|_\mathcal{H} = \frac{1}{\lambda}\Dinh.
    \end{equation*}
\end{proposition}
\begin{proof}
Let $\omega\in\Omega$. 
The function $h\mapsto\widehat{L}_{in}(\omega, h)$ is $\lambda$-strongly convex and Fr\'echet differentiable by  \cref{prop:strong_convexity_Lin,prop:fre_diff_L}. Moreover,  $\hat{h}_{\omega}$ is the minimizer of $h\mapsto\widehat{L}_{in}(\omega, h)$ by definition. 
Therefore, using \cref{lem:h_min_hstar}, we obtain a control on the distance in $\mathcal{H}$ to the optimum $\hat{h}_{\omega}$ of $h\mapsto\widehat{L}_{in}(\omega,h)$ in terms of the gradient $\partial_{h}\widehat{L}_{in}(\omega,h)$:
\begin{align*}
	\left\|h-\hat{h}_\omega\right\|_\mathcal{H}\leq\frac{1}{\lambda}\Verts{\partial_h\widehat{L}_{in}(\omega, h)}_\mathcal{H}, \qquad \forall h\in \mathcal{H}.
\end{align*}
In particular, choosing $h= h^\star_\omega$ yields the first inequality. The fact that $\Verts{\partial_h\widehat{L}_{in}(\omega, h_{\omega}^{\star})}_\mathcal{H}=\Dinh$ follows from the optimality of $h_{\omega}^{\star}$ which implies that  $\partial_h L_{in}(\omega, h_{\omega}^{\star})=0$. 
\end{proof}

\begin{proposition}\label{prop:lip_continuity_out}
Under \cref{assump:compact,assump:convexity_lin,assump:K_bounded,assump:reg_lin_lout}, the following inequalities hold for any $\omega\in\Omega$:
\begin{align*}
	\Eout &\coloneqq\verts{\widehat{L}_{out}(\omega, h^\star_\omega)-\widehat{L}_{out}(\omega, \hat{h}_\omega)}\leq C_{out}\Verts{h^\star_\omega-\hat{h}_\omega}_\mathcal{H},\\
	\Eouth&\coloneqq\Verts{\partial_h\widehat{L}_{out}(\omega, h^\star_\omega)-\partial_h\widehat{L}_{out}(\omega, \hat{h}_\omega)}_\mathcal{H} \leq C_{out}\Verts{h^\star_\omega-\hat{h}_\omega}_\mathcal{H},\\
	\Eoutw&\coloneqq\Verts{\partial_\omega\widehat{L}_{out}(\omega, h^\star_\omega)-\partial_\omega\widehat{L}_{out}(\omega, \hat{h}_\omega)} \leq C_{out}\Verts{h^\star_\omega-\hat{h}_\omega}_\mathcal{H},\\
	\Einhh&\coloneqq\Verts{\partial_h^2\widehat{L}_{in}(\omega, h^\star_\omega)-\partial_h^2\widehat{L}_{in}(\omega, \hat{h}_\omega)}_{\op}\leq C_{in}\Verts{h^\star_\omega-\hat{h}_\omega}_\mathcal{H},\\
	\Einwh&\coloneqq\Verts{\partial_{\omega, h}^2\widehat{L}_{in}(\omega, h^\star_\omega)-\partial_{\omega, h}^2\widehat{L}_{in}(\omega, \hat{h}_\omega)}_{\op} \leq C_{in}\Verts{h^\star_\omega-\hat{h}_\omega}_\mathcal{H}.
\end{align*}
The positive constants $C_{out}$ and $C_{in}$ are defined as:
\begin{align*}
    C_{out}&\coloneqq\max\left(M_{out}\sqrt{\kappa},\bar{M}_{out}\kappa,\bar{M}_{out}\sqrt{\kappa}\right)>0,\\
    C_{in}&\coloneqq\max\left(\bar{M}_{in}\kappa\sqrt{\kappa},\bar{M}_{in}\kappa,M_{in}\sqrt{d\kappa}\right)>0,
\end{align*}
where $M_{out}$, $\bar{M}_{out}$, and $\bar{M}_{in}$ are the positive constants defined in \cref{prop:uniform_boundedness,prop:uniform_Lipschitzness}.
\end{proposition}

\begin{proof}
\textbf{Lipschitz continuity of some functions of interest. }Let $\omega\in\Omega$. According to \cref{prop:bound_hstaromega}, both $h^\star_\omega(x)$ and $\hat{h}_\omega(x)$ lie in the compact interval $\mathcal{V}\coloneqq\left[-\frac{B\kappa}{\lambda},\frac{B\kappa}{\lambda}\right]\subset\mathbb{R}$, for any $x\in\mathcal{X}$, where $B\coloneqq\sup_{\omega\in\text{hull}(\Omega),y\in\mathcal{Y}}\left|\partial_v \ell_{in}(\omega, 0, y)\right|>0$. By \cref{assump:compact}, $\mathcal{Y}$ is a compact set. Hence $\Omega\times\mathcal{V}\times\mathcal{Y}$ is a compact set as well. Furthermore, by \cref{assump:reg_lin_lout}, $(\omega, v, y)\mapsto\ell_{in}(\omega, v, y)$, $(\omega, v, y)\mapsto\ell_{out}(\omega, v, y)$, and their derivatives are all continuous over the compact domain $\Omega\times\mathcal{V}\times\mathcal{Y}$. Therefore, these functions and their derivatives are bounded on this domain. In particular, this also holds when $v$ takes the specific values $h^\star_\omega(x)$ or $\hat{h}_\omega(x)$. Let $\bar{v}$ be either $h^\star_\omega(x)$ or $\hat{h}_\omega(x)$, for any $x\in\mathcal{X}$. For any $\omega\in\Omega$, and $y\in\mathcal{Y}$, we have:
\begin{align*}
    \verts{\partial_v\ell_{out}(\omega, \bar{v}, y)}&\leq\sup_{\omega\in\Omega,v\in\mathcal{V},y\in\mathcal{Y}}\verts{\partial_v\ell_{out}(\omega, v, y)}\leq M_{out}<+\infty,\\
   \verts{\partial_v^2\ell_{out}(\omega, \bar{v}, y)}&\leq\sup_{\omega\in\Omega,v\in\mathcal{V},y\in\mathcal{Y}}\verts{\partial_v^2\ell_{out}(\omega, v, y)}\leq\bar{M}_{out}<+\infty,\\
   \Verts{\partial_{\omega, v}^2\ell_{out}(\omega, \bar{v}, y)}&\leq\sup_{\omega\in\Omega,v\in\mathcal{V},y\in\mathcal{Y}}\Verts{\partial_{\omega, v}^2\ell_{out}(\omega, v, y)}\leq\bar{M}_{out}<+\infty,\\
   \verts{\partial_v^3\ell_{in}(\omega, \bar{v}, y)}&\leq\sup_{\omega\in\Omega,v\in\mathcal{V},y\in\mathcal{Y}}\verts{\partial_v^3\ell_{in}(\omega, v, y)}\leq\bar{M}_{in}<+\infty,\\
   \Verts{\partial_v\partial_{\omega, v}^2\ell_{in}(\omega, \bar{v}, y)}_{\op}&\leq\sup_{\omega\in\Omega,v\in\mathcal{V},y\in\mathcal{Y}}\Verts{\partial_\omega\partial_v^2\ell_{in}(\omega, v, y)}\leq\bar{M}_{in}<+\infty
\end{align*}
This means that $v\in\mathcal{V}\mapsto\ell_{out}(\omega, v, y)$, $v\in\mathcal{V}\mapsto\partial_v\ell_{out}(\omega, v, y)$, $v\in\mathcal{V}\mapsto\partial_\omega\ell_{out}(\omega, v, y)$, $v\in\mathcal{V}\mapsto\partial_v^2\ell_{in}(\omega, v, y)$, and $v\in\mathcal{V}\mapsto\partial_{\omega, v}^2\ell_{in}(\omega, v, y)$ are Lipschitz continuous, with Lipschitz constants $M_{out}$, $\bar{M}_{out}$, $\bar{M}_{out}$, $\bar{M}_{in}$, and $\bar{M}_{in}$, respectively, for any $\omega\in\Omega$ and $y\in\mathcal{Y}$.

\textbf{Upper-bounds. }We have:
\begin{align*}
    \Eout\coloneqq\verts{\widehat{L}_{out}(\omega, h^\star_\omega)-\widehat{L}_{out}(\omega, \hat{h}_\omega)}&=\verts{\frac{1}{m}\sum_{j=1}^m\ell_{out}(\omega, h^\star_\omega(\tilde{x}_j), \tilde{y}_j) - \frac{1}{m}\sum_{j=1}^m\ell_{out}(\omega, \hat{h}_\omega(\tilde{x}_j), \tilde{y}_j)}\\
    &\leq\frac{1}{m}\sum_{j=1}^m\verts{\ell_{out}(\omega, h^\star_\omega(\tilde{x}_j), \tilde{y}_j)-\ell_{out}(\omega, \hat{h}_\omega(\tilde{x}_j), \tilde{y}_j)}\\
    &\leq\frac{M_{out}}{m}\sum_{j=1}^m\verts{h^\star_\omega(\tilde{x}_j)-\hat{h}_\omega(\tilde{x}_j)}\\
    &\leq M_{out}\sqrt{\kappa}\Verts{h^\star_\omega-\hat{h}_\omega}_\mathcal{H},
\end{align*}
where the first line uses the definition of $(\omega, h)\mapsto\widehat{L}_{out}(\omega, h)$, the second line applies the triangle inequality, the third line leverages the fact that $v\mapsto\ell_{out}(\omega, v, y)$ is $M_{out}$-Lipschitz continuous, for any $\omega\in\Omega$ and $y\in\mathcal{Y}$, and the last line follows from the reproducing property of the RKHS $\mathcal{H}$, Cauchy-Schwarz's inequality, and \cref{assump:K_bounded} to bound $\Verts{K(x,\cdot)}_\mathcal{H}$ by $\sqrt{\kappa}$. Similarly, we obtain:
\begin{gather*}
    \partial_h\Eout\leq\bar{M}_{out}\kappa\Verts{h^\star_\omega-\hat{h}_\omega}_\mathcal{H},\quad\partial_\omega\Eout\leq\bar{M}_{out} \sqrt{\kappa}\Verts{h^\star_\omega-\hat{h}_\omega}_\mathcal{H},\quad\partial_h^2\Ein\leq\bar{M}_{in} \kappa\sqrt{\kappa}\Verts{h^\star_\omega-\hat{h}_\omega}_\mathcal{H},\\
    \partial_{\omega, h}^2\Ein\leq\bar{M}_{in} \kappa\Verts{h^\star_\omega-\hat{h}_\omega}_\mathcal{H}.
\end{gather*}
Combining all the bounds finishes the proof.
\end{proof}

\begin{proposition}\label{prop:bounded_derivatives}
Under \cref{assump:compact,assump:convexity_lin,assump:K_bounded,assump:reg_lin_lout}, the following inequalities hold for any $\omega\in\Omega$:
\begin{align*}
	\Verts{\partial_{h}L_{out}(\omega,h_{\omega}^{\star})}_{\mathcal{H}}\leq C_{out}, \quad \Verts{\partial^2_{\omega,h}L_{in}(\omega,h_{\omega}^{\star}) }_{\op}\leq C_{in},\quad 
	\Verts{\partial^2_{\omega,h}\widehat{L}_{in}(\omega,\hat{h}_{\omega}) }_{\op}\leq C_{in},
\end{align*}
where $C_{out}$ and $C_{in}$ are the positive constants defined in \cref{prop:lip_continuity_out}.
\end{proposition}

\begin{proof}Let $\omega\in\Omega$.

\textbf{Upper-bound on $\Verts{\partial_{h}L_{out}(\omega,h_{\omega}^{\star})}_{\mathcal{H}}$. }We have:
\begin{align*}
    \left\|\partial_h L_{out}(\omega, h^\star_\omega)\right\|_\mathcal{H}&=\Big\|\mathbb{E}_\mathbb{Q}\left[\partial_v \ell_{out}(\omega, h^\star_\omega(x), y)K(x,\cdot)\right]\Big\|_\mathcal{H}\\
    &\leq\mathbb{E}_\mathbb{Q}\Big[\left|\partial_v \ell_{out}(\omega, h^\star_\omega(x), y)\right|\left\|K(x,\cdot)\right\|_\mathcal{H}\Big]\\
    &\leq\sqrt{\kappa}\mathbb{E}_\mathbb{Q}\Big[\left|\partial_v \ell_{out}(\omega, h^\star_\omega(x), y)\right|\Big],
\end{align*}
where the first line follows from \cref{prop:fre_diff_L}, the second line results from the triangle inequality, and the last line uses \cref{assump:K_bounded} to bound $\Verts{K(x,\cdot)}_\mathcal{H}$ by $\sqrt{\kappa}$.  Furthermore, we know by \cref{prop:bound_hstaromega} that $(\omega,h_{\omega}^{\star}(x),y)$ belongs to the compact subset $\Omega\times \mathcal{V}\times \mathcal{Y}$ and by \cref{prop:uniform_boundedness} that $\partial_v \ell_{out}(\omega, h^\star_\omega(x), y)$ is bounded by a  constant $M_{out}$ on $\text{hull}(\Omega)\times \mathcal{V}\times \mathcal{Y}$. Hence, it follows that:
\begin{align*}
    \left\|\partial_h L_{out}(\omega, h^\star_\omega)\right\|_\mathcal{H}
    \leq \sqrt{\kappa} M_{out} \le C_{out},
\end{align*}
where $C_{out}$ is defined in \cref{prop:lip_continuity_out}. 

\textbf{Upper-boud on $\quad \Verts{\partial^2_{\omega,h}L_{in}(\omega,h_{\omega}^{\star}) }_{\op}$. }According to \cref{prop:fre_diff_L_v}, $\partial_{\omega, h}^2 L_{in}(\omega, h^\star_\omega)$ is a Hilbert-Schmidt operator, which points to:
\begin{equation}\label{eq:norm_op_partial_omega_h_Lin}
    \left\|\partial_{\omega, h}^2L_{in}(\omega, h^\star_\omega)\right\|_{\op}\leq\left\|\partial_{\omega, h}^2L_{in}(\omega, h^\star_\omega)\right\|_{\hs}=\sqrt{\sum_{l=1}^d\left\|\partial_{\omega_k, h}^2L_{in}(\omega, h^\star_\omega)\right\|_\mathcal{H}^2}.
\end{equation}
This means that to find an upper-bound on $\left\|\partial_{\omega, h}^2L_{in}(\omega, h^\star_\omega)\right\|_{\op}$, it suffices to establish an upper-bound on $\left\|\partial_{\omega_l, h}^2L_{in}(\omega, h^\star_\omega)\right\|_\mathcal{H}^2$ for any $l\in\{1,\ldots,d\}$. For a fixed $l\in\{1,\ldots,d\}$, we have:
\begin{align*}
    \left\|\partial_{\omega_l, h}^2L_{in}(\omega, h^\star_\omega)\right\|_\mathcal{H}^2&=\Big\|\mathbb{E}_\mathbb{P}\left[\partial_{\omega_l,v}^2 \ell_{in}(\omega, h^\star_\omega(x), y)K(x,\cdot)\right]\Big\|_\mathcal{H}^2\\
    &\leq\mathbb{E}_\mathbb{P}\left[\left|\partial_{\omega_l, v}^2 \ell_{in}(\omega, h^\star_\omega(x), y)\right|^2\left\|K(x,\cdot)\right\|_\mathcal{H}^2\right]\\
    &\leq\mathbb{E}_\mathbb{P}\left[\left\|\partial_{\omega, v}^2 \ell_{in}(\omega, h^\star_\omega(x), y)\right\|^2\right]\kappa,
\end{align*}
where the first line follows from \cref{prop:fre_diff_L_v}, the second line is a consequence of Jensen's inequality applied on the convex function $\|\cdot\|^2$, and the last line applies \cref{assump:K_bounded} to bound $\Verts{K(x,\cdot)}_\mathcal{H}^2$ by $\kappa$. Incorporating this upper-bound into \cref{eq:norm_op_partial_omega_h_Lin} yields:
\begin{equation*}
    \left\|\partial_{\omega, h}^2L_{in}(\omega, h^\star_\omega)\right\|_{\op}\leq\sqrt{\mathbb{E}_\mathbb{P}\left[\left\|\partial_{\omega, v}^2 \ell_{in}(\omega, h^\star_\omega(x), y)\right\|^2\right] d\kappa}\leq M_{in}\sqrt{d\kappa}\leq C_{out}, 
\end{equation*}
where we used \cref{prop:uniform_boundedness} to bound $\partial_{\omega, v}^2 \ell_{in}(\omega, h^\star_\omega(x), y)$ by the constant $M_{in}$.

\textbf{Upper-bound on $\Verts{\partial^2_{\omega,h}\widehat{L}_{in}(\omega,\hat{h}_{\omega}) }_{\op}$. }The derivation of this upper bound follows the same steps as the previous one, with the only differences being the use of $\widehat{L}_{in}$ instead of $L_{in}$, and $\hat{h}_\omega$ instead of $h^\star_\omega$.

Note that in the last step of each of the three upper-bounds, we used the fact that the functions we are dealing with are continuous (by \cref{assump:reg_lin_lout}) on $\Omega\times\mathcal{V}\times\mathcal{Y}$, which is compact because $\Omega$ and $\mathcal{Y}$ are compact by \cref{assump:compact} and $\mathcal{V}$ is a compact interval of $\mathbb{R}$ defined in \cref{prop:bound_hstaromega}. Hence, those functions are bounded.
\end{proof}


\begin{proposition}[{Approximation bounds}]\label{prop:grad_app_bound}
    Under \cref{assump:compact,assump:convexity_lin,assump:K_bounded,assump:reg_lin_lout}, the following holds for any $\omega\in\Omega$:
    \begin{gather*}
    \verts{\mathcal{F}(\omega)-\widehat{\mathcal{F}}(\omega)}\leq 
    \Dout
    +\frac{C_{out}}{\lambda}\Dinh,\\
 	\Verts{\nabla\mathcal{F}(\omega)-\widehat{\nabla\mathcal{F}}(\omega)}
 	\leq  
 	\Doutw  + \frac{C_{in}}{\lambda}\Douth + \frac{C_{out}C_{in}}{\lambda^2}\Dinhh + \frac{C_{out}}{\lambda}\Dinwh + \frac{C_{out}}{\lambda}\parens{1 + 2\frac{C_{in}}{\lambda}  + \frac{C_{in}^2}{\lambda^2} }\Dinh,
    \end{gather*}
 where the constants $C_{in}$ and $C_{out}$ are given in {\cref{prop:lip_continuity_out}}.
\end{proposition}
\begin{proof}
In all what follows, we fix a value for $\omega$ in $\Omega$. We start by controlling the value function, then its gradient.

{\bf Control on the value function.}
By the triangle inequality, we have:
\begin{equation}\label{eq:diff_f_int}
    \verts{\mathcal{F}(\omega)-\hat{\mathcal{F}}(\omega)}\leq\underbrace{\verts{L_{out}(\omega, h^\star_\omega)-\widehat{L}_{out}(\omega, h^\star_\omega)}}_{\delta_{\omega}^{out}} +\underbrace{\verts{\widehat{L}_{out}(\omega, h^\star_\omega)-\widehat{L}_{out}(\omega, \hat{h}_\omega)}}_{\Eout},
\end{equation}
According to \cref{prop:lip_continuity_out}, the error term $\Eout$ is controlled by the norm of the difference $h^\star_\omega-\hat{h}_\omega$, \textit{i.e.}, $\Eout\leq C_{out}\Verts{h^\star_\omega-\hat{h}_\omega}_\mathcal{H}$. 
Moreover, by \cref{prop:diff_hstar_hhat}, we know that $\Verts{h^\star_\omega-\hat{h}_\omega}_\mathcal{H}\leq \frac{1}{\lambda}\Dinh$. Therefore, combining both bounds yields: 
	$\Eout\leq \frac{C_{out}}{\lambda}\Dinh$. 
The upper-bound on value function follows by substituting the previous inequality into \cref{eq:diff_f_int}. 

{\bf Control on the gradient.}
By \cref{prop:tot_grad_int}, we have the following expression for the total gradient $\nabla\mathcal{F}$:
\begin{align*}
    \nabla\mathcal{F}(\omega)=\partial_\omega L_{out}(\omega, h^\star_\omega)-\partial_{\omega, h}^2 L_{in}(\omega, h^\star_\omega)\left(\partial_h^2 L_{in}(\omega, h^\star_\omega)\right)^{-1}\partial_h L_{out}(\omega, h^\star_\omega).
\end{align*}
{Similarly, the gradient estimator $\widehat{\nabla\mathcal{F}}$ is defined by replacing $L_{out}$ and $L_{in}$ by their empirical versions $\widehat{L}_{out}$ and $\widehat{L}_{in}$, and $h_{\omega}^{\star}$ by $\hat{h}_{\omega}\coloneqq\arg\min_{h\in \mathcal{H}} \widehat{L}_{in}(\omega,h)$ in the above expression, \textit{i.e.},}
\begin{align*}
    \widehat{\nabla\mathcal{F}}(\omega)=\partial_\omega\widehat{L}_{out}(\omega, \hat{h}_\omega)-\partial_{\omega, h}^2 \widehat{L}_{in}(\omega, \hat{h}_\omega)\left(\partial_h^2 \widehat{L}_{in}(\omega, \hat{h}_\omega)\right)^{-1}\partial_h \widehat{L}_{out}(\omega, \hat{h}_\omega).
\end{align*}
To simplify notations, for any $h\in\mathcal{H}$, we introduce the following operators $R(h), \hat{R}(h):\mathcal{H}\to\Omega$:    \begin{align*}
        R(h)=\partial_{\omega, h}^2L_{in}(\omega, h)\left(\partial_h^2 L_{in}(\omega, h)\right)^{-1}\quad\text{and}\quad 
        \hat{R}(h)=\partial_{\omega, h}^2\widehat{L}_{in}(\omega, h)\left(\partial_h^2 \widehat{L}_{in}(\omega, h)\right)^{-1}.
    \end{align*}
    The difference $\nabla\mathcal{F}(\omega)-\widehat{\nabla\mathcal{F}}(\omega)$ can be decomposed as:
    \begin{align*}
        \nabla\mathcal{F}(\omega)-\widehat{\nabla\mathcal{F}}(\omega)=&\left(\partial_\omega L_{out}(\omega,h^\star_\omega)-\partial_\omega \widehat{L}_{out}(\omega,h^\star_\omega)\right)+\left(\partial_\omega \widehat{L}_{out}(\omega,h^\star_\omega)-\partial_\omega \widehat{L}_{out}(\omega,\hat{h}_\omega)\right)\\
        &-\hat{R}(\hat{h}_\omega)\parens{\left(\partial_h L_{out}(\omega,h^\star_\omega)-\partial_h\widehat{L}_{out}(\omega,h^\star_\omega)\right)+ \left(\partial_h\widehat{L}_{out}(\omega,h^\star_\omega)-\partial_h\widehat{L}_{out}(\omega,\hat{h}_\omega)\right)}\\
                &-\left(R(h^\star_\omega)-\hat{R}(\hat{h}_\omega)\right)\partial_h L_{out}(\omega,h^\star_\omega).
    \end{align*}
By taking the norm of the above equality and using triangle inequality, we obtain the following upper-bound: 
    \begin{align}
    \begin{split}
        \Verts{\nabla\mathcal{F}(\omega)-\widehat{\nabla\mathcal{F}}(\omega)}
        & \leq
        \underbrace{\Verts{\partial_\omega L_{out}(\omega,h^\star_\omega)-\partial_\omega \widehat{L}_{out}(\omega,h^\star_\omega)}}_{\Doutw}+\underbrace{\Verts{\partial_\omega \widehat{L}_{out}(\omega,h^\star_\omega)-\partial_\omega \widehat{L}_{out}(\omega,\hat{h}_\omega)}}_{\Eoutw}\\
        +&\Verts{\hat{R}(\hat{h}_{\omega})}_{\op}\parens{\underbrace{\Verts{\partial_h L_{out}(\omega,h^\star_\omega)-\partial_h\widehat{L}_{out}(\omega,h^\star_\omega)}_{\mathcal{H}}}_{\Douth} + \underbrace{\Verts{\partial_h\widehat{L}_{out}(\omega,h^\star_\omega)-\partial_h\widehat{L}_{out}(\omega,\hat{h}_\omega)}_{\mathcal{H}}}_{\Eouth}}\\
                +& \Verts{R(h^\star_\omega)-\hat{R}(\hat{h}_\omega)}_{\op}\Verts{\partial_h L_{out}(\omega,h^\star_\omega)}_{\mathcal{H}}.
                \end{split}\label{eq:diff_grad_inter}
    \end{align}
    Next, we provide upper-bounds on $\Verts{R(h_{\omega}^{\star})- \hat{R}(\hat{h}_{\omega})}_{\op}$ and $\Verts{\hat{R}(\hat{h}_{\omega})}_{\op}$ in terms of derivatives of $L_{in}$ and $\widehat{L}_{in}$. 

    \textbf{Upper-bounds on $\Verts{R(h_{\omega}^{\star})- \hat{R}(\hat{h}_{\omega})}_{\op}$ and $\Verts{\hat{R}(\hat{h}_{\omega})}_{\op}$. } 
    %
    By application of {\cref{prop:fre_diff_L_v,prop:strong_convexity_Lin}}, we deduce that $\partial_{\omega,h}^2 L_{in}(\omega,h_{\omega}^{\star})$, $\partial_{h}^2 L_{in}(\omega,h_{\omega}^{\star})$, $\partial_{\omega,h}^2 \widehat{L}_{in}(\omega,\hat{h}_{\omega})$ and $\partial_{h}^2 \widehat{L}_{in}(\omega,\hat{h}_{\omega})$ are all bounded operators. Moreover, since $L_{in}$ and $\widehat{L}_{in}$ are $\lambda$-strongly  convex in their second argument by {\cref{prop:strong_convexity_Lin}}, it follows that $\partial_{h}^2 L_{in}(\omega,h_{\omega}^{\star})\geq \lambda\Id_{\mathcal{H}}$ and $\partial_{h}^2 \widehat{L}_{in}(\omega,\hat{h}_{\omega})\geq \lambda \Id_{\mathcal{H}}$. We can therefore apply \cref{lem:tech_res_1} which yields the following inequalities:  
    \begin{align*}
        \Verts{R(h_{\omega}^{\star})- \hat{R}(\hat{h}_{\omega})}_{\op}
        \leq & \frac{1}{\lambda^2}\Verts{\partial_{\omega,h}^2 L_{in}(\omega,h_{\omega}^{\star})}_{\op}\Verts{\partial_{h}^2 L_{in}(\omega,h_{\omega}^{\star}) - \partial_{h}^2 \widehat{L}_{in}(\omega,\hat{h}_{\omega})}_{\op}\\ 
        &+ \frac{1}{\lambda}\Verts{\partial_{\omega,h}^2 L_{in}(\omega,h_{\omega}^{\star}) - \partial_{\omega,h}^2 \widehat{L}_{in}(\omega,\hat{h}_{\omega})}_{\op},\\
        \Verts{\hat{R}(\hat{h}_{\omega})}_{\op}\leq&\frac{1}{\lambda} \Verts{\partial_{\omega,h}^2\widehat{L}_{in}(\omega,\hat{h}_{\omega})}_{\op}. 
    \end{align*}
By applying the triangle inequality to both terms of the first inequality above, we obtain:
\begin{align*}
        \Verts{R(h_{\omega}^{\star})- \hat{R}(\hat{h}_{\omega})}_{\op}
        \leq & \frac{1}{\lambda^2}\Verts{\partial_{\omega,h}^2 L_{in}(\omega,h_{\omega}^{\star})}_{\op}\parens{\underbrace{\Verts{\partial_{h}^2 L_{in}(\omega,h_{\omega}^{\star}) - \partial_{h}^2 \widehat{L}_{in}(\omega,h^\star_{\omega})}_{\op}}_{\Dinhh} + \underbrace{\Verts{\partial_{h}^2 \widehat{L}_{in}(\omega,h_{\omega}^{\star}) - \partial_{h}^2 \widehat{L}_{in}(\omega,\hat{h}_{\omega})}_{\op}}_{\Einhh}}\\ 
        &+ \frac{1}{\lambda}\parens{\underbrace{\Verts{\partial_{\omega,h}^2 L_{in}(\omega,h_{\omega}^{\star}) - \partial_{\omega,h}^2 \widehat{L}_{in}(\omega,h_{\omega}^{\star})}_{\op}}_{\Dinwh} + \underbrace{\Verts{\partial_{\omega,h}^2 \widehat{L}_{in}(\omega,h_{\omega}^{\star}) - \partial_{\omega,h}^2 \widehat{L}_{in}(\omega,\hat{h}_{\omega})}_{\op}}_{\Einwh}}.
    \end{align*}
{\bf Final bound.} We can now substitute the above bounds on $\Verts{R(h_{\omega}^{\star})- \hat{R}(\hat{h}_{\omega})}_{\op}$ and $\Verts{\hat{R}(\hat{h}_{\omega})}_{\op}$ into \cref{eq:diff_grad_inter} to obtain the following upper-bound on the gradient error:
 \begin{equation} \label{eq:diff_grad_inter_2}
\begin{aligned}
 	\Verts{\nabla\mathcal{F}(\omega)-\widehat{\nabla\mathcal{F}}(\omega)}
 	&\leq  
 	\Doutw + \Eoutw + \frac{1}{\lambda}\Verts{\partial_{\omega,h}^2 \widehat{L}_{in}(\omega,\hat{h}_{\omega})}_{\op}\parens{\Douth + \Eouth}\\
 	+& \Verts{\partial_{h}L_{out}(\omega,h_{\omega}^{\star})}_{\mathcal{H}}\parens{ \frac{1}{\lambda^2}\Verts{\partial^2_{\omega,h}L_{in}(\omega,h_{\omega}^{\star}) }_{\op} \parens{\Dinhh + \Einhh} + \frac{1}{\lambda}\parens{\Dinwh + \Einwh} }.
 \end{aligned}
 \end{equation}
Furthermore, by \cref{prop:bounded_derivatives}, we have the following upper-bounds on the derivatives of $L_{in}$ and $L_{out}$:
\begin{align*}
	\Verts{\partial_{h}L_{out}(\omega,h_{\omega}^{\star})}_{\mathcal{H}}\leq C_{out}, \quad \Verts{\partial^2_{\omega,h}L_{in}(\omega,h_{\omega}^{\star}) }_{\op}\leq C_{in},\quad 
	\Verts{\partial^2_{\omega,h}\widehat{L}_{in}(\omega,\hat{h}_{\omega}) }_{\op}\leq C_{in}. 
\end{align*}
Incorporating the above bounds into \cref{eq:diff_grad_inter_2}, we further get:
 \begin{align*}
 	\Verts{\nabla\mathcal{F}(\omega)-\widehat{\nabla\mathcal{F}}(\omega)}
 	\leq & 
 	\Doutw + \Eoutw + \frac{C_{in}}{\lambda}\parens{\Douth + \Eouth}\\
 	&+ C_{out}\parens{ \frac{C_{in}}{\lambda^2}\parens{\Dinhh + \Einhh} + \frac{1}{\lambda}\parens{\Dinwh + \Einwh} }.
 \end{align*}
By \cref{prop:lip_continuity_out}, we can upper-bound the error terms $\Eoutw, \Eouth$ by $C_{out}\Verts{h_{\omega}^{\star}-\hat{h}_{\omega}}_{\mathcal{H}}$ and $ \Einhh,\Einwh$ by the difference $C_{in}\Verts{h_{\omega}^{\star}-\hat{h}_{\omega}}_{\mathcal{H}}$. Furthermore, since  $\Verts{h_{\omega}^{\star}-\hat{h}_{\omega}}_{\mathcal{H}}\leq \frac{1}{\lambda}\Dinh$ by \cref{prop:diff_hstar_hhat}, we can further show that gradient error satisfies the desired bound:
\begin{align*}
 	\Verts{\nabla\mathcal{F}(\omega)-\widehat{\nabla\mathcal{F}}(\omega)}
 	\leq 
 	\Doutw  + \frac{C_{in}}{\lambda}\Douth + C_{out}\parens{ \frac{C_{in}}{\lambda^2}\Dinhh + \frac{1}{\lambda}\Dinwh}+ \frac{C_{out}}{\lambda}\parens{1 + 2\frac{C_{in}}{\lambda}  + \frac{C_{in}^2}{\lambda^2} }\Dinh.
\end{align*}
\end{proof}

\subsection{Maximal inequalities}\label{app_subsec:max_in}
\begin{proposition}[Maximal inequalities for  empirical processes]\label{prop:exp_uni_bound}
Let $\Lambda$ be a positive constant. 
Under \cref{assump:compact,assump:K_bounded,assump:reg_lin_lout,assump:convexity_lin}, the following maximal inequalities hold for any $0<\lambda \leq \Lambda$:
\begin{align*}
	\mathbb{E}_{\mathbb{Q}}\brackets{\sup_{\omega\in \Omega} \Dout }
	 &\leq {\sqrt{\frac{1}{\lambda^2 m}}c(\Omega)\max(M_{out}{\lipout}\diam(\Omega),\Lambda {M_{out}^2})}\\
	 \mathbb{E}_{\mathbb{Q}}\brackets{\sup_{\omega\in \Omega} \Doutw }
	&\leq {\sqrt{\frac{d}{\lambda^2m}}c(\Omega)\max(M_{out}\lipout\diam(\Omega),\Lambda M_{out}^2)}, 
\end{align*}
    where $c(\Omega)$ is a positive constant {greater than 1} that depends only on $\Omega$ and $d$, while {$\lipout$ and $M_{out}$ are positive constants defined in \cref{prop:uniform_boundedness,prop:uniform_Lipschitzness}}. 
\end{proposition}
\begin{proof}
We will apply the result of \cref{prop:empirical_process} which provides maximal inequalities for real-valued empirical processes that are uniformly bounded {and} Lipschitz in their parameter. To this end, consider the parametric families:
\begin{align*}
	\mathcal{T}_{l}^{out} &\coloneqq\braces{\mathcal{X}\times \mathcal{Y}\ni (x,y)\mapsto \partial_{w_l} \ell_{out}(\omega,h_{\omega}^{\star}(x),y)\mid\omega\in \Omega}, \qquad 1\leq l \leq d\\
	\mathcal{T}_{0}^{out} &\coloneqq\braces{\mathcal{X}\times \mathcal{Y}\ni (x,y)\mapsto  \ell_{out}(\omega,h_{\omega}^{\star}(x),y)\mid\omega\in \Omega}.
\end{align*} 
For any $0\leq l\leq d$, these real-valued functions are uniformly bounded by a positive constant $M_{out}$, thanks to {\cref{prop:uniform_boundedness}}. 
Moreover, by {\cref{prop:uniform_Lipschitzness}}, the functions $\omega\mapsto \partial_{\omega_l}\ell_{out}(\omega,h_{\omega}^{\star}(x),y)$, and $\omega\mapsto \ell_{out}(\omega,h_{\omega}^{\star}(x),y)$ are all $\lambda^{-1}\lipout$-Lipschitz for any $(x,y)\in \mathcal{X}\times \mathcal{Y}$. Hence, \cref{prop:empirical_process} is applicable to each of these families, with $\PP$ set to $\mathbb{Q}$ and $\mathcal{Z}$ set to $\mathcal{X}\times \mathcal{Y}$.  We treat both $\Dout$ and $\Doutw$ separately.

{\bf A maximal inequality for $\Dout$.}
For $l=0$, we readily apply \cref{prop:empirical_process} with $p=1$ to get the following maximal inequality for $\Dout$:

\begin{align*}
	\mathbb{E}_{\mathbb{Q}}\brackets{\sup_{\omega\in \Omega} \Dout } \coloneqq &
	\mathbb{E}_{\mathbb{Q}}\brackets{ \sup_{\omega\in \Omega} \verts{ \mathbb{E}_{(x,y)\sim \mathbb{Q}}\brackets{\ell_{out}(\omega,h_{\omega}^{\star}(x),y) } - \frac{1}{m}\sum_{j=1}^m \ell_{out}(\omega,h_{\omega}^{\star}(\tilde{x}_j),\tilde{y}_j) } }\\
	 \leq& \sqrt{\frac{1}{\lambda^2 m}}c(\Omega)\max({\lipout}\diam(\Omega),\Lambda {M_{out}^2}). 
\end{align*}

{\bf A maximal inequality for $\Doutw$.}
{We now turn to  $\Doutw$, which involves vector-valued processes (as an error between the gradient and its estimate).} While the maximal inequalities in \cref{prop:empirical_process} hold for real-valued processes, we will first obtain maximal inequalities for each component appearing in $\Doutw$ and then sum these to control $\Doutw$.
To this end, we first use the Cauchy-Schwarz inequality which implies that $\mathbb{E}_{\mathbb{Q}}\brackets{\sup_{\omega\in \Omega} (\Doutw) }\leq \mathbb{E}_{\mathbb{Q}}\brackets{\sup_{\omega\in \Omega} (\Doutw)^2 }^{\frac{1}{2}}$. Thus we only need to control $\mathbb{E}_{\mathbb{Q}}\brackets{\sup_{\omega\in \Omega} (\Doutw)^2 }$. Simple calculations show that:
\begin{align*}
	\mathbb{E}_{\mathbb{Q}}\brackets{\sup_{\omega\in \Omega} \Doutw }^2&\leq \mathbb{E}_{\mathbb{Q}}\brackets{\sup_{\omega\in \Omega} (\Doutw)^2 }\\
	&\leq  \sum_{l=1}^d \mathbb{E}_{\mathbb{Q}}\brackets{\sup_{\omega\in \Omega}\verts{\mathbb{E}_{(x,y)\sim\mathbb{Q}}\brackets{\partial_{w_l}\ell_{out}(\omega,h_{\omega}^{\star}(x),y)} - \frac{1}{m}\sum_{j=1}^m \partial_{\omega_l}\ell_{out}\parens{\omega,h_{\omega}^{\star}(\tilde{x}_j),\tilde{y}_j}  }^2}\\
	&\leq \parens{\sqrt{\frac{d}{\lambda^2m}}c(\Omega)\max(\lipout\diam(\Omega),\Lambda M_{out}^{{2}})}^2,
\end{align*}
where the last inequality follows by application of \cref{prop:empirical_process} with $p=2$ to each term in the right-hand side of the first inequality for $1\leq l\leq d$. We get the desired bound on  $\mathbb{E}_{\mathbb{Q}}\brackets{\sup_{\omega\in \Omega} \Doutw }$ by taking the square root of the above inequality.
\end{proof}

\begin{proposition}[Maximal inequalities for RKHS-valued empirical processes]\label{prop:exp_uni_bound_2}
Let $\Lambda$ be a positive constant. 
Under \cref{assump:compact,assump:K_bounded,assump:reg_lin_lout,assump:convexity_lin}, the following maximal inequalities hold for any $0<\lambda\leq \Lambda$:
 \begin{align*}
 	\mathbb{E}_{\mathbb{Q}}\brackets{\sup_{\omega\in \Omega} \Douth }&\leq 
 	{\lambda^{-\frac{1}{4}}m^{-\frac{1}{2}}  \parens{c(\Omega)\max\parens{\widetilde{M}_{out,1}\widetilde{L}_{out,1}\diam(\Omega),\Lambda\widetilde{M}_{out,1}^2}}^{\frac{1}{4}}},\\
 	\mathbb{E}_{\mathbb{P}}\brackets{\sup_{\omega\in \Omega} \Dinh }&\leq 
 	{\lambda^{-\frac{1}{4}}n^{-\frac{1}{2}} \parens{c(\Omega)\max\parens{\widetilde{M}_{in,1}\widetilde{L}_{in,1}\diam(\Omega),\Lambda\widetilde{M}_{in,1}^2}}^{\frac{1}{4}}},\\
 	 	\mathbb{E}_{\mathbb{P}}\brackets{\sup_{\omega\in \Omega} \Dinwh }&\leq 
 	{\lambda^{-\frac{1}{4}}n^{-\frac{1}{2}}d^{\frac{1}{2}} \parens{c(\Omega)\max\parens{\widetilde{M}_{in,1}\widetilde{L}_{in,1}\diam(\Omega),\Lambda\widetilde{M}_{in,1}^2}}^{\frac{1}{4}}},\\
 	\mathbb{E}_{\mathbb{P}}\brackets{\sup_{\omega\in \Omega} \Dinhh }&\leq 
 	{\lambda^{-\frac{1}{4}}n^{-\frac{1}{2}} \parens{c(\Omega)\max\parens{\widetilde{M}^2_{in,2}\widetilde{L}_{in,2}\diam(\Omega),\Lambda\widetilde{M}^2_{in,2}}}^{\frac{1}{4}}},
 \end{align*}
 where $c(\Omega)$ is a positive constant {greater than $1$} that depends only on $\Omega$ and $d$, {$\widetilde{L}_{out,1},\widetilde{L}_{in,1},\widetilde{L}_{in,2},\widetilde{M}_{out,1},\widetilde{M}_{in,1},\widetilde{M}_{in,2}$ are positive constants defined as:
 \begin{gather*}
     \widetilde{L}_{out,1}\coloneqq 2\lipout M_{out}\kappa,\quad\widetilde{L}_{in,1}\coloneqq 2\lipin M_{in}\kappa,\quad\widetilde{L}_{in,2}\coloneqq 2\lipin M_{in}\kappa^2,\\
     \widetilde{M}_{out,1}\coloneqq M_{out}^2\kappa,\quad\widetilde{M}_{in,1}\coloneqq M_{in}^2\kappa,\quad\widetilde{M}_{in,2}\coloneqq M_{in}^2\kappa^2,
 \end{gather*}
 and $\lipout,\lipin,M_{out},M_{in}$ are positive constants given in \cref{prop:uniform_boundedness,prop:uniform_Lipschitzness}.}
 
\end{proposition}

\begin{proof}
Consider parametric families of real-valued functions indexed by $\Omega$ of the form:
\begin{align*}
	\mathcal{T}_{s,a}\coloneqq\braces{ t_{\omega}: ((x,y),(x',y'))\mapsto  f_s(\omega,x,y)f_s(\omega,x',y')K^{a}(x,x')\mid\omega\in \Omega },
\end{align*}
where $a\in \{1,2\}$,  $s$ is an integer satisfying {$0\leq s\leq d+2$}, and $f_s(\omega,x,y)$ are real-valued functions given by:
\begin{gather*}
	f_0: (\omega,x,y)\mapsto\partial_v \ell_{out}(\omega, h^\star_\omega(x), y),\qquad 
	f_1:(\omega,x,y)\mapsto\partial_v \ell_{in}(\omega, h^\star_\omega(x), y),\qquad
	f_2:(\omega,x,y)\mapsto\partial_v^2 \ell_{in}(\omega, h^\star_\omega(x), y),\\
	f_{2+l}: (\omega,x,y)\mapsto\partial_{\omega_l, v}^2 \ell_{in}(\omega, h^\star_\omega(x), y), \quad 1\leq l\leq d.
\end{gather*}
For any $1\leq s\leq d+2$, the real-valued functions $f_s$ are uniformly bounded by a positive constant $M_{in}$ thanks to {\cref{prop:uniform_boundedness}}. Moreover, since the kernel $K$ is bounded by {$\kappa$ due to \cref{assump:K_bounded}}, it follows that all elements $t_{\omega}$ of $\mathcal{T}_{s,a}$ are uniformly bounded by {$\widetilde{M}_{in,a}\coloneqq M_{in}^2\kappa^a$}. Moreover, for $1\leq s\leq d+2$, the functions $\omega\mapsto f_{s}(\omega,x,y)$, are $\lambda^{-1}\lipin$-Lipschitz for any $(x,y)\in\mathcal{X}\times\mathcal{Y}$, by {\cref{prop:uniform_Lipschitzness}}. Hence, it follows that the maps $\omega\mapsto t_{\omega}((x,y),(x',y'))$ are $\lambda^{-1}\widetilde{L}_{in,a}$-Lipschitz with {$\widetilde{L}_{in,a}\coloneqq 2\lipin M_{in}\kappa^a$} for any $(x,y)$ and $(x',y')$ in $\mathcal{X}\times \mathcal{Y}$. Similarly, for $s=0$, we get the same properties, albeit, with different constants, \textit{i.e.}, the family {$\mathcal{T}_{0,a}$} is uniformly bounded by a constant {$\widetilde{M}_{out,a}\coloneqq M_{out}^2\kappa^a$} with {$M_{out}$ introduced in \cref{prop:uniform_boundedness}}, and is $\lambda^{-1}\widetilde{L}_{out,a}$-Lipschitz in its parameter with {$\widetilde{L}_{out,a}\coloneqq 2\lipout M_{out}\kappa^a$} where $\lipout$ is given in {\cref{prop:uniform_Lipschitzness}}.
Hence, the maximal inequality in \cref{lem:u_stat_sup_p_omega} is applicable to each of these families with $\mathcal{Z}$ set to $\mathcal{X}\times \mathcal{Y}$, and {$\PP$ set either to $\mathbb{P}$ for $1\leq s\leq d+2$, or to $\mathbb{Q}$ for $s=0$}.
For conciseness, in all what follows, we will write $z = (x,y)$ and $z_i = (x_i,y_i)$ and $\tilde{z}_j = (\tilde{x}_j,\tilde{y}_j)$ for $1\leq i\leq n$ and $1\leq j\leq m$. 

{\bf Maximal inequalities for $\Douth$ and $\Dinh$.}
We control $\Douth$ first as $\Dinh$ will be dealt with similarly. Using Cauchy-Schwarz inequality and standard calculus, we have that:
\begin{align*}
	\mathbb{E}_{\mathbb{Q}}\brackets{\sup_{\omega\in \Omega} \Douth }^2
	&\leq \mathbb{E}_{\mathbb{Q}}\brackets{\sup_{\omega\in \Omega} (\Douth)^2 }\\
	&\coloneqq \mathbb{E}_{\mathbb{Q}}\brackets{ \sup_{\omega\in \Omega} \Verts{ \mathbb{E}_{(x,y)\sim \mathbb{Q}}\brackets{\partial_v\ell_{out}(\omega,h_{\omega}^{\star}(x),y)K(x,\cdot)} - \frac{1}{m}\sum_{j=1}^m \partial_v\ell_{out}(\omega,h_{\omega}^{\star}(\tilde{x}_j),\tilde{y}_j)K(\tilde{x}_j,\cdot) }_{\mathcal{H}}^2 }\\
	&= \mathbb{E}_{\mathbb{Q}}\brackets{ \sup_{\omega\in \Omega} \mathbb{E}_{z,z'\sim\mathbb{Q}\otimes\mathbb{Q}}\left[t_\omega(z,z')\right]+\frac{1}{m^2}\sum_{i,j=1}^m t_\omega(z_i,z_j)-\frac{2}{m}\sum_{j=1}^m\mathbb{E}_{z\sim\mathbb{Q}}\left[t_\omega(z,\tilde{z}_j)\right]},
\end{align*}
 where $t_\omega(z,z') \coloneqq \partial_v\ell_{out}(\omega,h_{\omega}^{\star}(x),y)\partial_v\ell_{out}(\omega,h_{\omega}^{\star}(x'),y')K(x,x')\in \mathcal{T}_{0,1}$. The last term is precisely what \cref{lem:u_stat_sup_p_omega} controls when applying it to the family $\mathcal{T}_{0,1}$ and choosing $\PP$ to be $\mathbb{Q}$. Therefore, the following maximal inequality holds by application of \cref{lem:u_stat_sup_p_omega}:
 \begin{align*}
 	\mathbb{E}_{\mathbb{Q}}\brackets{\sup_{\omega\in \Omega} \Douth }\leq 
 	\lambda^{-\frac{1}{4}}m^{-\frac{1}{2}}  \parens{c(\Omega)\max\parens{\widetilde{M}_{out,1}\widetilde{L}_{out,1}\diam(\Omega),\Lambda\widetilde{M}_{out,1}^2}}^{\frac{1}{4}},
 \end{align*}
 where $c(\Omega)$ is a positive constant {greater than $1$} the depends only on $\Omega$ and $d$. We obtain a similar inequality for $\Dinh$ by carrying out similar calculations, then applying \cref{lem:u_stat_sup_p_omega} to the family $\mathcal{T}_{1,1}$ and choosing $\mathbb{P}$ for the probability distribution $\PP$. The resulting bound is then of the form:
\begin{align*}
 	\mathbb{E}_{\mathbb{P}}\brackets{\sup_{\omega\in \Omega} \Dinh }\leq 
 	\lambda^{-\frac{1}{4}}n^{-\frac{1}{2}} \parens{c(\Omega)\max\parens{\widetilde{M}_{in,1}\widetilde{L}_{in,1}\diam(\Omega),\Lambda\widetilde{M}_{in,1}^2}}^{\frac{1}{4}}.
 \end{align*}

{\bf A maximal inequality for $\Dinwh$.} We have:
\begin{align*}
	\mathbb{E}_{\mathbb{P}}\brackets{\sup_{\omega\in \Omega} \Dinwh }^2
	&\stackrel{(a)}{\leq} \mathbb{E}_{\mathbb{P}}\brackets{\sup_{\omega\in \Omega} (\Dinwh)^2 }\\
	&\stackrel{(b)}{\coloneqq} \mathbb{E}_{\mathbb{P}}\brackets{ \sup_{\omega\in \Omega} \Verts{ \mathbb{E}_{(x,y)\sim \mathbb{P}}\brackets{\partial_{\omega,v}^2\ell_{in}(\omega,h_{\omega}^{\star}(x),y)K(x,\cdot)} - \frac{1}{n}\sum_{i=1}^n \partial_{\omega,v}^2\ell_{in}(\omega,h_{\omega}^{\star}(x_i),y_i)K(x_i,\cdot) }_{\op}^2 }\\
	&\stackrel{(c)}{\leq}\mathbb{E}_{\mathbb{P}}\brackets{ \sup_{\omega\in \Omega} \Verts{ \mathbb{E}_{(x,y)\sim \mathbb{P}}\brackets{\partial_{\omega,v}^2\ell_{in}(\omega,h_{\omega}^{\star}(x),y)K(x,\cdot)} - \frac{1}{n}\sum_{i=1}^n \partial_{\omega,v}^2\ell_{in}(\omega,h_{\omega}^{\star}(x_i),y_i)K(x_i,\cdot) }_{\hs}^2 }\\
	&\stackrel{(d)}{=}\sum_{l=1}^d \mathbb{E}_{\mathbb{P}}\brackets{ \sup_{\omega\in \Omega} \Verts{ \mathbb{E}_{(x,y)\sim \mathbb{P}}\brackets{\partial_{\omega_l,v}^2\ell_{in}(\omega,h_{\omega}^{\star}(x),y)K(x,\cdot)} - \frac{1}{n}\sum_{i=1}^n \partial_{\omega_l,v}^2\ell_{in}(\omega,h_{\omega}^{\star}(x_i),y_i)K(x_i,\cdot) }_{\mathcal{H}}^2 }\\
	&\stackrel{(e)}{=} \sum_{l=1}^d \mathbb{E}_{\mathbb{P}}\brackets{ \sup_{\omega\in \Omega} \mathbb{E}_{z,z'\sim\mathbb{P}\otimes\mathbb{P}}\left[t_{\omega,l}(z,z')\right]+\frac{1}{n^2}\sum_{i,j=1}^n t_{\omega,l}(z_i,z_j)-\frac{2}{n}\sum_{i=1}^n\mathbb{E}_{z\sim\mathbb{P}}\left[t_{\omega,l}(z,z_i)\right]},
\end{align*}
where we introduced $t_{\omega,l}(z,z') \coloneqq \partial_{\omega_l,v}^2\ell_{in}(\omega,h_{\omega}^{\star}(x),y)\partial_{\omega_l,v}^2\ell_{in}(\omega,h_{\omega}^{\star}(x'),y')K(x,x')\in \mathcal{T}_{2+l,1}$. Here, (a) follows by Cauchy-Schwarz inequality, (b) is obtained by definition of $\Dinwh$, while (c) uses the general fact that the operator norm of an operator is upper-bounded by its Hilbert-Schmidt norm which is finite in our case by application of \cref{prop:fre_diff_L_v}. Moreover, (d) further uses the Hilbert-Schmidt norm of an operator  in terms of the norm of its rows, while (e) simply expands the squared RKHS norm and uses the reproducing property in the RKHS $\mathcal{H}$. Each term in the last item (e) is precisely what \cref{lem:u_stat_sup_p_omega} controls when applying it to the families $\mathcal{T}_{2+l,1}$ for $1\leq l\leq d$ and choosing $\PP$ to be $\mathbb{P}$.  Therefore, the following maximal inequality holds by a direct application of \cref{lem:u_stat_sup_p_omega}:
 \begin{align*}
 	\mathbb{E}_{\mathbb{P}}\brackets{\sup_{\omega\in \Omega} \Dinwh }\leq 
 	\lambda^{-\frac{1}{4}}n^{-\frac{1}{2}}d^{\frac{1}{2}} \parens{c(\Omega)\max\parens{\widetilde{M}_{in,1}\widetilde{L}_{in,1}\diam(\Omega),\Lambda\widetilde{M}_{in,1}^2}}^{\frac{1}{4}},
 \end{align*}
 where $c(\Omega)$ is a positive constant greater than $1$ that depends only on $\Omega$ and $d$.

{\bf A maximal inequality for $\Dinhh$.} We will use a similar approach as for $\Dinwh$. We have:
\begin{align*}
	&\mathbb{E}_{\mathbb{P}}\brackets{\sup_{\omega\in \Omega} \Dinhh }^2\\
	&\stackrel{(a)}{\leq} \mathbb{E}_{\mathbb{P}}\brackets{\sup_{\omega\in \Omega} (\Dinhh)^2 }\\
	&\stackrel{(b)}{\coloneqq} \mathbb{E}_{\mathbb{P}}\brackets{ \sup_{\omega\in \Omega} \Verts{ \mathbb{E}_{(x,y)\sim \mathbb{P}}\brackets{\partial_{v}^2\ell_{in}(\omega,h_{\omega}^{\star}(x),y)K(x,\cdot)\otimes K(x,\cdot)} - \frac{1}{n}\sum_{i=1}^n \partial_{v}^2\ell_{in}(\omega,h_{\omega}^{\star}(x_i),y_i)K(x_i,\cdot)\otimes K(x_i,\cdot) }_{\op}^2 }\\
	&\stackrel{(c)}{\leq}\mathbb{E}_{\mathbb{P}}\brackets{ \sup_{\omega\in \Omega} \Verts{ \mathbb{E}_{(x,y)\sim \mathbb{P}}\brackets{\partial_{v}^2\ell_{in}(\omega,h_{\omega}^{\star}(x),y)K(x,\cdot)\otimes K(x,\cdot)} - \frac{1}{n}\sum_{i=1}^n \partial_{v}^2\ell_{in}(\omega,h_{\omega}^{\star}(x_i),y_i)K(x_i,\cdot)\otimes K(x_i,\cdot) }_{\hs}^2 }\\
	&\stackrel{(d)}{=} \mathbb{E}_{\mathbb{P}}\brackets{\sup_{\omega\in \Omega} \mathbb{E}_{z,z'\sim\mathbb{P}\otimes\mathbb{P}}\left[t_\omega(z,z')\right]+\frac{1}{n^2}\sum_{i,j=1}^n t_\omega(z_i,z_j)-\frac{2}{n}\sum_{i=1}^n\mathbb{E}_{z\sim\mathbb{P}}\left[t_\omega(z,z_i)\right]},
\end{align*}
where we introduced $t_{\omega}(z,z')\coloneqq\partial_{v}^2\ell_{in}(\omega,x,y)\partial_{v}^2\ell_{in}(\omega,x',y')K^2(x,x')\in \mathcal{T}_{2,2}$. Here, (a) follows by Cauchy-Schwarz inequality, (b) is obtained by definition of $\Dinhh$, while (c) uses the general fact that the operator norm of an operator is upper-bounded by its Hilbert-Schmidt norm which is finite in our case by application of \cref{prop:fre_diff_L_v}. Moreover, (d) further uses the identity in \cref{lem:hs_identity} for computing the Hilbert-Schmidt norm of sum/expectation of tensor-product operators. 
The last item (d) is precisely what \cref{lem:u_stat_sup_p_omega} controls when applying it to the family $\mathcal{T}_{2,2}$  and choosing $\PP$ to be $\mathbb{P}$.  Therefore, the following maximal inequality holds by direct application of \cref{lem:u_stat_sup_p_omega}:
 \begin{align*}
 	\mathbb{E}_{\mathbb{P}}\brackets{\sup_{\omega\in \Omega} \Dinhh }\leq 
 	\lambda^{-\frac{1}{4}}n^{-\frac{1}{2}} \parens{c(\Omega)\max\parens{\widetilde{M}_{in,2}\widetilde{L}_{in,2}\diam(\Omega),\Lambda\widetilde{M}^2_{in,2}}}^{\frac{1}{4}},
 \end{align*}
 where $c(\Omega)$ is a positive constant greater than $1$ that depends only on $\Omega$ and $d$.
\end{proof}
\subsection{Proof of \cref{th:generalizationBounds}}\label{app_sub:main_proof}
\begin{theorem}[Generalization bounds]\label{th:gen_bound}
The following holds under \cref{assump:compact,assump:convexity_lin,assump:K_bounded,assump:reg_lin_lout}:
\begin{align*}
    \mathbb{E}\brackets{\sup_{\omega\in\Omega}\verts{\mathcal{F}(\omega)-\hat{\mathcal{F}}(\omega)}}
    &\lesssim
    \frac{1}{\lambda m^{\frac{1}{2}}}+\frac{C_{out}}{\lambda^{\frac{5}{4}} n^{\frac{1}{2}}}\\
    \mathbb{E}\left[\sup_{\omega\in\Omega}\left\|\nabla\mathcal{F}(\omega)-\widehat{\nabla\mathcal{F}}(\omega)\right\|\right]
    &\lesssim
    \frac{1}{\lambda}\left(d^{\frac{1}{2}} + \frac{C_{in}}{\lambda^{\frac{1}{4}}}\right)\frac{1}{m^{\frac{1}{2}}} + \frac{C_{out}}{\lambda^{\frac{5}{4}}}\parens{2 + 3\frac{C_{in}}{\lambda}+\frac{C_{in}^2}{\lambda^2}}\frac{1}{n^{\frac{1}{2}}},
\end{align*}
where the constants $C_{in}$ and $C_{out}$ are given in \cref{prop:lip_continuity_out}.
\end{theorem}

\begin{proof}
Using the point-wise estimates in \cref{prop:grad_app_bound} and taking their supremum over $\Omega$ followed by the expectations over data, the following error bounds hold:
\begin{align*}
   \mathbb{E}\brackets{\sup_{\omega\in\Omega} \verts{\mathcal{F}(\omega)-\hat{\mathcal{F}}(\omega)}}
   \leq & 
    \mathbb{E}_{\mathbb{Q}}\brackets{\sup_{\omega\in\Omega}\Dout}
    +\frac{C_{out}}{\lambda}\mathbb{E}_{\mathbb{P}}\brackets{\sup_{\omega\in\Omega}\Dinh},\\
    \mathbb{E}\brackets{\sup_{\omega\in\Omega}\Verts{\nabla\mathcal{F}(\omega)-\widehat{\nabla\mathcal{F}}(\omega)}}
 	\leq &  
    \mathbb{E}_{\mathbb{Q}}\brackets{\sup_{\omega\in\Omega}\Doutw}  + \frac{C_{in}}{\lambda}\mathbb{E}_{\mathbb{Q}}\brackets{\sup_{\omega\in\Omega}\Douth}\\ 
    &+\frac{C_{out}}{\lambda}\parens{1 + 2\frac{C_{in}}{\lambda}+\frac{C_{in}^2}{\lambda^2}} \mathbb{E}_{\mathbb{P}}\brackets{\sup_{\omega\in\Omega}\Dinh}\\
    &+ \frac{C_{out}C_{in}}{\lambda^2}\mathbb{E}_{\mathbb{P}}\brackets{\sup_{\omega\in\Omega}\Dinhh} + \frac{C_{out}}{\lambda}\mathbb{E}_{\mathbb{P}}\brackets{\sup_{\omega\in\Omega}\Dinwh}.
\end{align*}
Furthermore, we can use the maximal inequalities in  \cref{prop:exp_uni_bound,prop:exp_uni_bound_2} to control each term appearing in the right-hand side of the above inequalities:
\begin{align*}
    \mathbb{E}\brackets{\sup_{\omega\in\Omega} \verts{\mathcal{F}(\omega)-\hat{\mathcal{F}}(\omega)}}
   \leq & 
   R\parens{ m^{-\frac{1}{2}}\lambda^{-1}  + C_{out}n^{-\frac{1}{2}}\lambda^{-(1+\frac{1}{4})}}\\
    \mathbb{E}\brackets{\sup_{\omega\in\Omega}\Verts{\nabla\mathcal{F}(\omega)-\widehat{\mathcal{F}}(\omega)}}
 	\leq &  R\Big( m^{-\frac{1}{2}}\lambda^{-1}d^{\frac{1}{2}} + C_{in}m^{-\frac{1}{2}}\lambda^{-(1+\frac{1}{4})} + C_{out}n^{-\frac{1}{2}}\lambda^{-(1+\frac{1}{4})}\parens{1 + 2\frac{C_{in}}{\lambda}+\frac{C_{in}^2}{\lambda^2}}\\
 	&\qquad+ C_{out}C_{in}n^{-\frac{1}{2}}\lambda^{-(2+\frac{1}{4})}  +  C_{out}n^{-\frac{1}{2}}\lambda^{-(1+\frac{1}{4})}\Big),
\end{align*}
where the constant $R$ depends only on the Lipschitz constants $\lipin$, $\lipout$, the upper-bounds $M_{in}$, $M_{out}$, the bound $\kappa$ on the kernel, the set $\Omega$ and the dimension $d$. Rearranging the obtained upper-bounds concludes the proof.
\end{proof}
\begin{figure*}[t]
\centering
\begin{center} 
  \includegraphics[width=.63\textwidth]{figures/fidelity_legend.pdf}
\end{center}
\begin{subfigure}[b]{.3\textwidth}
  \centering
  \includegraphics[width=\linewidth]{figures/sentiment_fidelity3_small.pdf}
  \caption{\emph{Sentiment} $n\in [32,63]$}
  \label{fig:fidelity_sentiment}
\end{subfigure}%
\begin{subfigure}[b]{.3\textwidth}
  \centering
  \includegraphics[width=\linewidth]{figures/drop_fidelity_small.pdf}
  \caption{\emph{DROP} $n\in [32,63]$}
  \label{fig:fidelity_drop}
\end{subfigure}%
\begin{subfigure}[b]{.3\textwidth}
  \centering
  \includegraphics[width=\linewidth]{figures/hotpot_fidelity_small.pdf}
  \caption{\emph{HotpotQA} $n\in [32,63]$}
  \label{fig:fidelity_hotpot}
\end{subfigure}
\begin{subfigure}[b]{.3\textwidth}
  \centering
  \includegraphics[width=\linewidth]{figures/sentiment_fidelity3_large.pdf}
  \caption{\emph{Sentiment} $n\in [64,127]$}
  \label{fig:fidelity_sentiment2}
\end{subfigure}%
\begin{subfigure}[b]{.3\textwidth}
  \centering
  \includegraphics[width=\linewidth]{figures/drop_fidelity_large.pdf}
  \caption{\emph{DROP} $n\in [64,127]$}
  \label{fig:fidelity_drop2}
\end{subfigure}%
\begin{subfigure}[b]{.3\textwidth}
  \centering
  \includegraphics[width=\linewidth]{figures/hotpot_fidelity_large.pdf}
  \caption{\emph{HotpotQA} $n\in [64,127]$}
  \label{fig:fidelity_hotpot2}
\end{subfigure}
\vspace{-5pt}
\caption{On the removal task, \SpecExp{} performs competitively with 2\textsuperscript{nd} order methods on the \emph{Sentiment} dataset, and out-performs all approaches on \emph{DROP} and  \emph{HotpotQA} dataset for $n \in [32,63]$. When $n$ is too large to compute other interaction indices, we outperform marginal methods.}
\label{fig:fidelity}
\vspace{-12pt}
\end{figure*}

\vspace{-14pt}
\section{Experiments}
\label{sec:language}

\paragraph{Datasets} 
We use three popular datasets that require the LLM to understand interactions between features. 
\begin{enumerate}[ topsep=0pt, itemsep=0pt, leftmargin=*]
\item \emph{Sentiment} is primarily composed of the \emph{Large Movie Review Dataset} \cite{maas-EtAl:2011:ACL-HLT2011}, which contains both positive and negative IMDb movie reviews. The dataset is augmented with examples from the \emph{SST} dataset \cite{ socher2013recursive} to ensure coverage for small $n$. We treat the words of the reviews as the input features.
\item{\emph{HotpotQA} \cite{yang2018hotpotqa} is a question-answering dataset requiring multi-hop reasoning over multiple Wikipedia articles to answer complex questions. We use the sentences of the articles as the input features.}
\item{\emph{Discrete Reasoning Over Paragraphs} (DROP)} \cite{dua2019drop} is a comprehension benchmark requiring discrete reasoning operations like addition, counting, and sorting over paragraph-level content to answer questions. We use the words of the paragraphs as the input features. 
\end{enumerate}
%
%\emph{DROP} and \emph{HotpotQA} require , while \emph{Sentiment} is encoder-only. 
%
\vspace{-7pt}
\paragraph{Models} For \textit{DROP} and \textit{HotpotQA}, (generative question-answering tasks) we use \texttt{Llama-3.2-3B-Instruct} \cite{grattafiori2024llama3herdmodels} with $8$-bit quantization. For \emph{Sentiment} (classification), we use the encoder-only fine-tuned \texttt{DistilBERT} model \cite{Sanh2019DistilBERTAD,sentimentBert}.
%

\vspace{-7pt}
\paragraph{Baselines} We compare against popular marginal metrics LIME, SHAP, and the Banzhaf value. 
%
For interaction indices, we consider Faith-Shapley, Faith-Banzhaf, and the Shapley-Taylor Index. We compute all benchmarks where computationally feasible. That is, we always compute marginal attributions and interaction indices when $n$ is sufficiently small. In figures, we only show the best performing baselines. Results and implementation details for all baselines can be found in 
Appendix~\ref{apdx:experiments}.

\vspace{-6pt}
\paragraph{Hyperparameters} \SpecExp{} has several parameters to determine the number of model inferences (masks). We choose $C=3$, informed by \citet{li2015spright} under a simplified sparse Fourier setting. We fix $t = 5$, which is the error correction capability of \SpecExp{} and serves as an approximate bound on the maximum degree. 
%
We also set $b=8$; the total collected samples are $\approx C2^bt \log(n)$. 
%
For $\ell_1$ regression-based interaction indices, we choose the regularization parameter via $5$-fold cross-validation. 




\vspace{-3pt}
\subsection{Metrics}


We compare \SpecExp{} to other methods across a variety of well-established metrics to assess performance.
%\Efe{How about textbf rather than emph here?}

\textbf{Faithfulness}: To characterize how well the surrogate function $\hat{f}$ approximates the true function, we define \emph{faithfulness} \cite{zhang2023trade}:
\vspace{-3pt}
\begin{equation}
    R^2 = 1 -  \frac{\lVert \hat{f} - f \rVert^2}{\left\lVert f - \bar{f} \right\rVert^2},
\end{equation}
where $\left\lVert f  \right\rVert^2 = \sum_{\bbm \in \bbF_2^n}f(\bbm)^2$ and $\bar{f} = \frac{1}{2^n} \sum_{\bbm \in \bbF_2^n}f(\bbm)$.

Faithfulness measures the ability of different explanation methods to predict model output when masking random inputs. 
%
We measure faithfulness over 10,000 random \emph{test} masks per-sample, and report average $R^2$ across samples. 
%

\textbf{Top-$r$ Removal}: We measure the ability of methods to identify the top $r$ influential features to model output:
\vspace{-2pt}
\begin{align}
\begin{split}
    \mathrm{Rem}(r) = \frac{|f(\boldsymbol{1}) - f(\bbm^*)|}{|f(\boldsymbol{1})|}, \\
    \;\bbm^* = \argmax \limits_{\abs{\bbm} = n-r}|\hat{f}(\boldsymbol{1}) - \hat{f}(\bbm)|.
\end{split}
\end{align}
\vspace{-8pt}


\textbf{Recovery Rate@$r$:} 
%
Each question in \emph{HotpotQA} contains human-labeled annotations for the sentences required to correctly answer the question. 
%
We measure the ability of interaction indices to recover these human-labeled annotations. 
%
Let $S_{r^*} \subseteq [n]$ denote human-annotated sentence indices. %corresponding to the human-annotated sentences containing the answer. 
Let $S_{i}$ denote feature indices of the $i^{\text{th}}$ most important interaction for a given interaction index.
%
Define the recovery ability at $r$ for each method as follows
\vspace{-2pt}
\begin{equation}
\label{eq:recovery_k}
    \text{Recovery@}r = 
    \frac{1}{r}\sum^r_{i=1}\frac{\abs{S_r^*\cap S_i}}{|S_{i}|}.
\end{equation}
\vspace{-8pt}

Intuitively, \eqref{eq:recovery_k} measures how well interaction indices capture features that align with human-labels.   


\begin{figure*}[t]
\centering
\hfill
\begin{subfigure}[b]{.5\textwidth}
  \centering
    \hspace{0.82cm}\includegraphics[width=0.75\textwidth]{figures/recall_legend.pdf}
  \includegraphics[width=.9\linewidth]{figures/hotpot_recall.pdf}
  \caption{Recovery rate$@10$ for \emph{HotpotQA} }
  \label{fig:recovery_hotpot}
\end{subfigure}%
\hfill % To ensure space between the figures
\begin{subfigure}[b]{.46\textwidth}
  \centering
    \includegraphics[width=1\textwidth]{figures/hotpot.pdf}
  \caption{Human-labeled interaction identified by \SpecExp{}.}
  \label{fig:hotpot_additional}
\end{subfigure}
\hfill
\caption{(a) \SpecExp{} recovers more human-labeled features with significantly fewer training masks as compared to other methods. (b) For a long-context example ($n = 128$ sentences), \SpecExp{} identifies the three human-labeled sentences as the most important third order interaction while ignoring unimportant contextual information.}
\vspace{-8pt}
\end{figure*}

\vspace{-8pt}
\subsection{Faithfulness and Runtime}
\vspace{-3pt}

Fig.~\ref{fig:faith} shows the faithfulness of \SpecExp{} compared to other methods. We also plot the runtime of all approaches for the \emph{Sentiment} dataset for different values of $n$. 
%
All attribution methods are learned over a fixed number of training masks.
% 

\textbf{Comparison to Interaction Indices } \SpecExp{} maintains competitive performance with the best-performing interaction indices across datasets. 
%
Recall these indices enumerate \emph{all possible interactions}, whereas \SpecExp{} does not. 
%
This difference is reflected in the runtimes of Fig.~\ref{fig:faith}(a).
%
The runtime of other interaction indices explodes as $n$ increases while \SpecExp{} does not suffer any increase in runtime. 

\vspace{-2pt}
\textbf{Comparison to Marginal Attributions } For input lengths $n$ too large to run interaction indices, \SpecExp{} is significantly more faithful than marginal attribution approaches across all three datasets.

\vspace{-2pt}
\textbf{Varying number of training masks } Results in Appendix ~\ref{apdx:experiments} show that \SpecExp{} continues to out-perform other approaches as we vary the number of training masks. 

\vspace{-2pt}
\textbf{Sparsity of \SpecExp{} Surrogate Function} Results in Appendix ~\ref{apdx:experiments}, Table~\ref{tab:faith} show 
surrogate functions learned by \SpecExp{} have Fourier representations where only $\sim 10^{-100}$ percent of coefficients are non-zero. 


\vspace{-6pt}
\subsection{Removal}
\label{subsec:removal}

Fig.~\ref{fig:fidelity} plots the change in model output as we mask the top $r$ features for different regimes of $n$. 
%

\vspace{-2pt}
\textbf{Small $n$ } \SpecExp{} is competitive with other interaction indices for \textit{Sentiment}, and out-performs them for \textit{HotpotQA} and \textit{DROP}. 
%
Performance of \SpecExp{} in this task is particularly notable since Shapley-based methods are designed to identify a small set of influential features. 
%
On the other hand, \SpecExp{} does not optimize for this metric, but instead learns the function $f(\cdot)$ over all possible $2^n$ masks. 
%

\textbf{Large $n$ } \SpecExp{} out-performs all marginal approaches, indicating the utility of considering interactions.
%

\vspace{-10pt}
\subsection{Recovery Rate of Human-Labeled Interactions}

%
We compare the recovery rate \eqref{eq:recovery_k} for $r = 10$ of \SpecExp{} against third order Faith-Banzhaf and Faith-Shap interaction indices. 
%
We choose third order interaction indices because all examples 
are answerable with information from at most three sentences, i.e., maximum degree $d = 3$.
%
Recovery rate is measured as we vary the number of training masks. 

Results are shown in Fig.~\ref{fig:recovery_hotpot}, where \SpecExp{} has the highest recovery rate of all interaction indices across all sample sizes. 
%
Further, \SpecExp{} achieves close to its maximum performance with few samples, other approaches require many more samples to approach the recovery rate of \SpecExp{}. 

\textbf{Example of Learned Interaction by \SpecExp{}} Fig.~\ref{fig:hotpot_additional} displays a long-context example (128 sentences) from \emph{HotpotQA} whose answer is contained in the three highlighted sentences. 
%
\SpecExp{} identifies the three human-labeled sentences as the most important third order interaction while ignoring unimportant contextual information. 
%
Other third order methods are not computable at this length. 
%

\begin{figure*}[t]
    \centering
    \includegraphics[width=0.9\linewidth]{figures/case_studies.pdf}
    \caption{SHAP provides marginal feature attributions. Feature interaction attributions computed by SPEX provide a more comprehensive understanding of (above) words interactions that cause the model to answer incorrectly and (below) interactions between image patches that informed the model's output.}
    \label{fig:caseStudies}
\end{figure*}


\section{Experiments: Planning outperforms Heuristics}
\label{sec:experiment}

We begin our empirical demonstrations by showcasing the effectiveness of our planning framework on both synthetic and real datasets. We focus on the simplest planning algorithm, 1-step lookaheads (Algorithm~\ref{alg:complete}), and show that even basic planning can hold great promise. 
We illustrate our framework using two uncertainty quantification modules---GPs and 
\ensembles/ \ensembleplus. 

Throughout this section, we focus on evaluating the mean squared error of 
a regression model $\model$,  and develop adaptive policies that minimize uncertainty on $g(f)$ defined in~\eqref{eqn:l2-g-f}.
When GPs provide a valid model of uncertainty, 
our experiments show that our planning framework significantly outperforms other baselines. 
We further demonstrate that our conceptual framework extends to deep learning-based uncertainty quantification methods such as  \ensembleplus while highlighting computational challenges that need to be resolved in order to scale our ideas. 
For simplicity, we assume a naive predictor, i.e., $\psi(\cdot) \equiv 0$. However, we emphasize that this problem is just as complex as if we were using a sophisticated model $\psi(.)$. The performance gap between the algorithms 
primarily depends
on the level  of uncertainty in our prior beliefs.

To evaluate the performance of our algorithm, we benchmark it against several baselines. 
%Active learning baselines use an acquisition function $\ac$ to select points that have the highest   function value: $X\opt_t \in \argmax_{X \in \xpoolj{t}} \ac({X})$ at every step $t$. These methods may also need an UQ module, which we simply use the same UQ module as in our algorithm, and it  outputs $V(X)$ that measures the the uncertainty of each point $X \in \xpoolj{t}$.
Our first set of baselines are from active learning~\citep{AggarwalKoGuHaPh14}:
\\ % \noindent\textbf{Active Learning Heuristics:} 
\textbf{(1)} 
\textsf{Uncertainty Sampling (Static):}  In this approach, we query the samples for which the model is least certain about. Specifically, we estimate the variance of the latent output $f(X)$ for each $X \in \xpool$ using the UQ module and select the top-$K$ points with the highest uncertainty. \\
\textbf{(2)} \textsf{Uncertainty Sampling (Sequential):} This is a greedy heuristic that sequentially selects the points with the highest uncertainty within a batch, while updating the posterior beliefs using pseudo labels from the current posterior state. Unlike \textsf{Uncertainty Sampling (Static)}, this method takes into account the information gained from each point within batch, and hence tries to diversify the selected points within a batch. 

 
We also compare our approach to the  \textbf{(3)} \textsf{Random Sampling}, which selects each batch uniformly at random from the pool. Additionally, we compare solving the planning problem using  \textsf{REINFORCE}-based policy gradients with   $\mathsf{Smoothed\text{-}Autodiff}$ policy gradients.\footnote{Our code repository is available at
  \url{https://github.com/namkoong-lab/adaptive-labeling}.}
%Detailed experimental setups are provided in Section \ref{sec:details-experiments}.

%We repeat all experiments with 10 random seeds.




\begin{figure}[t]
\centering
\begin{minipage}[b]{0.49\textwidth}
\centering
\includegraphics[width=\textwidth, height=5cm]{figures/original_scale/Var_of_l_2_loss.pdf}
\caption{(Synthetic data) Variance of mean squared loss evaluated through the posterior belief $\mu_t$ at each horizon $t$. This is the objective that policy gradient methods like \textsf{REINFORCE} and $\ouralgo$ optimizes. 1-step lookaheads are surprisingly effective even in long horizons.}
\label{fig:var-l2-sim}
\end{minipage}
\hfill
\begin{minipage}[b]{0.49\textwidth}
\centering \includegraphics[width=\textwidth, height=5cm]{figures/original_scale/Error_of_estimated_model_l_2_loss.pdf}
\caption{(Synthetic data) Error between MSE calculated based on collected data $\mc{D}^{0:T}$ vs. population oracle MSE over $\mc{D}_{\rm eval} \sim P_X$. Reducing uncertainty over posteriors directly leads to better OOD evaluations. 1-step lookaheads significantly outperform active learning heuristics in small horizons.}
\label{fig:mean-l2-sim}
\end{minipage}
%\caption{Simulated data for GPs}
%\label{fig:both_plots}
\end{figure}

\subsection{Planning with Gaussian processes}
\label{sec:experiment-plan-GP}
We now briefly describe the data generation process for the GP experiments,  deferring a more detailed discussion of the dataset generation to Section~\ref{sec:details-experiments}. 
We use both the synthetic data and the real data to test our methodology.
For the \emph{simulated data},  we construct a setting where the general population is distributed across \emph{51 non-overlapping clusters} while the initial labeled data $\dtrain$ just comes from one cluster. In contrast, both $\dpool \defeq (\xpool,\ypool),\deval \defeq (\xeval,\yeval)$ are generated   from all the clusters. 
We begin with a low-dimensional scenario, generating a one-dimensional regression setting using a GP. %Gaussian Process (GP).
Although the data-generating process is not known to the algorithms,  we assume that the GP hyperparameters are known to all the algorithms
to ensure fair comparisons. This can be viewed as a setting where our prior is well-specified, allowing us to isolate the effects
of different policy optimization approaches
 without any concerns about the misspecified priors. We select $10$ batches, each of size $K=5$ across $T = 10$ time horizons.

To examine the robustness of our method against the distributional assumptions made  in the simulated case, we then move to a real dataset where the correct prior is not known. We simulate selection bias from the eICU dataset~\citep{PollardJoRaCeMaBa18}, which contains real-world patient data with in-hospital mortality outcomes. 
We conduct a $k$-means clustering to generate 51 clusters and then select data from those clusters. We view this to be a credible replication of practice, as severe distribution shifts are common due to selection bias in clinical labels.  To convert the binary mortality labels into a regression setting, we train a  random forest classifier and fit a GP on predicted scores, which serves as the UQ module for all the algorithms. As before, the task is to select 10 batches, each consisting of 5 samples, across 10 time horizons.

 In Figures~\ref{fig:var-l2-sim} and~\ref{fig:mean-l2-sim}, we present results for the simulated data. 
Figure~\ref{fig:var-l2-sim} shows the variance of $\ell_2$ loss, and Figure~\ref{fig:mean-l2-sim} presents the error in the estimated $\ell_2$ loss using $\mu_t$ (relative to true $\ell_2$ loss, that is unknown to the algorithm). 
As we can see from these plots, our method one-step lookahead  gives substantial improvements  over active learning baselines and random sampling. In addition,
compared to the one-step lookahead planning approach using \textsf{REINFORCE}-based policy gradients, 
we observe that $\mathsf{Smoothed\text{-}Autodiff}$-based policy gradients provide significantly more robust performance over all horizons.

In Figures~\ref{fig:var-l2-real}~and~\ref{fig:mean-l2-real}, we observe similar findings on the eICU data. We see that planning policies (\textsf{REINFORCE} and $\mathsf{Smoothed\text{-}Autodiff}$) consistently outperform other heuristics by a large margin.  Active learning baselines perform poorly in these small-horizon batched problems and can sometimes be even worse than the random search baselines.  Overall, our results show the importance of careful planning in adaptive labeling for reliable model evaluation. 

We offer some intuition as to why one-step lookahead planning may outperform other heuristic algorithms. 
 First,  \textsf{Uncertainty sampling (Static)} while myopically selects the
 top-$K$ inputs with the highest uncertainty, it fails to consider 
the overlap in information content among the ``best” instances; see \citep{AggarwalKoGuHaPh14} for more details. 
In other words,  it might acquire points from the same region with high uncertainty while failing to induce diversity among the batch.
Although \textsf{Uncertainty Sampling (Sequential)} somewhat addresses the issue of information overlap, a significant drawback of 
this algorithm
is the disconnect between the objective we aim to optimize and the algorithm. For example, it might sample from a region with high uncertainty but very low density. 

\begin{figure}[t]
\centering
\begin{minipage}[b]{0.48\textwidth}
\centering
\includegraphics[width=\textwidth, height=5cm]{figures/original_scale/Var_of_l_2_loss_real.pdf}
\caption{(Real-world eICU data) Variance of mean squared loss evaluated through the posterior belief $\mu_t$ at each horizon $t$. Even 1-step lookaheads are extremely effective planners, and auto-differentiation-based pathwise policy gradients provide a reliable optimization algorithm based on low-variance gradient estimates.}
\label{fig:var-l2-real}
\end{minipage}
\hfill
\begin{minipage}[b]{0.48\textwidth}
\centering \includegraphics[width=\textwidth, height=5cm]{figures/original_scale/Error_of_estimated_model_l_2_loss_real.pdf}
\caption{(Real-world eICU data) Error between MSE calculated based on collected data $\mc{D}^{0:T}$ vs. population oracle MSE over $\mc{D}_{\rm eval} \sim P_X$. Reducing uncertainty over posteriors directly leads to better OOD evaluations. Our method significantly outperforms active learning-based heuristics, and random sampling.}
\label{fig:mean-l2-real}
\end{minipage}
%\caption{Real data for GPs}
\end{figure}
 
%\vspace{-1.5cm}
% \begin{wrapfigure}{r}{.32\columnwidth}
%   \vspace{-.5cm} 
%   \centering
% \includegraphics[scale=.29]{figures/Var of l2l_2 loss.pdf}
%   \vspace{-0.2cm}
%   \caption{Results of GP}
% \label{fig:var-l2-gp}
%   \vspace{-0.1cm}
% \end{wrapfigure}


% Attempts have been made  in the past to address these  drawbacks heuristically  (see \citep{AggarwalKoGuHaPh14}). We give a unified computational framework while approaching the problem in a more principled manner and solving it more optimally.




\subsection{Planning with  neural network-based uncertainty quantification methods ($\ensembleplus$)}


We now provide a proof-of-concept that shows the generalizability of our conceptual framework  to the deep learning-based UQ modules, specifically focusing on $\ensembleplus$ due to their previously observed superior performance~\citep{OsbandWenAsDwIbLuRo23}. Recall that implementing our framework with deep learning-based UQ modules  requires us to retrain the model across multiple possible random actions $\bm{a}(\theta)$ sampled from the current policy $\pi_\theta$.
This requires significant computational resources, in sharp contrast to the GPs where the posteriors are in closed form and can be readily updated and differentiated. 

Due to the computational constraints, we test $\ensembleplus$ on a toy setting to demonstrate the generalizability of our framework. We consider a setting where the general population consists of four clusters, while the initial labeled data only comes from one cluster. Again we generate data using GPs.  The task is to select a batch of 2 points in one horizon. We detail the $\ensembleplus$ architecture in Section \ref{sec:details-experiments}, and we assume prior uncertainty to be large (depends on the scaling of the prior generating functions). 
The results are summarized in the Table~\ref{tab:UQ_ensemble}.

% \begin{table}[H]
% \vspace{-10pt}
% \caption{Performance under \ensembleplus as UQ module}
%     \centering
%     \begin{tabular}{|m{3cm}|m{2.5cm}|m{2cm}|} 
%     \hline
%       Algorithm   & Variance of $\loss_2$ loss estimate & Error of $\loss_2$ loss estimate  \\ \hline Random Sampling 
%          & $1710.9 \pm 1352.1$ & $8.67\pm6.62$ 
%       \\ \hline \ouralgo & $1.30 \pm 0.68$ & $0.91\pm0.25$ \\ \hline
%     \end{tabular}
%     \label{tab:UQ_ensemble}
%     %\vspace{-10pt}
% \end{table}




\begin{table}[h]
\vspace{-10pt}
\caption{Performance under \ensembleplus as the UQ module}
\centering
\begin{tabular}{|l|l|l|}
\hline
Algorithm   & Variance of $\loss_2$ loss estimate & Error of $\loss_2$ loss estimate  \\
\hline
\textsf{Random sampling} & 7129.8 $\pm$ 1027.0 & 136.2 $\pm$ 8.28 \\ \hline
\textsf{Uncertainty sampling (Static)} & 10852 $\pm$ 0.0 & 162.156 $\pm$ 0.0 \\ \hline
\textsf{Uncertainty sampling (Sequential)} & 8585.5 $\pm$ 898.9 & 144 $\pm$ 6.93 \\ \hline
\textsf{REINFORCE} & 1697.1 $\pm$ 0.0 & 45.27 $\pm$ 0.0 \\ \hline
\ouralgo & 1697.1 $\pm$ 0.0 & 45.27 $\pm$ 0.0 \\ \hline
\end{tabular}
%\caption{Comparison of different algorithms based on variance   and   error in $\ell_2$ loss estimation with Ensemble $+$ as the UQ module. Our results demonstrate that {\ouralgo} and REINFORCE outperformthe other active learning based heuristics, confirming the benefits of our MDP formulation for the adaptive labeling problem, as also demonstrated in Section 4.\\
%\footnotesize{Experimental details: We use Gaussian Processes as our data generating process, GP parameters are the same as in Section D.3.  The task is to select a batch of 2 points along one horizon.The marginal distribution $p_X$ has 4 \textit{non-overlapping} clusters. Initial data comes from one cluster, while pool and evaluation points comes from all the clusters. We have $20$ initial labeled data points, $10$ pool points, and $252$ evaluation points.  Training procedures are similar to the one in Section D.3.} }
\label{tab:UQ_ensemble}
\end{table}



% We faced  issues in scaling up these experiments which will be our focus in the future. 





% \begin{itemize}
%     \item Posteriors should be consistent. Two dimensions: even with less training,  
%     \item the inference should be  fast enough
% \end{itemize}


% Potential research directions for uncertainty quantification

% In this section we consider a simple setting We consider a simpler setting and 


% For synthetic dataset generation, we use ...... For real datasets, we use ...... We compare our methodolgy to several baselines ()    This Section is structured as follows:
% \begin{itemize}
%     \item \textbf{GPs, square loss objective} (Section \ref{}): 
%     %the broad aim of the experiments  in this section is to isolate the performance of our methodology without any concerns for the inefficiencies induced due to a mis-specified prior or imperfect posterior inference. To accomplish this we generate synthetic datasets using GPs (detailed later). We use the well specified prior (GPs - with same hyperparameter setting) as our UQ module.   
%      As GPs provide differentaible posterior inference - any errors induced due to imperfect posterior updates are also isolated. We note that under this setting
%      \item In Section\ref{} we demonstrate why our methodology performs better than other baselines - by devising various synthetic experiments ()
%     \item  \textbf{UQ Benchmarking }(Section \ref{}): Before diving into the experiments using $\ensembleplus$ and ENNs,  we showcase our benchmarking experiments in Section \ref{}. We use real datasets We observe that ENNs perform better
%      \item \textbf{Ensemble $+$}, objective: recall, accuracy
%     \item \textbf{ENN}, objective: recall, accuracy
% \end{itemize}




% In Section {}, we test 
% \subsection{Experimental details}

% \begin{itemize}
%     \item UQ methodologies - GPs, ENNs
%     \item Objectives - Recall,  ATE
%     \item Datasets - ATE-synthetic datasets, Recall-synthetic, real datasets
%     \item Baselines - 
%     \begin{itemize}
%         \item Random sampling
%         \item Active learning - Uncertainty based sampling - In regression setting almost all of the 
%         \item Myopic greedy - Greedy Batch based sampling
%         \item Policy Gradient
%     \end{itemize}
    
% \end{itemize}

% \subsection{Experiments}
%     \begin{itemize}
%     \item GPs with square loss
%     \item Benchmarking ENN
%         \item ENNs with ATE
%         \item ENNs with Recall
%     \end{itemize}

% \subsection{Benefits over other algorithms - intuition and experiments}

%Active learning - Myopic greedy / Don't rely on the objective rather some entropy version.


%%% Local Variables:
%%% mode: latex
%%% TeX-master: "main"
%%% End:

\noindent\textbf{Related Work: }
%Our work is closely connected to three strands in the broad RL literature: multi-task RL, meta-RL, and personalized RL.
%\luise{tension between negative interference and generalisation: Several works have focused on the problem of negative interference (Du et al., 2018; Suteu & Guo, 2019; Yu et al., 2020a) where the gradients corresponding to the different tasks interfere negatively with each other. }
%\cite{chu2024meta}Meta-Reinforcement Learning via Exploratory Task Clustering
Our work is closely related to three key areas within the broader reinforcement learning literature: multi-task RL, meta-RL, and personalized RL.

\noindent\textit{Multi-Task RL (MTRL):}
%A major advantage of MTRL over single-task learning is the ability to share knowledge across tasks, a concept extensively explored in various studies proposing different methods to utilize task relationships
The broad goal of MTRL approaches is to leverage inter-task relationships to learn policies that are effective across multiple tasks~\citep{yang2020multi,sodhani2021multi,sun2022paco}. 
%However, naive knowledge-sharing across tasks can lead to negative transfer, as not all tasks benefit from shared knowledge. Consequently, learning a task-specific skill may distract from the learning of other tasks. 
An important challenge in MTRL is task interference.
One class of approaches aims to mitigate this issue through gradient alignment techniques~\citep{hessel2019multi,yu2020gradient}.
%A notable area of research examines task interference in MTRL through the lens of gradient alignment. 
%\citet{yu2020gradient} tackles it by projecting the gradient of a task to the orthogonal direction of all the other tasks, while~\citet{hessel2019multi} addresses it via synchronizing the gradient magnitude across tasks. 
An alternative series of approaches addresses this issue
%Numerous methods in the literature aim to address task interference issues 
from a representation learning perspective, enabling the learning of policies that explicitly condition on task embeddings~\cite{hendawy2024multi,lan2024contrastive,sodhani2021multi}.
%\citet{sodhani2021multi} learn a mixture of state encoders shared across tasks, that helps generate diverse representations through an attention mechanism.
%\citet{lan2024contrastive} introduce the Contrastive Modules with Temporal Attention (CMTA) framework, which leverages contrastive learning to ensure the modules are distinct from one another and integrates shared modules at a finer granularity than the task level using temporal attention. 
%Recently,~\citet{hendawy2023multi} proposed an approach called Mixture of Orthogonal Experts (MOORE) that captures common structures among tasks by employing orthogonal representations to enhance diversity. MOORE utilizes a Gram-Schmidt process to create a shared subspace of representations derived from a mixture of experts. 
However, most MTRL methods still face challenges when tasks are diverse, particularly when it comes to generalizing to \emph{previously unseen tasks}.
%While all previous MTRL approaches focus on learning a policy to efficiently address a predefined set of tasks, our focus is to learn a set of policies such that at least one policy in the set is near-optimal for most \emph{previously unseen tasks}.

\noindent\textit{Meta-RL:}
The goal of meta-RL is an ability to quickly adapt to unseen tasks---what we refer to as \emph{few-shot learning}.
Meta-RL methods can be categorized broadly into two categories: (i) gradient-based and (ii) context-based (where context may include task-specific features).
Gradient-based approaches 
%have been developed to address the few-shot adaptation challenge. These approaches 
focus on learning a shared initialization of a model across tasks that is explicitly trained to facilitate few-shot learning~\citep{finn2017model,stadie2018importance,mendonca2019guided,zintgraf2019fast}.
However, they perform poorly in zero-shot generalization, and tend to require a large number of adaptation steps.
Context-based methods learn a context representation and use it as a policy input~\citep{bing2023meta,gupta2018meta,duan2016rl,lee2020stochastic,lee2020context,lee2023parameterizing,rakelly2019efficient}. %by employing RNN or LSTM-based neural networks to encode collected experiences into a latent context embedding, and then act by conditioning the policy on the learned context. 
However, these approaches often exhibit mode-seeking behavior and struggle to generalize, particularly when the number of training tasks is small.
While some recent approaches, such as \citet{bing2023meta}, attempt to improve performance by using natural language task embedding, they still require a large number of training tasks to succeed.
%However, they are susceptible to distribution shifts at inference time, as the encoded context and the policy derived from that context often struggle to generalize to out-of-distribution tasks. 
%Additionally, the parameters of the latent context encoder are trained to predict reward and/or transition dynamics based on the context, typically involving the minimization of a KL divergence-based loss. 
%Consequently, the learned context tends to exhibit mode-seeking behavior, which poses a significant limitation in situations that require capturing diverse, multi-modal context (such as in Meta-World). 
%Recently, \citet{bing2023meta} attempt to address this issue in non-parametric tasks by using task-specific detailed natural language instructions, but this approach still suffers from poor sample efficiency with respect to training tasks.
%Several gradient-based methods
%, allowing the agent to achieve strong performance on unseen target tasks with only a few gradient updates. 
%These approaches are not well-suited for zero-shot generalization problems, as they typically require numerous gradient steps through the policy to learn an effective policy for a given task. Finally, since meta-RL methods prioritize rapid adaptation, they often fall short of state-of-the-art MTRL performance on in-sample (training) tasks. In this work, we aim to close this gap by developing a framework that excels in both in-sample and out-of-sample tasks.

\noindent\textit{Personalized RL:}
\citet{ackermann2021unsupervised} and~\citet{ivanov2024personalized} proposed addressing task diversity by clustering RL tasks and training a distinct policy for each cluster.
Both use EM-style approaches to jointly cluster the tasks and learn a set of cluster-specific policies.
Our key contribution is to leverage an explicit parametric task representation, and reformulate the objective as a flexible  \emph{coverage} problem for an unknown distribution of tasks.
This enables us to achieve task sample efficiency both in theory and practice.
%, each with distinct preferences, through interaction with a small set of representative policies. 
%Although the personalized RL framework has some similarities to our approach, we adopt a broader setup allowing for variations both in rewards and transition dynamics, since many real-world scenarios warrant mastering a diverse set of tasks that comprise different dynamics.
%often necessitates acting optimally amidst varying transition dynamics and preferences. 
In particular, we empirically show that our \textsc{pacman} method significantly outperforms the state-of-the-art personalized RL approach.
%across diverse evaluation settings even when only rewards vary. 

\noindent\textit{Reward-Free RL:} Another related line of research is reward-free RL~\citep{agarwal2023provable,cheng2022provable,jin2020reward}.
However, results in this space make strong assumptions, such as assuming that the action space is finite or linear dynamics.
Our results, on the other hand, make no assumptions on the dynamics and do not depend on the size or dimension of the state or action space.
%A common example is that the action space is typically assumed to be finite. The papers that do not assume this require instead linearity (of both the dynamics and rewards), and/or make assumptions (such as low-rank) on the transition model. 
%We do not assume linearity, make no assumptions about the nature of the dynamics, and allow actions to be either discrete or in a continuous vector space (none of our results depend on the size or dimension of state or action space). Thus, one of our key results in few-shot generalization depends only on the number of policies 
 %(i.e., sample complexity is linear in 
%, along with other problem-specific features), but has no dependence on state or action space dimension.
\section*{Conclusion}
This paper aims to enhance our understanding of the computational complexity of computing various Shapley value variants. We found that for various ML models --- including decision trees, regression tree ensembles, weighted automata, and linear regression --- both local and global interventional and baseline SHAP can be computed in polynomial time under HMM modeled distributions. This extends popular algorithms, such as TreeSHAP, beyond their empirical distributional scope. We also establish strict complexity gaps between the various SHAP variants (baseline, interventional, and conditional) and prove the intractability of computing SHAP for tree ensembles and neural networks in simplified scenarios. Overall, we present SHAP as a versatile framework whose complexity depends on four key factors: \begin{inparaenum}[(i)] \item model type, \item SHAP variant, \item distribution modeling approach, \item and local vs. global explanations\end{inparaenum}. We believe this perspective provides deeper insight into the computational complexity of SHAP, paving the way for future work.




%We believe that our framework provides a more intricate understanding of SHAP computation complexity across different models, distributions, and variants, paving the way for further research.

Our work opens promising directions for future research. First, expanding our computational analysis to other SHAP-related metrics, such as asymmetric SHAP~\citep{frye20} and SAGE~\citep{covert2020understanding}, would be valuable. Additionally, we aim to explore more expressive distribution classes and relaxed assumptions beyond those in Section \ref{sec:tractable} while maintaining tractable SHAP computation. Finally, when exact computation is intractable (Section \ref{sec:intractable}), investigating the approximability of SHAP metrics through approximation and parameterized complexity theory~\citep{downey2012parameterized} is an important direction.

%Our work opens several promising avenues for future research on the computational properties of explainable AI methods, with a particular focus on SHAP. First, it would be interesting to broaden the computational analysis conducted in this work to include other popular SHAP-related metrics in the literature, such as asymmetric SHAP \cite{frye20} and SAGE \cite{covert2020understanding}. Also, in the future, we aim to explore more expressive distribution classes and relaxed distributional assumptions—extending beyond those examined in Section \ref{sec:tractable} —that still yield tractable SHAP computation. Finally, when exact computation proves intractable (Section \ref{sec:intractable}), it is worthwhile to theoretically investigate the question of the approximability of computing the SHAP metrics across various configurations, through the lens of approximation and parametrized complexity theory \cite{arora2009computational}.

%This paper aims to deepen our understanding of the computational complexity involved in obtaining different Shapley value variants. We found that for a variety of ML models, including decision trees, tree ensembles for regression, weighted automata, and linear regression models — computing both local and global interventional and baseline SHAP can be done in polynomial time when distributions are modeled by HMMs. This extends the distributional scope of popular algorithms like TreeSHAP, which is limited to empirical distributions. Additionally, we demonstrate a strict complexity gap between SHAP variants, showing that interventional and baseline SHAP can be strictly easier to compute than conditional SHAP. Despite these positive results, we uncovered intractability for various SHAP variants in neural networks and tree ensembles. Finally, we provided generalized complexity relations across SHAP variants. We believe that our framework offers a deeper understanding of the complexity involved in computing SHAP across various variants, models, distributions, as well as in both local and global computations, laying the groundwork for future research.

%%
%% The acknowledgments section is defined using the "acks" environment
%% (and NOT an unnumbered section). This ensures the proper
%% identification of the section in the article metadata, and the
%% consistent spelling of the heading.
%[TW] acmguide on the "\thanks" command:
%"This command is obsolete and should not be used in most cases.
%Do not list your acknowledgments or grant sponsors here.
%Put this information in the acks environment (see Section 2.13)."
\begin{acks}
    The authors thank Philipp Schroer for his support with \toolcaesar, Mirco Tribastone for his help with the initial conceptualization of the paper, and the anonymous referees for their detailed comments.
    %
    This work was partially funded by the \grantsponsor{DFG}{DFG}{https://www.dfg.de} \grantnum{DFG}{GRK 2236 UnRAVeL}, the EU's Horizon 2020 programme under the Marie Skłodowska-Curie grant No.\ 101008233 (MISSION), the ERC AdG No.\ 787914 (FRAPPANT), and the PNRR project iNEST (Interconnected Nord-Est Innovation Ecosystem) funded by the European Union Next-GenerationEU (Piano Nazionale di Ripresa e Resilienza (PNRR) – Missione 4 Componente 2, Investimento 1.5 – D.D. 1058 23/06/2022, ECS\_00000043).
\end{acks}

%% [TW] data availability statement and refereces don't count towards OOPSLA page limit (25 pages)
\iftoggle{arxiv}{}{%
\section*{Data Availability Statement}
A software artifact comprising the \toolcaesar verifier, the encodings of the case studies from \Cref{sec:case_studies}, and instructions for running the experiments is available~\cite{zenodo}.
}
%%
%% The next two lines define the bibliography style to be used, and
%% the bibliography file.
\bibliographystyle{ACM-Reference-Format}
\bibliography{references}


\iftoggle{arxiv}{
%%
%% If your work has an appendix, this is the place to put it.
\newpage
\appendix
\section*{\centering Appendix}
\bigskip
%% [TW] aligned equations are allow to break over pages in appendix, otherwise it becomes messy
\allowdisplaybreaks
\section{Additional Preliminary Material}
\label{app:background}
\subsection{Fixed Point Theory}

The following observation is key in several proofs in this paper:
%
\begin{restatable}[\textnormal{cf.~\cite[Proposition 2.1.12]{abramsky1994domain}}]{lemma}{oneVsTwoSups}
    \label{thm:oneVsTwoSups}
    Let $\poGen$ be an $\omega$-cpo and let $\poElem \colon \nats \times \nats \to \poDom$ be monotonic (with regards to the usual order on $\nats$ lifted pointwise to $\nats \times \nats$).
    Then
    \[
        \sup_{i \in \nats} \sup_{j \in \nats} \poElem(i,j)
        \eeq
        \sup_{i \in \nats} \poElem(i,i)
        \eeq
        \sup_{j \in \nats} \sup_{i \in \nats} \poElem(i,j)
        ~.
    \]
\end{restatable}
\begin{proof}
    \Cref{proof:oneVsTwoSups}.
\end{proof}


\subsection{Measure Theory and Lebesgue Integrals}
\label{sec:prelims:measure}


\subsubsection{Basic Notions of Measure Theory}
\label{sec:prelims:measure:basics}

Most of the following can be found in~\cite{ash2000probability}.
Let $\mUniv$ be a set.
A \proseSigmaAlgebra on $\mUniv$ is a set $\sAlg \subseteq 2^\mUniv$ of subsets of $\mUniv$ such that $\mUniv \in \sAlg$ and $\sAlg$ is closed under taking countable unions and complements relative to $\mUniv$.
If $\sAlg$ is a \proseSigmaAlgebra on $\mUniv$ then $(\mUniv, \sAlg)$ is called a \emph{measurable space} and the sets in $\sAlg$ are called \emph{measurable sets} (this is not to be confused with a \emph{measure space} which is introduced further below).
%
For an arbitrary set $\sigmaAlgebraGenerators \subseteq 2^\mUniv$ of subsets of $\mUniv$ we let $\sigmaAlgebraGeneratedBy{\sigmaAlgebraGenerators} = \bigcap \{\mathcal F \mid \sigmaAlgebraGenerators \subseteq \sAlg, \sAlg\text{ a \proseSigmaAlgebra on } \mUniv \}$ be the smallest \proseSigmaAlgebra (on $\mUniv$) containing $\sigmaAlgebraGenerators$.

The \emph{Borel \proseSigmaAlgebra} $\borelSets{\exReals}$ on the extended real numbers $\exReals$ is the smallest \proseSigmaAlgebra containing all intervals of the form $(\ivalL, \ivalR]$, $\ivalL, \ivalR \in \exReals$, i.e., $\borelSets{\exReals} = \sigmaAlgebraGeneratedBy{\{(\ivalL, \ivalR] \mid \ivalL, \ivalR \in \exReals\}}$.
The sets in $\borelSets{\exReals}$ are called \emph{Borel sets}.
Note that many common sets including various types of intervals (bounded, unbounded, open, closed, half-open, and so on), all countable subsets of $\exReals$, and many others are Borel sets.
However, $\borelSets{\exReals} \neq 2^\exReals$, i.e., there exist non-Borel sets.
%
For $\measurableSet \in \borelSets{\exReals}$ we define $\borelSets{\measurableSet} = \borelSets{\exReals} \cap \measurableSet$, where the intersection is taken element-wise.
It can be shown that $\borelSets{\measurableSet}$ is a \proseSigmaAlgebra on $\measurableSet$ and $\borelSets{\measurableSet} \subseteq \borelSets{\exReals}$, see, e.g.,~\cite[page 5]{ash2000probability}.
In this way we may obtain a \proseSigmaAlgebra on, say, $\uIval$.

Given a measurable space $(\mUniv, \sAlg)$, a \emph{measure} on $\sAlg$ is a function $\measure \colon \sAlg \to \exNonNegReals$ such that for all countable collections $\countableCollectionOfMeasurableSets \subseteq \sAlg$ of pairwise disjoint subsets of $\sAlg$ (i.e., $\forall \measurableSet,  \measurableSetb \in \countableCollectionOfMeasurableSets \colon \measurableSet = \measurableSetb \lor \measurableSet \cap \measurableSetb = \emptyset$) it holds that
$\measure\left( \bigcup_{\measurableSet \in \countableCollectionOfMeasurableSets} \measurableSet \right) = \sum_{\measurableSet \in \countableCollectionOfMeasurableSets} \measure(\measurableSet)$, where the infinite sum is allowed to take value $\infty$.
If $\measure(\mUniv) = 1$ then $\measure$ is a \emph{probability measure}.
%
We denote by $\lebmes$ the \emph{Lebesgue measure}%
\footnote{We remark that it is also common to define $\lebmes$ on $\lebesgueSets{\reals}$, the \emph{completion} of $\borelSets{\reals}$ w.r.t.\ $\lebmes$ (the sets in $\lebesgueSets{\reals}$ are often called \emph{Lebesgue measurable}), but this construction is not necessary for our purposes.}
%
on $\borelSets{\reals}$
is the unique measure $\lebmes$ satisfying $\lebmes((\ivalL, \ivalR]) = \ivalR - \ivalL$ for all $\ivalL \leq \ivalR \in \reals$.
Note that we define this only on the reals as we do not need it for the extended reals.
In fact, the only measure we consider (explicitly) in this paper is the Lebesgue measure on $\borelSets{\uIval}$.


\subsubsection{Lebesgue Integrals}
\label{sec:prelims:measure:integrals}

Let $(\mUniv_i, \sAlg_i)$, $i \in \{1,2\}$, be measurable spaces.
A function $\fun \colon \mUniv_1 \to \mUniv_2$ is called \emph{measurable} w.r.t.\ $\sAlg_1$ and $\sAlg_2$ if for all $\measurableSet \in \sAlg_2$ it holds that $\fun^{-1}(\measurableSet) \in \sAlg_1$.

For our purpose we only need to define integrals of non-negative functions.
Let $(\mUniv, \sAlg, \measure)$ be a measure space and $\fun \colon \mUniv \to \exNonNegReals$ be an arbitrary function.
A measurable function $\simpleFun \colon \mUniv \to \nonNegReals$ w.r.t.\ $\sAlg$ and $\borelSets{\nonNegReals}$ is called \emph{simple} if its image $\simpleFun(\mUniv)$ is a finite set $\{a_1,\ldots,a_n\}$; such an $\simpleFun$ can be written as $\simpleFun(x) = \sum_{i=1}^n a_i \cdot \iv{x \in \simpleFun^{-1}(a_i)}$ for all $x \in \mUniv$.
The Lebesgue integral of the simple function $\simpleFun$ on $\measurableSet \in \sAlg$ is defined as 
$
\int_{\measurableSet} \simpleFun \,d\measure
=
\sum_{i=1}^n a_i \cdot \measure(\simpleFun^{-1}(a_i) \cap \measurableSet)
$
.
The Lebesgue integral of $\fun$ on $\measurableSet$ is then defined as
\[
\int_{\measurableSet} \fun \,d\measure
\eeq
\sup \left\{\int_{A} \simpleFun \,d\measure \mid \simpleFun \text{ is simple, } \simpleFun \leq \fun \text{ (pointwise)} \right\}
\]
which can be any number in $\nonNegReals$ or $\infty$.
Note that $\fun$ itself does not have to be measurable; however, many fundamental properties of Lebesgue integrals  break for non-measurable $\fun$, hence it is customary to consider Lebesgue integrals only for measurable $\fun$.


\subsubsection{Multi- vs.\ Single-dimensional Integrals}
\label{sec:prelims:measure:multi}

Given two measurable spaces $(\mUniv_i, \sAlg_i)$, $i \in \{1,2\}$, define $\sAlg_1 \otimes \sAlg_2 = \sigmaAlgebraGeneratedBy{\{\measurableSet_1 \times \measurableSet_2 \mid \measurableSet_i \in \sAlg_i\}}$.
The measurable space $(\mUniv_1 \times \mUniv_2, \sAlg_1 \otimes \sAlg_2)$ is called the \emph{product} of $(\mUniv_1, \sAlg_1)$ and $(\mUniv_2, \sAlg_2)$.
%
For $n > 1$ and $\measurableSet \in \borelSets{\exReals}$, we obtain the Borel \proseSigmaAlgebra $\borelSets{\measurableSet^n}$ on $\measurableSet^n$ as the $n$-fold product of $(\measurableSet, \borelSets{\measurableSet})$ with itself.
The sets in $\borelSets{\measurableSet^n}$ are called Borel sets as well.

We call a function $\fun \colon \measurableSet \to \measurableSetb$ \emph{Borel measurable} if both $A$ and $B$ are Borel sets (in any dimension) and $\fun$ is measurable w.r.t.\ the respective Borel \proseSigmaAlgebras on $A$ and $B$. 
%
In this paper, we often consider Borel measurable functions of type $\fun \colon \reals^n \to \exNonNegReals$ and we would like to do something like \enquote{take a Lebesgue integral in one variable}.
To justify that this makes sense (and preserves measurability) we rely on the following:

\begin{theorem}[Part of Fubini's Theorem\textnormal{~\cite[Theorems 8.5 and 8.8]{10.5555/26851}}]
    \label{thm:fubini}
    Let $(\mUniv_i, \sAlg_i)$, $i \in \{1,2\}$, be measurable spaces and $\fun \colon \mUniv_1 \times \mUniv_2 \to \exNonNegReals$ be measurable w.r.t.\ the product space.
    Then:
    \begin{enumerate}
        \item For every $x \in \mUniv_1$, the function $\fun_x \colon \mUniv_2 \to \exNonNegReals, y \mapsto \fun(x,y)$ is measurable w.r.t.\ $\sAlg_2$.
        \item For every $\measurableSet \in \sAlg_2$ and measure $\measure$ on $\sAlg_2$, the function $F \colon \mUniv_1 \to \exNonNegReals, x \mapsto \int_{\measurableSet} \fun_x \,d\measure$ is measurable w.r.t.\ $\mathcal F _1$.
    \end{enumerate}
\end{theorem}
%
For a given $x \in \mUniv_1$, we also write $\int_{\measurableSet}\fun(x,y) \,d\measure(y)$ instead of $\int_{\measurableSet}\fun_{x} \,d\measure$.


\subsection{Riemann Integrals}

A useful aspect of (lower and upper) Riemann integrals is that one can choose partitions that induce sequences of upper (resp.\ lower) sums that converge \emph{monotonically}.
For partitions $\partition = (x_0,\ldots,x_{\partitionSize})$ and $\partitionb = (y_0,\ldots,y_{\partitionSizeb})$,
 both of the same interval, we say that $\partition$ \emph{refines} $\partitionb$, in symbols $\partition \partitionRefines \partitionb$, if for all indices $0 \leq j \leq \partitionSizeb$ there exists $0 \leq i \leq \partitionSize$ such that $x_i = y_j$.
In other words, the partition $\partition$ can be obtained from $\partitionb$ by subdividing some of the intervals in $\partitionb$ further.
\begin{lemma}[\textnormal{see,~e.g.,~\cite[Lemma 6.2]{fitzpatrick2009advanced}}]
    \label{thm:partitionRefine}
    Let $\fun \colon \clIvalGen \to \reals$ be bounded.
    Suppose that $\partition \partitionRefines \partitionb$.
    Then $\lowerSum{\fun}{\partitionb} \leq \lowerSum{\fun}{\partition}$
    and
    $\upperSum{\fun}{\partition} \leq \upperSum{\fun}{\partitionb}$.
\end{lemma}

To obtain the Riemann-Darboux integral, it is not actually necessary to consider \emph{all} partitions.
Indeed, it suffices to consider a set of partitions whose \emph{norm} (\enquote{fineness}) becomes arbitrarily small.
We now formalize this.
For a partition $\partition = (x_0,\ldots,x_{\partitionSize}) \in \partitions{\clIvalGen}$, let its norm be defined as
\[
\partitionNorm{\partition}
\eeq
\max_{0 \leq i < \partitionSize} x_{i+1} - x_i
~.
\]

\begin{restatable}[\textnormal{see,~e.g.,~\cite[Theorem 7.12]{fitzpatrick2009advanced}}]{theorem}{smallNormSuffices}
    \label{thm:smallNormSuffices}
    Let $\fun \colon \clIvalGen \to \reals$ be bounded and let $(\partition_n)_{n\in\nats}$ be a sequence of partitions of $\clIvalGen$ satisfying $\lim_{n \to \infty} \partitionNorm{\partition_n} = 0$.
    Then \[
    \lowerIntGen \fun(x) \,dx
    \eeq
    \lim_{n \to \infty} \lowerSum{\fun}{\partition_n}
    \eeq
    \sup_{n \in \nats} \lowerSum{\fun}{\partition_n}
    \qqand
    \upperIntGen \fun(x) \,dx
    \eeq
    \lim_{n \to \infty} \upperSum{\fun}{\partition_n}
    \eeq
    \inf_{n \in \nats} \upperSum{\fun}{\partition_n}
    ~.
    \]
\end{restatable}
\begin{proof}
    \Cref{proof:smallNormSuffices}
\end{proof}


\section{Omitted Remarks and Explanations}
\label{app:misc}
\subsection{On Redundancies in the Program Syntax}
\label{app:redudanciesSyntax}

Some of the programming constructs from \Cref{def:pwhile} could be regarded syntactic sugar.
However, this would have some unsatisfactory effects.
For example, $\PCHOICE{\prog_1}{\prob}{\prog_2}$ is equivalent to $\SEQ{\UNIFASSIGN{\pVarTmp}}{\ITE{\pVarTmp < \prob}{\prog_1}{\prog_2}}$, but this has the disadvantages that (i) it requires an additional program variable, (ii) it assumes that $(\pVarTmp < \prob) \in \guards$, and (iii) it replaces a simple convex combination by a way more difficult integration operation, which is a major drawback in practice.
%
Further, one could define $\DIVERGE$ as syntactic sugar for $\WHILE{true}{\SKIP}$.
However, in the context of this paper, it is more convenient to distinguish between loopy programs and loop-free programs possibly containing $\DIVERGE$.


\subsection{Formal Definition of Loop Unrolling}
\label{app:unrolling}

The following definition appears in \Cref{sec:approxwp:convloops}.
Let $\depth \in \nats$.
We define the \emph{$\depth$-fold unrolling} $\unfold{\prog}{\depth}$ of program $\prog$ inductively as follows:
For all atomic programs $\prog$, we let $\unfold{\prog}{\depth} = \prog$.
Further, we set (using some parentheses -- which are of course not part of the program syntax -- for clarity)
\begin{itemize}
    \item $\unfold{(\SEQ{\prog_1}{\prog_2})}{\depth} = \SEQ{\unfold{\prog_1}{\depth}}{\unfold{\prog_2}{\depth}}$,
    \item $\unfold{(\ITE{\guard}{\prog_1}{\prog_2})}{\depth} = \ITE{\guard}{\unfold{\prog_1}{\depth}}{\unfold{\prog_2}{\depth}}$, and
    \item $\unfold{(\PCHOICE{\prog_1}{\prob}{\prog_2})}{\depth} = \PCHOICE{\unfold{\prog_1}{\depth}}{\prob}{\unfold{\prog_2}{\depth}}$.
\end{itemize}
For loops, we let
\begin{itemize}
    \item $\unfold{(\WHILE{\guard}{\progBody})}{0} = \DIVERGE$, and for $\depth \geq 0$,
    \item $\unfold{(\WHILE{\guard}{\progBody})}{\depth+1} = \ITE{\guard}{\SEQ{\unfold{\progBody}{\depth}}{\unfold{(\WHILE{\guard}{\progBody})}}{\depth}}{\SKIP}$.
\end{itemize}

\section{Proofs}
\label{app:proofs}
%
\subsection{Proof of \Cref{thm:oneVsTwoSups}}
\label{proof:oneVsTwoSups}
%
\oneVsTwoSups*
%
\begin{proof}
    The suprema and infima exist because $\poGen$ is an $\omega$-cpo.
    The fact that the order of the two suprema can be exchanged is a standard property of suprema, so we only show the leftmost equality:
    %
    For all $i \in \nats$, $\sup_{j} \poElem(i,j) \geq \poElem(i,i)$, so $\sup_{i} \sup_{j} \poElem(i,j) \geq \sup_{i} \poElem(i,i)$.
    For the other inequality we argue as follows:
    Let $i \in \nats$ be arbitrary.
    \begin{align*}
        & \sup_j \poElem(i,j) \\
        \eeq & \sup_{i \leq j} \poElem(i,j) \tag{because $\poElem$ is non-decreasing in the 2nd argument} \\
        \lleq & \sup_{i \leq j} \poElem(j,j) \tag{because $i \leq j$ and $\poElem$ is non-decreasing in the 1st argument} \\
        \eeq & \sup_{j} \poElem(j,j)
    \end{align*}
    Since $i$ was arbitrary it follows that $\sup_i \sup_j \poElem(i,j) \leq \sup_{j} \poElem(j,j)$.
\end{proof}


\subsection{Proof of \Cref{thm:smallNormSuffices}}
\label{proof:smallNormSuffices}
%
\smallNormSuffices*
%
\begin{proof}
    We only show the equalities for the lower integral.
    The statement for the upper integral follows with \Cref{thm:upperLowerDuality}.
    
    We first show that $\lowerIntGen \fun(x) \,dx = \lim_{n \to \infty} \lowerSum{\fun}{\partition_n}$.
    Let us fix $\eps > 0$.
    We want to prove that there exists $n_0 \in \nats$ such that $\forall n \geq n_0$ it holds that $ \lowerIntGen \fun - \lowerSum{\fun}{\partition_n} < \eps$.
    
    By definition of $\lowerIntGen \fun$ as the supremum over all partitions there exists a partition $\partition_\eps$ such that 
    $\lowerIntGen \fun -\lowerSum{\fun}{\partition_\eps} < \frac{\eps}{2}.$ 
    %
    Let $\partitionSize$ be the number of points in $\partition_\eps$ and suppose that $\fun$ is bounded by $\boundedFunBound$, i.e., $|\fun| \leq \boundedFunBound$.
    Since $\lim_{n \to \infty} \partitionNorm{\partition_n} = 0$ there exists $n_0$ such that $\forall n \geq n_0$ we have $\partitionNorm{\partition_n} < \frac{\eps}{4 \partitionSize \boundedFunBound}$.
    Let $n \geq n_0$ be fixed for the rest of the proof.
    
    We set $\partition_n^* = \partition_n \cup \partition_\eps$, i.e., the partition obtained considering as extrema of the sub-intervals all the points in $\partition_n$ and $\partition_\eps$.
    Then $\partition_n^* \partitionRefines \partition_n$ and $\partition_n^* \partitionRefines \partition_\eps$,
    and thus $\lowerSum{\fun}{\partition_n^*} \geq \lowerSum{\fun}{\partition_n}$ and $\lowerSum{\fun}{\partition_n^*} \geq \lowerSum{\fun}{\partition_\eps}$.
    %
    Let us consider the difference $\lowerSum{\fun}{\partition_n^*} - \lowerSum{\fun}{\partition_n}$.
    Let $I$ be the set of indexes $i$ such that $x_i \in \partition_n$ and $[x_i, x_{i+1}]$ contains at least one point $x_i^{\eps}$ of $\partition_\eps$.
    Then $|I| \leq \partitionSize$ and 
    \begin{align*}
        & \lowerSum{\fun}{\partition_n^*} - \lowerSum{\fun}{\partition_n} \\
        %
        \eeq & \sum_{i \in I} \left( (x_{i+1} - x_i^{\eps}) \inf_{[x_i^{\eps},x_{i+1}]} \fun(x) + (x_i^{\eps} - x_i) \inf_{[x_i, x_i^{\eps}]} \fun(x) \mminus (x_{i+1} - x_i) \inf_{[x_i, x_{i+1}]} \fun(x) \right) \\
        %
        \begin{split}
            \eeq & \sum_{i \in I} \left( (x_{i+1} - x_i^{\eps}) \inf_{[x_i^{\eps}, x_{i+1}]} \fun(x) + (x_i^{\eps} - x_i) \inf_{[x_i, x_i^{\eps}]} \fun(x) \right. \\
            & \qquad\qquad\qquad\qquad \left. \mminus  (x_{i+1} - x_i^{\eps}) \inf_{[x_i, x_{i+1}]} \fun(x) - (x_i^{\eps} - x_i) \inf_{[x_i, x_{i+1}]} \fun(x) \right)
        \end{split} \\
        %
        \lleq & \sum_{i \in I} (x_{i+1} - x_i^{\eps}) \underbrace{\left(\inf_{[x_i^{\eps}, x_{i+1}]} \fun(x) - \inf_{[x_i, x_{i+1}]} \fun(x)\right)}_{\leq 2\boundedFunBound} \pplus \sum_{i \in I}  (x_i^{\eps} - x_i) \underbrace{\left( \inf_{[x_i, x_i^{\eps}]} \fun(x) - \inf_{[x_i, x_{i+1}]} \fun(x) \right)}_{\leq 2\boundedFunBound} \\
        %
        \lleq & \partitionSize \frac{\eps}{4 \partitionSize \boundedFunBound} 2\boundedFunBound \eeq \frac{\eps}{2} ~.
    \end{align*}
    It follows that
    \[
    \lowerIntGen \fun - \lowerSum{\fun}{\partition_n}
    \eeq
    \lowerIntGen \fun - \lowerSum{\fun}{\partition_n^*} + \lowerSum{\fun}{\partition_n^*} - \lowerSum{\fun}{\partition_n}
    \lleq
    \lowerIntGen \fun - \lowerSum{\fun}{\partition_\eps} + L_{f,\partition_n^*} - \lowerSum{\fun}{\partition_n} < \frac{\eps}{2} + \frac{\eps}{2} = \eps
    ~.
    \]
    It remains to show that $\lim_{n \to \infty} \lowerSum{\fun}{\partition_n} = \sup_{n \geq 0} \lowerSum{\fun}{\partition_n}$.
    By definition of the lower integral as the supremum over the lower sums w.r.t.\ \emph{every possible} partition it follows that $\sup_{n \in \nats} \lowerSum{\fun}{\partition_n} \leq \lowerIntGen \fun$.
    On the other hand, the limit of a convergent sequence is always bounded by its supremum, i.e., $\lim_{n \to \infty} \lowerSum{\fun}{\partition_n} \leq \sup_{n \in \nats} \lowerSum{\fun}{\partition_n}$
\end{proof}

\begin{lemma}
    \label{thm:upperLowerDuality}
    For bounded $\fun \colon \clIvalGen \to \reals$ and $\partition \in \partitions{\clIvalGen}$ it holds that
    \begin{enumerate}
        \item $\lowerSum{-\fun}{\partition} = -\upperSum{\fun}{\partition}$, and
        \item $\lowerIntGen -\fun(x) \,dx = -\upperIntGen \fun(x) \,dx$.
    \end{enumerate}
\end{lemma}
\begin{proof}
    Let $\partition = (x_0,\ldots,x_{\partitionSize})$.
    For (1) note that
    \begin{align*}
        \lowerSum{-\fun}{\partition} \eeq & \sum_{i=0}^{\partitionSize-1} (x_{i+1} - x_i) \inf_{x \in \clIval{x_i}{x_{i+1}}} -\fun(x) \\
        %
        \eeq & \sum_{i=0}^{\partitionSize-1} (x_{i+1} - x_i) \cdot \left(-\sup_{x \in \clIval{x_i}{x_{i+1}}} \fun(x) \right) \\
        %
        \eeq & -\sum_{i=0}^{\partitionSize-1} (x_{i+1} - x_i) \sup_{x \in \clIval{x_i}{x_{i+1}}} \fun(x) \\
        %
        \eeq & -\upperSum{\fun}{\partition} ~.
    \end{align*}
    For (2) we argue similarly:
    \begin{align*}
        \lowerIntGen-\fun(x) \,dx \eeq & \sup \left\{ \lowerSum{-\fun}{\partition} \mid \partition \in \partitions{\clIvalGen} \right\} \\
        %
        \eeq & \sup \left\{ -\upperSum{\fun}{\partition} \mid \partition \in \partitions{\clIvalGen} \right\} \tag{by (1)} \\
        %
        \eeq & -\inf \left\{ \upperSum{\fun}{\partition} \mid \partition \in \partitions{\clIvalGen} \right\} \\
        %
        \eeq & - \upperIntGen \fun(x) \,dx ~.
    \end{align*}
\end{proof}
\subsection{Proof of \Cref{thm:measexpsbicpo}}
\label{proof:measexpsbicpo}
%
\measexpsbicpo*
%
\begin{proof}
    
    To see that $\po{\exps}{\eleq}$ and $\po{\bexps}{\eleq}$ are complete lattices for any $\poSubset \subseteq \exps$ (resp. $\poSubset \subseteq \bexps$) set
    $$ \sup \poSubset = \lambda \st . \sup_{\ex \in \poSubset} \ex(\st), \hspace{1cm} \inf \poSubset = \lambda \st . \inf_{\ex \in \poSubset} \ex(\st).$$
    Then, $\sup \poSubset \in \exps$ and $\inf \poSubset \in \exps$ (resp. $\sup \poSubset \in \bexps$ and $\inf \poSubset \in \bexps$).
    
    Theorem 11.17 from \cite{rudin1953principles} ensures that for any sequence of measurable functions the supremum and infimum of the sequence is measurable. Specializing the theorem in the case of Borel measurability ensures that $\po{\expsmeas}{\eleq}$ and $\po{\bexpsmeas}{\eleq}$ are $\omega$-bicpos. 
\end{proof}


\subsection{Proof of \Cref{thm:wpWlpWellDefinedAndContinuous}}
\label{proof:wpWlpWellDefinedAndContinuous}
%
\wpWlpWellDefinedAndContinuous*
%
\begin{proof}
    Parts of this statement have been shown in in \cite[p.\ 88 (Proof of Lemma 3)]{DBLP:conf/setss/SzymczakK19} and \cite[Lemma 34]{DBLP:journals/pe/GretzKM14}. %\tw{the wlp stuff was not shown }
    Here we provide a mostly self-contained proof.
    Note that for well-definedness, one must show various points:
    preservation of measurable expectations, existence of the Lebesgue integrals, and existence of the least and greatest fixed points -- for the latter, $\omega$-(co)continuity is needed, which is why we show it simultaneously.
    
    In the following, we let $\ex \in \expsmeas$ be an arbitrary measurable expectation and $\ex_0 \eleq \ex_1 \eleq \ldots$ a (non-decreasing) $\omega$-chain in $\expsmeas$.
    Similarly, $\exb \in \bexpsmeas$ is a 1-bounded measurable expectation and $\exb_0 \egeq \exb_1 \egeq \ldots$ a (non-increasing) $\omega$-cochain in $\bexpsmeas$.
    We will show by structural induction that for all $\prog \in \pWhile$ it holds that $\wp{\prog}{\ex}$ and $\wlp{\prog}{\exb}$ are measurable and that $\sup_{i \in \nats}\wp{\prog}{\ex_i} = \wp{\prog}{\sup_{i \in \nats} \ex_i}$ as well as $\inf_{i \in \nats}\wlp{\prog}{\exb_i} = \wlp{\prog}{\inf_{i \in \nats} \exb_i}$.
    Throughout this proof we write $\sup_i$ as a shorthand for $\sup_{i\in \nats}$.
  
    
    \inductionCase{$\prog = \SKIP$}
    $\wpTrans{\SKIP} = \wlpTrans{\SKIP}$ is the identify function, hence it trivially preserves measurable expectations and is $\omega$-bicontinuous.
    
    
    \inductionCase{$\prog = \DIVERGE$}
    $\wpTrans{\DIVERGE}$ are $\wlpTrans{\DIVERGE}$ are constant expectations, hence they are both measurable and $\omega$-bicontinuous.
    
    
    \inductionCase{$\prog = \ASSIGN{\pVar}{\aExp}$}
    \newcommand{\auxiliaryStateTransformer}{U}
    \newcommand{\auxStTrans}{\auxiliaryStateTransformer}
    By definition, $\wp{\ASSIGN{\pVar}{\aExp}}{\ex} = \exSubsGen$ which can also be written as $\ex \circ \auxStTrans$ where $\auxStTrans \colon \states \to \states, \st \mapsto \pStUpdate{\st}{\pVar}{\aExp(\st)}$ is a function updating the program state according to $\ASSIGN{\pVar}{\aExp}$.
    Since $\aExp$ is measurable, $\auxStTrans$ is measurable as well, and thus $\exSubsGen$ is measurable as a composition of measurable functions. The argument for measurability of $\wlp{\ASSIGN{\pVar}{\aExp}}{\exb}$ is exactly the same. 
    
    For $\omega$-continuity we have:
    \begin{align*}
        & \wp{\ASSIGN{\pVar}{\aExp}}{\sup_i \ex_i}  \\
        \eeq & \exSubs{\left( \sup_i \ex_i \right)}{\pVar}{\aExp} \tag{definition of $\wpSymb$} \\
        \eeq & \sup_i \exSubs{\ex_i}{\pVar}{\aExp} \tag{$\sup$ defined pointwise}\\
        \eeq & \sup_i \wp{\ASSIGN{\pVar}{\aExp}}{\ex_i} \tag{definition of $\wpSymb$}
        ~.
    \end{align*} 
    An analogous argument holds for $\omega$-cocontinuity of $\wlpTrans{\ASSIGN{\pVar}{\aExp}}$.
    
    
    \inductionCase{$\prog = \UNIFASSIGN{\pVar}$}
    Measurability of $\wp{\UNIFASSIGN{\pVar}}{\ex} = \lam{\pSt}{\int_\uIval \ex (\pStUpdate{\st}{\pVar}{\xi}) \, d\lebmes(\xi)}$ and its liberal counterpart is a direct application of \Cref{thm:fubini}
    
    For $\omega$-continuity of $\wpTrans{\UNIFASSIGN{\pVar}}$ we argue as follows:
    \begin{align*}
        & \wp{\UNIFASSIGN{\pVar}}{\sup_i \ex_i} \\
        %
        \eeq & \lam{\pSt}{\int_\uIval \left(\sup_i \ex_i\right)(\pStUpdate{\st}{\pVar}{\xi}) \, d\lebmes(\xi) } \tag{definition of $\wpSymb$} \\
        %
        \eeq & \lam{\pSt}{\int_\uIval \sup_i \ex_i (\pStUpdate{\st}{\pVar}{\xi})  \, d\lebmes(\xi) } \tag{$\sup$ defined pointwise} \\
        %
        \eeq & \lam{\pSt}{\sup_i \int_\uIval \ex_i (\pStUpdate{\st}{\pVar}{\xi}) \, d\lebmes(\xi)} \tag{Monotone Convergence Theorem, e.g.,~\cite[Thm.\ 12.1]{schilling2017measures}} \\
        %
        \eeq & \sup_i \lam{\pSt}{\int_\uIval \ex_i (\pStUpdate{\st}{\pVar}{\xi}) \, d\lebmes(\xi)}  \tag{$\sup$ defined pointwise} \\
        %
        \eeq & \sup_i \wp{\UNIFASSIGN{\pVar}}{\ex_i} \tag{definition of $\wpSymb$}
    \end{align*}
    Note the application of the the Monotone Convergence Theorem, which ensures that for a non-decreasing sequence of measurable $\exNonNegReals$-valued functions it is possible to switch the supremum over the sequence and the Lebesgue integral. Co-continuity for $\wlpSymb$ follows by expressing $\inf_i \exb_i = 1 - \sup_i (1-\exb_i)$ and observing that the functions $(1-\exb_i) \in \bexps$ are a non-decreasing sequence, so Monotone Converge Theorem can be applied to them.
    
    
    \inductionCase{$\prog = \OBSERVE{\guard}$}
    Since we require the functions in $\guards$ to be Borel measurable, $\iv{\guard}$ is Borel measurable for any $\guard \in \guards$. Measurability of $\wpSymb$ and $\wlpSymb$ then follows from the fact that pointwise product of measurable functions is measurable. 
    For $\omega$-continuity we have:
    \begin{align*}
        & \wp{\OBSERVE{\guard}}{\sup_i \ex_i} \\ 
        & \eeq \iv{\guard} \cdot \sup_i \ex_i \tag{definition of $\wpSymb$} \\
        & \eeq \sup_i \left( \iv{\guard} \cdot \ex_i \right) \tag{$\sup$ defined pointwise} \\
        & \eeq \sup_i \wp{\OBSERVE{\guard}}{\ex_i} \tag{Definition of $\wpSymb$}
    \end{align*}
    where, again, we are using the properties of the supremum taken with respect to pointwise ordering as in \cite{DBLP:conf/setss/SzymczakK19}.  The same holds for $\omega$-cocontinuity of $\wlpSymb$ using the properties of the infimum taken with respect to pointwise ordering. 
    
    
    \inductionCase{$\prog = \ITE{\guard}{\prog_1}{\prog_2}$}
    By I.H.\ we know that $\wp{\prog_1}{\ex}$ and $\wp{\prog_2}{\ex}$ are measurable and $\omega$-continuous. Then, measurability of $\wp{\ITE{\guard}{\prog_1}{\prog_2}}{\ex} = \iv{\guard} \cdot \wp{\prog_1}{\ex} + \iv{\neg \guard} \cdot \wp{\prog_2}{\ex}$ follows from the fact that pointwise sum and product of measurable functions is measurable.        
    For $\omega$-continuity we have:
    \begin{align*}
        & \wp{\ITE{\guard}{\prog_1}{\prog_2}}{\sup_i \ex_i} \\
        & \eeq \iv{\guard} \cdot \wp{\prog_1}{\sup_i \ex_i} + \iv{\neg \guard} \cdot \wp{\prog_2}{\sup_i \ex_i}  \tag{definition of $\wpSymb$} \\
        & \eeq \iv{\guard}  \cdot \sup_i \wp{\prog_1}{\ex_i} + \iv{\neg \guard} \cdot \sup_i \wp{\prog_2}{\ex_i} \tag{I.H.}  \\
        & \eeq \sup_i \iv{\guard}  \cdot \wp{\prog_1}{\ex_i} + \sup_i \iv{\neg \guard} \cdot  \wp{\prog_2}{\ex_i} \tag{pointwise ordering}  \\
        & \ggeq \sup_i \iv{\guard}  \cdot \wp{\prog_1}{\ex_i} + \iv{\neg \guard} \cdot  \wp{\prog_2}{\ex_i} \tag{properties of $\sup$}
    \end{align*}
    To prove that inverse inequality holds, we proceed by contradiction, supposing $>$ holds. Then, for some $i_1, i_2 \in \nats$ it holds:
    $$  \iv{\guard}  \cdot \wp{\prog_1}{\ex_{i_1}} + \iv{\neg \guard} \cdot  \wp{\prog_2}{\ex_{i_2}}  > \sup_i \iv{\guard} \cdot \wp{\prog_1}{\ex_i} + \iv{\neg \guard} \cdot \wp{\prog_2}{\ex_i}.$$
    However, since $\ex_i$ is a $\omega$-chain we can take $k \in \nats$ such that $k > i_i$ and $k > i_2$ for which we have
    \begin{align*}
        & \iv{\guard}  \cdot \wp{\prog_1}{\ex_k} + \iv{\neg \guard} \cdot  \wp{\prog_2}{\ex_k} \\
        & \ge \iv{\guard}  \cdot \wp{\prog_1}{\ex_{i_1}} + \iv{\neg \guard} \cdot  \wp{\prog_2}{\ex_{i_2}} \tag{monotonicity of $\wpTrans{\prog_1}, \wpTrans{\prog_2}$} \\
        & > \sup_i \iv{\guard}  \cdot \wp{\prog_1}{\ex_i} + \iv{\neg \guard} \cdot  \wp{\prog_2}{\ex_i} \tag{hypothesis} 
    \end{align*}
    which is a contradiction since
    $$\iv{\guard}  \cdot \wp{\prog_1}{\ex_k} + \iv{\neg \guard} \cdot  \wp{\prog_2}{\ex_k} \le \sup_i \iv{\guard}  \cdot \wp{\prog_1}{\ex_i} + \iv{\neg \guard} \cdot  \wp{\prog_2}{\ex_i}.$$
    For $\wlpSymb$ both measurability and $\omega$-cocontinuity can be proved using the same arguments.
    
    
    \inductionCase{$\prog = \PCHOICE{\prog_1}{\prob}{\prog_2}$}
    As in the previous case, replacing $\iv{\guard}$ with $\prob$ and $\iv{\neg \guard}$ with $1 - \prob$.
    
    
    \inductionCase{$\prog = \SEQ{\prog_1}{\prog_2}$}
    Measurability of $\wp{\SEQ{\prog_1}{\prog_2}}{\ex}$ and $\wlp{\SEQ{\prog_1}{\prog_2}}{\exb}$ follows immediately from the I.H.. 
    For $\omega$-continuity we have:
    \begin{align*}
        & \wp{\SEQ{\prog_1}{\prog_2}}{\sup_i \ex_i} \\
        & \eeq \wp{\prog_1}{\wp{\prog_2}{\sup_i \ex_i}} \tag{definition of $\wpSymb$} \\
        & \eeq \wp{\prog_1}{\sup_i \wp{\prog_2}{\ex_i}} \tag{I.H.\ on $\wpTrans{\prog_2}$} \\
        & \eeq \sup_i \wp{\prog_1}{\wp{\prog_2}{\ex_i}} \tag{I.H.\ on $\wpTrans{\prog_1}$} \\
        & \eeq \sup_i \wp{\SEQ{\prog_1}{\prog_2}}{\ex_i} \tag{definition of $\wpSymb$}
    \end{align*}
    where $\wp{\prog_1}{\sup_i \wp{\prog_2}{\ex_i}}$ is well-defined since by I.H. $\wp{\prog_2}{\ex_i}$ is measurable for every $i$ and therefore $\sup_i \wp{\prog_2}{\ex_i}$ is measurable as well.
    
    
    \inductionCase{$\prog = \WHILE{\guard}{\progBody}$}
    \begin{align*}
        & \wp{\prog}{\sup_i \ex_i} \\
        & = \slfp{\lam{g}{\exlfp{\progBody}{\sup_i \ex_i}{\guard}(g)}} = \tag{definition of $\wpSymb$}\\
        & = \sup_n \left( \exlfp{\progBody}{\sup_i \ex_i}{\guard} \right)^n(\zerofun) = \tag{I.H. continuity of $\wpTrans{\progBody}$}  \\
        & = \sup_n \left( \sup_i \exlfp{\progBody}{\ex_i}{\guard} \right)^n(\zerofun) = \tag{continuity of $\lambda \ex . \iv{\neg \guard}\ex + \iv{\guard}\wp{\progBody}{\exb}$} \\
        & = \sup_n \sup_i \left(\exlfp{\progBody}{\ex_i}{\guard} \right)^n(\zerofun) = \tag{$*$} \\
        & = \sup_i \sup_n (\exlfp{\progBody}{\ex_i}{\guard})^n(\zerofun) = \tag{$**$} \\
        & = \sup_i \slfp{\lam{g}{\exlfp{\progBody}{\ex_i}{\guard}(g)}} = \tag{I.H. continuity of $\wpTrans{\progBody}$} \\
        &= \sup_i \wp{\prog}{\ex_i}. \tag{definition of $\wpSymb$}
    \end{align*}
    where $(*)$ is justified by proving with an inner induction on $n$ that:
    $$ \left(\sup_i \exlfp{\prog}{\ex_i}{\guard} \right)^n(\zerofun) = \sup_i \left( \exlfp{\prog}{\ex_i}{\guard}\right)^n(\zerofun).$$
    In fact, for $n=1$ the equality holds trivially, and assuming it holds for $n-1$ we get:
    \begin{align*}
        \text{L.H.S.} & = \left( \sup_i \exlfp{\prog}{\ex_i}{\guard} \right)\left(\sup_j \exlfp{\prog}{\ex_j}{\guard} \right)^{n-1}(\zerofun) =  \\
        & = \left( \sup_i \exlfp{\prog}{\ex_i}{\guard} \right)\left(\sup_j \left(\exlfp{\prog}{\ex_j}{\guard} \right)^{n-1}(\zerofun)\right) = \tag{case $n-1$} \\
        & = \sup_j \left( \sup_i \exlfp{\prog}{\ex_i}{\guard} \right)\left(\left(\exlfp{\prog}{\ex_j}{\guard} \right)^{n-1}(\zerofun)\right) = \tag{outer I.H.} \\
        & = \sup_j \sup_i \left( \exlfp{\prog}{\ex_i}{\guard} \right)\left(\left(\exlfp{\prog}{\ex_j}{\guard} \right)^{n-1}(\zerofun)\right) = \tag{case $n=1$}\\
        & = \sup_i \left( \exlfp{\prog}{\ex_i}{\guard} \right)\left(\left(\exlfp{\prog}{\ex_i}{\guard} \right)^{n-1}(\zerofun)\right) = \tag{Lemma \ref{thm:oneVsTwoSups}}  \\
        & = \sup_i \left(\left(\exlfp{\prog}{\ex_i}{\guard} \right)^{n}(\zerofun)\right) = \text{R.H.S.}.
    \end{align*}
    For $(**)$ switching the order of suprema is justified by the fact that both terms exists.
    
    To prove co-continuity of $\wlpSymb$ we can use the same arguments: $(*)$ and $(**)$ still hold if $\sup$ is replaced by $\inf$, since Lemma \ref{thm:oneVsTwoSups} still holds and the order of infima can be switched if both terms of the equality exist.
\end{proof}
\subsection{Proof of \Cref{thm:monriemann}}
\label{proof:monriemann}
%
\monriemann*
%
\begin{proof}
    We will show that the statement holds for $\lwpTrans{\N}{\prog}$ as the proof in the remaining cases is analogous. We proceed by structural induction on $\prog$ . For $\prog \in \{ \SKIP, \DIVERGE, \ASSIGN{\pVar}{\aExp}, \OBSERVE{\guard} \}$ $\lwpTrans{\N}{\prog}$ is equal to $\wpTrans{\prog}$ so the conclusion holds by monotonicity of the latter \cite{DBLP:series/mcs/McIverM05}. 
    
    For $\prog = \UNIFASSIGN{\pVar}$ suppose $\ex \eleq \exb$, then:
    \begin{align*}
        & \lwp{\N}{\prog}{\ex} \\
        \eeq & \frac{1}{\N} \sum_{i=0}^{\N-1} \inf_{\xi \in [\frac{i}{\N}, \frac{i+1}{\N}]} \exSubs{\ex}{\pVar}{\xi} \tag{definition of $\lwpSymb{\N}$}  \\
        \eleq & \frac{1}{\N} \sum_{i=0}^{\N-1} \inf_{\xi \in [\frac{i}{\N}, \frac{i+1}{\N}]} \exSubs{\exb}{\pVar}{\xi} \tag{$\ex \eleq \exb$} \\
        \eeq & \lwp{\N}{\prog}{\ex} \tag{definition of $\lwpSymb{\N}$}
    \end{align*}
    
    For $\prog \in \{ \ITE{\guard}{\prog_1}{\prog_2}, \PCHOICE{\prob}{\prog_1}{\prog_2}, \SEQ{\prog_1}{\prog_2} \}$ follows by I.H. on $\prog_1$ and $\prog_2$.
    
    For $\prog = \WHILE{\guard}{\progBody}$ consider the characteristic functions $\charfunuwp{\N}{\prog}{\ex}$ and $\charfunuwp{\N}{\prog}{\exb}$.
    Let us denote by $Y_\ex$ and $Y_\exb$ the least fixed points of $\charfunuwp{\N}{\prog}{\ex}$ and $\charfunuwp{\N}{\prog}{\exb}$ respectively. From I.H. it follows that $\charfunuwp{\N}{\prog}{\ex}(Y) \eleq \charfunuwp{\N}{\prog}{\exb}(Y)$ for all $Y$. Then, $ \charfunuwp{\N}{\prog}{\ex}(Y_\exb) \eleq \charfunuwp{\N}{\prog}{\exb}(Y_\exb) = Y_\exb.$ By Theorem \ref{thm:knasterTarski} the previous implies $Y_\ex \eleq Y_\exb$.
\end{proof}

\subsection{Counter-example to $\omega$-continuity of $\lwpSymb{\N}$}
\label{proof:lwpNotOmegaCont}

Consider $\prog = \UNIFASSIGN{\pVar}$ and the sequence of functions $\ex_i = \onefun - \sum_{k=i}^\infty \iv{\pVar = \frac{1}{k}}$. Then $\sup_i \ex_i = \onefun$ and therefore $\lwpTrans{\N}{\prog}\left( \sup_i \ex_i \right) = \onefun. $
On the other hand,
\begin{align*}
    & \sup_i \lwpTrans{\N}{\prog}(\ex_i) \\
    \eeq & \sup_i \frac{1}{\N} \sum_{j=0}^{\N -1} \inf_{\xi \in \left[ \frac{j}{\N}, \frac{j+1}{\N} \right]} \exSubs{\ex_i}{\pVar}{\xi} \tag{definition of $\lwpSymb{\N}$} \\
    \eeq & \sup_i \left( \frac{1}{\N} \inf_{\xi \in \left[ \frac{0}{\N}, \frac{1}{\N} \right]} \exSubs{\ex_i}{\pVar}{\xi} + \frac{1}{\N} \sum_{j=0}^{\N -1} \inf_{\xi \in \left[ \frac{j}{\N}, \frac{j+1}{\N} \right]} \exSubs{\ex_i}{\pVar}{\xi} \right) \tag{splitting the summation} \\
    \le & \sup_i \left( \frac{\N - 1}{\N} \right) \tag{$\inf_{\xi \in \left[ \frac{0}{\N}, \frac{1}{\N} \right]} \exSubs{\ex_i}{\pVar}{\xi} = 0 \, \forall i$} \\
    \eeq & \frac{\N-1}{\N} \tag{sup of a constant sequence}
\end{align*}
%\end{proof}

\subsection{Proof of \Cref{thm:soundness}}
\label{proof:soundness}
%
\soundness*
%
\begin{proof}
    We prove the statement for $\lwpTrans{\N}{\prog}$ since the remaining cases are analogous.
    
    We proceed by structural induction on $\prog$.
    
    For $\prog \in \{ \SKIP, \DIVERGE, \ASSIGN{\pVar}{\aExp}, \OBSERVE{\guard} \}$ the definitions of $\lwpTrans{\N}{\prog}$ and $\wpTrans{\prog}$ coincide so pointwise inequality holds trivially.
    
    For $\prog = \UNIFASSIGN{\pVar}$ we have
    \begin{align*}
        & \lwp{\N}{\prog}{\ex} \\
        \eeq & \frac{1}{\N} \sum_{i=0}^{\N-1} \inf_{\xi \in \left[ \frac{i}{\N}, \frac{i+1}{\N} \right]} \exSubs{\ex}{\pVar}{\xi} \tag{definition of $\lwpSymb{\N}$} \\
        \eeq & \sum_{i=0}^{\N -1} \int_{\frac{i}{\N}}^{\frac{i+1}{\N}}  \inf_{\xi \in \left[ \frac{i}{\N}, \frac{i+1}{\N} \right]} \exSubs{\ex}{\pVar}{\xi} \, d\lebmes(\xi) \tag{integral of constant function} \\
        \eleq & \sum_{i=0}^{\N - 1} \int_{\frac{i}{\N}}^{\frac{i+1}{\N}} \exSubs{\ex}{\pVar}{\xi} \, d\lebmes{\xi} \tag{monotonicity of integral} \\
        \eeq & \int_0^1 \exSubs{\ex}{\pVar}{\xi} \, d\lebmes(\xi) \tag{elementary property of integral} \\
        \eeq & \wp{\prog}{\ex}. \tag{definition of $\wpSymb$}
    \end{align*}
    
    
    For $\prog \in \{ \ITE{\guard}{\prog_1}{\prog_2}, \PCHOICE{\prob}{\prog_1}{\prog_2} \}$ conclusion follows applying the I.H. on $\prog_1$ and $\prog_2$.
    
    For $\prog = \SEQ{\prog_1}{\prog_2}$ we have
    \begin{align*}
        & \lwp{\N}{\prog}{\ex} = \\
        \eeq & \lwp{\N}{\prog_1}{\lwp{\N}{\prog_2}{\ex}} \tag{definition of $\lwpSymb{\N}$} \\
        \eleq \, & \lwp{\N}{\prog_1}{\wp{\prog_2}{\ex}} \tag{I.H. on $\prog_2$ + Lemma \ref{thm:monriemann}} \\
        \eleq \, & \wp{\prog_1}{\wp{\prog_2}{\ex}} \tag{I.H. on $\prog_1$} \\
        \eeq & \wp{\prog}{\ex}. \tag{definition of $\wpSymb$}
    \end{align*}
    
    Finally, for $\prog = \WHILE{\guard}{\progBody}$ consider the characteristic functions $\charfunwp{\prog}{\ex}$ and $\charfunlwp{\N}{\prog}{\ex}$.
    Let $Y^*$ and $Y^*_\N$ be the least fixed points of $\charfunwp{\prog}{\ex}$ and $\charfunlwp{\N}{\prog}{\ex}$ respectively.
    From I.H. on $\progBody$ it follows $\charfunlwp{\N}{\prog}{\ex}(Y) \eleq \charfunwp{\prog}{\ex}(Y)$ for all $Y$.
    Then, $\charfunlwp{\N}{\prog}{\ex}(Y^*) \eleq \charfunwp{\prog}{\ex}(Y^*) = Y^*$ and from Theorem \ref{thm:knasterTarski} it follows $Y^*_\N \eleq Y^*$.
\end{proof}


\subsection{Proof of \Cref{thm:powerTwoMonotonic}}
\label{proof:powerTwoMonotonic}

\begin{restatable}{lemma}{powerTwoMonotonic}
    \label{thm:powerTwoMonotonic}
    For all programs $\prog \in \pWhile$ and post-expectations $\ex \in \exps$, $\lwp{2^\N}{\prog}{\ex}$ and $\uwp{2^\N}{\prog}{\ex}$ are non-increasing (non-decreasing, respectively) sequences in $\N \geq 1$.
\end{restatable}
\begin{proof}
    Let $\N \geq 1$ be fixed.
    We show by induction on the structure of $\prog$ that for all $\ex \in \exps$ it holds that $\lwp{2^\N}{\prog}{\ex} \leq \lwp{2^{\N+1}}{\prog}{\ex}$.
    The base case $\prog = \UNIFASSIGN{\pVar}$ follows with \Cref{thm:partitionRefine} because the partition $0 < \frac{1}{2^\N} < \frac{2}{2^\N} \ldots  < 1$ is refined by $0 < \frac{1}{2^{\N+1}} < \frac{2}{2^{\N+1}} = \frac{1}{2^\N} < \ldots < 1$.
    For the other base cases there is nothing to show.
    The inductive cases are straightforward using monotonicity of $\lwpTrans{2^\N}{\prog}$.
    We exemplify this for sequential composition:
    \begin{align*}
        & \lwp{2^\N}{\SEQ{\prog_1}{\prog_2}}{\ex} \\
        \eeq & \lwp{2^\N}{\prog_1}{\lwp{2^\N}{\prog_2}{\ex}} \tag{definition of $\lwpSymb{2^\N}$} \\
        \eeleq & \lwp{2^\N}{\prog_1}{\lwp{2^{\N+1}}{\prog_2}{\ex}} \tag{I.H. on $\prog_2$ and monotonicity of $\lwpTrans{2^\N}{\prog_1}$ (\Cref{thm:monriemann})} \\
        \eeleq & \lwp{2^{\N+1}}{\prog_1}{\lwp{2^{\N+1}}{\prog_2}{\ex}} \tag{I.H. on $\prog_1$} \\
        \eeq & \lwp{2^{\N+1}}{\SEQ{\prog_1}{\prog_2}}{\ex} \tag{definition of $\lwpSymb{2^{\N+1}}$} ~.
    \end{align*}
    The argument for $\uwpSymb{}$ is similar.
\end{proof}

\subsection{Proof of \Cref{thm:wpLoopFreeCocontLocBounded}}
\label{proof:wpLoopFreeCocontLocBounded}

Let $\expsmeaslb$ be the set of measurable locally bounded expectations.

\begin{restatable}{lemma}{wpLoopFreeCocontLocBounded}
    \label{thm:wpLoopFreeCocontLocBounded}
    Let $\aExps$ be a set of locally bounded measurable arithmetic expressions (\enquote{right hand sides}) and $\guards$ be an arbitrary set of guards.
    Let $\prog \in \pWhileWith{\aExps}{\guards}$ be loop-free.
    Then $\wpTrans{\prog}$ preserves $\expsmeaslb$.
    Moreover, $\wpTrans{\prog}$ preserves limits of non-increasing sequences from $\expsmeaslb$.
\end{restatable}
\begin{proof}
    By induction on the structure of $\prog$.
    Let $\ex \in \expsmeaslb$ be locally bounded and let $\ex_0 \egeq \ex_1 \egeq \ldots$ be a non-increasing sequence of expectations from $\expsmeaslb$.
    
    \inductionCase{$\prog = \SKIP$}
    Trivial.
    
    \inductionCase{$\prog = \DIVERGE$}
    Trivial.
    
    \inductionCase{$\prog = \OBSERVE{\guard}$}
    Straightforward.
    
    \inductionCase{$\prog = \ASSIGN{\pVar}{\aExp}$}
    We have to show that $\wp{\ASSIGN{\pVar}{\aExp}}{\ex} = \exSubsGen = \lam{\pSt}{\ex(\pStUpdate{\pSt}{\pVar}{\aExp(\pSt)})}$ is locally bounded, i.e.,
    that
    $\sup_{\pSt \in \compactSet} \ex(\pStUpdate{\pSt}{\pVar}{\aExp(\pSt)}) < \infty$
    for all compact $\compactSet \subseteq \pStates$.
    
    Let us fix a compact $\compactSet \subseteq \pStates$.
    Since $\aExp$ is locally bounded by assumption, we have $\aExp(\compactSet) \subseteq [0,b]$ for some $b \in \nonNegReals$.
    Now let $\compactSet' \supseteq \compactSet$ be a compact set such that for all $\pSt \in \compactSet$ and $\xi \in [0,b]$ we have $\pStUpdate{\pSt}{\pVar}{\xi} \in \compactSet'$. To see that $\compactSet'$ exists consider the projection of $\compactSet$ on $\mathbb{R}^{\pVars \setminus \{ \pVar \}}$, denoted by $\pi(\compactSet)$. By Theorem 39 in \cite{pugh2002real} $\pi(\compactSet)$ is compact. Then set $\compactSet' = \pi(\compactSet) \times [0,b]$. By Theorem 31 in \cite{pugh2002real} $\compactSet'$ is compact and $\pStUpdate{\pSt}{\pVar}{\xi} \in \compactSet'$ for all $\pSt \in \compactSet$ and $\xi \in [0,b]$.
    
    Then we have
    \[
    \{\ex(\pStUpdate{\pSt}{\pVar}{\aExp(\pSt)}) \mid \pSt \in \compactSet\}
    ~\subseteq~
    \{\ex(\pStUpdate{\pSt}{\pVar}{\xi}) \mid \pSt \in \compactSet, \xi \in [0,b]\}
    ~\subseteq~
    \{\ex(\pSt) \mid \pSt \in \compactSet'\}
    \]
    and thus $\sup_{\pSt \in \compactSet} \ex(\pStUpdate{\pSt}{\pVar}{\aExp(\pSt)}) \leq \sup_{\pSt \in \compactSet'} \ex(\pSt) < \infty$ because $\ex$ is locally bounded.
    
    \inductionCase{$\prog = \UNIFASSIGN{\pVar}$}
    We first show that
    \[
    \wp{\UNIFASSIGN{\pVar}}{\ex} = \lam{\st}{\int_\uIval \ex(\pStUpdate{\st}{\pVar}{\xi}) \,d\lebmes(\xi)}
    \]
    remains locally bounded.
    To this end let $\compactSet \subseteq \pStates$ be compact.
    Recall that we have to show that $\sup \wp{\UNIFASSIGN{\pVar}}{\ex}(\compactSet) < \infty$.
    Let $\compactSet' \supseteq \compactSet$ be a compact set such that for all $\pSt \in \compactSet$ and $\xi \in \uIval$ we have $\pStUpdate{\pSt}{\pVar}{\xi} \in \compactSet'$, where $\compactSet'$ can be constructed as above.
    Since $\ex$ is locally bounded we have $\sup \ex(\compactSet') = b \in \nonNegReals$.
    Hence for all $\pSt \in \compactSet'$,
    \begin{align*}
        & \int_\uIval \ex(\pStUpdate{\st}{\pVar}{\xi}) \,d\lebmes(\xi) \\
        \lleq & \int_\uIval \sup \ex(\compactSet') \,d\lebmes(\xi) \\
        \eeq & \int_\uIval b \,d\lebmes(\xi) \\
        \eeq & b 
    \end{align*}
    It follows that $\sup \wp{\UNIFASSIGN{\pVar}}{\ex}(\compactSet) \leq \sup \wp{\UNIFASSIGN{\pVar}}{\ex}(\compactSet') \leq b < \infty$ and hence the claim.
    
    The fact that $\wpSymb$ preserve infima of non-increasing sequences follows from the properties of pointwise $\inf$.
    
    The fact that $\wpTrans{\UNIFASSIGN{\pVar}}$ preserves infima of non-increasing sequences $( \ex_i )$ follows from an extended version of the Monotone Convergence \cite[Theorem 12.1 (ii)]{schilling2017measures} adapted to non-increasing sequences of functions. Notably, the corollary requires that the integral of the first function in the sequence is smaller than $\infty$, but this is guaranteed by the restriction to locally bounded expectations, since for every $\pSt \in \pStates$ we can consider the compact set $\compactSet_\pSt = \{\pStUpdate{\pSt}{\pVar}{\xi} \mid \xi \in \uIval\}$, and since $\ex_0$ is locally bounded, we find that $\int_\uIval \ex_0(\pStUpdate{\st}{\pVar}{\xi}) \,d\lebmes(\xi) \leq \sup \compactSet_\pSt < \infty$.
    
    \inductionCase{$\prog = \ITE{\guard}{\prog_1}{\prog_2}$}
    Local boundedness follows by applying the I.H.\ to $\prog_1$ and $\prog_2$, while preservation of the limit can be proved using the same argument as in the proof \Cref{thm:wpWlpWellDefinedAndContinuous}.
    
    \inductionCase{$\prog = \PCHOICE{\prog_1}{\prob}{\prog_2}$}
    Local boundedness follows by applying the I.H.\ to $\prog_1$ and $\prog_2$, while preservation of the limit can be proved using the same argument as in the proof \Cref{thm:wpWlpWellDefinedAndContinuous}.
    
    \inductionCase{$\prog = \SEQ{\prog_1}{\prog_2}$}
    Since $\wp{\SEQ{\prog_1}{\prog_2}}{\ex} = \wp{\prog_1}{\wp{\prog_2}{\ex}}$ it follows immediately from the I.H.\ that $\wp{\prog_1}{\wp{\prog_2}{\ex}} \in \expsmeaslb$.
    
    To conclude the proof observe that
    \begin{align*}
        &\inf_{i \in \nats} \wp{\SEQ{\prog_1}{\prog_2}}{\ex_i} \\
        \eeq &\inf_{i \in \nats} \wp{\prog_1}{\wp{\prog_2}{\ex_i}} \\
        \eeq & \wp{\prog_1}{\inf_{i \in \nats} \wp{\prog_2}{\ex_i}} \tag{I.H.\ on $\prog_1$, using that $\wp{\prog_2}{\ex_i}$ is a non-decreasing sequence in $\expsmeaslb$ by I.H.\ on $\prog_2$} \\
        \eeq & \wp{\prog_1}{\wp{\prog_2}{\inf_{i \in \nats}  \ex_i}} \tag{I.H.\ on $\prog_2$} \\
        \eeq & \wp{\SEQ{\prog_1}{\prog_2}}{\inf_{i \in \nats}  \ex_i}
    \end{align*}
\end{proof}


\subsection{Proof of \Cref{thm:convLoopFree}}
\label{proof:convLoopFree}
%
\convLoopFree*
%
\begin{proof}
    In the rest of the proof we write  $\sup_\N$ and $\inf_\N$ instead of $\sup_{\N \geq 1}$ and $\inf_{\N \geq 1}$, respectively.
    We will show the following \emph{alternative} statement:
    For all loop-free $\prog \in \pWhileWith{\aExps}{\guards}$ and post-expectations $\ex \in \expsClass$:
    \eqref{eq:approxInClass} from above holds, and
    %
    \begin{align}
        \wp{\prog}{\ex}
        \eeq
        \sup_\N \lwp{2^\N}{\prog}{\ex}
        \eeq
        \inf_\N \uwp{2^\N}{\prog}{\ex}
        ~.
        \tag{\ref{eq:convLoopFree}'}
        \label{eq:convLoopFreeAlt}
    \end{align}
    %
    Note that the difference between \eqref{eq:convLoopFreeAlt} and $\eqref{eq:convLoopFree}$
    is that we consider the supremum over powers of two.
    This simplifies the situation as $\lwp{2^\N}{\prog}{\ex}$ and $\uwp{2^\N}{\prog}{\ex}$ are non-decreasing (non-increasing, respectively) sequences in $\N$ (see \Cref{thm:powerTwoMonotonic}), which allows us to apply \Cref{thm:oneVsTwoSups} at a crucial point in the proof.
    
    To show that the alternative statement implies the claim of \Cref{thm:convLoopFree} we argue as follows:
    By properties of suprema, $\sup_\N \lwp{2^\N}{\prog}{\ex} \leq \sup_\N \lwp{\N}{\prog}{\ex}$.
    We also have
    %
    \begin{align*}
        & \sup_\N \lwp{\N}{\prog}{\ex} \\
        %
        \lleq & \wp{\prog}{\ex} \tag{by \Cref{thm:soundness}} \\
        %
        \eeq & \sup_\N \lwp{2^\N}{\prog}{\ex} \tag{by the alternative statement \eqref{eq:convLoopFreeAlt}} ~,
    \end{align*}
    %
    and thus $\sup_\N \lwp{2^\N}{\prog}{\ex} = \sup_\N \lwp{\N}{\prog}{\ex}$.
    The argument for $\inf_\N \uwp{2^\N}{\prog}{\ex} = \inf_\N \uwp{\N}{\prog}{\ex}$ is analogous.
    
    The proof of the alternative statement \eqref{eq:convLoopFreeAlt} is by induction over the structure of $\prog$.
    In the following, we let $\ex \in \expsClass$ and $\N \in \nats$ be arbitrary.
    
    We consider the base cases first.
    For the base cases $\prog \in \{\SKIP, \DIVERGE, \ASSIGN{\pVar}{\aExp}, \OBSERVE{\guard} \}$ the definitions of $\lwpSymb \N$, $\uwpSymb \N$, and $\wpSymb$ coincide, so we only have to show that $\wp{\prog}{\ex} \in \expsClass$.
    
    
    \inductionCase{$\prog = \SKIP$}
    %
    By definition, $\wp \SKIP \ex = \ex \in \expsClass$.
    
    \inductionCase{$\prog = \DIVERGE$}
    %
    By definition, $\wp \DIVERGE \ex = 0$.
    By assumption, $0 \in \expsClass$.
    
    \inductionCase{$\prog = \ASSIGN{\pVar}{\aExp}$}
    %
    By definition, $\wp{\ASSIGN{\pVar}{\aExp}}{f} = \exSubs{\ex}{\pVar}{\aExp}$.
    By assumption, $\exSubs{\ex}{\pVar}{\aExp} \in \expsClass$.
    
    \inductionCase{$\prog = \OBSERVE{\guard}$}
    %
    By definition, $\wp{\OBSERVE{\guard}}{f} = \iv{\guard} \cdot \ex$.
    By assumption, $\iv{\guard} \cdot \ex + \iv{\neg\guard} \cdot 0 \in \expsClass$.
    
    \inductionCase{$\prog = \UNIFASSIGN{\pVar}$}
    %
    This base case is more interesting and crucially relies on Riemann-integrability.
    We first show that $\lwp{\N}{\UNIFASSIGN{\pVar}}{\ex} \in \expsClass$.
    By definition of $\lwpSymb{\N}$,
    \begin{align*}
        \lwp{\N}{\UNIFASSIGN{\pVar}}{\ex}
        \eeq
        \frac{1}{\N} \sum_{i=0}^{\N-1} \inf_{\xi \in [\frac{i}{\N}, \frac{i+1}{\N}]} \exSubs{\ex}{\pVar}{\xi}
        ~.
    \end{align*}
    The right-hand side is a finite convex combination of suprema over subintervals of $\uIval$, hence it is in $\expsClass$ by assumption.
    Next we show
    $\sup_\N \lwp{2^\N}{\UNIFASSIGN{\pVar}}{\ex} = \wp{\UNIFASSIGN{\pVar}}{\ex}$:
    \begin{align*}
        & \sup_\N \lwp{2^\N}{\UNIFASSIGN{\pVar}}{\ex} \\
        %
        \eeq & \sup_N \frac{1}{2^\N} \sum_{i=0}^{2^\N-1} \inf_{\xi \in [\frac{i}{2^\N}, \frac{i+1}{2^\N}]}
        \exSubs{\ex}{\pVar}{\xi} \tag{definition of $\lwpSymb{2^\N}$} \\
        %
        \eeq & \lam{\st}{} \lowerInt{0}{1} \ex(\pStUpdate{\st}{\pVar}{\xi}) \,d\xi \tag{by \Cref{thm:smallNormSuffices}} \\
        %
        \eeq & \lam{\st}{} \int_{0}^{1} \ex(\pStUpdate{\st}{\pVar}{\xi}) \,d\xi \tag{by assumption $\ex \in \expsClass$ is bounded and Riemann-integrable w.r.t.\ $\pVar$} \\
        %
        \eeq & \lam{\st}{} \int_\uIval  \ex(\pStUpdate{\st}{\pVar}{\xi}) \,d\lebmes(\xi) \tag{Riemann integral = Lebesgue integral, see \Cref{thm:riemannEqualsLebesgue}} \\
        %
        \eeq & \wp{\UNIFASSIGN{\pVar}}{\ex} \tag{definition of $\wpSymb$}
    \end{align*}
    The proof for $\uwpSymb{}$ is exactly analogous (replace all infima by suprema and vice versa).
    
    \inductionCase{$\prog = \ITE{\guard}{\prog_1}{\prog_2}$}
    %
    $\lwp{\N}{\ITE{\guard}{\prog_1}{\prog_2}}{\ex}
    = [\guard] \cdot \lwp{\N}{\prog_1}{\ex} + [\neg\guard] \cdot \lwp{\N}{\prog_2}{\ex} \in \expsClass$ by I.H. on $\prog_1$ and $\prog_2$ and assumption.
    Furthermore,
    \begin{align*}
        %
        & \sup_\N \lwp{2^\N}{\ITE{\guard}{\prog_1}{\prog_2}}{\ex} \\
        %
        \eeq & \sup_\N \left(\iv{\guard} \cdot \lwp{2^\N}{\prog_1}{\ex} + \iv{\neg\guard} \cdot \lwp{2^\N}{\prog_2}{\ex}\right) \tag{definition of $\lwpSymb{\N}$} \\
        %
        \eeq & \iv{\guard} \cdot \sup_\N \lwp{2^\N}{\prog_1}{\ex} + \iv{\neg\guard} \cdot \sup_N\lwp{2^\N}{\prog_2}{\ex} \tag{addition and multiplication by Iverson brackets is $\omega$-continuous} \\
        %
        \eeq & \iv{\guard} \cdot \wp{\prog_1}{\ex} + \iv{\neg\guard} \cdot \wp{\prog_2}{\ex} \tag{by I.H. on $\prog_1$ and $\prog_2$} \\
        %
        \eeq & \wp{\ITE{\guard}{\prog_1}{\prog_2}}{\ex} \tag{definition of $\wpSymb$}
        ~.
    \end{align*}
    %
    Again, the proof for $\uwpSymb{}$ is analogous (consider infima instead of suprema, and use that addition and multiplication by Iverson brackets is $\omega$-\underline{co}continuous).
    
    \inductionCase{$\prog = \PCHOICE{\prog_1}{\prob}{\prog_2}$}
    %
    $\lwp{\N}{\PCHOICE{\prog_1}{\prob}{\prog_2}}{\ex}
    = \prob \cdot \lwp{\N}{\prog_1}{\ex} + (1-\prob) \cdot \lwp{\N}{\prog_2}{\ex} \in \expsClass$ by I.H. on $\prog_1$ and $\prog_2$ and assumption.
    Furthermore,
    %
    \begin{align*}
        & \sup_\N \lwp{2^\N}{\PCHOICE{\prog_1}{\prob}{\prog_2}}{\ex} \\
        %
        \eeq & \sup_\N \left(\prob \cdot \lwp{2^\N}{\prog_1}{\ex} + (1-\prob) \cdot \lwp{2^\N}{\prog_2}{\ex}\right) \tag{definition of $\lwpSymb{\N}$} \\
        %
        \eeq & \prob \cdot \sup_\N \lwp{2^\N}{\prog_1}{\ex} + (1-\prob) \cdot \sup_\N\lwp{2^\N}{\prog_2}{\ex} \tag{addition and multiplication by constants is $\omega$-continuous} \\
        %
        \eeq & \prob \cdot \wp{\prog_1}{\ex} + (1-\prob) \cdot \wp{\prog_2}{\ex} \tag{by I.H. on $\prog_1$ and $\prog_2$} \\
        %
        \eeq & \wp{\PCHOICE{\prog_1}{\prob}{\prog_2}}{\ex} \tag{definition of $\wpSymb$}
        ~.
    \end{align*}
    %
    As before, the proof for $\uwpSymb{}$ is analogous (consider infima instead of suprema, and use that addition and multiplication by constants is $\omega$-\underline{co}continuous).
    
    \inductionCase{$\prog = \SEQ{\prog_1}{\prog_2}$}
    First, note that $\lwp{\N}{\SEQ{\prog_1}{\prog_2}}{\ex} = \lwp{\N}{\prog_1}{\lwp{\N}{\prog_2}{\ex}} \in \expsClass$ by I.H. on $\prog_2$ and $\prog_1$.
    For convergence we argue as follows:
    %
    \begin{align*}
        & \sup_\N \lwp{2^\N}{\SEQ{\prog_1}{\prog_2}}{\ex} \\
        %
        \eeq & \sup_\N \lwp{2^\N}{\prog_1}{\lwp{2^\N}{\prog_2}{\ex}}  \tag{definition of $\lwpSymb{\N}$} \\
        %
        \eeq & \sup_{\Nb} \sup_\N \lwp{2^\N}{\prog_1}{\lwp{2^\Nb}{\prog_2}{\ex}} \tag{\Cref{thm:oneVsTwoSups} and \Cref{thm:powerTwoMonotonic}} \\
        %
        \eeq & \sup_{\Nb} \wp{\prog_1}{\lwp{2^\Nb}{\prog_2}{\ex}} \tag{I.H.\ on $\prog_1$, using that $\lwp{2^\Nb}{\prog_2}{\ex} \in \expsClass$ (also by I.H.)} \\
        %
        \eeq & \wp{\prog_1}{\sup_{\Nb} \lwp{2^\Nb}{\prog_2}{\ex}} \tag{$\omega$-continuity of $\wpTrans{\prog_1}$, see \Cref{thm:wpWlpWellDefinedAndContinuous}} \\
        %
        \eeq & \wp{\prog_1}{\wp{\prog_2}{\ex}}{\ex} \tag{I.H.\ on $\prog_2$} \\
        %
        \eeq & \wp{\SEQ{\prog_1}{\prog_2}}{\ex} \tag{definition of $\wpSymb$}
    \end{align*}
    %
    The convergence proof of $\uwpSymb{}$ is not fully analogous as $\wpSymb$ is not $\omega$-\underline{co}continuous in general.
    Instead, we invoke \Cref{thm:wpLoopFreeCocontLocBounded}:
    %
    \begin{align*}
        & \inf_\N \uwp{2^\N}{\SEQ{\prog_1}{\prog_2}}{\ex} \\
        %
        \eeq & \inf_\N \uwp{2^\N}{\prog_1}{\uwp{2^\N}{\prog_2}{\ex}}  \tag{definition of $\uwpSymb{\N}$} \\
        %
        \eeq & \inf_{\Nb} \inf_\N \uwp{2^\N}{\prog_1}{\uwp{2^\Nb}{\prog_2}{\ex}} \tag{\enquote{$\inf$-version} of \Cref{thm:oneVsTwoSups} and \Cref{thm:powerTwoMonotonic}} \\
        %
        \eeq & \inf_{\Nb} \wp{\prog_1}{\uwp{2^\Nb}{\prog_2}{\ex}} \tag{I.H.\ on $\prog_1$, using that $\uwp{2^\Nb}{\prog_2}{\ex} \in \expsClass$ (also by I.H.)} \\
        %
        \eeq & \wp{\prog_1}{\inf_{\Nb} \uwp{2^\Nb}{\prog_2}{\ex}} \tag{\Cref{thm:wpLoopFreeCocontLocBounded}} \\
        %
        \eeq & \wp{\prog_1}{\wp{\prog_2}{\ex}}{\ex} \tag{I.H.\ on $\prog_2$} \\
        %
        \eeq & \wp{\SEQ{\prog_1}{\prog_2}}{\ex} \tag{definition of $\wpSymb$}
    \end{align*}
\end{proof}

\subsection{{$\wlpSymb$-Version of \Cref{thm:convLoopFree}}}
\label{app:wlpVersion}
\begin{lemma}
    \label{thm:convLoopFreeWlp}
    Let $(\aExps, \guards, \expsClass)$ be Riemann-suitable.
    Then for all loop-free $\prog \in \pWhileWith{\aExps}{\guards}$ and post-expectations $\exb \in \expsClass  \cap \bexpsmeas$ the following holds:
    \begin{align*}
        \forall \N \geq 1 \colon\quad \lwlp{\N}{\prog}{\exb} \in \expsClass \qand \uwlp{\N}{\prog}{\exb} \in \expsClass
    \end{align*}
    and
    \begin{align*}
        \wlp{\prog}{\exb}
        \eeq
        \sup_{\N \geq 1} \lwlp{\N}{\prog}{\exb}
        \eeq
        \inf_{\N \geq 1} \uwlp{\N}{\prog}{\exb}
        ~.
    \end{align*}
\end{lemma}
\begin{proof}
    Very similar to the proof of \Cref{thm:convLoopFree}.
    A minor difference is the straightforward base case $\prog = \DIVERGE$.
    Also recall that $\wlpSymb$ only applies to 1-bounded expectations only.
\end{proof}


\subsection{Proof of \Cref{thm:convUnfolding}}
\label{proof:convUnfolding}
%
\convUnfolding*
%
\begin{proof}
    The proof is by induction on the structure of $\prog$.
    We focus on $\wpSymb$, the proof for $\wlpSymb$ is analogous.
    For the bases cases $\prog = \SKIP$, $\prog = \DIVERGE$, $\prog = \ASSIGN{\pVar}{\aExp}$, $\prog = \UNIFASSIGN{\pVar}$ and $\prog = \OBSERVE{\guard}$ there is nothing to show as these atomic programs are unaffected by unfolding.
    We now consider the composite cases.
    Let $\ex \in \expsmeas$ be arbitrary but fixed throughout the following.
    
    \inductionCase{$\prog = \ITE{\guard}{\prog_1}{\prog_2}$ and $\prog = \PCHOICE{\prog_1}{\prob}{\prog_2}$}
    %
    These cases follow straightforwardly by induction and basic properties of $\wpSymb$.
    
    \inductionCase{$\prog = \SEQ{\prog_1}{\prog_2}$}
    %
    To show that for all $\depth \in \nats$ it holds that $\wp{\unfold{\prog}{\depth}}{\ex} \eleq \wp{\unfold{\prog}{\depth+1}}{\ex}$ we argue as follows:
    %
    \begin{align*}
        & \wp{(\SEQ{\prog_1}{\prog_2})^\depth}{\ex} \\
        %
        \eeq & \wp{\prog_1^\depth}{\wp{\prog_2^\depth}{\ex}} \tag{definition of $\wpSymb$ and of $(\SEQ{\prog_1}{\prog_2})^\depth$} \\
        %
        \eeleq & \wp{\prog_1^\depth}{\wp{\prog_2^{\depth + 1}}{\ex}} \tag{I.H. on $\prog_2$, monotonicity of $\wpSymb$}\\
        %
        \eeleq & \wp{\prog_1^{\depth + 1}}{\wp{\prog_2^{\depth + 1}}{\ex}} \tag{I.H. on $\prog_1$} \\
        %
        \eeq & \wp{(\SEQ{\prog_1}{\prog_2})^{\depth + 1}}{\ex} \tag{definition of $\wpSymb$ and of $(\SEQ{\prog_1}{\prog_2})^{\depth+1}$}
    \end{align*}
    %
    To see that the supremum over all unrollings is equal to the exact $\wpSymb$ consider the following:
    %
    \begin{align*}
        & \sup_{\depth \in \nats} \wp{(\SEQ{\prog_1}{\prog_2})^\depth}{\ex} \\
        %
        \eeq & \sup_{\depth \in \nats} \wp{\prog_1^\depth}{\wp{\prog_2^\depth}{\ex}} \tag{definition of $\wpSymb$ and of $(\SEQ{\prog_1}{\prog_2})^\depth$} \\
        %
        \eeq & \sup_{\depth_1 \in \nats} \sup_{\depth_2 \in \nats} \wp{\prog_1^{\depth_1}}{\wp{\prog_2^{\depth_2}}{\ex}} \tag{\Cref{thm:oneVsTwoSups}, monotonicity of $\wpTrans{\unfold{\prog_1}{\depth_1}}$, I.H.\ on $\prog_1$ and $\prog_2$} \\
        %
        \eeq & \sup_{\depth_1 \in \nats} \wp{\prog_1^{\depth_1}}{\sup_{\depth_2 \in \nats} \wp{\prog_2^{\depth_2}}{\ex}} \tag{$\omega$-continuity of $\wpTrans{\unfold{\prog_1}{\depth_1}}$, see \Cref{thm:wpWlpWellDefinedAndContinuous}} \\
        %
        \eeq & \sup_{\depth_1 \in \nats} \wp{\prog_1^{\depth_1}}{\wp{\prog_2}{\ex}} \tag{I.H.\ on $\prog_2$} \\
        %
        \eeq & \wp{\prog_1}{\wp{\prog_2}{\ex}} \tag{I.H.\ on $\prog_1$} \\
        %
        \eeq & \wp{\SEQ{\prog_1}{\prog_2}}{\ex} \tag{definition of $\wpSymb$} 
    \end{align*}
    
    \inductionCase{$\prog = \WHILE{\guard}{\progBody}$}
    %
    We first show by (an inner) induction on $\depth \in \nats$ that
    \[
    \wp{\unfold{(\WHILE{\guard}{\progBody})}{\depth}}{\ex}
    \eeleq
    \wp{\unfold{(\WHILE{\guard}{\progBody})}{\depth + 1}}{\ex}
    ~.
    \]
    %
    For $\depth = 0$ this follows as $\wp{\unfold{(\WHILE{\guard}{\progBody})}{0}}{\ex} = \wp{\DIVERGE}{\ex} = 0$ by definition.
    %
    For $\depth > 0$:
    %
    \begin{align*}
        & \wp{\unfold{(\WHILE{\guard}{\progBody})}{\depth}}{\ex} \\
        %
        \eeq & \wp{\ITE{\guard}{\SEQ{\unfold{\progBody}{\depth-1}}{\unfold{(\WHILE{\guard}{\progBody})}}{\depth-1}}{\SKIP}}{\ex} \tag{definition of unfolding} \\
        %
        \eeq & \iv{\guard} \cdot \wp{\unfold{\progBody}{\depth-1}}{\wp{\unfold{(\WHILE{\guard}{\progBody})}{\depth-1}}{\ex}} + \iv{\neg\guard} \cdot \wp{\SKIP}{\ex} \tag{definition of $\wpSymb$} \\
        %
        \eeleq & \iv{\guard} \cdot \wp{\unfold{\progBody}{\depth-1}}{\wp{\unfold{(\WHILE{\guard}{\progBody})}{\depth}}{\ex}} + \iv{\neg\guard} \cdot \wp{\SKIP}{\ex} \tag{inner I.H., monotonicity of $\wpSymb$} \\
        %
        \eeleq & \iv{\guard} \cdot \wp{\unfold{\progBody}{\depth}}{\wp{\unfold{(\WHILE{\guard}{\progBody})}{\depth}}{\ex}} + \iv{\neg\guard} \cdot \wp{\SKIP}{\ex} \tag{outer I.H. on $\progBody$} \\
        %
        \eeq & \wp{\ITE{\guard}{\SEQ{\unfold{\progBody}{\depth}}{\unfold{(\WHILE{\guard}{\progBody})}}{\depth}}{\SKIP}}{\ex} \tag{definition of $\wpSymb$} \\
        %
        \eeq & \wp{\unfold{(\WHILE{\guard}{\progBody})}{\depth+1}}{\ex} \tag{definition of unfolding}
    \end{align*}
    %
    To see that the supremum over all unrollings is equal to the exact $\wpSymb$ consider the following:
    %
    \begin{align*}
        & \sup_{\depth \in \nats} \wp{\unfold{(\WHILE{\guard}{\progBody})}{\depth}}{\ex} \\
        %
        \eeq & \sup_{\depth \in \nats} \wp{\unfold{(\WHILE{\guard}{\progBody})}{\depth+1}}{\ex} \tag{non-decreasing, see above} \\
        %
        \eeq & \sup_{\depth \in \nats} \wp{\ITE{\guard}{\SEQ{\unfold{\progBody}{\depth}}{\unfold{(\WHILE{\guard}{\progBody})}}{\depth}}{\SKIP}}{\ex} \tag{definition of unfolding} \\
        %
        \eeq & \sup_{\depth \in \nats} \left( \iv{\guard} \cdot \wp{\unfold{\progBody}{\depth}}{\wp{\unfold{(\WHILE{\guard}{\progBody})}{\depth}}{\ex}} + \iv{\neg\guard} \cdot \ex \right) \tag{definition of $\wpSymb$} \\
        %
        \eeq & \iv{\guard} \cdot \sup_{\depth \in \nats}  \wp{\unfold{\progBody}{\depth}}{\wp{\unfold{(\WHILE{\guard}{\progBody})}{\depth}}{\ex}} + \iv{\neg\guard} \cdot \ex \tag{$\omega$-continuity of $\cdot$ and $+$} \\
        %
        \eeq & \iv{\guard} \cdot \sup_{\depth_2 \in \nats} \sup_{\depth_1 \in \nats} \wp{\unfold{\progBody}{\depth_1}}{\wp{\unfold{(\WHILE{\guard}{\progBody})}{\depth_2}}{\ex}} + \iv{\neg\guard} \cdot \ex \tag{\Cref{thm:oneVsTwoSups}} \\
        %
        \eeq & \iv{\guard} \cdot \sup_{\depth_2 \in \nats}  \wp{\progBody}{\wp{\unfold{(\WHILE{\guard}{\progBody})}{\depth_2}}{\ex}} + \iv{\neg\guard} \cdot \ex \tag{I.H. on $\progBody$} \\
        %
        \eeq & \iv{\guard} \cdot  \wp{\progBody}{\sup_{\depth_2 \in \nats} \wp{\unfold{(\WHILE{\guard}{\progBody})}{\depth_2}}{\ex}} + \iv{\neg\guard} \cdot \ex \tag{$\omega$-continuity of $\wpSymb$}
    \end{align*}
    %
    By the definition of $\wpTrans{\WHILE{\guard}{\progBody}}$ in terms of a least fixed point, the above equation implies
    \[
    \wp{\WHILE{\guard}{\progBody}}{\ex}
    \eeleq
    \sup_{\depth \in \nats} \wp{\unfold{(\WHILE{\guard}{\progBody})}{\depth}}{\ex}
    ~. 
    \]
    
    To complete the proof we show by (an inner) induction on $\depth \in \nats$ that
    \[
    \wp{\unfold{(\WHILE{\guard}{\progBody})}{\depth}}{\ex}
    \eeleq
    \wp{\WHILE{\guard}{\progBody}}{\ex}
    ~.
    \]
    This implies that $\sup_{\depth \in \nats} \wp{\unfold{(\WHILE{\guard}{\progBody})}{\depth}}{\ex} \eleq \wp{\WHILE{\guard}{\progBody}}{\ex}$.
    For $\depth = 0$ this holds as $\wp{\unfold{(\WHILE{\guard}{\progBody})}{0}}{\ex} = \wp{\DIVERGE}{\ex} = 0$ by definition.
    For $\depth \geq 0$ the argument is as follows:
    %
    \begin{align*}
        & \wp{\unfold{(\WHILE{\guard}{\progBody})}{\depth+1}}{\ex}  \\
        %
        \eeq &  \wp{\ITE{\guard}{\SEQ{\unfold{\progBody}{\depth}}{\unfold{(\WHILE{\guard}{\progBody})}}{\depth}}{\SKIP}}{\ex} \tag{definition of unfolding} \\
        %
        \eeq & \iv{\guard} \cdot  \wp{\unfold{\progBody}{\depth}}{\wp{\unfold{(\WHILE{\guard}{\progBody})}{\depth}}{\ex}} + \iv{\neg\guard} \cdot \ex  \tag{definition of $\wpSymb$} \\
        %
        \eeleq & \iv{\guard} \cdot \wp{\unfold{\progBody}{\depth}}{\wp{\WHILE{\guard}{\progBody}}{\ex}} + \iv{\neg\guard} \cdot \ex  \tag{inner I.H., monotonicity of $\wpTrans{\unfold{\progBody}{\depth}}$} \\
        %
        \eeleq & \iv{\guard} \cdot \wp{\progBody}{\wp{\WHILE{\guard}{\progBody}}{\ex}} + \iv{\neg\guard} \cdot \ex  \tag{outer I.H. on $\progBody$} \\
        %
        \eeq & \wp{\WHILE{\guard}{\progBody}}{\ex} \tag{definition of $\wpSymb$}
    \end{align*}
\end{proof}


\subsection{Proof of \Cref{thm:pointwiseConv}}
\label{proof:pointwiseConv}
%
\pointwiseConv*
%
\begin{proof}
    We only show the equality involving $\wpSymb$ (the one for $\wlpSymb$ is analogous).    
    For all $n \in \nats$ we have
    \begin{align*}
        & \lwp{n}{\unfold{\prog}{n}}{\ex} \\
        \lleq & \wp{\unfold{\prog}{n}}{\ex} \tag{by \Cref{thm:soundness}} \\
        \lleq & \wp{\prog}{\ex} \tag{by \Cref{thm:convUnfolding}}
    \end{align*}
    and thus $\sup_{n \in \nats} \lwp{n}{\unfold{\prog}{n}}{\ex} \leq \wp{\prog}{\ex}$.
    But we also have
    \begin{align*}
        & \sup_{n \in \nats} \lwp{n}{\unfold{\prog}{n}}{\ex} \\
        \ggeq & \sup_{n \in \nats} \lwp{2^n}{\unfold{\prog}{2^n}}{\ex} \tag{elementary property of supremum} \\
        \eeq & \sup_{\depth \in \nats} \sup_{\N \in \nats} \lwp{2^\N}{\unfold{\prog}{2^\depth}}{\ex} \tag{by \Cref{thm:oneVsTwoSups} and \Cref{thm:powerTwoMonotonic}} \\
        \eeq & \sup_{\depth \in \nats} \wp{\unfold{\prog}{2^\depth}}{\ex} \tag{by \Cref{thm:convLoopFree}, using that $\unfold{\prog}{2^\depth}$ is loop-free and that $f \in \expsClass$} \\
        \eeq & \wp{\prog}{\ex} \tag{by \Cref{thm:convUnfolding}} ~.
    \end{align*}
\end{proof}

\subsection{Proof of \Cref{thm:expWellLocallyBounded}}
\label{proof:expWellLocallyBounded}
%
\expWellLocallyBounded*
%
\begin{proof}
    We show, by induction on the structure of $\synEx$, that $\sem{\synEx}$ is locally bounded when viewed as a function of type $\sem{\synEx} \colon \reals^{\lvarsSubset} \to \reals$.
    
    \inductionCase{$\synEx = \synTerm$}
    $\sem{\synTerm}$ is a piece-wise defined polynomial in finitely many variables, hence $\sem{\synTerm}$ is locally bounded.
    
    \inductionCase{$\synEx = \iv{\synGuard}\cdot \synEx$}
    It is $\sem{\iv{\synGuard}\cdot \synEx} \leq \sem{\synEx}$ and $\sem{\synEx}$ is locally bounded by the I.H.
    Hence $\sem{\iv{\synGuard}\cdot \synEx}$ is locally bounded, too.
    
    \inductionCase{$\synEx = \ratConst \cdot \synExb$}
    By the I.H., $\synExb$ is locally bounded, i.e., for every $\compactSet \subseteq \reals^\lvarsSubset = \reals^\lvarsSubset$ we have $\sup_{\st \in \compactSet} |\sem{\synExb}| < \infty$.
    But then we also have $\sup_{\st \in \compactSet} | \sem{\ratConst \cdot \synExb}| = \sup_{\st \in \compactSet} |\ratConst \cdot \sem{\synExb}|  = \sup_{\st \in \compactSet} |\ratConst|  \cdot | \sem{\synExb}| =  \sup_{\st \in \compactSet} \ratConst \cdot  | \sem{\synExb}| = \ratConst \cdot   \sup_{\st \in \compactSet} | \sem{\synExb}| < \infty$.
    
    \inductionCase{$\synEx = \synExb + \synExc$}
    Similar to the previous case, using I.H.\ for both $\synExb$ and $\synExc$.
    
    \inductionCase{$\synEx = \synSupBd{\lvar}{\ivalL}{\ivalR} \synExb$}
    Let $\compactSet \subseteq \reals^\lvarsSubset$ be compact.
    Let $\compactSet'$ be the (compact) cross product of $\compactSet$ and $\clIvalGen$, i.e., $\compactSet' = \{\pStUpdate{\st}{\lvar}{\xi} \in \reals^{\lvarsSubset \cup \{\lvar\}} \mid \st \in \compactSet, \xi \in \clIvalGen\}$.
    We have to show that $\sup_{\st \in \compactSet} |\sem{f}(\st)| < \infty$.
    We argue as follows:
    \begin{align*}
        & \sup_{\st \in \compactSet} \left|\sem{f}(\st) \right| \\ 
        \eeq & \sup_{\st \in \compactSet} \left| \sem{\synSupBd{\lvar}{\ivalL}{\ivalR} \synExb}(\st)  \right| \\
        \eeq & \sup_{\st \in \compactSet} \left| \sup_{\xi \in \clIvalGen}\sem{\synExb}(\pStUpdate{\st}{\lvar}{\xi})  \right| \tag{definition of $\sem{\synSupBd{\lvar}{\ivalL}{\ivalR} \synExb}$} \\
        \lleq & \sup_{\st \in \compactSet}  \sup_{\xi \in \clIvalGen} \left|\sem{\synExb}(\pStUpdate{\st}{\lvar}{\xi})  \right| \tag{absolute value of $\sup$ $\leq$ $\sup$ of absolute value} \\
        \lleq & \sup_{\st \in \compactSet'}  \left|\sem{\synExb}(\st)  \right| \tag{because $\compactSet' = \{\pStUpdate{\st}{\lvar}{\xi} \mid \st \in \compactSet, \xi \in \clIvalGen \}$} \\
        \llt & \infty \tag{$\sem{\synExb}$ locally bounded by I.H.} \\
    \end{align*}
    
    \inductionCase{$\synEx = \synInfBd{\lvar}{\ivalL}{\ivalR} \synExb$}
    This case is analogous to the $\supQuantifierSymbol$-case noticing that, for every set $A \subseteq \reals$, we have $|\inf A| = | - \sup - A| = |\sup - A| \leq \sup |-A| = \sup |A|$, where $- A = \{-a \mid a \in A\}$.
    
\end{proof}


\subsection{Proof of \Cref{thm:expToFo}}
\label{proof:expToFo}
%
\expToFo*
%
\begin{proof}
    By induction on the structure of $\synEx$.
    The crucial observation is that infima and suprema can be expressed in FO.
    In all of the following we assume that $\lvarb \notin \free{\synEx} \subseteq \{\lvar_1,\ldots,\lvar_n\}$. 

    \inductionCase{$\synEx(\lvar_1,\ldots,\lvar_n) = \synTerm(\lvar_1,\ldots,\lvar_n)$}
    We first define $\foForm_{\synTerm}(\lvar_1,\ldots,\lvar_n,\lvarb)$ by induction on the structure of the syntactic term $\synTerm$.
    
    \begin{itemize}
        \item $\synTerm = \ratConst$.
        We define
        \[
            \foForm_{\synTerm}(\lvarb)
            \quad\text{as}\quad
            \ratConst = \lvarb
        ~.
        \]
        %
        \item $\synTerm = \lvar$.
        We define
        \[
            \foForm_{\synTerm}(\lvar,\lvarb)
            \quad\text{as}\quad
            \lvar = \lvarb
            ~.
        \]
        %
        \item $\synTerm(\lvar_1,\ldots,\lvar_n) = \synTermb(\lvar_1,\ldots,\lvar_n) + \synTermc(\lvar_1,\ldots,\lvar_n)$.
        We define
        \[
            \foForm_\synTerm(\lvar_1,\ldots,\lvar_n,\lvarb)
            \quad\text{as}\quad
            \foExists \lvarb_1 \colon \foExists \lvarb_2 \colon \lvarb = \lvarb_1 + \lvarb_2 \land \foForm_\synTermb(\lvar_1,\ldots,\lvar_n,\lvarb_1) \land \foForm_\synTermc(\lvar_1,\ldots,\lvar_n,\lvarb_2)
            ~.
        \]
        %
        \item $\synTerm(\lvar_1,\ldots,\lvar_n) = \synTermb(\lvar_1,\ldots,\lvar_n) \cdot \synTermc(\lvar_1,\ldots,\lvar_n)$. Similar to the previous case.
        %
        \item $\synTerm(\lvar_1,\ldots,\lvar_n) = \synTermb(\lvar_1,\ldots,\lvar_n) \synMonus \synTermc(\lvar_1,\ldots,\lvar_n)$.
        This case is the most interesting one due to monus.
        We define $\foForm_\synTerm(\lvar_1,\ldots,\lvar_n,\lvarb)$ as 
        \begin{align*}
            \foExists \lvarb_1 \colon \foExists \lvarb_2 \colon &\foForm_\synTermb(\lvar_1,\ldots,\lvar_n,\lvarb_1) \land \foForm_\synTermc(\lvar_1,\ldots,\lvar_n,\lvarb_2) \\
            &\land
            \left( \lvarb_1 \geq \lvarb_2 \limplies \lvarb = \lvarb_1 - \lvarb_2 \right )
            \land
            \left( \lvarb_1 < \lvarb_2 \limplies \lvarb = 0 \right )
            ~.
        \end{align*}
    \end{itemize}
    Now that we have defined $\foForm_{\synTerm}$, we define
    \[
        \foForm_{\synEx}(\lvar_1,\ldots,\lvar_n,\lvarb)
        \quad\text{as}\quad
        \synTerm(\lvar_1,\ldots,\lvar_n) = \lvarb
        ~.
    \]
    
    \inductionCase{$\synEx(\lvar_1,\ldots,\lvar_n) = \iv{\synGuard(\lvar_1,\ldots,\lvar_n)}\cdot \synExb(\lvar_1,\ldots,\lvar_n)$}
    \newcommand{\lvarbool}{b}
    
    We first define an FO formula $\foForm_{\synGuard}(\lvar_1,\ldots,\lvar_n,\lvarbool)$ by induction on the structure of the syntactic guard $\synGuard$.
    The idea is to encode the (Boolean) result that $\synGuard$ evaluates to as a $\{0,1\}$-valued FO variable $\lvarbool$.
    
    \begin{itemize}
        \item $\synGuard(\lvar_1,\ldots,\lvar_n) = \synTerm(\lvar_1,\ldots,\lvar_n) < \synTermb(\lvar_1,\ldots,\lvar_n)$.
        We consider $\foForm_\synTerm(\lvar_1,\ldots,\lvar_n,\lvarb)$ and $\foForm_\synTermb(\lvar_1,\ldots,\lvar_n,\lvarb)$ as defined above and define $\foForm_{\synGuard}(\lvar_1,\ldots,\lvar_n,\lvarbool)$ as
        \begin{align*}
            \foExists \lvarb_1 \colon \foExists \lvarb_2 \colon &\foForm_\synTermb(\lvar_1,\ldots,\lvar_n,\lvarb_1) \land \foForm_\synTermc(\lvar_1,\ldots,\lvar_n,\lvarb_2) \\
            &\land
            \left( \lvarb_1 < \lvarb_2 \limplies \lvarbool = 1 \right )
            \land
            \left( \lvarb_1 \geq \lvarb_2 \limplies \lvarbool = 0 \right )
            ~.
        \end{align*}
        %
        \item $\synGuard(\lvar_1,\ldots,\lvar_n) = \synGuard_1(\lvar_1,\ldots,\lvar_n) \land \synGuard_2(\lvar_1,\ldots,\lvar_n)$.
        We consider $\foForm_{\synGuard_1}(\lvar_1,\ldots,\lvar_n,\lvarbool)$ and $\foForm_{\synGuard_2}(\lvar_1,\ldots,\lvar_n,\lvarbool)$ and define $\foForm_{\synGuard}(\lvar_1,\ldots,\lvar_n,\lvarbool)$ as
        \begin{align*}
            \foExists \lvarbool_1 \colon \foExists \lvarbool_2 \colon &\foForm_{\synGuard_1}(\lvar_1,\ldots,\lvar_n,\lvarbool_1) \land \foForm_{\synGuard_2}(\lvar_1,\ldots,\lvar_n,\lvarbool_2) \\
            &\land
            \left( \lvarbool_1 + \lvarbool_1 = 2 \limplies \lvarbool = 1 \right )
            \land
            \left( \lvarbool_1 + \lvarbool_1 \neq 2 \limplies \lvarbool = 0 \right )
            ~.
        \end{align*}
        %
        \item $\synGuard(\lvar_1,\ldots,\lvar_n) = \neg \synGuard'(\lvar_1,\ldots,\lvar_n)$.
        We consider $\foForm_{\synGuard'}(\lvar_1,\ldots,\lvar_n,\lvarbool)$ and define $\foForm_{\synGuard}(\lvar_1,\ldots,\lvar_n,\lvarbool)$ as
        \begin{align*}
            \foExists \lvarbool'\colon \foForm_{\synGuard'}(\lvar_1,\ldots,\lvar_n,\lvarbool')  
            \land
            \lvarbool = 1 - \lvarbool'
            ~.
        \end{align*}
    \end{itemize}
    
    Now that we have defined $\foForm_{\synGuard}$, we define
    \[
        \foForm_\synEx(\lvar_1,\ldots,\lvar_n,\lvarb)
        \quad\text{as}\quad
        \foExists \lvarbool \colon \foForm_\synGuard(\lvar_1,\ldots,\lvar_n,\lvarbool) \land (\lvarbool = 1 \limplies \foForm_\synEx(\lvar_1,\ldots,\lvar_n,\lvarb) )
        \land (\lvarbool = 0 \limplies \lvarb = 0 )
    \]
    
    \inductionCase{$\synEx(\lvar_1,\ldots,\lvar_n) = \ratConst \cdot \synExb(\lvar_1,\ldots,\lvar_n)$}
    We define
    \[
    \foForm_\synEx(\lvar_1,\ldots,\lvar_n,\lvarb)
    \quad\text{as}\quad
    \foExists \lvarb' \colon \lvarb = \ratConst \cdot \lvarb' \land \foForm_\synExb(\lvar_1,\ldots,\lvar_n,\lvarb')
    ~.
    \]
    
    \inductionCase{$\synEx(\lvar_1,\ldots,\lvar_n) = \synExb(\lvar_1,\ldots,\lvar_n) + \synExc(\lvar_1,\ldots,\lvar_n)$}
    We define
    \[
    \foForm_\synEx(\lvar_1,\ldots,\lvar_n,\lvarb)
    \quad\text{as}\quad
    \foExists \lvarb_1 \colon \foExists \lvarb_2 \colon \lvarb = \lvarb_1 + \lvarb_2 \land \foForm_\synExb(\lvar_1,\ldots,\lvar_n,\lvarb_1) \land \foForm_\synExc(\lvar_1,\ldots,\lvar_n,\lvarb_2)
    ~.
    \]
    
    \inductionCase{$\synEx = \synSupBd{\lvar}{\ivalL}{\ivalR} \synExb$}
    To give an intuition about how we handle this case, we recall the standard definition of the supremum:
    Let $A \subseteq \reals$ and $y \in \reals$.
    Then
    \begin{align*}
        & y = \sup A \\
        ~{}\iff{}~ & \text{for all } a\in A \colon a \leq y \tag{$y$ is an upper bound}\\
        & \text{and for all } y' \in \reals \colon( \text{for all } a \in A \colon a \leq y') \implies y \leq y' \tag{$y$ is not greater than any upper bound}
        ~.
    \end{align*}
    We now construct FO formulae corresponding to the above two conjuncts.
    Consider the formula $\foForm_\synExb(\lvar,\lvar_1,\ldots,\lvar_n,\lvarb)$ obtained by applying the I.H.\ to $\synExb(\lvar,\lvar_1,\ldots,\lvar_n)$.
    We define
    \[
    \foForm_{\synExb}^{ub}(\lvar_1,\ldots,\lvar_n,u)
    \quad\text{as}\quad
    \foForall \ivalL \leq \xi \leq \ivalR \colon \foForall \lvarb \colon \foForm_\synExb(\xi,\lvar_1,\ldots,\lvar_n,\lvarb) 	\limplies \lvarb \leq u
    ~.
    \]
    We have for all $\st \in \reals^{\free{\synExb}}$ and $\resVal \in \reals$ that $\st ,\resVal \models \foForm_{\synExb}^{ub}$ iff for all $\xi \in \clIvalGen \colon \sem{\synExb}(\pStUpdate{\st}{\lvar}{\xi}) \leq \resVal$.
    Now we define
    \[
    \foForm_\synEx(\lvar_1,\ldots,\lvar_n,\lvarb)
    \quad\text{as}\quad
    \foForm_{\synExb}^{ub}(\lvar_1,\ldots,\lvar_n,\lvarb) \land \foForall \lvarb ' \colon \foForm_{\synExb}^{ub}(\lvar_1,\ldots,\lvar_n,\lvarb') \limplies \lvarb \leq \lvarb'
    \]
    Then $\foForm_\synEx $ has the desired property.
    
    \inductionCase{$\synEx = \synInfBd{\lvar}{\ivalL}{\ivalR} \synExb$}
    Analogous to the previous case.
    \medskip
    
    Finally, observe that for \emph{quantifier-free} $\synEx$ (i.e., without $\supQuantifierSymbol$ and $\infQuantifierSymbol$), the above inductive construction of $\foForm_\synEx$ can be done in linear time in the size of $\synEx$.
    The so-obtained $\foForm_\synEx$ is not yet necessarily in existential prenex form, but it can be translated to the latter (again in linear time) by succesively applying the following standard prenexing rules which are true for all FO formulae $\foForm$ and $\foFormb$:
    \begin{align*}
        \foForm \limplies (\foExists \lvar \colon \foFormb)
        &\text{ is equivalent to }
        \foExists \lvar \colon (\foForm \limplies \foFormb)
        \\
        \foForm \land (\foExists \lvar \colon \foFormb)
        &\text{ is equivalent to }
        \foExists \lvar \colon (\foForm \land \foFormb)
    \end{align*}
    This completes the proof.
\end{proof}

\subsection{Proof of \Cref{thm:expBoundedAndRiemannIntegrable}}
\label{proof:expBoundedAndRiemannIntegrable}
%
\expBoundedAndRiemannIntegrable*
%
\begin{proof}
    Let $\pSt \in \nnReals^\lvarsSubset$ be arbitrary and
    consider the function $\ex_\pSt = \lam{\xi}{\iv{\xi \geq 0} \cdot \sem{\synEx}(\pStUpdate{\st}{\lvar}{\xi})}$ of type $\reals \to \reals$.
    By \Cref{thm:expWellLocallyBounded}, $\ex_\pSt$ is locally bounded.
    By \Cref{thm:expToFo}, $\ex_\pSt$ is semi-algebraic.
    It thus follows with \Cref{thm:semiAlgebraicFunctionAlmostContinuous} that $\ex_\pSt$ is Riemann-integrable on $\clIvalGen$.
    Hence, since $\st$ was arbitrary, we may conclude that $\sem{\synEx}$ is Riemann-integrable on $\clIvalGen$ w.r.t.\ $\lvar$.
\end{proof}


\subsection{Proof of \Cref{thm:checkEntailment}}
\label{proof:checkEntailment}
%
\checkEntailment*
%
\begin{proof}
    The fact that $\sem{\synEx} \eleq \sem{\synExb}$ iff the FO formula in \eqref{eq:smt-query} is valid follows directly from \Cref{thm:expToFo}.
    However, note that the formula in \eqref{eq:smt-query} is not necessarily quantifier-free as $\foForm_{\synEx'}$ and $\foForm_{\synExb'}$ may contain an $\foExists$-prefix (see \Cref{thm:expToFo}).
    By applying the standard prenexing rules
    \begin{align*}
        (\foExists \lvar \colon \foForm) \land \foFormb
        &\quad\text{is equivalent to}\quad
        (\foExists \lvar \colon \foForm \land \foFormb)
        \\
        \text{and}\qquad (\foExists \lvar \colon \foForm) \limplies \foFormb
        &\quad\text{is equivalent to}\quad
        \foForall \lvar \colon (\foForm \limplies \foFormb)
    \end{align*}
    which hold for all FO formulae $\foForm$ and $\foFormb$ with $\lvar \notin \free{\foFormb}$, we can transform the formula in \eqref{eq:smt-query} to universal prenex form.
    Checking validity of the latter is equivalent to checking validity of the \emph{quantifier-free} formula obtained by dropping the $\foForall$-prefix.
\end{proof}


\subsection{Substitution Lemma}

\begin{lemma}[{Substitution Lemma~\cite{DBLP:journals/pacmpl/BatzKKM21,DBLP:journals/pacmpl/SchroerBKKM23}}]
    \label{thm:syntacticSemanticSubs}
    Let $\synEx \in \synExps$ such that $\lvar \in \free{\synEx}$ and let $\synTerm \in \synTerms$ be a syntactic term.
    Consider the expression $\synSubs{\synEx}{\lvar}{\synTerm} \in \synExps$ obtained by (syntactically) substituting every free occurrence of $\lvar$ in $\synEx$ by $\synTerm$ in a capture-avoiding manner (i.e., by possibly renaming variables appropriately).
    Then
    \[
    \sem{\synSubs{\synEx}{\lvar}{\synTerm}}
    \eeq
    \exSubs{\sem{\synEx}}{\lvar}{\sem{\synTerm}}
    \gray{
        \eeq
        \lam{\pSt}{\sem{\synEx}(\pStUpdate{\pSt}{\lvar}{\sem{\synTerm}(\pSt)})}
    }
    ~.
    \]
\end{lemma}


\subsection{Proof of \Cref{thm:exprRiemannSuitable}}
\label{proof:exprRiemannSuitable}
%
\exprRiemannSuitable*
%
\begin{proof}
    Recall \Cref{def:riemannSuitable}.
    By \Cref{thm:expBoundedAndRiemannIntegrable}, all $\sem{\synEx}$ are bounded and Riemann integrable on $\uIval$ w.r.t.\ every $\pVar \in \pVars$.
    The arithmetic expressions representable by $\synTerms$, i.e., the functions $\sem{\synTerm}$ for $\synTerm \in \synTerms$, are locally bounded since they are piecewise defined polynomials.
    The closure properties from \Cref{def:riemannSuitable} all follow immediately from the definition of $\synExps$, except the substitution $\exSubs{\sem{\synEx}}{\pVar}{\aExp}$.
    Here, the crucial observation is that \emph{semantic substitution corresponds to syntactic substitution}, see \Cref{thm:syntacticSemanticSubs}.
\end{proof}


\subsection{Computing Riemann $\wpwlpSymb$ Syntactically}
\label{proof:computeSyntacticExpression}
%
\begin{restatable}[Computing Riemann $\wpwlpSymb$ Syntactically]{lemma}{computeSyntacticExpression}
	\label{thm:computeSyntacticExpression}
	For all $\prog \in \pWhileWith{\synTerms}{\synGuards}$, $\prog$ loop-free, and integers $\N \geq 1$ we can effectively do the following:
	For every given ...
	\begin{enumerate}
		\setlength\itemsep{0.2em}
		%
		\item ... $\synEx \in \syntacticExpectations$, compute $\synExb \in \syntacticExpectations$ such that \qquad \,\,\,\, $\lwp{\N}{\prog}{\sem{\synEx}} = \sem{\synExb}$ ~.
		%
		\item ... $\synEx \in \syntacticExpectations$, compute $\synExb \in \syntacticExpectations$ such that \qquad \,\,\,\, $\uwp{\N}{\prog}{\sem{\synEx}} = \sem{\synExb}$ ~.
		%
		\item ... $\synEx \in \bsyntacticExpectations$, compute $\synExb \in \bsyntacticExpectations$ such that \quad $\lwlp{\N}{\prog}{\sem{\synEx}} = \sem{\synExb}$ ~.
		%
		\item ...  $\synEx \in \bsyntacticExpectations$, compute $\synExb \in \bsyntacticExpectations$ such that \quad $\uwlp{\N}{\prog}{\sem{\synEx}} = \sem{\synExb}$ ~.
	\end{enumerate}
	In (1) and (3), if $\synEx$ is $\supQuantifierSymbol$-free, then so is $\synExb$.
	In (2) and (4), if $\synEx$ is $\infQuantifierSymbol$-free, then so is $\synExb$.
\end{restatable}
%
\begin{proof}
    By induction on the structure of $\prog$.
    Most cases are immediate as we have essentially designed $\synExps$ such that $\repExps$ ($\boundedRepresentableExpectations$) is closed under $\lwpTrans{\N}{\prog}$ and $\uwpTrans{\N}{\prog}$ ($\lwlpTrans{\N}{\prog}$ and $\uwlpTrans{\N}{\prog}$, resp.) for loop-free $\prog$.
    For the assignment $\ASSIGN{\pVar}{\aExp}$ we rely on syntactic substitution (\Cref{thm:syntacticSemanticSubs}).
\end{proof}


\subsection{Proof of \Cref{thm:boundsOnLoopFreeWpWlp}}
\label{proof:boundsOnLoopFreeWpWlp}
%
\boundsOnLoopFreeWpWlp*
%
\begin{proof}
    By computing a syntactic representation of the corresponding Riemann $\wpwlpSymb$ as in \Cref{thm:computeSyntacticExpression}, translating to PNF via \Cref{thm:pnf}, and applying the quantitative entailment check from \Cref{thm:checkEntailment}.
    The \gray{gray} implications follow from soundness of the Riemann $\wpwlpSymb$ (\Cref{thm:soundness}).
\end{proof}


\subsection{Proof of \Cref{thm:verificationInvariants}}
\label{proof:verificationInvariants}
%
\verificationInvariants*
%
\begin{proof}
    Since $\progBody$ is loop-free we can apply \Cref{thm:computeSyntacticExpression} to compute $i \in \synExps$ (resp. $j \in \bsyntacticExpectations$) such that $\sem{i} = \uwp{\N}{\progBody}{\sem{I}}$ (resp. $\sem{j} = \lwlp{\N}{\progBody}{\sem{J}}$) with $i$ is $\infQuantifierSymbol$-free (resp $\supQuantifierSymbol$-free). Then, applying \Cref{thm:invariant_approx}, we need to check that 
    $ [\guard] \cdot \sem{\synEx} + [\neg \guard] \cdot \sem{i} \leq \sem{I}$ (resp. $\sem{J} \leq [\guard] \cdot \sem{\synEx} + [\neg \guard] \cdot \sem{j} $). By \Cref{thm:boundsOnLoopFreeWpWlp} this inequality can be reduced to $\QFNRA$.
\end{proof}



\subsection{Proof of \Cref{thm:complexity}}
\label{proof:complexity}
%
\complexity*
%
\begin{proof}
    We show membership in $\coRE$ first (which is the more interesting direction).
    Consider the complement of the first problem:
    \begin{align}
        \text{``Does there exist } \pSt \in \pStates \text{ s.t.\ } \sem{\synExb}(\pSt) < \wp{\prog}{\sem{\synEx}}(\pSt) \text{ ?''}
        \tag{complement of problem \eqref{it:problemUpperBoundOnWp}}
    \end{align}
    If the answer to the above question is \emph{yes}, then, since 
    $\sup_{n \geq 1}\lwp{n}{\unfold{\prog}{n}}{\sem{\synEx}} = \wp{\prog}{\sem{\synEx}}$
    by \Cref{thm:pointwiseConv}, there must exist $\pSt \in \pStates$ and $n \geq 1$ such that $\sem{\synExb}(\pSt) < \lwp{n}{\unfold{\prog}{n}}{\sem{\synEx}}(\pSt)$.
    Hence our semi-decision procedure will check the latter inequality for increasing values of $n$.
    To do so for a fixed $n$, the procedure constructs $\synExc$ such that $\sem{\synExc} = \lwp{n}{\unfold{\prog}{n}}{\sem{\synEx}}$ as in \Cref{thm:computeSyntacticExpression} and then invokes \Cref{thm:expToFo} to obtain FO formulae $\foForm_\synExb(\free{\synExb}, \lvarb_\synExb)$ and $\foForm_\synExc(\free{\synExc}, \lvarb_\synExc)$ encoding $\synExb$ and $\synExc$.
    It then only remains to check if the FO formula $(\foForm_\synExb \land \foForm_\synExc) \limplies \lvarb_\synExb < \lvarb_\synExc$ is satisfiable, which is decidable.
    
    To see that the complement of problem \eqref{it:problemUpperBoundOnWp} is $\RE$-hard we reduce the halting problem to it.
    We proceed as in \cite{DBLP:conf/mfcs/KaminskiK15}.
    The standard halting problem is equivalent to checking if $\wp{\prog_{std}}{1}(\pSt) = 1$ where $\prog_{std} \in \pWhileWith{\synTerms}{\synGuards}$ is a given standard program, i.e., it does not contain any probabilistic constructs, and $\pSt \in \rats^\pVars$ is a given initial state.
    Now consider \[
    \prog
    \eeq
    \PCHOICE{\SEQ{\ASSIGN{\pVar_1 }{\pSt(\pVar_1)} \,\SEMICOLON\, \ldots \,\SEMICOLON\,\ASSIGN{\pVar_n }{\pSt(\pVar_n)} }{\prog_{std}}}{\tfrac 1 2}{\SKIP} 
    \]
    where $\{\pVar_1,\ldots,\pVar_n\} = \pVars$.
    Then $\tfrac 1 2 < \wp{\prog}{1}$ iff $\wp{\prog_{std}}{1}(\pSt) > 0$ iff $\prog_{std}$ terminates on input $\pSt$.
    
    The argument for $\coRE$ membership of problem \eqref{it:problemLowerBoundOnWlp} is completely analogous.
    For $\coRE$-hardness we argue dually (we define $\prog$ as above):
    $\wlp{\prog}{0} < \tfrac 1 2$ iff $\wlp{\prog_{std}}{0}(\pSt) < \tfrac 1 2$ iff $\prog_{std}$ terminates on input $\pSt$.
    The latter argument works because for standard programs we have $\wlp{\prog_{std}}{0}(\pSt) = 0$ iff $\prog_{std}$ terminates on $\pSt$, and $\wlp{\prog_{std}}{0}(\pSt) = 1$ iff $\prog_{std}$ does not terminate on $\pSt$.
\end{proof}
\subsection{Inputs to \toolcaesar}
\label{app:caesar_inputs}
%

\paragraph{The Monte Carlo $\pi$-approximator}
\phantom{a}

\begin{lstlisting}[language=C++, basicstyle=\ttfamily\small]
	coproc monte_carlo_pi(M : UReal) 
			-> (x : UReal, y : UReal, count : UReal)
	pre 0.85*(M)     
	post count
	{
		var i : UReal = 1
		count = 0
		@invariant(count + ite(0 <= i && i<=M, 0.85*((M-i) + 1), 0))
		while i <= M {
			
			var N : UInt = 16; 
			var j : UInt = unif(0, 15); //discrete_uniform(16)
			
			// --- Nondeterministic assignment x := [...]
			cohavoc x; 
			coassume ?!(j / N <= x && x <= (j+1) / N)
			// ---
			
			
			j  = unif(0, 15); //discrete_uniform(16)
			
			// --- Nondeterministic assignment y := [...]
			cohavoc y; 
			coassume ?!(j / N <= y && y <= (j+1) / N)
			// ---
			
			if x*x + y*y <= 1 {
				count = count +1
			}else{}
			
			i = i+1
			
		}
		
	}
\end{lstlisting}


\paragraph{Irwin-Hall without conditioning}
\phantom{a}
%
%
\begin{lstlisting}[language=C++, basicstyle=\ttfamily\small]
	coproc irwin_hall(M : UReal) -> (x : UReal)
	pre 1.1*((M)/2)
	post x
	{
		x = 0
		var i : UReal = 1
		@invariant(ite(i <= M, (x + 1.1*((M-i) + 1)/2), x))
		while i <= M {
			
			var inc : UReal; var N : UInt = 10; 
			var j : UInt = unif(0, 9); //discrete_uniform(10)
			
			// --- Nondeterministic assignment inc := [...]
			cohavoc inc; 
			coassume ?!(j / N <= inc && inc <= (j+1) / N)
			// ---
			
			x = x + inc
			i = i + 1
		}
	}
\end{lstlisting}


\paragraph{Irwin-Hall with conditioning}
\phantom{a}

\begin{lstlisting}[language=C++, basicstyle=\ttfamily\small]
	domain Exponentials {
		func exp(exponent: UReal): EUReal
		
		axiom exp_base exp(0) == 1
		axiom exp_step forall exponent: UReal. 
			exp(exponent + 1) == 0.5 * exp(exponent)
		axiom exp_antitone forall exp1: UReal. forall exp2: UReal. 
			(exp1 <= exp2) ==> (exp(exp2) <= exp(exp1))
	}
	
	
	proc irwin_hall_conditioning_wlp(M : UReal) -> (x : UReal)
	pre exp(M)
	post 1
	{
		x = 0
		var i : UReal = 1
		@invariant(ite(i <= M, exp((M-i) + 1), 1))
		while i <= M {
			var inc : UReal; var N : UInt = 2; 
			var j : UInt = unif(0, 1); //discrete_uniform(2)
			
			// --- Nondeterministic assignment inc := [...] 
			havoc inc; 
			assume ?(j / N <= inc && inc <= (j+1) / N) 
			// ---
			
			assert ?(inc <= 1/2) // observe
			x = x + inc
			i = i + 1
		}
	}
	

	coproc irwin_hall_conditioning_wp(M : UReal) -> (x : UReal)
	pre 1.5*((M)/8)
	post x
	{
		x = 0
		var i : UReal = 1
		@invariant(ite(i <= M, (x + 1.5*((M-i) + 1)/8), x))
		while i <= M {
			var inc : UReal; var N : UInt = 20; 
			var j : UInt = unif(0, 19); //discrete_uniform(20)
			
			// --- Nondeterministic assignment inc := [...] 
			cohavoc inc; 
			coassume ?!(j / N <= inc && inc <= (j+1) / N)
			// ---
			
			assert ?(inc <= 1/2) // observe
			x = x + inc
			i = i + 1
		}
	}
\end{lstlisting}


\paragraph{Probably Diverging Loop}
\phantom{a}

\begin{lstlisting}[language=C++, basicstyle=\ttfamily\small]
	domain Exponentials {
		func exp(exponent: UReal): EUReal
		
		axiom exp_base exp(0) == 1
		axiom exp_step forall exponent: UReal. 
			exp(exponent + 1) == 0.5 * exp(exponent)
		axiom exp_antitone forall exp1: UReal. forall exp2: UReal. 
			(exp1 <= exp2) ==> (exp(exp2) <= exp(exp1))
	}
	
	
	proc diverging(x_init : UReal, a : UReal, b : UReal) -> ()
	pre [a<=b]*(1-exp(x_init))
	post 0
	{
		var x : UReal = x_init 
		
		@invariant([a<=b]*(1-exp(x)))
		while(x >= 0){
			var y : UReal
			var N : UInt = 2; 
			var i : UInt = unif(0, 1); //discrete_uniform(2)
			
			// --- Nondeterministic assignment y := [...] 
			havoc y; 
			assume ?(i / N <= y && y <= (i+1) / N)
			// ---
			
			y = (b-a)*y + a        
			
			if (y <= (a+b)/2) {
				// --- diverge
				assert 1
				assume ?(false)
				// ---
			}else {}
			
			x = x - 1
			
		}
		
	}
\end{lstlisting}

\paragraph{Race between Tortoise and Hare}
\phantom{a}

\begin{lstlisting}[language=C++, basicstyle=\ttfamily\small]
	coproc tortoise_hare(h_init : UReal, t_init : UReal) 
			-> (count : UReal)
	pre 1.5*((t_init - h_init) + 2)*2 
	post count
	{
		var h : UReal = h_init
		var t : UReal = t_init
		
		count = 0
		
		@invariant(ite(h<=t, count + 1.5*((t-h) + 2)*2, count))
		while h <= t {
			var choice : Bool = flip(0.5)
			if choice {
				
				var inc : UReal; 
				var N : UInt = 25; 
				var j : UInt = unif(0, 24); //discrete_uniform(25)
				
				// --- Nondeterministic assignment inv := [...] 
				cohavoc inc; 
				coassume ?!(j / N <= inc && inc <= (j+1) / N)
				// ---
				
				inc = (10-0)*inc + 0 
				h = h + inc
			} else {}
			
			t = t+1
			count = count + 1
		}
	}
\end{lstlisting}

}{}

\end{document}
\endinput
