\subsection{Proof of \Cref{thm:measexpsbicpo}}
\label{proof:measexpsbicpo}
%
\measexpsbicpo*
%
\begin{proof}
    
    To see that $\po{\exps}{\eleq}$ and $\po{\bexps}{\eleq}$ are complete lattices for any $\poSubset \subseteq \exps$ (resp. $\poSubset \subseteq \bexps$) set
    $$ \sup \poSubset = \lambda \st . \sup_{\ex \in \poSubset} \ex(\st), \hspace{1cm} \inf \poSubset = \lambda \st . \inf_{\ex \in \poSubset} \ex(\st).$$
    Then, $\sup \poSubset \in \exps$ and $\inf \poSubset \in \exps$ (resp. $\sup \poSubset \in \bexps$ and $\inf \poSubset \in \bexps$).
    
    Theorem 11.17 from \cite{rudin1953principles} ensures that for any sequence of measurable functions the supremum and infimum of the sequence is measurable. Specializing the theorem in the case of Borel measurability ensures that $\po{\expsmeas}{\eleq}$ and $\po{\bexpsmeas}{\eleq}$ are $\omega$-bicpos. 
\end{proof}


\subsection{Proof of \Cref{thm:wpWlpWellDefinedAndContinuous}}
\label{proof:wpWlpWellDefinedAndContinuous}
%
\wpWlpWellDefinedAndContinuous*
%
\begin{proof}
    Parts of this statement have been shown in in \cite[p.\ 88 (Proof of Lemma 3)]{DBLP:conf/setss/SzymczakK19} and \cite[Lemma 34]{DBLP:journals/pe/GretzKM14}. %\tw{the wlp stuff was not shown }
    Here we provide a mostly self-contained proof.
    Note that for well-definedness, one must show various points:
    preservation of measurable expectations, existence of the Lebesgue integrals, and existence of the least and greatest fixed points -- for the latter, $\omega$-(co)continuity is needed, which is why we show it simultaneously.
    
    In the following, we let $\ex \in \expsmeas$ be an arbitrary measurable expectation and $\ex_0 \eleq \ex_1 \eleq \ldots$ a (non-decreasing) $\omega$-chain in $\expsmeas$.
    Similarly, $\exb \in \bexpsmeas$ is a 1-bounded measurable expectation and $\exb_0 \egeq \exb_1 \egeq \ldots$ a (non-increasing) $\omega$-cochain in $\bexpsmeas$.
    We will show by structural induction that for all $\prog \in \pWhile$ it holds that $\wp{\prog}{\ex}$ and $\wlp{\prog}{\exb}$ are measurable and that $\sup_{i \in \nats}\wp{\prog}{\ex_i} = \wp{\prog}{\sup_{i \in \nats} \ex_i}$ as well as $\inf_{i \in \nats}\wlp{\prog}{\exb_i} = \wlp{\prog}{\inf_{i \in \nats} \exb_i}$.
    Throughout this proof we write $\sup_i$ as a shorthand for $\sup_{i\in \nats}$.
  
    
    \inductionCase{$\prog = \SKIP$}
    $\wpTrans{\SKIP} = \wlpTrans{\SKIP}$ is the identify function, hence it trivially preserves measurable expectations and is $\omega$-bicontinuous.
    
    
    \inductionCase{$\prog = \DIVERGE$}
    $\wpTrans{\DIVERGE}$ are $\wlpTrans{\DIVERGE}$ are constant expectations, hence they are both measurable and $\omega$-bicontinuous.
    
    
    \inductionCase{$\prog = \ASSIGN{\pVar}{\aExp}$}
    \newcommand{\auxiliaryStateTransformer}{U}
    \newcommand{\auxStTrans}{\auxiliaryStateTransformer}
    By definition, $\wp{\ASSIGN{\pVar}{\aExp}}{\ex} = \exSubsGen$ which can also be written as $\ex \circ \auxStTrans$ where $\auxStTrans \colon \states \to \states, \st \mapsto \pStUpdate{\st}{\pVar}{\aExp(\st)}$ is a function updating the program state according to $\ASSIGN{\pVar}{\aExp}$.
    Since $\aExp$ is measurable, $\auxStTrans$ is measurable as well, and thus $\exSubsGen$ is measurable as a composition of measurable functions. The argument for measurability of $\wlp{\ASSIGN{\pVar}{\aExp}}{\exb}$ is exactly the same. 
    
    For $\omega$-continuity we have:
    \begin{align*}
        & \wp{\ASSIGN{\pVar}{\aExp}}{\sup_i \ex_i}  \\
        \eeq & \exSubs{\left( \sup_i \ex_i \right)}{\pVar}{\aExp} \tag{definition of $\wpSymb$} \\
        \eeq & \sup_i \exSubs{\ex_i}{\pVar}{\aExp} \tag{$\sup$ defined pointwise}\\
        \eeq & \sup_i \wp{\ASSIGN{\pVar}{\aExp}}{\ex_i} \tag{definition of $\wpSymb$}
        ~.
    \end{align*} 
    An analogous argument holds for $\omega$-cocontinuity of $\wlpTrans{\ASSIGN{\pVar}{\aExp}}$.
    
    
    \inductionCase{$\prog = \UNIFASSIGN{\pVar}$}
    Measurability of $\wp{\UNIFASSIGN{\pVar}}{\ex} = \lam{\pSt}{\int_\uIval \ex (\pStUpdate{\st}{\pVar}{\xi}) \, d\lebmes(\xi)}$ and its liberal counterpart is a direct application of \Cref{thm:fubini}
    
    For $\omega$-continuity of $\wpTrans{\UNIFASSIGN{\pVar}}$ we argue as follows:
    \begin{align*}
        & \wp{\UNIFASSIGN{\pVar}}{\sup_i \ex_i} \\
        %
        \eeq & \lam{\pSt}{\int_\uIval \left(\sup_i \ex_i\right)(\pStUpdate{\st}{\pVar}{\xi}) \, d\lebmes(\xi) } \tag{definition of $\wpSymb$} \\
        %
        \eeq & \lam{\pSt}{\int_\uIval \sup_i \ex_i (\pStUpdate{\st}{\pVar}{\xi})  \, d\lebmes(\xi) } \tag{$\sup$ defined pointwise} \\
        %
        \eeq & \lam{\pSt}{\sup_i \int_\uIval \ex_i (\pStUpdate{\st}{\pVar}{\xi}) \, d\lebmes(\xi)} \tag{Monotone Convergence Theorem, e.g.,~\cite[Thm.\ 12.1]{schilling2017measures}} \\
        %
        \eeq & \sup_i \lam{\pSt}{\int_\uIval \ex_i (\pStUpdate{\st}{\pVar}{\xi}) \, d\lebmes(\xi)}  \tag{$\sup$ defined pointwise} \\
        %
        \eeq & \sup_i \wp{\UNIFASSIGN{\pVar}}{\ex_i} \tag{definition of $\wpSymb$}
    \end{align*}
    Note the application of the the Monotone Convergence Theorem, which ensures that for a non-decreasing sequence of measurable $\exNonNegReals$-valued functions it is possible to switch the supremum over the sequence and the Lebesgue integral. Co-continuity for $\wlpSymb$ follows by expressing $\inf_i \exb_i = 1 - \sup_i (1-\exb_i)$ and observing that the functions $(1-\exb_i) \in \bexps$ are a non-decreasing sequence, so Monotone Converge Theorem can be applied to them.
    
    
    \inductionCase{$\prog = \OBSERVE{\guard}$}
    Since we require the functions in $\guards$ to be Borel measurable, $\iv{\guard}$ is Borel measurable for any $\guard \in \guards$. Measurability of $\wpSymb$ and $\wlpSymb$ then follows from the fact that pointwise product of measurable functions is measurable. 
    For $\omega$-continuity we have:
    \begin{align*}
        & \wp{\OBSERVE{\guard}}{\sup_i \ex_i} \\ 
        & \eeq \iv{\guard} \cdot \sup_i \ex_i \tag{definition of $\wpSymb$} \\
        & \eeq \sup_i \left( \iv{\guard} \cdot \ex_i \right) \tag{$\sup$ defined pointwise} \\
        & \eeq \sup_i \wp{\OBSERVE{\guard}}{\ex_i} \tag{Definition of $\wpSymb$}
    \end{align*}
    where, again, we are using the properties of the supremum taken with respect to pointwise ordering as in \cite{DBLP:conf/setss/SzymczakK19}.  The same holds for $\omega$-cocontinuity of $\wlpSymb$ using the properties of the infimum taken with respect to pointwise ordering. 
    
    
    \inductionCase{$\prog = \ITE{\guard}{\prog_1}{\prog_2}$}
    By I.H.\ we know that $\wp{\prog_1}{\ex}$ and $\wp{\prog_2}{\ex}$ are measurable and $\omega$-continuous. Then, measurability of $\wp{\ITE{\guard}{\prog_1}{\prog_2}}{\ex} = \iv{\guard} \cdot \wp{\prog_1}{\ex} + \iv{\neg \guard} \cdot \wp{\prog_2}{\ex}$ follows from the fact that pointwise sum and product of measurable functions is measurable.        
    For $\omega$-continuity we have:
    \begin{align*}
        & \wp{\ITE{\guard}{\prog_1}{\prog_2}}{\sup_i \ex_i} \\
        & \eeq \iv{\guard} \cdot \wp{\prog_1}{\sup_i \ex_i} + \iv{\neg \guard} \cdot \wp{\prog_2}{\sup_i \ex_i}  \tag{definition of $\wpSymb$} \\
        & \eeq \iv{\guard}  \cdot \sup_i \wp{\prog_1}{\ex_i} + \iv{\neg \guard} \cdot \sup_i \wp{\prog_2}{\ex_i} \tag{I.H.}  \\
        & \eeq \sup_i \iv{\guard}  \cdot \wp{\prog_1}{\ex_i} + \sup_i \iv{\neg \guard} \cdot  \wp{\prog_2}{\ex_i} \tag{pointwise ordering}  \\
        & \ggeq \sup_i \iv{\guard}  \cdot \wp{\prog_1}{\ex_i} + \iv{\neg \guard} \cdot  \wp{\prog_2}{\ex_i} \tag{properties of $\sup$}
    \end{align*}
    To prove that inverse inequality holds, we proceed by contradiction, supposing $>$ holds. Then, for some $i_1, i_2 \in \nats$ it holds:
    $$  \iv{\guard}  \cdot \wp{\prog_1}{\ex_{i_1}} + \iv{\neg \guard} \cdot  \wp{\prog_2}{\ex_{i_2}}  > \sup_i \iv{\guard} \cdot \wp{\prog_1}{\ex_i} + \iv{\neg \guard} \cdot \wp{\prog_2}{\ex_i}.$$
    However, since $\ex_i$ is a $\omega$-chain we can take $k \in \nats$ such that $k > i_i$ and $k > i_2$ for which we have
    \begin{align*}
        & \iv{\guard}  \cdot \wp{\prog_1}{\ex_k} + \iv{\neg \guard} \cdot  \wp{\prog_2}{\ex_k} \\
        & \ge \iv{\guard}  \cdot \wp{\prog_1}{\ex_{i_1}} + \iv{\neg \guard} \cdot  \wp{\prog_2}{\ex_{i_2}} \tag{monotonicity of $\wpTrans{\prog_1}, \wpTrans{\prog_2}$} \\
        & > \sup_i \iv{\guard}  \cdot \wp{\prog_1}{\ex_i} + \iv{\neg \guard} \cdot  \wp{\prog_2}{\ex_i} \tag{hypothesis} 
    \end{align*}
    which is a contradiction since
    $$\iv{\guard}  \cdot \wp{\prog_1}{\ex_k} + \iv{\neg \guard} \cdot  \wp{\prog_2}{\ex_k} \le \sup_i \iv{\guard}  \cdot \wp{\prog_1}{\ex_i} + \iv{\neg \guard} \cdot  \wp{\prog_2}{\ex_i}.$$
    For $\wlpSymb$ both measurability and $\omega$-cocontinuity can be proved using the same arguments.
    
    
    \inductionCase{$\prog = \PCHOICE{\prog_1}{\prob}{\prog_2}$}
    As in the previous case, replacing $\iv{\guard}$ with $\prob$ and $\iv{\neg \guard}$ with $1 - \prob$.
    
    
    \inductionCase{$\prog = \SEQ{\prog_1}{\prog_2}$}
    Measurability of $\wp{\SEQ{\prog_1}{\prog_2}}{\ex}$ and $\wlp{\SEQ{\prog_1}{\prog_2}}{\exb}$ follows immediately from the I.H.. 
    For $\omega$-continuity we have:
    \begin{align*}
        & \wp{\SEQ{\prog_1}{\prog_2}}{\sup_i \ex_i} \\
        & \eeq \wp{\prog_1}{\wp{\prog_2}{\sup_i \ex_i}} \tag{definition of $\wpSymb$} \\
        & \eeq \wp{\prog_1}{\sup_i \wp{\prog_2}{\ex_i}} \tag{I.H.\ on $\wpTrans{\prog_2}$} \\
        & \eeq \sup_i \wp{\prog_1}{\wp{\prog_2}{\ex_i}} \tag{I.H.\ on $\wpTrans{\prog_1}$} \\
        & \eeq \sup_i \wp{\SEQ{\prog_1}{\prog_2}}{\ex_i} \tag{definition of $\wpSymb$}
    \end{align*}
    where $\wp{\prog_1}{\sup_i \wp{\prog_2}{\ex_i}}$ is well-defined since by I.H. $\wp{\prog_2}{\ex_i}$ is measurable for every $i$ and therefore $\sup_i \wp{\prog_2}{\ex_i}$ is measurable as well.
    
    
    \inductionCase{$\prog = \WHILE{\guard}{\progBody}$}
    \begin{align*}
        & \wp{\prog}{\sup_i \ex_i} \\
        & = \slfp{\lam{g}{\exlfp{\progBody}{\sup_i \ex_i}{\guard}(g)}} = \tag{definition of $\wpSymb$}\\
        & = \sup_n \left( \exlfp{\progBody}{\sup_i \ex_i}{\guard} \right)^n(\zerofun) = \tag{I.H. continuity of $\wpTrans{\progBody}$}  \\
        & = \sup_n \left( \sup_i \exlfp{\progBody}{\ex_i}{\guard} \right)^n(\zerofun) = \tag{continuity of $\lambda \ex . \iv{\neg \guard}\ex + \iv{\guard}\wp{\progBody}{\exb}$} \\
        & = \sup_n \sup_i \left(\exlfp{\progBody}{\ex_i}{\guard} \right)^n(\zerofun) = \tag{$*$} \\
        & = \sup_i \sup_n (\exlfp{\progBody}{\ex_i}{\guard})^n(\zerofun) = \tag{$**$} \\
        & = \sup_i \slfp{\lam{g}{\exlfp{\progBody}{\ex_i}{\guard}(g)}} = \tag{I.H. continuity of $\wpTrans{\progBody}$} \\
        &= \sup_i \wp{\prog}{\ex_i}. \tag{definition of $\wpSymb$}
    \end{align*}
    where $(*)$ is justified by proving with an inner induction on $n$ that:
    $$ \left(\sup_i \exlfp{\prog}{\ex_i}{\guard} \right)^n(\zerofun) = \sup_i \left( \exlfp{\prog}{\ex_i}{\guard}\right)^n(\zerofun).$$
    In fact, for $n=1$ the equality holds trivially, and assuming it holds for $n-1$ we get:
    \begin{align*}
        \text{L.H.S.} & = \left( \sup_i \exlfp{\prog}{\ex_i}{\guard} \right)\left(\sup_j \exlfp{\prog}{\ex_j}{\guard} \right)^{n-1}(\zerofun) =  \\
        & = \left( \sup_i \exlfp{\prog}{\ex_i}{\guard} \right)\left(\sup_j \left(\exlfp{\prog}{\ex_j}{\guard} \right)^{n-1}(\zerofun)\right) = \tag{case $n-1$} \\
        & = \sup_j \left( \sup_i \exlfp{\prog}{\ex_i}{\guard} \right)\left(\left(\exlfp{\prog}{\ex_j}{\guard} \right)^{n-1}(\zerofun)\right) = \tag{outer I.H.} \\
        & = \sup_j \sup_i \left( \exlfp{\prog}{\ex_i}{\guard} \right)\left(\left(\exlfp{\prog}{\ex_j}{\guard} \right)^{n-1}(\zerofun)\right) = \tag{case $n=1$}\\
        & = \sup_i \left( \exlfp{\prog}{\ex_i}{\guard} \right)\left(\left(\exlfp{\prog}{\ex_i}{\guard} \right)^{n-1}(\zerofun)\right) = \tag{Lemma \ref{thm:oneVsTwoSups}}  \\
        & = \sup_i \left(\left(\exlfp{\prog}{\ex_i}{\guard} \right)^{n}(\zerofun)\right) = \text{R.H.S.}.
    \end{align*}
    For $(**)$ switching the order of suprema is justified by the fact that both terms exists.
    
    To prove co-continuity of $\wlpSymb$ we can use the same arguments: $(*)$ and $(**)$ still hold if $\sup$ is replaced by $\inf$, since Lemma \ref{thm:oneVsTwoSups} still holds and the order of infima can be switched if both terms of the equality exist.
\end{proof}