\subsection{Fixed Point Theory}

The following observation is key in several proofs in this paper:
%
\begin{restatable}[\textnormal{cf.~\cite[Proposition 2.1.12]{abramsky1994domain}}]{lemma}{oneVsTwoSups}
    \label{thm:oneVsTwoSups}
    Let $\poGen$ be an $\omega$-cpo and let $\poElem \colon \nats \times \nats \to \poDom$ be monotonic (with regards to the usual order on $\nats$ lifted pointwise to $\nats \times \nats$).
    Then
    \[
        \sup_{i \in \nats} \sup_{j \in \nats} \poElem(i,j)
        \eeq
        \sup_{i \in \nats} \poElem(i,i)
        \eeq
        \sup_{j \in \nats} \sup_{i \in \nats} \poElem(i,j)
        ~.
    \]
\end{restatable}
\begin{proof}
    \Cref{proof:oneVsTwoSups}.
\end{proof}


\subsection{Measure Theory and Lebesgue Integrals}
\label{sec:prelims:measure}


\subsubsection{Basic Notions of Measure Theory}
\label{sec:prelims:measure:basics}

Most of the following can be found in~\cite{ash2000probability}.
Let $\mUniv$ be a set.
A \proseSigmaAlgebra on $\mUniv$ is a set $\sAlg \subseteq 2^\mUniv$ of subsets of $\mUniv$ such that $\mUniv \in \sAlg$ and $\sAlg$ is closed under taking countable unions and complements relative to $\mUniv$.
If $\sAlg$ is a \proseSigmaAlgebra on $\mUniv$ then $(\mUniv, \sAlg)$ is called a \emph{measurable space} and the sets in $\sAlg$ are called \emph{measurable sets} (this is not to be confused with a \emph{measure space} which is introduced further below).
%
For an arbitrary set $\sigmaAlgebraGenerators \subseteq 2^\mUniv$ of subsets of $\mUniv$ we let $\sigmaAlgebraGeneratedBy{\sigmaAlgebraGenerators} = \bigcap \{\mathcal F \mid \sigmaAlgebraGenerators \subseteq \sAlg, \sAlg\text{ a \proseSigmaAlgebra on } \mUniv \}$ be the smallest \proseSigmaAlgebra (on $\mUniv$) containing $\sigmaAlgebraGenerators$.

The \emph{Borel \proseSigmaAlgebra} $\borelSets{\exReals}$ on the extended real numbers $\exReals$ is the smallest \proseSigmaAlgebra containing all intervals of the form $(\ivalL, \ivalR]$, $\ivalL, \ivalR \in \exReals$, i.e., $\borelSets{\exReals} = \sigmaAlgebraGeneratedBy{\{(\ivalL, \ivalR] \mid \ivalL, \ivalR \in \exReals\}}$.
The sets in $\borelSets{\exReals}$ are called \emph{Borel sets}.
Note that many common sets including various types of intervals (bounded, unbounded, open, closed, half-open, and so on), all countable subsets of $\exReals$, and many others are Borel sets.
However, $\borelSets{\exReals} \neq 2^\exReals$, i.e., there exist non-Borel sets.
%
For $\measurableSet \in \borelSets{\exReals}$ we define $\borelSets{\measurableSet} = \borelSets{\exReals} \cap \measurableSet$, where the intersection is taken element-wise.
It can be shown that $\borelSets{\measurableSet}$ is a \proseSigmaAlgebra on $\measurableSet$ and $\borelSets{\measurableSet} \subseteq \borelSets{\exReals}$, see, e.g.,~\cite[page 5]{ash2000probability}.
In this way we may obtain a \proseSigmaAlgebra on, say, $\uIval$.

Given a measurable space $(\mUniv, \sAlg)$, a \emph{measure} on $\sAlg$ is a function $\measure \colon \sAlg \to \exNonNegReals$ such that for all countable collections $\countableCollectionOfMeasurableSets \subseteq \sAlg$ of pairwise disjoint subsets of $\sAlg$ (i.e., $\forall \measurableSet,  \measurableSetb \in \countableCollectionOfMeasurableSets \colon \measurableSet = \measurableSetb \lor \measurableSet \cap \measurableSetb = \emptyset$) it holds that
$\measure\left( \bigcup_{\measurableSet \in \countableCollectionOfMeasurableSets} \measurableSet \right) = \sum_{\measurableSet \in \countableCollectionOfMeasurableSets} \measure(\measurableSet)$, where the infinite sum is allowed to take value $\infty$.
If $\measure(\mUniv) = 1$ then $\measure$ is a \emph{probability measure}.
%
We denote by $\lebmes$ the \emph{Lebesgue measure}%
\footnote{We remark that it is also common to define $\lebmes$ on $\lebesgueSets{\reals}$, the \emph{completion} of $\borelSets{\reals}$ w.r.t.\ $\lebmes$ (the sets in $\lebesgueSets{\reals}$ are often called \emph{Lebesgue measurable}), but this construction is not necessary for our purposes.}
%
on $\borelSets{\reals}$
is the unique measure $\lebmes$ satisfying $\lebmes((\ivalL, \ivalR]) = \ivalR - \ivalL$ for all $\ivalL \leq \ivalR \in \reals$.
Note that we define this only on the reals as we do not need it for the extended reals.
In fact, the only measure we consider (explicitly) in this paper is the Lebesgue measure on $\borelSets{\uIval}$.


\subsubsection{Lebesgue Integrals}
\label{sec:prelims:measure:integrals}

Let $(\mUniv_i, \sAlg_i)$, $i \in \{1,2\}$, be measurable spaces.
A function $\fun \colon \mUniv_1 \to \mUniv_2$ is called \emph{measurable} w.r.t.\ $\sAlg_1$ and $\sAlg_2$ if for all $\measurableSet \in \sAlg_2$ it holds that $\fun^{-1}(\measurableSet) \in \sAlg_1$.

For our purpose we only need to define integrals of non-negative functions.
Let $(\mUniv, \sAlg, \measure)$ be a measure space and $\fun \colon \mUniv \to \exNonNegReals$ be an arbitrary function.
A measurable function $\simpleFun \colon \mUniv \to \nonNegReals$ w.r.t.\ $\sAlg$ and $\borelSets{\nonNegReals}$ is called \emph{simple} if its image $\simpleFun(\mUniv)$ is a finite set $\{a_1,\ldots,a_n\}$; such an $\simpleFun$ can be written as $\simpleFun(x) = \sum_{i=1}^n a_i \cdot \iv{x \in \simpleFun^{-1}(a_i)}$ for all $x \in \mUniv$.
The Lebesgue integral of the simple function $\simpleFun$ on $\measurableSet \in \sAlg$ is defined as 
$
\int_{\measurableSet} \simpleFun \,d\measure
=
\sum_{i=1}^n a_i \cdot \measure(\simpleFun^{-1}(a_i) \cap \measurableSet)
$
.
The Lebesgue integral of $\fun$ on $\measurableSet$ is then defined as
\[
\int_{\measurableSet} \fun \,d\measure
\eeq
\sup \left\{\int_{A} \simpleFun \,d\measure \mid \simpleFun \text{ is simple, } \simpleFun \leq \fun \text{ (pointwise)} \right\}
\]
which can be any number in $\nonNegReals$ or $\infty$.
Note that $\fun$ itself does not have to be measurable; however, many fundamental properties of Lebesgue integrals  break for non-measurable $\fun$, hence it is customary to consider Lebesgue integrals only for measurable $\fun$.


\subsubsection{Multi- vs.\ Single-dimensional Integrals}
\label{sec:prelims:measure:multi}

Given two measurable spaces $(\mUniv_i, \sAlg_i)$, $i \in \{1,2\}$, define $\sAlg_1 \otimes \sAlg_2 = \sigmaAlgebraGeneratedBy{\{\measurableSet_1 \times \measurableSet_2 \mid \measurableSet_i \in \sAlg_i\}}$.
The measurable space $(\mUniv_1 \times \mUniv_2, \sAlg_1 \otimes \sAlg_2)$ is called the \emph{product} of $(\mUniv_1, \sAlg_1)$ and $(\mUniv_2, \sAlg_2)$.
%
For $n > 1$ and $\measurableSet \in \borelSets{\exReals}$, we obtain the Borel \proseSigmaAlgebra $\borelSets{\measurableSet^n}$ on $\measurableSet^n$ as the $n$-fold product of $(\measurableSet, \borelSets{\measurableSet})$ with itself.
The sets in $\borelSets{\measurableSet^n}$ are called Borel sets as well.

We call a function $\fun \colon \measurableSet \to \measurableSetb$ \emph{Borel measurable} if both $A$ and $B$ are Borel sets (in any dimension) and $\fun$ is measurable w.r.t.\ the respective Borel \proseSigmaAlgebras on $A$ and $B$. 
%
In this paper, we often consider Borel measurable functions of type $\fun \colon \reals^n \to \exNonNegReals$ and we would like to do something like \enquote{take a Lebesgue integral in one variable}.
To justify that this makes sense (and preserves measurability) we rely on the following:

\begin{theorem}[Part of Fubini's Theorem\textnormal{~\cite[Theorems 8.5 and 8.8]{10.5555/26851}}]
    \label{thm:fubini}
    Let $(\mUniv_i, \sAlg_i)$, $i \in \{1,2\}$, be measurable spaces and $\fun \colon \mUniv_1 \times \mUniv_2 \to \exNonNegReals$ be measurable w.r.t.\ the product space.
    Then:
    \begin{enumerate}
        \item For every $x \in \mUniv_1$, the function $\fun_x \colon \mUniv_2 \to \exNonNegReals, y \mapsto \fun(x,y)$ is measurable w.r.t.\ $\sAlg_2$.
        \item For every $\measurableSet \in \sAlg_2$ and measure $\measure$ on $\sAlg_2$, the function $F \colon \mUniv_1 \to \exNonNegReals, x \mapsto \int_{\measurableSet} \fun_x \,d\measure$ is measurable w.r.t.\ $\mathcal F _1$.
    \end{enumerate}
\end{theorem}
%
For a given $x \in \mUniv_1$, we also write $\int_{\measurableSet}\fun(x,y) \,d\measure(y)$ instead of $\int_{\measurableSet}\fun_{x} \,d\measure$.


\subsection{Riemann Integrals}

A useful aspect of (lower and upper) Riemann integrals is that one can choose partitions that induce sequences of upper (resp.\ lower) sums that converge \emph{monotonically}.
For partitions $\partition = (x_0,\ldots,x_{\partitionSize})$ and $\partitionb = (y_0,\ldots,y_{\partitionSizeb})$,
 both of the same interval, we say that $\partition$ \emph{refines} $\partitionb$, in symbols $\partition \partitionRefines \partitionb$, if for all indices $0 \leq j \leq \partitionSizeb$ there exists $0 \leq i \leq \partitionSize$ such that $x_i = y_j$.
In other words, the partition $\partition$ can be obtained from $\partitionb$ by subdividing some of the intervals in $\partitionb$ further.
\begin{lemma}[\textnormal{see,~e.g.,~\cite[Lemma 6.2]{fitzpatrick2009advanced}}]
    \label{thm:partitionRefine}
    Let $\fun \colon \clIvalGen \to \reals$ be bounded.
    Suppose that $\partition \partitionRefines \partitionb$.
    Then $\lowerSum{\fun}{\partitionb} \leq \lowerSum{\fun}{\partition}$
    and
    $\upperSum{\fun}{\partition} \leq \upperSum{\fun}{\partitionb}$.
\end{lemma}

To obtain the Riemann-Darboux integral, it is not actually necessary to consider \emph{all} partitions.
Indeed, it suffices to consider a set of partitions whose \emph{norm} (\enquote{fineness}) becomes arbitrarily small.
We now formalize this.
For a partition $\partition = (x_0,\ldots,x_{\partitionSize}) \in \partitions{\clIvalGen}$, let its norm be defined as
\[
\partitionNorm{\partition}
\eeq
\max_{0 \leq i < \partitionSize} x_{i+1} - x_i
~.
\]

\begin{restatable}[\textnormal{see,~e.g.,~\cite[Theorem 7.12]{fitzpatrick2009advanced}}]{theorem}{smallNormSuffices}
    \label{thm:smallNormSuffices}
    Let $\fun \colon \clIvalGen \to \reals$ be bounded and let $(\partition_n)_{n\in\nats}$ be a sequence of partitions of $\clIvalGen$ satisfying $\lim_{n \to \infty} \partitionNorm{\partition_n} = 0$.
    Then \[
    \lowerIntGen \fun(x) \,dx
    \eeq
    \lim_{n \to \infty} \lowerSum{\fun}{\partition_n}
    \eeq
    \sup_{n \in \nats} \lowerSum{\fun}{\partition_n}
    \qqand
    \upperIntGen \fun(x) \,dx
    \eeq
    \lim_{n \to \infty} \upperSum{\fun}{\partition_n}
    \eeq
    \inf_{n \in \nats} \upperSum{\fun}{\partition_n}
    ~.
    \]
\end{restatable}
\begin{proof}
    \Cref{proof:smallNormSuffices}
\end{proof}
