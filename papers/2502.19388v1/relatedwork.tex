\section{Related Work}
\label{sec:relwork}

\emph{Integral Approximation.}
Our approach relies on under- and over-approximation of Lebesgue integrals via Riemann sums, achieved by discretizing the domain of the continuous uniform distributions in the program.
Recently, \cite{DBLP:journals/pacmpl/GargHBM24} proposed an efficient \emph{bit blasting} discretization method for continuous mixed-gamma distributions (which subsume the uniform distributions) for programs with bounded ($\mathtt{for}$) loops.
In \cite{DBLP:conf/pldi/BeutnerOZ22}, integral approximation is used to derive guaranteed bounds on a program's posterior distribution.
Following up on this, \cite{wang2024static} extended the computation of such bounds to a class of programs with soft conditioning (\emph{score-at-end} programs).
Previously, \cite{DBLP:conf/pldi/SankaranarayananCG13} and \cite{DBLP:journals/pacmpl/AlbarghouthiDDN17} used integral approximations to bound the satisfaction probabilities of properties like safety and fairness.
However, these approaches are not able to deal with unbounded loops in full generality.
AQUA \cite{DBLP:journals/isse/HuangDM22} uses a different discretization method, where the posterior of a given program is approximated by quantizing the state space.
However, this approximation does not necessarily provide an upper or lower bound of the true distribution, and unbounded loops are not supported.
Finally, \cite{DBLP:conf/pldi/Wang0R21} propose a method to derive upper and lower bounds on central moments of cost accumulators.

\emph{Loop Invariant Analysis.} 
The main purpose of the current work is to enable automatic verification of inductive invariants for continuous probabilistic programs. Reasoning about probabilistic loops via quantitative invariants was first introduced by McIver and Morgan \cite{DBLP:series/mcs/McIverM05} for probabilistic programs with discrete probabilistic choice and extended by various authors to different settings \cite{DBLP:conf/setss/SzymczakK19,DBLP:journals/pe/GretzKM14,DBLP:conf/lics/KaminskiK17}.
A generalization of this framework to programs featuring continuous distribution was proposed in \cite{DBLP:conf/cav/ChakarovS13} and uses super-martingale ranking functions as the probabilistic analogue of classic ranking functions.
The super-martingale approach has been specialized to qualitative and quantitative termination problems for various classes of programs such as polynomial programs \cite{DBLP:conf/cav/ChatterjeeFG16}, affine programs admitting a linear ranking super-martingale \cite{DBLP:conf/popl/ChatterjeeFNH16}, and non-deterministic probabilistic programs \cite{DBLP:conf/popl/ChatterjeeNZ17,DBLP:journals/pacmpl/AgrawalC018}.
Besides termination, this approach has been successfully applied also to cost analysis \cite{DBLP:conf/pldi/Wang0GCQS19} and equivalence refutation \cite{DBLP:journals/pacmpl/ChatterjeeGNZ24}.
While building on the martingale approach, \cite{DBLP:conf/cav/ChatterjeeGMZ22} overcomes the necessity to compute ranking martingales, by introducing the notion of stochastic invariants indicators, that can be used to prove termination in a sound and relatively complete way.
A different generalization was proposed in \cite{DBLP:conf/sas/ChakarovS14}, where expectation invariants are proposed to introduce a generalized fixed-point framework for programs with continuous distributions and unbounded loops.
However, this analysis rules out invariants expressed as Iverson brackets and is restricted to piecewise linear assertion guards and updates. Similarly, \cite{DBLP:conf/pldi/WangS0CG21}, restricts to affine programs and invariants to find exponential bounds on assertion violation probabilities.

\emph{Exact Inference.}
Exact inference on probabilistic programs involving continuous distributions is especially hard because of the challenge of computing densities via integrals.
Exact symbolic inference engines like \cite{DBLP:conf/cav/GehrMV16, DBLP:conf/pldi/GehrSV20, narayanan2016probabilistic, DBLP:conf/pldi/SaadRM21} only support deterministically bounded loops.
Even within this constrained context, the occurrence of non-simplified integrals in the output can significantly impede the potential to perform appropriate analysis or further computations.
A different, yet still exact approach is pursued in \cite{DBLP:journals/pacmpl/MoosbruggerSBK22}, where the focus is on the computation of moments as functions of the loop iteration.
In order to compute the moments exactly, the syntax of the programs is restricted quite severely, though parametric models are supported. Under the same syntactic restrictions, it has been shown in \cite{DBLP:journals/fmsd/MoosbruggerBKK22} that termination probabilities can be computed automatically.