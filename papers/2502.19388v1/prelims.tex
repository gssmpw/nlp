\section{Preliminaries}
\label{sec:prelims}

In this section, we set up our notations and treat the fixed point-theoretic foundations we rely on.


\subsection{General Notation}

$\nats$ is the set of non-negative integers.
The set of non-negative \emph{extended reals} is $\ennReals = \nnReals \cup \{\infty\}$.
We adopt the following standard conventions:
For all $x \in \ennReals$, we let $x \leq \infty$ and $x + \infty = \infty + x = \infty$.
Moreover, we define $\infty \cdot 0 = 0 \cdot \infty = 0$ and $x \cdot \infty = \infty \cdot x = \infty$ for all $x > 0$.
Given $\ivalL, \ivalR \in \reals$, we denote by $\clIvalGen$ the real closed interval with endpoints $\ivalL$ and $\ivalR$.
The set of truth values is $\bools = \{\boolConstFalse, \boolConstTrue\}$.
Given a predicate $\guard \colon A \to \bools$ over a set $A$, we define the \emph{Iverson bracket}
%
\begin{align*}
    \iv{\guard} \colon A \to \{0,1\},~ a \mapsto
    \begin{cases}
        0 & \text{ if } \guard(a) = \boolConstFalse~, \\
        1 & \text{ if } \guard(a) = \boolConstTrue~. 
    \end{cases}
\end{align*}
%
For example, assuming that it is understood from context that $x$ is a real number, then $\iv{x \in \rats}$ denotes the \emph{Dirichlet function} that sends every $x \in \reals$ to $1$ if $x$ is rational, and to $0$ otherwise.

Lambda notation $\lam{x}{\aExp}$, where $\aExp$ is some mathematical expression with free variable $x$, is used to introduce unnamed functions whose domain and codomain will be clear from the context.


\subsection{Fixed Point Theory}
\label{sec:prelims:fixpoint}

We introduce the foundations from fixed point theory \cite[Section 5.4]{DBLP:books/daglib/0070910} required to sensibly reason about the semantics of loops and to obtain suitable notions of quantitative loop invariants.

Let $\poGen$ and $\poGenb$ be partial orders.
A function $\fun \colon \poDom \to \poDomb$ is called \emph{monotonic}, if for all $\poElem \poleq \poElemb$, we have $\fun(\poElem) \poleqb \fun(\poElemb)$.
%
An \emph{$\omega$-chain} in $\poGen$ is a monotonic function $\poElem \colon \nats \to \poDom$ (where $\nats$ is ordered by the usual $\leq$ relation), i.e., an $\omega$-chain is a non-decreasing sequence of elements $\poElem(0) \poleq \poElem(1) \poleq \ldots$ from $\poDom$.
%
The partial order $\poGen$ is an \emph{$\omega$-complete partial order} ($\omega$-cpo, for short), if all $\omega$-chains $\poElem$ in $\poDom$ have a \emph{supremum} (least upper bound) $\sup_{i \in \nats} \poElem(i)$ in $\poGen$. 
An $\omega$-cpo $\poGen$ with an element $\poBot \in \poDom$ satisfying $\poBot \poleq \poElem$ for all $\poElem$ is called \emph{$\omega$-cpo with bottom}.
%
A function $\fun \colon \poDom \to \poDomb$ between $\omega$-cpos $\poGen$ and $\poGenb$ is called \emph{$\omega$-continuous}, if for all $\omega$-chains $\poElem$ in $\poDom$ it holds that $\sup_{i \in \nats} \fun(\poElem(i)) =  \fun(\sup_{i \in \nats} \poElem(i))$.
Let $\fun \colon \poDom \to \poDom$ be a function where $\poDom$ is an arbitrary set.
An element $\poElem \in \poDom$ satisfying $\fun(\poElem) = \poElem$ is called a \emph{fixed point} of $\fun$.
The following is often attributed to Kleene:
%
\begin{theorem}[Kleene's Fixed Point Theorem~\textnormal{\cite[Theorem~5.11]{DBLP:books/daglib/0070910}}]
    \label{thm:kleene}
    Let $\poGen$ be an $\omega$-cpo with bottom and let $\fun \colon \poDom \to \poDom$ be $\omega$-continuous.
    Then $\fun$ has a least fixed point $\lfp\fun \in \poDom$ and $\lfp\fun = \sup_{i \in \nats} f^i(\poBot)$ where $\fun^i(\poBot)$ denotes the $i$-fold application of $\fun$ to $\poBot$.
\end{theorem}

A partial order $\poGen$ is said to be \emph{$\omega$-\underline{co}complete} ($\omega$-cocpo) if all \emph{$\omega$-\underline{co}chains}, i.e., non-\emph{increasing} sequences $\poElem(0) \pogeq \poElem(1) \pogeq \ldots$ in $\poDom$ have an infimum $\inf_{i \in \nats}\poElem(i) \in \poDom$.
An $\omega$-cocpo with a greatest element $\poTop$ is called $\omega$-cocpo \emph{with top}.
Note that $\poGen$ is an $\omega$-cocpo (with top) iff the reversed partial order $\po{\poDom}{\pogeq}$ is an $\omega$-cpo (with bottom).
Similarly, a function $\fun \colon \poDom \to \poDomb$ between $\omega$-cocpos $\poGen$ and $\poGenb$ is called \emph{$\omega$-\underline{co}continuous}, if $\inf_{i \in \nats} \fun(\poElem(i)) =  \fun(\inf_{i \in \nats} \poElem(i))$ for all non-increasing $\poElem$.
Note that for an $\omega$-cocpo $\po{\poDom}{\poleq}$ with top, \Cref{thm:kleene} reads as follows:
If $\fun \colon \poDom \to \poDom$ is $\omega$-\underline{co}continuous, then $\fun$ has a \emph{greatest} fixed point $\gfp\fun = \inf_{i \in \nats} \fun^i(\top)$.

A partial order is called \emph{$\omega$-bicomplete} ($\omega$-bicpo), if it is both an $\omega$-cpo and $\omega$-cocpo.
Similarly, a function $\fun$ between $\omega$-bicpos is called \emph{$\omega$-bicontinuous}, if it is both 
$\omega$-continuous and $\omega$-\underline{co}continuous.

A partial order $\poGen$ is a \emph{complete lattice} if for all $\poSubset \subseteq \poDom$ there exists $\sup \poSubset \in \poDom$ and $\inf \poSubset \in \poDom$.
Note that every complete lattice is an $\omega$-bicpo with bottom $\poBot = \sup \emptyset$ and top $\poTop = \inf \emptyset$.


\begin{theorem}[Knaster-Tarski Theorem~\textnormal{\cite[Theorems 5.15 and 5.16]{DBLP:books/daglib/0070910}}]
    \label{thm:knasterTarski}
    Let $\fun \colon \poDom \to \poDom$ be a monotonic function on the complete lattice $\poGen$.
    Then $\fun$ has a \emph{least} and \emph{greatest fixed point} $\lfp\fun \in \poDom$ and $\gfp\fun \in \poDom$.
    Moreover, for all $\poElem \in \poDom$ it holds that
    \begin{align*}
        \fun(\poElem) \poleq \poElem
        \quad\text{implies}\quad
        \lfp\fun \poleq \poElem
        \quad\qqand\quad
        \poElem \poleq \fun(\poElem)
        \quad\text{implies}\quad
        \poElem \poleq \gfp\fun
        ~.
    \end{align*}
\end{theorem}
%
The above implications are often referred to as \emph{Park (co)induction}~\cite{park1969fixpoint}.
