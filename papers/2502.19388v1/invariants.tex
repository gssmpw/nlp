\section{Invariant-Based Reasoning for Loops}
\label{sec:invariants_and_unrolling}
%
Deductive probabilistic program verification techniques typically bound expected outcomes of loops by means of \emph{quantitative loop invariants}. Intuitively, a quantitative loop invariant $\exI$ is an expectation whose pre-expectation w.r.t.\ \emph{one loop iteration} does not increase (or decrease, depending on whether one wants to establish upper or lower bones, see below).
While quantitative loop invariant-based reasoning can simplify the verification of loops significantly, 
%
the continuous setting poses challenges: Computing the pre-expectation of $\exI$ w.r.t.\ a loop's body requires reasoning about possibly complex Lebesgue integrals --- involving, e.g., indicator functions of predicates --- arising from the continuous sampling instructions.
%
We now develop invariant-based proof rules for our \emph{Riemann pre-expectations} (\Cref{def:riemannwpwlp}), which yield sound bounds on the \emph{Lebesgue pre-expectations} (\Cref{def:wpwlp}).
We will see in \Cref{sec:syntax,sec:case_studies} that these proof rules give rise to SMT-based techniques for verifying bounds on Lebesgue pre-expectations of loops in a semi-automated fashion.
%
%

%
%
Our first insight is that --- due to $(\exps, \, \eleq)$ and $(\bexps, \, \eleq)$ being complete lattices and the Riemann expectation transformers being monotonic by \Cref{thm:monriemann} --- bounds on \emph{Riemann} pre-expectations of loops can be established via Park induction (\Cref{thm:knasterTarski}):
%
%
\begin{lemma}
	\label{thm:invariant_approx}
	Let $\prog = \WHILE{\guard}{\progBody} \in\pWhile$ and $\N \geq 1$. Then:
	%
    \begin{enumerate}
		\item For all $\ex,\exI \in \exps$, we have
		%
		$
			\underbrace{\charfunuwp{\N}{\prog}{\ex}(\exI) \eleq \exI}_{\text{$\exI$ is $\uwpSymb{\N}$-superinvariant of $\prog$ w.r.t.\ $\ex$}}
			\qquad\text{implies}\qquad
			%
			\uwp{\N}{\prog}{\ex} \eleq \exI~.
		$
		%
		%
		\item For all $\ex,\exI \in \bexps$, we have
		%
		$
            \underbrace{\exI \eleq \charfunlwlp{\N}{\prog}{\ex}(\exI)}_{\text{$\exI$ is $\lwlpSymb{\N}$-subinvariant of $\prog$ w.r.t.\ $\ex$}}
            \qquad\text{implies}\qquad
            %
            \exI \eleq \lwlp{\N}{\prog}{\ex}~.
		$
	\end{enumerate}
\end{lemma}
%
%
More colloquially stated, \emph{super}invariants yield \emph{upper} bounds on \emph{upper} Riemann pre-expectations of loops and, dually, \emph{sub}invariants yield \emph{lower} bounds on \emph{lower liberal} Riemann pre-expectations. Notice that establishing the premise of the above proof rules only requires reasoning about the loop's {body} and avoids explicitly computing Lebesgue integrals.

It thus follows from the soundness of our Riemann expectation transformers (\Cref{thm:soundness}) that the above proof rules yield sound bounds on \emph{Lebesgue} pre-expectations.
%
\begin{theorem}
	\label{thm:invariant_cont}
	Let $\prog = \WHILE{\guard}{\progBody} \in\pWhile$ and $\N,\N' \geq 1$. We have:
	%
	\begin{enumerate}
		\item If $\exI \in \exps$ is a $\uwpSymb{\N}$-superinvariant of $\prog$ w.r.t.\ $\ex\in\expsmeas$, then
		%
		$\wp{\prog}{\ex} \eleq \exI~$.
		%
		\item If $\exI \in \bexps$ is a $\lwlpSymb{\N}$-subinvariant of $\prog$ w.r.t.\ $\ex\in\bexpsmeas$, then
		%
		$\exI \eleq \wlp{\prog}{\ex}$.
		%
		\item If $\exI \in \exps$ is a $\uwpSymb{\N}$-superinvariant of $\prog$ w.r.t.\ $\ex\in\expsmeas$ and $\exJ \in \bexps$ is a $\lwlpSymb{\N'}$-subinvariant of $\prog$ w.r.t.\ $1$, then, for all $\pState\in\pStates$, $\exJ(\pState) >0$ implies that $\cwp{\prog}{\ex}(\pState)$ is defined and
		%
		\[
			\cwp{\prog}{\ex}(\pState) \lleq \frac{\exI(\pState)}{\exJ(\pState)}~.
		\]
	\end{enumerate}
	%
\end{theorem}
%
It is important to note that in \Cref{thm:invariant_approx} we admit \emph{arbitrary} ($1$-bounded) post-expectations whereas in \Cref{thm:invariant_cont} we have to restrict to \emph{measurable} ($1$-bounded) post-expectations in order for the Lebesgue pre-expectations $\wp{\prog}{\ex}$ and $\wlp{\prog}{\ex}$ to be well-defined.
The quantitative loop invariant $\exI$, on the other hand, must \emph{not} necessarily be measurable since there is no need to plug $\exI$ into a Lebesgue expectation transformer.
%
%
%
\begin{example}
	\label{ex:invariant_monte_carlo} 
	%
	Consider the Monte Carlo $\pi$-approximator $\prog$ from \Cref{fig:intro}.
	%
	The expectation
	%
	\[
	   \exI \eeq \pVarcount + \iv{\pVari \leq \pVarm} \cdot \big(0.85\cdot ((\pVarm \monus \pVari) + 1)\big)
	\]
	%
	is a $\uwpSymb{16}$-superinvariant of $\prog$ w.r.t.\ $\pVarcount$ (here $\monus$ denote the \emph{monus} operator defined as $\synTerm_1 \synMonus \synTerm_2 = \max(\synTerm_1 - \synTerm_2, 0)$, detailed in \Cref{sec:syntax:defs}).
    Hence, we get by \Cref{thm:invariant_cont},
	%
	$\wp{\prog}{\pVarcount} \eleq \exI$,
	%
	i.e., if initially $\pVarcount = 0, \pVari = 1$, then $0.85\cdot \pVarm $ upper-bounds the expected final value of $\pVarcount$. 
\end{example}
