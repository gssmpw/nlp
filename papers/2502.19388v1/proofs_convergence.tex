\subsection{Proof of \Cref{thm:wpLoopFreeCocontLocBounded}}
\label{proof:wpLoopFreeCocontLocBounded}

Let $\expsmeaslb$ be the set of measurable locally bounded expectations.

\begin{restatable}{lemma}{wpLoopFreeCocontLocBounded}
    \label{thm:wpLoopFreeCocontLocBounded}
    Let $\aExps$ be a set of locally bounded measurable arithmetic expressions (\enquote{right hand sides}) and $\guards$ be an arbitrary set of guards.
    Let $\prog \in \pWhileWith{\aExps}{\guards}$ be loop-free.
    Then $\wpTrans{\prog}$ preserves $\expsmeaslb$.
    Moreover, $\wpTrans{\prog}$ preserves limits of non-increasing sequences from $\expsmeaslb$.
\end{restatable}
\begin{proof}
    By induction on the structure of $\prog$.
    Let $\ex \in \expsmeaslb$ be locally bounded and let $\ex_0 \egeq \ex_1 \egeq \ldots$ be a non-increasing sequence of expectations from $\expsmeaslb$.
    
    \inductionCase{$\prog = \SKIP$}
    Trivial.
    
    \inductionCase{$\prog = \DIVERGE$}
    Trivial.
    
    \inductionCase{$\prog = \OBSERVE{\guard}$}
    Straightforward.
    
    \inductionCase{$\prog = \ASSIGN{\pVar}{\aExp}$}
    We have to show that $\wp{\ASSIGN{\pVar}{\aExp}}{\ex} = \exSubsGen = \lam{\pSt}{\ex(\pStUpdate{\pSt}{\pVar}{\aExp(\pSt)})}$ is locally bounded, i.e.,
    that
    $\sup_{\pSt \in \compactSet} \ex(\pStUpdate{\pSt}{\pVar}{\aExp(\pSt)}) < \infty$
    for all compact $\compactSet \subseteq \pStates$.
    
    Let us fix a compact $\compactSet \subseteq \pStates$.
    Since $\aExp$ is locally bounded by assumption, we have $\aExp(\compactSet) \subseteq [0,b]$ for some $b \in \nonNegReals$.
    Now let $\compactSet' \supseteq \compactSet$ be a compact set such that for all $\pSt \in \compactSet$ and $\xi \in [0,b]$ we have $\pStUpdate{\pSt}{\pVar}{\xi} \in \compactSet'$. To see that $\compactSet'$ exists consider the projection of $\compactSet$ on $\mathbb{R}^{\pVars \setminus \{ \pVar \}}$, denoted by $\pi(\compactSet)$. By Theorem 39 in \cite{pugh2002real} $\pi(\compactSet)$ is compact. Then set $\compactSet' = \pi(\compactSet) \times [0,b]$. By Theorem 31 in \cite{pugh2002real} $\compactSet'$ is compact and $\pStUpdate{\pSt}{\pVar}{\xi} \in \compactSet'$ for all $\pSt \in \compactSet$ and $\xi \in [0,b]$.
    
    Then we have
    \[
    \{\ex(\pStUpdate{\pSt}{\pVar}{\aExp(\pSt)}) \mid \pSt \in \compactSet\}
    ~\subseteq~
    \{\ex(\pStUpdate{\pSt}{\pVar}{\xi}) \mid \pSt \in \compactSet, \xi \in [0,b]\}
    ~\subseteq~
    \{\ex(\pSt) \mid \pSt \in \compactSet'\}
    \]
    and thus $\sup_{\pSt \in \compactSet} \ex(\pStUpdate{\pSt}{\pVar}{\aExp(\pSt)}) \leq \sup_{\pSt \in \compactSet'} \ex(\pSt) < \infty$ because $\ex$ is locally bounded.
    
    \inductionCase{$\prog = \UNIFASSIGN{\pVar}$}
    We first show that
    \[
    \wp{\UNIFASSIGN{\pVar}}{\ex} = \lam{\st}{\int_\uIval \ex(\pStUpdate{\st}{\pVar}{\xi}) \,d\lebmes(\xi)}
    \]
    remains locally bounded.
    To this end let $\compactSet \subseteq \pStates$ be compact.
    Recall that we have to show that $\sup \wp{\UNIFASSIGN{\pVar}}{\ex}(\compactSet) < \infty$.
    Let $\compactSet' \supseteq \compactSet$ be a compact set such that for all $\pSt \in \compactSet$ and $\xi \in \uIval$ we have $\pStUpdate{\pSt}{\pVar}{\xi} \in \compactSet'$, where $\compactSet'$ can be constructed as above.
    Since $\ex$ is locally bounded we have $\sup \ex(\compactSet') = b \in \nonNegReals$.
    Hence for all $\pSt \in \compactSet'$,
    \begin{align*}
        & \int_\uIval \ex(\pStUpdate{\st}{\pVar}{\xi}) \,d\lebmes(\xi) \\
        \lleq & \int_\uIval \sup \ex(\compactSet') \,d\lebmes(\xi) \\
        \eeq & \int_\uIval b \,d\lebmes(\xi) \\
        \eeq & b 
    \end{align*}
    It follows that $\sup \wp{\UNIFASSIGN{\pVar}}{\ex}(\compactSet) \leq \sup \wp{\UNIFASSIGN{\pVar}}{\ex}(\compactSet') \leq b < \infty$ and hence the claim.
    
    The fact that $\wpSymb$ preserve infima of non-increasing sequences follows from the properties of pointwise $\inf$.
    
    The fact that $\wpTrans{\UNIFASSIGN{\pVar}}$ preserves infima of non-increasing sequences $( \ex_i )$ follows from an extended version of the Monotone Convergence \cite[Theorem 12.1 (ii)]{schilling2017measures} adapted to non-increasing sequences of functions. Notably, the corollary requires that the integral of the first function in the sequence is smaller than $\infty$, but this is guaranteed by the restriction to locally bounded expectations, since for every $\pSt \in \pStates$ we can consider the compact set $\compactSet_\pSt = \{\pStUpdate{\pSt}{\pVar}{\xi} \mid \xi \in \uIval\}$, and since $\ex_0$ is locally bounded, we find that $\int_\uIval \ex_0(\pStUpdate{\st}{\pVar}{\xi}) \,d\lebmes(\xi) \leq \sup \compactSet_\pSt < \infty$.
    
    \inductionCase{$\prog = \ITE{\guard}{\prog_1}{\prog_2}$}
    Local boundedness follows by applying the I.H.\ to $\prog_1$ and $\prog_2$, while preservation of the limit can be proved using the same argument as in the proof \Cref{thm:wpWlpWellDefinedAndContinuous}.
    
    \inductionCase{$\prog = \PCHOICE{\prog_1}{\prob}{\prog_2}$}
    Local boundedness follows by applying the I.H.\ to $\prog_1$ and $\prog_2$, while preservation of the limit can be proved using the same argument as in the proof \Cref{thm:wpWlpWellDefinedAndContinuous}.
    
    \inductionCase{$\prog = \SEQ{\prog_1}{\prog_2}$}
    Since $\wp{\SEQ{\prog_1}{\prog_2}}{\ex} = \wp{\prog_1}{\wp{\prog_2}{\ex}}$ it follows immediately from the I.H.\ that $\wp{\prog_1}{\wp{\prog_2}{\ex}} \in \expsmeaslb$.
    
    To conclude the proof observe that
    \begin{align*}
        &\inf_{i \in \nats} \wp{\SEQ{\prog_1}{\prog_2}}{\ex_i} \\
        \eeq &\inf_{i \in \nats} \wp{\prog_1}{\wp{\prog_2}{\ex_i}} \\
        \eeq & \wp{\prog_1}{\inf_{i \in \nats} \wp{\prog_2}{\ex_i}} \tag{I.H.\ on $\prog_1$, using that $\wp{\prog_2}{\ex_i}$ is a non-decreasing sequence in $\expsmeaslb$ by I.H.\ on $\prog_2$} \\
        \eeq & \wp{\prog_1}{\wp{\prog_2}{\inf_{i \in \nats}  \ex_i}} \tag{I.H.\ on $\prog_2$} \\
        \eeq & \wp{\SEQ{\prog_1}{\prog_2}}{\inf_{i \in \nats}  \ex_i}
    \end{align*}
\end{proof}


\subsection{Proof of \Cref{thm:convLoopFree}}
\label{proof:convLoopFree}
%
\convLoopFree*
%
\begin{proof}
    In the rest of the proof we write  $\sup_\N$ and $\inf_\N$ instead of $\sup_{\N \geq 1}$ and $\inf_{\N \geq 1}$, respectively.
    We will show the following \emph{alternative} statement:
    For all loop-free $\prog \in \pWhileWith{\aExps}{\guards}$ and post-expectations $\ex \in \expsClass$:
    \eqref{eq:approxInClass} from above holds, and
    %
    \begin{align}
        \wp{\prog}{\ex}
        \eeq
        \sup_\N \lwp{2^\N}{\prog}{\ex}
        \eeq
        \inf_\N \uwp{2^\N}{\prog}{\ex}
        ~.
        \tag{\ref{eq:convLoopFree}'}
        \label{eq:convLoopFreeAlt}
    \end{align}
    %
    Note that the difference between \eqref{eq:convLoopFreeAlt} and $\eqref{eq:convLoopFree}$
    is that we consider the supremum over powers of two.
    This simplifies the situation as $\lwp{2^\N}{\prog}{\ex}$ and $\uwp{2^\N}{\prog}{\ex}$ are non-decreasing (non-increasing, respectively) sequences in $\N$ (see \Cref{thm:powerTwoMonotonic}), which allows us to apply \Cref{thm:oneVsTwoSups} at a crucial point in the proof.
    
    To show that the alternative statement implies the claim of \Cref{thm:convLoopFree} we argue as follows:
    By properties of suprema, $\sup_\N \lwp{2^\N}{\prog}{\ex} \leq \sup_\N \lwp{\N}{\prog}{\ex}$.
    We also have
    %
    \begin{align*}
        & \sup_\N \lwp{\N}{\prog}{\ex} \\
        %
        \lleq & \wp{\prog}{\ex} \tag{by \Cref{thm:soundness}} \\
        %
        \eeq & \sup_\N \lwp{2^\N}{\prog}{\ex} \tag{by the alternative statement \eqref{eq:convLoopFreeAlt}} ~,
    \end{align*}
    %
    and thus $\sup_\N \lwp{2^\N}{\prog}{\ex} = \sup_\N \lwp{\N}{\prog}{\ex}$.
    The argument for $\inf_\N \uwp{2^\N}{\prog}{\ex} = \inf_\N \uwp{\N}{\prog}{\ex}$ is analogous.
    
    The proof of the alternative statement \eqref{eq:convLoopFreeAlt} is by induction over the structure of $\prog$.
    In the following, we let $\ex \in \expsClass$ and $\N \in \nats$ be arbitrary.
    
    We consider the base cases first.
    For the base cases $\prog \in \{\SKIP, \DIVERGE, \ASSIGN{\pVar}{\aExp}, \OBSERVE{\guard} \}$ the definitions of $\lwpSymb \N$, $\uwpSymb \N$, and $\wpSymb$ coincide, so we only have to show that $\wp{\prog}{\ex} \in \expsClass$.
    
    
    \inductionCase{$\prog = \SKIP$}
    %
    By definition, $\wp \SKIP \ex = \ex \in \expsClass$.
    
    \inductionCase{$\prog = \DIVERGE$}
    %
    By definition, $\wp \DIVERGE \ex = 0$.
    By assumption, $0 \in \expsClass$.
    
    \inductionCase{$\prog = \ASSIGN{\pVar}{\aExp}$}
    %
    By definition, $\wp{\ASSIGN{\pVar}{\aExp}}{f} = \exSubs{\ex}{\pVar}{\aExp}$.
    By assumption, $\exSubs{\ex}{\pVar}{\aExp} \in \expsClass$.
    
    \inductionCase{$\prog = \OBSERVE{\guard}$}
    %
    By definition, $\wp{\OBSERVE{\guard}}{f} = \iv{\guard} \cdot \ex$.
    By assumption, $\iv{\guard} \cdot \ex + \iv{\neg\guard} \cdot 0 \in \expsClass$.
    
    \inductionCase{$\prog = \UNIFASSIGN{\pVar}$}
    %
    This base case is more interesting and crucially relies on Riemann-integrability.
    We first show that $\lwp{\N}{\UNIFASSIGN{\pVar}}{\ex} \in \expsClass$.
    By definition of $\lwpSymb{\N}$,
    \begin{align*}
        \lwp{\N}{\UNIFASSIGN{\pVar}}{\ex}
        \eeq
        \frac{1}{\N} \sum_{i=0}^{\N-1} \inf_{\xi \in [\frac{i}{\N}, \frac{i+1}{\N}]} \exSubs{\ex}{\pVar}{\xi}
        ~.
    \end{align*}
    The right-hand side is a finite convex combination of suprema over subintervals of $\uIval$, hence it is in $\expsClass$ by assumption.
    Next we show
    $\sup_\N \lwp{2^\N}{\UNIFASSIGN{\pVar}}{\ex} = \wp{\UNIFASSIGN{\pVar}}{\ex}$:
    \begin{align*}
        & \sup_\N \lwp{2^\N}{\UNIFASSIGN{\pVar}}{\ex} \\
        %
        \eeq & \sup_N \frac{1}{2^\N} \sum_{i=0}^{2^\N-1} \inf_{\xi \in [\frac{i}{2^\N}, \frac{i+1}{2^\N}]}
        \exSubs{\ex}{\pVar}{\xi} \tag{definition of $\lwpSymb{2^\N}$} \\
        %
        \eeq & \lam{\st}{} \lowerInt{0}{1} \ex(\pStUpdate{\st}{\pVar}{\xi}) \,d\xi \tag{by \Cref{thm:smallNormSuffices}} \\
        %
        \eeq & \lam{\st}{} \int_{0}^{1} \ex(\pStUpdate{\st}{\pVar}{\xi}) \,d\xi \tag{by assumption $\ex \in \expsClass$ is bounded and Riemann-integrable w.r.t.\ $\pVar$} \\
        %
        \eeq & \lam{\st}{} \int_\uIval  \ex(\pStUpdate{\st}{\pVar}{\xi}) \,d\lebmes(\xi) \tag{Riemann integral = Lebesgue integral, see \Cref{thm:riemannEqualsLebesgue}} \\
        %
        \eeq & \wp{\UNIFASSIGN{\pVar}}{\ex} \tag{definition of $\wpSymb$}
    \end{align*}
    The proof for $\uwpSymb{}$ is exactly analogous (replace all infima by suprema and vice versa).
    
    \inductionCase{$\prog = \ITE{\guard}{\prog_1}{\prog_2}$}
    %
    $\lwp{\N}{\ITE{\guard}{\prog_1}{\prog_2}}{\ex}
    = [\guard] \cdot \lwp{\N}{\prog_1}{\ex} + [\neg\guard] \cdot \lwp{\N}{\prog_2}{\ex} \in \expsClass$ by I.H. on $\prog_1$ and $\prog_2$ and assumption.
    Furthermore,
    \begin{align*}
        %
        & \sup_\N \lwp{2^\N}{\ITE{\guard}{\prog_1}{\prog_2}}{\ex} \\
        %
        \eeq & \sup_\N \left(\iv{\guard} \cdot \lwp{2^\N}{\prog_1}{\ex} + \iv{\neg\guard} \cdot \lwp{2^\N}{\prog_2}{\ex}\right) \tag{definition of $\lwpSymb{\N}$} \\
        %
        \eeq & \iv{\guard} \cdot \sup_\N \lwp{2^\N}{\prog_1}{\ex} + \iv{\neg\guard} \cdot \sup_N\lwp{2^\N}{\prog_2}{\ex} \tag{addition and multiplication by Iverson brackets is $\omega$-continuous} \\
        %
        \eeq & \iv{\guard} \cdot \wp{\prog_1}{\ex} + \iv{\neg\guard} \cdot \wp{\prog_2}{\ex} \tag{by I.H. on $\prog_1$ and $\prog_2$} \\
        %
        \eeq & \wp{\ITE{\guard}{\prog_1}{\prog_2}}{\ex} \tag{definition of $\wpSymb$}
        ~.
    \end{align*}
    %
    Again, the proof for $\uwpSymb{}$ is analogous (consider infima instead of suprema, and use that addition and multiplication by Iverson brackets is $\omega$-\underline{co}continuous).
    
    \inductionCase{$\prog = \PCHOICE{\prog_1}{\prob}{\prog_2}$}
    %
    $\lwp{\N}{\PCHOICE{\prog_1}{\prob}{\prog_2}}{\ex}
    = \prob \cdot \lwp{\N}{\prog_1}{\ex} + (1-\prob) \cdot \lwp{\N}{\prog_2}{\ex} \in \expsClass$ by I.H. on $\prog_1$ and $\prog_2$ and assumption.
    Furthermore,
    %
    \begin{align*}
        & \sup_\N \lwp{2^\N}{\PCHOICE{\prog_1}{\prob}{\prog_2}}{\ex} \\
        %
        \eeq & \sup_\N \left(\prob \cdot \lwp{2^\N}{\prog_1}{\ex} + (1-\prob) \cdot \lwp{2^\N}{\prog_2}{\ex}\right) \tag{definition of $\lwpSymb{\N}$} \\
        %
        \eeq & \prob \cdot \sup_\N \lwp{2^\N}{\prog_1}{\ex} + (1-\prob) \cdot \sup_\N\lwp{2^\N}{\prog_2}{\ex} \tag{addition and multiplication by constants is $\omega$-continuous} \\
        %
        \eeq & \prob \cdot \wp{\prog_1}{\ex} + (1-\prob) \cdot \wp{\prog_2}{\ex} \tag{by I.H. on $\prog_1$ and $\prog_2$} \\
        %
        \eeq & \wp{\PCHOICE{\prog_1}{\prob}{\prog_2}}{\ex} \tag{definition of $\wpSymb$}
        ~.
    \end{align*}
    %
    As before, the proof for $\uwpSymb{}$ is analogous (consider infima instead of suprema, and use that addition and multiplication by constants is $\omega$-\underline{co}continuous).
    
    \inductionCase{$\prog = \SEQ{\prog_1}{\prog_2}$}
    First, note that $\lwp{\N}{\SEQ{\prog_1}{\prog_2}}{\ex} = \lwp{\N}{\prog_1}{\lwp{\N}{\prog_2}{\ex}} \in \expsClass$ by I.H. on $\prog_2$ and $\prog_1$.
    For convergence we argue as follows:
    %
    \begin{align*}
        & \sup_\N \lwp{2^\N}{\SEQ{\prog_1}{\prog_2}}{\ex} \\
        %
        \eeq & \sup_\N \lwp{2^\N}{\prog_1}{\lwp{2^\N}{\prog_2}{\ex}}  \tag{definition of $\lwpSymb{\N}$} \\
        %
        \eeq & \sup_{\Nb} \sup_\N \lwp{2^\N}{\prog_1}{\lwp{2^\Nb}{\prog_2}{\ex}} \tag{\Cref{thm:oneVsTwoSups} and \Cref{thm:powerTwoMonotonic}} \\
        %
        \eeq & \sup_{\Nb} \wp{\prog_1}{\lwp{2^\Nb}{\prog_2}{\ex}} \tag{I.H.\ on $\prog_1$, using that $\lwp{2^\Nb}{\prog_2}{\ex} \in \expsClass$ (also by I.H.)} \\
        %
        \eeq & \wp{\prog_1}{\sup_{\Nb} \lwp{2^\Nb}{\prog_2}{\ex}} \tag{$\omega$-continuity of $\wpTrans{\prog_1}$, see \Cref{thm:wpWlpWellDefinedAndContinuous}} \\
        %
        \eeq & \wp{\prog_1}{\wp{\prog_2}{\ex}}{\ex} \tag{I.H.\ on $\prog_2$} \\
        %
        \eeq & \wp{\SEQ{\prog_1}{\prog_2}}{\ex} \tag{definition of $\wpSymb$}
    \end{align*}
    %
    The convergence proof of $\uwpSymb{}$ is not fully analogous as $\wpSymb$ is not $\omega$-\underline{co}continuous in general.
    Instead, we invoke \Cref{thm:wpLoopFreeCocontLocBounded}:
    %
    \begin{align*}
        & \inf_\N \uwp{2^\N}{\SEQ{\prog_1}{\prog_2}}{\ex} \\
        %
        \eeq & \inf_\N \uwp{2^\N}{\prog_1}{\uwp{2^\N}{\prog_2}{\ex}}  \tag{definition of $\uwpSymb{\N}$} \\
        %
        \eeq & \inf_{\Nb} \inf_\N \uwp{2^\N}{\prog_1}{\uwp{2^\Nb}{\prog_2}{\ex}} \tag{\enquote{$\inf$-version} of \Cref{thm:oneVsTwoSups} and \Cref{thm:powerTwoMonotonic}} \\
        %
        \eeq & \inf_{\Nb} \wp{\prog_1}{\uwp{2^\Nb}{\prog_2}{\ex}} \tag{I.H.\ on $\prog_1$, using that $\uwp{2^\Nb}{\prog_2}{\ex} \in \expsClass$ (also by I.H.)} \\
        %
        \eeq & \wp{\prog_1}{\inf_{\Nb} \uwp{2^\Nb}{\prog_2}{\ex}} \tag{\Cref{thm:wpLoopFreeCocontLocBounded}} \\
        %
        \eeq & \wp{\prog_1}{\wp{\prog_2}{\ex}}{\ex} \tag{I.H.\ on $\prog_2$} \\
        %
        \eeq & \wp{\SEQ{\prog_1}{\prog_2}}{\ex} \tag{definition of $\wpSymb$}
    \end{align*}
\end{proof}

\subsection{{$\wlpSymb$-Version of \Cref{thm:convLoopFree}}}
\label{app:wlpVersion}
\begin{lemma}
    \label{thm:convLoopFreeWlp}
    Let $(\aExps, \guards, \expsClass)$ be Riemann-suitable.
    Then for all loop-free $\prog \in \pWhileWith{\aExps}{\guards}$ and post-expectations $\exb \in \expsClass  \cap \bexpsmeas$ the following holds:
    \begin{align*}
        \forall \N \geq 1 \colon\quad \lwlp{\N}{\prog}{\exb} \in \expsClass \qand \uwlp{\N}{\prog}{\exb} \in \expsClass
    \end{align*}
    and
    \begin{align*}
        \wlp{\prog}{\exb}
        \eeq
        \sup_{\N \geq 1} \lwlp{\N}{\prog}{\exb}
        \eeq
        \inf_{\N \geq 1} \uwlp{\N}{\prog}{\exb}
        ~.
    \end{align*}
\end{lemma}
\begin{proof}
    Very similar to the proof of \Cref{thm:convLoopFree}.
    A minor difference is the straightforward base case $\prog = \DIVERGE$.
    Also recall that $\wlpSymb$ only applies to 1-bounded expectations only.
\end{proof}


\subsection{Proof of \Cref{thm:convUnfolding}}
\label{proof:convUnfolding}
%
\convUnfolding*
%
\begin{proof}
    The proof is by induction on the structure of $\prog$.
    We focus on $\wpSymb$, the proof for $\wlpSymb$ is analogous.
    For the bases cases $\prog = \SKIP$, $\prog = \DIVERGE$, $\prog = \ASSIGN{\pVar}{\aExp}$, $\prog = \UNIFASSIGN{\pVar}$ and $\prog = \OBSERVE{\guard}$ there is nothing to show as these atomic programs are unaffected by unfolding.
    We now consider the composite cases.
    Let $\ex \in \expsmeas$ be arbitrary but fixed throughout the following.
    
    \inductionCase{$\prog = \ITE{\guard}{\prog_1}{\prog_2}$ and $\prog = \PCHOICE{\prog_1}{\prob}{\prog_2}$}
    %
    These cases follow straightforwardly by induction and basic properties of $\wpSymb$.
    
    \inductionCase{$\prog = \SEQ{\prog_1}{\prog_2}$}
    %
    To show that for all $\depth \in \nats$ it holds that $\wp{\unfold{\prog}{\depth}}{\ex} \eleq \wp{\unfold{\prog}{\depth+1}}{\ex}$ we argue as follows:
    %
    \begin{align*}
        & \wp{(\SEQ{\prog_1}{\prog_2})^\depth}{\ex} \\
        %
        \eeq & \wp{\prog_1^\depth}{\wp{\prog_2^\depth}{\ex}} \tag{definition of $\wpSymb$ and of $(\SEQ{\prog_1}{\prog_2})^\depth$} \\
        %
        \eeleq & \wp{\prog_1^\depth}{\wp{\prog_2^{\depth + 1}}{\ex}} \tag{I.H. on $\prog_2$, monotonicity of $\wpSymb$}\\
        %
        \eeleq & \wp{\prog_1^{\depth + 1}}{\wp{\prog_2^{\depth + 1}}{\ex}} \tag{I.H. on $\prog_1$} \\
        %
        \eeq & \wp{(\SEQ{\prog_1}{\prog_2})^{\depth + 1}}{\ex} \tag{definition of $\wpSymb$ and of $(\SEQ{\prog_1}{\prog_2})^{\depth+1}$}
    \end{align*}
    %
    To see that the supremum over all unrollings is equal to the exact $\wpSymb$ consider the following:
    %
    \begin{align*}
        & \sup_{\depth \in \nats} \wp{(\SEQ{\prog_1}{\prog_2})^\depth}{\ex} \\
        %
        \eeq & \sup_{\depth \in \nats} \wp{\prog_1^\depth}{\wp{\prog_2^\depth}{\ex}} \tag{definition of $\wpSymb$ and of $(\SEQ{\prog_1}{\prog_2})^\depth$} \\
        %
        \eeq & \sup_{\depth_1 \in \nats} \sup_{\depth_2 \in \nats} \wp{\prog_1^{\depth_1}}{\wp{\prog_2^{\depth_2}}{\ex}} \tag{\Cref{thm:oneVsTwoSups}, monotonicity of $\wpTrans{\unfold{\prog_1}{\depth_1}}$, I.H.\ on $\prog_1$ and $\prog_2$} \\
        %
        \eeq & \sup_{\depth_1 \in \nats} \wp{\prog_1^{\depth_1}}{\sup_{\depth_2 \in \nats} \wp{\prog_2^{\depth_2}}{\ex}} \tag{$\omega$-continuity of $\wpTrans{\unfold{\prog_1}{\depth_1}}$, see \Cref{thm:wpWlpWellDefinedAndContinuous}} \\
        %
        \eeq & \sup_{\depth_1 \in \nats} \wp{\prog_1^{\depth_1}}{\wp{\prog_2}{\ex}} \tag{I.H.\ on $\prog_2$} \\
        %
        \eeq & \wp{\prog_1}{\wp{\prog_2}{\ex}} \tag{I.H.\ on $\prog_1$} \\
        %
        \eeq & \wp{\SEQ{\prog_1}{\prog_2}}{\ex} \tag{definition of $\wpSymb$} 
    \end{align*}
    
    \inductionCase{$\prog = \WHILE{\guard}{\progBody}$}
    %
    We first show by (an inner) induction on $\depth \in \nats$ that
    \[
    \wp{\unfold{(\WHILE{\guard}{\progBody})}{\depth}}{\ex}
    \eeleq
    \wp{\unfold{(\WHILE{\guard}{\progBody})}{\depth + 1}}{\ex}
    ~.
    \]
    %
    For $\depth = 0$ this follows as $\wp{\unfold{(\WHILE{\guard}{\progBody})}{0}}{\ex} = \wp{\DIVERGE}{\ex} = 0$ by definition.
    %
    For $\depth > 0$:
    %
    \begin{align*}
        & \wp{\unfold{(\WHILE{\guard}{\progBody})}{\depth}}{\ex} \\
        %
        \eeq & \wp{\ITE{\guard}{\SEQ{\unfold{\progBody}{\depth-1}}{\unfold{(\WHILE{\guard}{\progBody})}}{\depth-1}}{\SKIP}}{\ex} \tag{definition of unfolding} \\
        %
        \eeq & \iv{\guard} \cdot \wp{\unfold{\progBody}{\depth-1}}{\wp{\unfold{(\WHILE{\guard}{\progBody})}{\depth-1}}{\ex}} + \iv{\neg\guard} \cdot \wp{\SKIP}{\ex} \tag{definition of $\wpSymb$} \\
        %
        \eeleq & \iv{\guard} \cdot \wp{\unfold{\progBody}{\depth-1}}{\wp{\unfold{(\WHILE{\guard}{\progBody})}{\depth}}{\ex}} + \iv{\neg\guard} \cdot \wp{\SKIP}{\ex} \tag{inner I.H., monotonicity of $\wpSymb$} \\
        %
        \eeleq & \iv{\guard} \cdot \wp{\unfold{\progBody}{\depth}}{\wp{\unfold{(\WHILE{\guard}{\progBody})}{\depth}}{\ex}} + \iv{\neg\guard} \cdot \wp{\SKIP}{\ex} \tag{outer I.H. on $\progBody$} \\
        %
        \eeq & \wp{\ITE{\guard}{\SEQ{\unfold{\progBody}{\depth}}{\unfold{(\WHILE{\guard}{\progBody})}}{\depth}}{\SKIP}}{\ex} \tag{definition of $\wpSymb$} \\
        %
        \eeq & \wp{\unfold{(\WHILE{\guard}{\progBody})}{\depth+1}}{\ex} \tag{definition of unfolding}
    \end{align*}
    %
    To see that the supremum over all unrollings is equal to the exact $\wpSymb$ consider the following:
    %
    \begin{align*}
        & \sup_{\depth \in \nats} \wp{\unfold{(\WHILE{\guard}{\progBody})}{\depth}}{\ex} \\
        %
        \eeq & \sup_{\depth \in \nats} \wp{\unfold{(\WHILE{\guard}{\progBody})}{\depth+1}}{\ex} \tag{non-decreasing, see above} \\
        %
        \eeq & \sup_{\depth \in \nats} \wp{\ITE{\guard}{\SEQ{\unfold{\progBody}{\depth}}{\unfold{(\WHILE{\guard}{\progBody})}}{\depth}}{\SKIP}}{\ex} \tag{definition of unfolding} \\
        %
        \eeq & \sup_{\depth \in \nats} \left( \iv{\guard} \cdot \wp{\unfold{\progBody}{\depth}}{\wp{\unfold{(\WHILE{\guard}{\progBody})}{\depth}}{\ex}} + \iv{\neg\guard} \cdot \ex \right) \tag{definition of $\wpSymb$} \\
        %
        \eeq & \iv{\guard} \cdot \sup_{\depth \in \nats}  \wp{\unfold{\progBody}{\depth}}{\wp{\unfold{(\WHILE{\guard}{\progBody})}{\depth}}{\ex}} + \iv{\neg\guard} \cdot \ex \tag{$\omega$-continuity of $\cdot$ and $+$} \\
        %
        \eeq & \iv{\guard} \cdot \sup_{\depth_2 \in \nats} \sup_{\depth_1 \in \nats} \wp{\unfold{\progBody}{\depth_1}}{\wp{\unfold{(\WHILE{\guard}{\progBody})}{\depth_2}}{\ex}} + \iv{\neg\guard} \cdot \ex \tag{\Cref{thm:oneVsTwoSups}} \\
        %
        \eeq & \iv{\guard} \cdot \sup_{\depth_2 \in \nats}  \wp{\progBody}{\wp{\unfold{(\WHILE{\guard}{\progBody})}{\depth_2}}{\ex}} + \iv{\neg\guard} \cdot \ex \tag{I.H. on $\progBody$} \\
        %
        \eeq & \iv{\guard} \cdot  \wp{\progBody}{\sup_{\depth_2 \in \nats} \wp{\unfold{(\WHILE{\guard}{\progBody})}{\depth_2}}{\ex}} + \iv{\neg\guard} \cdot \ex \tag{$\omega$-continuity of $\wpSymb$}
    \end{align*}
    %
    By the definition of $\wpTrans{\WHILE{\guard}{\progBody}}$ in terms of a least fixed point, the above equation implies
    \[
    \wp{\WHILE{\guard}{\progBody}}{\ex}
    \eeleq
    \sup_{\depth \in \nats} \wp{\unfold{(\WHILE{\guard}{\progBody})}{\depth}}{\ex}
    ~. 
    \]
    
    To complete the proof we show by (an inner) induction on $\depth \in \nats$ that
    \[
    \wp{\unfold{(\WHILE{\guard}{\progBody})}{\depth}}{\ex}
    \eeleq
    \wp{\WHILE{\guard}{\progBody}}{\ex}
    ~.
    \]
    This implies that $\sup_{\depth \in \nats} \wp{\unfold{(\WHILE{\guard}{\progBody})}{\depth}}{\ex} \eleq \wp{\WHILE{\guard}{\progBody}}{\ex}$.
    For $\depth = 0$ this holds as $\wp{\unfold{(\WHILE{\guard}{\progBody})}{0}}{\ex} = \wp{\DIVERGE}{\ex} = 0$ by definition.
    For $\depth \geq 0$ the argument is as follows:
    %
    \begin{align*}
        & \wp{\unfold{(\WHILE{\guard}{\progBody})}{\depth+1}}{\ex}  \\
        %
        \eeq &  \wp{\ITE{\guard}{\SEQ{\unfold{\progBody}{\depth}}{\unfold{(\WHILE{\guard}{\progBody})}}{\depth}}{\SKIP}}{\ex} \tag{definition of unfolding} \\
        %
        \eeq & \iv{\guard} \cdot  \wp{\unfold{\progBody}{\depth}}{\wp{\unfold{(\WHILE{\guard}{\progBody})}{\depth}}{\ex}} + \iv{\neg\guard} \cdot \ex  \tag{definition of $\wpSymb$} \\
        %
        \eeleq & \iv{\guard} \cdot \wp{\unfold{\progBody}{\depth}}{\wp{\WHILE{\guard}{\progBody}}{\ex}} + \iv{\neg\guard} \cdot \ex  \tag{inner I.H., monotonicity of $\wpTrans{\unfold{\progBody}{\depth}}$} \\
        %
        \eeleq & \iv{\guard} \cdot \wp{\progBody}{\wp{\WHILE{\guard}{\progBody}}{\ex}} + \iv{\neg\guard} \cdot \ex  \tag{outer I.H. on $\progBody$} \\
        %
        \eeq & \wp{\WHILE{\guard}{\progBody}}{\ex} \tag{definition of $\wpSymb$}
    \end{align*}
\end{proof}


\subsection{Proof of \Cref{thm:pointwiseConv}}
\label{proof:pointwiseConv}
%
\pointwiseConv*
%
\begin{proof}
    We only show the equality involving $\wpSymb$ (the one for $\wlpSymb$ is analogous).    
    For all $n \in \nats$ we have
    \begin{align*}
        & \lwp{n}{\unfold{\prog}{n}}{\ex} \\
        \lleq & \wp{\unfold{\prog}{n}}{\ex} \tag{by \Cref{thm:soundness}} \\
        \lleq & \wp{\prog}{\ex} \tag{by \Cref{thm:convUnfolding}}
    \end{align*}
    and thus $\sup_{n \in \nats} \lwp{n}{\unfold{\prog}{n}}{\ex} \leq \wp{\prog}{\ex}$.
    But we also have
    \begin{align*}
        & \sup_{n \in \nats} \lwp{n}{\unfold{\prog}{n}}{\ex} \\
        \ggeq & \sup_{n \in \nats} \lwp{2^n}{\unfold{\prog}{2^n}}{\ex} \tag{elementary property of supremum} \\
        \eeq & \sup_{\depth \in \nats} \sup_{\N \in \nats} \lwp{2^\N}{\unfold{\prog}{2^\depth}}{\ex} \tag{by \Cref{thm:oneVsTwoSups} and \Cref{thm:powerTwoMonotonic}} \\
        \eeq & \sup_{\depth \in \nats} \wp{\unfold{\prog}{2^\depth}}{\ex} \tag{by \Cref{thm:convLoopFree}, using that $\unfold{\prog}{2^\depth}$ is loop-free and that $f \in \expsClass$} \\
        \eeq & \wp{\prog}{\ex} \tag{by \Cref{thm:convUnfolding}} ~.
    \end{align*}
\end{proof}
