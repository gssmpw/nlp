\section{Effective Verification of $\pWhile$-Programs}
\label{sec:syntax}

In the previous sections, we have assumed that guards and arithmetical expressions in programs as well as the resulting (pre/post)-expectations are purely mathematical objects.
To enable \emph{effective}, i.e., automated, verification we now define, inspired by \cite{DBLP:journals/pacmpl/BatzKKM21,DBLP:journals/pacmpl/SchroerBKKM23}, concrete syntaxes for these objects.

\subsection{A Formal Language of Expressions}
\label{sec:syntax:defs}

Let $\lvars$ be a countably infinite set of logical variables ranged over by $\lvar, \lvarb, \ldots$, etc.
The sets $\syntacticTerms$ and $\syntacticGuards$ of (syntactic) \emph{terms} and \emph{guards} are defined as follows:
%
\begin{align*}
    \synTerm
    ~\grammarSymb~
    \ratConst \in \nonNegRats
    \mmid 
    \lvar \in \lvars
    \mmid
    \synTerm + \synTerm
    \mmid
    \synTerm \synMonus \synTerm
    \mmid
    \synTerm \cdot \synTerm
    %
    \qquad
    \qquad
    \synGuard
    ~\grammarSymb~
    \synTerm < \synTerm
    \mmid
    \neg \synGuard
    \mmid
    \synGuard \land \synGuard
\end{align*}
%
Note that the other standard comparison relations $\leq,\neq, =, >, \geq$ and Boolean connectives $\lor, \limplies, \liff$ can be expressed in terms of $<, \neg, \land$.
We allow further syntactic sugar such as $\synTerm^2$ for $\synTerm \cdot \synTerm$ and $1 \leq \lvar \leq 2$  for $1 \leq \lvar \land \lvar \leq 2$, etc.
Additional parentheses are admitted to clarify the order of precedence; to minimize the use of parentheses we assume that $\cdot$ takes precedence over $+$ and $\synMonus$, and $\neg$ binds stronger than the binary Boolean connectives, as is standard.
%
We let $\varsIn{\synTerm}$ and $\varsIn{\synGuard}$ be the variables occurring in term $\synTerm$ and guard $\synGuard$.

For every $\lvarsSubset \supseteq \free{\synTerm}$, the semantics $\sem{\synTerm} \colon \nnReals^\lvarsSubset \to \nnReals$ of $\synTerm \in \synTerms$ is standard except that $\synMonus$ is interpreted as \enquote{monus}, i.e., $\sem{\synTerm_1 \synMonus \synTerm_2} = \max(\synTerm_1 - \synTerm_2, 0)$.
%
The semantics of $\synGuard \in \synGuards$ can be viewed similarly as a function $\sem{\synGuard} \colon \nnReals^\lvarsSubset \to \bools$ for every $\lvarsSubset \supseteq \free{\synGuard}$.
For $\st \in \nnReals^\lvarsSubset$, we write $\st \models \synGuard$ and $\st \not\models \synGuard$ to indicate that $\sem{\synGuard}(\st) = \boolConstTrue$ and $\sem{\synGuard}(\st) = \boolConstFalse$, respectively.

Note that every $\synTerm \in \synTerms$ without monus is a non-negative polynomial in the --- finitely many --- variables $\free{\synTerm}$ with rational coefficients, possibly written in (partially) factorized form, whereas terms with monus can be seen as a piecewise defined non-negative polynomial.
Similarly, every $\synGuard \in \synGuards$ is a Boolean combination of (in)equations between such (piecewise) polynomials.

We are now ready to define our expression language similar to~\cite{DBLP:journals/pacmpl/BatzKKM21}.
The restriction to infima and suprema over \emph{compact} intervals $\clIvalGen$ resembles the definition of lower and upper Riemann sums.
%
\begin{definition}[The Expression Language $\synExps$]
    The set of $\synExps$ of (syntactic) \emph{expressions} is defined according to the following grammar:
    \begin{align*}
        \synEx
        &\quad\grammarSymb\quad
        \synTerm
        \mmid
        \iv{\synGuard} \cdot \synEx
        \mmid
        \ratConst \cdot \synEx
        \mmid
        \synEx + \synEx
        \mmid
        \synSupBd{\lvar}{\ivalL}{\ivalR} \,\synEx
        \mmid
        \synInfBd{\lvar}{\ivalL}{\ivalR} \,\synEx
    \end{align*}
    Here, $\synTerm \in \syntacticTerms$, $\synGuard \in \synGuards$, $\ratConst \in \nnRats$, $\ivalL, \ivalR \in \nnRats$, $\ivalL \leq \ivalR$, and $\lvar \in \lvars$.
    \qedDef
\end{definition}
%
We allow $\iv{\synGuard}$ as syntactic sugar for $\iv{\synGuard} \cdot 1$ and adopt the usual rules regarding parentheses and orders of precedence.
For $\synEx \in \synExps$ we let $\free{\synEx}$ be the variables in $\synEx$ that are not bound by a $\supQuantifierSymbol$ (supremum) or $\infQuantifierSymbol$ (infimum) \enquote{quantifier}.
As in standard first-order logic, it is possible for a variable to have both a free and a non-free occurrence in $\synEx$.
We may write $\synEx(\lvar_1,\ldots,\lvar_n)$ to indicate that $\synEx$ contains at most the pairwise distinct free variables $\lvar_1,\ldots,\lvar_n \in \lvars$.

\begin{definition}[Semantics of Expressions]
    \label{def:exprSem}
    We define the semantics of expressions $\sem{\synEx} \colon \nnReals^\lvarsSubset \to \nonNegReals$ for every finite $\lvarsSubset \subseteq \lvars$ with $\free{\synEx} \subseteq \lvarsSubset$ inductively as follows.
    For all $\st \in \nnReals^\lvarsSubset$:
    \begin{itemize}
        \item $\sem{\synTerm}(\st)$ is defined in the standard manner, see above.
        \item $\sem{\iv{\synGuard} \cdot \synEx}(\st) = \sem{\synEx}(\st)$ if $\st \models \synGuard$; $\sem{\iv{\synGuard} \cdot \synEx}(\st) = 0$ if $\st \not\models \synGuard$.
        \item $\sem{\ratConst \cdot \synEx}(\st) = \ratConst \cdot \sem{\synEx}(\st)$.
        \item $\sem{\synEx_1 + \synEx_2}(\st) = \sem{\synEx_1}(\st) + \sem{\synEx_2}(\st)$
        \item $\sem{\synSupBd{\lvar}{\ivalL}{\ivalR} \,\synEx}(\st) = \sup_{\xi \in \clIvalGen} \sem{\synEx}(\pStUpdate{\st}{\lvar}{\xi})$.
        \item $\sem{\synInfBd{\lvar}{\ivalL}{\ivalR} \,\synEx}(\st) = \inf_{\xi \in \clIvalGen} \sem{\synEx}(\pStUpdate{\st}{\lvar}{\xi})$.
    \end{itemize}
    In the last two cases, if $\lvar$ is not already in the domain of $\st$, we tacitly assume that the operation $\pStUpdate{\st}{\lvar}{\xi}$ extends the domain of $\st$ by $\lvar$.
    \qedDef
\end{definition}

It is not immediately clear that $\sem{\synEx}$ is well-defined due to the suprema in \Cref{def:exprSem} --- we have to ensure that those exist in $\nonNegReals$.
We show this now.

\begin{restatable}{lemma}{expWellLocallyBounded}
    \label{thm:expWellLocallyBounded}
    For every $\synEx \in \synExps$ and every finite $\lvarsSubset \subseteq \lvars$ with $\free{\synEx} \subseteq \lvarsSubset$ the semantics $\sem{\synEx} \colon \nnReals^\lvarsSubset \to \nonNegReals$ is a well-defined locally bounded function (cf.~\Cref{sec:approxwp:convloopfree}).
\end{restatable}

\begin{example}
    \label{ex:syntax}
    $\synEx = \synSupBd{\lvar}{0}{5} \iv{\lvard = \lvar^2} \cdot \lvar$ denotes the function $\sem{\synEx} = \lam{\st}{\iv{0 \leq \st(\lvard) \leq 25} \cdot \sqrt{\st(\lvard)}}$.
\end{example}


\subsection{Properties of Syntactic Expressions}
\label{sec:syntax:properties}

The set $\firstOrderFormulas$ of first-order formulae\footnote{In this paper, \emph{first-order formula} always refers to first-order logic over the signature of polynomial arithmetic with comparison; we interpret such formulae over the fixed structure $\reals$.} is defined according to the following grammar:
%
\begin{align*}
    \foForm
    \quad\grammarSymb\quad
    \synGuard
    \mmid
    \foExists \lvar \colon \foForm
    \mmid 
    \foForall \lvar \colon \foForm
    \mmid
    \neg \foForm
    \mmid
    \foForm \land \foForm
    \mmid
    \foForm \lor \foForm
    \mmid
    \foForm \limplies \foForm
\end{align*}
%
where $\synGuard$ is a Boolean combination of polynomial (in)equalities (similar to the guards defined in \Cref{sec:syntax:defs} but with standard minus instead of \enquote{monus}), and $\lvar \in \lvars$.
As usual, $\free{\foForm}$ is the set of free variables in $\foForm$, and we write $\foForm(\lvar_1,\ldots,\lvar_n)$ to indicate that $\foForm$ contains at most the (pairwise distinct) free variables  $\lvar_1,\ldots,\lvar_n \in \lvars$.
An FO formula $\foForm$ is called \emph{quantifier-free} if it does not contain any $\foExists$ and $\foForall$ quantifiers.
Moreover, $\foForm$ is said to be in \emph{prenex normal form} (PNF) if $\foForm = \foQuantifierSymbol_1\lvar_1 \colon \ldots \foQuantifierSymbol_n\lvar_n \colon \foForm'$ where $n \geq 0$, $\foQuantifierSymbol_1, \ldots, \foQuantifierSymbol_n \in \{\foExists, \foForall\}$, and $\foForm'$ is quantifier-free.
Similarly, $\foForm$ is in \emph{existential (universal) prenex form} if for all $1 \leq i \leq n$ we have $\foQuantifierSymbol_i = \foExists$ ($\foQuantifierSymbol_i = \foForall$, respectively).

The semantics of a formula $\foForm \in \foForms$ is standard and can be viewed as a function $\sem{\foForm} \colon \reals^\lvarsSubset \to \bools$ for all $\lvarsSubset \subseteq \lvars$ with $\free{\foForm} \subseteq \lvarsSubset$.
For $\st \in \reals^{\lvarsSubset}$ we write $\st \models \foForm$ to indicate that $\st$ is a model of $\foForm$, i.e., the \emph{sentence} (formula without free variables) obtained by substituting every free occurrence of $\lvar \in \free{\foForm}$ in $\foForm$ by $\st(\lvar)$ evaluates to $\boolConstTrue$ in $\reals$.
A formula $\foForm$ is called  \emph{satisfiable} if it has a model, \emph{unsatisfiable} if it does not have a model, and \emph{valid} if $\neg\foForm$ is unsatisfiable.

The decision problem of checking whether a given \emph{quantifier-free} $\foForm \in \foForms$ is satisfiable is called $\QFNRA$\footnote{Quantifier-free non-linear real arithmetic, a term coined by the SMT-community~\cite{BarFT-SMTLIB}.}.
It is known that $\QFNRA \in \PSPACE$.
Checking the satisfiability of general $\foForms$ formulae is decidable as well, but has higher complexity.

\begin{restatable}[{$\synExps$ to $\foForms$}]{lemma}{expToFo}
    \label{thm:expToFo}
    For every $\synEx(\lvar_1,\ldots,\lvar_n) \in \synExps$ there exists an FO formula $\foForm_\synEx(\lvar_1,\ldots,\lvar_n,\lvarb)$ encoding $\sem{\synEx}$ in the sense that for all $\st \in \nnReals^\free{\synEx}$ and $\resVal \in \nnReals$, we have that $\st,\resVal \models \foForm_\synEx$ iff $\sem{\synEx}(\st) = \resVal$.
    For quantifier-free $\synEx$, we can construct $\foForm_\synEx$ in existential prenex form in linear time.
\end{restatable}

FO-expressible functions in the sense of \Cref{thm:expToFo} have been extensively studied~\cite{coste2000introduction}.
Let $\foForms_\reals$ be defined like $\foForms$ with the only difference that \emph{arbitrary real numbers} are allowed as coefficients in the polynomials.
A set $\saSet \subseteq \reals^n$, $n \geq 1$, is called \emph{semi-algebraic} if there exists an $\foForms_\reals$ formula $\foForm(\lvar_1,\ldots,\lvar_n)$ such that $\saSet = \{ \st \in \reals^n \mid \st \models \foForm\}$.
A function $\fun \colon \reals^n \to \reals$ is called semi-algebraic if its graph $\{ ( \st, \fun(\st) ) \mid \st \in \reals^n \}$ is a semi-algebraic set.

\begin{lemma}[\textnormal{e.g.,~\cite[Exercise 2.22]{coste2000introduction}}]
    \label{thm:semiAlgebraicFunctionAlmostContinuous}
    Every semi-algebraic function $\fun \colon \reals \to \reals$ has at most finitely many discontinuities.
    In particular, $\fun$ is almost everywhere continuous and hence Riemann-integrable on every interval $\clIvalGen \subset \reals$ where $\fun$ is bounded.
\end{lemma}

By combining local boundedness from \Cref{thm:expWellLocallyBounded}, the translation to FO from \Cref{thm:expToFo}, and the fact that semi-algebraic functions are Riemann-integrable (\Cref{thm:semiAlgebraicFunctionAlmostContinuous}), we obtain:

\begin{restatable}{theorem}{expBoundedAndRiemannIntegrable}
    \label{thm:expBoundedAndRiemannIntegrable}
    For all $\synEx \in \synExps$, intervals $\clIvalGen \subset \nnReals$, finite $\lvarsSubset \subseteq \lvars$ with $\free{\synEx} \subseteq \lvarsSubset $, and variables $\lvar \in \lvarsSubset$, the function $\sem{\synEx} \colon \nnReals^\lvarsSubset \to \nonNegReals$ is Riemann-integrable on $\clIvalGen$ w.r.t.\ $\lvar$.
\end{restatable}

In analogy to standard FO, we say that an expression $\synEx \in \synExps$ is in \emph{prenex normal form} (PNF) if
\begin{align*}
    \synEx \eeq \synVarQuantBd{\lvar_1}{\ivalL_1}{\ivalR_1}{\quantitativeQuantifierSymbol^{\!1}} \ldots \synVarQuantBd{\lvar_n}{\ivalL_n}{\ivalR_n}{\quantitativeQuantifierSymbol^{\!n}} \synEx'
\end{align*}
where $n \geq 0$, $\quantitativeQuantifierSymbol^{\!1}, \ldots,\quantitativeQuantifierSymbol^{\!n} \in \{\infQuantifierSymbol, \supQuantifierSymbol\}$, and $\synEx'$ is quantifier-free.

\begin{lemma}[Prenex Normal Form\textnormal{~\cite[Lemma 8.1]{DBLP:journals/pacmpl/BatzKKM21}}]
    \label{thm:pnf}
    Every $\synEx \in \synExps$ can be transformed into an equivalent $\synEx' \in \synExps$ in PNF with the same free variables.
    Moreover, if $\synEx$ is $\infQuantifierSymbol$-free ($\supQuantifierSymbol$-free), then $\synEx'$ is $\infQuantifierSymbol$-free ($\supQuantifierSymbol$-free, respectively) as well.
    This transformation can be done in linear time.
\end{lemma}

Next, we formalize one of our key insights:
In order to check a \enquote{\emph{quantitative entailment}} $\sem{\synEx} \eleq \sem{\synExb}$ such that suprema occur only in $\synEx$ and infima only in $\synExb$, it is \emph{not} actually necessary to evaluate them exactly (which amounts to solving optimization problems).
Instead, one can, loosely speaking, replace $\sup$ and $\inf$ by $\foForall$-quantifiers.
This idea is based on the following elementary observation:
For arbitrary sets $A, B \subseteq \reals$, we have $\sup A \leq \inf B$ iff $\forall a \in A , b \in B \colon~  a \leq b$.

\begin{restatable}[Checking Quantitative Entailments\textnormal{, cf.~\cite[Section 5.2]{DBLP:journals/pacmpl/SchroerBKKM23}}]{lemma}{checkEntailment}
    \label{thm:checkEntailment}
    Let
    \begin{align*}
        \synEx \eeq \synSupBd{\lvar_1}{\ivalL_1}{\ivalR_1} \ldots \synSupBd{\lvar_n}{\ivalL_n}{\ivalR_n} \synEx'
        \qqand
        \synExb \eeq \synInfBd{\lvarc_1}{\ivalLb_1}{\ivalRb_1} \ldots \synInfBd{\lvarc_m}{\ivalLb_m}{\ivalRb_m} \synExb'
    \end{align*}
    be $\infQuantifierSymbol$-free ($\supQuantifierSymbol$-free, respectively) expressions in PNF.
    Further, let $\foForm_{\synEx'}(\free{\synEx'},\lvarb_\synEx)$ and $\foForm_{\synExb'}(\free{\synExb'},\lvarb_\synExb)$ be the FO formulae in existential prenex form encoding $\synEx'$ and $\synExb'$ from \Cref{thm:expToFo}.
    Then:
    \begin{align}
        & \sem{\synEx} \eeleq \sem{\synExb} \notag \\
        %
        \text{iff}\quad & 
        \Bigg(
            \bigwedge_{\substack{\lvar \,\in\, \free{\synEx} \\ \cup \, \free{\synExb}}} \lvar {\geq} 0
            ~\land~
            \bigwedge_{i=1}^n \ivalL_i {\leq} \lvar_i {\leq} \ivalR_i
            ~\land~
            \bigwedge_{i=1}^m \ivalLb_i {\leq} \lvarc_i {\leq} \ivalRb_i
            ~\land~
            \foForm_{\synEx'} \land \foForm_{\synExb'}
        \Bigg)
        ~\limplies~
        \lvarb_\synEx \leq \lvarb_\synExb
        \quad\text{is valid}
        \label{eq:smt-query}
    \end{align}
    As a consequence, the decision problem \enquote{ $\sem{\synEx} \eleq \sem{\synExb}$?} for $\synEx$ and $\synExb$ is linear-time reducible to $\QFNRA$.
\end{restatable}

\begin{example}
    Reconsider the expression $\synEx = \synSupBd{\lvar}{0}{5} \iv{\lvard = \lvar^2} \cdot \lvar$ from \Cref{ex:syntax}.
    Suppose we wish to check whether $\sem{\synEx}$ is upper-bounded by the constant function $\sem{\synExb}$, $\synExb = 4$.
    Since $\synEx$ and $\synExb$ have the form required by \Cref{thm:checkEntailment} we can achieve this by considering the FO formula
    \begin{align*}
        \left(
            \lvard \geq 0 \land
            0 \leq \lvar \leq 5 \land (\lvard = \lvar^2 \limplies \lvarb = x) \land (\lvard \neq \lvar^2 \limplies \lvarb = 0) \land \lvarb' = 4
        \right)
        ~\limplies~
        \lvarb \leq \lvarb' 
    \end{align*}
    which is \emph{not} valid as witnessed by the assignment $\{\lvar \mapsto 5, \lvard \mapsto 25, \lvarb \mapsto 5, \lvarb' \mapsto 4\}$.
    Hence $\sem{\synEx} \not\eleq 4$.
\end{example}


\subsection{Decidability and Complexity of $\pWhile$ Verification}

We now assume (w.l.o.g.) that the set of logical variables $\lvars$ used in syntactic expressions contains our fixed set $\pVars$ of program variables.

\begin{definition}[Representable and Syntactic Expectations]
    We define the following terminology:
    \begin{itemize}
        \item An expectation $\ex \in \exps$ is called \emph{representable} if $f = \sem{\synEx}$ for some $\synEx \in \synExps$ with $\free{\synEx} \subseteq \pVars$.
        The set of all representable (1-bounded) expectations is denoted $\repExps$ ($\brepExps$, respectively).
        \item A \emph{syntactic expectation} is an expression $\synEx \in \synExps$ whose free variables are program variables, i.e., $\free{\synEx} \subseteq \pVars$.
        A \emph{1-bounded} syntactic expectation  $\bsyntacticExpectations$ is $\synEx \in \synExps$ such that $\sem{\synEx} \in \boundedExpectations$.
        The set of all 1-bounded syntactic expectations is denoted $\bsyntacticExpectations$.
        \qedDef
    \end{itemize}
\end{definition}
%
Membership in $\bsyntacticExpectations$ is decidable by encoding $\synEx(\lvar_1,\ldots,\lvar_n)$ as the FO formula $\foForm_\synEx(\lvar_1,\ldots,\lvar_n,\lvarb)$ from \Cref{thm:expToFo} and checking validity of $(\lvar_1 \geq 0 \land \ldots \land \lvar_n \geq 0 \land \foForm_\synEx(\lvar_1,\ldots,\lvar_n, \lvarb)) \limplies \lvarb \leq 1$.

Our next theorem implies that if we restrict our framework to representable expectations and only allow the $\synTerms$ and $\synGuards$ defined in \Cref{sec:syntax:defs} as the arithmetic and Boolean, respectively, expressions in our programs, then we obtain the convergence guarantees from \Cref{sec:convergence}.

\begin{restatable}[Riemann-suitability of $\synExps$]{theorem}{exprRiemannSuitable}
    \label{thm:exprRiemannSuitable}
    The combination $(\expsClassExp, \synTerms, \synGuards)$ with $\synTerms$ and $\synGuards$ as defined in \Cref{sec:syntax:defs} is Riemann-suitable (see~\Cref{def:riemannSuitable}).
\end{restatable}


\subsubsection{Verifying Loop-free Programs}
\label{sec:verify_loop_free}

Since $\synExps$ is effectively closed under weakest (liberal) Riemann pre-expectation (formalized in\iftoggle{arxiv}{ \Cref{proof:computeSyntacticExpression}}{~\cite{arxiv}}), it follows that we can effectively prove upper and lower bounds on weakest (liberal) pre-expectations of loop-free programs:

\begin{restatable}[Bounds on Loop-free $\wpwlpSymb$]{theorem}{boundsOnLoopFreeWpWlp}
    \label{thm:boundsOnLoopFreeWpWlp}
    For all loop-free $\prog \in \pWhileWith{\synTerms}{\synGuards}$ and integers $\N \geq 1$ the following decision problems can be effectively translated%
    \footnote{In polynomial time if we assume $\prog$ to be of constant size and $\N$ given in unary. In general, an expression encoding $\lwp{\N}{\prog}{\sem{\synEx}}$ for a given $\synEx$ can be exponential in the size of $\prog$ --- consider $n$ sequential $\KWIF$-$\KWELSE$ constructs. However, our reduction is of practical interest as $\QFNRA$ is supported by many SMT solvers.}
    to $\QFNRA$:
    Given ...
    \begin{enumerate}
        \setlength\itemsep{0.2em}
        %
        \item ... an $\supQuantifierSymbol$-free $\synEx \in \syntacticExpectations$ and an $\infQuantifierSymbol$-free $\synExb \in \syntacticExpectations$, is \qquad $\sem{\synExb} \eleq \lwp{\N}{\prog}{\sem{\synEx}}$ ~? \\
        \gray{If yes, then $\sem{\synExb} \eleq \wp{\prog}{\sem{\synEx}}$, provided $\sem{\synEx} \in \expsmeas$.}
        %
        \item ... an $\infQuantifierSymbol$-free $\synEx \in \syntacticExpectations$ and an $\supQuantifierSymbol$-free $\synExb \in \syntacticExpectations$, is \qquad $\uwp{\N}{\prog}{\sem{\synEx}} \eleq \sem{\synExb}$ ~? \\
        \gray{If yes, then $\wp{\prog}{\sem{\synEx}} \eleq \sem{\synExb}$, provided $\sem{\synEx} \in \expsmeas$.}
        %
        \item ... an $\supQuantifierSymbol$-free $\synEx \in \bsyntacticExpectations$ and an $\infQuantifierSymbol$-free $\synExb \in \bsyntacticExpectations$, is \quad $\sem{\synExb} \eleq \lwlp{\N}{\prog}{\sem{\synEx}}$ ~? \\
        \gray{If yes, then $\sem{\synExb} \eleq \wlp{\prog}{\sem{\synEx}}$), provided $\sem{\synEx} \in \bexpsmeas$.}
        %
        \item ... an $\infQuantifierSymbol$-free $\synEx \in \bsyntacticExpectations$ and an $\supQuantifierSymbol$-free $\synExb \in \bsyntacticExpectations$, is \quad $\uwlp{\N}{\prog}{\sem{\synEx}} \eleq \sem{\synExb}$ ~? \\
        \gray{If yes, then $\wlp{\prog}{\sem{\synEx}} \eleq \sem{\synExb}$, provided $\sem{\synEx} \in \bexpsmeas$.}
    \end{enumerate}
\end{restatable}
%
\noindent
Moreover, for fixed $\prog$ and $\synEx$, the size of the syntactic expectation representing $\somewp{\prog}{\sem{\synEx}}$ for $\transSymb \in \{\lwpSymb{N},\uwpSymb{N},\lwlpSymb{N},\uwlpSymb{N}\}$ is in $\mathcal{O}\left( N^k\right)$, where $k$ is the number of $\UNIF$-statements \mbox{in $\prog$.}

Using \Cref{thm:boundsOnLoopFreeWpWlp} and \Cref{thm:cwpSound}, it is straightforward to derive bounds on $\cwpSymb$ as well.


\subsubsection{Verifying Loops}

We now state how to apply the proof rules from \Cref{sec:invariants_and_unrolling} effectively:

\begin{restatable}[Verification of Invariants]{theorem}{verificationInvariants}
    \label{thm:verificationInvariants}
    Let $\prog = \WHILE{\guard}{\progBody} \in \pWhileWith{\synTerms}{\synGuards}$ such that $\progBody$ is loop-free.
    Further, let $\synExI, \synEx \in \synExps$ and $\synExJ,\synExb \in \bsyntacticExpectations$ all be quantifier-free, and let $\N \geq 1$ be an integer.
    Then we can ...
    \begin{enumerate}
        \item ... decide if $\sem{\synExI}$ is a $\uwpSymb{\N}$-superinvariant of $\prog$ w.r.t.\ $\sem{\synEx}$ by a reduction to $\QFNRA$.\\
        \gray{If the superinvariant property holds, then $\wp{\prog}{\sem\synEx} \eleq \sem{I}$, provided $\sem{\synEx} \in \expsmeas$}
        \item ... decide if $\sem{\synExJ}$ is a $\lwlpSymb{\N}$-subinvariant of $\prog$ w.r.t.\ $\sem{\synExb}$ by a reduction to $\QFNRA$.\\
        \gray{If the subinvariant property holds, then $\sem{J} \eleq \wlp{\prog}{\sem\synExb}$, provided $\sem{\synExb} \in \bexpsmeas$.}
    \end{enumerate}
\end{restatable}

\begin{restatable}[Complexity of Verification Problems]{theorem}{complexity}
    \label{thm:complexity}
    The following decision problems are $\coRE$-complete\footnote{I.e., $\Pi_1^0$-complete in the arithmetical hierarchy.}:
    Given an arbitrary (not necessarily loop-free) $\prog \in \pWhileWith{\synTerms}{\synGuards}$ and ...
    \begin{enumerate}
        \item\label{it:problemUpperBoundOnWp} ... $\synEx, \synExb \in \syntacticExpectations$ such that $\sem{\synEx} \in \expsmeas$, does it hold that $\wp{\prog}{\sem{\synEx}} \eleq \sem{\synExb}$\,?
        \item\label{it:problemLowerBoundOnWlp} ... $\synEx, \synExb \in \bsyntacticExpectations$ such that $\sem{\synEx} \in \bexpsmeas$, does it hold that $\sem{\synExb} \eleq  \wlp{\prog}{\sem{\synEx}}$\,?
    \end{enumerate}
\end{restatable}
%
\noindent The proof of membership in $\coRE$ heavily relies on the convergence results established in \Cref{thm:pointwiseConv}.
The $\coRE$-hardness proofs are standard~\cite{DBLP:conf/mfcs/KaminskiK15} and follow from $\coRE$-hardness of the non-halting problem.
See\iftoggle{arxiv}{ \Cref{proof:complexity}}{~\cite{arxiv}} for details.
