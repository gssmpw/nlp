\subsection{Proof of \Cref{thm:expWellLocallyBounded}}
\label{proof:expWellLocallyBounded}
%
\expWellLocallyBounded*
%
\begin{proof}
    We show, by induction on the structure of $\synEx$, that $\sem{\synEx}$ is locally bounded when viewed as a function of type $\sem{\synEx} \colon \reals^{\lvarsSubset} \to \reals$.
    
    \inductionCase{$\synEx = \synTerm$}
    $\sem{\synTerm}$ is a piece-wise defined polynomial in finitely many variables, hence $\sem{\synTerm}$ is locally bounded.
    
    \inductionCase{$\synEx = \iv{\synGuard}\cdot \synEx$}
    It is $\sem{\iv{\synGuard}\cdot \synEx} \leq \sem{\synEx}$ and $\sem{\synEx}$ is locally bounded by the I.H.
    Hence $\sem{\iv{\synGuard}\cdot \synEx}$ is locally bounded, too.
    
    \inductionCase{$\synEx = \ratConst \cdot \synExb$}
    By the I.H., $\synExb$ is locally bounded, i.e., for every $\compactSet \subseteq \reals^\lvarsSubset = \reals^\lvarsSubset$ we have $\sup_{\st \in \compactSet} |\sem{\synExb}| < \infty$.
    But then we also have $\sup_{\st \in \compactSet} | \sem{\ratConst \cdot \synExb}| = \sup_{\st \in \compactSet} |\ratConst \cdot \sem{\synExb}|  = \sup_{\st \in \compactSet} |\ratConst|  \cdot | \sem{\synExb}| =  \sup_{\st \in \compactSet} \ratConst \cdot  | \sem{\synExb}| = \ratConst \cdot   \sup_{\st \in \compactSet} | \sem{\synExb}| < \infty$.
    
    \inductionCase{$\synEx = \synExb + \synExc$}
    Similar to the previous case, using I.H.\ for both $\synExb$ and $\synExc$.
    
    \inductionCase{$\synEx = \synSupBd{\lvar}{\ivalL}{\ivalR} \synExb$}
    Let $\compactSet \subseteq \reals^\lvarsSubset$ be compact.
    Let $\compactSet'$ be the (compact) cross product of $\compactSet$ and $\clIvalGen$, i.e., $\compactSet' = \{\pStUpdate{\st}{\lvar}{\xi} \in \reals^{\lvarsSubset \cup \{\lvar\}} \mid \st \in \compactSet, \xi \in \clIvalGen\}$.
    We have to show that $\sup_{\st \in \compactSet} |\sem{f}(\st)| < \infty$.
    We argue as follows:
    \begin{align*}
        & \sup_{\st \in \compactSet} \left|\sem{f}(\st) \right| \\ 
        \eeq & \sup_{\st \in \compactSet} \left| \sem{\synSupBd{\lvar}{\ivalL}{\ivalR} \synExb}(\st)  \right| \\
        \eeq & \sup_{\st \in \compactSet} \left| \sup_{\xi \in \clIvalGen}\sem{\synExb}(\pStUpdate{\st}{\lvar}{\xi})  \right| \tag{definition of $\sem{\synSupBd{\lvar}{\ivalL}{\ivalR} \synExb}$} \\
        \lleq & \sup_{\st \in \compactSet}  \sup_{\xi \in \clIvalGen} \left|\sem{\synExb}(\pStUpdate{\st}{\lvar}{\xi})  \right| \tag{absolute value of $\sup$ $\leq$ $\sup$ of absolute value} \\
        \lleq & \sup_{\st \in \compactSet'}  \left|\sem{\synExb}(\st)  \right| \tag{because $\compactSet' = \{\pStUpdate{\st}{\lvar}{\xi} \mid \st \in \compactSet, \xi \in \clIvalGen \}$} \\
        \llt & \infty \tag{$\sem{\synExb}$ locally bounded by I.H.} \\
    \end{align*}
    
    \inductionCase{$\synEx = \synInfBd{\lvar}{\ivalL}{\ivalR} \synExb$}
    This case is analogous to the $\supQuantifierSymbol$-case noticing that, for every set $A \subseteq \reals$, we have $|\inf A| = | - \sup - A| = |\sup - A| \leq \sup |-A| = \sup |A|$, where $- A = \{-a \mid a \in A\}$.
    
\end{proof}


\subsection{Proof of \Cref{thm:expToFo}}
\label{proof:expToFo}
%
\expToFo*
%
\begin{proof}
    By induction on the structure of $\synEx$.
    The crucial observation is that infima and suprema can be expressed in FO.
    In all of the following we assume that $\lvarb \notin \free{\synEx} \subseteq \{\lvar_1,\ldots,\lvar_n\}$. 

    \inductionCase{$\synEx(\lvar_1,\ldots,\lvar_n) = \synTerm(\lvar_1,\ldots,\lvar_n)$}
    We first define $\foForm_{\synTerm}(\lvar_1,\ldots,\lvar_n,\lvarb)$ by induction on the structure of the syntactic term $\synTerm$.
    
    \begin{itemize}
        \item $\synTerm = \ratConst$.
        We define
        \[
            \foForm_{\synTerm}(\lvarb)
            \quad\text{as}\quad
            \ratConst = \lvarb
        ~.
        \]
        %
        \item $\synTerm = \lvar$.
        We define
        \[
            \foForm_{\synTerm}(\lvar,\lvarb)
            \quad\text{as}\quad
            \lvar = \lvarb
            ~.
        \]
        %
        \item $\synTerm(\lvar_1,\ldots,\lvar_n) = \synTermb(\lvar_1,\ldots,\lvar_n) + \synTermc(\lvar_1,\ldots,\lvar_n)$.
        We define
        \[
            \foForm_\synTerm(\lvar_1,\ldots,\lvar_n,\lvarb)
            \quad\text{as}\quad
            \foExists \lvarb_1 \colon \foExists \lvarb_2 \colon \lvarb = \lvarb_1 + \lvarb_2 \land \foForm_\synTermb(\lvar_1,\ldots,\lvar_n,\lvarb_1) \land \foForm_\synTermc(\lvar_1,\ldots,\lvar_n,\lvarb_2)
            ~.
        \]
        %
        \item $\synTerm(\lvar_1,\ldots,\lvar_n) = \synTermb(\lvar_1,\ldots,\lvar_n) \cdot \synTermc(\lvar_1,\ldots,\lvar_n)$. Similar to the previous case.
        %
        \item $\synTerm(\lvar_1,\ldots,\lvar_n) = \synTermb(\lvar_1,\ldots,\lvar_n) \synMonus \synTermc(\lvar_1,\ldots,\lvar_n)$.
        This case is the most interesting one due to monus.
        We define $\foForm_\synTerm(\lvar_1,\ldots,\lvar_n,\lvarb)$ as 
        \begin{align*}
            \foExists \lvarb_1 \colon \foExists \lvarb_2 \colon &\foForm_\synTermb(\lvar_1,\ldots,\lvar_n,\lvarb_1) \land \foForm_\synTermc(\lvar_1,\ldots,\lvar_n,\lvarb_2) \\
            &\land
            \left( \lvarb_1 \geq \lvarb_2 \limplies \lvarb = \lvarb_1 - \lvarb_2 \right )
            \land
            \left( \lvarb_1 < \lvarb_2 \limplies \lvarb = 0 \right )
            ~.
        \end{align*}
    \end{itemize}
    Now that we have defined $\foForm_{\synTerm}$, we define
    \[
        \foForm_{\synEx}(\lvar_1,\ldots,\lvar_n,\lvarb)
        \quad\text{as}\quad
        \synTerm(\lvar_1,\ldots,\lvar_n) = \lvarb
        ~.
    \]
    
    \inductionCase{$\synEx(\lvar_1,\ldots,\lvar_n) = \iv{\synGuard(\lvar_1,\ldots,\lvar_n)}\cdot \synExb(\lvar_1,\ldots,\lvar_n)$}
    \newcommand{\lvarbool}{b}
    
    We first define an FO formula $\foForm_{\synGuard}(\lvar_1,\ldots,\lvar_n,\lvarbool)$ by induction on the structure of the syntactic guard $\synGuard$.
    The idea is to encode the (Boolean) result that $\synGuard$ evaluates to as a $\{0,1\}$-valued FO variable $\lvarbool$.
    
    \begin{itemize}
        \item $\synGuard(\lvar_1,\ldots,\lvar_n) = \synTerm(\lvar_1,\ldots,\lvar_n) < \synTermb(\lvar_1,\ldots,\lvar_n)$.
        We consider $\foForm_\synTerm(\lvar_1,\ldots,\lvar_n,\lvarb)$ and $\foForm_\synTermb(\lvar_1,\ldots,\lvar_n,\lvarb)$ as defined above and define $\foForm_{\synGuard}(\lvar_1,\ldots,\lvar_n,\lvarbool)$ as
        \begin{align*}
            \foExists \lvarb_1 \colon \foExists \lvarb_2 \colon &\foForm_\synTermb(\lvar_1,\ldots,\lvar_n,\lvarb_1) \land \foForm_\synTermc(\lvar_1,\ldots,\lvar_n,\lvarb_2) \\
            &\land
            \left( \lvarb_1 < \lvarb_2 \limplies \lvarbool = 1 \right )
            \land
            \left( \lvarb_1 \geq \lvarb_2 \limplies \lvarbool = 0 \right )
            ~.
        \end{align*}
        %
        \item $\synGuard(\lvar_1,\ldots,\lvar_n) = \synGuard_1(\lvar_1,\ldots,\lvar_n) \land \synGuard_2(\lvar_1,\ldots,\lvar_n)$.
        We consider $\foForm_{\synGuard_1}(\lvar_1,\ldots,\lvar_n,\lvarbool)$ and $\foForm_{\synGuard_2}(\lvar_1,\ldots,\lvar_n,\lvarbool)$ and define $\foForm_{\synGuard}(\lvar_1,\ldots,\lvar_n,\lvarbool)$ as
        \begin{align*}
            \foExists \lvarbool_1 \colon \foExists \lvarbool_2 \colon &\foForm_{\synGuard_1}(\lvar_1,\ldots,\lvar_n,\lvarbool_1) \land \foForm_{\synGuard_2}(\lvar_1,\ldots,\lvar_n,\lvarbool_2) \\
            &\land
            \left( \lvarbool_1 + \lvarbool_1 = 2 \limplies \lvarbool = 1 \right )
            \land
            \left( \lvarbool_1 + \lvarbool_1 \neq 2 \limplies \lvarbool = 0 \right )
            ~.
        \end{align*}
        %
        \item $\synGuard(\lvar_1,\ldots,\lvar_n) = \neg \synGuard'(\lvar_1,\ldots,\lvar_n)$.
        We consider $\foForm_{\synGuard'}(\lvar_1,\ldots,\lvar_n,\lvarbool)$ and define $\foForm_{\synGuard}(\lvar_1,\ldots,\lvar_n,\lvarbool)$ as
        \begin{align*}
            \foExists \lvarbool'\colon \foForm_{\synGuard'}(\lvar_1,\ldots,\lvar_n,\lvarbool')  
            \land
            \lvarbool = 1 - \lvarbool'
            ~.
        \end{align*}
    \end{itemize}
    
    Now that we have defined $\foForm_{\synGuard}$, we define
    \[
        \foForm_\synEx(\lvar_1,\ldots,\lvar_n,\lvarb)
        \quad\text{as}\quad
        \foExists \lvarbool \colon \foForm_\synGuard(\lvar_1,\ldots,\lvar_n,\lvarbool) \land (\lvarbool = 1 \limplies \foForm_\synEx(\lvar_1,\ldots,\lvar_n,\lvarb) )
        \land (\lvarbool = 0 \limplies \lvarb = 0 )
    \]
    
    \inductionCase{$\synEx(\lvar_1,\ldots,\lvar_n) = \ratConst \cdot \synExb(\lvar_1,\ldots,\lvar_n)$}
    We define
    \[
    \foForm_\synEx(\lvar_1,\ldots,\lvar_n,\lvarb)
    \quad\text{as}\quad
    \foExists \lvarb' \colon \lvarb = \ratConst \cdot \lvarb' \land \foForm_\synExb(\lvar_1,\ldots,\lvar_n,\lvarb')
    ~.
    \]
    
    \inductionCase{$\synEx(\lvar_1,\ldots,\lvar_n) = \synExb(\lvar_1,\ldots,\lvar_n) + \synExc(\lvar_1,\ldots,\lvar_n)$}
    We define
    \[
    \foForm_\synEx(\lvar_1,\ldots,\lvar_n,\lvarb)
    \quad\text{as}\quad
    \foExists \lvarb_1 \colon \foExists \lvarb_2 \colon \lvarb = \lvarb_1 + \lvarb_2 \land \foForm_\synExb(\lvar_1,\ldots,\lvar_n,\lvarb_1) \land \foForm_\synExc(\lvar_1,\ldots,\lvar_n,\lvarb_2)
    ~.
    \]
    
    \inductionCase{$\synEx = \synSupBd{\lvar}{\ivalL}{\ivalR} \synExb$}
    To give an intuition about how we handle this case, we recall the standard definition of the supremum:
    Let $A \subseteq \reals$ and $y \in \reals$.
    Then
    \begin{align*}
        & y = \sup A \\
        ~{}\iff{}~ & \text{for all } a\in A \colon a \leq y \tag{$y$ is an upper bound}\\
        & \text{and for all } y' \in \reals \colon( \text{for all } a \in A \colon a \leq y') \implies y \leq y' \tag{$y$ is not greater than any upper bound}
        ~.
    \end{align*}
    We now construct FO formulae corresponding to the above two conjuncts.
    Consider the formula $\foForm_\synExb(\lvar,\lvar_1,\ldots,\lvar_n,\lvarb)$ obtained by applying the I.H.\ to $\synExb(\lvar,\lvar_1,\ldots,\lvar_n)$.
    We define
    \[
    \foForm_{\synExb}^{ub}(\lvar_1,\ldots,\lvar_n,u)
    \quad\text{as}\quad
    \foForall \ivalL \leq \xi \leq \ivalR \colon \foForall \lvarb \colon \foForm_\synExb(\xi,\lvar_1,\ldots,\lvar_n,\lvarb) 	\limplies \lvarb \leq u
    ~.
    \]
    We have for all $\st \in \reals^{\free{\synExb}}$ and $\resVal \in \reals$ that $\st ,\resVal \models \foForm_{\synExb}^{ub}$ iff for all $\xi \in \clIvalGen \colon \sem{\synExb}(\pStUpdate{\st}{\lvar}{\xi}) \leq \resVal$.
    Now we define
    \[
    \foForm_\synEx(\lvar_1,\ldots,\lvar_n,\lvarb)
    \quad\text{as}\quad
    \foForm_{\synExb}^{ub}(\lvar_1,\ldots,\lvar_n,\lvarb) \land \foForall \lvarb ' \colon \foForm_{\synExb}^{ub}(\lvar_1,\ldots,\lvar_n,\lvarb') \limplies \lvarb \leq \lvarb'
    \]
    Then $\foForm_\synEx $ has the desired property.
    
    \inductionCase{$\synEx = \synInfBd{\lvar}{\ivalL}{\ivalR} \synExb$}
    Analogous to the previous case.
    \medskip
    
    Finally, observe that for \emph{quantifier-free} $\synEx$ (i.e., without $\supQuantifierSymbol$ and $\infQuantifierSymbol$), the above inductive construction of $\foForm_\synEx$ can be done in linear time in the size of $\synEx$.
    The so-obtained $\foForm_\synEx$ is not yet necessarily in existential prenex form, but it can be translated to the latter (again in linear time) by succesively applying the following standard prenexing rules which are true for all FO formulae $\foForm$ and $\foFormb$:
    \begin{align*}
        \foForm \limplies (\foExists \lvar \colon \foFormb)
        &\text{ is equivalent to }
        \foExists \lvar \colon (\foForm \limplies \foFormb)
        \\
        \foForm \land (\foExists \lvar \colon \foFormb)
        &\text{ is equivalent to }
        \foExists \lvar \colon (\foForm \land \foFormb)
    \end{align*}
    This completes the proof.
\end{proof}

\subsection{Proof of \Cref{thm:expBoundedAndRiemannIntegrable}}
\label{proof:expBoundedAndRiemannIntegrable}
%
\expBoundedAndRiemannIntegrable*
%
\begin{proof}
    Let $\pSt \in \nnReals^\lvarsSubset$ be arbitrary and
    consider the function $\ex_\pSt = \lam{\xi}{\iv{\xi \geq 0} \cdot \sem{\synEx}(\pStUpdate{\st}{\lvar}{\xi})}$ of type $\reals \to \reals$.
    By \Cref{thm:expWellLocallyBounded}, $\ex_\pSt$ is locally bounded.
    By \Cref{thm:expToFo}, $\ex_\pSt$ is semi-algebraic.
    It thus follows with \Cref{thm:semiAlgebraicFunctionAlmostContinuous} that $\ex_\pSt$ is Riemann-integrable on $\clIvalGen$.
    Hence, since $\st$ was arbitrary, we may conclude that $\sem{\synEx}$ is Riemann-integrable on $\clIvalGen$ w.r.t.\ $\lvar$.
\end{proof}


\subsection{Proof of \Cref{thm:checkEntailment}}
\label{proof:checkEntailment}
%
\checkEntailment*
%
\begin{proof}
    The fact that $\sem{\synEx} \eleq \sem{\synExb}$ iff the FO formula in \eqref{eq:smt-query} is valid follows directly from \Cref{thm:expToFo}.
    However, note that the formula in \eqref{eq:smt-query} is not necessarily quantifier-free as $\foForm_{\synEx'}$ and $\foForm_{\synExb'}$ may contain an $\foExists$-prefix (see \Cref{thm:expToFo}).
    By applying the standard prenexing rules
    \begin{align*}
        (\foExists \lvar \colon \foForm) \land \foFormb
        &\quad\text{is equivalent to}\quad
        (\foExists \lvar \colon \foForm \land \foFormb)
        \\
        \text{and}\qquad (\foExists \lvar \colon \foForm) \limplies \foFormb
        &\quad\text{is equivalent to}\quad
        \foForall \lvar \colon (\foForm \limplies \foFormb)
    \end{align*}
    which hold for all FO formulae $\foForm$ and $\foFormb$ with $\lvar \notin \free{\foFormb}$, we can transform the formula in \eqref{eq:smt-query} to universal prenex form.
    Checking validity of the latter is equivalent to checking validity of the \emph{quantifier-free} formula obtained by dropping the $\foForall$-prefix.
\end{proof}


\subsection{Substitution Lemma}

\begin{lemma}[{Substitution Lemma~\cite{DBLP:journals/pacmpl/BatzKKM21,DBLP:journals/pacmpl/SchroerBKKM23}}]
    \label{thm:syntacticSemanticSubs}
    Let $\synEx \in \synExps$ such that $\lvar \in \free{\synEx}$ and let $\synTerm \in \synTerms$ be a syntactic term.
    Consider the expression $\synSubs{\synEx}{\lvar}{\synTerm} \in \synExps$ obtained by (syntactically) substituting every free occurrence of $\lvar$ in $\synEx$ by $\synTerm$ in a capture-avoiding manner (i.e., by possibly renaming variables appropriately).
    Then
    \[
    \sem{\synSubs{\synEx}{\lvar}{\synTerm}}
    \eeq
    \exSubs{\sem{\synEx}}{\lvar}{\sem{\synTerm}}
    \gray{
        \eeq
        \lam{\pSt}{\sem{\synEx}(\pStUpdate{\pSt}{\lvar}{\sem{\synTerm}(\pSt)})}
    }
    ~.
    \]
\end{lemma}


\subsection{Proof of \Cref{thm:exprRiemannSuitable}}
\label{proof:exprRiemannSuitable}
%
\exprRiemannSuitable*
%
\begin{proof}
    Recall \Cref{def:riemannSuitable}.
    By \Cref{thm:expBoundedAndRiemannIntegrable}, all $\sem{\synEx}$ are bounded and Riemann integrable on $\uIval$ w.r.t.\ every $\pVar \in \pVars$.
    The arithmetic expressions representable by $\synTerms$, i.e., the functions $\sem{\synTerm}$ for $\synTerm \in \synTerms$, are locally bounded since they are piecewise defined polynomials.
    The closure properties from \Cref{def:riemannSuitable} all follow immediately from the definition of $\synExps$, except the substitution $\exSubs{\sem{\synEx}}{\pVar}{\aExp}$.
    Here, the crucial observation is that \emph{semantic substitution corresponds to syntactic substitution}, see \Cref{thm:syntacticSemanticSubs}.
\end{proof}


\subsection{Computing Riemann $\wpwlpSymb$ Syntactically}
\label{proof:computeSyntacticExpression}
%
\begin{restatable}[Computing Riemann $\wpwlpSymb$ Syntactically]{lemma}{computeSyntacticExpression}
	\label{thm:computeSyntacticExpression}
	For all $\prog \in \pWhileWith{\synTerms}{\synGuards}$, $\prog$ loop-free, and integers $\N \geq 1$ we can effectively do the following:
	For every given ...
	\begin{enumerate}
		\setlength\itemsep{0.2em}
		%
		\item ... $\synEx \in \syntacticExpectations$, compute $\synExb \in \syntacticExpectations$ such that \qquad \,\,\,\, $\lwp{\N}{\prog}{\sem{\synEx}} = \sem{\synExb}$ ~.
		%
		\item ... $\synEx \in \syntacticExpectations$, compute $\synExb \in \syntacticExpectations$ such that \qquad \,\,\,\, $\uwp{\N}{\prog}{\sem{\synEx}} = \sem{\synExb}$ ~.
		%
		\item ... $\synEx \in \bsyntacticExpectations$, compute $\synExb \in \bsyntacticExpectations$ such that \quad $\lwlp{\N}{\prog}{\sem{\synEx}} = \sem{\synExb}$ ~.
		%
		\item ...  $\synEx \in \bsyntacticExpectations$, compute $\synExb \in \bsyntacticExpectations$ such that \quad $\uwlp{\N}{\prog}{\sem{\synEx}} = \sem{\synExb}$ ~.
	\end{enumerate}
	In (1) and (3), if $\synEx$ is $\supQuantifierSymbol$-free, then so is $\synExb$.
	In (2) and (4), if $\synEx$ is $\infQuantifierSymbol$-free, then so is $\synExb$.
\end{restatable}
%
\begin{proof}
    By induction on the structure of $\prog$.
    Most cases are immediate as we have essentially designed $\synExps$ such that $\repExps$ ($\boundedRepresentableExpectations$) is closed under $\lwpTrans{\N}{\prog}$ and $\uwpTrans{\N}{\prog}$ ($\lwlpTrans{\N}{\prog}$ and $\uwlpTrans{\N}{\prog}$, resp.) for loop-free $\prog$.
    For the assignment $\ASSIGN{\pVar}{\aExp}$ we rely on syntactic substitution (\Cref{thm:syntacticSemanticSubs}).
\end{proof}


\subsection{Proof of \Cref{thm:boundsOnLoopFreeWpWlp}}
\label{proof:boundsOnLoopFreeWpWlp}
%
\boundsOnLoopFreeWpWlp*
%
\begin{proof}
    By computing a syntactic representation of the corresponding Riemann $\wpwlpSymb$ as in \Cref{thm:computeSyntacticExpression}, translating to PNF via \Cref{thm:pnf}, and applying the quantitative entailment check from \Cref{thm:checkEntailment}.
    The \gray{gray} implications follow from soundness of the Riemann $\wpwlpSymb$ (\Cref{thm:soundness}).
\end{proof}


\subsection{Proof of \Cref{thm:verificationInvariants}}
\label{proof:verificationInvariants}
%
\verificationInvariants*
%
\begin{proof}
    Since $\progBody$ is loop-free we can apply \Cref{thm:computeSyntacticExpression} to compute $i \in \synExps$ (resp. $j \in \bsyntacticExpectations$) such that $\sem{i} = \uwp{\N}{\progBody}{\sem{I}}$ (resp. $\sem{j} = \lwlp{\N}{\progBody}{\sem{J}}$) with $i$ is $\infQuantifierSymbol$-free (resp $\supQuantifierSymbol$-free). Then, applying \Cref{thm:invariant_approx}, we need to check that 
    $ [\guard] \cdot \sem{\synEx} + [\neg \guard] \cdot \sem{i} \leq \sem{I}$ (resp. $\sem{J} \leq [\guard] \cdot \sem{\synEx} + [\neg \guard] \cdot \sem{j} $). By \Cref{thm:boundsOnLoopFreeWpWlp} this inequality can be reduced to $\QFNRA$.
\end{proof}



\subsection{Proof of \Cref{thm:complexity}}
\label{proof:complexity}
%
\complexity*
%
\begin{proof}
    We show membership in $\coRE$ first (which is the more interesting direction).
    Consider the complement of the first problem:
    \begin{align}
        \text{``Does there exist } \pSt \in \pStates \text{ s.t.\ } \sem{\synExb}(\pSt) < \wp{\prog}{\sem{\synEx}}(\pSt) \text{ ?''}
        \tag{complement of problem \eqref{it:problemUpperBoundOnWp}}
    \end{align}
    If the answer to the above question is \emph{yes}, then, since 
    $\sup_{n \geq 1}\lwp{n}{\unfold{\prog}{n}}{\sem{\synEx}} = \wp{\prog}{\sem{\synEx}}$
    by \Cref{thm:pointwiseConv}, there must exist $\pSt \in \pStates$ and $n \geq 1$ such that $\sem{\synExb}(\pSt) < \lwp{n}{\unfold{\prog}{n}}{\sem{\synEx}}(\pSt)$.
    Hence our semi-decision procedure will check the latter inequality for increasing values of $n$.
    To do so for a fixed $n$, the procedure constructs $\synExc$ such that $\sem{\synExc} = \lwp{n}{\unfold{\prog}{n}}{\sem{\synEx}}$ as in \Cref{thm:computeSyntacticExpression} and then invokes \Cref{thm:expToFo} to obtain FO formulae $\foForm_\synExb(\free{\synExb}, \lvarb_\synExb)$ and $\foForm_\synExc(\free{\synExc}, \lvarb_\synExc)$ encoding $\synExb$ and $\synExc$.
    It then only remains to check if the FO formula $(\foForm_\synExb \land \foForm_\synExc) \limplies \lvarb_\synExb < \lvarb_\synExc$ is satisfiable, which is decidable.
    
    To see that the complement of problem \eqref{it:problemUpperBoundOnWp} is $\RE$-hard we reduce the halting problem to it.
    We proceed as in \cite{DBLP:conf/mfcs/KaminskiK15}.
    The standard halting problem is equivalent to checking if $\wp{\prog_{std}}{1}(\pSt) = 1$ where $\prog_{std} \in \pWhileWith{\synTerms}{\synGuards}$ is a given standard program, i.e., it does not contain any probabilistic constructs, and $\pSt \in \rats^\pVars$ is a given initial state.
    Now consider \[
    \prog
    \eeq
    \PCHOICE{\SEQ{\ASSIGN{\pVar_1 }{\pSt(\pVar_1)} \,\SEMICOLON\, \ldots \,\SEMICOLON\,\ASSIGN{\pVar_n }{\pSt(\pVar_n)} }{\prog_{std}}}{\tfrac 1 2}{\SKIP} 
    \]
    where $\{\pVar_1,\ldots,\pVar_n\} = \pVars$.
    Then $\tfrac 1 2 < \wp{\prog}{1}$ iff $\wp{\prog_{std}}{1}(\pSt) > 0$ iff $\prog_{std}$ terminates on input $\pSt$.
    
    The argument for $\coRE$ membership of problem \eqref{it:problemLowerBoundOnWlp} is completely analogous.
    For $\coRE$-hardness we argue dually (we define $\prog$ as above):
    $\wlp{\prog}{0} < \tfrac 1 2$ iff $\wlp{\prog_{std}}{0}(\pSt) < \tfrac 1 2$ iff $\prog_{std}$ terminates on input $\pSt$.
    The latter argument works because for standard programs we have $\wlp{\prog_{std}}{0}(\pSt) = 0$ iff $\prog_{std}$ terminates on $\pSt$, and $\wlp{\prog_{std}}{0}(\pSt) = 1$ iff $\prog_{std}$ does not terminate on $\pSt$.
\end{proof}