\section{Approximate Riemann Weakest Pre-Expectations}
\label{sec:approxwp}

We start by recalling lower and upper Riemann sums and integrals.%
\footnote{
    The definition of an integral in terms of these sums is in fact commonly attributed to Darboux, not to Riemann.
    However, Darboux's integral is equivalent to Riemann's which, rather than considering lower and upper sums, relies on evaluating $\fun$ at sample points within the intervals of a partition of the integration domain.
    We refer to~\cite[Chapter 3, Theorems 3.3.1 and 3.3.2]{burk2007garden} for an in-depth comparison.
    In this paper, we consistently use Darboux's definitions, but refer to them nonetheless as Riemann sums and integrals, since the latter terminology is more widespread.
}
%
Let $\clIvalGen \subseteq \reals$ and $\partitionSize \geq 1$.
A \emph{partition} of $\clIvalGen$ is a tuple of at least $\partitionSize+1$ real numbers $\partition = (x_0,x_1,\ldots,x_{\partitionSize})$ such that
\[
    \ivalL
    =
    x_0
    \llt
    x_1
    \llt
    \ldots
    \llt
    x_{\partitionSize}
    =
    \ivalR
    ~.
\]
The set of all partitions of the interval $\clIvalGen$ is denoted $\partitions{\clIvalGen}$.
%
For a bounded $\fun \colon \clIvalGen \to \reals$ and a partition $\partition = (x_0,\ldots,x_{\partitionSize}) \in \partitions{\clIvalGen}$, we define the \emph{lower} and \emph{upper sums} of $\fun$ w.r.t.\ $\partition$ as
\[
    \lowerSum{\fun}{\partition}
    \eeq
    \sum_{i=1}^{\partitionSize} (x_{i} - x_{i-1}) \inf_{\xi \in \clIval{x_{i-1}}{x_i}} \fun(\xi)
    %
    \qqand
    %
    \upperSum{\fun}{\partition}
    \eeq
    \sum_{i=1}^{\partitionSize} (x_{i} - x_{i-1}) \sup_{\xi \in \clIval{x_{i-1}}{x_i}} \fun(\xi)
    ~.
\]
Note that $\upperSum{\fun}{\partition}$ and $\lowerSum{\fun}{\partition}$ are well-defined real numbers because $\fun$ is bounded.

\begin{definition}[Riemann integral]
    \label{def:riemannIntegral}
    Let $\fun \colon \clIvalGen \to \reals$ be bounded.
    The \emph{lower-} and \emph{upper Riemann integrals} of $\fun$ are defined as follows:
    \[
        \lowerIntGen \fun(x) \,dx
        \eeq
        \sup \left\{ \lowerSum{\fun}{\partition} \mid \partition \in \partitions{\clIvalGen} \right\}
        %
        \qqand
        %
        \upperIntGen \fun(x) \,dx
        \eeq
        \inf \left\{ \upperSum{\fun}{\partition} \mid \partition \in \partitions{\clIvalGen} \right\}
        ~.
    \]
    If the upper integral equals the lower integral, then the common value is written $\int_{\ivalL}^{\ivalR} \fun(x) \,dx$ and called \emph{the} Riemann integral of $\fun$.
    In this case, $\fun$ is called \emph{Riemann-integrable}.
    \qedDef
\end{definition}

Finally, we  introduce the following terminology:
%
(i) For $\funb \colon \realDomain \to \reals$, $\realDomain \subseteq \reals$, we say that $\funb$ is Riemann-integrable \emph{on an interval} $\clIval{\ivalLb}{\ivalRb} \subset \realDomain$ if the restriction $\funb \colon \clIval{\ivalLb}{\ivalRb} \to \reals$ is Riemann-integrable.
%
(ii) In this paper we often consider functions of the form $\funb \colon \realDomain^\pVars \to \reals$ where $\pVars$ is a finite set, e.g.\ the set of program variables.
We then say that $\funb$ is Riemann-integrable on $\clIval{\ivalLb}{\ivalRb} \subseteq \realDomain$ w.r.t.\ some $\pVar \in \pVars$ if \emph{for all} $\pSt \in \realDomain^\pVars$ the function $\lam{\xi}{\funb(\pStUpdate{\pSt}{\pVar}{\xi})}$ of type $\clIval{\ivalLb}{\ivalRb} \to \reals$ is Riemann-integrable.

\iftoggle{arxiv}{
\begin{example}
    To illustrate the concepts defined above consider the following:
    \begin{itemize}
        \item A constant function $\fun \colon \clIvalGen \to \reals, x \mapsto \realConst$ for $\realConst \in \reals$ is Riemann-integrable because for \emph{all} partitions $\partition$ we have $\lowerSum{\fun}{\partition} = \upperSum{\fun}{\partition} = \realConst (\ivalL-\ivalR)$.
        \item $\fun(x,y) = \iv{x^2 + y^2 \leq 1}$ is Riemann-integrable on $\unitInterval$ w.r.t.\ $x$.
        Indeed, for all $y \in \reals$, we have $\int_{0}^{1} \fun(x,y) \, dx = \iv{-1 \leq y \leq 1} \sqrt{1-y^2}$.
        \item The Dirichlet function $\fun \colon \clIvalGen \to \reals, x \mapsto \iv{x \in \rats}$ is not Riemann-integrable because for all partitions $\partition$ we have $\lowerSum{\fun}{\partition} = 0$ and $\upperSum{\fun}{\partition} = 1$.
        Note that $\fun$ is nowhere continuous.
    \end{itemize}
\end{example}
}{}


\subsection{{$\lwpSymb{\N}$ and $\uwpSymb{\N}$: Lower and Upper Riemann Weakest Pre-Expectations}}
\label{sec:approxwp:def}

We are now ready to define our approximate expectation transformers.
They arise from the standard transformers defined in \Cref{tab:original} by replacing the \emph{Lebesgue} integrals in the $\UNIF_{\uIval}$ case by a lower or upper Riemann sum. 
Formally:

\begin{definition}[Lower and Upper Riemann $\wpwlpSymb$-Transformers]
	\label{def:riemannwpwlp}
	For all $\prog \in \pWhile$ and integers $\N \geq 1$, the expectation transformers $\lwpTrans{\N}{\prog} \colon \exps \to \exps$ and $\uwpTrans{\N}{\prog} \colon \exps \to \exps$ are defined by induction over the structure of $\prog$ as in \Cref{tab:original}, with the only exception that if $\prog$ is $\UNIFASSIGN{\pVar}$, then for all $\ex \in \exps$ we set%
	\footnote{Notice that $\inf_{\xi \in \clIvalGen}\exSubs{\ex}{\pVar}{\xi}$ is the same as $\lam{\st}{\inf_{\xi \in \clIvalGen} \ex(\pStUpdate{\st}{\pVar}{\xi})}$.}
	\begin{align*}
		&\lwp{\N}{\UNIFASSIGN{\pVar}}{\ex}
		\eeq
		\frac{1}{\N} \sum_{i=0}^{\N-1} \inf_{\xi \in [\frac{i}{\N}, \frac{i+1}{\N}]} \exSubs{\ex}{\pVar}{\xi} \qquad\text{and, similarly,} \\
		&\uwp{\N}{\UNIFASSIGN{\pVar}}{\ex}
		\eeq
		\frac{1}{\N} \sum_{i=0}^{\N-1} \sup_{\xi \in [\frac{i}{\N}, \frac{i+1}{\N}]} \exSubs{\ex}{\pVar}{\xi}
		~.
	\end{align*}
	The lower and upper Riemann weakest \emph{liberal} pre-expectation transformers $\lwlpTrans{\N}{\prog} \colon \bexps \to \bexps$ and $\uwlpTrans{\N}{\prog} \colon \bexps \to \bexps$ are defined analogously.
	\qedDef
\end{definition}

Our Riemann transformers approximate the Lebesgue integral in the definition of $\wpSymb$ (and $\wlpSymb$) by a lower or upper Riemann sum.
We work with the partitions $0 < \tfrac{1}{\N} < \tfrac{2}{\N} < \ldots < 1$ of the unit interval for the sake of concreteness.
Note that these partitions are \emph{not} successive refinements of each other (see\iftoggle{arxiv}{ \Cref{thm:partitionRefine}}{~\cite{arxiv}} for definitions).
Thus, increasing $\N$ by $1$ does \emph{not necessarily} yield \enquote{better} approximations, i.e., \emph{we might have} $\lwp{\N}{\prog}{\ex} \not\eleq \lwp{\N+1}{\prog}{\ex}$%
\iftoggle{arxiv}{%
, e.g:
%
\[
    \lwp{2}{\UNIFASSIGN{\pVar}}{\iv{x \geq \tfrac 1 2}} ~\not\eleq~ \lwp{3}{\UNIFASSIGN{\pVar}}{\iv{x \geq \tfrac 1 2}}
    ~.
\]
}{.}

Given a loop $\prog=\WHILE{\guard}{\progBody}$, $\N\geq 1$, $\ex\in\exps$, and $\transSymb\in\{\uwpSymb{\N},\lwpSymb{\N}\}$, we define the \emph{$\transSymb$-characteristic function of $\prog$ w.r.t.\ $\ex$} as 
%
\[
\charfuntrans{\prog}{\ex} \colon \exps \to \exps, 
\qquad 
\charfuntrans{\prog}{\ex}(\exb) \eeq \iv{\guard} \cdot \somewp{\progBody}{\exb} + \iv{\neg\guard}\cdot \ex~.
\]
%
For $\exb\in\bexps$ and $\transSymb\in\{\uwlpSymb{\N},\lwlpSymb{\N}\}$, $\charfuntrans{\prog}{\exb} \colon \bexps \to \bexps$ 
\iftoggle{arxiv}{%
we analogously define
%
\[
	\charfuntrans{\prog}{\exb} \colon \bexps \to \bexps, 
	\qquad 
	\charfuntrans{\prog}{\exb}(\exb) \eeq \iv{\guard} \cdot \somewp{\progBody}{\exb} + \iv{\neg\guard}\cdot \exb~.
\]
%
}{%
is defined analogously.
}
Note that unlike $\wpSymb$ and $\wlpSymb$, the approximate transformers are defined on the full sets of expectations $\exps$ and $\bexps$, respectively, not just on the measurable ones.
All transformers from \Cref{def:riemannwpwlp} are well-defined: this follows from the Knaster-Tarski \Cref{thm:knasterTarski} using that $\exps$ and $\bexps$ are complete lattices and monotonicity of the transformers, see \Cref{thm:monriemann} below.


\subsection{Healthiness Properties of $\lwpSymb{\N}$ and $\uwpSymb{\N}$}
\label{sec:approxwp:basic}

\begin{restatable}[Monotonicity of the Riemann $\wpwlpSymb$-Transformers]{lemma}{monriemann}
	\label{thm:monriemann}
	For all $\prog \in \pWhile$ and integers $\N \geq 1$, the functions $\lwpTrans{\N}{\prog}$, $\uwpTrans{\N}{\prog}$, $\lwlpTrans{\N}{\prog}$, and $\uwlpTrans{\N}{\prog}$ are monotonic w.r.t.\ $\eleq$.
\end{restatable}

Notably, the Riemann $\wpwlpSymb$-transformers do not possess the same continuity properties as their Lebesgue counterparts from \Cref{sec:programs:wp}.
For instance, due to the presence of infima in the lower Riemann sum, $\lwpTrans{\N}{\prog}$ is \emph{not} $\omega$-continuous in general (see\iftoggle{arxiv}{ \Cref{proof:lwpNotOmegaCont}}{~\cite{arxiv}} for a counter-example).

\begin{restatable}[Soundness of the Riemann $\wpwlpSymb$-Transformers]{lemma}{soundness}
	\label{thm:soundness}
	For all programs $\prog \in \pWhile$, post-expectations $\ex \in \expsmeas$, $\exb \in \bexpsmeas$, and integers $\N \geq 1$,
	\begin{align*}
		\lwp{\N}{\prog}{\ex}
		&\eeleq
		\wp{\prog}{\ex}
		\eeleq
		\uwp{\N}{\prog}{\ex} \qand 
		\\
		\lwlp{\N}{\prog}{\exb}
		&\eeleq
		\wlp{\prog}{\exb}
		\eeleq
		\uwlp{\N}{\prog}{\exb}
		~.
	\end{align*}
\end{restatable}

\noindent For \emph{conditional} weakest pre-expectations (\Cref{sec:wp:cwp}), we immediately get the following:
%
\begin{corollary}
    \label{thm:cwpSound}
	For all programs $\prog \in \pWhile$, post-expectations $\ex \in \expsmeas$, and integers $\N \geq 1$
	\begin{align*}
		\frac{\lwp{\N}{\prog}{\ex}(\pSt)}{\uwlp{\N}{\prog}{\onefun}(\pSt)}
		\lleq
		\cwp{\prog}{\ex}(\pSt)
		\lleq
		\frac{\uwp{\N}{\prog}{\ex}(\pSt)}{\lwlp{\N}{\prog}{\onefun}(\pSt)}
	\end{align*}
	for all $\pState \in \pStates$ such that none of $\lwlp{\N}{\prog}{1}(\pSt)$, $\wlp{\prog}{1}(\pSt)$, and $\uwlp{\N}{\prog}{1}(\pSt)$ is zero.
\end{corollary}
%
\noindent We stress that \Cref{thm:soundness} and \Cref{thm:cwpSound} hold even for non-Riemann-integrable post-expectations.
