\documentclass[conference]{IEEEtran}
\IEEEoverridecommandlockouts
% The preceding line is only needed to identify funding in the first footnote. If that is unneeded, please comment it out.
\usepackage{cite}
\usepackage{amsmath,amssymb,amsfonts}
\usepackage{algorithmic}
\usepackage{booktabs}
\usepackage{graphicx}
\usepackage{textcomp}
\usepackage{xcolor}
\usepackage{array}
\usepackage{float}

\usepackage{array} % Required for 'm' specifier
\usepackage{lipsum}
\usepackage{tabularray}

\usepackage{cuted}
\usepackage{caption}


\def\BibTeX{{\rm B\kern-.05em{\sc i\kern-.025em b}\kern-.08em
    T\kern-.1667em\lower.7ex\hbox{E}\kern-.125emX}}
    
% \newcommand{\CG}{\mathcal{G}\xspace}
\newcommand{\CV}{\mathcal{V}\xspace}
\newcommand{\CE}{\mathcal{E}\xspace}
\newcommand{\CA}{\mathcal{A}\xspace}
\newcommand{\CF}{\mathcal{F}\xspace}
\newcommand{\CR}{\mathcal{R}\xspace}
\newcommand{\CB}{\mathcal{B}\xspace}
\newcommand{\CX}{\mathcal{X}\xspace}
\newcommand{\CK}{\mathcal{K}\xspace}
\newcommand{\CM}{\mathcal{M}\xspace}
\newcommand{\CC}{\mathcal{C}\xspace}
\newcommand{\CL}{\mathcal{L}\xspace}
\newcommand{\CI}{\mathcal{I}\xspace}
\newcommand{\CQ}{\mathcal{Q}\xspace}
\newcommand{\CO}{\mathcal{O}\xspace}
\newcommand{\CP}{\mathcal{P}\xspace}
\newcommand{\CS}{\mathcal{S}\xspace}
\newcommand{\CT}{\mathcal{T}\xspace}
\newcommand{\CJ}{\mathcal{J}\xspace}
\usepackage[para]{footmisc}
\usepackage{subfig}
% \usepackage{subcaption}
% \usepackage{array}
% \usepackage{colortbl}


\newcommand{\ENACT}{\textit{ENACT-Heart}~}

\begin{document}

\title{ENACT-Heart -- ENsemble-based Assessment Using CNN and Transformer on Heart Sounds
% \thanks{Identify applicable funding agency here. If none, delete this.}
}

\author{\IEEEauthorblockN{Jiho Han\textsuperscript{1}\thanks{\textsuperscript{1}Jiho Han was with the Department of Computer Science and Engineering, The University of Michigan - Dearborn, Dearborn, MI, USA during the study.}}
\IEEEauthorblockA{\textit{Industrial AI Lab}\\
\textit{SimPlatform Co. Ltd. Affiliate Research Institute}\\
Geumcheon-gu, Seoul, Republic of Korea\\
jihohan@simplatform.com}
\and
\IEEEauthorblockN{Adnan Shaout}
\IEEEauthorblockA{\textit{Department of Electrical and Computer Engineering} \\
\textit{The University of Michigan – Dearborn}\\
Dearborn, MI, USA\\
shaout@umich.edu}}

\maketitle

\begin{strip}
    \centering
    \vspace{-6em}
    \includegraphics[width=\linewidth]{images/diagram_ENACT_Heart.png}
    \captionof{figure}{The pipeline of our \textit{ENACT-Heart} consists of three core steps: data augmentation, expert analysis of each modality, and analysis fusion. 1) Data augmentation through Gaussian Noise allows increased variability and generalization of the overall model. 2) Spectrogram analysis is performed through ViT, and audiovisual diagram analysis is done through CNN, allowing each model to leverage its strengths in feature extraction for different modalities.}
    \label{fig:banner}
\end{strip}

%%%%%%%% ABSTRACT
\begin{abstract}

% Recent works to jointly reconstruct 3D human and object from a single RGB image, are mostly model-based, that fail to capture the fine details of the clothed human body and object surface. In this paper, we introduce ReCHOR, a novel, model-free, first-method to produce realistic clothed human-object reconstructions from a monocular view. This is extremely challenging due to human-object occlusions, diverse interactions and depth ambiguity, as it needs to infer both 3D spatial awareness and high resolution details. Our core idea is based on estimating neural implicit representations for human and object respectively by an attention-based neural implicit model that attends to pixel-aligned features from both the global human-object image for spatial awareness and  the local separate view of human and object images for high quality details. Additionally, the network is conditioned on semantic features from an initial estimated human-object pose prior and a generative diffusion model that inpaints occluded regions, thus enabling the retrieval of details from them.
% We also propose a synthetic dataset with rendered scenes of diverse, inter-occluded 3D human and object scans, to train our network. We evaluate our method on the synthetic and real world BEHAVE dataset. Our experiments show that our method outperforms the SOTA in achieving realistic clothed human-object reconstructions.
Recent approaches to jointly reconstruct 3D humans and objects from a single RGB image represent 3D shapes with template-based or coarse models, which fail to capture details of loose clothing on human bodies. In this paper, we introduce a novel implicit approach for jointly reconstructing realistic 3D clothed humans and objects from a monocular view. For the first time, we model both the human and the object with an implicit representation, allowing to capture more realistic details such as clothing. This task is extremely challenging due to human-object occlusions and the lack of 3D information in 2D images, often leading to poor detail reconstruction and depth ambiguity. To address these problems, we propose a novel attention-based neural implicit model that leverages image pixel alignment from both the input human-object image for a global understanding of the human-object scene and from local separate views of the human and object images to improve realism with, for example, clothing details. Additionally, the network is conditioned on semantic features derived from an estimated human-object pose prior, which provides 3D spatial information about the shared space of humans and objects. To handle human occlusion caused by objects, we use a generative diffusion model that inpaints the occluded regions, recovering otherwise lost details. For training and evaluation, we introduce a synthetic dataset featuring rendered scenes of inter-occluded 3D human scans and diverse objects. Extensive evaluation on both synthetic and real-world datasets demonstrates the superior quality of the proposed human-object reconstructions over competitive methods.
\end{abstract}

%%%%%%%% BODY TEXT
\section{Introduction}\label{sec:intro}

In computational finance, Monte Carlo simulations are used extensively to estimate the expected value of financial payoffs based on the solution of stochastic differential equations (SDEs) which model the evolution of stock prices, interest rates, exchange rates and other quantities \cite{glasserman04}.  Monte Carlo methods are very general and flexible, but for high accuracy it requires generating a large number of costly SDE path approximations, which has motivated research into a number of variance reduction or, equivalently, cost reduction techniques. One such method is
Multilevel Monte Carlo (MLMC), which was proposed in \cite{GILES2008} and was adapted for various applications that are summarised in \cite{Giles_overview17} and successfully combined with other methods such as quasi-Monte Carlo methods. The main idea of MLMC is to approximate the payoff using different time stepping resolutions when numerically solving the underlying SDE and to generate an optimal number of samples on each level, such that the overall computational cost is minimised subject to the desired bound on the variance. %, such that the total computational cost is minimised. 
The computational savings come from the fact that most samples are computed on the coarser levels and hence are less expensive while only a few samples from the finest levels are required \cite{GILES2008}.


Among the directions in which the computational cost 
of MLMC methods could further be reduced, an important avenue is the use of lower precision calculations, especially for the first Monte Carlo levels where the targeted accuracy is relatively low. 
 An overview of the research on mixed precision for the standard Monte Carlo (MC) framework is provided in \cite{ChowMixedPrecisionStandardMC} but only a few references study the potential of low precision computation in the MLMC framework \cite{Rounding_error_oliver}. To the best of our knowledge, the only MLMC framework with customised precision in the literature is \cite{brugger2014mixed}, but they use a uniform precision for all operations on each Monte Carlo level instead of optimising 
 the precision of each intermediary variable to reduce as much as possible the cost of path generation.
 
An important motivation for an MLMC framework with variable precision would be performing the low precision computations on reconfigurable hardware devices such as Field Programmable Gate Arrays (FPGAs). FPGAs contain customizable logic blocks and connectors that make it easy to adapt the digital circuit architecture for a specific application, leading to a highly parallel and optimised implementation. Therefore they are successfully exploited in applications that require high speed and have high computational workload, such as signal processing \cite{woods2008fpga}, and real time applications like high frequency trading \cite{HFT1,HFT2}. That is why a number of previous works in hardware architecture design implemented the MLMC algorithm to price financial options using FPGAs as accelerators, which resulted in improved speed and power efficiency compared to full CPU architectures \cite{Schryver2013AMM}. The paper \cite{lindsey2016domain} also proposed 
a Domain Specific Language to automate the configuration of FPGAs for this specific application. However, only \cite{brugger2014mixed} proposed a heuristic to reduce the precision in calculations.

In addition, all aforementioned works considered that the random number generation (RNG) is performed in single or double precision. Yet in most cases an important portion of the workload in the overall MLMC simulation comes from the RNG and in \cite{brugger2014mixed} this limited the total computational savings.
To reduce the cost of MLMC simulations in particular those based on the Geometric Brownian Motion (GBM), \cite{approximateICDF_Oliver, NestedOliver} have proposed to use approximate random numbers that are generated by applying an approximation of the inverse CDF to uniform random numbers. In \cite{NestedOliver}, the authors proposed a way to integrate these lower precision random variables into a \textit{nested} MLMC framework and completed a numerical analysis to bound the resulting error at each MC level by a product of the time step and the error in the random number approximation. The same authors show in \cite{approximateICDF_Oliver} that using approximate random variables reduces the cost of path generation by a factor 7.


In this paper we propose a nested MLMC framework that combines the use of approximate random normal variables and lower precision calculations to reduce the computational cost of MLMC even further than \cite{brugger2014mixed,NestedOliver}. We illustrate the efficiency of our framework in Matlab, after making several assumptions on the cost of operations and size of the errors that we carefully justify. We focus on the case of GBM and use the approximate RNG methods presented in \cite{approximateICDF_Oliver} as well as a new slightly modified method that combines CDF inversion and the central limit theorem. To choose the precision of the variables in the low precision path generation, we introduce a novel method to optimise the bit-widths. This optimisation is performed before the main path generation loop is executed and is based on a linear model of the payoff error  
due to rounding when computing in low precision. The error model relies on algorithmic differentiation in a similar manner to \cite{unifying-bwoptim,bitwidth-AD,ADAPT}. The bit-width optimisation procedure can be performed off-line, so this stage can be excluded from the on-line time complexity of our framework. The user specified desired accuracy is then enforced by calculating on-line the number of samples that need to be generated.

In terms of hardware design, we suggest implementing the low precision path generation on FPGAs and the full-precision ones on a CPU or GPU. 
The FPGA offers enough flexibility to define a separate bit-width for every variable in the low precision path generation, and can be reconfigured periodically to update the bit-widths when the market parameters have changed considerably. 


The paper is organized as follows : \Cref{sec:MLMC} introduces MLMC and nested MLMC to make clear the estimator that is implemented in our framework. Then in \Cref{sec:RNG} we detail the methods that could be used to obtain approximate random normally distributed numbers very cheaply for the low precision path generation. In \Cref{sec:error_model} and \Cref{sec:costModel} we propose an error model and a cost model (resp.) that we then use to formulate the optimisation problem that is solved to obtain the optimal bit-widths of fixed point variables in \Cref{sec:optimisation}. Finally we summarise our results and future directions in \Cref{sec:conclusion}.



\section{Background}
\label{sec:background}


\subsection{Preliminaries}

{\color{red}[TODO: LLMs? in-context learning?]}

\subsection{Problem Definition}

{\color{red}[TODO: define the problem of citation intent]}


\section{Related Works}
\label{sec:related_works}
\paragraph{Next Day Wildfire Segmentation} To apply DNNs to fire spread prediction, recent research has framed the spread problem as a semantic segmentation one, and developed multiple datasets to support this framing. \cite{huot2022next} created Next-Day Wildfire Spread, a dataset for mono-temporal fire spread prediction. They developed a custom convolutional autoencoder that takes as input various explanatory variables and outputs a binary mask indicating fire presence at each pixel. Similarly, \cite{gerard2023wildfirespreadts} extended the dataset to include multi-temporal prediction, added more explanatory variables, and higher resolution fire masks. They achieved the best performance using a standard Unet model and a Unet model with temporal attention (UTAE) \cite{garnot2021panoptic}. Other datasets aim to predict beyond next-day and forecast fire behavior several days in advance, e.g., \cite{prapas2022deep} developed SeasFire Cube and trained Unet++ models \cite{zhou2018unet++} for medium-term fire prediction, between 8 and 64 days. \cite{kondylatos2022wildfire} improved upon the collected data cube and found that the LSTM and ConvLSTM models outperformed the Fire Weather Index (FWI). In FireSight, \cite{gottfriedsenfiresight} collected a dataset using remote sensing data from 20 datasets, and trained a 3D UNet model to model short-term fire hazard, between 3 and 8 days.  

\paragraph{Other Approaches}  Aside from segmentation, other formulations have been developed for modeling fire spread using deep learning (DL). \cite{ross2021being} used a CNN-based Reinforcement Learning model that predicts the best action as burn or no burn given current conditions. \cite{ghosh2024fire} developed a probabilistic cellular automata model to simulate wildfire spread. In \cite{lisim2real}, the authors developed Sim2Real-Fire, a synthetic, high-resolution dataset for fire spread forecast and backtracking and outperformed the considered baselines using a custom Transformer model. We refer the readers to \cite{xu2024wildfire} for a more comprehensive review of DL for wildfire prediction. 

\paragraph{Next Day Wildfire Prediction with Time-Series} In contrast to most existing work (e.g., \cite{li2024wildfire, fitzgerald2023paying, shah2023wildfire, xiao2024wildfire}), we focus on utilizing a time-series of features for next-day wildfire spread prediction, which has been cited as an important emerging area of research \cite{fitzgerald2023paying, gerard2023wildfirespreadts, li2024wildfire}.  Historically, time-series modeling has been challenging due to the lack of appropriate public datasets to train and evaluate models for this task. Recently, \cite{gerard2023wildfirespreadts}, building upon the work of \cite{huot2022next}, developed the first multi-temporal dataset for time-series prediction.  Notably, they found that models using a time-series of input tend to outperform those using a single day, reinforcing the importance of this research direction. %We extend the existing time-series prediction research by investigating attention-based models: specifically the recent SwinUnet model \cite{liu2021swin}, which has been found extremely effective in a variety of contexts for segmentation tasks. 


\section{Proposed Approach} \label{proposed approach}
The choice of heart sound analysis in this study is driven by its unique diagnostic value, which complements other modalities such as ECG. Despite the advent of modern diagnostic techniques and sophisticated imaging modalities, cardiac auscultation and heart sounds remain invaluable diagnostic tools. While ECG is widely regarded as the gold standard for diagnosing cardiac rhythm disorders and ischemic heart disease, it may not capture certain aspects of cardiac function that heart sound analysis can, such as detecting murmurs, rubs, and other abnormal heart sounds indicative of structural abnormalities like valvular heart diseases or ventricular hypertrophy. Therefore, heart sound analysis provides additional, complementary information that can enhance diagnostic accuracy.

Researches has demonstrated that it is
possible to process spectrograms from audio data as images
and apply computer vision algorithms such as CNN \cite{verma_neural_2018, cabrera-ponce_detection_2020, hyder_acoustic_2017}. The core problem of the current approaches in using regular CNN-based computer vision methods on audio spectrogram representation lies in the distinctiveness of the spectrogram in comparison to other image data.

Visual transformers leverage attention mechanisms to capture dependencies between different parts of the input data. This allows them to model long-range dependencies more effectively than traditional CNNs, whose feature extraction is limited to local receptive fields. By aggregating information from across the entire spectrogram, transformers can show a global contextual understanding of the audio signal, enabling them to capture non-local dependencies and extract meaningful features from spectrograms.

\subsection{Spectral Data Visualization \& Analysis}
In spectral visualization and analysis, researchers employ various techniques to gain insights into the frequency content of signals. These methodologies enable the examination of how frequencies evolve over time, providing valuable information for tasks such as audio processing, speech recognition, and biomedical signal analysis.

\textbf{Spectrogram.}
Spectrograms stand as one of the primary tools in spectral visualization. They offer detailed representations of frequency spectra over time, revealing how the frequency composition of a signal changes temporally. By plotting frequency on the vertical axis, time on the horizontal axis, and intensity or magnitude using color or brightness, spectrograms provide a comprehensive view of signal dynamics. This detailed visualization allows analysts to identify specific features, patterns, and transient events within the signal, making spectrograms invaluable for tasks requiring fine-grained temporal frequency analysis.

\textbf{Spectral Centroid.}
In contrast to the detailed temporal-frequency mapping provided by spectrograms, spectral centroids offer a simplified summary of a signal's frequency content. The spectral centroid indicates the "center of mass" or average frequency of a signal within each time frame. This single-value representation reduces the complexity of the data while still providing a concise summary of the signal’s frequency characteristics. Spectral centroids are particularly useful for enhancing computational efficiency and maintaining robustness against noise and variations in the signal. However, they lack the detailed temporal information that spectrograms provide.

The synergy in using spectrograms and spectral centroids with different models lies in their ability to capture distinct and complementary features of audio signals. Spectrograms provide a comprehensive visualization of the frequency content over time, highlighting complex, high-dimensional patterns. In contrast, spectral centroids and waveforms represent simpler, more repetitive features, which are well-suited to the strengths of CNNs in learning local patterns through convolution and pooling operations.

By employing a MoE approach, the proposed model effectively combines these diverse representations. The spectrograms allow the model to capture detailed, global time-frequency information, while the spectral centroids and waveforms facilitate the extraction of robust, localized features. This integration leverages the strengths of both ViT and CNNs, resulting in a more accurate and holistic analysis of heart sounds.

\subsection{MoE}
MoE is a powerful ensemble learning methodology used in machine learning and statistical modeling. Within ensemble methods, multiple models are combined to improve predictive performance compared to any individual model. MoE takes this concept a step further by combining various models and adjusting the weight of their contributions adaptively per the input data.

In MoE, the "experts" are individual models or learners, each specializing in a particular region of the input space or addressing specific patterns in the data. These experts make predictions independently based on their specialized knowledge. The key innovation of MoE lies in the gating network, which dynamically selects the most relevant expert or combination of experts for each input instance.

The gating network, often implemented as a neural network, learns to assign weights to the experts based on the input data. These weights determine the contribution of each expert to the final prediction. By adaptively combining the predictions of multiple experts, MoE can capture complex relationships in the data and achieve superior predictive performance compared to traditional ensemble methods. The flowchart of the proposed experiment, depicted in Figure \ref{fig:flowchart}, illustrates the entire process, from input data processing to the final output generated by the MoE.


\section{Experimental Results}
\begin{table*}[t]
\centering
\caption{Total Variation Distance on CIFAR-10-LT ($N_l = 500$, $M_l = 4000$) with different class imbalance ratios $\gamma_l$ and $\gamma_u$ under five different unlabeled class distributions.}
\label{tab:cifar10-tv}
\resizebox{\textwidth}{!}{
\begin{tabular}{lccccccccccc}
\toprule
& & \multicolumn{2}{c}{consistent} & \multicolumn{2}{c}{uniform} & \multicolumn{2}{c}{reversed} & \multicolumn{2}{c}{middle} & \multicolumn{2}{c}{head-tail} \\
\cmidrule(lr){3-4} \cmidrule(lr){5-6} \cmidrule(lr){7-8} \cmidrule(lr){9-10} \cmidrule(lr){11-12}
& & $\gamma_l = 150$ & $\gamma_l = 100$ & $\gamma_l = 150$ & $\gamma_l = 100$ & $\gamma_l = 150$ & $\gamma_l = 100$ & $\gamma_l = 150$ & $\gamma_l = 100$ & $\gamma_l = 150$ & $\gamma_l = 100$ \\
Model & Estimator & $\gamma_u = 150$ & $\gamma_u = 100$ & $\gamma_u = 1$ & $\gamma_u = 1$ & $\gamma_u = 1/150$ & $\gamma_u = 1/100$ & $\gamma_u = 150$ & $\gamma_u = 100$ & $\gamma_u = 150$ & $\gamma_u = 100$ \\
\midrule
Supervised & MLLS & 0.269 ± 0.252 & 0.038 ± 0.006 & 0.251 ± 0.046 & 0.255 ± 0.060 & 0.429 ± 0.028 & 0.493 ± 0.050 & 0.333 ± 0.042 & 0.320 ± 0.009 & 0.457 ± 0.034 & 0.444 ± 0.043 \\
Supervised & RLLS & 0.043 ± 0.001 & 0.044 ± 0.010 & 0.348 ± 0.034 & 0.305 ± 0.068 & 0.769 ± 0.016 & 0.678 ± 0.028 & 0.430 ± 0.008 & 0.368 ± 0.013 & 0.539 ± 0.018 & 0.503 ± 0.020 \\
\midrule
MLE & IPW & 0.027 ± 0.001 & 0.027 ± 0.000 & 0.319 ± 0.072 & 0.243 ± 0.010 & 0.674 ± 0.020 & 0.646 ± 0.041 & 0.438 ± 0.020 & 0.454 ± 0.026 & 0.547 ± 0.049 & 0.491 ± 0.059 \\
MLE & OR & 0.045 ± 0.004 & 0.042 ± 0.000 & 0.215 ± 0.026 & 0.203 ± 0.032 & 0.433 ± 0.017 & 0.395 ± 0.033 & 0.193 ± 0.006 & 0.209 ± 0.037 & 0.307 ± 0.147 & 0.249 ± 0.130 \\
MLE & DR & 0.090 ± 0.002 & 0.079 ± 0.000 & 0.407 ± 0.027 & 0.360 ± 0.007 & 0.425 ± 0.007 & 0.421 ± 0.029 & 0.256 ± 0.001 & 0.286 ± 0.031 & 0.435 ± 0.136 & 0.362 ± 0.122 \\
\midrule
EM & IPW & 0.035 ± 0.002 & 0.040 ± 0.001 & 0.021 ± 0.001 & 0.029 ± 0.015 & 0.303 ± 0.187 & 0.091 ± 0.010 & 0.119 ± 0.011 & 0.105 ± 0.022 & 0.104 ± 0.026 & 0.104 ± 0.051 \\
EM & OR & 0.037 ± 0.003 & 0.042 ± 0.002 & 0.016 ± 0.001 & 0.024 ± 0.012 & 0.269 ± 0.183 & 0.090 ± 0.008 & 0.122 ± 0.012 & 0.103 ± 0.022 & 0.072 ± 0.012 & 0.073 ± 0.024 \\
EM & DR & 0.034 ± 0.004 & 0.037 ± 0.001 & 0.014 ± 0.001 & 0.027 ± 0.020 & 0.264 ± 0.191 & 0.092 ± 0.005 & 0.111 ± 0.019 & 0.097 ± 0.026 & 0.077 ± 0.016 & 0.073 ± 0.028 \\
\midrule
SimPro & IPW & 0.070 ± 0.011 & 0.058 ± 0.000 & 0.046 ± 0.001 & 0.049 ± 0.005 & 0.254 ± 0.074 & 0.223 ± 0.098 & 0.097 ± 0.025 & 0.067 ± 0.002 & 0.105 ± 0.066 & 0.110 ± 0.079 \\
SimPro & OR & 0.071 ± 0.012 & 0.058 ± 0.000 & 0.045 ± 0.001 & 0.049 ± 0.006 & 0.040 ± 0.003 & 0.059 ± 0.017 & 0.074 ± 0.006 & 0.075 ± 0.002 & 0.033 ± 0.003 & 0.033 ± 0.003 \\
SimPro & DR & 0.017 ± 0.004 & 0.026 ± 0.001 & 0.019 ± 0.002 & 0.018 ± 0.003 & 0.039 ± 0.003 & 0.058 ± 0.025 & 0.091 ± 0.007 & 0.031 ± 0.001 & 0.015 ± 0.003 & 0.019 ± 0.007 \\
\bottomrule
\end{tabular}
}
\end{table*}


\begin{table*}[t]
\centering
\caption{Total Variation Distance on CIFAR-100-LT ($N_l = 50$, $M_l = 400$) with different class imbalance ratios $\gamma_l$ and $\gamma_u$ under five different unlabeled class distributions.}
\label{tab:cifar100-tv}
\resizebox{\textwidth}{!}{
\begin{tabular}{lccccccccccc}
\toprule
& & \multicolumn{2}{c}{consistent} & \multicolumn{2}{c}{uniform} & \multicolumn{2}{c}{reversed} & \multicolumn{2}{c}{middle} & \multicolumn{2}{c}{head-tail} \\
\cmidrule(lr){3-4} \cmidrule(lr){5-6} \cmidrule(lr){7-8} \cmidrule(lr){9-10} \cmidrule(lr){11-12}
& & $\gamma_l = 20$ & $\gamma_l = 10$ & $\gamma_l = 20$ & $\gamma_l = 10$ & $\gamma_l = 20$ & $\gamma_l = 10$ & $\gamma_l = 20$ & $\gamma_l = 10$ & $\gamma_l = 20$ & $\gamma_l = 10$ \\
Model & Estimator & $\gamma_u = 20$ & $\gamma_u = 10$ & $\gamma_u = 1$ & $\gamma_u = 1$ & $\gamma_u = 1/20$ & $\gamma_u = 1/10$ & $\gamma_u = 20$ & $\gamma_u = 10$ & $\gamma_u = 20$ & $\gamma_u = 10$ \\
\midrule
Supervised & MLLS & 0.707 ± 0.016 & 0.313 ± 0.100 & 0.445 ± 0.172 & 0.309 ± 0.119 & 0.383 ± 0.075 & 0.397 ± 0.006 & 0.570 ± 0.001 & 0.373 ± 0.107 & 0.543 ± 0.009 & 0.231 ± 0.057 \\
Supervised & RLLS & 0.520 ± 0.007 & 0.133 ± 0.003 & 0.337 ± 0.125 & 0.253 ± 0.082 & 0.424 ± 0.060 & 0.463 ± 0.003 & 0.454 ± 0.021 & 0.306 ± 0.074 & 0.460 ± 0.028 & 0.241 ± 0.040 \\
\midrule
MLE & IPW & 0.075 ± 0.000 & 0.071 ± 0.001 & 0.229 ± 0.001 & 0.167 ± 0.002 & 0.565 ± 0.005 & 0.443 ± 0.007 & 0.415 ± 0.000 & 0.311 ± 0.005 & 0.343 ± 0.000 & 0.280 ± 0.001 \\
MLE & OR & 0.065 ± 0.002 & 0.061 ± 0.001 & 0.200 ± 0.007 & 0.143 ± 0.001 & 0.526 ± 0.011 & 0.399 ± 0.023 & 0.360 ± 0.003 & 0.256 ± 0.012 & 0.328 ± 0.003 & 0.266 ± 0.005 \\
MLE & DR & 0.149 ± 0.019 & 0.145 ± 0.010 & 0.243 ± 0.004 & 0.214 ± 0.019 & 0.568 ± 0.005 & 0.464 ± 0.014 & 0.403 ± 0.014 & 0.309 ± 0.012 & 0.365 ± 0.007 & 0.320 ± 0.004 \\
\midrule
EM & IPW & 0.097 ± 0.008 & 0.092 ± 0.004 & 0.239 ± 0.007 & 0.179 ± 0.003 & 0.478 ± 0.012 & 0.329 ± 0.020 & 0.262 ± 0.016 & 0.202 ± 0.003 & 0.312 ± 0.002 & 0.227 ± 0.001 \\
EM & OR & 0.121 ± 0.007 & 0.108 ± 0.005 & 0.261 ± 0.007 & 0.189 ± 0.004 & 0.489 ± 0.013 & 0.335 ± 0.020 & 0.274 ± 0.016 & 0.211 ± 0.004 & 0.336 ± 0.003 & 0.235 ± 0.001 \\
EM & DR & 0.125 ± 0.005 & 0.111 ± 0.004 & 0.269 ± 0.007 & 0.194 ± 0.005 & 0.497 ± 0.010 & 0.336 ± 0.024 & 0.281 ± 0.019 & 0.219 ± 0.008 & 0.336 ± 0.007 & 0.233 ± 0.004 \\
\midrule
SimPro & IPW & 0.125 ± 0.001 & 0.100 ± 0.005 & 0.166 ± 0.007 & 0.141 ± 0.009 & 0.353 ± 0.023 & 0.261 ± 0.008 & 0.202 ± 0.003 & 0.158 ± 0.005 & 0.277 ± 0.009 & 0.197 ± 0.003 \\
SimPro & OR & 0.133 ± 0.005 & 0.100 ± 0.004 & 0.160 ± 0.007 & 0.138 ± 0.010 & 0.322 ± 0.014 & 0.253 ± 0.008 & 0.202 ± 0.003 & 0.156 ± 0.005 & 0.269 ± 0.006 & 0.191 ± 0.004 \\
SimPro & DR & 0.122 ± 0.003 & 0.106 ± 0.006 & 0.188 ± 0.009 & 0.149 ± 0.006 & 0.343 ± 0.023 & 0.257 ± 0.007 & 0.219 ± 0.010 & 0.172 ± 0.002 & 0.279 ± 0.007 & 0.198 ± 0.004 \\
\bottomrule
\end{tabular}
}
\end{table*}
\begin{table*}[t]
\centering
\caption{Top-1 accuracy (\%) on CIFAR-10-LT ($N_l = 500$, $M_l = 4000$) with different class imbalance ratios $\gamma_l$ and $\gamma_u$ under five different unlabeled class distributions. In most settings, our two stage algorithm improves SimPro (9 / 10) and BOAT (8 / 10). We use {\green green} to indicate when our plug-in improves and {\red red} when it degrades the base model.}
\label{tab:cifar10-acc}
\resizebox{\textwidth}{!}{
\begin{tabular}{lcccccccccc}
\toprule

& \multicolumn{2}{c}{consistent} & \multicolumn{2}{c}{uniform} & \multicolumn{2}{c}{reversed} & \multicolumn{2}{c}{middle} & \multicolumn{2}{c}{head-tail} \\
\cmidrule(lr){2-3} \cmidrule(lr){4-5} \cmidrule(lr){6-7} \cmidrule(lr){8-9} \cmidrule(lr){10-11}

& $\gamma_l = 150$ & $\gamma_l = 100$ & $\gamma_l = 150$ & $\gamma_l = 100$ & $\gamma_l = 150$ & $\gamma_l = 100$ & $\gamma_l = 150$ & $\gamma_l = 100$ & $\gamma_l = 150$ & $\gamma_l = 100$ \\
& $\gamma_u = 150$ & $\gamma_u = 100$ & $\gamma_u = 1$ & $\gamma_u = 1$ & $\gamma_u = 1/150$ & $\gamma_u = 1/100$ & $\gamma_u = 150$ & $\gamma_u = 100$ & $\gamma_u = 150$ & $\gamma_u = 100$ \\

\midrule

FixMatch & 62.9 $\pm$ 0.36 & 67.8 $\pm$ 1.13 & 67.6 $\pm$ 2.56 & 73.0 $\pm$ 3.81 & 59.9 $\pm$ 0.82 & 62.5 $\pm$ 0.94 & 64.3 $\pm$ 0.63 & 71.7 $\pm$ 0.46 & 58.3 $\pm$ 1.46 & 66.6 $\pm$ 0.87 \\
CReST+ & 67.5 $\pm$ 0.45 & 76.3 $\pm$ 0.86 & 74.9 $\pm$ 0.90 & 82.2 $\pm$ 1.53 & 62.0 $\pm$ 1.18 & 62.9 $\pm$ 1.39 & 58.5 $\pm$ 0.68 & 71.4 $\pm$ 0.60 & 59.3 $\pm$ 0.72 & 67.2 $\pm$ 0.48 \\
DASO & 70.1 $\pm$ 1.81 & 76.0 $\pm$ 0.37 & 83.1 $\pm$ 0.47 & 86.6 $\pm$ 0.84 & 64.0 $\pm$ 0.11 & 71.0 $\pm$ 0.95 & 69.0 $\pm$ 0.31 & 73.1 $\pm$ 0.68 & 70.5 $\pm$ 0.59 & 71.1 $\pm$ 0.32 \\
% w/ ACR$\dagger$ (Wei \& Gan, 2023) & 70.9 $\pm$ 0.37 & 76.1 $\pm$ 0.42 & 91.9 $\pm$ 0.02 & 92.5 $\pm$ 0.19 & 83.2 $\pm$ 0.39 & 85.2 $\pm$ 0.12 & 77.6 $\pm$ 0.20 & 79.3 $\pm$ 0.30 & 73.8 $\pm$ 0.83 & 79.3 $\pm$ 0.48 \\
% w/ SimPro & 74.2 $\pm$ 0.90 & 80.7 $\pm$ 0.30 & 93.6 $\pm$ 0.08 & 93.8 $\pm$ 0.10 & 83.5 $\pm$ 0.95 & 85.8 $\pm$ 0.48 & 82.6 $\pm$ 0.38 & 84.8 $\pm$ 0.54 & 81.0 $\pm$ 0.27 & 83.0 $\pm$ 0.36 \\
Supervised & 63.2 $\pm$ 0.14 & 66.0 $\pm$ 0.27 & 63.3 $\pm$ 0.28 & 65.8 $\pm$ 0.19 & 63.1 $\pm$ 0.19 & 65.9 $\pm$ 0.51 & 63.5 $\pm$ 0.22 & 65.8 $\pm$ 0.03 & 63.0 $\pm$ 0.18 & 66.4 $\pm$ 0.07 \\
\midrule
EM & 69.1 $\pm$ 1.29 & 73.8 $\pm$ 0.71 & 94.0 $\pm$ 0.08 & 93.2 $\pm$ 0.94 & 76.6 $\pm$ 2.72 & 82.2 $\pm$ 0.24 & 79.5 $\pm$ 0.35 & 81.6 $\pm$ 0.58 & 79.2 $\pm$ 0.50 & 79.8 $\pm$ 0.17 \\
\midrule
SimPro & 74.4 $\pm$ 0.71 & 79.7 $\pm$ 0.45 & 93.3 $\pm$ 0.10 & 93.3 $\pm$ 0.47 & 83.8 $\pm$ 0.80 & 84.1 $\pm$ 0.24 & 78.7 $\pm$ 0.30 & 84.2 $\pm$ 0.26 & 81.2 $\pm$ 0.20 & 82.0 $\pm$ 1.07 \\
% \midrule
SimPro+ & \green 77.8 $\pm$ 1.50 & \green 81.2 $\pm$ 0.39 & \green 93.7 $\pm$ 0.07 & \green 93.7 $\pm$ 0.24 & \red 83.3 $\pm$ 0.38 & \green 84.7 $\pm$ 0.78 & \green 79.2 $\pm$ 0.70 & \green 85.4 $\pm$ 0.66 & \green 81.3 $\pm$ 0.27 & \green 82.5 $\pm$ 0.56 \\
\midrule
BOAT & 80.5 $\pm$ 0.39 & 83.3 $\pm$ 0.27 & 93.9 $\pm$ 0.03 & 94.1 $\pm$ 0.10 & 79.7 $\pm$ 0.25 & 81.1 $\pm$ 0.15 & 79.7 $\pm$ 1.15 & 81.6 $\pm$ 0.09 & 79.4 $\pm$ 0.44 & 80.9 $\pm$ 0.16 \\
% \midrule
BOAT+ & \green 81.6 $\pm$ 0.15 & \green 83.8 $\pm$ 0.04 & \red 93.7 $\pm$ 0.23 & 94.1 $\pm$ 0.17 & \green 80.4 $\pm$ 0.71 & \green 81.7 $\pm$ 0.38 & \green 80.3 $\pm$ 0.28 & \green 83.1 $\pm$ 0.45 & \green 79.7 $\pm$ 0.29 & \green 81.0 $\pm$ 0.36 \\
\bottomrule
\end{tabular}
}
\end{table*}

\begin{table*}[t]
\centering
\caption{Top-1 accuracy (\%) on CIFAR-100-LT ($N_l = 50$, $M_l = 400$) with different class imbalance ratios $\gamma_l$ and $\gamma_u$ under five different unlabeled class distributions. Despite poor estimation in stage 1, our approach does not degrade the accuracy for most of the settings. We use {\green green} to indicate when our plug-in improves and {\red red} when it degrades the base method.}
\label{tab:cifar100-acc}
\resizebox{\textwidth}{!}{
\begin{tabular}{lccccccccccc}
\toprule

& \multicolumn{2}{c}{consistent} & \multicolumn{2}{c}{uniform} & \multicolumn{2}{c}{reversed} & \multicolumn{2}{c}{middle} & \multicolumn{2}{c}{head-tail} \\
\cmidrule(lr){2-3} \cmidrule(lr){4-5} \cmidrule(lr){6-7} \cmidrule(lr){8-9} \cmidrule(lr){10-11}

& $\gamma_l = 20$ & $\gamma_l = 10$ & $\gamma_l = 20$ & $\gamma_l = 10$ & $\gamma_l = 20$ & $\gamma_l = 10$ & $\gamma_l = 20$ & $\gamma_l = 10$ & $\gamma_l = 20$ & $\gamma_l = 10$ \\
& $\gamma_u = 20$ & $\gamma_u = 10$ & $\gamma_u = 1$ & $\gamma_u = 1$ & $\gamma_u = 1/20$ & $\gamma_u = 1/10$ & $\gamma_u = 20$ & $\gamma_u = 10$ & $\gamma_u = 20$ & $\gamma_u = 10$ \\

\midrule
% FixMatch & 40.0 $\pm$ 0.96 & 45.2 $\pm$ 0.55 & 39.6 $\pm$ 1.16 & \\
% CReST+ & 40.1 $\pm$ 1.28 & 44.5 $\pm$ 0.94 & 37.6 $\pm$ 0.88 & \\
% DASO & 43.0 $\pm$ 0.15 & 49.8 $\pm$ 0.24 & 49.4 $\pm$ 0.93 & \\
Supervised & 32.4 $\pm$ 0.40 & 38.4 $\pm$ 0.18 & 32.7 $\pm$ 0.25 & 38.0 $\pm$ 0.22 & 32.5 $\pm$ 0.51 & 38.4 $\pm$ 0.43 & 32.3 $\pm$ 0.08 & 37.9 $\pm$ 0.43 & 32.1 $\pm$ 0.33 & 38.2 $\pm$ 0.38 \\
% \midrule
EM & 42.4 $\pm$ 0.43 & 49.6 $\pm$ 0.30 & 50.9 $\pm$ 0.27 & 58.0 $\pm$ 0.35 & 42.1 $\pm$ 0.16 & 49.8 $\pm$ 0.47 & 42.8 $\pm$ 0.41 & 49.6 $\pm$ 0.36 & 41.5 $\pm$ 1.26 & 49.5 $\pm$ 0.18 \\
\midrule
SimPro & 42.5 $\pm$ 0.58 & 49.6 $\pm$ 0.22 & 51.7 $\pm$ 0.22 & 58.1 $\pm$ 0.53 & 44.9 $\pm$ 0.21 & 51.8 $\pm$ 0.42 & 42.7 $\pm$ 0.06 & 49.8 $\pm$ 0.45 & 43.3 $\pm$ 0.76 & 50.9 $\pm$ 0.19 \\
% \midrule
SimPro+ & \green 42.8 $\pm$ 0.49 & \green 50.1 $\pm$ 0.33 & \red 51.6 $\pm$ 0.63 & \red 57.8 $\pm$ 0.38 & \red 44.7 $\pm$ 0.51 & \red 51.4 $\pm$ 0.88 & \green 43.4 $\pm$ 0.58 & \green 50.4 $\pm$ 0.28 & \green 43.8 $\pm$ 0.50 & \red 50.7 $\pm$ 0.76 \\
\midrule
BOAT & 43.7 $\pm$ 0.16 & 51.4 $\pm$ 0.32 & 55.1 $\pm$ 0.95 & 60.5 $\pm$ 0.15 & 43.1 $\pm$ 0.49 & 52.7 $\pm$ 0.23 & 43.6 $\pm$ 0.19 & 51.4 $\pm$ 0.39 & 43.9 $\pm$ 0.42 & 51.4 $\pm$ 0.14 \\
% \midrule
BOAT+ & \green 44.8 $\pm$ 0.13 & 51.4 $\pm$ 0.51 & \red 53.8 $\pm$ 0.32 & 60.5 $\pm$ 0.69 & \green 43.4 $\pm$ 0.56 & \red 52.4 $\pm$ 0.36 & \green 43.9 $\pm$ 0.59 & \red 50.8 $\pm$ 0.09 & \red 43.6 $\pm$ 0.50 & \green 51.9 $\pm$ 0.49 \\
\bottomrule
\end{tabular}
}
\end{table*}

We perform experiments for each stage of our algorithm. In the first stage, we compare among various methods to estimate the unlabeled class distribution $P(Y|A=0)$, showing that SimPro + DR performs well. In the second stage, we freeze the unlabeled class distribution, using our best estimator  SimPro + DR, and plug it into 2 SOTA semi-supervised learning algorithms, SimPro and BOAT~\cite{boat}. We show that this simple procedure improves the existing methods, and is even capable of improving the original SimPro when used for both stages.


% \textbf{Datasets} We adopt 4 standard benchmarks used frequently in other semi-supervised learning work: CIFAR-10, CIFAR-100~\cite{cifar}, STL-10~\cite{stl10} and Imagenet-127~\cite{cossl}. To match our RTSSL setting, we create long-tailed labeled and unlabeled sets from CIFAR-10 and CIFAR-100. Specifically, we use $\gamma_l$ and $n_1$ to denote the imbalance ratio and the head class's number of samples of the labeled data, the remaining class's size is computed as $n_c = n_1 \times \gamma_l^{-\frac{c-1}{C-1}}$ and likewise, $\gamma_u$ and $m_1$ of the unlabeled data. For CIFAR-10, we fix $n_1=500$ and $m_1=4000$. We test 2 different configurations $\gamma_l=\gamma_c=150$ and $\gamma_l=\gamma_c=100$. We further permute classes the unlabeled sets in 5 ways: consistent, uniform, reversed, middle and headtail, similar to \cite{simpro} and visualized in figure~\ref{fig:distribution}, which results in 10 different datasets in total. Similarly for CIFAR-100, we fix $n_1=500$ and $m_1=4000$, use 2 configurations $\gamma_l=\gamma_c=20$ and $\gamma_l=\gamma_c=10$, and permute the classes in 5 ways, resulting in 10 datasets as well. For STL-10, the unlabeled set has no ground truth labels, therefore we use all samples in the head class and set the imbalance ratio $\gamma_l$ to $10$ or $20$. Imagenet-127 is a naturally long-tailed dataset with 127 classes. We train on 32x32 and 64x64 image resolutions following ~\cite{cossl}.


\textbf{Datasets} We evaluate our method on four standard semi-supervised learning benchmarks: CIFAR-10, CIFAR-100~\cite{cifar}, STL-10~\cite{stl10}, and Imagenet-127~\cite{cossl}. To simulate RTSSL, we construct long-tailed labeled and unlabeled sets for CIFAR-10 and CIFAR-100. The labeled data follows an imbalance ratio $\gamma_l$ with head class size $n_1$, while the remaining class sizes are computed as $n_c = n_1 \times \gamma_l^{-\frac{c-1}{C-1}}$. The unlabeled data follows a similar setup with $\gamma_u$ and $m_1$.  

For CIFAR-10, we set $n_1 = 500$, $m_1 = 4000$, and test two configurations: $\gamma_l = \gamma_u = 150$ and $\gamma_l = \gamma_u = 100$. We generate 10 datasets by permuting the unlabeled class distributions in five ways: \textit{consistent, uniform, reversed, middle}, and \textit{head-tail}, as in~\cite{simpro}. CIFAR-100 follows the same setup with $n_1 = 50$, $m_1 = 400$, and $\gamma_l, \gamma_u$ values of 20 and 10.  

For STL-10, where unlabeled data lacks ground-truth labels, we use all head-class samples and set $\gamma_l$ to 10 or 20. Imagenet-127 is naturally long-tailed with 127 classes, and we train on 32$\times$32 and 64$\times$64 resolutions as in~\cite{cossl}.


\paragraph{Training.} We follow the implementation and hyperparameter settings of \cite{simpro}. We defer these details in \cref{subsec:training-setting}. One important exception is that for Imagenet-127, we use the smaller Wide ResNet-28-2 in stage 1 and the larger ResNet-50 for stage 2, to demonstrate that a smaller model is sufficient for distribution estimation.


\begin{table}[t]
\small
\centering
\caption{Top-1 Accuracy (\%) on STL-10. Our two-stage algorithms improves both SimPro and BOAT for both settings.}
\label{tab:stl10-acc}
% \resizebox{\linewidth}{!}{
\begin{tabular}{lcc}
\toprule
Method & $\gamma_l=10$ & $\gamma_l=20$ \\ \hline
Supervised & 73.9 $\pm$ 0.57 & 70.4 $\pm$ 0.95 \\
\midrule
MLE & 67.6 $\pm$ 0.57 & 58.9 $\pm$ 4.05 \\
\midrule
EM & 84.9 $\pm$ 0.14 & 83.6 $\pm$ 0.25 \\
\midrule
SimPro & 82.4 $\pm$ 1.57 & 80.5 $\pm$ 0.96 \\
SimPro+ & \green 83.9 $\pm$ 0.76 & \green 82.7 $\pm$ 0.86 \\
\midrule
BOAT & 83.8 $\pm$ 0.20 & 82.0 $\pm$ 0.34 \\
BOAT+ & \green 84.1 $\pm$ 0.38 & \green 82.4 $\pm$ 0.10 \\
\bottomrule
\end{tabular}
\end{table}















\begin{table}[t]
% \setlength{\tabcolsep}{3.5pt}
\small
\centering
\caption{Top-1 Accuracy (\%) on Imagenet-127. Our two-stage approach improves both SimPro and BOAT for both resolutions.}
\label{tab:imagenet-127-acc}
% \resizebox{\linewidth}{!}{
\begin{tabular}{lcc}
\toprule
Method & $32 \times 32$ & $64 \times 64$ \\ \hline
SimPro & 54.8 & 63.7 \\
SimPro+ & \green 55.1 & \green 64.2 \\
\midrule
BOAT & 51.6 & 58.7 \\
BOAT+ & \green 52.0 & \green 59.2 \\

\bottomrule
\end{tabular}
% }
\end{table}


\begin{table}[t]
% \setlength{\tabcolsep}{3.5pt}
\small\centering
\caption{Total Variation Distance on Imagenet-127}
\label{tab:imagenet-127-tv}
% \resizebox{\linewidth}{!}{
\begin{tabular}{cccc}
\toprule
Method & Estimator & $32 \times 32$ & $64 \times 64$ \\ \hline
MLE & IPW  & 0.103 $\pm$ 0.034 & 0.051 $\pm$ 0.000 \\
MLE & OR  & 0.153 $\pm$ 0.052 & 0.041 $\pm$ 0.000 \\
MLE & DR  & \green 0.100 $\pm$ 0.029 & \green 0.075 $\pm$ 0.003 \\
\midrule
EM & IPW  & 0.141 $\pm$ 0.006 & 0.163 $\pm$ 0.010 \\
EM & OR  & 0.205 $\pm$ 0.006 & 0.236 $\pm$ 0.011 \\
EM & DR  & \green 0.024 $\pm$ 0.001 & \green 0.042 $\pm$ 0.004 \\
\midrule
SimPro & IPW  & 0.041 $\pm$ 0.012 & 0.224 $\pm$ 0.040 \\
SimPro & OR  & 0.036 $\pm$ 0.014 & 0.291 $\pm$ 0.079 \\
SimPro & DR  & \green 0.017 $\pm$ 0.000 & \green 0.037 $\pm$ 0.004 \\
\bottomrule
\end{tabular}
% }
\end{table}

\subsection{Better results on label distribution} 
\label{subsec:label}
We have mentioned various ways throughout the papers to estimate the unlabeled class distribution. In what follows, method consists of a model, which is how the learning is done, and an estimator, which is how the final distribution is estimated using parameters learned from the model.

%\begin{enumerate}
%\item 
\noindent
\textbf{Supervised}. The model is trained on the labeled set only and used to estimate the unlabeled class distribution \cite{unifiedlabelshift}. 2 successful estimators for this setting are \textbf{RLLS} \cite{rlls} and \textbf{MLLS} \cite{mlls}. 

%\item 
\noindent\textbf{MLE}. The model is trained by directly maximizing the likelihood \cref{eq:likelihood}. We also use the decomposition $P(Y|X)$ and $P(A|Y)$, and write the unlabeled term as $P(A=0, X) = \sum_{c} P(Y=c|X) P(A=0|Y=c)$, which enables gradient descent training on these parameters. This is also the MLE method to estimate $P(A|Y)$ in \cite{arelabelsinformative}.

%\item 
\noindent\textbf{EM}. We further test the EM algorithm in \cref{subsec:em}. In particular we also use strong and weak augmentations similar to FixMatch, but not the pseudo labeling operator. Confidence thresholding removes the soft predictions of the non-dominant classes, which may be better to keep since our target of the first stage is the global class statistics. We also try 3 estimators on this model.

%\item 
\noindent\textbf{SimPro} \cite{simpro} can be seen as our previous EM but also with FixMatch's confidence thresholding and logit adjustment loss in \cref{subsec:simpro}. Confidence thresholding is a powerful regularization technique that encodes the assumption that classes are well separated \cite{entropyminimization}, but can introduce bias to the estimation, which justifies the use of DR.
%\end{enumerate}

% For semi-supervised methods MLE, EM and SimPro, as we now have additional information on the missingness mechanism, we can use 3 estimators OR, IPW and DR presented in \cref{subsec:2-stage}


Results on \cref{tab:cifar10-tv} presents the performance of various models and estimators on CIFAR-10. We can see that SimPro + DR performs best. In contrast, SimPro + OR, SimPro's original way of estimating $P(Y|A=0)$, and SimPro + IPW tend to underperform EM on the consistent and uniform datasets. The consistent setting is worth noting, since it arises when data is sampled uniformly at random for labeling,  representative of a large number of real world situations. EM is competitive to SimPro as well even without pseudo labeling, but overall we found this regularization to offer significant gains in the reversed, middle and head-tail settings. Finally, Supervised with either MLLS or RLLS estimators performs much worse than the semi-supervise methods.

\cref{tab:imagenet-127-tv} aligns with the observations  made in \cref{tab:cifar10-tv}. In particular, SimPro + DR is the best method for class distribution estimation of the much larger Imagenet-127. We also found that a small neural network and a small image resolution is sufficient for the distribution estimation of the much larger dataset Imagenet-127. This matches our intuition that the finite-dimensional parameter is easier to learn.

\cref{tab:cifar100-tv} shows that most methods understandably struggle to estimate the class distributions in CIFAR-100. This is because there are few samples in each class, the head class has 10 times less samples while the number of classes multiplies 10 times compared to CIFAR-10. We see here that SimPro + DR is not the best method, but the relative gap between estimators are small.

% Among the models, the supervised baseline do not perform well even in the consistent setting, showing that when unlabeled data is available during training, learning from them can be valuable for class distribution estimation, especially in the cases with little labeled data like ours. Both the MLE and supervised models perform badly on the reversed, middle and head-tail settings

% Among the estimators, we see that DR boosts the performance of SimPro and EM in CIFAR-10, and of all semi-supervised models in Imagenet-127. It does not improve MLE on CIFAR-10, and it does not improve on CIFAR-100. However, for most of the time, the decrease is not much. In constrast, IPW estimators can be significantly worse, for example in the reversed setting of CIFAR-10, where the distance is $0.254$ for $\gamma_l=150$ and $0.233$ for $\gamma_l=100$, compared to OR's 0.040 and 0.059. 

% Both the MLE and supervised models perform badly on the reversed, middle and head-tail settings. EM does a decent job, though not as well as SimPro, on all 5 distribution settings of CIFAR-10. However, on Imagenet-127, EM without DR performs worse than MLE.

% We note that the performance on DR is similar to OR in these cases, showing that DR has a double robustness property. While IPW only relies on the finite-dimensional $P(A|Y)$, which intuitively is easy to estimate, we found that the inverse probability weight can nevertheless be unstable when some probabilities are small, and this is where DR shows its strength by combining both IPW and OR.



\subsection{Two-stage algorithm improves accuracy}

In the second stage of our algorithm, we freeze our estimation and plug it in SimPro and BOAT. We denote SimPro+ and BOAT+ for algorithms that use our first stage estimate.



\cref{tab:cifar10-acc} shows that for CIFAR-10 SimPro+ and BOAT+ improve over their original versions across most settings, leading to large improvements in both the consistent and middle class distribution settings. In particular, our two-stage approach improves SimPro in 9 / 10 settings and BOAT in 8 / 10 settings.
We also observe consistent improvements ove both base algorithms, SimPro and BOAT, for several other datasets. \cref{tab:stl10-acc} demonstrates improvements for 2 / 2 class imbalance ratios in STL-10 and \cref{tab:imagenet-127-acc} for 2 / 2  different resolutions of ImageNet-127. 


We also evaluate on CIFAR-100 for multiple unlabeled  class distribution settings and with mediocre class label distribution estimates in stage 1, demonstrate no degradation in accuracy in stage 2. As shown in \cref{tab:cifar100-acc}, the two stage algorithm with a mediocre stage 1 estimation leads to parity with the baseline. Stage 2 provides small improvements in 5 / 10 settings for SimPro and in 4 / 10 (with 2 ties) for BOAT.


\subsection{Ablation Study: Alternative implementations.}
\label{subsec:ablation-1}
In this section, we ablate on our 2-stage choice. Specifically, we consider 2 alternative implementations:
\paragraph{\textbf{Doubly-robust risk}}  
This approach is \cite{arelabelsinformative, onnonrandommissinglabels}, as discussed in \cref{sec:background}. we consider the doubly-robust risk as our training loss. We use the missingness mechanism estimation from stage-1 of SimPro+ for fair comparison. \cref{eq:dr-risk} is used for training in which the pseudo-labeling operators can be applied straightforwardly. More detail in \cref{subsec:dr-risk}
\paragraph{\textbf{Batch-update doubly-robust $P(Y|A)$}} Different from SimPro+, here we remove the first stage and instead update our doubly robust estimation of the unlabeled class distribution using a moving average of the batch statistics.

\cref{tab:cifar10-ablation-1} shows that the batch-update version of SimPro+ is significantly worse on the consistent and uniform settings, while the doubly-robust risk is worst overall, especially in the reversed setting where $P(A|Y)$ is very small for the labeled tail classes, causing instability issues during training. In conclusion, our 2-stage approach is the best.


\begin{table}[t]
\small
\centering
\caption{Top-1 Accuracy (\%) on CIFAR-10. We compare our 2-stage SimPro+ with 1) an 1-stage alternative that updates and uses the doubly-robust estimation on-the-fly and 2) SimPro with doubly-robust risk. We use $\gamma_l=150$. {\green green} color indicates that our method performs best.}
\label{tab:cifar10-ablation-1}
\resizebox{\linewidth}{!}{
\begin{tabular}{lccccc}
\toprule
Method & consistent & uniform & reversed & middle & headtail\\ \hline
SimPro+ & \green 77.8 & \green 93.7 & \green 83.3 & \green 79.2 & \green 81.3 \\
batch-update & 71.9 & 91.4 & 82.6 & 78.6 & 81.2 \\
DR-risk & 72.1 & 89.8 & 67.1 & 75.6 & 79.5 \\
\bottomrule
\end{tabular}
}
\end{table}
\section{Evaluation Results} \label{section:restuls}

\subsection{Interference-Free Analysis}
\noindent
\textbf{Performance of the \texttt{Exchange} primitive.}
Figure~\ref{fig:io-bandwidth} illustrates a comparison of the IO throughput achieved by our optimized \texttt{Exchange} and the baseline solution, which solely relies on the GPU runtime. 
We vary the total amount of data transferred from 2GB to 16GB and adjust the packet size from 10MB to 80MB. 
The combination of data size and packet size determines the total number of packets, which in turn affects the number of pipeline stages required for data transfer. 
Too few pipeline stages can lead to significant overhead in the prologue and epilogue phases of the pipeline. 
Conversely, utilizing excessively small packets is also inefficient, as each memory copy incurs a fixed overhead from the runtime, regardless of the transferred data volume. 
Therefore, small packet sizes exacerbate this overhead, making it disproportionately large.

Our solution achieves up to 140GB/s throughput when transferring 8GB or more of data. 
When the total amount of data transferred is small, we observe a decrease in throughput due to the reduced number of pipeline stages. 
As previously explained, this issue cannot be alleviated by simply reducing the packet size. 
For instance, while a packet size of 10MB provides better performance compared to an 80MB packet size when transferring 2GB of data, it delivers lower throughput when the data size exceeds 8GB. 
Empirically, we find that a packet size of 20MB strikes a balance, achieving desirable performance for small and large data transfers.
Consequently, we use a packet size of 20MB for all the applications evaluated below.

Compared to the baseline, which fully relies on the GPU runtime, our solution is not only more performant but also more stable. 
Such a baseline fails to fully exploit the full-duplex capabilities of PCIe links, achieving only about 110-130GB/s throughput when transferring data bidirectionally. 
Additionally, its performance is highly unstable due to the irregular PCIe bandwidth, especially when the CPU DRAM bandwidth becomes saturated.

%%%%%%%%%%%% OLD TEXT START %%%%%%%%%%%%
\begin{comment}

Figure \ref{fig:io-bandwidth} compares the IO throughput achieved by our optimized \texttt{Exchange} with the one achieved by the baseline solution only relying on the GPU runtime.
We vary the total amount of data transferred from 2GB to 16GB, and the packet size from 10MB to 80MB.
The amount of data and packet size determines the total number of packets, which consequently determines the number of pipeline stages for the data transfer.
If there are too few pipeline stages, the overhead in the prologue and epilogue of the pipeline becomes considerable.
On the other hand, using tiny packets is also unacceptable.
Each memory copy pays a fixed overhead for the runtime regardless of the amount of data being transferred.
Tiny packets make such overhead significant.

Our solution achieves up to around 140GB/s throughput when transferring 8GB or more data.
When the total amount of data transferred is less, we observe decreased throughput due to fewer pipeline stages.
As explained, this can not be relieved by reducing the packet size.
While the case of 10 MB packet size achieves a better performance than the case of 80MB packet size when the total amount of data being transferred is 2GB, it delivers less throughput when the data size is larger than 8GB.
Empirically, we find 20MB is a sweet point that achieves desirable performance in transferring small and large amounts of data.
Thus, we use 20MB for all the applications evaluated below.

Compared to the baseline that fully depends on the GPU runtime, our solution is not only more performant but also more stable.
The baseline solution fails to take advantage of the full-duplex property of PCIe links properly, thus it only achieves around 110-130GB/s throughput when transferring the traffic in both directions. 
Besides, its performance is highly unstable due to the irregular PCIe bandwidth when the CPU DRAM bandwidth is saturating.
\end{comment}
%%%%%%%%%%%% OLD TEXT END %%%%%%%%%%%%

% \begin{figure}
%     \centering
%     \includegraphics[width=0.8\linewidth]{figures/sort-result.pdf}
%     \caption{Results for Sort. (a) the throughput achieved by different solutions, (b) the time breakdown for the \THISWORK\ sort, and (c) the time taken by on-GPU kernel execution of a typical pipeline stage.}
%     \label{fig:sort-perf}
% \end{figure}

\noindent
\textbf{Performance of Sort.}
We compare our sort implementation with CPU and GPU baselines in Figure~\ref{fig:sort-perf}(a). 
Our implementation achieves a throughput of 2.7 billion elements per second, which is 27.9$\times$ faster than TBB, 6.3$\times$ faster than PARADIS, and 1.7$\times$ faster than the configuration using only one GPU's IO resources. 
Figure~\ref{fig:sort-perf}(b) provides a breakdown of the sort operation, revealing that 65.1\% of the time is consumed by the \texttt{SortExKernel}. 
This occurs because, after enhancing the IO throughput, the sorting operation becomes bounded by the GPU processing throughput, as illustrated in Figure~\ref{fig:sort-perf}(c). 
While it takes the GPU approximately 208ms to sort a partition of 500 million 8-byte integers, transferring that partition to the GPU using four GPUs' IO resources requires only about 113ms. 
This limitation explains why we do not achieve nearly a 4$\times$ speedup compared to the single GPU IO solution. 
Conversely, the \texttt{MergeExKernel} remains IO-bound, with the on-GPU kernel completing in approximately 67ms.

%%%%%%%%%%%% OLD TEXT START %%%%%%%%%%%%
\begin{comment}
We compare our sort implementation with the CPU and GPU baselines in Figure~\ref{fig:sort-perf}(a).
Our sort implementation achieves 2.67B elements per second throughput, which is 27.9$\times$ compared to TBB, 6.3$\times$ compared to PARADIS, and 1.7$\times$ compared to the case using only one GPU's IO.
Figure~\ref{fig:sort-perf}(b) is the time breakdown of the sort operation, where 65.1\% of time is spent on the \texttt{SortExOperation}.
The reason is that after we enhance the IO throughput, sorting the array by partition is bounded by GPU-processing throughput, which is showcased in Figure~\ref{fig:sort-perf}(c).
It takes the GPU ~208ms to sort a partition of 500M 8-byte integers, but only ~113ms to transfer that partition to GPU using 4 GBUs' IO resources.
This is why we do not achieve close to 4$\times$ speedup compared to the single GPU IO solution.
On the other hand, \texttt{MergeExOperation} is still IO-bound, which finishes the on-GPU kernel in ~67ms.
\end{comment}
%%%%%%%%%%%% OLD TEXT END %%%%%%%%%%%%

% \begin{figure}
%     \centering
%     \includegraphics[width=0.8\linewidth]{figures/join-result.pdf}
%     \caption{Results for Hash Join. (a) the throughput achieved by different solutions, (b) the time breakdown for the \THISWORK\ hash join, and (c) the time taken by on-GPU kernel execution of a typical pipeline stage.}
%     \label{fig:hash-join-perf}
% \end{figure}

\begin{figure*}[t]
\centerline{\includegraphics[width=\linewidth]{figures/ssb-result.pdf}}
\caption{Star Schema Benchmark execution time and speedup.}
\label{fig:ssb-perf}
\end{figure*}
\begin{figure}
    \centering
    \includegraphics[width=0.86\linewidth]{figures/interference.pdf}
    \caption{Interference between \THISWORK\ on the target GPU and the deep learning applications on the forwarding GPUs. 
    % (a) The slowdown for the deep learning applications (x-axis) when the IO traffic (y-axis) runs in the background. 
    % (b) The slowdown for the \THISWORK\ applications (y-axis) when the deep learning applications (x-axis) run in the background.
    }
    \label{fig:interference}
\end{figure}
\noindent
\textbf{Performance of Hash Join.}
In contrast to sorting, hash join remains an IO-bound kernel even with our IO optimization technique. 
As shown in Figure~\ref{fig:sort-perf}(d), our solution achieves a throughput of 2.3 billion tuples per second. 
This is 24.1$\times$ faster than DuckDB, 2.4$\times$ faster than Triton Join (CPU), 1.3$\times$ faster than the CPU-GPU-NVLink-based Triton Join (GPU), and 3.2$\times$ faster than the single GPU solution using a standard PCIe link.
The speedup over the single GPU IO solution is more pronounced because all phases of hash join are IO-bound. 
This is evident in Figure~\ref{fig:sort-perf}(f). 
The \texttt{HashJoinExKer} requires only 34ms to complete the on-GPU join kernel, which is significantly less than the 61ms required for data transfer.
Similarly, it takes 90ms to partition a chunk of data, which is transferred in around 113ms. 
All phases scale uniformly with the improvement of IO throughput, as depicted in the time breakdown in Figure~\ref{fig:sort-perf}(e), where they consume a comparable amount of time. 
Notably, \THISWORK\ outperforms Triton Join without using proprietary CPU-GPU interconnects by exploiting untapped PCIe bandwidth.
% Notably, while surpassing the performance of Triton Join, our solution relies solely on commodity PCIe links, without utilizing any proprietary CPU-GPU connections to enhance IO throughput.


%%%%%%%%%%%% OLD TEXT START %%%%%%%%%%%%
\begin{comment}
Unlike sort, hash join is still an IO-bound kernel even with our IO-redistribution technique.
As shown in Figure~\ref{fig:hash-join-perf}(a), our solution achieves 2.3 billion tuples per second throughput.
This is around 24.1$\times$ over DuckDB, 2.4$\times$ over the CPU implementation of Triton Join, 1.3$\times$ over the CPU-GPU-NVlink based GPU Triton Join~\cite{triton-join}, and 3.2$\times$ over the single GPU solution with a common PCIe link.
The speedup over the single GPU IO solution is more significant because all hash join phases are IO-bound.
This can be observed from Figure~\ref{fig:hash-join-perf}(c).
The \texttt{HashJoinExOp} takes only ~34ms to finish the on-GPU join kernel, which is much lower than the ~61ms data transfer time.
Similarly, it only takes ~90ms to partition a chunk of data transferred in ~113ms.
All phases scale uniformly with the improvement of IO throughput, thus the time breakdown in Figure~\ref{fig:hash-join-perf}(b) shows that they take a similar amount of time.
While achieving better results than Triton Join, we do not use any proprietary CPU-GPU links to improve the IO through, but solely based on commodity PCIe links.
\end{comment}
%%%%%%%%%%%% OLD TEXT END %%%%%%%%%%%%

\noindent
\textbf{Performance of SSB queries.}
Figure~\ref{fig:ssb-perf} illustrates the comparison of SSB query performance between our solution and the baseline approaches.
On average, our solution achieves a 3.4$\times$ speedup over DuckDB, with all data dynamically fetched from CPU DRAM.
When examining individual query flights, the speedup is 2.4$\times$ for Q1.*, 3.6$\times$ for Q2.*, 3.9$\times$ for Q3.*, and 3.7$\times$ for Q4.*. 
The higher speedup observed in Q2.*, Q3.*, and Q4.* is attributed to their inclusion of more complex multi-way joins.
The more complex multi-way join demands higher memory throughput for hash table probing, thus favoring GPU-based solutions more as they can operate in high-bandwidth GPU memory.
The CPU-based solution has to use the limited DRAM bandwidth on hash table probing and fact table reading, while our solution only uses DRAM bandwidth for the latter.
Lightweight queries like Q11 only filter the fact table based on some predicates, whose only DRAM traffic is reading the fact table once.
Thus, the benefit of high-bandwidth GPU memory is minimized, and we observe less speedup. 


By comparing the bars of \texttt{navie} and \texttt{Proteus-GPU} with \texttt{DuckDB}, it becomes evident that GPU-based solutions struggle to achieve performance comparable to the CPU-based DuckDB without utilizing our IO optimization technique. 
However, this technique alone is insufficient, as indicated by the comparison between the \THISWORK\ and \texttt{DuckDB} bars. 
It only achieves a 1.6$\times$ speedup against \texttt{DuckDB} because it transfers unused data to the GPU without considering column selectivity. 
While zero-copy can exploit selectivity, it falls short of maximizing throughput because it relies on a single PCIe link. 
Notably, using zero-copy alone results in worse performance than \THISWORK\ .
Our final solution dynamically switches between SDMA-based data transfer for columns with selectivity greater than a threshold \(TH\) and zero-copy data transfer for columns with selectivity less than \(TH\).
Our solution also achieves 5.7$\times$ speedup over \texttt{Proteus-Hybrid}, despite that it uses both CPU and GPU.
It is difficult for such a hybrid solution to divide work between CPU and GPU and efficiently utilize the CPU DRAM bandwidth.
Our solution achieves 6.2$\times$ speedup over \texttt{Proteus-Lazy}, which enhances \texttt{Proteus-GPU} with late materialization techniques.
After we resolve the IO bottleneck and fully utilize CPU-side DRAM, a pure GPU-based solution can achieve highly competitive results.


%%%%%%%%%%%% OLD TEXT START %%%%%%%%%%%%
\begin{comment}
Figure~\ref{fig:ssb-perf} shows the comparison of SSB query performance between our solution and the baselines.
On average, our solution achieves 3.4$\times$ speedup over DuckDB, with all the data fetched from CPU DRAM on the fly.
Broken down into each query flight, the speedup is 2.4$\times$ for Q1.*, 3.6$\times$ for Q2.*, 3.9$\times$ for Q3.* and 3.7$\times$ for Q4.*.
More speedup is observed in Q2.*, Q3.*, and Q4.* because they include more complicated multi-way joins.

By comparing the bars of \texttt{navie} and \texttt{Proteus-GPU} with \texttt{DuckDB}, note that GPU-based solutions fail to achieve comparable performance to CPU-based DuckDB without using our IO redistribution technique.
However, this technique only is not enough, as we can see by comparing the bars of \texttt{GPU-IO} with \texttt{DuckDB}. 
It only achieves a 1.6$\times$ speedup against DuckDB, as it transfers unused data to GPU ignoring columns' selectivity.
While we can use zero-copy to exploit selectivity, it fails short in maximum throughput because it only uses one PCIe link.
We can notice that using zero-copy alone only delivers worse performance than \texttt{GPU-IO}.
Our final solution switches between SDMA-based data transfer for columns with selectivity larger than a threshold $TH$ and zero-copy data transfer for columns with selectivity lower than $TH$.
\end{comment}
%%%%%%%%%%%% OLD TEXT END %%%%%%%%%%%%

% \begin{figure}
%     \centering
%     \includegraphics[width=\linewidth]{figures/zero-copy-vs-gpu-io.pdf}
%     \caption{Zero copy vs GPU IO}
%     \label{fig:selectivity-perf}
% \end{figure}

% In our study, we set the threshold \(TH = 64\) based on the formula outlined in \S\ref{sec:design-ssb}. 
% To ensure the accuracy and effectiveness of this threshold, we developed the following micro-benchmark specifically designed for validation purposes.
% \begin{verbatim}
% for i in range(16e9)
%   sum += pred[i % 2e9] % SEL == 0 ? v[i] : 0
% \end{verbatim}
% The \texttt{pred} array resides in GPU memory, and \texttt{SEL} is a hyperparameter that is inversely related to selectivity. 
% We implement this micro-benchmark using both GPU-IO and zero-copy data transfer techniques, varying \texttt{SEL} from 1 to 128. 
% The results are presented in Figure~\ref{fig:selectivity-perf}. 
% Notably, when \texttt{SEL} \(> 64\), zero-copy becomes more efficient. 
% This aligns with the threshold \(TH < \frac{1}{64}\), corroborating the results derived from our formula.

%%%%%%%%%%%% OLD TEXT START %%%%%%%%%%%%
\begin{comment}
The \texttt{pred} array on GPU memory, and \texttt{SEL} is a hyperparameter that is inverse to the selectivity.
We implement this micro-benchmark using both GPU-IO and zero-copy data transfer and varies \texttt{SEL} from 1 to 128.
The result is presented in Figure~\ref{fig:selectivity-perf}.
We notice when \texttt{SEL} $>64$ zero-copy becomes more efficient. 
This corresponds to $TH < \frac{1}{64}$ and matches the result from our formula.
\end{comment}
%%%%%%%%%%%% OLD TEXT END %%%%%%%%%%%%

\subsection{Interference Analysis}
\label{sec:interference}
\noindent
While \THISWORK\ utilizes additional GPUs and their IO resources to forward data to a target GPU, running AI workloads on these auxiliary GPUs can lead to a slowdown of these workloads.
Figure~\ref{fig:interference}(a) presents the slowdown for the AI applications (x-axis) when the IO traffic (y-axis) runs in the background, and (b) shows the slowdown for the \THISWORK\ applications (y-axis) when the deep learning applications (x-axis) run in the background.
(1) Compared to single-direction IO traffic, bidirectional IO traffic has a more significant impact on the performance of foreground applications. This is likely due to the increased stress placed on the memory subsystems of the forwarding GPUs.
(2) Memory-intensive workloads are more susceptible to interference from data forwarding activities, as their performance is constrained by the memory bandwidth available on the GPUs. 
Background data forwarding consumes a portion of the memory bandwidth, leading to an average slowdown of 6.8\%.
Compared to SD3, text embedding generation, and LLM prefilling, LLM decoding experiences a greater degree of slowdown.

Figure~\ref{fig:interference} illustrates that current hardware may not optimize for our IO optimization techniques due to two key observations.
First, although the memory subsystem is theoretically stressed to the same degree in both scenarios, forwarding IO traffic from the device to the host results in a more significant slowdown compared to traffic from the host to the device.
Second, to support the 140GB/s IO throughput we achieved, each GPU incurs an additional memory bandwidth cost of $\frac{140 \times 2}{4} = 70$GB/s, which constitutes only $\frac{70}{1200} \approx 5.8\%$ of the MI100's total bandwidth.
However, empirical observations reveal slowdowns of 7.2\%, 13.4\%, and 16.9\% for \texttt{SD3}, \texttt{Llama3} decoding with a batch size of 32, and \texttt{Llama3} decoding with a batch size of 1, respectively.
We hypothesize that this discrepancy arises because our programming model generates atypical memory traffic that hinders the GPU memory controller's ability to fully utilize bandwidth for the foreground application.

We analyze the slowdown of data analytics applications caused by DL applications on forwarding GPUs. 
As shown in Figure~\ref{fig:interference}, the target GPU experiences less slowdown, with a maximum of 10.4\%. 
However, the slowdown patterns are more irregular compared to forwarding GPUs. 
Text embedding generation and \texttt{Llama3} prefilling cause more interference than \texttt{SD3}, despite all being compute-bound workloads. 
Interestingly, the memory-bound \texttt{Llama3} decoding shows less interference on the target GPU, contrasting with the significant interference on the forwarding GPUs.

%%%%%%%%%%%% OLD TEXT START %%%%%%%%%%%%
\begin{comment}
Besides, Figure~\ref{fig:interference} also shows current hardware may not be able to handle our novel use cases efficiently.
(1) While stressing the memory subsystem to the same degree theoretically, the forwarding IO traffic from device to host causes a greater slowdown than from host to device.
(2) To support the ~140GB/s IO throughput we achieved, each GPU only needs to pay $\frac{140 \times 2}{4} = 70$ GB/s additional memory bandwidth, which is only $\frac{70}{1200} \approx 5.8\%$ of MI100's total bandwidth.
However, we observe 7.2\%, 13.4\%, and 16.9\% slowdown for \texttt{SD3}, \texttt{Llama3} decoding with batch size 32, and \texttt{Llama3} decoding with batch size 1.
We speculate that this is because our new way of programming generates uncommon memory traffic to the GPU memory controller, and prevents it from fully utilizing the maximum memory bandwidth.

Next, we also analyze the slowdown of data analytics applications influenced by DL applications running on the forwarding GPUs.
Less slowdown is observed on the target GPU as shown in Figure~\ref{fig:interference}, where the maximum slowdown is 10.4\%.
However, the slowdown numbers become more irregular compared to the case of forwarding GPUs.
We observe that text embedding generation and \texttt{Llama3} prefilling cause more interference than \texttt{SD3}, although all of them are compute-bound workloads.
Surprisingly, memory-bound \texttt{Llama3} decoding shows less interference on the target GPU, in contrast to the high degree of interference on the forwarding GPUs.
\end{comment}
%%%%%%%%%%%% OLD TEXT END %%%%%%%%%%%%

% \noindent
% \textbf{How are \THISWORK\ applications influenced by the DL applications on the forwarding GPUs?}


\noindent
\textbf{Overall system efficiency.}
Given that our technique can accelerate heavily IO-bound applications by 3 to 4 times, we argue that the system is still more efficient even with a slowdown of up to 16.9\% on the other GPUs.
The improvement of overall system efficiency in a 4-GPU system can be quantified as shown below.
% We discuss how our technique enhances the overall efficiency of a 4-GPU system, as quantified by the following formula.
% Given that our technique can speed up the heavily IO-bounded applications by 3~4$\times$, up to 16.9\% slowdown on the other three GPUs is acceptable. 
% We discuss how much our technique improves the 4-GPU system's efficiency as a whole.
% The improvement of the whole 4-GPU system's efficiency is given by the following formula
$$
\text{speedup}_\text{sys} = \frac{\text{speedup}_\text{t} * \text{slowdown}_\text{t} + 3 * \text{slowdown}_\text{f}}{4}
$$
The subscripts `t' and `f' denote the target GPU and forwarding GPUs, respectively. 
Consider the scenario where \texttt{SD3} and hash join, both with primarily bidirectional IO traffic, are collocated.
The overall system speedup is $\frac{3.2 * (1 - 0.051) + 3 * (1 - 0.072)}{4} \approx 1.45$.
In our setup, the least favorable combination is running \texttt{Llama3} decoding without batching alongside sort. 
Despite this, we still achieve a modest speedup of$\frac{1.7 * (1 - 0.032) + 3 * (1 - 0.169)}{4} \approx 1.03$ speedup.
Note that these speedup values refer to the entire 4-GPU system. 
For a single GPU, they correspond to speedups of 2.8$\times$ and 1.12$\times$, respectively.

%%%%%%%%%%%% OLD TEXT START %%%%%%%%%%%%
\begin{comment}
where the subscript ``t'' and ``f'' mean the target GPU and the forwarding GPUs.
Consider the case of collocating \texttt{SD3} and hash join, whose IO traffic is mainly bidirectional.
The whose system speedup is $\frac{3.2 * (1 - 0.051) + 3 * (1 - 0.072)}{4} \approx 1.45$.
In our setup, the worst combination is running \texttt{Llama3} decoding without batching with sort, but we still achieve a minor $\frac{1.7 * (1 - 0.032) + 3 * (1 - 0.169)}{4} \approx 1.03$ speedup.
Note that the speedup here is in terms of all 4 GPUs, and the speedup above translates to 2.8$\times$ and 1.12$\times$ in terms of a single GPU.
\end{comment}
%%%%%%%%%%%% OLD TEXT END %%%%%%%%%%%%
\section{Conclusion and future directions} \label{sec:conclusion}

In this paper we proposed a nested MLMC framework that offers important computational savings by performing most calculations in low precision and exploiting approximate random normal variables for the low precision path calculations. The low precision calculations could be performed in fixed precision on an FPGA for greater efficiency, and we suggested a procedure to optimise the bit-widths of every variable at each Monte Carlo level. This is an important improvement over previous mixed precision MLMC frameworks which held the lower precision fixed \cite{Rounding_error_oliver} or defined uniform bit-width at every level heuristically \cite{brugger2014mixed}. Our numerical results suggest that for the first levels our procedure reduces the cost at these levels by a factor 5 or 7. Hence the overall savings are significant since most paths are calculated on the first levels. Our approach would be even more efficient for the Milstein scheme because its higher order strong convergence leads to a greater proportion of the computational costs being on the coarsest levels.

The next stage of the research project will be to implement the RNG methods and the nested framework on FPGAs to determine the hardware requirements and confirm the extent of the computational savings. It would also be good to compare the performance benefits to using half-precision floating point arithmetic on GPUs or CPUs for the low-accuracy computations.




%%%%%%%%% REFERENCES
{
    \small
    \bibliographystyle{IEEEtran}
    \bibliography{refs}
}

\end{document}