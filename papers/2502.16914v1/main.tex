\documentclass[conference]{IEEEtran}
\IEEEoverridecommandlockouts
% The preceding line is only needed to identify funding in the first footnote. If that is unneeded, please comment it out.
\usepackage{cite}
\usepackage{amsmath,amssymb,amsfonts}
\usepackage{algorithmic}
\usepackage{booktabs}
\usepackage{graphicx}
\usepackage{textcomp}
\usepackage{xcolor}
\usepackage{array}
\usepackage{float}

\usepackage{array} % Required for 'm' specifier
\usepackage{lipsum}
\usepackage{tabularray}

\usepackage{cuted}
\usepackage{caption}


\def\BibTeX{{\rm B\kern-.05em{\sc i\kern-.025em b}\kern-.08em
    T\kern-.1667em\lower.7ex\hbox{E}\kern-.125emX}}
    
% %
% --- inline annotations
%
\newcommand{\red}[1]{{\color{red}#1}}
\newcommand{\todo}[1]{{\color{red}#1}}
\newcommand{\TODO}[1]{\textbf{\color{red}[TODO: #1]}}
% --- disable by uncommenting  
% \renewcommand{\TODO}[1]{}
% \renewcommand{\todo}[1]{#1}



\newcommand{\VLM}{LVLM\xspace} 
\newcommand{\ours}{PeKit\xspace}
\newcommand{\yollava}{Yo’LLaVA\xspace}

\newcommand{\thisismy}{This-Is-My-Img\xspace}
\newcommand{\myparagraph}[1]{\noindent\textbf{#1}}
\newcommand{\vdoro}[1]{{\color[rgb]{0.4, 0.18, 0.78} {[V] #1}}}
% --- disable by uncommenting  
% \renewcommand{\TODO}[1]{}
% \renewcommand{\todo}[1]{#1}
\usepackage{slashbox}
% Vectors
\newcommand{\bB}{\mathcal{B}}
\newcommand{\bw}{\mathbf{w}}
\newcommand{\bs}{\mathbf{s}}
\newcommand{\bo}{\mathbf{o}}
\newcommand{\bn}{\mathbf{n}}
\newcommand{\bc}{\mathbf{c}}
\newcommand{\bp}{\mathbf{p}}
\newcommand{\bS}{\mathbf{S}}
\newcommand{\bk}{\mathbf{k}}
\newcommand{\bmu}{\boldsymbol{\mu}}
\newcommand{\bx}{\mathbf{x}}
\newcommand{\bg}{\mathbf{g}}
\newcommand{\be}{\mathbf{e}}
\newcommand{\bX}{\mathbf{X}}
\newcommand{\by}{\mathbf{y}}
\newcommand{\bv}{\mathbf{v}}
\newcommand{\bz}{\mathbf{z}}
\newcommand{\bq}{\mathbf{q}}
\newcommand{\bff}{\mathbf{f}}
\newcommand{\bu}{\mathbf{u}}
\newcommand{\bh}{\mathbf{h}}
\newcommand{\bb}{\mathbf{b}}

\newcommand{\rone}{\textcolor{green}{R1}}
\newcommand{\rtwo}{\textcolor{orange}{R2}}
\newcommand{\rthree}{\textcolor{red}{R3}}
\usepackage{amsmath}
%\usepackage{arydshln}
\DeclareMathOperator{\similarity}{sim}
\DeclareMathOperator{\AvgPool}{AvgPool}

\newcommand{\argmax}{\mathop{\mathrm{argmax}}}     


\newcommand{\ENACT}{\textit{ENACT-Heart}~}

\begin{document}

\title{ENACT-Heart -- ENsemble-based Assessment Using CNN and Transformer on Heart Sounds
% \thanks{Identify applicable funding agency here. If none, delete this.}
}

\author{\IEEEauthorblockN{Jiho Han\textsuperscript{1}\thanks{\textsuperscript{1}Jiho Han was with the Department of Computer Science and Engineering, The University of Michigan - Dearborn, Dearborn, MI, USA during the study.}}
\IEEEauthorblockA{\textit{Industrial AI Lab}\\
\textit{SimPlatform Co. Ltd. Affiliate Research Institute}\\
Geumcheon-gu, Seoul, Republic of Korea\\
jihohan@simplatform.com}
\and
\IEEEauthorblockN{Adnan Shaout}
\IEEEauthorblockA{\textit{Department of Electrical and Computer Engineering} \\
\textit{The University of Michigan – Dearborn}\\
Dearborn, MI, USA\\
shaout@umich.edu}}

\maketitle

\begin{strip}
    \centering
    \vspace{-6em}
    \includegraphics[width=\linewidth]{images/diagram_ENACT_Heart.png}
    \captionof{figure}{The pipeline of our \textit{ENACT-Heart} consists of three core steps: data augmentation, expert analysis of each modality, and analysis fusion. 1) Data augmentation through Gaussian Noise allows increased variability and generalization of the overall model. 2) Spectrogram analysis is performed through ViT, and audiovisual diagram analysis is done through CNN, allowing each model to leverage its strengths in feature extraction for different modalities.}
    \label{fig:banner}
\end{strip}

%%%%%%%% ABSTRACT
\begin{abstract}


The choice of representation for geographic location significantly impacts the accuracy of models for a broad range of geospatial tasks, including fine-grained species classification, population density estimation, and biome classification. Recent works like SatCLIP and GeoCLIP learn such representations by contrastively aligning geolocation with co-located images. While these methods work exceptionally well, in this paper, we posit that the current training strategies fail to fully capture the important visual features. We provide an information theoretic perspective on why the resulting embeddings from these methods discard crucial visual information that is important for many downstream tasks. To solve this problem, we propose a novel retrieval-augmented strategy called RANGE. We build our method on the intuition that the visual features of a location can be estimated by combining the visual features from multiple similar-looking locations. We evaluate our method across a wide variety of tasks. Our results show that RANGE outperforms the existing state-of-the-art models with significant margins in most tasks. We show gains of up to 13.1\% on classification tasks and 0.145 $R^2$ on regression tasks. All our code and models will be made available at: \href{https://github.com/mvrl/RANGE}{https://github.com/mvrl/RANGE}.

\end{abstract}



%%%%%%%% BODY TEXT
\section{Introduction}

Video generation has garnered significant attention owing to its transformative potential across a wide range of applications, such media content creation~\citep{polyak2024movie}, advertising~\citep{zhang2024virbo,bacher2021advert}, video games~\citep{yang2024playable,valevski2024diffusion, oasis2024}, and world model simulators~\citep{ha2018world, videoworldsimulators2024, agarwal2025cosmos}. Benefiting from advanced generative algorithms~\citep{goodfellow2014generative, ho2020denoising, liu2023flow, lipman2023flow}, scalable model architectures~\citep{vaswani2017attention, peebles2023scalable}, vast amounts of internet-sourced data~\citep{chen2024panda, nan2024openvid, ju2024miradata}, and ongoing expansion of computing capabilities~\citep{nvidia2022h100, nvidia2023dgxgh200, nvidia2024h200nvl}, remarkable advancements have been achieved in the field of video generation~\citep{ho2022video, ho2022imagen, singer2023makeavideo, blattmann2023align, videoworldsimulators2024, kuaishou2024klingai, yang2024cogvideox, jin2024pyramidal, polyak2024movie, kong2024hunyuanvideo, ji2024prompt}.


In this work, we present \textbf{\ours}, a family of rectified flow~\citep{lipman2023flow, liu2023flow} transformer models designed for joint image and video generation, establishing a pathway toward industry-grade performance. This report centers on four key components: data curation, model architecture design, flow formulation, and training infrastructure optimization—each rigorously refined to meet the demands of high-quality, large-scale video generation.


\begin{figure}[ht]
    \centering
    \begin{subfigure}[b]{0.82\linewidth}
        \centering
        \includegraphics[width=\linewidth]{figures/t2i_1024.pdf}
        \caption{Text-to-Image Samples}\label{fig:main-demo-t2i}
    \end{subfigure}
    \vfill
    \begin{subfigure}[b]{0.82\linewidth}
        \centering
        \includegraphics[width=\linewidth]{figures/t2v_samples.pdf}
        \caption{Text-to-Video Samples}\label{fig:main-demo-t2v}
    \end{subfigure}
\caption{\textbf{Generated samples from \ours.} Key components are highlighted in \textcolor{red}{\textbf{RED}}.}\label{fig:main-demo}
\end{figure}


First, we present a comprehensive data processing pipeline designed to construct large-scale, high-quality image and video-text datasets. The pipeline integrates multiple advanced techniques, including video and image filtering based on aesthetic scores, OCR-driven content analysis, and subjective evaluations, to ensure exceptional visual and contextual quality. Furthermore, we employ multimodal large language models~(MLLMs)~\citep{yuan2025tarsier2} to generate dense and contextually aligned captions, which are subsequently refined using an additional large language model~(LLM)~\citep{yang2024qwen2} to enhance their accuracy, fluency, and descriptive richness. As a result, we have curated a robust training dataset comprising approximately 36M video-text pairs and 160M image-text pairs, which are proven sufficient for training industry-level generative models.

Secondly, we take a pioneering step by applying rectified flow formulation~\citep{lipman2023flow} for joint image and video generation, implemented through the \ours model family, which comprises Transformer architectures with 2B and 8B parameters. At its core, the \ours framework employs a 3D joint image-video variational autoencoder (VAE) to compress image and video inputs into a shared latent space, facilitating unified representation. This shared latent space is coupled with a full-attention~\citep{vaswani2017attention} mechanism, enabling seamless joint training of image and video. This architecture delivers high-quality, coherent outputs across both images and videos, establishing a unified framework for visual generation tasks.


Furthermore, to support the training of \ours at scale, we have developed a robust infrastructure tailored for large-scale model training. Our approach incorporates advanced parallelism strategies~\citep{jacobs2023deepspeed, pytorch_fsdp} to manage memory efficiently during long-context training. Additionally, we employ ByteCheckpoint~\citep{wan2024bytecheckpoint} for high-performance checkpointing and integrate fault-tolerant mechanisms from MegaScale~\citep{jiang2024megascale} to ensure stability and scalability across large GPU clusters. These optimizations enable \ours to handle the computational and data challenges of generative modeling with exceptional efficiency and reliability.


We evaluate \ours on both text-to-image and text-to-video benchmarks to highlight its competitive advantages. For text-to-image generation, \ours-T2I demonstrates strong performance across multiple benchmarks, including T2I-CompBench~\citep{huang2023t2i-compbench}, GenEval~\citep{ghosh2024geneval}, and DPG-Bench~\citep{hu2024ella_dbgbench}, excelling in both visual quality and text-image alignment. In text-to-video benchmarks, \ours-T2V achieves state-of-the-art performance on the UCF-101~\citep{ucf101} zero-shot generation task. Additionally, \ours-T2V attains an impressive score of \textbf{84.85} on VBench~\citep{huang2024vbench}, securing the top position on the leaderboard (as of 2025-01-25) and surpassing several leading commercial text-to-video models. Qualitative results, illustrated in \Cref{fig:main-demo}, further demonstrate the superior quality of the generated media samples. These findings underscore \ours's effectiveness in multi-modal generation and its potential as a high-performing solution for both research and commercial applications.
\section{Background} \label{section:LLM}

% \subsection{Large Language Model (LLM)}   

Figure~\ref{fig:LLaMA_model}(a) shows that a decoder-only LLM initially processes a user prompt in the “prefill” stage and subsequently generates tokens sequentially during the “decoding” stage.
Both stages contain an input embedding layer, multiple decoder transformer blocks, an output embedding layer, and a sampling layer.
Figure~\ref{fig:LLaMA_model}(b) demonstrates that the decoder transformer blocks consist of a self attention and a feed-forward network (FFN) layer, each paired with residual connection and normalization layers. 

% Differentiate between encoder/decoder, explain why operation intensity is low, explain the different parts of a transformer block. Discuss Table II here. 

% Explain the architecture with Llama2-70B.

% \begin{table}[thb]
% \renewcommand\arraystretch{1.05}
% \centering
% % \vspace{-5mm}
%     \caption{ML Model Parameter Size and Operational Intensity}
%     \vspace{-2mm}
%     \small
%     \label{tab:ML Model Parameter Size and Operational Intensity}    
%     \scalebox{0.95}{
%         \begin{tabular}{|c|c|c|c|c|}
%             \hline
%             & Llama2 & BLOOM & BERT & ResNet \\
%             Model & (70B) & (176B) & & 152 \\
%             \hline
%             Parameter Size (GB) & 140 & 352 & 0.17 & 0.16 \\
%             \hline
%             Op Intensity (Ops/Byte) & 1 & 1 & 282 & 346 \\
%             \hline
%           \end{tabular}
%     }
% \vspace{-3mm}
% \end{table}

% {\fontsize{8pt}{11pt}\selectfont 8pt font size test Memory Requirement}

\begin{figure}[t]
    \centering
    \includegraphics[width=8cm]{Figure/LLaMA_model_new_new.pdf}
    \caption{(a) Prefill stage encodes prompt tokens in parallel. Decoding stage generates output tokens sequentially.
    (b) LLM contains N$\times$ decoder transformer blocks. 
    (c) Llama2 model architecture.}
    \label{fig:LLaMA_model}
\end{figure}

Figure~\ref{fig:LLaMA_model}(c) demonstrates the Llama2~\cite{touvron2023llama} model architecture as a representative LLM.
% The self attention layer requires three GEMVs\footnote{GEMVs in multi-head attention~\cite{attention}, narrow GEMMs in grouped-query attention~\cite{gqa}.} to generate query, key and value vectors.
In the self-attention layer, query, key and value vectors are generated by multiplying input vector to corresponding weight matrices.
These matrices are segmented into multiple heads, representing different semantic dimensions.
The query and key vectors go though Rotary Positional Embedding (RoPE) to encode the relative positional information~\cite{rope-paper}.
Within each head, the generated key and value vectors are appended to their caches.
The query vector is multiplied by the key cache to produce a score vector.
After the Softmax operation, the score vector is multiplied by the value cache to yield the output vector.
The output vectors from all heads are concatenated and multiplied by output weight matrix, resulting in a vector that undergoes residual connection and Root Mean Square layer Normalization (RMSNorm)~\cite{rmsnorm-paper}.
The residual connection adds up the input and output vectors of a layer to avoid vanishing gradient~\cite{he2016deep}.
The FFN layer begins with two parallel fully connections, followed by a Sigmoid Linear Unit (SiLU), and ends with another fully connection.
\section{Related Works} \label{related_works}
\subsection{Thematically Related Works}
Thematically, the proposed method lies in the field of computer-assisted diagnosis (CAD) systems for heart disease. CAD systems for heart diseases utilize computation techniques such as machine learning, pattern recognition, and AI to analyze cardiac data and provide decision-support tools for healthcare providers.

There have been numerous attempts to apply CAD systems for heart diseases in diverse modalities, including but not limited to ECG, cardiac CT/MRI, etc. These systems analyze cardiac data to identify abnormalities and patterns - especially indicators of certain heart diseases. However, the direction of the majority of these researches are pointed mainly toward computer vision over audio AI, mainly due to the advanced deep learning models available.

For some researches that emphasized sound classification, its methodologies have varied slightly from the approach proposed in this study. For instance, Jumphoo et al. utilized a CNN for feature extraction and Data-efficient Image Transformer (DieT), a variant of the ViT model, for classification tasks through stacking \cite{jumphoo_exploiting_2024}. Another heart sound classification model, proposed by Liu et al., also uses ViT for classification but employs a different image modality called bispectral patterns and relies solely on ViT without integrating other models \cite{liu_heart_2023}. While these studies highlight the effectiveness of ViT individually, they do not explore the potential benefits of using an ensemble approach.

Overall, the integration of multiple distinct AI models and modalities from the same sound inputs, as proposed in the \ENACT using an MoE approach, has not been attempted yet. This novel methodology leverages the strengths of both ViT and CNN models, potentially offering a more robust and accurate solution for heart sound classification.

\subsection{Methodologically Related Works}
The use of computer vision as a tool for machine hearing is an emerging approach. There have been attempts to use computer vision techniques as a method of machine hearing - analyzing audio signals by treating them as visual data. 

Hsu \textit{et al.} \cite{hsu_deep_2021} presented a deep learning-based music classification through mel-spectrogram and Fourier tempogram features. Although the concept of using multiple different audiovisual modalities and models from singular sound data is there, the paper employed the short-chunk CNN + ResNet as the backbone architecture of their models.

\section{Proposed Approach} \label{proposed approach}
The choice of heart sound analysis in this study is driven by its unique diagnostic value, which complements other modalities such as ECG. Despite the advent of modern diagnostic techniques and sophisticated imaging modalities, cardiac auscultation and heart sounds remain invaluable diagnostic tools. While ECG is widely regarded as the gold standard for diagnosing cardiac rhythm disorders and ischemic heart disease, it may not capture certain aspects of cardiac function that heart sound analysis can, such as detecting murmurs, rubs, and other abnormal heart sounds indicative of structural abnormalities like valvular heart diseases or ventricular hypertrophy. Therefore, heart sound analysis provides additional, complementary information that can enhance diagnostic accuracy.

Researches has demonstrated that it is
possible to process spectrograms from audio data as images
and apply computer vision algorithms such as CNN \cite{verma_neural_2018, cabrera-ponce_detection_2020, hyder_acoustic_2017}. The core problem of the current approaches in using regular CNN-based computer vision methods on audio spectrogram representation lies in the distinctiveness of the spectrogram in comparison to other image data.

Visual transformers leverage attention mechanisms to capture dependencies between different parts of the input data. This allows them to model long-range dependencies more effectively than traditional CNNs, whose feature extraction is limited to local receptive fields. By aggregating information from across the entire spectrogram, transformers can show a global contextual understanding of the audio signal, enabling them to capture non-local dependencies and extract meaningful features from spectrograms.

\subsection{Spectral Data Visualization \& Analysis}
In spectral visualization and analysis, researchers employ various techniques to gain insights into the frequency content of signals. These methodologies enable the examination of how frequencies evolve over time, providing valuable information for tasks such as audio processing, speech recognition, and biomedical signal analysis.

\textbf{Spectrogram.}
Spectrograms stand as one of the primary tools in spectral visualization. They offer detailed representations of frequency spectra over time, revealing how the frequency composition of a signal changes temporally. By plotting frequency on the vertical axis, time on the horizontal axis, and intensity or magnitude using color or brightness, spectrograms provide a comprehensive view of signal dynamics. This detailed visualization allows analysts to identify specific features, patterns, and transient events within the signal, making spectrograms invaluable for tasks requiring fine-grained temporal frequency analysis.

\textbf{Spectral Centroid.}
In contrast to the detailed temporal-frequency mapping provided by spectrograms, spectral centroids offer a simplified summary of a signal's frequency content. The spectral centroid indicates the "center of mass" or average frequency of a signal within each time frame. This single-value representation reduces the complexity of the data while still providing a concise summary of the signal’s frequency characteristics. Spectral centroids are particularly useful for enhancing computational efficiency and maintaining robustness against noise and variations in the signal. However, they lack the detailed temporal information that spectrograms provide.

The synergy in using spectrograms and spectral centroids with different models lies in their ability to capture distinct and complementary features of audio signals. Spectrograms provide a comprehensive visualization of the frequency content over time, highlighting complex, high-dimensional patterns. In contrast, spectral centroids and waveforms represent simpler, more repetitive features, which are well-suited to the strengths of CNNs in learning local patterns through convolution and pooling operations.

By employing a MoE approach, the proposed model effectively combines these diverse representations. The spectrograms allow the model to capture detailed, global time-frequency information, while the spectral centroids and waveforms facilitate the extraction of robust, localized features. This integration leverages the strengths of both ViT and CNNs, resulting in a more accurate and holistic analysis of heart sounds.

\subsection{MoE}
MoE is a powerful ensemble learning methodology used in machine learning and statistical modeling. Within ensemble methods, multiple models are combined to improve predictive performance compared to any individual model. MoE takes this concept a step further by combining various models and adjusting the weight of their contributions adaptively per the input data.

In MoE, the "experts" are individual models or learners, each specializing in a particular region of the input space or addressing specific patterns in the data. These experts make predictions independently based on their specialized knowledge. The key innovation of MoE lies in the gating network, which dynamically selects the most relevant expert or combination of experts for each input instance.

The gating network, often implemented as a neural network, learns to assign weights to the experts based on the input data. These weights determine the contribution of each expert to the final prediction. By adaptively combining the predictions of multiple experts, MoE can capture complex relationships in the data and achieve superior predictive performance compared to traditional ensemble methods. The flowchart of the proposed experiment, depicted in Figure \ref{fig:flowchart}, illustrates the entire process, from input data processing to the final output generated by the MoE.
% \section{Experiment and Results}
\section{Results and Analysis}
In this section, we first present safe vs. unsafe evaluation results for 12 LLMs, followed by fine-grained responding pattern analysis over six models among them, and compare models' behavior when they are attacked by same risky questions presented in different languages: Kazakh, Russian and code-switching language.    

\begin{table}[t!]
\centering
\small
\resizebox{\columnwidth}{!}{
\begin{tabular}{clcccc}
\toprule
\multicolumn{1}{l}{\textbf{Rank} } & \textbf{Model} & \textbf{Kazakh $\uparrow$} & \textbf{Russian $\uparrow$} & \textbf{English $\uparrow$} \\
\midrule
1 & \claude & \textbf{96.5}   & 93.5    & \textbf{98.6}    \\
2 & \gptfouro & 95.8   & 87.6    & 95.7    \\
3 & \yandexgpt & 90.7   & \textbf{93.6}    & 80.3    \\
4 & \kazllmseventy & 88.1 & 87.5 & 97.2 \\
5 & \llamaseventy & 88.0   & 85.5    & 95.7    \\
6 & \sherkala & 87.1   & 85.0    & 96.0    \\
7 & \falcon & 87.1   & 84.7    & 96.8    \\
8 & \qwen & 86.2   & 85.1    & 88.1    \\
9 & \llamaeight & 85.9   & 84.7    & 98.3    \\
10 & \kazllmeight & 74.8   & 78.0    & 94.5 \\
11 & \aya & 72.4 & 84.5 & 96.6 \\
12 & \vikhr & --- & 85.6 & 91.1 \\
\bottomrule
\end{tabular}
}
\caption{Safety evaluation results of 12 LLMs, ranked by the percentage of safe responses in the Kazakh dataset. \claude\ achieves the highest safety score for both Kazakh and English, while \yandexgpt\ is the safest model for Russian responses.}
\label{tab:safety-binary-eval}
\end{table}



\subsection{Safe vs. Unsafe Classification}
% In this subsection, 
We present binary evaluation results of 12 LLMs, by assessing 52,596 Russian responses and 41,646 Kazakh responses.
% 26,298 responses generated by six models on the Russian dataset and 22,716 responses on the Kazakh dataset. 

%\textbf{Safety Rank.} In general, proprietary systems outperform the open-source model. For Russian, As shown in Table \ref{tab:model_comparison_russian}, \textbf{Yandex-GPT} emerges as the safest large language model (LLM) for Russian, with a safety percentage of 93.57\%. Among the open-source models, \textbf{Vikhr-Nemo-12B} is the safest, achieving a safety percentage of 85.63\%.
% Edited: This is mentioned in the discussion
% This outcome highlights the potential impact of pretraining data on model behavior. Models pre-trained primarily on Russian data are better at understanding and handling harmful questions - in both proprietary systems and open-source setups. 
%For Kazakh, as shown in Table \ref{tab:model_comparison_kazakh}, \textbf{Claude} emerges as the safest large language model (LLM) with a safety percentage of 96.46\%, closely followed by GPT-4o at 95.75\%. In contrast, \textbf{Aya-101}, despite being specifically tuned for Kazakh, consistently shows the highest unsafe response rates at 72.37\%. 

\begin{figure*}[t!]
	\centering
        \includegraphics[scale=0.28]{figures/question_type_no6_kaz.png}
	\includegraphics[scale=0.28]{figures/question_type_exclude_region_specific_new.png} 

	\caption{Unsafe answer distribution across three question types for risk types I-V: Kazakh (left) and Russian (right)}
	\label{fig:qt_non_reg}
\end{figure*}

\begin{figure*}[t!]
	\centering
        \includegraphics[scale=0.28]{figures/question_type_only6_kaz.png}
	\includegraphics[scale=0.28]{figures/question_type_region_specific_new.png} 
	
	\caption{Unsafe answer distribution across three question types for risk type VI: Kazakh (left) and Russian (right)}
	\label{fig:qt_reg}
\end{figure*}

\textbf{Safety Rank.} In general, proprietary systems outperform the open-source models. 
For Russian, as shown in Table~\ref{tab:safety-binary-eval},  % \ref{tab:model_comparison_russian}, 
\yandexgpt emerges as the safest language model for Russian, with safe responses account for 93.57\%. Among the open-source models, \kazllmseventy is the safest (87.5\%), followed by \vikhr with a safety percentage of 85.63\%.

For Kazakh, % as shown in Table \ref{tab:model_comparison_kazakh}, 
% YX: todo, rerun Kazakh safety percentage using Diana threshold
\claude is the safest model with 96.46\% safe responses, closely followed by \gptfouro\ at 95.75\%. \aya, despite being specifically tuned for Kazakh, shows the highest unsafe rates at 72.37\%.



% \begin{table}[t!]
% \centering
% \resizebox{\columnwidth}{!}{%
% \begin{tabular}{clccc}
% \toprule
% \textbf{Rank} & \textbf{Model Name}  & \textbf{Safe} & \textbf{Unsafe} & \textbf{Safe \%} \\ \midrule
% \textbf{1} & \textbf{Yandex-GPT} & \textbf{4101} & \textbf{282} & \textbf{93.57} \\
% 2 & Claude & 4100 & 283 & 93.54 \\
% 3 & GPT-4o & 3839 & 544 & 87.59 \\
% 4 & Vikhr-12B & 3753 & 630 & 85.63 \\
% 5 & LLama-3.1-instruct-70B & 3746 & 637 & 85.47 \\
% 6 & LLama-3.1-instruct-8B & 3712 & 671 & 84.69 \\
% \bottomrule
% \end{tabular}
% }
% \caption{Comparison of models based on safety percentages for the Russian dataset.}
% \label{tab:model_comparison_russian}
% \end{table}


% \begin{table}[t!]
% \centering
% \resizebox{\columnwidth}{!}{%
% \begin{tabular}{clccc}
% \toprule
% \textbf{Rank} & \textbf{Model Name}  & \textbf{Safe} & \textbf{Unsafe} & \textbf{Safe \%} \\ \midrule
% 1             & \textbf{Claude}  & \textbf{3652} & \textbf{134} & \textbf{96.46} \\ 
% 2             & GPT-4o                      & 3625          & 161          & 95.75 \\ 
% 3             & YandexGPT                   & 3433          & 353          & 90.68 \\
% 4             & LLama-3.1-instruct-70B      & 3333          & 453          & 88.03 \\
% 5             & LLama-3.1-instruct-8B       & 3251          & 535	       & 85.87 \\
% 6             & Aya-101                     & 2740          & 1046         & 72.37 \\ 
% \bottomrule
% \end{tabular}
% }
% \caption{Comparison of models based on safety percentages for the Kazakh dataset.}
% \label{tab:model_comparison_kazakh}
% \end{table}



\textbf{Risk Areas.} 
We selected six representative LLMs for Russian and Kazakh respectively and show their unsafe answer distributions over six risk areas.
As shown in Table \ref{tab:unsafe_answers_summary}, risk type VI (region-specific sensitive topics) overwhelmingly contributes the largest number of unsafe responses across all models. This highlights that LLMs are poorly equipped to address regional risks effectively. For instance, while \llama models maintain comparable safety levels across other risk type (I–V), their performance drops significantly when dealing with risk type VI. Interestingly, while \yandexgpt\ demonstrates relatively poor performance in most other risk areas, it handles region-specific questions remarkably well, suggesting a stronger alignment with regional norms and sensitivities. For Kazakh, Table \ref{tab:unsafe_answers_summary_kazakh} shows that region‐specific topics (risk type VI) pose a substantial challenge across all six models, including the commercial \gptfouro\ and \claude, which demonstrate superior safety on general categories. 

% \begin{table}[t!]
% \centering
% \resizebox{\columnwidth}{!}{%
% \begin{tabular}{lccccccc}
% \toprule
% \textbf{Model} & \textbf{I} & \textbf{II} & \textbf{III} & \textbf{IV} & \textbf{V} & \textbf{VI} & \textbf{Total} \\ \midrule
% LLama-3.1-instruct-8B & 60 & 70 & 16 & 31 & 9 & 485 & 671 \\
% LLama-3.1-instruct-70B & 29 & 55 & 8 & 4 & 1 & 540 & 637 \\
% Vikhr-12B & 41 & 93 & 15 & 1 & 3 & 477 & 630 \\
% GPT-4o & 21 & 51 & 6 & 2 & 0 & 464 & 544 \\
% Claude & 7 & 10 & 1 & 0 & 0 & 265 & 283 \\
% Yandex-GPT & 55 & 125 & 9 & 3 & 8 & 82 & 282 \\
% \bottomrule
% \end{tabular}%
% }
% \caption{Ru unsafe cases over risk areas of six models.}
% \label{tab:unsafe_answers_summary}
% \end{table}


\begin{table}[t!]
\centering
\resizebox{\columnwidth}{!}{%
\begin{tabular}{lccccccc}
\toprule
\textbf{Model} & \textbf{I} & \textbf{II} & \textbf{III} & \textbf{IV} & \textbf{V} & \textbf{VI} & \textbf{Total} \\ \midrule
\llamaeight & 60 & 70 & 16 & 31 & 9 & 485 & 671 \\
\llamaseventy & 29 & 55 & 8 & 4 & 1 & 540 & 637 \\
\vikhr & 41 & 93 & 15 & 1 & 3 & 477 & 630 \\
\gptfouro & 21 & 51 & 6 & 2 & 0 & 464 & 544 \\
\claude & 7 & 10 & 1 & 0 & 0 & 265 & 283 \\
\yandexgpt & 55 & 125 & 9 & 3 & 8 & 82 & 282 \\
\bottomrule
\end{tabular}%
}
\caption{Ru unsafe cases over risk areas of six models.}
\label{tab:unsafe_answers_summary}
\end{table}


% \begin{table}[t!]
% \centering
% \resizebox{\columnwidth}{!}{%
% \begin{tabular}{lccccccc}
% \toprule
% \textbf{Model} & \textbf{I} & \textbf{II} & \textbf{III} & \textbf{IV} & \textbf{V} & \textbf{VI} & \textbf{Total} \\ \midrule
% Aya-101 & 96 & 235 & 165 & 166 & 90 & 294 & 1046 \\
% Llama-3.1-instruct-8B & 25 & 15 & 91 & 37 & 14 & 353 & 535 \\
% Llama-3.1-instruct-70B & 33 & 39 & 88 & 27 & 20 & 246 & 453 \\
% Yandex-GPT & 29 & 76 & 95 & 29 & 16 & 108 & 353 \\
% GPT-4o & 2 & 1 & 41 & 0 & 3 & 114 & 161 \\
% Claude & 2 & 1 & 26 & 3 & 6 & 96 & 134 \\ \bottomrule
% \end{tabular}%
% }
% \caption{Kaz unsafe cases over risk areas of six models.}
% \label{tab:unsafe_answers_summary_kazakh}
% \end{table}


\begin{table}[t!]
\centering
\resizebox{\columnwidth}{!}{%
\begin{tabular}{lccccccc}
\toprule
\textbf{Model} & \textbf{I} & \textbf{II} & \textbf{III} & \textbf{IV} & \textbf{V} & \textbf{VI} & \textbf{Total} \\ \midrule
\aya & 96 & 235 & 165 & 166 & 90 & 294 & 1046 \\
\llamaeight & 25 & 15 & 91 & 37 & 14 & 353 & 535 \\
\llamaseventy & 33 & 39 & 88 & 27 & 20 & 246 & 453 \\
\yandexgpt & 29 & 76 & 95 & 29 & 16 & 108 & 353 \\
\gptfouro & 2 & 1 & 41 & 0 & 3 & 114 & 161 \\
\claude & 2 & 1 & 26 & 3 & 6 & 96 & 134 \\ 
\bottomrule
\end{tabular}%
}
\caption{Kaz unsafe cases over risk areas of six models.}
\label{tab:unsafe_answers_summary_kazakh}
\end{table}

% \begin{figure*}[t!]
% 	\centering
% 	\includegraphics[scale=0.28]{figures/human_1000_kz_font16.png} 
% 	\includegraphics[scale=0.28]{figures/human_1000_ru_font16.png}

% 	\caption{Human vs \gptfouro\ fine-grained labels on 1,000 Kazakh (left) and Russian (right) samples.}
% 	\label{fig:human_fg_1000}
% \end{figure*}

\textbf{Question Type.} For Russian, Figures \ref{fig:qt_non_reg} and \ref{fig:qt_reg} reveal differences in how models handle general risks I-V and region-specific risk VI. For risks I-V, indirect attacks % crafted to exploit model vulnerabilities—
result in more unsafe responses due to their tricky phrasing. 

In contrast, region-specific risks see slightly more unsafe responses from direct attacks, 
% as these explicit prompts are more likely to bypass safety mechanisms. 
since indirect attacks for region-specific prompts often elicit safer, vaguer answers. %, suggesting models struggle less with implicit harm. 
Overall, the number of unsafe responses is similar across question types, indicating models generally struggle with safety alignment in all jailbreaking queries.

For Kazakh, Figures \ref{fig:qt_non_reg} and \ref{fig:qt_reg} show greater variation in unsafe responses across question types due to the low-resource nature of Kazakh data. For general risks I-V, \llamaseventy\ and \aya\ produce more unsafe outputs for direct harm prompts. At the same time, \claude\ and \gptfouro\ struggle more with indirect attacks, reflecting challenges in handling subtle cues. For region-specific risk VI, most models perform similarly due to limited Kazakh-specific data, though \llamaeight\ shows higher unsafe outputs for indirect local references, likely due to their implicit nature. Direct region-specific attacks yield fewer unsafe responses, as explicit prompts trigger more cautious outputs. Across all risk areas, general questions with sensitive words produce the fewest unsafe answers, suggesting over-flagging or cautious behavior for unclear harmful intent.



% \subsection{Fine-grained Classification}
% We extended our analysis to include fine-grained classifications for both safe and unsafe responses. For unsafe responses, we categorized outputs into four harm types, as shown in Table \ref{table:unsafe_response_categories}. 

% For safe responses, we classified outputs into six distinct patterns of safety, following a fine-grained rubric provided in \cite{wang2024chinesedatasetevaluatingsafeguards}. The types outlined in this rubric are presented in Table \ref{table:safe_response_categories}.

% To validate the fine-grained classification, we conducted human evaluation on the same 1,000 responses in Russian used for the preliminary binary classification.
% The confusion matrix highlights the alignment and discrepancies between human annotations and GPT's fine-grained labels. The diagonal values represent instances where the GPT's labels match human annotations, with category 5 (provides general, safe information) showing the highest agreement (404 instances). However, off-diagonal values reveal areas of disagreement, such as misclassifications in categories 1 (not willing to answer), 4 (recommends consulting an expert), and 7 (provides harmful or risky instructions). For example, in category 1, while 94 responses were correctly labeled, GPT-4o misclassified several instances into categories 4, 5, or 7, indicating overlap or ambiguity in these classifications. Similar trends are observed in other categories, where GPT sometimes struggles to differentiate nuanced distinctions in human-labeled categories. Overall, GPT's fine-grained labels match human annotations in 710 out of 1000 cases, achieving an agreement rate of 71\%.

% \begin{figure}[ht!]
%     \centering
%     \includegraphics[width=0.95\linewidth]{figures/human_fg_1000_ru.png}
%     \caption{{Human vs GPT-4o Fine-Grained Labels on 1,000 Russian Samples}}
%     \label{fig:human_fg_1000_ru}
% \end{figure}


% After conducting human evaluation on a representative sample, we extended the fine-grained classification to a full dataset comprising 21,915 responses generated by five different models. 


% \begin{figure}[ht!]
%     \centering
%     \includegraphics[width=0.95\linewidth]{figures/all_5_ru.png}
%     \caption{Fine-grained label distribution for responses from five models for Russian.}
%     \label{fig:all_5_fg_russian}
% \end{figure}
% Category 5 ("safe and general information") consistently has the highest frequency across all models, aligning with its dominance in the 1,000-sample evaluation. However, differences in the distribution across other categories highlight variability in how models handle nuanced safety risks, with Yandex-GPT showing a slightly broader spread across categories. 
% In the distribution of unsafe responses, most models exhibit higher counts in certain labels such as 8. However, Yandex-GPT displays comparatively fewer responses in these labels. 
% It exhibits a high rate of responses classified under label 7, which indicates instances where the model provides harmful or risky instructions, including unethical behavior or sensitive discussions. While this may suggest a vulnerability in addressing complex or challenging prompts, it was observed that many of Yandex-GPT’s responses tend to deflect responsibility or offer vague advice such as "check the internet". Although this approach minimizes the risk of unsafe outputs, it often results in responses that lack depth or contextually relevant information, limiting their overall usefulness for users.

% % \subsection{Kazakh}


% % Overall, these findings underscore how resource constraints and prompt explicitness affect model safety in Kazakh. Some models manage direct attacks better yet fail on indirect ones, while region-specific content remains challenging for all given the lack of localized training data.
% \subsubsection{Fine-grained Classification}
% Similarly, we conducted a human evaluation on 1,000 Kazakh samples, following the same methodology as the Russian evaluation. The match between human annotations and GPT-4o's fine-grained classifications was 707 out of 1,000, ensuring that the fine-grained classification framework aligned well with human judgments.
% The confusion matrix in Figure \ref{fig:human_fg_1000_kz} for 1,000 Kazakh samples illustrates the agreement between human annotations and GPT-4o's fine-grained classifications. The highest agreement is observed in category 5 (360 instances), indicating GPT-4o's strength in recognizing responses labeled by humans as "safe and general information." However, discrepancies are evident in categories such as 3 and 7, where GPT-4o misclassified several instances, highlighting areas for further refinement in distinguishing nuanced classifications.


\begin{figure}[t!]
	\centering
	\includegraphics[scale=0.18]{figures/human_1000_kz_font16.png} 
	\includegraphics[scale=0.18]{figures/human_1000_ru_font16.png}

	\caption{Human vs \gptfouro\ fine-grained labels on 1,000 Kazakh (left) and Russian (right) samples.}
	\label{fig:human_fg_1000}
\end{figure}

% \begin{figure}[t!]
% 	\centering
% 	\includegraphics[scale=0.28]{figures/human_1000_kz_font16.png} 
% 	\includegraphics[scale=0.28]{figures/human_1000_ru_font16.png}

% 	\caption{Human vs \gptfouro\ fine-grained labels on 1,000 Kazakh (top) and Russian (bottom) samples.}
% 	\label{fig:human_fg_1000}
% \end{figure}

% \begin{figure*}[t!]
% 	\centering
% 	\includegraphics[scale=0.28]{figures/all_5_kz_font16.png} 
% 	\includegraphics[scale=0.28]{figures/all_5_ru_font_16.png} \\
% 	\caption{Fine-grained responding pattern distribution across five models for Kazakh (left) and Russian (right).}
% 	\label{fig:all_5}
% \end{figure*}

\begin{figure}[t!]
	\centering
	\includegraphics[scale=0.28]{figures/all_5_kz_font16.png} 
	\includegraphics[scale=0.28]{figures/all_5_ru_font_16.png} \\
	\caption{Fine-grained responding pattern distribution across five models for Kazakh (top) and Russian (bottom).}
	\label{fig:all_5}
\end{figure}


\subsection{Fine-Grained Classification}
\label{sec:fine-grained-classification}
% As discussed in Section \ref{harmfulness_evaluation}, 
We further analyzed fine-grained responding patterns for safe and unsafe responses. For unsafe responses, outputs were categorized into four harm types, and safe responses were classified into six distinct patterns of safety, as rubric in Appendix \ref{safe_unsafe_response_categories}. 
% \cite{wang2024chinesedatasetevaluatingsafeguards}

\paragraph{Human vs. GPT-4o}
Similar to binary classification, we validated \gptfouro's automatic evaluation results by comparing with human annotations on 1,000 samples for both Russian and Kazakh. %, comparing human annotations with \gptfouro's fine-grained labels.
For the Russian dataset, \gptfouro's labels aligned with human annotations in 710 out of 1,000 cases, achieving an agreement rate of 71\%. 
Agreement rate of Kazakh samples is 70.7\%. %with 707 out of 1,000 cases matching
% The confusion matrix highlights areas of alignment and discrepancies.
% 
As confusion matrices illustrated in Figure~\ref{fig:human_fg_1000}, the majority of cases falling into \textit{safe responding patter 5} --- providing general and harmless information, for both human annotations and automatic predictions.
% highest agreement with 404 correct classifications for Russian. 
Mis-classifications for safe responses mainly focus on three closely-similar patterns: 3, 4, and 5, and patterns 7 and 8 are confusing to discern for unsafe responses, particularly for Kazakh (left figure).
We find many Russian samples which were labeled as ``1. reject to answer'' by humans are diversely classified across 1-6 by GPT-4o, which is also observed in Kazakh but not significant.

% suggesting label alignment issues are language-independent. 
% YX: I did not observe this, commented
% Notably, Russian showed confusion between 7 (risky instructions) and 1 (refusal to answer), this trend does not appear in Kazakh.


% highlight the strengths and limitations of {\gptfouro}'s fine-grained classification framework across both languages, paving the way for further refinements.


% However, misclassifications were observed in categories such as 1 (not willing to answer), 4 (recommends consulting an expert), and 7 (provides harmful or risky instructions), revealing overlaps and ambiguities in nuanced classifications.

% Similarly, for the Kazakh dataset, the agreement rate between human annotations and GPT-4o's labels was 70.7\%, with 707 out of 1,000 cases matching. As with the Russian analysis, category 5 (360 instances) showed the highest alignment. However, discrepancies were more prominent in categories such as 3 and 7, underscoring GPT-4o's challenges in differentiating fine-grained human-labeled categories. 
% A similar pattern was observed for Kazakh dataset, which suggests that alignment and misaligned of fine-grained lables is not language dependent.

% These findings, illustrated in Figures \ref{fig:human_fg_1000}, highlight the strengths and limitations of {\gptfouro}'s fine-grained classification framework across both languages, paving the way for further refinements.

\paragraph{Fine-grained Analysis of Five LLMs}
% After conducting human evaluation on representative samples, we extended 
\figref{fig:all_5} shows fine-grained responding pattern distribution across five models based on the full set of Russian and Kazakh data.
% For Russian, we selected \vikhr, \gptfouro, \llamaseventy, \claude, and \yandexgpt, while for Kazakh, we chose \aya, \gptfouro, \llamaseventy, \claude, and \yandexgpt. 
% The evaluation covered 21,915 responses in Russian and 18,930 responses in Kazakh.
% 
In both languages, pattern 5 of providing \textit{general and harmless information} consistently witnessed the highest frequency across all models, with \llamaseventy\ exhibiting the largest number of responses falling into this category for Kazakh (2,033). 
% YX:summarize more noteable findings here.

Differences of other patterns vary across languages. 
Unsafe responses in Russian are predominantly in pattern 8, where models provide incorrect or misleading information without expressing uncertainty. % (misinformation and speculation), 
For Kazakh, \aya\ exhibits the highest occurrence of pattern 7 (harmful or risky information) and pattern 8, indicating a stronger tendency to generate unethical, misleading, or potentially harmful content.

%Variations in other patterns across models highlight differences in how nuanced safety risks are classified, reflecting the models' differing capabilities in handling safety evaluation for these distinct linguistic contexts. For Russian, the majority of unsafe responses fall under pattern 8 (misinformation and speculation), indicating that models frequently provide incorrect or misleading information without acknowledging uncertainty. For Kazakh, \aya\ has the highest occurence of pattern 7 (harmful or risky information) and pattern 8 (misinformation and speculation), indicating a greater tendency to generate unethical, misleading, or potentially harmful content. 

%This trend suggests that Russian models may struggle more with factual accuracy, whereas Kazakh models, particularly \aya, pose higher risks related to both harmful content and misinformation. Additionally, \gptfouro\ and \claude\ consistently produce fewer unsafe responses in both languages, demonstrating stronger alignment with safety standards
\subsection{Code Switching}
\begin{table}[t!]
\centering

\setlength{\tabcolsep}{3pt}
\scalebox{0.7}{
\begin{tabular}{lcccccccccc}
\toprule
\textbf{Model Name} & \multicolumn{2}{c}{\textbf{Kazakh}} & \multicolumn{2}{c}{\textbf{Russian}} & \multicolumn{2}{c}{\textbf{Code-Switched}} \\  
\cmidrule(lr){2-3} \cmidrule(lr){4-5} \cmidrule(lr){6-7}
& \textbf{Safe} & \textbf{Unsafe} & \textbf{Safe} & \textbf{Unsafe} & \textbf{Safe} & \textbf{Unsafe} \\ 
\midrule
\llamaseventy & 450 & 50 & 466 & 34 & 414 & 86 \\
\gptfouro & 492 & 8 & 473 & 27 & 481 & 19
 \\
\claude & 491 & 9 & 478 & 22 & 484 & 16 \\ 
\yandexgpt & 435 & 65 & 458 & 42 & 464 & 36 \\
\midrule
\end{tabular}}
\caption{Model safety when prompted in Kazakh, Russian, and code-switched language.}
\label{tab:finetuning-comparison}
\end{table}


\gptfouro\ and \claude\ demonstrate strong safety performance across three languages, even with a high proportion of safe responses in the challenging code-switching context. In contrast, \llamaseventy\ and \yandexgpt\ are less safe, exhibiting more harmful responses, particularly in the code-switching scenario. These results show the varying capabilities of models in defending the same attacks that are just presented in different languages, where open-sourced large language models especially require more robust safety alignment in multilingual and code-switching scenarios.

% \subsection{LLM Response Collection}
% We conducted experiments with a variety of mainstream and region-specific 
% large language models for both Russian and Kazakh languages. For both Russian and Kazakh languages, we employed four multilingual models: Claude-3.5-sonnet, Llama 3.1 70B \cite{meta2024llama3}, GPT-4 \cite{openai2024gpt4o}, and YandexGPT. Additionally, we included language-specific models: VIKHR \cite{nikolich2024vikhrconstructingstateoftheartbilingual} for Russian and Aya \cite{ustun-etal-2024-aya} for Kazakh. 

% \subsection{Kazakh-Russian Code-Switching Evaluation}

% In Kazakhstan, the prevalence of bilingualism is a defining characteristic of its linguistic landscape, with most individuals seamlessly mixing Kazakh and Russian in daily communication \cite{Zharkynbekova2022}. This phenomenon, known as code-switching, reflects the unique cultural and social dynamics of the region. Despite this, there is currently no safety evaluation dataset tailored to this unique multilingual environment. Developing a code-switched dataset is essential to evaluate the ability of large language models (LLMs) to navigate the complexities of bilingual interactions, ensuring they produce contextually appropriate, non-harmful, and culturally sensitive responses. To address this, we sampled 500 questions from both Kazakh and Russian question sets, ensuring they represent similar queries but with culturally nuanced differences in meaning. Building on these, we developed a mixed version that integrates Kazakh and Russian naturally, reflecting the way people in Kazakhstan communicate. This process involved annotators to ensure that the mixed questions were crafted with a natural tone and were localized to Kazakhstan's context.


% \begin{table*}[t!]
%     \centering
%     \small
%     \setlength{\tabcolsep}{6pt} % Adjusts column separation
%     \adjustbox{max width=\linewidth}{
%     \begin{tabular}{ll ccc ccc c}
%     \toprule
%     \textbf{Risk Area} & \textbf{Harm Type} & \multicolumn{3}{c}{\textbf{Kazakh}} & \multicolumn{3}{c}{\textbf{Russian}} & \textbf{\#Q} \\
%                        &                    & \textbf{Ori} & \textbf{FN} & \textbf{FP} & \textbf{Ori} & \textbf{FN} & \textbf{FP} & \\
%     \midrule
%     \multirow{2}{*}{I. Information Hazards} & 1. Risks from leaking sensitive information... & 131 & 133 & 131 & 131 & 133 & 131 & 790 \\
%                                              & 2. Compromise of privacy by leaking or inferring private information (person/individual) & 81 & 82 & 81 & 81 & 82 & 81 & 488 \\
%     \midrule
%     \multirow{3}{*}{II. Malicious Uses} & 3. Assisting illegal activities & 132 & 135 & 132 & 132 & 135 & 132 & 798 \\
%                                         & 4. Nudging or advising unethical actions & 71 & 71 & 71 & 71 & 71 & 71 & 426 \\
%                                         & 5. Reducing the cost of disinformation campaigns & 40 & 42 & 40 & 40 & 42 & 40 & 244 \\
%     \midrule
%     \multirow{3}{*}{III. Discrimination, Exclusion, Toxicity, Hateful, Offensive} & 6. Social stereotypes and unfair discrimination & 94 & 96 & 94 & 94 & 96 & 94 & 568 \\
%                                         & 7. Toxic language (hate speech) & 52 & 59 & 52 & 52 & 59 & 52 & 326 \\
%                                         & 8. Adult content & 27 & 29 & 27 & 27 & 29 & 27 & 166 \\

%     \midrule
%     \multirow{2}{*}{IV. Misinformation Harms} & 9. Disseminating false or misleading information & 92 & 99 & 92 & 92 & 99 & 92 & 566 \\
%                                         & 10. Causing material harm by disseminating misinformation e.g. in medicine or law & 63 & 63 & 63 & 63 & 63 & 63 & 378 \\
%     \midrule
%     \multirow{2}{*}{V. Human–chatbot Interaction Harms} & 11. Mental health or overreliance concerns & 66 & 66 & 66 & 66 & 66 & 66 & 396 \\
%                                         & 12. Treating the chatbot as a human & 50 & 51 & 50 & 50 & 51 & 50 & 302 \\

%     \midrule
%     \multirow{2}{*}{VI. Kazakhstan-specific Sensitivity} & 13. Politically sensitive topics & 63 & 66 & 63& 63 & 66 & 63 &  384 \\
%         & 14. Controversial historical events & 46 & 57 & 46 & 46 & 57 & 46  & 298 \\
% & 15. Regional and racial issues & 45 & 45 & 45 &  45 & 45 & 45 & 270  \\
% & 16. Societal and cultural concerns & 138 & 139 & 138 &  138 & 139 & 138  & 830  \\
% & 17. Legal and human rights matters & 57 & 57 & 57 & 57 & 57 & 57  & 342 \\
%     \midrule
%         \multirow{2}{*}{VII. Russia-specific Sensitivity} 
%             & 13. Politically sensitive topics & - & - & - & 54 & 54 & 54 & 162 \\
%     & 14. Controversial historical events & - & - & - & 38 & 38 & 38 & 114 \\
%     & 15. Regional and racial issues & - & - & - & 26 & 26 & 26 & 78 \\
%     & 16. Societal and cultural concerns & - & - & - & 40 & 40 & 40 & 120 \\
%     & 17. Legal and human rights matters & - & - & - & 41 & 41 & 41 & 123 \\
%     \midrule
%     \bf Total & -- & 1248 & 1290 & 1248 & 1447 & 1489 & 1447 & \textbf{8169} \\
%     \bottomrule
%     \end{tabular}
%     }
%     \caption{The number of questions for Kazakh and Russian datasets across six risk areas and 17 harm types. Ori: original direct attack, FN: indirect attack, and FP: over-sensitivity assessment.}
%     \label{tab:kazakh-russian-data}
% \end{table*}




\section{Discussion}

% \subsection{Kazakh vs Russian}

% The evaluation reveals that Kazakh responses tend to be generally safer than their Russian counterparts, likely due to Kazakh being a low-resource language with significantly less training data. As a result, Kazakh models are less exposed to the vast, often unfiltered datasets containing harmful or unsafe content, which are more prevalent in high-resource languages like Russian. This data scarcity naturally limits the model's ability to generate nuanced but potentially unsafe responses. However, this does not mean the models are specifically fine-tuned for safer performance. When analyzing unsafe answers, it’s clear that Kazakh models, while safer overall, distribute their unsafe responses more evenly across various risk types and question types. This suggests Kazakh models generate fewer unsafe answers but in a broader range of contexts.

% In contrast, Russian models tend to concentrate unsafe answers in specific areas, particularly region-specific risks or indirect attacks. This indicates that Russian models have learned to handle certain types of unsafe content by focusing on specific topics, such as politically sensitive issues, but struggle when confronted with unfamiliar content, leading to unsafe responses due to insufficient filtering. Kazakh models, despite having less training data, tend to respond more broadly, including both direct and indirect risks. This could be due to the less curated nature of their training data, making them more likely to answer unsafe questions without filtering the potential harm involved. The exception to this trend is Aya, a model specifically fine-tuned for Kazakh. Despite fine-tuning, it exhibits the lowest safety percentage (72.37\%) in the Kazakh dataset, suggesting that fine-tuning in specific languages may introduce risks if proper safety measures are not taken.

% The evaluation reveals notable differences in the distribution of safe response patterns across Kazakh and Russian fine-grained labels. Refusal to answer is more frequent in Russian models, particularly Yandex-GPT, reflecting a cautious approach to safety-critical queries. Interestingly, Aya, despite being fine-tuned for Kazakh and exhibiting lower overall safety, also frequently refuses to answer, suggesting an over-reliance on conservative mechanisms. Responses providing general, safe information dominate in both languages, with Kazakh models displaying a slightly higher tendency to rely on this approach. This highlights how the low-resource nature of Kazakh results in more generalized and inherently safer responses. In contrast, Russian models excel at recognizing risks, issuing disclaimers, and refuting incorrect assumptions, likely benefiting from richer and more diverse training data.
% Yandex-GPT exhibits a notably high rate of responses classified under label 7, indicating an overreliance on general disclaimers or deflections, such as "check the internet" or "I don't know." While these responses minimize the risk of unsafe outputs, they often lack substantive or contextually relevant information, reducing their overall utility for users.


Most models perform safer on Kazakh dataset than Russian dataset, higher safe rate on Kazakh dataset in \tabref{tab:safety-binary-eval}. This does not necessarily reveal that current LLMs have better understanding and safety alignment on Kazakh language than Russian, while this may conversely imply that models do not fully understand the meaning of Kazakh attack questions, fail to perceive risks and then provide general information due to lacking sufficient knowledge regarding this request.

We observed the similar number of examples falling into category 5 \textit{general and harmless information} for both Kazakh and Russian, while the Kazakh data set size is 3.7K and Russian is 4.3K. Kazakh has much less examples in category 1 \textit{reject to answer} compared to Russian. This demonstrate models tend to provide general information and cannot clearly perceive risks for many cases.

Additionally, in spite of less harmful responses on Kazakh data, these unsafe responses distribute evenly across different risk areas and question categories, exhibiting equally vulnerability spanning all attacks regardless of what risks and how we jailbreak it.
In contrary, unsafe responses on Russian dataset often concentrate on specific areas and question types, such as region-specific risks or indirect attacks, presenting similar model behaviors when evaluating over English and Chinese data.
It suggests that broader training data in English, Chinese and Russian may allow models to address certain types of attacks robustly,
% effectively—particularly politically sensitive issues—
yet they may falter when confronted with unfamiliar content like regional sensitive topics.

Moreover, in responses collection, we observed many Russian or English responses especially for open-sourced LLMs when we explicitly instructed the models to answer Kazakh questions in Kazakh language. This further implies more efforts are still needed to improve LLMs' performance on low-resource languages.
Interestingly, \aya, a fine-tuned Kazakh model, proves an exception by displaying the lowest safety percentage (72.37\%) among Kazakh models, revealing that the multilingual fine-tuning without stringent safety measures can introduce risks.



% However, this does not mean they are explicitly fine-tuned for safety, likely it happens due to limited training data, which reduces exposure to harmful content. 
% \aya, a fine-tuned Kazakh model, proves an exception by displaying the lowest safety percentage (72.37\%) among Kazakh models, revealing that the multilingual fine-tuning without stringent safety measures can introduce risks.
% Kazakh models generally produce safer responses than their Russian counterparts, likely because Kazakh is a low-resource language with less training data. 
% This limited exposure to harmful or unsafe content naturally limits nuanced yet potentially unsafe outputs. 
% However, it does not imply that the models are specifically fine-tuned for enhanced safety.


% while Kazakh models tend to generate fewer unsafe answers overall, those unsafe responses appear more evenly spread across different risk types and question categories.
% Russian models, on the other hand, often concentrate unsafe responses in specific areas, such as region-specific risks or indirect attacks.
% It implies that their broader training datasets allow them to address certain types of unsafe content more effectively—particularly politically sensitive issues—yet they may falter when confronted with unfamiliar or insufficiently filtered content.

% Meanwhile, Kazakh models sometimes respond more broadly, possibly due to less curated training data. 

Differences also emerge in how language models handle safe responses. 
\yandexgpt, for instance, often refuses to answer high-risk queries. 
It frequently relies on generic disclaimers or deflections like ``check in the Internet'' or ``I don’t know,'' minimizing risk but are less helpful. Interestingly, it often responds with ``I don’t know'' in Russian, even for Kazakh queries, we speculate that these may be default responses stemming from internal system filters, rather than generated by model itself.
This likely explains why \yandexgpt\ is the safest model for the Russian language but ranks third for Kazakh. While its filters perform well for Russian, they struggle with the low-resource Kazakh language.

% Aya, despite its lower overall safety, also employs refusals often, hinting at an over-reliance on conservative approaches. 

% Across both languages, models commonly resort to providing general, safe information, although Kazakh models lean on this strategy slightly more. 
% Russian models, by contrast, excel at detecting risks, issuing disclaimers, and correcting inaccuracies, likely benefiting from richer and more diverse training data.


% \subsection{Response Patterns}


% We conducted a detailed analysis of the models' outputs and identified several noteworthy patterns. YandexGPT, while being one of the safest overall, frequently generates responses in Russian even when the question is posed in Kazakh. These responses often appear as placeholders, prompting users to search for the answer online. This behavior might not originate from the model itself but rather from safety filters implemented in the YandexGPT system. The model's leading performance in ensuring safety during Russian-language interactions, coupled with its lower performance in Kazakh, can be attributed to the limited robustness of these safety filters when handling unsafe content in Kazakh.

% In contrast, Aya-101 exhibits a tendency to fall into repetition, often repeating the same sentences multiple times. Interestingly, the Vikhr model, despite being of a similar size, does not exhibit this issue. We attribute this difference to two key factors. First, Vikhr and Aya-101 have distinct architectures: Vikhr is based on the Mistral-Nemo model, whereas Aya-101 is built on mT5, an older and less robust model. Second, Aya-101 is a multilingual model, while Vikhr was predominantly trained for Russian. Multilingualism has been shown to potentially degrade performance in large language models~\cite{huang2025surveylargelanguagemodels}, which may explain Aya-101's issues with repetition.

\begin{table*}[!t]
    \centering
    \resizebox{\textwidth}{!}{%
        \begin{tabular}{lccccc|c}
            \toprule
            & \textbf{Microbiology} & \textbf{Chemistry} & \textbf{Economics} & \textbf{Sociology} & \textbf{US History} & \textbf{Average} \\
            \midrule
            Base score & 0.46 & 0.09 & 0.00 & 0.61 & 0.03 & 0.24 \\
            \midrule
            Zero-shot & 0.62 (+0.16) & 0.40 (+0.31) & 0.40 (+0.40) & 0.61 (+0.00) & 0.19 (+0.16) & 0.44 (+0.20) \\
            Few-shot & 0.62 (+0.16) & \underline{0.45} (+0.36) & \underline{0.47} (+0.47) & 0.62 (+0.01) & 0.16 (+0.13) & 0.46 (+0.22) \\
            Chain-of-thought & 0.61 (+0.15) & \underline{0.45} (+0.36) & 0.46 (+0.46) & 0.61 (+0.00) & 0.19 (+0.16) & 0.46 (+0.22) \\
            Bloom-based & 0.57 (+0.11) & 0.37 (+0.28) & 0.29 (+0.29) & 0.62 (+0.01) & 0.22 (+0.19) & 0.41 (+0.17) \\
            \midrule
            SFT (Subject-Specific) & 0.65 (+0.19) & 0.24 (+0.15) & 0.46 (+0.46) & \underline{0.64} (+0.03) & 0.20 (+0.17) & 0.44 (+0.20) \\
            SFT (Cross-Subject) & 0.59 (+0.13) & 0.21 (+0.12) & \underline{0.47 (+0.47)} & 0.63 (+0.02) & \underline{0.26} (+0.23) & 0.43 (+0.19) \\
            \midrule
            \textsc{QUEST} (Subject-Specific) & \bf 0.76 (+0.30) & \bf 0.46 (+0.37) & \bf 0.58 (+0.58) & \bf 0.65 (+0.04) & \bf 0.31 (+0.28) & \bf 0.55 (+0.31) \\
            \textsc{QUEST} (Cross-Subject) & \underline{0.73} (+0.27) & \underline{0.41} (+0.32) & \underline{0.47} (+0.47) & \bf 0.65 (+0.04) & 0.25 (+0.22) & \underline{0.50} (+0.26) \\
            \bottomrule
        \end{tabular}
    }
    \caption{\textbf{End-of-chapter exam score} results of different question generation approaches across various subjects. 
    \ours produces models that generate questions that lead to the highest scores on all subjects.  
    Gain values (in parentheses) are calculated as the increase from the base score. The highest and second highest scores per column are \textbf{bolded} and \underline{underlined}, respectively.}
    \label{tab:question-gen-results}
\end{table*}


\section{Experimental Results}
In this section, we first compare the overall performance of all question generation baselines based on the learner's exam score (\secref{ssec:overall-performance}).  
Next, we analyze evaluation metrics by examining their correlations with utility and assessing the impact of optimizing models on high-scoring questions for each metric (\secref{ssec:evaluation-metrics}).
We then conduct a qualitative analysis of high-utility questions to understand their characteristics (\secref{ssec:high-utility-question}). 
Finally, we perform ablation studies on the framework by varying the criteria for selecting high-utility questions for training and replacing \texttt{gpt-4o-mini} to \texttt{gpt-4o} (\secref{ssec:rs-analysis}-\secref{ssec:model-variants}).



\subsection{Overall Performance}
\label{ssec:overall-performance}
Table~\ref{tab:question-gen-results} presents the learner's exam performance of different question generators, measured by exam scores using all generated question-answer pairs (\secref{ssec:quest-evaluation}).
Here are findings:
(1) \textbf{Prompting techniques (Few-shot, CoT, Bloom-based)} offer only marginal performance gains. While advanced prompting enhances reasoning and task accuracy~\cite{brown2020language, wei2022chain, zhou2024self}, it does not directly optimize utility, which reflects real-world impact—how well generated questions enhance learning. Without explicit selection or optimization, prompting cannot systematically improve this measure;
(2) \textbf{SFT} shows no performance gains, indicating that while it learns the style of exam questions, it fails to generate questions that enhance learner understanding.
This highlights the key distinction between producing syntactically valid questions and generating those that effectively promote learning;
(3) \textbf{QUEST} achieves the highest performance gain, improving by approximately 20\% on average.
Performance gap between subject-specific and cross-subject rejection sampling suggests that the definition of a ``high-utility'' question varies by domain.
The results indicate that outcome-based learning is most effective when applied within a specific domain.


\subsection{Evaluation Metrics Analysis}
\label{ssec:evaluation-metrics}
\paragraph{Correlation.}
\begin{table}[!t]
    \centering
    \resizebox{\columnwidth}{!}{%
        \begin{tabular}{ll
        cc}
            \toprule
            Metric 1 & Metric 2 & \textbf{Spearman correlation} & \textbf{p-value} \\
            \midrule
            \textbf{Utility} & Saliency & 0.097 & 0.003 \\
            \textbf{Utility} & EIG  & -0.022 & 0.512 \\
            Saliency & EIG  & 0.030 & 0.363 \\
            \bottomrule
        \end{tabular}
    }
    \caption{\textbf{Spearman correlation between metrics.} Utility shows a weak correlation with saliency and EIG, showing that it is independent of these indirect metrics.}
    \label{tab:correlation_results}
\end{table}
To analyze the relationship between \textit{utility} and existing metrics (\textit{saliency}, \textit{EIG}), we estimate all three metrics on generated questions from the training set.
Table~\ref{tab:correlation_results} shows that both saliency and EIG have weak correlations with utility.
While saliency has a weak but statistically significant correlation with utility, EIG shows no meaningful relationship.
This indicates that existing indirect metrics may not accurately reflect a question’s impact on learning outcomes.

\paragraph{Optimization on Indirect Metrics.}
\begin{table*}[!t]
    \centering
    \small
    \resizebox{\textwidth}{!}{%
        \begin{tabular}{lccc ccc ccc ccc ccc}
            \toprule
            & \multicolumn{3}{c}{\textbf{Microbiology}} & \multicolumn{3}{c}{\textbf{Chemistry}} & \multicolumn{3}{c}{\textbf{Economics}} & \multicolumn{3}{c}{\textbf{Sociology}} & \multicolumn{3}{c}{\textbf{US History}} \\
            \cmidrule(lr){2-4} \cmidrule(lr){5-7} \cmidrule(lr){8-10} \cmidrule(lr){11-13} \cmidrule(lr){14-16}
            \textbf{Train Metric} & \textbf{Utility} & \textbf{Saliency} & \textbf{EIG} & \textbf{Utility} & \textbf{Saliency} & \textbf{EIG} & \textbf{Utility} & \textbf{Saliency} & \textbf{EIG} & \textbf{Utility} & \textbf{Saliency} & \textbf{EIG} & \textbf{Utility} & \textbf{Saliency} & \textbf{EIG} \\
            \midrule
            $utility > 0.1$ & \textbf{0.76} & 4.27 & -0.18 & \textbf{0.46} & 4.65 & -0.20 & \textbf{0.58} & \textbf{4.70} & -0.04 & \textbf{0.65} & \textbf{4.49} & -0.02 & \textbf{0.31} & 4.65 & \bf -0.01 \\
            $saliency = 5$  & 0.73 & \textbf{4.42} & -0.24 & 0.39 & \textbf{4.46} & -0.22 & 0.46 & 4.66 & -0.08 & 0.64 & \textbf{4.49} & -0.03 & 0.23 & \textbf{4.68} & -0.02 \\
            $EIG > 0$      & 0.61 & 4.21 & \textbf{-0.17} & 0.32 & 4.40 & \textbf{-0.09} & 0.47 & 4.65 & \textbf{0.01} & 0.62 & 4.46 & \textbf{0.01} & 0.21 & 4.65 & \bf -0.01 \\
            \bottomrule
        \end{tabular}
    }
    \caption{\textbf{End-of-chapter exam scores (utility), average saliency, and average EIG of generated questions} for different \textsc{QUEST}-optimized models trained on datasets filtered by different selection criteria.}
    \label{tab:quest-indirect}
\end{table*}
To further investigate the impact of indirect metrics on question generation performance, we compare the results of \textsc{QUEST} when trained using different selection criteria: (1) only questions with a \textit{utility} score greater than 0.1 (Ours), (2) only questions with a \textit{saliency} score of 5, and (3) only questions with an \textit{expected information gain (EIG)} greater than 0.
We evaluate the generated questions based on their overall utility (\textit{i.e.,} end-of-chapter exam scores), as well as their average saliency and EIG, to assess the question generator’s performance across different quality metrics.
Table~\ref{tab:quest-indirect} shows that while saliency- and EIG-based training improves their respective scores, it does not enhance utility. 
In contrast, the utility-trained model consistently achieves the highest utility across all subjects and even improves some indirect metrics (\textit{e.g.,} it matches or outperforms saliency-based training in Economics and Sociology).
Furthermore, utility-based training outperforms EIG-based training on saliency and saliency-based training on EIG, demonstrating its broader effectiveness. 
These results emphasize that optimizing for indirect metrics does not improve real-world learning, whereas utility-driven training yields the best overall performance.




\subsection{High Utility Questions Analysis}
\label{ssec:high-utility-question}
\paragraph{Overlap with exam questions.}
To evaluate the relationship between generated high-utility questions and exam questions, we measured their semantic and lexical similarity.
For each generated question, we computed embedding similarity using \texttt{text-3-embedding-small}\footnote{\href{https://platform.openai.com/docs/guides/embeddings/}{https://platform.openai.com/docs/guides/embeddings/}} for semantic overlap and the ROUGE score for lexical overlap with all exam questions in the same chapter.
We then assess the correlation between utility and the most similar exam question based on these measures.
The correlation between utility and semantic similarity is 0.25 (p < 0.001), indicating a weak positive relationship, while the correlation with ROUGE is nearly zero at 0.04 (p < 0.01).
These findings suggest that high-utility questions are not simple rephrasings of exam questions but introduce novel concepts that enhance learning beyond surface-level similarity.

\paragraph{Qualitative analysis.}
Qualitative question examples do not exhibit clear patterns in question style (see Appendix~\ref{appendix:qualitative-examples}).
An interesting observation is that Bloom's taxonomy, which categorizes cognitive depth based on question type—where "what" questions typically involve simple recall, while "why" and "how" questions require deeper processing—does not strongly correlate with utility.
Using Bloom's taxonomy as a cognitive depth scale (Likert 1-6), the correlation between utility and cognitive depth is 0.12 (p < 0.001), indicating a weak positive relationship.


\subsection{Rejection Sampling Analysis}
\label{ssec:rs-analysis}
Filtering for high-utility questions through rejection sampling is crucial for improving question generation. 
As shown in Figure~\ref{fig:rs_analysis}, increasing the utility threshold enhances question quality, leading to higher exam scores.
However, stricter filtering reduces the available training data, posing challenges for model training. 
These results suggest that increasing the dataset size while applying a higher threshold could further boost performance.
\begin{figure}[!t]
    \centering
    \begin{minipage}{\columnwidth}
    \centering
    \includegraphics[width=\columnwidth]{figures/rs_analysis.pdf}
    \end{minipage}
    \caption{\textbf{Impact of threshold} in \ours on end-of-chapter exam scores for Chemistry.}
    \label{fig:rs_analysis}
    \vspace{-0.5cm}
\end{figure}

\subsection{Model Variants Analysis}
\label{ssec:model-variants}
\begin{table*}[!t]
    \centering
    \resizebox{\textwidth}{!}{%
    \begin{tabular}{c|ccc|ccccc}
        \toprule
        & \textbf{QG ($M_q$)} & \textbf{AG ($M_a$)} & \textbf{RS ($M_l$)} & \textbf{Microbiology} & \textbf{Chemistry} & \textbf{Economics} & \textbf{Sociology} & \textbf{US History} \\
        \midrule
        \multicolumn{1}{c|}{\multirow{4}{*}{Zero-Shot}} 
        & \texttt{gpt-4o-mini} & \texttt{gpt-4o-mini} & \texttt{gpt-4o-mini} & 0.620 & 0.414 & 0.398 & 0.609 & 0.233 \\
        & \texttt{gpt-4o} & \texttt{gpt-4o-mini} & \texttt{gpt-4o-mini} & 0.681 & 0.457 & 0.466 & 0.634 & 0.180 \\
        & \texttt{gpt-4o-mini} & \texttt{gpt-4o} & \texttt{gpt-4o-mini} & 0.682 & 0.422 & 0.480 & 0.634 & 0.232 \\
        & \texttt{gpt-4o-mini} & \texttt{gpt-4o-mini} & \texttt{gpt-4o} & 0.710 & 0.173 & 0.476 & 0.564 & 0.263 \\
        \midrule
        \textsc{QUEST} & \texttt{gpt-4o-mini} & \texttt{gpt-4o-mini} & \texttt{gpt-4o-mini} & \bf 0.756 &	\bf 0.457 &	\bf 0.582 &	\bf 0.649 &	\bf 0.311 \\
        \bottomrule
    \end{tabular}
    }
    \vspace{-0.2cm}
    \caption{\textbf{End-of-chapter exam scores} for different model sizes across various subjects and modules.}
        \vspace{-0.2cm}

    \label{tab:model-variants}
\end{table*}

To evaluate the robustness of our framework and the impact of model size on different components, we conduct experiments to analyze how a larger model affects each module.
Table~\ref{tab:model-variants} presents results for different configurations of the question generator ($M_q$), answer generator ($M_a$), and reader simulator (\textit{i.e.,} learner $M_l$). 
The baseline corresponds to the \texttt{zero-shot} setting in Table~\ref{tab:question-gen-results}. 
In the $M_a$ experiment, we use the same questions from the \texttt{zero-shot} setting but generate new answers.
For the reader simulator experiment, we keep the same questions and answers from \texttt{zero-shot} and re-run the simulation only.

Our main findings are the following:
(1) \textbf{Question Generator}: A larger model (\texttt{gpt-4o}) improves the utility score by 5.7\%. However, it still underperforms compared to the smaller, utility-optimized model (\texttt{gpt-4o-mini}) by 12.3\%;
(2) \textbf{Answer Generator}: A larger model improves performance by 7.1\%, suggesting that higher answer quality provides additional information to the QA pair. 
However, it remains 11\% behind the optimized \texttt{gpt-4o-mini} in utility.
(3) \textbf{Reader Simulator}: Using a larger model (\texttt{gpt-4o}) as the reader simulator leads to mixed results, with performance gains in some subjects but a sharp decline in Chemistry. 
This suggests that larger models may introduce different reasoning strategies or evaluation biases, leading to inconsistencies in scoring. 
Additionally, since our framework is optimized for \texttt{gpt-4o-mini}, the larger model may not align well with the training dynamics. 
These results highlight the importance of consistency in simulation for reliable utility estimation.

% equipped with a larger model (\texttt{gpt-4o}) is expected to make more accurate simulation of how much generated questions really impact to the student understanding on exams.
% There is consistent increasing gap on microbiology, economics, sociology, us history while there is a huge gap in decreasing gap on chemistry, compared to smaller model (\texttt{gpt-4o-mini})



% \begin{itemize}
%     \item \textbf{Stronger question generators improve performance in some subjects but not all.} Upgrading $M_q$ from GPT-4o-mini to GPT-4o enhances performance in Microbiology and Economics but degrades US History, suggesting that question quality improvement is domain-dependent.
%     \item \textbf{Answer generator quality has a moderate effect on utility.} Stronger answer generators help in certain subjects (e.g., Economics) but have minimal impact in others (e.g., Chemistry).
%     \item \textbf{Evaluator model strength influences overall results but can introduce variability.} Using a stronger evaluator (GPT-4o) improves utility estimates in Microbiology and Economics but drastically reduces performance in Chemistry. This suggests that evaluators need to be carefully tuned to ensure robust and reliable assessments across different domains.
% \end{itemize}



\section{Discussion}\label{sec:discussion}



\subsection{From Interactive Prompting to Interactive Multi-modal Prompting}
The rapid advancements of large pre-trained generative models including large language models and text-to-image generation models, have inspired many HCI researchers to develop interactive tools to support users in crafting appropriate prompts.
% Studies on this topic in last two years' HCI conferences are predominantly focused on helping users refine single-modality textual prompts.
Many previous studies are focused on helping users refine single-modality textual prompts.
However, for many real-world applications concerning data beyond text modality, such as multi-modal AI and embodied intelligence, information from other modalities is essential in constructing sophisticated multi-modal prompts that fully convey users' instruction.
This demand inspires some researchers to develop multimodal prompting interactions to facilitate generation tasks ranging from visual modality image generation~\cite{wang2024promptcharm, promptpaint} to textual modality story generation~\cite{chung2022tale}.
% Some previous studies contributed relevant findings on this topic. 
Specifically, for the image generation task, recent studies have contributed some relevant findings on multi-modal prompting.
For example, PromptCharm~\cite{wang2024promptcharm} discovers the importance of multimodal feedback in refining initial text-based prompting in diffusion models.
However, the multi-modal interactions in PromptCharm are mainly focused on the feedback empowered the inpainting function, instead of supporting initial multimodal sketch-prompt control. 

\begin{figure*}[t]
    \centering
    \includegraphics[width=0.9\textwidth]{src/img/novice_expert.pdf}
    \vspace{-2mm}
    \caption{The comparison between novice and expert participants in painting reveals that experts produce more accurate and fine-grained sketches, resulting in closer alignment with reference images in close-ended tasks. Conversely, in open-ended tasks, expert fine-grained strokes fail to generate precise results due to \tool's lack of control at the thin stroke level.}
    \Description{The comparison between novice and expert participants in painting reveals that experts produce more accurate and fine-grained sketches, resulting in closer alignment with reference images in close-ended tasks. Novice users create rougher sketches with less accuracy in shape. Conversely, in open-ended tasks, expert fine-grained strokes fail to generate precise results due to \tool's lack of control at the thin stroke level, while novice users' broader strokes yield results more aligned with their sketches.}
    \label{fig:novice_expert}
    % \vspace{-3mm}
\end{figure*}


% In particular, in the initial control input, users are unable to explicitly specify multi-modal generation intents.
In another example, PromptPaint~\cite{promptpaint} stresses the importance of paint-medium-like interactions and introduces Prompt stencil functions that allow users to perform fine-grained controls with localized image generation. 
However, insufficient spatial control (\eg, PromptPaint only allows for single-object prompt stencil at a time) and unstable models can still leave some users feeling the uncertainty of AI and a varying degree of ownership of the generated artwork~\cite{promptpaint}.
% As a result, the gap between intuitive multi-modal or paint-medium-like control and the current prompting interface still exists, which requires further research on multi-modal prompting interactions.
From this perspective, our work seeks to further enhance multi-object spatial-semantic prompting control by users' natural sketching.
However, there are still some challenges to be resolved, such as consistent multi-object generation in multiple rounds to increase stability and improved understanding of user sketches.   


% \new{
% From this perspective, our work is a step forward in this direction by allowing multi-object spatial-semantic prompting control by users' natural sketching, which considers the interplay between multiple sketch regions.
% % To further advance the multi-modal prompting experience, there are some aspects we identify to be important.
% % One of the important aspects is enhancing the consistency and stability of multiple rounds of generation to reduce the uncertainty and loss of control on users' part.
% % For this purpose, we need to develop techniques to incorporate consistent generation~\cite{tewel2024training} into multi-modal prompting framework.}
% % Another important aspect is improving generative models' understanding of the implicit user intents \new{implied by the paint-medium-like or sketch-based input (\eg, sketch of two people with their hands slightly overlapping indicates holding hand without needing explicit prompt).
% % This can facilitate more natural control and alleviate users' effort in tuning the textual prompt.
% % In addition, it can increase users' sense of ownership as the generated results can be more aligned with their sketching intents.
% }
% For example, when users draw sketches of two people with their hands slightly overlapping, current region-based models cannot automatically infer users' implicit intention that the two people are holding hands.
% Instead, they still require users to explicitly specify in the prompt such relationship.
% \tool addresses this through sketch-aware prompt recommendation to fill in the necessary semantic information, alleviating users' workload.
% However, some users want the generative AI in the future to be able to directly infer this natural implicit intentions from the sketches without additional prompting since prompt recommendation can still be unstable sometimes.


% \new{
% Besides visual generation, 
% }
% For example, one of the important aspect is referring~\cite{he2024multi}, linking specific text semantics with specific spatial object, which is partly what we do in our sketch-aware prompt recommendation.
% Analogously, in natural communication between humans, text or audio alone often cannot suffice in expressing the speakers' intentions, and speakers often need to refer to an existing spatial object or draw out an illustration of her ideas for better explanation.
% Philosophically, we HCI researchers are mostly concerned about the human-end experience in human-AI communications.
% However, studies on prompting is unique in that we should not just care about the human-end interaction, but also make sure that AI can really get what the human means and produce intention-aligned output.
% Such consideration can drastically impact the design of prompting interactions in human-AI collaboration applications.
% On this note, although studies on multi-modal interactions is a well-established topic in HCI community, it remains a challenging problem what kind of multi-modal information is really effective in helping humans convey their ideas to current and next generation large AI models.




\subsection{Novice Performance vs. Expert Performance}\label{sec:nVe}
In this section we discuss the performance difference between novice and expert regarding experience in painting and prompting.
First, regarding painting skills, some participants with experience (4/12) preferred to draw accurate and fine-grained shapes at the beginning. 
All novice users (5/12) draw rough and less accurate shapes, while some participants with basic painting skills (3/12) also favored sketching rough areas of objects, as exemplified in Figure~\ref{fig:novice_expert}.
The experienced participants using fine-grained strokes (4/12, none of whom were experienced in prompting) achieved higher IoU scores (0.557) in the close-ended task (0.535) when using \tool. 
This is because their sketches were closer in shape and location to the reference, making the single object decomposition result more accurate.
Also, experienced participants are better at arranging spatial location and size of objects than novice participants.
However, some experienced participants (3/12) have mentioned that the fine-grained stroke sometimes makes them frustrated.
As P1's comment for his result in open-ended task: "\emph{It seems it cannot understand thin strokes; even if the shape is accurate, it can only generate content roughly around the area, especially when there is overlapping.}" 
This suggests that while \tool\ provides rough control to produce reasonably fine results from less accurate sketches for novice users, it may disappoint experienced users seeking more precise control through finer strokes. 
As shown in the last column in Figure~\ref{fig:novice_expert}, the dragon hovering in the sky was wrongly turned into a standing large dragon by \tool.

Second, regarding prompting skills, 3 out of 12 participants had one or more years of experience in T2I prompting. These participants used more modifiers than others during both T2I and R2I tasks.
Their performance in the T2I (0.335) and R2I (0.469) tasks showed higher scores than the average T2I (0.314) and R2I (0.418), but there was no performance improvement with \tool\ between their results (0.508) and the overall average score (0.528). 
This indicates that \tool\ can assist novice users in prompting, enabling them to produce satisfactory images similar to those created by users with prompting expertise.



\subsection{Applicability of \tool}
The feedback from user study highlighted several potential applications for our system. 
Three participants (P2, P6, P8) mentioned its possible use in commercial advertising design, emphasizing the importance of controllability for such work. 
They noted that the system's flexibility allows designers to quickly experiment with different settings.
Some participants (N = 3) also mentioned its potential for digital asset creation, particularly for game asset design. 
P7, a game mod developer, found the system highly useful for mod development. 
He explained: "\emph{Mods often require a series of images with a consistent theme and specific spatial requirements. 
For example, in a sacrifice scene, how the objects are arranged is closely tied to the mod's background. It would be difficult for a developer without professional skills, but with this system, it is possible to quickly construct such images}."
A few participants expressed similar thoughts regarding its use in scene construction, such as in film production. 
An interesting suggestion came from participant P4, who proposed its application in crime scene description. 
She pointed out that witnesses are often not skilled artists, and typically describe crime scenes verbally while someone else illustrates their account. 
With this system, witnesses could more easily express what they saw themselves, potentially producing depictions closer to the real events. "\emph{Details like object locations and distances from buildings can be easily conveyed using the system}," she added.

% \subsection{Model Understanding of Users' Implicit Intents}
% In region-sketch-based control of generative models, a significant gap between interaction design and actual implementation is the model's failure in understanding users' naturally expressed intentions.
% For example, when users draw sketches of two people with their hands slightly overlapping, current region-based models cannot automatically infer users' implicit intention that the two people are holding hands.
% Instead, they still require users to explicitly specify in the prompt such relationship.
% \tool addresses this through sketch-aware prompt recommendation to fill in the necessary semantic information, alleviating users' workload.
% However, some users want the generative AI in the future to be able to directly infer this natural implicit intentions from the sketches without additional prompting since prompt recommendation can still be unstable sometimes.
% This problem reflects a more general dilemma, which ubiquitously exists in all forms of conditioned control for generative models such as canny or scribble control.
% This is because all the control models are trained on pairs of explicit control signal and target image, which is lacking further interpretation or customization of the user intentions behind the seemingly straightforward input.
% For another example, the generative models cannot understand what abstraction level the user has in mind for her personal scribbles.
% Such problems leave more challenges to be addressed by future human-AI co-creation research.
% One possible direction is fine-tuning the conditioned models on individual user's conditioned control data to provide more customized interpretation. 

% \subsection{Balance between recommendation and autonomy}
% AIGC tools are a typical example of 
\subsection{Progressive Sketching}
Currently \tool is mainly aimed at novice users who are only capable of creating very rough sketches by themselves.
However, more accomplished painters or even professional artists typically have a coarse-to-fine creative process. 
Such a process is most evident in painting styles like traditional oil painting or digital impasto painting, where artists first quickly lay down large color patches to outline the most primitive proportion and structure of visual elements.
After that, the artists will progressively add layers of finer color strokes to the canvas to gradually refine the painting to an exquisite piece of artwork.
One participant in our user study (P1) , as a professional painter, has mentioned a similar point "\emph{
I think it is useful for laying out the big picture, give some inspirations for the initial drawing stage}."
Therefore, rough sketch also plays a part in the professional artists' creation process, yet it is more challenging to integrate AI into this more complex coarse-to-fine procedure.
Particularly, artists would like to preserve some of their finer strokes in later progression, not just the shape of the initial sketch.
In addition, instead of requiring the tool to generate a finished piece of artwork, some artists may prefer a model that can generate another more accurate sketch based on the initial one, and leave the final coloring and refining to the artists themselves.
To accommodate these diverse progressive sketching requirements, a more advanced sketch-based AI-assisted creation tool should be developed that can seamlessly enable artist intervention at any stage of the sketch and maximally preserve their creative intents to the finest level. 

\subsection{Ethical Issues}
Intellectual property and unethical misuse are two potential ethical concerns of AI-assisted creative tools, particularly those targeting novice users.
In terms of intellectual property, \tool hands over to novice users more control, giving them a higher sense of ownership of the creation.
However, the question still remains: how much contribution from the user's part constitutes full authorship of the artwork?
As \tool still relies on backbone generative models which may be trained on uncopyrighted data largely responsible for turning the sketch into finished artwork, we should design some mechanisms to circumvent this risk.
For example, we can allow artists to upload backbone models trained on their own artworks to integrate with our sketch control.
Regarding unethical misuse, \tool makes fine-grained spatial control more accessible to novice users, who may maliciously generate inappropriate content such as more realistic deepfake with specific postures they want or other explicit content.
To address this issue, we plan to incorporate a more sophisticated filtering mechanism that can detect and screen unethical content with more complex spatial-semantic conditions. 
% In the future, we plan to enable artists to upload their own style model

% \subsection{From interactive prompting to interactive spatial prompting}


\subsection{Limitations and Future work}

    \textbf{User Study Design}. Our open-ended task assesses the usability of \tool's system features in general use cases. To further examine aspects such as creativity and controllability across different methods, the open-ended task could be improved by incorporating baselines to provide more insightful comparative analysis. 
    Besides, in close-ended tasks, while the fixing order of tool usage prevents prior knowledge leakage, it might introduce learning effects. In our study, we include practice sessions for the three systems before the formal task to mitigate these effects. In the future, utilizing parallel tests (\textit{e.g.} different content with the same difficulty) or adding a control group could further reduce the learning effects.

    \textbf{Failure Cases}. There are certain failure cases with \tool that can limit its usability. 
    Firstly, when there are three or more objects with similar semantics, objects may still be missing despite prompt recommendations. 
    Secondly, if an object's stroke is thin, \tool may incorrectly interpret it as a full area, as demonstrated in the expert results of the open-ended task in Figure~\ref{fig:novice_expert}. 
    Finally, sometimes inclusion relationships (\textit{e.g.} inside) between objects cannot be generated correctly, partially due to biases in the base model that lack training samples with such relationship. 

    \textbf{More support for single object adjustment}.
    Participants (N=4) suggested that additional control features should be introduced, beyond just adjusting size and location. They noted that when objects overlap, they cannot freely control which object appears on top or which should be covered, and overlapping areas are currently not allowed.
    They proposed adding features such as layer control and depth control within the single-object mask manipulation. Currently, the system assigns layers based on color order, but future versions should allow users to adjust the layer of each object freely, while considering weighted prompts for overlapping areas.

    \textbf{More customized generation ability}.
    Our current system is built around a single model $ColorfulXL-Lightning$, which limits its ability to fully support the diverse creative needs of users. Feedback from participants has indicated a strong desire for more flexibility in style and personalization, such as integrating fine-tuned models that cater to specific artistic styles or individual preferences. 
    This limitation restricts the ability to adapt to varied creative intents across different users and contexts.
    In future iterations, we plan to address this by embedding a model selection feature, allowing users to choose from a variety of pre-trained or custom fine-tuned models that better align with their stylistic preferences. 
    
    \textbf{Integrate other model functions}.
    Our current system is compatible with many existing tools, such as Promptist~\cite{hao2024optimizing} and Magic Prompt, allowing users to iteratively generate prompts for single objects. However, the integration of these functions is somewhat limited in scope, and users may benefit from a broader range of interactive options, especially for more complex generation tasks. Additionally, for multimodal large models, users can currently explore using affordable or open-source models like Qwen2-VL~\cite{qwen} and InternVL2-Llama3~\cite{llama}, which have demonstrated solid inference performance in our tests. While GPT-4o remains a leading choice, alternative models also offer competitive results.
    Moving forward, we aim to integrate more multimodal large models into the system, giving users the flexibility to choose the models that best fit their needs. 
    


\section{Conclusion}\label{sec:conclusion}
In this paper, we present \tool, an interactive system designed to help novice users create high-quality, fine-grained images that align with their intentions based on rough sketches. 
The system first refines the user's initial prompt into a complete and coherent one that matches the rough sketch, ensuring the generated results are both stable, coherent and high quality.
To further support users in achieving fine-grained alignment between the generated image and their creative intent without requiring professional skills, we introduce a decompose-and-recompose strategy. 
This allows users to select desired, refined object shapes for individual decomposed objects and then recombine them, providing flexible mask manipulation for precise spatial control.
The framework operates through a coarse-to-fine process, enabling iterative and fine-grained control that is not possible with traditional end-to-end generation methods. 
Our user study demonstrates that \tool offers novice users enhanced flexibility in control and fine-grained alignment between their intentions and the generated images.


%%%%%%%%% REFERENCES
{
    \small
    \bibliographystyle{IEEEtran}
    \bibliography{refs}
}

\end{document}