\section{Introduction} \label{introduction}
Cardiac diagnostics have undergone remarkable advancements over the centuries, yet the analysis of heart sounds remains a fundamental aspect of assessing cardiovascular health. These sounds, primarily associated with the closure of heart valves, offer crucial insights into cardiac function. The first heart sound (S1), produced by the closure of the atrioventricular valves, and the second heart sound (S2), associated with the closure of the semilunar valves, are commonly recognized as the characteristic 'lub' and 'dub.' In a healthy heart, these sounds provide clear indications of proper valve function. Over time, the practice of analyzing these sounds has become an indispensable non-invasive tool in medical diagnostics, offering a reliable means to detect abnormalities and evaluate cardiac performance.

\subsection{Anomalies in the Heart Sounds}
Heart murmurs, extra heart sounds, and extrasystoles are common anomalies detected during cardiac auscultation. Each type of anomaly provides valuable information about potential underlying cardiac conditions.

\textbf{Heart murmurs} are among the most common anomalies detected through auscultation. Produced by turbulent blood flow strong enough to generate audible noise, these "whooshing" sounds can be heard in various scenarios, including some healthy individuals. While innocent murmurs, also known as functional or benign murmurs, are typically harmless and not associated with structural heart abnormalities, pathologic murmurs indicate underlying conditions such as valve defects, congenital heart defects, or abnormal blood flow patterns. 

\textbf{Extra Heart Sounds} refer to additional heart sounds beyond the normal "lub-dub" pattern. These may manifest as "lub-lub dub" or "lub dub-dub" sequences. Extra heart sounds can sometimes indicate underlying conditions, although they may also occur in healthy individuals. Detection of these sounds is crucial as they may not be easily identified through other diagnostic tools like ultrasound.

\textbf{Extrasystole} involves irregular heart rhythms, typically presenting as extra or skipped heartbeats. These can be heard as sequences such as "lub-lub dub" or "lub dub-dub." While extrasystoles may be benign, they can also signify underlying heart disease, making early detection important for effective treatment.

A detailed explanation of these anomalies can be found in Table \ref{tab:heart_anomalies}.

\subsection{Organization of the Paper}
This paper is organized as follows: Section \ref{background} covers the background of the study, providing the necessary theoretical foundation. \ref{related_works} provides an in-depth review of related work, discussing the advancements and methodologies in cardiac sound analysis and the integration of machine learning models in cardiovascular diagnostics. Section \ref{proposed approach} details the proposed approach, including the data preprocessing techniques, the generation of audiovisual data, and the model architectures. Section \ref{experiments} describes the experimental setup, including the dataset used, training procedures, and evaluation metrics. The results of the experiments are presented in \ref{results}, highlighting the performance of individual models and the ensemble method. Finally, Section \ref{conclusion} concludes the paper with a summary of findings, potential implications for clinical practice, and directions for future research.

\subsection{Contributions}
The primary impacts of the proposed experiment are the following:
\begin{itemize}
    \item We developed \ENACT (See Fig. \ref{fig:banner}) -- a novel transformer-based ensemble method specifically designed for diagnosing medical audio data through advanced visualization techniques. \ENACT demonstrates state-of-the-art performance, surpassing other existing ensemble methods in the field.
    \item We explored the feasibility of employing the Mixture of Experts (MoE) approach across different AI architectures, combining CNN and ViT. This integration effectively leverages multiple image modalities derived from the same input data, enhancing diagnostic accuracy.
    \item We applied a ViT model to medical time-series sound data by converting it into audiovisual representations. This innovative approach, still an active area of research, opens new avenues for analyzing and interpreting complex medical signals.

\end{itemize}

\begin{table}[t]
\caption{Types of anomalies in heart sound \cite{malik_multi-classification_2022}.}
\begin{tabular}{|p{0.08\textwidth}|p{0.36\textwidth}|}
\hline
\centering\textbf{Category} & \ \ \textbf{Description} \\
\hline
\rule{0pt}{4ex} \centering Normal & 
\begin{itemize}
    \vspace{-0.4cm} 
    \item Healthy heart sounds with a clear "lub dub" pattern.
    \item May contain background noises and occasional random noise.
    \vspace{-0.2cm}
\end{itemize} \\
\hline
\rule{0pt}{4ex} \centering Murmur & 
\begin{itemize}
    \vspace{-0.4cm}
    \item Abnormal heart sounds with a "whooshing, roaring, rumbling, or turbulent fluid" noise between "lub" and "dub", or between "dub" and "lub".
    \item "Lub" and "dub" are still present.
    \item Murmurs do not occur directly on "lub" or "dub".
    \vspace{-0.2cm}
\end{itemize} \\
\hline
\rule{0pt}{4ex} \centering Extra Heart Sound & 
\begin{itemize}
    \vspace{-0.4cm}
    \item Additional heart sounds such as "lub-lub dub" or "lub dub-dub".
    \item May or may not be a sign of disease.
    \item Important to detect as it may not be detected well by ultrasound.
    \vspace{-0.2cm}
\end{itemize} \\
% \hline
% \rule{0pt}{4ex} \centering Artifact & 
% \begin{itemize}
%     \vspace{-0.4cm}
%     \item Wide range of different sounds including feedback squeals, echoes, speech, music, and noise.
%     \item No discernable heart sounds.
%     \item Little or no temporal periodicity below 195 Hz.
%     \vspace{-0.2cm}
% \end{itemize} \\
\hline
\rule{0pt}{2ex} \centering Extrasystole & 
\begin{itemize}
    \vspace{-0.2cm}
    \item Heart sounds out of rhythm involving extra or skipped heartbeats, such as "lub-lub dub" or "lub dub-dub".
    \item May or may not be a sign of disease.
    \item Treatment is likely to be more effective if diseases are detected earlier.
    \vspace{-0.2cm}
\end{itemize} \\
\hline
\end{tabular}
\label{tab:heart_anomalies}
\end{table}