\section{Conclusion and Discussion} \label{conclusion}
In conclusion, the combination of ViT and CNN models using an ensemble method improved the classification performance in every aspect in general. This study highlights the importance of evaluating individual models to identify their strengths and the potential benefits of using ensemble methods to achieve superior results. 

The data augmentation techniques employed also played a key role in enhancing model robustness and performance. These findings can inform future research and development of advanced classification systems in the medical field.

The proposed \ENACT model demonstrates significant promise in the field of heart sound classification, particularly when considered alongside the advancements in smart wearable devices. With the proliferation of wearable technology, the collection and analysis of audio data have become more accessible and widespread. This is especially pertinent in the medical field, where heart sound data can be continuously monitored and analyzed in real-time, offering invaluable insights into a patient's cardiovascular health.

Moreover, from a practical point of view, the advancement of smart wearable devices presents a significant opportunity for improving healthcare accessibility, especially in low-resource settings. In many developing countries, access to advanced medical diagnostics is limited due to the lack of infrastructure and trained healthcare professionals. Wearable devices equipped with advanced models like \ENACT can bridge this gap by enabling non-invasive, continuous monitoring of heart health, thus providing timely and accurate diagnostics without the need for expensive and bulky equipment.

This technology can revolutionize the practice of medicine in poorer regions, making high-quality healthcare more approachable and affordable. The ability to monitor and analyze heart sounds continuously can lead to early detection of cardiovascular issues, prompt intervention, and ultimately, better health outcomes. As wearable devices become more affordable and their usage more prevalent, the integration of sophisticated models like \ENACT can play a crucial role in democratizing access to advanced medical diagnostics globally.

In summary, the synergy between the \ENACT model and smart wearable technology holds great potential for enhancing healthcare delivery, particularly in underserved regions. By providing a reliable and efficient means of heart sound classification, this approach not only advances the field of medical diagnostics but also contributes to the broader goal of equitable healthcare access.