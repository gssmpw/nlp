\section{Results} \label{results}

\subsection{Individual Model Performance}

In this study, we evaluated the performance of two state-of-the-art models, ViT and CNN, on a dataset of heart sound recordings. The goal was to classify the recordings into five categories: artifact, extrahls, extrastole, murmur, and normal.

\textbf{ViT}. Although the ViT model demonstrated strong performance across most classes, it was outperformed by the CNN in several key areas. This discrepancy is primarily due to the repetitive nature of heart sounds, which consist of recurring local patterns that CNNs are particularly adept at capturing and analyzing. Consequently, CNN's ability to effectively recognize these local patterns contributed to its superior performance in this context. This, however, doesn't mean the usage of ViT model is futile, as ViT might have identified characteristics that is not evident through the CNN model.

\textbf{CNN.} The CNN model, on the other hand, achieved higher overall precision and demonstrated more balanced performance across all classes. This proves the point mentioned earlier: CNN on centroids can provide a more robust model in comparison to ViT.


\subsection{Ensemble Model Performance}

To leverage the strengths of both models, we implemented an ensemble method by combining the predictions of the ViT and CNN models using a Mixture of Experts approach. The ensemble was created using an additive weighted approach, where different weights were tested to find the optimal combination.

\ENACT achieved the highest accuracy of 97.52\%, significantly outperforming both individual models. The improvement in accuracy demonstrates the effectiveness of the ensemble approach, particularly in enhancing the model's robustness and generalization capabilities. The ensemble method effectively combined the strengths of both ViT and CNN. The effectiveness of the \ENACT compared to the individual ViT and CNN models is illustrated in Table \ref{tab:perf_comp}. Additionally, the performance of the \ENACT, as illustrated by its confusion matrix in Fig. \ref{fig:conf_mat}, further demonstrates its statistics over different types of diseases.

\begin{figure}[t]
    \centering
    \vspace{-0.5em}
    \includegraphics[width=0.45\textwidth]{images/confusion_matrix.png}
    \caption{\centering Confusion matrix of the performance of \ENACT}
    \vspace{-0.5em}
    \label{fig:conf_mat}
\end{figure}

\begin{table}[ht!]
\caption{UPL performance summary: Comparison of UPL in terms of the number of \textcolor{red}{Best}\textbar Second rank performance.}
\begin{center}
\begin{scriptsize}
\setlength{\columnsep}{1pt}%
\resizebox{0.4\linewidth}{!}{
\begin{tabular}{@{\extracolsep{1pt}}lc|c|c@{}}
\toprule 
Datasets Types & GCN & GAT & SAGE  \\ 
\cline{1-4}
Homophilic & \textcolor{red}{$9$}$|2$ & \textcolor{red}{$10$}$|1$ & \textcolor{red}{$11$}$|0$ 
\\
% \cline{2-4}
Heterophilic&\textcolor{red}{$7$}$|2$ & \textcolor{red}{$9$}$|0$ & \textcolor{red}{$7$}$|2$
\\
\cdashline{1-4}
Total & \textcolor{red}{$16$}$|4$ & \textcolor{red}{$19$}$|1$ & \textcolor{red}{$18$}$|2$ \\
\bottomrule

\end{tabular}
}


\end{scriptsize}
\end{center}
\label{tb:performance_comparison}
\vspace{-0.1in}
\end{table}

\subsection{Comparison with State-of-the-Art}
The proposed \ENACT demonstrates competitive performance when compared to other state-of-the-art models in heart sound classification. Utilizing a MoE ensemble approach that integrates ViT and CNN, the \ENACT achieves an impressive accuracy of 97.52\%. This surpasses the accuracy of individual ViT (93.88\%) and CNN (95.45\%) models. Additionally, the \ENACT maintains high precision (0.98), recall (0.97), and F1-score (0.98), indicating a balanced performance across various metrics.


In comparison, other notable studies in the field exhibit slightly different accuracies. For instance, the work by Liu et al. \cite{liu_heart_2023} utilizing bispectrum features and ViT reported an accuracy of 91\%, while Yang et al. \cite{yang_assisting_2023} achieved an accuracy of 98.74\% using a combination of Transformer and CNN models. Similarly, Wang et al. \cite{wang_pctmf-net_2023} presented the PCTMF-Net model, which recorded an accuracy of 99.36\% on the Yansen dataset but only 93\% on the PhysioNet Challenge dataset, highlighting variability across different datasets. Furthermore, Jumphoo et al. \cite{jumphoo_exploiting_2024} reported an accuracy of 99.44\% with Conv-DeiT, and their precision, recall, and F1-score metrics are comparable to those of the \textit{ENACT-Heart}.

As summarized in Table \ref{tab:sota_comp}, while some studies report higher accuracies, the balanced performance of \ENACT across multiple metrics underscores its reliability and robustness in heart sound classification tasks. The high accuracy and comprehensive performance metrics indicate the potential of the MoE ensemble method in enhancing diagnostic accuracy and reliability in cardiovascular health monitoring.

\begin{table}[t]
    \centering
    \caption{\centering Comparison of other State-of-the-Art MoE heart sound classification studies.}
    \resizebox{0.48\textwidth}{!}{
    \begin{tabular}{>{\raggedright\arraybackslash}p{3cm} >{\raggedright\arraybackslash}p{2cm} >{\raggedright\arraybackslash}p{2cm} >{\raggedright\arraybackslash}p{2.5cm}}
        \toprule
        \textbf{Study} & \textbf{Authors} & \textbf{Model(s) Used} & \textbf{Metrics} \\ \midrule
        \textbf{ENACT-Heart \newline(Proposed Model)} & J. Han, \newline A. Shaout & MoE\newline (Ensemble of ViT \& CNN) & \textbf{Accuracy}: 0.9752 \newline (PASCAL DB)\newline \textbf{Precision}: 0.98 \newline \textbf{Recall}: 0.97 \newline \textbf{F1-Score} : 0.98 \\ \midrule
        \textbf{Heart sound classification based on bispectrum features and Vision Transformer model (Nov. 2023) \cite{liu_heart_2023}} & Z. Liu, H. Jiang,\newline F. Zhang,\newline W. Ouyang, \newline X. Li & ViT, CNN & \textbf{Accuracy}: 0.91 \newline AUC: 0.98\\ \midrule
        \textbf{Assisting Heart Valve Diseases Diagnosis via Transformer-Based Classification \newline (May. 2023) \cite{yang_assisting_2023}} & D. Yang, Y. Lin,\newline J. Wei, X. Lin,\newline X. Zhao, Y. Yao & Transformer, CNN & \textbf{Accuracy}: 0.9874 \newline AUC: 0.99 \\ \midrule
        \textbf{PCTMF-Net: heart sound classification with parallel CNNs-transformer and spectral analysis \newline (Jul. 2023) \cite{wang_pctmf-net_2023}} & R. Wang, \newline Y. Duan, \newline Y. Li, D. Zheng,\newline X. Liu, C.T. Lam & CNN, Transformer & \textbf{Accuracy}: 0.9936 \newline (Yansen Dataset) \newline 0.93 (PhysioNet Challenge) \\ \midrule
        \textbf{Exploring Data-Efficient Image Transformer-based Transfer Learning \newline (Jan. 2024) \cite{jumphoo_exploiting_2024}} & T. Jumphoo, \newline K. Phapatanaburi, \newline W. Pathonsuwan & Conv-DeiT & \textbf{Accuracy}: 0.9944 \newline \textbf{Precision}: 0.9852  \newline \textbf{Recall}: 0.9854 \newline \textbf{F1-Score}: 0.9851 \\ \midrule
        \textbf{Heart Sound Classification Network Based on Convolution and Transformer (Aug. 2023) \cite{cheng_heart_2023}} & J. Cheng, \newline K. Sun & CNN, Transformer & \textbf{Accuracy}: 0.964  \newline 0.997 \newline 0.957 (3 distinct dataset) \\ \midrule
        \textbf{Multi-classification neural network model for detection of abnormal heartbeat audio signals \newline (Jul. 2022) \cite{malik_multi-classification_2022}} & H. Malik, \newline U. Bashir,\newline A. Ahmad & RNN \newline LSTM & \textbf{Accuracy}: 0.99771 (PASCAL DB) \newline 0.9870 (PhysioNet Challenge) \\ \bottomrule
    \end{tabular}
    }
    \vspace{-1em}
    \label{tab:sota_comp}
\end{table}

