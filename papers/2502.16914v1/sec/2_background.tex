\section{Historical Background} \label{background}
The diagnosis of heart conditions dates back to the early days of medicine, where physicians relied on palpation and pulse assessment to detect abnormalities. A significant breakthrough occurred in the 1700s when Jean Baptiste de Senac, physician to King Louis XV of France, established the connection between atrial fibrillation and mitral valve disease. Senac's work laid the foundation for cardiology as a distinct field of study \cite{mcmichael_history_1982}.

The invention of the stethoscope by René Laennec in 1816 marked a pivotal moment in cardiac diagnostics. Laennec introduced the technique of "mediate auscultation" using his newly created paper acoustic device, allowing for more accurate detection of heart sounds and abnormalities \cite{laennec_treatise_nodate}. This innovation remains a cornerstone in the history of cardiology.

These early diagnostic methods, based on manual interpretation of heart sounds, evolved significantly over the centuries. With technological advancements, traditional auscultation has been augmented by electrocardiograms (ECGs) and other imaging modalities. In recent years, the integration of artificial intelligence (AI) and machine learning has opened new avenues for the analysis of heart sounds, leading to more precise and efficient diagnostic tools \cite{esbin_overcoming_2020, huang_fusion_2020}.