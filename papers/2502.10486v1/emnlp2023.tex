% This must be in the first 5 lines to tell arXiv to use pdfLaTeX, which is strongly recommended.
\pdfoutput=1
% In particular, the hyperref package requires pdfLaTeX in order to break URLs across lines.

\documentclass[11pt]{article}

% Remove the "review" option to generate the final version.
% \usepackage[review]{EMNLP2023}
\usepackage{EMNLP2023}
\usepackage{tabularx}
\usepackage{booktabs}
% Standard package includes
\usepackage{times}
\usepackage{latexsym}
\usepackage{amsmath,amsfonts,bm}
%\usepackage{newclude}

% For proper rendering and hyphenation of words containing Latin characters (including in bib files)
\usepackage[T1]{fontenc}
% For Vietnamese characters
% \usepackage[T5]{fontenc}
% See https://www.latex-project.org/help/documentation/encguide.pdf for other character sets

% This assumes your files are encoded as UTF8
\usepackage[utf8]{inputenc}

% This is not strictly necessary and may be commented out.
% However, it will improve the layout of the manuscript,
% and will typically save some space.
\usepackage{microtype}

% This is also not strictly necessary and may be commented out.
% However, it will improve the aesthetics of text in
% the typewriter font.
\usepackage{inconsolata}

\usepackage{amsmath}
\usepackage{multirow}
\usepackage{graphicx}
\usepackage{subfigure}

% tabel color
\usepackage{colortbl}
\definecolor{Ocean}{RGB}{129,194,234}

\usepackage{color, soul}
\definecolor{tri_red}{RGB}{187,39,26}
\definecolor{tri_blue}{RGB}{75,119,209}
\definecolor{tri_green}{RGB}{120,166,90}
\definecolor{pipeline_red}{RGB}{187,39,26}
\definecolor{pipeline_green}{RGB}{71,116,44}
\definecolor{table_ocean}{RGB}{229,242,250}
% \definecolor{table_ocean}{RGB}{232,242,254}
\sethlcolor{table_ocean}

\interfootnotelinepenalty=10000
% internal link
\usepackage{cleveref}
\crefformat{section}{\S#2#1#3}
\crefformat{subsection}{\S#2#1#3}
\crefformat{subsubsection}{\S#2#1#3}
\crefrangeformat{section}{\S#3#1#4 to~\S#5#2#6}
\crefmultiformat{section}{\S#2#1#3}{ and~\S#2#1#3}{, #2#1#3}{ and~#2#1#3}
\Crefformat{figure}{#2Fig.~#1#3}
\Crefmultiformat{figure}{Figs.~#2#1#3}{ and~#2#1#3}{, #2#1#3}{ and~#2#1#3}
\Crefformat{table}{#2Tab.~#1#3}
\Crefmultiformat{table}{Tabs.~#2#1#3}{ and~#2#1#3}{, #2#1#3}{ and~#2#1#3}
\Crefformat{appendix}{#2Appx.~\S#1#3}
\crefformat{algorithm}{Alg.~#2#1#3}
\Crefformat{equation}{#2Eq.~#1#3}

% special commands
\newcommand{\stitle}[1]{\vspace{1ex} \noindent{\bf #1.}}
\newcommand{\muhao}[1]{{\color{blue}[{MC:} #1]}}
\newcommand{\todo}[1]{{\color{red}[{TODO:} #1]}}
\newcommand{\fei}[1]{{\color{brown}[{FW:} #1]}}

\newcommand{\chaowei}[1]{{\color{brown}[{Chaowei:} #1]}}

\newcommand{\ucd}{\raisebox{4pt}{\includegraphics[]{figures/1F334.pdf}}}
\newcommand{\usc}{\raisebox{4pt}{\includegraphics[]{figures/1F335.pdf}}}
\newcommand{\uwm}{\raisebox{4pt}{\includegraphics[]{figures/1F341.pdf}}}


% =============== command for method name ==================
\usepackage{xspace}
\newcommand{\MODEL}{\mbox{\textsc{VLM-Guard}}\xspace}


% If the title and author information does not fit in the area allocated, uncomment the following
%
%\setlength\titlebox{<dim>}
%
% and set <dim> to something 5cm or larger.

\title{\MODEL: Safeguarding Vision-Language Models \\ via Fulfilling Safety Alignment Gap}

% Author information can be set in various styles:
% For several authors from the same institution:
% \author{Author 1 \and ... \and Author n \\
%         Address line \\ ... \\ Address line}
% if the names do not fit well on one line use
%         Author 1 \\ {\bf Author 2} \\ ... \\ {\bf Author n} \\
% For authors from different institutions:
% \author{Author 1 \\ Address line \\  ... \\ Address line
%         \And  ... \And
%         Author n \\ Address line \\ ... \\ Address line}
% To start a separate ``row'' of authors use \AND, as in
% \author{Author 1 \\ Address line \\  ... \\ Address line
%         \AND
%         Author 2 \\ Address line \\ ... \\ Address line \And
%         Author 3 \\ Address line \\ ... \\ Address line}

\author{
Qin Liu\ucd~~~
Fei Wang\usc~~~
Chaowei Xiao\uwm~~~
Muhao Chen\ucd\\
{\ucd}UC Davis;\;{\usc}USC;\;{\uwm}UW-Madison\\
\texttt{\{qinli, muhchen\}@ucdavis.edu};~~~fwang598@usc.edu;~~~
\texttt{cxiao34@wisc.edu}\\
  }

\begin{document}
\maketitle

\begin{abstract}


The choice of representation for geographic location significantly impacts the accuracy of models for a broad range of geospatial tasks, including fine-grained species classification, population density estimation, and biome classification. Recent works like SatCLIP and GeoCLIP learn such representations by contrastively aligning geolocation with co-located images. While these methods work exceptionally well, in this paper, we posit that the current training strategies fail to fully capture the important visual features. We provide an information theoretic perspective on why the resulting embeddings from these methods discard crucial visual information that is important for many downstream tasks. To solve this problem, we propose a novel retrieval-augmented strategy called RANGE. We build our method on the intuition that the visual features of a location can be estimated by combining the visual features from multiple similar-looking locations. We evaluate our method across a wide variety of tasks. Our results show that RANGE outperforms the existing state-of-the-art models with significant margins in most tasks. We show gains of up to 13.1\% on classification tasks and 0.145 $R^2$ on regression tasks. All our code and models will be made available at: \href{https://github.com/mvrl/RANGE}{https://github.com/mvrl/RANGE}.

\end{abstract}



\section{Introduction}

Video generation has garnered significant attention owing to its transformative potential across a wide range of applications, such media content creation~\citep{polyak2024movie}, advertising~\citep{zhang2024virbo,bacher2021advert}, video games~\citep{yang2024playable,valevski2024diffusion, oasis2024}, and world model simulators~\citep{ha2018world, videoworldsimulators2024, agarwal2025cosmos}. Benefiting from advanced generative algorithms~\citep{goodfellow2014generative, ho2020denoising, liu2023flow, lipman2023flow}, scalable model architectures~\citep{vaswani2017attention, peebles2023scalable}, vast amounts of internet-sourced data~\citep{chen2024panda, nan2024openvid, ju2024miradata}, and ongoing expansion of computing capabilities~\citep{nvidia2022h100, nvidia2023dgxgh200, nvidia2024h200nvl}, remarkable advancements have been achieved in the field of video generation~\citep{ho2022video, ho2022imagen, singer2023makeavideo, blattmann2023align, videoworldsimulators2024, kuaishou2024klingai, yang2024cogvideox, jin2024pyramidal, polyak2024movie, kong2024hunyuanvideo, ji2024prompt}.


In this work, we present \textbf{\ours}, a family of rectified flow~\citep{lipman2023flow, liu2023flow} transformer models designed for joint image and video generation, establishing a pathway toward industry-grade performance. This report centers on four key components: data curation, model architecture design, flow formulation, and training infrastructure optimization—each rigorously refined to meet the demands of high-quality, large-scale video generation.


\begin{figure}[ht]
    \centering
    \begin{subfigure}[b]{0.82\linewidth}
        \centering
        \includegraphics[width=\linewidth]{figures/t2i_1024.pdf}
        \caption{Text-to-Image Samples}\label{fig:main-demo-t2i}
    \end{subfigure}
    \vfill
    \begin{subfigure}[b]{0.82\linewidth}
        \centering
        \includegraphics[width=\linewidth]{figures/t2v_samples.pdf}
        \caption{Text-to-Video Samples}\label{fig:main-demo-t2v}
    \end{subfigure}
\caption{\textbf{Generated samples from \ours.} Key components are highlighted in \textcolor{red}{\textbf{RED}}.}\label{fig:main-demo}
\end{figure}


First, we present a comprehensive data processing pipeline designed to construct large-scale, high-quality image and video-text datasets. The pipeline integrates multiple advanced techniques, including video and image filtering based on aesthetic scores, OCR-driven content analysis, and subjective evaluations, to ensure exceptional visual and contextual quality. Furthermore, we employ multimodal large language models~(MLLMs)~\citep{yuan2025tarsier2} to generate dense and contextually aligned captions, which are subsequently refined using an additional large language model~(LLM)~\citep{yang2024qwen2} to enhance their accuracy, fluency, and descriptive richness. As a result, we have curated a robust training dataset comprising approximately 36M video-text pairs and 160M image-text pairs, which are proven sufficient for training industry-level generative models.

Secondly, we take a pioneering step by applying rectified flow formulation~\citep{lipman2023flow} for joint image and video generation, implemented through the \ours model family, which comprises Transformer architectures with 2B and 8B parameters. At its core, the \ours framework employs a 3D joint image-video variational autoencoder (VAE) to compress image and video inputs into a shared latent space, facilitating unified representation. This shared latent space is coupled with a full-attention~\citep{vaswani2017attention} mechanism, enabling seamless joint training of image and video. This architecture delivers high-quality, coherent outputs across both images and videos, establishing a unified framework for visual generation tasks.


Furthermore, to support the training of \ours at scale, we have developed a robust infrastructure tailored for large-scale model training. Our approach incorporates advanced parallelism strategies~\citep{jacobs2023deepspeed, pytorch_fsdp} to manage memory efficiently during long-context training. Additionally, we employ ByteCheckpoint~\citep{wan2024bytecheckpoint} for high-performance checkpointing and integrate fault-tolerant mechanisms from MegaScale~\citep{jiang2024megascale} to ensure stability and scalability across large GPU clusters. These optimizations enable \ours to handle the computational and data challenges of generative modeling with exceptional efficiency and reliability.


We evaluate \ours on both text-to-image and text-to-video benchmarks to highlight its competitive advantages. For text-to-image generation, \ours-T2I demonstrates strong performance across multiple benchmarks, including T2I-CompBench~\citep{huang2023t2i-compbench}, GenEval~\citep{ghosh2024geneval}, and DPG-Bench~\citep{hu2024ella_dbgbench}, excelling in both visual quality and text-image alignment. In text-to-video benchmarks, \ours-T2V achieves state-of-the-art performance on the UCF-101~\citep{ucf101} zero-shot generation task. Additionally, \ours-T2V attains an impressive score of \textbf{84.85} on VBench~\citep{huang2024vbench}, securing the top position on the leaderboard (as of 2025-01-25) and surpassing several leading commercial text-to-video models. Qualitative results, illustrated in \Cref{fig:main-demo}, further demonstrate the superior quality of the generated media samples. These findings underscore \ours's effectiveness in multi-modal generation and its potential as a high-performing solution for both research and commercial applications.

% \section{Introduction}

% % 1. Background: what are backdoors in NLP? Why are they harmful? why should we defend?

% % 2. Existing methods, why do they commonly fall short?
% % 2.1 Invisible triggers (with examples).
% % 2.2 Diverse triggers.
% % (perhaps) 2.3 some methods even need a clean dev set, which is not realistic.

% % 3. Inspired by PoE for spurious correlation mitigation in model debaising, we propose ...
% % 3.1 the commonality (and a bit about the difference) between shortcuts and triggers.
% % 3.2 however, additional challenges (correspond to the techniques you propose):
%      % 3.2.1 co-efficient between bias and main models (leads to psuedo dev set selection)
%      % 3.2.2 diverse trigger -> MoE of bias models (to be decided)
%      % 3.2.3 noisy inductive bias (not easy to explain, but may try)

% % 4. In this paper, we propose ... (can write later)
% % 4.1 composition of the method (which techniques are components are incorporated?)
% % 4.2 results and observations

% % 5. the technical contributions are 2/3/4-folds. First: proposing PoE for handling invisible and diverse backdoor triggers. Second: (something related to key techniques that necessarily adapt PoE for backdoor defense). Third: the first method that effectively handle mix triggers, and is generally robust and potent.




\section{Methodology}

\subsection{Problem Definition}

Given a multivariate time series input $X \in \mathbb{R}^{C  \times T}$, multivariate time series forecasting tasks are designed to predict its future $F$ time steps $\hat{Y}\in \mathbb{R}^{C \times F}$ using past $T$ steps. $C $ is the number of variates or channels.

\subsection{Preliminary Analysis}

This section presents why RevIN~\citep{Kim_revin,liu2022non}, High-pass, and Low-pass filters fail to address the Mid-Frequency Spectrum Gap. Let the input univariate time series be $ x(t) $ with length $ T $ and target $ y(t) $ with length $ F $. 

\begin{definition}[Frequency Spectral Energy]\label{def:energy}
The Fourier transform of $x(t)$, $X(f)$, and its spectral energy $E_X(f)$ is given by:
\vspace{-0.2cm}
\begin{align}
X(f) = \sum_{t=0}^{T-1} x(t) e^{-i 2 \pi f t / {T-1}}, \quad &f = 0, 1, \dots, T-1\notag\\
E_X(f) = |X(f)|^2.
\end{align}
\vspace{-0.2cm}
\end{definition}

\textbf{Impact of RevIN on Frequency Spectrum \quad}
\begin{definition}[Reversible Instance Normalization]\label{def:RevIN}
Given a \textbf{forecast model} $ f: \mathbb{R}^T \rightarrow \mathbb{R}^F $ that generates a forecast $ \hat{y}(t) $ from a given input $x(t)$, RevIN is defined as:
\vspace{-0.2cm}
\begin{align}
&\hat{x}(t) = \frac{x(t) - \mu}{\sigma},\quad t = 0, 1, \dots, T-1\notag\\
&\hat{y}(t) = f(\hat{x}(t)), \quad \hat{y}(t)_{rev}= \hat{y}(t) \cdot \sigma + \mu,\notag\\
&\mu = \frac{1}{T} \sum_{t=0}^{T-1} x(t), \quad \sigma = \sqrt{\frac{1}{T} \sum_{t=0}^{T-1} (x(t) - \mu)^2}.
\end{align}
\vspace{-0.2cm}
\end{definition}

\begin{theorem} [Frequency Spectrum after RevIN] \label{theorem:RevIN}
\vspace{-0.2cm}
The spectral energy of $\hat{x}(t)$ (transformed using RevIN):
\begin{align}
E_{\hat{X}}(0)=0,& \quad f=0, \notag\\
E_{\hat{X}}(f) = \left( \frac{1}{\sigma} \right)^2 |X(f)|^2,&\quad f = 1,2,\dots, T-1 . 
\end{align}
\vspace{-0.2cm}
\end{theorem}
The proof is in Appendix~\ref{app:RevIN}. Theorem~\ref{theorem:RevIN} suggests that RevIN scales the absolute spectral energy by $ \sigma^2 $ but does not affect its relative distribution except $E_{\hat{X}}(0)=0$. Thus, RevIN preserves the relative spectral energy distribution and leaves the Mid-Frequency Spectrum Gap unresolved. \textit{However, our experiments still employ RevIN to ensure a fair comparison with other baselines.}
\begin{figure*}[h]
  \centering
  \includegraphics[width=1.\linewidth]{Faker/source/assets/jpg/ReFocus.jpg}
  \caption{General structure of \textbf{ReFocus}. `Adaptive Mid-Frequency Energy Optimizer (AMEO)' enhances mid-frequency components modeling, and `Energy-based Key-Frequency Picking Block' (EKPB) effectively captures shared Key-Frequency across channels}
  \label{fig:refocus}
\end{figure*}

\begin{figure*}[h]
  \centering
  \includegraphics[width=0.7\linewidth]{Faker/source/assets/jpg/ket.jpg}
  \caption{General process of the \textbf{Key-Frequency Enhanced Training strategy (KET)}, where spectral information from other channels is randomly introduced into each channel, to enhance the extraction of the shared Key-Frequency.}
  \label{fig:reshuffle}
\end{figure*}
\textbf{Impact of High- and Low-pass filter \quad}
We still define $\hat{x}(t)$ to be the filtered (processed) signal, obtained by applying a filter $H(f)$ (High/Low-pass filter). The filter $ H(f) $ is 1 in the passband (High/Low frequency) and 0 in the stopband (Middle frequency). So $E_{\hat{X}}(f)=0,\quad E_{\hat{X}}\leq E_X(f)$ for middle frequencies, which creates even larger gap.

\subsection{Overall Structure of The Proposed ReFocus}

In this section, we elucidate the overall architecture of \textbf{ReFocus}, depicted in Figure \ref{fig:refocus}. We define frequency domain projection as $D1\rightarrow D2$ representing a projection from dimension $D1$ to $D2$ in the frequency domain~\citep{xu2024fits}. Initially, we apply \textbf{AMEO} to the input $X \in \mathbb{R}^{C \times T}$, yielding the processed spectrum $ X_{am} \in \mathbb{R}^{C  \times T} $. Next, we use a projection $T\rightarrow D$ to transform $ X_{am}$ into the Variate Embedding $ X_{em} \in \mathbb{R}^{C  \times D}$~\citep{LiuiTransformer}. Then, $X_{em}$ go through $N$ \textbf{EKPB} to generate representation $H_{N+1}$, which is projected to obtain final prediction $\hat{Y}$. 

\textbf{Adaptive Mid-Frequency Energy Optimizer \quad}
Building upon the \textbf{Preliminary Analysis}, we propose a convolution- and residual learning-based solution to address the Mid-Frequency Spectrum Gap, which we denoted as AMEO. 
\begin{definition}[Adaptive Mid-Frequency Energy Optimizer]\label{def:AMEO}
AMEO is defined as:
\begin{align}
&\hat{x}(t) = x(t)-\frac{\beta}{K}\sum_{k=0}^{K-1} \tilde{x}(t+K-1-k),\notag\\
&\tilde{x}(t) =\notag\\
&\begin{cases}
x(t-(\frac{K}{2}+1)), \quad \text{if } \frac{K}{2}+1 \leq t < T+\frac{K}{2}+1, \\
0,  \quad\text{if } 0 \leq t < \frac{K}{2}+1 \text{ or } T+\frac{K}{2}+1 \leq t < T+K.
\end{cases}
\end{align}
\vspace{-0.2cm}
\end{definition}

It is equivalent to $x=x-\beta \cdot Conv(x)$. $Conv$ is a 1D convolution (Zero-padding at both ends, stride $s=1$, kernel size $K$, with values initialized as $ \frac{1}{K} $). $\beta \in \mathbb{R}^{1}$ is a hyperparameter.

\begin{theorem} [Frequency Spectrum after AMEO] \label{theorem:AMEO}
The spectral energy of $\hat{x}(t)$ obtained using AMEO:
\begin{align}
E_{\hat{X}}(f) =|X(f)|^2 \left\{1 - \beta \cdot \underbrace{\frac{1}{K} \sum_{k=0}^{K-1} e^{i 2 \pi f (\frac{3K}{2}-k -2) / {T-1}}}_{G(f)}\right\}^2
\end{align}
\vspace{-0.2cm}
\end{theorem}

The proof is in Appendix~\ref{app:AMEO}. We have $E_{\hat{X}}(f) =|X(f)|^2(1-\beta  \cdot G(f))^2$. Generally, $ G(f) $ behaves as a decay function, gradually reducing its value from \textbf{One} to \textbf{Zero}. Such \textbf{decay behavior} makes AMEO relatively enhances mid-frequency components, thus addressing the Mid-Frequency Spectrum Gap.

\textbf{Energy-based Key-Frequency Picking Block \quad} In each \textbf{EKPB}, the input $ H_i \in \mathbb{R}^{C  \times D} (H_1=X_{em}) $ is first processed through an MLP to generate $ H_i^k \in \mathbb{R}^{C  \times Q}$. Then, FFT is applied to get $ H_i^f \in \mathbb{R}^{C  \times (Q/2+1)}$. For $ H_i^f$, we calculate its energy, denoted as $ H_i^e \in \mathbb{R}^{C  \times (Q/2+1)}$. A cross-channel softmax is then applied to $ H_i^e$ per frequency to obtain a probability distribution $ H_i^{soft} \in \mathbb{R}^{C  \times (Q/2+1)}$. Using $H_i^{soft}$, we select values from $ H_i^f$ across channels for each frequency, resulting in $K^f_i \in \mathbb{R}^{1  \times (Q/2+1)}$, which represents the Shared Key-Frequency across all channels. Then iFFT is performed on $K^f_i$ to get $K_i\in \mathbb{R}^{1  \times Q}$, followed by projection $Q\rightarrow D$ and repeating (C times) to get $\hat{K}_i \in \mathbb{R}^{C  \times D}$. This $\hat{K}_i$ is point-wisely added to $\hat{H_i}\in \mathbb{R}^{C  \times D}$ , which is the projection of $ H_i$ using projection $D\rightarrow D$. Then, an MLP and $Add\&Norm$ is applied to the result $HK\in \mathbb{R}^{C  \times D}$ to fuse inter-series dependencies information, and another MLP and $Add\&Norm$ is used to capture intra-series variations~\citep{LiuiTransformer}. The output of each \textbf{EKPB} is $\hat{O_i} \in \mathbb{R}^{C  \times D}$, where $H_{i+1}=\hat{O_i}$.

\subsection{Key-Frequency Enhanced Training strategy}

In real-world time series, certain channels often exhibit spectral dependencies, which may not be fully captured in the training set, and the specific channels with such dependencies are also unknown~\citep{geweke1984freqchannel,Zhao2024freqchannel}. So this work borrows insight from recent advancement of mix-up in time series~\citep{zhou2023mixup,ansari2024mixup}, randomly introducing spectral information from other channels into each channel, to enhance the extraction of the shared Key-Frequency, as in Figure~\ref{fig:reshuffle}. Given a multivariate time series input $X \in \mathbb{R}^{C \times T}$ and its ground-truth $Y \in \mathbb{R}^{C \times F}$, we generate a pseudo sample pair: 

\begin{align}
X' = iFFT(FFT(X) +\alpha \cdot FFT(X[\text{perm},:]))&,  \notag\\ 
Y' = iFFT(FFT(Y) +\alpha \cdot FFT(Y[\text{perm},:]))&.
\end{align}

$\alpha \in \mathbb{R}^{C \times 1}$ is a weight vector sampled from a normal distribution, $\text{perm}$ is a reshuffled channel index. Since $FFT$ and $iFFT$ are linear operations, this mix-up process can be equivalently simplified in the \textbf{Time Domain}:
\begin{align}
X' = X +\alpha \cdot X[\text{perm},:]&,  \notag\\
Y' = Y +\alpha \cdot Y[\text{perm},:]&
 \end{align}
We alternate training between real and synthetic data to preserve the spectral dependencies in real samples. This combines the advantages of data augmentation, such as improved generalization, while mitigating potential drawbacks like over-smoothing and training instability~\citep{ryu2024tf,alkhalifah2022tf}.













\section{Experiments}

\subsection{Setups}
\subsubsection{Implementation Details}
We apply our FDS method to two types of 3DGS: 
the original 3DGS, and 2DGS~\citep{huang20242d}. 
%
The number of iterations in our optimization 
process is 35,000.
We follow the default training configuration 
and apply our FDS method after 15,000 iterations,
then we add normal consistency loss for both
3DGS and 2DGS after 25000 iterations.
%
The weight for FDS, $\lambda_{fds}$, is set to 0.015,
the $\sigma$ is set to 23,
and the weight for normal consistency is set to 0.15
for all experiments. 
We removed the depth distortion loss in 2DGS 
because we found that it degrades its results in indoor scenes.
%
The Gaussian point cloud is initialized using Colmap
for all datasets.
%
%
We tested the impact of 
using Sea Raft~\citep{wang2025sea} and 
Raft\citep{teed2020raft} on FDS performance.
%
Due to the blurriness of the ScanNet dataset, 
additional prior constraints are required.
Thus, we incorporate normal prior supervision 
on the rendered normals 
in ScanNet (V2) dataset by default.
The normal prior is predicted by the Stable Normal 
model~\citep{ye2024stablenormal}
across all types of 3DGS.
%
The entire framework is implemented in 
PyTorch~\citep{paszke2019pytorch}, 
and all experiments are conducted on 
a single NVIDIA 4090D GPU.

\begin{figure}[t] \centering
    \makebox[0.16\textwidth]{\scriptsize Input}
    \makebox[0.16\textwidth]{\scriptsize 3DGS}
    \makebox[0.16\textwidth]{\scriptsize 2DGS}
    \makebox[0.16\textwidth]{\scriptsize 3DGS + FDS}
    \makebox[0.16\textwidth]{\scriptsize 2DGS + FDS}
    \makebox[0.16\textwidth]{\scriptsize GT (Depth)}

    \includegraphics[width=0.16\textwidth]{figure/fig3_img/compare3/gt_rgb/frame_00522.jpg}
    \includegraphics[width=0.16\textwidth]{figure/fig3_img/compare3/3DGS/frame_00522.jpg}
    \includegraphics[width=0.16\textwidth]{figure/fig3_img/compare3/2DGS/frame_00522.jpg}
    \includegraphics[width=0.16\textwidth]{figure/fig3_img/compare3/3DGS+FDS/frame_00522.jpg}
    \includegraphics[width=0.16\textwidth]{figure/fig3_img/compare3/2DGS+FDS/frame_00522.jpg}
    \includegraphics[width=0.16\textwidth]{figure/fig3_img/compare3/gt_depth/frame_00522.jpg} \\

    % \includegraphics[width=0.16\textwidth]{figure/fig3_img/compare1/gt_rgb/frame_00137.jpg}
    % \includegraphics[width=0.16\textwidth]{figure/fig3_img/compare1/3DGS/frame_00137.jpg}
    % \includegraphics[width=0.16\textwidth]{figure/fig3_img/compare1/2DGS/frame_00137.jpg}
    % \includegraphics[width=0.16\textwidth]{figure/fig3_img/compare1/3DGS+FDS/frame_00137.jpg}
    % \includegraphics[width=0.16\textwidth]{figure/fig3_img/compare1/2DGS+FDS/frame_00137.jpg}
    % \includegraphics[width=0.16\textwidth]{figure/fig3_img/compare1/gt_depth/frame_00137.jpg} \\

     \includegraphics[width=0.16\textwidth]{figure/fig3_img/compare2/gt_rgb/frame_00262.jpg}
    \includegraphics[width=0.16\textwidth]{figure/fig3_img/compare2/3DGS/frame_00262.jpg}
    \includegraphics[width=0.16\textwidth]{figure/fig3_img/compare2/2DGS/frame_00262.jpg}
    \includegraphics[width=0.16\textwidth]{figure/fig3_img/compare2/3DGS+FDS/frame_00262.jpg}
    \includegraphics[width=0.16\textwidth]{figure/fig3_img/compare2/2DGS+FDS/frame_00262.jpg}
    \includegraphics[width=0.16\textwidth]{figure/fig3_img/compare2/gt_depth/frame_00262.jpg} \\

    \includegraphics[width=0.16\textwidth]{figure/fig3_img/compare4/gt_rgb/frame00000.png}
    \includegraphics[width=0.16\textwidth]{figure/fig3_img/compare4/3DGS/frame00000.png}
    \includegraphics[width=0.16\textwidth]{figure/fig3_img/compare4/2DGS/frame00000.png}
    \includegraphics[width=0.16\textwidth]{figure/fig3_img/compare4/3DGS+FDS/frame00000.png}
    \includegraphics[width=0.16\textwidth]{figure/fig3_img/compare4/2DGS+FDS/frame00000.png}
    \includegraphics[width=0.16\textwidth]{figure/fig3_img/compare4/gt_depth/frame00000.png} \\

    \includegraphics[width=0.16\textwidth]{figure/fig3_img/compare5/gt_rgb/frame00080.png}
    \includegraphics[width=0.16\textwidth]{figure/fig3_img/compare5/3DGS/frame00080.png}
    \includegraphics[width=0.16\textwidth]{figure/fig3_img/compare5/2DGS/frame00080.png}
    \includegraphics[width=0.16\textwidth]{figure/fig3_img/compare5/3DGS+FDS/frame00080.png}
    \includegraphics[width=0.16\textwidth]{figure/fig3_img/compare5/2DGS+FDS/frame00080.png}
    \includegraphics[width=0.16\textwidth]{figure/fig3_img/compare5/gt_depth/frame00080.png} \\



    \caption{\textbf{Comparison of depth reconstruction on Mushroom and ScanNet datasets.} The original
    3DGS or 2DGS model equipped with FDS can remove unwanted floaters and reconstruct
    geometry more preciously.}
    \label{fig:compare}
\end{figure}


\subsubsection{Datasets and Metrics}

We evaluate our method for 3D reconstruction 
and novel view synthesis tasks on
\textbf{Mushroom}~\citep{ren2024mushroom},
\textbf{ScanNet (v2)}~\citep{dai2017scannet}, and 
\textbf{Replica}~\citep{replica19arxiv}
datasets,
which feature challenging indoor scenes with both 
sparse and dense image sampling.
%
The Mushroom dataset is an indoor dataset 
with sparse image sampling and two distinct 
camera trajectories. 
%
We train our model on the training split of 
the long capture sequence and evaluate 
novel view synthesis on the test split 
of the long capture sequences.
%
Five scenes are selected to evaluate our FDS, 
including "coffee room", "honka", "kokko", 
"sauna", and "vr room". 
%
ScanNet(V2)~\citep{dai2017scannet}  consists of 1,613 indoor scenes
with annotated camera poses and depth maps. 
%
We select 5 scenes from the ScanNet (V2) dataset, 
uniformly sampling one-tenth of the views,
following the approach in ~\citep{guo2022manhattan}.
To further improve the geometry rendering quality of 3DGS, 
%
Replica~\citep{replica19arxiv} contains small-scale 
real-world indoor scans. 
We evaluate our FDS on five scenes from 
Replica: office0, office1, office2, office3 and office4,
selecting one-tenth of the views for training.
%
The results for Replica are provided in the 
supplementary materials.
To evaluate the rendering quality and geometry 
of 3DGS, we report PSNR, SSIM, and LPIPS for 
rendering quality, along with Absolute Relative Distance 
(Abs Rel) as a depth quality metrics.
%
Additionally, for mesh evaluation, 
we use metrics including Accuracy, Completion, 
Chamfer-L1 distance, Normal Consistency, 
and F-scores.




\subsection{Results}
\subsubsection{Depth rendering and novel view synthesis}
The comparison results on Mushroom and 
ScanNet are presented in \tabref{tab:mushroom} 
and \tabref{tab:scannet}, respectively. 
%
Due to the sparsity of sampling 
in the Mushroom dataset,
challenges are posed for both GOF~\citep{yu2024gaussian} 
and PGSR~\citep{chen2024pgsr}, 
leading to their relative poor performance 
on the Mushroom dataset.
%
Our approach achieves the best performance 
with the FDS method applied during the training process.
The FDS significantly enhances the 
geometric quality of 3DGS on the Mushroom dataset, 
improving the "abs rel" metric by more than 50\%.
%
We found that Sea Raft~\citep{wang2025sea}
outperforms Raft~\citep{teed2020raft} on FDS, 
indicating that a better optical flow model 
can lead to more significant improvements.
%
Additionally, the render quality of RGB 
images shows a slight improvement, 
by 0.58 in 3DGS and 0.50 in 2DGS, 
benefiting from the incorporation of cross-view consistency in FDS. 
%
On the Mushroom
dataset, adding the FDS loss increases 
the training time by half an hour, which maintains the same
level as baseline.
%
Similarly, our method shows a notable improvement on the ScanNet dataset as well using Sea Raft~\citep{wang2025sea} Model. The "abs rel" metric in 2DGS is improved nearly 50\%. This demonstrates the robustness and effectiveness of the FDS method across different datasets.
%


% \begin{wraptable}{r}{0.6\linewidth} \centering
% \caption{\textbf{Ablation study on geometry priors.}} 
%         \label{tab:analysis_prior}
%         \resizebox{\textwidth}{!}{
\begin{tabular}{c| c c c c c | c c c c}

    \hline
     Method &  Acc$\downarrow$ & Comp $\downarrow$ & C-L1 $\downarrow$ & NC $\uparrow$ & F-Score $\uparrow$ &  Abs Rel $\downarrow$ &  PSNR $\uparrow$  & SSIM  $\uparrow$ & LPIPS $\downarrow$ \\ \hline
    2DGS&   0.1078&  0.0850&  0.0964&  0.7835&  0.5170&  0.1002&  23.56&  0.8166& 0.2730\\
    2DGS+Depth&   0.0862&  0.0702&  0.0782&  0.8153&  0.5965&  0.0672&  23.92&  0.8227& 0.2619 \\
    2DGS+MVDepth&   0.2065&  0.0917&  0.1491&  0.7832&  0.3178&  0.0792&  23.74&  0.8193& 0.2692 \\
    2DGS+Normal&   0.0939&  0.0637&  0.0788&  \textbf{0.8359}&  0.5782&  0.0768&  23.78&  0.8197& 0.2676 \\
    2DGS+FDS &  \textbf{0.0615} & \textbf{ 0.0534}& \textbf{0.0574}& 0.8151& \textbf{0.6974}&  \textbf{0.0561}&  \textbf{24.06}&  \textbf{0.8271}&\textbf{0.2610} \\ \hline
    2DGS+Depth+FDS &  0.0561 &  0.0519& 0.0540& 0.8295& 0.7282&  0.0454&  \textbf{24.22}& \textbf{0.8291}&\textbf{0.2570} \\
    2DGS+Normal+FDS &  \textbf{0.0529} & \textbf{ 0.0450}& \textbf{0.0490}& \textbf{0.8477}& \textbf{0.7430}&  \textbf{0.0443}&  24.10&  0.8283& 0.2590 \\
    2DGS+Depth+Normal &  0.0695 & 0.0513& 0.0604& 0.8540&0.6723&  0.0523&  24.09&  0.8264&0.2575\\ \hline
    2DGS+Depth+Normal+FDS &  \textbf{0.0506} & \textbf{0.0423}& \textbf{0.0464}& \textbf{0.8598}&\textbf{0.7613}&  \textbf{0.0403}&  \textbf{24.22}& 
    \textbf{0.8300}&\textbf{0.0403}\\
    
\bottomrule
\end{tabular}
}
% \end{wraptable}



The qualitative comparisons on the Mushroom and ScanNet dataset 
are illustrated in \figref{fig:compare}. 
%
%
As seen in the first row of \figref{fig:compare}, 
both the original 3DGS and 2DGS suffer from overfitting, 
leading to corrupted geometry generation. 
%
Our FDS effectively mitigates this issue by 
supervising the matching relationship between 
the input and sampled views, 
helping to recover the geometry.
%
FDS also improves the refinement of geometric details, 
as shown in other rows. 
By incorporating the matching prior through FDS, 
the quality of the rendered depth is significantly improved.
%

\begin{table}[t] \centering
\begin{minipage}[t]{0.96\linewidth}
        \captionof{table}{\textbf{3D Reconstruction 
        and novel view synthesis results on Mushroom dataset. * 
        Represents that FDS uses the Raft model.
        }}
        \label{tab:mushroom}
        \resizebox{\textwidth}{!}{
\begin{tabular}{c| c c c c c | c c c c c}
    \hline
     Method &  Acc$\downarrow$ & Comp $\downarrow$ & C-L1 $\downarrow$ & NC $\uparrow$ & F-Score $\uparrow$ &  Abs Rel $\downarrow$ &  PSNR $\uparrow$  & SSIM  $\uparrow$ & LPIPS $\downarrow$ & Time  $\downarrow$ \\ \hline

    % DN-splatter &   &  &  &  &  &  &  &  & \\
    GOF &  0.1812 & 0.1093 & 0.1453 & 0.6292 & 0.3665 & 0.2380  & 21.37  &  0.7762  & 0.3132  & $\approx$1.4h\\ 
    PGSR &  0.0971 & 0.1420 & 0.1196 & 0.7193 & 0.5105 & 0.1723  & 22.13  & 0.7773  & 0.2918  & $\approx$1.2h \\ \hline
    3DGS &   0.1167 &  0.1033&  0.1100&  0.7954&  0.3739&  0.1214&  24.18&  0.8392& 0.2511 &$\approx$0.8h \\
    3DGS + FDS$^*$ & 0.0569  & 0.0676 & 0.0623 & 0.8105 & 0.6573 & 0.0603 & 24.72  & 0.8489 & 0.2379 &$\approx$1.3h \\
    3DGS + FDS & \textbf{0.0527}  & \textbf{0.0565} & \textbf{0.0546} & \textbf{0.8178} & \textbf{0.6958} & \textbf{0.0568} & \textbf{24.76}  & \textbf{0.8486} & \textbf{0.2381} &$\approx$1.3h \\ \hline
    2DGS&   0.1078&  0.0850&  0.0964&  0.7835&  0.5170&  0.1002&  23.56&  0.8166& 0.2730 &$\approx$0.8h\\
    2DGS + FDS$^*$ &  0.0689 &  0.0646& 0.0667& 0.8042& 0.6582& 0.0589& 23.98&  0.8255&0.2621 &$\approx$1.3h\\
    2DGS + FDS &  \textbf{0.0615} & \textbf{ 0.0534}& \textbf{0.0574}& \textbf{0.8151}& \textbf{0.6974}&  \textbf{0.0561}&  \textbf{24.06}&  \textbf{0.8271}&\textbf{0.2610} &$\approx$1.3h \\ \hline
\end{tabular}
}
\end{minipage}\hfill
\end{table}

\begin{table}[t] \centering
\begin{minipage}[t]{0.96\linewidth}
        \captionof{table}{\textbf{3D Reconstruction 
        and novel view synthesis results on ScanNet dataset.}}
        \label{tab:scannet}
        \resizebox{\textwidth}{!}{
\begin{tabular}{c| c c c c c | c c c c }
    \hline
     Method &  Acc $\downarrow$ & Comp $\downarrow$ & C-L1 $\downarrow$ & NC $\uparrow$ & F-Score $\uparrow$ &  Abs Rel $\downarrow$ &  PSNR $\uparrow$  & SSIM  $\uparrow$ & LPIPS $\downarrow$ \\ \hline
    GOF & 1.8671  & 0.0805 & 0.9738 & 0.5622 & 0.2526 & 0.1597  & 21.55  & 0.7575  & 0.3881 \\
    PGSR &  0.2928 & 0.5103 & 0.4015 & 0.5567 & 0.1926 & 0.1661  & 21.71 & 0.7699  & 0.3899 \\ \hline

    3DGS &  0.4867 & 0.1211 & 0.3039 & 0.7342& 0.3059 & 0.1227 & 22.19& 0.7837 & 0.3907\\
    3DGS + FDS &  \textbf{0.2458} & \textbf{0.0787} & \textbf{0.1622} & \textbf{0.7831} & 
    \textbf{0.4482} & \textbf{0.0573} & \textbf{22.83} & \textbf{0.7911} & \textbf{0.3826} \\ \hline
    2DGS &  0.2658 & 0.0845 & 0.1752 & 0.7504& 0.4464 & 0.0831 & 22.59& 0.7881 & 0.3854\\
    2DGS + FDS &  \textbf{0.1457} & \textbf{0.0679} & \textbf{0.1068} & \textbf{0.7883} & 
    \textbf{0.5459} & \textbf{0.0432} & \textbf{22.91} & \textbf{0.7928} & \textbf{0.3800} \\ \hline
\end{tabular}
}
\end{minipage}\hfill
\end{table}


\begin{table}[t] \centering
\begin{minipage}[t]{0.96\linewidth}
        \captionof{table}{\textbf{Ablation study on geometry priors.}}
        \label{tab:analysis_prior}
        \resizebox{\textwidth}{!}{
\begin{tabular}{c| c c c c c | c c c c}

    \hline
     Method &  Acc$\downarrow$ & Comp $\downarrow$ & C-L1 $\downarrow$ & NC $\uparrow$ & F-Score $\uparrow$ &  Abs Rel $\downarrow$ &  PSNR $\uparrow$  & SSIM  $\uparrow$ & LPIPS $\downarrow$ \\ \hline
    2DGS&   0.1078&  0.0850&  0.0964&  0.7835&  0.5170&  0.1002&  23.56&  0.8166& 0.2730\\
    2DGS+Depth&   0.0862&  0.0702&  0.0782&  0.8153&  0.5965&  0.0672&  23.92&  0.8227& 0.2619 \\
    2DGS+MVDepth&   0.2065&  0.0917&  0.1491&  0.7832&  0.3178&  0.0792&  23.74&  0.8193& 0.2692 \\
    2DGS+Normal&   0.0939&  0.0637&  0.0788&  \textbf{0.8359}&  0.5782&  0.0768&  23.78&  0.8197& 0.2676 \\
    2DGS+FDS &  \textbf{0.0615} & \textbf{ 0.0534}& \textbf{0.0574}& 0.8151& \textbf{0.6974}&  \textbf{0.0561}&  \textbf{24.06}&  \textbf{0.8271}&\textbf{0.2610} \\ \hline
    2DGS+Depth+FDS &  0.0561 &  0.0519& 0.0540& 0.8295& 0.7282&  0.0454&  \textbf{24.22}& \textbf{0.8291}&\textbf{0.2570} \\
    2DGS+Normal+FDS &  \textbf{0.0529} & \textbf{ 0.0450}& \textbf{0.0490}& \textbf{0.8477}& \textbf{0.7430}&  \textbf{0.0443}&  24.10&  0.8283& 0.2590 \\
    2DGS+Depth+Normal &  0.0695 & 0.0513& 0.0604& 0.8540&0.6723&  0.0523&  24.09&  0.8264&0.2575\\ \hline
    2DGS+Depth+Normal+FDS &  \textbf{0.0506} & \textbf{0.0423}& \textbf{0.0464}& \textbf{0.8598}&\textbf{0.7613}&  \textbf{0.0403}&  \textbf{24.22}& 
    \textbf{0.8300}&\textbf{0.0403}\\
    
\bottomrule
\end{tabular}
}
\end{minipage}\hfill
\end{table}




\subsubsection{Mesh extraction}
To further demonstrate the improvement in geometry quality, 
we applied methods used in ~\citep{turkulainen2024dnsplatter} 
to extract meshes from the input views of optimized 3DGS. 
The comparison results are presented  
in \tabref{tab:mushroom}. 
With the integration of FDS, the mesh quality is significantly enhanced compared to the baseline, featuring fewer floaters and more well-defined shapes.
 %
% Following the incorporation of FDS, the reconstruction 
% results exhibit fewer floaters and more well-defined 
% shapes in the meshes. 
% Visualized comparisons
% are provided in the supplementary material.

% \begin{figure}[t] \centering
%     \makebox[0.19\textwidth]{\scriptsize GT}
%     \makebox[0.19\textwidth]{\scriptsize 3DGS}
%     \makebox[0.19\textwidth]{\scriptsize 3DGS+FDS}
%     \makebox[0.19\textwidth]{\scriptsize 2DGS}
%     \makebox[0.19\textwidth]{\scriptsize 2DGS+FDS} \\

%     \includegraphics[width=0.19\textwidth]{figure/fig4_img/compare1/gt02.png}
%     \includegraphics[width=0.19\textwidth]{figure/fig4_img/compare1/baseline06.png}
%     \includegraphics[width=0.19\textwidth]{figure/fig4_img/compare1/baseline_fds05.png}
%     \includegraphics[width=0.19\textwidth]{figure/fig4_img/compare1/2dgs04.png}
%     \includegraphics[width=0.19\textwidth]{figure/fig4_img/compare1/2dgs_fds03.png} \\

%     \includegraphics[width=0.19\textwidth]{figure/fig4_img/compare2/gt00.png}
%     \includegraphics[width=0.19\textwidth]{figure/fig4_img/compare2/baseline02.png}
%     \includegraphics[width=0.19\textwidth]{figure/fig4_img/compare2/baseline_fds01.png}
%     \includegraphics[width=0.19\textwidth]{figure/fig4_img/compare2/2dgs04.png}
%     \includegraphics[width=0.19\textwidth]{figure/fig4_img/compare2/2dgs_fds03.png} \\
      
%     \includegraphics[width=0.19\textwidth]{figure/fig4_img/compare3/gt05.png}
%     \includegraphics[width=0.19\textwidth]{figure/fig4_img/compare3/3dgs03.png}
%     \includegraphics[width=0.19\textwidth]{figure/fig4_img/compare3/3dgs_fds04.png}
%     \includegraphics[width=0.19\textwidth]{figure/fig4_img/compare3/2dgs02.png}
%     \includegraphics[width=0.19\textwidth]{figure/fig4_img/compare3/2dgs_fds01.png} \\

%     \caption{\textbf{Qualitative comparison of extracted mesh 
%     on Mushroom and ScanNet datasets.}}
%     \label{fig:mesh}
% \end{figure}












\subsection{Ablation study}


\textbf{Ablation study on geometry priors:} 
To highlight the advantage of incorporating matching priors, 
we incorporated various types of priors generated by different 
models into 2DGS. These include a monocular depth estimation
model (Depth Anything v2)~\citep{yang2024depth}, a two-view depth estimation 
model (Unimatch)~\citep{xu2023unifying}, 
and a monocular normal estimation model (DSINE)~\citep{bae2024rethinking}.
We adapt the scale and shift-invariant loss in Midas~\citep{birkl2023midas} for
monocular depth supervision and L1 loss for two-view depth supervison.
%
We use Sea Raft~\citep{wang2025sea} as our default optical flow model.
%
The comparison results on Mushroom dataset 
are shown in ~\tabref{tab:analysis_prior}.
We observe that the normal prior provides accurate shape information, 
enhancing the geometric quality of the radiance field. 
%
% In contrast, the monocular depth prior slightly increases 
% the 'Abs Rel' due to its ambiguous scale and inaccurate depth ordering.
% Moreover, the performance of monocular depth estimation 
% in the sauna scene is particularly poor, 
% primarily due to the presence of numerous reflective 
% surfaces and textureless walls, which limits the accuracy of monocular depth estimation.
%
The multi-view depth prior, hindered by the limited feature overlap 
between input views, fails to offer reliable geometric 
information. We test average "Abs Rel" of multi-view depth prior
, and the result is 0.19, which performs worse than the "Abs Rel" results 
rendered by original 2DGS.
From the results, it can be seen that depth order information provided by monocular depth improves
reconstruction accuracy. Meanwhile, our FDS achieves the best performance among all the priors, 
and by integrating all
three components, we obtained the optimal results.
%
%
\begin{figure}[t] \centering
    \makebox[0.16\textwidth]{\scriptsize RF (16000 iters)}
    \makebox[0.16\textwidth]{\scriptsize RF* (20000 iters)}
    \makebox[0.16\textwidth]{\scriptsize RF (20000 iters)  }
    \makebox[0.16\textwidth]{\scriptsize PF (16000 iters)}
    \makebox[0.16\textwidth]{\scriptsize PF (20000 iters)}


    % \includegraphics[width=0.16\textwidth]{figure/fig5_img/compare1/16000.png}
    % \includegraphics[width=0.16\textwidth]{figure/fig5_img/compare1/20000_wo_flow_loss.png}
    % \includegraphics[width=0.16\textwidth]{figure/fig5_img/compare1/20000.png}
    % \includegraphics[width=0.16\textwidth]{figure/fig5_img/compare1/16000_prior.png}
    % \includegraphics[width=0.16\textwidth]{figure/fig5_img/compare1/20000_prior.png}\\

    % \includegraphics[width=0.16\textwidth]{figure/fig5_img/compare2/16000.png}
    % \includegraphics[width=0.16\textwidth]{figure/fig5_img/compare2/20000_wo_flow_loss.png}
    % \includegraphics[width=0.16\textwidth]{figure/fig5_img/compare2/20000.png}
    % \includegraphics[width=0.16\textwidth]{figure/fig5_img/compare2/16000_prior.png}
    % \includegraphics[width=0.16\textwidth]{figure/fig5_img/compare2/20000_prior.png}\\

    \includegraphics[width=0.16\textwidth]{figure/fig5_img/compare3/16000.png}
    \includegraphics[width=0.16\textwidth]{figure/fig5_img/compare3/20000_wo_flow_loss.png}
    \includegraphics[width=0.16\textwidth]{figure/fig5_img/compare3/20000.png}
    \includegraphics[width=0.16\textwidth]{figure/fig5_img/compare3/16000_prior.png}
    \includegraphics[width=0.16\textwidth]{figure/fig5_img/compare3/20000_prior.png}\\
    
    \includegraphics[width=0.16\textwidth]{figure/fig5_img/compare4/16000.png}
    \includegraphics[width=0.16\textwidth]{figure/fig5_img/compare4/20000_wo_flow_loss.png}
    \includegraphics[width=0.16\textwidth]{figure/fig5_img/compare4/20000.png}
    \includegraphics[width=0.16\textwidth]{figure/fig5_img/compare4/16000_prior.png}
    \includegraphics[width=0.16\textwidth]{figure/fig5_img/compare4/20000_prior.png}\\

    \includegraphics[width=0.30\textwidth]{figure/fig5_img/bar.png}

    \caption{\textbf{The error map of Radiance Flow and Prior Flow.} RF: Radiance Flow, PF: Prior Flow, * means that there is no FDS loss supervision during optimization.}
    \label{fig:error_map}
\end{figure}




\textbf{Ablation study on FDS: }
In this section, we present the design of our FDS 
method through an ablation study on the 
Mushroom dataset to validate its effectiveness.
%
The optional configurations of FDS are outlined in ~\tabref{tab:ablation_fds}.
Our base model is the 2DGS equipped with FDS,
and its results are shown 
in the first row. The goal of this analysis 
is to evaluate the impact 
of various strategies on FDS sampling and loss design.
%
We observe that when we 
replace $I_i$ in \eqref{equ:mflow} with $C_i$, 
as shown in the second row, the geometric quality 
of 2DGS deteriorates. Using $I_i$ instead of $C_i$ 
help us to remove the floaters in $\bm{C^s}$, which are also 
remained in $\bm{C^i}$.
We also experiment with modifying the FDS loss. For example, 
in the third row, we use the neighbor 
input view as the sampling view, and replace the 
render result of neighbor view with ground truth image of its input view.
%
Due to the significant movement between images, the Prior Flow fails to accurately 
match the pixel between them, leading to a further degradation in geometric quality.
%
Finally, we attempt to fix the sampling view 
and found that this severely damaged the geometric quality, 
indicating that random sampling is essential for the stability 
of the mean error in the Prior flow.



\begin{table}[t] \centering

\begin{minipage}[t]{1.0\linewidth}
        \captionof{table}{\textbf{Ablation study on FDS strategies.}}
        \label{tab:ablation_fds}
        \resizebox{\textwidth}{!}{
\begin{tabular}{c|c|c|c|c|c|c|c}
    \hline
    \multicolumn{2}{c|}{$\mathcal{M}_{\theta}(X, \bm{C^s})$} & \multicolumn{3}{c|}{Loss} & \multicolumn{3}{c}{Metric}  \\
    \hline
    $X=C^i$ & $X=I^i$  & Input view & Sampled view     & Fixed Sampled view        & Abs Rel $\downarrow$ & F-score $\uparrow$ & NC $\uparrow$ \\
    \hline
    & \ding{51} &     &\ding{51}    &    &    \textbf{0.0561}        &  \textbf{0.6974}         & \textbf{0.8151}\\
    \hline
     \ding{51} &           &     &\ding{51}    &    &    0.0839        &  0.6242         &0.8030\\
     &  \ding{51} &   \ding{51}  &    &    &    0.0877       & 0.6091        & 0.7614 \\
      &  \ding{51} &    &    & \ding{51}    &    0.0724           & 0.6312          & 0.8015 \\
\bottomrule
\end{tabular}
}
\end{minipage}
\end{table}




\begin{figure}[htbp] \centering
    \makebox[0.22\textwidth]{}
    \makebox[0.22\textwidth]{}
    \makebox[0.22\textwidth]{}
    \makebox[0.22\textwidth]{}
    \\

    \includegraphics[width=0.22\textwidth]{figure/fig6_img/l1/rgb/frame00096.png}
    \includegraphics[width=0.22\textwidth]{figure/fig6_img/l1/render_rgb/frame00096.png}
    \includegraphics[width=0.22\textwidth]{figure/fig6_img/l1/render_depth/frame00096.png}
    \includegraphics[width=0.22\textwidth]{figure/fig6_img/l1/depth/frame00096.png}

    % \includegraphics[width=0.22\textwidth]{figure/fig6_img/l2/rgb/frame00112.png}
    % \includegraphics[width=0.22\textwidth]{figure/fig6_img/l2/render_rgb/frame00112.png}
    % \includegraphics[width=0.22\textwidth]{figure/fig6_img/l2/render_depth/frame00112.png}
    % \includegraphics[width=0.22\textwidth]{figure/fig6_img/l2/depth/frame00112.png}

    \caption{\textbf{Limitation of FDS.} }
    \label{fig:limitation}
\end{figure}


% \begin{figure}[t] \centering
%     \makebox[0.48\textwidth]{}
%     \makebox[0.48\textwidth]{}
%     \\
%     \includegraphics[width=0.48\textwidth]{figure/loss_Ignatius.pdf}
%     \includegraphics[width=0.48\textwidth]{figure/loss_family.pdf}
%     \caption{\textbf{Comparison the photometric error of Radiance Flow and Prior Flow:} 
%     We add FDS method after 2k iteration during training.
%     The results show
%     that:  1) The Prior Flow is more precise and 
%     robust than Radiance Flow during the radiance 
%     optimization; 2) After adding the FDS loss 
%     which utilize Prior 
%     flow to supervise the Radiance Flow at 2k iterations, 
%     both flow are more accurate, which lead to
%     a mutually reinforcing effects.(TODO fix it)} 
%     \label{fig:flowcompare}
% \end{figure}






\textbf{Interpretive Experiments: }
To demonstrate the mutual refinement of two flows in our FDS, 
For each view, we sample the unobserved 
views multiple times to compute the mean error 
of both Radiance Flow and Prior Flow. 
We use Raft~\citep{teed2020raft} as our default optical flow model
for visualization.
The ground truth flow is calculated based on 
~\eref{equ:flow_pose} and ~\eref{equ:flow} 
utilizing ground truth depth in dataset.
We introduce our FDS loss after 16000 iterations during 
optimization of 2DGS.
The error maps are shown in ~\figref{fig:error_map}.
Our analysis reveals that Radiance Flow tends to 
exhibit significant geometric errors, 
whereas Prior Flow can more accurately estimate the geometry,
effectively disregarding errors introduced by floating Gaussian points. 

%





\subsection{Limitation and further work}

Firstly, our FDS faces challenges in scenes with 
significant lighting variations between different 
views, as shown in the lamp of first row in ~\figref{fig:limitation}. 
%
Incorporating exposure compensation into FDS could help address this issue. 
%
 Additionally, our method struggles with 
 reflective surfaces and motion blur,
 leading to incorrect matching. 
 %
 In the future, we plan to explore the potential 
 of FDS in monocular video reconstruction tasks, 
 using only a single input image at each time step.
 


\section{Conclusions}
In this paper, we propose Flow Distillation Sampling (FDS), which
leverages the matching prior between input views and 
sampled unobserved views from the pretrained optical flow model, to improve the geometry quality
of Gaussian radiance field. 
Our method can be applied to different approaches (3DGS and 2DGS) to enhance the geometric rendering quality of the corresponding neural radiance fields.
We apply our method to the 3DGS-based framework, 
and the geometry is enhanced on the Mushroom, ScanNet, and Replica datasets.

\section*{Acknowledgements} This work was supported by 
National Key R\&D Program of China (2023YFB3209702), 
the National Natural Science Foundation of 
China (62441204, 62472213), and Gusu 
Innovation \& Entrepreneurship Leading Talents Program (ZXL2024361)

% \section{Analysis}
This section analyzes the impact of disparity score distribution and the choice of score threshold $\lambda$ on the resulting steering vectors' quality and intervention performance.

\subsection{Impact of Disparity Score Distribution}

\begin{figure}[bt]
    \centering
    \includegraphics[width=\linewidth]{figs/bias-distribution-1.pdf}
    \caption{Probability distribution of disparity scores over the entire training set from which the prompts used for extracting vectors are sampled.}
    \label{fig:bias-distribution}
\end{figure}

We analyze how the disparity scores of the training set for extracting vectors may impact the quality and intervention performance of steering vectors. \autoref{fig:bias-distribution} (and \autoref{fig:bias-distribution-2} in \autoref{app:bias-distribution}) shows the disparity score probability distribution over the entire training set for each model. Most models exhibit a similar tri-modal distribution pattern with three distinct peaks located around -1, 0, and 1, except for \model{Qwen-1.8B} which shows a unimodal distribution (see \autoref{fig:bias-distribution-2}).  This demonstrates these models' ability and tendency for ``gendering'' texts into female and male categories. We compute the mode intervals of the distribution using the SkinnyDip algorithm~\citep{maurus2016skinny}, based on the dip test of unimodality~\citep{hartigan1985dip}, as shown by the shaded areas in \autoref{fig:bias-distribution}. Our results suggest that models with a wider center modal interval, like \model{Llama-3.1-8B} and \model{OLMo-2-7B}, show less effective debiasing performance with steering (\autoref{tab:steering-performance}). Furthermore, we find that models with less prominent peaks in their distribution, such as \model{Llama-2-13B} and \model{Qwen} in \autoref{fig:bias-distribution-2}, also show a lower projection correlation in their steering vectors.


\subsection{Varying Disparity Score Threshold}
\label{sec:bias-threshold}

\begin{figure}[tb]
    \centering
    \includegraphics[width=\linewidth]{figs/threshold_test.pdf}
    \caption{Bias scores after intervention using steering vectors computed by eight different threshold scores for constructing the training set, where $\delta=[0.01,0.3]$.}
    \label{fig:varying-threshold}
\end{figure}

Results shown in both \autoref{sec:extract-results} and \autoref{sec:steering-results} are based on the same score threshold $\delta$ of 0.05. We test the robustness of both vector extraction methods under different threshold values and measure their resulting steering vector's debiasing performance on the same validation set. We use eight different values of $\delta$ from 0.01 to 0.3 with increasing increments. \autoref{fig:varying-threshold} shows the range of RMS bias scores after debiasing under different $\delta$ across all eight models. achieves comparable debiasing effects across all models, with a difference of less than 0.05 in bias scores for the same model. MD exhibits the largest discrepancy in bias scores for the \model{Llama-3.1-8B} model, with a difference of 0.1. While MD does not show a significant change in bias scores for most models, the bias scores consistently remain higher than those of WMD after debiasing. 



\section{Related Work}

\subsection{Large 3D Reconstruction Models}
Recently, generalized feed-forward models for 3D reconstruction from sparse input views have garnered considerable attention due to their applicability in heavily under-constrained scenarios. The Large Reconstruction Model (LRM)~\cite{hong2023lrm} uses a transformer-based encoder-decoder pipeline to infer a NeRF reconstruction from just a single image. Newer iterations have shifted the focus towards generating 3D Gaussian representations from four input images~\cite{tang2025lgm, xu2024grm, zhang2025gslrm, charatan2024pixelsplat, chen2025mvsplat, liu2025mvsgaussian}, showing remarkable novel view synthesis results. The paradigm of transformer-based sparse 3D reconstruction has also successfully been applied to lifting monocular videos to 4D~\cite{ren2024l4gm}. \\
Yet, none of the existing works in the domain have studied the use-case of inferring \textit{animatable} 3D representations from sparse input images, which is the focus of our work. To this end, we build on top of the Large Gaussian Reconstruction Model (GRM)~\cite{xu2024grm}.

\subsection{3D-aware Portrait Animation}
A different line of work focuses on animating portraits in a 3D-aware manner.
MegaPortraits~\cite{drobyshev2022megaportraits} builds a 3D Volume given a source and driving image, and renders the animated source actor via orthographic projection with subsequent 2D neural rendering.
3D morphable models (3DMMs)~\cite{blanz19993dmm} are extensively used to obtain more interpretable control over the portrait animation. For example, StyleRig~\cite{tewari2020stylerig} demonstrates how a 3DMM can be used to control the data generated from a pre-trained StyleGAN~\cite{karras2019stylegan} network. ROME~\cite{khakhulin2022rome} predicts vertex offsets and texture of a FLAME~\cite{li2017flame} mesh from the input image.
A TriPlane representation is inferred and animated via FLAME~\cite{li2017flame} in multiple methods like Portrait4D~\cite{deng2024portrait4d}, Portrait4D-v2~\cite{deng2024portrait4dv2}, and GPAvatar~\cite{chu2024gpavatar}.
Others, such as VOODOO 3D~\cite{tran2024voodoo3d} and VOODOO XP~\cite{tran2024voodooxp}, learn their own expression encoder to drive the source person in a more detailed manner. \\
All of the aforementioned methods require nothing more than a single image of a person to animate it. This allows them to train on large monocular video datasets to infer a very generic motion prior that even translates to paintings or cartoon characters. However, due to their task formulation, these methods mostly focus on image synthesis from a frontal camera, often trading 3D consistency for better image quality by using 2D screen-space neural renderers. In contrast, our work aims to produce a truthful and complete 3D avatar representation from the input images that can be viewed from any angle.  

\subsection{Photo-realistic 3D Face Models}
The increasing availability of large-scale multi-view face datasets~\cite{kirschstein2023nersemble, ava256, pan2024renderme360, yang2020facescape} has enabled building photo-realistic 3D face models that learn a detailed prior over both geometry and appearance of human faces. HeadNeRF~\cite{hong2022headnerf} conditions a Neural Radiance Field (NeRF)~\cite{mildenhall2021nerf} on identity, expression, albedo, and illumination codes. VRMM~\cite{yang2024vrmm} builds a high-quality and relightable 3D face model using volumetric primitives~\cite{lombardi2021mvp}. One2Avatar~\cite{yu2024one2avatar} extends a 3DMM by anchoring a radiance field to its surface. More recently, GPHM~\cite{xu2025gphm} and HeadGAP~\cite{zheng2024headgap} have adopted 3D Gaussians to build a photo-realistic 3D face model. \\
Photo-realistic 3D face models learn a powerful prior over human facial appearance and geometry, which can be fitted to a single or multiple images of a person, effectively inferring a 3D head avatar. However, the fitting procedure itself is non-trivial and often requires expensive test-time optimization, impeding casual use-cases on consumer-grade devices. While this limitation may be circumvented by learning a generalized encoder that maps images into the 3D face model's latent space, another fundamental limitation remains. Even with more multi-view face datasets being published, the number of available training subjects rarely exceeds the thousands, making it hard to truly learn the full distibution of human facial appearance. Instead, our approach avoids generalizing over the identity axis by conditioning on some images of a person, and only generalizes over the expression axis for which plenty of data is available. 

A similar motivation has inspired recent work on codec avatars where a generalized network infers an animatable 3D representation given a registered mesh of a person~\cite{cao2022authentic, li2024uravatar}.
The resulting avatars exhibit excellent quality at the cost of several minutes of video capture per subject and expensive test-time optimization.
For example, URAvatar~\cite{li2024uravatar} finetunes their network on the given video recording for 3 hours on 8 A100 GPUs, making inference on consumer-grade devices impossible. In contrast, our approach directly regresses the final 3D head avatar from just four input images without the need for expensive test-time fine-tuning.



\paragraph{Summary}
Our findings provide significant insights into the influence of correctness, explanations, and refinement on evaluation accuracy and user trust in AI-based planners. 
In particular, the findings are three-fold: 
(1) The \textbf{correctness} of the generated plans is the most significant factor that impacts the evaluation accuracy and user trust in the planners. As the PDDL solver is more capable of generating correct plans, it achieves the highest evaluation accuracy and trust. 
(2) The \textbf{explanation} component of the LLM planner improves evaluation accuracy, as LLM+Expl achieves higher accuracy than LLM alone. Despite this improvement, LLM+Expl minimally impacts user trust. However, alternative explanation methods may influence user trust differently from the manually generated explanations used in our approach.
% On the other hand, explanations may help refine the trust of the planner to a more appropriate level by indicating planner shortcomings.
(3) The \textbf{refinement} procedure in the LLM planner does not lead to a significant improvement in evaluation accuracy; however, it exhibits a positive influence on user trust that may indicate an overtrust in some situations.
% This finding is aligned with prior works showing that iterative refinements based on user feedback would increase user trust~\cite{kunkel2019let, sebo2019don}.
Finally, the propensity-to-trust analysis identifies correctness as the primary determinant of user trust, whereas explanations provided limited improvement in scenarios where the planner's accuracy is diminished.

% In conclusion, our results indicate that the planner's correctness is the dominant factor for both evaluation accuracy and user trust. Therefore, selecting high-quality training data and optimizing the training procedure of AI-based planners to improve planning correctness is the top priority. Once the AI planner achieves a similar correctness level to traditional graph-search planners, strengthening its capability to explain and refine plans will further improve user trust compared to traditional planners.

\paragraph{Future Research} Future steps in this research include expanding user studies with larger sample sizes to improve generalizability and including additional planning problems per session for a more comprehensive evaluation. Next, we will explore alternative methods for generating plan explanations beyond manual creation to identify approaches that more effectively enhance user trust. 
Additionally, we will examine user trust by employing multiple LLM-based planners with varying levels of planning accuracy to better understand the interplay between planning correctness and user trust. 
Furthermore, we aim to enable real-time user-planner interaction, allowing users to provide feedback and refine plans collaboratively, thereby fostering a more dynamic and user-centric planning process.


\section{Limitations}
Our method's reliance on semantic embeddings introduces inherent biases present in encoder's training data. While these embeddings enable semantic consistency, they may not capture certain culturally-specific or nuanced artistic concepts. This highlights the need for more careful study on choices of the semantic embeddings and their effects on SliderSpace discovery. The current discovery process requires significant computational time ($\approx$ 2 hrs on A100), which may limit rapid experimentation and iteration. This computational overhead opens avenues for future research into training time optimizations. We also note that our method trains 4 times faster than  Concept Sliders for same number of sliders. For art style discovery, it is possible that the discovered directions are not one-to-one matched with the original artists. Further work can address discovery that nudges the directions to be aligned with real artists. 
% This can be used as an interpretability tool or an attribution tool if the discover directions are aligned with real artists. 

% \section*{Ethics Statement}

% A solution for xxx

% contributions to ethics







%\section*{Acknowledgements}


% Entries for the entire Anthology, followed by custom entries
\bibliography{anthology,custom}
\bibliographystyle{acl_natbib}

\appendix



\newpage
\appendix
\section{Applicability of SparseTransX for dense graphs} 
\label{A:density}
Even for fully dense graphs, our KGE computations remain highly sparse. This is because our SpMM leverages the incidence matrix for triplets, rather than the graph's adjacency matrix. In the paper, the sparse matrix $A \in \{-1,0,1\}^{M \times (N+R)}$ represents the triplets, where $N$ is the number of entities, $R$ is the number of relations, and $M$ is the number of triplets. This representation remains extremely sparse, as each row contains exactly three non-zero values (or two in the case of the "ht" representation). Hence, the sparsity of this formulation is independent of the graph's structure, ensuring computational efficiency even for dense graphs.

\section{Computational Complexity}
\label{A:complexity}
 For a sparse matrix $A$ with $m \times k$ having $nnz(A)=$ number of non zeros and dense matrix $X$ with $k \times n$ dimension, the computational complexity of the SpMM is $O(nnz(A) \cdot n)$ since there are a total of $nnz(A)$ number of dot products each involving $n$ components. Since our sparse matrix contains exactly three non-zeros in each row, $nnz(A) = 3m$. Therefore, the complexity of SpMM is $O(3m \cdot n)$ or $O(m \cdot n)$, meaning the complexity increases when triplet counts or embedding dimension is increased. Memory access pattern will change when the number of entities is increased and it will affect the runtime, but the algorithmic complexity will not be affected by the number of entities/relations.

\section{Applicability to Non-translational Models}
\label{A:non_trans}
Our paper focused on translational models using sparse operations, but the concept extends broadly to various other knowledge graph embedding (KGE) methods. Neural network-based models, which are inherently matrix-multiplication-based, can be seamlessly integrated into this framework. Additionally, models such as DistMult, ComplEx, and RotatE can be implemented with simple modifications to the SpMM operations. Implementing these KGE models requires modifying the addition and multiplication operators in SpMM, effectively changing the semiring that governs the multiplication.   

In the paper, the sparse matrix $A \in \{-1,0,1\}^{M \times (N+R)}$ represents the triplets, and the dense matrix $E \in \mathbb{R}^{(N+R) \times d}$ represents the embedding matrix, where $N$ is the number of entities, $R$ is the number of relations, and $M$ is the number of triplets. TransE’s score function, defined as $h + r - t$, is computed by multiplying $A$ and $E$ using an SpMM followed by the L2 norm. This operation can be generalized using a semiring-based SpMM model: $Z_{ij} = \bigoplus_{k=1}^{n} (A_{ik} \otimes E_{kj})$

Here, $\oplus$ represents the semiring addition operator, and $\otimes$ represents the semiring multiplication operator. For TransE, these operators correspond to standard arithmetic addition and multiplication, respectively.

\subsection*{DistMult} 
DistMult’s score function has the expression $h \odot r \odot t$. To adapt SpMM for this model, two key adjustments are required: The sparse matrix $A$ stores $+1$ at the positions corresponding to $h_{\text{idx}}$, $t_{\text{idx}}$, and $r_{\text{idx}}$. Both the semiring addition and multiplication operators are set to arithmetic multiplication. These changes enable the use of SpMM for the DistMult score function.

\subsection*{ComplEx} 
ComplEx’s score function has $h \odot r \odot \bar{t}$, where embeddings are stored as complex numbers (e.g., using PyTorch). In this case, the semiring operations are similar to DistMult, but with complex number multiplication replacing real number multiplication.

\subsection*{RotatE} 
RotatE’s score function has $h \odot r - t$. For this model, the semiring requires both arithmetic multiplication and subtraction for $\oplus$. With minor modifications to our SpMM implementation, the semiring addition operator can be adapted to compute $h \odot r - t$.

\subsection*{Support from other libraries}
Many existing libraries, such as GraphBLAS (Kimmerer, Raye, et al., 2024), Ginkgo (Anzt, Hartwig, et al., 2022), and Gunrock (Wang, Yangzihao, et al., 2017), already support custom semirings in SpMM. We can leverage C++ templates to extend support for KGE models with minimal effort.


\begin{figure*}[t]
\centering     %%% not \center
\includegraphics[width=\textwidth]{figures/all-eval.pdf}
\caption{Loss curve for sparse and non-sparse approach. Sparse approach eventually reaches the same loss value with similar Hits@10 test accuracy.}
\label{fig:loss_curve}
\end{figure*}

\section{Model Performance Evaluation and Convergence}
\label{A:eval}
SpTransX follows a slightly different loss curve (see Figure \ref{fig:loss_curve}) and eventually converges with the same loss as other non-sparse implementations such as TorchKGE. We test SpTransX with the WN18 dataset having embedding size 512 (128 for TransR and TransH due to memory limitation) and run 200-1000 epochs. We compute average Hits@10 of 9 runs with different initial seeds and a learning rate scheduler. The results are shown below. We find that Hits@10 is generally comparable to or better than the Hits@10 achieved by TorchKGE.

\begin{table}[h]
\centering
\caption{Average of 9 Hits@10 Accuracy for WN18 dataset}
\begin{tabular}{|c|c|c|}
\hline
\textbf{Model} & \textbf{TorchKGE} & \textbf{SpTransX} \\ \hline
TransE         & 0.79 ± 0.001700   & 0.79 ± 0.002667   \\ \hline
TransR         & 0.29 ± 0.005735   & 0.33 ± 0.006154   \\ \hline
TransH         & 0.76 ± 0.012285   & 0.79 ± 0.001832   \\ \hline
TorusE         & 0.73 ± 0.003258   & 0.73 ± 0.002780   \\ \hline
\end{tabular}
\label{table:perf_eval}
\end{table}

% We also plot the loss curve for different models in Figure \ref{fig:loss_curve}. We observe that the sparse approach follows a similar loss curve and eventually converges to the same final loss.

\section{Distributed SpTransX and Its Applicability to Large KGs}
\label{A:dist}
SpTransX framework includes several features to support distributed KGE training across multi-CPU, multi-GPU, and multi-node setups. Additionally, it incorporates modules for model and dataset streaming to handle massive datasets efficiently. 

Distributed SpTransX relies on PyTorch Distributed Data Parallel (DDP) and Fully Sharded Data Parallel (FSDP) support to distribute sparse computations across multiple GPUs. 

\begin{table}[h]
\centering
\caption{Average Time of 15 Epochs (seconds). Training time of TransE model with Freebase dataset (250M triplets, 77M entities. 74K relations, batch size 393K)  on 32 NVIDIA A100 GPUs. FSDP enables model training with larger embedding when DDP fails.}
\begin{tabular}{|p{2cm}|p{2.5cm}|p{2.5cm}|}
\hline
\textbf{Embedding Size} & \textbf{DDP (Distributed Data Parallel)} & \textbf{FSDP (Fully Sharded Data Parallel)} \\ \hline
16                      & 65.07 ± 1.641                            & 63.35 ± 1.258                               \\ \hline
20                      & Out of Memory                            & 96.44 ± 1.490                               \\ \hline
\end{tabular}
\end{table}

We run an experiment with a large-scale KG to showcase the performance of distributed SpTransX. Freebase (250M triplets, 77M entities. 74K relations, batch size 393K) dataset is trained using the TransE model on 32 NVIDIA A100 GPUs of NERSC using various distributed settings. SpTransX’s Streaming dataset module allows fetching only the necessary batch from the dataset and enables memory-efficient training. FSDP enables model training with larger embedding when DDP fails.

\section{Scaling and Communication Bottlenecks for Large KG Training}
\label{A:scaling}
Communication can be a significant bottleneck in distributed KGE training when using SpMM. However, by leveraging Distributed Data-Parallel (DDP) in PyTorch, we successfully scale distributed SpTransX to 64 NVIDIA A100 GPUs with reasonable efficiency. The training time for the COVID-19 dataset with 60,820 entities, 62 relations, and 1,032,939 triplets is in Table \ref{table:scaling}. 
% \vspace{-.3cm}
\begin{table}[h]
\centering
\caption{Scaling TransE model on COVID-19 dataset}
\begin{tabular}{|c|c|}
\hline
\textbf{Number of GPUs} & \textbf{500 epoch time (seconds)} \\ \hline
4                       & 706.38                            \\ \hline
8                       & 586.03                            \\ \hline
16                      & 340.00                               \\ \hline
32                      & 246.02                            \\ \hline
64                      & 179.95                            \\ \hline
\end{tabular}
\label{table:scaling}
\end{table}
% \vspace{-.2cm}
It indicates that communication is not a bottleneck up to 64 GPUs. If communication becomes a performance bottleneck at larger scales, we plan to explore alternative communication-reducing algorithms, including 2D and 3D matrix distribution techniques, which are known to minimize communication overhead at extreme scales. Additionally, we will incorporate model parallelism alongside data parallelism for large-scale knowledge graphs.

\section{Backpropagation of SpMM}
\label{A:backprop}
 Our main computational kernel is the sparse-dense matrix multiplication (SpMM). The computation of backpropagation of an SpMM w.r.t. the dense matrix is also another SpMM. To see how, let's consider the sparse-dense matrix multiplication $AX = C$ which is part of the training process. As long as the computational graph reduces to a single scaler loss $\mathfrak{L}$, it can be shown that $\frac{\partial C}{\partial X} = A^T$. Here, $X$ is the learnable parameter (embeddings), and $A$ is the sparse matrix. Since $A^T$ is also a sparse matrix and $\frac{\partial \mathfrak{L}}{\partial C}$ is a dense matrix, the computation $\frac{\partial \mathfrak{L}}{\partial X} = \frac{\partial C}{\partial X} \times \frac{\partial \mathfrak{L}}{\partial C} = A^T \times \frac{\partial \mathfrak{L}}{\partial C} $ is an SpMM. This means that both forward and backward propagation of our approach benefit from the efficiency of a high-performance SpMM.

\subsection*{Proof that $\frac{\partial C}{\partial X} = A^T$}
 To see why $\frac{\partial C}{\partial X} = A^T$ is used in the gradient calculation, we can consider the following small matrix multiplication without loss of generality.
\begin{align*}
A &= \begin{bmatrix}
a_1 & a_2 \\
a_3 & a_4
\end{bmatrix} \\ 
 X &= \begin{bmatrix}
x_1 & x_2 \\
x_3 & x_4
\end{bmatrix} \\
 C &=  \begin{bmatrix}
c_1 & c_2 \\
c_3 & c_4
\end{bmatrix}
\end{align*}
Where $C=AX$, thus-
\begin{align*}
c_1&=f(x_1, x_3) \\
c_2&=f(x_2, x_4) \\
c_3&=f(x_1, x_3) \\
c_4&=f(x_2, x_4) \\
\end{align*}
Therefore-
\begin{align*}
\frac{\partial \mathfrak{L}}{\partial x_1} &= \frac{\partial \mathfrak{L}}{\partial c_1} \times \frac{\partial c_1}{\partial x_1} + \frac{\partial \mathfrak{L}}{\partial c_2} \times \frac{\partial c_2}{\partial x_1} + \frac{\partial \mathfrak{L}}{\partial c_3} \times \frac{\partial c_3}{\partial x_1} + \frac{\partial \mathfrak{L}}{\partial c_4} \times \frac{\partial c_4}{\partial x_1}\\
&= \frac{\partial \mathfrak{L}}{\partial c_1} \times \frac{\partial \mathfrak{c_1}}{\partial x_1} + 0 + \frac{\partial \mathfrak{L}}{\partial c_3} \times \frac{\partial \mathfrak{c_3}}{\partial x_1} + 0\\
&= a_1 \times \frac{\partial \mathfrak{L}}{\partial c_1} + a_3 \times \frac{\partial \mathfrak{L}}{\partial c_3}\\
\end{align*}

Similarly-
\begin{align*}
\frac{\partial \mathfrak{L}}{\partial x_2}
&= a_1 \times \frac{\partial \mathfrak{L}}{\partial c_2} + a_3 \times \frac{\partial \mathfrak{L}}{\partial c_4}\\
\frac{\partial \mathfrak{L}}{\partial x_3}
&= a_2 \times \frac{\partial \mathfrak{L}}{\partial c_1} + a_4 \times \frac{\partial \mathfrak{L}}{\partial c_3}\\
\frac{\partial \mathfrak{L}}{\partial x_4}
&= a_2 \times \frac{\partial \mathfrak{L}}{\partial c_2} + a_4 \times \frac{\partial \mathfrak{L}}{\partial c_4}\\
\end{align*}
This can be expressed as a matrix equation in the following manner-
\begin{align*}
\frac{\partial \mathfrak{L}}{\partial X} &= \frac{\partial C}{\partial X} \times \frac{\partial \mathfrak{L}}{\partial C}\\
\implies \begin{bmatrix}
\frac{\partial \mathfrak{L}}{\partial x_1} & \frac{\partial \mathfrak{L}}{\partial x_2} \\
\frac{\partial \mathfrak{L}}{\partial x_3} & \frac{\partial \mathfrak{L}}{\partial x_4}
\end{bmatrix} &= \frac{\partial C}{\partial X} \times \begin{bmatrix}
\frac{\partial \mathfrak{L}}{\partial c_1} & \frac{\partial \mathfrak{L}}{\partial c_2} \\
\frac{\partial \mathfrak{L}}{\partial c_3} & \frac{\partial \mathfrak{L}}{\partial c_4}
\end{bmatrix}
\end{align*}
By comparing the individual partial derivatives computed earlier, we can say-

\begin{align*}
\begin{bmatrix}
\frac{\partial \mathfrak{L}}{\partial x_1} & \frac{\partial \mathfrak{L}}{\partial x_2} \\
\frac{\partial \mathfrak{L}}{\partial x_3} & \frac{\partial \mathfrak{L}}{\partial x_4}
\end{bmatrix} &= \begin{bmatrix}
a_1 & a_3 \\
a_2 & a_4
\end{bmatrix} \times \begin{bmatrix}
\frac{\partial \mathfrak{L}}{\partial c_1} & \frac{\partial \mathfrak{L}}{\partial c_2} \\
\frac{\partial \mathfrak{L}}{\partial c_3} & \frac{\partial \mathfrak{L}}{\partial c_4}
\end{bmatrix}\\
\implies \begin{bmatrix}
\frac{\partial \mathfrak{L}}{\partial x_1} & \frac{\partial \mathfrak{L}}{\partial x_2} \\
\frac{\partial \mathfrak{L}}{\partial x_3} & \frac{\partial \mathfrak{L}}{\partial x_4}
\end{bmatrix} &= A^T \times \begin{bmatrix}
\frac{\partial \mathfrak{L}}{\partial c_1} & \frac{\partial \mathfrak{L}}{\partial c_2} \\
\frac{\partial \mathfrak{L}}{\partial c_3} & \frac{\partial \mathfrak{L}}{\partial c_4}
\end{bmatrix}\\
\implies \frac{\partial \mathfrak{L}}{\partial X} &= A^T \times \frac{\partial \mathfrak{L}}{\partial C}\\
\therefore \frac{\partial C}{\partial X} &= A^T \qed
\end{align*}



\end{document}

