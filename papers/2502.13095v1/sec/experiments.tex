\section{Experiments} \label{sec:exp}

\subsection{Models and Baseline Methods} 

We compare \OursMethod\ with recent inference-time VLM defense frameworks, AdaShield \cite{wang2024adashield} and ECSO \cite{gou2025eyes} on five open-source VLMs: LLaVA-1.5-7B \cite{liu2024visual, liu2024improved}, LLaVA-1.6-34B \cite{liu2024llava}, MiniGPT-4-7B \cite{zhu2023minigpt}, ShareGPT4V-7B \cite{chen2024sharegpt4v}, and Qwen-VL-7B \cite{bai2023qwen}.

\subsection{Main Results on Safety} 

\textbf{Evaluation Metric.} 
To evaluate the effectiveness of a jailbreak attack under a defense framework, we measure the \textbf{Attack Success Rate (ASR)}, defined as the ratio of harmful responses to the total number of input queries. A lower ASR indicates a stronger defense against attacks. Following \cite{liu2025mm, wang2024adashield}, we classify harmful responses by checking for the presence of rejection keywords in the response, predefined in 
% such as phrases like “I am sorry” and “I apologize”. Details can be found in 
Appendix \ref{appendix-implementation}.

\documentclass{article}

% Recommended, but optional, packages for figures and better typesetting:
\usepackage{graphicx}
\usepackage{subfigure}
\usepackage{booktabs} % for professional tables
\usepackage[a4paper,top=3cm,bottom=2cm,left=2.5cm,right=2.5cm,marginparwidth=1.75cm]{geometry}
\usepackage{amsmath, amssymb, natbib, graphicx, url,dsfont,datetime,cases,mathtools,amsthm}
\usepackage{thmtools,thm-restate}
\usepackage{hyperref}
\usepackage{authblk}


\usepackage{amsmath}
\usepackage{amssymb}
\usepackage{mathtools}
\usepackage{amsthm}
\usepackage{enumitem}
\usepackage{macros}
\usepackage{tikz}
\usepackage{booktabs}
\usepackage{subfigure,algorithm2e,algorithmic}

% if you use cleveref..
\usepackage[capitalize,noabbrev]{cleveref}

%%%%%%%%%%%%%%%%%%%%%%%%%%%%%%%%
% THEOREMS
%%%%%%%%%%%%%%%%%%%%%%%%%%%%%%%%
\theoremstyle{plain}
\newtheorem{theorem}{Theorem}[section]
\newtheorem{proposition}[theorem]{Proposition}
\newtheorem{lemma}[theorem]{Lemma}
\newtheorem{corollary}[theorem]{Corollary}
\theoremstyle{definition}
\newtheorem{definition}[theorem]{Definition}
\newtheorem{assumption}[theorem]{Assumption}
\theoremstyle{remark}
\newtheorem{remark}[theorem]{Remark}

\title{The Batch Complexity of Bandit Pure Exploration}
\date{}
\author[1]{Adrienne Tuynman}
\author[1]{Rémy Degenne}
\affil[1]{Univ. Lille, Inria, CNRS, Centrale Lille, UMR 9189-CRIStAL, F-59000 Lille, France}
\begin{document}
	\maketitle

\begin{abstract}
In a fixed-confidence pure exploration problem in stochastic multi-armed bandits, an algorithm iteratively samples arms and should stop as early as possible and return the correct answer to a query about the arms distributions.
We are interested in batched methods, which change their sampling behaviour only a few times, between batches of observations.
We give an instance-dependent lower bound on the number of batches used by any sample efficient algorithm for any pure exploration task.
We then give a general batched algorithm and prove upper bounds on its expected sample complexity and batch complexity.
We illustrate both lower and upper bounds on best-arm identification and thresholding bandits.
\end{abstract}
\section{Introduction}
\label{sec:introduction}
The business processes of organizations are experiencing ever-increasing complexity due to the large amount of data, high number of users, and high-tech devices involved \cite{martin2021pmopportunitieschallenges, beerepoot2023biggestbpmproblems}. This complexity may cause business processes to deviate from normal control flow due to unforeseen and disruptive anomalies \cite{adams2023proceddsriftdetection}. These control-flow anomalies manifest as unknown, skipped, and wrongly-ordered activities in the traces of event logs monitored from the execution of business processes \cite{ko2023adsystematicreview}. For the sake of clarity, let us consider an illustrative example of such anomalies. Figure \ref{FP_ANOMALIES} shows a so-called event log footprint, which captures the control flow relations of four activities of a hypothetical event log. In particular, this footprint captures the control-flow relations between activities \texttt{a}, \texttt{b}, \texttt{c} and \texttt{d}. These are the causal ($\rightarrow$) relation, concurrent ($\parallel$) relation, and other ($\#$) relations such as exclusivity or non-local dependency \cite{aalst2022pmhandbook}. In addition, on the right are six traces, of which five exhibit skipped, wrongly-ordered and unknown control-flow anomalies. For example, $\langle$\texttt{a b d}$\rangle$ has a skipped activity, which is \texttt{c}. Because of this skipped activity, the control-flow relation \texttt{b}$\,\#\,$\texttt{d} is violated, since \texttt{d} directly follows \texttt{b} in the anomalous trace.
\begin{figure}[!t]
\centering
\includegraphics[width=0.9\columnwidth]{images/FP_ANOMALIES.png}
\caption{An example event log footprint with six traces, of which five exhibit control-flow anomalies.}
\label{FP_ANOMALIES}
\end{figure}

\subsection{Control-flow anomaly detection}
Control-flow anomaly detection techniques aim to characterize the normal control flow from event logs and verify whether these deviations occur in new event logs \cite{ko2023adsystematicreview}. To develop control-flow anomaly detection techniques, \revision{process mining} has seen widespread adoption owing to process discovery and \revision{conformance checking}. On the one hand, process discovery is a set of algorithms that encode control-flow relations as a set of model elements and constraints according to a given modeling formalism \cite{aalst2022pmhandbook}; hereafter, we refer to the Petri net, a widespread modeling formalism. On the other hand, \revision{conformance checking} is an explainable set of algorithms that allows linking any deviations with the reference Petri net and providing the fitness measure, namely a measure of how much the Petri net fits the new event log \cite{aalst2022pmhandbook}. Many control-flow anomaly detection techniques based on \revision{conformance checking} (hereafter, \revision{conformance checking}-based techniques) use the fitness measure to determine whether an event log is anomalous \cite{bezerra2009pmad, bezerra2013adlogspais, myers2018icsadpm, pecchia2020applicationfailuresanalysispm}. 

The scientific literature also includes many \revision{conformance checking}-independent techniques for control-flow anomaly detection that combine specific types of trace encodings with machine/deep learning \cite{ko2023adsystematicreview, tavares2023pmtraceencoding}. Whereas these techniques are very effective, their explainability is challenging due to both the type of trace encoding employed and the machine/deep learning model used \cite{rawal2022trustworthyaiadvances,li2023explainablead}. Hence, in the following, we focus on the shortcomings of \revision{conformance checking}-based techniques to investigate whether it is possible to support the development of competitive control-flow anomaly detection techniques while maintaining the explainable nature of \revision{conformance checking}.
\begin{figure}[!t]
\centering
\includegraphics[width=\columnwidth]{images/HIGH_LEVEL_VIEW.png}
\caption{A high-level view of the proposed framework for combining \revision{process mining}-based feature extraction with dimensionality reduction for control-flow anomaly detection.}
\label{HIGH_LEVEL_VIEW}
\end{figure}

\subsection{Shortcomings of \revision{conformance checking}-based techniques}
Unfortunately, the detection effectiveness of \revision{conformance checking}-based techniques is affected by noisy data and low-quality Petri nets, which may be due to human errors in the modeling process or representational bias of process discovery algorithms \cite{bezerra2013adlogspais, pecchia2020applicationfailuresanalysispm, aalst2016pm}. Specifically, on the one hand, noisy data may introduce infrequent and deceptive control-flow relations that may result in inconsistent fitness measures, whereas, on the other hand, checking event logs against a low-quality Petri net could lead to an unreliable distribution of fitness measures. Nonetheless, such Petri nets can still be used as references to obtain insightful information for \revision{process mining}-based feature extraction, supporting the development of competitive and explainable \revision{conformance checking}-based techniques for control-flow anomaly detection despite the problems above. For example, a few works outline that token-based \revision{conformance checking} can be used for \revision{process mining}-based feature extraction to build tabular data and develop effective \revision{conformance checking}-based techniques for control-flow anomaly detection \cite{singh2022lapmsh, debenedictis2023dtadiiot}. However, to the best of our knowledge, the scientific literature lacks a structured proposal for \revision{process mining}-based feature extraction using the state-of-the-art \revision{conformance checking} variant, namely alignment-based \revision{conformance checking}.

\subsection{Contributions}
We propose a novel \revision{process mining}-based feature extraction approach with alignment-based \revision{conformance checking}. This variant aligns the deviating control flow with a reference Petri net; the resulting alignment can be inspected to extract additional statistics such as the number of times a given activity caused mismatches \cite{aalst2022pmhandbook}. We integrate this approach into a flexible and explainable framework for developing techniques for control-flow anomaly detection. The framework combines \revision{process mining}-based feature extraction and dimensionality reduction to handle high-dimensional feature sets, achieve detection effectiveness, and support explainability. Notably, in addition to our proposed \revision{process mining}-based feature extraction approach, the framework allows employing other approaches, enabling a fair comparison of multiple \revision{conformance checking}-based and \revision{conformance checking}-independent techniques for control-flow anomaly detection. Figure \ref{HIGH_LEVEL_VIEW} shows a high-level view of the framework. Business processes are monitored, and event logs obtained from the database of information systems. Subsequently, \revision{process mining}-based feature extraction is applied to these event logs and tabular data input to dimensionality reduction to identify control-flow anomalies. We apply several \revision{conformance checking}-based and \revision{conformance checking}-independent framework techniques to publicly available datasets, simulated data of a case study from railways, and real-world data of a case study from healthcare. We show that the framework techniques implementing our approach outperform the baseline \revision{conformance checking}-based techniques while maintaining the explainable nature of \revision{conformance checking}.

In summary, the contributions of this paper are as follows.
\begin{itemize}
    \item{
        A novel \revision{process mining}-based feature extraction approach to support the development of competitive and explainable \revision{conformance checking}-based techniques for control-flow anomaly detection.
    }
    \item{
        A flexible and explainable framework for developing techniques for control-flow anomaly detection using \revision{process mining}-based feature extraction and dimensionality reduction.
    }
    \item{
        Application to synthetic and real-world datasets of several \revision{conformance checking}-based and \revision{conformance checking}-independent framework techniques, evaluating their detection effectiveness and explainability.
    }
\end{itemize}

The rest of the paper is organized as follows.
\begin{itemize}
    \item Section \ref{sec:related_work} reviews the existing techniques for control-flow anomaly detection, categorizing them into \revision{conformance checking}-based and \revision{conformance checking}-independent techniques.
    \item Section \ref{sec:abccfe} provides the preliminaries of \revision{process mining} to establish the notation used throughout the paper, and delves into the details of the proposed \revision{process mining}-based feature extraction approach with alignment-based \revision{conformance checking}.
    \item Section \ref{sec:framework} describes the framework for developing \revision{conformance checking}-based and \revision{conformance checking}-independent techniques for control-flow anomaly detection that combine \revision{process mining}-based feature extraction and dimensionality reduction.
    \item Section \ref{sec:evaluation} presents the experiments conducted with multiple framework and baseline techniques using data from publicly available datasets and case studies.
    \item Section \ref{sec:conclusions} draws the conclusions and presents future work.
\end{itemize}
\section{Lower Bound}
This section presents our main lower bound. As stated above, the lower bound is nearly tight, apart from lower-order terms and the dependency on $\eps$.

\begin{theorem}
\label{thm:lb}
There exists a distribution $\calP$ and a loss function $f$  satisfying Assumption~\ref{assum:lispchitz_smooth} and Assumption~\ref{assump:dia_dominant}, such that for any $(\epsilon,\delta)$-User-level-DP algorithm $\calM$, given i.i.d. dataset $\calD$ drawn from $\calP$, the output of $\calM$ satisfies
\begin{align*}
    \E[F(\calM(\calD))-F(x^*)]\ge GD\cdot \Tilde{\Omega}\Big(\min\Big\{d,\frac{d}{\sqrt{mn}}+\frac{d^{3/2}}{n\epsilon\sqrt{m}}\Big\}\Big).
\end{align*}
%where $F(x):=\E_{z\sim \calP}f(x;z)$ and $x^*=\arg\min_{x\in\calX}f(x)$.
\end{theorem}

The non-private term $GD\frac{d}{\sqrt{mn}}$ represents the information-theoretic lower bound for SCO under these assumptions (see, e.g., Theorem 1 in \cite{agarwal2009information}).  


We construct the hard instance as follows:
let $\calX=[-1,1]^d$ be unit $\ell_\infty$-ball and let $f(x;z)=-\langle x,z\rangle$ for any $x\in \calX$ be the linear function.
Let $z\in[-\sqrt{m},\sqrt{m}]^d$  with $\E_{z\sim\calP}[z]=\mu$.
Then one can easily verify that $f$ satisfies Assumptions~\ref{assum:lispchitz_smooth} and~\ref{assump:dia_dominant} with $G=\sqrt{m},D=1$ and $\beta=0$.
We have
\begin{align}
    F(\calM(\calD))-F(x^*) &= \sum_{i=1}^{d} (\sign(\mu[i])-\calM(\calD)[i])\cdot\mu[i]
   \nonumber \\ &\ge \sum_{i=1}^{d}  |\mu[i]|.\ind\big(\sign(\mu[i])\neq\sign(\calM(\calD)[i])\big). \label{eq:opt_error_to_sign_error}
\end{align}

By~\eqref{eq:opt_error_to_sign_error}, we reduce the optimization error to the weighted sign estimation error.  
Most existing lower bounds rely on the $\ell^2_2$-error of mean estimation.  
We adapt their techniques, especially the fingerprinting lemma, and provide the proof in the Appendix~\ref{sec:lbproof}.
\begin{algorithm}[ht!]
\caption{\textit{NovelSelect}}
\label{alg:novelselect}
\begin{algorithmic}[1]
\State \textbf{Input:} Data pool $\mathcal{X}^{all}$, data budget $n$
\State Initialize an empty dataset, $\mathcal{X} \gets \emptyset$
\While{$|\mathcal{X}| < n$}
    \State $x^{new} \gets \arg\max_{x \in \mathcal{X}^{all}} v(x)$
    \State $\mathcal{X} \gets \mathcal{X} \cup \{x^{new}\}$
    \State $\mathcal{X}^{all} \gets \mathcal{X}^{all} \setminus \{x^{new}\}$
\EndWhile
\State \textbf{return} $\mathcal{X}$
\end{algorithmic}
\end{algorithm}

% !TeX root = ../all.tex
\begin{figure*}[!ht]
	\centering
	\subfigure[Number of samples before stopping in a random BAI instance, logarithmic scale]{\includegraphics[width=0.3\textwidth]{plot_samp.png}\label{fig:exp}
	}\hspace{1em}
	\subfigure[Number of rounds before stopping in a random BAI instance]{\includegraphics[width=0.3\textwidth]{plot_round.png}
		\label{fig:expr}} \hspace{1em}
	\subfigure[Number of samples before stopping in the min. threshold setting, hard instance]{\includegraphics[width=0.3\textwidth]{TaSbad.png}
		
		\label{fig:exptas}}\label{fig:experiments}\caption{Experimental results, $\delta=0.05$, $N=1000$ runs}\end{figure*}
	
\subsection{Experiments on the BAI setting}


	
	
	

Our algorithm PET is near-optimal in round and sample complexities for many pure exploration problems, and has theoretical guarantees for any pure exploration problem. To ascertain its practical performances, we compare it to baselines and state of the art algorithms for best arm identification and thresholding bandits.	

Each experiment is repeated over 1000 runs. All reward distributions are Gaussian with variance 1 and we use the confidence level $\delta = 0.05$, which is chosen for its relevance to statistical practice. We compare
\begin{itemize}[noitemsep]
	\item Round Robin (or uniform sampling), where the stopping rule is checked only at timesteps $(900\times2^r)_{r \ge 1}$;
	\item Track-and-Stop (TaS) \citep{garivierOptimalBestArm2016}, where the empirical value of $w$ is updated only at timesteps $(900\times 2^r)_r$, and the stopping rule is only checked at those times;
	\item Our algorithm PET, with $T_0 = 1$;
	\item Opt-BBAI \citep{jinOptimalBatchedBest2023} with $\alpha = 1.05$ and the quantities described in their Theorem 4.2.
\end{itemize}
The initial batch sizes for TaS and Round Robin were chosen to approximate the initial batch size of our algorithm, to not disadvantage them in terms of round complexity. We modified TaS in order to turn it into a batch algorithm. Note that there is no formal guarantee for the batch or sample complexity of that modification of TaS, but we use it as a sensible baseline. 

For the BAI experiment, we run each algorithm on $10$-arm instances where the best arm has mean $1$, and each other arm $i$ has mean uniformly sampled between $0.6$ and $0.9$.
See Figure~\ref{fig:exp} for the box plots of the sample complexities. The mean is indicated by a black cross.
While both our algorithm and Opt-BBAI use similarly few batches, PET outperforms Opt-BBAI for the sample complexity.
That algorithm is asymptotically optimal as $\delta\rightarrow 0$ but it uses batches that seem to be too large for moderate values of $\delta$ like the $0.05$ we use.

While the batch modification of TaS might seem to be a good alternative for the BAI experiment, there are instances of the thresholding setting where it performs sub-optimally.
That effect that was first observed in \citep{degenneNonAsymptoticPureExploration2019} for the fully online TaS and reflects that, contrary to our results, the sample complexity guarantees of TaS are only asymptotic. 
We run the algorithms on a thresholding bandit with threshold 0.6 and two arms with means 0.5 and 0.6 and observe that batched TaS has high average sample complexity (see Figure~\ref{fig:exptas}; the mean is the black cross), while PET does not.




\section{Conclusion}
In this work, we propose a simple yet effective approach, called SMILE, for graph few-shot learning with fewer tasks. Specifically, we introduce a novel dual-level mixup strategy, including within-task and across-task mixup, for enriching the diversity of nodes within each task and the diversity of tasks. Also, we incorporate the degree-based prior information to learn expressive node embeddings. Theoretically, we prove that SMILE effectively enhances the model's generalization performance. Empirically, we conduct extensive experiments on multiple benchmarks and the results suggest that SMILE significantly outperforms other baselines, including both in-domain and cross-domain few-shot settings.

\section*{Acknowledgements}
	The authors acknowledge the funding of the French National Research Agency under the project FATE (ANR22-CE23-0016-01) and the PEPR IA FOUNDRY project (ANR-23-PEIA-0003). 
	The authors are members of the Inria team Scool.


\bibliographystyle{apalike}
\bibliography{bibli}


%%%%%%%%%%%%%%%%%%%%%%%%%%%%%%%%%%%%%%%%%%%%%%%%%%%%%%%%%%%%%%%%%%%%%%%%%%%%%%%
%%%%%%%%%%%%%%%%%%%%%%%%%%%%%%%%%%%%%%%%%%%%%%%%%%%%%%%%%%%%%%%%%%%%%%%%%%%%%%%
% APPENDIX
%%%%%%%%%%%%%%%%%%%%%%%%%%%%%%%%%%%%%%%%%%%%%%%%%%%%%%%%%%%%%%%%%%%%%%%%%%%%%%%
%%%%%%%%%%%%%%%%%%%%%%%%%%%%%%%%%%%%%%%%%%%%%%%%%%%%%%%%%%%%%%%%%%%%%%%%%%%%%%%
\newpage
\appendix
\onecolumn
% !TeX root = ../all.tex

\section{Proofs of the lower bounds}\label{app:lb}
\subsection{Preliminary lemmas}

For the sake of completeness, we start by restating and proving some results from \citep{taoCollaborativeLearningLimited2019} in slightly more general language.
\begin{definition}
	For some integers $r$ and $n$, define $\tau_\delta^r$ the number of samples before the end of round $r$.
\end{definition}
\begin{lemma}[Generalization of Lemma 27 of \citep{taoCollaborativeLearningLimited2019}]\label{lem:27f}
	For an algorithm, two instances ${\bm\nu}$ and ${\bm\nu}'$ and $r\in\bN$, \[\bP_{{\bm\nu}'}\{R_\delta\geq r+1,\tau_\delta^{r+1} \leq n+m\}\geq \bP_{\bm\nu}\{R_\delta \geq r+1,\tau_\delta^r\leq m\}-\bP_{\bm\nu} \{\tau_\delta>n\} -\Vert\cD_{\bm\nu}^m -\cD_{{\bm\nu}'}^m\Vert_{TV}\] where $\cD^m_{\bm\nu}$ is the distribution of rewards the algorithm got from $\bm\nu$ over $m$ steps.
\end{lemma}

\begin{proof}
	Fix a deterministic algorithm. 
	
	
	First of all, \begin{equation}\label{eq:mpntomn}(R_\delta \geq r+1,\tau_\delta^{r} \leq m,\tau_\delta^{r+1}-\tau_\delta^r < n) \subseteq (R_\delta \geq r+1,\tau_\delta^{r+1}\leq n+m)\end{equation}
	
	And, since $(R_\delta \geq r+1,\tau_\delta^{r} \leq m,\tau_\delta^{r+1}-\tau_\delta^r < n)$ is determined by the first $m$ rewards (at the end of round $r$ using less than $m$ samples, the algorithm must choose the length of round $r+1$), \begin{equation}\label{eq:distprob} \bP_{{\bm\nu}'}\{R_\delta \geq r+1,\tau_\delta^{r} \leq m,\tau_\delta^{r+1}-\tau_\delta^r < n\} \geq \bP_{\bm\nu} \{R_\delta \geq r+1,\tau_\delta^{r} \leq m,\tau_\delta^{r+1}-\tau_\delta^r < n\}  -\Vert\cD_{\bm\nu}^m -\cD_{{\bm\nu}'}^m\Vert_{TV}\end{equation}

On the other hand, \begin{align*}
	(R_\delta \geq r+1,\tau_\delta^r\leq m)\setminus (R_\delta\geq r+1,\tau_\delta^r\leq m,\tau_\delta^{r+1}-\tau_\delta^r < n) &= (R_\delta\geq r+1,\tau_\delta^r\leq m, \tau_\delta^{r+1}-\tau_\delta^r \geq n) \\
	&\subseteq (\tau_\delta >n)
\end{align*} hence \begin{equation}\label{eq:ajoutround}\bP_{\bm\nu}\{R_\delta\geq r+1,\tau_\delta^r\leq m,\tau_\delta^{r+1}-\tau_\delta^r < n\}\geq \bP_{\bm\nu}\{R_\delta \geq r+1,\tau_\delta^r\leq m\}-\bP_{\bm\nu}\{\tau_\delta >n\}\end{equation}

	Hence, using Equations \eqref{eq:mpntomn},~\eqref{eq:distprob} then~\eqref{eq:ajoutround}, \begin{align*}
		\bP_{{\bm\nu}'}\{R_\delta \geq r+1,\tau_\delta^{r+1}\leq n+m\}&\geq \bP_{\bm\nu'} \{R_\delta \geq r+1,\tau_\delta^{r} \leq m,\tau_\delta^{r+1}-\tau_\delta^r < n\}\\
		&\geq \bP_{\bm\nu} \{R_\delta \geq r+1,\tau_\delta^{r} \leq m,\tau_\delta^{r+1}-\tau_\delta^r < n\} -\Vert\cD_{\bm\nu}^m -\cD_{{\bm\nu}'}^m\Vert_{TV} \\
		&\geq \bP_{\bm\nu}\{R_\delta \geq r+1,\tau_\delta^r\leq m\}-\bP_{\bm\nu} \{\tau_\delta>n\} -\Vert\cD_{\bm\nu}^m -\cD_{{\bm\nu}'}^m\Vert_{TV}
	\end{align*}
	
\end{proof}



\begin{lemma}[Generalization of Lemma 26 of \citep{taoCollaborativeLearningLimited2019}]\label{lem:26f_aux}
	For any $\delta$-correct algorithm, for all $m,r\in\bN$ and any two bandit instances ${\bm\nu}, {\bm\nu}'$, 
	we have
	\begin{align*}
	\bP_{\bm\nu}\{R_\delta\geq r+1,\tau_\delta^r\leq m\}
	\ge \bP_{\bm\nu}\{R_\delta\geq r,\tau_\delta^r\leq m\} - 2\delta - \Vert \cD_{\bm\nu}^m - \cD_{{\bm\nu}'}^m \Vert_{TV}
	\: .
	\end{align*}
\end{lemma}

\begin{proof}
Consider the event $\mathcal{F}_1=(R_\delta=r,\tau_\delta \leq m)$.
Denote by $\mathcal{F}_2$ the event that the algorithm returns the best arm of instance ${\bm\nu}$.
Then \(\bP_{\bm\nu}\{\mathcal{F}_1\}=\bP_{\bm\nu}\{\mathcal{F}_1\wedge \mathcal{F}_2\}+\bP_{\bm\nu}\{\mathcal{F}_1\wedge \overline{\mathcal{F}_2}\}\)
	
With $\mathcal{D}_{\bm\nu}^m$ the distribution of rewards over $m$ samples and some ${\bm\nu}'\in Alt_{\bm\nu}$,
\begin{align*}
\bP_{\bm\nu}\{\mathcal{F}_1\wedge \mathcal{F}_2\}
&\leq \bP_{{\bm\nu}'}\{\mathcal{F}_1\wedge \mathcal{F}_2\}+\Vert\cD_{\bm\nu}^m -\cD_{{\bm\nu}'}^m\Vert_{TV}
\\
&\leq \bP_{{\bm\nu}'}\{\mathcal{F}_2\} +\Vert\cD_{\bm\nu}^m -\cD_{{\bm\nu}'}^m\Vert_{TV}
\\
&\leq \delta+\Vert\cD_{\bm\nu}^m -\cD_{{\bm\nu}'}^m\Vert_{TV}
\: .
\end{align*}
On the other hand, $\mathbb{P}_{\bm\nu}\{\mathcal F_1 \wedge \overline{\mathcal F}_2\} \le \mathbb{P}_{\bm\nu}\{\overline{\mathcal F}_2\} \le \delta$.
Therefore $\bP_{\bm\nu}\{\mathcal{F}_1\}\leq 2\delta + \Vert\cD_{\bm\nu}^m -\cD_{{\bm\nu}'}^m\Vert_{TV}$.
Using \( \bP_{\bm\nu}\{R_\delta\geq r+1,\tau_\delta^r\leq m\}\geq \bP_{\bm\nu}\{R_\delta\geq r,\tau_\delta^r\leq m\}-\bP_{\bm\nu}(\mathcal{F}_1)\), we conclude.
\end{proof}


\begin{lemma}\label{lem:26f}
	For any $\delta$-correct algorithm, for all $m,r\in\bN$ and any bandit instance ${\bm\nu}$, 
	we have
	\begin{align*}
	\bP_{\bm\nu}\{R_\delta\geq r+1,\tau_\delta^r\leq m\}
	\ge \bP_{\bm\nu}\{R_\delta\geq r,\tau_\delta^r\leq m\} - 2\delta - \sqrt{\frac{m}{2} (T^\star(\bm\nu))^{-1}}
	\: .
	\end{align*}
\end{lemma}



\begin{proof}
First apply Lemma~\ref{lem:26f_aux} to an arbitrary instance ${\bm\nu}' \in Alt_{\bm\nu}$. Then using Pinsker's inequality yields
\begin{align*}
\Vert \cD_{\bm\nu}^m - \cD_{{\bm\nu}'}^m \Vert_{TV}
\leq \sqrt{\frac{1}{2}\KL(\cD_{\bm\nu}^m\Vert \cD_{{\bm\nu}'}^m)}
= \sqrt{\frac{1}{2} \sum_{i\in[K]} \bE_{\bm\nu}[N_{m,i}] \frac{(\mu_i-\mu_i')^2}{2}}
\end{align*} with $N_{m,i}$ the number of times arm $i$ is pulled before time $m$.

As this is true for all instances ${\bm\nu}'\in Alt_{{\bm\nu}}$, we can obtain an inequality using the infimum over those instances,
\begin{align*}
\inf_{{\bm\nu}' \in Alt_{\bm\nu}} \Vert \cD_{\bm\nu}^m - \cD_{{\bm\nu}'}^m \Vert_{TV}
&\le \sqrt{\frac{m}{2} \inf_{\bm\lambda \in Alt_{{\bm\nu}}} \sum_{i\in[K]} \frac{\bE_{\bm\nu}[N_{m,i}]}{m} \frac{(\mu_i-\lambda_i)^2}{2}}
\\
&\le \sqrt{\frac{m}{2} \sup_{w\in \Sigma_K} \inf_{\bm\lambda \in Alt_{{\bm\nu}}} \sum_{i\in[K]} w_i \frac{(\mu_i-\lambda_i)^2}{2}}
\\
&=   \sqrt{\frac{m}{2} (T^\star(\bm\nu))^{-1}}
\: ,
\end{align*}
by definition of $T^\star$.
\end{proof}





Finally, we also give a technical result to solve inequalities of the form $(k+N^2(a+b\ln N))^N\leq \rho$.

\begin{lemma}\label{lem:suffN}
	Let $\rho \ge e$, $a,b\geq 0$ and $k$ be real numbers, and let $A=\max\{e,k+a\}$.
	Then $N \coloneqq \left\lfloor \frac{\ln \rho}{\ln((\ln \rho)^2(A+b\ln \ln \rho))}\right\rfloor$ satisfies $(k+N^2(a+b\ln N))^N\leq \rho$~.
\end{lemma}

\begin{proof}
	If $N=0$, the equality is $1 \le \rho$, which is true since $\rho \ge e$. Otherwise, $N\geq 1$ and $(\ln \rho)^2(A+b\ln\ln \rho)\geq A\geq e$,
	so $N \le \lfloor \ln\rho / \ln e \rfloor \leq \ln \rho$. Therefore
	\begin{align*}
		N\ln(k+N^2(a+b\ln N))&\leq N\ln (N^2(A+b\ln N))
		\\
		&\leq N\ln((\ln \rho)^2(A+b\ln \ln \rho))
		\\
		&\leq \ln \rho
	\end{align*}
	and finally $(k+N^2(a+b\ln N))^N\leq \rho$~.
\end{proof}

\subsection{The lower bound in the general cases}

We give here a result for any sequence of instances.

\begin{restatable}[]{lemma}{lemrec}\label{lem:rec} 
	Let there be a sequence of instances $({\bm\nu}^n)_{0\leq n\leq N}$ such that the probability of error is bounded by $\delta$ and for any $n\in[0,N-1]$, $c_n \geq \bP_{{\bm\nu}^n} [\tau_\delta >x_n]$.
	Then \begin{align*} \bP_{{\bm\nu}^N}[R_\delta>N] &\geq 1-2N\delta -\sum_{i=0}^{N-1}\left[  c_n+ \sqrt{\frac{X_{n-1}}{2}} \left(\sqrt{ \frac{1}{T^\star(\bm\mu^n)} }  +\sqrt{\sum_{i\in[K]}\frac{\bE[N_{X_{n-1},i}]}{X_{n-1}} \frac{(\mu_i^{n+1}-\mu_i^n)^2}{2\sigma^2}}\right)\right]\end{align*} where $X_n=\sum_{i=-1}^n x_i$, $x_{-1}$ is any positive real number, and $N_{t,i}$ is the number of times arm $i$ is sampled before time $t$.
\end{restatable}

\begin{proof}[Proof of Lemma~\ref{lem:rec}] 
	By lemmas \ref{lem:27f} and \ref{lem:26f}, for any $m$, \begin{align*} 
		\bP_{{\bm\nu}^{n+1}}\{R_\delta\geq n+1,\tau_\delta^{n+1}\leq m+x_n\}  &\geq \bP_{{\bm\nu}^n}\{R_\delta\geq n+1,\tau_\delta^n\leq m\}-c_n-\Vert\cD_{{\bm\nu}^n}^m-\cD_{{\bm\nu}^{n+1}}^m\Vert_{TV}\\
		&\geq \bP_{{\bm\nu}^{n}}\{R_\delta \geq n,\tau_\delta^n\leq m\}-2\delta-\sqrt{\frac{m}{2} (T^\star(\bm\mu^n))^{-1}}\\
		&\hspace{1.5em}-c_n-\sqrt{\frac{1}{2}\sum_{i\in[K]}\bE[N_{m,i}] \frac{(\mu_i^{n+1}-\mu_i^n)^2}{2\sigma^2}}
	\end{align*} and with $X_n=\sum_{i=-1}^{n} x_i$, \begin{align*}\bP_{{\bm\nu}^{n+1}}\{R_\delta\geq n+1,\tau_\delta^{n+1}\leq X_n\} & \geq \bP_{{\bm\nu}^{n}}\{R_\delta\geq n,\tau_\delta^n\leq X_{n-1}\}-2\delta-c_n -\sqrt{\frac{X_{n-1}}{2} (T^\star(\bm\mu))^{-1}}\\
	&\hspace{1.5em} -\sqrt{\frac{X_{n-1}}{2}\sum_{i\in[K]}\frac{\bE[N_{X_{n-1},i}]}{X_{n-1}} \frac{(\mu_i^{n+1}-\mu_i^n)^2}{2\sigma^2}}\end{align*} So that finally \begin{align*}
	\bP_{{\bm\nu}^{N}}\{R_\delta\geq N,\tau_\delta^N\leq X_{N-1}\} &\geq \bP_{{\bm\nu}^{0}}\{R_\delta\geq 0,\tau_\delta^0\leq x_{-1}\}-2N\delta\\
	&\hspace{1.5em} -\sum_{i=0}^{N-1}\left[ c_n+ \sqrt{\frac{X_{n-1}}{2}}\left(\sqrt{ (T^\star(\bm\mu^n))^{-1} } +\sqrt{\sum_{i\in[K]}\frac{\bE[N_{X_{n-1},i}]}{X_{n-1}} \frac{(\mu_i^{n+1}-\mu_i^n)^2}{2\sigma^2}}\right)\right]\end{align*} and we conclude since for any $x_{-1}\geq 0$, $\bP_{{\bm\nu}^0}\{R_\delta\geq 1,\tau_\delta^0\leq x_{-1}\}=1$ (we always use at least 1 round).
\end{proof}

From there, we derive the result for $T^\star(\bm\mu^n)=\zeta^{-n} T^\star(\bm\mu^0)$.

\theorec* 


\begin{proof}[Proof of Lemma~\ref{th:theorec}]
	We apply Lemma~\ref{lem:rec} on the sequence $(\bm\nu^n)_{0\leq n\leq N}$ with $x_{-1}=\gamma T^\star(\bm\mu^0)\log(1/\delta)\frac{1}{\zeta^{-1}-1}$. That way, \begin{align*} X_{n} &= x_{-1} +\sum_{i=0}^n \gamma T^\star(\bm\mu^i)\log(1/\delta)\\
		&=\gamma T^\star(\bm\mu^0)\log(1/\delta)\left( \frac{1}{\zeta^{-1}-1}+\sum_{i=0}^n \zeta^{-i}\right)\\
		&=\gamma T^\star(\bm\mu^0)\log(1/\delta)\frac{\zeta^{-(n+1)}}{\zeta^{-1}-1}
	\end{align*}
	
\end{proof}

Under Assumption~\ref{asm:aff}, we can pick a sequence of instances of means $\bm\mu^{n+1}=x\bm\mu^n+(1-x)\bm y$ and control the sequence of $T^\star(\bm\mu^n)$. That way, we get the following result:
\begin{restatable}[Batch lower bound on affine sequences]{lemma}{lembar}\label{lem:bar}
	For problems on which Assumption~\ref{asm:aff} is satisfied;
	for any algorithm such that, for any Gaussian instance $\bm\nu$ satisfying $T^\star(\bm\mu)\in (T_{\min},T_{\max})$ the probability of error is smaller than $\delta$ and such that $\bP_{\bm\nu}(\tau_\delta>\gamma\log(1/\delta) T^\star(\bm\mu))\leq c$; we have for any $\sigma$-Gaussian instance $\bm\nu$ of complexity $T^\star(\bm\mu)\in (T_{\min},T_{\max})$, for the corresponding $y\in \R$ given by Assumption~\ref{asm:aff} for $\bm\mu$, that $\bP_{\bm\nu} (R_\delta\geq N)\geq 1/2$ for \[N = \min\left\{\frac{\ln \frac{T^\star(\bm\mu)}{T_{\min}}}{\ln\left( \left(\ln \frac{T^\star(\bm\mu)}{T_{\min}}\right)^2 \max\{e,C\} \right)},\frac{1}{2\delta+c}\right\}\] with $C=1+4\gamma\log(\frac{1}{\delta})\left(1+\sqrt{\frac{T^\star(\bm\mu)\Delta^2}{2\sigma^2}}\right)^2$ and $\Delta = \max_i |\mu_i -y|$.
\end{restatable}

\begin{proof}
	Fix some ($\sigma$-Gaussian) instance $\bm\nu^0=\bm\nu$ of complexity $T^\star(\bm\mu)=T_0\in(T_{\min},T_{\max})$. 
	
	For some $\zeta\in (0,1)$ to be fixed later, define the instance of mean $\bm\mu^{n+1}=\zeta^{-1/2}\bm\mu^n+(1-\zeta^{-1/2})\bm y$. We then have $\zeta^nT^\star(\bm\mu^0)= T^\star(\bm\mu^{n})$ by hypothesis. We can thus construct a sequence of instances of length $N$ as long as $\zeta^{N} > \frac{T_{\min}}{T^\star(\bm\mu^0)}$.
	
	\begin{align*} \frac{(\mu_i^{N-n-1}-\mu_i^{N-n})^2}{2\sigma^2} &= \frac{(\zeta^{-(N-n-1)/2}-\zeta^{-(N-n)/2})^2(\mu_i^0-y)^2}{2\sigma^2}\\ &\leq \frac{\zeta^{n-N}\Delta^2}{2\sigma^2} (1-\zeta^{1/2})^2\end{align*}
	We apply Theorem~\ref{th:theorec} on the reversed sequence $\left( \bm\nu_{N-i}\right)_{0\leq i\leq N}$:
	\begin{align*}\bP_{{\bm\nu}}(R_\delta>N) &\geq 1-N(2\delta+c)-\sqrt{\frac{\gamma\log(1/\delta)}{2(\zeta^{-1}-1)}}\times \sum_{i=0}^{N-1} \left[ 1 +\sqrt{T^\star(\bm\mu)\sup_{w\in \Delta_K}\sum_{i\in[K]}w_i \frac{\Delta^2}{2\sigma^2} (1-\zeta^{1/2})^2}\right]\\
		&\geq 1-N\left(2\delta+c+\sqrt{\frac{\gamma\log(1/\delta)}{2(\zeta^{-1}-1)}} \left(1+\sqrt{\frac{T^\star(\bm\mu^0)\Delta^2}{2\sigma^2}}\right)\right)\\
		&\geq 5/8-N(2\delta+c)\end{align*} for \[\zeta =\Bigg( 1+4N^2\gamma\log\left(\frac{1}{\delta}\right)\left(1+\sqrt{\frac{T^\star(\bm\mu^0)\Delta^2}{2\sigma^2}}\right)^2\Bigg)^{-1}\]
	
	
	We can apply Lemma~\ref{lem:suffN} with $\rho = \frac{T^\star(\bm\mu^0)}{T_{\min}}$, $k=1$, $b=0$, and $a=4\gamma\log(1/\delta)\left(1+\sqrt{\frac{T^\star(\bm\mu^0)\Delta^2}{2\sigma^2}}\right)^2$. We get that a sufficient condition is \begin{equation} \label{eq:Nbai} N\leq \frac{\ln \frac{T^\star(\bm\mu^0)}{T_{\min}}}{\ln\left( \left(\ln \frac{T^\star(\bm\mu^0)}{T_{\min}}\right)^2 \max\left\{e,C\right\} \right)}\end{equation} with $C=1+4\gamma\log(1/\delta)\left(1+\sqrt{\frac{T^\star(\bm\mu^0)\Delta^2}{2\sigma^2}}\right)^2$.
	
	Therefore, by picking $N$ that satisfies \eqref{eq:Nbai} and $N\leq \frac{1}{8(2\delta+c)}$, we have that $\bP_{\bm\nu}(R_\delta > N)>\frac{1}{2}$.
\end{proof}




\begin{restatable}[Batch lower bound on affine sequences, expectation constraint]{lemma}{lembarexp}\label{lem:barexp}
	For problems on which Assumption~\ref{asm:aff} is satisfied; 
	for any $\delta$-correct algorithm such that, for any Gaussian instance $\bm\nu$ satisfying $T^\star(\bm\mu)\in (T_{\min},T_{\max})$ $\bE_{\bm\nu}[\tau_\delta]\leq \gamma\log(1/\delta) T^\star(\bm\mu)$, we have for any $\sigma$-Gaussian instance $\bm\nu$ of complexity $T^\star(\bm\mu)\in (T_{\min},T_{\max})$, for the corresponding $y\in\R$ given by Assumption~\ref{asm:aff} for $\bm\mu$ that $\bP_{\bm\nu} (R_\delta\geq N)\geq 1/2$ for \[ N \geq \min\left\{ \frac{\ln \frac{T^\star(\bm\mu)}{T_{\min}}}{\ln\left( \left(\ln \frac{T^\star(\bm\mu)}{T_{\min}}\right)^2 \max\{e,C_\delta'\}  \right)},\frac{1}{3} \ln \frac{T^\star(\bm\mu)}{T_{\min}},\frac{1}{3\delta}\right\}\] with $C_\delta'=\max\left\{e,1+4\gamma\log(1/\delta)\ln \frac{T^\star(\bm\mu)}{T_{\min}}\left(1+\sqrt{\frac{T^\star(\bm\mu)\Delta^2}{2\sigma^2}}\right)^2 \right\}$ and $\Delta = \max_i |\mu_i -y|$.
\end{restatable}

\begin{proof}
For instance $\bm\nu$, for some algorithm satisfying $\bE_{\bm\nu}[\tau_\delta]\leq \gamma \log(1/\delta) T^\star(\bm\mu)$, we have by the Markov inequality that $\bP_{\bm\nu}(\tau_\delta \geq (\gamma/c) T^\star(\bm\mu) \log(1/\delta)) \leq c$.
Applying Lemma~\ref{lem:bar}, $\bP_{\bm\nu} (R_\delta\geq N)\geq 1/2$ for \[N = \min\left\{\frac{\ln \frac{T^\star(\bm\mu)}{T_{\min}}}{\ln\left( \left(\ln \frac{T^\star(\bm\mu)}{T_{\min}}\right)^2 \max\{e,C\} \right)},\frac{1}{2\delta+c}\right\}\] with $C=1+4\gamma/c\log(\frac{1}{\delta})\left(1+\sqrt{\frac{T^\star(\bm\mu)\Delta^2}{2\sigma^2}}\right)^2$ and $\Delta = \max_i |\mu_i -y|$.

Choosing $c = \max\left\{ \delta,\left( \log\frac{T^\star(\bm\mu)}{T_{\min}}\right)^{-1}\right\}$, if $\delta < \left( \log\frac{T^\star(\bm\mu)}{T_{\min}}\right)^{-1}$, then \[ N \geq \min\left\{ \frac{\ln \frac{T^\star(\bm\mu)}{T_{\min}}}{\ln\left( \left(\ln \frac{T^\star(\bm\mu)}{T_{\min}}\right)^2 \max\{e,C_\delta'\}  \right)},\frac{1}{3} \ln \frac{T^\star(\bm\mu)}{T_{\min}}\right\}\] with $C_\delta'=\max\left\{e,1+4\gamma\log(1/\delta)\ln \frac{T^\star(\bm\mu)}{T_{\min}}\left(1+\sqrt{\frac{T^\star(\bm\mu)\Delta^2}{2\sigma^2}}\right)^2 \right\}$.

If $\delta \geq \left( \log\frac{T^\star(\bm\mu)}{T_{\min}}\right)^{-1}$, \[N \geq \min\left\{\frac{\ln \frac{T^\star(\bm\mu)}{T_{\min}}}{\ln\left( \left(\ln \frac{T^\star(\bm\mu)}{T_{\min}}\right)^2 \max\{e,C_\delta'\} \right)},\frac{1}{3\delta}\right\}\]



\end{proof}











\subsection{The top-$k$ and BAI settings}

All that remains is to show that our problems satisfy Assumption~\ref{asm:aff}. We start with top-$k$, and first give a technical result giving a simple formula for $T^\star(\bm\mu)$.

\begin{lemma}\label{lem:baiw}
	For any $\bm w\in \Sigma_K$, \begin{align*}\inf_{\bm\lambda \in Alt_{\bm\mu}} \left( \sum_{i\in [K]} w_i d(\mu_i,\lambda_i)\right) = \min_{\substack{b\geq k+1 \\ a\leq k}} w_a d(\mu_a,\mu_{ab})+w_b d(\mu_b,\mu_{ab})\end{align*} where $\mu_{ab}=\frac{w_a\mu_a +w_b\mu_b}{w_a+w_b}$ (arms are assumed to be ordered, $\mu_1\geq\mu_2\geq \dots$).
\end{lemma}

\begin{lemma}\label{lem:topkgoodbar}
	In the top-$k$ problem, setting $\bm\mu' = x\bm\mu + (1-x)\bm y$ where $\bm y$ is a constant vector and $x>0$, $Alt_{\bm\mu'}=Alt_{\bm\mu}$, $\Delta_i^{\bm\mu'}=x\Delta_i^{\bm\mu}$ and $(T^\star(\bm\mu'))^{-1}=x^2(T^\star(\bm\mu))^{-1}$.
\end{lemma}
\begin{proof}
	First of all, for any two arms $i,j$, $\mu'_i-\mu'_j = x(\mu_i-\mu_j)$ with $x>0$. Therefore, the ordering of arms is conserved, and $Alt_{\bm\mu'}=Alt_{\bm\mu}$. Moreover, since $\Delta_i^{\bm\mu'} = \mu'_i - \mu'_{k+1} =x(\mu_i-\mu_{k+1})=x\Delta_i^{\bm\mu}$ for $i\leq k$ and $\Delta_i^{\bm\mu'} = \mu'_k-\mu'_i=x\Delta_i^{\bm\mu}$ otherwise, we do have $\Delta_i^{\bm\mu'}=x\Delta_i^{\bm\mu}$.
	
	Furthermore,
	\begin{align*}
		(T^\star(\bm\mu'))^{-1}&= \sup_{w\in \Sigma_K}\inf_{\bm\lambda \in Alt_{\bm\mu}} \left( \sum_{i\in [K]} w_i \frac{(\mu_i'-\lambda_i)^2}{2\sigma^2}\right)\\
		&= \sup_{w\in \Sigma_K} \min_{\substack{b\geq k+1 \\ a\leq k}} w_a \frac{(\mu_a'-\mu_{ab}')^2}{2\sigma^2}+w_b \frac{(\mu_b'-\mu_{ab}')^2}{2\sigma^2}
	\end{align*} with \begin{align*}\mu_{ab}'(w)&=\frac{w_a\mu_a' +w_b\mu_b'}{w_a+w_b}=x\mu_{ab}+(1-x)y\end{align*}
	So that \begin{align*}
		(T^\star(\bm\mu'))^{-1}&=x^2\sup_{w\in \Sigma_K}\min_{\substack{b\geq k+1 \\ a\leq k}} w_a \frac{(\mu_a-\mu_{ab})^2}{2\sigma^2}+w_b \frac{(\mu_b-\mu_{ab})^2}{2\sigma^2}\\
		&=x^2(T^\star(\bm\mu))^{-1}
	\end{align*}


\end{proof}

With these results, we can apply Lemmas~\ref{lem:bar} and \ref{lem:barexp}. We see that the value of $y$ does not impact the proof: we thus choose the value that minimizes $\max_i |\mu_i-y|$, which is $y = \frac{\max_i \mu_i+\min_i\mu_i}{2}$.

%\thlbbaib*
	

	
%\begin{proof}\textbf{of Theorem~\ref{th:lbbaib}}
%	Thanks to Lemma~\ref{lem:topkgoodbar}, it suffices to apply Lemma~\ref{lem:bar} with $\nu = \frac{\mu_1+\mu_K}{2}$. We have $\Delta = |\mu_1-\mu_K|/2$.
%\end{proof}	

\subsection{The thresholding setting}

\begin{lemma}\label{lem:tbpgoodbar}
	In the thresholding bandit problem, setting $\bm\mu' = x\bm\mu + (1-x)\bm \tau$ where $\bm \tau$ is the constant vector of value $\tau$ the threshold and $x>0$, $Alt_{\bm\mu'}=Alt_{\bm\mu}$, $\Delta_i^{\bm\mu'}=x\Delta_i^{\bm\mu}$ and $(T^\star(\bm\mu'))^{-1}=x^2(T^\star(\bm\mu))^{-1}$.
\end{lemma}

\begin{proof}
	First of all, for any arm $i$, $\mu'_i -\tau = x(\mu_i-\tau)$ with $x>0$. Therefore, $Alt_{\bm\mu}=Alt_{\bm\mu'}$. Moreover, $\Delta_i^{\bm\mu'}=|\mu'_i-\tau|=x|\mu_i-\tau|=x\Delta_i^{\bm\mu'}$.
	
	Furthermore, \begin{align*}
		(T^\star(\bm\mu'))^{-1}&=\sup_{w\in \Sigma_K} \inf_{\bm\lambda\in Alt_{\bm\mu}} \left( \sum_{i\in[K]} w_i\frac{(\mu'_i-\lambda_i)^2}{2\sigma^2}\right)\\
		&=\sup_{w\in \Sigma_K} \sup_{i\in [K]} w_i \frac{(\mu'_i-\tau)^2}{2\sigma^2}\\
		&=x^2 \sup_{w\in \Sigma_K} \sup_{i\in [K]} w_i \frac{(\mu_i-\tau)^2}{2\sigma^2}\\
		&=x^2(T^\star(\bm\mu))^{-1}
	\end{align*}
\end{proof}



% !TeX root = ../all.tex


\section{Concentration and threshold for the stopping rule}
\label{app:concentration}

We suppose that each arm is sampled once during the first $K$ time steps.
\begin{theorem}
  \label{thm:bound_delta}
  Suppose that the arm distributions are $\sigma^2$-sub-Gaussian. Let $\hat{\mu}_{t,k}$ be the average of arm $k$ at time $t$ and $N_{t,k}$ be the number of times arm $k$ is sampled up to time $t$.
  With probability $1 - \delta$, for all $t > K$,
  \begin{align*}
  \frac{1}{2} \sum_{k=1}^K N_{t,k}\frac{(\hat{\mu}_{t,k}- \mu_k)^2}{2 \sigma^2}
  &\le \frac{K}{2} \overline{W}\left(2\ln \left(\frac{e\pi^2}{6}\right) + \frac{2}{K}\ln \left(\prod_{k=1}^K (1 + \ln N_{t,k})^2\right) + \frac{2}{K}\ln \frac{1}{\delta}\right)
  \: .
  \end{align*}
\end{theorem}



\subsection{Proof of the concentration theorem}

We can assume $w.l.o.g.$ that $\mu_k = 0$ for all $k$ and $\sigma^2 = 1$.

Let $S_{t,k} = \sum_{s=1}^t X_{s,k} \mathbb{I}\{k_s = k\}$.
We want a bound on $\frac{1}{2} \sum_{k=1}^K \frac{S_{t,k}^2}{N_{t,k}}$.


We first remark that $\frac{1}{2}x^2 = \sup_{\lambda} \lambda x - \frac{1}{2}\lambda^2$~. Apply that to $x = S_{t,k}/\sqrt{N_{t,k}}$ to get
\begin{align*}
\sum_{k=1}^K \frac{1}{2}\frac{S_{t,k}^2}{N_{t,k}}
&= \sup_{\lambda_1, \ldots, \lambda_K} \sum_{k=1}^K \left( \lambda_k S_{t,k} - \frac{1}{2}N_{t,k} \lambda_k^2 \right)
\\
&= \sup_{\lambda_1, \ldots, \lambda_K} \sum_{s=1}^t \lambda_{k_s}X_{s,k_s} - \frac{1}{2}\lambda_{k_s}^2
\: .
\end{align*}
The advantage of that formulation is that we can concentrate the sum for any fixed value of $\lambda$ (or any distribution on $\lambda$) thanks to a martingale argument.

\begin{lemma}
For all $\rho \in \mathcal P(\mathbb{R}^K)$, the process $t \mapsto \mathbb{E}_{\lambda \sim \rho}\left[\exp\left(\sum_{s=1}^t \lambda_{k_s}X_{s,k_s} - \frac{1}{2}\lambda_{k_s}^2\right)\right]$ is a non-negative supermartingale with expectation bounded by 1.
\end{lemma}


\begin{corollary}\label{cor:prob_exists_log_ge_le}
For all $\rho \in \mathcal P(\mathbb{R}^K)$ and $x \ge 0$,
\begin{align*}
\mathbb{P}\left(\exists t, \ \ln \mathbb{E}_{\lambda \sim \rho}\left[\exp\left(\sum_{s=1}^t \lambda_{k_s}X_{s,k_s} - \frac{1}{2}\lambda_{k_s}^2\right)\right] \ge x\right) \le e^{-x}
\: .
\end{align*}
Equivalently, for all $\delta \in (0,1]$,
\begin{align*}
\mathbb{P}\left(\exists t, \ \ln \mathbb{E}_{\lambda \sim \rho}\left[\exp\left(\sum_{s=1}^t \lambda_{k_s}X_{s,k_s} - \frac{1}{2}\lambda_{k_s}^2\right)\right] \ge \ln\frac{1}{\delta}\right) \le \delta
\: .
\end{align*}
\end{corollary}

\begin{proof}
Use Ville's inequality and the fact that the process is a non-negative supermartingale.
\end{proof}

We don't want to bound an integral over $\lambda \sim \rho$, but the supremum over $\lambda$, so we need to relate the two quantities.
We do that for Gaussian priors over $\lambda$.

\begin{lemma}\label{lem:integral_exp_eq_log_add}
For $\rho = \mathcal N(0, \mathrm{diag}(\sigma_k^{-2}))$,
\begin{align*}
\ln \mathbb{E}_{\lambda \sim \rho}\left[\exp\left(\sum_{s=1}^t \lambda_{k_s}X_{s,k_s} - \frac{1}{2}\lambda_{k_s}^2\right)\right]
&= -\frac{1}{2}\sum_{k=1}^K \ln(1 + N_{t,k}\sigma_k^{-2}) + \frac{1}{2} \sum_{k=1}^K \frac{S_{t,k}^2}{(N_{t,k} + \sigma_k^2)}
\: .
\end{align*}
\end{lemma}

\begin{proof}
\begin{align*}
&\mathbb{E}_{\lambda \sim \rho}\left[\exp\left(\sum_{s=1}^t \lambda_{k_s}X_{s,k_s} - \frac{1}{2}\lambda_{k_s}^2\right)\right]
\\
&= \prod_k \mathbb{E}_{\lambda_k \sim \mathcal N(0, \sigma_k^{-2})}\left[\exp\left(\lambda_k S_{t,k} - \frac{1}{2}N_{t,k} \lambda_k^2\right)\right]
\\
&= \prod_k \frac{1}{\sqrt{2 \pi \sigma_k^{-2}}}\int_{\lambda_k}\exp\left(\lambda_k S_{t,k} - \frac{1}{2}N_{t,k} \lambda_k^2 - \frac{\sigma_k^2}{2}\lambda_k^2\right)\mathrm{d}\lambda_k
\\ 
&= \prod_k \frac{1}{\sqrt{(1 + N_{t,k}\sigma_k^{-2})}} \frac{1}{\sqrt{2 \pi (N_{t,k} + \sigma_k^2)^{-1}}}
  \int_{\lambda_k} \exp\left(-\frac{1}{2}(N_{t,k} + \sigma_k^2)\left( \lambda_k - \frac{S_{t,k}}{(N_{t,k} + \sigma_k^2)} \right)^2 + \frac{1}{2}\frac{S_{t,k}^2}{N_{t,k} + \sigma_k^2}\right)\mathrm{d}\lambda_k
\\
&= \prod_k \frac{1}{\sqrt{(1 + N_{t,k}\sigma_k^{-2})}} \exp\left(\frac{1}{2}\frac{S_{t,k}^2}{N_{t,k} + \sigma_k^2}\right)
\\ 
\end{align*}

\end{proof}

\begin{corollary}\label{cor:sum_le_eta_mul}
Let $\rho = \mathcal N(0, \mathrm{diag}(\sigma_k^{-2}))$, $\eta_{t,\max} = \max_k \frac{\sigma_k^2}{N_{t,k}}$ and $\eta_{t,\min} = \min_k \frac{\sigma_k^2}{N_{t,k}}$. Then
\begin{align*}
\frac{1}{2} \sum_{k=1}^K \frac{S_{t,k}^2}{N_{t,k}}
&\le (1 + \eta_{t,\max}) \left(\ln \mathbb{E}_{\lambda \sim \rho}\left[\exp\left(\sum_{s=1}^t \lambda_{k_s}X_{s,k_s} - \frac{1}{2}\lambda_{k_s}^2\right)\right]
  + \frac{K}{2}\ln(1 + \eta_{t,\min}^{-1})\right)
\end{align*}

\end{corollary}

\begin{proof}
Using Lemma~\ref{lem:integral_exp_eq_log_add},
\begin{align*}
\frac{1}{2} \sum_{k=1}^K \frac{S_{t,k}^2}{N_{t,k} + \sigma_k^2}
&= \ln \mathbb{E}_{\lambda \sim \rho}\left[\exp\left(\sum_{s=1}^t \lambda_{k_s}X_{s,k_s} - \frac{1}{2}\lambda_{k_s}^2\right)\right]
  + \frac{1}{2} \sum_{k=1}^K \ln(1 + N_{t,k}\sigma_k^{-2})
\: .
\end{align*}
Then
\begin{align*}
\frac{1}{2} \sum_{k=1}^K \frac{S_{t,k}^2}{N_{t,k} + \sigma_k^2}
\ge \frac{1}{2} \sum_{k=1}^K \frac{S_{t,k}^2}{N_{t,k}(1 + \eta_{t,\max})}
= \frac{1}{1 + \eta_{t,\max}}\frac{1}{2} \sum_{k=1}^K \frac{S_{t,k}^2}{N_{t,k}}
\end{align*}
Finally, $N_{t,k} \sigma_k^{-2} \le \eta_{t,\min}^{-1}$.
\end{proof}

If $N_{t,k}$ was a known, unchanging number, we could choose $\sigma_k^2 \propto N_{t,k}$ to get $\eta_{t,\max} = \eta_{t, \min}$, and we would choose it to minimize the right hand side.
The strategy to use that ``known $N_{t,k}$'' case even if they are random is to put geometric grids on the number of pulls of each arm, define distributions that are adapted to each cell of the grid, and combine them into a mixture of Gaussians.

Let $(\eta_{n_1, \ldots, n_K})_{n_1, \ldots, n_K \in \mathbb{N}}$ be non-negative real numbers that will be chosen later.
For $i \in \mathbb{N}$, let $w_i = \frac{6}{\pi^2}\frac{1}{(i+1)^2}$. The weights $(w_i)$ satisfy $\sum_{i \in \mathbb{N}} w_i = 1$, hence also $\sum_{n_1, \ldots, n_K} (\prod_{k=1}^K w_{n_k}) = 1$.

Let $\rho_{n_1, \ldots, n_K} = \bigotimes_{k=1}^K \mathcal N(0, e^{- n_k} \eta_{n_1, \ldots, n_K}^{-1})$. This is a product distribution, with each marginal being a Gaussian with mean 0 and variance that depends on the number of grid cell.

With probability $1 - \delta$, for all $(n_1, \ldots, n_K) \in \mathbb{N}^K$ and all $t$,
\begin{align*}
\ln \mathbb{E}_{\lambda \sim \rho_{n_1, \ldots, n_K}}\left[\exp\left(\sum_{s=1}^t \lambda_{k_s}X_{s,k_s} - \frac{1}{2}\lambda_{k_s}^2\right)\right]
\le \ln \frac{1}{\delta} + \sum_{k=1}^K \ln \frac{1}{w_{n_k}}
\: .
\end{align*}
This is simply an union bound using Corollary~\ref{cor:prob_exists_log_ge_le}, with weight $\prod_{k=1}^K w_{n_k}$ for $\rho_{n_1, \ldots, n_K}$.

In particular, there exists $(n_1, \ldots, n_k)$ such that for all $k \in [K]$, $e^{n_k} \le N_{t,k} \le e^{n_k+1}$.
For that choice, $e^{-1}\eta_{n_1, \ldots, n_K} \le \frac{e^{n_k}\eta_{n_1, \ldots, n_K}}{N_{t,k}} \le \eta_{n_1, \ldots, n_K}$.
For those values, using Corollary~\ref{cor:sum_le_eta_mul} with $\sigma_k^2 = e^{n_k}\eta_{n_1, \ldots, n_K}$, with probability $1 - \delta$,
\begin{align*}
\frac{1}{2} \sum_{k=1}^K \frac{S_{t,k}^2}{N_{t,k}}
&\le (1 + \eta_{n_1, \ldots, n_K}) \left(\ln \frac{1}{\delta} + \sum_{k=1}^K \ln \frac{1}{w_{n_k}}
  + \frac{K}{2}\ln(1 + e \eta_{n_1, \ldots, n_K}^{-1})\right)
\\
&\le (1 + \eta_{n_1, \ldots, n_K}) \left(\ln \frac{e^{K/2}}{\delta} + \sum_{k=1}^K \ln \frac{1}{w_{n_k}}
  + \frac{K}{2}\ln(1 + \eta_{n_1, \ldots, n_K}^{-1})\right)
\\
&= (1 + \eta_{n_1, \ldots, n_K}) \left(\ln \frac{(\sqrt{e}\pi^2/6)^K \prod_{k=1}^K (n_k+1)^2}{\delta}
  + \frac{K}{2}\ln(1 + \eta_{n_1, \ldots, n_K}^{-1})\right)
\end{align*}

This is where we choose $\eta_{n_1, \ldots, n_K}$ to minimize the right hand side.

By Lemma A.3 of \citep{degenneImpactStructureDesign2019}, the minimal value is attained at some $\eta_{n_1, \ldots, n_K}$ such that
\begin{align*}
&(1 + \eta_{n_1, \ldots, n_K}) \left(\ln \frac{(\sqrt{e}\pi^2/6)^K \prod_{k=1}^K (n_k+1)^2}{\delta}
  + \frac{K}{2}\ln(1 + \eta_{n_1, \ldots, n_K}^{-1})\right)
\\
&= \frac{K}{2} \overline{W}\left( 1 + \frac{2}{K}\ln \frac{(\sqrt{e}\pi^2/6)^K \prod_{k=1}^K (n_k+1)^2}{\delta}\right)
\end{align*}

By the choice of $n_k$, it satisfies $n_k \le \ln N_{t,k}$.
We get that with probability $1 - \delta$, for all $t$,
\begin{align*}
\frac{1}{2} \sum_{k=1}^K \frac{S_{t,k}^2}{N_{t,k}}
&\le \frac{K}{2} \overline{W}\left( 1 + \frac{2}{K}\ln \frac{(\sqrt{e}\pi^2/6)^K \prod_{k=1}^K (1 + \ln N_{t,k})^2}{\delta}\right)
\\
&= \frac{K}{2} \overline{W}\left(\frac{2}{K}\ln \left((e\pi^2/6)^K \prod_{k=1}^K (1 + \ln N_{t,k})^2\right) + \frac{2}{K}\ln \frac{1}{\delta}\right)
\: .
\end{align*}

This ends the proof of the theorem.



\subsection{Upper bounds on $\beta(t, \delta)$ and on $\gamma_r$}

We choose the threshold
\begin{align*}
\beta(t, \delta)
&= \frac{K}{2} \overline{W}\left(2\ln \left(\frac{e\pi^2}{6}\right) + \frac{2}{K}\ln \left(\prod_{k=1}^K (1 + \ln N_{t,k})^2\right) + \frac{2}{K}\ln \frac{1}{\delta}\right)
\: .
\end{align*}

We can get an upper bound that is not random by maximizing over $(N_{t,k})_{k \in [K]}$ under the constraint $\sum_{k=1}^K N_{t,k} = t$. We get
\begin{align*}
\beta(t, \delta)
&\le \frac{K}{2} \overline{W}\left(2\ln \left(\frac{e\pi^2}{6}\right) + 4\ln \left(1 + \ln \frac{t}{K}\right) + \frac{2}{K}\ln \frac{1}{\delta}\right)
\: .
\end{align*}
We can get further upper bounds by using $\overline{W}(x) \le x + \ln x + 1/2 \le 2x$. This gives
\begin{align*}
\beta(t, \delta)
&\le 2 K\ln \left(\frac{e\pi^2}{6}\right) + 4 K\ln \left(1 + \ln \frac{t}{K}\right) + 2\ln \frac{1}{\delta}
\\
&\le 2 K\ln \left(\frac{e\pi^2}{6}\right) + 4 K\ln \frac{t}{K} + 2\ln \frac{1}{\delta}
\: .
\end{align*}
The right asymptotic for $\beta(t, \delta)$ as $\delta \to 0$ is $\ln(1/\delta)$. We lost a factor 2 in the upper bound above.




\begin{lemma}\label{lem:gamma_ub_bis}
Let $\gamma_r$ be the solution to $\beta(\bar{t}_r, \delta) = \gamma_r$ , for $\bar{t}_r = 2(K l_{1,r}/T_0 + \gamma_r) T_r$ and $l_{1,r} = 32 T_0 \ln (2\sqrt{2K} T_r)$. Then
\begin{align*}
\gamma_r \le 4 \ln \frac{1}{\delta} + 8K \ln(T_r) + 4K (11 + \ln K)
\: .
\end{align*}
\end{lemma}

\begin{proof}
We use an upper bound for $\beta(t, \delta)$: $\gamma_r$ is bounded from above by the solution $\gamma'_r$ to
\begin{align*}
\gamma = 2 K\ln \left(\frac{e\pi^2}{6}\right) + 4 K \ln \left(2(32 \ln (2\sqrt{2K} T_r) + \frac{\gamma}{K}) T_r\right) + 2\ln \frac{1}{\delta}
\: .
\end{align*}
Then either $\gamma'_r \le 8 K \ln (2\sqrt{2K} T_r)$ or $\gamma'_r$ is less than the solution to
\begin{align*}
\gamma
&= 2 K\ln \left(\frac{e\pi^2}{6}\right) + 4 K \ln \left(10\frac{\gamma T_r}{K}\right) + 2\ln \frac{1}{\delta}
\\
&= 2 K\ln \left(\frac{50 e\pi^2}{3}\right) + 4 K \ln \left(\frac{\gamma T_r}{K}\right) + 2\ln \frac{1}{\delta} 
\: .
\end{align*}


That is,
\begin{align*}
\gamma'_r
&= 4K \overline{W}\left( \frac{1}{2K} \ln \frac{1}{\delta} + \ln(T_r) + \frac{1}{2}\ln\frac{800 e \pi^2}{3} \right)
\\
&\le 4 \ln \frac{1}{\delta} + 8K \ln(T_r) + 4K \ln\frac{800 e \pi^2}{3}
\: .
\end{align*}

At this point, we get
\begin{align*}
\gamma_r
&\le \max\left\{ 4 \ln \frac{1}{\delta} + 8K \ln(T_r) + 4K \ln\frac{800 e \pi^2}{3},  8 K \ln (2\sqrt{2K} T_r)\right\}
\\
&\le 8K \ln(T_r) + \max\left\{ 4 \ln \frac{1}{\delta} + 4K \ln\frac{800 e \pi^2}{3}, 8 K \ln (2\sqrt{2K})\right\}
\\
&\le 8K \ln(T_r) + 4 \ln \frac{1}{\delta} + 4K \ln\frac{800 e \pi^2}{3} + 8 K \ln (2\sqrt{2K})
\\
&\le 8K \ln(T_r) + 4 \ln \frac{1}{\delta} + 4K \ln\left(\frac{6400 e \pi^2}{3}K\right)
\\
&\le 8K \ln(T_r) + 4 \ln \frac{1}{\delta} + 4K (11 + \ln K)
\: .
\end{align*}


\end{proof}

% !TeX root = ../all.tex


\section{Proofs related to the algorithm}\label{app:ub} 

\subsection{Additional Lemmas}

\begin{lemma}\label{lem:subG_concentration}
Let $(X_i)_{i \in \mathbb{N}}$ be i.i.d. $\sigma^2$-sub-Gaussian random variables with mean $\mu$. For $n \in \mathbb{N}$, let $\hat{\mu}_n$ be the average of the first $n$ random variables. Then
\begin{align*}
\mathbb{P}(\exists n \ge N, \ \hat{\mu}_n \ge \mu + \varepsilon) \le e^{- \frac{N \varepsilon^2}{2\sigma^2}}
\: , \\
\mathbb{P}(\exists n \ge N, \ \hat{\mu}_n \le \mu - \varepsilon) \le e^{- \frac{N \varepsilon^2}{2\sigma^2}} \: .
\end{align*}
\end{lemma}

\begin{proof}
Given $(X_1, \ldots, X_N)$, the process $M_n(\lambda) : n \mapsto e^{\lambda\sum_{i=1}^{N+n} (X_i - \mu) - \frac{1}{2}(N+n) \sigma^2 \lambda^2}$ is a nonnegative supermartingale for any $\lambda \in \mathbb{R}$ by the sub-Gaussian hypothesis, with expectation $e^{\lambda\sum_{i=1}^{N} (X_i - \mu) - \frac{1}{2}N \sigma^2 \lambda^2}$ at $n=0$.

By Ville's inequality,
\begin{align*}
\mathbb{P}(\exists n, \  M_n(\lambda) \ge 1/\delta \mid X_1, \ldots, X_N)
\le \delta e^{\lambda\sum_{i=1}^{N} (X_i - \mu) - \frac{1}{2}N \sigma^2 \lambda^2}
\end{align*}

For all $\lambda \in \mathbb{R}$ and all $\delta \in (0,1)$,
\begin{align*}
\mathbb{P}\left(\exists n \ge N, \  \lambda\sum_{i=1}^{n} (X_i - \mu) - \frac{1}{2}n\sigma^2 \lambda^2 \ge \ln(1/\delta)\right)
&= \mathbb{E}\left[\mathbb{P}(\exists n \ge 0, \  M_n(\lambda) \ge 1/\delta \mid X_1, \ldots, X_N)\right]
\\
&\le \delta \mathbb{E}\left[e^{\lambda\sum_{i=1}^{N} (X_i - \mu) - \frac{1}{2}N \sigma^2 \lambda^2}\right]
\\
&\le \delta
\: .
\end{align*}
Reordering, we get, for $\lambda \ge 0$,
\begin{align*}
\mathbb{P}\left(\exists n \ge N, \  \hat{\mu}_n \ge \mu + \frac{1}{2}\sigma^2 \lambda + \frac{1}{N \lambda}\ln\frac{1}{\delta}\right)
&\le \mathbb{P}\left(\exists n \ge N, \  \hat{\mu}_n \ge \mu + \frac{1}{2}\sigma^2 \lambda + \frac{1}{n \lambda}\ln\frac{1}{\delta}\right)
\\
&\le \delta
\: .
\end{align*}

Choose $\delta = e^{-\frac{N \varepsilon^2}{2 \sigma^2}}$ and $\lambda = \frac{\varepsilon}{\sigma^2}$ to obtain
\begin{align*}
\mathbb{P}\left(\exists n \ge N, \  \hat{\mu}_n \ge \mu + \varepsilon \right)
\le e^{-\frac{N \varepsilon^2}{2 \sigma^2}}
\: .
\end{align*}

The second inequality is obtained similarly, with $\lambda \le 0$.
\end{proof}

\begin{lemma}\label{lem:probaE_bis}
	The probability of $\mathcal E_r$ satisfies
	\begin{align*}
	\mathbb{P}(\mathcal E_r) \ge 1-2K\exp(- 2^r l_{1,r} \varepsilon_r^2/2\sigma^2)
	\: .
	\end{align*}
\end{lemma}

\begin{proof}
$\mathcal E_r$ is the event that $\Vert \bm\mu - \tilde{\bm\mu}^r \Vert_\infty \le \varepsilon_r$ and $\Vert \bm\mu - \hat{\bm\mu}^r \Vert_\infty \le \varepsilon_r$.
We use an union bound over the arms to bound the probability of the complement $\mathcal{E}_r^c$.
For each $i \in [K]$, $\tilde{\mu}^r_i$ and $\hat{\mu}^r_i$ are empirical means of at least $2^r l_{1,r}$ samples. We can thus apply Lemma~\ref{lem:subG_concentration} (twice, once for deviations from above and once for deviations from below).
\end{proof}

\begin{lemma}\label{lem:proba_Er}
	Let $p_r \in (0,1]$.
	For the choice $\varepsilon_r = \sqrt{\frac{2\sigma^2}{2^r l_1}\ln\frac{2K}{p_r}}$, the probability of the event $\mathcal E_r$ is $\mathbb{P}(\mathcal E_r) \ge 1 - p_r$.
\end{lemma}


\subsection{Proof of Theorem~\ref{th:alggen}}\label{app:ub_proof}

If $\overline{T}^\star(\hat{B}_r) > T_r$ then the algorithm does not enter the second batch of the phase by definition of the algorithm.

If $\overline{T}^\star(\hat{B}_r) \le T_r$ then by the choice of $\gamma_r$ and Lemma~\ref{lem:sufficientsampling}, under $\mathcal E_r$ the stopping condition is triggered.

We now prove the complexity upper bounds. Let $\mathcal C_r$ be the event that the algorithm attains phase $r$ and does not stop at that phase.
We proved that $\{\overline{T}^\star(\hat{B}_r) \le T_r\} \cap \mathcal E_r \subseteq \mathcal C_r^c$ for all $r$. That is, $\mathcal C_r \subseteq \mathcal E_r^c \cup \{\overline{T}^\star(\hat{B}_r) > T_r\}$.

Recall that $R^* = \min \{r \mid \forall r' \ge r, \ \mathcal E_{r'} \implies \overline{T}^\star(\hat{B}_{r'}) \le T_{r'}\}$.
\begin{align*}
R_\delta
&= \sum_{r=1}^{+\infty} \mathbb{I}(\mathcal C_{r-1}) + \mathbb{I}(\mathcal C_{r-1} \wedge \{\overline{T}^\star(\hat{B}_r) \le T_r\}) \: .
\\
&\le R^* + 2 \sum_{r=R^*+1}^{+\infty} \mathbb{I}(\mathcal C_{r-1}) + \sum_{r=1}^{R^*} \mathbb{I}(\mathcal C_{r-1} \wedge \{\overline{T}^\star(\hat{B}_r) \le T_r\})
\: .
\end{align*}
By definition of $R^*$, for $r \ge R^*$ we have $ \{\overline{T}^\star(\hat{B}_r) > T_r\} \subseteq \mathcal E_r^c$.
Using that property and the inclusion we proved on $\mathcal C_r$ we have, for $r > R^*$,
\begin{align*}
\mathcal C_{r-1}
\subseteq \mathcal E_{r-1}^c \cup \{\overline{T}^\star(\hat{B}_{r-1}) > T_{r-1}\}
\subseteq \mathcal E_{r-1}^c
\: .
\end{align*}
Therefore,
\begin{align*}
R_\delta
&\le R^* + 2 \sum_{r=R^*+1}^{+\infty} \mathbb{I}(\mathcal E_{r-1}^c) + \sum_{r=1}^{R^*} \mathbb{I}(\mathcal C_{r-1} \wedge \{\overline{T}^\star(\hat{B}_r) \le T_r\})
\: .
\end{align*}
When $\mathcal E_r$ happens $T^\star(\bm\mu) \le \overline{T}^\star(\hat{B}_r)$, hence $\{\overline{T}^\star(\hat{B}_r) \le T_r\} \subseteq \mathcal E_r^c \cup \{T^\star(\bm\mu) \le T_r\}$.
\begin{align*}
R_\delta
&\le R^* + 2 \sum_{r=R^*+1}^{+\infty} \mathbb{I}(\mathcal E_{r-1}^c) + \sum_{r=1}^{R^*} \mathbb{I}(\mathcal E_{r}^c)
	+ \sum_{r=1}^{R^*} \mathbb{I}(\{T^\star(\bm\mu) \le T_r\})
\\
&\le R^* + 1 + 2 \sum_{r=1}^{+\infty} \mathbb{I}(\mathcal E_{r}^c) + \sum_{r=1}^{R^*} \mathbb{I}(\{T^\star(\bm\mu) \le T_r\})
\: .
\end{align*}
Finally, $ \sum_{r=1}^{R^*} \mathbb{I}(\{T^\star(\bm\mu) \le T_r\}) = \max\{0, R^* - \lceil \log_2 \frac{T^\star(\bm\mu)}{T_0} \rceil \}$, and the definition of $R^*$ implies $R^* \ge \lceil \log_2 \frac{T^\star(\bm\mu)}{T_0} \rceil$.

We now bound the sample complexity $\tau_\delta$. Since $\bar{t}_r$ is an upper bound on the sample complexity up to phase $r$,
\begin{align*}
\tau_\delta
&\le \sum_{r=1}^{+\infty} \bar{t}_r \mathbb{I}\{C_{r-1}\}
\\
&\le \bar{t}_{R^*} + \sum_{r=R^* + 1}^{+\infty} \bar{t}_r \mathbb{I}\{C_{r-1}\}
\\
&\le \bar{t}_{R^*} + \sum_{r=R^* + 1}^{+\infty} \bar{t}_r \mathbb{I}\{E_{r-1}^c\}
\: .
\end{align*}



\subsection{Proof of the batch and complexity upper bounds}

We finish the proof of Theorem~\ref{thm:compexity_upper_bounds} from where we stopped in the main text. We have
\begin{align*}
\mathbb{E}\left[R_\delta\right]
&\le 2r^* - \lceil \log_2 \frac{T^\star(\bm\mu)}{T_0} \rceil + 1 + 2\sum_{r = 1}^{+\infty} p_r
\: , \\
\mathbb{E}\left[\tau_\delta\right]
&\le \bar{t}_{r^*} + \sum_{r = 1}^{+\infty} p_{r} \bar{t}_{r+1}
\: .
\end{align*}
In those expressions, $r^* = \max\{r_0, r_1\}$ with $r_0 = \min\{r \mid 2 \varepsilon_r \le b(\bm\mu)\}$, $r_1 =  \min \{r \mid T_r \ge e T^\star(\bm\mu)\}$, and
$\bar{t}_r = (K l_{1,r}/T_0 + 2 \gamma_r) T_r$ with $l_{1,r}/T_0 = 32 \ln(2 \sqrt{2K} T_r)$.

The choice of $p_r$ is a trade-off between the sums and $r_0$. We choose $p_r = T_{r+1}^2$.

\paragraph{Bounding the sums}
The sum in the batch complexity is bounded by $T_0^{-2}/3$.
The sum that appears in the sample complexity is
\begin{align*}
\sum_{r = \max\{r_0, r_1\}+1}^{+\infty} p_{r-1} \bar{t}_r
= \sum_{r = \max\{r_0, r_1\}+1}^{+\infty} \frac{\bar{t}_r}{T_r^2}
\: .
\end{align*}
We will need the values of a few sums.
\begin{align*}
\sum_{r=1}^{+\infty} \frac{1}{T_r}
&= \sum_{r=1}^{+\infty} \frac{1}{2^r T_0}
= \frac{1}{T_0}
\: , \\
\sum_{r=1}^{+\infty} \frac{\ln T_r}{T_r}
&= \frac{1}{T_0} \sum_{r=1}^{+\infty} \frac{r \ln2 + \ln T_0}{2^r}
= \frac{\ln (4T_0)}{T_0}
\: .
\end{align*}

Let $c_{K,\delta} = 4 \ln \frac{1}{\delta} + 4K (11 + \ln K)$.
By Lemma~\ref{lem:gamma_ub_bis}, $\gamma_r \le 8K \ln(T_r) + c_{K,\delta}$ .
An upper bound on the sample complexity until the end of phase $r$ is then
\begin{align*}
\bar{t}_r
&= (K l_{1,r}/T_0 + 2\gamma_r) T_r
\\
&= (32 K \ln(2\sqrt{2K}T_r) + 2 \gamma_r) T_r
\\
&\le (48 K \ln(T_r) + 32 K \ln(2\sqrt{2K}) + c_{K,\delta})T_r
\: .
\end{align*}
The sum that appears in the sample complexity is at most
\begin{align*}
\sum_{r = 1}^{+\infty} \frac{\bar{t}_r}{T_r^2}	
&\le \frac{48 K \ln(4T_0) + 32 K \ln(2\sqrt{2K}) + c_{K,\delta}}{T_0}
\: .
\end{align*}

\paragraph{Bound on $r^*$ and $\bar{t}_{r^*}$}

\begin{align*}
\bar{t}_{r^*}
\le (48 K \ln (\max\{T_{r_0}, T_{r_1}\}) + 32 K \ln(2\sqrt{2K}) + c_{K,\delta}) \max\{T_{r_0}, T_{r_1}\}
\: .
\end{align*}


Recall that $r_0 = \min\{r \mid 2 \varepsilon_r \le b(\bm\mu)\}$, $r_1 =  \min \{r \mid T_r \ge e T^\star(\bm\mu)\}$.
If we get an upper bound $n$ on $T_i$, we then have $r_i \le \log_2 \frac{n}{T_0}$.
We get a bound on $T_{r_1}$ from its definition: $T_{r_1-1} \le e T^*(\bm\mu)$ hence $T_{r_1} \le 2 e T^\star(\bm\mu)$.

Since $\varepsilon_{r_0 - 1} \ge b(\bm\mu)/2$, we get an inequality on $T_{r_0 - 1}$.
\begin{align*}
\sqrt{\frac{2\sigma^2}{2^{r_0 - 1}l_{1, r_0-1}} \ln \left( 2K T_{r_0}^2 \right)} \ge \frac{b(\bm\mu)}{2}
\: .
\end{align*}
With the value of $l_{1,r}$ and using $2 T_{r_0 - 1} = T_{r_0}$, this becomes
\begin{align*}
T_{r_0} \le \frac{\sigma^2}{b(\bm\mu)^2}
\: .
\end{align*}

Let $T^\star_b(\bm\mu) = \max\{\frac{\sigma^2}{b(\bm\mu)^2}, 2 e T^\star(\bm\mu)\}$.
We have proved $\max\{T_{r_0}, T_{r_1}\} \le T^\star_b(\bm\mu)$, hence
\begin{align*}
\bar{t}_{r^*}
\le (48 K \ln T^\star_b(\bm\mu) + 32 K \ln(2\sqrt{2K}) + c_{K,\delta}) T^\star_b(\bm\mu)
\end{align*}
and $r^* \le \log_2 \frac{T^\star_b(\bm\mu)}{T_0}$.

\paragraph{Putting things together}

\begin{align*}
\mathbb{E}\left[R_\delta\right]
&\le \log_2 \frac{T^\star_b(\bm\mu)}{T_0} + \log_2 \frac{T^\star_b(\bm\mu)}{T^\star(\bm\mu)} + 1 + T_0^{-2}
\: , \\
\mathbb{E}\left[\tau_\delta\right]
&\le (48 K \ln T^\star_b(\bm\mu) + 32 K \ln(2\sqrt{2K}) + c_{K,\delta}) T^\star_b(\bm\mu)
\\ & \quad + \frac{48 K \ln(4T_0) + 32 K \ln(2\sqrt{2K}) + c_{K,\delta}}{T_0}
\: .
\end{align*}

Let's simplify the sample complexity.
\begin{align*}
32 K \ln(2\sqrt{2K}) + c_{K,\delta}
&= 32 K \ln(2\sqrt{2K}) + 4 \ln \frac{1}{\delta} + 4K (11 + \ln K)
\\
&= 4 \ln \frac{1}{\delta} + 4K (5\ln K + 11 + 4 \ln(8))
\\
&\le 4 \ln \frac{1}{\delta} + 20K (\ln K + 4)
\: .
\end{align*}

\begin{align*}
\mathbb{E}\left[\tau_\delta\right]
&\le (48 K \ln T^\star_b(\bm\mu) + 4 \ln \frac{1}{\delta} + 20K (\ln K + 4)) T^\star_b(\bm\mu)
\\ & \quad + (48 K \ln(4T_0) + 4 \ln \frac{1}{\delta} + 20K (\ln K + 4)) T_0^{-1}
\\
&= 4 \ln \left(\frac{1}{\delta}\right) (T^\star_b(\bm\mu) + T_0^{-1}) + 20K (\ln K + 4) (T^\star_b(\bm\mu) + T_0^{-1})
	+ 48 K (T^\star_b(\bm\mu) \ln T^\star_b(\bm\mu) + T_0^{-1} \ln(4T_0))
\: .
\end{align*}



\subsection{Implication between the two assumptions}

We prove Lemma~\ref{lem:asm2_implies_asm1}.

\begin{lemma}\label{lem:sub_sqrt_inv_TStar} 
Let $\bm\nu$ and $\bm\nu'$ be two instances and let $\omega_{\bm \mu} = \arg\max_\omega \inf_{\lambda \in Alt_{\bm\mu}}\sum_{i=1}^K \omega_i (\mu_i - \lambda_i)^2$. Then
\begin{align*}
\sqrt{T^\star(\bm\mu)^{-1}} - \sqrt{T^\star(\bm\mu')^{-1}}
\le \frac{1}{\sqrt{2\sigma^2}} \Vert \mu - \mu' \Vert_\infty
\: .
\end{align*}
\end{lemma}

\begin{proof}
For $\omega \in \Sigma_K$ and $\bm x \in \mathbb{R}^K$, let $\Vert \bm x \Vert_\omega = \sqrt{\sum_{i=1}^K \omega_i x_i^2}$.
It satisfies the triangle inequality and $\Vert \bm x \Vert_\omega \le \Vert \bm x \Vert_\infty$.
For any $\bm\lambda$ and $\omega$,
\begin{align*}
\Vert \bm\mu - \bm\lambda\Vert_\omega \le \Vert \bm\mu' - \bm\lambda\Vert_\omega + \Vert \bm\mu - \bm\mu'\Vert_\omega
\: .
\end{align*}
We can take an infimum on both sides over lambda in $Alt_{\bm\mu}$ and then apply the result to $\omega_{\bm\mu}$ to get
\begin{align*}
\sqrt{2\sigma^2 T^\star(\bm\mu)^{-1}} \le \inf_{\lambda \in Alt_{\bm\mu}}\Vert \bm\mu' - \bm\lambda\Vert_{\omega_{\bm\mu}} + \Vert \bm\mu - \bm\mu'\Vert_{\omega_{\bm\mu}}
\: .
\end{align*}
Either $Alt_{\bm\mu} = Alt_{\bm\mu'}$ and we can replace by that on the right hand side, or $\bm\mu' \in Alt_{\bm\mu}$. In that second case $\inf_{\lambda \in Alt_{\bm\mu}}\Vert \bm\mu' - \bm\lambda\Vert_{\omega_{\bm\mu}} = 0 \le \inf_{\lambda \in Alt_{\bm\mu'}}\Vert \bm\mu' - \bm\lambda\Vert_{\omega_{\bm\mu}}$. We thus have
\begin{align*}
\sqrt{2\sigma^2 T^\star(\bm\mu)^{-1}} \le \inf_{\lambda \in Alt_{\bm\mu'}}\Vert \bm\mu' - \bm\lambda\Vert_{\omega_{\bm\mu}} + \Vert \bm\mu - \bm\mu'\Vert_{\omega_{\bm\mu}}
\: .
\end{align*}
We maximize over $\omega$ to get $\inf_{\lambda \in Alt_{\bm\mu'}}\Vert \bm\mu' - \bm\lambda\Vert_{\omega_{\bm\mu}} \le \sqrt{2\sigma^2 T^\star(\bm\mu')^{-1}}$, hence
\begin{align*}
\sqrt{2\sigma^2 T^\star(\bm\mu)^{-1}}
&\le \sqrt{2\sigma^2 T^\star(\bm\mu')^{-1}} + \Vert \bm\mu - \bm\mu'\Vert_{\omega_{\bm\mu}}
\\
&\le \sqrt{2\sigma^2 T^\star(\bm\mu')^{-1}} + \Vert \bm\mu - \bm\mu'\Vert_{\infty}
\: .
\end{align*}
After dividing by $\sqrt{2\sigma^2}$, this is the inequality of the lemma.
\end{proof}


\begin{corollary}\label{cor:sub_ln_TStar_le}
For all $\bm\mu$ and $\bm\mu'$,
\begin{align*}
\ln T^\star(\bm\mu') - \ln T^\star(\bm\mu)
\le \sqrt{\frac{2}{\sigma^2}T^\star(\bm\mu')} \ \Vert \bm\mu - \bm\mu' \Vert_\infty
\: .
\end{align*}
\end{corollary}

\begin{proof}
\begin{align*}
\ln T^\star(\bm\mu') - \ln T^\star(\bm\mu)
&= 2 \ln \left( 1 + \frac{\sqrt{T^\star(\bm\mu)^{-1}} - \sqrt{T^\star(\bm\mu')^{-1}}}{\sqrt{T^\star(\bm\mu')^{-1}}} \right)
\\
&\le 2 \frac{\sqrt{T^\star(\bm\mu)^{-1}} - \sqrt{T^\star(\bm\mu')^{-1}}}{\sqrt{T^\star(\bm\mu')^{-1}}}
\\
&\le \sqrt{\frac{2}{\sigma^2}T^\star(\bm\mu')} \ \Vert \bm\mu - \bm\mu' \Vert_\infty
\: .
\end{align*}

\end{proof}


\begin{corollary}\label{cor:abs_sub_ln_TStar_le}
For all $\bm\mu$ and $\bm\mu'$ with $\Vert \bm\mu - \bm\mu' \Vert_\infty \le \sqrt{\sigma^2 / (2 T^\star(\bm\mu))}$,
\begin{align*}
\left\vert \ln T^\star(\bm\mu') - \ln T^\star(\bm\mu) \right\vert
\le \sqrt{\frac{8}{\sigma^2}T^\star(\bm\mu)} \ \Vert \bm\mu - \bm\mu' \Vert_\infty
\: .
\end{align*}
\end{corollary}

\begin{proof}
One of the two inequalities we need to prove is due to Corollary~\ref{cor:sub_ln_TStar_le}. For the other, by the same corollary,
\begin{align*}
\ln T^\star(\bm\mu') - \ln T^\star(\bm\mu)
\le \sqrt{\frac{2}{\sigma^2}T^\star(\bm\mu')} \ \Vert \bm\mu - \bm\mu' \Vert_\infty
\: .
\end{align*}
It remains to show $T^\star(\bm\mu') \le 4 T^\star(\bm\mu)$. By Lemma~\ref{lem:sub_sqrt_inv_TStar} and the the hypothesis on $\Vert \bm\mu - \bm\mu' \Vert_\infty$,
\begin{align*}
\sqrt{T^\star(\bm\mu)^{-1}} - \sqrt{T^\star(\bm\mu')^{-1}}
\le \frac{1}{\sqrt{2\sigma^2}} \Vert \bm\mu - \bm\mu' \Vert_\infty
\le \frac{1}{2}\sqrt{T^\star(\bm\mu)^{-1}} \: .
\end{align*}
Reordering proves the inequality.
\end{proof}



\subsection{Proofs for Top-K and thresholding bandits}
\label{app:topk_threshold}

This section is devoted to the proof of Lemma~\ref{lem:constrBbai}. We start with a preliminary result allowing us to compute $\overline{w}^\star(B)$ once we have found a suitable instance in $B$.


\begin{lemma}\label{lem:wbwst}
	If all instances in $B$ share the same correct answer $i^\star$ and if there exists some mean vector $\bm b\in B$ such that\begin{equation}\label{eq:bworst} \inf_{\bm\nu\in B}\inf_{\bm{\lambda}\in Alt_{\bm b}} \sum_i w_i(\bm{b}) \frac{(\mu_i-\lambda_i)^2}{2\sigma^2}\geq \inf_{\bm{\lambda}\in Alt_{\bm b}} \sum_i w_i(\bm{b}) \frac{(b_i-\lambda_i)^2}{2\sigma^2}\end{equation} where $w(\bm\mu)=\argmax_{w\in \Sigma_K} \inf_{\bm\lambda \in Alt_{\bm\mu}} \sum_i w_i \frac{(\mu_i-\lambda_i)^2}{2\sigma^2}$, then $\overline{T}^\star(B) = \max_{\bm \mu\in B}T^\star(\bm \mu)$.
\end{lemma}

\begin{proof}
	For some $w\in \Sigma_K$, writing $f(w,\bm\mu')=\inf_{\bm\lambda\in Alt_{\bm b}} \sum_i w_i \frac{(\mu'_i-\lambda_i)^2}{2\sigma^2}$ for clarity,
	\begin{align*}
		\inf_{\bm \nu\in \hat{B}_r} f( w,\bm\mu) &\leq f( w,\bm b) &\text{because }\bm b\in \hat{B}_r\\
		&\leq f(w_{\bm b},\bm b)& \text{from the definition of }w_{\bm b}\\
		&\leq \inf_{\bm\mu'\in\hat{B}_r} f(w_{\bm b},\bm\mu')&\text{from the hypothesis}
	\end{align*} so that $w^\star(\bm b) = \argmax_{w\in \Sigma_K} \inf_{\bm\nu\in\hat{B}_r} f(\bm w,\nu)=\overline w^\star(B)$.

\begin{align*}
	\overline{T}^\star(B) &= \left( \inf_{\bm\nu'\in B} \inf_{\bm\lambda \in Alt_{\bm\nu'}} \sum_i \overline{w}^\star_i(B) \frac{(\mu'_i-\lambda_i)^2}{2\sigma^2}\right)^{-1} &\\
	&=\left( \inf_{\bm\nu'\in B} \inf_{\bm\lambda \in Alt_{\bm\nu'}} \sum_i w^\star_i(\bm b) \frac{(\mu'_i-\lambda_i)^2}{2\sigma^2}\right)^{-1}&\\
	&=\left(  \inf_{\bm\lambda \in Alt_{\bm b}} \sum_i w^\star_i(\bm b) \frac{(b_i-\lambda_i)^2}{2\sigma^2}\right)^{-1}&\hspace{-5em}\text{by Equation~\eqref{eq:bworst}}\\
	&= T^\star(\bm b)&
\end{align*}
hence $\overline{T}(B) \leq \max_{\bm\nu\in B} T^\star(\bm\nu)$. By definition, we have the other inequality, and we conclude.
\end{proof}



\lemconstrBbai*

We prove the result separately for top-$k$ and TBP. In both cases, we give a certain mean vector $\bm b$, and then we show it satisfies the premise of Lemma~\ref{lem:wbwst}, then we use that result to show that $\overline{T}^\star(\mathcal{B}_\infty(\bm\mu,\varepsilon))=T^\star(\bm b)$.

\begin{proof}[Proof of Lemma~\ref{lem:constrBbai} for top-$k$]
	Assume without loss of generality that the arms are well ordered, $\mu_1\geq \mu_2\geq \cdots \geq \mu_K$.
	
	If $\mu_{k}-\mu_{k+1}\leq 2\varepsilon$, then there exists $\bm b\in \mathcal B_{\infty}(\bm\mu, \varepsilon)$ such that $b_k=b_{k+1}$. $\overline{T}^\star(\mathcal B_{\infty}(\bm\mu, \varepsilon))\geq \max_{\bm\nu'\in \mathcal B_{\infty}(\bm\mu, \varepsilon)} T^\star(\bm\nu')\geq T^\star(\bm b)=+\infty$, therefore \[\overline{T}^\star(\mathcal B_{\infty}(\bm\mu, \varepsilon))= \max_{\bm\nu'\in \mathcal B_{\infty}(\bm\mu, \varepsilon)} T^\star(\bm\nu')\: .\]
	
	When $\mu_k-\mu_{k+1}>2\varepsilon$, define \[\left\{ \begin{aligned} &b_i= \mu_i-\varepsilon \qquad\text{if }i\leq k \\ &b_i =\mu_i +\varepsilon \qquad\text{if }i\geq k+1\end{aligned}\right. \]
	and, for any $\bm\nu'$, \[w_{\bm\nu'} =\argmax_{w\in \Sigma_K} \inf_{\bm\lambda\in Alt_{\bm\nu'}} \sum_i w_i \frac{(\mu_i'-\lambda_i)^2}{2\sigma^2}.\] 
	
	
	Let there be some $\bm{\nu}'\in \mathcal B_{\infty}(\bm\mu, \varepsilon)$. Then for $i\leq k$, $\mu'_{i} \geq \mu_i-\varepsilon=b_{i}$ and for $i\geq l+1$, $\mu'_i \leq \mu_i +\varepsilon = b_i$, and $Alt_{\bm\nu'}=Alt_{\bm b}$.
	
	We know from Lemma~\ref{lem:baiw} \begin{equation}\label{eq:argmin} \min_{\bm{\lambda}\in Alt_{\bm b}}\sum_i  w_i(\bm{b})\frac{(\mu'_i-\lambda_i)^2}{2\sigma^2} = w_a(\bm b)\frac{(\mu'_a-\mu'_{aj})^2}{2\sigma^2}+w_j(\bm b)\frac{(\mu'_j-\mu'_{aj})^2}{2\sigma^2}\end{equation} for some $a\leq k<k+1\leq j$, and $\mu'_{aj}= \frac{w_a(\bm{b})}{w_a(\bm{b})+w_j(\bm{b})}\mu'_a + \frac{w_j(\bm{b})}{w_a(\bm{b})+w_j(\bm{b})}\mu'_j$. 
	

	\begin{itemize}
		\item If $\mu'_{aj}\in (b_a,\mu'_a)$, then \begin{align*}
			w_a(\bm{b}) (\mu'_a-\mu'_{aj})^2+w_j(\bm{b}) (\mu'_j-\mu'_{aj})^2 &\geq 0+w_j(\bm{b}) (\mu'_j-b_a)^2 \\
			&\geq w_a(\bm{b})(b_a-b_a)^2+w_j(\bm{b})(b_j-b_a)^2\\
		\end{align*}
		\item If $\mu'_{aj}\in (b_j,b_a)$, \begin{align*}
			w_a(\bm{b}) (\mu'_a-\mu'_{aj})^2+w_j(\bm{b}) (\mu'_j-\mu'_{aj})^2 &\geq w_a(\bm{b})(b_a-\mu'_{aj})^2 +w_j(\bm{b})(b_j-\mu'_{aj})^2
		\end{align*}
		\item If $\mu'_{aj} \in (\mu'_j,b_j)$, \begin{align*} w_a(\bm{b}) (\mu'_a-\mu'_{aj})^2+w_j(\bm{b}) (\mu'_j-\mu'_{aj})^2 &\geq w_a(\bm{b})(\mu'_a-b_j)^2+0 \\
			&\geq w_a(\bm{b}) (b_a-b_j)^2 +w_j(\bm{b})(b_j-b_j)^2\end{align*}
	\end{itemize}
	In all three cases, \begin{align*} w_a(\bm{b}) (\mu'_a-\mu'_{aj})^2+w_j(\bm{b}) (\mu'_j-\mu'_{aj})^2 &\geq \inf_{\lambda\in [b_j,b_a]} w_a(\bm{b}) (b_a-\lambda)^2+w_j(\bm{b})(b_j-\lambda)^2\\
		&\geq \inf_{\bm{\lambda}\in Alt_{\bm b}} \sum_i w_i(\bm{b}) (b_i-\lambda_i)^2
	\end{align*}
	and therefore, by Equation~\eqref{eq:argmin}, \[\forall \bm\nu' \in B_{\infty}(\bm\mu, \varepsilon),\; \inf_{\bm{\lambda}\in Alt_{\bm b}} \sum_i w_i(\bm{b}) \frac{(\nu_i-\lambda_i)^2}{2\sigma^2}\geq \inf_{\bm{\lambda}\in Alt_{\bm b}} \sum_i w_i(\bm{b}) \frac{(b_i-\lambda_i)^2}{2\sigma^2} \] and therefore \begin{equation} \inf_{\bm\nu\in\mathcal B_{\infty}(\bm\mu, \varepsilon)}\inf_{\bm{\lambda}\in Alt_{\bm b}} \sum_i w_i(\bm{b}) \frac{(\nu_i-\lambda_i)^2}{2\sigma^2}\geq \inf_{\bm{\lambda}\in Alt_{\bm b}} \sum_i w_i(\bm{b}) \frac{(b_i-\lambda_i)^2}{2\sigma^2}\end{equation}
	
We can thus apply Lemma~\ref{lem:wbwst}, and conclude

\[\overline{T}(\mathcal B_{\infty}(\bm\mu, \varepsilon)) = \max_{\bm\nu\in \mathcal B_{\infty}(\bm\mu, \varepsilon)} T^\star(\bm\nu) \: .\]
\end{proof}









\begin{proof}[Proof of Lemma~\ref{lem:constrBbai} for thresholding bandits]
	If for some $i$, $|\mu_i -\tau|\leq \varepsilon$, then there exists $\bm b\in \mathcal{B}_\infty(\bm\mu,\varepsilon)$ such that $b_i = \tau$. Therefore, $\overline{T}^\star(\mathcal{B}_\infty(\bm\mu,\varepsilon)) \geq \max_{\bm\nu'\in \mathcal{B}_\infty(\bm\mu,\varepsilon)} T^\star(\bm\nu')\geq T^\star(\bm b) =+\infty$, therefore \[\overline{T}^\star(\mathcal{B}_\infty(\bm\mu,\varepsilon)) = \max_{\bm\nu'\in \mathcal{B}_\infty(\bm\mu,\varepsilon)}T^\star(\bm\nu') \: .\]
	
	When $\min_k |\mu_k-\tau|>\varepsilon$, define $U=\{i\in[K]:\mu_i >\tau\}$ and $L=[K]\setminus U$. Define \[\left\{ \begin{aligned} &b_i= \mu_i-\varepsilon &\text{if }i\in U \\ &b_i =\mu_i +\varepsilon &\text{if }i\in L.\end{aligned}\right. \] and for any $\bm\nu'$, $w_{\bm\nu'} =\argmax_{w\in \Sigma_K} \inf_{\bm\lambda \in Alt_{\bm\nu'}} \sum_i w_i\frac{(\mu_i'-\lambda_i)^2}{2\sigma^2}$.
	
	
	
	
	Let there be some $\bm\nu\in \mathcal{B}_\infty(\bm\mu,\varepsilon)$.	We know $\min_{\bm\lambda \in Alt_{\bm b}} \sum_i w_i(\bm b)\frac{(\mu'_i-\lambda_i)^2}{2\sigma^2}=w_j(\bm b) \frac{(\mu'_j-\tau)^2}{2\sigma^2}$ for some $j$. 
	
	For all $i\in U$, $\mu'_i \geq \mu_i-\varepsilon = b_i>\tau$; for all $i\in L$, $\mu'_i \leq \mu_i+\varepsilon = b_i<\tau$. Therefore, \[w_j(\bm b) \frac{(\mu'_j-\tau)^2}{2\sigma^2}\geq w_j(\bm b) \frac{(b_j-\tau)^2}{2\sigma^2}\geq \inf_{\bm\lambda \in Alt_{\bm b}} \sum_i w_i(\bm b) \frac{(b_i-\lambda_i)^2}{2\sigma^2}\]
	We thus have $\forall \bm\nu' \in \mathcal{B}_\infty(\bm\mu,\varepsilon)$, $\inf_{\bm\lambda\in Alt_{\bm b}} \sum_i w_i(\bm b)\frac{(\mu'_i-\lambda_i)^2}{2\sigma^2} \geq \inf_{\bm\lambda \in Alt_{\bm b}} \sum_i w_i(\bm b) \frac{(b_i-\lambda_i)^2}{2\sigma^2}$, and \[\inf_{\nu\in\hat{B}_r}\inf_{\bm\lambda\in Alt_{\bm b}} \sum_i w_i(\bm b)\frac{(\nu_i-\lambda_i)^2}{2\sigma^2} \geq \inf_{\bm\lambda \in Alt_{\bm b}} \sum_i w_i(\bm b) \frac{(b_i-\lambda_i)^2}{2\sigma^2}\] and by Lemma~\ref{lem:wbwst},
\[\overline{T}(\mathcal B_{\infty}(\bm\mu, \varepsilon)) = \max_{\bm\nu\in \mathcal B_{\infty}(\bm\mu, \varepsilon)} T^\star(\bm\nu) \: .\]
\end{proof}

%%%%%%%%%%%%%%%%%%%%%%%%%%%%%%%%%%%%%%%%%%%%%%%%%%%%%%%%%%%%%%%%%%%%%%%%%%%%%%%
%%%%%%%%%%%%%%%%%%%%%%%%%%%%%%%%%%%%%%%%%%%%%%%%%%%%%%%%%%%%%%%%%%%%%%%%%%%%%%%


\end{document}

\begin{table*}[t]
    \centering
    \caption{Attack success rate (ASR) on LLaVA-1.5-7B for MM-SafetyBench. Lower values indicate stronger defense performance.}

    \label{table:mm-safetybench-llava-1.5-7b-part}

    \resizebox{\textwidth}{!}{
    \begin{tabular}{lccccccccccccc}
        
    \toprule
    
    \multirow{2}{*}{Scenarios} & \multirow{2}{*}{Text} & \multicolumn{4}{c}{SD} & \multicolumn{4}{c}{OCR} & \multicolumn{4}{c}{SD+OCR} \\
    
    \cmidrule(lr){3-6} \cmidrule(lr){7-10} \cmidrule(lr){11-14}
    & & Direct & ECSO & AdaSheild & \Highlight \Highlight \OursMethod & Direct & ECSO & AdaSheild & \Highlight \OursMethod & Direct & ECSO & AdaSheild & \Highlight \OursMethod \\
    
    \midrule
    
    01: Illegal Activity & 10.2 & 25.1 & 6.6 & 10.6 & \Highlight 6.2 & 70.3 & 6.0 & 7.5 & \Highlight 6.4 & 78.3 & 12.4 & 10.9 & \Highlight 7.2 \\
    
    02: HateSpeech & 8.7 & 19.5 & 4.3 & 10.6 & \Highlight 6.4 & 44.8 & 16.2 & 7.8 & \Highlight 5.3 & 51.5 & 17.0 & 9.6 & \Highlight 10.5 \\
    
    03: Malware Generation & 59.6 & 18.8 & 7.5 & 4.5 & \Highlight 4.5 & 72.1 & 15.9 & 9.6 & \Highlight 12.6 & 65.8 & 19.0 & 8.1 & \Highlight 10.2 \\
    
    04: Physical Harm & 34.9 & 20.0 & 10.4 & 15.7 & \Highlight 8.8 & 64.9 & 15.0 & 16.2 & \Highlight 10.5 & 60.1 & 18.3 & 13.5 & \Highlight 7.4 \\
    
    05: Economic Harm & 8.4 & 6.8 & 7.9 & 10.3 & \Highlight 8.1 & 14.0 & 7.9 & 15.6 & \Highlight 8.1 & 17.5 & 10.5 & 14.2 & \Highlight 7.9 \\
    
    06: Fraud & 15.2 & 23.8 & 10.4 & 13.3 & \Highlight 9.4 & 72.6 & 12.2 & 9.4 & \Highlight 9.7 & 64.1 & 22.2 & 13.6 & \Highlight 10.8 \\
    
    07: Pornography & 15.2 & 12.2 & 9.5 & 10.1 & \Highlight 9.7 & 25.1 & 16.0 & 13.2 & \Highlight 8.8 & 28.8 & 25.9 & 13.3 & \Highlight 10.8 \\

    09: Privacy Violence & 27.6 & 15.1 & 14.6 & 18.2 & \Highlight 10.2 & 57.4 & 16.6 & 22.4 & \Highlight 15.0 & 60.0 & 25.3 & 21.8 & \Highlight 17.7 \\

    \midrule
    
    \textbf{Average} & 49.2 & 45.4 & 40.3 & 42.6 & \Highlight \textbf{38.0} & 69.3 & 43.0 & 42.6 & \Highlight \textbf{39.7} & 70.5 & 48.8 & 45.8 & \Highlight \textbf{43.6} \\
    
    \bottomrule
    
    \end{tabular}
    }
    \vspace{-10pt}
\end{table*}
\begin{table}[t]
    \centering
    \vspace{-10pt}
    \caption{Attack success rates on the FigStep benchmark \cite{gong2023figstep}. Lower values indicate stronger defense performance.}

    \label{table:figstep}

    \resizebox{0.9\linewidth}{!}{
    \begin{tabular}{lccccccccc}
        
    \toprule
    
    Models & Direct & ECSO & AdaShield & \Highlight \OursMethod \\
    
    \midrule
    
    LLaVA-1.5-7B &  62.4 & 9.7 & 12.4 & \Highlight \textbf{8.5}  \\
    
    ShareGPT4V-7B  & 28.7 & 10.9 & 14.3 & \Highlight \textbf{9.2} \\
    
    MiniGPT-4-7B &  12.5 & 8.3 & 8.0 & \Highlight \textbf{6.3}  \\
    
    Qwen-VL-7B & 25.3 & 9.5 & 10.5 & \Highlight \textbf{8.4} \\
    
    \bottomrule
    
    \end{tabular}
    }
    \vspace{-20pt}
\end{table}

\textbf{Safety Benchmarks.}
Experiments evaluating the safety of VLMs' responses are conducted on the \textbf{MM-SafetyBench} \cite{liu2025mm} and \textbf{FigStep} \cite{gong2023figstep} benchmarks. MM-SafetyBench assesses VLM safety across 13 commonly prohibited scenarios. Each query is represented in three input formats: (1) Stable-diffusion images (SD); (2) Typography (OCR) images and (3) SD+OCR images.  FigStep rephrases harmful instructions to encourage the model to generate answers item-by-item and converts them into images using typography. More details are in Appendix \ref{appendix-datasets}.

\textbf{Evaluation Results.}
For MM-SafetyBench, the average ASR across 13 scenarios for all VLMs is shown in Table \ref{table:mm-safetybench-all}, while Table \ref{table:mm-safetybench-llava-1.5-7b-part} presents ASR results for 8 out of 13 scenarios using LLaVA-1.5-7B, following \cite{gou2025eyes}. Table~\ref{table:figstep} shows ASR results on FigStep across different VLMs. Complete results are available in the Appendix \ref{appendix-results}.

Most VLM backbones exhibit a high ASR when processing vision-language inputs. While SD images cause only a slight increase in ASR, typography-based attacks (OCR \& FigStep) are highly effective. After applying \OursMethod, ASR is significantly reduced across all VLMs and attack types, demonstrating its effectiveness in reactivating safety alignment and defending against attacks. \OursMethod\ also outperforms ECSO and AdaShield, highlighting the effectiveness of its activation calibration.


\subsection{Main Results on Utility}


To ensure we audit the privacy of synthetic text data in a realistic setup, the synthetic data needs to bear high utility. We measure the synthetic data utility by comparing the downstream classification performance of RoBERTa-base~\citep{DBLP:journals/corr/abs-1907-11692} when fine-tuned exclusively on real or synthetic data. We fine-tune models for binary (SST-2) and multi-class classification (AG News) for 1 epoch on the same number of real or synthetic data records using a batch size of $16$ and learning rate $\eta = \num{1e-5}$. We report the macro-averaged AUC score and accuracy on a held-out test dataset of real records. 

Table~\ref{tab:utility_no_canaries} summarizes the results for synthetic data generated based on original data which does not contain any canaries. While we do see a slight drop in downstream performance when considering synthetic data instead of the original data, AUC and accuracy remain high for both tasks. 

\begin{table}[ht]
    \centering
    \begin{tabular}{ccrr}
    \toprule
        & \multirow{2}{*}{Fine-tuning data} & \multicolumn{2}{c}{Classification} \\
        \cmidrule(lr){3-4}
        Dataset &  & AUC & Accuracy \\
        \midrule 
        \multirow{2}{*}{\parbox{2cm}{\centering SST-2}} & Real & $0.984$ & \SI{92.3}{\percent} \\ 
         & Synthetic & $0.968$ & \SI{91.5}{\percent} \\
         \midrule
        \multirow{2}{*}{\parbox{2cm}{\centering AG News}} & Real & $0.992$ & \SI{94.4}{\percent} \\ 
         & Synthetic & $0.978$ & \SI{90.0}{\percent} \\ 
        \bottomrule
    \end{tabular}
    \caption{Utility of synthetic data generated from real data \emph{without} canaries. We compare the performance of text classifiers trained on real or synthetic data---both evaluated on real, held-out test data.}
    \label{tab:utility_no_canaries}
\end{table}

We further measure the synthetic data utility when the original data contains standard canaries (see Sec.~\ref{sec:baseline_results}). Specifically, we consider synthetic data generated from a target model trained on data containing \num{500} canaries repeated $n_\textrm{rep} = 12$ times, so \num{6000} data records. When inserting canaries with an artificial label, we remove all synthetic data associated with labels not present originally when fine-tuning the RoBERTa-base model. 

\begin{table}[h]
    \centering
    \begin{tabular}{ccc@{\hskip 15pt}rr}
    \toprule
        & \multicolumn{2}{c}{Canary injection} & \multicolumn{2}{c}{Classification}\\
        \cmidrule(lr){2-3} \cmidrule(lr){4-5}
        Dataset & Source & Label & AUC & Accuracy \\
        \midrule
        \multirow{3}{*}{\parbox{1cm}{\centering SST-2}} & \multicolumn{2}{l}{In-distribution} & $0.972$ & \SI{91.6}{\percent} \\ 
        \cmidrule{2-5}
         & \multirow{2}{*}{\parbox{1.8cm}{Synthetic}} & Natural & $0.959$ & \SI{89.3}{\percent} \\ 
         & & Artificial & $0.962$ & \SI{89.9}{\percent} \\ 
        \midrule
        \multirow{3}{*}{\parbox{2cm}{\centering AG News}} & \multicolumn{2}{l}{In-distribution} & $0.978$ & \SI{89.8}{\percent}\\ 
        \cmidrule{2-5} 
         & \multirow{2}{*}{\parbox{1.8cm}{Synthetic}} & Natural & $0.977$ & \SI{88.6}{\percent} \\ 
         & & Artificial & $0.980$ & \SI{90.1}{\percent} \\         
         \bottomrule
    \end{tabular}
    \caption{Utility of synthetic data generated from real data \emph{with} canaries ($n_\textrm{rep}=12$). We compare the performance of text classifiers trained on real or synthetic data---both evaluated on real, held-out test data.}
    \label{tab:utility_canaries}
\end{table}

Table~\ref{tab:utility_canaries} summarizes the results. Across all canary injection methods, we find limited impact of canaries on the downstream utility of synthetic data. While the difference is minor, the natural canary labels lead to the largest utility degradation. This makes sense, as the high perplexity synthetic sequences likely distort the distribution of synthetic text associated with a certain real label. In contrast, in-distribution canaries can be seen as up-sampling certain real data points during fine-tuning, while canaries with artificial labels merely reduce the capacity of the model to learn from real data and do not interfere with this process as much as canaries with natural labels do.


\OursMethod\ is designed to not compromise VLM visual utility, thus the model is also evaluated on utility benchmarks.

\textbf{Utility Benchmarks.}
Experiments are conducted on popular VLM utility benchmarks, \textbf{MME} and \textbf{MM-Vet}, which assess essential VLM capabilities. MME evaluates performance using accuracy (per question) and accuracy+ (per image, requiring both questions to be correct). MM-Vet, which requires open-ended responses, is scored based on the average GPT-4 rating (0 to 1) across all samples. Details are provided in Appendix \ref{appendix-datasets}.


\textbf{Evaluation Results.}
Table \ref{table:utility} presents the utility scores of all VLMs on the MME and MM-Vet benchmarks. On these benchmarks, \OursMethod\ performs similarly to the original models and outperforms other baselines. This demonstrates that \OursMethod\ successfully preserves visual reasoning utility by maintaining modality shifts in the activation space.


\subsection{Does \OursMethod\ Truly Correct Safety Perception?}  \label{sec:exp-truely-correct-safety-perception}


\begin{figure}[t]
    \begin{minipage}{0.49\linewidth}
        \begin{center}
        \includegraphics[width=\linewidth]{images/safety-cls-bar.pdf}
        \end{center}
    \end{minipage}
    \hfill
    \begin{minipage}{0.49\linewidth}
        \begin{center}
        \vspace{-7pt}
        \includegraphics[width=\linewidth]{images/activation_vis_llava_after.pdf}
        \end{center}
    \end{minipage}
    \caption{\textbf{Left}: Binary safety classification accuracy across VLMs. \textbf{Right}: t-SNE visualization of LLaVA-1.5-7B activation on \yellowcircle\ $\mathcal{D}_\text{tt}^\text{safe}$, \purplecircle\ $\mathcal{D}_\text{tt}^\text{unsafe}$, \greencircle\ $\mathcal{D}_\text{vl}^\text{safe}$, and \bluecircle\ $\mathcal{D}_\text{vl}^\text{unsafe}$ after applying \OursMethod.}
    \vspace{-20pt}
    \label{fig:exp-safety-cls-acc-vis}
\end{figure}

% \begin{figure}[t] 
% \begin{center}
%     \includegraphics[width=\linewidth]{images/safety-cls-bar.pdf}
% \end{center}
% \caption{Binary safety classification accuracies. \lu{you can make figure 8 and 9 slimmer and put them side by side to save space}}
% \label{fig:exp-safety-cls-unsafe}
% \end{figure}

% \begin{figure}[t] 
% \begin{center}
%     \includegraphics[width=0.75\linewidth]{images/activation_vis_llava_after.pdf}
% \end{center}
% \caption{t-SNE visualization of LLaVA-1.5-7B activations on \yellowcircle\ $\mathcal{D}_\text{tt}^\text{safe}$, \purplecircle\ $\mathcal{D}_\text{tt}^\text{unsafe}$, \greencircle\ $\mathcal{D}_\text{vl}^\text{safe}$, and \bluecircle\ $\mathcal{D}_\text{vl}^\text{unsafe}$ after applying \OursMethod.}
% \label{fig:exp-safety-cls-vis}
% \end{figure}

\OursMethod\ eliminates the safety shift in activations caused by the visual modality, allowing VLMs to accurately identify unsafe instructions. To evaluate its effectiveness, we measure the binary safety classification accuracy of each VLM on LLaVA-Instruct-80k (safe) and MM-SafetyBench (unsafe) after applying \OursMethod. Here we use each VLM as a classifier to determine whether inputs are safe or unsafe, detailed in Appendix \ref{appendix-implementation}.
% by prompting it with “Is the instruction harmful, unsafe, and unethical? Answer yes or no:”  
The classification accuracies for different models are presented in Figure~\ref{fig:exp-safety-cls-acc-vis} (left). Results for text-only input accuracy and accuracy before applying \OursMethod\ are also provided for reference. After applying \OursMethod, the accuracy for image-text inputs improve significantly to match the text-only accuracy, as expected.

We also visualize LLaVA-1.5-7B's activations after applying \OursMethod\ in Figure~\ref{fig:exp-safety-cls-acc-vis} (right). The visualization shows that the activations for unsafe and safe image-text instructions are now separable, contrary to the previous intermixed state shown in Figure \ref{fig:tsne}. Additionally, most unsafe image-text activations are positioned correctly on the “unsafe” side of the boundary derived from text-only activations, demonstrating that \OursMethod\ works as intended.



\subsection{Does \OursMethod\ Cause False Alarms on Safe Datasets?} \label{sec:exp-false-alarms}

\begin{table}[t]
    \centering
    \caption{Changes in misclassification rates of VLMs predicting safe queries as unsafe on benign datasets after applying \OursMethod.}

    \label{table:benign-miscls}

    \resizebox{\linewidth}{!}{
    \begin{tabular}{lcccccccc}
        
    \toprule
    
    Datasets & MME & MM-Vet & LLaVA-Instruct-80K \\
    
    \midrule
    
    LLaVA-1.5-7B & \textcolor{Green}{-0.0\%} & \textcolor{Green}{-0.4\%} & \textcolor{Green}{-0.0\%} \\
    
    ShareGPT4V-7B & \textcolor{Green}{-0.0\%} & \textcolor{red}{+1.6\%} & \textcolor{Green}{-0.0\%} \\
    
    MiniGPT-4-7B & \textcolor{red}{+0.7\%} & \textcolor{Green}{-0.0\%} & \textcolor{Green}{-0.0\%} \\
    
    Qwen-VL-7B & \textcolor{Green}{-0.2\%} & \textcolor{Green}{-0.0\%} & \textcolor{Green}{-0.1\%} \\
    
    \bottomrule
    
    \end{tabular}
    }
    \vspace{-18pt}
\end{table}

To ensure that \OursMethod\ maintains \textbf{VLM helpfulness} on benign instructions, Table~\ref{table:benign-miscls} reports the changes in the misclassification rate (safe samples misclassified as unsafe) on MME, MM-Vet, and instructions sampled from LLaVA-Instruct-80K after applying \OursMethod. Since these datasets are entirely benign and do not trigger harmful responses, any detection of harm is considered a false alarm. The results show that \OursMethod\ rarely increases the misclassification rate in most cases, indicating that it preserves the activations of benign instructions in their correct safe positions.


\subsection{Mechanism of How Defensive Prompts Work}

AdaShield operates by prepending a defensive prompt to the inputs, guiding the VLM to thoroughly analyze the image and instruction before responding. Defensive prompt-based methods have been shown to risk rejection of benign requests. Here we analyze the mechanism of defensive prompt-based strategies, specifically AdaShield \cite{wang2024adashield}, from the perspective of activation shifts. 

\begin{figure}[t]
    \centering
    \includegraphics[height=9em, width=0.8\linewidth]{images/cosine-adashield.pdf}
    \vspace{-15pt}
    \caption{Cosine similarity between the activation shift induced by the defensive prompt and the safety-relevant shift $\mathbf{s}^\ell$.}
    \vspace{-20pt}
    \label{fig:defense-prompt}
\end{figure}

For each layer, we calculate the activation shift contrasting inputs with and without the defensive prompt, and compute its cosine similarity to the safety-relevant shift $\mathbf{s}^\ell$. Figure \ref{fig:defense-prompt} shows a negative cosine similarity across most layers for both safe and unsafe datasets, indicating that defensive prompts consistently push activations toward the unsafe side.  While this helps VLMs correctly identify unsafe inputs, it causes
safe inputs to be misclassified as unsafe and rejected. 
% Additionally, it pushes harmless activations too far from their original distribution, increasing perplexity and reducing utility. 
In contrast, \OursMethod\ uses the safety-related direction as an anchor, ensuring that activations are not excessively shifted toward the unsafe side, effectively mitigating this issue.

\subsection{Inference Efficiency}
We report the average inference time per response for \OursMethod\ and ECSO \cite{gou2025eyes} across all inputs on MM-SafetyBench and MME in Table \ref{table:inference-time}. \OursMethod\ increases inference time compared to the backbone, as it requires two additional forward passes to obtain image captions and input activations. However, the second forward pass is faster since it does not require autoregressive text generation, only activation extraction. The increase in inference time is smaller than ECSO, which requires two full autoregressive generations for response safety checks and image captioning.