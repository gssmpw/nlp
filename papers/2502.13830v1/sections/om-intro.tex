%!TEX root = ../main.tex


% \begin{xiaoenv}{Other Plans}
% Other Plans
% \begin{enumerate}
% \item
% Need to add the poly-round ExtCom from (constant-round OT) plus (poly-round ZK) construction.

% Then, we can formalize the results in this paper as a black-box compiler that converts
% (ZK $+$ OT) to MPC, adding at most a constant rounds. (This is inaccurate. We actually start from BB Com-n-Prove, instead of standard ZK. But anyway we get the first poly-round PQ-MPC in black-box?)
% \item
% Also, check with @Takashi if I'm correct: currently, there is no black-box PQ-MPC in polynomial rounds. I know that there are black-box MPQC (i.e., fully quantum MPC) in poly rounds, but those works makes heavy use of quantum communication. So, if we focus on post-quantum setting, the problem is still open.
% \end{enumerate}
% \end{xiaoenv}


\section{Introduction}

{\em Secure multi-party computation} (MPC) allows two or more mutually distrustful parties to compute any functionality without compromising the privacy of their inputs \cite{FOCS:Yao86,STOC:GolMicWig87}. We study foundational questions pertaining to the efficiency of secure multiparty computation in the {\em post-quantum} regime where honest parties and communication channels are classical but the adversary can be a quantum machine. We focus on two specific efficiency criteria: (1) the round-complexity, and (2) the black-box nature of the protocols. We are only concerned with general-purpose protocols in this work, i.e., protocols that can compute any well-defined multiparty functionality.

The black-box nature of the protocols manifests itself in at least two ways. First, the MPC protocol is said to have a {\em black-box construction}, if it only relies on the input-output behaviour of the underlying cryptographic primitives/assumptions. That is, the description of the MPC protocol is independent of the {\em implementation} level details of the underlying cryptographic primitives. This ensures that the efficiency of the protocol does not change with the implementation details of the underlying primitives. Moreover, such constructions remain valid even if the building-block primitives are based on a physical object such as a noisy channel or tamper-proof hardware \cite{wyner1975wire,FOCS:CreKil88,TCC:GLMMR04}.



Second, the MPC protocol is said to have a {\em black-box security-proof (or reduction)} if the security proof uses the adversary only as a black-box (i.e., only relies on its input/output functionality). We are concerned with MPC protocols that are {\em fully} black-box \cite{STOC:ImpRud89,TCC:ReiTreVad04}, i.e., they have a black-box construction as well as a black-box reduction to the underlying cryptographic primitives. Protocols that admit black-box reductions are often simpler and tend to result in more efficient implementations.

% \para{Black-Box Post-Quantum MPC.} 
The complexity of black-box MPC protocols is well understood in the classical setting, resulting in fully black-box constructions in a {\em constant} number of rounds under standard polynomial hardness assumptions \cite{STOC:Goyal11}, obtained after a long sequence of works in this direction \cite{STOC:IKLP06,STOC:IKOS07,TCC:Haitner08,C:IshPraSah08,TCC:PasWee09,TCC:CDMW09,FOCS:Wee10}.


However, these questions are wide open in the post-quantum MPC (PQ-MPC) setting where honest parties and communication channels are still classical but the adversary is allowed to be a quantum machine. This is in part due to the fact that classical techniques for performing simulation and extraction in MPC and Zero-Knowledge protocols rarely work when the adversary is a quantum machine. In fact, Chia, Chung, Liu, and Yamakawa \cite{chia2022impossibility} recently showed that standard (expected) polynomial-time black-box simulation is impossible to achieve by constant-round constructions in the post-quantum setting (unless $\NP \subseteq \BQP$). Indeed, this impossibility holds even in scenarios where honest parties have access to quantum capabilities \cite{arXiv:CCLL}. These strong results still leave  glaring feasibility questions unresolved, which we discuss next.

\para{Full simulation but non-constant rounds:} In this regime, all known constructions achieving full (or standard) simulation \cite{EC:ABGKM21,FOCS:LPY23,goyal2023concurrent}\footnote{Note that \cite{arXiv:LPY23} and \cite{FOCS:LPY23} refer to the same paper. We include two separate bibliography entries because certain lemmas appear exclusively in the arXiv version \cite{arXiv:LPY23} but not in the conference version \cite{FOCS:LPY23}, and we occasionally need to cite them specifically.} make extensive use of non-black-box techniques. Indeed, \cite{EC:ABGKM21,FOCS:LPY23} even achieve constant rounds by relying on non-black-box simulation. However, no results are known if we insist on black-box constructions, {\em even} for the two-party setting.\footnote{We note that fully black-box constructions exist if the honest parties are allowed to leverage quantum power (e.g., \cite{C:BCKM21b,EC:GLSV21}). However, this falls outside the scope of our focus on {\em post-quantum} protocols.} This raises the following question:
\begin{quote}
 {\bf Question 1:} Do there exist black-box constructions of post-quantum 2PC (and MPC) with full simulation (in more than constant number of rounds)?
\end{quote} 

\para{Relaxed simulation in constant rounds:} The everpresent desire for constant-round secure protocols has prompted exploration of alternative notions such as {\em with $\epsilon$-simulation}, which is a relaxed form of standard simulation-based security that allows for an arbitrarily small noticeable simulation error $\epsilon$. This is an extensively well-studied notion in the literature \cite{DBLP:conf/focs/DworkNRS99,C:JKKR17,STOC:BitKhuPan19} that implies other important security notions --- e.g., $\epsilon$-zero-knowledge protocols imply witness indistinguishability \cite{STOC:FeiSha90} and $\epsilon$-simulatable MPCs imply input-indistinguishable computation \cite{FOCS:MicPasRos06}. 
% We note that using {\em non-black} simulation techniques, recent work \cite{EC:ABGKM21,FOCS:LPY23} \rohit{Is ABG necessary here?} obtained a constant-round PQ-MPC protocol (where both the construction and the reduction are non-black-box). In contrast, little is known in the black-box regime. 
In this $\epsilon$-simulation regime, the recent work of \cite{C:CCLY22} made initial progress by presenting a constant-round fully black-box protocol for the two-party setting. However, obtaining similar results in the {\em multi-party setting} has remained an unsolved challenge, {\em even with stronger hardness assumptions than those in the classical setting}. This motivates our second question:
\begin{quote}
 {\bf Question 2:} Do there exist black-box, constant-round constructions of post-quantum MPC with $\epsilon$-simulation?
 \end{quote} 

% Other possible simulation-based notions not explored in this work include super-polynomial simulation (SPS) \cite{EC:Pass03,STOC:PraSah04}, and coherently expected-polynomial time black-box simulation \cite{lombardi2022post}. We see these as promising open questions in the burgeoning post-quantum security regime. 

We remark that the recent breakthrough by Lombardi, Ma, and Spooner \cite{lombardi2022post} proposed a new model for post-quantum simulation, called coherent-runtime expected quantum polynomial time simulation. In this model, a simulator is allowed to coherently run multiple computational branches with different runtime so that they can interfere with one another.   
They show a set of results in this model that bypass the impossibility result of \cite{chia2022impossibility}. 
We emphasize that in the current work, we focus on the traditional notion of quantum \emph{strict}, rather than \emph{expected}, polynomial-time simulation. It is also worth mentioning that although the \cite{lombardi2022post}'s coherent-runtime expected QPT simulation implies $\epsilon$-simulation, the round complexity of {\em fully} black-box PQ-MPC has not been resolved in their model either. We leave it as an interesting direction to investigate the implications of the \cite{lombardi2022post} model on the round complexity of black-box PQ-MPC.


% We leave it as an interesting direction to investigate the further potential of the round complexity of black-box post-quantum MPC with coherent-runtime expected quantum polynomial time  simulation.


\iffalse
\begin{xiaoenv}{}
@Rohit: pls finish:
\begin{itemize}
\item 
Full simulation but polynomial rounds: All known \cite{EC:ABGKM21,FOCS:LPY23,goyal2023concurrent} constructions makes heavy use of non-black-box techniques. Indeed, \cite{EC:ABGKM21,FOCS:LPY23} even achieves constant rounds using non-black-box simulation. However, no results is known if we insist on black-box constructions, then for the two-party setting.

This raise the first question:
\begin{quote}
 {\bf Question 1:} does there exist black-box constructions of Post-Quantum 2PC (and MPC) in polynomial rounds?
 \end{quote} 

 \item 
Another direction is to target at $\epsilon$-simulation. Now, you can simply copy the two paragraph below the red line. Then, say that This raise the second question:
\begin{quote}
 {\bf Question 2:} does there exist black-box, constant-round constructions of Post-Quantum  MPC with $\epsilon$-simulation?
 \end{quote} 

\end{itemize}
\end{xiaoenv}
\fi


\subsection{Our Results}
In this work, we give a positive resolution of these two questions.

\subsubsection{Black-Box PQ-2PC and PQ-MPC with Full Simulation}

We obtain the first fully black-box PQ-2PC protocol from minimal assumptions, in any super-constant number of rounds, which is (asymptotically) optimal for black-box simulation (due to the lower bound of \cite{chia2022impossibility}):
\begin{theorem}\label{thm:informal:full2PC}
There exists a $\omega(1)$-round,\footnote{While the term $\omega(1)$ is typically used for lower bounds, in our context, we use it to mean that ``any super-constant value suffices.''} black-box construction of PQ-2PC (with full simulation), from the minimal assumption of post-quantum, semi-honest oblivious transfers (OTs). 
\end{theorem}
To build this protocol, we follow the approach of \cite{C:CCLY22} originally designed for black-box PQ-2PC with {\em $\epsilon$-simulation}. Very roughly speaking, the most crucial component in their approach is a {\em post-quantum extractable commitment} with $\epsilon$-simulation. This primitive is similar to the standard notion of extractable commitments in the classical setting, but it additionally requires that the post-extraction state of $C^*$ (the malicious committer) should be $\epsilon$-indistinguishable from that in the real execution.\footnote{We remark that while simulating for $C^*$'s post-extraction state is trivial in the classical setting, this task is particularly challenging when $C^*$ is a quantum machine (see \cite{C:CCLY22}).} We observe that we can use \cite{C:CCLY22} template to also achieve the standard notion of fully simulatable PQ-2PC (instead of just $\epsilon$-simulatability) if we can just make the underlying extractable commitment fully-simulatable. 

While the goal is clear, achieving this turns out to be quite non-trivial. To the best of our knowledge, all existing black-box constructions for this task crucially utilize quantum communication in their protocol \cite{C:BCKM21b,EC:GLSV21}. Since our aim is to build a {\em post-quantum} protocol, this does not suit us. To address this issue, we introduce the first black-box construction of post-quantum extractable commitments with full simulation. Our construction makes use of post-quantum semi-honest OTs. We note that while semi-honest OTs may not be the minimal assumption for extractable commitments per se, it is however minimal for our eventual goal of PQ-2PC.
\begin{lemma}\label{lemma:informal:fullExtCom}
    Assuming the existence of post-quantum semi-honest OTs, there exists a $\omega(1)$-round, black-box construction of post-quantum extractable commitments with full simulation. 
\end{lemma}

 

Given our construction of black-box PQ-2PC, we can use it to get a construction for fully simulatable PQ-MPC. This is done by invoking the \cite{C:IshPraSah08} black-box compiler to get a polynomial round PQ-MPC --- the key thing to notice is that our 2PC construction can also serve as the kind of OT protocol that is required by this compiler, albeit necessitating sequential composition for multiple OT calls. We refer the reader to \Cref{sec:full-MPC} for further details.

\begin{theorem}\label{thm:informal:fullMPC}
There exists a black-box construction of PQ-MPC with full simulation, from the minimal assumption of post-quantum semi-honest OTs. 
\end{theorem}





\iffalse 
\begin{xiaoenv}{}
@Rohit:

\begin{itemize}
\item 
We present the first $\omega(1)$-round, black-box construction of PQ-2PC from the minimal assumption of post-quantum semi-honest OT. Make a formal theorem for this result.
\item
Say that we follow the paradigm established in \cite{C:CCLY22} for black-box PQ-2PC with $\epsilon$-simulation. Very roughly speak, \cite{C:CCLY22} makes use of a post-quantum extractable commiment with $\epsilon$-simulation, and we find that we can achieve the standard notion of PQ-2PC (instead of the $\epsilon$-simulatable one) if we make their extractable commitment fully-simulatale. Thus, the problem can be reduced to build a black-box post-quantum extractable commitements (with full simulation).

\item 
However, this turns out to be non-trivial. To the best of our knowledge, all exsiting black-box constructions for this task \cite{C:BCKM21b,EC:GLSV21} utilized quantum communication. This does not suffice for our purpose to build a post-quantum protocol (where honest parties are classical).

\item 
As a technical contribution, we introduce the first $\omega(1)$-round, black-box construction of post-quantum ExtCom (with full simulation). Our construction makes use of post-quantum semi-honest OT. Though this may not be the minimal assumption for extractable commitments, but this is minimal for our eventual goal of post-quatum 2PC. 

Make a lemma for this results of post-quantum ExtCom from OT.	

\item
Once black-box PQ-2PC is obtained, invoke the \cite{C:IshPraSah08} compiler to get the poly round black-box PQ-MPC. Refer to \Cref{sec:full-MPC} for details. Make a theorem for this results of post-quantum MPC (from the minimal assumption of semi-honest OT).
\end{itemize}
\end{xiaoenv}
\fi 

\subsubsection{Application I: LOCC MPC without OWFs}

A recent breakthrough by Kretschmer, Qian, and Tal \cite{STOC:KreQiaTal25} constructed a classical oracle relative to which $\mathbf{P} = \NP$, yet $\BQP$-computable (and quantum-secure) trapdoor OWFs exist, making them impossible to ``de-quantize'' in a black-box manner. This relativized world is particularly surprising when contrasted with its classical counterpart, where $\BPP$-computable OWFs can be de-randomized in a black-box manner \cite{FOCS:ImpLub89}. 

\cite{STOC:KreQiaTal25} established their main theorem via a fully black-box reduction. Consequently, relative to the same classical oracle, their theorem extends to demonstrate the existence of any ``LOCC'' cryptographic object that admits a fully black-box reduction to trapdoor OWFs in the post-quantum setting.  Here, LOCC stands for ``local operations and classical communication,'' meaning that parties can perform local quantum operations, but all communication must be classical.

 By combining our \Cref{thm:informal:full2PC} (and \Cref{thm:informal:fullMPC}) above {\em and} the post-quantum fully black-box reduction from semi-honest OTs to trapdoor OWFs from \cite{FOCS:GKMRV00}\footnote{Although the original work \cite{FOCS:GKMRV00} was focused on the classical setting, it is straightforward to see that their reduction holds in the post-quantum setting as well.}, the authors of \cite{STOC:KreQiaTal25} were able to derive the following \Cref{cor:application-I} as a corollary of their main theorem. As explained in \cite{STOC:KreQiaTal25}, our \Cref{thm:informal:full2PC} (and \Cref{thm:informal:fullMPC}) are essential for this result, as previous 2PC/MPC constructions either make non-black-box use of semi-honest OTs or lack security proofs in the presence of a quantum attacker.

 \begin{corollary}[{\cite[Corollary 39]{STOC:KreQiaTal25}, strengthened\footnote{The original \cite[Corollary 39]{STOC:KreQiaTal25} relied only on our \Cref{thm:informal:full2PC} to obtain maliciously secure OTs (and thus 2PC). Here, we extend it to MPC using the stronger \Cref{thm:informal:fullMPC}.}}]
 \label{cor:application-I}
There exists a classical oracle relative to which classical-communication and quantum-secure MPC exist, yet $\mathbf{P} = \NP$. 
 \end{corollary}



\subsubsection{Constant-Round Black-Box PQ-MPC with $\epsilon$-Simulation}
As for $\epsilon$-simulation, we study the general multi-party setting, and obtain the first constant-round fully black-box construction for PQ-MPC by relying on the same (more accurately, the post-quantum analog of) hardness assumptions as for the state-of-the-art {\em classical} MPC protocols:
\begin{theorem}\label{thm:informal:main}
There exists a constant-round black-box construction of $\epsilon$-simulatable PQ-MPC from a variety of standard post-quantum cryptographic primitives, such as lossy public-key encryption, linearly homomorphic public-key encryption, or dense cryptosystems.\footnote{We did not mention post-quantum (enhanced) trapdoor permutations as they are not known from standard quantum hardness assumptions yet. But as long as they exist, they can be included in \Cref{thm:informal:main} as well.}
\end{theorem}
Our approach to \Cref{thm:informal:main} follows a pipeline established for classical constant-round black-box MPC, which has evolved through a series of prior work \cite{C:IshPraSah08,TCC:PasWee09,FOCS:Wee10,STOC:Goyal11,FOCS:GLOV12}. In broad terms, we demonstrate that if the building components used in this pipeline are properly instantiated using their post-quantum equivalents, the outcome can be extended to the post-quantum realm. Further insights into this process are elaborated upon in \Cref{sec:tech-oeverview:reduction-to-NMC}. For now, it is worth noting that a critical step in this framework is the development of a black-box {\em 1-many non-malleable} commitment scheme in constant rounds. This constitutes the primary technical challenge in the post-quantum setting.

\para{Post-Quantum 1-Many Non-Malleability.} Non-malleable commitments \cite{STOC:DolDwoNao91} are commitments secure in the so-called {\em man-in-the-middle} (MIM) setting: An adversary $\mcal{M}$ plays the role of a receiver in one instance of a commitment (referred to as the {\em left session}), while simultaneously acting as a committer in another session (referred to as the {\em right session}). During the execution, $\mcal{M}$ can potentially make the value committed in the right session depend on that in the left session, in a malicious manner that is to her advantage. Notice that this is not breaking the hiding property of the commitment scheme, as $\mcal{M}$ may be able to conduct the above attack without explicitly learning the value committed in the left session. Furthermore, a commitment is said to be {\em 1-many} non-malleable if it is secure in the MIM setting with one left session but {\em polynomially many} right sessions, i.e., the adversary $\mcal{M}$ cannot make the {\em joint distribution} of the values committed across all right sessions depend on the one committed in the left session.


In the classical setting, the existence of black-box constant-round 1-many non-malleable commitments was established under the minimal assumption of one-way functions \cite{STOC:Goyal11,FOCS:GLOV12}. Such commitments played a pivotal role in enabling black-box constant-round MPC. However, in the post-quantum context, achieving non-malleability (even in the 1-1 MIM setting) with constant rounds proves to be an exceptionally challenging task. A recent result by \cite{FOCS:LPY23} succeeded in obtaining a post-quantum 1-1 non-malleable commitment in constant rounds. Yet, their construction relies significantly on {\em non-black-box} usage of post-quantum one-way functions, and it remains uncertain if their scheme can maintain non-malleability in the more demanding 1-many scenario.


In this work, we obtain a black-box and constant-round construction for a {\em weak version} (explained shortly) of 1-many post-quantum non-malleable commitments, from the minimal assumption of post-quantum one-way functions. Compared to the standard notion of 1-many non-malleability, our construction is restricted in the following sense:
\begin{itemize}
 \item 
 It supports a polynomial {\em tag space}\footnote{Each execution of non-malleable commitments requires a unique tag; otherwise, it is impossible to protect against MIM attacks (see \cite{STOC:Pass04} for related discussions).}, instead of a exponential-size tag space as required by  the standard definition.
 \item 
It is non-malleable only in the {\em synchronous} setting, meaning that all the messages of the left session and the polynomially many right sessions are sent in parallel.
\item
It is non-malleable conditioned on the fact that the honest receiver {\em in every right session} accepts. That is, if there is some right session where the receiver rejects during the commit stage (the committed value for this session is then defined to be $\bot$), then our protocol does not provide any non-malleability guarantee. (We refer to \Cref{def:NMCom:weak:pq} for a formal treatment.)
 \end{itemize} 
We emphasize that while our construction may not be as powerful as the standard 1-many post-quantum non-malleable commitments, it already has non-trivial applications. Firstly, such a scheme suffices for our main focus of post-quantum MPC. Additionally, it also reduces the assumption utilized in a lower bound of quantum parallel repetition (as we will discuss shortly). We believe it will find more applications in the future.
\begin{theorem}\label{thm:informal:1-many-NMC}
Assuming the existence of post-quantum one-way functions, there exists a black-box and constant-round construction of {\em weak} (as explained above) post-quantum 1-many non-malleable commitments.
\end{theorem}

It is known that 1-many non-malleability implies the seemingly more demanding {\em many-many} non-malleability, using a standard hybrid argument. This reduction holds even in the post-quantum setting (see e.g., \cite[Lemma 7.3]{EC:ABGKM21}). This yields the following corollary of \Cref{thm:informal:1-many-NMC}. 
% But we remark that the 1-many notion in \Cref{thm:informal:1-many-NMC} already suffices for all the applications in this paper.

\begin{corollary}\label{cor:informal:many-many-NMC}
Assuming the existence of post-quantum one-way functions, there exists a black-box and constant-round construction of {\em weak} (as explained above) post-quantum many-many non-malleable commitments.
\end{corollary}


\subsubsection{Application II: Quantum Parallel Repetition Lower Bound} 

Interestingly, our many-many non-malleability commitments find further application in establishing the lower bound for parallel repetition of post-quantum arguments. The recent work by Bostanci, Qian, Spooner, and Yuen  \cite{bostanci2023efficient} shows that parallel repetition does not always reduce the soundness error of post-quantum interactive argument systems. In particular, for any polynomial $k(\secpar)$, the authors of \cite{bostanci2023efficient} constructed a {\em constant-round} interactive argument for which a $k$-fold parallel repetition does not reduce the (post-quantum) soundness at all. Their construction makes use of many-many post-quantum (synchronous) non-malleable commitments in constant rounds, which were not known previously. Now, the above \Cref{cor:informal:many-many-NMC} reduces the assumption used in \cite{bostanci2023efficient} to the existence of post-quantum one-way functions. We state the result in the following \Cref{cor:informal:parallel-rep} and refer the interested reader to \cite[Theorem 1.6 and Section 6]{bostanci2023efficient}\footnote{We remark that \cite[Theorem 1.6]{bostanci2023efficient} assumes `concurrent-secure' many-to-many non-malleable commitments. But as the authors have shown in \cite[Section 6]{bostanci2023efficient}, `parallel-secure' (i.e., synchronous) many-to-many non-malleable commitments suffice. Moreover, in their application, if the verifier in one session rejects, the entire execution is considered rejected. Thus, our weak many-many non-malleability notion suffices.} for more information.
\begin{corollary}\label{cor:informal:parallel-rep}
Assume the existence of post-quantum one-way functions. Then, for every polynomial $k(\secpar)$,
there is a constant-round post-quantum interactive argument such that a $k(\secpar)$-fold repetition does not decrease the soundness error compared to the original protocol.
\end{corollary}

\subsection{More Related Work on Non-Black-Box Constructions}


Besides the aforementioned works \cite{EC:ABGKM21,FOCS:LPY23,goyal2023concurrent}, other non-black-box constructions of PQ-2PC also exist, such as \cite{AFRICACRYPT:LunNie11,C:HalSmiSon11}. This naturally raises the question: how large is the gap between these non-black-box PQ-2PC protocols and our black-box PQ-2PC in \Cref{thm:informal:full2PC}? What are the key obstacles preventing the removal of non-black-box components in these constructions?

In fact, these works adopt a fundamentally different approach from ours, as we elaborate below.


\cite{AFRICACRYPT:LunNie11} primarily focused on feasibility results rather than the black-box nature of the construction. Indeed, it is unclear how to remove the non-black-box components from the \cite{AFRICACRYPT:LunNie11} approach. This is because \cite{AFRICACRYPT:LunNie11} builds PQ-2PC following the GMW approach \cite{STOC:GolMicWig87}: first constructing a semi-honest protocol and then achieving active security by adding ZK proofs on each message to enforce honest behavior from the parties. This GMW approach is inherently non-black-box due to its reliance on ZK proofs for cryptographic statements (i.e., the parties' next-message functions).

Even in the classical setting, black-box constructions of 2PC/MPC move away from the GMW approach and instead follow a very different path established by the line of works \cite{C:IshPraSah08,TCC:PasWee09,TCC:CDMW09,TCC:Haitner08,STOC:IKLP06,FOCS:Wee10,STOC:Goyal11}. Briefly, the key advantage of this line of work lies in the development of techniques that enforce honest behavior without requiring ZK proofs for cryptographic statements, while achieving a constant number of interactions. Our constructions follow this line of work in the post-quantum setting, and therefore have little overlap with the \cite{AFRICACRYPT:LunNie11} approach.

A similar situation applies to \cite{C:HalSmiSon11}. Essentially, the PQ-2PC from \cite{C:HalSmiSon11} follows the classical approach established by \cite{STOC:CLOS02}. This is another inherently non-black-box approach where a commit-and-prove protocol is executed on cryptographic languages to enforce honest behavior from the parties. This can be viewed as a variant of the GMW compiler in the Universal-Composable (UC) framework. As such, there is little common ground for further comparison.



