%!TEX root = ../main.tex

\section{Post-Quantum Non-Malleable Commitments: One-to-One and Two-sided}
\label{sec:two-sided:main-body}

In this section, we show how to remove the `one-sided' restriction from \Cref{protocol:BB-NMCom}. 
 
Recall that our proof for the non-malleability of \Cref{protocol:BB-NMCom} works only if $t<\tilde{t}$. However, this is not guaranteed in the real main-in-the-middle attack---the adversary can of course use a smaller tag in the right session. Fortunately, this problem can be addressed by the so-called `two-slot' technique proposed by Pass and Rosen \cite{STOC:PasRos05}. The idea is to create a situation where no matter how the MIM adversary $\mcal{M}$ schedules the messages, there is always a `slot' for which the `$t<\tilde{t}$' condition holds; As long as this is true, non-malleability can be proven using the same techniques as we did for \Cref{protocol:BB-NMCom}.

To do that, first observe that the only place where \Cref{protocol:BB-NMCom} makes use of the tag $t$ is \Cref{bbnmc:hard-puzzle:puzzle-setup} {\bf Hard Puzzle Setup}: The receiver is required to setup a $t$-solution hard puzzle where $t$ is determined by the tag. Of course, \Cref{prot:bbnmc:puzzle-sol-reveal,prot:bbnmc:PoC} also depend on $t$ but that is rather a consequence of \Cref{bbnmc:hard-puzzle:puzzle-setup} using a $t$-solution hard puzzle. 

This observation allows us to instantiate the \cite{STOC:PasRos05} technique for \Cref{protocol:BB-NMCom} as follows. We view \Cref{bbnmc:hard-puzzle:puzzle-setup} as a `slot' in \cite{STOC:PasRos05} terminology. We ask the receiver to repeat this `slot' twice sequentially, using $t$ and $(T-t)$ as their respective tag, where recall that $T$ is the upper-bound for the size of tag space and is a polynomial on the security parameter $\secpar$. That is,
\begin{itemize}
 \item
 {\bf Slot-A:}
$R$ first executes \Cref{bbnmc:hard-puzzle:puzzle-setup} as it is, setting a $t$-solution hard puzzle;
\item
{\bf Slot-B:}
$R$ then executes \Cref{bbnmc:hard-puzzle:puzzle-setup} again, but using $(T-t)$ in place of $t$ in the first execution. This sets a $(T-t)$-solution hard puzzle. 
 \end{itemize} 
We also modify \Cref{prot:bbnmc:puzzle-sol-reveal,prot:bbnmc:PoC} as follows:
\begin{itemize}
 \item
In \Cref{prot:bbnmc:puzzle-sol-reveal}, $R$ reveals the solutions to {\em both} the $t$ solutions w.r.t.\ {\bf Slot-A} {\em and} the $(T-t)$ solutions w.r.t.\ {\bf Slot-B};
\item
In \Cref{prot:bbnmc:PoC}, we change the trapdoor statement from `$C$ manages to commit to a puzzle solution in \Cref{prot:bbnmc:extcom}' to `$C$ manages to commit to a puzzle solution {\em either} for {\bf Slot-A} {\em or} for {\bf Slot-B} in \Cref{prot:bbnmc:extcom}'.
 \end{itemize} 
By the above design, it is easy to see that one of the following case must happen no matter how $\mcal{M}$ sets the tags $t$ and $\tilde{t}$:
\begin{enumerate}
\item
{$t = \tilde{t}$:} This is the trivial case that is already ruled out by the definition of non-malleability.
\item
{$t < \tilde{t}$:} In this case, non-malleability follows by applying the same argument as we did for \Cref{protocol:BB-NMCom} to {\bf Slot-A}. 
\item
{$t > \tilde{t}$:} In this case, it must hold that $(T-t) < (T-\tilde{t})$. In other words, the tag for the left {\bf Slot-B} is smaller than the tag for the right {\bf Slot-B}. Therefore,  non-malleability follows by applying the same argument as we did for \Cref{protocol:BB-NMCom} to {\bf Slot-B}. 
\end{enumerate}
Therefore, the modified protocol is non-malleable without the `one-sided' restriction.

We remark that the same technique has been employed by \cite{FOCS:LPY23} to remove the `one-sided' restriction in their original protocol as well. Our application does not encounter any new challenges compared with the same step in \cite{FOCS:LPY23}.  Thus, we omit the proof details and only present the formal description of this updated protocol in \Cref{sec:two-sided:full}.

We summarize the result of this section as the following theorem.

\begin{theorem}\label{thm:two-sided:non-malleability}
  Assuming the existence of post-quantum one-way functions, there exists (i.e., \Cref{protocol:BB-NMCom:two-sided}) a black-box, constant-round construction of 1-1 (two-sided) post-quantum non-malleable commitments (as per \Cref{def:NMCom:pq} with $k=1$) in the synchronous setting,  supporting tag space $[T]$ with $T(\secpar)$ being any polynomial in the security parameter $\secpar$.
\end{theorem} 