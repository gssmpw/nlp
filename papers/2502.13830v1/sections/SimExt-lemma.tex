%!TEX root = ../main.tex
\section{Simulation-Extractor \textnormal{$\SimExt$}: 1-1 Settings}
\label{sec:simext:1-1}

\subsection{Noisy Simulation-Extraction Lemma}
\label{sec:noisy-sim-ext}

\begin{lemma}[Noisy Simulatable-Extraction Lemma]
\label{lem:Noisy-SimExt}
Let $\mcal{G}$ be a QPT algorithm that takes the security parameter $1^\secpar$, an error parameter $1^{\gamma^{-1}}$, a quantum state $\rho$, {and a classical string $z$} as input,  and outputs  $d\in \{\top,\bot\}$ and a quantum state $\rho_\out$. 

Suppose that there exists a QPT algorithm $\mcal{K}$ (referred to as the simulation-less extractor) that takes as input the security parameter $1^\secpar$, two error parameters $1^{\gamma^{-1}}$ and $1^{\zeta^{-1}}$, a quantum state $\rho$, {and a classical string $z$}, and outputs $s\in \bit^{\poly(\secpar)}\cup \{\bot\}$   
satisfying the following w.r.t.\ some sequence of classical strings {$\{s^*_{z}\}_{z\in \bit^*}$.}
\takashi{The index of $s^*$ is changed from $\secpar$ to $z$. (Imagine that $s^*$ is the committed message and $z$ is the partial transcript. Then this should be a more natural formalization.)}

%\footnote{Strictly speaking, we consider a sequence $\{s^*_\secpar\}_{\secpar\in \mathbb{N}}$. We simply denote by $s^*$ to refer to $s^*_\secpar$. \label{footnote:sequence_s}}   
\begin{enumerate}
 \item  \label{item:s_star_or_bot}
    For any $\secpar$,  $\rho_\secpar$,  $z_\secpar$, and any noticeable functions $\gamma(\secpar)$ and $\zeta(\secpar)$, it holds that  

$$\Pr[s \notin \Set{s^*_{z_\secpar}, \bot}~:~s \la \mcal{K}(1^\secpar,1^{\gamma^{-1}}, 1^{\zeta^{-1}}, \rho_\secpar,{z_\secpar})]\le \zeta(\secpar) +\negl(\secpar).$$


    \item \label{item:gamma_delta} \takashi{I modified the statement to match \Cref{lem:Simultaneous-SimExt}.}
 For any noticeable function $\gamma(\secpar)$, there exists a noticeable function $\delta(\secpar)$, 
 which is efficiently computable from $\gamma(\secpar)$, so that the following requirement is satisfied: For 
 any noticeable function $\zeta(\secpar)$ and
 any sequence $\{\rho_\secpar,{z_\secpar}\}_{\secpar\in\mathbb{N}}$ of polynomial-size quantum states and classical strings, %\footnote{Similarly to \Cref{footnote:sequence_s}, we consider a sequence $\{\rho_\secpar\}_{\secpar\in \mathbb{N}}$ and denote by $\rho$ to mean $\rho_\secpar$.}    
 if 
$$
\Pr[d=\top ~:~ (d,\rho_\out) \leftarrow \mcal{G}(1^\secpar,1^{\gamma^{-1}},\rho_\secpar,{z_\secpar})]\geq  8\gamma(\secpar), 
$$  
then 
$$
\Pr[s =s^*_{{z_\secpar}}~:~ s\la \mcal{K}(1^\secpar,1^{\gamma^{-1}}, 1^{\zeta^{-1}}, \rho_\secpar,{z_\secpar})]\geq   \delta(\secpar)-\zeta(\secpar)-\negl(\secpar).
$$
\end{enumerate}
Then, there exists a QPT algorithm $\SimExt$ such that for any noticeable function $\epsilon=\epsilon(\secpar)$, there exists a noticeable function $\gamma=\gamma(\secpar)\le \epsilon(\secpar)$ that is efficiently computable from $\epsilon$ and satisfies the following:
For any sequence $\{\rho_\secpar,{z_\secpar}\}_{\secpar\in\mathbb{N}}$ of polynomial-size quantum states and classical strings,  
$$
\{\SimExt(1^\secpar,1^{\epsilon^{-1}},\rho_\secpar,{z_\secpar})\}_{\secpar \in \Naturals}
~\statind_{\epsilon}~ 
\{(\rho_\out,\Gamma_d(s^*_{{z_\secpar}}))~:~(d,\rho_\out)\leftarrow \mcal{G}(1^\secpar,1^{\gamma^{-1}},\rho_\secpar,{z_{\secpar}})\}_{\secpar \in \Naturals},
$$
where  $
\Gamma_d(s^*_{{z_\secpar}})\defeq 
\begin{cases}
s^*_{{z_\secpar}} & \text{if}~ d=\top \\
\bot & \text{otherwise}
\end{cases}
$.
\end{lemma}
\takashi{This is almost identical to (the updated version of) the proof of Lemma 20 in LPY.
The full proof takes 7 pages, but I guess the difference will be just a few words. Please check if the following proof sketch is okay.
}

\begin{proof}[Proof sketch]
    Since the proof is almost identical to that of \cite[Lemma 20]{arXiv:LPY23}, we only describe the differences.\footnote{\cite{arXiv:LPY23} is the full version of \cite{FOCS:LPY23} on arXiv.} 
    There are the following two differences in the statement:
    \begin{itemize}
    \item We introduce an additional error parameter $\zeta$, which gives an upper bound of the probability that $\mcal{K}$ outputs $s\notin \{s^*_{z_\secpar},\bot\}$. In \cite[Lemma 20]{arXiv:LPY23}, the probability was assumed to be $0$.
    \item The lower bound of $\mcal{G}'s$ success probability in \Cref{item:gamma_delta} is $8\gamma(\secpar)$ instead of $\gamma(\secpar)$. 
    \end{itemize}
The second point can be easily dealt with by simply replacing $\gamma$ with $8\gamma$ in the original proof.
%setting $\delta$ in such a way that \Cref{item:gamma_delta} holds for $\gamma=\frac{1}{8}\left(\frac{\epsilon}{5}\right)^4$ instead of $\gamma=\left(\frac{\epsilon}{5}\right)^4$ at the beginning of the proof of \cite[Lemma 26]{FOCS:LPY23}. 
The first point introduces an additional noticeable error polynomially related to $\zeta$ in the simulation for the case of $b=\top$.
Since $\zeta$ can be chosen to be an arbitrarily small noticeable function, we can manage the additional error by appropriately setting the parameters. 

Below, we give more concrete explanation for the readers who are familiar with the proof of \cite[Lemma 20]{arXiv:LPY23}. %We remark that we are referring to the latest version of \cite{arXiv:LPY23} on arXiv which was updated after the original publication.\footnote{\url{https://arxiv.org/abs/2207.05861v3}} 
We only need to modify the proof of \cite[Lemma 26]{arXiv:LPY23}, which claims that the simulation for the case $b=\top$ works.  
The first difference causes an error probability $\zeta+\negl(\secpar)$ in \cite[Claims 29 and 30]{arXiv:LPY23}, which eventually causes an error $\zeta^{1/2}+\negl(\secpar)$ in \cite[Eq. (84)]{arXiv:LPY23} where the square root appears due to the gentle measurement lemma.
As a result, \cite[Eq. (84)]{arXiv:LPY23} should be replaced with  $\left(12(8\gamma)^{1/2}+2\nu^{1/2}\right)^{1/2}+\zeta^{1/2}+\negl(\secpar)$ instead of $\left(12\gamma^{1/2}+2\nu^{1/2}\right)^{1/2}$.
(Note that $\gamma$ is replaced with $8\gamma$ to deal with the second point as explained above.) 
It suffices to set 
$
\gamma:=\frac{1}{8}\left(\frac{\epsilon}{10}\right)^4$, 
$\nu:=\left(\frac{\epsilon}{4}\right)^4$, and 
$\zeta:=\left(\frac{\epsilon}{2}\right)^2$ so that $\left(12(8\gamma)^{1/2}+2\nu^{1/2}\right)^{1/2}+\zeta^{1/2}<\epsilon$.
\end{proof}

