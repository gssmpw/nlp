%!TEX root = ../main.tex
\section{Post-Quantum $\epsilon$-Simulatable MPC}
\label{sec:MPC}

In this section we will describe and show security of a black-box, constant-round $\epsilon$-MPC protocol. In fact we have gathered essentially all the ingredients needed for this task. The sole remaining component is the black-box compiler given in \cite{C:IshPraSah08}. Their protocol is a constant-round black-box MPC protocol, albeit with {\em UC security} and additionally assuming {\em ideal} OT channels. Here we will argue that this protocol when initialized with our malicious parallel OT protocol\footnote{Indeed, this observation has been employed in the classical setting \cite{FOCS:Wee10,STOC:Goyal11}.}, will give us an MPC protocol with all the desired properties. We capture this in the following lemma. 

\begin{lemma}
    The MPC protocol described in \cite{C:IshPraSah08}, instantiated with the OT construction given in \Cref{prot:mal-OT}  (in lieu of an ideal OT functionality) is a \emph{black-box}, {\em constant-round}, \emph{post-quantum} $\epsilon$-simulatable MPC protocol as defined in \Cref{def:mpc}. 
\end{lemma}

%\rohit{need def for MPC}
\begin{proof}
    The construction simply involves instantiating the MPC protocol from \cite{C:IshPraSah08} using our OT scheme, as is already stated. We refer to their protocol as IPS for convenience. 
    
    We begin with a brief overview of the IPS MPC protocol. This involves composing {\em two} MPC protocols (titled the {\em inner} and {\em outer} protocols) in a specific fashion. The outer protocol uses the so-called {\em client-server} model, which involves parties called servers that have no input of their own but carry out the majority of the computation in the protocol. The key stratagem devised in IPS is to {\em emulate} the function of these servers distributedly using the inner MPC protocol. To ensure honesty, the protocol uses a mechanism introduced in IPS known as {\em watchlists}, which ensure that each party is able to monitor some emulated servers. 

    Thus, in the running of the IPS protocol, there are various OT calls that are of two kinds. The first kind is used to initialize the watchlist mechanism, and this can be performed at the start of the protocol. The second kind is in the operation of the inner OT protocol that is used to emulate the servers (the inner protocol is in the OT-hybrid model, and needs to make calls to the ideal OT functionality). 

    We make the following two observations about the IPS protocol. These are easily verifiable from the descriptions present in \cite{C:IshPraSah08}. The first is that the watchlist setup can be initialized with an $n$-party OT functionality (this is observed in their work), i.e., $2\cdot\binom{n}{2}$ OT calls in totally where each pair of parties run two OT calls with reversed role of sender and receiver. So we can use an $n$-fold parallel execution \Cref{prot:mal-OT} in the beginning that suffices to setup the watchlists. The second is a {\em randomized OT trick} that can be used to `prepone' the OT executions required by the inner protocol (this is also observed in their work). This modification is also needed for security. The idea is to basically initially perform $n$-fold OT executions with {\em random} values for the senders and receivers. Subsequently, the sender can send appropriately offset values (that encode OT inputs of its choice) to the receiver and the latter can recover its intended message from this. 

    This is to say that using this trick, the inner OT calls can also be pushed to the beginning of the protocol where we perform an $n$-fold parallel execution of \Cref{prot:mal-OT} (for sufficiently long sender inputs). Therefore, we can complete an execution of the IPS protocol by beginning with two $n$-fold parallel (randomized) OT executions, for the watchlist and for the inner MPC protocol respectively. Subsequently, we proceed with the IPS protocol, setting up the watchlists and then executing the composed MPC protocol. Everytime the inner protocol would make an OT call, we use the random OT transformation and consume a predefined portion of the initial parallel randomized OT call to perform the actual OT interaction in the protocol. \xiao{check this part with rohit}
    
    %More precisely, we run the IPS protocol with due care as to run carefully demarcated \emph{OT rounds} whenever the IPS protocol chooses to make an ideal OT call. We stress that the OT rounds involve \emph{no other steps} of the protocol execution - OT rounds only execute an instance of the parallel malicious OT protocol, while suspending all other steps. The bound on the number of parallel sessions is a fixed polynomial quantity \xiao{check the structure of IPs with Rohit.} that is specified by the IPS protocol and depends on the number of MPC parties. It is clear to see that this is an admissible way to run the IPS protocol in the standalone model. It is also easy to verify correctness of this composed protocol. 
    
    Next we will argue why this achieves the desired security guarantee - security is already somewhat apparent and straightforward to establish, and we limit ourselves to addressing the more prominent concerns in this regard. We treat these in turn. 

    \para{Constant-Round:} The first thing to determine is simply whether the composed protocol is still constant round. While the total number of atomic OT calls made in the IPS protocol does depend on the number of parties (and hence grows polynomially with $\secpar$), these can be \emph{batched} into a single parallel OT execution and shunted to the start of the protocol as described above. Now the IPS protocol itself is constant round (this includes the interactions made due to the randomized OT trick). In turn, our OT protocol from \Cref{prot:mal-OT} also runs in constant rounds. The resulting protocol is therefore constant rounds. Indeed, this exact pipeline has been used in previous work on black-box MPC protocols to get protocols with {\em constant round overhead} (over the parallel OT part) in the classical standalone setting (see \cite{FOCS:Wee10,STOC:Goyal11}). 

    \para{Security:} As pointed out, the security of the IPS protocol when initialized with a parallel OT protocol has been noted and employed in previous work (\cite{FOCS:Wee10,STOC:Goyal11}). Our setting however presents two new challenges that are not present in the more standard setting, and we tackle them in turn. 

    \subpara{Post-Quantum security:} A simple examination of the IPS security proof and that of our OT protocol reveals that both of these are \emph{black-box} and also \emph{quantum compatible} - namely, they enjoy straightline simulation and are not reliant on classical rewinding. 
    % Roughly, this is the case due to the UC based nature of the IPS protocol, and by design for our OT (since we use post-quantum simulation-extraction techniques in it for this exact reason). The observation regarding the quantum compatibility of the IPS protocol is not new and has been recorded in \cite{EC:ABGKM21}. 
     The combined security proof for the composed MPC protocol inherits these properties. 

    \subpara{Sequential composition:} A detail we have elided so far is that our OT is limited to $\varepsilon$-simulatability. This can affect the hybrids where we \emph{sequentially} simulate various parallel OT executions in the protocol. 
    % This is because there is no folklore \emph{sequential composition lemma} for primitives with only $\varepsilon$-security: indeed, more careful consideration is needed to control the error growth in such composition, since we do not have the nice composition properties of negligible errors. 
     Fortunately, this exact kind of post-quantum sequential composition guarantee for $\varepsilon$-simulation has been shown in the work of \cite[Section 7.2]{C:CCLY22}. 
     
\end{proof}
 