%!TEX root = ../main.tex
\begin{abstract}\addcontentsline{toc}{section}{Abstract}

We study the round-complexity of secure multi-party computation (MPC) in the post-quantum regime where honest parties and communication channels are classical but the adversary can be a quantum machine. Our focus is on the {\em fully} black-box setting where both the construction as well as the security reduction are black-box in nature. In this context, Chia, Chung, Liu, and Yamakawa [FOCS'22] demonstrated the infeasibility of achieving standard simulation-based security within constant rounds, unless $\NP \subseteq \BQP$. This outcome leaves crucial feasibility questions unresolved. Specifically, it remains unknown whether black-box constructions are achievable within polynomial rounds; additionally, the existence of constant-round constructions with respect to {\em $\epsilon$-simulation}, a relaxed yet useful alternative to the standard simulation notion, remains unestablished. 



\hspace{1em}
This work provides positive answers to the aforementioned questions. We introduce the first black-box construction for post-quantum MPC in polynomial rounds, from the minimal assumption of post-quantum semi-honest oblivious transfers. In the two-party scenario, our construction requires only $\omega(1)$ rounds. 
 % As the primary component driving these results, we develop the first black-box post-quantum extractable commitments from post-quantum semi-honest oblivious transfers, which are an interesting primitive in their own right. 
% \xiao{Remove the previous sentence. Replace it with Luowen's new results that cite this paper.}
These results have already found application in the oracle separation between classical-communication quantum MPC and $\mathbf{P} = \NP$ in the recent work of Kretschmer, Qian, and Tal [STOC'25].

\hspace{1em} 
As for $\epsilon$-simulation, Chia, Chung, Liang, and Yamakawa [CRYPTO'22] resolved the issue for the two-party setting, leaving the general multi-party setting as an open question. We complete the picture by presenting the first black-box and constant-round construction in the multi-party setting. Our construction can be instantiated using various standard post-quantum primitives including lossy public-key encryption, linearly homomorphic public-key encryption, or dense cryptosystems.





\hspace{1em} 
En route, we obtain a black-box and constant-round post-quantum commitment that achieves a weaker version of the standard 1-many non-malleability, from the minimal assumption of post-quantum one-way functions. Besides its utility in our post-quantum MPC construction, this commitment scheme also reduces the assumption used in the lower bound of quantum parallel repetition recently established by Bostanci, Qian, Spooner, and Yuen [STOC'24]. We anticipate that it will find more applications in the future.

\keywords{Multi-Party Computation \and Post-Quantum \and Non-Malleability \and Black-Box \and Round Complexity}
\end{abstract}


% Plaintext abstract

% We study the round-complexity of secure multiparty computation in the post-quantum regime where honest parties and communication channels are classical but the adversary can be a quantum machine. Our focus is on the $\mathit{fully}$ black-box setting where both the construction as well as the security reduction are black-box in nature. In this context, Chia, Chung, Liu, and Yamakawa [FOCS'22] demonstrated the infeasibility of achieving standard simulation-based security within constant rounds, unless $\mathbf{NP} \subseteq \mathbf{BQP}$. This outcome leaves crucial feasibility questions unresolved. Specifically, it remains unknown whether black-box constructions are achievable within polynomial rounds; additionally, the existence of constant-round constructions with respect to $\epsilon$-simulation, a relaxed yet useful alternative to the standard simulation notion, remains unestablished. Answers to these questions are pivotal to our understanding of the nature of black-box post-quantum secure computation.

% This work provides positive answers to the aforementioned questions. We introduce the first black-box construction for post-quantum multi-party computation in polynomial rounds, from the minimal assumption of post-quantum semi-honest oblivious transfers. In the two-party scenario, our construction requires only $\omega(1)$ rounds (i.e., any super-constant number of rounds suffices). As the primary component driving these results, we develop the first black-box post-quantum extractable commitments from post-quantum semi-honest oblivious transfers, which are an interesting primitive in their own right.

% As for $\epsilon$-simulation, Chia, Chung, Liang, and Yamakawa [CRYPTO'22] resolved the issue for the two-party setting, leaving the general multi-party setting as an open question. We complete the picture by presenting the first black-box and constant-round construction in the multi-party setting. Our construction can be instantiated using various standard post-quantum primitives including lossy public-key encryption, linearly homomorphic public-key encryption, or dense cryptosystems.

% En route, we obtain a black-box and constant-round post-quantum commitment that achieves a weaker version of the standard 1-many non-malleability, from the minimal assumption of post-quantum one-way functions. Besides its utility in our post-quantum multi-party computation, this commitment scheme also reduces the assumption used in the lower bound of quantum parallel repetition recently established by Bostanci, Qian, Spooner, and Yuen [STOC'24]. We anticipate that it will find more applications in the future.
