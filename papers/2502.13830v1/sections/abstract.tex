%!TEX root = ../main.tex
\begin{abstract}
We present the first {\em constant-round} construction of post-quantum multi-party computation (PQ-MPC) that makes only {\em black-box} use of the underlying primitives. Our targeted security notion is {\em $\epsilon$-simulatability}, a relaxed form of standard simulation-based security that allows for an arbitrarily small noticeable simulation error $\epsilon$. Our construction can be instantiated using a variety of standard post-quantum cryptographic primitives, including lossy public-key encryption, linearly homomorphic public-key encryption, or dense cryptosystem. Notably, prior to our work, PQ-MPC with both black-box construction {\em and} constant round complexity was unattained, {\em even from stronger hardness assumptions}.

\vspace{1em}

\hspace{1em} En route, we obtain the first post-quantum non-malleable commitment that is secure {\em even when multiple instances are executed in parallel}. In this scenario, an adversary acts as the receiver in some instances (dubbed left sessions) and as the committer in the others (dubbed right sessions). Our scheme prevents such a {\em man-in-the-middle} adversary from correlating the values committed in any right session with those in any left session. This is an interesting primitive in itself, recognized in the literature as {\em many-to-many} non-malleability. Our construction is a special {\em synchronous} case of it (i.e.,  all the sessions are executed in parallel), which already suffices for most applications. This construction is both black-box and constant-round, and is based on {\em the minimal assumption} of post-quantum one-way functions.

\keywords{Multi-Party Computation \and Post-Quantum \and Non-Malleability}
\end{abstract}


% Plaintext abstract

% We study the round-complexity of secure multiparty computation in the post-quantum regime where honest parties and communication channels are classical but the adversary can be a quantum machine. Our focus is on the {\em fully} black-box setting where both the construction as well as the security reduction are black-box in nature. In this context, Chia, Chung, Liu, and Yamakawa [FOCS'22] demonstrated the infeasibility of achieving standard simulation-based security within constant rounds, unless $\NP \subseteq \BQP$. This outcome leaves crucial feasibility questions unresolved. Specifically, it remains unknown whether black-box constructions are achievable within polynomial rounds; additionally, the existence of constant-round constructions with respect to {\em $\epsilon$-simulation}, a relaxed yet useful alternative to the standard simulation notion, remains unestablished. Answers to these questions are pivotal to our understanding of the nature of black-box post-quantum secure computation.

% This work provides positive answers to the aforementioned questions. We introduce the first black-box construction for post-quantum multi-party computation in polynomial rounds, from the minimal assumption of post-quantum semi-honest oblivious transfers. In the two-party scenario, our construction requires only $\omega(1)$ rounds (i.e., any super-constant number of rounds suffices). As the primary component driving these results, we develop the first black-box post-quantum extractable commitments from post-quantum semi-honest oblivious transfers, which are an interesting primitive in their own right.

% As for $\epsilon$-simulation, Chia, Chung, Liang, and Yamakawa [CRYPTO'22] resolved the issue for the two-party setting, leaving the general multi-party setting as an open question. We complete the picture by presenting the first black-box and constant-round construction in the multi-party setting. Our construction can be instantiated using various standard post-quantum primitives including lossy public-key encryption, linearly homomorphic public-key encryption, or dense cryptosystems.

% En route, we obtain a black-box and constant-round post-quantum commitment that achieves a weaker version of the standard 1-many non-malleability, from the minimal assumption of post-quantum one-way functions. Besides its utility in our post-quantum multi-party computation, this commitment scheme also reduces the assumption used in the lower bound of quantum parallel repetition recently established by Bostanci, Qian, Spooner, and Yuen [STOC'24]. We anticipate that it will find more applications in the future.

