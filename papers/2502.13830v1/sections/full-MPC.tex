%!TEX root = ../main.tex

\section{Post-Quantum Black-Box MPC with Full Simulation}
\label{sec:full-MPC}

\subsection{Black-Box PQ-2PC with Full Simulation}
\label{sec:full-MPC:2PC}
In this part, we prove the following theorem.
\begin{theorem}\label{thm:2pc:main}
Assuming the existence of a constant-round, semi-honest post-quantum OT, there exists a black-box, $\omega(1)$-round construction of post-quantum 2PC. 
\end{theorem}

To prove \Cref{thm:2pc:main}, we follow the paradigm established in earlier works, in particular \cite{EC:GLSV21,C:CCLY22}. This involves two steps.

\para{Step-1:} In \cite{C:CCLY22}, the authors first define an ideal functionality $\Func^t_\textsc{so-com}$ for ``selective-opening secure'' commitments, which is shown in \Cref{figure:functionality:so-com}. More descriptively, this is an idealization of a commitment that offers {\em selective opening} security in a {\em bounded-parallel} execution. That is, it can be used by a committer to commit to an a-priori bounded number, say a polynomial $t(\secpar)$, of strings within a single invocation; later, the receiver may specify an arbitrary subset $I \subset [t]$ of positions, and the committer must decommit to the $i$-th commitment it made, for each $i \in I$.

The intuitive benefit in having access to such a construct arises from the fact that it allows for implementations of {\em cut-and-choose} protocols which naturally involve committing to several instances of certain data and then later opening a receiver-chosen subset of these committed instances. The techniques we use to get 2PC will involve these techniques. 
% \rohit{This may be superfluous --- also maybe we can cite [CDMW] here.} 

% We describe this functionality formally in \Cref{figure:functionality:para-ot}. \rohit{This figure is for OT -- will change this to SO-COM.}

% \xiao{Explain this notion alittle bit. Can copy from Page 53 in \cite{C:CCLY22}.}

\begin{FigureBox}[label={figure:functionality:so-com}]{The Ideal Functionality \textnormal{$\Func^t_\textsc{so-com}$} \cite{EC:GLSV21,C:CCLY22}}
% \para{Parties:} a committer $C$ and a receiver $R$.
\para{Commit Stage:} $\Func^t_\textsc{so-com}$ receives from the committer $C$ a query $\big(\algo{Commit}, sid, (m_1, \ldots, m_t)\big)$. $\Func^t_\textsc{so-com}$ records $\big(sid, (m_1, \ldots, m_t)\big)$ and sends $(\algo{Receipt}, sid)$ to the receiver $R$. $\Func^t_\textsc{so-com}$ ignores further $\algo{Commit}$ messages with the same $sid$.

\para{Decommit Stage:} $\Func^t_\textsc{so-com}$ receives from $R$ a query $(\algo{Reveal}, sid, I)$, where $I$ is a subset of $[t]$. If no $\big(sid, (m_1, \ldots, m_t)\big)$ has been recorded, $\Func^t_\textsc{so-com}$ does nothing; otherwise, it sends to $R$ the message $\big(\algo{Open}, sid, \Set{m_i}_{i\in I} \big)$. 
\end{FigureBox}

% \begin{FigureBox}[label={figure:functionality:para-ot}]{The Ideal Functionality \textnormal{$\Func^t_\textsc{ot}$} \cite{EC:GLSV21,C:CCLY22}}
% \para{Sender's Message:}
% $\Func^t_\textsc{ot}$ receives from the sender $S$ a query $\big(\algo{Send}, sid, \Set{(x^i_0, x^i_1)}_{i \in [t]}\big)$. $\Func^t_\textsc{ot}$ records $\big(sid, \Set{(x^i_0, x^i_1)}_{i \in [t]}\big)$. $\Func^t_\textsc{ot}$ ignores further $\algo{Send}$ messages with the same $sid$.

% \para{Receiver's Message:}
% $\Func^t_\textsc{ot}$ receives from the receiver $R$ a query $\big(\algo{Receive}, sid, c \in \bits^t\big)$. If no $\big(sid, \Set{(x^i_0, x^i_1)}_{i \in [t]}\big)$ has been recorded, $\Func^t_\textsc{ot}$ does nothing; otherwise, it sends to $R$ the message $\big(\algo{Open}, sid, \Set{x^i_{c_i}}_{i\in [t]} \big)$, where $c_i$ is the $i$-th bit of $c$.
% \end{FigureBox}


\para{Step 2:} Then, it is shown in \cite[Section 7.4]{C:CCLY22} that $\Func^t_\textsc{so-com}$ is indeed black-box 2PC-complete. Namely, \cite[Section 7.4]{C:CCLY22} shows that given a protocol $\pi$ that securely implements $\Func^t_\textsc{so-com}$ against QPT adversaries, one can construct a {\em general-purpose} 2PC protocol (i.e, computing any efficient 2-party functionality) that is secure against QPT adversaries. Moreover, the 2PC construction makes only {\em black-box} use of $\pi$ and involves only a constant multiplicative blow up in the number
of rounds (as compared to $\pi$).  

We must keep in mind the following caveat: the protocol $\pi$ as described in \cite[Section 7.4]{C:CCLY22} in fact implements $\Func^t_\textsc{so-com}$ w.r.t.\ $\epsilon$-simulation. Hence the final 2PC they obtained is also w.r.t.\ $\epsilon$-simulation. It is straightforward however to see that the same proof works w.r.t.\ standard negligibly-close simulation as well. Namely, if one starts with a $\pi$ that implements $\Func^t_\textsc{so-com}$ w.r.t.\ the standard notion of negligible-close simulation, then the resulting 2PC protocol will also be secure w.r.t.\ the standard notion of negligible-close simulation.


\para{Implementing \textnormal{$\Func^t_\textsc{so-com}$}.} From the above discussion we can see that to prove \Cref{thm:2pc:main}, it suffices to construct a $\omega(1)$-round, black-box, post-quantum protocol implementing the $\Func^t_\textsc{so-com}$ functionality. For that, we will make use of the $\omega(1)$-round ExtCom-and-Prove protocol described in \Cref{protocol:ExtBCom}. Note that it is okay to make use of this ExtCom-and-Prove protocol because this protocol makes only black-box use of a semi-honest post-quantum OT protocol, which is indeed the minimal assumption for our current goal of 2PC.

We can then conclude the proof of \Cref{thm:2pc:main} using the following lemma.

\begin{lemma}[\text{\cite[Lemma 26]{C:CCLY22}}]
\label{lem:ExtCin-and-Prove:to:SO-Com}
Assume the existence of Post-Quantum ExtCom-and-Prove (as per \Cref{def:com-n-prove}). Then, for any polynomial $t(\secpar)$, there exists a post-quantum protocol implementing $\Func^t_\textsc{so-com}$. Moreover, this construction makes only black-box use of the ExtCom-and-Prove protocol and incurs only a constant blow up in the number of rounds.
\end{lemma}
\begin{proof}
This proof is essentially identical to the proof of \cite[Lemma 26]{C:CCLY22}, relying on the extractable commit-and-prove protocol given in \Cref{protocol:ExtCom-n-Prove}. The idea is simple: the committer uses the Commit Stage of the Extcom-and-Prove protocol to commit to different messages $(m_1,\dots,m_t)$ of its choice. Next, when required to decommit to a certain subset $I \subset [t]$ of messages, the committer reveals these messages $\Set{m_i}_{i \in I}$ to the receiver and then uses to Prove Stage to prove that the revealed messages are indeed the committed ones for the appropriate positions. Note that this protocol then is purely black-box and only adds a small constant number of rounds of communication (relating to sending the revealed subset and the decommitment information) over the underlying Extcom-and-Prove protocol.   

Security against a cheating committer is obtained via the soundness of \Cref{protocol:ExtCom-n-Prove}, and that against a cheating reciever can be seen from the zero-knowledge of the same underlying protocol. 
The crucial (and in fact only) difference is that \cite[Lemma 26]{C:CCLY22} uses an underlying ExtCom-and-Prove protocol that offers $\epsilon$-simulation. This is why they only manage to obtain a protocol implementing $\Func^t_\textsc{so-com}$ w.r.t.\ $\epsilon$-simulation. In contrast, \Cref{protocol:ExtCom-n-Prove} does achieve the standard notion of negligibly close simulation. It is then easy to verify that our protocol for $\Func^t_\textsc{so-com}$ achieves the standard notion of full simulation as well, using the same proof.
\end{proof}





% \begin{xiaoenv}{Road Map to fully simulatable Black-Box PQ-2PC}
% Road Map to fully simulatable PQ-BBMPC in $\omega(1)$ rounds.	(need to add a footnote for this use of $\omega(1)$. Typically, it is used as a lower bound. Here, we simply mean that any super-constant number of rounds suffices):
% \begin{enumerate}
% \item \label{item:para-ExtCom-from-OT}
% First, notice that in \Cref{sec:ExtCom-from-OT}, we have a $\omega(\log\secpar)$-round weakly-parallel PQ-ExtCom with full simulation, constructed in black-box from any constant-round malicious-sender OT (question: is this the minimal assumption? can we get it from semi-honest OTs)?
% \item \label{item:ENMC-from-OT}
% Use the above fully-simulatable PQ-ExtCom to replace the $\ExtCom$ (i.e., \Cref{prot:bbnmc:extcom} of \Cref{protocol:BB-NMCom}) in our 1-many PQ-NMCom yields a black-box 1-many PQ-NMCom in  $\omega(\log\secpar)$ rounds. Moreover, this PQ-NMCom would be (weakly?) parallel extractable with full simulation.
% \item \label{item:full-para-OT}
% In \Cref{prot:mal-OT}, use \Cref{item:para-ExtCom-from-OT} to replace the ExtCom in \Cref{prot:parOT:cointoss:rec-extcom}. Also, use \Cref{item:ENMC-from-OT} to replace the ENMC in \Cref{prot:parOT:cut-n-choose:enmc}. This yields a $\omega(\log\secpar)$-round multi-party parallel-OT with {\em full simulation}.
% \item
% Plug the multi-party parallel-OT with {\em full simulation} (i.e., \Cref{item:full-para-OT}) into the IPS compiler. This gives the first $\omega(\log\secpar)$-round black-box PQ-MPC (with full simulation).
% \end{enumerate}
% \end{xiaoenv}

% \xiao{remark taht such a black-box MPC is not known previously, even if the round complexity can be a polynomial in the number of parties!}


\subsection{Black-Box PQ-MPC with Full Simulation}
\label{sec:full-MPC:MPC}
Here we turn to the problem of obtaining a {\em fully simulatable} black-box post-quantum MPC protocol. More precisely, we show the following theorem.

\begin{theorem}\label{thm:mpc:main}
Assuming the existence of a semi-honest post-quantum OT, there exists a black-box construction of post-quantum MPC in polynomial rounds. 
\end{theorem}

This theorem follows directly from \Cref{thm:2pc:main} and \cite{C:IshPraSah08}. To start, we observe that \Cref{thm:2pc:main} provides a black-box construction of post-quantum maliciously secure and fully simulatable OT --- via the constructed 2PC protocol (recall that the latter can be made to implement {\em any} 2 party functionality). We have further seen in \Cref{sec:MPC} that the black-box compiler given in \cite{C:IshPraSah08} from OT to MPC works in the post-quantum setting as is.  

There is however a caveat: recall that the original \cite{C:IshPraSah08} result is in the OT hybrid model in the UC setting, where the OT primitive is modeled as an ideal UC functionality. Such modeling indeed allows parallel OT calls (this has been observed and discussed in \cite[Section 7]{C:CCLY22} and in \Cref{sec:MPC}). But the OT protocol we obtain from \Cref{thm:2pc:main} is only secure in the {\em standalone} setting. 

Fortunately, this does not become a problem for our application. Recall that the IPS compiler involves carrying out a certain polynomial number of OT calls or executions at the start of the protocol, which are carried out in parallel as observed above. For our purposes, we simply make required number of OT calls in sequence instead of in parallel, which adds to the round complexity of our protocol but preserves the desired order asymptotics --- it is easy to check that our overall MPC protocol still takes only polynomial rounds in $n$ (and thus also in $\secpar$).