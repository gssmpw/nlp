%!TEX root = ../main.tex

\section{Introduction}

{\em Secure multi-party computation} (MPC) allows two or more mutually distrustful parties to compute any functionality without compromising the privacy of their inputs \cite{FOCS:Yao86,STOC:GolMicWig87}. Its remarkable versatility positions it at the core of cryptographic research.

\para{Black-Box Constructions.} An important aspect impacting the efficiency of MPC is how protocols utilize their building blocks. A cryptographic construction is {\em black-box} if it does not refer to the code of any cryptographic primitive it employs, relying solely on their input/output behavior. Such constructions typically outperform {\em non-black-box} alternatives by avoiding resource-intensive operations like running {\em zero-knowledge} protocols or {\em circuit-garbling} schemes with other cryptographic primitives. Furthermore, they maintain their validity even when the foundational components are based on {\em physical} objects, such as noisy channels or tamper-proof hardware \cite{wyner1975wire,FOCS:CreKil88,TCC:GLMMR04}. Indeed, a significant focus in cryptographic research has been on developing black-box constructions for MPC and related primitives, e.g., \cite{STOC:Kilian88,C:DamIsh05,STOC:IKLP06,STOC:IKOS07,TCC:Haitner08,C:IshPraSah08,TCC:PasWee09,FOCS:Wee10,STOC:Goyal11,FOCS:GLOV12,C:LinPas12,C:Kiyoshima14,STOC:GOSV14,TCC:GGMP16,C:HazVen16,EC:GarKiyPan18,TCC:KhuOstSri18,ICALP:ChaLiaPan20,SCN:GLPV20,EC:GKLW21,C:LiaPan21,C:ChiChuYam21,C:CCLY22}.  
% \cite{STOC:Kilian88,STOC:IKLP06,STOC:IKOS07,TCC:Haitner08,C:IshPraSah08,FOCS:Wee10,STOC:Goyal11} etc.

%\cite{STOC:Kilian88,C:DamIsh05,STOC:IKLP06,STOC:IKOS07,TCC:Haitner08,C:IshPraSah08,TCC:PasWee09,FOCS:Wee10,STOC:Goyal11,FOCS:GLOV12,C:LinPas12,C:Kiyoshima14,STOC:GOSV14,TCC:GGMP16,C:HazVen16,EC:GarKiyPan18,TCC:KhuOstSri18,ICALP:ChaLiaPan20,SCN:GLPV20,EC:GKLW21,C:LiaPan21,C:ChiChuYam21,C:CCLY22}.  

\para{Black-Box Reductions.} A closely related `black-box' notion pertains to security proofs in modern cryptography. The foundation of cryptographic schemes often relies on {\em reductions} to problems believed to be inherently difficult. To establish confidence in a proposed scheme, we can reason as follows: If an adversary can compromise the scheme, it implies an ability to exploit a hardness assumption. Given the presumed difficulty of the assumption, such an outcome would be highly surprising. This positions us in a win-win scenario. A security proof (or reduction) is considered black-box if it relies solely on the I/O behavior of the adversary. Such reductions are a natural choice and encompass the majority of security proofs in the literature, encapsulating the `essence' of cryptographic tasks. Since the seminal work of \cite{STOC:ImpRud89}, black-box reductions have systematically contributed to understanding the relationships between different cryptographic tasks, identifying fundamental primitives, and recognizing tasks requiring significantly stronger assumptions and techniques. It is worth noting that this concept is closely linked to black-box {\em constructions} because black-box constructions typically go hand-in-hand with black-box security proofs.\footnote{One notable exception is the work of \cite{STOC:GOSV14}. But it relies on the \cite{FOCS:Barak01} technique, for which a quantum analog remains an unresolved question.} Henceforth, we use the term `black-box' to refer to both the construction and the security proof. 


In this work, we are interested in the fundamental problem of the round complexity of black-box MPC protocols.

\para{Round Complexity of Black-Box MPC.} This problem is well-studied in the classical setting, where constant-round, black-box constructions has been achieved \cite{STOC:Goyal11}, together with a black-box security proof. This construction is based on a weaker form of {\em malicious-sender} oblivious transfers (OTs), which demand security against malicious senders but only {\em semi-honest} receivers. Moreover, the associated simulator for security proof is required to be `straight-line' (i.e., not performing any rewindings). Such OT protocols are available under several standard assumptions, including certifiable enhanced trapdoor permutations, dense cryptosystems, linearly homomorphic public-key encryption, or lossy public-key encryption.

However, the landscape changes dramatically when we consider {\em post-quantum} MPC (PQ-MPC), where the honest parties and their communication channels are entirely classical, but the adversary is a quantum machine. Recent works \cite{EC:ABGKM21,FOCS:LPY23} obtained constant-round PQ-MPC via heavy use of non-black-box construction {\em and} reduction techniques. In contrast, little is known in the black-box regime. 

Indeed, the recent lower bound \cite{chia2022impossibility} has demonstrated the impossibility of achieving constant-round PQ-MPC with black-box security proof (or unless $\NP \subseteq \BQP$). Faced with this impossibility result, an interesting question is to investigate the round complexity of black-box PQ-MPC {\em with $\epsilon$-simulation}, which is a relaxed form of standard simulation-based security that allows for an arbitrarily small noticeable simulation error $\epsilon$. This is an interesting notion because it implies indistinguishability. For example, $\epsilon$-zero-knowledge protocols are witness indistinguishable \cite{STOC:FeiSha90} and $\epsilon$-simulatable MPCs satisfy input-indistinguishable security \cite{FOCS:MicPasRos06}. In this regard, the recent work \cite{C:CCLY22} achieves constant-round using black-box techniques {\em in the two party setting}. However, obtaining similar results in the multi-part setting remains an unsolved challenge, {\em even with more robust hardness assumptions than those in the classical setting}. 
\begin{quote}
{\bf Question:} Does there exist constant-round black-box $\epsilon$-simulatable PQ-MPC?
\end{quote}
The answer to this question is of fundamental importance to our understanding of MPC in the quantum era.




% In the classical realm, constant-round black-box MPC has been achieved \cite{STOC:Goyal11}, assuming only a weaker form of {\em malicious-sender} oblivious transfers (OTs), which demand security against malicious senders but only {\em semi-honest} receivers. Moreover, the associated simulator for security proof is required to be `straight-line' (i.e., not performing any rewindings). Such OT protocols are available under several standard assumptions, including certifiable enhanced trapdoor permutations, dense cryptosystems, linearly homomorphic public-key encryption, or lossy public-key encryption.

% However, the landscape changes dramatically when we consider {\em post-quantum} MPC (PQ-MPC), where the honest parties and their communication channels are entirely classical, but the adversary is a quantum machine. Achieving constant-round black-box PQ-MPC, {\em even with more robust hardness assumptions than those in the classical setting}, remains an unsolved challenge.


% The recent lower bound on post-quantum zero-knowledge arguments \cite{chia2022impossibility} has demonstrated the impossibility of achieving constant-round PQ-MPC, unless the security proof utilizes non-black-box simulation techniques (or unless $\NP \subseteq \BQP$). This suggests that constant-round and black-box construction may necessitate a relaxation of the model.\footnote{It is worth noting that while the \cite{chia2022impossibility} impossibility pertains to black-box {\em security proof}, rather than the construction itself, known black-box constructions typically rely on black-box security proofs. One notable exception is the work of \cite{STOC:GOSV14}. But it relies on the \cite{FOCS:Barak01} technique, for which a quantum analog remains an unresolved question.}

% The closest work in this regard is \cite{C:CCLY22}, which achieves an {\em $\epsilon$-simulatable} protocol, specifically {\em in the two party setting}. The $\epsilon$-simulatability is a relaxed form of standard simulation-based security that allows for an arbitrarily small noticeable simulation error $\epsilon$. This notion is not only necessary in light of \cite{chia2022impossibility}, but also valuable as a security concept in its own right (e.g., $\epsilon$-simulatable MPC implies input-indistinguishable MPC \cite{FOCS:MicPasRos06}). However, the more general {\em multi-party} scenario remains unclear. Indeed, \cite{C:CCLY22} left it as an open question:
% \begin{quote}
% {\bf Question:} Does there exist constant-round black-box $\epsilon$-simulatable PQ-MPC?
% \end{quote}




\subsection{Our Results}

We answer the above {\bf Question} affirmatively, relying on the same (more accurately, the post-quantum analog of) hardness assumptions as for the state-of-the-art of classical MPC:
\begin{theorem}\label{thm:informal:main}
There exists a black-box and constant-round construction of $\epsilon$-simulatable PQ-MPC from a variety of standard post-quantum cryptographic primitives, such as lossy public-key encryption, linearly homomorphic public-key encryption, or dense cryptosystem.\footnote{We did not mention post-quantum (enhanced) trapdoor permutations as they are not known from standard quantum hardness assumptions yet. But as long as they exist, they can be included in \Cref{thm:informal:main} as well.}
\end{theorem}

Our approach to \Cref{thm:informal:main} follows a pipeline established for classic constant-round black-box MPC, which has evolved through a series of prior work \cite{C:IshPraSah08,TCC:PasWee09,FOCS:Wee10,STOC:Goyal11,FOCS:GLOV12}. In broad terms, we demonstrate that if the building components used in this pipeline are properly instantiated using their post-quantum equivalents, the outcome can be extended to the post-quantum realm. Further insights into this process are elaborated upon in \Cref{sec:tech-oeverview:reduction-to-NMC}. For now, it is worth noting that a critical step in this framework is the development of a black-box {\em 1-many non-malleable} commitment scheme in constant rounds. This constitutes the primary technical challenge in the post-quantum setting.

\para{Post-Quantum 1-Many Non-Malleability.} Non-malleable commitments \cite{STOC:DolDwoNao91} are commitments secure in the so-called {\em man-in-the-middle} (MIM) setting: An adversary $\mcal{M}$ plays the role of a receiver in one instance of a commitment (referred to as the {\em left session}), while simultaneously acting as a committer in another session (referred to as the {\em right session}). During the execution, $\mcal{M}$ can potentially make the value committed in the right session depend on that in the left session, in a malicious manner that is to her advantage. Notice that this is not breaking the hiding property of the commitment scheme, as $\mcal{M}$ may be able to conduct the above attack without explicitly learning the value committed in the left session. Furthermore, a commitment is said to be {\em 1-many} non-malleable if it is secure in the MIM setting with one left session but {\em polynomially many} right sessions, i.e., the adversary $\mcal{M}$ cannot make the {\em joint distribution} of the values committed across all right sessions depend on the one committed in the left session.


In the classical setting, the existence of black-box constant-round 1-many non-malleable commitments is established under the minimal assumption of one-way functions \cite{STOC:Goyal11,FOCS:GLOV12}, which plays a pivotal role in enabling black-box constant-round MPC. However, in the post-quantum context, achieving non-malleability (even in the 1-1 MIM setting) with constant rounds proves to be an exceptionally challenging task. A recent breakthrough by \cite{FOCS:LPY23} succeeded in obtaining a post-quantum 1-1 non-malleable commitment in constant rounds. Yet, their construction relies significantly on {\em non-black-box} usage of post-quantum one-way functions, and it remains uncertain if their scheme can maintain non-malleability in the more demanding 1-many scenario.


In this work, we obtain the first black-box and constant-round post-quantum non-malleable commitments in the 1-many setting, from the minimal assumption of post-quantum one-way functions. Our construction supports a polynomial {\em tag space}\footnote{Each execution of non-malleable commitments requires a unique tag; otherwise, it is impossible to protect against MIM attacks (see \cite{STOC:Pass04} for related discussions).}, and is non-malleable only in the {\em synchronous} setting, meaning that all the messages of the left session and polynomially many right sessions are send in parallel. Such a scheme already suffices for post-quantum MPC, and we believe it will find more applications in the future.

% Moreover, it finds further application in a recent work for the impossibility of parallel repetition of post-quantum arguments (see \xiao{Section XXX} for details) \xiao{This is a new paper by Luowen Qian at el. Luowen told me that they will post the paper on arxiv soon. I'll add the citation when it's on-line}. We believe it will find more applications in the future.

\begin{theorem}\label{thm:informal:1-many-NMC}
Assuming the existence of post-quantum one-way functions, there exists a black-box and constant-round construction of post-quantum 1-many non-malleable commitments in the synchronous setting, supporting a polynomial tag space.
\end{theorem}

It is known that 1-many non-malleability implies the seeming more demanding {\em many-many} non-malleability, using a standard hybrid argument. This reduction holds even in the post-quantum setting (see e.g., \cite[Lemma 7.3]{EC:ABGKM21}). This yields the following corollary of \Cref{thm:informal:1-many-NMC}. But we remark that the 1-many notion in \Cref{thm:informal:1-many-NMC} already suffices for all the applications in this paper.

\begin{corollary}
Assuming the existence of post-quantum one-way functions, there exists a black-box and constant-round construction of post-quantum many-many non-malleable commitments in the synchronous setting, supporting a polynomial tag space.
\end{corollary}


