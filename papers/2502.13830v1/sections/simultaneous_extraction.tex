%!TEX root = ../main.tex

%%%%%%%%%%%%%%%%%%%%%%%%%%%%%%%%%%%
% Takashi's Macro
%%%%%%%%%%%%%%%%%%%%%%%%%%%%%%%%%%%
\newcommand{\projimp}{\mathsf{ProjImp}}
\newcommand{\API}{\mathsf{API}}
\newcommand{\shiftdis}[1]{\Delta_{\mathsf{Shift}}^{#1}}
\newcommand{\repair}{\mathsf{Repair}}
\newcommand{\repairA}{\mcal{A}\text{-}\mathsf{Repair}}

\section{Simultaneous Extraction Lemma}


So far, we have obtained a post-quantum non-malleable commitment in the 1-1 MIM setting. Recall that our final goal is to obtain a construction secure in the more demanding 1-many MIM setting. Jupping ahead, we will manage to show (in \Cref{sec:BB-NMCom:one-many}) that the same protocol \Cref{protocol:BB-NMCom} (more accurately, its two-sided version \Cref{protocol:BB-NMCom:two-sided}), without any modifications, is indeed already secure in the 1-many MIM setting. However, this or course requires a different security proof. 

In this section, our develop a `simultaneous extraction lemma' (\Cref{lem:Simultaneous-SimExt}). This lemma will play a crucial role later when we upgrade the security proof for \Cref{protocol:BB-NMCom} (or \Cref{protocol:BB-NMCom:two-sided}) to the 1-many setting in \Cref{sec:BB-NMCom:one-many}.




\subsection{Lemma Statement}

\begin{lemma}[Simultaneous Extraction Lemma]
\label{lem:Simultaneous-SimExt}
Let $\mcal{V}$ be a QPT algorithm that takes the security parameter $1^\secpar$, an error parameter $1^{\gamma^{-1}}$, 
a quantum state $\rho$, {and a classical string $z$}  
as input,  and outputs  $d\in \{\top,\bot\}$. %and a quantum state $\rho_\out$. 
\takashi{
This is essentially the same as $\mcal{G}$ as in the simulatable extraction lemma. 
I changed the name and syntax since the quantum output $\rho_\out$ is redundant in this lemma.
But it also makse sense to use the same $\mcal{G}$ for consistency.
}

Suppose that for $i\in [n]$, 
there exists a QPT algorithm $\mcal{K}_i$ (referred to as the extractor) that takes as input the security parameter $1^\secpar$, two error parameters $1^{\gamma^{-1}}$ and $1^{\zeta^{-1}}$, 
a quantum state $\rho$ %, {and a classical string $z$}, 
and outputs $s\in \bit^{\poly(\secpar)}\cup \{\bot\}$   
satisfying the following w.r.t.\ some sequence of classical strings {$\{s^*_{z,i}\}_{z\in \bit^*,i\in[n]}$.} \takashi{This notation may be slightly weird since $n$ may be a function of $\secpar$.} 
\begin{itemize}
 \item    \label{item:simultaneous_s_star_or_bot} 
    {\bf Assumption 1:} For any $\secpar$,  $\rho_\secpar$,  $z_\secpar$, $i\in [n]$, and any noticeable functions $\gamma(\secpar)$ and $\zeta(\secpar)$, it holds that  

$$\Pr[s \notin \Set{s^*_{z_\secpar,i}, \bot}~:~s \la \mcal{K}_i(1^\secpar,1^{\gamma^{-1}}, 1^{\zeta^{-1}}, \rho_\secpar,{z_\secpar})]\le \zeta(\secpar) +\negl(\secpar).$$


    \item \label{item:simultaneous_gamma_delta}
    {\bf Assumption 2:} For any noticeable function $\gamma(\secpar)$, there exists a noticeable function  $\delta(\secpar)$, 
 which is efficiently computable from $\gamma(\secpar)$, so that the following requirement is satisfied: 
 For any noticeable function $\zeta(\secpar)$
 and any sequence $\{\rho_\secpar,{z_\secpar}\}_{\secpar\in\mathbb{N}}$ of polynomial-size quantum states and classical strings and $i\in[n]$, %\footnote{Similarly to \Cref{footnote:sequence_s}, we consider a sequence $\{\rho_\secpar\}_{\secpar\in \mathbb{N}}$ and denote by $\rho$ to mean $\rho_\secpar$.}    
 if 
$$
\Pr[d=\top ~:~ d \leftarrow \mcal{V}(1^\secpar,1^{\gamma^{-1}},\rho_\secpar,{z_\secpar})]\geq  \gamma(\secpar), 
$$  
then 
$$
\Pr[s =s^*_{{z_\secpar,i}}~:~ s\la \mcal{K}_i(1^\secpar,1^{\gamma^{-1}}, 1^{\zeta^{-1}}, \rho_\secpar,{z_\secpar})]\geq   \delta(\secpar)-\zeta(\secpar)-\negl(\secpar).
$$
\end{itemize}
Then, there exists a QPT algorithm $\mcal{K}$ that satisfies the following:  
\begin{enumerate}
 \item  \label[Property]{item:simultaneous_conclusion_s_star_or_bot}
    For any $\secpar$,  $\rho_\secpar$,  $z_\secpar$,  and any noticeable functions $\gamma(\secpar)$ and $\zeta(\secpar)$, it holds that  

$$\Pr[\bar{s} \notin \Set{s^*_{z_\secpar,1}||s^*_{z_\secpar,2}...||s^*_{z_\secpar,n}, \bot}~:~\bar{s} \la \mcal{K}(1^\secpar,1^{\gamma^{-1}}, 1^{\zeta^{-1}}, \rho_\secpar,{z_\secpar})]\le \zeta(\secpar) +\negl(\secpar).$$

    \item \label[Property]{item:simultaneous_conclusion_gamma_delta}
  For any noticeable function $\gamma(\secpar)$,  there exists a noticeable function  $\delta'(\secpar)$,  
 which is efficiently computable from $\gamma(\secpar)$, so that 
 the following requirement is satisfied: For 
 any noticeable funntion $\zeta(\secpar)$ and 
 any sequence $\{\rho_\secpar,{z_\secpar}\}_{\secpar\in\mathbb{N}}$ of polynomial-size quantum states and classical strings, %\footnote{Similarly to \Cref{footnote:sequence_s}, we consider a sequence $\{\rho_\secpar\}_{\secpar\in \mathbb{N}}$ and denote by $\rho$ to mean $\rho_\secpar$.}    
 if 
$$
\Pr[d=\top ~:~ d \leftarrow \mcal{V}(1^\secpar,1^{\gamma^{-1}},\rho_\secpar,{z_\secpar})]\geq  8\gamma(\secpar), 
$$  
then 
$$
\Pr[\bar{s} =s^*_{z_\secpar,1}||s^*_{z_\secpar,2}...||s^*_{z_\secpar,n}~:~ \bar{s}\la \mcal{K}(1^\secpar,1^{\gamma^{-1}}, 1^{\zeta^{-1}}, \rho_\secpar,{z_\secpar})]\geq   \delta'(\secpar)-\zeta(\secpar)-\negl(\secpar).
$$
\takashi{Indeed, $8\gamma(\secpar)$ and $\delta'(\secpar)$ can be made arbitrarily close to $\gamma(\secpar)$.}
\end{enumerate}
\end{lemma}

\subsection{Preparation}
We take several tools and lemmas from \cite{TCC:Zhandry20,FOCS:CMSZ21} and give slight extensions of them.   
\begin{definition}[Projective Implementation \cite{TCC:Zhandry20}]
Let $\mcal{M}=(M_0,M_1)$ be a binary outcome POVM. Let $\mcal{E}=\{E_p\}_{p\in S}$ be a projective measurement indexed by $p\in S$ for some finite subset $S$ of $[0,1]$.\footnote{In \cite{TCC:Zhandry20}, $\mcal{E}$ is labeled by a distribution $D$. The definition here is identical to theirs if we interpret $p$ as a distribution that takes $1$ with probability $p$ and otherwise takes $0$.} Consider the following experiment:
\begin{enumerate}
\item Apply the measurement $\mcal{E}$ to obtain $p\in S$. 
\item Output $1$ with probability $p$ and output $0$ with probability $1-p$.  
\end{enumerate}
We say that $\mcal{E}$ is a projective implementation of $\mcal{M}$ if for any initial state, the above experiment yields the identical distribution to that obtained by applying the POVM $\mcal{M}$.  
\end{definition}

\begin{lemma}[{\cite[Lemma 3.3]{TCC:Zhandry20}}]
Any binary outcome POVM $\mcal{M}$ has a unique projective implementation. 
\end{lemma} \xiao{It appears to me that \cite[Lemma 3.3]{TCC:Zhandry20} only claims the existence for projective implementation for {\em commutative} $\mcal{M}$. Are you suggesting that the commutative condition is redundant? Or you actually mean that ``If a binary outcome POVM $\mcal{M}$ has a  projective implementation, then it must be unique''?}
\takashi{I'm relying on the fact that any binary-outcome POVM commute. This is implicitly used in \cite{TCC:Zhandry20}. For example, see the paragraph starting from "In our case,..." in page 7.} \xiao{I see. You're right!}

For a binary outcome POVM $\mcal{M}$, 
we write $\projimp(\mcal{M})$ to mean its projective implementation.

\begin{definition}[Shift Distance~\cite{TCC:Zhandry20}]\label{def:shift_distance}
For two distributions $D_0,D_1$, with cumulative density functions $f_0,f_1$, respectively, 
the shift distance with parameter $\epsilon$
is defined as
\begin{align*}
\shiftdis{\epsilon}(D_0,D_1) \coloneqq \sup_{x\in \mathbb{R}}\min_{y\in [f_1(x-\epsilon),f_1(x+\epsilon)]}|f_0(x)-y|.
\end{align*}
For two real-valued measurements $\mcal{M}$ and $\mcal{N}$ over the same quantum system, the shift distance between $\mcal{M}$ and $\mcal{N}$ with parameter $\epsilon$ is
\[
\shiftdis{\epsilon}(\mcal{M},\mcal{N}) \coloneqq  \sup_{\ket{\psi}}\shiftdis{\epsilon}(\mcal{M}(\ket{\psi}),\mcal{N}(\ket{\psi})).
\]
\end{definition}

By the definition, we can see the following: If $\shiftdis{\epsilon}(\mcal{M},\mcal{N})\le \eta$, then for any state $\ket{\psi}$ and $x \in \mathbb{R}$, 
\begin{align*}
\Pr[\mcal{M}(\ket{\psi}) \le x] & \le \Pr[\mcal{N}(\ket{\psi})\le x + \epsilon] + \eta,&& \Pr[\mcal{M}(\ket{\psi}) \ge x]  \le \Pr[\mcal{N}(\ket{\psi})\ge x - \epsilon] + \eta,\\
\Pr[\mcal{N}(\ket{\psi})\le x] & \le \Pr[\mcal{M}(\ket{\psi}\le x + \epsilon] + \eta,&& \Pr[\mcal{N}(\ket{\psi})\ge x]  \le \Pr[\mcal{M}(\ket{\psi}\ge x - \epsilon] + \eta.
\end{align*}

\begin{definition}[Almost Projective Measurements~\cite{TCC:Zhandry20}]
A real-valued measurement $\mcal{M}=(M_i)_{i\in I}$ is $(\epsilon,\eta)$-almost projective if the following is true: for any quantum state $\ket{\psi}$, apply $\mcal{M}$ twice in a row to $\ket{\psi}$, obtaining outcomes $x,y$. Then $\Pr[|x-y|\le \epsilon]\ge 1-\eta$. 
\end{definition}

\xiao{@Takashi: Just for my understanding: It seems \cite{TCC:Zhandry20} and \cite{FOCS:CMSZ21} works with the original \cite[Theorem 6.2]{TCC:Zhandry20}. However we need this variant \Cref{lem:API}. I want to learn the reason behind this difference. (I saw your remarks after this lemma. But I'd like to discuss more about it.)}

\xiao{@Takashi: Also, why doesn't \Cref{lem:API} follow directly from \cite[Lemma 4.9]{FOCS:CMSZ21}?}
\takashi{
As I explained in the first item of the remark, 
the difference is that they focus on a mixture of projective measurements.
I remark that the second point of the remark was already implicitly dealt with in \cite{FOCS:CMSZ21} (see \cite[Remark 4.8(a)]{FOCS:CMSZ21}).}

The following is a variant of \cite[Theorem 6.2]{TCC:Zhandry20}.  
\begin{lemma}\label{lem:API}
For any binary-outcome POVM $\mcal{M}=(M_0,M_1)$ and reals $0<\epsilon,\eta<1$, there is a real-valued measurement $\API_{\mcal{M}}^{\epsilon,\eta}$ that satisfies the following: 
\begin{enumerate}
\item \label{item:API_shiftdis}
$\shiftdis{\epsilon}(\API_{\mcal{M}}^{\epsilon,\eta},\projimp(\mcal{M}))\le \eta$.
\item  \label{item:API_almost_projective}
$\API_{\mcal{M}}^{\epsilon,\eta}$ is $(\epsilon,\eta)$-almost projective.
\item \label{item:API_run_time}
The run time of $\API_{\mcal{M}}^{\epsilon,\eta}$ is $T_{\mcal{M}}\cdot \poly(\epsilon^{-1},\log (\eta^{-1}))$, where $T_{\mcal{M}}$ is the run time of the POVM $\mcal{M}$. 
\end{enumerate}
\end{lemma}

There are the following two differences from the original statement of \cite[Theorem 6.2]{TCC:Zhandry20}.
\begin{enumerate}
\item We consider general binary-outcome POVM whereas they focuses on a special case called  ``mixture of projective measurements." \xiao{@Takashi: Why cannot we view our case of binary-outcome POVM  as a special case of "mixture of projective measurements"? For example, in \cite[Section 6]{TCC:Zhandry20} notation, if the index set $\mcal{I} = \Set{1}$, then doesn't it correspond to our case?}\takashi{No, because $\mathcal{M}$ may not be a projective measurement.
Note that if $\mathcal{M}$ is a projective measurement, its projective implementation is trivial since it is projective from the beginning.
}


\item We require the run time of $\API_{\mcal{M}}^{\epsilon,\eta}$ is  $T_{\mcal{M}}\cdot \poly(\epsilon^{-1},\log (\eta^{-1}))$ whereas they require it only for the \emph{expected} run time. \xiao{@Takashi: did you actually strengthen Zhandry? It seems that Zhandry only claimed strict QPT for $p$ far from 0 or 1.}
\takashi{I don't think he made any formal claim about strict QPT. If you are talking about \cite[Remark 6.4]{TCC:Zhandry20}, I believe this means that 1. if $p\in [1/4,3/4]$, achieving strict QPT is straightforward, and 2. we can reduce the general case to the case of $p\in [1/4,3/4]$ by introducing "dummy projections". The proof of \Cref{lem:API} exactly follows this idea. }
\end{enumerate}
%In \cite[Theorem 6.2]{TCC:Zhandry20}, the above theorem is proven for the case where $\mcal{M}$ is what is called a ``mixture of projective measurements". We observe that it can be extended to any binary outcome with essentially the same proof. 
For the first difference, we observe that the original proof can be easily extended to general binary-outcome POVM by using Jordan's lemma.  
The second difference can be resolved by using an idea of ``scaling down" as sketched in \cite[Remark 6.4]{TCC:Zhandry20}.

For completeness, we prove \Cref{lem:API}. We note that the proof is based on the proof of \cite[Theorem 6.2]{TCC:Zhandry20} and we often repeat very similar arguments to theirs. 

\begin{proof}[Proof of \Cref{lem:API}]
First, we construct $\tilde{\API}_{\mcal{M}}^{\epsilon,\eta}$ that satisfies the requirements if $\projimp(\mcal{M})$ is supported by $p\in [1/4,3/4]$, i.e., for any state $\rho$, it holds that 
$$\Pr[\frac{1}{4} \le p \le \frac{3}{4} ~:~p\gets \projimp(\mcal{M})(\rho)]=1.$$ 
Looking ahead, this assumption is used to make sure that $\tilde{\API}_{\mcal{M}}^{\epsilon,\eta}$ runs in strict QPT (rather than expected QPT as in \cite[Theorem 6.2]{TCC:Zhandry20}). 
At the end of the proof, we modify it to $\API_{\mcal{M}}^{\epsilon,\eta}$ that works for any binary-outcome measurement. 

Suppose that $\projimp(\mcal{M})$ is supported by $p\in [1/4,3/4]$. 
Let $\regX$ be a quantum register for states on which $\mcal{M}=(M_0,M_1)$ acts.
Let $U$ be a purification of $\mcal{M}$ on $\regX$ and an ancilla register $\regY$. That is, we define the unitary $U$ in such a way that for any state $\rho_\regX$ on $\regX$ and $b\in \bit$, we have 
\begin{align*}
    \Tr(M_b \rho_\regX)=\Tr(U^\dagger(\ket{b}\bra{b}\otimes I) U (\rho_\regX \otimes \ket{0}\bra{0}_{\regY}))
\end{align*}
where $\ket{b}\bra{b}\otimes I$ means the operator that projects the first qubit of $\regX$ onto $\ket{b}$. 
We define two projectors $\Pi_0$ and $\Pi_1$ over $\regX$ and $\regY$ as:
\begin{align*}
\Pi_0 \coloneqq I_\regX \otimes \ket{0^n}\bra{0^n}_\regY,~~~\Pi_1 \coloneqq U^\dagger(\ket{1}\bra{1}\otimes I) U
\end{align*}
where $n$ is the number of qubits in $\regY$. 
By applying Jordan's lemma to $\Pi_0$ and $\Pi_1$, we can see that there is an orthogonal decomposition of the Hilbert space over $\regX$ and $\regY$ into two-dimensional subspaces $\{S_j\}_j$ that satisfies the following:\footnote{In general, there may also appear one-dimensional subspaces. However, by our assumption that $\projimp(\mcal{M})$ is supported by $p\in [1/4,3/4]$, 
all eigenvalues of $\Pi_0 \Pi_1 \Pi_0$ belongs to $[1/4,3/4]$, and thus one-dimensional subspaces do not appear in our case. 
}
For each two-dimensional subspace $S_j$, there exist two orthonormal bases $(\ket{\alpha_j},\ket{\alpha_j^{\bot}})$ and $(\ket{\beta_j},\ket{\beta_j^{\bot}})$ of $S_j$ such that 
\begin{align*}
    \Pi_0\ket{\alpha_j}=\ket{\alpha_j},~~~ \Pi_0\ket{\alpha_j^{\bot}}=0,\\
    \Pi_1\ket{\beta_j}=\ket{\beta_j},~~~ \Pi_1\ket{\beta_j^{\bot}}=0.
\end{align*}
Moreover, if we let 
\begin{align*}
    p_j\defeq \bra{\alpha_j}\Pi_1 \ket{\alpha_j},
\end{align*}
then we have $1/4\le p_i \le 3/4$ and
\begin{align*}
\ket{\alpha_j}=\sqrt{p_j}\ket{\beta_j}+\sqrt{1-p_j}\ket{\beta_j^{\bot}},~~~
\ket{\beta_j}=\sqrt{p_j}\ket{\alpha_j}+\sqrt{1-p_j}\ket{\alpha_j^{\bot}}.
\end{align*} 
In particular, this implies that 
\begin{align}
\label{eq:transition_alpha_beta}
\begin{split}
\Pi_1\ket{\alpha_j}=\sqrt{p_j}\ket{\beta_j},~~~ &(I-\Pi_1)\ket{\alpha_j}=\sqrt{1-p_j}\ket{\beta_j^\bot},\\
\Pi_1\ket{\alpha_j^\bot}=\sqrt{1-p_j}\ket{\beta_j},~~~ &(I-\Pi_1)\ket{\alpha_j^\bot}=\sqrt{p_j}\ket{\beta_j^\bot},\\
\Pi_0\ket{\beta_j}=\sqrt{p_j}\ket{\alpha_j},~~~ &(I-\Pi_0)\ket{\beta_j}=\sqrt{1-p_j}\ket{\alpha_j^\bot},\\
\Pi_0\ket{\beta_j^\bot}=\sqrt{1-p_j}\ket{\alpha_j},~~~ &(I-\Pi_0)\ket{\beta_j}=\sqrt{p_j}\ket{\alpha_j^\bot}
\end{split}
\end{align}
Since $\Pi_0\ket{\alpha_j}=\ket{\alpha_j}$, we can write $\ket{\alpha_j}=\ket{\alpha'_j}_{\regX}\ket{0}_{\regY}$ for each $j$. 
For each $p\in [1/4,3/4]$, 
we define a projector $E_p$ on $\regX$ as 
$$
E_p \coloneqq \sum_{j:p_j=p} \ket{\alpha'_j}\bra{\alpha'_j}. 
$$
Then one can see that $\mcal{E}=\{E_p\}_{p\in S}$ is the projective implementation of $\mcal{M}$ where \xiao{@Takashi: in terms of writing, I think this part might need some ``guidance sentence'' to let the reader know why you're building $E_p$. If I understand it correctly, you are constructing the ``unique'' projective implementation $\projimp(\mcal{M})$? The explicit form of this $\projimp(\mcal{M})$ you provide here will be used in later part of the proof?}
\takashi{Yes. So strictly speaking, we should have proven that $\mcal{E}$ satisfies the definition of projective implementation, but I thought this was almost obvious.}

\begin{takashienv}{The reason why $\mathcal{E}$ is the projective implementation}
    {Here is a brief explanation:
Given any state $\ket{\psi}$, we can decompose the state as
\[
\ket{\psi}\ket{0^n}=\sum_{j} c_j \ket{\alpha_j}.
\]
We have 
\[
\Tr(M_1\ket{\psi}\bra{\psi})=\|\Pi_1\ket{\psi_1}\|^2
=\|\sum_{j} c_j \sqrt{p_j} \ket{\beta_j}\|^2
=\sum_{j} c_j^2 p_j
\]
where the first equality follows from the definition of $\Pi_1$, the second from $\Pi_1\ket{\alpha_j}=\sqrt{p_j} \ket{\beta_j}$, 
and the third from the orthogonality of $\ket{\beta_j}$'s and $\|\ket{\beta_j}\|=1$.So applying the POVM $(M_0,M_1)$ on $\ket{\psi}$ results in the outcome $1$ with probability  $\sum_{j} c_j^2 p_j$.

On the other hand, if we first apply $\mathcal{E}$ on $\ket{\psi}$, then we get an outcome $p$ with probability 
\[
\sum_{j:p_{j}=p}c_{j}^2.
\]
Thus, if we output $1$ with probability $p$, the overall probability of outputting $1$ is
\[
\sum_{p}(\sum_{j:p_{j}=p}c_{j}^2)p
=\sum_{j}c_j^2 p_j
\]
where the first sum is taken over all $p$ such that $p=p_j$ for some $j$. 
So the probability of outputting $1$ is the same as that by the POVM $(M_0,M_1)$ for any initial state. (I focused on pure states, but this immediately extends to the mixed states by decomposing it into a mix of pure states.) This means that $\mathcal{E}$ is the projective implementation of $(M_0,M_1)$. 
}
\end{takashienv}

$$S \coloneqq \{p\in [1/4,3/4] ~:~ \exists j~\text{s.t.}~p_j=p\}.$$ 

We describe the algorithm $\tilde{\API}_{\mcal{M}}^{\epsilon,\eta}$ on register $\regX$ below:
\begin{enumerate}
\item Prepare and initialize the register $\regY$ to the all-zero state.
\item Initialize a classical list $L=(0)$.
\item \label{step:main_loop}
Repeat the following ``main loop"  
for $i=1,2, \ldots, T$, where $T\coloneqq \lceil  \ln(6/\eta)/\epsilon^2\rceil$: 
\begin{enumerate}
\item Apply the projective measurement $(I-\Pi_1,\Pi_1)$, obtaining an outcome $b_{2i-1}$, and append $b_{2i-1}$ to the end of $L$. 
\item Apply the projective measurement $(\Pi_0,I-\Pi_0)$, obtaining an outcome $b_{2i}$, and append $b_{2i}$ to the end of $L$. 
\end{enumerate}
\item Let $t$ be the number of bit flips in the sequence $L=(0,b_1,b_2,...,b_{2T})$, and let $\tilde{p} \coloneqq t/2T$. 
\item \label[Step]{step:API_recover}
If $b_{2T}=1$, repeat the ``main loop" until the first time $b_{2i}=0$ or it is repeated $T'=\lceil\log_{5/8}(\eta/3)\rceil$ times.   
We say that it fails if $b_{2i}=0$ does not occur within $T'$ times repetition. 
%In the case of latter, declare the failure and halt. 
\item Discard  $\regY$ and output $\tilde{p}$. 
\end{enumerate} 
The run time requirement of \Cref{item:API_run_time} is clear from the description. We next establish one by one. First, we remark that  $\tilde{\API}_{\mcal{M}}^{\epsilon,\eta}$ just applies projective measurements $(I-\Pi_1, \Pi_1)$ and $(\Pi_0,I-\Pi_0)$  on registers $\regX,\regY$. Therefore, when proving \Cref{item:API_shiftdis,item:API_almost_projective}, we can analyze each subspace separately. \xiao{@Takashi: just for my understanding: Though I believe this claim, I never go through the rigorous reasoning to prove it myself. I may need to discuss with you about it, especially for the case where there are multiple subspaces sharing the same $p_j$.}
\takashi{I don't think there's any issue in that case, but it might be better to write down the details. I was just too lazy to do so. It seems that Mark's proof writes a little bit more details on how it works. (His proof is essentially identical to ours except for how to define the decomposition into subspaces.)
} 

\begin{takashienv}{More details on why we can analyze each subspace separately.}
 We would be able to argue this formally as follows:
For any state $\ket{\psi}$, we decompose the state as
\[
\sum_{j} c_j \ket{\alpha'_j}.
\]
\begin{itemize}
\item Let $D$ be the distribution of the outcome of $\tilde{\API}_{\mcal{M}}^{\epsilon,\eta}(\ket{\psi})$ and $D_j$ be the distribution of outcome of $\tilde{\API}_{\mcal{M}}^{\epsilon,\eta}(\ket{\alpha'_j})$. 
Then $D$ can be written as a convex sum of $D_j$:
\[
D=\sum_{j} |c_j|^2 D_j.
\]
(A formal proof of this fact may need a tedious calculation, but this seems obvious to me from the fact that "different subspaces do not interfere each other" during the execution of $\tilde{\API}_{\mcal{M}}^{\epsilon,\eta}$.)
\item Let $D'$ be the distribution of outcome of $\mathsf{ProjImp}(\mathcal{M})(\ket{\psi})$ and $D'_j$ be the distribution whose density function concentrates on $p_j$. Then, by the fact that $\mathcal{E}$ is the projective implementation of $\mathcal{M}$,  we have 
\[
D'=\sum_{j} |c_j|^2 D'_j.
\]
\end{itemize}
Then we have 
\[
\shiftdis{\epsilon}(D,D')\le
\sum_{j}|c_j|^2\shiftdis{\epsilon}(D_j,D'_j)
\le 
\sum_{j}|c_j|^2 \eta =\eta
\]
where the first inequality follows from the definition of shift distance and the second inequality follows from the analysis of the case where the initial state is $\ket{\alpha'_j}$.  
This completes the proof of Item 1. 

For Item 2, let $D''$ be the distribution of $(p,p')$ obtained by a consecutive execution of $\tilde{\API}_{\mcal{M}}^{\epsilon,\eta}$ on the initial state $\ket{\psi}$. 
Let $D''_j$ be the similar distribution for the initial state $\ket{\alpha'_j}$. 
Then, 
\[
D''=\sum_{j}c_j D''_j.
\]
(Again, rely on the non-interference among different subspaces.)
Then, 
\[
\Pr[|p-p'|> \epsilon:(p,p')\leftarrow D'']
\le 
\sum_{j}|c_j|^2\Pr[|p-p'|> \epsilon:(p,p')\leftarrow D''_j]
\le 
\sum_{j}|c_j|^2 \eta =\eta
\]
where the first inequality is obvious and the second inequality follows from the analysis for the case where the initial state is $\ket{\alpha'_j}$.  
\end{takashienv}
%Below, we assume that the initial state $\ket{\psi}_{\regX}$ is such that $\ket{\psi}_{\regX}\ket{0^n}_{\regY}\in S_j$. 
That is, we can focus on the case where the initial state is $\ket{\alpha'_j}$ for some $j$. 

\subpara{Proving \Cref{item:API_shiftdis} of \Cref{lem:API}.}  Note that $\projimp(\mcal{M})$ on  $\ket{\alpha'_j}$ 
results in $p_j$ with probability $1$. \xiao{@Takashi: I guess this is one of the places where you utilize the explicit form of $\projimp(\mcal{M})$ you provided previously?}  \takashi{Yes, and I beleieve this is the only place.}
%Since each subspace $S_j$ is invariant under projections $\Pi_0$ and $\Pi_1$, we can analyze 
%For any state $\ket{\psi}_\regX$, 
%by considering the Jordan decomposition, one can see that
By \Cref{eq:transition_alpha_beta}, we can see that the list $L$ obtained by applying $\tilde{\API}_{\mcal{M}}^{\epsilon,\eta}$ on $\ket{\alpha'_j}$  is according to the following distribution:
\begin{itemize}
%\item Apply $\projimp(\mcal{M})$ on $\ket{\psi}_\regX$, obtaining an outcome $p$.
\item Let $K$ be a list of $2T$ independent coin flips with expected value $p_j$.
\item Set $L_i$ be the parity of the first $i$ bits of $K$.
\end{itemize}
Then $t=2T\tilde{p}$ is the number of $1$s in $K$. Thus, by Hoeffding's bound, we have 
$$
\Pr[|p_j-\tilde{p}|\ge \epsilon/2]\le 2e^{-2(2T)(\epsilon/2)^2}\le \eta/3<\eta
$$
where we used $T\ge \ln(6/\eta)/\epsilon^2$. 
This implies 
$\shiftdis{\epsilon}(\tilde{\API}_{\mcal{M}}^{\epsilon,\eta},\projimp(\mcal{M}))\le \eta$, finishing the proof of \Cref{item:API_shiftdis}. 

\subpara{Proving \Cref{item:API_almost_projective} of \Cref{lem:API}.}  Suppose that we sequentially run $\tilde{\API}^{\epsilon,\eta}_{\mcal{M}}$ twice on the initial state $\ket{\alpha'_j}$.  
Let $\tilde{p}_0$ and $\tilde{p}_1$ be the measurement outcome of the first and second application, respectively. 
If the first application of $\tilde{\API}^{\epsilon,\eta}_{\mcal{M}}$ does not fail, then the state in $\regX$ goes back to $\ket{\alpha'_j}$ at the end of the first application. 
%after the application is in the image of $\Pi_0=I_\regX\otimes \ket{0^n}\bra{0^n}_{\regY}$. 
Thus, by repeating a similar analysis to the above, we can see that 
$$
\Pr[|\tilde{p}_b-p_j|\ge \epsilon/2]\le \eta/3
$$
for $b\in \bit$ 
conditioned on that the first application does not fail. 
%In each subspace $S_j$, 
Moreover, each trial in \Cref{step:API_recover} of the description of $\tilde{\API}^{\epsilon,\eta}_{\mcal{M}}$  
succeeds with probability $2p_j(1-p_j)\ge 3/8$ \xiao{@Takashi: Why's that the success probability for each trial is $2p_j(1-p_j)$? Oh, I see, you actually meant to say ``succeeds with probability at least  $2p_j(1-p_j)$''?} 
\takashi{I menat that the probability is exactly equal to $2p_j(1-p_j)$ when the initial state is $\ket{\alpha'_j}$. 
Note that the probability of observing a "bit flip" in one step is $p_j$. Here, we are interested in the probability of observing a "bit flip" (from $1$ to $0$) in $2$ steps. This can happen in either way of "flip -> non-flip" or "non-flip -> flip". So its probability is $p_j(1-p_j)+(1-p_j)p_j=2p_j(1-p_j).$
}
where we used $p_j\in [1/4,3/4]$, and thus the probability of failure is at most $(1-3/8)^{T'}\le \eta/3$  
where we used $T'\ge \log_{5/8}(\eta/3)$. 
Combining the above, we have 
$$
\Pr[|\tilde{p}_0-\tilde{p}_1|\ge \epsilon]\le 
\eta/3+\eta/3+\eta/3=
\eta,
$$
which implies \Cref{item:API_almost_projective}. 
This finishes the proof of \Cref{lem:API} for the case where $\projimp(\mcal{M})$ is supported by $p\in [1/4,3/4]$. 

\para{For the General Case of $p \in [0,1]$.} Finally, we extend it to general binary-outcome POVMs. 
For any binary-outcome POVM $\mcal{M}=(M_0,M_1)$,  let  $\mcal{M}' \coloneqq (\frac{I}{4}+\frac{M_0}{2},\frac{I}{4}+\frac{M_1}{2})$. 
That is, $\mcal{M}'$ corresponds to the process that either outputs a uniformly random bit or applies $\mcal{M}$  
with probability $1/2$ for each.   
Let $\mcal{E}=\{E_p\}_{p\in S}$ be the projective implementation of $\mcal{M}$. Then it is easy to see that 
the projective implementation of $\mcal{M}'$ is 
$\mcal{E'}=\{E'_{p'}\}_{p'\in S'}$
where $E'_{p'} \coloneqq E_{2p'-1/2}$ and $S' \coloneqq \{p': 2p'-1/2\in S\}$. 
For any $p'\in S'$, 
since 
$2p'-1/2\in [0,1]$, we have $p'\in [1/4,3/4]$.  
Thus, $\projimp(\mcal{M}')$ is supported by $p'\in [1/4,3/4]$ and $\tilde{\API}$ is applicable for $\mcal{M}'$. 
Based on this observation, we construct $\API_\calM^{\epsilon,\eta}$ as follows:
\begin{enumerate}
\item Apply $\tilde{\API}_{\calM'}^{\epsilon/2,\eta}$, obtaining an outcome $p'$.
\item Output $p \coloneqq 2p'-1/2$.
\end{enumerate}
Then the properties of $\tilde{\API}_{\calM'}^{\epsilon/2,\eta}$ which we showed above are directly translated into those of $\API_\calM^{\epsilon,\eta}$, which concludes the proof of \Cref{lem:API}. 

\end{proof}



\begin{lemma}[{\cite[Lemma 4.10]{FOCS:CMSZ21}}]\label{lem:repair}
Let $\mcal{N}$ be an $(\epsilon,\eta)$-almost projective measurement on a Hilbert space $\mcal{H}$, and $\mcal{P}=(P_0,P_1)$ be a binary-outcome projective measurement on $\mcal{H}$. %and $T$ be a positive integer. 
Then there is a quantum algorithm $\mathsf{Repair}^{\mcal{N},\mcal{P}}$ on $\mcal{H}$ satisfying the following: 
\begin{itemize}
\item For a positive integer $T$, consider the following procedure $\mathsf{RepairExpt}^{\mcal{N},\mcal{P}}(1^T)$ on $\mcal{H}$: 
\begin{enumerate}
\item Apply $\mcal{N}$, obtaining outcome $p$; 
\item Apply $\mcal{P}$, obtaining outcome $b$;
\item Apply $\mathsf{Repair}^{\mcal{N},\mcal{P}}(1^T,b,p)$.
\item Output $p$.
\end{enumerate}
Then $\mathsf{RepairExpt}^{\mcal{N},\mcal{P}}(1^T)$ is $(2\epsilon,2(\eta+1/T)+4\sqrt{\eta})$-almost projective.
\item The expected run time of  $\mathsf{Repair}^{\mcal{N},\mcal{P}}(1^T,b,p)$ is at most $(T_{\mcal{N}}+T_{\mcal{P}})\cdot (4T\sqrt{\eta}+3)$ where  $T_{\mcal{N}}$ and $T_{\mcal{P}}$ are run times of $\mcal{
N}$ and $\mcal{P}$, respectively. \takashi{We may need to carefully clarify what is meant by an expected run time of a quantum algorithm. 
This is briefly mentioned in the Preliminaries of \cite{FOCS:CMSZ21}
}
\end{itemize}
\end{lemma}

We show the following corollary.
\begin{corollary}\label{cor:repair}
Let $\mcal{N}$ be an $(\epsilon,\eta)$-almost projective measurement on a Hilbert space $\mcal{H}$, and $\mcal{A}$ be a quantum algorithm that takes a quantum state in $\mcal{H}$ as input and outputs a classical string satisfying the following:
There are some classical string $s^*$ and $0\le \zeta \le 1$ such that for any state $\rho$, 
$$\Pr[s \notin \Set{s^*, \bot}~:~s \la \mcal{A}(\rho)]\le \zeta.$$
Then for any positive integer $T$ and $p\in[0,1]$, there is a measurement $\repairA(1^T,p)$ 
%a quantum algorithm $\tilde{\mcal{A}}$ 
%$\mathsf{Repair}^{\mcal{N},\mcal{P}}$ on $\mcal{H}$ 
satisfying the following: 
%For any state $\rho$, the output distribution of $\tilde{\mcal{A}}(\rho)$ is identical to that of $\mcal{A}(\rho)$.
\begin{itemize}
\item For any $T$, $p$, and any state $\rho$ in $\mcal{H}$, if we apply $\repairA(1^T,p)$ on $\rho$, then the distribution of the measurement outcome is identical to that of $\mcal{A}(\rho)$. 
\item For a positive integer $T$, consider the following procedure $\mathsf{RepairExpt}^{\mcal{N},\repairA}(1^T)$ on $\mcal{H}$: 
\begin{enumerate}
\item Apply $\mcal{N}$, obtaining outcome $p$; 
\item Apply $\repairA(1^T,p)$, obtaining outcome $s$; 
%\item Apply $\mathsf{Repair}^{\mcal{N},\mcal{A}}(1^T,s,p)$.
\item Output $p$.
\end{enumerate}
Then $\mathsf{RepairExpt}^{\mcal{N},\repairA}(1^T)$ is $\big(2\epsilon,2(\eta+1/T)+4\sqrt{\eta}+\sqrt{\zeta}\big)$-almost projective. 
\item The expected run time of  %$\mathsf{Repair}^{\mcal{N},\mcal{P}}(1^T,b,p)$ 
$\repairA(1^T,p)$
is at most $(T_{\mcal{N}}+T_{\mcal{A}})\cdot (4T\sqrt{\eta}+3)$ where  $T_{\mcal{N}}$ and $T_{\mcal{A}}$ are run times of $\mcal{
N}$ and $\mcal{A}$, respectively. 
\end{itemize}
\end{corollary}
\begin{proof}
%Let $U$ be the purification of $\mcal{A}$. More precisely, we define the unitary $U$ over registers $\reginp,\regS,\regW$ in such a way that $\mcal{A}$ can be described as follows:  
Intuitively, $\repairA$ first runs $\mcal{A}$ and then applies the repair procedure of \Cref{lem:repair}. A formal proof is given below. 

We can describe $\mcal{A}$ by using a unitary $U$ over the input register $\reginp$, output register $\regS$, and working register $\regW$ as follows:
\begin{description}
\item $\mcal{A}(\rho):$ Set $\rho$ in $\reginp$, initialize $\regS$ and $\regW$ to be all-zero states, apply $U$, measure $\regS$, and output the outcome $s$. 
\end{description}
We define a binary projective measurement $\mcal{P}=(P_0,P_1)$ on $\reginp$, $\regS$, and $\regW$ as 
$$
P_1 \coloneqq U^\dagger (\sum_{s\neq \bot}\ket{s}\bra{s})_{\regS} U
$$
and $P_0 \coloneqq I-P_1$. 
We apply \Cref{lem:repair} for $\mcal{N}$ and $\mcal{P}$ to get $\mathsf{Repair}^{\mcal{N},\mcal{P}}$ satisfying the requirements of \Cref{lem:repair}.\footnote{
Strictly speaking, $\mcal{N}$ is a POVM on $\reginp$ but $\mcal{P}$ is a projector on  $(\reginp,\regS,\regW)$ and thus  \Cref{lem:repair} is not directly applicable.  
We abuse the notation to simply write $\mcal{N}$ to mean its trivial extension  to registers $(\reginp,\regS,\regW)$ that does not touch $(\regS,\regW)$.
} By using it, we construct $\repairA(1^T,p)$ on $\reginp$ as follows:
\begin{enumerate}
\item Initialize $\regS$ and $\regW$ to all-zero states. 
\item Apply $\mcal{P}$, obtaining an outcome $b$.
\item \label[Step]{item:measure_s} If $b=0$, then set $s \coloneqq \bot$. If $b=1$, then apply $U$, measure $\regS$ to obtain $s$, and apply $U^\dagger$.
\item Apply $\mathsf{Repair}^{\mcal{N},\mcal{P}}(1^T,b,p)$.
\item \label[Step]{item:output_s}
Output $s$ as the measurement outcome. 
\end{enumerate}
It is clear from the construction that the distribution of $s$ obtained by applying $\repairA(p)$ on $\rho$ is identical to the distribution of $\mcal{A}(\rho)$. 
The requirement about the run time directly follows from that of \Cref{lem:repair}. 
Below, we show that $\mathsf{RepairExpt}^{\mcal{N},\repairA}(1^T)$ is $(2\epsilon,2(\eta+1/T)+4\sqrt{\eta}+\sqrt{\zeta})$-almost projective. 

Let $\repairA'(p)$ be a quantum process that works similarly to $\repairA(p)$ except that \Cref{item:measure_s,item:output_s} are removed. 
Then, it is not hard to see that $\mathsf{RepairExpt}^{\mcal{N},\repairA'}(1^T)$ is identical to 
$\mathsf{RepairExpt}^{\mcal{N},\mcal{P}}(1^T)$, and thus it is $(2\epsilon,2(\eta+1/T)+4\sqrt{\eta})$-almost projective by \Cref{lem:repair}. 
Moreover, we observe that the measurement in \Cref{item:measure_s} of $\repairA(p)$ for the case of $b=1$ yields a fixed value $s^*$ with probability except for $\zeta$ by the assumption about $\mcal{A}$. Thus, by the gentle measurement lemma \cite[Lemma 2.2]{DBLP:journals/toc/Aaronson05}, the trace distance between the states before and after the step is at most $\sqrt{\zeta}$.    
This implies that $\mathsf{RepairExpt}^{\mcal{N},\repairA}(1^T)$  is $(2\epsilon,2(\eta+1/T)+4\sqrt{\eta}+\sqrt{\zeta})$-almost projective.

This finishes the proof of \Cref{cor:repair}.
\end{proof}

\subsection{Proof of \Cref{lem:Simultaneous-SimExt}}

Let $\mcal{M}_{\secpar,\gamma,z}$ be the binary-outcome POVM corresponding to $\mcal{V}(1^\secpar,1^{\gamma^{-1}},\cdot,z)$. That is, it is defined in such a way that 
$
\Pr[\mcal{M}_{\secpar,\gamma,z}(\rho)=1]=
\Pr[\mcal{V}(1^\secpar,1^{\gamma^{-1}},\rho,z)=\top]$
for any state $\rho$.
Let $\API_{\mcal{M}_{\secpar,\gamma,z}}^{\epsilon,\eta}$ be the $(\epsilon,\eta)$-almost projective measurement
as given in \Cref{lem:API}.
For each $i,\secpar,\gamma,\zeta,\rho,z,\epsilon,\eta$, we apply \Cref{cor:repair} to 
the $(\epsilon,\eta)$-almost projective measurement $\API_{\mcal{M}_{\secpar,\gamma,z}}^{\epsilon,\eta}$ and
the algorithm $\mcal{K}_i(1^\secpar,1^{\gamma^{-1}},1^{\zeta^{-1}},\cdot,z)$, 
and we denote the corresponding repairing measurement by $\mcal{K}_i(1^\secpar,1^{\gamma^{-1}},1^{\zeta^{-1}},\cdot,z)\text{-}\repair$
and the corresponding repairing experiment by $\mathsf{RepairExpt}_{\secpar,\gamma,\zeta,z}^{\epsilon,\eta}(1^T)$.\footnote{If we strictly follow the notation in \Cref{cor:repair}, then  the experiment should be written as $\mathsf{RepairExpt}^{\API_{\mcal{M}_{\secpar,\gamma,z}}^{\epsilon,\eta},\mcal{K}_i(1^\secpar,1^{\gamma^{-1}},1^{\zeta^{-1}},\cdot,z)\text{-}\repair}(1^T)$, but we simply write $\mathsf{RepairExpt}_{\secpar,\gamma,\zeta,z}^{\epsilon,\eta}(1^T)$ for brevity.}
By the assumption about $\mcal{K}_i$ and \Cref{cor:repair}, $\mathsf{RepairExpt}_{\secpar,\gamma,\zeta,z}^{\epsilon,\eta}(1^T)$ 
%$\mathsf{RepairExpt}^{\API_{\mcal{M}_{\secpar,\gamma,z}}^{\epsilon,\eta},\mcal{K}_i(1^\secpar,1^{\gamma^{-1}},1^{\zeta^{-1}},\cdot,z)\text{-}\repair^{\epsilon,\eta}}$
is $\big(2\epsilon,2(\eta+1/T)+4\sqrt{\eta}+\sqrt{\zeta}+\negl(\secpar)\big)$-almost projective. \xiao{@Takashi: According to the statement of the current \Cref{lem:Simultaneous-SimExt}, the $\mcal{K}_i$'s probability is bounded by $\zeta + \negl$. However, the \Cref{cor:repair} does not take care of the $\negl$ term. Do we need to say something about it?}
\takashi{You are right. I added a negligible term.}

%For proving \Cref{lem:Simultaneous-SimExt}, it suffices to construct \emph{expected} QPT algorithm $\mcal{K}$ that satisfies the requirements since we could truncate it to make it strict QPT while satisfying the requirements.  \takashi{Is this okay? I argue in this way since the repair procedure of \Cref{lem:repair} is only shown to be expected QPT.}


We first construct \emph{expected} QPT algorithm $\mcal{K}$ that satisfies the requirements, after which we argue that we can modify it to be \emph{strict} QPT by truncation. 

The \emph{expected} QPT algorithm $\mcal{K}$ is described as follows:
\begin{description}
\item $\mcal{K}(1^\secpar,1^{\gamma^{-1}},1^{\zeta^{-1}},\rho,z)$: 
%Set $\rho$ in register $\reginp$, initialize $\regS$ and $\regW$  to all-zero states, and 
Do the following:
\begin{enumerate}
\item \label{item:set_delta}
Compute $\delta$ as in {\bf Assumption 2} of \Cref{lem:Simultaneous-SimExt} from the given $\gamma$. 
%when $\gamma$ is replaced with $\gamma' \coloneqq \gamma/8$.  \takashi{Here is a subtle issue. We want to replace the $\gamma$ with $\gamma'$ for the upper bound of the success probability of $\mcal{V}$, but we still want to use the original $\gamma$ as the input to $\mcal{V}$.}
\item 
%Set   
%$T' \coloneqq  \lfloor 2\delta^{-1}\secpar \rfloor$, 
Take an positive integer $T'$ in such a way that 
$(1-\delta/3)^{T'}\le \gamma/n$ holds. (For example, $T'=O(\delta^{-1} \log \gamma^{-1} \log n)$ suffices).
\item 
Set parameters as follows:
\begin{align*}
T & \coloneqq \lceil 6nT'\gamma^{-1} \rceil \\ 
\epsilon
& \coloneqq  \min\{2\gamma/(2nT'+1),\gamma/4\} \\ 
\eta
& \coloneqq  \left( \gamma/(18nT')\right)^2 \\ 
\zeta' 
& \coloneqq  \min \{\zeta/(nT'),\left( \gamma/(3nT')\right)^2,\delta/2\}
\end{align*} 
%so that $\mathsf{RepairExpt}_{\secpar,\gamma,\zeta',z}^{\epsilon,\eta}(1^T)$ is $(2\epsilon,\eta/(8nT'))$-almost projective. 

\item \label[Step]{item:check_initial}
Apply $\API_{\mcal{M}_{\secpar,\gamma,z}}^{\epsilon,\eta}$ on $\reginp$ to obtain an outcome $\tilde{p}$. 
 If $\tilde{p}<4\gamma-\epsilon$, output $\bot$ and halt.     
\item For $i=1,2,...,n$, do the following:
\begin{enumerate}
\item For $j=1,2,...,T' $, do the following  \label[Step]{item:K_loop_j}
\begin{enumerate}
\item Apply $\API_{\mcal{M}_{\secpar,\gamma,z}}^{\epsilon,\eta}$ to obtain an outcome $\tilde{p}_{i,j}$.

\item \label[Step]{item:check_prob}
If $\tilde{p}_{i,j}<\tilde{p}_{i,j-1}-2\epsilon$, output $\bot$ and halt, 
where when $i=1$ and $j=1$, $\tilde{p}_{i,j-1} \coloneqq \tilde{p}$ and when  $i\ge 2$ and $j=1$, $\tilde{p}_{i,j-1} \coloneqq \tilde{p}_{i-1,T'}$.  

\item \label[Step]{item:apply_K_i}
Apply $\mcal{K}_i(1^\secpar,1^{\gamma^{-1}},1^{\zeta'^{-1}},\cdot,z)\text{-}\repair^{\epsilon,\eta}(1^T,\tilde{p}_{i,j})$ to obtain an outcome $s_{i,j}$. 

\item \label[Step]{item:check_s}
If $s_{i,j}\ne \bot$, set $s_i \coloneqq s_{i,j}$, break the inner loop, and proceed to the outer loop for $i+1$. 
\end{enumerate}

\item \label[Step]{item:check_time_out}
If $s_{i,j}=\bot$ for all $j\in[T']$, output $\bot$ and halt.  
\end{enumerate}

\item Output $s_{1}||s_{2}||...||s_{n}$.
\end{enumerate}
\end{description}
We can see that $\mcal{K}$ runs in expected QPT 
%since $\API_{\mcal{M}_{\secpar,\gamma,z}}^{\epsilon,\eta}$ and $\mcal{K}_i(1^\secpar,1^{\gamma^{-1}},1^{\zeta'^{-1}},\cdot,z)\text{-}\repair^{\epsilon,\eta}(\tilde{p}_{i,j})$
%run in expected QPT  
by \Cref{lem:API} and \Cref{cor:repair}. %As remarked above, this can be converted into strict QPT one by truncation.

\subpara{Proving \Cref{item:simultaneous_conclusion_s_star_or_bot} of \Cref{item:simultaneous_gamma_delta}.} We observe that whenever  $\mcal{K}$  does not output $\bot$, 
%the $i$-th component of the 
for each $i\in [n]$, 
$s_i$ is a non-$\bot$ value obtained by $\mcal{K}_i(1^\secpar,1^{\gamma^{-1}},1^{\zeta'^{-1}},\cdot,z)\text{-}\repair^{\epsilon,\eta}(\tilde{p}_{i,j})$. By \Cref{cor:repair}, its distribution is identical to the output distribution of $\mcal{K}_i(1^\secpar,1^{\gamma^{-1}},1^{\zeta'^{-1}},\cdot,z)$. 
\xiao{ @Takashi: 
I have confusion regarding this step: I think \Cref{cor:repair} means that (correct me if I'm wrong): if we run $\repairA$ on the ``original'' state $\rho$, then it leads to the same distribution as $\mathcal{A}(\rho)$. However, I'm not sure whether the ``same distribution'' claim holds when we compare $\mcal{A}(\rho)$ with $\repairA(\rho')$, {\em where $\rho'$ is the result we obtained after applying $\mcal{N}$ to the original $\rho$}. Do we need the $\mcal{A}(\rho) = \repairA(\rho')$ condition here to make your argument go through? 
}


\xiao{Oh, I think I see it now. You are talking about {\bf Assumption 1}, which holds for any $\rho$. So what I concerned does not matter here. Is this correct?}

\takashi{Yes!}

By {\bf Assumption 1} of \Cref{lem:Simultaneous-SimExt}, it outputs non-$\bot$ value other than $s^*_{z,i}$ with probability at most $\zeta'+\negl(\secpar)$. 
Since we apply it at most $T'$ times, the probability that it ever occurs is at most $T' (\zeta'+\negl(\secpar))$. %by Assumption \Cref{item:simultaneous_s_star_or_bot} of \Cref{item:simultaneous_gamma_delta}.  
By taking union bound over all $i\in [n]$, the probability that it occurs for some $i\in [n]$ is at most $n T' (\zeta'+\negl(\secpar))\le \zeta+\negl(\secpar)$. This finishes the proof of \Cref{item:simultaneous_conclusion_s_star_or_bot}. 

\subpara{Proving  \Cref{item:simultaneous_conclusion_gamma_delta} of \Cref{item:simultaneous_gamma_delta}.} 
Suppose that $\rho$ and $z$ satisfy the requirement of \Cref{item:simultaneous_conclusion_gamma_delta}, i.e., we have
\begin{align}\label{eq:_assumption_V}
\Pr[d=\top ~:~ d \leftarrow \mcal{V}(1^\secpar,1^{\gamma^{-1}},\rho,{z})]\geq  8\gamma. 
\end{align}  
%Below, we give an upper bound for the probability that $\mcal{K}(1^\secpar,1^{\gamma^{-1}},1^{\zeta^{-1}},\rho,z)$ returns $\bot$. 
We define the following events in the execution of $\mcal{K}(1^\secpar,1^{\gamma^{-1}},1^{\zeta^{-1}},\rho,z)$: 
\begin{itemize}
\item $\mathsf{Bad}_1$: The event that $\mcal{K}$ returns $\bot$ in   \Cref{item:check_initial}.
\item $\mathsf{Bad}_2$: The event that $\mcal{K}$ returns $\bot$ in \Cref{item:check_prob} for some $i,j$.    
\item $\mathsf{Bad}_3$: The event that $\mcal{K}$ returns $\bot$ in  \Cref{item:check_time_out} for some $i$.  
\end{itemize}  
Note that we have 
\begin{align}\label{eq:bot_prob}
\Pr[\mcal{K}(1^\secpar,1^{\gamma^{-1}},1^{\zeta^{-1}},\rho,z)=\bot]=\Pr[\mathsf{Bad}_1]+\Pr[\mathsf{Bad}_2]+\Pr[\mathsf{Bad}_3].
\end{align}
Below (\Cref{lem:bad_1,lem:bad_2,lem:bad_3}), we upper bound each term in the RHS of \Cref{eq:bot_prob}.

\begin{lemma}\label{lem:bad_1}
$\Pr[\mathsf{Bad}_1]\le 1-4\gamma+\eta$
\end{lemma}
\begin{proof}[Proof of \Cref{lem:bad_1}]
\Cref{eq:_assumption_V} implies
$$
\Pr[\mcal{M}_{\secpar,\gamma,z}(\rho)=1]\geq 8\gamma. 
$$
By the definition of $\projimp(\mcal{M}_{\secpar,\gamma,z})$ and an averaging argument, we have 
$$
\Pr[p\ge 4\gamma :p\gets \projimp(\mcal{M}_{\secpar,\gamma,z})(\rho)]\ge 4\gamma.
$$
By \Cref{item:API_shiftdis} of \Cref{lem:API}, we have 
$$
\Pr[\tilde{p}\ge 4\gamma- \epsilon :\tilde{p}\gets \API_{\mcal{M}_{\secpar,\gamma,z}}^{\epsilon,\eta}(\rho)]\ge 4\gamma-\eta.
$$
This completes the proof of \Cref{lem:bad_1}.

\end{proof}

\begin{lemma}\label{lem:bad_2}
$\Pr[\mathsf{Bad}_2]\le \gamma+\negl(\secpar)$. 
%nT'\left(2(\eta+1/T)+4\sqrt{\eta}+\sqrt{\zeta}\right)$
\end{lemma}
\begin{proof}[Proof of \Cref{lem:bad_2}]
Since $\API_{\mcal{M}_{\secpar,\gamma,z}}^{\epsilon,\eta}$ is $(\epsilon,\eta)$-almost projective, we have 
$$\Pr[\tilde{p}_{1,1}<\tilde{p}-2\epsilon]\le \eta.$$ 
Note that the loop done in \Cref{item:K_loop_j} of $\mcal{K}(1^\secpar,1^{\gamma^{-1}},1^{\zeta^{-1}},\rho,z)$ is identical to $\mathsf{RepairExpt}_{\secpar,\gamma,\zeta',z}^{\epsilon,\eta}(1^T)$ except for an additional check in \Cref{item:check_prob}.  
Since $\mathsf{RepairExpt}_{\secpar,\gamma,\zeta',z}^{\epsilon,\eta}(1^T)$ 
is $\big(2\epsilon,2(\eta+1/T)+4\sqrt{\eta}+\sqrt{\zeta'}+\negl(\secpar)\big)$-almost projective by \Cref{cor:repair}, it holds for each 
$(i,j)\ne (1,1)$ that
%$(i,j)\in ([n]\times [T'])\setminus \{(1,1)\}$, 
$$\Pr[\tilde{p}_{i,j}<\tilde{p}_{i,j-1}-2\epsilon]\le 2(\eta+1/T)+4\sqrt{\eta}+\sqrt{\zeta'}+\negl(\secpar)\le \gamma/(nT')+\negl(\secpar)$$
where we used $\eta\le \left( \gamma/(18nT')\right)^2$, $T=\lceil 6nT'\gamma^{-1} \rceil$, and $\zeta'\le \left( \gamma/(3nT')\right)^2$.

Noting that 
$$\eta\le \left( \gamma/(18nT')\right)^2\le \gamma/(nT'),$$ 
the union bound over all $(i,j)\in [n]\times [T']$
gives \Cref{lem:bad_2}.

\end{proof}

\begin{lemma}\label{lem:bad_3}
$\Pr[\mathsf{Bad}_3]\le \gamma$.
\end{lemma}
\begin{proof}[Proof of \Cref{lem:bad_3}]
   For each $i,j$, let $\rho_{i,j}$ be the state just before applying 
$\mcal{K}_i(1^\secpar,1^{\gamma^{-1}},1^{\zeta'^{-1}},\cdot,z)\text{-}\repair^{\epsilon,\eta}(\tilde{p}_{i,j})$ in
 \Cref{item:apply_K_i}. 
 Note that whenever \Cref{item:apply_K_i} is invoked, either of $\mathsf{\Bad}_1$ or $\mathsf{Bad}_2$ has not occurred by that point, which implies $\tilde{p}_{i,j}\ge 4\gamma-(2nT'+1)\epsilon \ge 2\gamma$ where we used $\epsilon\le 2\gamma/(2nT'+1)$.  
 Since $\API_{\mcal{M}_{\secpar,\gamma,z}}^{\epsilon,\eta}$ is $(\epsilon,\eta)$-almost projective by \Cref{item:API_almost_projective} of \Cref{lem:API}, 
 $$
 \Pr[
 \tilde{p}'_{i,j}\ge 2\gamma- \epsilon
 :
 \tilde{p}'_{i,j}\gets \API_{\mcal{M}_{\secpar,\gamma,z}}^{\epsilon,\eta}(\rho_{i,j})]\ge 1-\eta.
 $$
 Since we have $\shiftdis{\epsilon}(\API_{\mcal{M}_{\secpar,\gamma,z}}^{\epsilon,\eta},\projimp(\mcal{M}_{\secpar,\gamma,z}))\le \eta$ by \Cref{item:API_shiftdis} of \Cref{lem:API}, we have  
  $$
 \Pr[
 p_{i,j}\ge 2\gamma- 2\epsilon
 :
 p_{i,j}\gets \projimp(\mcal{M}_{\secpar,\gamma,z})(\rho_{i,j})]\ge 1-2\eta.
 $$
 This implies 
 $$
 \Pr[d=\top ~:~ d \leftarrow \mcal{V}(1^\secpar,1^{\gamma^{-1}},\rho_{i,j},{z})]\geq (1-2\eta)(2\gamma- 2\epsilon)
 \ge \gamma
 $$
 where we used $\eta\le \left( \gamma/(18nT')\right)^2\le \gamma/4$ and $\epsilon\le \gamma/4$. 
Thus, by {\bf Assumption 2} of \Cref{lem:Simultaneous-SimExt}, %\footnote{Recall that we set $\delta$ in such a way that Assumption~2 holds for $\gamma/8$ instead of $\gamma$ as stated in \Cref{item:set_delta} of the description of $\mcal{K}$.}  
we have 
$$
\Pr[s_i =s^*_{{z,i}}~:~ s_i\la \mcal{K}_i(1^\secpar,1^{\gamma^{-1}}, 1^{\zeta'^{-1}}, \rho_{i,j},{z})]\geq   \delta-\zeta'-\negl(\secpar) \ge \delta/3
$$
for sufficiently large $\secpar$ 
where we used $\zeta'\le \delta/2$. %and $\negl(\secpar)\le \zeta'/6$ for sufficiently large $\secpar$. 
Noting that $\mcal{K}_i(1^\secpar,1^{\gamma^{-1}},1^{\zeta'^{-1}},\cdot,z)\text{-}\repair^{\epsilon,\eta}(1^T,\tilde{p}_{i,j})$ on $\rho_{i,j}$ yields the identical distribution as  $\mcal{K}_i(1^\secpar,1^{\gamma^{-1}}, 1^{\zeta'^{-1}}, \rho_{i,j},{z})$ by \Cref{cor:repair} \xiao{@Takashi: I have the same confusion as the last time you invoked \Cref{cor:repair}: $\rho_{i,j}$ is different from $\rho$; does it require $\rho_{i,j}=\rho$ to apply the first item of \Cref{cor:repair}? If I understand your writing correctly, this is not a real issue here either --- it is just a typo that you should have written $\rho_{i, j}$ in place of $\rho$ in the input ot $\mcal{K}_i$?}\takashi{Yes, you are right, I fixed it.}, for each $i,j$, the probability of breaking the inner loop in \Cref{item:check_s} is at least $\delta/3$. Thus, for each $i$, the probability that this does not happen for all $j\in [T']$ is at most $(1-\delta/3)^{T'}\le \gamma/n$. \xiao{@Takashi: Just for my own understanding: is it true that if I'm willing to set $T'$ large enough, I can actually make this bad-3 probability negligible?}\takashi{I believe so.} %by $T'=\lfloor 2\delta^{-1} \secpar \rfloor$. 
By taking the union bound over all $i\in [n]$, \Cref{lem:bad_3} holds.

\end{proof}

Combining \Cref{eq:bot_prob} and \Cref{lem:bad_1,lem:bad_2,lem:bad_3}, we have 
\begin{align*}
\Pr[\mcal{K}(1^\secpar,1^{\gamma^{-1}},1^{\zeta^{-1}},\rho,z)=\bot]\le 1-4\gamma+ \eta+ \gamma + \gamma+\negl(\secpar)\le 1-\gamma+\negl(\secpar)
\end{align*}
where we used $\eta \le \left( \gamma/(18nT')\right)^2\le \gamma$. 
Combined with \Cref{item:simultaneous_conclusion_s_star_or_bot} of \Cref{lem:Simultaneous-SimExt} which is already proven, we have 
\begin{align*}
\Pr[\mcal{K}(1^\secpar,1^{\gamma^{-1}},1^{\zeta^{-1}},\rho,z)=s_{z,1}^*||...||s_{z,n}^*]\ge \gamma-\zeta-\negl(\secpar). 
\end{align*} 
This finishes the proof that $\mcal{K}$ satisfies  \Cref{item:simultaneous_conclusion_gamma_delta} of \Cref{lem:Simultaneous-SimExt}.
\xiao{@Takashi: It appears to me that the current proof (so far) does not mention $\delta'$ at all? If I understand it correct, you mean that $\delta' = \gamma$ in the expected QPT case, and $\delta' = \gamma/2$ in the strictly QPT case?}
\takashi{Yes.}


\para{On Strictly QPT.} Finally, we argue how to convert $\mcal{K}$ into a \emph{strict} QPT one. 
For some polynomial $C(\secpar)$, 
suppose that we modify $\mcal{K}$ so that if it runs $C(\secpar)$ times longer 
than its expected run time, then it immediately outputs $\bot$ and halts. 
Then $\mcal{K}$ now runs in strict QPT. This modification does not affect \Cref{item:simultaneous_conclusion_s_star_or_bot} since $\mcal{K}$ only outputs $\bot$ in the case of the time out. 
By Markov's inequality, the time out occurs with probability at most $C(\secpar)^{-1}$, which may decrease the probability in \Cref{item:simultaneous_conclusion_gamma_delta} by at most $C(\secpar)^{-1}$. Thus, if we set $C(\secpar)$ in such a way that $C(\secpar)^{-1}\le \delta'(\secpar)/2$, then  \Cref{item:simultaneous_conclusion_gamma_delta} is still satisfied if we replace $\delta'(\secpar)$ with $\delta'(\secpar)/2$.  

This completes the proof of \Cref{lem:Simultaneous-SimExt}. 
\takashi{Please check if this argument is okay.} \xiao{I just finished reading. It looks good modulo the comments I left.}

