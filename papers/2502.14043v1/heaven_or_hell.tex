\begin{figure}[H]
    \centering
    \small
    \begin{tikzpicture}[>={Stealth[scale=1]}, auto, node distance=3cm, semithick]
        % Styles for uniform, larger nodes
        \tikzstyle{state} = [ellipse, minimum width=1.3cm,minimum height=.6cm, inner sep=0,path picture={\fill[left color=blue!8, right color=blue!2] (path picture bounding box.south west) rectangle (path picture bounding box.north east);}]

        % States
        \node[state] (S1) {Start};
        \node[state, above right=0cm and 3cm of S1] (Heaven) {Heaven};
        \node[state, below right=0cm and 3cm of S1] (Hell) {Hell};

        % Arrows
        \path[->] (S1) edge[bend left=12] node[midway, above=.1cm] {Action 1} (Heaven)
                       edge[bend right=12] node[midway, below=.1 cm] {Action 2} (Hell);
        \path[->] (Heaven) edge[loop right] node[right] {Reward = 1} (Heaven);
        \path[->] (Hell) edge[loop right] node[right] {Reward = 0} (Hell);

    \end{tikzpicture}
    \caption{The ``Heaven or Hell'' problem demonstrates why online RL is doomed without further assumptions. Action 1 leads to high reward forever and Action 2 leads to low reward forever. Without further assumptions (or external help), the agent no way to tell which action leads to Heaven and which leads to Hell, so it is impossible to guarantee high reward.\looseness=-1}
    \label{fig:heaven_hell_problem}
\end{figure}