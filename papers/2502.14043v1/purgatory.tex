\begin{figure}[tb]
    \centering
\hspace{-1.2 in}
    \begin{tikzpicture}[>=stealth', shorten >=1pt, auto, node distance=2.9cm, semithick]
        % Styles for uniform, larger nodes
        \tikzstyle{state} = [ellipse, minimum width=1.5cm,minimum height=.7cm, inner sep=0,path picture={\fill[left color=blue!8, right color=blue!2] (path picture bounding box.south west) rectangle (path picture bounding box.north east);}]

        % States
        \node[state] (S1) {Start};
        \node[state, above right=0cm and 2.9cm of S1] (Heaven) {Heaven};
        \node[state, below right=0cm and 2.9cm of S1] (Hell) {Hell};

        % Arrows
        \path[->] (S1) edge[bend left=15] node[midway, above=.1cm] {Action 1} (Heaven)
                       edge[bend right=15] node[midway, below=.1 cm] {Action 2} (Hell);
        \path[->] (Heaven) edge[loop right] node[right] {Reward = 1} (Heaven);
        \path[->] (S1) edge[loop left] node[left, align=center] {Action 3\\ Reward = 0} (S1);        
        \path[->] (Hell) edge[loop right] node[right] {Reward = 0} (Hell);

    \end{tikzpicture}
    \caption{A variant of \Cref{fig:heaven_hell_problem} which could be called the ``Heaven or Hell or Purgatory'' problem. Now a third action is available which keeps the agent in the starting state and provides reward 0. Action 3 is always safe, as it avoids Hell and retains the option to enter Heaven. However, Action 1 leads to much higher reward and is also safe. This example shows that knowing a safe policy upfront does not resolve the Heaven or Hell problem, since the agent can only obtain high reward if it reaches Heaven, and knowing that Action 3 is safe does not help the agent to determine whether Action 1 or Action 2 will lead to Heaven.}
    \label{fig:purgatory}
\end{figure}