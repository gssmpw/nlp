\usepackage{fullpage}
\usepackage{thmtools}
\usepackage{thm-restate}
\usepackage{graphicx}
\usepackage{float}
%\usepackage{subfigure}
\usepackage{booktabs} 
\usepackage{wrapfig}   
\usepackage[table]{xcolor}% for wrapfigure
\usepackage{subcaption}
\usepackage{placeins}
\usepackage{adjustbox}            % To adjust table width
\usepackage{multirow} 
\usepackage{makecell}


\usepackage{color-edits}
\usepackage{enumerate}
\addauthor{ev}{purple}
\addauthor{ishani}{teal}
\addauthor{mika}{green}


\usepackage{amsmath,amssymb,amsfonts,mathtools,amsthm}
\usepackage{bbm}
\usepackage{color}
%\usepackage{verbatim}
\usepackage{natbib}

\usepackage[algo2e,linesnumbered,ruled,vlined]{algorithm2e}
\usepackage{algorithmic}
%\usetikzlibrary{arrows}
%\usetikzlibrary{decorations.pathreplacing,angles,quotes}
%\usepackage{setspace}
\usepackage{caption}
\usepackage{fancyhdr}
% \usepackage[colorinlistoftodos]{todonotes}
%\usepackage{rotating}
\usepackage{enumitem, comment}
\usepackage{soul} %highlight and underline
 \usepackage{dsfont}
%\usepackage{bookmark}
\usepackage[utf8]{inputenc} % allow utf-8 input
\usepackage{csquotes}
\usepackage{fnpct} %for footnotes
\usepackage{float}
\usepackage[utf8]{inputenc}
\usepackage[english]{babel}
%\usepackage{parskip}
% \usepackage{cleveref}
%\usepackage[mathcal]{euscript}
%\usepackage{dutchcal}

%fonts
% \usepackage[T1]{fontenc}
%\usepackage{libertine} \usepackage[libertine]{newtxmath}
% \usepackage{lmodern}
%\usepackage{mathpazo}
%\usepackage[theoremfont]{newpxmath}
%\usepackage{newpxmath}
%\usepackage{kpfonts}
\usepackage{xspace}    % Smart spacing with \xspace % to define text macros
% \usepackage[protrusion=true,expansion=true]{microtype}  % Improve font rendering
%\usepackage{libertine}
%\usepackage[libertine]{newtxmath}
%\usepackage[varg]{txfonts}
%\usepackage{pifont}
%\renewcommand{\rmdefault}{ptm}
%\usepackage{mathpazo}
% theorem environments
% \let\openbox\relax
% \usepackage{amsthm}
% \usepackage{thm-restate}
% \usepackage{ifdraft}

%\usepackage[color]{showkeys}
\usepackage{nicefrac}

\usepackage{mdframed}
\newmdtheoremenv{theomd}{Theorem}

\newtheorem{theorem}{Theorem}[section]
\newtheorem*{theorem*}{Theorem}

\newtheorem{Claim}[theorem]{Claim}
\newtheorem*{claim*}{Claim}
\newtheorem{subclaim}{Claim}[theorem]
\newtheorem{proposition}[theorem]{Proposition}
\newtheorem*{proposition*}{Proposition}
\newtheorem{lemma}[theorem]{Lemma}
\newtheorem*{lemma*}{Lemma}
\newtheorem{corollary}[theorem]{Corollary}
\newtheorem{conjecture}[theorem]{Conjecture}
\newtheorem*{conjecture*}{Conjecture}
\newtheorem{observation}[theorem]{Observation}
\newtheorem{fact}[theorem]{Fact}
\newtheorem{hypothesis}[theorem]{Hypothesis}
\newtheorem*{hypothesis*}{Hypothesis}

\theoremstyle{definition}
\newtheorem{definition}[theorem]{Definition}
\newmdtheoremenv{defmd}[theorem]{Definition}
\newtheorem{construction}[theorem]{Construction}
\newtheorem{reduction}[theorem]{Reduction}
\newtheorem{example}[theorem]{Example}
%\newtheorem{algorithm}[theorem]{Algorithm}
\newtheorem{SDP}[theorem]{SDP}
\newtheorem{problem}[theorem]{Problem}
\newtheorem{goal}[theorem]{Goal}
\newtheorem{protocol}[theorem]{Protocol}
\newtheorem{remark}[theorem]{Remark}
\newtheorem{assumption}[theorem]{Assumption}
\newtheorem{question}[theorem]{Question}

\usepackage{prettyref}
\newcommand{\savehyperref}[2]{\texorpdfstring{\hyperref[#1]{#2}}{#2}}

\newrefformat{eq}{\savehyperref{#1}{\textup{(\ref*{#1})}}}
% \newrefformat{lem}{\savehyperref{#1}{Lemma~\ref*{#1}}}
\newrefformat{def}{\savehyperref{#1}{Definition~\ref*{#1}}}
\newrefformat{thm}{\savehyperref{#1}{Theorem~\ref*{#1}}}
\newrefformat{cor}{\savehyperref{#1}{Corollary~\ref*{#1}}}
\newrefformat{cha}{\savehyperref{#1}{Chapter~\ref*{#1}}}
\newrefformat{sec}{\savehyperref{#1}{Section~\ref*{#1}}}
\newrefformat{app}{\savehyperref{#1}{Appendix~\ref*{#1}}}
\newrefformat{tab}{\savehyperref{#1}{Table~\ref*{#1}}}
\newrefformat{fig}{\savehyperref{#1}{Figure~\ref*{#1}}}
\newrefformat{hyp}{\savehyperref{#1}{Hypothesis~\ref*{#1}}}
\newrefformat{alg}{\savehyperref{#1}{Algorithm~\ref*{#1}}}
\newrefformat{sdp}{\savehyperref{#1}{SDP~\ref*{#1}}}
\newrefformat{rmk}{\savehyperref{#1}{Remark~\ref*{#1}}}
\newrefformat{item}{\savehyperref{#1}{Item~\ref*{#1}}}
\newrefformat{step}{\savehyperref{#1}{step~\ref*{#1}}}
\newrefformat{conj}{\savehyperref{#1}{Conjecture~\ref*{#1}}}
\newrefformat{fact}{\savehyperref{#1}{Fact~\ref*{#1}}}
\newrefformat{prop}{\savehyperref{#1}{Proposition~\ref*{#1}}}
\newrefformat{claim}{\savehyperref{#1}{Claim~\ref*{#1}}}
\newrefformat{relax}{\savehyperref{#1}{Relaxation~\ref*{#1}}}
\newrefformat{red}{\savehyperref{#1}{Reduction~\ref*{#1}}}
\newrefformat{part}{\savehyperref{#1}{Part~\ref*{#1}}}
\newrefformat{prob}{\savehyperref{#1}{Problem~\ref*{#1}}}
\newrefformat{ass}{\savehyperref{#1}{Assumption~\ref*{#1}}}
\newrefformat{ques}{\savehyperref{#1}{Question~\ref*{#1}}}

\newrefformat{ex}{\savehyperref{#1}{Example~\ref*{#1}}}
% bold math package - provides command to use boldmath font
\usepackage{bm}
\newcommand{\im}{\mathrm{i}\mkern1mu}


%\hypersetup{colorlinks,linkcolor=blue,filecolor = blue, citecolor = violet, urlcolor  = magenta}

\usepackage{thmtools}
\usepackage{thm-restate}
% \declaretheorem[name=Lemma,numberwithin=section]{lemma}
% \declaretheorem[name=Lemma,numberwithin=section]{lemma}

% % BibLatex
% \usepackage[backend=biber, style=alphabetic,
% sorting=nyt, natbib=true, backref=false, eprint=false, url=false, doi = false, isbn = false, maxbibnames = 99, maxcitenames = 2, maxalphanames = 4]{biblatex}
% \addbibresource{ref.bib}

% % If eprint, then show, else show doi, else show url
% \DeclareSourcemap{
%   \maps[datatype=bibtex]{
%     \map[overwrite]{
%       \step[fieldsource=eprint, final]
%       \step[fieldset=doi, null]
%     }
%   }
% }

% \DeclareSourcemap{
%   \maps[datatype=bibtex]{
%     \map[overwrite]{
%       \step[fieldsource=doi, final]
%       \step[fieldset=url, null]
%     }
%   }
% }

% \DeclareSourcemap{
%   \maps[datatype=bibtex]{
%     \map[overwrite]{
%       \step[fieldsource=eprint, final]
%       \step[fieldset=url, null]
%     }
%   }
% }

% % Do not show adress, location
% \DeclareSourcemap{
%   \maps[datatype=bibtex, overwrite]{
%     \map{
%       \step[fieldset=address, null]
%       \step[fieldset=location, null]
%     }
%   }
% }
\newenvironment{keywords}
{\bgroup\leftskip 20pt\rightskip 20pt \small\noindent{\bfseries
Keywords:} \ignorespaces}%
{\par\egroup\vskip 0.25ex}
\newlength\aftertitskip     \newlength\beforetitskip
\newlength\interauthorskip  \newlength\aftermaketitskip

\renewcommand{\qedsymbol}{\ensuremath{\blacksquare}}


% punctation at the end of displayed formulas
\newcommand{\mper}{\,.}
\newcommand{\mcom}{\,,}


% boldface vectors
\renewcommand{\vec}[1]{{\bm{#1}}}

% prime/tilde vector
\newcommand{\bvec}[1]{{\bar{\vec{#1}}}}
\newcommand{\pvec}[1]{\vec{#1}'}
\newcommand{\ppvec}[1]{\vec{#1}''}
\newcommand{\tvec}[1]{{\tilde{\vec{#1}}}}

% parentheses
\newcommand{\paren}[1]{\left(#1 \right )}
\newcommand{\Paren}[1]{\left(#1 \right )}

% brackets
\newcommand{\brac}[1]{[#1 ]}
\newcommand{\Brac}[1]{\left[#1\right]}

% set braces
\newcommand{\set}[1]{\left\{#1\right\}}
\newcommand{\Set}[1]{\left\{#1\right\}}


% absolute value sign
\newcommand{\abs}[1]{\left\lvert#1\right\rvert}
\newcommand{\Abs}[1]{\left\lvert#1\right\rvert}

% ceil floor
\newcommand{\ceil}[1]{\lceil #1 \rceil}
\newcommand{\floor}[1]{\lfloor #1 \rfloor}

% norm
\newcommand{\norm}[1]{\left\lVert#1\right\rVert}
\newcommand{\Norm}[1]{\left\lVert#1\right\rVert}
\newcommand{\fnorm}[1]{\norm{#1}_\mathrm{F}}
\newcommand{\opnorm}[1]{\norm{#1}_\mathrm{op}}
\newcommand{\spnorm}[1]{\norm{#1}_\mathrm{2}}
\newcommand{\nuclear}[1]{\norm{#1}_{*}}

%nnz non zeros
\newcommand{\nnz}[1]{\mathrm{nnz}(#1)}


% define symbol for definition
\newcommand{\defeq}{\mathrel{\mathop:}=}%{\stackrel{\textup{def}}{=}}
\newcommand{\seteq}{\gets}%{\mathrel{\mathop:}=}
\newcommand{\iseq}{\stackrel{\textup{?}}{=}}
\newcommand{\isgeq}{\stackrel{\textup{?}}{\geq}}
\newcommand{\isleq}{\stackrel{\textup{?}}{\leq}}
% big vertical space
\newcommand{\vbig}{\vphantom{\bigoplus}}

% linear algebra
\newcommand{\inprod}[1]{\left\langle #1\right\rangle}

% trace
\newcommand{\Tr}[1]{\mathrm {Tr}\paren{#1}}
\newcommand{\tr}{\mathrm {Tr}}


% norm
\newcommand{\snorm}[1]{\norm{#1}^2}
\newcommand{\dist}[1]{{\sf dist }\paren{#1}}

% L2 norm
\newcommand{\normt}[1]{\norm{#1}_{\scriptstyle 2}}
\newcommand{\snormt}[1]{\norm{#1}^2_2}

% Projection Operators
\newcommand{\projsymb}{{\sf Proj}}
\newcommand{\proj}[2]{\projsymb_{#1}\paren{#2}}
\newcommand{\projperp}[2]{\projsymb_{#1}^{\perp}\paren{#2}}
\newcommand{\projpar}[2]{\projsymb_{#1}^{\parallel}\paren{#2}}

% norms
\newcommand{\normo}[1]{\norm{#1}_{\scriptstyle 1}}
\newcommand{\normi}[1]{\norm{#1}_{\scriptstyle \infty}}
\newcommand{\normb}[1]{\norm{#1}_{\scriptstyle \square}}

% number sets
\newcommand{\Z}{{\mathbb Z}}
\newcommand{\N}{{\mathbb N}}
\newcommand{\R}{\mathbb R}
\newcommand{\Q}{\mathbb Q}
\newcommand{\Rnn}{\R_+}

\newcommand{\subjectto}{\text{subject to}}
\newcommand{\suchthat}{~:~}

% probability symbols
\newcommand{\Esymb}{\mathbb{E}}
\newcommand{\Psymb}{\mathbb{P}}
\newcommand{\Vsymb}{\mathbb{V}}
\newcommand{\Isymb}{\mathbb{I}}
\DeclareMathOperator*{\E}{\Esymb}
\DeclareMathOperator*{\Var}{{ Var}}
\DeclareMathOperator*{\ProbOp}{\Psymb}
\newcommand{\var}[1]{\Var \paren{#1}}

%conditioning
\newcommand{\given}{\mathrel{}\middle|\mathrel{}}
\newcommand{\Given}{\given}

% probability of an event \prob{e}=IP{e}
\newcommand{\prob}[1]{\ProbOp\Set{#1}}
\newcommand{\ber}{\mathsf{Ber}}
\newcommand{\Prob}{\mathbb{P}}
\newcommand{\probSub}[2]{\mathbb{P}_{#2}\Set{#1}}

%indicator functions
\newcommand{\indicator}[1]{\mathds{1}{\set{#1}}}
\newcommand{\1}[1]{\mathds{1}_{\set{#1}}}

% Expectation of variable \ex{X} = IE[X]
\newcommand{\ex}[1]{\E\brac{#1}}
\newcommand{\Ex}[1]{\E\Brac{#1}}
\newcommand{\exSub}[2]{\E_{#2}\brac{#1}}


\renewcommand{\Pr}[1]{\ProbOp\Brac{#1}}
\newcommand{\pr}[2]{\ProbOp_{#1}\Set{#2}}


\newcommand{\ind}[2]{\Isymb_{#1}\brac{#2}}
\newcommand{\Ind}[1]{\Isymb\Brac{#1}}

\newcommand{\varex}[1]{\E\paren{#1}}
\newcommand{\varEx}[1]{\E\Paren{#1}}
\newcommand{\eset}{\emptyset}
\newcommand{\e}{\epsilon}
\newcommand{\ebf}{\mathbf{e}}
\newcommand{\Abar}{\bar{A}}
\newcommand{\Dbar}{\bar{D}}
\newcommand{\zero}{\mathbf{0}}

%probability distributions
\newcommand{\bern}[1]{\ensuremath{\operatorname{Bern}\paren{ #1 } }}
\newcommand{\unif}{\ensuremath{\operatorname{Unif}}}


% superscript with parentheses
\newcommand{\super}[2]{#1^{\paren{#2}}}

% bits
\newcommand{\bits}{\{0,1\}}

\definecolor{DSgray}{cmyk}{0,0,0,0.7}
\newcommand{\Authornote}[2]{{\small\textcolor{red}{\sf$<<<${  {\sc #1}: #2 }$>>>$}}}
\newcommand{\Authormarginnote}[2]{\marginpar{\parbox{2cm}{\raggedright\tiny \textcolor{red}{#1: #2}}}}

\let\e\varepsilon

% {{{ alphabet }}}

% \newcommand{\tI}{\mathcal I}

\newcommand{\cA}{\mathcal A}
\newcommand{\cB}{\mathcal B}
\newcommand{\cC}{\mathcal C}
\newcommand{\cD}{\mathcal D}
\newcommand{\cE}{\mathcal E}
\newcommand{\cF}{\mathcal F}
\newcommand{\cG}{\mathcal G}
\newcommand{\cH}{\mathcal H}
\newcommand{\cI}{\mathcal I}
\newcommand{\cJ}{\mathcal J}
\newcommand{\cK}{\mathcal K}
\newcommand{\cL}{\mathcal L}
\newcommand{\cM}{\mathcal M}
\newcommand{\cN}{\mathcal N}
\newcommand{\cO}{\mathcal O}
\newcommand{\cP}{\mathcal P}
\newcommand{\cQ}{\mathcal Q}
\newcommand{\cR}{\mathcal R}
\newcommand{\cS}{\mathcal S}
\newcommand{\cT}{\mathcal T}
\newcommand{\cU}{\mathcal U}
\newcommand{\cV}{\mathcal V}
\newcommand{\cW}{\mathcal W}
\newcommand{\cX}{\mathcal X}
\newcommand{\cY}{\mathcal Y}
\newcommand{\cZ}{\mathcal Z}
\newcommand{\hatN}{\hat{N}}

\newcommand{\bbB}{\mathbb B}
\newcommand{\bbS}{\mathbb S}
\newcommand{\bbR}{\mathbb R}
\newcommand{\bbZ}{\mathbb Z}
\newcommand{\bbI}{\mathbb I}
\newcommand{\bbQ}{\mathbb Q}
\newcommand{\bbP}{\mathbb P}
\newcommand{\bbE}{\mathbb E}
\newcommand{\bbC}{\mathbb C}

\newcommand{\sfE}{\mathsf E}


% {{{ names }}}
% Hungarian/Polish/East European names
\newcommand{\Erdos}{Erd\H{o}s\xspace}
\newcommand{\Renyi}{R\'enyi\xspace}
\newcommand{\Lovasz}{Lov\'asz\xspace}
\newcommand{\cdeg}{\mathrm{cdeg}}
\newcommand{\bigO}{O}
\newcommand{\bigo}[1]{\bigO\!\left(#1\right)}
\newcommand{\tbigO}{\tilde{\mathcal{O}}}
\newcommand{\tbigo}[1]{\tbigO\!\left(#1\right)}
\newcommand{\tensor}{\otimes}
\newcommand{\eigvec}{{\sf v}}

\DeclareMathOperator*{\argmin}{argmin} %limit displayed underneath
\DeclareMathOperator*{\argmax}{argmax} %limit displayed underneath
%\newcommand{\argmax}{{\sf argmax}} %limit displayed on side
%\newcommand{\argmin}{{\sf argmin}} %limit displayed on side
\newcommand{\poly}{{\sf poly}}
\newcommand{\polylog}{{\sf polylog}}
\newcommand{\supp}{{\sf supp}}

\newcommand{\rank}{{\sf rank}}
\newcommand{\tk}{t_{1/k}} % gausssian cap
\newcommand{\rd}{{\sf d}}


% Porject Specific Macros added below
\newcommand{\tildeN}{\widetilde{N}}
\newcommand{\tildeA}{\widetilde{A}}
\newcommand{\tX}{\widetilde{X}}
\newcommand{\tA}{\widetilde{A}}
\newcommand{\tB}{\widetilde{B}}
\newcommand{\calI}{\mathcal{I}}
\newcommand{\calS}{\mathcal{S}}
\newcommand{\barA}{\bar{A}}
\newcommand{\barB}{\bar{B}}
\newcommand{\Bbd}{B_{\text{bd}}}

\usepackage{dsfont}
\usepackage{bbm}
\newcommand{\calE}{\mathcal{E}}
\newcommand{\calB}{\mathcal{B}}
\newcommand{\blambda}{\mathbf{\lambda}}
\newcommand{\xto}{x^{(i)}}
\newcommand{\Ebar}{\bar{E}}

% Sparsifier replacement macros
\newcommand{\spectralApx}{additive spectral norm approximation }
\newcommand{\spectralApxEnd}{additive spectral norm approximation}
\newcommand{\nuclearApx}{nuclear norm approximation }
\newcommand{\nuclearApxEnd}{nuclear norm approximation}

\newcommand{\lap}{L}
\newcommand{\normAdj}{N}
\newcommand{\normLap}{L_N}
\newcommand{\query}{\mathcal{Q}}
% comments
\newcommand{\chris}[1]{{\small\textcolor{violet}{\bf$\lll${  Chris: #1 }$\ggg$}}}
 \newcommand{\sidford}[1]{{\small\textcolor{red}{\bf$\lll${  Sidford: #1 }$\ggg$}}}
 \newcommand{\avs}[1]{{\small\textcolor{orange}{\bf$\lll${  Apoorv: #1 }$\ggg$}}}
\newcommand{\ishani}[1]{{\small\textcolor{teal}{\bf$\lll${  Ishani: #1 }$\ggg$}}}
\newcommand{\joyee}[1]{{\small\textcolor{brown}{\bf$\lll${  Joyee: #1 }$\ggg$}}}

\newcommand{\luana}[1]{{\small\textcolor{red}{\bf$\lll${  L: #1 }$\ggg$}}}

\usepackage{subcaption}
%\renewcommand{\yujia}[1]{{}}
%\renewcommand{\chris}[1]{{}}
%\renewcommand{\sidford}[1]{{}}
%\renewcommand{\avs}[1]{{}}


\newcommand{\was}{Wasserstein}
\renewcommand{\epsilon}{\varepsilon}



% ================

% % ================

% % ================

% \crefname{algorithm}{algorithm}{algorithms}
% \Crefname{algorithm}{Algorithm}{Algorithms}
% \crefname{lemma}{lemma}{lemma}
% \Crefname{lemma}{Lemma}{Lemma}
% \crefname{def}{definition}{definition}
% \Crefname{def}{Definition}{Definition}
% \Crefname{problem}{Problem}{Problem}
% \crefname{problem}{problem}{problem}
% \crefname{fact}{fact}{fact}
% \Crefname{fact}{Fact}{Fact}

\newcommand{\vtilde}{\tilde{v}}
\newcommand{\pitilde}{\tilde{\pi}}
\newcommand{\sample}{\code{sample}}
\newcommand{\sampleTrans}{\code{Sample}}
\newcommand{\sampleProd}{\code{ApxUtility}}
\newcommand{\apxVal}{\code{ApxVal}}
\newcommand{\truncatedRandomizedVI}{\code{TruncatedVRVI}}
\newcommand{\Toff}{\mathcal{T}_{\mathrm{offset}}}
\newcommand{\apxOffset}{\code{ApxOffset}}
\newcommand{\SublineartruncatedRandomizedVI}{\code{SamplingTruncatedVRVI}}
\newcommand{\ProblemDependentRandomizedVI}{\code{ProblemDependentTruncatedVRVI}}
\newcommand{\HighPrecision}{\code{OfflineTruncatedVRVI}}
\newcommand{\bv}{\bm{v}}
\newcommand{\bxi}{\bm{\xi}}
\newcommand{\bx}{\bm{x}}
\newcommand{\bg}{\bm{g}}
\newcommand{\br}{\bm{r}}
\newcommand{\bP}{\mathbf{P}}
\newcommand{\bu}{\bm{u}}
\newcommand{\bsigma}{\bm{\sigma}}
\newcommand{\bmp}{\bm{p}}
\newcommand{\Atot}{\cA_{\mathrm{tot}}}%{{\cA_{\mathrm{tot}}}}
\newcommand{\Aset}{\cA}
\newcommand{\vones}{\mathbf{1}}
\newcommand{\vzero}{\mathbf{0}}
\newcommand{\tmix}{t_{\mathrm{mix}}}

\newcommand{\reward}{\bm{r}}
\newcommand{\transProb}{\bm{p}}
\newcommand{\otilde}{\tilde{O}}
\newcommand{\transMat}{\mathbf{P}}

\newcommand{\vals}{\bv}
\newcommand{\optVals}{\bv^*}
\newcommand{\diffVec}{\bm{\Delta}}


\newcommand{\code}[1]{\textnormal{\texttt{#1}}}
\newcommand{\median}{\code{median}}

\SetKwInput{KwInput}{Input}
\SetKwInput{KwOutput}{Output}
\SetKwInput{Return}{return}
\usepackage{algorithmic}

\newcommand{\VarOf}[1]{\textnormal{Var}\Brac{#1}}

\usepackage{booktabs} % For prettier tables
\usepackage{array} % For new column types

\newcommand{\reviewerone}{\textcolor{red}{\textbf{(R1)}}}
\newcommand{\reviewerthree}{\textcolor{purple}{\textbf{(R3)}}}
\newcommand{\reviewerfour}{\textcolor{orange}{\textbf{(R4)}}}
\newcommand{\Igamma}{(\bm{I} - \gamma \bP^{\star})^{-1}}
\newcommand{\normsigma}{\norm{(\bm{I} - \gamma \bP^{\star})^{-1}\sqrt{\bsigma_{\bv^\star}}}_\infty}
\newcommand{\tv}[2]{d_{TV}\paren{#1, #2}}

\newcommand{\bI}{\bm{I}}

% Define the Proof Sketch environment
\newenvironment{hproof}{%
  \renewcommand{\proofname}{Proof sketch}\proof}{\endproof}


\newcommand{\cOT}{\textnormal{OT}}
\newcommand{\cTD}{\textnormal{TD}}
\newcommand{\cTMD}{\textnormal{TMD}}
\newcommand{\cTreeNorm}[1]{\normInline{#1}_{\textnormal{TN}_w^L}}
\newcommand{\blankTree}{T_{\bm{0}}}
\newcommand{\neighborhood}{\textnormal{N}}
\newcommand{\normInline}[1]{\Vert #1 \Vert}
\newcommand{\absInline}[1]{\lvert #1 \rvert}
\newcommand{\cOTbar}{\smash{\overline{\textnormal{OT}}}}