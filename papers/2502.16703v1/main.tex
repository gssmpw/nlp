\documentclass[11pt]{article}


%
\setlength\unitlength{1mm}
\newcommand{\twodots}{\mathinner {\ldotp \ldotp}}
% bb font symbols
\newcommand{\Rho}{\mathrm{P}}
\newcommand{\Tau}{\mathrm{T}}

\newfont{\bbb}{msbm10 scaled 700}
\newcommand{\CCC}{\mbox{\bbb C}}

\newfont{\bb}{msbm10 scaled 1100}
\newcommand{\CC}{\mbox{\bb C}}
\newcommand{\PP}{\mbox{\bb P}}
\newcommand{\RR}{\mbox{\bb R}}
\newcommand{\QQ}{\mbox{\bb Q}}
\newcommand{\ZZ}{\mbox{\bb Z}}
\newcommand{\FF}{\mbox{\bb F}}
\newcommand{\GG}{\mbox{\bb G}}
\newcommand{\EE}{\mbox{\bb E}}
\newcommand{\NN}{\mbox{\bb N}}
\newcommand{\KK}{\mbox{\bb K}}
\newcommand{\HH}{\mbox{\bb H}}
\newcommand{\SSS}{\mbox{\bb S}}
\newcommand{\UU}{\mbox{\bb U}}
\newcommand{\VV}{\mbox{\bb V}}


\newcommand{\yy}{\mathbbm{y}}
\newcommand{\xx}{\mathbbm{x}}
\newcommand{\zz}{\mathbbm{z}}
\newcommand{\sss}{\mathbbm{s}}
\newcommand{\rr}{\mathbbm{r}}
\newcommand{\pp}{\mathbbm{p}}
\newcommand{\qq}{\mathbbm{q}}
\newcommand{\ww}{\mathbbm{w}}
\newcommand{\hh}{\mathbbm{h}}
\newcommand{\vvv}{\mathbbm{v}}

% Vectors

\newcommand{\av}{{\bf a}}
\newcommand{\bv}{{\bf b}}
\newcommand{\cv}{{\bf c}}
\newcommand{\dv}{{\bf d}}
\newcommand{\ev}{{\bf e}}
\newcommand{\fv}{{\bf f}}
\newcommand{\gv}{{\bf g}}
\newcommand{\hv}{{\bf h}}
\newcommand{\iv}{{\bf i}}
\newcommand{\jv}{{\bf j}}
\newcommand{\kv}{{\bf k}}
\newcommand{\lv}{{\bf l}}
\newcommand{\mv}{{\bf m}}
\newcommand{\nv}{{\bf n}}
\newcommand{\ov}{{\bf o}}
\newcommand{\pv}{{\bf p}}
\newcommand{\qv}{{\bf q}}
\newcommand{\rv}{{\bf r}}
\newcommand{\sv}{{\bf s}}
\newcommand{\tv}{{\bf t}}
\newcommand{\uv}{{\bf u}}
\newcommand{\wv}{{\bf w}}
\newcommand{\vv}{{\bf v}}
\newcommand{\xv}{{\bf x}}
\newcommand{\yv}{{\bf y}}
\newcommand{\zv}{{\bf z}}
\newcommand{\zerov}{{\bf 0}}
\newcommand{\onev}{{\bf 1}}

% Matrices

\newcommand{\Am}{{\bf A}}
\newcommand{\Bm}{{\bf B}}
\newcommand{\Cm}{{\bf C}}
\newcommand{\Dm}{{\bf D}}
\newcommand{\Em}{{\bf E}}
\newcommand{\Fm}{{\bf F}}
\newcommand{\Gm}{{\bf G}}
\newcommand{\Hm}{{\bf H}}
\newcommand{\Id}{{\bf I}}
\newcommand{\Jm}{{\bf J}}
\newcommand{\Km}{{\bf K}}
\newcommand{\Lm}{{\bf L}}
\newcommand{\Mm}{{\bf M}}
\newcommand{\Nm}{{\bf N}}
\newcommand{\Om}{{\bf O}}
\newcommand{\Pm}{{\bf P}}
\newcommand{\Qm}{{\bf Q}}
\newcommand{\Rm}{{\bf R}}
\newcommand{\Sm}{{\bf S}}
\newcommand{\Tm}{{\bf T}}
\newcommand{\Um}{{\bf U}}
\newcommand{\Wm}{{\bf W}}
\newcommand{\Vm}{{\bf V}}
\newcommand{\Xm}{{\bf X}}
\newcommand{\Ym}{{\bf Y}}
\newcommand{\Zm}{{\bf Z}}

% Calligraphic

\newcommand{\Ac}{{\cal A}}
\newcommand{\Bc}{{\cal B}}
\newcommand{\Cc}{{\cal C}}
\newcommand{\Dc}{{\cal D}}
\newcommand{\Ec}{{\cal E}}
\newcommand{\Fc}{{\cal F}}
\newcommand{\Gc}{{\cal G}}
\newcommand{\Hc}{{\cal H}}
\newcommand{\Ic}{{\cal I}}
\newcommand{\Jc}{{\cal J}}
\newcommand{\Kc}{{\cal K}}
\newcommand{\Lc}{{\cal L}}
\newcommand{\Mc}{{\cal M}}
\newcommand{\Nc}{{\cal N}}
\newcommand{\nc}{{\cal n}}
\newcommand{\Oc}{{\cal O}}
\newcommand{\Pc}{{\cal P}}
\newcommand{\Qc}{{\cal Q}}
\newcommand{\Rc}{{\cal R}}
\newcommand{\Sc}{{\cal S}}
\newcommand{\Tc}{{\cal T}}
\newcommand{\Uc}{{\cal U}}
\newcommand{\Wc}{{\cal W}}
\newcommand{\Vc}{{\cal V}}
\newcommand{\Xc}{{\cal X}}
\newcommand{\Yc}{{\cal Y}}
\newcommand{\Zc}{{\cal Z}}

% Bold greek letters

\newcommand{\alphav}{\hbox{\boldmath$\alpha$}}
\newcommand{\betav}{\hbox{\boldmath$\beta$}}
\newcommand{\gammav}{\hbox{\boldmath$\gamma$}}
\newcommand{\deltav}{\hbox{\boldmath$\delta$}}
\newcommand{\etav}{\hbox{\boldmath$\eta$}}
\newcommand{\lambdav}{\hbox{\boldmath$\lambda$}}
\newcommand{\epsilonv}{\hbox{\boldmath$\epsilon$}}
\newcommand{\nuv}{\hbox{\boldmath$\nu$}}
\newcommand{\muv}{\hbox{\boldmath$\mu$}}
\newcommand{\zetav}{\hbox{\boldmath$\zeta$}}
\newcommand{\phiv}{\hbox{\boldmath$\phi$}}
\newcommand{\psiv}{\hbox{\boldmath$\psi$}}
\newcommand{\thetav}{\hbox{\boldmath$\theta$}}
\newcommand{\tauv}{\hbox{\boldmath$\tau$}}
\newcommand{\omegav}{\hbox{\boldmath$\omega$}}
\newcommand{\xiv}{\hbox{\boldmath$\xi$}}
\newcommand{\sigmav}{\hbox{\boldmath$\sigma$}}
\newcommand{\piv}{\hbox{\boldmath$\pi$}}
\newcommand{\rhov}{\hbox{\boldmath$\rho$}}
\newcommand{\upsilonv}{\hbox{\boldmath$\upsilon$}}

\newcommand{\Gammam}{\hbox{\boldmath$\Gamma$}}
\newcommand{\Lambdam}{\hbox{\boldmath$\Lambda$}}
\newcommand{\Deltam}{\hbox{\boldmath$\Delta$}}
\newcommand{\Sigmam}{\hbox{\boldmath$\Sigma$}}
\newcommand{\Phim}{\hbox{\boldmath$\Phi$}}
\newcommand{\Pim}{\hbox{\boldmath$\Pi$}}
\newcommand{\Psim}{\hbox{\boldmath$\Psi$}}
\newcommand{\Thetam}{\hbox{\boldmath$\Theta$}}
\newcommand{\Omegam}{\hbox{\boldmath$\Omega$}}
\newcommand{\Xim}{\hbox{\boldmath$\Xi$}}


% Sans Serif small case

\newcommand{\Gsf}{{\sf G}}

\newcommand{\asf}{{\sf a}}
\newcommand{\bsf}{{\sf b}}
\newcommand{\csf}{{\sf c}}
\newcommand{\dsf}{{\sf d}}
\newcommand{\esf}{{\sf e}}
\newcommand{\fsf}{{\sf f}}
\newcommand{\gsf}{{\sf g}}
\newcommand{\hsf}{{\sf h}}
\newcommand{\isf}{{\sf i}}
\newcommand{\jsf}{{\sf j}}
\newcommand{\ksf}{{\sf k}}
\newcommand{\lsf}{{\sf l}}
\newcommand{\msf}{{\sf m}}
\newcommand{\nsf}{{\sf n}}
\newcommand{\osf}{{\sf o}}
\newcommand{\psf}{{\sf p}}
\newcommand{\qsf}{{\sf q}}
\newcommand{\rsf}{{\sf r}}
\newcommand{\ssf}{{\sf s}}
\newcommand{\tsf}{{\sf t}}
\newcommand{\usf}{{\sf u}}
\newcommand{\wsf}{{\sf w}}
\newcommand{\vsf}{{\sf v}}
\newcommand{\xsf}{{\sf x}}
\newcommand{\ysf}{{\sf y}}
\newcommand{\zsf}{{\sf z}}


% mixed symbols

\newcommand{\sinc}{{\hbox{sinc}}}
\newcommand{\diag}{{\hbox{diag}}}
\renewcommand{\det}{{\hbox{det}}}
\newcommand{\trace}{{\hbox{tr}}}
\newcommand{\sign}{{\hbox{sign}}}
\renewcommand{\arg}{{\hbox{arg}}}
\newcommand{\var}{{\hbox{var}}}
\newcommand{\cov}{{\hbox{cov}}}
\newcommand{\Ei}{{\rm E}_{\rm i}}
\renewcommand{\Re}{{\rm Re}}
\renewcommand{\Im}{{\rm Im}}
\newcommand{\eqdef}{\stackrel{\Delta}{=}}
\newcommand{\defines}{{\,\,\stackrel{\scriptscriptstyle \bigtriangleup}{=}\,\,}}
\newcommand{\<}{\left\langle}
\renewcommand{\>}{\right\rangle}
\newcommand{\herm}{{\sf H}}
\newcommand{\trasp}{{\sf T}}
\newcommand{\transp}{{\sf T}}
\renewcommand{\vec}{{\rm vec}}
\newcommand{\Psf}{{\sf P}}
\newcommand{\SINR}{{\sf SINR}}
\newcommand{\SNR}{{\sf SNR}}
\newcommand{\MMSE}{{\sf MMSE}}
\newcommand{\REF}{{\RED [REF]}}

% Markov chain
\usepackage{stmaryrd} % for \mkv 
\newcommand{\mkv}{-\!\!\!\!\minuso\!\!\!\!-}

% Colors

\newcommand{\RED}{\color[rgb]{1.00,0.10,0.10}}
\newcommand{\BLUE}{\color[rgb]{0,0,0.90}}
\newcommand{\GREEN}{\color[rgb]{0,0.80,0.20}}

%%%%%%%%%%%%%%%%%%%%%%%%%%%%%%%%%%%%%%%%%%
\usepackage{hyperref}
\hypersetup{
    bookmarks=true,         % show bookmarks bar?
    unicode=false,          % non-Latin characters in AcrobatÕs bookmarks
    pdftoolbar=true,        % show AcrobatÕs toolbar?
    pdfmenubar=true,        % show AcrobatÕs menu?
    pdffitwindow=false,     % window fit to page when opened
    pdfstartview={FitH},    % fits the width of the page to the window
%    pdftitle={My title},    % title
%    pdfauthor={Author},     % author
%    pdfsubject={Subject},   % subject of the document
%    pdfcreator={Creator},   % creator of the document
%    pdfproducer={Producer}, % producer of the document
%    pdfkeywords={keyword1} {key2} {key3}, % list of keywords
    pdfnewwindow=true,      % links in new window
    colorlinks=true,       % false: boxed links; true: colored links
    linkcolor=red,          % color of internal links (change box color with linkbordercolor)
    citecolor=green,        % color of links to bibliography
    filecolor=blue,      % color of file links
    urlcolor=blue           % color of external links
}
%%%%%%%%%%%%%%%%%%%%%%%%%%%%%%%%%%%%%%%%%%%





\title{Subsampling Graphs with GNN Performance Guarantees}

\author{
	Mika Sarkin Jain \\ Stanford University \\ \texttt{mjain4@stanford.edu} \and 
	Stefanie Jegelka \\ TUM and MIT \\ \texttt{stefje@csail.mit.edu}
    \and
	Ishani Karmarkar \\ Stanford University \\ \texttt{ishanik@stanford.edu }
  \and
  	Luana Ruiz \\ Johns Hopkins University \\ \texttt{lrubini1@jhu.edu}
    \and
	Ellen Vitercik \\ Stanford University \\ \texttt{vitercik@stanford.edu}}

\begin{document}

\maketitle

\begin{abstract}
How can we subsample graph data so that a graph neural network (GNN) trained on the subsample achieves performance comparable to training on the full dataset? This question is of fundamental interest, as smaller datasets reduce labeling costs, storage requirements, and computational resources needed for training. Selecting an effective subset is challenging: a poorly chosen subsample can severely degrade model performance, and empirically testing multiple subsets for quality obviates the benefits of subsampling. Therefore, it is critical that subsampling comes with guarantees on model performance. In this work, we introduce new subsampling methods for graph datasets that leverage the \emph{Tree Mover’s Distance} to reduce both the number of graphs and the size of individual graphs. To our knowledge, our approach is the first that is supported by rigorous theoretical guarantees: we prove that training a GNN on the subsampled data results in a bounded increase in loss compared to training on the full dataset. Unlike existing methods, our approach is both \emph{model-agnostic}, requiring minimal assumptions about the GNN architecture, and \emph{label-agnostic}, eliminating the need to label the full training set. This enables subsampling early in the model development pipeline—before data annotation, model selection, and hyperparameter tuning—reducing costs and resources needed for storage, labeling, and training. We validate our theoretical results with experiments showing that our approach outperforms existing subsampling methods across multiple datasets.
 \end{abstract}


\section{Introduction}
\label{sec:introduction}
The business processes of organizations are experiencing ever-increasing complexity due to the large amount of data, high number of users, and high-tech devices involved \cite{martin2021pmopportunitieschallenges, beerepoot2023biggestbpmproblems}. This complexity may cause business processes to deviate from normal control flow due to unforeseen and disruptive anomalies \cite{adams2023proceddsriftdetection}. These control-flow anomalies manifest as unknown, skipped, and wrongly-ordered activities in the traces of event logs monitored from the execution of business processes \cite{ko2023adsystematicreview}. For the sake of clarity, let us consider an illustrative example of such anomalies. Figure \ref{FP_ANOMALIES} shows a so-called event log footprint, which captures the control flow relations of four activities of a hypothetical event log. In particular, this footprint captures the control-flow relations between activities \texttt{a}, \texttt{b}, \texttt{c} and \texttt{d}. These are the causal ($\rightarrow$) relation, concurrent ($\parallel$) relation, and other ($\#$) relations such as exclusivity or non-local dependency \cite{aalst2022pmhandbook}. In addition, on the right are six traces, of which five exhibit skipped, wrongly-ordered and unknown control-flow anomalies. For example, $\langle$\texttt{a b d}$\rangle$ has a skipped activity, which is \texttt{c}. Because of this skipped activity, the control-flow relation \texttt{b}$\,\#\,$\texttt{d} is violated, since \texttt{d} directly follows \texttt{b} in the anomalous trace.
\begin{figure}[!t]
\centering
\includegraphics[width=0.9\columnwidth]{images/FP_ANOMALIES.png}
\caption{An example event log footprint with six traces, of which five exhibit control-flow anomalies.}
\label{FP_ANOMALIES}
\end{figure}

\subsection{Control-flow anomaly detection}
Control-flow anomaly detection techniques aim to characterize the normal control flow from event logs and verify whether these deviations occur in new event logs \cite{ko2023adsystematicreview}. To develop control-flow anomaly detection techniques, \revision{process mining} has seen widespread adoption owing to process discovery and \revision{conformance checking}. On the one hand, process discovery is a set of algorithms that encode control-flow relations as a set of model elements and constraints according to a given modeling formalism \cite{aalst2022pmhandbook}; hereafter, we refer to the Petri net, a widespread modeling formalism. On the other hand, \revision{conformance checking} is an explainable set of algorithms that allows linking any deviations with the reference Petri net and providing the fitness measure, namely a measure of how much the Petri net fits the new event log \cite{aalst2022pmhandbook}. Many control-flow anomaly detection techniques based on \revision{conformance checking} (hereafter, \revision{conformance checking}-based techniques) use the fitness measure to determine whether an event log is anomalous \cite{bezerra2009pmad, bezerra2013adlogspais, myers2018icsadpm, pecchia2020applicationfailuresanalysispm}. 

The scientific literature also includes many \revision{conformance checking}-independent techniques for control-flow anomaly detection that combine specific types of trace encodings with machine/deep learning \cite{ko2023adsystematicreview, tavares2023pmtraceencoding}. Whereas these techniques are very effective, their explainability is challenging due to both the type of trace encoding employed and the machine/deep learning model used \cite{rawal2022trustworthyaiadvances,li2023explainablead}. Hence, in the following, we focus on the shortcomings of \revision{conformance checking}-based techniques to investigate whether it is possible to support the development of competitive control-flow anomaly detection techniques while maintaining the explainable nature of \revision{conformance checking}.
\begin{figure}[!t]
\centering
\includegraphics[width=\columnwidth]{images/HIGH_LEVEL_VIEW.png}
\caption{A high-level view of the proposed framework for combining \revision{process mining}-based feature extraction with dimensionality reduction for control-flow anomaly detection.}
\label{HIGH_LEVEL_VIEW}
\end{figure}

\subsection{Shortcomings of \revision{conformance checking}-based techniques}
Unfortunately, the detection effectiveness of \revision{conformance checking}-based techniques is affected by noisy data and low-quality Petri nets, which may be due to human errors in the modeling process or representational bias of process discovery algorithms \cite{bezerra2013adlogspais, pecchia2020applicationfailuresanalysispm, aalst2016pm}. Specifically, on the one hand, noisy data may introduce infrequent and deceptive control-flow relations that may result in inconsistent fitness measures, whereas, on the other hand, checking event logs against a low-quality Petri net could lead to an unreliable distribution of fitness measures. Nonetheless, such Petri nets can still be used as references to obtain insightful information for \revision{process mining}-based feature extraction, supporting the development of competitive and explainable \revision{conformance checking}-based techniques for control-flow anomaly detection despite the problems above. For example, a few works outline that token-based \revision{conformance checking} can be used for \revision{process mining}-based feature extraction to build tabular data and develop effective \revision{conformance checking}-based techniques for control-flow anomaly detection \cite{singh2022lapmsh, debenedictis2023dtadiiot}. However, to the best of our knowledge, the scientific literature lacks a structured proposal for \revision{process mining}-based feature extraction using the state-of-the-art \revision{conformance checking} variant, namely alignment-based \revision{conformance checking}.

\subsection{Contributions}
We propose a novel \revision{process mining}-based feature extraction approach with alignment-based \revision{conformance checking}. This variant aligns the deviating control flow with a reference Petri net; the resulting alignment can be inspected to extract additional statistics such as the number of times a given activity caused mismatches \cite{aalst2022pmhandbook}. We integrate this approach into a flexible and explainable framework for developing techniques for control-flow anomaly detection. The framework combines \revision{process mining}-based feature extraction and dimensionality reduction to handle high-dimensional feature sets, achieve detection effectiveness, and support explainability. Notably, in addition to our proposed \revision{process mining}-based feature extraction approach, the framework allows employing other approaches, enabling a fair comparison of multiple \revision{conformance checking}-based and \revision{conformance checking}-independent techniques for control-flow anomaly detection. Figure \ref{HIGH_LEVEL_VIEW} shows a high-level view of the framework. Business processes are monitored, and event logs obtained from the database of information systems. Subsequently, \revision{process mining}-based feature extraction is applied to these event logs and tabular data input to dimensionality reduction to identify control-flow anomalies. We apply several \revision{conformance checking}-based and \revision{conformance checking}-independent framework techniques to publicly available datasets, simulated data of a case study from railways, and real-world data of a case study from healthcare. We show that the framework techniques implementing our approach outperform the baseline \revision{conformance checking}-based techniques while maintaining the explainable nature of \revision{conformance checking}.

In summary, the contributions of this paper are as follows.
\begin{itemize}
    \item{
        A novel \revision{process mining}-based feature extraction approach to support the development of competitive and explainable \revision{conformance checking}-based techniques for control-flow anomaly detection.
    }
    \item{
        A flexible and explainable framework for developing techniques for control-flow anomaly detection using \revision{process mining}-based feature extraction and dimensionality reduction.
    }
    \item{
        Application to synthetic and real-world datasets of several \revision{conformance checking}-based and \revision{conformance checking}-independent framework techniques, evaluating their detection effectiveness and explainability.
    }
\end{itemize}

The rest of the paper is organized as follows.
\begin{itemize}
    \item Section \ref{sec:related_work} reviews the existing techniques for control-flow anomaly detection, categorizing them into \revision{conformance checking}-based and \revision{conformance checking}-independent techniques.
    \item Section \ref{sec:abccfe} provides the preliminaries of \revision{process mining} to establish the notation used throughout the paper, and delves into the details of the proposed \revision{process mining}-based feature extraction approach with alignment-based \revision{conformance checking}.
    \item Section \ref{sec:framework} describes the framework for developing \revision{conformance checking}-based and \revision{conformance checking}-independent techniques for control-flow anomaly detection that combine \revision{process mining}-based feature extraction and dimensionality reduction.
    \item Section \ref{sec:evaluation} presents the experiments conducted with multiple framework and baseline techniques using data from publicly available datasets and case studies.
    \item Section \ref{sec:conclusions} draws the conclusions and presents future work.
\end{itemize}
\section{Preliminaries}
\label{sec:prelim}
\label{sec:term}
We define the key terminologies used, primarily focusing on the hidden states (or activations) during the forward pass. 

\paragraph{Components in an attention layer.} We denote $\Res$ as the residual stream. We denote $\Val$ as Value (states), $\Qry$ as Query (states), and $\Key$ as Key (states) in one attention head. The \attlogit~represents the value before the softmax operation and can be understood as the inner product between  $\Qry$  and  $\Key$. We use \Attn~to denote the attention weights of applying the SoftMax function to \attlogit, and ``attention map'' to describe the visualization of the heat map of the attention weights. When referring to the \attlogit~from ``$\tokenB$'' to  ``$\tokenA$'', we indicate the inner product  $\langle\Qry(\tokenB), \Key(\tokenA)\rangle$, specifically the entry in the ``$\tokenB$'' row and ``$\tokenA$'' column of the attention map.

\paragraph{Logit lens.} We use the method of ``Logit Lens'' to interpret the hidden states and value states \citep{belrose2023eliciting}. We use \logit~to denote pre-SoftMax values of the next-token prediction for LLMs. Denote \readout~as the linear operator after the last layer of transformers that maps the hidden states to the \logit. 
The logit lens is defined as applying the readout matrix to residual or value states in middle layers. Through the logit lens, the transformed hidden states can be interpreted as their direct effect on the logits for next-token prediction. 

\paragraph{Terminologies in two-hop reasoning.} We refer to an input like “\Src$\to$\brga, \brgb$\to$\Ed” as a two-hop reasoning chain, or simply a chain. The source entity $\Src$ serves as the starting point or origin of the reasoning. The end entity $\Ed$ represents the endpoint or destination of the reasoning chain. The bridge entity $\Brg$ connects the source and end entities within the reasoning chain. We distinguish between two occurrences of $\Brg$: the bridge in the first premise is called $\brga$, while the bridge in the second premise that connects to $\Ed$ is called $\brgc$. Additionally, for any premise ``$\tokenA \to \tokenB$'', we define $\tokenA$ as the parent node and $\tokenB$ as the child node. Furthermore, if at the end of the sequence, the query token is ``$\tokenA$'', we define the chain ``$\tokenA \to \tokenB$, $\tokenB \to \tokenC$'' as the Target Chain, while all other chains present in the context are referred to as distraction chains. Figure~\ref{fig:data_illustration} provides an illustration of the terminologies.

\paragraph{Input format.}
Motivated by two-hop reasoning in real contexts, we consider input in the format $\bos, \text{context information}, \query, \answer$. A transformer model is trained to predict the correct $\answer$ given the query $\query$ and the context information. The context compromises of $K=5$ disjoint two-hop chains, each appearing once and containing two premises. Within the same chain, the relative order of two premises is fixed so that \Src$\to$\brga~always precedes \brgb$\to$\Ed. The orders of chains are randomly generated, and chains may interleave with each other. The labels for the entities are re-shuffled for every sequence, choosing from a vocabulary size $V=30$. Given the $\bos$ token, $K=5$ two-hop chains, \query, and the \answer~tokens, the total context length is $N=23$. Figure~\ref{fig:data_illustration} also illustrates the data format. 

\paragraph{Model structure and training.} We pre-train a three-layer transformer with a single head per layer. Unless otherwise specified, the model is trained using Adam for $10,000$ steps, achieving near-optimal prediction accuracy. Details are relegated to Appendix~\ref{app:sec_add_training_detail}.


% \RZ{Do we use source entity, target entity, and mediator entity? Or do we use original token, bridge token, end token?}





% \paragraph{Basic notations.} We use ... We use $\ve_i$ to denote one-hot vectors of which only the $i$-th entry equals one, and all other entries are zero. The dimension of $\ve_i$ are usually omitted and can be inferred from contexts. We use $\indicator\{\cdot\}$ to denote the indicator function.

% Let $V > 0$ be a fixed positive integer, and let $\vocab = [V] \defeq \{1, 2, \ldots, V\}$ be the vocabulary. A token $v \in \vocab$ is an integer in $[V]$ and the input studied in this paper is a sequence of tokens $s_{1:T} \defeq (s_1, s_2, \ldots, s_T) \in \vocab^T$ of length $T$. For any set $\mathcal{S}$, we use $\Delta(\mathcal{S})$ to denote the set of distributions over $\mathcal{S}$.

% % to a sequence of vectors $z_1, z_2, \ldots, z_T \in \real^{\dout}$ of dimension $\dout$ and length $T$.

% Let $\mU = [\vu_1, \vu_2, \ldots, \vu_V]^\transpose \in \real^{V\times d}$ denote the token embedding matrix, where the $i$-th row $\vu_i \in \real^d$ represents the $d$-dimensional embedding of token $i \in [V]$. Similarly, let $\mP = [\vp_1, \vp_2, \ldots, \vp_T]^\transpose \in \real^{T\times d}$ denote the positional embedding matrix, where the $i$-th row $\vp_i \in \real^d$ represents the $d$-dimensional embedding of position $i \in [T]$. Both $\mU$ and $\mP$ can be fixed or learnable.

% After receiving an input sequence of tokens $s_{1:T}$, a transformer will first process it using embedding matrices $\mU$ and $\mP$ to obtain a sequence of vectors $\mH = [\vh_1, \vh_2, \ldots, \vh_T] \in \real^{d\times T}$, where 
% \[
% \vh_i = \mU^\transpose\ve_{s_i} + \mP^\transpose\ve_{i} = \vu_{s_i} + \vp_i.
% \]

% We make the following definitions of basic operations in a transformer.

% \begin{definition}[Basic operations in transformers] 
% \label{defn:operators}
% Define the softmax function $\softmax(\cdot): \real^d \to \real^d$ over a vector $\vv \in \real^d$ as
% \[\softmax(\vv)_i = \frac{\exp(\vv_i)}{\sum_{j=1}^d \exp(\vv_j)} \]
% and define the softmax function $\softmax(\cdot): \real^{m\times n} \to \real^{m \times n}$ over a matrix $\mV \in \real^{m\times n}$ as a column-wise softmax operator. For a squared matrix $\mM \in \real^{m\times m}$, the causal mask operator $\mask(\cdot): \real^{m\times m} \to \real^{m\times m}$  is defined as $\mask(\mM)_{ij} = \mM_{ij}$ if $i \leq j$ and  $\mask(\mM)_{ij} = -\infty$ otherwise. For a vector $\vv \in \real^n$ where $n$ is the number of hidden neurons in a layer, we use $\layernorm(\cdot): \real^n \to \real^n$ to denote the layer normalization operator where
% \[
% \layernorm(\vv)_i = \frac{\vv_i-\mu}{\sigma}, \mu = \frac{1}{n}\sum_{j=1}^n \vv_j, \sigma = \sqrt{\frac{1}{n}\sum_{j=1}^n (\vv_j-\mu)^2}
% \]
% and use $\layernorm(\cdot): \real^{n\times m} \to \real^{n\times m}$ to denote the column-wise layer normalization on a matrix.
% We also use $\nonlin(\cdot)$ to denote element-wise nonlinearity such as $\relu(\cdot)$.
% \end{definition}

% The main components of a transformer are causal self-attention heads and MLP layers, which are defined as follows.

% \begin{definition}[Attentions and MLPs]
% \label{defn:attn_mlp} 
% A single-head causal self-attention $\attn(\mH;\mQ,\mK,\mV,\mO)$ parameterized by $\mQ,\mK,\mV \in \real^{{\dqkv\times \din}}$ and $\mO \in \real^{\dout\times\dqkv}$ maps an input matrix $\mH \in \real^{\din\times T}$ to
% \begin{align*}
% &\attn(\mH;\mQ,\mK,\mV,\mO) \\
% =&\mO\mV\layernorm(\mH)\softmax(\mask(\layernorm(\mH)^\transpose\mK^\transpose\mQ\layernorm(\mH))).
% \end{align*}
% Furthermore, a multi-head attention with $M$ heads parameterized by $\{(\mQ_m,\mK_m,\mV_m,\mO_m) \}_{m=1}^M$ is defined as 
% \begin{align*}
%     &\Attn(\mH; \{(\mQ_m,\mK_m,\mV_m,\mO_m) \}_{m\in[M]}) \\ =& \sum_{m=1}^M \attn(\mH;\mQ_m,\mK_m,\mV_m,\mO_m) \in \real^{\dout \times T}.
% \end{align*}
% An MLP layer $\mlp(\mH;\mW_1,\mW_2)$ parameterized by $\mW_1 \in \real^{\dhidden\times \din}$ and $\mW_2 \in \real^{\dout \times \dhidden}$ maps an input matrix $\mH = [\vh_1, \ldots, \vh_T] \in \real^{\din \times T}$ to
% \begin{align*}
%     &\mlp(\mH;\mW_1,\mW_2) = [\vy_1, \ldots, \vy_T], \\ \text{where } &\vy_i = \mW_2\nonlin(\mW_1\layernorm(\vh_i)), \forall i \in [T].
% \end{align*}

% \end{definition}

% In this paper, we assume $\din=\dout=d$ for all attention heads and MLPs to facilitate residual stream unless otherwise specified. Given \Cref{defn:operators,defn:attn_mlp}, we are now able to define a multi-layer transformer.

% \begin{definition}[Multi-layer transformers]
% \label{defn:transformer}
%     An $L$-layer transformer $\transformer(\cdot): \vocab^T \to \Delta(\vocab)$ parameterized by $\mP$, $\mU$, $\{(\mQ_m^{(l)},\mK_m^{(l)},\mV_m^{(l)},\mO_m^{(l)})\}_{m\in[M],l\in[L]}$,  $\{(\mW_1^{(l)},\mW_2^{(l)})\}_{l\in[L]}$ and $\Wreadout \in \real^{V \times d}$ receives a sequence of tokens $s_{1:T}$ as input and predict the next token by outputting a distribution over the vocabulary. The input is first mapped to embeddings $\mH = [\vh_1, \vh_2, \ldots, \vh_T] \in \real^{d\times T}$ by embedding matrices $\mP, \mU$ where 
%     \[
%     \vh_i = \mU^\transpose\ve_{s_i} + \mP^\transpose\ve_{i}, \forall i \in [T].
%     \]
%     For each layer $l \in [L]$, the output of layer $l$, $\mH^{(l)} \in \real^{d\times T}$, is obtained by 
%     \begin{align*}
%         &\mH^{(l)} =  \mH^{(l-1/2)} + \mlp(\mH^{(l-1/2)};\mW_1^{(l)},\mW_2^{(l)}), \\
%         & \mH^{(l-1/2)} = \mH^{(l-1)} + \\ & \quad \Attn(\mH^{(l-1)}; \{(\mQ_m^{(l)},\mK_m^{(l)},\mV_m^{(l)},\mO_m^{(l)}) \}_{m\in[M]}), 
%     \end{align*}
%     where the input $\mH^{(l-1)}$ is the output of the previous layer $l-1$ for $l > 1$ and the input of the first layer $\mH^{(0)} = \mH$. Finally, the output of the transformer is obtained by 
%     \begin{align*}
%         \transformer(s_{1:T}) = \softmax(\Wreadout\vh_T^{(L)})
%     \end{align*}
%     which is a $V$-dimensional vector after softmax representing a distribution over $\vocab$, and $\vh_T^{(L)}$ is the $T$-th column of the output of the last layer, $\mH^{(L)}$.
% \end{definition}



% For each token $v \in \vocab$, there is a corresponding $d_t$-dimensional token embedding vector $\embed(v) \in \mathbb{R}^{d_t}$. Assume the maximum length of the sequence studied in this paper does not exceed $T$. For each position $t \in [T]$, there is a corresponding positional embedding  







\newcommand{\xmark}{\ensuremath{\boldsymbol{\times}}}

\section{Graph Subsampling}\label{sec:graph_drop}

Here we present our approach for subsampling graphs in a dataset while preserving the performance of a downstream GNN.  Our method relies on the following Lipschitz bound that relates a GNN's stability to TMD\footnote{While Theorem 8 in \citet{Chuang22:Tree} is stated for GINs, they extend it to other GNN architectures.}.

\begin{theorem}[Informal, Theorem 8 by \citet{Chuang22:Tree}, restated]\label{thm:stable} {There exists a weight function $w$ such that} for any $(L-1)$-layer message-passing GNN with readout $h: \cG \mapsto \R^d$ and layer-wise Lipschitz constants $\phi_\ell$, the following holds for any two graphs $G_a, G_b$: \[\norm{h(G_a) - h(G_b)}\leq \cTMD_{w}^{L} (G_a, G_b) \cdot \prod_{\ell=1}^{L-1} \phi_\ell.\] 
\end{theorem}

Let $X= \{G_1, ..., G_n\}$ be our graph dataset. Given a budget $k$, we aim to choose a subset $\cI \subset [n]$ of size $k$ such that a GNN trained on the subsampled set {$\{G_i\}_{i \in \cI}$} obtains similar readouts and loss as if it were trained on the entire set $X$.
To ensure $\cI$ is representative of $X$, we select $\cI$ by optimizing a medoids objective, which quantifies how well each graph $G_i \in X$ is represented by at least one graph in $\cI$. 

\begin{definition}[Medoids objective] Let $X= \{G_1, ..., G_n\}$ be a graph dataset and $\cI \subset [n]$ with $\absInline{\cI} \leq k$. The medoids objective value of $\cI$ with respect to distance $D : X \times X \to \R_{\geq 0}$ is defined by 
\begin{align*}
    f_D(\cI; X) = \frac{1}{\abs{X}}\sum\nolimits_{i \in [n]} \min_{j \in\cI} D(G_i, G_j).
\end{align*}
For $j \in \cI$, let $\tau_j$ be the number of graphs closest to $G_j$:
\[\tau_j \defeq \abs{\{i \in [n]: D(G_i, G_j) < D(G_i, G_k),~\forall k \neq j\}},\]
breaking ties arbitrarily. We call $\tau_j$ the size of cluster $j$. 
\end{definition}

The provide a lemma that bounds the difference in a GNN's loss on the (weighted) subsampled dataset $\cI \subset X$ and the full dataset $X$ in terms of the medoids objective with respect to TMD. Proofs of results in this section are in Appendix~\ref{sec:apx_graph_drop}.

\begin{restatable}{lemma}{tmdgen}\label{lemma:tmd-gen} Let $\cH$ be a hypothesis class of $(L-1)$-layer GNNs $h: \cG \to \R^d$, where the $\ell$-th layer has Lipschitz constant at most $\Phi_\ell$. Let $\cL: \R^d \to \R$ be an $M$-Lipschitz loss function, let {$y = (y_1, ..., y_n)$ be labels for $X$}, and $\cI$ be a subset of $[n]$. For any GNN $h \in \cH$ and graph $G \in \cG$,  
\begin{align}
    \Big\lvert{\frac{1}{n} \sum\nolimits_{i \in \cI} \tau_{i} \cL(h(G_i); y_i) - \frac{1}{n}\sum\nolimits_{i \in [n]}\cL(h(G_i); y_i)}\Big\rvert \nonumber \leq  M \cdot f_{\cTMD_w^L}(\cI; X) \cdot {\prod\nolimits_{\ell \in [L-1]} \Phi_\ell} . \label{eq:mediods_bound}
\end{align}
\end{restatable}
\begin{hproof} We use the Lipschitz constant $M$ of $\cL$ and Theorem~\ref{thm:stable} to bound the average deviation in the loss on $X$ from the loss on $\cI$ (reweighted according to the $\tau_i$'s). 
\end{hproof}

If $f_{\cTMD_w^L}(\cI; X)$ is small, then each $G_i \in X$ is close to some subsampled graph, which keeps the overall loss on the subsampled set similar to the loss on $X$. If we minimize loss over $\cI$, the next corollary shows that the resulting hypothesis will incur only a small increase in loss on $X$.
\begin{restatable}{corollary}{erm}\label{cor:ERM}
    Suppose $M \cdot \prod_{\ell \in [L-1]} \Phi_\ell \leq c$, $f_{\cTMD_w^L}(\cI; X) \leq \epsilon$, and $\hat{h} \in \cH$ minimizes the weighted loss $\sum_{i \in \cI} \tau_{i} \cL(h(G_i); y_i).$ Then $
    \frac{1}{n}\sum\nolimits_{i \in [n]}\cL\big(\hat{h}(G_i); y_i\big) \leq \min_{h \in \cH}\frac{1}{n}\sum\nolimits_{i \in [n]}\cL(h(G_i); y_i) + 2c\epsilon.$
\end{restatable}
% \paragraph{Graph subsampling.}
% \begin{figure}[t]
% \centering
%     \includegraphics[width=\textwidth]{figures/GraphSubsampling.png}
% \caption{Test accuracy versus fraction of graphs in the training set, subsampled with our approach and existing model-agnostic methods.}
% \label{fig:graph_subsampling}
% \end{figure}
%Remark~\ref{remark:tmd-gen-remark}
Hence, the additional training loss incurred by training on $\cI$ instead of $X$ is bounded by an error proportional to $f_{\cTMD_w^L}(\cI; X).$ This bound could be combined with prior work on the VC dimension of GNNs~\citep[e.g.,][]{scarselli2018vapnik} to obtain bounds on the \emph{test loss} as well.

Corollary~\ref{cor:ERM} motivates selecting $\cI$ to minimize $f_{\cTMD_w^L}(\cI; X).$
% \begin{definition}[Graph subsampling problem]\label{def:graph-subsampling} Let $X= \{G_1, ..., G_n\}$ be a set of graphs and $k$ be a budget. The \emph{graph subsampling problem} is to select a subset $\cI \subset [n]$ of $k$ graphs that minimizes the objective $f_{\cTMD_w^L}(\cI; X).$
% \end{definition}
Importantly, this optimization problem is independent of graph labels. Training a GNN on the subsampled graphs in only requires knowing the labels of those graphs. Additionally, our results make mild assumptions on the GIN architecture---we need only know its depth $L$. Though the $k$-medoids problem---and thus optimizing $f_{\cTMD_w^L}(\cI; X)$---is NP-hard~\citep{kazakovtsev2020application}, {there are} efficient approximation algorithms, for example Python's $k$-medoids algorithm from the sklearn package \citep{sklearn_api}, which we use in our experiments.

\paragraph{Other pseudo-metrics.} Given these results, a natural question is whether the TMD pseudo-metric is essential for the stability result in Theorem~\ref{thm:stable}. One might hope that other graph pseudo-metrics could also be used to bound GNN stability, thereby yielding analogs of Corollary~\ref{cor:ERM}.  We express this intuition as the following conjecture.

\begin{conjecture}\label{conj:TMD-not-needed} Let $D$ denote a graph pseudo-metric. There exists a function $C : \{\phi_\ell\}_{\ell=1}^{L-1} \to \R_{> 0}$ such that for any $(L-1)$-layer message-passing GNN with readout function $h: \cG \to \R^d$ and layer-wise Lipschitz constants $\phi_\ell$, the following holds for any two graphs $G_a, G_b$: $\norm{h(G_a) - h(G_b)} \leq C\paren{\phi_1, ..., \phi_{L-1}} \cdot D(G_a, G_b)$. 
\end{conjecture}

Surprisingly, we show that this conjecture is \emph{provably} false for four of the most widely used graph pseudo-metrics: Weisfeiler–Lehman (WL) \citep{shervashidze2011weisfeiler}, WL-Optimal Transport (WL-OA) \citep{kriege2016valid}, shortest-paths (SP) \citep{borgwardt2005shortest}, and graphlet sampling (GS) \citep{shervashidze2009efficient}. 

\begin{restatable}{theorem}{tmdneeded}\label{thm:TMD-is-needed}
    Let $D^L$ denote any of the following pseudo-metrics: GS, $L$-layer WL, $L$-layer WL-OA, or the $L$-layer SP. Then Conjecture~\ref{conj:TMD-not-needed} is false for $D = D^L$.
\end{restatable}

Consequently, our theoretical findings and the main result in Corollary~\ref{cor:ERM} rely \emph{crucially} on TMD as the underlying pseudo-metric. See Appendix~\ref{sec:thm-tmd-is-needed} for details.
\section{Node Subsampling}\label{sec:node_drop}

In this section, we present our approach for subsampling nodes of a graph dataset so that the performance of a downstream GNN trained on the subsample is preserved. Proofs for results in this section are in Appendix~\ref{sec:apx-node}.

Suppose we have a graph dataset $X= \{G_1, ..., G_n\}$ where $G_i = (V_i, E_i)$, and a node budget $k$. For a subset $S_i \subseteq V_i$ of $k$ nodes, let $G_i[S_i]$ denote the subgraph of $G_i$ induced by $S_i.$ Our goal is to {select these subsets so} that a GNN produces similar readouts on the original dataset $X$ and on the induced subgraphs $X' = \{G_1[S_1], ..., G_n[S_n]\}$. The following corollary shows that if the GNN layers have small Lipschitz constants and the distances $\cTMD(G_i[S_i], G_i)$ are small, then training over $X'$ yields a nearly optimal predictor over $X$. 

\begin{restatable}{corollary}{corrermfirst}\label{corr:erm-first} Let $\cH$ be a set of $(L-1)$-layer GNNs $h: \cG \to \R^d$, where the $\ell$-th layer has Lipschitz constant at most $\Phi_\ell$. Let $\cL: \R^d \to \R$ be an $M$-Lipschitz loss function. 

Suppose that $M \cdot \prod_{\ell \in [L-1]} \Phi_\ell \leq c$ and $\cTMD_w^L(G_i, G_i[S_i]) \leq \epsilon_i$ for all $i \in [n]$. Finally, let $\hat{h} \in \cH$ be a GNN with minimum loss over the subsampled training set with respect to the training labels $y_1, \dots, y_n$: $\hat{h} = \argmin\nolimits_{h \in \cH}\sum\nolimits_{i \in [n]} \cL(h(G_i[S_i]); y_i).$
Then $\hat{h}$ has near-optimal loss over the original training set:
\begin{align*}
    \frac{1}{n}\sum\nolimits_{i \in [n]} \cL\Big({\hat{h}(G_i); y_i}\Big) \leq  \min_{h \in \cH} \frac{1}{n}\sum\nolimits_{i \in [n]} \cL(h(G_i); y_i) + \frac{2c}{n}\sum\nolimits_{i \in [n]} \epsilon_i.
\end{align*}
\end{restatable}

Thus, the additional {loss incurred by} training on $X'$ rather than $X$ is proportional to the average of {$\cTMD_{w}^L(G_i, G_i[S_i])$}, so for each $G \in X$, we would ideally solve:
\begin{equation}
    \min_{S \subset V: \abs{S} \leq k} \cTMD_w^L(G, G[S]).\label{eq:node-subsampling}
\end{equation}

We face {two} key challenges in doing so. First, the number of candidate subsets $\{S \subset V: \abs{S} \leq {k}\}$ grows exponentially.  Indeed, solving \eqref{eq:node-subsampling} is NP-hard (see Appendix~\ref{sec:hardness}).
 We therefore 
 restrict
 our search to an appropriately chosen feasible set $\cS$.
In our experiments, we combine well-motivated, fast heuristics for selecting small candidate sets $\cS$ based on prior analyses \citep{salha2022degeneracy, razin2023ability, alimohammadi2023local}. (Details are in Appendix~\ref{sec:heuristic}.)

The next challenge is that computing $\cTMD_w^L(G, G[S])$ for each $S \in \cS$ can be expensive: the algorithm by \citet{Chuang22:Tree} takes $\cO(L|V|^4)$ time. To address this, we prove that, surprisingly, computing $\cTMD_w^L(G, G[S])$ is equivalent to a much simpler optimization problem. 

\begin{restatable}{theorem}{nodesamplingrelaxed}\label{thm:node-subsampling-relaxed} Let $G = (V, E, f)$ be a graph and $\cS$ be a set of candidate node subsets. Then \begin{align}\label{eq:final-opt-problem}
    \argmin_{S \in \cS} \cTMD_w^L(G, G[S]) = \argmax_{S \in \cS} \cTreeNorm{G[S]}.
\end{align}
\end{restatable}

We prove Theorem~\ref{thm:node-subsampling-relaxed} in Section~\ref{sec:proof} and in Section~\ref{sec:alg}, we provide a linear-time algorithm for computing tree norms and, thereby, the solution to Equation~\eqref{eq:final-opt-problem}.

\begin{restatable}{theorem}{algorithm}\label{thm:node-subsampling-relaxed-algorithm} Given a graph $G = (V, E, f)$, Algorithm~\ref{alg:1} computes $\cTreeNorm{G}$ in $\bigO(\abs{E}L)$ time.
\end{restatable}

\subsection{TMD Between Graphs and Subgraphs}\label{sec:proof}

In this section, we sketch our proof of Theorem~\ref{thm:node-subsampling-relaxed}. In doing so, we prove several properties of the TMD which may be of independent interest given the broad applications of TMD to out-of-distribution generalization. Our analysis crucially relies on the following lemma, which shows that an appealing simple \emph{identity} transport plan can be used to compute $\cTMD$ between a graph and its induced subgraphs.

\begin{restatable}{lemma}{generaldecomposition}\label{lemma:general-decomposition} Let {$w: \N \to \R$ be a weight function and} $G = (V, E, f)$ be a graph.
For $S \subseteq V$, define the identity transportation plan $\bm{I}$ that maps $T_v(G)$ to $T_v(G[S])$ if $v \in S$ and to $\emptyset$ otherwise. Then $\bm{I}$ is an optimal transport plan for
\begin{align}\label{eq:ot-problem}
    \cTMD_w^L(G, G[S]) := \cOTbar\paren{\cT_G^L, \cT_{G[S]}^L}.
\end{align}
Consequently, we can decompose $\cTMD_w^L(G, G[S])$ as: 
\begin{align}
    \cTMD_w^L(G, G[S]) \label{eq:decomposition} &= \underbrace{\sum_{v \notin S} \|f^v\|}_{\textnormal{Deleted nodes' features}} +\underbrace{w(L-1)\sum_{v \notin S} \cOTbar
    \left(\cT_v\left(T_v^L(G)\right),\,\emptyset\right)}_{\textnormal{Cost of removing deleted nodes' trees}}\\
    &+\;
    \underbrace{w(L-1)\sum_{v \in S} \cOTbar
    \left(\cT_v\left(T_v^L(G)\right),\, \cT_v\left(T_v^L(G[S])\right)\right)}_{\textnormal{Cost of matching retained nodes' trees}}.\nonumber
\end{align}
\end{restatable}
\begin{hproof} We prove the statement by induction on $L$. In the base case $(L= 1)$, by definition, $\cOT(\cT_G^1, \cT_{G[S]}^1)$ 
equals the sum of features of the nodes that are in $G$ but not in $G[S]$, i.e., $\cOT(\cT_G^1, \cT_{G[S]}^1) = \sum_{v \notin S} \Vert f^v \Vert = \langle C, \bm{I} \rangle$, so the base case holds. Inductively, we use the TD's recursive formulation to reduce OT problems of depth $L$ to those of depth $L-1$.
\end{hproof}

Next, we use Lemma~\ref{lemma:general-decomposition} to characterize how TMD accumulates as we remove a sequence of nodes. Lemma~\ref{lemma:chain-simple} shows that this accumulation is \emph{additive} under certain conditions, which is unexpected given the TMD's combinatorial nature.


\begin{restatable}{lemma}{finegraineddecomp}\label{lemma:chain-simple} 
For $T \subset S \subset V$, $\cTMD_w^L(G, G[S \setminus T ]) = \cTMD_w^L(G, G[S]) + \cTMD_w^L(G[S], G[S\setminus T])$.  
\end{restatable}

 This equality is noteworthy because the triangle inequality only guarantees $\cTMD_w^L(G, G[S \setminus T ]) \leq \cTMD_w^L(G, G[S]) + \cTMD_w^L(G[S], G[S\setminus T])$. However, we show that the triangle inequality is \emph{always} tight when $T \subset S \subset V$.

\begin{hproof}[Proof sketch of Lemma~\ref{lemma:chain-simple}] We apply Lemma~\ref{lemma:general-decomposition} {inductively over $L$}. When $L=1$, the second and third summands of \eqref{eq:decomposition} equal 0: $\cT_G^1$ and $\cT_{G[S]}^1$ are multisets of depth 1, so for any $v \in V$, $T_v^1(G) = \{v\}$ and thus $\cT_v(T_v^1(G)) = \emptyset$. Likewise, for any $v \in S$, $\cT_v(T_v^1(G[S])) = \emptyset$. Therefore, 
\begin{align*}
    \cTMD_w^1(G, G[S \setminus T]) = \sum\nolimits_{v \notin (S \setminus T) } \norm{f^v} = \sum\nolimits_{v\notin S} \norm{f^v} + \sum\nolimits_{v \in T} \norm{f^v}, 
\end{align*} 
because for $T \subset S$, $\{v \in V : v \notin (S \setminus T) \} = \{v \in V : v \notin S\} \sqcup T$, where $\sqcup$ denotes the \emph{disjoint union}.
Similarly, 
\begin{align*}
    \cTMD_w^1(G, G[S]) = \sum\nolimits_{v \notin S} \norm{f^v} \text{\,\,\,\,and\,\,\,\,} \cTMD_w^1(G[S], G[S \setminus T]) = \sum\nolimits_{v \in T} \norm{f^v}, 
\end{align*} 
thus verifying the base case. The intuition is similar for $L>1$ but requires
care to handle the recursive OT terms.
\end{hproof}

Finally, we can prove Theorem~\ref{thm:node-subsampling-relaxed}.

\begin{proof}[Proof of Theorem~\ref{thm:node-subsampling-relaxed}] Let $S \in \cS$. By Lemma~\ref{lemma:chain-simple} with $T = S$, $\cTreeNorm{G} =  \cTMD_w^L(G, \emptyset) = \cTMD_w^L(G, G[S]) + \cTMD_w^L(G[S], \emptyset) = \cTMD_w^L(G, G[S]) + \cTreeNorm{G[S]}.$  Thus, 
\begin{align*}
    \max_{\substack{S \in \cS \\ \abs{S} = k}}
 \cTreeNorm{G[S]} \equiv \max_{\substack{S \in \cS \\ \abs{S} = k}} \cTreeNorm{G} - \cTMD_w^L(G, G[S]) \equiv \min_{\substack{S \in \cS \\ \abs{S} = k}} \cTMD_w^L(G, G[S]). \tag*{\qedhere} 
\end{align*}
\end{proof}





\subsection{Faster Algorithm for Tree Norms}\label{sec:alg}
We next leverage Lemma~\ref{lemma:general-decomposition} to obtain  Algorithm~\ref{alg:1}, a faster algorithm for computing $\cTreeNorm{G}$ for $G = (V, E, f).$ The algorithm by \citet{Chuang22:Tree} requires $\bigO(L \abs{V}^4)$ time, whereas ours {has runtime} $\bigO(L |E|)$. Algorithm~\ref{alg:1} is based on the fact that by Lemma~\ref{lemma:general-decomposition}, $\cTreeNorm{G}$ is essentially a weighted sum of the number of vertices in all of its depth-$L$ computation trees, which Algorithm~\ref{alg:1} {computes}. Its runtime is dominated by the cost of $L$ matrix-vector {multiplies} with the adjacency matrix, which takes $\bigO(|E|)$-time. Proofs of correctness and runtime are in Appendix~\ref{sec:apx-node}. \\

\RestyleAlgo{ruled}
\SetKwComment{Comment}{/* }{ */}
\begin{algorithm2e}[H]
\caption{\text{TreeNorm}($G, L, w$)}\label{alg:1}
\KwInput{Graph $G = (V, E, f)$ {with adjacency matrix $A$}, weights $w: \{1, ..., L-1\} \to \R_+, L\geq 1$}
Define $x \in \R^{|V|}$ such that $x_v = \norm{f^v}$ \label{line:x}\; 
Initialize $z^{(0)} = x$\; 
\For{$\ell \in [{L-1}]$}{
    $z^{(\ell)} \gets A z^{(\ell-1)}$\label{line:z}\;
}
$b \gets z^{(0)} + \sum_{\ell = 1}^{L-1} \paren{\prod_{t=1}^\ell w(L-t)} \cdot z^{(\ell)}$\;  
\Return{$\normInline{b}_1$}
\end{algorithm2e}



\section{Experiments}\label{sec:experiments}

\paragraph{Graph subsampling.} We compare our approach with several graph subsampling and condensation methods. KiDD \citep{kidd} and DosCond \citep{jin2022condensing} condense datasets into small synthetic graphs but are not label-agnostic. For a fair comparison, we modify them to operate in a label-agnostic manner by removing their label-aware components in the ``without labels'' section of Tables~\ref{tab:all-results-graph-subsampling1} and \ref{tab:all-results-graph-subsampling2}, as detailed in Appendix~\ref{sec:additional-background-experiments-vallabel}. Since these methods do not support randomized train/test splits and rely on fixed initializations, we do not randomize over splits but do randomize over GNN parameter initializations. MIRAGE \citep{mirage} is another condensation method that subsamples computation trees, but persistent runtime errors in its implementation prevented thorough benchmarking, an issue also reported by \citet{sun2024gc}. We successfully ran MIRAGE on two datasets for certain sampling percentages, as detailed in Appendix~\ref{sec:additional-background-experiments-mirage}.  In addition to these methods, we construct three additional baselines. \emph{WL} applies $k$-medoids clustering using the widely-used Weisfeiler-Lehman (WL) kernel \citep{shervashidze2011weisfeiler}. \emph{Random} performs uniform graph subsampling. \emph{Feature} applies $k$-medoids clustering based only on node features, ignoring graph structure (see Appendix~\ref{sec:additional-background-experiments-feature}). This baseline is particularly useful for evaluating the impact of structure-aware methods such as TMD.  We use an 80/20 train/test split and select between 1\% and 10\% of the training graphs. A GNN is trained on the selected graphs, and we report the average test performance with 95\% confidence intervals over 20 trials with random train/test splits and network initializations.  

Our findings are summarized in Tables~\ref{tab:all-results-graph-subsampling1} and \ref{tab:all-results-graph-subsampling2}. Following \citet{jin2022condensing}, we measure test AUC-ROC on OGBG datasets and classification test accuracy on the remaining datasets. ``Wins'' count how often a method outperforms others, while ``Fails'' count how often a method falls below 50\% accuracy, as all tasks involve binary classification.  TMD consistently ranks first or second across nearly all datasets and subsampling percentages, except on NCI1, where no method outperforms Random. While Random achieves strong performance on NCI1, it performs poorly on most other datasets, a trend also observed with DosCond, which achieves strong results on OGBG-MOLBACE but struggles without labels. Overall, TMD attains the highest total wins across datasets, as shown in Table~\ref{tab:wins-fails-graph}. Additionally, while DosCond and KiDD exhibit high variance, particularly on PROTEINS, TMD maintains stable performance across different subsampling percentages. This consistency is reflected in its low fail count in Table~\ref{tab:wins-fails-graph}, underscoring the robustness of our approach.  

\begin{table}[t]
\caption{Summary of graph and node subsampling performance. The ``Wins" column counts the number of times each method achieves the best test performance across datasets and sampling percentages. The ``Fails" column counts instances where test performance falls below 50\% (worse than random chance). No fails were observed for node subsampling. TMD achieves the highest number of wins across both tasks, while the strong performance of Random is primarily observed on the NCI1 dataset. Full results are presented in Tables~\ref{tab:all-results-graph-subsampling1}, \ref{tab:all-results-graph-subsampling2}, and \ref{tab:combined-results-nodes}.}
\centering
\label{tab:summary}
\begin{tabular}{cc} 
\begin{subtable}{0.48\textwidth}
    \centering
    \caption{Graph subsampling}
    \label{tab:wins-fails-graph}
    \begin{tabular}{l | c | c}
    \hline
    \textbf{Method} & \textbf{Wins} $\uparrow$ & \textbf{Fails} $\downarrow$ \\
    \hline
    DosCond & 11 & 20 \\
    KiDD & 7  & 7 \\
    Feature & 13  & 3 \\
    WL & 8 & 4 \\
    Random & 18 & 7 \\
    \textbf{TMD} & \textbf{23} & \textbf{2} \\
    \hline
    \end{tabular}
\end{subtable}
\begin{subtable}{0.48\textwidth}
    \centering
    \caption{Node subsampling}
    \label{tab:wins-fails-node}
    \begin{tabular}{l | c }
    \hline
    \textbf{Method} & \textbf{Wins} $\uparrow$  \\
    \hline
    RW & 15 \\
    k-cores & 5  \\
    Random & 16 \\
    \textbf{TMD} & \textbf{25} \\
    \hline
    \end{tabular}
\end{subtable}
\end{tabular}
\end{table}


\paragraph{Node subsampling.} To our knowledge, no existing methods focus on node subsampling for graph classification, so we adapt two related approaches as benchmarks. \emph{K-cores}, proposed by \citet{razin2023ability}, is a node selection heuristic based on $k$-core decomposition, which identifies structurally important nodes by iteratively pruning low-degree nodes. \emph{RW}, introduced by \citet{salha2022degeneracy}, is a random-walk-based heuristic originally designed for subgraph sampling in large graph autoencoders. Additionally, we compare against \emph{Random}, which performs uniform node subsampling.  For our proposed node subsampling method (Theorems~\ref{thm:node-subsampling-relaxed} and~\ref{thm:node-subsampling-relaxed-algorithm}), we construct the candidate set $\mathcal{S}$ using breadth-first search (BFS) trees, leveraging the fact that BFS preserves critical motifs in real-world networks~\citep{alimohammadi2023local}. We further augment $\mathcal{S}$ with subsets generated by the RW and k-cores heuristics. The final subset is then selected using TMD. Additional details are provided in Appendix~\ref{sec:heuristic}.  Datasets with relatively larger graphs, such as COX2, PROTEINS, and DD, are well-suited for node subsampling, whereas MUTAG, NCI1, OGBG-MOLBACE, and OGBG-MOLBBBP contain fewer than 35 nodes per graph. For completeness, we include results on MUTAG to demonstrate that our method remains competitive even on smaller graphs.  

Table~\ref{tab:combined-results-nodes} compares test accuracy across these methods for sampling fractions ranging from 10\% to 90\%. We report averages with 95\% confidence intervals over 20 trials with random neural network initializations and train/test splits. The ``W'' row counts the number of times a method outperforms others. TMD consistently ranks among the top two methods and achieves the best overall performance (Table~\ref{tab:wins-fails-node}).  \\

%\newpage
\begin{table}[H]
\centering
\caption{Graph subsampling performance across sampling percentages for TUDatasets, reported as mean ± confidence bar of accuracy (ACC) or area under the ROC (ROC-AUC). Best and second-best per column are in dark/light green, respectively. W is total wins per method.}
\vspace{.1in}
\label{tab:all-results-graph-subsampling1} 
\begin{subtable}{\textwidth}
\captionsetup[subtable]{aboveskip=0pt, belowskip=0pt}
\centering
\resizebox{1.0\textwidth}{!} & \textbf{2\%} & \textbf{3\%} & \textbf{4\%} & \textbf{5\%} & \textbf{6\%} & \textbf{7\%} & \textbf{8\%} & \textbf{9\%} & \textbf{10\%} & \textbf{W} & \textbf{F} \\
\hline
\multicolumn{12}{c}{\textbf{With labels}} \\
\hline
\textbf{Doscond} & 0.66±0.01 & 0.64±0.03 & 0.63±0.00 & 0.68±0.03 & 0.66±0.00 & 0.64±0.01 & 0.65±0.00 & 0.66±0.01 & 0.65±0.00 & 0.66±0.00 & - & 0 \\
\textbf{Kidd} & 0.66±0.03 & 0.68±0.03 & 0.71±0.03 & 0.69±0.03 & 0.70±0.03 & 0.68±0.03 & 0.66±0.02 & 0.64±0.02 & 0.69±0.03 & 0.68±0.03 &  - & 0\\
\hline
\multicolumn{12}{c}{\textbf{Without labels}} \\
\hline
\textbf{Doscond} & 0.48±0.03 & \cellcolor{green!80}{0.67±0.01} & 0.65±0.01 & 0.66±0.00 & 0.55±0.02 & 0.63±0.00 & 0.61±0.01 & 0.65±0.01 & 0.61±0.01 & 0.63±0.01 & 1 & 0\\
\textbf{Kidd} & \cellcolor{green!25}{0.60±0.02} & 0.58±0.02 & 0.55±0.01 & 0.56±0.02 & 0.60±0.02 & 0.61±0.03 & 0.60±0.02 & 0.62±0.02 & 0.58±0.01 & 0.56±0.02 & 0 & 0\\
\textbf{Feature} & 0.57±0.04 & 0.60±0.05 & 0.62±0.03 & 0.65±0.03 & \cellcolor{green!80}{0.67±0.02} & 0.67±0.04 & \cellcolor{green!80}{0.70±0.02} & \cellcolor{green!25}{0.67±0.04} & \cellcolor{green!25}{0.68±0.03} & \cellcolor{green!80}{0.73±0.01} & 3 & 0 \\
\textbf{WL} & \cellcolor{green!25}{0.60±0.03} & 0.62±0.04 & \cellcolor{green!25}{0.67±0.03} & \cellcolor{green!25}{0.68±0.02} & \cellcolor{green!25}{0.65±0.03} & \cellcolor{green!25}{0.68±0.02} & \cellcolor{green!25}{0.69±0.02} & \cellcolor{green!80}{0.69±0.04} & \cellcolor{green!80}{0.69±0.02} & 0.68±0.02 & 2 & 0\\
\textbf{Random} & \cellcolor{green!80}{0.62±0.02} & 0.62±0.02 & 0.63±0.03 & 0.65±0.02 & \cellcolor{green!80}{0.67±0.02} & 0.63±0.03 & 0.67±0.02 & \cellcolor{green!25}{0.67±0.02} & \cellcolor{green!80}{0.69±0.02} & 0.68±0.02 & 3 & 0\\
\textbf{TMD} & \cellcolor{green!80}{0.62±0.02} & \cellcolor{green!25}{0.63±0.03} & \cellcolor{green!80}{0.69±0.02} & \cellcolor{green!80}{0.69±0.02} & \cellcolor{green!80}{0.67±0.02} & \cellcolor{green!80}{0.70±0.02} & 0.67±0.01 & \cellcolor{green!80}{0.69±0.02} & \cellcolor{green!80}{0.69±0.03} & \cellcolor{green!25}{0.71±0.20} & 7 & 0\\
\hline
\end{tabular}
}
\subcaption{COX2. \vspace{-.4cm}}\label{tab:COX2}
\end{subtable}

\vspace{1em}


\begin{subtable}{\textwidth}
\centering
\resizebox{1.0\textwidth}{!}{%
\begin{tabular}{l | c | c | c | c | c | c | c | c | c | c | c | c}

\hline
\multicolumn{12}{c}{\textbf{With labels}} \\
\hline
\textbf{Doscond} & 0.74±0.05 & 0.77±0.02 & 0.78±0.00 & 0.79±0.01 & 0.78±0.01 & 0.78±0.00 & 0.78±0.02 & 0.76±0.03 & 0.78±0.01 & 0.78±0.01 & - & 0 \\
\textbf{Kidd} & 0.17±0.00 & 0.17±0.00 & 0.39±0.31 & 0.83±0.00 & 0.83±0.00 & 0.83±0.00 & 0.83±0.00 & 0.83±0.00 & 0.83±0.00 & 0.39±0.31 &  - & 0 \\
\hline
\multicolumn{12}{c}{\textbf{Without labels}} \\
\hline
\textbf{Doscond} & 0.44±0.14 & 0.22±0.00 & 0.25±0.03 & 0.25±0.03 & 0.77±0.02 & 0.39±0.08 & 0.65±0.09 & 0.22±0.00 & 0.22±0.00 & 0.21±0.00 & 0 & 8\\
\textbf{Kidd} & 0.17±0.00 & \cellcolor{green!80}{0.83±0.00} & 0.17±0.00 & 0.61±0.31 & \cellcolor{green!80}{0.83±0.00} & \cellcolor{green!80}{0.83±0.00} & 0.17±0.00 & 0.17±0.00 & 0.17±0.00 & 0.61±0.31 & 3 & 5\\
\textbf{Feature} & \cellcolor{green!25}{0.65±0.06} & 0.72±0.05 & \cellcolor{green!25}{0.71±0.04} & 0.72±0.02 & \cellcolor{green!25}{0.78±0.01} & 0.75±0.03 & \cellcolor{green!25}{0.74±0.04} & 0.73±0.03 & \cellcolor{green!80}{0.78±0.02} & 0.76±0.02 & 1 & 0 \\
\textbf{WL} & 0.57±0.06 & 0.70±0.04 & \cellcolor{green!80}{0.75±0.02} & \cellcolor{green!25}{0.76±0.02} & 0.77±0.02 & \cellcolor{green!25}{0.77±0.02} & \cellcolor{green!80}{0.77±0.02} & \cellcolor{green!25}{0.77±0.02} & \cellcolor{green!80}{0.78±0.02} & \cellcolor{green!80}{0.78±0.01} & 4 & 0\\
\textbf{Random} & \cellcolor{green!25}{0.65±0.07} & 0.70±0.04 & 0.69±0.05 & 0.68±0.06 & 0.76±0.04 & 0.74±0.04 & \cellcolor{green!25}{0.74±0.03} & 0.74±0.04 & 0.75±0.06 & \cellcolor{green!25}{0.77±0.03} & 0 & 0 \\
\textbf{TMD} & \cellcolor{green!80}{0.70±0.04} & \cellcolor{green!25}{0.76±0.02} & \cellcolor{green!80}{0.75±0.03} & \cellcolor{green!80}{0.78±0.02} & \cellcolor{green!25}{0.78±0.02} & \cellcolor{green!25}{0.77±0.01} & \cellcolor{green!80}{0.77±0.01} & \cellcolor{green!80}{0.78±0.02} & \cellcolor{green!25}{0.77±0.02} & \cellcolor{green!80}{0.78±0.02} & 6 & 0 \\
\hline
\end{tabular}
}
\subcaption{PROTEINS. \vspace{-.4cm}}\label{tab:proteins}
\end{subtable}

\vspace{1em}


\begin{subtable}{\textwidth}
\centering
\resizebox{1.0\textwidth}{!}{%
\begin{tabular}{l | c | c | c | c | c | c | c | c | c | c | c| c} 
\hline
\multicolumn{12}{c}{\textbf{With labels}} \\
\hline
\textbf{Doscond} & 0.71±0.02 & 0.69±0.01 & 0.69±0.03 & 0.70±0.03 & 0.67±0.02 & 0.69±0.03 & 0.70±0.02 & 0.70±0.02 & 0.70±0.01 & 0.69±0.04 & - & 0 \\
\textbf{Kidd} & 0.68±0.00 & 0.68±0.00 & 0.68±0.00 & 0.68±0.00 & 0.68±0.00 & 0.75±0.10 & 0.68±0.00 & 0.68±0.00 & 0.68±0.00 & 0.74±0.04 & - & 0 \\
\hline
\multicolumn{12}{c}{\textbf{Without labels}} \\
\hline
\textbf{Doscond} & \cellcolor{green!25}{0.67±0.00} & \cellcolor{green!80}{0.74±0.03} & \cellcolor{green!80}{0.73±0.01} & \cellcolor{green!80}{0.73±0.01} & \cellcolor{green!25}{0.71±0.01} & \cellcolor{green!25}{0.70±0.01} & 0.70±0.00 & 0.68±0.01 & 0.72±0.01 & 0.70±0.01 & 3 & 0\\
\textbf{Kidd} & \cellcolor{green!80}{0.68±0.00} & 0.68±0.00 & 0.68±0.00 & 0.44±0.17 & 0.68±0.00 & 0.68±0.00 & 0.32±0.00 & 0.70±0.03 & 0.32±0.00 & 0.32±0.00 & 1 & 0\\
\textbf{Feature} & 0.62±0.05 & 0.65±0.06 & 0.67±0.07 & 0.69±0.03 & 0.70±0.04 & 0.68±0.04 & \cellcolor{green!25}{0.72±0.04} & \cellcolor{green!25}{0.71±0.04} & \cellcolor{green!80}{0.74±0.04} & \cellcolor{green!80}{0.74±0.04} & 2 & 3 \\
\textbf{WL} & \cellcolor{green!25}{0.67±0.04} & 0.70±0.04 & \cellcolor{green!25}{0.71±0.02} & \cellcolor{green!25}{0.70±0.03} & \cellcolor{green!25}{0.71±0.03} & \cellcolor{green!25}{0.70±0.03} & 0.68±0.03 & 0.70±0.04 & 0.70±0.04 & \cellcolor{green!80}{0.74±0.04} & 1 & 0\\
\textbf{Random} & 0.62±0.05 & 0.65±0.06 & 0.67±0.07 & 0.69±0.03 & 0.70±0.04 & 0.68±0.04 & \cellcolor{green!25}{0.72±0.04} & \cellcolor{green!25}{0.71±0.04} & \cellcolor{green!80}{0.74±0.04} & \cellcolor{green!80}{0.74±0.04} & 2 & 0\\
\textbf{TMD} & 0.64±0.07 & \cellcolor{green!25}{0.71±0.04} & 0.70±0.03 & \cellcolor{green!80}{0.73±0.04} & \cellcolor{green!80}{0.76±0.04} & \cellcolor{green!80}{0.72±0.05} & \cellcolor{green!80}{0.75±0.04} & \cellcolor{green!80}{0.75±0.03} & \cellcolor{green!25}{0.73±0.04} & \cellcolor{green!25}{0.73±0.03} & 5 & 0\\
\hline
\end{tabular}
}
\subcaption{MUTAG. \vspace{-.4cm}}\label{tab:mutag}
\end{subtable}

\vspace{1em}

\begin{subtable}{\textwidth}
\centering
\resizebox{1.0\textwidth}{!}{%
\begin{tabular}{l | c | c | c | c | c | c | c | c | c | c | c | c}
\hline
\multicolumn{12}{c}{\textbf{With labels}} \\
\hline
\textbf{Doscond} & 0.56±0.01 & 0.56±0.02 & 0.57±0.02 & 0.58±0.02 & 0.58±0.02 & 0.56±0.03 & 0.59±0.02 & 0.61±0.02 & 0.61±0.02 & 0.61±0.02 & - & 0\\
\textbf{Kidd} & 0.60±0.01 & 0.60±0.01 & 0.61±0.02 & 0.61±0.01 & 0.61±0.02 & 0.62±0.02 & 0.62±0.01 & 0.62±0.01 & 0.62±0.01 & 0.62±0.01 & - & 0 \\
\hline
\multicolumn{12}{c}{\textbf{Without labels}} \\
\hline
\textbf{Doscond} & 0.50±0.02 & 0.51±0.03 & 0.49±0.03 & 0.52±0.03 & 0.52±0.02 & 0.54±0.02 & 0.54±0.02 & 0.54±0.02 & 0.51±0.04 & 0.55±0.03 & 0 & 0\\
\textbf{Kidd} & \cellcolor{green!80}{0.55±0.03} & \cellcolor{green!25}{0.56±0.03} & \cellcolor{green!25}{0.56±0.03} & \cellcolor{green!25}{0.54±0.05} & \cellcolor{green!25}{0.57±0.02} & \cellcolor{green!25}{0.57±0.02} & \cellcolor{green!25}{0.57±0.02} & \cellcolor{green!25}{0.57±0.02} & \cellcolor{green!25}{0.58±0.02} & \cellcolor{green!25}{0.58±0.02} & 1 & 0\\
\textbf{Feature} & \cellcolor{green!25}{0.51±0.01} & 0.53±0.01 & 0.52±0.01 & 0.53±0.02 & 0.53±0.02 & 0.53±0.02 & 0.54±0.02 & 0.53±0.01 & 0.56±0.03 & 0.54±0.03 & 0 & 0\\
\textbf{WL} & \cellcolor{green!25}{0.51±0.01} & 0.51±0.01 & 0.52±0.01 & 0.50±0.01 & 0.51±0.01 & 0.51±0.01 & 0.50±0.00 & 0.51±0.01 & 0.52±0.01 & 0.53±0.02 & 0 & 0\\
\textbf{Random} & \cellcolor{green!80}{0.55±0.02} & \cellcolor{green!80}{0.59±0.02} & \cellcolor{green!80}{0.62±0.01} & \cellcolor{green!80}{0.60±0.02} & \cellcolor{green!80}{0.63±0.02} & \cellcolor{green!80}{0.62±0.02} & \cellcolor{green!80}{0.61±0.02} & \cellcolor{green!80}{0.61±0.02} & \cellcolor{green!80}{0.64±0.01} & \cellcolor{green!80}{0.65±0.01} & 10 & 0\\
\textbf{TMD} & \cellcolor{green!25}{0.51±0.01} & 0.53±0.01 & 0.52±0.01 & 0.53±0.02 & 0.53±0.02 & 0.53±0.02 & 0.54±0.02 & 0.53±0.01 & 0.56±0.03 & 0.54±0.03 & 0 & 0\\
\hline
\end{tabular}
}
\subcaption{NCI1.\vspace{-.4cm}}\label{tab:NCI1}
\end{subtable}

\end{table}































\clearpage
\newpage

\begin{table}[t!]
\centering
\caption{Graph subsampling performance across sampling percentages for OGBG datasets, reported as mean ± confidence bar of accuracy (ACC) or area under the ROC (ROC-AUC). Best and second-best per column are in dark/light green, respectively. W is total wins per method.}
\vspace{.1in}
\label{tab:all-results-graph-subsampling2} 




\begin{subtable}{\textwidth}
\centering
\resizebox{1.0\textwidth}{!}{%
\begin{tabular}{l | c | c | c | c | c | c | c | c | c | c | c | c}

\hline
\multicolumn{12}{c}{\textbf{With labels}} \\
\hline
\textbf{Doscond} & 0.51±0.01 & 0.51±0.02 & 0.52±0.02 & 0.52±0.02 & 0.40±0.02 & 0.54±0.02 & 0.55±0.02 & 0.56±0.02 & 0.58±0.02 & 0.62±0.02 & - & 0 \\
\textbf{Kidd} & 0.62±0.02 & 0.62±0.04 & 0.62±0.05 & 0.62±0.06 & 0.62±0.07 & 0.58±0.15 & 0.62±0.08 & 0.63±0.10 & 0.63±0.11 & 0.63±0.12 & - & 0 \\
\hline
\multicolumn{12}{c}{\textbf{Without labels}} \\
\hline
\textbf{Doscond} & 0.43±0.02 & 0.44±0.02 & 0.30±0.02 & 0.45±0.02 & 0.46±0.03 & 0.46±0.03 & 0.32±0.01 & 0.47±0.03 & 0.47±0.04 & 0.48±0.03 & 0 & 10\\
\textbf{Kidd} & 0.57±0.04 & 0.57±0.05 & 0.58±0.04 & 0.58±0.05 & 0.58±0.06 & 0.58±0.05 & 0.58±0.06 & 0.59±0.07 & 0.59±0.07 & 0.59±0.08 & 0 & 0\\
\textbf{Feature} & \cellcolor{green!80}{0.78±0.01} & \cellcolor{green!80}{0.76±0.01} & \cellcolor{green!25}{0.76±0.04} & \cellcolor{green!80}{0.77±0.00} & \cellcolor{green!80}{0.79±0.00} & 0.77±0.01 & \cellcolor{green!25}{0.76±0.01} & \cellcolor{green!25}{0.76±0.00} & \cellcolor{green!80}{0.78±0.00} & \cellcolor{green!80}{0.78±0.01} & 6 & 0\\
\textbf{WL} & 0.57±0.08 & 0.69±0.05 & 0.73±0.06 & 0.74±0.05 & 0.76±0.01 & 0.77±0.01 & 0.72±0.06 & 0.70±0.07 & \cellcolor{green!80}{0.78±0.01} & 0.76±0.01 & 1 & 0\\
\textbf{Random} & 0.62±0.09 & \cellcolor{green!80}{0.76±0.02} & \cellcolor{green!80}{0.77±0.01} & \cellcolor{green!25}{0.76±0.01} & \cellcolor{green!25}{0.77±0.00} & \cellcolor{green!25}{0.78±0.00} & \cellcolor{green!25}{0.76±0.03} & \cellcolor{green!80}{0.77±0.00} & \cellcolor{green!25}{0.77±0.00} & \cellcolor{green!25}{0.77±0.00} & 3 & 0\\
\textbf{TMD} & \cellcolor{green!25}{0.69±0.05} & \cellcolor{green!25}{0.75±0.03} & \cellcolor{green!25}{0.76±0.02} & \cellcolor{green!25}{0.76±0.01} & 0.73±0.04 & \cellcolor{green!80}{0.79±0.02} & \cellcolor{green!80}{0.78±0.01} & \cellcolor{green!80}{0.77±0.01} & \cellcolor{green!25}{0.77±0.01} & 0.73±0.07 & 3 & 0\\
\hline
\end{tabular}
}
\subcaption{OGBG-MOLBBBP. \vspace{-.4cm}}\label{tab:molbbbp}
\end{subtable}
\vspace{1em}

\begin{subtable}{\textwidth}
\centering
\resizebox{1.0\textwidth}{!}{
\begin{tabular}{l | c | c | c | c | c | c | c | c | c | c | c | c}
\hline
\textbf{Doscond} & 0.67±0.01 & 0.67±0.03 & 0.64±0.01 & 0.67±0.01 & 0.65±0.03 & 0.68±0.02 & 0.66±0.02 & 0.67±0.01 & 0.67±0.04 & 0.66±0.02 & - & 0 \\
\textbf{Kidd} & 0.64±0.03 & 0.66±0.03 & 0.65±0.03 & 0.64±0.03 & 0.61±0.03 & 0.63±0.03 & 0.66±0.03 & 0.67±0.03 & 0.69±0.03 & 0.68±0.03 & - & 0 \\
\hline
\multicolumn{12}{c}{\textbf{Without labels}} \\
\hline
\textbf{Doscond} & \cellcolor{green!80}{0.53±0.02} & 0.45±0.04 & \cellcolor{green!80}{0.64±0.03} & \cellcolor{green!80}{0.60±0.02} & 0.37±0.01 & \cellcolor{green!80}{0.56±0.06} & \cellcolor{green!80}{0.61±0.02} & 0.54±0.01 & \cellcolor{green!80}{0.59±0.03} & \cellcolor{green!80}{0.62±0.03} & 7 & 2\\
\textbf{Kidd} & \cellcolor{green!80}{0.53±0.03} & \cellcolor{green!25}{0.51±0.03} & 0.46±0.02 & 0.48±0.02 & \cellcolor{green!80}{0.56±0.03} & \cellcolor{green!25}{0.54±0.03} & \cellcolor{green!25}{0.58±0.03} & 0.55±0.02 & 0.50±0.03 & 0.52±0.03 & 2 & 2\\
\textbf{Feature} & \cellcolor{green!25}{0.50±0.03} & \cellcolor{green!80}{0.52±0.03} & 0.53±0.02 & \cellcolor{green!25}{0.55±0.02} & \cellcolor{green!25}{0.55±0.02} & 0.51±0.02 & 0.55±0.02 & \cellcolor{green!25}{0.56±0.03} & 0.56±0.02 & 0.56±0.02 & 1 & 0\\
\textbf{WL} & 0.48±0.01 & 0.49±0.02 & 0.48±0.02 & 0.51±0.03 & 0.48±0.02 & 0.50±0.02 & 0.53±0.03 & 0.50±0.03 & 0.55±0.03 & 0.53±0.03 & 0 & 4 \\
\textbf{Random} & 0.49±0.02 & 0.49±0.03 & 0.51±0.03 & 0.49±0.02 & 0.48±0.03 & 0.50±0.04 & 0.48±0.02 & 0.48±0.02 & 0.49±0.03 & 0.51±0.03 & 0 & 7\\
\textbf{TMD} & 0.45±0.01 & \cellcolor{green!80}{0.52±0.03} & \cellcolor{green!25}{0.55±0.04} & 0.49±0.02 & 0.54±0.03 & \cellcolor{green!25}{0.54±0.03} & \cellcolor{green!25}{0.58±0.03} & \cellcolor{green!80}{0.60±0.03} & \cellcolor{green!25}{0.57±0.02} & \cellcolor{green!25}{0.58±0.02} & 2 & 2\\
\hline
\end{tabular}
}
\subcaption{OGBG-MOLBACE. \vspace{-.4cm}}\label{tab:molbace}
\end{subtable}
\vspace{1em}


\end{table}



















\begin{table}[H]
\scriptsize
\caption{Node subsampling performance across datasets and sampling percentages, reported as mean ± confidence bar of accuracy (ACC). Best and second-best per column are in dark and light green, respectively. W is total wins per method.}
\label{tab:combined-results-nodes}
\vspace{0.1in}
\centering

\begin{subtable}{\textwidth}
\centering
\resizebox{1.0\textwidth}{!} & \textbf{2\%} & \textbf{3\%} & \textbf{4\%} & \textbf{5\%} & \textbf{6\%} & \textbf{7\%} & \textbf{8\%} & \textbf{9\%} & \textbf{W} \\
\hline
\textbf{RW}
    & \cellcolor{green!80}{0.68±0.02}
    & \cellcolor{green!80}{0.67±0.03}
    & 0.66±0.03
    & 0.65±0.04
    & 0.66±0.04
    & 0.68±0.08
    & 0.71±0.05
    & \cellcolor{green!25}{0.77±0.03}
    & 0.80±0.04
    & 2 \\
\textbf{k-cores}
    & 0.56±0.07
    & \cellcolor{green!25}{0.63±0.04}
    & 0.48±0.07
    & 0.64±0.06
    & \cellcolor{green!25}{0.67±0.05}
    & 0.65±0.06
    & \cellcolor{green!25}{0.73±0.04}
    & 0.73±0.03
    & 0.74±0.03
    & 0 \\
\textbf{Random}
    & \cellcolor{green!80}{0.68±0.04}
    & \cellcolor{green!80}{0.67±0.03}
    & \cellcolor{green!80}{0.70±0.03}
    & \cellcolor{green!80}{0.71±0.03}
    & 0.66±0.03
    & \cellcolor{green!25}{0.69±0.03}
    & \cellcolor{green!80}{0.74±0.03}
    & 0.75±0.03
    & \cellcolor{green!25}{0.81±0.04}
    & 4 \\
\textbf{TMD}
    & \cellcolor{green!25}{0.66±0.05}
    & \cellcolor{green!80}{0.67±0.03}
    & \cellcolor{green!25}{0.68±0.04}
    & \cellcolor{green!25}{0.69±0.02}
    & \cellcolor{green!80}{0.68±0.04}
    & \cellcolor{green!80}{0.72±0.03}
    & \cellcolor{green!80}{0.74±0.04}
    & \cellcolor{green!80}{0.78±0.03}
    & \cellcolor{green!80}{0.84±0.02}
    & 6 \\
\hline
\end{tabular}
}
\caption{MUTAG}
\label{tab:mutag-pivot}
\end{subtable}

\vspace{0em}


\begin{subtable}{\textwidth}
\centering
\resizebox{1.0\textwidth}{!}{%
\begin{tabular}{l|c|c|c|c|c|c|c|c|c|c}

\hline
\textbf{RW}
    & \cellcolor{green!80}{0.61±0.01}
    & \cellcolor{green!80}{0.60±0.01}
    & \cellcolor{green!80}{0.63±0.01}
    & 0.63±0.01
    & \cellcolor{green!25}{0.67±0.02}
    & \cellcolor{green!25}{0.69±0.01}
    & \cellcolor{green!25}{0.68±0.02}
    & \cellcolor{green!80}{0.72±0.01}
    & 0.70±0.02
    & 4 \\
\textbf{k-cores}
    & 0.59±0.01
    & \cellcolor{green!80}{0.60±0.01}
    & 0.60±0.01
    & \cellcolor{green!25}{0.64±0.02}
    & 0.65±0.01
    & \cellcolor{green!25}{0.69±0.02}
    & \cellcolor{green!25}{0.68±0.02}
    & 0.70±0.01
    & \cellcolor{green!25}{0.71±0.01}
    & 1 \\
\textbf{Random}
    & \cellcolor{green!25}{0.60±0.01}
    & \cellcolor{green!80}{0.60±0.01}
    & \cellcolor{green!25}{0.62±0.02}
    & \cellcolor{green!25}{0.64±0.02}
    & \cellcolor{green!25}{0.67±0.02}
    & 0.68±0.01
    & \cellcolor{green!80}{0.71±0.01}
    & 0.70±0.02
    & 0.70±0.02
    & 2 \\
\textbf{TMD}
    & \cellcolor{green!25}{0.60±0.01}
    & \cellcolor{green!80}{0.60±0.01}
    & \cellcolor{green!80}{0.63±0.01}
    & \cellcolor{green!80}{0.65±0.01}
    & \cellcolor{green!80}{0.68±0.01}
    & \cellcolor{green!80}{0.70±0.01}
    & \cellcolor{green!80}{0.71±0.01}
    & \cellcolor{green!25}{0.71±0.01}
    & \cellcolor{green!80}{0.73±0.01}
    & 7 \\
\hline
\end{tabular}
}
\caption{PROTEINS}
\label{tab:proteins-pivot}
\end{subtable}

\vspace{0em}

\begin{subtable}{\textwidth}
\centering
\resizebox{1.0\textwidth}{!}{%
\begin{tabular}{l|c|c|c|c|c|c|c|c|c|c}

\hline
\textbf{RW}
    & \cellcolor{green!25}{0.58±0.01}
    & \cellcolor{green!80}{0.59±0.01}
    & \cellcolor{green!80}{0.61±0.01}
    & 0.63±0.01
    & \cellcolor{green!80}{0.66±0.02}
    & \cellcolor{green!80}{0.69±0.01}
    & \cellcolor{green!80}{0.72±0.02}
    & \cellcolor{green!25}{0.73±0.02}
    & \cellcolor{green!25}{0.74±0.02}
    & 5 \\
\textbf{k-cores}
    & \cellcolor{green!80}{0.59±0.01}
    & \cellcolor{green!80}{0.59±0.01}
    & \cellcolor{green!25}{0.59±0.02}
    & 0.59±0.01
    & 0.60±0.01
    & \cellcolor{green!25}{0.59±0.01}
    & 0.59±0.01
    & 0.60±0.01
    & 0.59±0.01
    & 2 \\
\textbf{Random}
    & \cellcolor{green!80}{0.59±0.01}
    & \cellcolor{green!25}{0.58±0.01}
    & \cellcolor{green!80}{0.61±0.01}
    & \cellcolor{green!25}{0.64±0.01}
    & \cellcolor{green!80}{0.66±0.02}
    & \cellcolor{green!80}{0.69±0.02}
    & \cellcolor{green!25}{0.71±0.02}
    & \cellcolor{green!25}{0.73±0.02}
    & 0.73±0.03
    & 7 \\
\textbf{TMD}
    & \cellcolor{green!25}{0.58±0.02}
    & \cellcolor{green!80}{0.59±0.01}
    & \cellcolor{green!80}{0.61±0.01}
    & \cellcolor{green!80}{0.65±0.02}
    & \cellcolor{green!25}{0.65±0.02}
    & \cellcolor{green!80}{0.69±0.02}
    & 0.70±0.02
    & \cellcolor{green!80}{0.74±0.01}
    & \cellcolor{green!80}{0.76±0.01}
    & 6 \\
\hline
\end{tabular}
}
\caption{DD}
\label{tab:dd-pivot}
\end{subtable}

\vspace{0em}


\begin{subtable}{\textwidth}
\centering
\resizebox{1.0\textwidth}{!}{%
\begin{tabular}{l|c|c|c|c|c|c|c|c|c|c}

\hline
\textbf{RW}
    & \cellcolor{green!25}{0.78±0.01}
    & 0.77±0.01
    & \cellcolor{green!80}{0.79±0.01}
    & \cellcolor{green!80}{0.80±0.02}
    & \cellcolor{green!80}{0.79±0.02}
    & \cellcolor{green!80}{0.78±0.02}
    & \cellcolor{green!25}{0.79±0.02}
    & \cellcolor{green!25}{0.79±0.02}
    & 0.77±0.02
    & 4 \\
\textbf{k-cores}
    & \cellcolor{green!25}{0.78±0.01}
    & \cellcolor{green!25}{0.78±0.02}
    & 0.77±0.01
    & \cellcolor{green!25}{0.79±0.02}
    & 0.74±0.05
    & \cellcolor{green!80}{0.78±0.01}
    & 0.77±0.02
    & 0.78±0.02
    & \cellcolor{green!80}{0.79±0.02}
    & 2 \\
\textbf{Random}
    & \cellcolor{green!80}{0.79±0.02}
    & \cellcolor{green!25}{0.78±0.02}
    & \cellcolor{green!25}{0.78±0.01}
    & \cellcolor{green!25}{0.79±0.02}
    & \cellcolor{green!80}{0.79±0.01}
    & \cellcolor{green!80}{0.78±0.02}
    & 0.77±0.02
    & 0.78±0.01
    & \cellcolor{green!25}{0.78±0.01}
    & 3 \\
\textbf{TMD}
    & \cellcolor{green!80}{0.79±0.02}
    & \cellcolor{green!80}{0.79±0.02}
    & \cellcolor{green!25}{0.78±0.01}
    & 0.77±0.02
    & \cellcolor{green!25}{0.78±0.01}
    & \cellcolor{green!80}{0.78±0.02}
    & \cellcolor{green!80}{0.80±0.02}
    & \cellcolor{green!80}{0.80±0.01}
    & \cellcolor{green!80}{0.79±0.02}
    & 6 \\
\hline
\end{tabular}
}
\caption{COX2}
\label{tab:cox2-pivot}
\end{subtable}

\end{table}


%\FloatBarrier


\section{Conclusion}
In this work, we propose a simple yet effective approach, called SMILE, for graph few-shot learning with fewer tasks. Specifically, we introduce a novel dual-level mixup strategy, including within-task and across-task mixup, for enriching the diversity of nodes within each task and the diversity of tasks. Also, we incorporate the degree-based prior information to learn expressive node embeddings. Theoretically, we prove that SMILE effectively enhances the model's generalization performance. Empirically, we conduct extensive experiments on multiple benchmarks and the results suggest that SMILE significantly outperforms other baselines, including both in-domain and cross-domain few-shot settings.

\section*{Acknowledgements}
The authors thank Ching-Yao Chuang and Joshua Robinson for valuable discussions.  This work was supported in part by NSF grant CCF-2338226, the NSF AI Institute TILOS and an Alexander von Humboldt fellowship.

%\FloatBarrier

\bibliography{ref}
\bibliographystyle{plainnat}



\newpage
\appendix

\section{Additional background}\label{sec:additional-background}


In this Appendix, we discuss additional helpful background for the discussions in the main body. 

\subsection{Additional details about the TMD}\label{app:TMD}

Here we describe the padding function $\rho$ using in $\cOTbar$.

We use $\blankTree^n$ to denote $n$ disjoint copies of a blank tree $\blankTree.$ Given two tree multisets $\cT_u(T)$, $\cT_v(T')$, we define $\rho$ to be the following augmentation function, which returns two multi-sets of the same size: 
\begin{align}\label{eq:rho}
    \rho: (\cT_v(T'), \cT_u(T)) \mapsto \paren{\cT_v(T') \cup {\blankTree}^{\max(|\cT_u(T)| - |\cT_v(T')|, 0)}, \cT_u(T) \cup {\blankTree}^{\max(|\cT_v(T)| - |\cT_u(T')|, 0)} }.
\end{align}

Equipped with this definition, we define $\cOTbar$ for a given weight function $w: \N \to \R_{> 0}$ as follows: 
\begin{align}\label{eq:OT-bar}
    \cOTbar(\cdot, \cdot) = \cOT_{\cTD_w}(\rho(\cdot, \cdot)). 
\end{align}

\subsection{Visual notation aids}\label{sec:notation}

\begin{figure}
    \centering
    \includegraphics[width=0.75\linewidth]{figures/ComputationTrees.png}
    \caption{Computation trees up to depth $L=3$ for an example 4-node graph (Definition~\ref{def:tree}).}
    \label{fig:ctrees}
\end{figure}

We have included some visualizations to aid in understanding the notations introduced in Section~\ref{sec:prelim}. In the following figures, we use black outlines to denote the roots of rooted trees.  Figure~\ref{fig:ctrees} gives a visualization of computation trees for a simple four-node graph, as per Definition~\ref{def:tree}. Figure~\ref{fig:multiset} gives a visualization of a tree multiset as per Definition~\ref{def:multiset}. Figure~\ref{fig:td} gives a visualization of the recursive definition of the tree distance (TD) Definition~\ref{def:TD}. Figure~\ref{fig:tmd} shows a visualization of computing TMD as an OT problem between multiset of computation trees of two graphs. We hope that these figures help the reader to develop a more intuitive understanding of the TMD.

\begin{figure}
    \centering
    \includegraphics[width=0.4\linewidth]{figures/TreeMultiset.png}
    \caption{The tree multiset associated with an example depth 3 rooted tree (Definition~\ref{def:multiset}).}
    \label{fig:multiset}
\end{figure}

\begin{figure}[ht]
    \centering
    \includegraphics[width=1\linewidth]{figures/TD.png}
    \caption{The tree distance between example graphs (Definition~\ref{def:TD}).}
    \label{fig:td}
\end{figure}

\begin{figure}[ht]
    \centering
    \includegraphics[width=.25\linewidth]{figures/TMD.png}
    \caption{TMD computation between example graphs (Definition~\ref{def:TMD}). We first construct the computation trees of the original graphs, followed by computing the OT cost with respect to the tree distance. Since the graphs have unequal number of nodes, we pad the computation trees of $G_2$ with a \emph{blank} tree (see also \eqref{eq:rho}).}
    \label{fig:tmd}
\end{figure}

\subsection{Heuristic for selecting $\cS$ in the relaxed node subsampling problem}\label{sec:heuristic}

Here we present
our heuristic for constructing the set of candidate subsets $\cS$ in our experiments. Most real-world networks are scale-free and small-world, and hence can be well-modeled by random graphs, such as preferential attachment, configuration models, and {inhomogeneous} random graphs \citep{bollobas2004coupling,newman1999scaling}. {These} 
graphs
converge \textit{locally} to {graph limits} \citep[Vol.~2, Ch.~2]{van2024random}. By a ``transitive'' argument, {we can} view real-world networks as parts of sequences converging to graph limits in the same sense. 
\citet{alimohammadi2023local} showed that for 
such graphs, sampling {nodes'} {local} breadth-first search (BFS) trees asymptotically preserves critical motifs of the {graph limit} (see Appendix~\ref{sec:random-graph-limits} for an overview). Thus, in our experiments, to construct our candidate subset $\cS$ for the \emph{relaxed node subsampling problem}, we include the BFS search trees rooted at the graph's nodes up to a certain node budget.  

\begin{definition}[$k$-BFS subset] {Given $G = (V, E, f)$ and $v \in V$}, let $\ell_k$ be the deepest level such that the $v$-rooted BFS tree of depth $\ell_k$ has at most $k$ nodes (breaking ties in a fixed but arbitrary way). {The $k$-BFS subset of $v$, denoted $S_{\mathrm{BFS}(v;k)}$, is} the set of nodes at distance $\leq \ell_{k}$ from $v$ in $G$.
\end{definition}

In addition, we augment the candidate subsets with additional candidate subgraphs proposed in prior random-walk-inspired heuristics based on random walks (RW) \citep{razin2023ability} and graph cores (k-cores) \citep{salha2022degeneracy}. Concretely, given a node budget $k$, \citet{razin2023ability} provide an algorithm to construct a subset $S_{\mathrm{RW}}$ of size at most $k$; and \citet{salha2022degeneracy} provide an algorithm to construct a subset $S_{\mathrm{k-core}}$ of size at most $k$. 

Combining these three heuristics, we use: 
\begin{align*}
    \cS = \cup_{v \in V} S_{\mathrm{BFS}(v;k)} \cup \{S_{\mathrm{RW}}, S_{\mathrm{k-core}}\} 
\end{align*}
as our candidate subsets for the \emph{relaxed node subsampling problem} in our node subsampling experiments. We then apply Theorem~\ref{thm:node-subsampling-relaxed} to select among these candidate node subsets in $\cS$ using the TMD. Note that $\cS$ contains a total of $|V| + 2 = O(|V|)$ graphs, each consisting of $k$ nodes. Thus, using Theorem~\ref{thm:node-subsampling-relaxed} and Theorem~\ref{thm:node-subsampling-relaxed-algorithm}, we obtain an overall runtime of $O(L |V| |E|)$ for our node subsampling procedure.

\subsection{Random graph limits}\label{sec:random-graph-limits}

As we discussed in Section~\ref{sec:heuristic}, most real networks are scale-free and small-world, and hence well-modeled by random graph models, such as preferential attachment, configuration models and inhomogenous random graphs. As shown by \citet{van2024random}, such random graph models produce graphs that can be shown to converge \textit{locally} to a limit $(G,o)$, where $G$ is a graph to which we assign a root node $o$. By a ``transitive'' argument, it makes sense to see real networks as parts of sequences converging to graph limits in a similar sense. 

To be precise, let $\mathcal{G}_*$ be the set of all possible rooted graphs. A limit graph is defined as a measure over the space $\mathcal{G}_*$ with respect to the local metric
\begin{align*}
    d_{loc}((G_1,o_1),(G_2,o_2)) = \frac{1}{1+\inf_k\{k:B_k(G_1,o_1)\not\simeq B_k(G_2,o_2)\}}
\end{align*}
 where $B_k(G,v)$ is the $k$-hop neighborhood of node $v$, and $\simeq$ is the graph isomorphism. A sequence of graphs converging to this limit is defined as follows.

\begin{definition}[Local convergence \citep{alimohammadi2023local}] Let $G_n = (V_n, E_n)$ denote a finite connected graph. Let $(G_n, o_n)$ be the rooted graph obtained by letting $o_n \in V_n$ be chosen uniformly at random. We say that $(G_n, o_n)$ converges locally to the connected rooted graph $(G, o)$, which is a (possibly random) element of $\mathcal{G}_*$ having law $\mu$, when, for every bounded and continuous function $h: \mathcal{G}_* \to \mathbb{R}$,
$$
\mathbb{E}[h(G_n,o_n)] \to \mathbb{E}_\mu(G,o)
$$
where the expectation on the right-hand-side is with respect to $(G, o)$ having law $\mu$, while the expectation on the left-hand-side is with respect to the random vertex $o_n$.
\end{definition}

\begin{table}[ht]
\centering
\caption{Overview of Graph Datasets}
\vskip 0.1in
\begin{tabular}{lccccc}
\toprule
\textbf{Dataset} & \textbf{\#Graphs} & \textbf{Avg. Nodes} & \textbf{Avg. Edges} & \textbf{\#Classes} & \textbf{Domain}\\
\midrule
MUTAG & 188 & 17.93 & 19.79 & 2 & Molecules\\
PROTEINS & 1,113 & 39.06 & 72.82 & 2 & Molecules \\
COX2 & 467 & 43.45 & 43.45 & 2 & Molecules \\
NCI1 & 4,110 & 29.87 & 32.30 & 2 & Molecules \\
DD & 1,178 & 284.32 & 715.66 & 2 & Proteins \\
OGBG-MOLBACE & 1,513 & 34.1 & 36.9 & 2 & Molecules \\
OGBG-MOLBBBP & 2,039 & 24.1 & 26.0 & 2 & Molecules \\
\bottomrule
\end{tabular}
\label{table:graph_datasets_summary}
\end{table}

\section{Additional experiments and experimental details}\label{sec:additional-background-experiments}

In this Appendix, we cover additional experimental details and experimental results to complement our discussion in the main body. 

\subsection{Details about datasets considered in empirical evaluation}\label{sec:dataset_details}

In Table~\ref{table:graph_datasets_summary}, we provide statistics pertaining to the datasets we consider in our empirical study. 

\subsection{Additional details regarding experimental setup for Table~\ref{tab:all-results-graph-subsampling1} and \ref{tab:all-results-graph-subsampling2}}\label{sec:additional-experiment-details}

All experiments were run on an NIVIDIA A6000 GPU with 1TB of RAM. The GNNs were implemented using Pytorch Geometric \citep{he2024pytorch}.  The TMD weight function was set according to Pascal's triangle rule~\citep[][Theorem 8]{Chuang22:Tree} and our implementation of TMD was based on \citet{Chuang22:Tree}. We used the Graph Kernel Library \citep{siglidis2020grakel} for the kernel distances in our experiments.  We will make the code for all of our experiments publicly available if the paper is accepted. For now, we have included an anonymous repository link in the main body of the paper. 

For the experiments in Table~\ref{tab:all-results-graph-subsampling1} using TUDatasets \citep{morris2020tudataset} (MUTAG, PROTEINS, COX2, NCI1), we trained a graph isomorphism network (GIN) with three layers. The model was optimized using the Adam optimizer with a learning rate of 0.01 and a binary cross-entropy (BCE) loss with logits. The batch size was set to 16 for TUD datasets and 64 for OGBG datasets. Each layer contained 128 hidden channels for TUD datasets and 256 for OGBG datasets. We used a global add pooling function for aggregation, with no weight regularization or dropout applied. Additionally, batch normalization was not used in the model.








For the TUDatasets, this is the same architecture used in the original paper by \citet{Chuang22:Tree} for evaluating the TMD as a measure of GNN robustness, since all of their experiments were also on TUDatasets. For the OGBG datasets, (OGBG-MOLBACE, OGBG-MOLBBBP) because they are significantly larger in terms of the \emph{number} of graphs, we use the same architecture and hyperparameters with the exception that we set the batch size to 64 to accommodate the large number of graphs and increase training efficiency, and we used a larger number of hidden channels (2x) to handle the larger, more complex distribution over graphs. We refrained from using batch normalization in both models to best align with our theoretical analysis. Models were implemented using standard GIN implementations in PytorchGeometric \citep{fey2019fast}. We focus on the GIN architecture because it is known to come with strong theoretical expressiveness guarantees and is a common choice for achieving state-of-the-art message-passing GNN performance \citep{xu19gin}. However, as the results of \citet{Chuang22:Tree} can be extended to other architectures, such as Graph Convolutional Networks \citep{kipf2016semi}, our results can easily be extended to other message-passing architectures. 

\subsection{Definition of feature-medoids distance metric}\label{sec:additional-background-experiments-feature}

For the ``Feature'' rows of our empirical results we use the following feature-distance function for graphs $G_1 = (V_1, E_1, f_1), G_2 = (V_2, E_2, f_2)$
\begin{align*}
    D_{\mathrm{feature}}(G_1, G_2) \defeq \norm{\frac{1}{\abs{V_1}} \sum_{v \in V_1} f_1 - \frac{1}{\abs{V_2}}\sum_{v \in V_2} f_2}_2. 
\end{align*}
To interpret this formula, note that it is essentially the Euclidean distance between the \emph{average} feature value in $G_1$ and the \emph{average} feature value in $G_2$ where the average is taken across all nodes in the graph. We then run $k$-medoids clustering in the distance metric $D_\mathrm{feature}$ to construct our subsampled graph datasets. 






\subsection{Effects of validation and label usage in KiDD and DosCond}\label{sec:additional-background-experiments-vallabel}

The original KiDD and DosCond methods rely on access to graph labels for both the training dataset and a held-out validation dataset. In contrast, our approach is fully label-agnostic, requiring neither a labeled validation dataset nor labeled training data. To ensure a fair comparison, we adapt KiDD and DosCond by removing their dependence on labeled validation and training datasets. Specifically, we eliminate the validation set by selecting the GNN at the final epoch, after the default number of epochs used for the original model, rather than relying on validation-based model selection. We also remove the need for training labels by modifying the loss function to exclude label-dependent terms and initializing the sampled graphs randomly instead of using class-dependent initialization.

These adapted versions of KiDD and DosCond, which operate without labeled validation and training data, serve as the most direct comparison to our label-agnostic approach. As shown in Tables~\ref{tab:kidd} and~\ref{tab:doscond}, the removal of validation and label information in these methods each results in a measurable decline in performance. The simultaneous removal of both further exacerbates this decline, highlighting the challenges of achieving strong performance in fully label-agnostic settings. These results underscore the performance of our approach, which achieves competitive outcomes under label constraints.


\begin{table}[ht]
\centering
\caption{Performance comparison of KiDD across multiple datasets and percentages of graphs sampled. Performance is reported as mean ± standard deviation of accuracy (ACC) and area under the receiver operating curve (ROC-AUC), for configurations of KiDD with and without labels (lab.) and validation (val.).}
\vskip .1in
\label{tab:kidd}
\resizebox{1.0\textwidth}{!}{%
\begin{tabular}{cc cccccccccc}
\toprule
\multicolumn{2}{c}{} & \multicolumn{10}{c}{\textbf{\% of graphs sampled}} \\
\cmidrule(lr){3-12}
\textbf{lab.} & \textbf{val.} 
& \textbf{1} & \textbf{2} & \textbf{3} & \textbf{4} & \textbf{5} 
& \textbf{6} & \textbf{7} & \textbf{8} & \textbf{9} & \textbf{10} \\
\midrule
\multicolumn{12}{c}{\textbf{MUTAG} (ACC)} \\
\midrule

\xmark & \xmark
& 0.684±0.000 & 0.684±0.000 & 0.684±0.000 & 0.439±0.174 & 0.684±0.000
& 0.684±0.000 & 0.316±0.000 & 0.702±0.025 & 0.316±0.000 & 0.316±0.000 \\

\checkmark & \xmark
& 0.684±0.000 & 0.316±0.000 & 0.684±0.020 & 0.684±0.000 & 0.684±0.000
& 0.684±0.000 & 0.316±0.000 & 0.700±0.030 & 0.316±0.000 & 0.684±0.010 \\

\xmark & \checkmark
& 0.684±0.000 & 0.684±0.000 & 0.684±0.000 & 0.684±0.000 & 0.684±0.000
& 0.684±0.000 & 0.684±0.000 & 0.684±0.000 & 0.684±0.000 & 0.684±0.000 \\

\checkmark & \checkmark
& 0.684±0.000 & 0.684±0.000 & 0.684±0.000 & 0.684±0.000 & 0.684±0.000
& 0.754±0.099 & 0.684±0.000 & 0.684±0.000 & 0.684±0.000 & 0.737±0.043 \\
\midrule


\multicolumn{12}{c}{\textbf{COX2} (ACC)} \\
\midrule

\xmark & \xmark
& 0.170±0.000 & 0.170±0.000 & 0.390±0.311 & 0.830±0.000 & 0.830±0.000
& 0.830±0.000 & 0.830±0.000 & 0.830±0.000 & 0.830±0.000 & 0.390±0.311 \\

\checkmark & \xmark
& 0.610±0.311 & 0.390±0.311 & 0.500±0.200 & 0.610±0.311 & 0.830±0.000
& 0.830±0.000 & 0.610±0.311 & 0.390±0.311 & 0.390±0.311 & 0.610±0.311 \\

\xmark & \checkmark
& 0.390±0.311 & 0.170±0.170 & 0.170±0.000 & 0.170±0.000 & 0.830±0.000
& 0.170±0.000 & 0.170±0.000 & 0.830±0.000 & 0.170±0.000 & 0.830±0.000 \\

\checkmark & \checkmark
& 0.170±0.000 & 0.830±0.000 & 0.170±0.000 & 0.610±0.311 & 0.830±0.000
& 0.830±0.000 & 0.170±0.000 & 0.170±0.000 & 0.170±0.000 & 0.610±0.311 \\
\midrule

\multicolumn{12}{c}{\textbf{NCI1} (ACC)} \\
\midrule

\xmark & \xmark
& 0.550±0.030 & 0.555±0.028 & 0.560±0.026 
& 0.540±0.050  & 0.565±0.024 & 0.568±0.022 & 0.570±0.023 & 0.574±0.022 & 0.578±0.020 & 0.580±0.021 \\

\checkmark & \xmark
& 0.572±0.020 & 0.578±0.022 & 0.582±0.023 & 0.585±0.020 & 0.588±0.021
& 0.590±0.019 & 0.591±0.020 & 0.593±0.022 & 0.595±0.018 & 0.598±0.021 \\

\xmark & \checkmark
& 0.585±0.020 
& 0.552±0.044 
& 0.590±0.018 & 0.592±0.013 & 0.593±0.017
& 0.595±0.015 & 0.596±0.012 & 0.598±0.016 & 0.599±0.014 & 0.600±0.013 \\

\checkmark & \checkmark
& 0.602±0.013 & 0.605±0.010 & 0.608±0.015 & 0.612±0.012 & 0.613±0.016
& 0.615±0.015 & 0.615±0.013 & 0.617±0.010 & 0.619±0.011 & 0.621±0.012 \\
\midrule


\multicolumn{12}{c}{\textbf{PROTEINS} (ACC)} \\
\midrule

\xmark & \xmark
& 0.600±0.023 & 0.580±0.018 & 0.550±0.013 & 0.560±0.018 & 0.600±0.020
& 0.610±0.028 & 0.600±0.023 & 0.620±0.023 & 0.580±0.013 & 0.560±0.020 \\

\checkmark & \xmark
& 0.560±0.020 & 0.610±0.030 & 0.590±0.030 & 0.590±0.025 & 0.600±0.025
& 0.620±0.030 & 0.620±0.025 & 0.600±0.025 & 0.610±0.030 & 0.610±0.025 \\

\xmark & \checkmark
& 0.640±0.020 & 0.660±0.030 & 0.660±0.025 & 0.650±0.020 & 0.640±0.030
& 0.620±0.025 & 0.630±0.030 & 0.660±0.020 & 0.680±0.030 & 0.670±0.025 \\

\checkmark & \checkmark
& 0.660±0.030 & 0.680±0.025 & 0.710±0.030 & 0.690±0.030 & 0.700±0.030
& 0.680±0.025 & 0.660±0.020 & 0.640±0.020 & 0.690±0.030 & 0.680±0.025 \\
\midrule


\multicolumn{12}{c}{\textbf{OGBG-molbace} (ROC-AUC)} \\
\midrule

\xmark & \xmark
& 0.53±0.03 & 0.51±0.03 & 0.46±0.02 & 0.48±0.02 & 0.56±0.03
& 0.54±0.03 & 0.58±0.03 & 0.55±0.02 & 0.50±0.03 & 0.52±0.03 \\

\checkmark & \xmark
& 0.58±0.03 & 0.56±0.03 & 0.61±0.02 & 0.62±0.03 & 0.60±0.03
& 0.59±0.03 & 0.61±0.02 & 0.62±0.03 & 0.58±0.03 & 0.56±0.03 \\

\xmark & \checkmark
& 0.60±0.03 & 0.59±0.03 & 0.56±0.03 & 0.58±0.03 & 0.61±0.03
& 0.62±0.03 & 0.63±0.03 & 0.62±0.03 & 0.60±0.03 & 0.61±0.03 \\

\checkmark & \checkmark
& 0.64±0.03 & 0.66±0.03 & 0.65±0.03 & 0.64±0.03 & 0.61±0.03
& 0.63±0.03 & 0.66±0.03 & 0.67±0.03 & 0.69±0.03 & 0.68±0.03 \\
\midrule

\multicolumn{12}{c}{\textbf{OGBG-molbbp} (ROC-AUC)} \\
\midrule

\xmark & \xmark
& 0.57±0.04 & 0.57±0.05 & 0.58±0.04 & 0.58±0.05 & 0.58±0.06
& 0.58±0.05 & 0.58±0.06 & 0.59±0.07 & 0.59±0.07 & 0.59±0.08 \\

\checkmark & \xmark
& 0.59±0.04 & 0.59±0.04 & 0.60±0.05 & 0.60±0.05 & 0.60±0.06
& 0.60±0.06 
& 0.57±0.10
& 0.60±0.08 & 0.60±0.08 & 0.60±0.09 \\

\xmark & \checkmark
& 0.61±0.03 & 0.61±0.03 & 0.61±0.04 & 0.61±0.05 & 0.61±0.06
& 0.61±0.07 & 0.61±0.07 & 0.61±0.08 & 0.61±0.09 & 0.61±0.09 \\

\checkmark & \checkmark
& 0.62±0.02 & 0.62±0.04 & 0.62±0.05 & 0.62±0.06 & 0.62±0.07
& 0.58±0.15
& 0.62±0.08 & 0.63±0.10 & 0.63±0.11 & 0.63±0.12 \\
\bottomrule
\end{tabular}
}
\end{table}


\begin{table}[h]
\vskip .1in
\centering
\caption{Performance comparison of DosCond across multiple datasets and percentages of graphs sampled. Performance is reported as mean ± standard deviation of accuracy (ACC) and area under the receiver operating curve (ROC-AUC), for configurations of DosCond with and without labels (lab.) and validation (val.).}
\vspace{.1in}
\resizebox{1.0\textwidth}{!}{%
\begin{tabular}{cc cccccccccc}
\toprule
\multicolumn{2}{c}{} & \multicolumn{10}{c}{\textbf{\% of graphs sampled}} \\
\cmidrule(lr){3-12}
\textbf{lab.} & \textbf{val.} 
& \textbf{1} & \textbf{2} & \textbf{3} & \textbf{4} & \textbf{5} 
& \textbf{6} & \textbf{7} & \textbf{8} & \textbf{9} & \textbf{10} \\
\midrule
\multicolumn{12}{c}{\textbf{MUTAG} (ACC)} \\
\midrule

\xmark & \xmark
& 0.671±0.003 & 0.736±0.027 & 0.731±0.008 & 0.727±0.000 & 0.709±0.008
& 0.702±0.008 & 0.696±0.021 & 0.680±0.009 & 0.716±0.006 & 0.704±0.011 \\

\checkmark & \xmark
& 0.682±0.015 & 0.730±0.020 & 0.726±0.018 & 0.719±0.016 & 0.698±0.018
& 0.691±0.021 & 0.684±0.020 & 0.679±0.015 & 0.712±0.015 & 0.705±0.020 \\

\xmark & \checkmark
& 0.656±0.035 & 0.687±0.009 & 0.687±0.024 & 0.698±0.018 & 0.667±0.022
& 0.698±0.016 & 0.700±0.020 & 0.707±0.019 & 0.704±0.017 & 0.684±0.023 \\

\checkmark & \checkmark
& 0.707±0.020 & 0.693±0.005 & 0.687±0.025 & 0.704±0.027 & 0.667±0.020
& 0.691±0.028 & 0.700±0.016 & 0.704±0.017 & 0.702±0.019 & 0.693±0.036 \\
\midrule

\multicolumn{12}{c}{\textbf{COX2} (ACC)} \\
\midrule

\xmark & \xmark
& 0.439±0.142 & 0.217±0.004 & 0.248±0.031 & 0.254±0.025 & 0.772±0.017
& 0.394±0.077 & 0.650±0.090 & 0.219±0.003 & 0.216±0.003 & 0.215±0.000 \\

\checkmark & \xmark
& 0.600±0.060 & 0.456±0.033 & 0.500±0.050 & 0.600±0.044 & 0.700±0.033
& 0.650±0.022 & 0.720±0.040 & 0.500±0.056 & 0.550±0.030 & 0.750±0.022 \\

\xmark & \checkmark
& 0.754±0.021 & 0.760±0.009 & 0.678±0.080 & 0.678±0.083 & 0.755±0.034
& 0.776±0.014 & 0.786±0.000 & 0.763±0.028 & 0.765±0.022 & 0.783±0.002 \\

\checkmark & \checkmark
& 0.744±0.051 & 0.769±0.024 & 0.783±0.004 & 0.787±0.005 & 0.780±0.006
& 0.784±0.003 & 0.780±0.016 & 0.759±0.034 & 0.784±0.005 & 0.784±0.007 \\
\midrule

\multicolumn{12}{c}{\textbf{NCI1} (ACC)} \\
\midrule
%---- (no val, no labels)
\xmark & \xmark
& 0.503±0.022 & 0.510±0.025 & 0.490±0.030
& 0.520±0.028 & 0.521±0.021
& 0.535±0.019 & 0.537±0.025 & 0.540±0.022 
& 0.512±0.040 
& 0.545±0.026 \\
%---- (no val, labels)
\checkmark & \xmark
& 0.521±0.031 & 0.530±0.028 & 0.570±0.022 & 0.512±0.039
& 0.555±0.021
& 0.560±0.033 & 0.526±0.040
& 0.575±0.020 & 0.576±0.019 & 0.580±0.022 \\

\xmark & \checkmark
& 0.540±0.020 & 0.550±0.025 & 0.555±0.019 & 0.557±0.021 
& 0.518±0.035
& 0.560±0.018 & 0.563±0.024 & 0.580±0.020 & 0.578±0.026 & 0.590±0.022 \\

\checkmark & \checkmark
& 0.562±0.012 & 0.560±0.015 
& 0.570±0.018 & 0.575±0.021 & 0.583±0.020
& 0.555±0.030
& 0.590±0.024 & 0.605±0.019 & 0.608±0.022 & 0.612±0.018 \\
\midrule


\multicolumn{12}{c}{\textbf{PROTEINS} (ACC)} \\
\midrule

\xmark & \xmark
& 0.481±0.033 & 0.670±0.009 & 0.651±0.010 & 0.663±0.004 & 0.551±0.016
& 0.634±0.003 & 0.610±0.013 & 0.645±0.006 & 0.610±0.006 & 0.628±0.009 \\

\checkmark & \xmark
& 0.620±0.010 & 0.630±0.020 & 0.645±0.025 & 0.660±0.018 & 0.583±0.012
& 0.633±0.017 & 0.650±0.020 & 0.655±0.012 & 0.650±0.023 & 0.645±0.015 \\

\xmark & \checkmark
& 0.647±0.010 & 0.653±0.009 & 0.647±0.017 & 0.668±0.003 & 0.658±0.020
& 0.641±0.002 & 0.669±0.005 & 0.647±0.016 & 0.642±0.023 & 0.663±0.007 \\

\checkmark & \checkmark
& 0.661±0.011 & 0.643±0.033 & 0.633±0.003 & 0.681±0.027 & 0.655±0.004
& 0.637±0.011 & 0.653±0.003 & 0.658±0.006 & 0.653±0.003 & 0.664±0.002 \\
\midrule

\multicolumn{12}{c}{\textbf{OGBG-MOLBACE} (ROC-AUC)} \\
\midrule

\xmark & \xmark
& 0.53±0.02 & 0.45±0.04 & 0.64±0.03 & 0.60±0.02 & 0.37±0.01
& 0.56±0.06 & 0.62±0.02 & 0.54±0.01 & 0.60±0.03 & 0.62±0.03 \\

\checkmark & \xmark
& 0.62±0.02 & 0.61±0.02 & 0.65±0.01 & 0.65±0.03 & 0.60±0.03
& 0.64±0.02 & 0.63±0.02 & 0.65±0.03 & 0.64±0.02 & 0.66±0.02 \\
\xmark & \checkmark
& 0.68±0.03 & 0.66±0.02 & 0.64±0.01 & 0.64±0.03 & 0.62±0.02
& 0.67±0.01 & 0.65±0.02 & 0.65±0.03 & 0.66±0.01 & 0.65±0.04 \\

\checkmark & \checkmark
& 0.67±0.01 & 0.67±0.03 & 0.64±0.01 & 0.67±0.01 & 0.65±0.03
& 0.68±0.02 & 0.66±0.02 & 0.66±0.01 & 0.67±0.04 & 0.66±0.02 \\
\midrule

\multicolumn{12}{c}{\textbf{OGBG-MOLBBBP} (ROC-AUC)} \\
\midrule

\xmark & \xmark
& 0.43±0.02 & 0.44±0.02 & 0.30±0.02
& 0.45±0.02 & 0.46±0.03
& 0.46±0.03 & 0.32±0.01 
& 0.47±0.03 & 0.47±0.04 & 0.48±0.03 \\

\checkmark & \xmark
& 0.45±0.03 & 0.46±0.03 & 0.46±0.03 & 0.40±0.02 
& 0.46±0.03
& 0.47±0.03 & 0.48±0.03 & 0.44±0.03 
& 0.48±0.03 & 0.49±0.03 \\

\xmark & \checkmark
& 0.48±0.02 & 0.48±0.02 & 0.49±0.02 & 0.49±0.02
& 0.50±0.03 & 0.47±0.03 
& 0.50±0.02 & 0.50±0.02 & 0.51±0.02 & 0.51±0.02 \\

\checkmark & \checkmark
& 0.51±0.01 & 0.51±0.02
& 0.52±0.02 & 0.52±0.02 
& 0.40±0.02 
& 0.54±0.02 & 0.55±0.02 & 0.56±0.02 & 0.58±0.02 & 0.62±0.02 \\
\bottomrule
\end{tabular}%
}
\end{table}


\subsection{Performance of MIRAGE}\label{sec:additional-background-experiments-mirage}

Table~\ref{tab:mirage} reports the performance of Mirage \citep{mirage} on NCI1 and OGBG-MOLBACE for certain subsample percentages, which were the only cases where we were able to execute the publicly available referenced in the original paper. Unresolved errors were encountered when running Mirage on MUTAG, PROTEINS, NCI1, DD, OGBG-MOLBBBP, and the omitted percentages in Table~\ref{tab:mirage}, due to the MP Tree search either returning an empty selection set or generating trees in a format incompatible with GNN training.

Unlike our method, MIRAGE is not label-agnostic, as it explicitly relies on a labeled validation set with an 80\%-train / 10\%-validation / 10\%-test split. Despite this supervision, MIRAGE underperforms compared to TMD and other methods, typically achieving accuracy below 0.5 for NCI1 and ROC-AUC below 0.5 for OGBG-MOLBBBP.


\begin{table}[ht]
\centering

\caption{Performance of Mirage on the NCI1 and molbbp datasets across percentages of graphs sampled, using the original implementation with labels and validation. Performance is reported as mean ± standard deviation of accuracy (ACC) and area under the receiver operating curve (ROC-AUC). Results for '--', other datasets, and configurations without labels or without validation could not be generated due to unresolved errors in execution.}
\vskip .1in
\label{tab:doscond}
\resizebox{1.0\textwidth}{!}{%
\begin{tabular}{cc cccccccccc}
\toprule
\multicolumn{1}{c}{} & \multicolumn{10}{c}{\textbf{\% of graphs sampled}} \\
\cmidrule(lr){2-11}

\textbf{Dataset} & \textbf{1} & \textbf{2} & \textbf{3} & \textbf{4} & \textbf{5} 
& \textbf{6} & \textbf{7} & \textbf{8} & \textbf{9} & \textbf{10} \\
\midrule

\makecell{\textbf{NCI1} (ACC)} & 0.509±0.021 & 0.513±0.010 & -- & 0.492±0.015 & 0.531±0.022
& 0.488±0.030 & -- & -- & 0.488±0.022 & 0.501±0.018 \\
\midrule

\makecell{\textbf{OGBG-MOLBBBP} (ROC-AUC)} & 0.51±0.0 & 0.42±0.1 & 0.49±0.0 & 0.41±0.0 & 0.42±0.1 
& 0.58±0.0 & 0.43±0.1 & 0.21±0.0 & 0.41±0.0 & 0.37±0.2 \\
\bottomrule
\end{tabular}%
}
\label{tab:mirage}
\end{table}

%\FloatBarrier

\subsection{Demonstration of generalization to alternative GNN architecture}\label{def:generalize-architecture}

To demonstrate the effect of modifying the model architecture, we also consider the effect of increasing the size of the neural network--from 128 neurons to 256 for TUDatasets (MUTAG, PROTEINS, COX2, NCI1), and from 256 to 512 for OGBG--and modify the aggregation function to global mean pool, which is another popular choice of pooling function in the applied GNN literature. 

\paragraph{Graph subsampling} Because the graph condensation methods are significantly more time-intensive and require re-running the entire distillation process for a new model architecture, we did not run KiDD and DOSCOND on this alternative architecture. 

Our results for all of the medoids-methods as well as Random are shown in Table~\ref{tab:performance_comparison_large}. We use the same experimental setup as for Table~\ref{tab:all-results-graph-subsampling1} and \ref{tab:all-results-graph-subsampling2}, except for modifying the pooling layer and number of neurons per layer, as described above. Each entry reports the mean performance (measured in test accuracy for the TUDatasets and and test AUC-ROC for the OGBG datasets) and 95\% confidence bars across 20 trials, randomized over train/test splits as well as neural network initialization. 

Combined with Table~\ref{tab:all-results-graph-subsampling1} and \ref{tab:all-results-graph-subsampling2} in the main body, our results indicate that our method can perform well on different-sized neural networks with different aggregation functions, despite not factoring this information into the choice of the graph subsamples. 

\paragraph{Node subsampling} We use the same experimental setup as for Table~\ref{tab:combined-results-nodes}, except for modifying the pooling layer and number of neurons per layer, as previously described. 

Results are shown in Table~\ref{tab:combined-results-nodes-2}. Each entry reports the mean and 95\% confidence bars across 20 trials, randomized over train/test splits as well as neural network initialization. Combined with Table~\ref{tab:combined-results-nodes} in the main body, our results indicate that our method can perform well on different-sized neural networks with different aggregation functions, despite not factoring this information into the choice of the node subsamples.

\begin{table}[ht]
\caption{Performance comparison for different methods across the percentage of subsampled graphs and various datasets on alternative GIN archiecture. Performance for MUTAG, PROTEINS, COX2, NCI1 is reported in test-accuracy. Performance for OGBG-MOLBACE and OGBG-MOLBBP is reported in test-AUC-ROC. TMD almost always performs best or second-best among the compared methods.  Dark and light green highlight best and second best performance in each row, respectively. All performances are reported as average $\pm$ 95\% confidence bars.}
\vspace{.1in}
\centering
\resizebox{1.0\textwidth}{!}{
\scriptsize
\begin{tabular}{lccccccccccc}
\toprule
Method & 1\% & 2\% & 3\% & 4\% & 5\% & 6\% & 7\% & 8\% & 9\% & 10\% \\
\midrule
\multicolumn{11}{c}{\textbf{MUTAG}} \\
\midrule
TMD & 0.58±0.07 & \cellcolor{green!80}{0.71±0.03} & \cellcolor{green!80}{0.71±0.02} & \cellcolor{green!25}{0.71±0.04} & \cellcolor{green!80}{0.76±0.04} & \cellcolor{green!80}{0.77±0.03} & \cellcolor{green!80}{0.77±0.03} & \cellcolor{green!25}{0.74±0.06} & \cellcolor{green!80}{0.75±0.04} & \cellcolor{green!80}{0.76±0.02} \\
WL & \cellcolor{green!80}{0.67±0.05} & \cellcolor{green!80}{0.71±0.04} & \cellcolor{green!25}{0.70±0.02} & \cellcolor{green!80}{0.72±0.03} & 0.72±0.03 & \cellcolor{green!25}{0.69±0.03} & 0.69±0.04 & 0.69±0.04 & \cellcolor{green!25}{0.73±0.03} & 0.72±0.04 \\
Random & \cellcolor{green!25}{0.65±0.05} & \cellcolor{green!25}{0.65±0.06} & 0.66±0.07 & 0.67±0.03 & 0.72±0.04 & \cellcolor{green!25}{0.69±0.04} & 0.72±0.04 & 0.73±0.03 & \cellcolor{green!80}{0.75±0.04} & 0.72±0.03 \\
Feature & 0.55±0.07 & 0.62±0.08 & \cellcolor{green!80}{0.71±0.03} & 0.69±0.04 & \cellcolor{green!25}{0.74±0.04} & \cellcolor{green!80}{0.77±0.03} & \cellcolor{green!25}{0.76±0.02} & \cellcolor{green!80}{0.75±0.02} & \cellcolor{green!25}{0.73±0.03} & \cellcolor{green!25}{0.74±0.03} \\
\midrule
\multicolumn{11}{c}{\textbf{PROTEINS}} \\
\midrule
TMD & \cellcolor{green!80}{0.63±0.03} & \cellcolor{green!25}{0.64±0.04} & \cellcolor{green!80}{0.69±0.02} & \cellcolor{green!80}{0.70±0.02} & \cellcolor{green!25}{0.67±0.03} & \cellcolor{green!80}{0.69±0.01} & \cellcolor{green!25}{0.69±0.02} & \cellcolor{green!80}{0.69±0.02} & \cellcolor{green!25}{0.70±0.02} & \cellcolor{green!25}{0.70±0.02} \\
WL & \cellcolor{green!25}{0.62±0.02} & \cellcolor{green!80}{0.65±0.03} & \cellcolor{green!25}{0.68±0.02} & \cellcolor{green!25}{0.68±0.02} & \cellcolor{green!80}{0.68±0.01} & \cellcolor{green!25}{0.67±0.02} & 0.68±0.02 & \cellcolor{green!80}{0.69±0.02} & 0.66±0.04 & 0.67±0.02 \\
Random & 0.61±0.02 & \cellcolor{green!25}{0.64±0.02} & 0.64±0.02 & 0.64±0.04 & \cellcolor{green!25}{0.67±0.02} & 0.65±0.02 & \cellcolor{green!25}{0.69±0.02} & \cellcolor{green!25}{0.67±0.02} & 0.68±0.02 & 0.66±0.03 \\
Feature & \cellcolor{green!25}{0.62±0.03} & 0.60±0.04 & 0.64±0.03 & 0.65±0.04 & \cellcolor{green!80}{0.68±0.03} & \cellcolor{green!25}{0.67±0.03} & \cellcolor{green!80}{0.70±0.02} & 0.66±0.04 & \cellcolor{green!80}{0.71±0.02} & \cellcolor{green!80}{0.72±0.01} \\
\midrule
\multicolumn{11}{c}{\textbf{COX2}} \\
\midrule
TMD & \cellcolor{green!80}{0.70±0.04} & \cellcolor{green!80}{0.76±0.02} & \cellcolor{green!80}{0.75±0.03} & \cellcolor{green!80}{0.78±0.02} & \cellcolor{green!80}{0.78±0.02} & \cellcolor{green!80}{0.77±0.01} & \cellcolor{green!80}{0.77±0.01} & \cellcolor{green!80}{0.78±0.02} & \cellcolor{green!25}{0.77±0.02} & \cellcolor{green!80}{0.78±0.02} \\
WL & 0.57±0.06 & 0.70±0.04 & \cellcolor{green!80}{0.75±0.02} & \cellcolor{green!25}{0.76±0.02} & \cellcolor{green!25}{0.77±0.02} & \cellcolor{green!80}{0.77±0.02} & \cellcolor{green!80}{0.77±0.02} & \cellcolor{green!25}{0.77±0.02} & \cellcolor{green!80}{0.78±0.02} & \cellcolor{green!80}{0.78±0.01} \\
Random & \cellcolor{green!25}{0.65±0.07} & 0.70±0.04 & 0.69±0.05 & 0.68±0.06 & 0.76±0.04 & 0.74±0.04 & \cellcolor{green!25}{0.74±0.03} & 0.74±0.04 & 0.75±0.06 & \cellcolor{green!25}{0.77±0.03} \\
Feature & \cellcolor{green!25}{0.65±0.06} & \cellcolor{green!25}{0.72±0.05} & \cellcolor{green!25}{0.71±0.04} & 0.72±0.02 & \cellcolor{green!80}{0.78±0.01} & \cellcolor{green!25}{0.75±0.03} & \cellcolor{green!25}{0.74±0.04} & 0.73±0.03 & \cellcolor{green!80}{0.78±0.02} & 0.76±0.02 \\
\midrule
\multicolumn{11}{c}{\textbf{NCI1}} \\
\midrule
TMD & 0.52 ± 0.01 & \cellcolor{green!25}{0.54 ± 0.02} & 0.53 ± 0.02 & \cellcolor{green!25}{0.51 ± 0.01} & \cellcolor{green!25}{0.52 ± 0.01} & 0.54 ± 0.02 & 0.55 ± 0.02 & \cellcolor{green!25}{0.52 ± 0.01} & 0.54 ± 0.02 & 0.53 ± 0.02 \\
WL & 0.50 ± 0.01 & 0.51 ± 0.00 & 0.52 ± 0.01 & \cellcolor{green!25}{0.51 ± 0.01} & 0.51 ± 0.01 & 0.50 ± 0.00 & 0.50 ± 0.01 & 0.50 ± 0.01 & 0.52 ± 0.01 & 0.51 ± 0.01 \\
Random & \cellcolor{green!80}{0.56 ± 0.02} & \cellcolor{green!80}{0.60 ± 0.02} & \cellcolor{green!25}{0.60 ± 0.02} & \cellcolor{green!80}{0.61 ± 0.00} & \cellcolor{green!80}{0.63 ± 0.01} & \cellcolor{green!80}{0.62 ± 0.01} & \cellcolor{green!80}{0.63 ± 0.01} & \cellcolor{green!80}{0.62 ± 0.01} & \cellcolor{green!25}{0.62 ± 0.02} & \cellcolor{green!80}{0.64 ± 0.01} \\
Feature & \cellcolor{green!25}{0.53 ± 0.01} & \cellcolor{green!80}{0.60 ± 0.02} & \cellcolor{green!80}{0.61 ± 0.01} & \cellcolor{green!80}{0.61 ± 0.02} & \cellcolor{green!80}{0.63 ± 0.01} & \cellcolor{green!25}{0.60 ± 0.02} & \cellcolor{green!25}{0.62 ± 0.02} & \cellcolor{green!80}{0.62 ± 0.01} & \cellcolor{green!80}{0.63 ± 0.00} & \cellcolor{green!25}{0.62 ± 0.01} \\
\midrule
\multicolumn{11}{c}{\textbf{OGBG-MOLBACE}} \\
\midrule
TMD & 0.45±0.01 & \cellcolor{green!80}{0.52±0.03} & \cellcolor{green!80}{0.55±0.04} & 0.49±0.02 & \cellcolor{green!25}{0.54±0.03} & \cellcolor{green!80}{0.54±0.03} & \cellcolor{green!80}{0.58±0.03} & \cellcolor{green!80}{0.60±0.03} & \cellcolor{green!80}{0.57±0.02} & \cellcolor{green!80}{0.58±0.02} \\
WL & 0.48±0.01 & \cellcolor{green!25}{0.49±0.02} & 0.48±0.02 & \cellcolor{green!25}{0.51±0.03} & 0.48±0.02 & 0.50±0.02 & 0.53±0.03 & 0.50±0.03 & 0.55±0.03 & 0.53±0.03 \\
Random & \cellcolor{green!25}{0.49±0.02} & \cellcolor{green!25}{0.49±0.03} & 0.51±0.03 & 0.49±0.02 & 0.48±0.03 & 0.50±0.04 & 0.48±0.02 & 0.48±0.02 & 0.49±0.03 & 0.51±0.03 \\
Feature & \cellcolor{green!80}{0.50±0.03} & \cellcolor{green!80}{0.52±0.03} & \cellcolor{green!25}{0.53±0.02} & \cellcolor{green!80}{0.55±0.02} & \cellcolor{green!80}{0.55±0.02} & \cellcolor{green!25}{0.51±0.02} & \cellcolor{green!25}{0.55±0.02} & \cellcolor{green!25}{0.56±0.03} & \cellcolor{green!25}{0.56±0.02} & \cellcolor{green!25}{0.56±0.02} \\
\midrule
\multicolumn{11}{c}{\textbf{OGBG-MOLBBBP}} \\
\midrule
TMD & 0.69 ± 0.07 & 0.67 ± 0.07 & \cellcolor{green!25}{0.75 ± 0.02} & \cellcolor{green!80}{0.76 ± 0.00} & \cellcolor{green!80}{0.76 ± 0.01} & \cellcolor{green!80}{0.77 ± 0.02} & \cellcolor{green!25}{0.77 ± 0.01} & \cellcolor{green!25}{0.76 ± 0.01} & \cellcolor{green!25}{0.76 ± 0.01} & 0.76 ± 0.01 \\
WL & 0.44 ± 0.10 & \cellcolor{green!25}{0.74 ± 0.04} & 0.73 ± 0.05 & \cellcolor{green!25}{0.75 ± 0.03} & \cellcolor{green!80}{0.76 ± 0.01} & \cellcolor{green!25}{0.76 ± 0.01} & \cellcolor{green!80}{0.78 ± 0.01} & \cellcolor{green!25}{0.76 ± 0.01} & \cellcolor{green!80}{0.78 ± 0.01} & \cellcolor{green!80}{0.78 ± 0.00} \\
Random & \cellcolor{green!25}{0.74 ± 0.02} & 0.65 ± 0.08 & 0.68 ± 0.07 & \cellcolor{green!80}{0.76 ± 0.01} & \cellcolor{green!80}{0.76 ± 0.01} & 0.72 ± 0.07 & 0.67 ± 0.09 & 0.73 ± 0.04 & \cellcolor{green!25}{0.76 ± 0.00} & \cellcolor{green!25}{0.77 ± 0.01} \\
Feature & \cellcolor{green!80}{0.78 ± 0.01} & \cellcolor{green!80}{0.77 ± 0.01} & \cellcolor{green!80}{0.77 ± 0.01} & 0.71 ± 0.07 & \cellcolor{green!80}{0.76 ± 0.02} & 0.72 ± 0.07 & 0.76 ± 0.01 & \cellcolor{green!80}{0.77 ± 0.01} & 0.72 ± 0.07 & \cellcolor{green!80}{0.78 ± 0.00} \\
\bottomrule
\end{tabular}
\label{tab:performance_comparison_large}
}
\end{table}




\begin{table*}[t]
\scriptsize
\caption{ Performance comparison for different methods across the percentage of subsampled graphs and various datasets on alternative GIN archiecture. All performances are reported in test accuracy. TMD tends to perform better than other methods, but is comparable with RW on COX-2 and PROTEINS on this alternative architecture.  Dark and light green highlight best and second best performance in each row, respectively. All results display average $\pm$ 95\% confidence bars. }
\vspace{0.1in}
\centering

\begin{subtable}{\textwidth}
\centering
\resizebox{1.0\textwidth}{!} & \textbf{2\%} & \textbf{3\%} & \textbf{4\%} & \textbf{5\%} & \textbf{6\%} & \textbf{7\%} & \textbf{8\%} & \textbf{9\%} & \textbf{Wins} \\
\hline
\textbf{RW}
  & \cellcolor{green!25}{0.66±0.04} 
  & \cellcolor{green!80}{0.67±0.03} 
  & 0.66±0.03 
  & 0.66±0.04
  & 0.61±0.06
  & 0.64±0.07
  & \cellcolor{green!25}{0.72±0.03}
  & \cellcolor{green!25}{0.75±0.04}
  & 0.79±0.04
  & 1 \\
\textbf{k-cores}
  & 0.59±0.07 
  & \cellcolor{green!25}{0.63±0.04} 
  & 0.45±0.07 
  & \cellcolor{green!25}{0.69±0.03}
  & \cellcolor{green!25}{0.63±0.06}
  & 0.62±0.08
  & \cellcolor{green!25}{0.72±0.03}
  & 0.74±0.04
  & 0.74±0.05
  & 0 \\
\textbf{Random}
  & \cellcolor{green!80}{0.70±0.03}
  & \cellcolor{green!80}{0.67±0.03} 
  & \cellcolor{green!80}{0.70±0.03} 
  & \cellcolor{green!80}{0.70±0.03}
  & \cellcolor{green!80}{0.68±0.04}
  & \cellcolor{green!25}{0.67±0.04}
  & 0.71±0.03
  & \cellcolor{green!25}{0.75±0.03}
  & \cellcolor{green!25}{0.81±0.03}
  & 5 \\
\textbf{TMD}
  & 0.42±0.07 
  & \cellcolor{green!80}{0.67±0.03} 
  & \cellcolor{green!25}{0.68±0.04} 
  & 0.68±0.02
  & \cellcolor{green!80}{0.68±0.04}
  & \cellcolor{green!80}{0.74±0.03}
  & \cellcolor{green!80}{0.75±0.04}
  & \cellcolor{green!80}{0.80±0.02}
  & \cellcolor{green!80}{0.83±0.03}
  & 6 \\
\hline
\end{tabular}
}
\caption{MUTAG}
\label{tab:mutag-node-2-pivot}
\end{subtable}

\vspace{1em}


\begin{subtable}{\textwidth}
\centering
\resizebox{1.0\textwidth}{!} & \textbf{2\%} & \textbf{3\%} & \textbf{4\%} & \textbf{5\%} & \textbf{6\%} & \textbf{7\%} & \textbf{8\%} & \textbf{9\%} & \textbf{Wins} \\
\hline
\textbf{RW}
  & \cellcolor{green!80}{0.61±0.01}
  & \cellcolor{green!80}{0.60±0.01}
  & \cellcolor{green!80}{0.63±0.01}
  & 0.63±0.01
  & \cellcolor{green!25}{0.67±0.02}
  & \cellcolor{green!25}{0.69±0.01}
  & \cellcolor{green!25}{0.68±0.02}
  & \cellcolor{green!80}{0.72±0.01}
  & \cellcolor{green!80}{0.70±0.02}
  & 5 \\
\textbf{k-cores}
  & 0.59±0.01
  & \cellcolor{green!80}{0.60±0.01}
  & 0.60±0.01
  & \cellcolor{green!25}{0.64±0.02}
  & 0.65±0.01
  & 0.68±0.02
  & 0.68±0.02
  & \cellcolor{green!25}{0.69±0.02}
  & \cellcolor{green!25}{0.69±0.02}
  & 1 \\
\textbf{Random}
  & \cellcolor{green!25}{0.60±0.01}
  & \cellcolor{green!80}{0.60±0.01}
  & \cellcolor{green!25}{0.62±0.02}
  & \cellcolor{green!80}{0.65±0.02}
  & \cellcolor{green!25}{0.67±0.02}
  & \cellcolor{green!25}{0.69±0.02}
  & \cellcolor{green!80}{0.69±0.02}
  & 0.68±0.02
  & \cellcolor{green!80}{0.70±0.02}
  & 4 \\
\textbf{TMD}
  & \cellcolor{green!25}{0.60±0.01}
  & \cellcolor{green!80}{0.60±0.01}
  & 0.61±0.01
  & \cellcolor{green!25}{0.64±0.01}
  & \cellcolor{green!80}{0.67±0.01}
  & \cellcolor{green!80}{0.70±0.02}
  & \cellcolor{green!80}{0.70±0.02}
  & \cellcolor{green!25}{0.69±0.02}
  & \cellcolor{green!80}{0.70±0.03}
  & 5 \\
\hline
\end{tabular}
}
\caption{PROTEINS}
\label{tab:proteins-node-2-pivot}
\end{subtable}

\vspace{1em}


\begin{subtable}{\textwidth}
\centering
\resizebox{1.0\textwidth}{!} & \textbf{2\%} & \textbf{3\%} & \textbf{4\%} & \textbf{5\%} & \textbf{6\%} & \textbf{7\%} & \textbf{8\%} & \textbf{9\%} & \textbf{Wins} \\
\hline
\textbf{RW}
  & \cellcolor{green!25}{0.58±0.01}
  & \cellcolor{green!80}{0.59±0.01}
  & \cellcolor{green!80}{0.61±0.01}
  & \cellcolor{green!25}{0.64±0.02}
  & \cellcolor{green!80}{0.67±0.02}
  & 0.65±0.02
  & \cellcolor{green!25}{0.72±0.02}
  & \cellcolor{green!25}{0.73±0.02}
  & 0.73±0.03
  & 3 \\
\textbf{k-cores}
  & \cellcolor{green!80}{0.59±0.01}
  & \cellcolor{green!80}{0.59±0.01}
  & 0.59±0.02
  & 0.59±0.01
  & 0.60±0.01
  & 0.59±0.01
  & 0.59±0.01
  & 0.60±0.01
  & 0.59±0.01
  & 2 \\
\textbf{Random}
  & \cellcolor{green!80}{0.59±0.01}
  & \cellcolor{green!25}{0.58±0.01}
  & \cellcolor{green!25}{0.60±0.01}
  & \cellcolor{green!25}{0.64±0.02}
  & \cellcolor{green!25}{0.66±0.02}
  & \cellcolor{green!80}{0.69±0.02}
  & 0.71±0.02
  & \cellcolor{green!80}{0.74±0.01}
  & \cellcolor{green!25}{0.74±0.01}
  & 3 \\
\textbf{TMD}
  & \cellcolor{green!25}{0.58±0.01}
  & \cellcolor{green!80}{0.59±0.01}
  & \cellcolor{green!25}{0.60±0.01}
  & \cellcolor{green!80}{0.65±0.01}
  & 0.65±0.01
  & \cellcolor{green!25}{0.67±0.03}
  & \cellcolor{green!80}{0.73±0.02}
  & 0.71±0.03
  & \cellcolor{green!80}{0.75±0.02}
  & 4 \\
\hline
\end{tabular}
}
\caption{DD}
\label{tab:dd-node-2-pivot}
\end{subtable}

\vspace{1em}


\begin{subtable}{\textwidth}
\centering
\resizebox{1.0\textwidth}{!} & \textbf{2\%} & \textbf{3\%} & \textbf{4\%} & \textbf{5\%} & \textbf{6\%} & \textbf{7\%} & \textbf{8\%} & \textbf{9\%} & \textbf{Wins} \\
\hline
\textbf{RW}
  & \cellcolor{green!25}{0.78±0.01}
  & 0.77±0.01
  & \cellcolor{green!80}{0.79±0.01}
  & \cellcolor{green!80}{0.80±0.02}
  & \cellcolor{green!80}{0.79±0.02}
  & \cellcolor{green!80}{0.78±0.02}
  & \cellcolor{green!25}{0.79±0.02}
  & \cellcolor{green!80}{0.79±0.02}
  & 0.77±0.02
  & 5 \\
\textbf{k-cores}
  & \cellcolor{green!25}{0.78±0.01}
  & \cellcolor{green!25}{0.78±0.02}
  & 0.77±0.01
  & \cellcolor{green!25}{0.79±0.02}
  & 0.74±0.05
  & \cellcolor{green!80}{0.78±0.01}
  & 0.77±0.02
  & \cellcolor{green!25}{0.78±0.02}
  & \cellcolor{green!80}{0.79±0.02}
  & 2 \\
\textbf{Random}
  & \cellcolor{green!80}{0.79±0.02}
  & \cellcolor{green!25}{0.78±0.02}
  & \cellcolor{green!25}{0.78±0.01}
  & \cellcolor{green!25}{0.79±0.02}
  & \cellcolor{green!80}{0.79±0.01}
  & \cellcolor{green!80}{0.78±0.02}
  & 0.77±0.02
  & \cellcolor{green!25}{0.78±0.01}
  & \cellcolor{green!25}{0.78±0.01}
  & 3 \\
\textbf{TMD}
  & \cellcolor{green!80}{0.79±0.02}
  & \cellcolor{green!80}{0.79±0.02}
  & \cellcolor{green!25}{0.78±0.01}
  & 0.77±0.02
  & \cellcolor{green!25}{0.78±0.01}
  & \cellcolor{green!80}{0.78±0.02}
  & \cellcolor{green!80}{0.80±0.02}
  & \cellcolor{green!25}{0.78±0.03}
  & \cellcolor{green!80}{0.79±0.02}
  & 5 \\
\hline
\end{tabular}
}
\caption{COX2}
\label{tab:cox2-node-2-pivot}
\end{subtable}
\label{tab:combined-results-nodes-2}
\end{table*}



\subsection{Lloyd-Max Algorithm}
\label{subsec:Lloyd-Max}
For a given quantization bitwidth $B$ and an operand $\bm{X}$, the Lloyd-Max algorithm finds $2^B$ quantization levels $\{\hat{x}_i\}_{i=1}^{2^B}$ such that quantizing $\bm{X}$ by rounding each scalar in $\bm{X}$ to the nearest quantization level minimizes the quantization MSE. 

The algorithm starts with an initial guess of quantization levels and then iteratively computes quantization thresholds $\{\tau_i\}_{i=1}^{2^B-1}$ and updates quantization levels $\{\hat{x}_i\}_{i=1}^{2^B}$. Specifically, at iteration $n$, thresholds are set to the midpoints of the previous iteration's levels:
\begin{align*}
    \tau_i^{(n)}=\frac{\hat{x}_i^{(n-1)}+\hat{x}_{i+1}^{(n-1)}}2 \text{ for } i=1\ldots 2^B-1
\end{align*}
Subsequently, the quantization levels are re-computed as conditional means of the data regions defined by the new thresholds:
\begin{align*}
    \hat{x}_i^{(n)}=\mathbb{E}\left[ \bm{X} \big| \bm{X}\in [\tau_{i-1}^{(n)},\tau_i^{(n)}] \right] \text{ for } i=1\ldots 2^B
\end{align*}
where to satisfy boundary conditions we have $\tau_0=-\infty$ and $\tau_{2^B}=\infty$. The algorithm iterates the above steps until convergence.

Figure \ref{fig:lm_quant} compares the quantization levels of a $7$-bit floating point (E3M3) quantizer (left) to a $7$-bit Lloyd-Max quantizer (right) when quantizing a layer of weights from the GPT3-126M model at a per-tensor granularity. As shown, the Lloyd-Max quantizer achieves substantially lower quantization MSE. Further, Table \ref{tab:FP7_vs_LM7} shows the superior perplexity achieved by Lloyd-Max quantizers for bitwidths of $7$, $6$ and $5$. The difference between the quantizers is clear at 5 bits, where per-tensor FP quantization incurs a drastic and unacceptable increase in perplexity, while Lloyd-Max quantization incurs a much smaller increase. Nevertheless, we note that even the optimal Lloyd-Max quantizer incurs a notable ($\sim 1.5$) increase in perplexity due to the coarse granularity of quantization. 

\begin{figure}[h]
  \centering
  \includegraphics[width=0.7\linewidth]{sections/figures/LM7_FP7.pdf}
  \caption{\small Quantization levels and the corresponding quantization MSE of Floating Point (left) vs Lloyd-Max (right) Quantizers for a layer of weights in the GPT3-126M model.}
  \label{fig:lm_quant}
\end{figure}

\begin{table}[h]\scriptsize
\begin{center}
\caption{\label{tab:FP7_vs_LM7} \small Comparing perplexity (lower is better) achieved by floating point quantizers and Lloyd-Max quantizers on a GPT3-126M model for the Wikitext-103 dataset.}
\begin{tabular}{c|cc|c}
\hline
 \multirow{2}{*}{\textbf{Bitwidth}} & \multicolumn{2}{|c|}{\textbf{Floating-Point Quantizer}} & \textbf{Lloyd-Max Quantizer} \\
 & Best Format & Wikitext-103 Perplexity & Wikitext-103 Perplexity \\
\hline
7 & E3M3 & 18.32 & 18.27 \\
6 & E3M2 & 19.07 & 18.51 \\
5 & E4M0 & 43.89 & 19.71 \\
\hline
\end{tabular}
\end{center}
\end{table}

\subsection{Proof of Local Optimality of LO-BCQ}
\label{subsec:lobcq_opt_proof}
For a given block $\bm{b}_j$, the quantization MSE during LO-BCQ can be empirically evaluated as $\frac{1}{L_b}\lVert \bm{b}_j- \bm{\hat{b}}_j\rVert^2_2$ where $\bm{\hat{b}}_j$ is computed from equation (\ref{eq:clustered_quantization_definition}) as $C_{f(\bm{b}_j)}(\bm{b}_j)$. Further, for a given block cluster $\mathcal{B}_i$, we compute the quantization MSE as $\frac{1}{|\mathcal{B}_{i}|}\sum_{\bm{b} \in \mathcal{B}_{i}} \frac{1}{L_b}\lVert \bm{b}- C_i^{(n)}(\bm{b})\rVert^2_2$. Therefore, at the end of iteration $n$, we evaluate the overall quantization MSE $J^{(n)}$ for a given operand $\bm{X}$ composed of $N_c$ block clusters as:
\begin{align*}
    \label{eq:mse_iter_n}
    J^{(n)} = \frac{1}{N_c} \sum_{i=1}^{N_c} \frac{1}{|\mathcal{B}_{i}^{(n)}|}\sum_{\bm{v} \in \mathcal{B}_{i}^{(n)}} \frac{1}{L_b}\lVert \bm{b}- B_i^{(n)}(\bm{b})\rVert^2_2
\end{align*}

At the end of iteration $n$, the codebooks are updated from $\mathcal{C}^{(n-1)}$ to $\mathcal{C}^{(n)}$. However, the mapping of a given vector $\bm{b}_j$ to quantizers $\mathcal{C}^{(n)}$ remains as  $f^{(n)}(\bm{b}_j)$. At the next iteration, during the vector clustering step, $f^{(n+1)}(\bm{b}_j)$ finds new mapping of $\bm{b}_j$ to updated codebooks $\mathcal{C}^{(n)}$ such that the quantization MSE over the candidate codebooks is minimized. Therefore, we obtain the following result for $\bm{b}_j$:
\begin{align*}
\frac{1}{L_b}\lVert \bm{b}_j - C_{f^{(n+1)}(\bm{b}_j)}^{(n)}(\bm{b}_j)\rVert^2_2 \le \frac{1}{L_b}\lVert \bm{b}_j - C_{f^{(n)}(\bm{b}_j)}^{(n)}(\bm{b}_j)\rVert^2_2
\end{align*}

That is, quantizing $\bm{b}_j$ at the end of the block clustering step of iteration $n+1$ results in lower quantization MSE compared to quantizing at the end of iteration $n$. Since this is true for all $\bm{b} \in \bm{X}$, we assert the following:
\begin{equation}
\begin{split}
\label{eq:mse_ineq_1}
    \tilde{J}^{(n+1)} &= \frac{1}{N_c} \sum_{i=1}^{N_c} \frac{1}{|\mathcal{B}_{i}^{(n+1)}|}\sum_{\bm{b} \in \mathcal{B}_{i}^{(n+1)}} \frac{1}{L_b}\lVert \bm{b} - C_i^{(n)}(b)\rVert^2_2 \le J^{(n)}
\end{split}
\end{equation}
where $\tilde{J}^{(n+1)}$ is the the quantization MSE after the vector clustering step at iteration $n+1$.

Next, during the codebook update step (\ref{eq:quantizers_update}) at iteration $n+1$, the per-cluster codebooks $\mathcal{C}^{(n)}$ are updated to $\mathcal{C}^{(n+1)}$ by invoking the Lloyd-Max algorithm \citep{Lloyd}. We know that for any given value distribution, the Lloyd-Max algorithm minimizes the quantization MSE. Therefore, for a given vector cluster $\mathcal{B}_i$ we obtain the following result:

\begin{equation}
    \frac{1}{|\mathcal{B}_{i}^{(n+1)}|}\sum_{\bm{b} \in \mathcal{B}_{i}^{(n+1)}} \frac{1}{L_b}\lVert \bm{b}- C_i^{(n+1)}(\bm{b})\rVert^2_2 \le \frac{1}{|\mathcal{B}_{i}^{(n+1)}|}\sum_{\bm{b} \in \mathcal{B}_{i}^{(n+1)}} \frac{1}{L_b}\lVert \bm{b}- C_i^{(n)}(\bm{b})\rVert^2_2
\end{equation}

The above equation states that quantizing the given block cluster $\mathcal{B}_i$ after updating the associated codebook from $C_i^{(n)}$ to $C_i^{(n+1)}$ results in lower quantization MSE. Since this is true for all the block clusters, we derive the following result: 
\begin{equation}
\begin{split}
\label{eq:mse_ineq_2}
     J^{(n+1)} &= \frac{1}{N_c} \sum_{i=1}^{N_c} \frac{1}{|\mathcal{B}_{i}^{(n+1)}|}\sum_{\bm{b} \in \mathcal{B}_{i}^{(n+1)}} \frac{1}{L_b}\lVert \bm{b}- C_i^{(n+1)}(\bm{b})\rVert^2_2  \le \tilde{J}^{(n+1)}   
\end{split}
\end{equation}

Following (\ref{eq:mse_ineq_1}) and (\ref{eq:mse_ineq_2}), we find that the quantization MSE is non-increasing for each iteration, that is, $J^{(1)} \ge J^{(2)} \ge J^{(3)} \ge \ldots \ge J^{(M)}$ where $M$ is the maximum number of iterations. 
%Therefore, we can say that if the algorithm converges, then it must be that it has converged to a local minimum. 
\hfill $\blacksquare$


\begin{figure}
    \begin{center}
    \includegraphics[width=0.5\textwidth]{sections//figures/mse_vs_iter.pdf}
    \end{center}
    \caption{\small NMSE vs iterations during LO-BCQ compared to other block quantization proposals}
    \label{fig:nmse_vs_iter}
\end{figure}

Figure \ref{fig:nmse_vs_iter} shows the empirical convergence of LO-BCQ across several block lengths and number of codebooks. Also, the MSE achieved by LO-BCQ is compared to baselines such as MXFP and VSQ. As shown, LO-BCQ converges to a lower MSE than the baselines. Further, we achieve better convergence for larger number of codebooks ($N_c$) and for a smaller block length ($L_b$), both of which increase the bitwidth of BCQ (see Eq \ref{eq:bitwidth_bcq}).


\subsection{Additional Accuracy Results}
%Table \ref{tab:lobcq_config} lists the various LOBCQ configurations and their corresponding bitwidths.
\begin{table}
\setlength{\tabcolsep}{4.75pt}
\begin{center}
\caption{\label{tab:lobcq_config} Various LO-BCQ configurations and their bitwidths.}
\begin{tabular}{|c||c|c|c|c||c|c||c|} 
\hline
 & \multicolumn{4}{|c||}{$L_b=8$} & \multicolumn{2}{|c||}{$L_b=4$} & $L_b=2$ \\
 \hline
 \backslashbox{$L_A$\kern-1em}{\kern-1em$N_c$} & 2 & 4 & 8 & 16 & 2 & 4 & 2 \\
 \hline
 64 & 4.25 & 4.375 & 4.5 & 4.625 & 4.375 & 4.625 & 4.625\\
 \hline
 32 & 4.375 & 4.5 & 4.625& 4.75 & 4.5 & 4.75 & 4.75 \\
 \hline
 16 & 4.625 & 4.75& 4.875 & 5 & 4.75 & 5 & 5 \\
 \hline
\end{tabular}
\end{center}
\end{table}

%\subsection{Perplexity achieved by various LO-BCQ configurations on Wikitext-103 dataset}

\begin{table} \centering
\begin{tabular}{|c||c|c|c|c||c|c||c|} 
\hline
 $L_b \rightarrow$& \multicolumn{4}{c||}{8} & \multicolumn{2}{c||}{4} & 2\\
 \hline
 \backslashbox{$L_A$\kern-1em}{\kern-1em$N_c$} & 2 & 4 & 8 & 16 & 2 & 4 & 2  \\
 %$N_c \rightarrow$ & 2 & 4 & 8 & 16 & 2 & 4 & 2 \\
 \hline
 \hline
 \multicolumn{8}{c}{GPT3-1.3B (FP32 PPL = 9.98)} \\ 
 \hline
 \hline
 64 & 10.40 & 10.23 & 10.17 & 10.15 &  10.28 & 10.18 & 10.19 \\
 \hline
 32 & 10.25 & 10.20 & 10.15 & 10.12 &  10.23 & 10.17 & 10.17 \\
 \hline
 16 & 10.22 & 10.16 & 10.10 & 10.09 &  10.21 & 10.14 & 10.16 \\
 \hline
  \hline
 \multicolumn{8}{c}{GPT3-8B (FP32 PPL = 7.38)} \\ 
 \hline
 \hline
 64 & 7.61 & 7.52 & 7.48 &  7.47 &  7.55 &  7.49 & 7.50 \\
 \hline
 32 & 7.52 & 7.50 & 7.46 &  7.45 &  7.52 &  7.48 & 7.48  \\
 \hline
 16 & 7.51 & 7.48 & 7.44 &  7.44 &  7.51 &  7.49 & 7.47  \\
 \hline
\end{tabular}
\caption{\label{tab:ppl_gpt3_abalation} Wikitext-103 perplexity across GPT3-1.3B and 8B models.}
\end{table}

\begin{table} \centering
\begin{tabular}{|c||c|c|c|c||} 
\hline
 $L_b \rightarrow$& \multicolumn{4}{c||}{8}\\
 \hline
 \backslashbox{$L_A$\kern-1em}{\kern-1em$N_c$} & 2 & 4 & 8 & 16 \\
 %$N_c \rightarrow$ & 2 & 4 & 8 & 16 & 2 & 4 & 2 \\
 \hline
 \hline
 \multicolumn{5}{|c|}{Llama2-7B (FP32 PPL = 5.06)} \\ 
 \hline
 \hline
 64 & 5.31 & 5.26 & 5.19 & 5.18  \\
 \hline
 32 & 5.23 & 5.25 & 5.18 & 5.15  \\
 \hline
 16 & 5.23 & 5.19 & 5.16 & 5.14  \\
 \hline
 \multicolumn{5}{|c|}{Nemotron4-15B (FP32 PPL = 5.87)} \\ 
 \hline
 \hline
 64  & 6.3 & 6.20 & 6.13 & 6.08  \\
 \hline
 32  & 6.24 & 6.12 & 6.07 & 6.03  \\
 \hline
 16  & 6.12 & 6.14 & 6.04 & 6.02  \\
 \hline
 \multicolumn{5}{|c|}{Nemotron4-340B (FP32 PPL = 3.48)} \\ 
 \hline
 \hline
 64 & 3.67 & 3.62 & 3.60 & 3.59 \\
 \hline
 32 & 3.63 & 3.61 & 3.59 & 3.56 \\
 \hline
 16 & 3.61 & 3.58 & 3.57 & 3.55 \\
 \hline
\end{tabular}
\caption{\label{tab:ppl_llama7B_nemo15B} Wikitext-103 perplexity compared to FP32 baseline in Llama2-7B and Nemotron4-15B, 340B models}
\end{table}

%\subsection{Perplexity achieved by various LO-BCQ configurations on MMLU dataset}


\begin{table} \centering
\begin{tabular}{|c||c|c|c|c||c|c|c|c|} 
\hline
 $L_b \rightarrow$& \multicolumn{4}{c||}{8} & \multicolumn{4}{c||}{8}\\
 \hline
 \backslashbox{$L_A$\kern-1em}{\kern-1em$N_c$} & 2 & 4 & 8 & 16 & 2 & 4 & 8 & 16  \\
 %$N_c \rightarrow$ & 2 & 4 & 8 & 16 & 2 & 4 & 2 \\
 \hline
 \hline
 \multicolumn{5}{|c|}{Llama2-7B (FP32 Accuracy = 45.8\%)} & \multicolumn{4}{|c|}{Llama2-70B (FP32 Accuracy = 69.12\%)} \\ 
 \hline
 \hline
 64 & 43.9 & 43.4 & 43.9 & 44.9 & 68.07 & 68.27 & 68.17 & 68.75 \\
 \hline
 32 & 44.5 & 43.8 & 44.9 & 44.5 & 68.37 & 68.51 & 68.35 & 68.27  \\
 \hline
 16 & 43.9 & 42.7 & 44.9 & 45 & 68.12 & 68.77 & 68.31 & 68.59  \\
 \hline
 \hline
 \multicolumn{5}{|c|}{GPT3-22B (FP32 Accuracy = 38.75\%)} & \multicolumn{4}{|c|}{Nemotron4-15B (FP32 Accuracy = 64.3\%)} \\ 
 \hline
 \hline
 64 & 36.71 & 38.85 & 38.13 & 38.92 & 63.17 & 62.36 & 63.72 & 64.09 \\
 \hline
 32 & 37.95 & 38.69 & 39.45 & 38.34 & 64.05 & 62.30 & 63.8 & 64.33  \\
 \hline
 16 & 38.88 & 38.80 & 38.31 & 38.92 & 63.22 & 63.51 & 63.93 & 64.43  \\
 \hline
\end{tabular}
\caption{\label{tab:mmlu_abalation} Accuracy on MMLU dataset across GPT3-22B, Llama2-7B, 70B and Nemotron4-15B models.}
\end{table}


%\subsection{Perplexity achieved by various LO-BCQ configurations on LM evaluation harness}

\begin{table} \centering
\begin{tabular}{|c||c|c|c|c||c|c|c|c|} 
\hline
 $L_b \rightarrow$& \multicolumn{4}{c||}{8} & \multicolumn{4}{c||}{8}\\
 \hline
 \backslashbox{$L_A$\kern-1em}{\kern-1em$N_c$} & 2 & 4 & 8 & 16 & 2 & 4 & 8 & 16  \\
 %$N_c \rightarrow$ & 2 & 4 & 8 & 16 & 2 & 4 & 2 \\
 \hline
 \hline
 \multicolumn{5}{|c|}{Race (FP32 Accuracy = 37.51\%)} & \multicolumn{4}{|c|}{Boolq (FP32 Accuracy = 64.62\%)} \\ 
 \hline
 \hline
 64 & 36.94 & 37.13 & 36.27 & 37.13 & 63.73 & 62.26 & 63.49 & 63.36 \\
 \hline
 32 & 37.03 & 36.36 & 36.08 & 37.03 & 62.54 & 63.51 & 63.49 & 63.55  \\
 \hline
 16 & 37.03 & 37.03 & 36.46 & 37.03 & 61.1 & 63.79 & 63.58 & 63.33  \\
 \hline
 \hline
 \multicolumn{5}{|c|}{Winogrande (FP32 Accuracy = 58.01\%)} & \multicolumn{4}{|c|}{Piqa (FP32 Accuracy = 74.21\%)} \\ 
 \hline
 \hline
 64 & 58.17 & 57.22 & 57.85 & 58.33 & 73.01 & 73.07 & 73.07 & 72.80 \\
 \hline
 32 & 59.12 & 58.09 & 57.85 & 58.41 & 73.01 & 73.94 & 72.74 & 73.18  \\
 \hline
 16 & 57.93 & 58.88 & 57.93 & 58.56 & 73.94 & 72.80 & 73.01 & 73.94  \\
 \hline
\end{tabular}
\caption{\label{tab:mmlu_abalation} Accuracy on LM evaluation harness tasks on GPT3-1.3B model.}
\end{table}

\begin{table} \centering
\begin{tabular}{|c||c|c|c|c||c|c|c|c|} 
\hline
 $L_b \rightarrow$& \multicolumn{4}{c||}{8} & \multicolumn{4}{c||}{8}\\
 \hline
 \backslashbox{$L_A$\kern-1em}{\kern-1em$N_c$} & 2 & 4 & 8 & 16 & 2 & 4 & 8 & 16  \\
 %$N_c \rightarrow$ & 2 & 4 & 8 & 16 & 2 & 4 & 2 \\
 \hline
 \hline
 \multicolumn{5}{|c|}{Race (FP32 Accuracy = 41.34\%)} & \multicolumn{4}{|c|}{Boolq (FP32 Accuracy = 68.32\%)} \\ 
 \hline
 \hline
 64 & 40.48 & 40.10 & 39.43 & 39.90 & 69.20 & 68.41 & 69.45 & 68.56 \\
 \hline
 32 & 39.52 & 39.52 & 40.77 & 39.62 & 68.32 & 67.43 & 68.17 & 69.30  \\
 \hline
 16 & 39.81 & 39.71 & 39.90 & 40.38 & 68.10 & 66.33 & 69.51 & 69.42  \\
 \hline
 \hline
 \multicolumn{5}{|c|}{Winogrande (FP32 Accuracy = 67.88\%)} & \multicolumn{4}{|c|}{Piqa (FP32 Accuracy = 78.78\%)} \\ 
 \hline
 \hline
 64 & 66.85 & 66.61 & 67.72 & 67.88 & 77.31 & 77.42 & 77.75 & 77.64 \\
 \hline
 32 & 67.25 & 67.72 & 67.72 & 67.00 & 77.31 & 77.04 & 77.80 & 77.37  \\
 \hline
 16 & 68.11 & 68.90 & 67.88 & 67.48 & 77.37 & 78.13 & 78.13 & 77.69  \\
 \hline
\end{tabular}
\caption{\label{tab:mmlu_abalation} Accuracy on LM evaluation harness tasks on GPT3-8B model.}
\end{table}

\begin{table} \centering
\begin{tabular}{|c||c|c|c|c||c|c|c|c|} 
\hline
 $L_b \rightarrow$& \multicolumn{4}{c||}{8} & \multicolumn{4}{c||}{8}\\
 \hline
 \backslashbox{$L_A$\kern-1em}{\kern-1em$N_c$} & 2 & 4 & 8 & 16 & 2 & 4 & 8 & 16  \\
 %$N_c \rightarrow$ & 2 & 4 & 8 & 16 & 2 & 4 & 2 \\
 \hline
 \hline
 \multicolumn{5}{|c|}{Race (FP32 Accuracy = 40.67\%)} & \multicolumn{4}{|c|}{Boolq (FP32 Accuracy = 76.54\%)} \\ 
 \hline
 \hline
 64 & 40.48 & 40.10 & 39.43 & 39.90 & 75.41 & 75.11 & 77.09 & 75.66 \\
 \hline
 32 & 39.52 & 39.52 & 40.77 & 39.62 & 76.02 & 76.02 & 75.96 & 75.35  \\
 \hline
 16 & 39.81 & 39.71 & 39.90 & 40.38 & 75.05 & 73.82 & 75.72 & 76.09  \\
 \hline
 \hline
 \multicolumn{5}{|c|}{Winogrande (FP32 Accuracy = 70.64\%)} & \multicolumn{4}{|c|}{Piqa (FP32 Accuracy = 79.16\%)} \\ 
 \hline
 \hline
 64 & 69.14 & 70.17 & 70.17 & 70.56 & 78.24 & 79.00 & 78.62 & 78.73 \\
 \hline
 32 & 70.96 & 69.69 & 71.27 & 69.30 & 78.56 & 79.49 & 79.16 & 78.89  \\
 \hline
 16 & 71.03 & 69.53 & 69.69 & 70.40 & 78.13 & 79.16 & 79.00 & 79.00  \\
 \hline
\end{tabular}
\caption{\label{tab:mmlu_abalation} Accuracy on LM evaluation harness tasks on GPT3-22B model.}
\end{table}

\begin{table} \centering
\begin{tabular}{|c||c|c|c|c||c|c|c|c|} 
\hline
 $L_b \rightarrow$& \multicolumn{4}{c||}{8} & \multicolumn{4}{c||}{8}\\
 \hline
 \backslashbox{$L_A$\kern-1em}{\kern-1em$N_c$} & 2 & 4 & 8 & 16 & 2 & 4 & 8 & 16  \\
 %$N_c \rightarrow$ & 2 & 4 & 8 & 16 & 2 & 4 & 2 \\
 \hline
 \hline
 \multicolumn{5}{|c|}{Race (FP32 Accuracy = 44.4\%)} & \multicolumn{4}{|c|}{Boolq (FP32 Accuracy = 79.29\%)} \\ 
 \hline
 \hline
 64 & 42.49 & 42.51 & 42.58 & 43.45 & 77.58 & 77.37 & 77.43 & 78.1 \\
 \hline
 32 & 43.35 & 42.49 & 43.64 & 43.73 & 77.86 & 75.32 & 77.28 & 77.86  \\
 \hline
 16 & 44.21 & 44.21 & 43.64 & 42.97 & 78.65 & 77 & 76.94 & 77.98  \\
 \hline
 \hline
 \multicolumn{5}{|c|}{Winogrande (FP32 Accuracy = 69.38\%)} & \multicolumn{4}{|c|}{Piqa (FP32 Accuracy = 78.07\%)} \\ 
 \hline
 \hline
 64 & 68.9 & 68.43 & 69.77 & 68.19 & 77.09 & 76.82 & 77.09 & 77.86 \\
 \hline
 32 & 69.38 & 68.51 & 68.82 & 68.90 & 78.07 & 76.71 & 78.07 & 77.86  \\
 \hline
 16 & 69.53 & 67.09 & 69.38 & 68.90 & 77.37 & 77.8 & 77.91 & 77.69  \\
 \hline
\end{tabular}
\caption{\label{tab:mmlu_abalation} Accuracy on LM evaluation harness tasks on Llama2-7B model.}
\end{table}

\begin{table} \centering
\begin{tabular}{|c||c|c|c|c||c|c|c|c|} 
\hline
 $L_b \rightarrow$& \multicolumn{4}{c||}{8} & \multicolumn{4}{c||}{8}\\
 \hline
 \backslashbox{$L_A$\kern-1em}{\kern-1em$N_c$} & 2 & 4 & 8 & 16 & 2 & 4 & 8 & 16  \\
 %$N_c \rightarrow$ & 2 & 4 & 8 & 16 & 2 & 4 & 2 \\
 \hline
 \hline
 \multicolumn{5}{|c|}{Race (FP32 Accuracy = 48.8\%)} & \multicolumn{4}{|c|}{Boolq (FP32 Accuracy = 85.23\%)} \\ 
 \hline
 \hline
 64 & 49.00 & 49.00 & 49.28 & 48.71 & 82.82 & 84.28 & 84.03 & 84.25 \\
 \hline
 32 & 49.57 & 48.52 & 48.33 & 49.28 & 83.85 & 84.46 & 84.31 & 84.93  \\
 \hline
 16 & 49.85 & 49.09 & 49.28 & 48.99 & 85.11 & 84.46 & 84.61 & 83.94  \\
 \hline
 \hline
 \multicolumn{5}{|c|}{Winogrande (FP32 Accuracy = 79.95\%)} & \multicolumn{4}{|c|}{Piqa (FP32 Accuracy = 81.56\%)} \\ 
 \hline
 \hline
 64 & 78.77 & 78.45 & 78.37 & 79.16 & 81.45 & 80.69 & 81.45 & 81.5 \\
 \hline
 32 & 78.45 & 79.01 & 78.69 & 80.66 & 81.56 & 80.58 & 81.18 & 81.34  \\
 \hline
 16 & 79.95 & 79.56 & 79.79 & 79.72 & 81.28 & 81.66 & 81.28 & 80.96  \\
 \hline
\end{tabular}
\caption{\label{tab:mmlu_abalation} Accuracy on LM evaluation harness tasks on Llama2-70B model.}
\end{table}

%\section{MSE Studies}
%\textcolor{red}{TODO}


\subsection{Number Formats and Quantization Method}
\label{subsec:numFormats_quantMethod}
\subsubsection{Integer Format}
An $n$-bit signed integer (INT) is typically represented with a 2s-complement format \citep{yao2022zeroquant,xiao2023smoothquant,dai2021vsq}, where the most significant bit denotes the sign.

\subsubsection{Floating Point Format}
An $n$-bit signed floating point (FP) number $x$ comprises of a 1-bit sign ($x_{\mathrm{sign}}$), $B_m$-bit mantissa ($x_{\mathrm{mant}}$) and $B_e$-bit exponent ($x_{\mathrm{exp}}$) such that $B_m+B_e=n-1$. The associated constant exponent bias ($E_{\mathrm{bias}}$) is computed as $(2^{{B_e}-1}-1)$. We denote this format as $E_{B_e}M_{B_m}$.  

\subsubsection{Quantization Scheme}
\label{subsec:quant_method}
A quantization scheme dictates how a given unquantized tensor is converted to its quantized representation. We consider FP formats for the purpose of illustration. Given an unquantized tensor $\bm{X}$ and an FP format $E_{B_e}M_{B_m}$, we first, we compute the quantization scale factor $s_X$ that maps the maximum absolute value of $\bm{X}$ to the maximum quantization level of the $E_{B_e}M_{B_m}$ format as follows:
\begin{align}
\label{eq:sf}
    s_X = \frac{\mathrm{max}(|\bm{X}|)}{\mathrm{max}(E_{B_e}M_{B_m})}
\end{align}
In the above equation, $|\cdot|$ denotes the absolute value function.

Next, we scale $\bm{X}$ by $s_X$ and quantize it to $\hat{\bm{X}}$ by rounding it to the nearest quantization level of $E_{B_e}M_{B_m}$ as:

\begin{align}
\label{eq:tensor_quant}
    \hat{\bm{X}} = \text{round-to-nearest}\left(\frac{\bm{X}}{s_X}, E_{B_e}M_{B_m}\right)
\end{align}

We perform dynamic max-scaled quantization \citep{wu2020integer}, where the scale factor $s$ for activations is dynamically computed during runtime.

\subsection{Vector Scaled Quantization}
\begin{wrapfigure}{r}{0.35\linewidth}
  \centering
  \includegraphics[width=\linewidth]{sections/figures/vsquant.jpg}
  \caption{\small Vectorwise decomposition for per-vector scaled quantization (VSQ \citep{dai2021vsq}).}
  \label{fig:vsquant}
\end{wrapfigure}
During VSQ \citep{dai2021vsq}, the operand tensors are decomposed into 1D vectors in a hardware friendly manner as shown in Figure \ref{fig:vsquant}. Since the decomposed tensors are used as operands in matrix multiplications during inference, it is beneficial to perform this decomposition along the reduction dimension of the multiplication. The vectorwise quantization is performed similar to tensorwise quantization described in Equations \ref{eq:sf} and \ref{eq:tensor_quant}, where a scale factor $s_v$ is required for each vector $\bm{v}$ that maps the maximum absolute value of that vector to the maximum quantization level. While smaller vector lengths can lead to larger accuracy gains, the associated memory and computational overheads due to the per-vector scale factors increases. To alleviate these overheads, VSQ \citep{dai2021vsq} proposed a second level quantization of the per-vector scale factors to unsigned integers, while MX \citep{rouhani2023shared} quantizes them to integer powers of 2 (denoted as $2^{INT}$).

\subsubsection{MX Format}
The MX format proposed in \citep{rouhani2023microscaling} introduces the concept of sub-block shifting. For every two scalar elements of $b$-bits each, there is a shared exponent bit. The value of this exponent bit is determined through an empirical analysis that targets minimizing quantization MSE. We note that the FP format $E_{1}M_{b}$ is strictly better than MX from an accuracy perspective since it allocates a dedicated exponent bit to each scalar as opposed to sharing it across two scalars. Therefore, we conservatively bound the accuracy of a $b+2$-bit signed MX format with that of a $E_{1}M_{b}$ format in our comparisons. For instance, we use E1M2 format as a proxy for MX4.

\begin{figure}
    \centering
    \includegraphics[width=1\linewidth]{sections//figures/BlockFormats.pdf}
    \caption{\small Comparing LO-BCQ to MX format.}
    \label{fig:block_formats}
\end{figure}

Figure \ref{fig:block_formats} compares our $4$-bit LO-BCQ block format to MX \citep{rouhani2023microscaling}. As shown, both LO-BCQ and MX decompose a given operand tensor into block arrays and each block array into blocks. Similar to MX, we find that per-block quantization ($L_b < L_A$) leads to better accuracy due to increased flexibility. While MX achieves this through per-block $1$-bit micro-scales, we associate a dedicated codebook to each block through a per-block codebook selector. Further, MX quantizes the per-block array scale-factor to E8M0 format without per-tensor scaling. In contrast during LO-BCQ, we find that per-tensor scaling combined with quantization of per-block array scale-factor to E4M3 format results in superior inference accuracy across models. 


\end{document}

