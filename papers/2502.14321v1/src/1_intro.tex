\section{Introduction}
Large language models (LLMs) have recently demonstrated significant potential across diverse domains. Building upon these strengths, LLMs have been integrated into autonomous agents equipped with modules for profiling, memorization, planning, and action~\cite{llm_agent_define}. To make LLM-based agents more coordinated and scalable when handling complex or dynamic tasks~\cite{single_limit}, LLM-based multi-agent systems (LLM-MAS) have been proposed, where multiple agents interact to achieve goals that exceed the capacity of a single agent. Recent studies underscore LLM-MAS effectiveness in contexts ranging from social simulation\cite{social_media_regulation}, to software engineering\cite{metagpt}, and recommendation systems~\cite{jd_recommendation_system}, illustrating the growing demand for more coordinated and intelligent multi-agent systems. 

% \begin{table}[htbp]
% \centering
% \label{tab:table1}
% \resizebox{\columnwidth}{!}{%
% \begin{tabular}{p{2cm} |p{2.8cm}| p{4.3cm}| p{4cm}}
% \toprule
% \textbf{Category} & \textbf{References} & \textbf{Focus} & \textbf{Limitation} \\
% \midrule
% \multirow{3}{*}{\parbox[t]{2cm}{General Overviews}} 
%   & \cite{mas_survey_1} 
%     & \multirow{3}{*}{\parbox[t]{4.3cm}{Organizational structures, agent-level actions, application scenarios.}} 
%     & \multirow{3}{*}{\parbox[t]{4cm}{Limited discussion on inter-agent communication and coordination workflows.}} \\
%   & \cite{mas_survey_2} 
%     &  &  \\
%   & \cite{mas_survey_3} 
%     &  &  \\
% \midrule
% \multirow{3}{*}{\parbox[t]{2cm}{Domain-Specific Investigations}} 
%   & \cite{mas_application_survey_simulation} 
%     & \multirow{3}{*}{\parbox[t]{4.3cm}{In-depth exploration of LLM-MAS usage in specific fields.}} 
%     & \multirow{3}{*}{\parbox[t]{4cm}{Less generalizable; does not provide a holistic framework for diverse applications.}} \\
%   & \cite{mas_application_survey_1} 
%     &  &  \\
%   & \cite{mas_application_survey_2} 
%     &  &  \\
% \midrule
% \multirow{4}{*}{\parbox[t]{2cm}{Ours}} 
%   & \multirow{4}{*}{---} 
%     & \multirow{4}{*}{\parbox[t]{4.3cm}{Propose a macro to micro framework from a communication perspective applicable to all LLM-MAS workflows.}} 
%     & \multirow{4}{*}{---} \\
%   &  &  &  \\
%   &  &  &  \\
%   &  &  &  \\
% \bottomrule
% \end{tabular}}
% \caption{Comparison of Existing LLM-MAS Surveys: Categories, Focus, and Limitations.}
% \end{table}


% \begin{table}[htbp]
%   \centering
%   \resizebox{\columnwidth}{!}{%
%     \begin{tabular}{c|c|c|c}
%     \toprule
%     \multirow{2}[0]{*}{\textbf{Category}} & \multirow{2}[0]{*}{\textbf{References}} & \multirow{2}[0]{*}{\textbf{Focus}} & \multirow{2}[0]{*}{\textbf{Limitation}} \\
%           &       &       &  \\
%     \midrule
%     {\multirow{3}[0]{*}{\makecell{General\\Overviews}}} & \cite{mas_survey_1} & \multirow{3}[0]{*}{\makecell{Organizational structures,\\agent-level actions,\\application scenarios}} & \multirow{3}[0]{*}{\makecell{Limited discussion on inter-\\agent communication and\\coordination workflows.}} \\
%           & \cite{mas_survey_2} &       &  \\
%           & \cite{mas_survey_3} &       &  \\
%     \midrule
%     \multirow{3}[0]{*}{\makecell{Domain-Specific\\Investigations}} & \cite{mas_application_survey_simulation} & \multirow{3}[0]{*}{\makecell{In-depth exploration\\of LLM-MAS usage\\in specific fields.}} & \multirow{3}[0]{*}{\makecell{Less generalizable; Doesn't\\provide a holistic framework\\for diverse applications.}} \\
%           & \cite{mas_application_survey_1} &       &  \\
%           & \cite{mas_application_survey_2} &       &  \\
%     \midrule
%     \multirow{5}[0]{*}{Ours} & \multirow{5}[0]{*}{---} & \multirow{5}[0]{*}{\makecell{Propose a macro to\\micro framework from\\a communication perspective\\applicable to all\\LLM-MAS workflows.}} & \multirow{5}[0]{*}{---} \\
%           &       &       &  \\
%           &       &       &  \\
%           &       &       &  \\
%           &       &       &  \\
%     \bottomrule
%     \end{tabular}%
%     }
%   \label{tab:addlabel}%
  
% \end{table}%


\begin{table}[htbp]
  \centering
  \resizebox{\columnwidth}{!}{%
    \begin{tabular}{c|c|c}
    \toprule
    \multirow{2}[0]{*}{\textbf{Category}} & \multirow{2}[0]{*}{\textbf{Main Content}} & \multirow{2}[0]{*}{\textbf{Limitation}} \\
          &       &  \\
    \midrule
    \multirow{3}[0]{*}{\makecell{General Overviews}} & \multirow{3}[0]{*}{\makecell[l]{Organizational structures,\\agent-level actions, and\\application scenarios}} & \multirow{3}[0]{*}{\makecell[l]{Limited inter-agent comm-\\unication and workflows}} \\
        &       &  \\
        &       &  \\
    \midrule
    \multirow{3}[0]{*}{\makecell{Domain-Specific\\Investigations}} & \multirow{3}[0]{*}{\makecell[l]{LLM-MAS workflow\\in specific domains}} & \multirow{3}[0]{*}{\makecell[l]{Limited generalizability}} \\
        &       &  \\
        &       &  \\
    \midrule
    \multirow{5}[0]{*}{\makecell[l]{Communication-Centric\\Analysis of LLM-MAS}} & \multirow{5}[0]{*}{\makecell[l]{Propose a framework\\applicable to all LLM-\\MAS workflows.}} & \multirow{5}[0]{*}{---} \\
           &       &  \\
           &       &  \\
           &       &  \\
           &       &  \\
    \bottomrule
    \end{tabular}%
    }
  \label{tab:addlabel}%
  \caption{Comparison of Existing LLM-MAS Surveys: Categories, Main Content, and Limitation.}
\end{table}



Given the broad application scope and developmental potential of LLM-MAS, several surveys have appeared to help researchers quickly grasping this emerging research area. As shown in Table 1, these surveys can be broadly categorized into two types: general overviews and domain-specific surveys. General overviews typically focus on system architectures, agent-level actions, and potential application scenarios~\cite{mas_survey_1,mas_survey_2,mas_survey_3}. However, they often provide limited coverage of the communication and coordination workflows among agents, which are critical for effective multi-agent collaboration. In contrast, domain-specific surveys delve into particular use cases such as social simulation~\cite{mas_application_survey_simulation} or software engineering~\cite{mas_application_survey_1,mas_application_survey_2}. While they go into great details about the workflow of LLM-MAS in specific domains, they are less generalizable and do not provide a holistic framework for understanding LLM-MAS across diverse applications.


% Subsequent works \cite{mas_survey_3,mas_survey_2} further summarize LLM-MAS. The former introduces LLM-MAS structures (equal-level, hierarchical, nested, dynamic) and emphasizes current challenges. The latter describes interfaces, profiling, and communication structures, but focuses primarily on application scenarios. Both discussions are largely confined to surface-level features such as system architecture but without providing sufficiently detailed insights into the internal workflows. In contrast, \cite{mas_survey_1} does introduce an LLM-MAS workflow, but from a single-agent perspective in the system—covering agent roles, action classifications, and organizational structures. While informative, this single-agent viewpoint cannot fully illustrate multi-agent interactions and workflows, particularly regarding communication paradigms and strategies. Other surveys target specific domains, such as \cite{mas_application_survey_simulation} for simulation and emulation, or \cite{mas_application_survey_1,mas_application_survey_2} for software engineering. Although these works offer valuable insights within their respective domains, they are less generalizable for broader applications.

We observe that the key to LLM-MAS being able to accomplish more complex tasks compared to single-agent systems is inter-agent communication, which enables agents to exchange ideas and coordinate plans. Inspired by ~\cite{communication_1,communication_2}, we find that the definitions in the traditional communication domain such as source and channel can be matched with the communication process in the LLM-MAS workflow. Therefore, we propose to decompose the LLM-MAS workflow from the perspective of communication. Specifically, we define LLM-MAS as an automated system driven by communication goals within a predefined communication architecture. Agents in this system have multiple communication strategies and paradigms, interacting with various communication objects to exchange diverse content to for task completion.

% 缺一段架构介绍
As shown in Figure 1, We propose a two-level framework distinguished between system-level communication(Figure 1a) and system internal communication(Figure 1b). The system-level communication focus on how agents are organized and their overarching communication goals. The system internal communication delve into the internal dynamics of communication within the system, including strategies, paradigms, objects and content. This structure provides a holistic understanding of the workflow across varied tasks both at the macro and micro levels of the LLM-MAS.

Our main contributions include:
\begin{itemize}
    \item \textbf{Comprehensive Framework:} From the perspective of communication, we designed a comprehensive framework applicable to all types of LLM-MAS, analyzing the system workflow from macro to micro.
    \item \textbf{Deep Analysis of Communication Processes:} We dissect real-world examples and prototypes to illustrate how well-orchestrated communication leads to more effective multi-agent behavior.
    \item \textbf{Identification of Challenges and Opportunities:} We shed light on open issues like scalability, security, and multimodal integration, and offer potential research directions for both academia and industry.
\end{itemize}



% A key factor behind LLM-MAS success lies in the ability of agents to communicate with each other in natural language, enabling them to share knowledge, negotiate decisions, and collaboratively plan actions in real time. Unlike traditional multi-agent systems that rely on constrained messaging or symbolic languages, LLMs’ advanced linguistic capacity opens up a more flexible and adaptive interaction paradigm. This flexibility is crucial for handling large-scale, highly dynamic, or open-ended tasks where continuous information exchange is required for decision making.

% Consequently, inter-agent communication has become a defining characteristic of LLM-MAS, allowing agents to exhibit emergent behaviors, from coordinated teamwork to sophisticated competitive strategies. However, designing and analyzing such communication workflows—covering how messages are structured, how information is shared, and how conflicts or collaborations emerge—remain open challenges.

% Several recent surveys have reviewed LLM-based multi-agent systems from different angles. For instance, some works \cite{agent_survey_2,agent_survey_1} focus on agent composition (e.g., perception, action, memory) or highlight prospective applications across industries. Subsequent works \cite{mas_survey_3,mas_survey_2} emphasize system architectures—distinguishing between flat, hierarchical, or nested forms—and discuss challenges in a broad sense. Meanwhile, domain-specific surveys \cite{mas_application_survey_simulation,mas_application_survey_1,mas_application_survey_2} delve into specialized areas like simulation or software engineering. Although these works offer valuable insights within their respective domains, they are less generalizable for broader applications. Despite these efforts, two critical limitations become apparent. First, most discussions offer only surface-level overviews of LLM-MAS architectures or high-level features, without systematically dissecting how agents actually communicate (e.g., the sequence of message exchanges, the content structure, or the paradigms employed). Second, even when workflows are mentioned \cite{mas_survey_1}, they tend to be framed from a single-agent perspective, overlooking the cooperative, competitive, or hybrid multi-agent communication that defines more complex systems. Consequently, a cohesive, communication-centered framework is needed to elucidate how multi-agent interactions unfold in LLM-powered environments.

% In response to these gaps, our survey aims to systematically examine how LLM-MAS can be understood and optimized through the lens of communication. Specifically, we observe that inter-agent communication is a defining characteristic of LLM-MAS, enabling agents to exchange ideas and coordinate strategies at scale. Drawing inspiration from prior communication research~\cite{communication_1,communication_2} and classical multi-agent theories, we propose a two-level framework. First, we investigate system-level aspects such as communication architectures (e.g., hierarchical, team, or society-based) and communication goals (cooperation vs. competition vs. mixed). Next, we delve into system-internal perspectives, dissecting strategies, paradigms, objects, and content in agent interactions.


% Through this macro-to-micro perspective, we bridge theory and practical design, offering a detailed taxonomy that can be applied across varied tasks, including collaborative code generation and complex social simulations. Our main contributions include:
% \begin{itemize}
%     \item \textbf{Comprehensive Framework:} We unify diverse LLM-MAS studies under a \textit{communication-centered} taxonomy, highlighting the roles of architecture, strategy, and content.
%     \item \textbf{Deep Analysis of Communication Processes:} We dissect real-world examples and prototypes to illustrate how well-orchestrated communication leads to more effective multi-agent behavior.
%     \item \textbf{Identification of Challenges and Opportunities:} We shed light on open issues like scalability, security, and multimodal integration, and offer potential research directions for both academia and industry.
% \end{itemize}




