\section{Background}
\subsection{Single-Agent Systems}
Single-agent systems represent the cornerstone of multi-agent architectures, as they lay the groundwork for understanding how individual agents reason, plan, and act. In line with the taxonomy proposed by ~\cite{agent_survey_1}, this section outlines the composition and functionalities of LLM-based single-agent systems, providing the foundational concepts needed before delving into multi-agent scenarios.

An LLM-based single-agent system typically consists of three key components: \textit{Brain}, \textit{Perception}, and \textit{Action}, each playing a distinct yet complementary role. 1) \textbf{Brain} is a LLM that integrates short-term and long-term memory modules to reduce potential hallucinations and enhance the agent's reasoning and planning capabilities. Meanwhile, various retrieval augmented generation(RAG) methods~\cite{rag_1} and RAG enhancement research~\cite{rag_2} play an important role in enhancing memory modules. 2) \textbf{Perception} is text or multimodal input (visual/auditory) for richer state perception. 
3) \textbf{Actions} are represented as integrations with external tools such as web APIs or real-world actuators~\cite{embodied_agents} to expand the agent’s problem-solving scope.

\subsection{Multi-Agent Systems}
While single-agent systems exhibit strong individual reasoning, they often struggle with tasks requiring collective intelligence or large-scale coordination. Multi-agent systems (MAS)~\cite{mas_define} can address these limits by orchestrating multiple intelligent agents and leveraging communication as a key mechanism for goal alignment. 

Agents can cooperate, compete, or negotiate, depending on system objectives and architectural choices. Such flexibility in communication design sets the stage for LLM-MAS, where advanced language models enable more sophisticated inter-agent interactions across diverse application domains.


