\section{Challenges and Opportunities}
As LLM-MAS continue to garner increasing attention in both research and application domains, their further development faces several significant challenges and emerging opportunities. We analyze several key challenges and research directions based on the content of the article.
\subsection{Optimizing the System Design}
The communication architecture is the foundational component of LLM-MAS. As task complexity increases, traditional communication architectures may no longer suffice. Therefore, the design of hybrid architectures is expected to be a key focus of future research. With more complex structures, the number of agents increases, leading to greater demands on computational resources. Therefore, a major challenge lies in developing communication paradigms that are both efficient and scalable. Meanwhile, how to optimize the allocation of computational resources is also need to consider. Concurrently, the increasing volume of internal system information poses another challenge. Ensuring that agents correctly interpret and understand this information, while minimizing the risk of hallucinations or misunderstandings, will be a crucial area of investigation.
\subsection{Advancing Research on Agent Competition}
In a competitive environment, agents can develop more complex strategies, improve decision-making, and promote innovative behaviors by employing techniques such as game theory. However, a key challenge lies in balancing competition and cooperation, as excessive competition may lead to inefficiency or instability. Future research can focus on finding the optimal balance between competition and cooperation, developing scalable competition strategies, and exploring how to safely and effectively integrate competition into real-world applications.
\subsection{Communicate Multimodal Content}
With the development of large multimodal models, agents in LLM-MAS should not be limited to text-based communication. Communication of multimodal content (text, images, audio, and video) should also be considered.This expansion into multimodal content enables more natural and context-aware interactions, thereby enhancing agents' adaptability and decision-making capabilities. However, there are some challenges to integrating multimodal content. A major issue is how to effectively present and coordinate different modalities in a coherent way that is comprehensible to all agents. In addition, agents not only have to process these different modalities, but also communicate them effectively to one another. Future research should focus on improving the fusion of multimodal data and designing stronger agents in key components for handing multimodal content. 
\subsection{Communication Security}
As LLM-MAS becomes more complex and integrated into real-world applications, ensuring the confidentiality, integrity, and authenticity of this communication is essential for maintaining the system's overall security and functionality. One of the main challenges of communication security is protection against malicious attacks such as eavesdropping, data tampering, or spoofing. A malicious attacker may exploit vulnerabilities in the communication protocol to inject misleading or harmful content. Furthermore, as agents interact in an increasingly open environment, the risk of unauthorized access to communication channels and data increases. Future research should focus on developing encryption and authentication protocols tailored for decentralized multi-agent environments. Another key challenge is integrating secure communication protocols that can adapt to the dynamic nature of LLM-MAS, where agents frequently join or leave the system. By addressing these security challenges, we can enable LLM-MAS to operate safely in scenarios that require higher security, such as autonomous vehicles and healthcare. 
\subsection{Benchmarks and Evaluation}
In recent years, the application of LLM-MAS has expanded across a growing number of domains. However, the lack of benchmarks that encompass multiple domains poses a significant challenge for comprehensive evaluation. At the same time, system-level and system internal communication metrics remain limited, with most assessments focusing on the performance of individual agents. Consequently, current approaches do not fully capture the system’s collaborative and interactive capabilities, leading to incomplete evaluations of the overall system.