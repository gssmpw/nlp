\section{Related Work}
 
Various approaches have been suggested for image restoration \cite{zhang2018learning,zhang2021designing,luo2020unfolding,liang2021swinir,zhou2022codeformer,lin2023diffbir,yue2024difface,zhu2024flowie,ohayon2024posterior}. In recent years, solutions for IR based on generative methods, including GANs \cite{goodfellow2014generative}, diffusion models \cite{song2021denoising} and flow matching \cite{lipman2023flow}, have emerged, yielding impressive results.

\textbf{GAN-based methods.} GAN-based techniques have been proposed to address image restoration. BSRGAN \cite{zhang2021designing} and Real-ESRGAN \cite{wang2021realesrgan} are GAN-based methods that use effective degradation modeling process for blind super-resolution. GFPGAN \cite{wang2021gfpgan} and GPEN \cite{yang2021gan} proposed to leverage GAN priors for blind face restoration. GPEN suggested training a GAN network for high-quality face generation and then embedding it to a network as a decoder before blind face restoration. GFPGAN connected a degradation removal module and a pre-trained face GAN by direct latent code mapping. CodeFormer \cite{zhou2022codeformer} also uses GAN priors by learning a discrete codebook before using a vector-quantized autoencoder.
Similarly, VQFR \cite{gu2022vqfr} uses a combination of vector quantization and parallel decoding, enabling efficient and effective restoration.
 

\textbf{Diffusion-based methods.}
DDRM \cite{kawar2022denoising}, DDNM \cite{wangzero}, and GDP \cite{fei2023generative} are diffusion-based methods that have superior generative capabilities compared to GAN-based methods by incorporating the powerful diffusion model as an additional prior. 
Under the assumption of known degradations, these methods can effectively restore images in a zero-shot manner. ResShift \cite{10681246} proposed an efficient diffusion model that facilitates the transitions between HQ and LQ images by shifting their residuals. 
Recently, several approaches have suggested two-stage pipeline algorithms. DifFace \cite{yue2024difface} suggested such a method for blind face restoration,  performing sampling from a transition distribution followed by a diffusion process. DiffBIR \cite{lin2023diffbir} proposed to solve blind image restoration by first applying a restoration module for degradation removal and then generating the lost content using a latent diffusion model.


\textbf{Flow-based methods.}
Recently, FlowIE \cite{zhu2024flowie} and PMRF \cite{ohayon2024posterior} introduced two-stage algorithms for image restoration based on rectified flows \cite{liu2023flow}. 
FlowIE relies on the computationally intensive Stable Diffusion \cite{rombach2022high}, which limits its suitability for deployment on edge devices.
PMRF has shown impressive results on both perception and distortion metrics by minimizing the MSE under a perfect perceptual index constraint. It alleviates the issues of solving the ODE by adding Gaussian noise to the posterior mean predictions. Nevertheless, PMRF uses sophisticated attention patterns that pose significant challenges for efficient execution on resource-constrained edge devices because of intensive shape and indexing operations \cite{li2022rethinking}. Our work introduces an efficient flow-based method designed with a hardware-friendly architecture, enabling its deployment on resource-constrained edge devices.