\begin{table*}[]
\centering
\footnotesize
\caption{\textbf{CelebA-Test Evaluation for BFR.} Comparison between \name and baseline models for blind face restoration. \textcolor{red}{Red}, \textcolor{blue}{blue} and {\color[HTML]{009901} green} indicate the best, the second best, and the third best scores, respectively.}
\begin{tabular}{llcccccccc}
\toprule
\multicolumn{1}{c}{} &  & \multicolumn{2}{c}{Efficiency} & \multicolumn{3}{c}{Perceptual Quality} & \multicolumn{3}{c}{Distortion} \\ \cmidrule(l){3-4} \cmidrule(l){5-7}  \cmidrule(l){8-10}
\multicolumn{1}{l}{\multirow{-2}{*}{Model}} & \multirow{-2}{*}{Type} & \begin{tabular}[c]{@{}c@{}}\#Params[M]($\downarrow$)\end{tabular} & \multicolumn{1}{l}{FPS($\uparrow$)} & \multicolumn{1}{l}{FID($\downarrow$)} & \multicolumn{1}{l}{NIQE($\downarrow$)} & MUSIQ($\uparrow$) & PSNR($\uparrow$) & SSIM($\uparrow$) & LPIPS($\downarrow$) \\ \midrule\midrule
CodeFormer & GAN & 94 & \textbf{\color[HTML]{009901}12.79} & 55.85 & 4.73 & \textbf{\color[HTML]{009901} 74.99} & 25.21 & \textbf{\color[HTML]{009901} 0.6964} & \textbf{\color[HTML]{FE0000} 0.3402} \\
GFPGAN & GAN & \textbf{\color[HTML]{009901} 86} & \textbf{\color[HTML]{FE0000}26.37} & 47.60 & 4.34 & \textbf{\color[HTML]{3531FF} 75.30} & 24.98 & 0.6932 & 0.3627 \\
VQFRv2 & GAN & \textbf{\color[HTML]{3531FF} 83} & 8.54 & 47.96 & 4.19 & 73.85 & 23.76 & 0.6749 & 0.3536 \\
Difface & Diffusion & 176 & 0.20 & \textbf{\color[HTML]{FE0000} 37.44} & \textbf{\color[HTML]{3531FF} 4.05} & 69.34 & 24.83 & 0.6872 & 0.3932 \\
DiffBIR & Diffusion & 1667 & 0.07 & 56.61 & 6.16 & \textbf{\color[HTML]{FE0000} 76.51} & 25.23 & 0.6556 & 0.3839 \\
ResShift & Diffusion & 195 & 4.26 & 46.95 &  4.28 & 72.85 & \textbf{\color[HTML]{3531FF} 25.75} & \textbf{\color[HTML]{3531FF} 0.7048} & \textbf{\color[HTML]{3531FF} 0.3437} \\
PMRF & Flow & 176 & 0.63 & \textbf{\color[HTML]{3531FF} 38.52} & \textbf{\color[HTML]{FE0000} 3.78} & 71.47 & \textbf{\color[HTML]{FE0000} 26.25} & \textbf{\color[HTML]{FE0000} 0.7095} & \textbf{\color[HTML]{009901} 0.3465} \\
\midrule
\textbf{ELIR (Ours)} & LCFM & \textbf{\color[HTML]{FE0000}37} & \textbf{\color[HTML]{3531FF}19.51} & \textbf{\color[HTML]{009901} 39.75} & \textbf{\color[HTML]{009901} 4.07} & 71.45 & \textbf{\color[HTML]{009901} 25.55} & 0.6933 & 0.3753 \\ 
\bottomrule
\end{tabular}
\label{bfr_celeb}
\end{table*}



\begin{table*}[]
\centering
\footnotesize
\caption{\textbf{CelebA-Test Evaluation.} Comparison between \name and PMRF \cite{ohayon2024posterior} for super-resolution, denoising, inpainting and colorization.}
\resizebox{\textwidth}{!}{%
\begin{tabular}{llcccccccc}
\toprule
 & \multicolumn{1}{c}{} & \multicolumn{2}{c}{Efficiency} & \multicolumn{3}{c}{Perceptual Quality} & \multicolumn{3}{c}{Distortion} \\ \cmidrule(l){3-4} \cmidrule(l){5-7}  \cmidrule(l){8-10} 
\multirow{-2}{*}{Task} & \multicolumn{1}{c}{\multirow{-2}{*}{Model}} & \begin{tabular}[c]{@{}c@{}}\#Params[M]($\downarrow$)\end{tabular} & \multicolumn{1}{c}{FPS($\uparrow$)} & FID($\downarrow$) & \multicolumn{1}{l}{NIQE($\downarrow$)} & MUSIQ($\uparrow$) & PSNR($\uparrow$) & SSIM($\uparrow$) & LPIPS($\downarrow$) \\ \midrule\midrule
 & PMRF & {\color[HTML]{333333} 126} & {\color[HTML]{333333} 1.08} & {\color[HTML]{333333} 43.24} & {\color[HTML]{333333} 5.45} & {\color[HTML]{333333} 63.17} & {\color[HTML]{333333} 24.33} & {\color[HTML]{333333} 0.6776} & {\color[HTML]{333333} 0.2997} \\
\multirow{-2}{*}{Super Resolution} & \cellcolor[HTML]{FFFFFF}\textbf{\name (Ours)} & \cellcolor[HTML]{FFFFFF}{\color[HTML]{333333} 27} & \cellcolor[HTML]{FFFFFF}{\color[HTML]{333333} 49.26} & \cellcolor[HTML]{FFFFFF}{\color[HTML]{333333} 44.81} & \cellcolor[HTML]{FFFFFF}{\color[HTML]{333333} 5.01} & \cellcolor[HTML]{FFFFFF}{\color[HTML]{333333} 64.06} & \cellcolor[HTML]{FFFFFF}{\color[HTML]{333333} 23.87} & \cellcolor[HTML]{FFFFFF}{\color[HTML]{333333} 0.6579} & \cellcolor[HTML]{FFFFFF}{\color[HTML]{333333} 0.3256} \\ \hline
 & \cellcolor[HTML]{FFFFFF}PMRF & \cellcolor[HTML]{FFFFFF}{\color[HTML]{333333} 126} & \cellcolor[HTML]{FFFFFF}{\color[HTML]{333333} 1.08} & \cellcolor[HTML]{FFFFFF}{\color[HTML]{333333} 41.42} & \cellcolor[HTML]{FFFFFF}{\color[HTML]{333333} 4.99} & \cellcolor[HTML]{FFFFFF}{\color[HTML]{333333} 65.73} & \cellcolor[HTML]{FFFFFF}{\color[HTML]{333333} 27.87} & \cellcolor[HTML]{FFFFFF}{\color[HTML]{333333} 0.7888} & \cellcolor[HTML]{FFFFFF}{\color[HTML]{333333} 0.2381} \\
\multirow{-2}{*}{Denoising} & \cellcolor[HTML]{FFFFFF}\textbf{\name (Ours)} & \cellcolor[HTML]{FFFFFF}{\color[HTML]{333333} 27} & \cellcolor[HTML]{FFFFFF}{\color[HTML]{333333} 49.26} & \cellcolor[HTML]{FFFFFF}{\color[HTML]{333333} 39.73} & \cellcolor[HTML]{FFFFFF}{\color[HTML]{333333} 5.04} & \cellcolor[HTML]{FFFFFF}{\color[HTML]{333333} 66.21} & \cellcolor[HTML]{FFFFFF}{\color[HTML]{333333} 27.13} & \cellcolor[HTML]{FFFFFF}{\color[HTML]{333333} 0.7737} & \cellcolor[HTML]{FFFFFF}{\color[HTML]{333333} 0.2537} \\ \hline
 & \cellcolor[HTML]{FFFFFF}PMRF & \cellcolor[HTML]{FFFFFF}126 & \cellcolor[HTML]{FFFFFF}{\color[HTML]{333333} 1.08} & \cellcolor[HTML]{FFFFFF}39.60 & \cellcolor[HTML]{FFFFFF}5.20 & \cellcolor[HTML]{FFFFFF}65.86 & \cellcolor[HTML]{FFFFFF}25.86 & \cellcolor[HTML]{FFFFFF}0.7411 & \cellcolor[HTML]{FFFFFF}0.2632 \\
\multirow{-2}{*}{Inpainting} & \cellcolor[HTML]{FFFFFF}\textbf{\name (Ours)} & \cellcolor[HTML]{FFFFFF}{\color[HTML]{333333} 27} & \cellcolor[HTML]{FFFFFF}{\color[HTML]{333333} 49.26} & \cellcolor[HTML]{FFFFFF}40.17 & \cellcolor[HTML]{FFFFFF}{\color[HTML]{333333} 4.95} & \cellcolor[HTML]{FFFFFF}{\color[HTML]{333333} 66.17} & \cellcolor[HTML]{FFFFFF}{\color[HTML]{333333} 25.40} & \cellcolor[HTML]{FFFFFF}{\color[HTML]{333333} 0.7302} & \cellcolor[HTML]{FFFFFF}{\color[HTML]{333333} 0.2779} \\ \hline
 & \cellcolor[HTML]{FFFFFF}PMRF & \cellcolor[HTML]{FFFFFF}126 & \cellcolor[HTML]{FFFFFF}{\color[HTML]{333333} 1.08} & \cellcolor[HTML]{FFFFFF}41.34 & \cellcolor[HTML]{FFFFFF}5.00 & \cellcolor[HTML]{FFFFFF}67.16 & \cellcolor[HTML]{FFFFFF}23.39 & \cellcolor[HTML]{FFFFFF}0.7378 & \cellcolor[HTML]{FFFFFF}0.3432 \\
\multirow{-2}{*}{Colorization} & \cellcolor[HTML]{FFFFFF}\textbf{\name (Ours)} & \cellcolor[HTML]{FFFFFF}{\color[HTML]{333333} 27} & \cellcolor[HTML]{FFFFFF}{\color[HTML]{333333} 49.26} & \cellcolor[HTML]{FFFFFF}46.34 & \cellcolor[HTML]{FFFFFF}{\color[HTML]{333333} 4.91} & \cellcolor[HTML]{FFFFFF}{\color[HTML]{333333} 65.12} & \cellcolor[HTML]{FFFFFF}{\color[HTML]{333333} 22.83} & \cellcolor[HTML]{FFFFFF}{\color[HTML]{333333} 0.7303} & \cellcolor[HTML]{FFFFFF}{\color[HTML]{333333} 0.3705} \\ 
\bottomrule
\end{tabular}}
\label{task_celeb}
\end{table*}


% \begin{table*}[]
% \centering
% \footnotesize
% \caption{\textbf{CelebA-Test Evaluation.} Comparison between \name and baseline models for blind face restoration. \textcolor{red}{Red}, \textcolor{blue}{blue} and {\color[HTML]{009901} green} indicate the best, the second best, and the third best scores, respectively.}
% \begin{tabular}{llcccccccl}
% \toprule
% \multicolumn{1}{c}{} &  & \multicolumn{3}{c}{Perceptual Quality} & \multicolumn{3}{c}{Distortion} &  &  \\ \cmidrule(l){3-5}  \cmidrule(l){6-8}
% \multicolumn{1}{c}{\multirow{-2}{*}{Model}} & \multirow{-2}{*}{Type} & FID($\downarrow$) & \multicolumn{1}{l}{NIQE($\downarrow$)} & MUSIQ($\uparrow$) & PSNR($\uparrow$) & SSIM($\uparrow$) & LPIPS($\downarrow$) & \multirow{-2}{*}{\begin{tabular}[c]{@{}c@{}}\#Params\\ {[}M{]}\end{tabular}} & \multirow{-2}{*}{FPS} \\
% \midrule\midrule
% CodeFormer & GAN & 55.85 & 4.73 &  \textbf{\color[HTML]{009901} 74.99} & 25.21 &  \textbf{\color[HTML]{009901} 0.6964} &  \textbf{\color[HTML]{FE0000} 0.3402} & 94 & {\color[HTML]{333333} 12.79} \\
% GFPGAN & GAN & 47.60 & 4.34 &  \textbf{\color[HTML]{3531FF} 75.30} & 24.98 & 0.6932 & 0.3627 & 86 & {\color[HTML]{333333} 26.37} \\
% VQFRv2 & GAN & 47.96 & 4.19 & 73.85 & 23.76 & 0.6749 & 0.3536 & 83 & 8.54 \\
% \midrule
% Difface & Diffusion & \textbf{\color[HTML]{FE0000} 37.44} &  \textbf{\color[HTML]{3531FF} 4.05} & 69.34 & 24.83 & 0.6872 & 0.3932 & 176 & 0.20 \\
% DiffBIR & Diffusion & \cellcolor[HTML]{FFFFFF}56.61 & 6.16 &  \textbf{\color[HTML]{FE0000} 76.51} & \cellcolor[HTML]{FFFFFF}25.23 & \cellcolor[HTML]{FFFFFF}0.6556 & \cellcolor[HTML]{FFFFFF}0.3839 & \cellcolor[HTML]{FFFFFF}1667 & 0.07 \\
% ResShift & Diffusion & 46.95 & 4.28 & 72.85 &  \textbf{\color[HTML]{3531FF} 25.75} &  \textbf{\color[HTML]{3531FF} 0.7048} &  \textbf{\color[HTML]{3531FF} 0.3437} & 195 & 4.26 \\
% PMRF & Flow &  \textbf{\color[HTML]{3531FF} 38.52} &  \textbf{\color[HTML]{FE0000} 3.78} & 71.47 &  \textbf{\color[HTML]{FE0000} 26.25} &  \textbf{\color[HTML]{FE0000} 0.7095} &  \textbf{\color[HTML]{009901} 0.3465} & 176 & 0.63 \\
% \midrule
% \textbf{\name (Ours)} & LCFM &  \textbf{\color[HTML]{009901} 39.75} &  \textbf{\color[HTML]{009901} 4.07} & 71.45 &  \textbf{\color[HTML]{009901} 25.55} & 0.6933 & 0.3753 & 37 & 19.51 \\ 
% \bottomrule
% \end{tabular}
% \label{bfr_celeb}
% \end{table*}







% \begin{table*}[]
% \centering
% \small
% \caption{\textbf{CelebA-Test Evaluation.} Comparison between \name and PMRF \cite{ohayon2024posterior} for super-resolution, denoising, inpainting and colorization.}
% \begin{tabular}{llcccccccc}
% \toprule
%  & \multicolumn{1}{c}{} & \multicolumn{3}{c}{Perceptual Quality} & \multicolumn{3}{c}{Distortion} &  & \multicolumn{1}{l}{} \\ \cmidrule(l){3-5}  \cmidrule(l){6-8}
% \multirow{-2}{*}{Task} & \multicolumn{1}{c}{\multirow{-2}{*}{Model}} & FID($\downarrow$) & \multicolumn{1}{l}{NIQE($\downarrow$)} & MUSIQ($\uparrow$) & PSNR($\uparrow$) & SSIM($\uparrow$) & LPIPS($\downarrow$) & \multirow{-2}{*}{\begin{tabular}[c]{@{}c@{}}\#Params\\ {[}M{]}\end{tabular}} & \multicolumn{1}{l}{\multirow{-2}{*}{FPS}} \\ 
% \midrule\midrule
%  & PMRF & {\color[HTML]{333333} 43.24} & {\color[HTML]{333333} 5.45} & {\color[HTML]{333333} 63.17} & {\color[HTML]{333333} 24.33} & {\color[HTML]{333333} 0.6776} & {\color[HTML]{333333} 0.2997} & {\color[HTML]{333333} 126} & {\color[HTML]{333333} 1.08} \\
% \multirow{-2}{*}{Super Resolution} & \cellcolor[HTML]{FFFFFF}\textbf{\name (Ours)} & \cellcolor[HTML]{FFFFFF}{\color[HTML]{333333} 44.81} & \cellcolor[HTML]{FFFFFF}{\color[HTML]{333333} 5.01} & \cellcolor[HTML]{FFFFFF}{\color[HTML]{333333} 64.06} & \cellcolor[HTML]{FFFFFF}{\color[HTML]{333333} 23.87} & \cellcolor[HTML]{FFFFFF}{\color[HTML]{333333} 0.6579} & \cellcolor[HTML]{FFFFFF}{\color[HTML]{333333} 0.3256} & \cellcolor[HTML]{FFFFFF}{\color[HTML]{333333} 27} & \cellcolor[HTML]{FFFFFF}{\color[HTML]{333333} 49.26} \\ \hline
%  & \cellcolor[HTML]{FFFFFF}PMRF & \cellcolor[HTML]{FFFFFF}{\color[HTML]{333333} 41.42} & \cellcolor[HTML]{FFFFFF}{\color[HTML]{333333} 4.99} & \cellcolor[HTML]{FFFFFF}{\color[HTML]{333333} 65.73} & \cellcolor[HTML]{FFFFFF}{\color[HTML]{333333} 27.87} & \cellcolor[HTML]{FFFFFF}{\color[HTML]{333333} 0.7888} & \cellcolor[HTML]{FFFFFF}{\color[HTML]{333333} 0.2381} & \cellcolor[HTML]{FFFFFF}{\color[HTML]{333333} 126} & \cellcolor[HTML]{FFFFFF}{\color[HTML]{333333} 1.08} \\
% \multirow{-2}{*}{Denoising} & \cellcolor[HTML]{FFFFFF}\textbf{\name (Ours)} & \cellcolor[HTML]{FFFFFF}{\color[HTML]{333333} 39.73} & \cellcolor[HTML]{FFFFFF}{\color[HTML]{333333} 5.04} & \cellcolor[HTML]{FFFFFF}{\color[HTML]{333333} 66.21} & \cellcolor[HTML]{FFFFFF}{\color[HTML]{333333} 27.13} & \cellcolor[HTML]{FFFFFF}{\color[HTML]{333333} 0.7737} & \cellcolor[HTML]{FFFFFF}{\color[HTML]{333333} 0.2537} & \cellcolor[HTML]{FFFFFF}{\color[HTML]{333333} 27} & \cellcolor[HTML]{FFFFFF}{\color[HTML]{333333} 49.26} \\ \hline
%  & \cellcolor[HTML]{FFFFFF}PMRF & \cellcolor[HTML]{FFFFFF}39.60 & \cellcolor[HTML]{FFFFFF}5.20 & \cellcolor[HTML]{FFFFFF}65.86 & \cellcolor[HTML]{FFFFFF}25.86 & \cellcolor[HTML]{FFFFFF}0.7411 & \cellcolor[HTML]{FFFFFF}0.2632 & \cellcolor[HTML]{FFFFFF}126 & \cellcolor[HTML]{FFFFFF}{\color[HTML]{333333} 1.08} \\
% \multirow{-2}{*}{Inpainting} & \cellcolor[HTML]{FFFFFF}\textbf{\name (Ours)} & \cellcolor[HTML]{FFFFFF}40.17 & \cellcolor[HTML]{FFFFFF}{\color[HTML]{333333} 4.95} & \cellcolor[HTML]{FFFFFF}{\color[HTML]{333333} 66.17} & \cellcolor[HTML]{FFFFFF}{\color[HTML]{333333} 25.40} & \cellcolor[HTML]{FFFFFF}{\color[HTML]{333333} 0.7302} & \cellcolor[HTML]{FFFFFF}{\color[HTML]{333333} 0.2779} & \cellcolor[HTML]{FFFFFF}{\color[HTML]{333333} 27} & \cellcolor[HTML]{FFFFFF}{\color[HTML]{333333} 49.26} \\ \hline
%  & \cellcolor[HTML]{FFFFFF}PMRF & \cellcolor[HTML]{FFFFFF}41.34 & \cellcolor[HTML]{FFFFFF}5.00 & \cellcolor[HTML]{FFFFFF}67.16 & \cellcolor[HTML]{FFFFFF}23.39 & \cellcolor[HTML]{FFFFFF}0.7378 & \cellcolor[HTML]{FFFFFF}0.3432 & \cellcolor[HTML]{FFFFFF}126 & \cellcolor[HTML]{FFFFFF}{\color[HTML]{333333} 1.08} \\
% \multirow{-2}{*}{Colorization} & \cellcolor[HTML]{FFFFFF}\textbf{\name (Ours)} & \cellcolor[HTML]{FFFFFF}46.34 & \cellcolor[HTML]{FFFFFF}{\color[HTML]{333333} 4.91} & \cellcolor[HTML]{FFFFFF}{\color[HTML]{333333} 65.12} & \cellcolor[HTML]{FFFFFF}{\color[HTML]{333333} 22.83} & \cellcolor[HTML]{FFFFFF}{\color[HTML]{333333} 0.7303} & \cellcolor[HTML]{FFFFFF}{\color[HTML]{333333} 0.3705} & \cellcolor[HTML]{FFFFFF}{\color[HTML]{333333} 27} & \cellcolor[HTML]{FFFFFF}{\color[HTML]{333333} 49.26} \\
% \bottomrule
% \end{tabular}
% \label{task_celeb}
% \end{table*}


\section{Experiments}
In this section, we present experiments for the following tasks: blind face restoration (BFR), super-resolution, image denoising, inpainting, and colorization. We train our model for each task with the FFHQ \cite{karras2019style} dataset which contains 70k high-quality images. The model is trained using the AdamW \cite{loshchilovdecoupled} optimizer. Both losses $\mathcal{L}_{2}$ and $\mathcal{L}_{LCFM}$ are optimized jointly where the gradients of $\vectorsym{\theta}$ are detached from $\vectorsym{\omega},\vectorsym{\phi}$. We set $\lambda_i=1$ for all the experiments. During training, we only use random horizontal flips for data augmentation. In addition, we incorporate collapsible linear blocks \cite{bhardwaj2022collapsible}, to improve training efficiency without affecting inference time. We use an exponential moving average (EMA) with a decay of 0.999. The final EMA weights are then used in all evaluations. During inference, we set $M=K$ for Euler steps, unless mentioned otherwise. We report FID (vs FFHQ) \cite{heusel2017gans}, NIQE \cite{mittal2012making}, and MUSIQ \cite{ke2021musiq} for perception metrics and PSNR, SSIM \cite{wang2004image} and LPIPS \cite{zhang2018perceptual} for distortion metrics. All evaluation metrics are computed using \citet{pyiqa}. In addition, we report the number of parameters and the frames per second (FPS). FPS are evaluated by injecting images into an NVIDIA GeForce RTX 2080 Ti and recording its process time. The training hyperparameters are provided in the Appendix (Table~\ref{hyperparams}).





\begin{figure*}
\centering
\hspace*{0.0cm}\includegraphics[scale=0.09]{images/celeba_bfr.jpg}
\caption{\textbf{Visual Results for BFR.} Visual comparisons between \name and baseline models sampled from CelebA-Test for the blind face restoration task. HQ and LQ refer to high-quality (groud-truth) and low-quality (inputs) images.}
\label{fig:scheme}
\label{bfr_figure}
\end{figure*}


\subsection{Implementation Details}\label{impl}
\subsubsection{Blind Face Restoration (BFR)}\label{bfr_imp}
\textbf{Training}.\quad The training process is conducted on $512\times512$ resolution with a first-order degradation model to synthesize LQ images. The degradation \cite{zhang2021designing} is approximated by
\begin{align}\label{degradation}
\vectorsym{y}=\{[(\vectorsym{x} \circledast \vectorsym{k}_\sigma)\downarrow r+\vectorsym{n}_\delta]_{\mathbf{JPEG}_{Q}}\}\uparrow r
\end{align}
where $\circledast$ denotes convolution, $\vectorsym{k}_\sigma$ is a Gaussian blur kernel of size $41 \times 41$ with variance $\sigma^2$, $\downarrow r$ and $\uparrow r$ are down-sampling and up-sampling by a factor $r$, respectively. $\vectorsym{n}_\delta$ is Gaussian noise with variance $\delta^2$ and $[\cdot]_{\mathbf{JPEG}_{Q}}$ is JPEG compression-decompression with quality factor Q. We choose $\sigma$,$r$,$\delta$,$Q$ uniformly from [0.1, 12], [1, 12], [0, 15], and [30, 100], respectively. The noise level is $\sigma_s=0.1$ and we set $\sigma_{min}=10^{-5}$. The consistency loss is applied with multi-segments \cite{yang2024consistencyfm} of $K=5$. The model is trained with a learning rate of $10^{-4}$ and a batch size of 64.
\\
\textbf{Evaluation}.\quad We evaluate our method on synthetic CelebA-Test \cite{liu2015faceattributes}. CelebA-Test consists of 3000 pairs of low and high-quality images taken from CelebA and degraded by \citet{wang2021towards}. 


\subsubsection{Super Resolution, Image Denoising, Inpainting, Colorization}\label{tasks_imp}
\textbf{Training}.\quad Similar to the training process outlined in PMRF \cite{ohayon2024posterior}, we employ a 256 x 256 resolution and utilize the degradation model as follows: For super-resolution, we use $8\times$ bicubic downscale and add Gaussian noise with a standard deviation of $0.05$. Note that the downscaled images are first $8\times$ bicubic upscaled (back to $256\times256$) before feeding them into the model. For image denoising, we add Gaussian noise with a standard deviation of $0.35$. For inpainting, we randomly mask 90\% of the pixels in the ground-truth image and add Gaussian noise with a standard deviation of $0.1$. For colorization, we average the RGB channels and add Gaussian noise with a standard deviation of $0.25$. We set $\sigma_s=0.025$ for image denoising and $\sigma_s=0.1$ for the rest of the tasks. The consistency loss is applied with multi-segments \cite{yang2024consistencyfm} of $K=3$. The model is trained with a learning rate of $2\cdot10^{-4}$ and a batch size of 128.
\\
\textbf{Evaluation}.\quad We test our method on synthetic CelebA-Test, where the same training degradations have been used in the evaluation.


\subsection{Results}
\subsubsection{Blind Face Restoration (BFR)}\label{bfr_result}
We compare our method with the following baseline models: CodeFormer \cite{zhou2022codeformer}, GFPGAN \cite{wang2021gfpgan}, VQFRv2 \cite{gu2022vqfr}, Difface \cite{yue2024difface}, DiffBIR \cite{lin2023diffbir}, ResShift \cite{10681246} and PMRF \cite{ohayon2024posterior}. In Table~\ref{bfr_celeb} we present a comparative evaluation showing that \name is competitive with state-of-the-art methods for blind face restoration. Our method achieves a notably high PSNR without compromising FID, indicating its ability to balance perception and distortion. Moreover, \name has the smallest number of parameters compared to all other methods. In terms of latency, \name is faster by 4-270$\times$ compared to diffusion \& flow-based methods. Note that GAN-based methods have shown similar latency to \name but with large degradations in FID and PSNR. In addition, Figure~\ref{bfr_figure} presents visual results of \name compared to baseline methods. \name demonstrates competitive performance while having the smallest model size and fast inference time, which makes it ideally suited for deployment on resource-constrained edge devices. Additional results can be found in the Appendix~\ref{apx:results}.

\subsubsection{Super Resolution, Image Denoising, Inpainting, Colorization}\label{tasks_result}
Table~\ref{task_celeb} compares our method and PMRF \cite{ohayon2024posterior} across various image restoration tasks. Our method achieves competitive performance with PMRF in terms of perceptual quality metrics, while exhibiting a slight performance gap in distortion metrics. In colorization, we observe a performance gap in FID, which we attribute to the crucial role of global context in this specific task. Nevertheless, our method demonstrates a $4.6\times$ reduction in model size and a $45\times$ speedup compared to PMRF, facilitating efficient deployment on edge devices. Visual results are shown in Appendix~\ref{apx:results}.
%In addition, Figures~\ref{sr_figure},\ref{id_figure},\ref{ip_figure},\ref{color_figure} in the Appendix show visual results compared to PMRF for super-resolution, denoising, inpainting, and colorization, respectively. \name produces sharp and visually appealing details while being small and fast.










\subsection{Ablation}
Here, we evaluate ELIR's performance through an ablation study, which examines the contribution of its components. Additional ablations can be found in the Appendix~\ref{apx:ablation}.

\textbf{\name steps}.\quad To evaluate \name's performance in latent space, Table~\ref{steps} presents PSNR and FID values at each processing step. As expected, the highest PSNR is achieved after Latent MMSE, confirming its effectiveness. Subsequently, PSNR gradually decreases while FID improves, reflecting the expected distortion-perception trade-off. Figure~\ref{fig:flow_bfr} in the Appendix, illustrates the restoration process, visualizing the process from LQ images to visually appealing results.
\begin{table}[]
\centering
\caption{\textbf{\name steps.} \name steps from low-quality image (LQ) to restoration on CelebA-Test for blind face restoration. The observation of the highest PSNR after MMSE aligns with expectations and validates the effective implementation of Latent MMSE. }
\begin{tabular}{lcc}
\toprule
\multicolumn{1}{c}{} &  &  \\
\multicolumn{1}{c}{\multirow{-2}{*}{Step}} & \multirow{-2}{*}{FID($\downarrow$)} & \multirow{-2}{*}{PSNR($\uparrow$)} \\ \midrule\midrule
LQ & 145.29 & 25.26 \\
Latent MMSE & \cellcolor[HTML]{FFFFFF}{\color[HTML]{000000} 72.57} & \cellcolor[HTML]{FFFFFF}{\color[HTML]{000000} 26.43} \\
Latent MMSE + Noise & 78.91 & 26.20 \\
Iteration 1 & 63.19 & 26.25 \\
Iteration 2 & 55.66 & 25.85 \\
Iteration 3 & 46.36 & 25.63 \\
Iteration 4 & 40.21 & 25.56 \\
Restored & \cellcolor[HTML]{FFFFFF}{\color[HTML]{000000} 39.75} & \cellcolor[HTML]{FFFFFF}{\color[HTML]{000000} 25.55} \\ \bottomrule
\label{steps}
\end{tabular}
\end{table}


\textbf{Efficieny of Latent CFM}.\quad Figure~\ref{fm_cfm_figure} compares the performance of FM and CFM in latent space by plotting PSNR and FID for varying NFEs. Both methods exhibit a similar trend: PSNR decreases while FID improves with increasing NFE, reflecting the expected distortion-perception trade-off. While FM requires 25 NFEs to reach a comparable FID, CFM achieves the same FID with only 5 NFEs, highlighting CFM's superior efficiency.
\begin{figure}
\centering
\hspace*{0.0cm}\includesvg[scale=0.42]{images/bfr_iters.svg} 
\caption{\textbf{Efficieny of LCFM.} PSNR and FID are evaluated in several NFEs for Latent FM and Latent CFM for blind face restoration. }
\label{fm_cfm_figure}
\end{figure}

\textbf{Model Size}.\quad Figure~\ref{fig:model_size} presents different model sizes for super-resolution. We vary the vector field size while keeping the latent MMSE constant. Our results indicate a diminishing return in FID improvement beyond 27M parameters. For additional metrics see Table~\ref{table:size} in the Appendix.

\begin{figure}
\centering
\hspace*{0.0cm}\includesvg[scale=0.37]{images/size_ablation.svg} 
\caption{\textbf{Model Size.} FID and FPS are evaluated for several model sizes on CelebA-Test for super-resolution. The area of each circle is proportional to the model size. }
\label{fig:model_size}
\end{figure}

\textbf{Effectiveness of trainable encoder}.\quad This ablation study demonstrates the importance of fine-tuning the encoder. Given that the encoder was initially trained on HQ images, it struggles to represent the LQ images encountered in various tasks. This limitation is evident in Table~\ref{table:encoder_short}, where fixed encoders exhibit significantly lower performance, with PSNR values 1.5-2 dB lower and FID scores 2-3 points higher compared to trainable encoders. Additional metrics can be found in the Appendix (Table~\ref{table:encoder_long}).
\begin{table}[]
\centering
\caption{\textbf{Effectiveness of trainable encoder.} Experiments of \name for denoising and inpainting on the CelebA-Test dataset \emph{w/} and \emph{w/o} training the encoder. }
\begin{tabular}{lccc}
\toprule
 & \multicolumn{1}{l}{} &  &  \\
\multirow{-2}{*}{Task} & \multicolumn{1}{l}{\multirow{-2}{*}{\begin{tabular}[c]{@{}l@{}}Trainable\\  Encoder\end{tabular}}} & \multirow{-2}{*}{FID($\downarrow$)} & \multirow{-2}{*}{PSNR($\uparrow$)} \\ \midrule\midrule
 & \cellcolor[HTML]{FFFFFF}\xmark  & \cellcolor[HTML]{FFFFFF}{\color[HTML]{333333} 40.89} & \cellcolor[HTML]{FFFFFF}{\color[HTML]{333333} 26.55} \\
\multirow{-2}{*}{Denoising} & \cellcolor[HTML]{FFFFFF}\cmark  & \cellcolor[HTML]{FFFFFF}{\color[HTML]{333333} 39.73} & \cellcolor[HTML]{FFFFFF}{\color[HTML]{333333} 27.13} \\ \hline
 & \cellcolor[HTML]{FFFFFF}\xmark  & \cellcolor[HTML]{FFFFFF}43.00 & \cellcolor[HTML]{FFFFFF}23.46 \\
\multirow{-2}{*}{Inpainting} & \cellcolor[HTML]{FFFFFF}\cmark & \cellcolor[HTML]{FFFFFF}40.17 & \cellcolor[HTML]{FFFFFF}{\color[HTML]{333333} 25.40} \\ \bottomrule
\end{tabular}
\label{table:encoder_short}
\end{table}


