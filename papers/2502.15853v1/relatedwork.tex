\section{Literature Review}
Predicting the stock market is challenging yet crucial for investors, traders, and researchers. Various methods, including mathematical, statistical, and Artificial Intelligence (AI) techniques, have been proposed to forecast stock prices. Parmar et al. \cite{parmar2018stock} discovered that machine learning and deep learning prediction methods significantly outperform traditional stock market forecasting techniques in terms of both speed and accuracy. W. Khan \cite{khan2022stock} noted that social media significantly influences stock predictions, with Random Forest consistently performing well in all scenarios. Additionally, utilizing larger datasets and incorporating sentiment analysis into ML and DL models can enhance the precision of stock price predictions. These findings lead to development of many machine learning models, one being the Support Vector Machines (SVM) which belongs to the class of  supervised learning algorithms. Mankar et al. \cite{mankar2018stock} found that SVM proved to be a more effective and practical machine learning model for predicting stock price movements based on the sentiment expressed in tweets. Implementation of other deep learning techniques like artificial neural networks (ANNs) shows significant improvement over the previous machine learning solutions. Rao P. S. et al. \cite{rao2020survey} improved upon the existing model by proposing an artificial neural networks (ANNs), utilizing the backpropagation algorithm. The model outperformed the traditional regression methods and exhibited lower prediction errors. Naik N. and Mohan B. R. \cite{nikou2019stock} found that deep learning techniques outperformed machine learning techniques in terms of results. Further development into deep learning techniques suggested usage of RNN which can capture temporal dependencies and can further improve upon the prediction accuracy. 

Selvamuthu D., Kumar V., and Mishra A. \cite{selvamuthu2019indian} discovered that tick data provided more accurate predictions compared to 15-minute data. They reported that the Levenberg-Marquardt, Scaled Conjugate Gradient, and Bayesian Regularization methods achieved an impressive 99.9\% accuracy with tick data. Additionally, they suggested that using LSTM RNN would be advantageous. W. Fang et al. \cite{fang2019combine} proposed the LSTM predictive model, which includes an embedded layer and an automatic encoder. They note that stock news is not fully leveraged and that the approach has only been applied to the Chinese stock market. Their findings demonstrate that shallow machine learning algorithms, such as SVM and backpropagation, deliver lower accuracy compared to the LSTM and embedded LSTM methods. There have been several variations of LSTM RNN applications over stock prediction which proved it as a robust method over previous implementations. S. Mohan, S. Mullapudi, et al. \cite{mohan2019stock} demonstrated that RNN models incorporating LSTM outperformed the ARIMA model and showed a slight advantage over the Facebook Prophet algorithm. According to the research by Vignesh CK \cite{vignesh2018applying}, the LSTM method achieved a mean accuracy of 66.83 for Yahoo Finance, while the SVM method only attained an accuracy of 65.20. Training with smaller datasets while increasing the number of epochs can enhance testing outcomes across different datasets. M. Nikou, G. Mansourfar, and J. Bagherzadeh \cite{nikou2019stock} demonstrated that the LSTM-RNN block provides superior predictions of the closing price for the company's dataset compared to other methods. They recommend using combined models, such as the SVR model paired with a genetic algorithm, along with other hybrid approaches from machine learning algorithms. The findings of N. Sirimevan et al. \cite{sirimevan2019stock} indicates that the LSTM-RNN model, when combined with weighted average and differential evolution techniques, effectively predicted stock prices. 

A. Moghar and M. Hamiche \cite{moghar2020stock} demonstrated that the accuracy of stock price forecasting using LSTM models improves as the number of training epochs increases, specifically for GOOGL and NKE assets. J. Eapen, D. Bein, and A. Verma \cite{eapen2019novel} found that integrating Convolutional Neural Networks (CNN) with BiLSTM models enhanced stock market prediction accuracy by 9 percentage compared to using a single deep learning pipeline. Similarly, Adil Moghar and Mhamed Hamiche \cite{aguirre2018proceedings} concluded that BiLSTM models produce a lower Root Mean Squared Error (RMSE) compared to standard LSTM models, making them a more favorable option for stock prediction tasks. This hybrid approach also outperformed traditional Support Vector Machine (SVM) regressors in forecasting temporal sequences. Further optimizations have been proposed by incorporating external features into deep learning models to improve predictive accuracy. For instance, Khan, W., Malik, et al. \cite{khan2020predicting} discovered that adding sentiment analysis as an attribute had a minimal impact on stock price predictions but still improved the accuracy of machine learning algorithms by approximately 2 percentage. Moreover, M. Nabipour et al. \cite{nabipour2020predicting} found that using binary data instead of continuous data resulted in a significant enhancement in stock price predictions, highlighting the importance of data representation in predictive modeling. The reviewed literature highlights the growing effectiveness of deep learning models, such as LSTM and BiLSTM, in improving stock price forecasting accuracy through model integration and external optimizations. The reviewed literature highlights the growing effectiveness of deep learning models in improving stock price forecasting accuracy through model integration and external optimizations. However the field is evolving with time abd further research needs to be done to deploy machine learning models that can mimic humans and ensure that profit is generated with minimal human intervention.

% \cite{aguirre2018proceedings} demonstrated that employing Particle Swarm Optimization (PSO) enhances the accuracy of financial market predictions. 

% \begin{figure}[htbp]
%     \centering
%     \includegraphics[width = 0.9\textwidth]{Figures/METHODLOGYFRAMEWORL.drawio.pdf}
%     \caption{Proposed Framework For Anomaly Detection}
%     \label{fig:1}
% \end{figure}