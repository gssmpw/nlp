\section{Introduction}

In the rapidly advancing field of robotics, simulations play an important role in research and practical applications, offering a controlled and cost-effective environment for testing and development~\cite{liu2021role}. This is particularly true in the realm of marine robotics, where the harsh and unpredictable conditions of underwater environments present significant challenges ~\cite{Grimaldi2023InvestigationOT}. Remotely-operated vehicles (ROVs) and autonomous underwater vehicles (AUVs) are integral to a wide range of applications including oceanographic research, underwater infrastructure inspection, and deep-sea exploration~\cite{petillot2019underwater,10801590}. The development and deployment of subsea robots demand rigorous testing to ensure reliability and performance under various environmental conditions. However, real-world underwater testing is often impractical due to high costs, logistical complexities, and the potential for damage to expensive equipment. Consequently, simulators capable of capturing the features of underwater environments have emerged as indispensable tools in the field of subsea robotics \cite{sim_review}\cite{adetunji2024digitaltwinssurfaceenhancing}. Moreover, new algorithms based on deep learning and other machine learning approaches are being extensively used nowadays and their development is hindered by the lack of sufficient amounts of annotated data. 
This data includes images from cameras operating in and out of the visual spectrum, like the thermal camera, from event-based sensors, and from sonars, as well as, e.g., navigation ground truth.

\begin{figure}[t]
    \centering
    \includegraphics[width=0.95\linewidth, height=7.5cm,  trim=0cm 7.5cm 0cm 7.5cm,clip]{images/Trying-to-make-better-stonefish-fig-1_portait_v2.pdf}
    \caption{Stonefish: key features and improvements.}
    \label{fig:stonefish}
\end{figure}


This paper presents recent advancements in the Stonefish simulator~\cite{cieslak2019stonefish}, a state-of-the-art marine robotics simulation platform. Significant improvements, as shown in Fig.~\ref{fig:stonefish}, span across various key areas, making it one of the most versatile and accurate simulators for marine and especially underwater robotics. Notable enhancements include the integration of new GPU-accelerated sensors, implementation of visual light communication (VLC) and enhancement of sonar simulation. The aforementioned new sensors include an event-based camera (EBC), a thermal camera and an optical flow sensor. The existing forward-looking sonar (FLS) has been significantly upgraded, resulting in increased accuracy and reliability of the simulated data. 
All these additions and improvements notably expand the platform’s capabilities in sensor simulation, thus opening new possibilities for generating datasets for machine learning research. Additionally, the simulator now supports linking robots with tethers, providing more realistic behaviour of robots involved in subsea operations. The introduction of configurable thruster dynamics has extended the flexibility and control over the utilised mathematical models, enhancing insights into vehicle dynamics and control strategies. Furthermore, the simulator now includes an automatic annotation tool, streamlining data analysis and improving the efficiency of training and testing environments. These comprehensive upgrades solidify the position of Stonefish as a leading tool for marine robotics research and development.




\begin{figure*}[tp] 
\centering
    \includegraphics[width=\textwidth]{images/simulator_comparison.pdf}
    \caption{Comparison of availability of the most important features in selected simulators.}
    \label{fig:comparison}
\end{figure*}

\section{Related Work}




\subsection{Classical simulators}

One of the most well-known underwater robotics simulators is the UWSim \cite{dhurandher2008uwsim}.  It enabled the simulation of underwater inspection and intervention missions with one or more robots. One of its main advantages is that it delivers a ROS (Robot Operating System \cite{quigley2009ros}) interface allowing for easy integration with ROS-based software architectures, commonly used in robots developed by the scientific community. 
However, in practice, UWSim is mostly used as a visualisation tool and for simulating underwater sensors. It lacks accurate simulation of dynamics and hydrodynamics of vehicles and it does not support simulation of manipulator dynamics (only kinematics). Moreover, this simulation tool has been deprecated and not updated in several years.

A more modern approach is to use the ROS standard simulator Gazebo \cite{gazebo}. Gazebo relies on DART and Bullet Physics for the physics and OGREv2 for the graphics. It lacks native support for hydrodynamics and has poor rendering quality. It can be used either with a custom dynamics code or combined, e.g., with the UUV Simulator plugin \cite{Manhaes_2016}, which adds simple hydrodynamics to the Gazebo physics engine. However, this solution is lacking the simulation of manipulator hydrodynamics and buoyancy. 

Another simulator, Project DAVE \cite{zhang_dave_2022} was built upon the foundation of UUV Simulator \cite{manhaesUUVSimulatorGazebobased2016} and \texttt{ds\_sim} tool package \cite{WhoidslDs_simBitbucket}, featuring a comprehensive array of environments, robots, sensors, and demonstrations, with a particular focus on underwater manipulation tasks.

Another classical simulator is the Marine Systems Simulator~\cite{perez2006overview}. It is a Matlab/Simulink toolbox that delivers accurate dynamic simulation of underwater vehicles. However, the lack of visualisation and no direct interface with ROS makes it unpopular in the community.


\subsection{Unity-Based Underwater Simulators}
Several underwater simulators have been developed using the Unity3D game engine, taking advantage of its realistic visuals, accurate rigid-body physics, and user-friendly development environment. Notable examples include:

\begin{itemize}


\item UWRoboticsSimulator~\cite{UwRoboticsSim}: Specifically developed for marine environments, this simulator supports both ROS and ROS2 and is still actively maintained. Despite its active development, it lacks essential sensors like LiDAR, sonar, and DVL.

\item URSim~\cite{URSim}: This simulator also focuses on the marine environment but the last supported ROS version was Kinetic, and development has ceased since 2019. URSim is therefore outdated and lacks support for recent ROS versions.

\item MARUS~\cite{Marus}: Developed to address the limitations of existing Unity-based simulators, it leverages the engine's capabilities to offer high-end graphics, computation parallelized on GPU, simplicity of test scene design, and many out-of-the-box tools for physics simulation. Unity’s popularity is supported by a broad and versatile community that provides numerous examples and solutions for common problems in game design. MARUS extends Unity3D’s physics engine with water body physics (buoyancy, waves, etc.), simulation of various sensors, propulsion emulation, and visualization tools.

\end{itemize}

Unity3D's strengths in realistic visuals and ease of development have made it a popular choice for simulator development. However, as an engine targeted at game development, it delivers many tools that are not necessary for researchers and were created for artists, who do not have the coding experience. The amount of additional tools results in vast download sizes. Moreover, many of the advanced features of Unity3D are not free but require a commercial license.



\subsection{Unreal Engine-Based Underwater Simulators}

Unreal Engine (UE) is probably the most powerful game engine available today. It is known for its high-fidelity graphics and GPU-accelerated physics engine. Key simulators include:

\begin{itemize}
\item HoloOcean~\cite{potokar2022holoocean}: Built on top of UE4, HoloOcean provides high-fidelity imagery and accurate dynamics using the PhysX physics engine. It features a simple Python interface, making it easy to install and use across various systems. HoloOcean augments Holodeck with accurate underwater dynamics, multi-agent communications, and realistic imaging sonar implementations. 

\item UNav-Sim~\cite{amer2023unav}: The first open-source underwater simulator based on Unreal Engine 5 (UE5), UNav-Sim offers superior rendering quality essential for developing AI and vision-based navigation algorithms for underwater vehicles. It supports robotics tools such as ROS2 and autopilot firmware, making it suitable for robotics research and development. UNav-Sim uses open-source AirSim \cite{AirSim} extensions to add custom vehicle models and integrates AirSim with UE5. This simulator excels in creating photorealistic environments and supports a wide range of underwater scenarios and models, making it highly effective for developing vision-based localization and navigation methods for underwater robots.
\end{itemize}

The use of UE for simulators brings high-quality visuals and highly parallelised physics calculations to underwater simulation. Moreover, UE can be used completely free of charge.
%
However, similarly to the Unity3D, the UE requires a download of a vast amount of tools that are targeted at artists and the development of a custom simulator based on UE requires deep understanding of its code-base and thus a lot of work hours. Moreover, the default physics engine (PhysX) is non-deterministic, resulting in different simulation results in each run.
% 
In both cases, the Unity3D and the UE, the researchers must rely on the implemented physics engines or replace them with their own implementations designed specifically for underwater environments. All the sensor simulations have to be implemented as well, e.g., in form of plugins.




