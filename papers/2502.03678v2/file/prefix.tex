\usepackage{graphicx}
\usepackage[normalem]{ulem}
\usepackage{float}
\usepackage{microtype}
% \usepackage{subfigure} % outdated, use subcaption instead
\usepackage{subcaption}
\usepackage{stackengine}

\usepackage{verbatim} % comment things

\usepackage{amsmath, amssymb, bm, bbm, mathtools}
\usepackage{booktabs} % for professional tables
\usepackage{color}
\usepackage[english]{babel}
\usepackage{times}
\usepackage{nicefrac}

\usepackage{authblk} % author affiliation
\usepackage{xurl} % long url

% Change subfigure numbering style to (a), (b), ...
\renewcommand\thesubfigure{(\alph{subfigure})}

% Apply the format to the sub-figure caption
\captionsetup[subfigure]{labelformat=simple}

% Change equation numbering style to (1), (2), ...
\renewcommand{\theequation}{\arabic{equation}}

% Define new command for equation references
\newcommand{\eqnref}[1]{%
    \ifnum\pdfstrcmp{(}{\unexpanded\expandafter{\@car#1\@nil}}=0
        Equation~\ref{#1}%
    \else
        Equation~(\ref{#1})%
    \fi
}

\newcommand{\theHalgorithm}{\arabic{algorithm}}
\usepackage[linesnumbered,lined,ruled]{algorithm2e}
\SetKwComment{Comment}{$\triangleright$ }{}
% example https://www.overleaf.com/learn/latex/Algorithms#The_algorithm2e_package
\SetKwInOut{Input}{Input}
\SetKwInOut{Output}{Output}
% \SetKwFor{ForEach}{ForEach}{Do}{End}
\SetKwFor{ForEach}{ForEach}{Do}{End}
\SetKwFor{While}{While}{Do}{}
% \SetKwIF{If}{ElseIf}{Else}{If}{Then}{Else If}{Else}{End}
\SetKwIF{If}{ElseIf}{Else}{If}{Then}{Else If}{Else}{}
\SetKwFunction{Return}{\textbf{Return}}
\SetInd{1.5ex}{1.5ex} % Set the indentation width

% --> toc for appendix only
\usepackage[header,page,toc]{appendix}
\usepackage{titletoc}

\definecolor{darkblue}{rgb}{0,0.08,0.45}
\usepackage{hyperref}  % should appear after titletoc

\usepackage{amsthm}
\theoremstyle{plain}
\newtheorem{theorem}{Theorem}[section]
\newtheorem{corollary}[theorem]{Corollary}
\newtheorem{lemma}[theorem]{Lemma}
\newtheorem{proposition}[theorem]{Proposition}

% unnumbered in appendix
\newtheorem*{thm}{Theorem}
\newtheorem*{prop}{Proposition}
\newtheorem*{lma}{Lemma}
\newtheorem*{coro}{Corollary}

\theoremstyle{definition}
% \newtheorem{strategy}{Strategy}[section]
% \renewcommand{\thestrategy}{\Roman{strategy}}
\newtheorem{definition}[theorem]{Definition}
\newtheorem{assumption}[theorem]{Assumption}
\newtheorem{fact}[theorem]{Fact}
\newtheorem{claim}[theorem]{Claim}
\newtheorem{question}[theorem]{Question}

\theoremstyle{remark}
\newtheorem{remark}[theorem]{Remark}

% unnumbered in appendix
\newtheorem*{asmp}{Assumption}

\usepackage{enumitem}

\usepackage[globalcitecopy]{bibunits}
\usepackage{chngcntr}

% Todonotes is useful during development; simply uncomment the next line
%    and comment out the line below the next line to turn off comments
%\usepackage[disable,textsize=normalsize]{todonotes}
\usepackage[disable,textwidth=1.3in,textsize=footnotesize]{todonotes}

% % if you use cleveref..
\usepackage[capitalize,noabbrev]{cleveref}

\usepackage{xcolor}

%%%%%%%%%%%%%%%%%%%%%%%%%%%%%%%%%%%%%%%%%%%%%%%%%%%%%%%%
%%%%%%%%%% ---------- New Commands ---------- %%%%%%%%%%
%%%%%%%%%%%%%%%%%%%%%%%%%%%%%%%%%%%%%%%%%%%%%%%%%%%%%%%%
\usepackage{multirow}
\usepackage{makecell}  % use \multirowcell to handle auto line break in a cell

\usepackage{array}
% Define new column types for left and center alignment with column width as fraction of \textwidth
\newcolumntype{L}[1]{>{\raggedright\let\newline\\\arraybackslash\hspace{0pt}}m{#1\textwidth-2\tabcolsep}}
\newcolumntype{C}[1]{>{\centering\let\newline\\\arraybackslash\hspace{0pt}}m{#1\textwidth-2\tabcolsep}}
\newcolumntype{R}[1]{>{\raggedleft\let\newline\\\arraybackslash\hspace{0pt}}m{#1\textwidth-2\tabcolsep}}

% \newcolumntype{L}[1]{>{\raggedright\arraybackslash}m{#1}}
% \newcolumntype{C}[1]{>{\centering\arraybackslash}m{#1}}


% \newcommand{\ubar}[1]{\text{\b{$#1$}}}
\newcommand{\ubar}[1]{\underline{#1}}

\newcommand{\rulesep}{\unskip\ \vrule\ }

\makeatletter
\newcommand*{\indep}{
    \mathbin{
        \mathpalette{\@indep}{}
    }
}
\newcommand*{\nindep}{
    \mathbin{% The final symbol is a binary math operator
        \mathpalette{\@indep}{\not}% \mathpalette helps for the adaptation of the symbol to the different math styles.
    }
}
\newcommand*{\@indep}[2]{%
    % #1: math style
    % #2: empty or \not
    \sbox0{$#1\perp\m@th$}%        box 0 contains \perp symbol
    \sbox2{$#1=$}%                 box 2 for the height of =
    \sbox4{$#1\vcenter{}$}%        box 4 for the height of the math axis
    \rlap{\copy0}%                 first \perp
    \dimen@=\dimexpr\ht2-\ht4-.2pt\relax
    % The equals symbol is centered around the math axis.
    % The following equations are used to calculate the
    % right shift of the second \perp:
    % [1] ht(equals) - ht(math_axis) = line_width + 0.5 gap
    % [2] right_shift(second_perp) = line_width + gap
    % The line width is approximated by the default line width of 0.4pt
    \kern\dimen@
    {#2}%
    % {\not} in case of \nindep;
    % the braces convert the relational symbol \not to an ordinary
    % math object without additional horizontal spacing.
    \kern\dimen@
    \copy0 %                       second \perp
}
\makeatother

\DeclareMathOperator*{\argmax}{argmax} % thin space, limits underneath in displays
\DeclareMathOperator*{\argmin}{argmin}

\newcommand{\fracpartial}[2]{\frac{\partial #1}{\partial  #2}}

% for creating titles for both main paper and appendix
\makeatletter
\newcommand{\settitle}{\@maketitle}
\makeatother

%% \begin{}\label{bunch of symbols}
% mathbb
\newcommand{\Ebb}{\mathbb{E}}
\newcommand{\Nbb}{\mathbb{N}}
\newcommand{\Rbb}{\mathbb{R}}


% mathbf
\newcommand{\Fbf}{\mathbf{F}}
\newcommand{\Ibf}{\mathbf{I}}
\newcommand{\Kbf}{\mathbf{K}}
\newcommand{\Lbf}{\mathbf{L}}
\newcommand{\Vbf}{\mathbf{V}}
\newcommand{\Zbf}{\mathbf{Z}}

\newcommand{\abf}{\mathbf{a}}
\newcommand{\cbf}{\mathbf{c}}
\newcommand{\fbf}{\mathbf{f}}
\newcommand{\sbf}{\mathbf{s}}
\newcommand{\tbf}{\mathbf{t}}
\newcommand{\ubf}{\mathbf{u}}
\newcommand{\vbf}{\mathbf{v}}
\newcommand{\wbf}{\mathbf{w}}
\newcommand{\xbf}{\mathbf{x}}
\newcommand{\ybf}{\mathbf{y}}
\newcommand{\zbf}{\mathbf{z}}

% boldsymbol
\newcommand{\Kbs}{\boldsymbol{K}}

\newcommand{\abss}{\boldsymbol{a}}
\newcommand{\fbs}{\boldsymbol{f}}
\newcommand{\kbs}{\boldsymbol{k}}
\newcommand{\wbs}{\boldsymbol{w}}
\newcommand{\xbs}{\boldsymbol{x}}
\newcommand{\ybs}{\boldsymbol{y}}
\newcommand{\zbs}{\boldsymbol{z}}

\newcommand{\Phibs}{\boldsymbol{\Phi}}
\newcommand{\Psibs}{\boldsymbol{\Psi}}
\newcommand{\alphabs}{\boldsymbol{\alpha}}
\newcommand{\betabs}{\boldsymbol{\beta}}
\newcommand{\phibs}{\boldsymbol{\phi}}
\newcommand{\xibs}{\boldsymbol{\xi}}
\newcommand{\psibs}{\boldsymbol{\psi}}


% mathcal
\newcommand{\Acal}{\mathcal{A}}
\newcommand{\Dcal}{\mathcal{D}}
\newcommand{\Ecal}{\mathcal{E}}
\newcommand{\Fcal}{\mathcal{F}}
\newcommand{\Gcal}{\mathcal{G}}
\newcommand{\Hcal}{\mathcal{H}}
\newcommand{\Ical}{\mathcal{I}}
\newcommand{\Kcal}{\mathcal{K}}
\newcommand{\Lcal}{\mathcal{L}}
\newcommand{\Ncal}{\mathcal{N}}
\newcommand{\Qcal}{\mathcal{Q}}
\newcommand{\Rcal}{\mathcal{R}}
\newcommand{\Scal}{\mathcal{S}}
\newcommand{\Vcal}{\mathcal{V}}
\newcommand{\Wcal}{\mathcal{W}}
\newcommand{\Xcal}{\mathcal{X}}
\newcommand{\Ycal}{\mathcal{Y}}
\newcommand{\Zcal}{\mathcal{Z}}


\newcommand{\E}{\mathrm{E}}
\newcommand{\Var}{\mathrm{Var}}
\newcommand{\Cov}{\mathrm{Cov}}
\newcommand{\indicator}{\mathbbm{1}}

\usepackage{pifont}
\newcommand{\cmark}{\color{StrokeColorCBGreen} \small \ding{51}}
\newcommand{\xmark}{\color{FontColorCBRed} \scriptsize \ding{56}}

%\end{}

\errorcontextlines=1000  % easier to debug

% For double column
\setlist[enumerate]{leftmargin=*}