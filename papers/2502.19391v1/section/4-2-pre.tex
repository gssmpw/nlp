\section{Problem Formulation}
\label{sec:preliminary}
Antibodies are Y-shaped proteins with a distinctive molecular architecture: they comprise heavy and light chains, each containing constant domains that are conserved across antibodies and variable domains that determine binding specificity. Within these variable domains, the structure alternates between framework regions (FRs) that maintain structural stability and {\em complementarity-determining regions (CDRs)} that are responsible for antigen recognition, with CDR positions defined by the IMGT numbering scheme \cite{lefranc2003imgt}. Among these regions, the CDR-H3 loop in the heavy chain variable domain exhibits the highest variability and plays a pivotal role in antigen recognition \cite{narciso2011analysis, tsuchiya2016diversity, frederic2023abmpnn}. In the context of antibody-antigen interactions, the binding surface of the antibody is termed the {\em paratope}, while its corresponding target region on the antigen is called the {\em epitope}, as illustrated in Figure~\ref{fig:fig1-complex}.

\begin{figure}[t]
  \centering
  \hspace{-2mm}
  \includegraphics[width=0.8\textwidth]{figure/antibody-igformer.pdf}
  \vspace{-3mm}
  \caption{Illustration of antibody-antigen complex.}
  \vspace{-3mm}
  \label{fig:fig1-complex}
  \vspace{-2mm}
\end{figure}


The objective of antibody co-design is to simultaneously predict the amino acid sequence and 3D structure of antibody CDRs that optimally bind a given epitope on a target antigen, as illustrated in Figure~\ref{fig:fig2-task}. 
We formulate this problem using a dual-graph representation of the binding region. The first component is an antibody graph $G_{ab} = \{V_{ab}, E_{ab}\}$, while the second is an antigen epitope graph $G_{ae} = \{V_{ae}, E_{ae}\}$. The vertices $V_{ab}$ and $V_{ae}$ represent amino acid residues, where each residue $v_i$ is characterized by its amino acid type $s_i$ and a coordinate matrix $\vect{X}_i \in \mathbb{R}^{3\times c_i}$ encoding the 3D positions of its $c_i$ atoms, including both backbone and side chain atoms. Spatial proximities between residues are captured through edges $E_{ab}$ and $E_{ae}$ constructed using a {\em $k$-nearest neighbors (kNN)} approach over all atomic positions. The complete graph construction process is detailed in Appendix~\ref{appendix-sec:graph-construct}. Within this framework, we denote the paratope residues as a subset $V_p \subseteq V_{ab}$ of the antibody graph, where we specifically focus on {\em CDR-H3} as the paratope region. 
\begin{definition}
\label{def:antibody-codesign}
    Given partial antibody sequence $\{ s_i \mid v_i\in V_{ab}, v_i\notin V_p \}$ and the target antigen epitope sequence $\{s_j \mid v_j\in V_{ae}\}$ with corresponding structural information $\vect{X}_{ab}$ and $\vect{X}_{ae}$, the goal of antibody co-design is to predict both the 1D sequence $\{s_k \mid v_k \in V_p\}$ of the paratope and 3D structure $\{\vect{X}_k \mid v_k\in V_p\}$ of the entire antibody in the context of the antibody-antigen complex.
\end{definition}
This formulation defines a joint optimization problem where the model simultaneously predicts both the amino acid sequence of the antibody paratope and its complete 3D coordinates within the antibody-antigen binding region. Through this formulation, we aim to harness the expressive power of equivariant graph neural architectures to capture the intricate spatial and chemical relationships that govern antibody-antigen interactions, enabling end-to-end structure-based antibody co-design.

