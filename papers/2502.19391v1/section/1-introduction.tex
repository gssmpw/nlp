\section{Main}

Antibodies are Y-shaped proteins that play a pivotal role in the immune system, capable of binding with high specificity to antigens, including pathogens and foreign molecules \cite{murphy2016janeway}. The design of antibodies for specific epitopes in antigens is an essential yet challenging task with broad applications in therapeutic development and drug discovery \cite{feldmann2003tnf,adams2005cancer,mullard2022fda,wang2024human}. 
The challenges arise primarily from the variability and structural complexity of complementarity-determining regions (CDRs), where binding primarily occurs \cite{he2024novo}. Understanding and modeling the intricate interactions between antibodies and antigens is crucial for developing an effective computational antibody design approach \cite{baran2017principles,tennenhouse2024computational,yan2024antibodies}.

\begin{wrapfigure}{r}{0.42\textwidth}
  \centering
  \hspace{-4mm}
  \includegraphics[width=\linewidth]{figure/task-igformer.pdf}  \vspace{-3mm}
  \caption{End-to-end antibody co-design task.}  \label{fig:fig2-task}
  \vspace{-3mm}
\end{wrapfigure}

The past decade has witnessed a significant evolution in antibody design approaches. Traditional methods, including sequence-based models \cite{alley2019unirep,shin2021protein} and energy-based optimization frameworks \cite{pantazes2010optcdr,lapidoth2015abdesign,adolf2018rabd}, provided initial solutions but fell short of fully capturing the structural and functional interactions between antibodies and antigens \cite{ingraham2019generative}. 
More recently, learning-based approaches, particularly equivariant graph neural networks, deep generative models, and diffusion models \cite{saka2021antibody,jin2022refinegnn,watson2022broadly,luo2022diffab,kong2023dymean,martinkus2023abdiffuser,wang2024iggm}, have emerged as powerful tools for simultaneously co-designing CDR 1D sequences and 3D structures.
These models represent a significant advancement over traditional approaches by leveraging deep learning to model complex sequential and structural relationships. However, many of these methods simplify the problem by focusing exclusively on backbone modeling or specific CDR regions, overlooking side-chain interactions and the full-atom geometry that are crucial to accurately modeling antibody-antigen binding \cite{ruffolo2023fast,martinkus2023abdiffuser}.




Traditional computational antibody design relies on multi-stage pipelines that separately handle structure prediction, docking, and CDR generation. A recent work, dyMEAN \cite{kong2023dymean}, integrates these tasks into a unified end-to-end framework, providing a more streamlined approach to antibody co-design. However, despite these advances, dyMEAN does not fully take advantage of intricate antibody-antigen interactions, leaving opportunities for further improvement in capturing the detailed molecular interplay at binding interfaces. These limitations highlight the need for a more sophisticated approach that can simultaneously handle full-atom geometry and complex hierarchical interactions in antibody-antigen complexes. Further discussion of related works can be found in Appendix~\ref{appendix-sec:related-work}.



To address these limitations, in this work, we introduce Igformer\footnote{\em \underline{I}mmuno\underline{g}lobulin Trans\underline{former} (Igformer)}, a novel end-to-end framework for antibody co-design that advances antibody co-design through three key innovations. At its core, Igformer introduces a sophisticated antibody-antigen inter-graph refinement strategy that uniquely integrates personalized propagation with a global attention mechanism. 
The personalized propagation scheme models local chemical properties and geometric constraints, while the complementary global attention mechanism captures long-range structural dependencies. This novel dual-scale architecture enables Igformer to maintain atomic-level precision while effectively capturing intricate residue-level interactions within the antibody-antigen binding interface. Furthermore, Igformer incorporates E(3)-equivariance throughout the framework, ensuring that all structural predictions adhere to the physical and geometric constraints of biological systems.


Through comprehensive experimental evaluation, Igformer demonstrates superior effectiveness across multiple challenging antibody design tasks. For epitope-binding CDR design, Igformer shows a 2.02\% improvements over state-of-the-art models in terms of amino acid recovery rate. For antigen-antibody complex structure prediction, Igformer achieves remarkable accuracy with an 11.84\% reduction in RMSD compared to existing approaches. These substantial improvements validate the effectiveness of our technical innovations in addressing the challenges of full-atom geometry and sequence-structure co-design. Igformer thus represents a significant advancement in computational antibody design, offering promising new directions for accelerating drug discovery and therapeutic development. 


