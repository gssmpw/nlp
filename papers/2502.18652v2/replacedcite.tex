\section{Literature Review}
The demand to derive traffic movements and patterns from data efficiently in order to ameliorate urban mobility challenges enables traffic agencies to adopt advanced ML methodologies. In this context, LLMs serve as a pivotal tool, potentially enhancing the deployment of ML models in mobility data analysis. This section seeks to encapsulate and evaluate contemporary research on the role of LLMs in two key areas: 1) the challenges associated with these approaches, and 2) the emerging role of LLMs and the multi-agent frameworks in overcoming these challenges. Prior to further discussion on LLMs, this section provides an overview of the current state of research on ML applications in the traffic field, which may serve as a foundational basis for the potential integration of LLMs.

\subsection{ML in mobility analysis}

Traditional statistical models have long been used for traffic forecasting and analysis; however, the non-linear and complex nature of traffic patterns often limits their effectiveness. The advent of ML, particularly deep learning (DL) models, has provided more robust tools for capturing spatiotemporal dependencies in traffic data. Given such nature of traffic data, researchers frequently utilize RNNs, especially LSTM networks, which excel at modeling temporal sequences and have been employed to predict traffic flow based on historical data ____. The accuracy of results can be improved by integrating RNNs into various analyses or combining them with other methodologies. For example, combining time series analysis with a 3-layer LSTM-BiLSTM model ____ and integrating Federated Deep Learning with LSTM ____ have yielded promising outcomes. Additionally, some researchers applied attention mechanisms to RNNs, as exemplified by the BiGRU-BiGRU model ____, hybrid LSTM, and sequential LSTM ____. These approaches enable the models to capture intricate dependencies within trajectory data, which is crucial for effective network traffic management and congestion mitigation.

To improve the performance of RNN-based models, CNNs have been incorporated into time-series analysis for spatial feature extraction due to their proficiency in handling grid-like data structures, making them suitable for analyzing traffic flow in urban grids ____. For instance, the CNN-LSTM hybrid model has been shown to provide accurate predictions by capturing both spatial and temporal traffic features ____. In another study, LSTM was utilized to capture both inter-day and intra-day patterns, while CNN was employed to integrate contextual information ____.

Graph Neural Networks (GNNs) have also gained attention for modeling traffic networks, representing road networks as graphs to capture the relational information between different nodes (e.g., intersections, road segments) ____. Recent GNN-related research has focused on addressing the uncertainty of traffic data, thereby enhancing the robustness of GNN models under varying and sparse data conditions. For instance, the Progressive Graph Convolutional Network ____ learned similarities and extracted temporal features from graph nodes. The Ensemble-Based Spatiotemporal GNN ____ optimized hyperparameter configurations and decomposed uncertainty from GNN ensembles. The Probabilistic GNN ____ quantified the uncertainty of traffic demand while simultaneously revealing the spatiotemporal pattern of this uncertainty. These models have shown promise in improving the accuracy of traffic prediction and anomaly detection.

Each of these neural network approaches brings distinct advantages to the analysis of traffic data, effectively handling the complex spatial and temporal dependencies within such datasets. However, the wide variety of models in these categories, each with its own strengths and limitations, poses challenges for agencies in terms of scalability, efficiency, the need for expertise in both transportation engineering and ML, as well as concerns about data privacy and security. In contrast, AI agents powered by LLMs could offer a more cost-effective solution, potentially streamlining traffic mobility analysis.

\subsection{AI Agents and LLMs}

The emergence of AI agents empowered by LLMs offers potential solutions to the above-mentioned challenges. AI agents are autonomous systems that interact with their environment through sensors and actuators to make decisions and perform tasks, aiming to achieve specific objectives efficiently and effectively ____. LLMs, such as OpenAI's GPT series and LLaMA, have demonstrated remarkable capabilities in understanding and generating human-like text ____. In transportation, LLMs can serve as intermediaries between users and complex ML models, enabling natural language interfaces, automated data processing, and enhanced decision support. Current LLM-related research in the transportation field can be classified into LLMs as predictors, synthetic data generators, assistants, and evaluators.

\subsubsection{LLM-as-a-Predictor}
LLMs are well-suited for predicting spatial-temporal traffic data due to their ability to capture complex dependencies, integrate multimodal data, and model sequential patterns effectively. The R2T-LLM ____ was proposed as a novel traffic flow prediction model leveraging LLMs for both accurate and explainable traffic forecasting. However, as noted in the paper, generating coherent and contextually relevant explanations alongside predictions remains a challenge. To help LLMs better understand time-series data, Ren et al. ____ effectively leveraged pre-trained LLMs for traffic prediction, showing strong performance in both full-sample and few-shot scenarios. However, combining CNNs and GCNs to handle spatiotemporal features increases computational complexity, which may pose challenges for implementation in resource-constrained or real-time systems. Another approach to improving LLMs’ understanding of spatial-temporal data is to tokenize timesteps and locations for embedding in the models. For instance, ST-LLM ____ introduced a partially frozen attention strategy that effectively captures global spatial-temporal dependencies, though its reliance on spatial-temporal embeddings can hinder its suitability for real-time decision-making applications such as dynamic traffic management systems. UrbanGPT ____ utilized spatiotemporal instruction-tuning to achieve accurate zero-shot predictions in urban scenarios. Although this model is designed for generalization, fine-tuning may still be needed to optimize results for specific urban contexts, which could be challenging for organizations lacking the necessary expertise or computational resources.

\subsubsection{LLM-as-a-Generator}  
LLMs are adept at generating realistic sequence data, making them particularly useful for generating traffic data in simulations. While challenges remain in ensuring the fidelity and diversity of the generated data, LLMs have been widely employed to simulate complex systems such as urban traffic flows and human mobility patterns ____. For example, SeGPT, a framework leveraging ChatGPT, excels at generating dynamic and complex scenarios for autonomous vehicle trajectory prediction, aiding in model testing and adaptation across various real-world conditions ____. However, generalizing across different driving scenarios, particularly in low-data environments, remains a challenge. Similarly, Chang et al. presented a framework for generating rare and complex corner scenarios for autonomous vehicle testing using LLMs, enhancing scenario diversity and interoperability ____. Nonetheless, the reliance on LLMs to generate complex multi-agent scenarios introduces significant complexity in model design and implementation, which may make it difficult to ensure that the scenarios accurately reflect real-world conditions. MobilityGPT ____, by leveraging a gravity model and road connectivity matrix, successfully generated realistic synthetic trajectories. However, the absence of formal privacy guarantees for the generated data raises concerns when such synthetic data is intended to preserve privacy, potentially exposing sensitive information.

\subsubsection{LLM-as-an-Assistant}  
LLMs can act as powerful assistants, enhancing various aspects of operations, management, and user interaction. Zhang et al. ____ introduced TrafficGPT, which combines ChatGPT with traffic foundation models (TFMs) to enable natural language interaction with traffic data analysis and decision-making processes. The model translates natural language queries into prompts for TFMs, which then analyze traffic data to produce actionable insights and recommendations. This innovative approach improves the accuracy, efficiency, and reliability of urban traffic management decisions. Similarly, DriveLLM ____ is a framework that integrates LLMs into autonomous driving systems, enhancing commonsense reasoning and decision-making in dynamic environments. However, in complex environments with multiple moving objects, DriveLLM tends to make overly cautious decisions.

Research on LLM-as-an-Assistant in the transportation field remains relatively underdeveloped compared to their use as predictors or generators. Although some frameworks have explored LLMs for enhancing traffic management and autonomous driving through natural language interaction and decision-making, the scope and depth of this research remain limited. The input methods in these studies may not always be flexible or generic enough, as users might be uncertain about the best approach to achieve traffic project objectives. Additionally, the frameworks require substantial human interaction, which may not always be the most efficient use of time and effort. 

\subsubsection{LLM-as-a-Judge}
To evaluate LLM-generated results, traditional methods like BLEU ____, ROUGE ____, and METEOR ____ focus on word and phrase matching but lack semantic and reasoning evaluation. Recent approaches like BERTScore ____ and MoverScore ____ use contextual embeddings to assess semantics but are sensitive to nuances in word embeddings and may misinterpret out-of-context phrases. To address these issues, LLM-based evaluation methods such as QAGScore, GPTScore, G-Eval, and Prometheus have been developed. These methods offer better semantic understanding and interpretability. QAGScore ____ evaluates factual consistency through a question-answering approach but is dependent on question-and-answer quality. Prometheus ____ combines multiple metrics like factuality, fluency, and coherence, but its complex implementation and high computational cost are drawbacks. GPTScore ____ offers flexibility through fine-tuning for different criteria. Similarly, G-Eval ____, based on GPT models, excels at evaluating complex qualities like logical flow and coherence, aligning closely with human judgments. In this paper, though GPTScore is fast and flexible, G-Eval is preferred for its better alignment with human evaluations and logical thinking and reliable and consistent performance in scoring tasks such as coherence, factuality, and relevance.

This paper aims to address the research gap by advancing LLM-as-an-Assistant research and developing a self-motivating, generic framework—IDM-GPT—that assists users in identifying optimal analysis strategies, enhancing data analysis efficiency, decision-making, and user interaction across various transportation tasks.