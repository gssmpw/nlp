In this section, we first introduce the motivation for watermarking LLM-generated code. Next, we present several criteria the watermarking framework shall meet and the potential threats from the adversaries. 

\subsection{Scenario and Motivation}
Watermarking is a viable technique to protect the IP of code LLM owners and trace the distribution of their generated code. A general scenario of the code LLM watermarking is shown in Figure~\ref{fig:overview}. As seen, the code LLM API encodes watermarks on the generated code before distributing it to end users. If the LLM owner suspects the code snippet was used without acquiring permission, they send the snippet to arbitrators to determine the ownership. 
Third-party arbitrators decode the signatures from the watermarked code snippet and determine if it matches the owner-provided watermark. 

\begin{figure}[!ht]
    \centering
    \includegraphics[width=\columnwidth]{./figs/overview.pdf}
    \caption{A general scenario of watermark insertion and verification.}
    \label{fig:overview}
\end{figure}

The watermarking framework can be used to:

\textbf{Protect the IP of Code LLM}: Developing efficient and powerful code large language models requires high-quality training data~\cite{li2023starcoder,javaheripi2023phi}, tokenizer/model architecture optimization~\cite{liu2023chipnemo,roziere2023code}, and better training strategy~\cite{li2023starcoder,roziere2023code}. 
As such, the code LLM constitutes valuable IP and requires watermarking techniques to protect the generated outputs.


\textbf{Trace LLM-generated Code Distribution}: The LLM-generated code can contain vulnerabilities, like overflow or underflow. The buggy statements injected intentionally~\cite{lee2023wrote} or generated unintentional~\cite{sandoval2023lost} can result in security threats. Watermarking helps developers identify the source of the code, and raise their attention when the code is machine-generated.

\textbf{Detect Code Plagiarism}: In the class or hiring assessment that involves coding, the students or candidates may use code from the LLM API to complete the assignment. Detecting watermarks from the uploaded code snippets helps teachers and recruiters to find students or candidates who are plagiarized. 

\subsection{Watermarking Criteria}

An ideal watermarking framework should follow these criteria~\cite{zhang2024emmark}: 

\begin{itemize}
    \item \textbf{Extractability}: the encoded watermarks should be successfully extracted from the watermarked code. 
    
    \item \textbf{Fidelity}: the watermarked code shall preserve the original code's executability and correctness.

    \item \textbf{Stealthiness}: the encoded watermarks are imperceptible upon human or machine inspection.   

    \item \textbf{Robustness}: the watermarks should withstand various watermark removal or forging attacks.

    \item \textbf{Efficiency}: the watermark insertion should be efficient both in terms of time and computation overheads    
\end{itemize}

\sys{} incorporates the above watermarking criteria into the system design and emerges as an effective framework for LLM-generated code IP protection. 

\subsection{Threats}

Following prior work~\cite{lee2023wrote,yang2023towards}, we consider the adversaries as the end-users with black-box access to the code LLM API. The code generation modules and the watermarking framework are beyond their reach. The adversaries target to detect, remove, or forge the watermarks, all while preserving the executability and correctness of the code. We perform comprehensive evaluations of \sys{} resiliency under the subsequent attacks:

\begin{itemize}
    \item \textbf{Watermark Detect Attack}: The adversaries detect the existence of the watermark by calculating the syntax-level and structural-level distance~\cite{li2023protecting} or using machine-learning based approaches~\cite{zhang2023remark}.

    \item  \textbf{Watermark Remove Attack}: The adversaries remove the watermark by renaming the variable~\cite{lee2023wrote}, applying code optimization toolkit~\cite{li2024resilient}, transferring code styles~\cite{yang2023towards}, or de-watermarking~\cite{yang2023towards}.
 
    \item  \textbf{Watermark Forge Attack}: The adversaries re-watermark the code with another watermarking framework and falsely claim code ownership~\cite{li2024resilient}.
\end{itemize}