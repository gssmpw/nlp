%%%%%%%% ICML 2025 EXAMPLE LATEX SUBMISSION FILE %%%%%%%%%%%%%%%%%

\documentclass{article}

% Recommended, but optional, packages for figures and better typesetting:
\usepackage{microtype}
\usepackage{graphicx}
\usepackage{subfigure}
\usepackage{booktabs} % for professional tables
\usepackage{listings}
\usepackage{xcolor}

\lstset{
    language=Python,
    basicstyle=\ttfamily\scriptsize,
    keywordstyle=\color{blue},
    commentstyle=\color{gray},
    stringstyle=\color{red},
    showstringspaces=false,
    numbers=left,
    numberstyle=\tiny,
    breaklines=true,
    frame=single,
}

% hyperref makes hyperlinks in the resulting PDF.
% If your build breaks (sometimes temporarily if a hyperlink spans a page)
% please comment out the following usepackage line and replace
% \usepackage{icml2025} with \usepackage[nohyperref]{icml2025} above.
\usepackage{hyperref}


% Attempt to make hyperref and algorithmic work together better:
\newcommand{\theHalgorithm}{\arabic{algorithm}}

% Use the following line for the initial blind version submitted for review:
%\usepackage{icml2025}

% If accepted, instead use the following line for the camera-ready submission:
\usepackage[accepted]{icml2025}

% For theorems and such
\usepackage{amsmath}
\usepackage{amssymb}
\usepackage{mathtools}
\usepackage{amsthm}
\usepackage{multirow}
\usepackage{booktabs}
\usepackage{multirow}
\usepackage{color,soul}
 
\usepackage{capt-of}



%\usepackage{algorithm} 
% if you use cleveref..
\usepackage[capitalize,noabbrev]{cleveref}
\usepackage{xspace}
\usepackage{sidecap}
\usepackage{soul}
%%%%%%%%%%%%%%%%%%%%%%%%%%%%%%%%
% THEOREMS
%%%%%%%%%%%%%%%%%%%%%%%%%%%%%%%%
\theoremstyle{plain}
\usepackage{pifont}
\definecolor{failed}{gray}{0.7}
\newtheorem{theorem}{Theorem}[section]
\newtheorem{proposition}[theorem]{Proposition}
\newtheorem{lemma}[theorem]{Lemma}
\newtheorem{corollary}[theorem]{Corollary}
\theoremstyle{definition}
\newtheorem{definition}[theorem]{Definition}
\newtheorem{assumption}[theorem]{Assumption}
\theoremstyle{remark}
\newtheorem{remark}[theorem]{Remark}
\newcommand{\cmark}{\ding{51}}%
\newcommand{\xmark}{\ding{55}}%
\colorlet{listing-highlight}{yellow!50} 
\newcommand{\sys}{RoSeMary}
% Todonotes is useful during development; simply uncomment the next line
%    and comment out the line below the next line to turn off comments
%\usepackage[disable,textsize=tiny]{todonotes}
\usepackage[textsize=tiny]{todonotes}

\newcommand{\Prv}{$\mathcal{P}$\xspace}
\newcommand{\Cir}{$\mathcal{C}$\xspace}
\newcommand{\Vrf}{$\mathcal{V}$\xspace}

\lstdefinestyle{mypy}{ %My Python style definition (based on Google Colaboratory)
    language=Python,
    escapeinside={\%*}{*)},
    numbers=left, 
    numberstyle=\footnotesize, 
    numbersep=1em, 
    xleftmargin=1em, 
    framextopmargin=2em, 
    framexbottommargin=2em, 
    showspaces=false, 
    showtabs=false, 
    showstringspaces=false, 
    columns=flexible,
    keepspaces=true,
    tabsize=4,
    basicstyle=\ttm,
    backgroundcolor=\color{Background},
    keywords={as,assert,async,await,break,continue,del,elif,else,except,finally,for,from,if,import,pass,raise,return,try,while,with,yield}, 
    keywordstyle={\ttm\color{purpy}},
    keywords={[2]@invariant,False,None,True,and,class,def,global,in,is,lambda,nonlocal,not,or}, 
    keywordstyle={[2]\ttm\color{bluepy}},
    keywords={[3]@invariant,abs,all,any,ascii,bin,bool,bytearray,bytes,callable,chr,classmethod,compile,complex,delattr,dict,dir,divmod,enumerate,eval,exec,filter,float,format,frozenset,getattr,globals,hasattr,hash,help,hex,id,input,int,isinstance,issubclass,iter,len,list,locals,map,max,memoryview,min,next,object,oct,open,ord,pow,print,property,range,repr,reversed,roundset,setattr,slice,sorted,@staticmethod,str,sum,super,tuple,type,vars,zip, myfun}, % myfun should be brown in its definition
    keywordstyle={[3]\ttm\color{brownpy}},   
    stringstyle=\color{strpy}, 
    commentstyle=\color{commentpy},
}

\lstnewenvironment{python}[1][]{ 
\lstset{style=mypy, frame=l, numbers=none}
}{}
% The \icmltitle you define below is probably too long as a header.
% Therefore, a short form for the running title is supplied here:
\icmltitlerunning{}

\begin{document}

\twocolumn[
\icmltitle{\emph{Ro}bust and \emph{Se}cure Code Water\emph{ma}rking for Large Language Models via ML/C\emph{ry}pto Codesign}

% It is OKAY to include author information, even for blind
% submissions: the style file will automatically remove it for you
% unless you've provided the [accepted] option to the icml2025
% package.

% List of affiliations: The first argument should be a (short)
% identifier you will use later to specify author affiliations
% Academic affiliations should list Department, University, City, Region, Country
% Industry affiliations should list Company, City, Region, Country

% You can specify symbols, otherwise they are numbered in order.
% Ideally, you should not use this facility. Affiliations will be numbered
% in order of appearance and this is the preferred way.
\icmlsetsymbol{equal}{*}

\begin{icmlauthorlist}
\icmlauthor{Ruisi Zhang}{equal,yyy}
\icmlauthor{Neusha Javidnia}{equal,yyy}
\icmlauthor{Nojan Sheybani}{yyy}
\icmlauthor{Farinaz Koushanfar}{yyy}
%\icmlauthor{}{sch}
%\icmlauthor{}{sch}
\end{icmlauthorlist}

\icmlaffiliation{yyy}{University of California, San Diego}

\icmlcorrespondingauthor{Ruisi Zhang}{ruz032@ucsd.edu}

% You may provide any keywords that you
% find helpful for describing your paper; these are used to populate
% the "keywords" metadata in the PDF but will not be shown in the document
\icmlkeywords{Machine Learning, ICML}

\vskip 0.3in
]

% this must go after the closing bracket ] following \twocolumn[ ...

% This command actually creates the footnote in the first column
% listing the affiliations and the copyright notice.
% The command takes one argument, which is text to display at the start of the footnote.
% The \icmlEqualContribution command is standard text for equal contribution.
% Remove it (just {}) if you do not need this facility.

%\printAffiliationsAndNotice{}  % leave blank if no need to mention equal contribution
\printAffiliationsAndNotice{\icmlEqualContribution} % otherwise use the standard text.

\begin{abstract}
This paper introduces \sys{}, the first-of-its-kind ML/Crypto codesign watermarking framework that regulates LLM-generated code to avoid intellectual property rights violations and inappropriate misuse in software development. 
High-quality watermarks adhering to the detectability-fidelity-robustness tri-objective are limited due to codes' low-entropy nature. Watermark verification, however, often needs to reveal the signature and requires re-encoding new ones for code reuse, which potentially compromising the system's usability.
To overcome these challenges, \sys{} obtains high-quality watermarks by training the watermark insertion and extraction modules end-to-end to ensure (i) unaltered watermarked code functionality and (ii) enhanced detectability and robustness leveraging pre-trained CodeT5 as the insertion backbone to extract better code features. In the deployment, \sys{} uses zero-knowledge proofs for secure verification without revealing the underlying signatures. 
Extensive evaluations demonstrated \sys{} achieves high detection accuracy while preserving the code functionality. \sys{} is also robust against attacks and provides efficient secure watermark verification.


%Existing works struggle to balance the detectability-fidelity-robustness tri-objective for practical code watermarking. Moreover,
%disclosing watermarks for verification often requires re-encoding new signatures to reuse the code, potentially compromising its functionality.
 
\end{abstract}


\section{Introduction}\label{intro}
\section{Introduction}
Backdoor attacks pose a concealed yet profound security risk to machine learning (ML) models, for which the adversaries can inject a stealth backdoor into the model during training, enabling them to illicitly control the model's output upon encountering predefined inputs. These attacks can even occur without the knowledge of developers or end-users, thereby undermining the trust in ML systems. As ML becomes more deeply embedded in critical sectors like finance, healthcare, and autonomous driving \citep{he2016deep, liu2020computing, tournier2019mrtrix3, adjabi2020past}, the potential damage from backdoor attacks grows, underscoring the emergency for developing robust defense mechanisms against backdoor attacks.

To address the threat of backdoor attacks, researchers have developed a variety of strategies \cite{liu2018fine,wu2021adversarial,wang2019neural,zeng2022adversarial,zhu2023neural,Zhu_2023_ICCV, wei2024shared,wei2024d3}, aimed at purifying backdoors within victim models. These methods are designed to integrate with current deployment workflows seamlessly and have demonstrated significant success in mitigating the effects of backdoor triggers \cite{wubackdoorbench, wu2023defenses, wu2024backdoorbench,dunnett2024countering}.  However, most state-of-the-art (SOTA) backdoor purification methods operate under the assumption that a small clean dataset, often referred to as \textbf{auxiliary dataset}, is available for purification. Such an assumption poses practical challenges, especially in scenarios where data is scarce. To tackle this challenge, efforts have been made to reduce the size of the required auxiliary dataset~\cite{chai2022oneshot,li2023reconstructive, Zhu_2023_ICCV} and even explore dataset-free purification techniques~\cite{zheng2022data,hong2023revisiting,lin2024fusing}. Although these approaches offer some improvements, recent evaluations \cite{dunnett2024countering, wu2024backdoorbench} continue to highlight the importance of sufficient auxiliary data for achieving robust defenses against backdoor attacks.

While significant progress has been made in reducing the size of auxiliary datasets, an equally critical yet underexplored question remains: \emph{how does the nature of the auxiliary dataset affect purification effectiveness?} In  real-world  applications, auxiliary datasets can vary widely, encompassing in-distribution data, synthetic data, or external data from different sources. Understanding how each type of auxiliary dataset influences the purification effectiveness is vital for selecting or constructing the most suitable auxiliary dataset and the corresponding technique. For instance, when multiple datasets are available, understanding how different datasets contribute to purification can guide defenders in selecting or crafting the most appropriate dataset. Conversely, when only limited auxiliary data is accessible, knowing which purification technique works best under those constraints is critical. Therefore, there is an urgent need for a thorough investigation into the impact of auxiliary datasets on purification effectiveness to guide defenders in  enhancing the security of ML systems. 

In this paper, we systematically investigate the critical role of auxiliary datasets in backdoor purification, aiming to bridge the gap between idealized and practical purification scenarios.  Specifically, we first construct a diverse set of auxiliary datasets to emulate real-world conditions, as summarized in Table~\ref{overall}. These datasets include in-distribution data, synthetic data, and external data from other sources. Through an evaluation of SOTA backdoor purification methods across these datasets, we uncover several critical insights: \textbf{1)} In-distribution datasets, particularly those carefully filtered from the original training data of the victim model, effectively preserve the model’s utility for its intended tasks but may fall short in eliminating backdoors. \textbf{2)} Incorporating OOD datasets can help the model forget backdoors but also bring the risk of forgetting critical learned knowledge, significantly degrading its overall performance. Building on these findings, we propose Guided Input Calibration (GIC), a novel technique that enhances backdoor purification by adaptively transforming auxiliary data to better align with the victim model’s learned representations. By leveraging the victim model itself to guide this transformation, GIC optimizes the purification process, striking a balance between preserving model utility and mitigating backdoor threats. Extensive experiments demonstrate that GIC significantly improves the effectiveness of backdoor purification across diverse auxiliary datasets, providing a practical and robust defense solution.

Our main contributions are threefold:
\textbf{1) Impact analysis of auxiliary datasets:} We take the \textbf{first step}  in systematically investigating how different types of auxiliary datasets influence backdoor purification effectiveness. Our findings provide novel insights and serve as a foundation for future research on optimizing dataset selection and construction for enhanced backdoor defense.
%
\textbf{2) Compilation and evaluation of diverse auxiliary datasets:}  We have compiled and rigorously evaluated a diverse set of auxiliary datasets using SOTA purification methods, making our datasets and code publicly available to facilitate and support future research on practical backdoor defense strategies.
%
\textbf{3) Introduction of GIC:} We introduce GIC, the \textbf{first} dedicated solution designed to align auxiliary datasets with the model’s learned representations, significantly enhancing backdoor mitigation across various dataset types. Our approach sets a new benchmark for practical and effective backdoor defense.



  
\section{Background and Related Work}\label{bg}
In this section, we introduce the related work and background for code LLMs and watermarking code generated by LLMs.

\subsection{LLM-generated Code Watermarking}
Code large language models are widely applied in the programming language domain to assist software development, e.g., code generation~\cite{copilot,roziere2023code,luo2023wizardcoder}, code translation~\cite{pan2023understanding,roziere2021leveraging}, code summarization~\cite{ahmed2022few,gao2023makes}, automated testing~\cite{yu2023llm,kang2023large}. 
The goal of code LLMs is to learn a probabilistic model $p(c|x)$ that predicts a code snippet $c$ given a context $x$, which can include natural language descriptions, partial code, or other relevant information. Code LLMs are typically trained on large corpora of code data using self-supervised learning objectives to maximize the likelihood of the observed code sequences. Formally, given a dataset $D = {c_1, c_2, ..., c_N}$ of code snippets, the training objective is to minimize the negative log-likelihood, where $p(c_i)$ is modeled by the large language models.

\begin{equation}
\mathcal{L} = -\frac{1}{N}\sum_{i=1}^{N}\log p(c_i)
\end{equation}


Similar to natural language watermarking, watermarking code data can generally be categorized into three types~\cite{zhang2023remark}: (i) rule-based watermarking, (ii) inference-based watermarking, and, (iii) neural-based watermarking.  
The rule-based watermarking ACW~\cite{li2024resilient} maintains a transformation table and uses the code style transfer as the watermarks. However, the methodology requires additional engineering for new programming languages and exhibits low transferability. The inference-based watermarking SWEET~\cite{lee2023wrote} encodes signatures during the LLM inference stages. By watermarking on the high-entropy tokens, SWEET~\cite{lee2023wrote} maintains the correctness and the executability of the watermarked code. However, it comes at the cost of the lower watermarking strength the framework can provide. The neural-based watermarking SrcMarks~\cite{yang2023towards} leverages an end-to-end learning technique to integrate the watermarking signatures into the codes while maintaining AST-invariant.
However, it falls short of the watermarking capacity it can provide and requires re-training for new programming languages.  

In this paper, we improve upon prior work and build an efficient, high-capacity, and robust end-to-end neural-based watermarking framework that demonstrates high watermarking capacity, cross-programming language transferability, and robustness.

\subsection{Zero-knowledge Proofs}

(ZKPs) are a cryptographic primitive that allows a prover to prove knowledge of a secret value $w$ to a verifier. In a standard ZKP scheme, the prover convinces convinces a verifier that $w$ is a valid input such that $y=\mathcal{C}(x, w)$, in which $C$ is an arbitrary computation and $x$ and $y$ are public inputs and outputs, respectively. Zero-knowledge proof generation can be performed in an interactive or non-interactive manner, depending on the application. One of the main drawbacks of interactive schemes is it limits proofs to \textit{designated-verifier} settings, meaning proof generation, which is the most computationally heavy process in ZKP workflows must be repeated for every new verifier. Non-interactive ZKPs allow for the \textit{publicly verifiable} setting, meaning once a proof is generated attesting correct computation or valid data, it can be verified by any third party. ZKROWNN~\cite{sheybani2023zkrownn} shows the feasibility of non-interactive ZKPs for watermark verification in deep neural networks, requiring only KBs of communication for a user to verify a proof. 

\textcolor{red}{@Nojan, I've copied your texts, do you want to add more?}

\section{Method}\label{method}
\section{Study Design}
% robot: aliengo 
% We used the Unitree AlienGo quadruped robot. 
% See Appendix 1 in AlienGo Software Guide PDF
% Weight = 25kg, size (L,W,H) = (0.55, 0.35, 06) m when standing, (0.55, 0.35, 0.31) m when walking
% Handle is 0.4 m or 0.5 m. I'll need to check it to see which type it is.
We gathered input from primary stakeholders of the robot dog guide, divided into three subgroups: BVI individuals who have owned a dog guide, BVI individuals who were not dog guide owners, and sighted individuals with generally low degrees of familiarity with dog guides. While the main focus of this study was on the BVI participants, we elected to include survey responses from sighted participants given the importance of social acceptance of the robot by the general public, which could reflect upon the BVI users themselves and affect their interactions with the general population \cite{kayukawa2022perceive}. 

The need-finding processes consisted of two stages. During Stage 1, we conducted in-depth interviews with BVI participants, querying their experiences in using conventional assistive technologies and dog guides. During Stage 2, a large-scale survey was distributed to both BVI and sighted participants. 

This study was approved by the University’s Institutional Review Board (IRB), and all processes were conducted after obtaining the participants' consent.

\subsection{Stage 1: Interviews}
We recruited nine BVI participants (\textbf{Table}~\ref{tab:bvi-info}) for in-depth interviews, which lasted 45-90 minutes for current or former dog guide owners (DO) and 30-60 minutes for participants without dog guides (NDO). Group DO consisted of five participants, while Group NDO consisted of four participants.
% The interview participants were divided into two groups. Group DO (Dog guide Owner) consisted of five participants who were current or former dog guide owners and Group NDO (Non Dog guide Owner) consisted of three participants who were not dog guide owners. 
All participants were familiar with using white canes as a mobility aid. 

We recruited participants in both groups, DO and NDO, to gather data from those with substantial experience with dog guides, offering potentially more practical insights, and from those without prior experience, providing a perspective that may be less constrained and more open to novel approaches. 

We asked about the participants' overall impressions of a robot dog guide, expectations regarding its potential benefits and challenges compared to a conventional dog guide, their desired methods of giving commands and communicating with the robot dog guide, essential functionalities that the robot dog guide should offer, and their preferences for various aspects of the robot dog guide's form factors. 
For Group DO, we also included questions that asked about the participants' experiences with conventional dog guides. 

% We obtained permission to record the conversations for our records while simultaneously taking notes during the interviews. The interviews lasted 30-60 minutes for NDO participants and 45-90 minutes for DO participants. 

\subsection{Stage 2: Large-Scale Surveys} 
After gathering sufficient initial results from the interviews, we created an online survey for distributing to a larger pool of participants. The survey platform used was Qualtrics. 

\subsubsection{Survey Participants}
The survey had 100 participants divided into two primary groups. Group BVI consisted of 42 blind or visually impaired participants, and Group ST consisted of 58 sighted participants. \textbf{Table}~\ref{tab:survey-demographics} shows the demographic information of the survey participants. 

\subsubsection{Question Differentiation} 
Based on their responses to initial qualifying questions, survey participants were sorted into three subgroups: DO, NDO, and ST. Each participant was assigned one of three different versions of the survey. The surveys for BVI participants mirrored the interview categories (overall impressions, communication methods, functionalities, and form factors), but with a more quantitative approach rather than the open-ended questions used in interviews. The DO version included additional questions pertaining to their prior experience with dog guides. The ST version revolved around the participants' prior interactions with and feelings toward dog guides and dogs in general, their thoughts on a robot dog guide, and broad opinions on the aesthetic component of the robot's design. 


\section{Experiment}\label{exp}
\section{Experiments}
\label{sec:exp}
\subsection{Experimental settings}

\noindent\textbf{Benchmark.}  We conduct experiments on two established 3D occupancy benchmarks: (i) nuScenes~\cite{nuScenes}, which provides instance-level annotations with manually labeled 3D bounding boxes (position/size/orientation) for dynamic objects, and (ii) Occ3D~\cite{Occ3D}, which generates voxel-level occupancy labels (0.4m resolution) through automated LiDAR point cloud aggregation and mesh reconstruction, including occlusion states. Both benchmarks share identical scene configurations of 1,050 driving scenes, each containing up to 40 timestamped frames. Every frame includes six synchronized camera views (front, front-left, front-right, back, back-left, back-right) at 1600$\times$900 resolution. In our experiments, we extend single-frame baselines~\cite{MonoScene,surroundOcc,viewformer} by aggregating features from $N$ historical keyframes. Additionally, we extract unlabeled intermediate frames from the ``sweeps'' folder~\cite{nuScenes} to provide implicit motion cues, enabling self-supervised temporal consistency learning.

\noindent\textbf{Implementation details.} For the nuScenes benchmark~\cite{nuScenes}, we follow the parameter settings of SurroundOcc~\cite{surroundOcc}, using $Cam=6$, $p=6$, $v=32$,$L=116$, and $W=200$. For the Occ3D benchmark~\cite{Occ3D}, we adopt ViewFormer's~\cite{viewformer} standard setup with $Cam=6$, $p=6$, $v=32$, $L=32$, and $W=88$. The output of the occupancy result on both benchmarks is formatted into a vector with dimensions $[200, 200, 16]$. In this vector, the first two dimensions (200 and 200) represent the length and width, while the third (16) indicates the height. The occupancy result covers a range from -50 meters to 50 meters in both width and length, and the vertical height varies from -5 meters to 3 meters. Each voxel corresponds to a cube measuring 0.5 meters on each side. Occupied voxels are categorized into one of 17~\cite{nuScenes,surroundOcc} and 18~\cite{Occ3D} semantic classes.
More details on implementation can be found in the supplementary material.

\subsection{Evaluation Metrics}

To validate the temporal consistency and occupancy accuracy of moving and static objects, objects are divided into two general classes~\cite{Cam4docc}: General Moving Objects (GMO) and General Static Objects (GSO). Detailed classification classes are introduced in the supplementary material.

\noindent\textbf{Occupancy Accuracy Metric.} To ensure rigorous evaluation across different benchmarks, we employ both Intersection over Union (IoU) and Mean Intersection over Union (mIoU) metrics. These metrics are widely adopted in 3D semantic occupancy prediction tasks~\cite{PASCAL, Microsoft_COCO, Cityscapes_dataset, Mask_R_CNN}. The mIoU are calculated separately for three category groups: All classes, GMO classes, and GSO classes.

\noindent\textbf{Temporal Consistency Metric.} \label{para:consistency_metric} To evaluate the effect achieved by integrating \ours\ with baseline models, we propose a temporal consistency metric. We aim to detect and measure changes in a scene from one frame to the next. This metric reflects the stability of prediction results, which directly impacts the user's visual experience. Let $\sigma_{i,n}^{(x,y,z)}$ denote the semantic label of the $n$-th voxel point (with coordinates $(x,y,z)$) in frame $i$, and define the indicator function $\delta(e_1,e_2) = \mathbb{I}(e_1 \neq e_2)$. 

In the occupancy results of frames $i$ and $j$, voxels at corresponding positions may undergo changes, which are categorized into two types: ``Static Object Change"~(SOC) and ``Moving Object Change"~(MOC). The definitions of these changes are as table \ref{tab:moc-soc}.

\begin{wrapfigure}[7]{l}{80mm}
\centering
% \setlength{\tabcolsep}{8pt}
\captionsetup{type=table}
    \begin{tabular}{c|c}
    \toprule
    \textbf{Type} & \textbf{Condition} \\  
    \midrule
    MOC & $\sigma_{i,n}^{(x,y,z)} \in \text{GMO} \lor \sigma_{j,n}^{(x,y,z)} \in \text{GMO}$ \\
    % \midrule
    SOC & $\sigma_{i,n}^{(x,y,z)} \wedge \sigma_{j,n}^{(x,y,z)} \in \text{GSO}$ \\  
    \bottomrule
    \end{tabular}
    \vspace{8pt}
    \caption{Definition of MOC and SOC. $N_{mc}$/$N_{sc}$ denote the number of MOC/SOC voxels, respectively.}
    \label{tab:moc-soc}
    % \vspace{-10pt}
\end{wrapfigure}

Based on these definitions, we can define disparity metrics~($\Delta_{m}$/$\Delta_{s}$) to quantify temporal inconsistencies across frames~($i$ and $j$). The process is defined as:
\begin{equation}
\begin{dcases}
\Delta_{m}(i,j) = \dfrac{1}{N_{mc}} \sum\limits_{n=1}^{N_{mc}} \delta\left(\sigma_{i,n}^{(x,y,z)}, \sigma_{j,n}^{(x,y,z)}\right) \\
\Delta_{s}(i,j) = \dfrac{1}{N_{sc}} \sum\limits_{n=1}^{N_{sc}} \delta\left(\sigma_{i,n}^{(x,y,z)}, \sigma_{j,n}^{(x,y,z)}\right).
\end{dcases}
\label{eq:disparity}
\end{equation}

The temporal consistency metrics -- $S_m$ (moving) and $S_s$ (static) -- are derived through aggregation of $\Delta_{m}$ and $\Delta_{s}$ across sequential frames. Formally, we have:
\begin{equation}
S_{m/s} = 1 - \dfrac{1}{M-1} \sum\limits_{k=1}^{M-1} \Delta_{m/s}(k,k+1),
\label{eq:consistency_scores}
\end{equation}
where $M$ is the scene's total frame count. Final metrics $\overline{S_m}$/$\overline{S_s}$ average across all scenes. A higher temporal consistency score indicates that the predictions within the scene are smoother and more consistent over time.

\begin{table}[t]
    % \centering
    \footnotesize 
    \begin{minipage}[t]{0.48\textwidth}
        \centering
        % \captionsetup{justification=centering, singlelinecheck=false}
        \setlength{\tabcolsep}{2pt}
        \renewcommand{\arraystretch}{1.25}
        \begin{tabular}{r|cccc|cc}
            \toprule
            \multicolumn{1}{r|}{\multirow{2}{*}[-0.4em]{Method}} & \multicolumn{1}{c|}{\multirow{2}{*}[-0.4em]{IoU~$\uparrow$}} & \multicolumn{3}{c|}{mIoU~$\uparrow$} & \multicolumn{1}{c}{\multirow{2}{*}[-0.4em]{$\overline{S_m}\uparrow$}} & \multicolumn{1}{c}{\multirow{2}{*}[-0.4em]{$\overline{S_s}\uparrow$}} \\ \cmidrule(lr){3-5}
            \multicolumn{1}{c|}{} & \multicolumn{1}{c|}{} & All & GMO & GSO & \multicolumn{1}{c}{} & \multicolumn{1}{c}{} \\ 
            \midrule        
            Atlas~\cite{Atlas} & 28.66 & 15.00 & 12.64 & 17.35 & \text{--} & \text{--}  \\
            BEVFormer{~\cite{BEVFormer}} & 30.50 & 16.75 & 14.17 & 19.33 & \text{--} & \text{--}  \\
            TPVFormer~\cite{TPVFormer} & 30.86 & 17.10 & 14.04 & 20.15 & \text{--} & \text{--} \\
            BEVDet4D-Occ~\cite{bevdet4d} & 24.26 & 14.22 & 11.10 & 17.34 & \text{--} & \text{--} \\
            MonoScene~\cite{MonoScene} & 10.04 & 1.15 & 0.24 & 2.07 & 46.53 & 81.77 \\
            % Cam4DOcc~\cite{Cam4docc} & 23.92 & 7.12 & 4.71 & 10.17 & 60.34 & 91.15 \\
            SurroundOcc~\cite{surroundOcc} & 31.49 & 20.30  & \cellcolor{gray!20}18.39 & 22.20 & 58.33 & 91.71 \\
            \midrule
            \makecell[r]{MonoScene \\ \textbf{+\ours}} & \makecell{13.10\\\textbf{+3.06}} & \makecell{1.69\\\textbf{+0.54}}  & \makecell{0.34\\\textbf{+0.10}} & \makecell{3.04\\\textbf{+0.98}} & \makecell{54.21\\\textbf{+7.68}} & \makecell{83.84\\\textbf{+2.07}} \\
            \midrule
            \makecell[r]{SurroundOcc \\ \textbf{+\ours}} & \cellcolor{gray!20}\makecell{33.12 \\ \textbf{+1.63}} & \cellcolor{gray!20}\makecell{20.67\\\textbf{+0.37}}  & \makecell{18.26\\-0.13} & \cellcolor{gray!20}\makecell{23.08\\\textbf{+0.88}} & \cellcolor{gray!20}\makecell{60.64\\\textbf{+2.31}} & \cellcolor{gray!20}\makecell{92.54\\\textbf{+0.83}} \\
            \bottomrule
        \end{tabular}
        \vspace{2mm}
        \caption{Occupancy prediction accuracy on \textbf{nuScenes benchmark~\cite{nuScenes}}. For a fair comparison, we ensure that all models have uniform input data. The best performance is highlighted in gray.}
        \label{tab:main-res-a}
    \end{minipage}\hfill
    \begin{minipage}[t]{0.48\textwidth}
        \centering
        % \captionsetup{justification=centering, singlelinecheck=false}
        \setlength{\tabcolsep}{2pt}
        \begin{tabular}{r|cccc|cc}
            \toprule
            \multicolumn{1}{r|}{\multirow{2}{*}[-0.4em]{Method}} & \multicolumn{1}{c|}{\multirow{2}{*}[-0.4em]{IoU~$\uparrow$}} & \multicolumn{3}{c|}{mIoU~$\uparrow$} & \multicolumn{1}{c}{\multirow{2}{*}[-0.4em]{$\overline{S_m}\uparrow$}} & \multicolumn{1}{c}{\multirow{2}{*}[-0.4em]{$\overline{S_s}\uparrow$}} \\ \cmidrule(lr){3-5}
            \multicolumn{1}{c|}{} & \multicolumn{1}{c|}{} & All & GMO & GSO & \multicolumn{1}{c}{} & \multicolumn{1}{c}{} \\ 
            \midrule    
            MonoScene~\cite{MonoScene} & \text{--} & 6.06 & 5.36 & 6.68 & \text{--} & \text{--} \\
            OccFormer~\cite{OccFormer} & \text{--} & 21.93 & 21.78 & 22.06 & \text{--} & \text{--} \\
            % CTF-Occ~\cite{Occ3D} & 28.53 & 27.42 & 29.52 & \text{--} & \text{--} \\
            FB-OCC~\cite{fb_occ} & \text{--} & 39.11  & 33.74 & 43.88 & \text{--} & \text{--} \\
            SparseOcc~\cite{SparseOcc_Liu} & \text{--} & 30.10  & \text{--} & \text{--} & \text{--} & \text{--} \\
            BEVDet4D-Occ~\cite{bevdet4d} & \text{--} & 39.30  & 29.09 & 42.16 & \text{--} & \text{--} \\ 
            OPUS-L~\cite{opus} & \text{--} & 36.20  & 31.25 & 40.44 & \text{--} & \text{--} \\      
            SurroundOcc~\cite{surroundOcc} & 51.89 & 7.24  & 0.36 & 13.35 & 65.35 & 89.54 \\
            ViewFormer~\cite{viewformer} & 70.39 & 40.46  & 33.73 & 46.45 & 67.26 & 86.06 \\
            \midrule
            \makecell[r]{SurroundOcc \\ \textbf{+\ours}} & \makecell{52.13\\\textbf{+0.24}}& \makecell{10.33\\\textbf{+3.09}}  & \makecell{1.98\\\textbf{+1.62}} & \makecell{17.76\\\textbf{+4.41}} & \makecell{69.60\\\textbf{+4.25}} & \cellcolor{gray!20}\makecell{90.91\\\textbf{+1.37}} \\
            \midrule
            \makecell[r]{ViewFormer \\ \textbf{+\ours}} & \cellcolor{gray!20}\makecell{70.63\\\textbf{+0.24}} & \cellcolor{gray!20}\makecell{41.30\\\textbf{+0.84}}  & \cellcolor{gray!20}\makecell{34.33\\\textbf{+0.60}} & \cellcolor{gray!20}\makecell{47.50\\\textbf{+1.05}} & \cellcolor{gray!20}\makecell{70.13\\\textbf{+2.87}} & \makecell{87.10\\\textbf{+1.04}} \\
            \bottomrule
        \end{tabular}
        \vspace{2mm}
        \caption{Occupancy prediction accuracy on \textbf{Occ3D benchmark~\cite{Occ3D}}. For a fair comparison, we ensure that all models have uniform input data. The best performance is highlighted in gray.}
        \label{tab:main-res-b}
    \end{minipage}
    \vspace{-3mm}
    % \caption{Occupancy prediction accuracy on two benchmarks. The best performance is highlighted in gray.}
    \label{tab:main-res}
    \vspace{-5mm}
\end{table}

\subsection{Comparison Results}

\noindent\textbf{Occupancy accuracy on nuScenes.} We compare our method against several SOTA models, including Atlas~\cite{Atlas}, BEVFormer~\cite{BEVFormer}, TPVFormer~\cite{TPVFormer}, MonoScene~\cite{MonoScene}, and SurroundOcc~\cite{surroundOcc}. For a fair comparison, all methods are trained on the same ground truth and follow the same training procedure. By combining methods such as MonoScene~\cite{MonoScene} and SurroundOcc~\cite{surroundOcc} with \ours, we evaluate the effect of \ours\ in performance enhancement. The results presented in \cref{tab:main-res-a} show that our performance improvement is significant. Notably, the incorporation of \ours\ into SurroundOcc~\cite{surroundOcc} has led to improved metrics that surpass those of all other models listed in this table. The results are improved by 1.63\% and 0.37\% compared with SurroundOcc~\cite{surroundOcc} in IoU and mIoU~(All), respectively.

\noindent\textbf{Occupancy accuracy on Occ3D.} We also conduct experiments on Occ3D~\cite{Occ3D} in \cref{tab:main-res-b}. To validate \ours, we conducted two sets of experiments: First, integrating \ours\ with the 3D VONs~\cite{surroundOcc,viewformer} improved one of the original models'~\cite{viewformer} performance by 0.24\% in IoU and 0.84\% in mIoU. Second, \ours\ consistently outperforms existing history-aware VONs~\cite{opus,bevdet,SparseOcc_Liu,fb_occ} by over 2\% mIoU, demonstrating the efficacy of the \ours.


\noindent\textbf{Temporal Consistency.} The results of $\overline{S_m}$ and $\overline{S_s}$ shown in \cref{tab:main-res} indicate that the integration of \ours\ improved the temporal consistency of occupancy across all frames in all scenes for all models, demonstrating \ours' effectiveness. This enhancement can be attributed to the incorporation of previous keyframes from the dataset~\cite{nuScenes,Occ3D}, along with the addition of intermediate frames from the ``sweeps''~\cite{nuScenes} directory for the SFE and MFE modules. These elements provide critical historical information and motion clues for the model.

\subsection{Ablation study}

Our ablation experiments are all conducted on the nuScenes benchmark~\cite{nuScenes}. The results are presented in~\cref{tab:ablation-studies}.

\begin{wrapfigure}{l}{90mm}
\centering
\captionsetup{type=table}
    \begin{subtable}[t]{0.48\textwidth}
    \centering
    \footnotesize
    \begin{tabular}{c|ccc|cccc}
    \toprule 
    Idx. & Pre & Cur & Mid & IoU$\uparrow$ & mIoU$\uparrow$ & $\overline{S_m}\uparrow$ & $\overline{S_s}\uparrow$ \\
    \midrule
    \textbf{M0} & \ding{55} & \ding{55} & \ding{55} & 31.49 & 20.30 & 58.33 & 91.71 \\
    \textbf{M1} & \ding{55} & \ding{51} & \ding{51} & 33.04 & 20.04 & 60.59 & 92.25 \\
    \textbf{M2} & \ding{51} & \ding{55} & \ding{51} & 33.05 & 19.98 & 60.09 & 92.44 \\
    \textbf{M3} & \ding{51} & \ding{51} & \ding{55} & 32.88 & 20.10 & 60.24 & 92.24 \\
    \textbf{M4} & \ding{51} & \ding{51} & \ding{51} & 31.97 & 20.11 & 60.19 & 92.01 \\
    \textbf{M5} & \ding{51} & \ding{51} & \ding{51} & \textbf{33.12} & \textbf{20.67} & \textbf{60.64} & \textbf{92.54} \\
    \bottomrule
    \end{tabular}
    \vspace{1mm}
    \caption{Ablation study of \ours. \textbf{Cur}, \textbf{Pre} and \textbf{Mid} represent the $f_{cur}$, $f_{pre}$ and $f_{mid}$ input, respectively, in the MSI.}
    \label{tab:ablation-modules}
    \end{subtable}
% \hfill
    \vspace{2mm}
    
    \begin{subtable}[t]{0.48\textwidth}
    \centering
    \footnotesize
    \setlength{\tabcolsep}{7pt}
    \begin{tabular}{c|c|cccc}
    \toprule 
    Idx. & \makecell{Type of\\motion info.} & IoU~$\uparrow$ & mIoU~$\uparrow$ & $\overline{S_m}\uparrow$ & $\overline{S_s}\uparrow$ \\
    \midrule
    \textbf{I0} & -  & 31.49 & 20.30 & 58.33 & 91.71 \\
    \textbf{I1} & Raw Image  & 32.39 & 19.45 & 59.01 & 91.15 \\
    \textbf{I2} & Optical Flow  & 32.80 & 20.27 & 60.53 & 92.13 \\
    \textbf{I3} & Frame Diff.  & \textbf{33.12} & \textbf{20.67}  & \textbf{60.64} & \textbf{92.54} \\
    \bottomrule
    \end{tabular}
    \vspace{1mm}
    \caption{Effect of different types of motion information.}
    \label{tab:ablation-motion}
    \end{subtable}
    % \vspace{-2mm}
\caption{Ablation studies on \ours\ modules and motion information. Best results are \textbf{bolded}.}
\label{tab:ablation-studies}
\end{wrapfigure}

\noindent\textbf{Different combinations of \ours.} \label{para:aba-comb}\cref{tab:ablation-modules} presents the performance results of different combination of \ours's components for $N$=$1$. In \cref{tab:ablation-modules}, there are 6 different combinations: \textbf{M0} shows results from SurroundOcc~\cite{surroundOcc}, which represents the basic model without our method. \textbf{M1} means the model variant in which the part responsible for processing previous keyframes is removed, thereby excluding the input data \(F_{pre}^{1}\). \textbf{M2} refers to the model variant that omits the current feature \(F_{cur}\). \textbf{M3} indicates the model configuration that has \(F_{mid}^{1}\) removed. \textbf{M4} indicates that $I_{cur}$ is used to compute MHAM's query, while $I_{pre}$ and $I_{mid}$ are utilized to compute MHAM's key and value, which differs from the standard design. \textbf{M5} represents the full model with all components included.
% 验证了我们三种输入的必要性
\cref{tab:ablation-modules} clearly demonstrates that the removal of any single input from \ours\ module significantly reduces performance both in prediction accuracy and in temporal consistency. This validates the necessity of the three inputs. Furthermore, the comparison between \textbf{M4} and \textbf{M5} confirms that the cues provided by the previous keyframes and the intermediate frames are crucial for occupancy prediction.

\noindent\textbf{Impact of different types of motion information.} \label{para:motion-extracting} This experiment was conducted on the MFE module to investigate the effects of various types of motion information for $N=1$. The results are presented in \cref{tab:ablation-motion}. Specifically, \textbf{I0} served as the base model~\cite{surroundOcc} without using any motion information. \textbf{I1} employed raw intermediate frames as the input for the MFE. \textbf{I2} used optical flow~\cite{OpticalFlow} as the motion information input. \textbf{I3} used frame difference~\cite{Frame_difference} to capture motion information. It is clear that \textbf{I1} surpasses \textbf{I0} in terms of IoU metrics; however, it exhibits the lowest performance in mIoU, $\overline{S_m}$, and $\overline{S_s}$ metrics compared with \textbf{I1}, \textbf{I2}, and \textbf{I3}. This discrepancy is mainly because of the substantial amount of irrelevant information in the raw, intermediate frames, which complicates the extraction of motion features by the MFE. In addition, the results show that \textbf{I3} significantly outperforms \textbf{I2} in both IoU and mIoU metrics and slightly improves in $\overline{S_m}$ and $\overline{S_s}$ metrics. This indicates that frame difference more effectively captures sudden changes in a scene, such as the abrupt appearance of pedestrians or vehicles exiting intersections, while optical flow may experience delays in processing these sudden events. Furthermore, given the lightweight design of \ours, the frame difference method~\cite{Frame_difference} reduces data processing complexity by only processing simple differential data, thereby contributing to computing speed.

\begin{wrapfigure}[22]{l}{90mm}
    \centering
    \includegraphics[width=90mm]{assets/case_single_light.png}
    \caption{
    Several challenging scenarios are presented: pedestrians are partially occluded by vehicles in the first and second columns, and the road boundary appears visually obscure in the third column. Ours achieves more accurate predictions, while SOTA methods display significant artifacts.
    }
    \label{fig:case-extra}
\end{wrapfigure}

\noindent\textbf{Impact of different numbers of previous keyframes.}\label{para:ntrack} We conduct ablation experiments on $N$ to explore the performance of the model when $N$=$0$, $N$=$1$ and $N$=$2$. $N$=$0$ represents SurroundOcc~\cite{surroundOcc}, which does not use any previous keyframes. Detailed experiment results are documented in the supplementary material.

\begin{figure}[!t]
\centering
% \fbox{\rule{0pt}{2in} \rule{0.9\linewidth}{0pt}}
\includegraphics[width=\linewidth]{assets/case_big_light.png}
% \vspace{-7mm}
\caption{
Comparison under a T-junction scenario, where a pedestrian is partially and dynamically occluded in certain frames. Ours showcases robust predictions, with the pedestrian being consistently tracked, while SOTA methods show a flickering phenomenon.
}
\label{fig:case-study-big}
\vspace{-20pt}
\end{figure}

\subsection{Case analysis}
To visually evaluate the effectiveness of our method~(SurroundOcc+\ours), we compare it with the SOTA 3D VONs~\cite{surroundOcc} and the SOTA history-aware VONs~\cite{bevdet4d}.

\noindent\textbf{Temporal visualization case.} As shown in~\cref{fig:case-study-big} (Scene 277, Frames \#7-\#11), a pedestrian traversing the sidewalk parallel to the ego-motion trajectory is intermittently occluded by roadside vegetation. SurroundOcc~\cite{surroundOcc} exhibits severe instability in predictions (missing in Frames \#7/\#9), revealing fundamental limitations in temporal modeling. BEVDet4D-Occ~\cite{bevdet4d} alleviates this issue through data fusion but still suffers from occasional inconsistencies, such as detection dropout in Frame \#8. In contrast, our method completely eliminates flickering artifacts and maintains consistent detection across all occlusion states.

\noindent\textbf{Extra single frame visualization case.} ~\cref{fig:case-extra} highlights challenging scenarios: 
(i) Vehicle-pedestrian occlusion (Scene-0911 Frame \#15, Scene-0928 Frame \#14): Both SurroundOcc~\cite{surroundOcc} and BEVDet4D-Occ~\cite{bevdet4d} fail to recover the occluded pedestrian’s occupancy, while our method successfully localizes the target with precise geometry.
(ii) Curved road prediction (Scene-0923 Frame \#28): Our approach correctly anticipates the right-turn road geometry where baselines produce fragmented or erroneous occupancy, achieving superior shape consistency with real-world conditions.




\subsection{Overhead analysis}

For a fair comparison, all overhead analysis experiments are performed on a single NVIDIA L20 GPU.

\begin{figure}[htbp]
    \centering
    \begin{minipage}{0.48\textwidth}
        \footnotesize
        \begin{tabular}{r|ccc}
            \toprule
            Model & mIoU~$\uparrow$ & \makecell{Memory (MB)\\ Train~/~Test}~$\downarrow$ & Latency~$\downarrow$ \\
            \midrule
            FB-Occ~\cite{fb_occ} & 39.11 & 32,915~/~5,933 & 0.09s \\
            % SparseOcc~\cite{SparseOcc_Liu} & 30.10 & $>$49,140~/~7,147 & \cellcolor{Gray} 0.05s \\
            OPUS-L~\cite{opus} & 36.20 & OOM~/~10,579 & 0.16s \\
            OPUS-T~\cite{opus} & 33.20 & 48,532~/~6,711 & \textbf{0.03s} \\
            BEVDet4D-Occ~\cite{bevdet4d} & 39.30 & 22,833~/~4,689 & 0.26s \\
            \midrule
            ViewFormer+Ours & \textbf{41.30} & \textbf{16,619~/~4,687} & 0.12s \\
            \bottomrule
        \end{tabular}
        \vspace{2mm}
        \caption{Comparison of computational overhead. All models are benchmarked with ResNet-50 backbones. Our result (ViewFormer+\ours) in this table is measured for $N = 1$. OOM indicates out of CUDA memory. Best results are \textbf{bolded}.}
        \label{tab:efficiency}
    \end{minipage}\hfill
    \begin{minipage}{0.48\textwidth}
        \centering
        \includegraphics[width=\linewidth]{assets/bubble_0304.png}  % 替换成你的图片
        % \vspace{-8mm}
        \caption{Comparison of memory and latency overheads. Lower-left positions indicate superior performance with reduced memory consumption and faster inference. Large circles indicate better mIoU quality.}
        \label{fig:bubble}
    \end{minipage}
\end{figure}



As illustrated in \cref{tab:efficiency} and \cref{fig:bubble}, we conducted a comparative study to evaluate the computational overhead of our model against existing temporal methods~\cite{bevdet4d,opus,fb_occ}. The analysis focuses on GPU memory consumption during the training/testing phases and per-sample inference latency. The result shows that our method establishes an optimal accuracy-memory balance, achieving state-of-the-art mIoU while maintaining minimal GPU memory consumption alongside sustained computational efficiency that avoids runtime bottlenecks. For quantitative benchmarking, we compare two baseline frameworks:
\begin{itemize}
    \item ViewFormer on Occ3D: (i) Training memory: ViewFormer+\ours\ requires 16 GB of GPU memory, with the \ours\ module consuming only 0.22 GB, accounting for \textbf{1.4\%} of total usage; (ii) Inference latency: Full sample processing takes 0.1218s, where \ours\ contributes merely 0.0043s, accounting for \textbf{3.5\%} of total computation.
    \item SurroundOcc on nuScenes: (i) Training memory: SurroundOcc+\ours\ consumes 39 GB of GPU memory, with \ours\ occupying only 0.69 GB, which is \textbf{1.8\%} of total memory; (ii) Inference latency: Complete sample inference requires 0.9200s, while \ours\ takes 0.0065s, contributing to \textbf{0.7\%} of total latency.
\end{itemize}

These measurements confirm that our architecture introduces negligible computational overhead while delivering competitive performance.

\section{Conclusion}
In this paper, we present \sys{}, the first-of-its-kind ML/Crypto codesign secure watermarking framework with enhanced and balanced detectability-fidelity-robustness. We train the watermark insertion and extraction modules end-to-end, aiming to ensure the watermarked codes' functionality-invariant, while maintaining the detectability of the watermark in the adversarial environment. We also design a zero-knowledge proof-based watermark verification in the system deployment to ensure correct ownership proofs without disclosing the signature details. Extensive evaluations of various coding benchmarks demonstrated the effectiveness of our proposed approach. 


% We recommend future work in exploring \sys's generalizability to watermarking other coding languages. We also recommend exploring adaptive watermarking verification approaches to verify watermarked code in large engineering projects. 

\section*{Impact Statement}

Our code watermarking framework has potential societal implications.  
By building secure and public-verifiable code watermarking framework, our approach can enhance the usability of existing watermarking systems, especially for low-entropy code data. As such, it helps to detect code plagiarism for academic dishonesty, protect the intelligent property of the LLM owners, and monitor the distribution of the watermarked content. However, there might be cases where human-written code can be erroneously detected as LLM-generated, leading to false accusations. 

% In the unusual situation where you want a paper to appear in the
% references without citing it in the main text, use \nocite
\nocite{langley00}

\bibliography{references,example_paper}
\bibliographystyle{icml2025}


%%%%%%%%%%%%%%%%%%%%%%%%%%%%%%%%%%%%%%%%%%%%%%%%%%%%%%%%%%%%%%%%%%%%%%%%%%%%%%%
%%%%%%%%%%%%%%%%%%%%%%%%%%%%%%%%%%%%%%%%%%%%%%%%%%%%%%%%%%%%%%%%%%%%%%%%%%%%%%%
% APPENDIX
%%%%%%%%%%%%%%%%%%%%%%%%%%%%%%%%%%%%%%%%%%%%%%%%%%%%%%%%%%%%%%%%%%%%%%%%%%%%%%%
%%%%%%%%%%%%%%%%%%%%%%%%%%%%%%%%%%%%%%%%%%%%%%%%%%%%%%%%%%%%%%%%%%%%%%%%%%%%%%%

\subsection{Lloyd-Max Algorithm}
\label{subsec:Lloyd-Max}
For a given quantization bitwidth $B$ and an operand $\bm{X}$, the Lloyd-Max algorithm finds $2^B$ quantization levels $\{\hat{x}_i\}_{i=1}^{2^B}$ such that quantizing $\bm{X}$ by rounding each scalar in $\bm{X}$ to the nearest quantization level minimizes the quantization MSE. 

The algorithm starts with an initial guess of quantization levels and then iteratively computes quantization thresholds $\{\tau_i\}_{i=1}^{2^B-1}$ and updates quantization levels $\{\hat{x}_i\}_{i=1}^{2^B}$. Specifically, at iteration $n$, thresholds are set to the midpoints of the previous iteration's levels:
\begin{align*}
    \tau_i^{(n)}=\frac{\hat{x}_i^{(n-1)}+\hat{x}_{i+1}^{(n-1)}}2 \text{ for } i=1\ldots 2^B-1
\end{align*}
Subsequently, the quantization levels are re-computed as conditional means of the data regions defined by the new thresholds:
\begin{align*}
    \hat{x}_i^{(n)}=\mathbb{E}\left[ \bm{X} \big| \bm{X}\in [\tau_{i-1}^{(n)},\tau_i^{(n)}] \right] \text{ for } i=1\ldots 2^B
\end{align*}
where to satisfy boundary conditions we have $\tau_0=-\infty$ and $\tau_{2^B}=\infty$. The algorithm iterates the above steps until convergence.

Figure \ref{fig:lm_quant} compares the quantization levels of a $7$-bit floating point (E3M3) quantizer (left) to a $7$-bit Lloyd-Max quantizer (right) when quantizing a layer of weights from the GPT3-126M model at a per-tensor granularity. As shown, the Lloyd-Max quantizer achieves substantially lower quantization MSE. Further, Table \ref{tab:FP7_vs_LM7} shows the superior perplexity achieved by Lloyd-Max quantizers for bitwidths of $7$, $6$ and $5$. The difference between the quantizers is clear at 5 bits, where per-tensor FP quantization incurs a drastic and unacceptable increase in perplexity, while Lloyd-Max quantization incurs a much smaller increase. Nevertheless, we note that even the optimal Lloyd-Max quantizer incurs a notable ($\sim 1.5$) increase in perplexity due to the coarse granularity of quantization. 

\begin{figure}[h]
  \centering
  \includegraphics[width=0.7\linewidth]{sections/figures/LM7_FP7.pdf}
  \caption{\small Quantization levels and the corresponding quantization MSE of Floating Point (left) vs Lloyd-Max (right) Quantizers for a layer of weights in the GPT3-126M model.}
  \label{fig:lm_quant}
\end{figure}

\begin{table}[h]\scriptsize
\begin{center}
\caption{\label{tab:FP7_vs_LM7} \small Comparing perplexity (lower is better) achieved by floating point quantizers and Lloyd-Max quantizers on a GPT3-126M model for the Wikitext-103 dataset.}
\begin{tabular}{c|cc|c}
\hline
 \multirow{2}{*}{\textbf{Bitwidth}} & \multicolumn{2}{|c|}{\textbf{Floating-Point Quantizer}} & \textbf{Lloyd-Max Quantizer} \\
 & Best Format & Wikitext-103 Perplexity & Wikitext-103 Perplexity \\
\hline
7 & E3M3 & 18.32 & 18.27 \\
6 & E3M2 & 19.07 & 18.51 \\
5 & E4M0 & 43.89 & 19.71 \\
\hline
\end{tabular}
\end{center}
\end{table}

\subsection{Proof of Local Optimality of LO-BCQ}
\label{subsec:lobcq_opt_proof}
For a given block $\bm{b}_j$, the quantization MSE during LO-BCQ can be empirically evaluated as $\frac{1}{L_b}\lVert \bm{b}_j- \bm{\hat{b}}_j\rVert^2_2$ where $\bm{\hat{b}}_j$ is computed from equation (\ref{eq:clustered_quantization_definition}) as $C_{f(\bm{b}_j)}(\bm{b}_j)$. Further, for a given block cluster $\mathcal{B}_i$, we compute the quantization MSE as $\frac{1}{|\mathcal{B}_{i}|}\sum_{\bm{b} \in \mathcal{B}_{i}} \frac{1}{L_b}\lVert \bm{b}- C_i^{(n)}(\bm{b})\rVert^2_2$. Therefore, at the end of iteration $n$, we evaluate the overall quantization MSE $J^{(n)}$ for a given operand $\bm{X}$ composed of $N_c$ block clusters as:
\begin{align*}
    \label{eq:mse_iter_n}
    J^{(n)} = \frac{1}{N_c} \sum_{i=1}^{N_c} \frac{1}{|\mathcal{B}_{i}^{(n)}|}\sum_{\bm{v} \in \mathcal{B}_{i}^{(n)}} \frac{1}{L_b}\lVert \bm{b}- B_i^{(n)}(\bm{b})\rVert^2_2
\end{align*}

At the end of iteration $n$, the codebooks are updated from $\mathcal{C}^{(n-1)}$ to $\mathcal{C}^{(n)}$. However, the mapping of a given vector $\bm{b}_j$ to quantizers $\mathcal{C}^{(n)}$ remains as  $f^{(n)}(\bm{b}_j)$. At the next iteration, during the vector clustering step, $f^{(n+1)}(\bm{b}_j)$ finds new mapping of $\bm{b}_j$ to updated codebooks $\mathcal{C}^{(n)}$ such that the quantization MSE over the candidate codebooks is minimized. Therefore, we obtain the following result for $\bm{b}_j$:
\begin{align*}
\frac{1}{L_b}\lVert \bm{b}_j - C_{f^{(n+1)}(\bm{b}_j)}^{(n)}(\bm{b}_j)\rVert^2_2 \le \frac{1}{L_b}\lVert \bm{b}_j - C_{f^{(n)}(\bm{b}_j)}^{(n)}(\bm{b}_j)\rVert^2_2
\end{align*}

That is, quantizing $\bm{b}_j$ at the end of the block clustering step of iteration $n+1$ results in lower quantization MSE compared to quantizing at the end of iteration $n$. Since this is true for all $\bm{b} \in \bm{X}$, we assert the following:
\begin{equation}
\begin{split}
\label{eq:mse_ineq_1}
    \tilde{J}^{(n+1)} &= \frac{1}{N_c} \sum_{i=1}^{N_c} \frac{1}{|\mathcal{B}_{i}^{(n+1)}|}\sum_{\bm{b} \in \mathcal{B}_{i}^{(n+1)}} \frac{1}{L_b}\lVert \bm{b} - C_i^{(n)}(b)\rVert^2_2 \le J^{(n)}
\end{split}
\end{equation}
where $\tilde{J}^{(n+1)}$ is the the quantization MSE after the vector clustering step at iteration $n+1$.

Next, during the codebook update step (\ref{eq:quantizers_update}) at iteration $n+1$, the per-cluster codebooks $\mathcal{C}^{(n)}$ are updated to $\mathcal{C}^{(n+1)}$ by invoking the Lloyd-Max algorithm \citep{Lloyd}. We know that for any given value distribution, the Lloyd-Max algorithm minimizes the quantization MSE. Therefore, for a given vector cluster $\mathcal{B}_i$ we obtain the following result:

\begin{equation}
    \frac{1}{|\mathcal{B}_{i}^{(n+1)}|}\sum_{\bm{b} \in \mathcal{B}_{i}^{(n+1)}} \frac{1}{L_b}\lVert \bm{b}- C_i^{(n+1)}(\bm{b})\rVert^2_2 \le \frac{1}{|\mathcal{B}_{i}^{(n+1)}|}\sum_{\bm{b} \in \mathcal{B}_{i}^{(n+1)}} \frac{1}{L_b}\lVert \bm{b}- C_i^{(n)}(\bm{b})\rVert^2_2
\end{equation}

The above equation states that quantizing the given block cluster $\mathcal{B}_i$ after updating the associated codebook from $C_i^{(n)}$ to $C_i^{(n+1)}$ results in lower quantization MSE. Since this is true for all the block clusters, we derive the following result: 
\begin{equation}
\begin{split}
\label{eq:mse_ineq_2}
     J^{(n+1)} &= \frac{1}{N_c} \sum_{i=1}^{N_c} \frac{1}{|\mathcal{B}_{i}^{(n+1)}|}\sum_{\bm{b} \in \mathcal{B}_{i}^{(n+1)}} \frac{1}{L_b}\lVert \bm{b}- C_i^{(n+1)}(\bm{b})\rVert^2_2  \le \tilde{J}^{(n+1)}   
\end{split}
\end{equation}

Following (\ref{eq:mse_ineq_1}) and (\ref{eq:mse_ineq_2}), we find that the quantization MSE is non-increasing for each iteration, that is, $J^{(1)} \ge J^{(2)} \ge J^{(3)} \ge \ldots \ge J^{(M)}$ where $M$ is the maximum number of iterations. 
%Therefore, we can say that if the algorithm converges, then it must be that it has converged to a local minimum. 
\hfill $\blacksquare$


\begin{figure}
    \begin{center}
    \includegraphics[width=0.5\textwidth]{sections//figures/mse_vs_iter.pdf}
    \end{center}
    \caption{\small NMSE vs iterations during LO-BCQ compared to other block quantization proposals}
    \label{fig:nmse_vs_iter}
\end{figure}

Figure \ref{fig:nmse_vs_iter} shows the empirical convergence of LO-BCQ across several block lengths and number of codebooks. Also, the MSE achieved by LO-BCQ is compared to baselines such as MXFP and VSQ. As shown, LO-BCQ converges to a lower MSE than the baselines. Further, we achieve better convergence for larger number of codebooks ($N_c$) and for a smaller block length ($L_b$), both of which increase the bitwidth of BCQ (see Eq \ref{eq:bitwidth_bcq}).


\subsection{Additional Accuracy Results}
%Table \ref{tab:lobcq_config} lists the various LOBCQ configurations and their corresponding bitwidths.
\begin{table}
\setlength{\tabcolsep}{4.75pt}
\begin{center}
\caption{\label{tab:lobcq_config} Various LO-BCQ configurations and their bitwidths.}
\begin{tabular}{|c||c|c|c|c||c|c||c|} 
\hline
 & \multicolumn{4}{|c||}{$L_b=8$} & \multicolumn{2}{|c||}{$L_b=4$} & $L_b=2$ \\
 \hline
 \backslashbox{$L_A$\kern-1em}{\kern-1em$N_c$} & 2 & 4 & 8 & 16 & 2 & 4 & 2 \\
 \hline
 64 & 4.25 & 4.375 & 4.5 & 4.625 & 4.375 & 4.625 & 4.625\\
 \hline
 32 & 4.375 & 4.5 & 4.625& 4.75 & 4.5 & 4.75 & 4.75 \\
 \hline
 16 & 4.625 & 4.75& 4.875 & 5 & 4.75 & 5 & 5 \\
 \hline
\end{tabular}
\end{center}
\end{table}

%\subsection{Perplexity achieved by various LO-BCQ configurations on Wikitext-103 dataset}

\begin{table} \centering
\begin{tabular}{|c||c|c|c|c||c|c||c|} 
\hline
 $L_b \rightarrow$& \multicolumn{4}{c||}{8} & \multicolumn{2}{c||}{4} & 2\\
 \hline
 \backslashbox{$L_A$\kern-1em}{\kern-1em$N_c$} & 2 & 4 & 8 & 16 & 2 & 4 & 2  \\
 %$N_c \rightarrow$ & 2 & 4 & 8 & 16 & 2 & 4 & 2 \\
 \hline
 \hline
 \multicolumn{8}{c}{GPT3-1.3B (FP32 PPL = 9.98)} \\ 
 \hline
 \hline
 64 & 10.40 & 10.23 & 10.17 & 10.15 &  10.28 & 10.18 & 10.19 \\
 \hline
 32 & 10.25 & 10.20 & 10.15 & 10.12 &  10.23 & 10.17 & 10.17 \\
 \hline
 16 & 10.22 & 10.16 & 10.10 & 10.09 &  10.21 & 10.14 & 10.16 \\
 \hline
  \hline
 \multicolumn{8}{c}{GPT3-8B (FP32 PPL = 7.38)} \\ 
 \hline
 \hline
 64 & 7.61 & 7.52 & 7.48 &  7.47 &  7.55 &  7.49 & 7.50 \\
 \hline
 32 & 7.52 & 7.50 & 7.46 &  7.45 &  7.52 &  7.48 & 7.48  \\
 \hline
 16 & 7.51 & 7.48 & 7.44 &  7.44 &  7.51 &  7.49 & 7.47  \\
 \hline
\end{tabular}
\caption{\label{tab:ppl_gpt3_abalation} Wikitext-103 perplexity across GPT3-1.3B and 8B models.}
\end{table}

\begin{table} \centering
\begin{tabular}{|c||c|c|c|c||} 
\hline
 $L_b \rightarrow$& \multicolumn{4}{c||}{8}\\
 \hline
 \backslashbox{$L_A$\kern-1em}{\kern-1em$N_c$} & 2 & 4 & 8 & 16 \\
 %$N_c \rightarrow$ & 2 & 4 & 8 & 16 & 2 & 4 & 2 \\
 \hline
 \hline
 \multicolumn{5}{|c|}{Llama2-7B (FP32 PPL = 5.06)} \\ 
 \hline
 \hline
 64 & 5.31 & 5.26 & 5.19 & 5.18  \\
 \hline
 32 & 5.23 & 5.25 & 5.18 & 5.15  \\
 \hline
 16 & 5.23 & 5.19 & 5.16 & 5.14  \\
 \hline
 \multicolumn{5}{|c|}{Nemotron4-15B (FP32 PPL = 5.87)} \\ 
 \hline
 \hline
 64  & 6.3 & 6.20 & 6.13 & 6.08  \\
 \hline
 32  & 6.24 & 6.12 & 6.07 & 6.03  \\
 \hline
 16  & 6.12 & 6.14 & 6.04 & 6.02  \\
 \hline
 \multicolumn{5}{|c|}{Nemotron4-340B (FP32 PPL = 3.48)} \\ 
 \hline
 \hline
 64 & 3.67 & 3.62 & 3.60 & 3.59 \\
 \hline
 32 & 3.63 & 3.61 & 3.59 & 3.56 \\
 \hline
 16 & 3.61 & 3.58 & 3.57 & 3.55 \\
 \hline
\end{tabular}
\caption{\label{tab:ppl_llama7B_nemo15B} Wikitext-103 perplexity compared to FP32 baseline in Llama2-7B and Nemotron4-15B, 340B models}
\end{table}

%\subsection{Perplexity achieved by various LO-BCQ configurations on MMLU dataset}


\begin{table} \centering
\begin{tabular}{|c||c|c|c|c||c|c|c|c|} 
\hline
 $L_b \rightarrow$& \multicolumn{4}{c||}{8} & \multicolumn{4}{c||}{8}\\
 \hline
 \backslashbox{$L_A$\kern-1em}{\kern-1em$N_c$} & 2 & 4 & 8 & 16 & 2 & 4 & 8 & 16  \\
 %$N_c \rightarrow$ & 2 & 4 & 8 & 16 & 2 & 4 & 2 \\
 \hline
 \hline
 \multicolumn{5}{|c|}{Llama2-7B (FP32 Accuracy = 45.8\%)} & \multicolumn{4}{|c|}{Llama2-70B (FP32 Accuracy = 69.12\%)} \\ 
 \hline
 \hline
 64 & 43.9 & 43.4 & 43.9 & 44.9 & 68.07 & 68.27 & 68.17 & 68.75 \\
 \hline
 32 & 44.5 & 43.8 & 44.9 & 44.5 & 68.37 & 68.51 & 68.35 & 68.27  \\
 \hline
 16 & 43.9 & 42.7 & 44.9 & 45 & 68.12 & 68.77 & 68.31 & 68.59  \\
 \hline
 \hline
 \multicolumn{5}{|c|}{GPT3-22B (FP32 Accuracy = 38.75\%)} & \multicolumn{4}{|c|}{Nemotron4-15B (FP32 Accuracy = 64.3\%)} \\ 
 \hline
 \hline
 64 & 36.71 & 38.85 & 38.13 & 38.92 & 63.17 & 62.36 & 63.72 & 64.09 \\
 \hline
 32 & 37.95 & 38.69 & 39.45 & 38.34 & 64.05 & 62.30 & 63.8 & 64.33  \\
 \hline
 16 & 38.88 & 38.80 & 38.31 & 38.92 & 63.22 & 63.51 & 63.93 & 64.43  \\
 \hline
\end{tabular}
\caption{\label{tab:mmlu_abalation} Accuracy on MMLU dataset across GPT3-22B, Llama2-7B, 70B and Nemotron4-15B models.}
\end{table}


%\subsection{Perplexity achieved by various LO-BCQ configurations on LM evaluation harness}

\begin{table} \centering
\begin{tabular}{|c||c|c|c|c||c|c|c|c|} 
\hline
 $L_b \rightarrow$& \multicolumn{4}{c||}{8} & \multicolumn{4}{c||}{8}\\
 \hline
 \backslashbox{$L_A$\kern-1em}{\kern-1em$N_c$} & 2 & 4 & 8 & 16 & 2 & 4 & 8 & 16  \\
 %$N_c \rightarrow$ & 2 & 4 & 8 & 16 & 2 & 4 & 2 \\
 \hline
 \hline
 \multicolumn{5}{|c|}{Race (FP32 Accuracy = 37.51\%)} & \multicolumn{4}{|c|}{Boolq (FP32 Accuracy = 64.62\%)} \\ 
 \hline
 \hline
 64 & 36.94 & 37.13 & 36.27 & 37.13 & 63.73 & 62.26 & 63.49 & 63.36 \\
 \hline
 32 & 37.03 & 36.36 & 36.08 & 37.03 & 62.54 & 63.51 & 63.49 & 63.55  \\
 \hline
 16 & 37.03 & 37.03 & 36.46 & 37.03 & 61.1 & 63.79 & 63.58 & 63.33  \\
 \hline
 \hline
 \multicolumn{5}{|c|}{Winogrande (FP32 Accuracy = 58.01\%)} & \multicolumn{4}{|c|}{Piqa (FP32 Accuracy = 74.21\%)} \\ 
 \hline
 \hline
 64 & 58.17 & 57.22 & 57.85 & 58.33 & 73.01 & 73.07 & 73.07 & 72.80 \\
 \hline
 32 & 59.12 & 58.09 & 57.85 & 58.41 & 73.01 & 73.94 & 72.74 & 73.18  \\
 \hline
 16 & 57.93 & 58.88 & 57.93 & 58.56 & 73.94 & 72.80 & 73.01 & 73.94  \\
 \hline
\end{tabular}
\caption{\label{tab:mmlu_abalation} Accuracy on LM evaluation harness tasks on GPT3-1.3B model.}
\end{table}

\begin{table} \centering
\begin{tabular}{|c||c|c|c|c||c|c|c|c|} 
\hline
 $L_b \rightarrow$& \multicolumn{4}{c||}{8} & \multicolumn{4}{c||}{8}\\
 \hline
 \backslashbox{$L_A$\kern-1em}{\kern-1em$N_c$} & 2 & 4 & 8 & 16 & 2 & 4 & 8 & 16  \\
 %$N_c \rightarrow$ & 2 & 4 & 8 & 16 & 2 & 4 & 2 \\
 \hline
 \hline
 \multicolumn{5}{|c|}{Race (FP32 Accuracy = 41.34\%)} & \multicolumn{4}{|c|}{Boolq (FP32 Accuracy = 68.32\%)} \\ 
 \hline
 \hline
 64 & 40.48 & 40.10 & 39.43 & 39.90 & 69.20 & 68.41 & 69.45 & 68.56 \\
 \hline
 32 & 39.52 & 39.52 & 40.77 & 39.62 & 68.32 & 67.43 & 68.17 & 69.30  \\
 \hline
 16 & 39.81 & 39.71 & 39.90 & 40.38 & 68.10 & 66.33 & 69.51 & 69.42  \\
 \hline
 \hline
 \multicolumn{5}{|c|}{Winogrande (FP32 Accuracy = 67.88\%)} & \multicolumn{4}{|c|}{Piqa (FP32 Accuracy = 78.78\%)} \\ 
 \hline
 \hline
 64 & 66.85 & 66.61 & 67.72 & 67.88 & 77.31 & 77.42 & 77.75 & 77.64 \\
 \hline
 32 & 67.25 & 67.72 & 67.72 & 67.00 & 77.31 & 77.04 & 77.80 & 77.37  \\
 \hline
 16 & 68.11 & 68.90 & 67.88 & 67.48 & 77.37 & 78.13 & 78.13 & 77.69  \\
 \hline
\end{tabular}
\caption{\label{tab:mmlu_abalation} Accuracy on LM evaluation harness tasks on GPT3-8B model.}
\end{table}

\begin{table} \centering
\begin{tabular}{|c||c|c|c|c||c|c|c|c|} 
\hline
 $L_b \rightarrow$& \multicolumn{4}{c||}{8} & \multicolumn{4}{c||}{8}\\
 \hline
 \backslashbox{$L_A$\kern-1em}{\kern-1em$N_c$} & 2 & 4 & 8 & 16 & 2 & 4 & 8 & 16  \\
 %$N_c \rightarrow$ & 2 & 4 & 8 & 16 & 2 & 4 & 2 \\
 \hline
 \hline
 \multicolumn{5}{|c|}{Race (FP32 Accuracy = 40.67\%)} & \multicolumn{4}{|c|}{Boolq (FP32 Accuracy = 76.54\%)} \\ 
 \hline
 \hline
 64 & 40.48 & 40.10 & 39.43 & 39.90 & 75.41 & 75.11 & 77.09 & 75.66 \\
 \hline
 32 & 39.52 & 39.52 & 40.77 & 39.62 & 76.02 & 76.02 & 75.96 & 75.35  \\
 \hline
 16 & 39.81 & 39.71 & 39.90 & 40.38 & 75.05 & 73.82 & 75.72 & 76.09  \\
 \hline
 \hline
 \multicolumn{5}{|c|}{Winogrande (FP32 Accuracy = 70.64\%)} & \multicolumn{4}{|c|}{Piqa (FP32 Accuracy = 79.16\%)} \\ 
 \hline
 \hline
 64 & 69.14 & 70.17 & 70.17 & 70.56 & 78.24 & 79.00 & 78.62 & 78.73 \\
 \hline
 32 & 70.96 & 69.69 & 71.27 & 69.30 & 78.56 & 79.49 & 79.16 & 78.89  \\
 \hline
 16 & 71.03 & 69.53 & 69.69 & 70.40 & 78.13 & 79.16 & 79.00 & 79.00  \\
 \hline
\end{tabular}
\caption{\label{tab:mmlu_abalation} Accuracy on LM evaluation harness tasks on GPT3-22B model.}
\end{table}

\begin{table} \centering
\begin{tabular}{|c||c|c|c|c||c|c|c|c|} 
\hline
 $L_b \rightarrow$& \multicolumn{4}{c||}{8} & \multicolumn{4}{c||}{8}\\
 \hline
 \backslashbox{$L_A$\kern-1em}{\kern-1em$N_c$} & 2 & 4 & 8 & 16 & 2 & 4 & 8 & 16  \\
 %$N_c \rightarrow$ & 2 & 4 & 8 & 16 & 2 & 4 & 2 \\
 \hline
 \hline
 \multicolumn{5}{|c|}{Race (FP32 Accuracy = 44.4\%)} & \multicolumn{4}{|c|}{Boolq (FP32 Accuracy = 79.29\%)} \\ 
 \hline
 \hline
 64 & 42.49 & 42.51 & 42.58 & 43.45 & 77.58 & 77.37 & 77.43 & 78.1 \\
 \hline
 32 & 43.35 & 42.49 & 43.64 & 43.73 & 77.86 & 75.32 & 77.28 & 77.86  \\
 \hline
 16 & 44.21 & 44.21 & 43.64 & 42.97 & 78.65 & 77 & 76.94 & 77.98  \\
 \hline
 \hline
 \multicolumn{5}{|c|}{Winogrande (FP32 Accuracy = 69.38\%)} & \multicolumn{4}{|c|}{Piqa (FP32 Accuracy = 78.07\%)} \\ 
 \hline
 \hline
 64 & 68.9 & 68.43 & 69.77 & 68.19 & 77.09 & 76.82 & 77.09 & 77.86 \\
 \hline
 32 & 69.38 & 68.51 & 68.82 & 68.90 & 78.07 & 76.71 & 78.07 & 77.86  \\
 \hline
 16 & 69.53 & 67.09 & 69.38 & 68.90 & 77.37 & 77.8 & 77.91 & 77.69  \\
 \hline
\end{tabular}
\caption{\label{tab:mmlu_abalation} Accuracy on LM evaluation harness tasks on Llama2-7B model.}
\end{table}

\begin{table} \centering
\begin{tabular}{|c||c|c|c|c||c|c|c|c|} 
\hline
 $L_b \rightarrow$& \multicolumn{4}{c||}{8} & \multicolumn{4}{c||}{8}\\
 \hline
 \backslashbox{$L_A$\kern-1em}{\kern-1em$N_c$} & 2 & 4 & 8 & 16 & 2 & 4 & 8 & 16  \\
 %$N_c \rightarrow$ & 2 & 4 & 8 & 16 & 2 & 4 & 2 \\
 \hline
 \hline
 \multicolumn{5}{|c|}{Race (FP32 Accuracy = 48.8\%)} & \multicolumn{4}{|c|}{Boolq (FP32 Accuracy = 85.23\%)} \\ 
 \hline
 \hline
 64 & 49.00 & 49.00 & 49.28 & 48.71 & 82.82 & 84.28 & 84.03 & 84.25 \\
 \hline
 32 & 49.57 & 48.52 & 48.33 & 49.28 & 83.85 & 84.46 & 84.31 & 84.93  \\
 \hline
 16 & 49.85 & 49.09 & 49.28 & 48.99 & 85.11 & 84.46 & 84.61 & 83.94  \\
 \hline
 \hline
 \multicolumn{5}{|c|}{Winogrande (FP32 Accuracy = 79.95\%)} & \multicolumn{4}{|c|}{Piqa (FP32 Accuracy = 81.56\%)} \\ 
 \hline
 \hline
 64 & 78.77 & 78.45 & 78.37 & 79.16 & 81.45 & 80.69 & 81.45 & 81.5 \\
 \hline
 32 & 78.45 & 79.01 & 78.69 & 80.66 & 81.56 & 80.58 & 81.18 & 81.34  \\
 \hline
 16 & 79.95 & 79.56 & 79.79 & 79.72 & 81.28 & 81.66 & 81.28 & 80.96  \\
 \hline
\end{tabular}
\caption{\label{tab:mmlu_abalation} Accuracy on LM evaluation harness tasks on Llama2-70B model.}
\end{table}

%\section{MSE Studies}
%\textcolor{red}{TODO}


\subsection{Number Formats and Quantization Method}
\label{subsec:numFormats_quantMethod}
\subsubsection{Integer Format}
An $n$-bit signed integer (INT) is typically represented with a 2s-complement format \citep{yao2022zeroquant,xiao2023smoothquant,dai2021vsq}, where the most significant bit denotes the sign.

\subsubsection{Floating Point Format}
An $n$-bit signed floating point (FP) number $x$ comprises of a 1-bit sign ($x_{\mathrm{sign}}$), $B_m$-bit mantissa ($x_{\mathrm{mant}}$) and $B_e$-bit exponent ($x_{\mathrm{exp}}$) such that $B_m+B_e=n-1$. The associated constant exponent bias ($E_{\mathrm{bias}}$) is computed as $(2^{{B_e}-1}-1)$. We denote this format as $E_{B_e}M_{B_m}$.  

\subsubsection{Quantization Scheme}
\label{subsec:quant_method}
A quantization scheme dictates how a given unquantized tensor is converted to its quantized representation. We consider FP formats for the purpose of illustration. Given an unquantized tensor $\bm{X}$ and an FP format $E_{B_e}M_{B_m}$, we first, we compute the quantization scale factor $s_X$ that maps the maximum absolute value of $\bm{X}$ to the maximum quantization level of the $E_{B_e}M_{B_m}$ format as follows:
\begin{align}
\label{eq:sf}
    s_X = \frac{\mathrm{max}(|\bm{X}|)}{\mathrm{max}(E_{B_e}M_{B_m})}
\end{align}
In the above equation, $|\cdot|$ denotes the absolute value function.

Next, we scale $\bm{X}$ by $s_X$ and quantize it to $\hat{\bm{X}}$ by rounding it to the nearest quantization level of $E_{B_e}M_{B_m}$ as:

\begin{align}
\label{eq:tensor_quant}
    \hat{\bm{X}} = \text{round-to-nearest}\left(\frac{\bm{X}}{s_X}, E_{B_e}M_{B_m}\right)
\end{align}

We perform dynamic max-scaled quantization \citep{wu2020integer}, where the scale factor $s$ for activations is dynamically computed during runtime.

\subsection{Vector Scaled Quantization}
\begin{wrapfigure}{r}{0.35\linewidth}
  \centering
  \includegraphics[width=\linewidth]{sections/figures/vsquant.jpg}
  \caption{\small Vectorwise decomposition for per-vector scaled quantization (VSQ \citep{dai2021vsq}).}
  \label{fig:vsquant}
\end{wrapfigure}
During VSQ \citep{dai2021vsq}, the operand tensors are decomposed into 1D vectors in a hardware friendly manner as shown in Figure \ref{fig:vsquant}. Since the decomposed tensors are used as operands in matrix multiplications during inference, it is beneficial to perform this decomposition along the reduction dimension of the multiplication. The vectorwise quantization is performed similar to tensorwise quantization described in Equations \ref{eq:sf} and \ref{eq:tensor_quant}, where a scale factor $s_v$ is required for each vector $\bm{v}$ that maps the maximum absolute value of that vector to the maximum quantization level. While smaller vector lengths can lead to larger accuracy gains, the associated memory and computational overheads due to the per-vector scale factors increases. To alleviate these overheads, VSQ \citep{dai2021vsq} proposed a second level quantization of the per-vector scale factors to unsigned integers, while MX \citep{rouhani2023shared} quantizes them to integer powers of 2 (denoted as $2^{INT}$).

\subsubsection{MX Format}
The MX format proposed in \citep{rouhani2023microscaling} introduces the concept of sub-block shifting. For every two scalar elements of $b$-bits each, there is a shared exponent bit. The value of this exponent bit is determined through an empirical analysis that targets minimizing quantization MSE. We note that the FP format $E_{1}M_{b}$ is strictly better than MX from an accuracy perspective since it allocates a dedicated exponent bit to each scalar as opposed to sharing it across two scalars. Therefore, we conservatively bound the accuracy of a $b+2$-bit signed MX format with that of a $E_{1}M_{b}$ format in our comparisons. For instance, we use E1M2 format as a proxy for MX4.

\begin{figure}
    \centering
    \includegraphics[width=1\linewidth]{sections//figures/BlockFormats.pdf}
    \caption{\small Comparing LO-BCQ to MX format.}
    \label{fig:block_formats}
\end{figure}

Figure \ref{fig:block_formats} compares our $4$-bit LO-BCQ block format to MX \citep{rouhani2023microscaling}. As shown, both LO-BCQ and MX decompose a given operand tensor into block arrays and each block array into blocks. Similar to MX, we find that per-block quantization ($L_b < L_A$) leads to better accuracy due to increased flexibility. While MX achieves this through per-block $1$-bit micro-scales, we associate a dedicated codebook to each block through a per-block codebook selector. Further, MX quantizes the per-block array scale-factor to E8M0 format without per-tensor scaling. In contrast during LO-BCQ, we find that per-tensor scaling combined with quantization of per-block array scale-factor to E4M3 format results in superior inference accuracy across models. 



%%%%%%%%%%%%%%%%%%%%%%%%%%%%%%%%%%%%%%%%%%%%%%%%%%%%%%%%%%%%%%%%%%%%%%%%%%%%%%%
%%%%%%%%%%%%%%%%%%%%%%%%%%%%%%%%%%%%%%%%%%%%%%%%%%%%%%%%%%%%%%%%%%%%%%%%%%%%%%%


\end{document}


% This document was modified from the file originally made available by
% Pat Langley and Andrea Danyluk for ICML-2K. This version was created
% by Iain Murray in 2018, and modified by Alexandre Bouchard in
% 2019 and 2021 and by Csaba Szepesvari, Gang Niu and Sivan Sabato in 2022.
% Modified again in 2023 and 2024 by Sivan Sabato and Jonathan Scarlett.
% Previous contributors include Dan Roy, Lise Getoor and Tobias
% Scheffer, which was slightly modified from the 2010 version by
% Thorsten Joachims & Johannes Fuernkranz, slightly modified from the
% 2009 version by Kiri Wagstaff and Sam Roweis's 2008 version, which is
% slightly modified from Prasad Tadepalli's 2007 version which is a
% lightly changed version of the previous year's version by Andrew
% Moore, which was in turn edited from those of Kristian Kersting and
% Codrina Lauth. Alex Smola contributed to the algorithmic style files.
