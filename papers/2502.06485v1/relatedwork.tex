\section{Related Work}
\paragraph{CDVAE}
The Crystal Diffusion Variational Autoencoder (CDVAE) \citep{xie2022crystal} is a generative model for crystal structures that combines a variational autoencoder (VAE) with a diffusion model. Generation from CDVAE starts with sampling from the VAE: a vector $z\sim\mathcal{N}(0, I)$ is sampled from which the lattice vectors $L$, the number of atoms $M$, and the initial composition are decoded. The positions of the $M$ atoms are randomly initialized, and the elements are randomly assigned according to the decoded composition. The diffusion process then consists of denoising the positions and elements, conditioned on $z$, while keeping $L$ fixed during the full process. The positions and atoms are updated without any explicit or built-in constraints with respect to symmetries. 


\paragraph{DiffCSP and DiffCSP++}
DiffCSP \citep{jiao_crystal_2023} builds upon CDVAE by replacing the VAE with a diffusion model that jointly learns the lattice and coordinates, enabling more precise modeling of crystal geometry. \mbox{DiffCSP++} \cite{jiao_space_2024} further incorporates space group symmetry by leveraging pre-defined structural templates from the training data to learn atomic types and coordinates aligned with these templates. However, this might limit the diversity and novelty of the generated materials.

\paragraph{SymmCD}
To address this limitation, SymmCD \citep{levy_symmcd_2024} introduces a physically-motivated representation of symmetries as binary matrices, enabling efficient information-sharing and generalization across both crystal and site symmetries. By explicitly incorporating crystallographic symmetry into the generative process, SymmCD can generate diverse and valid crystals with realistic symmetries and predicted properties.