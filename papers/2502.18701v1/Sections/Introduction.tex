\section{Introduction}

People who are blind or have low vision (BLV) rely on the Internet to access information and perform everyday tasks, such as shopping for essential items and managing personal finances \cite{10.1145/3234695.3236337}. However, many BLV users face significant challenges when navigating websites with screen readers, which are tools that convert text to speech for auditory access \cite{10.1145/2207676.2207736}. Common issues include inaccessible table formats \cite{10.1145/274497.274521}, inadequate image descriptions that lack important visual context \cite{10.1145/3173574.3174092}, insufficient navigation aids \cite{10.1145/3173574.3173585}, and confusing or unclear explanations \cite{8651676, 10.1145/3167132.3167349}. These accessibility barriers can lead to frustration and a sense of helplessness for BLV users when interacting with websites \cite{baker2005building}.

For BLV users, navigating unfamiliar websites can be particularly challenging due to the need to understand and interact with unique website elements and layouts, which can be confusing and time-consuming. This problem is especially evident on shopping websites, where visually appealing but complex designs often take precedence over accessibility \cite{haubl2000consumer}. Shopping websites frequently use intricate layouts, dynamic content, and numerous filters or categories that may not be accessible to screen readers, further complicating the navigation experience for BLV users. In addition, shopping tasks such as finding product details, accessing reviews, managing shopping carts, and checking out involve multiple steps and specific interactions that are difficult to execute without clear and consistent navigation aids. Although simplified website designs are recommended to improve accessibility \cite{10.1145/1978942.1979001}, the lack of inclusive design on many online shopping platforms makes it difficult for BLV users to perform common tasks, such as comparing products across multiple sites, exploring deals or reading detailed product information.

Researchers have explored various methods to improve web accessibility for BLV users within the field of Human-Computer Interaction (HCI). Studies have focused on creating interactive systems to help users navigate specific elements on product pages or retrieve essential information using AI-based assistants~\cite{10.1145/3411764.3445547, 10.1145/3234695.3236337}. However, these existing solutions primarily target specific elements, such as product descriptions or image retrieval, rather than ensuring a comprehensive, end-to-end accessible experience across the entire web page, particularly for users who may be unfamiliar with a new platform. They often rely on external tools or assistants that interact with existing site structures, which may not fully align with users' browsing needs and preferences. To fill this gap, we propose a novel approach that uses Generative AI (GenAI) to automate the optimization of shopping websites for screen reader users. Our research goes beyond previous efforts by focusing on improving the overall structure and content of websites, ensuring a more inclusive and user-friendly experience for BLV users using a screen reader. Our study aims to answer the following research questions:
\begin{enumerate}
    \item[\textbf{RQ1:}] What specific accessibility barriers do screen reader users face on shopping websites?
    \item[\textbf{RQ2:}] What are the current practices for addressing these barriers?
    \item[\textbf{RQ3:}] How can GenAI be utilized to improve the accessibility of these websites?
    \item[\textbf{RQ4:}] How effective is our proposed GenAI-based solution in enhancing accessibility?
\end{enumerate}

To address these research questions, we conducted a three-phase study that involved formative interviews, system development, and user evaluations. In the formative study, we interviewed six screen reader users to identify the specific challenges they face on shopping websites and to understand their current strategies to overcome these barriers. The insights gained from this phase informed the development of a GenAI-based tool that automatically revises HTML content to improve accessibility for screen reader users. The tool was implemented as a browser extension that dynamically reorganizes web page elements to align with screen reader navigation patterns.

Our findings indicate that the proposed GenAI solution significantly improves accessibility by restructuring web page content in ways that make it more intuitive and easier to navigate for BLV users. In user evaluations, seven screen reader users tested both the original and revised versions of a shopping website. The results demonstrated that our tool's regenerated HTML version led to more efficient navigation and a clearer understanding of website hierarchy compared to the original version. The participants particularly appreciated the logical reordering of sections, the addition of summary headings, and enhanced navigation flexibility, highlighting the potential of our approach to provide a more comprehensive and inclusive web experience for BLV users.

Our study makes several important contributions to the field of web accessibility. First, we provide an in-depth analysis of the specific accessibility barriers that screen reader users encounter on shopping websites, particularly on non-mainstream platforms with complex layouts and dynamic content. This analysis extends the current understanding of the unique challenges faced by BLV users in navigating e-commerce environments, such as inconsistent header navigation across different websites, browsing and item comparison fatigue, and how BLV users adopt the latest GenAI technologies to complement the lack of product information in images and description. Second, we introduce a novel application of GenAI for enhancing web accessibility. Unlike previous approaches that focus on isolated elements or rely on external tools, our GenAI-based solution offers an automated, comprehensive optimization of website structure and content directly on the web page. Third, we develop and validate a browser extension that can be used to improve the accessibility of shopping websites, providing a practical tool which can make an immediate impact. Finally, our user evaluations show that the revised HTML content generated by our tool leads to substantial improvements in usability and navigation efficiency for screen reader users.
%v2 intro by ayaka
% \fixme{
% Blind or low vision (BLV) individuals routinely utilize websites for information gathering and purchasing everyday necessities \cite{10.1145/3234695.3236337}. While many BLV users employ screen readers to access information aurally when using websites, they frequently encounter various obstacles when navigating web pages \cite{10.1145/2207676.2207736}. Specifically, issues such as inaccessible table information \cite{10.1145/274497.274521}, inadequate image descriptions \cite{10.1145/3173574.3174092}, lack of navigation aids \cite{10.1145/3173574.3173585}, and inappropriate explanations \cite{8651676、10.1145/3167132.3167349} are common, potentially leading to feelings of helplessness among visually impaired users when interacting with websites \cite{baker2005building}. 

% Notably, when accessing unfamiliar websites, comprehending and effectively utilizing site-specific features can be an exceptionally challenging and time-consuming task for visually impaired users. Consequently, simplified website designs are recommended \cite{10.1145/1978942.1979001}. However, this issue is particularly pronounced in shopping websites, which often prioritize visual appeal and employ complex layouts and designs \cite{haubl2000consumer}. Furthermore, while product comparisons across multiple websites are common practice, BLV users find this process difficult on unfamiliar online shopping platforms, highlighting a lack of inclusive design.

% In response, the field of Human-Computer Interaction (HCI) has explored various approaches to enhance web accessibility for visually impaired users. These include interactive information retrieval systems for product pages \cite{10.1145/3411764.3445547} and AI-driven conversion of e-commerce websites into structured formats \cite{10.1145/3234695.3236337}. However, these solutions often focus on specific elements or functionalities and do not comprehensively address the overall accessibility of entire websites, particularly for unfamiliar or new platforms. To address this gap, we propose an approach utilizing generative AI (GenAI) to automate accessibility design for shopping websites tailored to screen reader users. Our research aims to optimize the overall structure and content of shopping websites, diverging from previous studies that focused on individual elements. By leveraging GenAI, we aspire to create inherently accessible websites, enabling developers and designers to incorporate accessibility considerations directly into the design phase.

% This study investigates the following research questions:


%add the findings %add the findings %add the findings
% To address these questions, we conducted a three-phase study: (1) a formative study with six screen reader users to identify accessibility challenges and user strategies, (2) development of a GenAI-powered system to optimize website HTML structure, tags, and descriptions, and (3) a user evaluation with seven visually impaired participants to assess the effectiveness of our approach.
%add the findings %add the findings %add the findings

% Our research contributes to the fields of HCI and web accessibility by:
% \begin{itemize}
%   \item Providing detailed insights into specific accessibility challenges faced by blind and low vision (BLV) users on e-commerce websites.
%   \item Introducing a novel approach using GenAI to automatically enhance website accessibility.
%   \item Offering empirical evidence of this approach's effectiveness through user evaluation.
%   \item Presenting design implications for developing accessible e-commerce platforms.
% \end{itemize}

% By addressing these research questions and providing these contributions, our study aims to advance the field of web accessibility and facilitate effective utilization of both new and existing websites by screen reader users.}