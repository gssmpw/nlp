\section{Methodology Overview}

To address our research questions (RQ), we designed a structured methodology comprising four key steps, summarized in Figure~\ref{fig:method}. In step \ding{182}, we conducted a formative study to uncover the challenges screen reader users face, their coping strategies, and their specific needs when navigating shopping websites. These findings informed step \ding{183}, where we developed a browser extension system powered by GenAI to automatically revise website HTML for improved accessibility. In steps \ding{184} and \ding{185}, we evaluated the improved webpages through technical assessments and user evaluations to validate both their accessibility enhancements and content integrity.
%We conducted a three-phase study to address the proposed research questions:

% recover
\begin{figure}[h]
\centering
\includegraphics[width=13cm]{figures/method.png}
\caption[A flowchart depicting the four main steps of the study methodology. Step 1: Formative study to identify challenges, coping strategies, and needs of screen reader users navigating shopping websites. Step 2: Development of a browser extension system powered by GenAI to automatically revise website HTML for improved accessibility. Step 3: Technical assessments to evaluate accessibility improvements and content integrity of the revised webpages. Step 4: User evaluations to validate accessibility enhancements and gather qualitative feedback. Arrows connect the steps sequentially, showing the flow from formative research to development, technical assessment, and user validation.]{
\textbf{Methodology Overview---%
    \textmd{{\small Note that participants of the user evaluation (step \ding{185}) do not overlap with those in steps \ding{182}}. 
    }
 }
 }
\label{fig:method}
\vspace{-0.16in}
\end{figure}


% \begin{itemize}
% \item \textbf{User Study 1: Formative Study (Section \ref{sec:phase1})} We conducted a formative study with six screen reader users to (i) identify the specific accessibility barriers that BLV users face on non-mainstream online shopping websites, and (ii) understand their current strategies for overcoming these challenges.

% \item \textbf{System Implementation (Section \ref{sec:phase2})}: Based on the findings from the formative study, we developed a pipeline that uses GenAI to automatically revise the HTML of websites to eliminate or mitigate the identified issues. Additionally, we built a browser extension system that applies these HTML changes to live websites in real-time for user testing.

% \item  \textbf{User Study 2: User Evaluation (Section \ref{sec:phase3})} We conducted a user evaluation of the prototype revised website with seven participants, where they navigated a non-mainstream and unfamiliar shopping website. We chose Mercari\footnote{\url{https://www.mercari.com/}} as an example because it is relatively similar to the most popular examples such as Ebay\footnote{\url{https://www.ebay.com/}} or Amazon, available to participants in United States, and is relatively unknown compared to the other shopping websites. Participants provided feedback through a survey and online interviews, allowing us to assess the effectiveness of the accessibility improvements.
% \end{itemize}


\begin{comment}
\begin{itemize}
    \item \textbf{Formative Study (Section \ref{sec:phase1})}: We conducted a formative study with six screen reader users to (i) identify the specific accessibility barriers that BLV users face on non-mainstream online shopping websites, and (ii) understand their current strategies for overcoming these challenges.
   
    \item \textbf{System Implementation (Section \ref{sec:phase2})}: Based on the findings from the formative study, we developed a pipeline that uses GenAI to automatically revise the HTML of websites to eliminate or mitigate the identified issues. Additionally, we built a browser extension system that applies these HTML changes to live websites in real-time for user testing.
    
    \item \textbf{User Evaluation (Section \ref{sec:phase3})}: We conducted a user evaluation of the prototype revised website with seven participants, where they navigated a non-mainstream and unfamiliar shopping website. We chose Mercari\footnote{\url{https://www.mercari.com/}} as an example because it is relatively similar to the most popular examples such as Ebay\footnote{\url{https://www.ebay.com/}} or Amazon, available to participants in United States, and is relatively unknown compared to the other shopping websites. Participants provided feedback through a survey and online interviews, allowing us to assess the effectiveness of the accessibility improvements.
\end{itemize}
\end{comment}

\subsection{Participants recruitment}
\label{sec:recruit}
A total of 21 participants from the United States participated in our study, with 6 in the formative study and 15 in the user evaluation. Participants were recruited via a screening survey shared on platforms like Prolific\footnote{\url{https://www.prolific.com/}}, as well as mailing lists and word of mouth. The survey assessed vision ability, screen reader usage, and online shopping frequency to identify BLV users of shopping websites. User evaluation participants were distinct from those in the formative study to avoid priming bias. Each participant received a \$25 Amazon gift card per hour. Study sessions were audio-recorded, transcribed, and conducted with participants’ consent. The study was approved by the Institutional Review Board (IRB).


% A total of \fixme{21} participants located in the United States took part in our formative study (n = 6) and user evaluation (n = \fixme{15}). We recruited participants by sharing a screening survey on online platforms such as Facebook\footnote{\url{https://www.facebook.com/}} and Prolific\footnote{\url{https://www.prolific.com/}}, as well as through mailing lists and word of mouth. The screening survey included questions about participants' vision ability, screen reader usage, and frequency of online shopping to help identify those who met our inclusion criteria as BLV users of shopping websites. Participants in the user evaluation were different from those in the formative study to avoid priming bias. Each participant was compensated with a \$25 Amazon gift card per hour. Participants provided their consent to participate in the study before each interview session. All study sessions were audio recorded and transcribed with the participants' consent. The study was approved by the Institutional Review Board (IRB).

\subsection{Data Analysis}
\label{sec:data_ana}
The recordings from the formative study and user evaluation sessions were transcribed using Zoom\footnote{\url{https://zoom.us/}}. We then performed a thematic analysis \cite{braun2006using} on these transcriptions. Three authors were involved in the data analysis: two authors coded the formative study data, with one of them also participating in coding the user evaluation data alongside another author. We employed a deductive analysis method to explore predefined research questions as high-level themes, such as \textit{``challenges on browsing shopping websites''}. Subthemes were added while reading through the transcriptions, and new high-level themes were proposed when necessary.

For both the formative study and user evaluation, two authors independently reviewed two interview transcripts, developed initial codes, and compared them to establish a consistent preliminary codebook. Each author then independently applied the codebook to code the remaining transcriptions. We met regularly to discuss emerging subthemes, resolve discrepancies, and update the codebook. This iterative process resulted in approximately 30 themes for the formative study codebook and 50 themes for the user evaluation codebook. Some example themes include: \textit{``Too many headings on shopping websites,''} \textit{``Hard to locate product information,''} and \textit{``Preference for generated HTML version.''}


% We performed coding using xxxxx \footnote{\url{xxxxx}}. For instance, a participant's statement such as ``xxxx'' was considered a single code. Quotes from other participants expressing similar opinions might be tagged with the same code. This code, along with other related codes, was organized under the theme ``xxxxxx''. Through this process, we identified xxx codes. Subsequently, we employed the affinity diagramming method \cite{beyer1999contextual} to categorize these codes into themes. This process culminated in the identification of xxx primary themes: (1) xxxx,  and (2) xxxxx. It is worth noting that the number and composition of themes in thematic analysis can vary depending on factors such as the volume of data and the research content \cite{braun2006using}.

% Based on the observed literature we designed a three-part study to investigate and address issues experienced by people with visual impairments when accessing online shopping website with screen readers. First we conducted an initial investigation into challenges experienced by screen reader users. Then we developed a prototype in the form of a web browser plugin that analyzes a website and prompts a Large Language Model (LLM) to restructure or rewrite a website to make it more suitable for screen readers. Then finally we tested the developed prototype in a series of user tests followed by semi-structured user interviews.

\section{Formative Study}
\label{sec:phase1}
\label{sec:formative study}
Prior research has examined the accessibility barriers that BLV users face on commonly used shopping websites~\cite{10.1145/3313831.3376404, 10.1145/3411764.3445547}, primarily focusing on identifying challenges and improving image accessibility. One study~\cite{10.1145/3234695.3236337} specifically noted that BLV users experience difficulties accessing product information due to missing alt text for images and non-compliance with WCAG standards \cite{wcag21}. However, there is limited research on the specific challenges that BLV users face throughout their shopping process due to website design and development, especially on new or unfamiliar websites. Addressing this gap is crucial, as BLV users often want to explore a broader range of shopping options and compare products but are limited to a very narrow selection of accessible websites. There is also a need to understand their strategies for overcoming these challenges and their desired solutions to address these issues. We conducted a formative study to explore these questions.

% Based on the observed literature, we outline two primary research questions: (1) ``\textbf{What problems do screen reader users experience when shopping online?}'' and (2) ``\textbf{How can generative AI applications address these problems?}'' 


% \begin{enumerate}
%     \item \textbf{Initial investigation into challenges experienced by screen reader users} when browsing online shopping websites,
%     \item \textbf{Developing a browser plugin prototype} that analyzes a website and prompts a Large Language Model (LLM) to restructure or rewrite a website to make it more suitable for screen readers,
%     \item \textbf{Prototype validation} in a series of user tests followed by semi-structured interviews
% \end{enumerate}

% \subsection{Initial Investigation Into Screen Reader Issues}

% \subsubsection{Study Design}
%introduction
\subsection{Semi-Structured Interview Procedure} 
The interviews were conducted online using the Zoom video calling platform and lasted between 45-60 minutes. At the beginning of each interview, we obtained participants' consent for the study and for recording the session. The interview was divided into two parts. In the first part, we asked participants about their experiences with online shopping, the challenges they encountered, the accessibility issues they faced on previous shopping websites, how they handled these issues, and their suggestions for improvements. In the second part, participants were asked to \textit{``screen share and think aloud''} while browsing a new shopping website, Mercari. We instructed them to start by browsing the homepage and exploring the website, then presented them with a scenario of searching for and selecting a blender on the website. Participants were asked to think aloud about the keys they used and their thoughts on the browsing experience with screen readers while performing these tasks. Data analysis details are in section~\ref{sec:data_ana}.



% The first phase of the study focuses on understanding the challenges and needs of screen reader users on shopping websites. Similar works on accessibility of web and screen readers\cite{10.1145/3234695.3236337,chartreader,schaadhardt2021understanding} suggest using a participatory design approach in which users are initially interviewed about the issues they experience when using screen readers online. Thus, we implemented a survey that included questions on participants' vision ability and screen reader usage, frequency of online shopping, and an open comment on challenges they previously experienced. Upon completion of a survey, participants were invited for a series of semi-structured interviews to understand users' previous experience and challenges on online shopping or selling. Finally, the interview was then followed by a think-aloud observation of a participant's experience on a relatively new online shopping website that participants were not previously familiar with (e.g., Mercari\footnote{https://www.mercari.com/}).
%popular shopping website (e.g., Amazon\footnote{https://www.amazon.com/}) and a relatively newer shopping platform (e.g., Mercari\footnote{https://www.mercari.com/}).

% , each participant was paid 25 currency\footnote{currency name removed for manuscript anonymization purposes} per hour. All participants were recruited through the Prolific\footnote{https://www.prolific.com/} platform. The responses of all participants were then transcribed, analyzed, and grouped into emerging themes. 

% \yaman{We only tested on Mercari, please also justify}

%Results
%would be nice to include a table with participant info
\subsection{Participants Demographics}
The formative study included six BLV users who regularly used screen readers for online shopping. These participants, aged 25 to 54 years, were recruited through social networks and mailing lists. Half of them were legally blind (n = 3), and the other half had significant visual impairments requiring assistive technologies (n = 3). Four participants self-identified as male and two as female. Beyond basic demographics, we collected detailed background information on their online shopping habits to ensure a diverse group of participants. This included people who shop online once a week and those who did so 2-3 times a week, on various e-commerce platforms. More details are provided in the demographic Table \ref{tab:phase1demo}.

% recover
\begin{table}[H]
\begin{tabular}{llllll}
\hline
\textbf{ID} &
  \textbf{Age} &
  \textbf{Gender} &
  \textbf{Visual Ability} &
  \textbf{\begin{tabular}[c]{@{}l@{}}Online shopping\\ Frequency\end{tabular}} &
  \textbf{\begin{tabular}[c]{@{}l@{}}Used Screen\\ Reader\end{tabular}}  \\ \hline
\textbf{P1} & 25-34 & Male   & Completely blind           & Once a week      & NVDA, Google Talkback                \\
\textbf{P2} & 25-34 & Male   & Blind                      & Once a week      & JAWS, NVDA, VoiceOver \\
\textbf{P3} & 35-44 & Male   & Totally blind              & Once a week      & JAWS, VoiceOver                \\
\textbf{P4} & 25-34 & Female & I have light perception    & Once a week      & JAWS, NVDA, VoiceOver                \\
\textbf{P5} & 25-34 & Male   & Only some light perception & 2-3 times a week & JAWS, NVDA, VoiceOver                 \\
\textbf{P6} & 45-54 & Female & Limited light perception   & Once a week      & JAWS, NVDA           \\ \hline
\end{tabular}
\caption[Table 1. Formative Study Participants Demographic Table. The table contains demographic information for six participants labeled P1 through P6. It has six columns: ID, Age, Gender, Visual Ability, Online shopping Frequency, and Used Screen Reader. The participants’ ages range from 25-54, with four males and two females. Their visual abilities vary from completely blind to limited light perception. All participants shop online at least once a week, with one shopping 2-3 times a week. They use various screen readers, including NVDA, JAWS, Google Talkback, and VoiceOver, with most participants using multiple screen readers. This table provides a comprehensive overview of the diverse group of visually impaired participants in the formative study, highlighting their varied demographics and screen reader usage patterns.]{Formative Study Participants Demographic Table}
\label{tab:phase1demo}
\end{table}


% \subsection{Data Analysis}
% We conducted six semi-structured interviews. The interviews were automatically transcribed via built-in audio transcription feature on the Zoom platform. Similarly to previous studies on screen readers and accessibility\cite{schaadhardt2021understanding,10.1145/3234695.3236337}, we used thematic analysis and open coding to analyze the responses of interview participants. To perform the coding, two authors independently organized the transcribed interviews into emerging themes and then compared the codes. We highlight five different themes that emerged during the coding process. 

\subsection{Interview Findings}
% \subsubsection{Current Challenges and Practices}
All participants encountered accessibility barriers when using online shopping websites. These barriers included inappropriate use of HTML tags and labels, a lack of comprehensive alt text for images, and an offline return process. In the following sections, we will discuss each of these challenges in detail and report the strategies participants used to counter them.

\subsubsection{Current Accessibility Challenges on Shopping Websites}

\paragraph{\textbf{\textit{Inappropriate Use of HTML Tags and Labels.}}}
All participants highlighted  the inappropriate use of HTML tags and labels as the most common accessibility issue on shopping websites. Screen reader users reported relying on shortcut keys to navigate HTML components efficiently. P1 explained, \textit{``Most of the time, if I enter a website, I press H to check if they have headings or not, because this is the fastest and easiest way for us to browse.''} Common shortcuts include pressing "H" to iterate through all headings or numbers 1 to 6 to navigate specific heading levels ~\cite{yu2023design, kaushik2023guardlens, webaim_screen_reader_survey_10}. Misuse of HTML tags like headings frustrates users with excessive, irrelevant information, making it hard to locate key details or understand crucial information needed for purchase decisions.

% All participants highlighted that the most prevalent accessibility issue on shopping websites is the inappropriate use of HTML tags and labels. Screen reader users reported that they typically use shortcut keys to quickly navigate between specific HTML components based on their tags. P1 explained in the interview, \textit{``Most of the times if I enter website, I press H to check if they have headings or not, because this is the fastest and the easiest way for us to browse.''} The most commonly used shortcuts involve jumping between headings ~\cite{yu2023design, kaushik2023guardlens, webaim_screen_reader_survey_10}: pressing ``H'' allows users to sequentially iterate over all headings regardless of their level, while pressing numbers 1 to 6 enables them to browse through headings of a specific level one at a time. Inappropriate use of HTML tags such as heading leads to screen reader users' frustration from hearing excessive unnecessary information, difficulty in locating key details, and challenges in understanding crucial information needed for making purchase decisions.

\paragraph{\textbf{Too many headings.}}
%P5 "And then I search, and then I press heading. I'm assuming the search results are heading by headings. Yes, this is the heading filter, by oh, 1,000 results or so."
%P2: "00:56:48.000 There's just like so many headings. Felt like.Unnecessary like. I would have liked.

Five out of six participants reported that shopping websites often feel overcrowded due to excessive emphasis on non-essential information through headings, causing confusion about the website's structure and flow, particularly during initial browsing. Overuse of headings makes it hard for screen reader users to navigate quickly, as they must listen to irrelevant content. For instance, some websites place long lists of category headings, like electronics or home, before the main content, forcing users to navigate through numerous headings—often over ten—before reaching key information. P2 noted, \textit{``Categories don’t have to be headings; there could just be a heading titled `Categories,' with items listed as links underneath.''} Similarly, P1 mentioned, \textit{``Sometimes all the filters are marked as headings, making it take a long time to reach the product search results.''}
% Many participants (5 out of 6) reported that shopping websites often feel overcrowded due to excessive emphasis on non-essential information through headings, which creates confusion for screen reader users regarding the website's structure and flow. This issue is particularly problematic during their initial browsing experience. The overuse of headings makes it difficult for screen reader users to quickly navigate to key information, as they are forced to listen to irrelevant content. For example, some websites place long lists of category headings, such as electronics or home, before the main content. As a result, screen reader users must navigate through numerous category headings—often more than ten—before reaching the essential information they are interested in, significantly increasing navigation time and effort. P2 specifically pointed out that some websites unnecessarily use headings for every category, \textit{``Categories do not have to be headings; there could just be a heading titled ``Categories'', with each item listed as a link underneath.''}  P1 also mentioned that \textit{``Sometimes all the filters are marked as headings, which makes it take a long time to reach the product search results.''}

\paragraph{\textbf{Lack of headings.}}
%P4: So, right? So on mobile, it's always been the same, like, you know, like, for example, like, swipe right, it tells me the button or heading or whatever. Like, headings are usually more for there's not, you know, something here to enter. So there's not as many headings on mobile, this might be buttons or links.
On the contrary, some informal yet essential information is not placed in headings or appropriate HTML tags, misaligning with screen reader users' navigation preferences. Screen reader users typically jump between headings, links, and buttons to quickly access key information ~\cite{yu2023design,webaim_screen_reader_survey_10}. P3 explained, \textit{``When I am searching for something, I really expect it to be in a heading.''} For example, some shopping websites fail to include product titles in headings. Participants often attempted to navigate by headings to locate products but, failing to do so, resorted to using the down arrow key to iterate through elements on the search results page. P4 noted, \textit{``Some smaller store websites are not accessible at all; they are too graphical with not enough headings and links to navigate.''}

% other informal and essential information sometimes is not in headings or other appropriate HTML tags that align with screen reader users' browsing preference. This mismatch does not align with the way screen reader users typically navigate by jumping between headings, links, and buttons to quickly access the most important information ~\cite{yu2023design,webaim_screen_reader_survey_10}. P3 explained in the interview, \textit{``when I am searching for something, I am really expecting that it will be in heading.''}  For example, some shopping websites do not include the product title in the headings. Participants found first navigated by headings to find the products and failed, so they need to use down arrow key to iterate each element to locate where the products start on the search result page. P4 explicitly mentioned \textit{``Some of the smaller store websites are not accessible at all; they are too graphical with not enough headings and links to navigate.''} 

\paragraph{\textbf{Disorganized Heading Hierarchy.}}
% P4: So escape. You know, it takes it back to the top of the page every time. So once I get in, you know, when I pick my selection, then it goes take me back to the top, and then I have to do h again, popular searches, heading level six, TV, remote. And again, I'm arrowing down. So I did H couple of times. Now I'm arrowing down once, heading up any for so, so at all, like one, TV, one, tell one, dad, five, cluster three. DVD, one.
Even when appropriate headings are used, inconsistent heading levels and poorly structured layouts can still pose challenges for screen reader users. For instance, a website might assign a level 1 heading to reviews and a level 6 heading to product descriptions, confusing users who rely on specific heading levels for navigation.

Many participants navigate headings using the ``H'' key, following their sequence in the HTML code rather than their visual layout. This often leads to a disjointed experience, as visually oriented layouts may prioritize sighted users. For example, if the layout places images and reviews on the left and product names and details on the right, screen reader users will hear about reviews and vague image descriptions first, instead of the crucial product names and details. Screen reader users may encounter irrelevant information, like reviews or vague image descriptions, before crucial product details. P5 noted, \textit{``I just don’t understand why unnecessary information like likes, comments, and share buttons is positioned before the product information.''}

% Even when shopping websites use appropriate headings for content, the assignment of heading levels and the layout of these elements can still be challenging for screen reader users. For example, a website might use headings for seller information, reviews, and product descriptions but inconsistently assign heading levels, such as using a level 1 heading for reviews and a level 6 heading for the product description. This inconsistency can confuse users who rely on navigating by specific heading levels. 

% Many participants use the ``H'' key to navigate through headings in the order in which they appear in the HTML code, without considering the heading levels. This can lead to a disjointed browsing experience if the visual layout of the page is designed primarily for sighted users. For example, if the layout places images and reviews on the left and product names and details on the right, screen reader users will hear about reviews and vague image descriptions first, instead of the crucial product names and details. This mismatch in the layout and navigation order creates confusion and makes it harder for screen reader users to find key information efficiently. P5 explicitly mentioned the example and noted \textit{``I just do not understand why unnecessary information like likes, comments, and share buttons is positioned before the product information.''}

\paragraph{\textbf{Unclear Labels on Images and Buttons.}}
Participants reported frustration with the lack of clear, descriptive labels for buttons and images, a common issue on less mainstream shopping websites. Unclear labels confuse screen reader users, who rely on meaningful descriptions to navigate interactive elements. For instance, a button might be labeled generically as ``button'', providing no indication of its function, such as ``add to cart'' or ``view details.'' Similarly, images may lack alt text, leaving their content or relevance unclear. This forces users to guess button functions or spend unnecessary time exploring, reducing usability and accessibility issues. P2 noted, \textit{``Amazon has a few unlabeled buttons that can be a bit confusing sometimes. You can usually figure them out from the context, but they are not labeled correctly.''}


% Participants reported frustration with the lack of clear and descriptive labels for buttons and images, a common issue on less mainstream shopping websites. Unclear labels create confusion and hinder navigation for screen reader users who rely on meaningful text descriptions to understand the function of interactive elements. For example, a button might be generically labeled as the ``button'' without any additional context, making it impossible for users to know what action it performs, such as ``add to cart'' or ``view details.'' Similarly, images may lack descriptive alt text, preventing users from understanding their content or relevance. This lack of clarity forces screen reader users to guess the purpose of buttons and images or to spend time exploring them unnecessarily, significantly diminishing the overall usability and accessibility of the website. Without meaningful labels, these users cannot efficiently navigate the site or complete tasks, which can lead to frustration and abandonment of the site altogether. For example, P2 mentioned that \textit{``Amazon has a few unlabeled buttons, that can be a bit confusing sometimes. You can usually figure them out from the context, but they are not labeled correctly.''}

\paragraph{\textbf{\textit{Lack of comprehensive and accurate text and image description for products.}}}
Participants noted difficulties understanding product details on shopping websites due to insufficient text and image descriptions. Many sites rely heavily on images to convey key information, such as color, style, condition, and features, without providing adequate text. This creates a significant barrier for screen reader users, who depend on alt text to interpret images. P2 shared, \textit{``The condition of items on eBay is hard to identify because it's all about images.''} Participants also mentioned inconsistencies between product descriptions and reviews, adding to the confusion. P3 explained, \textit{``When it comes to clothing, I want more detail than just `blue shirt.' I need to know about the fit and style, which is often missing in online descriptions.''}

% Participants highlighted a challenge in trying to understand product details on shopping websites due to a lack of comprehensive text and image descriptions. Many websites often rely heavily on images to convey important information about products, such as color, style, condition and specific features, without providing adequate accompanying text descriptions. This practice poses a major accessibility barrier for screen reader users who cannot see these images and instead rely on alt text to understand the content. For example, P2 shared that \textit{``The condition of items on eBay is hard to identify because it's all about images.''} The participants also shared that product descriptions sometimes contradict product reviews, creating confusion. They often need to verify details shown in images, but the lack of clear and comprehensive image descriptions makes this verification difficult. P3 shared that \textit{``When it comes to clothing, I want more detail than just ``blue shirt.'' I need to know about the fit and style, which is often missing in online descriptions.''}

\paragraph{\textbf{\textit{Hard to compare different products.}}}
Three out of six participants reported significant challenges comparing similar products on shopping websites. While sighted users can quickly assess visual and descriptive information, screen reader users must repeatedly switch between tabs to find key details like prices, descriptions, and specifications. This process is time-consuming and mentally exhausting. P3 explained, \textit{``Comparing products, especially similar ones, is just exhausting. I have to remember all the details in my head and switch back and forth between tabs to check things like the price or specific features.''} Improper use of HTML tags and labels further complicates navigation, making the task even more inefficient. P2 similarly noted, \textit{``It’s really hard to compare products that are very similar. I have to keep switching between tabs to find the price or some specific details. It’s not efficient at all.''}
% Some participants (3 out of 6) reported significant challenges when comparing different products, particularly similar models, on shopping websites. While sighted users can quickly compare visual information and descriptive text, screen reader users must repeatedly switch between tabs to locate key details, such as prices, descriptions, and product specifications. This appears to be a unique challenge for screen reader users that was relatively unexplored in previous literature. Users report that the process is not only time-consuming, but also mentally exhausting. For example, P3 explained, \textit{``Comparing products, especially similar ones, is just exhausting. I have to remember all the details in my head and switch back and forth between tabs to check things like the price or specific features.''} The issue is further compounded by the improper use of HTML tags and labels on these websites, which makes navigation even more cumbersome. P2 shared a similar sentiment, stating, \textit{``It’s really hard to compare products that are very similar. I have to keep switching between tabs to find the price or some specific details. It’s not efficient at all.''}


\subsubsection{Strategies for Overcoming Accessibility Challenges}

\paragraph{\textbf{\textit{Relying on External Video Reviews.}}
\normalfont{Participants reported relying on external resources, such as YouTube reviews, to counter accessibility challenges and gain a clearer understanding of products. These video reviews provide in-depth demonstrations and visual details that are often lacking in the brief text descriptions found on shopping sites. For example, P4 mentioned, }\textit{``When I’m not sure about how a product looks or functions, I usually check YouTube. The videos give me a much better idea than the short descriptions on the website.''} \normalfont{This approach helps participants make more informed decisions, especially when considering the purchase of products where visual details are important, such as electronics or clothing.}}

\paragraph{\textbf{\textit{Utilizing User Ratings and Reviews}}
\normalfont{Additionally, participants closely follow user ratings and reviews on shopping platforms to gauge the condition and quality of products. Since product images or brief descriptions can be misleading or incomplete, user reviews become a crucial source of information. These reviews often provide first-hand experiences and specific details that are not immediately visible or described in the official product listings. As P2 explained, }\textit{``The product pictures don’t always tell the full story. I go through the reviews to see what people are actually saying about the product condition and quality.''}}

\paragraph{\textbf{\textit{Seeking Assistance from Sighted Individuals}}
\normalfont{Participants might ask sighted friends or family members for assistance in describing visual details that are important for making purchase decisions. This practice is necessary when the product’s visual information is critical but inaccessible. P5 shared their experience, stating,} \textit{``Sometimes I have to ask someone to look at it to make sure I am seeing the right thing, especially when it is not clear on the website.''} \normalfont{This indicates a reliance on others when the digital content is insufficient, which adds another layer of dependency and effort for screen reader users.}}

\paragraph{\textbf{\textit{Leveraging GenAI for Accessibility Enhancements}}
\normalfont Interestingly, most of the participants (4 out of 6) have used GenAI to help mitigate accessibility challenges on shopping websites. They reported using GenAI tools such as ChatGPT ~\cite{chatgpt} and Be My Eyes ~\cite{bemyeyes} to get detailed descriptions of product images and clarify ambiguous information that was not mentioned in the text description or even in the image alt text. P5 shared that \textit{``I used ChatGPT to describe pictures of the product and got a better idea of what the product consists of. It would be great to have an AI that can answer questions about how a product looks.''} \normalfont{Participants also used GenAI to compare different models of a product by generating a navigable table that highlights key differences, making it easier to understand these differences and make informed purchase decisions}. P6 also shared, \textit{``I usually make the AI compare two products by giving it links or screenshots of the pictures to find the differences that are not mentioned in the descriptions.''} \normalfont{However, they also reported limited trust in AI to conduct product searches because it often does not bring up results from preferred websites such as Amazon. P6 explained,} \textit{``I do not trust AI to search for products on Amazon. I usually search for the product myself and then use AI to make sure it has the features I am looking for.''}}

\paragraph{\textbf{\textit{Preferred Improvements for Screen Reader Users}}
\normalfont{Participants preferred browsing websites themselves over using virtual assistants, which have been proposed as potential solutions~\cite{virtual_assistant_vtyurina2019verse}. They cited concerns about reliability, language compatibility, and additional accessibility barriers. P1 explained, \textit{``I prefer browsing myself because I can control what I do with my screen reader.''} P2 noted issues with virtual assistants, stating, \textit{``If you are in the same window with a virtual assistant, you are not aware of new messages. That is not friendly for screen reader users at all.''} Instead, participants suggested improvements like a screen reader mode'' that simplifies web pages by showing only essential elements, such as product descriptions. P3 shared, \textit{``It would be helpful if shopping websites had a screen reader mode that strips away images and just shows the product descriptions.''} This highlights their preference for streamlined, user-controlled experiences that focus on relevant information.}}

% \subsubsection{Use of Generative AI to Enhance Accessibility}
% Surprisingly, most of the participants (4 out of 6) have used Generative AI to help mitigate accessibility challenges on shopping websites. They reported using Generative AI tools like ChatGPT and Be My AI get detailed descriptions of product images and clarify ambiguous information that were not mentioned in the text description or even in the image alt text. P5 shared that \textit{``I used ChatGPT to describe pictures of the product and got a better idea of what the product consists of. It would be great to have an AI that can answer questions about how a product looks.''} Participants also used Generative AI to compare different models of a product by generating a navigable table that highlights key differences, making it easier to understand these differences and make informed purchase decisions. P6 also shared that \textit{``I usually make the AI compare two products by giving it links or screenshots of the pictures to find the differences that are not mentioned in the descriptions''} But they also reported limited trust in AI to conduct product searches because it often does not bring up results from preferred websites like Amazon. P6 explained \textit{``I don’t trust AI to search for products on Amazon. I usually search for the product myself and then use AI to make sure it has the features I’m looking for.''}


% \subsubsection{Participant-Desired Improvements}

% Participants expressed a clear preference for browsing websites themselves rather than relying on virtual assistants, which have been proposed by previous literature as potential solutions~\fixme{add citations}\cite{xx, yy}. They believe virtual assistants may not always be reliable, might not operate in the users' preferred language, and could introduce new accessibility barriers and learning costs. For instance, P1 noted, \textit{“I prefer browsing myself because I can control what I do with my screen reader.”} Additionally, P2 highlighted concerns about the effectiveness of virtual assistants, stating, \textit{“If you're in the same window with a virtual assistant, you’re not aware of new messages. That's not friendly for screen reader users at all.”} Instead of relying on virtual assistants, participants suggested improvements such as a "screen reader mode" that simplifies web pages by displaying only essential elements, like product descriptions, to enhance usability. P3 shared, \textit{“It would be helpful if shopping websites had a screen reader mode that strips away images and just shows the product descriptions.”} This suggestion reflects a desire for more streamlined, user-controlled experiences that minimize distractions and focus on relevant information.


% Participants have expressed explicitly preference for browsing themselves rather than using virtual assistants which have been proposed by previous literature as solutions~\fixme{add citations , please find more than one\cite{xx}}. They believe virtual assistant might not always be reliable or in the users' prefered language. P1 explained \textit{``I prefer browsing myself because I can control what I do with my screen reader.''} They also shared concern on virtual assistant itself introducing more accessibility barriers and learning cost. For example, P2 shared that \textit{``If you're in the same window with virtual assistant, you’re not aware of new messages. That's not friendly for screen reader users at all.''} Participants have suggested having a ``screen reader mode'' that simplifies the page by displaying only essential elements, such as product descriptions. P3 mentioned that \textit{``It would be helpful if shopping websites had a screen reader mode that strips away images and just shows the product descriptions.''}


% N=X participants reported using screen readers for over X hours a day in such scenarios as online browsing. N=Y participants reported utilizing screen readers for online shopping. When it comes to the difficulties experienced when browsing and shopping online with screen readers, the participants reported issues with tags (N=X), [Any other findings?].In order to adress aforementioned issues some users reported using X while others (N=X) suggested using [Something else] to improve their browsing experience.

\subsection{Design Implications}
Our findings strongly support our design intuition that leveraging GenAI to improve HTML tags and labels can significantly enhance accessibility for screen reader users. Participants already use GenAI to tackle certain accessibility challenges, such as extracting detailed descriptions from images or comparing similar products. However, directing GenAI to effectively revise the layout and information hierarchy through HTML tags is challenging and often unfeasible for them. This difficulty highlights the need for more intuitive, user-friendly solutions that seamlessly integrate with existing browsing practices, reducing the burden on users to manually adjust or work around accessibility issues.

Participants expressed a preference for accessibility solutions that operate within their familiar browsing environments rather than relying on external tools that add layers of complexity, such as AI assistants designed to help locate information. This suggests a need for back-end solutions where developers revise a website’s structure to better align with how screen reader users navigate, creating a more seamless and integrated user experience. However, this is a challenging task for developers, as it requires specialized knowledge of both accessibility standards and screen reader behavior, along with substantial time and effort to implement these changes manually.

We developed a GenAI-driven tool to enhance online shopping website accessibility by optimizing HTML tags, labels, and information hierarchy. The system addresses common barriers for screen reader users, such as inconsistent headings, missing alt text, and unclear button labels. By automating these adjustments, the tool reduces developer effort and bridges knowledge gaps, offering a more efficient way to create accessible web experiences.

% Our findings strongly supported our design intuition that leveraging Generative AI to improve HTML tags and labels to improve accessibility for screen reader users. Participants already use Generative AI to address accessibility challenges, such as extracting details from images and comparing products. However, it is challenging for them to direct Generative AI to effectively revise the layout and information hierarchy through HTML tags. This difficulty points to the need for more intuitive solutions that integrate seamlessly with the existing browsing practices of users.

% Rather than relying on external tools that add layers of complexity, such as AI assistants that help locate information, participants expressed a preference for solutions that operate within their familiar settings. This means revising the website’s structure in the back-end by developers to better align with how screen reader users navigate, ensuring a smoother and more integrated experience. For instance, reordering headings, improving alt text, and restructuring content flow based on user navigation patterns can directly address their needs. Thus, we proposed and implemented a Generative AI-driven tool designed to automatically enhance the accessibility of online shopping websites by optimizing HTML tags, labels, and the overall information hierarchy. The system focuses on making websites more navigable for screen reader users by addressing common accessibility barriers such as inconsistent heading structures, missing or vague alt text, and unclear button labels.