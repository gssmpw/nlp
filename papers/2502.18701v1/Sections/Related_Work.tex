\section{Related Work}
This section clarifies the challenges faced by BLV users who utilize screen readers for navigating websites and distinguishes our research from previous HCI studies. Section \ref{subsection_2.1} identifies the difficulties encountered when using unfamiliar websites, with a particular focus on the accessibility of shopping sites that feature complex web designs. Section \ref{subsecction_2.2} discusses approaches for web accessibility and design using AI and compares these with our research.

\subsection{Online Web Accessibility}
\label{subsection_2.1}
Recent efforts in making shopping accessible can be divided into online and offline categories. While issues encountered in offline shopping have been addressed through third-party assistance \cite{offlineshopping1} and combination of technological solutions \cite{offlineshopping2,offlineshopping3}, still for BLV users who experience difficulties with mobility, visiting a store to find desired products without assistance can be challenging \cite{10.1145/2971648.2971723, 10.1145/2513383.2513449}. Consequently, many visually impaired people regularly use websites for purchasing daily necessities and gathering information \cite{10.1145/3234695.3236337}. However, screen reader users with visual impairments often encounter various obstacles when navigating new websites \cite{10.1145/2207676.2207736}. Understanding and effectively using the unique interface of a new website can be a highly challenging and time-consuming task \cite{10.1145/1805986.1806005}. Websites are often visually oriented and accessible primarily via mouse \cite{10.1145/1805986.1806005}, requiring significant adaptation time for new website configurations. As screen readers sometimes read out incomprehensible strings of characters and numbers instead of meaningful text, visually impaired users may feel helpless and vulnerable while using websites \cite{baker2005building}. Blas et al. revealed that there are unidentified areas to make websites accessible for BLV users \cite{Blas2004UsableAT}. Major problems include the lack of navigation aids \cite{10.1145/3173574.3173585}, improper table design \cite{10.1145/274497.274521}, absence of alternative text for images \cite{10.1145/3173574.3174092, 10.1145/2470654.2481291, 10.1145/1866029.1866080, 10.1145/2702123.2702437}, or overly verbose descriptions \cite{8651676, 10.1145/3167132.3167349}. Due to these issues, visually impaired users might not have full access to the vast informational resources available online \cite{leuthold2008beyond, 10.5555/1162223}. Moreover, since website designs vary significantly, unfamiliar websites can impose an additional burden on visually impaired users and it is recommended that websites use simple structures to be more accessible  \cite{10.1145/1978942.1979001}. Interfaces with advanced interactive features, such as complex layouts or designs, particularly in shopping websites \cite{haubl2000consumer}, exacerbate this issue. Stangl et al. revealed that while information on shopping websites is often presented through images, the accompanying image descriptions are frequently inadequate, leading visually impaired users to seek more detailed information. Furthermore, one study indicated that the familiarity and usability of a website significantly impact the difficulty of information retrieval  \cite{10.1145/3313831.3376404}. This suggests that gathering information is particularly challenging on new or unfamiliar websites. Additionally, visually impaired users reported that shopping websites face notable accessibility issues \cite{webaim_survey}. In HCI, various approaches have been explored to improve the usability of websites for users with visual impairments. The following details these approaches:

\subsection{\textbf{Online Shopping Websites}}
Regarding shopping sites, Wang et al. developed an interactive information retrieval system to support review-based QA on restructured product pages for online shopping \cite{10.1145/3411764.3445547}. This system extracts descriptions to help visually impaired users understand the appearance of products and is implemented as a browser extension for Amazon\footnote{https://www.amazon.com/}. This approach has proven effective in purchasing visually important items such as apparel. Additionally, Stangl et al. developed a system using AI to automatically convert online shopping websites into structured formats \cite{10.1145/3234695.3236337}. Users can interact with the system to ask questions about products, allowing them to gather specific information instead of unrelated details. This study highlighted the extensive benefits of command-based browser assistants. These previous studies focus on adopting command-based conversational approaches and browser extensions or interactive systems to reconstitute information from existing shopping websites to support their use. While useful, these approaches do not fundamentally address the difficulties associated with using unfamiliar new websites. They might also fall short of fully meeting the needs of visually impaired users who want to freely browse detailed information about numerous products. In contrast, our research aims to use GenAI to automate website design, creating accessible websites from the ground up. This approach allows developers and designers to directly create websites with built-in accessibility considerations during the design phase.

\subsection{\textbf{Improving Web Browsing Accessibility}}
In the field of web accessibility research, numerous studies have focused on improving accessibility for screen reader users. Most of these studies simulate user web browsing behaviors or enhance existing web content. Examples include transcoding technologies such as WebInSight \cite{bigham2006webinsight}, which automatically creates and inserts alternative text for images on web pages, and SADIe \cite{harper2007sadie}, which developed high-level ontology extensions for web documents that reference CSS. These aim to automatically modify the original content to improve accessibility before it reaches the end user. These studies differ from our approach of automatically generating web designs, as they focus on the automatic insertion of alternative text for images and extensions to CSS. There are also tools focused on specific functionalities or elements, such as command-based browsing \cite{10.1145/3519032}, and tools for converting HTML tables into accessible formats \cite{10.1145/3491102.3517469}, which have limitations in enhancing the accessibility of entire websites comprehensively. Furthermore, these methods primarily focus on improving existing websites and do not sufficiently incorporate accessibility into the development process of new websites. In light of previous research highlighting usability issues in shopping sites, we developed a system that leverages GenAI to automatically generate accessible designs for new shopping sites. This system aims to automatically generate accessible designs from new websites, with future applications targeted at various websites and applications.

\begin{comment} %old version
\subsection{AI-Driven Web Accessibility and Design}
\label{subsecction_2.2} 
The advancement of Artificial Intelligence (AI) technology has brought new possibilities for improving web accessibility. Various approaches focusing on supporting visually impaired individuals have been developed, offering solutions to conventional accessibility challenges. Progress in image recognition technology has enabled methods for visually impaired users to interactively explore images by identifying the location of objects within images and providing descriptions for each object. This technology has been applied to improve image search on websites \cite{jing2015visual, zhai2017visual} and generate spatial explanations for complex images \cite{zhong2015regionspeak}. Furthermore, the development of item recommendation systems combining collaborative filtering and content-based ranking is progressing, contributing to enhanced user experiences \cite{kislyuk2015human}. In HCI, diverse approaches are being taken to support visually impaired individuals in web accessibility. These include research on restructuring web data tables into formats accessible to visually impaired users \cite{10.1145/3491102.3517469}, technology for converting PDF documents to HTML \cite{10.1145/3441852.3476545,wang2021improving}, AI-generated image descriptions \cite{huh2023genassist}, development of systems to help visually impaired users select and understand short videos \cite{van2024making}, and creation of audio descriptions for videos \cite{lei2020mart, 10.1145/3411764.3445347, 10.1145/3357236.3395433, 10.1145/3334480.3382821}. Hara et al. developed an interface combining manual audio description authoring with automatic review and feedback generation based on computer vision and natural language processing \cite{10.1145/3544548.3581023}. This research demonstrated the effectiveness of automated feedback in creating high-quality scripted descriptions.

New solutions are also being proposed for challenges specific to online shopping websites. For example, in C2C shopping applications, mobile applications have been developed that apply online AI assistants for information retrieval and communication \cite{10.1145/3604571.3604586}. Additionally, in C2C marketplaces, the need for investigating methods to encourage sellers to provide higher quality and more accessible product descriptions and photos has been pointed out \cite{10.1145/3517428.3550390}. Within this research trend, we are proposing a new approach. Specifically, we are advancing research to automate accessibility design for shopping websites aimed at screen reader users using GenAI. This initiative aims to further develop existing research and comprehensively improve the accessibility of entire websites. Unlike previous research that focused on individual elements (such as images, videos, text, etc.), our research aims to optimize the overall structure and content of shopping websites to achieve a seamless and stress-free shopping experience for screen reader users. It is believed that this automation approach can realize high-quality accessibility design while reducing the burden on developers.
\end{comment}



%new version for RR
\subsection{AI-Driven Web Accessibility and Design}
\label{subsecction_2.2}
% \subsubsection{AI Approaches for Enhancing Web Accessibility}
Advancements in Artificial Intelligence (AI) have opened new avenues for improving web accessibility. AI technologies have been utilized to enhance image search capabilities \cite{jing2015visual, zhai2017visual} and generate spatial explanations for complex visuals \cite{zhong2015regionspeak}. Additionally, AI-driven methods facilitate the restructuring of web data tables \cite{10.1145/3491102.3517469} and the conversion of PDF documents to HTML \cite{10.1145/3441852.3476545, wang2021improving}, ensuring better compatibility with assistive technologies. AI-generated image and audio descriptions \cite{huh2023genassist, lei2020mart, 10.1145/3411764.3445347, 10.1145/3357236.3395433, 10.1145/3334480.3382821} further enhance content accessibility, while systems designed to help visually impaired users navigate and understand short videos \cite{van2024making} contribute to a more inclusive online experience. Furthermore, because manual accessibility evaluation and remediation require substantial time and effort \cite{9809423}, AI-driven methods for automatic adjustments have gained attention in recent years. For example, tools have been proposed that use ChatGPT for automatic web accessibility repair \cite{10.1145/3594806.3596542}, automated web accessibility evaluation tools \cite{10.1145/3650212.3652113}, and systems providing context-aware image descriptions \cite{10.1145/3663548.3675658}.

However, LLMs are not without their limitations. They can sometimes generate incorrect or misleading information, a phenomenon known as "hallucination," which can compromise the reliability of accessibility features \cite{bigham2017effects, 10.1145/3663548.3675631}. Additionally, without a user-centered design approach, AI tools may fail to meet the nuanced needs of visually impaired users, leading to solutions that are theoretically sound but practically ineffective. For instance, accessibility overlays—AI-driven tools intended to improve website accessibility may inadvertently creating new barriers by interfering with screen readers and other assistive technologies \cite{10.1007/978-3-031-08645-8_2, 10.1145/3663548.3675650}.

To address these limitations, recent advancements in AI have explored targeted applications to improve accessibility and user experience. For example, studies have investigated how AI tools can support accessibility in specific contexts, such as C2C shopping platforms where accessibility remains a critical concern. These tools help with tasks like information retrieval and communication~\cite{10.1145/3604571.3604586}, as well as improving the accessibility of seller-provided descriptions and photos~\cite{10.1145/3517428.3550390}. Building on these efforts, our research adopts a broader perspective, proposing a GenAI-powered approach to automate accessibility design for shopping websites, focusing specifically on screen reader users. Unlike prior studies that addressed isolated HTML features, such as inserting headings based on visual properties~\cite{10.1145/1866029.1866041} or fixing missing labels in form elements~\cite{10.1145/1414471.1414508}, our method uses LLMs to holistically understand web semantics and apply comprehensive modifications to the HTML structure. By leveraging GenAI, our approach regenerates or reorganizes the HTML of existing websites, creating a clearer information hierarchy and facilitating navigation that aligns with users' natural browsing behaviors. This system is designed to enhance compatibility with screen readers and eliminate information gaps, allowing visually impaired users to gather adequate information even on unfamiliar websites.

% \aya{Building on our proposed solutions, we specifically target challenges in online shopping websites where accessibility remains a critical concern}. 

% \aya{a system to investigate the visual features of text rendered in web browsers \cite{10.1145/1414471.1414508}, a study presenting the finite mixture model (FMM) formulation for label association problems \cite{10.1145/1866029.1866041}} 

% For example, tools have been proposed that use ChatGPT for automatic web accessibility repair \cite{10.1145/3594806.3596542}, automated web accessibility evaluation tools \cite{10.1145/3650212.3652113}, and systems providing context-aware image descriptions \cite{10.1145/3663548.3675658}. These AI-driven approaches have shown some effectiveness in improving web accessibility for users with visual impairments \cite{10.1145/1216295.1216364}.

% \fixme{Although AI serves as a powerful tool for automatically enhancing accessibility design, it is not a default or flawless solution. 

% One problematic example of using AI is the implementation of accessibility overlays that control display elements. These overlays can unintentionally exacerbate existing accessibility issues, such as preventing screen readers from correctly interpreting information~\cite{10.1007/978-3-031-08645-8_2, 10.1145/3663548.3675650}. Additionally, AI's inherent limitations, such as missing information or hallucinations in generating responses, can further impede the effectiveness of screen readers for users with visual impairments~\cite{bigham2017effects, 10.1145/3663548.3675631}. These challenges highlight that while AI can significantly improve accessibility, it must be implemented thoughtfully and in conjunction with user-centered design practices to avoid unintended consequences.}

% Various approaches focusing on supporting visually impaired individuals have been developed, offering solutions to conventional accessibility challenges. Progress in image recognition technology has enabled methods for visually impaired users to interactively explore images by identifying the location of objects within images and providing descriptions for each object. Furthermore, the development of item recommendation systems combining collaborative filtering and content-based ranking is progressing, contributing to enhanced user experiences \cite{kislyuk2015human}.
 % In HCI, diverse approaches are being taken to support visually impaired individuals in web accessibility. 
 
 %  Hara et al. developed an interface combining manual audio description authoring with automatic review and feedback generation based on computer vision and natural language processing \cite{10.1145/3544548.3581023}. This research demonstrated the effectiveness of automated feedback in creating high-quality scripted descriptions. 


%\aya{However, the use of AI in accessibility comes with several challenges. For instance, it remains uncertain how much knowledge of accessibility is required for developers to effectively use AI assistants to generate accessible code \cite{10.1145/3663548.3688513, 10.1145/3597503.3639219}. If developers must explicitly instruct the AI assistant to create accessible code, the advantages would be limited to those users who are already aware of accessibility needs. While it is known that significant programming knowledge is still required to effectively utilize AI-assisted coding, the level of accessibility expertise necessary to fully leverage the assistant's embedded accessibility features is yet to be defined. To effectively harness AI's potential, it is essential to address these critical issues.}

% \subsubsection{Challenges of Using AI for Accessibility}

% The use of accessibility overlays that control display elements can unintentionally worsen existing accessibility issues \cite{10.1007/978-3-031-08645-8_2}. For example, these overlays have been reported to prevent screen readers from correctly reading out information, leading to compatibility issues \cite{10.1145/3663548.3675650}. 
% Furthermore, Adnin et al. point out that while current GenAI tools allow for a basic level of non-visual access, the lack of labeled buttons and appropriate UI design still makes them difficult to use for visually impaired individuals \cite{10.1145/3663548.3675631}. This is particularly evident in situations where important information is ambiguous or hard to find within the UI. 



% \aya{Enhancing accessibility through AI involves several significant challenges. Firstly, accessibility overlays, which use JavaScript to control display elements, can unintentionally exacerbate existing accessibility issues \cite{10.1007/978-3-031-08645-8_2}. These overlays have been found to disrupt the functionality of screen readers, leading to compatibility issues \cite{10.1145/3663548.3675650}.  Additionally, there is uncertainty about the level of accessibility knowledge developers need to effectively use AI assistants to generate accessible code \cite{10.1145/3663548.3688513, 10.1145/3597503.3639219}. If developers must explicitly instruct AI to address accessibility requirements, the benefits may be confined to those who already have a background in accessibility. Addressing these issues from these perspectives is crucial to ensure that AI-driven adjustments truly benefit users.}

%  To ensure that AI genuinely supports a diverse range of user needs, it is important to adopt a human-centered approach to technology design. Other studies \cite{10.1145/3663548.3675650, 10.1145/3597638.3608395} emphasize the importance of actively involving users with diverse abilities in the development and testing processes of AI-based accessibility tools \cite{10.1145/3178855}. In light of these considerations, our study conducted semi-structured interviews to identify the challenges faced by screen reader users on websites \cite{10.1145/3178855}. Through the evaluation of three developed systems, the study advanced user-centered development.



% \subsubsection{Enhancing Accessibility in Online Shopping Website}


% Compared to traditional rule-based algorithms for the correction of HTML labels~\cite{10.1145/1866029.1866041, 10.1145/1414471.1414508}, Large Language Models (LLMs) offer a more flexible and context-aware approach. LLMs can comprehend the semantic structure of web content and generate meaningful modifications, such as accurate alt-text for images or improved content organization, which are essential for screen reader compatibility.}

% individual elements (such as images, videos, text, heading tag etc.), our research aims to optimize the overall structure and content of shopping websites to achieve a seamless and stress-free shopping experience for screen reader users. Through our research, we not only aim to develop automated solutions but also to set a new standard in comprehensive accessibility design. It is believed that this automation approach can realize high-quality accessibility design while reducing the burden on developers.



