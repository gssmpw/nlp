\section{Discussion}
\label{sec:discuss}



Our formative study identified specific mismatches between website design and the browsing behaviors of screen reader users, particularly highlighting the difficulties they face when navigating unfamiliar shopping websites. Although many websites technically adhere to WCAG guidelines, their design still often poses problems for screen reader users. Participants who are blind or have low vision (BLV) frequently encountered confusion and interruptions, caused by a mismatch between HTML tags, labels, and their screen reader's navigation patterns.

Websites may intend to use headings and structure to aid screen reader navigation, but they often overload headings with excessive information, failing to emphasize the key content that users are looking for. This creates a cluttered experience that is not user-friendly. Additionally, websites tend to prioritize visual design over functional HTML organization, resulting in a disorganized content hierarchy that does not match how screen reader users expect to navigate. This disconnect makes it especially challenging for users when they are browsing unfamiliar websites, are new to using screen readers, or are trying to locate specific information needed for decision-making. Shopping websites, in particular, magnify these issues due to their complex layouts, varied content types, and numerous options. When the design does not support clear and intuitive navigation for screen reader users, it leads to a more frustrating and inefficient experience.
Based on our formative study findings, we developed a pipeline that uses LLMs to regenerate and reorganize the HTML structure of existing websites. This LLM-based approach significantly improved the information hierarchy and labeling, and all participants preferred the revised websites, finding them better aligned with screen reader navigation patterns. We reflect on our findings to discuss limitations and opportunities for future work:


\subsection{Handling LLM’s Info Drift for Better Web Accessibility}
While large language models (LLMs) can effectively revise and improve accessibility design, they often face the common challenge of \textbf{information drift}~\cite{8496795}—a tendency to unintentionally add or omit details from the original content when generating new outputs. In the context of regenerating HTML, this challenge may result in missing content or the inclusion of unintended details in the output. To address this challenge, we implemented two strategies: (1) refining prompts and incorporating a similarity check function to compare the generated output with the original website, and (2) limiting changes to only HTML tags to avoid altering the visual text and functionality.

For strategy (1), although participants perceived the three versions—original, regenerated HTML, and reorganized HTML tags—as very similar or identical, researchers identified some missing details in the generated version through a thorough comparison with the original. For example, the regenerated version might omit certain products if the page is too long. In another case, a search results page that originally displays only product names and prices might have the regenerated version include additional details, such as product condition (e.g., ``like new''), extracted from HTML metadata and added to the visible text. For strategy (2), we discussed the limitations of accessibility improvements in the Results section (see Section \ref{sec:phase3}), as this approach does not allow for changes to the website's layout or the sequence of information sections.

These inconsistencies could create challenges for screen reader users if the extension is applied directly to live websites. However, if developers use this pipeline for testing before publishing the website, these issues can be mitigated through automated or manual checks. Participants also expressed a preference for website hosts to address accessibility barriers directly, rather than requiring screen reader users to rely on additional solutions, like virtual assistants, to navigate these issues. Additionally, the system interface could be enhanced to clearly highlight new information added in the output and any missing details compared to the original. This would allow developers to review these differences, choose the LLM-generated version that best meets their needs, and make further adjustments as necessary. Future systems or research could focus on designing interfaces that better support developers in utilizing this pipeline. For instance, future work could explore real-time feedback mechanisms that provide accessibility suggestions and quick fixes, allowing developers to continuously refine their systems and enhance their adaptability across different browsing environments. Specifically, the pipeline could be used to generate or reorganize a developer's HTML product and visualize the changes and reasoning in a sidebar, enabling developers to make informed decisions. This system could then be further evaluated and improved based on developer feedback. By extending the pipeline’s capabilities in these ways, the tool could become both a practical solution for developers looking to create more inclusive web experiences.
% While Large language model can handle the revision excellently and make significant improvements on accessibility design, in regenerated HTML they can add or miss some information from original website in their generated output. We have tried to combat this challenge by (1) adjusting prompt and add similarity check function between generated output and original website. (2) only adjust the HTML tags on the website to avoid the change of other visual text and functions. But for (1), although all participants perceived very similar or the same among the three versions (original website, regenerated HTML website and reorganized HTML tags website), researchers still noticed some omited details in the generated version in a thoughral comparision with original website. For example, the regenerated version might miss some products if the page is too long. In another case, on the search results page, the original website may display only the product name and price, but the regenerated version could extract additional details, such as the product condition (e.g., ``like new''), from the HTML metadata and add it to the visible text. These inconsistencies could pose problems if screen reader users directly use the extension on live websites. However, for developers using this pipeline for testing, it could provide an opportunity to explore different ways to visualize these differences and select the most suitable version generated by the LLM. 
% \subsection{Interaction Improvements}
% Further development of this pipeline could involve creating customizable interfaces that allow both end users and developers to set specific parameters for HTML restructuring, based on their unique needs and constraints. Additionally, future work could explore real-time user feedback mechanisms to continuously refine the system, enhancing its adaptability and effectiveness in diverse browsing environments. By extending the pipeline’s capabilities in these directions, the tool could serve as both a practical solution for users and a comprehensive resource for developers aiming to create more inclusive web experiences.



% However, the regenerated HTML version also presented challenges that offset some of the excellent improvements. The first challenge is the similarity to the original website. Although participants perceived a high similarity among the three versions, it is difficult to generate an exact replica of the original website, including precise visual text and identical functions. While essential functions like search, login, and purchase are present in the regenerated website, it may randomly omit some details. For example, the regenerated version might miss some products if the page is too long. In another case, on the search results page, the original website may display only the product name and price, but the regenerated version could extract additional details, such as the product condition (e.g., ``like new''), from the HTML metadata and add it to the visible text. These inconsistencies could pose problems if screen reader users directly use the extension on live websites. However, for developers using this pipeline for testing, it could provide an opportunity to explore different ways to visualize these differences and select the most suitable version generated by the LLM. 
% Further development of this pipeline could involve creating customizable interfaces that allow both end users and developers to set specific parameters for HTML restructuring, based on their unique needs and constraints. Additionally, future work could explore real-time user feedback mechanisms to continuously refine the system, enhancing its adaptability and effectiveness in diverse browsing environments. By extending the pipeline’s capabilities in these directions, the tool could serve as both a practical solution for users and a comprehensive resource for developers aiming to create more inclusive web experiences.
\subsection{Broadening GenAI Accessibility Enhancements to Diverse Websites}
Our study primarily explored how GenAI can be used to improve website design to better match screen reader users' browsing patterns, particularly in shopping scenarios. However, the pipeline we developed, which leverages GenAI to dynamically regenerate and reorganize HTML structures via a browser extension, has the potential to be applied across a broader range of websites beyond just shopping platforms. Many of the accessibility challenges identified in shopping websites—such as poor information hierarchy, irrelevant or missing headings, and navigation inefficiencies—are common across different types of websites, including news platforms, educational resources, and service-based sites. By applying our GenAI-powered pipeline to these varied contexts, we could address similar issues, such as enhancing the logical flow of content, improving section labeling, and ensuring important information is easily accessible to screen reader users. Future work could involve adapting and testing our pipeline on different types of websites to identify context-specific accessibility barriers and refine the GenAI model’s ability to address them. For example, on a news website, the system could focus on reorganizing article sections, headlines, and sidebars to make news consumption more efficient. On educational sites, the pipeline could improve the layout of course materials and resources to better support learners using screen readers. This broader application could significantly impact the overall web experience for screen reader users, moving beyond niche applications to provide more comprehensive, web-wide solutions.

\subsection{Limitations and Future Work}
\label{sec:limitation}
We recognize that our work also has certain limitations. Our sample size for interviews and evaluations is relatively small due to the challenges of recruiting BLV (blind and low vision) participants who use screen readers and shop online. However, we achieved data saturation and obtained valuable real-life insights and feedback from the participants. We suggest further research with a larger and more diverse group of screen reader users to validate our findings and explore additional accessibility improvements.




% Our evaluation was carried out with only six participants in the formative study and seven participants in the evaluation, due to the difficulty in participant recruitment from a limited pool of potential BLV participants who use screen readers and shop online. We recognize that our sample size might be insufficient for a quantitative evaluation of our method. However, we believe that even with such a small group of test participants we could gain valuable qualitative insights. We plan to conduct a study with a larger group of participants in the near future. 

We also recognize some technical limitations of our current system. In the LLM-regenerated outputs, these dynamic elements are sometimes omitted or rendered non-functional, leading to reduced usability. Future work should focus on addressing these challenges by improving the integration of dynamic JavaScript functionalities in LLM-regenerated outputs. Furthermore, while our solution works for web browser plugins, certain shopping platforms require users to shop via native mobile applications, which can be inaccessible to the modification method presented in our paper. Furthermore, although image description generation is out of this study scope, a holistic solution to web browsing experience should provide some form of image recognition in addition to our method. Finally, a commercial or production version of the application should also consider request processing times and inferred computing and maintenance costs related to the security and usage of LLM API. 