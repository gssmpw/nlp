\section{Related Work}
As we propose the tasks of detecting and tracing LLM code paraphrasing, 
we have not found directly related works. 
Instead, we conducted a literature survey on detecting LLM-generated code, 
a binary classification task that determines whether a given code is human-written or 
LLM-generated.
\citet{lee2024wrote} divided the vocabulary in code generation through LLM 
into a Green/Red list and inserted detectable patterns in LLM-generated code by 
promoting the generation of tokens belonging to the Green list.
\citet{yang2023zero} proposed DetectGPT4Code, a modification of the existing 
machine-generated text detection method, 
DetectGPT~\citep{mitchell2023detectgpt}, using a code-specialized language model. 
\citet{wang2023evaluating} evaluated the effectiveness of existing 
machine-generated text detectors in detecting machine-generated code.
They reported that the existing detectors exhibit degraded performance 
in identifying machine-generated code in contrast to 
their performance in detecting machine-generated natural language text.
These studies focused on detecting LLM-generated code derived from 
natural language descriptions 
or function headers and did not address the detection of paraphrasing between 
human-written and LLM-generated code.
\citet{krishna2023paraphrasing} reported that applying paraphrasing to 
LLM-generated text can bypass existing detection methods. 
Therefore, studying the detection of LLM paraphrasing
is necessary to enhance the robustness of LLM-generated code detection methods.