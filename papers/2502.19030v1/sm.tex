
\begin{center}
\vspace*{12pt}
{\Large Supplementary Materials for:\\
\vspace{12pt}
Sampling nodes and hyperedges via random walks on large hypergraphs
}
\vspace{12pt} \\
\end{center}

\setcounter{figure}{0}
\setcounter{table}{0}
\setcounter{section}{0}

\renewcommand{\thesection}{S\arabic{section}}
\renewcommand{\thefigure}{S\arabic{figure}}
\renewcommand{\thetable}{S\arabic{table}}
\renewcommand{\theequation}{S\arabic{equation}}

\begin{center}
\author{Kazuki Nakajima, Masanao Kodakari, and Masaki Aida}
\vspace{24pt} \\
\end{center}

\section{Simulation results for other hypergraphs}

We show the simulation results for other hypergraphs.
Figure \ref{fig:s1} shows the number of queries generated to hyperedges as a function of $r$.
Figure \ref{fig:s2} shows the repetition rate of hyperedge samples as a function of $r$.
Figure \ref{fig:s3} shows the NRMSE of the estimator of the average node degree as a function of $r$.
Figure \ref{fig:s4} shows the NRMSE of the estimator of the node degree distribution as a function of $r$.
Figure \ref{fig:s5} shows the NRMSE of the estimator of the average hyperedge size as a function of $r$.
Figure \ref{fig:s6} shows the NRMSE of the estimator of the hyperedge size distribution as a function of $r$.
Figure \ref{fig:s7} shows the NRMSE of the estimator of the probability that a node has degree $d$ as a function of $d$ for $r = 10^4$. 
Figure \ref{fig:s8} shows the NRMSE of the estimator of the probability that a hyperedge has size $s$ as a function of $s$ for $r = 10^4$. 

\begin{figure}[t]
    \centering
    \includegraphics[width=1.0\linewidth]{fig_s1.pdf}\\
    \caption{Comparison of the number of queries to hyperedges. (a) MAG-geology hypergraph. (b) MAG-history hypergraph. (c) Amazon hypergraph. (d) stack-overflow hypergraph.}
\label{fig:s1}
\end{figure}

\begin{figure}[t]
    \centering
    \includegraphics[width=1.0\linewidth]{fig_s2.pdf}\\
    \caption{Comparison of the repetition rate of hyperedge samples. (a) MAG-geology hypergraph. (b) MAG-history hypergraph. (c) Amazon hypergraph. (d) stack-overflow hypergraph.}
\label{fig:s2}
\end{figure}

\begin{figure}[t]
    \centering
    \includegraphics[width=1.0\linewidth]{fig_s3.pdf}\\
    \caption{NRMSE of the estimator of the average node degree as a function of $r$. (a) MAG-geology hypergraph. (b) MAG-history hypergraph. (c) Amazon hypergraph. (d) stack-overflow hypergraph.}
\label{fig:s3}
\end{figure}

\begin{figure}[t]
    \centering
    \includegraphics[width=1.0\linewidth]{fig_s4.pdf}\\
    \caption{NRMSE of the estimator of the node degree distribution as a function of $r$. (a) MAG-geology hypergraph. (b) MAG-history hypergraph. (c) Amazon hypergraph. (d) stack-overflow hypergraph.}
\label{fig:s4}
\end{figure}

\begin{figure}[t]
    \centering
    \includegraphics[width=1.0\linewidth]{fig_s5.pdf}\\
    \caption{NRMSE of the estimator of the average hyperedge size as a function of $r$. (a) MAG-geology hypergraph. (b) MAG-history hypergraph. (c) Amazon hypergraph. (d) stack-overflow hypergraph.}
\label{fig:s5}
\end{figure}

\begin{figure}[t]
    \centering
    \includegraphics[width=1.0\linewidth]{fig_s6.pdf}\\
    \caption{NRMSE of the estimator of the hyperedge size distribution as a function of $r$. (a) MAG-geology hypergraph. (b) MAG-history hypergraph. (c) Amazon hypergraph. (d) stack-overflow hypergraph.}
\label{fig:s6}
\end{figure}

\begin{figure}[t]
    \centering
    \includegraphics[width=0.85\linewidth]{fig_s7.pdf}\\
    \caption{NRMSE of the estimator of the probability that a node has degree $d$ for $r = 10^4$. (a) and (b) MAG-geology hypergraph. (c) and (d) MAG-history hypergraph. (e) and (f) Amazon hypergraph. (g) and (h) stack-overflow hypergraph.}
\label{fig:s7}
\end{figure}

\begin{figure}[t]
    \centering
    \includegraphics[width=0.85\linewidth]{fig_s8.pdf}\\
    \caption{NRMSE of the estimator of the probability that a hyperedge has size $s$ for $r = 10^4$. (a) and (b) MAG-geology hypergraph. (c) and (d) MAG-history hypergraph. (e) and (f) Amazon hypergraph. (g) and (h) stack-overflow hypergraph.}
\label{fig:s8}
\end{figure}

