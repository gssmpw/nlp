\documentclass{article}

% Recommended, but optional, packages for figures and better typesetting:
\usepackage{microtype}
\usepackage{graphicx}
% \usepackage{subfigure}
\usepackage{subfig}
% \usepackage[subfigure]
\usepackage{booktabs} % for professional tables

% hyperref makes hyperlinks in the resulting PDF.
% If your build breaks (sometimes temporarily if a hyperlink spans a page)
% please comment out the following usepackage line and replace
% \usepackage{icml2025} with \usepackage[nohyperref]{icml2025} above.
% \usepackage{hyperref}
\usepackage{mathtools}
% \usepackage{titletoc}
% Attempt to make hyperref and algorithmic work together better:
% \newcommand{\theHalgorithm}{\arabic{algorithm}}

% Use the following line for the initial blind version submitted for review:
% \usepackage{icml2025}

% If accepted, instead use the following line for the camera-ready submission:
% \usepackage{icml2025}
\usepackage[accepted]{icml2025}
% 在导言区添加以下参数控制
\usepackage{float}
\setlength{\textfloatsep}{5pt}  % 文本与浮动体间距
\setlength{\floatsep}{3pt}      % 浮动体之间间距
% For theorems and such
\usepackage{bm}
\newcommand{\bigparallel}{\parallel}  % Or define it however you want

\usepackage{amsmath}
\usepackage{amssymb}
\usepackage{mathtools}
\usepackage{amsthm}
\usepackage[algo2e,ruled,vlined]{algorithm2e}
\usepackage{multirow}
\usepackage{colortbl}
\usepackage{microtype}
\usepackage[american]{babel}
\usepackage{wrapfig}
\newcommand{\ie}{{\emph{i.e.}}}
\newcommand{\eg}{{\emph{e.g.}}}
\newcommand{\wrt}{{\emph{w.r.t.}}}
\usepackage{mdframed}
\newmdenv[
  topline=false,
  bottomline=false,
  rightline=false,
  linewidth=2pt,
  linecolor=black,
  leftmargin=0pt,
  rightmargin=0pt,
  innertopmargin=5pt,
  innerbottommargin=5pt,
  innerrightmargin=0pt,
  innerleftmargin=10pt,
  skipabove=\medskipamount,
  skipbelow=2pt
]{findingbox}


%%%%%%%%%%%%%%%%%%%%%%%%%%%%%%%%
% THEOREMS
%%%%%%%%%%%%%%%%%%%%%%%%%%%%%%%%
\theoremstyle{plain}
\newtheorem{theorem}{Theorem}[section]
\newtheorem{proposition}[theorem]{Proposition}
\newtheorem{lemma}[theorem]{Lemma}
\newtheorem{corollary}[theorem]{Corollary}
\theoremstyle{definition}
\newtheorem{definition}[theorem]{Definition}
\newtheorem{assumption}[theorem]{Assumption}
\theoremstyle{remark}
\newtheorem{remark}[theorem]{Remark}

% 设置超链接颜色
% \usepackage[dvipsnames]{xcolor}
\definecolor{CColor}{rgb}{0.01,0.31,0.59}
\definecolor{GGray}{rgb}{0.80,0.90,1}
\definecolor{Shady}{rgb}{0.9,0.9,0.9}
\definecolor{kaistblue}{RGB}{20,135,200}
\definecolor{kaistdarkblue}{RGB}{0,65,145}
\definecolor{urbanablue}{RGB}{19,41,75}
\definecolor{urbanaorange}{RGB}{232,74,39}
\definecolor{drp}{rgb}{0.53,0.15,0.34}
\usepackage[plainpages=false,pdfpagelabels,colorlinks=true,linkcolor=CColor,citecolor=CColor,urlcolor=CColor]{hyperref}


% 设置目录
\usepackage{titletoc}
\usepackage[subfigure]{tocloft}
% Adjust the lengths to control the spacing
\setlength{\cftbeforesecskip}{0.2em}  % Adjust the space before sections
\setlength{\cftbeforesubsecskip}{0.2em}  % Adjust the space before subsections
\setlength{\cftbeforesubsubsecskip}{0.2em}  % Adjust the space before subsubsections

% if you use cleveref..
\usepackage[capitalize,noabbrev]{cleveref}
\crefname{section}{Sec.}{Secs.}
\Crefname{section}{Section}{Sections}
\Crefname{table}{Table}{Tables}
\crefname{table}{Tab.}{Tabs.}
\crefname{equation}{Eq.}{Eqs.}
\crefname{algorithm}{Alg.}{Algs.}
\crefname{figure}{Fig.}{Figs.}
\crefname{appendix}{App.}{Apps.}


\SetCommentSty{myCommentStyle}
\newcommand\myCommentStyle[1]{\footnotesize\ttfamily\textcolor{blue}{#1}}
% Todonotes is useful during development; simply uncomment the next line
%    and comment out the line below the next line to turn off comments
%\usepackage[disable,textsize=tiny]{todonotes}
\usepackage[textsize=tiny]{todonotes}

\begin{document}

\twocolumn[
\icmltitle{Mix Data or Merge Models? Balancing the Helpfulness, Honesty, and Harmlessness of Large Language Model via Model Merging}

% It is OKAY to include author information, even for blind
% submissions: the style file will automatically remove it for you
% unless you've provided the [accepted] option to the icml2025
% package.

% List of affiliations: The first argument should be a (short)
% identifier you will use later to specify author affiliations
% Academic affiliations should list Department, University, City, Region, Country
% Industry affiliations should list Company, City, Region, Country

% You can specify symbols, otherwise they are numbered in order.
% Ideally, you should not use this facility. Affiliations will be numbered
% in order of appearance and this is the preferred way.
% \icmlsetsymbol{equal}{*}

\begin{icmlauthorlist}
\icmlauthor{Jinluan Yang}{ant_intern,zju}
\icmlauthor{Dingnan Jin}{comp}
\icmlauthor{Anke Tang}{whu}
\icmlauthor{Li Shen}{personal}
\icmlauthor{Didi Zhu}{zju}
\icmlauthor{Zhengyu Chen}{zju}
\icmlauthor{Daixin Wang}{comp}
\icmlauthor{Qing Cui}{comp}
\icmlauthor{Zhiqiang Zhang}{comp}
\icmlauthor{Jun Zhou}{comp}
\icmlauthor{Fei Wu}{zju}
\icmlauthor{Kun Kuang}{zju}

\end{icmlauthorlist}
\icmlaffiliation{ant_intern}{This work was supported by Ant Group Research Intern Program}
\icmlaffiliation{zju}{Zhejiang University}
\icmlaffiliation{whu}{Whuhan University}
\icmlaffiliation{personal}{}
\icmlaffiliation{comp}{Ant Group}


% \icmlcorrespondingauthor{Kun Kuang}{kunkuang@zju.edu.cn}

% You may provide any keywords that you
% find helpful for describing your paper; these are used to populate
% the "keywords" metadata in the PDF but will not be shown in the document
\icmlkeywords{Machine Learning, ICML}

\vskip 0.3in
]

% this must go after the closing bracket ] following \twocolumn[ ...

% This command actually creates the footnote in the first column
% listing the affiliations and the copyright notice.
% The command takes one argument, which is text to display at the start of the footnote.
% The \icmlEqualContribution command is standard text for equal contribution.
% Remove it (just {}) if you do not need this facility.

%\printAffiliationsAndNotice{}  % leave blank if no need to mention equal contribution
\printAffiliationsAndNotice{} % otherwise use the standard text.

\newcommand{\xmark}{\text{\sffamily X}}

\begin{abstract}


Achieving balanced alignment of large language models (LLMs) in terms of Helpfulness, Honesty, and Harmlessness (3H optimization) constitutes a cornerstone of responsible AI, with existing methods like data mixture strategies facing limitations including reliance on expert knowledge and conflicting optimization signals. While model merging offers a promising alternative by integrating specialized models, its potential for 3H optimization remains underexplored.
This paper establishes the first comprehensive benchmark for model merging in 3H-aligned LLMs, systematically evaluating 15 methods (12 training-free merging and 3 data mixture techniques) across 10 datasets associated with 5 annotation dimensions, 2 LLM families, and 2 training paradigms. Our analysis reveals three pivotal insights: (i) previously overlooked collaborative/conflicting relationships among 3H dimensions, (ii) the consistent superiority of model merging over data mixture approaches in balancing alignment trade-offs, and (iii) the critical role of parameter-level conflict resolution through redundant component pruning and outlier mitigation. Building on these findings, we propose R-TSVM, a \textbf{R}eweighting-enhanced \textbf{T}ask \textbf{S}ingular \textbf{V}ector \textbf{M}erging method that incorporates outlier-aware parameter weighting and sparsity-adaptive rank selection strategies adapted to the heavy-tailed parameter distribution and sparsity for LLMs, further improving LLM alignment across multiple evaluations. Our models will be available at \href{https://huggingface.co/Jinluan}{3H\_Merging}.

\end{abstract}

\section{Introduction}

\begin{figure}[h]
    \centering
    \begin{overpic}[trim=0cm 0cm 0cm 0cm,clip,angle=0,origin=c,width=.4\linewidth]{images/teaser_absolute.png}
        %  trim={<left> <lower> <right> <upper>}
        %  \put(horiz, vert)
        %  \put(horiz, vert){\rotatebox{90}{Text}}
        %
        \put(107, 32){$\mathbf{\to}$}
    \end{overpic}\hspace{1cm}
    \begin{overpic}[trim=0cm 0cm 0cm 0cm,clip,angle=0,origin=c,width=.4\linewidth]{images/teaser_translated_yellow.png}
        %  trim={<left> <lower> <right> <upper>}
        %  \put(horiz, vert)
        %  \put(horiz, vert){\rotatebox{90}{Text}}
        %
    \end{overpic}
    \caption{Using translation methods, a controller trained on an environment with a given visual variation \textit{(left)} can be reused without any training or fine-tuning on a different environment (\textit{right}) with comparable performance. In red we see the trajectory of a car driven by the same controller when connected to two different encoders, one for each visual variation.
    }
    \label{fig:teaser}
\end{figure}

Deep Reinforcement Learning (RL) has enabled agents to achieve remarkable performance in complex decision-making tasks, from robotic manipulation to high-dimensional games (Mnih et al., 2015; Silver et al., 2017). 
Although recent RL techniques achieved strong improvements over sample efficiency \citep{yarats2021drqv2, kostrikov2020image}, training new agents remains a costly process, both in computational and temporal terms.
Despite these advances, most methods still require at least partial retraining when dealing with domain shifts such as visual appearance, reward functions, or action spaces \citep{pmlr-v97-cobbe19a, zhang2020learning}. These domain changes typically require expensive retraining, which can be prohibitive for real-world settings that require millions of interactions.

A variety of approaches have been proposed to address these shifting conditions. Domain randomization \citep{tobin2017domain, sadeghi2016cad2rl} trains agents across diverse visual styles or physics settings, promoting invariant features but demanding broader coverage of possible variations. Multi-task RL \citep{parisotto2015actor, teh2017distral} attempts to learn shared representations across multiple tasks.

In the supervised setting, recent representation learning techniques \citep{Moschella2022-yf,maiorca2023latent, norelli2022b, cannistraci2023bricks}, show that it is possible to zero-shot recombine encoders and decoders to perform new tasks across different modalities (images, text..) and tasks (classification, reconstruction) and even architectures.
In RL, methods adopting the relative representation framework \citep{Moschella2022-yf} have shown promising results in adapting encoders to different controllers with zero or few-shots adaptation, for robotic control from proprioceptive states \citep{jian2021adversarial} or for playing games in the Gymnasium suite \citep{towers2024gymnasium} from pixels \citep{ricciardi2025r3lrelativerepresentationsreinforcement}.
These methods, however, still require training models to use the new relative representations.

By contrast, \cite{maiorca2023latent} suggest that modules from independently trained neural networks can be connected via a simple linear or affine transformation, with no training constraint or fine-tuning required, if such transformations can be reliably estimated from a small set of “anchor” samples, pairs of states or observations deemed semantically equivalent.

Our main contribution is the implementation of a RL method based on semantic alignment to map between latent spaces of different neural models, so that their encoders and controllers can be stitched with the goal of creating new agents that can act on visual-task combinations never seen together in training. This includes the use of the transformations to map modules from different networks, and the collection of anchor samples used to estimate these transformations. We call our method Semantic Alignment for Policy Stitching (\textbf{SAPS}).
We perform analyses and empirical tests on the CarRacing and LunarLander environments to show the performance of new agents created via zero-shot stitching of encoders and controllers trained on different visual-task variations, demonstrating significant gains compared to existing zero-shot methods.
\section{Related Work}
\label{sec:related}


\noindentbold{2D visual foundation models}
In recent years, we have witnessed the emergence of large pretrained models—so-called foundation models that are trained on large-scale datasets and serve as a \textit{foundation} for many downstream tasks.
These models demonstrate remarkable versatility across multiple modalities, including language~\cite{team2023gemini,touvron2023llama,touvron2023llama2,dubey2024llama3,vicuna2023,radford2019language,brown2020language,chung2024scaling,achiam2023gpt,bai2023qwen,yang2024qwen2,jiang2023mistral,jiang2024mixtral}, vision~\cite{sam,ravi2024sam,dino_v1,oquab2023dinov2,zou2024segment,rombach2022high,ho2020denoising,nichol2021improved,songdenoising,songscore}, audio~\cite{deshmukh2023pengi,zhang2023speechgpt,rubenstein2023audiopalm,borsos2023audiolm}. 
Furthermore, they enable multi-modal reasoning capabilities that bridge across different modalities~\cite{girdharImageBindOneEmbedding2023,Qwen-VL,llava,radfordLearningTransferableVisual2021,jia2021scaling,team2024gemini}.
Among these models, those that operate on visual modalities are known as visual foundation models (VFM).
VFMs excel in various computer vision tasks such as image segmentation~\cite{sam,ravi2024sam,zou2024segment,zou2023generalized,cheng2021per,cheng2022masked,jain2023oneformer,li2024semantic}, object detection~\cite{liu2023grounding,carion2020end}, representation learning~\cite{dino_v1,oquab2023dinov2}, and open-vocabulary understanding~\cite{radfordLearningTransferableVisual2021,li2022language,ghiasi2022scaling,ram,ram_pp,yu2023convolutions,kang2024defense,naeem2024silc,cho2024cat}.
When integrated with large language models, they enable sophisticated visual reasoning and natural language interactions~\cite{llava,Qwen-VL,girdharImageBindOneEmbedding2023,team2024gemini,guo2024regiongpt,yuan2024osprey,you2023ferret}.
We use such vision language models to construct open vocabulary segmentation and captions for point clouds based on multiview images.







\noindentbold{Open-vocabulary 3D segmentation}
Building on the success of 2D VFMs, recent work have extended open-vocabulary capabilities to 3D scene understanding.
OpenScene~\cite{Peng2023OpenScene} first introduced zero-shot 3D semantic segmentation by distilling knowledge from language-aligned image encoders~\cite{li2022language,ghiasi2022scaling}.
Subsequent methods~\cite{ding2022pla,yang2024regionplc,jiang2024open} leverage multiview images to generate textual captions, which then serve as training supervision.
However, these methods face challenges in generating high-quality 3D mask-text pairs at scale.
For open-vocabulary 3D instance segmentation, existing methods~\cite{takmaz2023openmask3d,nguyen2024open3dis,huang2024openins3d} typically rely on closed-vocabulary proposal networks such as Mask3D~\cite{schult2023mask3d}, which inherently constrains their ability to detect novel object categories. 
Moreover, these methods leverage 2D VFMs like CLIP~\cite{radfordLearningTransferableVisual2021} for region classification by projecting 3D regions onto multiple 2D views.
This approach requires both 2D images and 3D point clouds during inference. Additionally, it necessitates multiple inferences of large 2D models on projected masks, resulting in high computational costs. 
We address these limitations by developing the first single-stage open-vocabulary 3D instance segmentation model that operates directly in 3D without ground truth labels, using our \dataname dataset and Segment3D~\cite{huang2024segment3d} proposals.

\noindentbold{3D vision-language datasets}
Several datasets align 3D scenes with textual annotations to facilitate language-driven 3D understanding. 
ScanRefer~\cite{chen2020scanrefer}, ReferIt3D~\cite{achlioptas2020referit_3d} and EmbodiedScan~\cite{wangEmbodiedScanHolisticMultiModal2023} provide fine-grained object-level localization through detailed referential phrases, while ScanQA~\cite{azuma2022scanqa} targets spatially grounded question-answering. 
In contrast, SceneVerse~\cite{jiaSceneVerseScaling3D2024} and MMScan~\cite{lyu2024mmscan} employ large-language models or vision-language models to partially automate annotation.
Despite leveraging advanced models, these datasets depend significantly on costly human annotations derived from closed-vocabulary sources, limiting their support for open-vocabulary and scalability for large-scale 3D segmentation tasks.

\section{An Overview of \our}
\label{sec:preliminary}

This section presents an overview of \our from a probabilistic modeling perspective. We first examine how traditional reasoning chains work and then introduce our dependency-based graph structures and their contraction mechanisms to enhance the modeling capability of reasoning processes. 

\subsection{Reasoning Chain}
Chain-of-Thought (CoT) prompting enables LLMs to progressively propose intermediate thoughts $T_i$ when solving a problem. As discussed earlier, this approach requires maintaining a complete reasoning history, which can be formalized as a probabilistic sampling procedure:
\begin{align}  
A \sim p(A|\mathcal{T}, Q_0) \prod_{i=0}^N p(T_i|\mathcal{T}_{<i}, Q_0)  
\end{align}
Here, $\mathcal{T} = \{T_0, T_1, \dots, T_N\}$ represents the sequence of intermediate thoughts generated by the LLM. Each thought $T_i$ depends on the previous thoughts $\mathcal{T}_{<i}$ and the initial question $Q_0$.

To explore chain-based methods with different node definitions, Least-to-Most~\cite{zhou2023least} replaces the intermediate thoughts $T_i$ with subquestions $Q_i$, resulting in a different formulation of the reasoning chain:
\begin{align}  
A \sim p(A|\mathcal{Q}) \prod_{i=0}^N p(Q_i|\mathcal{Q}_{<i})  
\end{align}  
where $\mathcal{Q} = \{Q_0, Q_1, \dots, Q_N\}$ is the sequence of subquestions.

In an ideal scenario where the reasoning chain $\mathcal{Q}$ exhibits the Markov property, each subquestion $Q_{i+1}$ would only depend on its immediate predecessor $Q_i$, similar to how humans naturally solve complex problems by resolving independent subquestions and reformulating simplified states. This leads to:
\begin{align}
A \sim p(A|Q_N) \prod_{i=0}^N p(Q_{i+1}|Q_i)
\end{align}
However, achieving true Markov property in real-world reasoning tasks is challenging. We adopt the subquestion-based node structure from reasoning chains as states while exploring a two-phase state transition mechanism consisting of decomposition and contraction to address this challenge.

\subsection{Dependency Directed Acyclic Graph}
\our utilizes temporary DAG structures to decompose the current question, unlike existing methods that maintain complex dependencies throughout the reasoning process. This DAG structure serves as a scaffold during state transitions, providing rich structural information to guide the complete state transition process, specifically functioning as the decomposition phase to facilitate the subsequent contraction phase.

The DAG $\mathcal{G}$ is defined as:

\begin{align}
\mathcal{G} = (\mathcal{Q}, E), \quad \mathcal{Q} = \{Q_i\}_{i=1}^n , \quad E \subseteq \mathcal{Q} \times \mathcal{Q} 
\end{align}

In our DAG definition, nodes represent subquestions $Q_i$, and edges $(Q_j, Q_i)$ indicate that $Q_j$ contains necessary information for solving $Q_i$. A major challenge in constructing Markov processes stems from the dependencies of various information in complex reasoning scenarios, and this definition provides structural information for identifying dependencies through rule-based determination.

Based on their dependency relationships, all subquestion nodes can be categorized into two types:

Independent subquestions $\mathcal{Q}_{\text{ind}}$ (nodes without incoming edges):

\begin{align}
\mathcal{Q}_{\text{ind}} = \{Q_i \in \mathcal{Q} \mid \nexists Q_j \in \mathcal{Q}, (Q_j, Q_i) \in E\}
\end{align}

Dependent subquestions $\mathcal{Q}_{\text{dep}}$ (nodes with incoming edges):

\begin{align}
\mathcal{Q}_{\text{dep}} = \{Q_i \in \mathcal{Q} \mid \exists Q_j \in \mathcal{Q}, (Q_j, Q_i) \in E\}
\end{align}

The key assumption of acyclicity in our DAG is guaranteed by this edge definition: since subquestions are generated following natural language order, any subquestion $Q_i$ can only depend on previously generated subquestions $Q_{<i}$. Even in the maximally connected case where each subquestion links to all its predecessors, acyclicity is maintained, as any additional edges would create cycles by connecting to future nodes while violating the natural language order.

\subsection{Contraction}
The contraction phase transforms the temporary DAG structure into the next atomic state while preserving the Markov property. To ensure this Markov process is meaningful, we must maintain state atomicity while ensuring progress in the reasoning process. As the reasoning progresses, new conclusions and information are continuously derived, necessitating the selective discarding of information to maintain atomic states. \our addresses this by treating results from $\mathcal{Q}_{\text{ind}}$ as either given conditions or eliminated process information, while contracting $\mathcal{Q}_{\text{dep}}$ into an independent question as the next state. This contracted question maintains solution equivalence to $Q_i$, ensuring the reasoning process stays on track.

The reasoning process is formally described in Algorithm \ref{algo:main}, which shows how \our iterates through decomposition and contraction steps. This iterative process continues until it reaches a maximum number $D$, which is assigned by the depth of the first generated graph $\mathcal{G}_0$ to prevent infinite decomposition. The process can be formalized as:

\begin{align}  
A \sim p(A|Q_D) \prod_{i=0}^D p(Q_{i+1}|\mathcal{G}_i)\  p(\mathcal{G}_i|Q_i)  
\end{align}


\begin{algorithm}[t]
\caption{Algorithm of \our}
\begin{algorithmic}[1]
\label{algo:main}
\small
\REQUIRE Initial question $Q_0$
\ENSURE Final answer $A$
\STATE Iteration counter $i \gets 0$
\STATE max depth $D \gets \text{None}$
\WHILE{$i < D$ or $D$ is None}
    \STATE $\mathcal{G}_i \gets \texttt{decompose}_{\text{LLM}}(Q_i)$ \\ // Generate dependency DAG
    \IF{$D$ is None}
        \STATE $D \gets \texttt{GetMaxPathLength}(\mathcal{G}_i)$ \\ // Rule-based path length calculation
    \ENDIF
    \STATE $\mathcal{Q}_{ind} \gets \{Q_i \in \mathcal{Q} \mid \nexists Q_j \in \mathcal{Q}, (Q_j, Q_i) \in E\}$
    \STATE $\mathcal{Q}_{dep} \gets \{Q_i \in \mathcal{Q} \mid \exists Q_j \in \mathcal{Q}, (Q_j, Q_i) \in E\}$
    \STATE $Q_{i+1} \gets \texttt{contract}_{\text{LLM}}(\mathcal{Q}_{ind}, \mathcal{Q}_{dep})$ \\ // Contract subquestions into a independent question
    \STATE $i \gets i + 1$
\ENDWHILE
\STATE $A \gets \texttt{solve}_{\text{LLM}}(Q_D)$ \\ // Generate final answer
\RETURN $A$
\end{algorithmic}
\end{algorithm}

\section{Experiments}\label{sec:experiments}
We now evaluate SAPS using both qualitative and quantitative analyses. We first compare its zero-shot performance to R3L on benchmark tasks, then 
%perform delve into ablation studies and
an analysis of how our alignment approach behaves under different conditions.

\paragraph{Environments}
Our agents act by receiving pixel images as input observation, consisting of four consecutive $84 \times 84$ RGB images, stacked along the channel dimension to capture dynamic information such as velocity and acceleration.
We consider environments where we can freely change visual features (background color, camera perspective) or task (rewards, dynamics), therefore we use CarRacing \citep{klimov2016carracing} and LunarLander as both implemented in R3L.
CarRacing requires the agent to drive in a track using pixel observations, whose variations can be in the background color or the target speed, while LunarLander requires the agent to land on a platform, with variations comprising background color and different gravities.
% \AR{appendice per dettagli approfonditi su variazioni}.
No context is provided, hence the agents do not receive any information about the task.
% In the appendix we have other tests with atari env: \Cref{appendix:atari} \AR{riscrivi frase}

\paragraph{Baselines}
We mainly compare SAPS to (R3L), another zero-shot stitching method using relative representations whose approach is similar to ours.
For an additional baseline we also compare to naive zero-shot stitching, where we stitch encoders and controllers with no additional processing, to showcase the progress reached by the methods performing latent alignment techniques.

\section{Extened Study to Improve Task Singular Vector Merging for 3H Optimization}

Beyond the benchmarking effort in Sec. \ref{benchmark}, we also explore algorithmic advancements to improve model merging for 3H optimization. Specifically, we first select the most effective and stable merging algorithm, TSVM, to reform the 3H optimization problem and then discuss its drawbacks integrated with the ignored property of LLM (sparsity and heavy-tail) while merging LLM. Then, we leverage outlier weighting and sparsity-adaptive rank selection strategies to improve the effect of TSVM for 3H optimization. Finally, we provide a discussion for future optimized directions.

\subsection{Limitation for Task Singular Vector Merging}
As stated above, TSVM can achieve stable and good results for 3H Optimization. We reform the 3H optimization problem based on its implementation. Given $n$ alignment processes that produce model variants $\{\bm{\theta}^{i}\}_{i=1}^n$ from initial aligned model $\bm{\theta}^{0}$, for each layer $l \in \{1,...,L\}$, we can calculate alignment vector capturing optimization adjustments as Eq. \ref{alignment vector} and perform the singular value decomposition of the task vectors as Eq. \ref{eq:task_svd}, where $\bm{\theta}^{0}_{l}$ are the Parameters of the initial model at layer $l$, $\bm{U}^{(i)}_l, \bm{V}^{(i)}_l$ are respectively left and right singular vectors of task $i$'s parameter change matrix $\bm{\Delta}^{(i)}_l$, $\bm{S}^{(i)}_l$ are the singular values of $\bm{\Delta}^{(i)}_l$ and $d_l$ is the hidden dimension of layer and $\bm{\Delta}_l^{(i)}$ is a square matrix with row and column indices $(r,c) \in \{1,...,d_l\} \times \{1,...,d_l\}$. 
\begin{align}
    \label{alignment vector}
    \bm{\Delta}_l^{(i)} = \bm{\theta}^{i}_{l} - \bm{\theta}^{0}_{l} \in \mathbb{R}^{d_l \times d_l} \\[0.5ex] % 压缩行间距
    \label{eq:task_svd}
    \bm{\Delta}^{(i)}_{l} = \bm{U}^{(i)}_{l} \bm{S}^{(i)}_{l} \bm{V}_{l}^{(i)\top}
\end{align}

TSVM reduces interference through two steps: First, compress the layer task matrics with a fixed rank truncation $k_{\text{fixed}}$. Then, project task-specific parameter changes into orthogonal subspace by whitening these matrices to minimize their correlations as Eq. \ref{seek_orthogonal_u} and Eq. \ref{seek_orthogonal_v}, where $U_{l}$ and $V_{l}$ are the concatenated results of the left and right singular
vectors for different tasks. Thus, the layer-wise parameter of the merged model can be computed as Eq. \ref{eq:tsvm}
\vspace{-0.3cm}
\begin{align}
    \label{seek_orthogonal_u}
    \min_{U_{l\bot}} & \quad \| U_{l\bot} - U_{l} \|_F 
        \quad \text{s.t.} \quad U_{l\bot}^\top U_{l\bot} = I, \\[0.5ex] % 压缩行间距
    \label{seek_orthogonal_v}
    \min_{V_{l\bot}} & \quad \| V_{l\bot} - V_{l} \|_F 
        \quad \text{s.t.} \quad V_{l\bot}^\top V_{l\bot} = I, \\[0.5ex] % 压缩行间距
    \label{eq:tsvm}
    \bm{\theta}_{l}^{\text{TSVM}} & = \bm{\theta}^{0}_{l} + 
        \sum_{\mathclap{i=1}}^n U_{l\bot}^{(i)}[:,:k_{\text{fixed}}] \bm{S}^{(i)}_l V_{l\bot}^{(i)\top}[:k_{\text{fixed}},:]
\end{align}

\vspace{-0.3cm}


Based on Eq. \ref{eq:tsvm}, we can draw two limitations of TSVM while merging LLMs: \textbf{Limitations(i) Isotropic Treatment of Parameters}: TSVM ignores the heavy-tailed distribution of LLM parameters, where the outlier introduces additional challenges to capture true optimization direction through task vectors for model merging \cite{li2024owlore}; \textbf{Limitations(ii) Fixed Rank Selection}: TSVM's fixed rank truncation $k$ fails to adapt to layer-specific sparsity, ignoring significant structural heterogeneity across layers for LLM (Attention and FNN layers own different level of sparsity and parameter importance) \cite{li2024discovering}. We introduce reweighting-based optimization on TSVM (called R-TSVM), with theoretical foundations in outlier detection and sparse pattern analysis to address these problems.


\subsection{Our Reweight Task Singular Vector Merging}

% To cope with these limitations, as shown in Algorithm~\ref{alg:OWLM}, we propose a new reweight-based task singular vector merging algorithm (called R-TSVM), with theoretical foundations in outlier detection and sparse pattern analysis, aiming to further improve model merging for 3H optimization.


\textbf{Outlier-Aware Weighting for Limitation(i)}, we utilize layer-wise outlier detection to aggregate the Outlier-Aware Weight for subsequently weighting the singular values.  This means we should check which parts of the singular values can truly represent the optimization towards alignment process. Considering the heavy-tailed distribution of LLM parameter updates where few parameters undergo significant changes while most exhibit minor adjustments, we can refer to the 3 $\sigma$ principle compatible with this property. Thus, we adopt the statistical significance filtering and competitive weight normalization to identify significant true optimization adjustments as follows:
\begin{align}
    \mu_r^{(i)}    & = \mathbb{E}_c[|\Delta_{l,r,c}|]                            \\
    \sigma_r^{(i)} & = \sqrt{\mathbb{E}_c[|\Delta_{l,r,c}|^2] - (\mu_r^{(i)})^2}
\end{align}
\begin{equation}
    \alpha^{(i)}_l = \frac{\sum_{r=1}^{d_l} \|\textsc{Threshold}(\bm{\Delta}_{l,r,:}^{(i)}, \mu_r^{(i)}+3\sigma_r^{(i)})\|_1}{\sum_{j=1}^n \sum_{r=1}^{d_l} \|\textsc{Threshold}(\bm{\Delta}_{l,r,:}^{(j)}, \mu_r^{(j)}+3\sigma_r^{(j)})\|_1}
\end{equation}
where $\Delta_{l,r,c}^{(i)} \in \mathbb{R}$ denotes the weight deviation at row $r$, column $c$ of layer $l$ for model $i$ relative to initial model, $\mu_r^{(i)}$ and $\sigma_r^{(i)}$ represent the mean and standard deviation of deviations in row $r$, quantifying central tendency and dispersion, $\alpha_l^{(i)} \in [0,1]$ computes layer-wise aggregation weights via $L_1$-normalized sparse outlier magnitudes, and $\textsc{Threshold}(\bm{M}, \tau)$ applies hard-thresholding to suppress elements in matrix $\bm{M}$ with absolute values below $\tau$.

Compared with TSVM's direct utilization of full singular values, our outlier-aware weighting provides two key advantages:
\emph{Noise Suppression}: By thresholding parameter deviations via the $3\sigma$ rule, we filter out low-magnitude fluctuations that predominantly encode noise, forcing the singular vectors $\bm{u}_r^{(i)}$ to align with statistically significant task features.
\emph{Task Equilibrium}: The layer-wise aggregation weights $\alpha_l^{(i)}$ are globally normalized across all models, ensuring balanced contributions from diverse tasks and preventing dominance by high-magnitude updates that may obscure subtle yet critical features.

\textbf{Sparsity-Adaptive Rank Selection for Limitation (ii)}, we aim to adaptively decide the level of rank truncation based on the layer sparsity. We can first compute the sparsity consensus for all models first and then achieve the dynamic rank as Eq. \ref{rank_selection}, where $\gamma$ is the sparsity-rank coupling factor controlling the strength of rank reduction, $k_l$ is defined as the dynamic rank for layer $l$, adaptively determined by sparsity $\Omega_l$. The $\epsilon$ is set to 0.1 by default.
\begin{align}
    \label{sparsity}
    \Omega_l &= \frac{1}{n d_l^2}\sum_{i=1}^n\sum_{r,c=1}^{d_l} \mathbb{I}\left(|\Delta_{l,r,c}^{(i)}| < \epsilon\right) \\
    \label{rank_selection}
    k_l      &= \left\lfloor d_l(1-\gamma \Omega_l) \right\rfloor
\end{align}
% \begin{equation}
% \label{rank_selection}
% \bm{\Delta}_l^{(i)} \approx \bm{U}^{(i)}_l[:,:k_l] \bm{\tilde{S}}^{(i)}_l \bm{V}^{(i)\top}_l[:k_l,:]
% \end{equation}

Compared with TSVM's fixed-rank truncation, our sparsity-adaptive rank selection offers two enhancements:
\emph{Information Preservation}: The dynamic rank $k_l = \lfloor d_l(1-\gamma \Omega_l)\rfloor$ (Eq.~\ref{rank_selection}) adapts to layer-specific sparsity $\Omega_l$—retaining more singular directions in sparse layers (where updates concentrate on critical subspaces) while aggressively truncating redundant components in dense layers, thereby balancing information retention and noise suppression.
\emph{Conflict Mitigation}: By preserving dominant singular directions in sparse layers and enforcing orthogonality through Eq.~\ref{seek_orthogonal_u}, we reduce overlaps between task-specific parameters, decoupling interference-prone optimization trajectories.



\textbf{Overal:} With the above two reweighting designs, we can reform the layer-wise weight of the final merged model of Eq. \ref{eq:tsvm} to Eq.\ref{R-TSVM} with highlighting the improved parts:
{
\footnotesize
\begin{equation}\label{R-TSVM}
    \bm{\theta}^{\text{R-TSVM}}_l =
    \bm{\theta}^{0}_{l} + \sum_{i=1}^n
    \underbrace{
        \bm{{{{U}_l}_\bot}^{(i)}}[:,:\textcolor{blue}{k_l}]
    }_{\mathclap{\text{Adaptive rank selection}}}
    \underbrace{
        \biggl(
        \textcolor{red}{\alpha^{(i)}_l}
        \bm{S}^{(i)}_l
        \biggr)
    }_{\mathclap{\text{Outlier-aware weighting}}}
    \underbrace{
        \bm{{{{V}_l}_\bot}^{(i)\top}}[:\textcolor{blue}{k_l},:]
    }_{\mathclap{\text{Adaptive rank selection}}}
\end{equation}
}
More details of R-TSVM can be shown in Appendix \ref{appendix_method}.


\begin{algorithm2e}[t]
    \small
    \DontPrintSemicolon
    \SetKwInOut{Input}{Input}
    \SetKwInOut{Output}{Output}

    \Input{Initial model $\bm{\theta}^0$ and Further Aligned models $\{\bm{\theta}^i\}_{i=1}^n$ with same layers $L$, Sparsity factor $\gamma \in [0,1]$, $\epsilon > 0$}
    \Output{Merged model $\bm{\theta}^*$}
    % $\bm{\theta}^* \leftarrow \bm{\theta}_0$ \tcp*{\textcolor{gray}{Preserve base alignment}}

    \For{layer $l \leftarrow 1$ \KwTo $L$}{
    \tcp{Step1:Alignment Vector Extraction}
    $\bm{\Delta}_l^{(1:n)} \leftarrow [\bm{\theta}^{i}_{l} - \bm{\theta}^{0}_{l}]_{i=1}^n$

    \tcp{Step2:Outlier-Aware Weighting}
    \For{model $i \leftarrow 1$ \KwTo $n$}{
    Compute row-wise statistics: \\
    $\bm{\mu}^{(i)}, \bm{\sigma}^{(i)} \leftarrow \textsc{RowOutlierScore}(|\bm{\Delta}_l^{(i)}|)$ \\
    Calculate sparse aggregation weights: \\
    $\alpha^{(i)}_l \leftarrow \frac{\sum_{r=1}^{d_l} \|\textsc{Threshold}(\bm{\Delta}_{l,r,:}^{(i)}, \mu_r^{(i)}+3\sigma_r^{(i)})\|_1}{\sum_{j=1}^n \sum_{r=1}^{d_l} \|\textsc{Threshold}(\bm{\Delta}_{l,r,:}^{(j)}, \mu_r^{(j)}+3\sigma_r^{(j)})\|_1}$
    }
    \tcp{Step 3: Sparsity-Adaptive Rank Selection}
    Compute layer sparsity consensus: \\
    $\Omega_l \leftarrow \frac{1}{n d_l^2} \sum_{i=1}^n \sum_{r,c=1}^{d_l} \mathbb{I}(|\Delta_{l,r,c}^{(i)}| < \epsilon)$ \\
    Determine dynamic rank: \\
    $k_l \leftarrow \lfloor d_l(1 - \gamma \Omega_l) \rfloor$

    \tcp{Step 4:Reweight Optimization}
    \For{model $i \leftarrow 1$ \KwTo $n$}{
    Decompose:
    $[\bm{U}^{(i)}_l, \bm{S}^{(i)}_l, \bm{V}^{(i)}_l] \leftarrow \textsc{SVD}(\bm{\Delta}_l^{(i)})$ \\
    Compute orthogonal projections $\bm{{{U}_l}_\bot^{(i)}} $ and $\bm{{{V}_l}_\bot^{(i)}}$ via Eq.\ref{seek_orthogonal_u} via Eq.\ref{seek_orthogonal_v} \\
    Reweight for Outlier Weight:
    $\bm{S}^{(i)}_l \leftarrow \alpha^{(i)}_l \cdot \bm{S}^{(i)}_l$\\
    Reweight for Rank Selection:
    $\bm{{{U}_l}_\bot^{(i)}} \leftarrow \bm{{{U}_l}_\bot^{(i)}}[:,:k_l]$,
    $\bm{{{V}_l}_\bot^{(i)}} \leftarrow \bm{{{V}_l}_\bot^{(i)}}[:k_l,:]$,
    $\bm{S}^{(i)}_l \leftarrow \bm{S}^{(i)}_l[:k_l, :k_l]$}
    Merge Components:
    $\bm{M}_l \leftarrow \sum_{i=1}^n \bm{{{{U}_l}_\bot}^{(i)}}\bm{{S}_l}^{(i)} \bm{{{{V}_l}_\bot}^{(i)\top}}$ \\
    Update the Layer for the Merged Model:
    $\bm{\theta}^*_l \leftarrow \bm{\theta}^{0}_{l} + \bm{M}_l$}
    \caption{Reweight Task Singular Vector Merging}
    \label{alg:OWLM}
\end{algorithm2e}



\subsection{Comparsion Results and Discussions}

\textbf{Reweighting benefits TSVM for 3H optimization in LLM alignment}.
Our experiments demonstrate that the reweighting mechanisms—outlier weighting and sparsity-adaptive rank selection collectively enhance TSVM's capability for 3H optimization. As shown in Table \ref{tab:llama3_results} and \ref{tab:mistral_results}, we report the average score of helpfulness, honesty, and harmless under static optimization settings. The full results on various evaluation datasets can be shown in the Appendix \ref{appendix_reweighting}. From the results, we can observe that integrating both outlier weighting and sparsity-adaptive rank selection can collectively elevate the average score of 3H metric from 64.70 to 65.61 (Llama3) and from 61.67 to 62.30 (Mistral) on extensive evaluations. Notably, R-TSVM outperforms data mixture baselines on Mistral, achieving a higher relative 2.1 x improvement. These results validate model merging as a viable pathway for LLM alignment, particularly when balancing multi-dimensional objectives.

\textbf{Discussions}.
Our experimental results demonstrate that the main improvement of R-TSVM comes from the honest and harmless aspects. This can reflect the decrease in conflict between them, which can be defined as inter-aspect conflict reduction. But for helpfulness, R-TSVM is still worse than data mixture methods on Llama3 and the improvement on Mistral compared with existing merging strategy is also merely. Though the initial goal of honest and harmless training is not designed for helpfulness, modern preference datasets inherently encode helpfulness as a baseline annotation, forcing the alignment process to optimize towards this dimension regardless of their primary target (honesty/harmlessness). This means every alignment vector can represent helpfulness and one or more other dimensions' optimization directions, which may lead to conflict between alignment vectors only from the helpful dimension (e.g. code and commonsense QA abilities for LLM), which can be defined as intra-dimension conflict. This phenomenon necessitates a hierarchical conflict resolution framework to improve model merging for 3H optimization considering these two categories of conflicts simultaneously.




\begin{table}
    \centering
    \caption{Reweighting-Induced Improvements on Llama3 Under Static Optimization Settings.}
    \label{tab:llama3_results}
    \resizebox{\columnwidth}{!}{%
        \begin{tabular}{lcccc}
            \toprule
            \textbf{Method}              & \textbf{Helpfulness} & \textbf{Honesty} & \textbf{Harmlessness} & \textbf{Avg} \\ \midrule
            Llama3-8B-Instruct           & 58.79                & 53.50            & 59.07                 & 57.12        \\
            Hummer (best mixture)        & 60.35                & 55.60            & 73.21                 & 63.05        \\
            TSVM (best merging)          & 59.30                & 56.20            & 78.60                 & 64.70        \\
            R-TSVM w/o Outlier Weighting & 59.52                & 56.20            & 79.45                 & 65.06        \\
            R-TSVM w/o Rank Selection    & 59.45                & 56.80            & 79.05                 & 65.10        \\
            R-TSVM                       & 59.72                & 57.20            & 79.60                 & 65.51        \\
            \bottomrule
        \end{tabular}}
\end{table}

\begin{table}
    \centering
    \caption{Reweighting-Induced Improvements on Mistral Under Static Optimization Settings.}
    \label{tab:mistral_results}
    \resizebox{\columnwidth}{!}{%
        \begin{tabular}{lcccc}
            \toprule
            \textbf{Method}              & \textbf{Helpfulness} & \textbf{Honesty} & \textbf{Harmlessness} & \textbf{Avg} \\ \midrule
            Mistral-7B-Instruct-v0.2                & 41.70                & 62.17            & 76.38                 & 60.08        \\
            Hummer (best mixture)        & 42.50                & 62.05            & 78.57                 & 61.04        \\
            TSVM (best merging)          & 43.02                & 61.10            & 80.88                 & 61.67        \\
            R-TSVM w/o Outlier Weighting & 42.90                & 61.80            & 81.25                 & 61.98        \\
            R-TSVM w/o Rank Selection    & 43.32                & 61.20            & 81.75                 & 62.09        \\
            R-TSVM                       & 43.15                & 61.50            & 82.25                 & 62.30        \\
            \bottomrule
        \end{tabular}
    }
\end{table}

% \section{Methodology}
% \label{sec:method}

% \subsection{Task Singular Vector Merging Revisited}
% \label{ssec:tsvm_review}

% The TSVM framework \cite{ilharco2022editing} addresses task interference through geometric analysis of parameter subspaces. Given $n$ task-specific models $\{\bm{\theta}_i\}$ fine-tuned from base model $\bm{\theta}_0$, define task vectors as:

% \begin{equation}
% \bm{\tau}^{(i)} = \bm{\theta}_i - \bm{\theta}_0 \in \mathbb{R}^D
% \end{equation}

% The core innovation of TSVM lies in the singular value decomposition (SVD) of stacked task vectors:

% \begin{equation}
% [\bm{\tau}^{(1)}|\cdots|\bm{\tau}^{(n)}] = \bm{U}\bm{\Sigma}\bm{V}^\top
% \end{equation}

% Where the left singular vectors $\bm{U} = [\bm{u}_1,\cdots,\bm{u}_k]$ identify principal directions in parameter space. The merged model is constructed by:

% \begin{equation}
% \bm{\theta}^* = \bm{\theta}_0 + \sum_{r=1}^k \sigma_r \bm{u}_r
% \label{eq:tsvm_base}
% \end{equation}

% This achieves two key properties:
% \begin{itemize}
% \item \textbf{Constructive Alignment}: Directions shared across tasks ($\bm{u}_r$ with large $\sigma_r$) are amplified
% \item \textbf{Conflict Suppression}: Orthogonal components ($\bm{u}_r^\top\bm{u}_s \approx 0$) undergo natural cancellation
% \end{itemize}

% \subsection{Limitations and Our Improvements}
% \label{ssec:limitations}

% While effective, TSVM exhibits three limitations our method addresses:

% \paragraph{Isotropic Treatment of Parameters} 
% TSVM applies uniform SVD across all layers, ignoring the heavy-tailed distribution of LLM parameters \cite{dettmers2022sparsity}. We propose layer-wise outlier detection:

% \begin{equation}
% w_l^{(i)} = \frac{\|\mathcal{T}_{3\sigma}(\bm{\tau}_l^{(i)})\|_1}{\sum_j \|\mathcal{T}_{3\sigma}(\bm{\tau}_l^{(j)})\|_1},\ \ 
% \mathcal{T}_\tau(\bm{x}) = \begin{cases} 
% x_k & |x_k| \geq \mu + 3\sigma \\
% 0 & \text{otherwise}
% \end{cases}
% \end{equation}

% \paragraph{Fixed Rank Selection} 
% TSVM's fixed rank truncation ($k=0.8d$) fails to adapt to layer-specific sparsity. Our dynamic rank mechanism:

% \begin{equation}
% k_l = \lfloor d_l(1 - \gamma s_l)\rfloor,\ \ 
% s_l = \frac{1}{nd_l^2}\sum_{i,j,k} \mathbb{I}(|\tau_{l,jk}^{(i)}| < \epsilon)
% \end{equation}

% % \paragraph{Uniform Singular Value Scaling} 
% % Original TSVM equally weights surviving directions. We introduce competitive weighting:

% % \begin{equation}
% % \bm{\theta}^* = \bm{\theta}_0 + \sum_{l=1}^L \sum_{r=1}^{k_l} \underbrace{w_l^{(i)}\sigma_r^{(i)}}_{\text{Adaptive Scaling}} \bm{u}_r^{(i)}
% % \label{eq:ours}
% % \end{equation}











\section{Conclusion}
This paper introduces \textbf{ReLearn}, a novel unlearning framework via positive optimization that balances forgetting, retention, and linguistic capabilities. 
Our key contributions encompass a practical unlearning paradigm, comprehensive metrics (KFR, KRR, LS), and a mechanistic analysis comparing reverse and positive optimization. 
%As underscored, unlearning should not only erase knowledge but also relearn knowledge for constructive outputs.

\label{sec:bibtex}
\section*{Limitations}
While ReLearn shows promising performance, several limitations remain.
(1) Computational Overhead: Data synthesis may hinder scalability.
(2) Metric Sensitivity: Our metrics still have limited sensitivity to subtle knowledge nuances.
(3) Theoretical Grounding: Understanding the dynamics of knowledge restructuring requires deeper theoretical investigation, which we plan to explore in the future work.

% Aligning large language models (LLMs) with Helpfulness, Harmlessness, and Honesty (3H) is critical for responsible AI. While conventional data mixture methodologies are hampered by expertise-intensive manual configuration and computationally prohibitive scaling demands in balancing these objectives, model merging emerges as a performant and resource-efficient alternative. This paper first expands the exploration to a wider array of model merging techniques for 3H Optimization, through a comprehensive, first-of-its-kind benchmarking study spanning 15 representative methods (12 training-free merging methods and 3 representative data mixture methods), 10 training tasks associated with 5 annotation dimensions of preference datasets, 2 classific families of LLMs, and 2 different training settings. Unlike previous merging works that discuss the merits and mechanism of linear or spherical interpolation with different rewarded policies, this paper unveils previously overlooked conflict issues for 3H optimization, establishing the theoretical and experimental principle and insight for further studies. We subsequently propose a novel \textbf{R}eweighting-based optimization including outlier weighting and sparsity-ware rank selection based on \textbf{T}ask \textbf{S}ingular \textbf{V}ector \textbf{M}erging method (\textbf{R-TSVM}), further improving LLM alignment across multiple evaluations. Our findings offer a promising direction for LLM alignment, advancing the development of ethically-constrained language models while maintaining computational efficiency.


\section*{Impact Statement}
This paper explores model merging for 3H optimization of large language models (LLMs) during the alignment stage. Its potential impacts are contingent on how these merging methods are utilized. On the positive side, assisted by model merging techniques, achieving a better-aligned model without retraining or adjusting the data mixture ratio many times could lead to significant reductions in energy consumption, contributing to the development of green AI and achieving improved performance in resource-constrained environments. However, there is a potential negative aspect in terms of private protection, as the competitors may steal your model parameters through model merging without prior notice. However, given the technical focus of this work, there are no specific societal consequences directly stemming from it that need to be highlighted here.

% In the unusual situation where you want a paper to appear in the
% % references without citing it in the main text, use \nocite
% \nocite{daheim2024model,chen2024local}

\bibliography{example_paper}
\bibliographystyle{icml2025}


%%%%%%%%%%%%%%%%%%%%%%%%%%%%%%%%%%%%%%%%%%%%%%%%%%%%%%%%%%%%%%%%%%%%%%%%%%%%%%%
%%%%%%%%%%%%%%%%%%%%%%%%%%%%%%%%%%%%%%%%%%%%%%%%%%%%%%%%%%%%%%%%%%%%%%%%%%%%%%%
% APPENDIX
%%%%%%%%%%%%%%%%%%%%%%%%%%%%%%%%%%%%%%%%%%%%%%%%%%%%%%%%%%%%%%%%%%%%%%%%%%%%%%%
%%%%%%%%%%%%%%%%%%%%%%%%%%%%%%%%%%%%%%%%%%%%%%%%%%%%%%%%%%%%%%%%%%%%%%%%%%%%%%%
\newpage
\appendix
\onecolumn

The appendix is structured into multiple sections, each offering supplementary information and further clarification on topics discussed in the main body of the manuscript. 

\startcontents[sections]  % 开始定义附录目录
\printcontents[sections]{}{1}{\setcounter{tocdepth}{3}}  % 打印附录目录
\vskip 0.2in
\hrule

\section{More Details for Method}\label{appendix_method}
\subsection{More Details for Outlier-aware Weighting}\label{app:outlier}
\paragraph{Interpretation of Dual Objectives for outlier weighting}
The mathematical framework achieves cross-model consensus and intra-model saliency through its hierarchical thresholding mechanism:

(i) \textbf{Cross-Model Consensus}:
The denominator in Eq. (3) normalizes each model's contribution by the total sparse outlier magnitude across all $n$ models:
\begin{equation}
    \sum_{j=1}^n \sum_{c=1}^{d_l} \|\textsc{Threshold}(\bm{\Delta}_{l,c}^{(j)}, \mu_c^{(j)}+3\sigma_c^{(j)})\|_1
\end{equation}
This forces models with greater sparse deviation magnitudes (potential task conflicts) to receive proportionally reduced aggregation weights $\alpha_l^{(i)}$, effectively suppressing outlier-dominated models in the merged output.

(ii) \textbf{Intra-Model Saliency}:
The $3\sigma$ threshold in $\textsc{Threshold}(\bm{\Delta}_{l,c}^{(i)}, \mu_c^{(i)}+3\sigma_c^{(i)})$ implements statistical outlier detection within each model's parameter distribution. For Gaussian-distributed $\Delta_{l,c,k}^{(i)}$ (per Central Limit Theorem), this retains only the top 0.3\% extreme deviations that likely correspond to:
\begin{itemize}
    \item Task-specific knowledge carriers ($\Delta > \mu+3\sigma$)
    \item Catastrophic interference sources ($\Delta < \mu-3\sigma$)
\end{itemize}
The $L_1$ norm aggregation $\sum_{c=1}^{d_l}\|\cdot\|_1$ then amplifies layers containing concentrated outlier parameters.

\textbf{Synergistic Effect}: The normalization in (i) prevents any single model's outliers from dominating the merger, while the saliency detection in (ii) preserves critical task-specific features within each model. This dual mechanism reduces interference by selectively blending statistically significant parameters across models.

\subsection{More Details for the Reasonability of R-TSVM}
\label{ssec:analysis}

Building on TSVM's theoretical framework, our method provides enhanced guarantees through statistical awareness and adaptive computation.

\paragraph{Conflict Probability Bound}
Let $p_{\text{conflict}}^{(l)}$ denote the probability of directional conflicts in layer $l$. Our rank adaptation yields as follows. We can observe that , compared to TSVM's fixed $\frac{1}{\sqrt{d_l}}$, our bound adapts to layer sparsity.

\begin{equation}
    \mathbb{E}[p_{\text{conflict}}^{(l)}] \leq \frac{1}{\sqrt{k_l}} \propto \frac{1}{\sqrt{d_l(1-\gamma s_l)}}
\end{equation}

\paragraph{Weight Concentration}
The 3$\sigma$ thresholding induces weight concentration on critical parameters. For any layer $l$:

\begin{equation}
    \frac{\mathbb{V}[w_l^{(i)}]}{\mathbb{E}[w_l^{(i)}]^2} \leq \frac{1}{\|\mathcal{T}_{3\sigma}(\bm{\tau}_l^{(i)})\|_0}
\end{equation}

This variance-to-mean ratio decreases as outliers become sparser, stabilizing training.

\begin{table}[ht]
    \centering
    \footnotesize  % 使用更小字号
    \setlength{\tabcolsep}{5pt}  % 压缩列间距
    \caption{Theoretical Comparison between our reweight optimization and TSVM.}
    \label{tab:theory}
    \begin{tabular}{@{} l >{\centering\arraybackslash}p{1.8cm} >{\centering\arraybackslash}p{2cm} @{}} % 自定义列宽
        \toprule
        \textbf{Property}    & \textbf{TSVM} & \textbf{R-TSVM}                    \\
        \midrule
        Layer adaptivity     & $\times$      & $\checkmark$                       \\
        Sparsity awareness   & $\times$      & $\checkmark$                       \\
        Conflict bound       & $O(d^{-1/2})$ & $O(d^{-1/2}(1{-}\gamma s)^{-1/2})$ \\
        Weight concentration & Uniform       & Heavy-tailed                       \\
        Comp.\ complexity    & $O(d^3)$      & $O(k d^2)$                         \\
        \bottomrule
    \end{tabular}
\end{table}

% \subsection{Orthogonal Subspace Mechanism}
% The key to TSVM's conflict resolution lies in the orthogonality of singular vectors:
% \begin{equation}
% \langle \bm{u}_r^{(i)}, \bm{u}_s^{(j)} \rangle \approx 0 \quad (r \neq s)
% \label{eq:orthogonality}
% \end{equation}
% This property enables:
% \begin{itemize}
%     \item \textbf{Constructive Alignment}: Shared directions ($\bm{u}_r^{(i)} \approx \bm{u}_r^{(j)}$) are amplified.
%     \item \textbf{Conflict Suppression}: Orthogonal directions ($\bm{u}_r^{(i)} \perp \bm{u}_s^{(j)}$) are naturally attenuated.
% \end{itemize}


\subsection{Order of Orthogonalization and Rank Truncation/Selection}
\label{sec:order}

A critical design choice in our R-TSVM algorithm lies in the sequential relationship between orthogonalization (Eq.~\ref{seek_orthogonal_u}-\ref{seek_orthogonal_v}) and rank truncation (Eq.~\ref{rank_selection}). Through theoretical analysis and empirical validation, we establish that \textbf{orthogonalization should precede truncation} to ensure optimal subspace alignment and information preservation. This ordering stems from three fundamental considerations:
\textbf{Global Orthogonality Constraints}: The orthogonal projection in Eq.~\ref{seek_orthogonal_u} minimizes the Frobenius norm difference $\| {U_{l}}_\bot - U_{l} \|_F$ under strict orthogonality constraints. Performing this projection \textit{before} truncation preserves the complete singular vector structure, enabling accurate modeling of cross-task interference patterns. Early truncation would discard directional components essential for constructing the orthogonal basis, particularly when task-specific updates exhibit heterogeneous rank distributions.

\textbf{Dynamic Rank Adaptation}: Our sparsity-adaptive rank selection (Eq.~\ref{rank_selection}) requires layer-wise sparsity measurement $\Omega_l$, computed from the full parameter deviation matrix $\bm{\Delta}_l^{(i)}$. Truncating $\bm{\Delta}_l^{(i)}$ prematurely would bias $\Omega_l$ by excluding contributions from low-magnitude parameters, thereby undermining the adaptive rank calculation. As shown in Algorithm~\ref{alg:OWLM}, orthogonalization (Step~4) utilizes the full-rank SVD decomposition to maintain statistical fidelity.

\textbf{Outlier Weighting Integrity}: The outlier-aware weighting mechanism (Eq.~6) operates on the complete parameter deviation matrix to identify statistically significant updates. Truncation prior to outlier detection would risk eliminating subtle yet critical features masked within lower-rank components, particularly in layers with heavy-tailed parameter distributions.

\section{More Details for Related Work}\label{more_relatedwork}
\subsection{Discussion with the Alignment Tax.}
\begin{figure}[htbp]
    \centering
        \setlength{\abovecaptionskip}{0cm}   %调整图片标题与图距离
    \setlength{\belowcaptionskip}{0cm}   %调整图片标题与
    \includegraphics[width=0.99\linewidth]{fig/subsequent_DPO.pdf} % 调整宽度为栏宽的 90%
    \caption{Illustration of Training Stage of 3H Optimization, which aims to further enhance LLMs alignment from three perspectives based on the existing Initially Aligned LLMs.}
    \label{fig:3H_stage}
\end{figure}
We would like to further clarify the main difference between 3H trade-off and previously defined alignment tax \cite{lin2024mitigating,lu2024online}. In general, the alignment tax describes the phenomenon of RLHF training leading to \emph{the forgetting of pre-trained abilities during the first alignment stage}. However, as shown in Figure \ref{fig:3H_stage}, we mainly focus on how can we further \emph{enhance the 3H-related abilities of the existing already-aligned model during the second or subsequent stages.} The trade-off mainly comes from the conflict of different alignment objects without dealing with the pre-trained knowledge. Take the Llama3 series for example, alignment tax mainly analyzes the pre-trained ability degradation on the SFT version of the Base LLM (e.g. train the Llama-3-8B on the Ultrachat) while performing DPO training, which refers to the \textbf{green arrow} of the Figure \ref{fig:3H_stage}. However, in this paper, we mainly focus on how can we further enhance the 3H-related abilities of the existing already aligned model (e.g. Llama3-8B-Instruct) during the second or subsequent alignment stages (\textbf{orange arrow} of the Figure \ref{fig:3H_stage}), which can meet more strict demands for specific applications.

\subsection{Discussion with the MOE-Based Merging Methods (e.g. H3 fusion).} To further distinguish our work from previous ones and strengthen our contribution, we provide more detailed discussions about the MOE-Based Merging methods \cite{zhao2024loraretriever,zhao2024merging,zhou2025mergeme}. Specifically, most of MOE-based merging works, such as SMILES \cite{tang2024smile}, Free-Merging \cite{zheng2024free}, and Twin-Merging \cite{zheng2024free}, aim to balance the performance and deployment costs through modular expertise identification and integration adapted to the input data, which is not designed for our setting about 3H optimization in LLM alignment. 

Recently,  we have noticed a concurrent MOE-fusion work called H3 fusion \cite{tekin2024h} related to our theme. It includes three main steps:(i) Adopt the instruction tuning and summarization fusion as two modern ensemble learning in the context of helpful-harmless-honest (H3) alignment (ii) \textbf{Merge} the aligned model weights with an expert router \textbf{according to the type of
input} instruction and dynamically select a subset of experts. (iii) Utilize the gating loss and regularization terms to enhance performance. But our work mainly focuses on how can we address the conflict issued for 3H optimization to construct a multi-object aligned LLM rather than dynamically adapted to the input data. Simultaneously, considering that the constraints of data availability and data leak will limit the generalization of existing merging methods for LLMs, in the paper we mainly adopt the well-known and latest \textbf{training-
free and data-free} merging strategies for dense LLM, while H3 fusion needs the data for training and only utilizes the merging techniques for efficiently adapting to the input data. Thus, \textbf{H3 fusion is indeed different from our work from the perspective of problem and technique contributions.}

% (e.g.as SMILES \cite{tang2024smile}, WEMOE \cite{shen2024efficient}, Free-Merging \cite{zheng2024free}, Twin-Merging \cite{zheng2024free}, H3 fusion \cite{cf208ab262f0affc4f5581b9a18901265d9728ab}}
% Notably, those mixtures of experts(MOE) based merging methods, such as SMILES \cite{tang2024smile}, WEMOE \cite{shen2024efficient}, Free-Merging \cite{zheng2024free} and Twin-Merging \cite{zheng2024free} are not designed for our settings, we provide discussions in the Appendix.


\section{More Details for Experiments}\label{training_details}
\subsection{The Training Details for Model Constructions and Baselines}
\textbf{Training hyperparameters for model constructions:} following SimPO \cite{meng2024simpo}, based on Llama-3-8B-Instruct and Mistral-7B-Instruct-V2, we conduct preference optimization adopting the fixed batch size 128 for 1 epoch training with the Adam optimizer. We set the max sequence length to 4096 and apply a cosine learning rate schedule with 10 percent warmup steps for each dataset. Specially, we adjust $\beta \in \left[ 0.1, 0.5, 1.0, 2.0 \right]$ and learning rate $lr \in \left[3e-7,5e-7 \right]$ for model constructions and report the best individual training models corresponding to different annotation dimensions.


\textbf{The Implementation of Baselines:} For Heuristic data mixture methods, we control the ratio between Honesty\&Harmlessness and Helpfulness to 1/5,1/10 and 1/20 by default and report the best average score (usually 1/10 according to our experiments). For ArmoRM, we follow the process of SimPO \cite{meng2024simpo} to achieve refined full mixture data. For hummer \cite{jiang2024hummer}, we refine the alignment dimension conflict (ADC) among preference datasets leveraging the powerful ability of AI feedback(e.g. GPT4) as the paper stated. For the full mixture datasets of Table \ref{tab:dpo_data_stats}, we control the ADC lower than 20 percent.




\textbf{Computation environment:} All of our experiments in this paper were conducted on 16×A100 GPUs based on the LLaMA-Factory \cite{zheng2024llamafactory}, MergeKit \cite{goddard2024arcee} and fusion\_bench \cite{tang2024fusionbench}.


\textbf{Reproducibility:} We have made significant efforts to ensure the reproducibility of our work. Upon acceptance, we will release all of the trained models and the complete training and testing code to facilitate the full reproducibility of our results. We are committed to advancing this work and will provide updates on its accessibility in the future. 



\subsection{The Evaluation Details for the Judged Models}\label{evaluation_datails}
We provide detailed descriptions for the evaluation that needs the judged models. For MT-Bench, we report scores following its evaluation
protocol to grade single answers from 1 to 10 scores assisted by GPT4. For HaluEval-Wild, given prompts to our trained model, we utilize the judged model to check whether the output of our trained model is a hallucination or not and then calculate the no hallucination rate. Similarly, we utilize the prompts from SaladBench and OR-Bench to instruct our trained models and then let the judged models check whether the replies of our trained models are safe/unsafe or refusal/answer. Based on the check results, we can naturally calculate the safe score and refusal score by counting all results. The detailed descriptions of the evaluation can be shown in Table \ref{tab:evaluation_comparison}.  More details can be shown in the original paper.

\begin{table}[ht]
    \centering
    \footnotesize  % 使用更小字号
    \setlength{\tabcolsep}{5pt}  % 压缩列间距
    \caption{Evaluation details corresponding judge models, scoring types, and metrics.}
    \label{tab:evaluation_comparison}
    \begin{tabular}{@{} l l c c c @{}} % 自定义列宽
        \toprule
        \textbf{Evaluation Datasets} &\textbf{Examples} & \textbf{Judge Models} & \textbf{Scoring Type} & \textbf{Metrics} \\
        \midrule
         MT-Bench \cite{zheng2023judging}          & 80    & GPT-4  & Single Answer Grade  & Rating of 1-10 \\
         HaluEval-Wild \cite{zhu2024halueval}         & 500   & GPT4   & Classify \& Calculate Ratio  & Rating of 0-100 \\
         SaladBench \cite{li2024salad}        & 1817  & MD-Judge-V0.2 & Classify \& Calculate Ratio  & Rating of 0-100 \\
         OR-Bench \cite{cui2024or}           & 1319  & GPT4-o & Classify \& Calculate Ratio  & Rating of 0-100 \\
        \bottomrule
    \end{tabular}
\end{table}

\begin{table*}
    % \setlength{\abovecaptionskip}{0cm}
    % \setlength{\belowcaptionskip}{0cm}
    \captionsetup{font={small,stretch=1.25}, labelfont={bf}}
    \renewcommand{\arraystretch}{1}
    \caption{\textbf{3H Results on Mistral Under Continuous Optimization Setting where we sequentially perform DPO training using data with annotations about Helpfulness\&Honesty (Stage1), Helpfulness\&Harmlessness (Stage2) and Helpful (Stage3).For merging methods, we highlight the best score in bold and the second score with underlining.}}
    \label{continuous_mistral}
    \centering
    \resizebox{0.99\textwidth}{!}{
        \begin{tabular}{lcccccccc|c|cc|cccc}
            \toprule
            \multirow{2}{*}{\textbf{Methods}} & \multicolumn{8}{c|}{\textbf{Helpfulness}} & \textbf{Honesty} & \multicolumn{2}{c|}{\textbf{Harmlessness}} & \multirow{2}{*}{\textbf{Helpful\_Avg}} & \multirow{2}{*}{\textbf{Honest\_Avg}} & \multirow{2}{*}{\textbf{Harmless\_Avg}} & \multirow{2}{*}{\textbf{AVG}}                                                                                                          \\ \cmidrule{2-12}
                                              & Math                                      & GSM8K            & ARC-E                                      & ARC-C                                  & MMLU                                  & MBPP\_Plus                              & HumanEval\_Plus               & MT-Bench & HaluEval\_Wild & Salad\_Bench(↑) & OR-Bench(↑) &       &       &       &                   \\ \midrule
            \textbf{Mistral-7B-Instruct-V2}   & 9.54                                      & 46.17            & 82.36                                      & 72.88                                  & 59.97                                 & 26.46                                   & 28.66                         & 7.55     & 62.17           & 78.07           & 74.68       & 41.70 & 62.17 & 76.38 & 60.08             \\ \midrule
            Continual DPO Training Stage1     & 8.76                                      & 43.14            & 82.01                                      & 74.92                                  & 59.78                                 & 25.93                                   & 27.33                         & 7.59     & 61.33           & 78.74           & 77.23       & 41.18 & 61.33 & 77.99 & 60.17             \\
            Continual DPO Training Stage2     & 9.26                                      & 36.16            & 82.54                                      & 75.59                                  & 60.38                                 & 29.88                                   & 33.33                         & 7.86     & 56.40           & 82.76           & 78.54       & 41.88 & 56.40 & 80.65 & 59.64             \\
            Continual DPO Training Stage3     & 9.60                                      & 40.49            & 82.54                                      & 77.29                                  & 60.51                                 & 26.25                                   & 34.15                         & 7.46     & 57.40           & 80.77           & 83.16       & 42.29 & 57.40 & 81.97 & 60.52             \\ \midrule
            Weight Average                    & 10.04                                     & 45.72            & 82.36                                      & 75.25                                  & 61.03                                 & 26.46                                   & 31.71                         & 7.56     & 59.20           & 78.02           & 81.43       & 42.52 & 59.20 & 79.73 & 60.48             \\
            Rewarded Soup                     & 9.72                                      & 46.02            & 82.19                                      & 75.25                                  & 61.03                                 & 26.46                                   & 32.93                         & 7.61     & 58.60           & 77.94           & 81.34       & 42.65 & 58.60 & 79.64 & 60.30             \\
            Model Stock                       & 9.74                                      & 47.69            & 82.36                                      & 73.56                                  & 59.77                                 & 24.87                                   & 27.44                         & 7.68     & 61.00           & 78.51           & 76.44       & 41.64 & 61.00 & 77.48 & 60.04             \\
            Task Arithmetic                   & 9.76                                      & 43.06            & 82.54                                      & 75.93                                  & 61.27                                 & 25.66                                   & 32.93                         & 7.46     & 57.80           & 78.32           & 82.35       & 42.33 & 57.80 & 80.34 & 60.15             \\
            Ties                              & 10.48                                     & 41.55            & 84.66                                      & 76.27                                  & 61.60                                 & 26.19                                   & 30.49                         & 7.46     & 53.80           & 78.99           & 85.43       & 42.34 & 53.80 & 82.21 & 59.45             \\
            DARE                              & 10.40                                     & 42.99            & 85.36                                      & 75.93                                  & 61.54                                 & 24.60                                   & 33.54                         & 7.54     & 56.00           & 78.81           & 85.21       & 42.74 & 56.00 & 82.01 & 60.25             \\
            DARE Ties                         & 10.28                                     & 42.00            & 85.01                                      & 76.27                                  & 61.61                                 & 27.25                                   & 32.32                         & 7.43     & 53.00           & 79.17           & 86.50       & 42.77 & 53.00 & 82.84 & 59.54             \\
            DELLA                             & 10.18                                     & 43.14            & 84.83                                      & 75.25                                  & 61.46                                 & 26.46                                   & 31.71                         & 7.58     & 55.25           & 79.35           & 86.04       & 42.58 & 55.25 & 82.70 & 60.18             \\
            DELLA  Ties                       & 10.50                                     & 40.18            & 85.89                                      & 77.97                                  & 61.37                                 & 30.16                                   & 30.48                         & 7.30     & 54.80           & 79.90           & 87.49       & 42.98 & 54.80 & 83.70 & \underline{60.49} \\
            Breadcrumbs                       & 10.56                                     & 42.53            & 84.83                                      & 75.59                                  & 64.50                                 & 24.60                                   & 32.32                         & 7.53     & 52.40           & 79.42           & 84.34       & 42.81 & 52.40 & 81.88 & 59.03             \\
            Breadcrumbs Ties                  & 10.54                                     & 42.46            & 84.66                                      & 76.95                                  & 61.47                                 & 26.72                                   & 29.88                         & 7.45     & 53.40           & 79.80           & 84.57       & 42.52 & 53.40 & 82.19 & 59.37             \\
            TSVM                              & 10.52                                     & 41.25            & 85.28                                      & 77.21                                  & 61.57                                 & 29.22                                   & 30.48                         & 7.55     & 54.95           & 79.90           & 87.49       & 42.89 & 54.95 & 83.70 & \textbf{60.51}    \\


            \bottomrule
        \end{tabular}
    }
\end{table*}

\begin{table*}
  % \setlength{\abovecaptionskip}{0cm}
  % \setlength{\belowcaptionskip}{0cm}
  \captionsetup{font={small,stretch=1.25}, labelfont={bf}}
  \renewcommand{\arraystretch}{1}
  \caption{\textbf{The detailed 3H Results on Llama3 Under Static Optimization Setting adopting our proposed reweighting-based optimization.}}
  \label{detailed_reweight_llama3}
  \centering
  \setlength{\tabcolsep}{2pt}
  \resizebox{0.99\textwidth}{!}{
    \begin{tabular}{lcccccccc|c|cc|cccc}
      \toprule
      \multirow{2}{*}{\textbf{Methods}}      & \multicolumn{8}{c|}{\textbf{Helpfulness}} & \textbf{Honesty} & \multicolumn{2}{c|}{\textbf{Harmlessness}} & \multirow{2}{*}{\textbf{Helpful\_Avg}} & \multirow{2}{*}{\textbf{Honest\_Avg}} & \multirow{2}{*}{\textbf{Harmless\_Avg}} & \multirow{2}{*}{\textbf{AVG}}                                                                                                          \\ \cmidrule{2-12}
                                             & Math                                      & GSM8K            & ARC-E                                      & ARC-C                                  & MMLU                                  & MBPP\_Plus                              & HumanEval\_Plus               & MT-Bench & HaluEval\_Wild & Salad\_Bench & OR-Bench &       &       &       &                   \\ \midrule
      \textbf{llama3-8B-Instruct}            & 28.08                                     & 78.09            & 93.65                                      & 82.03                                  & 68.20                                 & 58.99                                   & 53.05                         & 8.25     & 53.50           & 91.16           & 26.97       & 58.79 & 53.50 & 59.07 & 57.12             \\ \midrule
      Hummer (best mixture training)          & 29.41                                     & 78.95            & 93.65                                      & 82.69                                  & 68.59                                 & 60.41                                   & 58.15                         & 8.58     & 55.60           & 92.10           & 50.11       & \textbf{60.35} & 55.60 & 73.21 & 63.05             \\ \midrule
      TSVM (best merging)                                   & 29.92                                     & 77.63            & 93.12                                      & 82.17                                  & 68.51                                 & 59.26                                   & 55.49                         & 8.29     & 56.20           & 89.43           & 67.76       & 59.30 & 56.20 & 78.60 & 64.70    \\
      \textbf{R-TSVM (ours)}                                   & 29.89                                     & 78.89            & 93.65                                      & 82.37                                  & 68.51                                 & 59.56                                   & 56.63                         & 8.29     & 57.20          & 89.92           & 69.27       & 59.72 & \textbf{57.20} & \textbf{79.60} & \textbf{65.51}    \\
      \bottomrule
    \end{tabular}
  }
\end{table*}

\begin{table*}
  % \setlength{\abovecaptionskip}{0cm}
  % \setlength{\belowcaptionskip}{0cm}
  \captionsetup{font={small,stretch=1.25}, labelfont={bf}}
  \renewcommand{\arraystretch}{1}
  \caption{\textbf{The detailed 3H Results on Mistral Under Static Optimization Setting adopting our proposed reweighting-based optimization.}}
  \label{detailed_reweight_mistral}
  \centering
  \setlength{\tabcolsep}{2pt}
  \resizebox{0.99\textwidth}{!}{
    \begin{tabular}{lcccccccc|c|cc|cccc}
      \toprule
      \multirow{2}{*}{\textbf{Methods}}      & \multicolumn{8}{c|}{\textbf{Helpfulness}} & \textbf{Honesty} & \multicolumn{2}{c|}{\textbf{Harmlessness}} & \multirow{2}{*}{\textbf{Helpful\_Avg}} & \multirow{2}{*}{\textbf{Honest\_Avg}} & \multirow{2}{*}{\textbf{Harmless\_Avg}} & \multirow{2}{*}{\textbf{AVG}}                                                                                                          \\ \cmidrule{2-12}
                                             & Math                                      & GSM8K            & ARC-E                                      & ARC-C                                  & MMLU                                  & MBPP\_Plus                              & HumanEval\_Plus               & MT-Bench & HaluEval\_Wild & Salad\_Bench & OR-Bench &       &       &       &                   \\ \midrule
      \textbf{Mistral-7B-Instruct-v0.2}        & 9.54                                      & 46.17            & 82.36                                      & 72.88                                  & 59.97                                 & 26.46                                   & 28.66                         & 7.55     & \textbf{62.17}           & 78.07           & 74.68       & 41.70 & 62.17 & 76.38 & 60.08             \\ \midrule
      Hummer (best mixture training)           & 9.79                                      & 44.50            & 83.72                                      & 74.89                                  & 60.53                                 & 25.85                                   & 33.15                         & 7.56     & 62.05           & 81.85           & 75.28       & 42.50 & 62.05 & 78.57 & 61.04             \\ \midrule
      TSVM (best merging)                                    & 10.40                                     & 44.88            & 84.29                                      & 75.24                                  & 60.87                                 & 28.50                                   & 32.32                         & 7.65     & 61.10           & 83.25           & 78.51       & 43.02 & 61.10 & 80.88 & 61.67  \\
        \textbf{R-TSVM (ours)}                                   & 10.44                                    & 45.00            & 84.35                                      & 75.79                                  & 60.87                                 & 28.50                                   & 32.52                         & 7.71     & 61.50           & 84.25           & 80.25       & \textbf{43.15} & 61.50 & \textbf{82.25} & \textbf{62.30}  \\
      \bottomrule
    \end{tabular}
  }
\end{table*}


\subsection{More Experiments under the Continual DPO Training Settings} \label{appendix_continual}
As shown in Table \ref{continuous_mistral}, we provide additional results under the continual training settings. Through comparison results between different training stages, we can observe the honesty, helpfulness, and harmlessness of LLMs are interactively enhanced due to forgetting during continual training. Moreover, model merging methods can achieve comparable results to these continual training methods without the need to consider the optimized status at a specific training stage. In other words, model merging paves a new way for continual DPO training, advocating training multiple models from the same start point and then merging them, rather than continually optimizing the model from the previous optimization. 

\subsection{The detailed results of Reweighting-based optimization.}\label{appendix_reweighting}
Due to the page limit in the main content, we provide the detailed results of Reweighting-based optimization over TSVM to further verify its effectiveness for 3H optimization in LLM alignment.



\subsection{Hyper-Parameter Analysis to Sparsity} \label{appendix_sparsity}
The sparsity-based strategy is closely related to the merging effect. As shown in Table \ref{static_llama3} and Table \ref{static_mistral}, apart from the SVD-based methods, the most effective merging methods are DARE and DELLA, both of which depend on random sparsification as shown in Table \ref{tab:methods}. However, we conduct extended studies to check the robustness and stability concerning random seed and sparsity factors. As shown in Figure \ref{seed} and Figure \ref{sparsity_sensitivity}, we can observe that R-TSVM can achieve better and more robust results than previous random sparsification-based methods, further verifying the effectiveness of our methods.


\begin{figure}[tb]
\centering
\subfloat[DARE-Ties]{
\includegraphics[width=0.3\linewidth]{fig/seed_dare_ties.pdf}
}
\subfloat[R-TSVM]{
\includegraphics[width=0.3\linewidth]{fig/seed_R_TSVM.pdf}
}
\caption{Comparisons between the random sparsification strategy (e.g.DARE-Ties) and SVD-based strategy (R-TSVM) on Mistral under static optimization settings adopting different seeds. R-TSVM can achieve more stable results than random sparsification methods. }
\label{seed}
\end{figure}

\begin{figure}[tb]
\centering
\subfloat[DARE-Ties]{
\includegraphics[width=0.3\linewidth]{fig/sparsity_dare_ties.pdf}
}
\subfloat[R-TSVM]{
\includegraphics[width=0.3\linewidth]{fig/sparsity_R_TSVM.pdf}
}
\caption{Paramerter sensitive analysis concerning sparsity factor for model merging methods on Mistral under static optimization settings.}
\label{sparsity_sensitivity}
\end{figure}

%%%%%%%%%%%%%%%%%%%%%%%%%%%%%%%%%%%%%%%%%%%%%%%%%%%%%%%%%%%%%%%%%%%%%%%%%%%%%%%
%%%%%%%%%%%%%%%%%%%%%%%%%%%%%%%%%%%%%%%%%%%%%%%%%%%%%%%%%%%%%%%%%%%%%%%%%%%%%%%

\end{document}


% This document was modified from the file originally made available by
% Pat Langley and Andrea Danyluk for ICML-2K. This version was created
% by Iain Murray in 2018, and modified by Alexandre Bouchard in
% 2019 and 2021 and by Csaba Szepesvari, Gang Niu and Sivan Sabato in 2022.
% Modified again in 2023 and 2024 by Sivan Sabato and Jonathan Scarlett.
% Previous contributors include Dan Roy, Lise Getoor and Tobias
% Scheffer, which was slightly modified from the 2010 version by
% Thorsten Joachims & Johannes Fuernkranz, slightly modified from the
% 2009 version by Kiri Wagstaff and Sam Roweis's 2008 version, which is
% slightly modified from Prasad Tadepalli's 2007 version which is a
% lightly changed version of the previous year's version by Andrew
% Moore, which was in turn edited from those of Kristian Kersting and
% Codrina Lauth. Alex Smola contributed to the algorithmic style files.
