\section{Related Works}
\label{sec:related_works}


\subsection{Simulatenous Localization and Mapping}

\textbf{Metric-Semantic SLAM.} The most referenced LiDAR SLAM pipelines include LOAM \cite{loam}, LIO-SAM \cite{lio_sam}, BALM \cite{BALM}, Fast-LIO \cite{fast_lio}, \cite{fast_lio2}. All the above techniques utilize low-level geometric features in the environment to estimate the pose and map of the robot, due to which such techniques can be limited in their accuracy when exploring large and complex indoor environments. Additionally several other SLAM techniques exist in the literature that utilize high-level semantic features additional to geometric features to improve the environment understanding and the map and pose accuracy. Some of such techniques include LeGO-LOAM \cite{lego-loam}, SA-LOAM \cite{sa-loam}, SegMap \cite{segmap}, SUMA++ \cite{suma}. LeGO-LOAM \cite{lego-loam} utilizes different planar semantics in the environment like ground plane to improve the map and pose estimate. While SA-LOAM \cite{sa-loam} utilizes high-level semantics to improve the loop closure accuracy of the underlying metric SLAM, SegMap \cite{segmap} utilizes learned high-level descriptors from the environment to perform robust localization with respect to the high-level descriptors. SUMA++ \cite{suma} completely segments the environment in different semantic features to perform object level semantic SLAM removing dynamic entities from the scene. 

However, one of the major limitations of the metric-semantic SLAM techniques is that they do no exploit the different semantic entities to perform a better map management strategies for improved optimization and map accuracy for large scale environments. Most of these techniques either clear major map elements as the map size increases to maintain real-time performance or do not provide a real-time analysis of the underlying SLAM when managing large scale maps.    

\textbf{Hierarchical SLAM.} 
To overcome the inherent problem of improving the computation with increasing map size, works like \cite{hierarchical_optimization}, \cite{information_theoretic} exploit the methods to compress the graph into different sub-graphs to maintain real-time performance without the loss of map and pose accuracy.  \cite{hierarchical_optimization} present a technique grouping nodes into different sub-graphs based on a simple distance based criteria. While \cite{information_theoretic} present an information-theoretic approach for factor graph compression where laser scans measurements
and their corresponding robot poses are removed such that
the expected loss of information with respect to the current
map is minimized. In \cite{continous_time_slam} authors present a hierarchical continuous time SLAM algorithm dividing it into local sub-graphs aggregating measurements from 3D LiDAR, the generated sub-graphs are aligned with each other using local surfel based registration techniques. In the above technique authors also use distance based heuristic to create different sub-graphs for the optimization problem. \cite{globally_consistent} propose a local and global hierarchical optimization technique using sub-map strategies similar to \cite{cartographer}. They generate local sub-maps at given distance intervals performing local optimization and then connecting different local sub-maps using global registration to perform efficient global optimization. The authors in \cite{globally_consistent} also choose a heuristic based stragety to generate the local and global maps. 

Although hierarchical SLAM shows efficient optimization of the graph for real-time performance, currently most of the techniques rely on time-distance based heuristics to formulate and optimize the hierarchical graph. 


\subsection{3D Scene Graphs}
Recently, 3D scene graphs have shown great potential in representing the environment in a more meaningful and compact manner \cite{armeni, 3dssg, 3dscene_graph, scene_graph_fusion, 3ddsg}. Additionally methods such as, \cite{hydra}, \cite{curb_sg}, \cite{s_graphs+} tightly couple the 3D scene graph with SLAM graphs exploiting the hierarchy in the environment to improve the final pose and map accuracy. \cite{hydra} focuses on generation of 3D scene graph mainly using RBG-D cameras incorporating different hierarchies of objects, places, rooms and buildings, furthermore it utilizes the hierarchy to perform a top-down and bottom-up search improving the search of loop closure candidates, thus improving the final map accuracy. \cite{curb_sg} presents a 3D scene graph for outdoor environment dividing it in hierarchies such as lanes, landmarks, intersections and environment. They factor different connections of the hierarchical graph as a factor graph, continuously optimizing the pose and map. \cite{situational_graphs, s_graphs+} represent an indoor scene as a four-layered optimizable factor graph dividing it into layers of keyframes, walls, rooms and floors, simultaneously optimizing all the layers to obtain improved map and pose accuracy. 

However, most of the works based on 3D scene graphs only focus on generating the hierarchical representation of the environment and do not exploit the intuitive nature of these hierarchical graphs to perform enhanced management/optimization of the different map elements which eventually leads to the problem of increased computation with the increase in the size of the explored scene. 

Thus in this work, we apply the concept of hierarchical SLAM to 3D scene graphs. The 3D scene graph structure allows to generate meaningful hierarchical SLAM graph instead of using distance-time based heuristics. This hierarchical SLAM graph improves the management of generated map elements to scale for large scale environments, while maintaining the pose and map accuracy with real-time performance.