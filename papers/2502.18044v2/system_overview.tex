\section{Overview}
\label{overview}


\subsection{Background}

\textbf{Situational Graphs.} We build over S-Graphs+ \cite{s_graphs+}, that estimates a layered graph representation of the environment from 3D LiDAR readings. The four layers of the graph, illustrated in Fig.~\ref{fig:system_architecture}, are as follows. The \textbf{\textit{Keyframes Layer}},  containing robot poses recorded at different distance-time intervals. The \textbf{\textit{Walls Layer}},  representing the walls in the environment as planar surfaces connected to the keyframes from which they were observed. The \textbf{\textit{Rooms Layer}}, modeling two-wall and four-wall rooms in the environment, connecting in this manner the previous walls layer. Finally, the \textbf{\textit{Floors Layer}} that represents the different floors in the building, connecting them with the underlying rooms layer.  

\subsection{System Architecture} \label{subsec:system_architecture}
Fig.~\ref{fig:system_architecture} shows the complete architecture of our approach. The front-end implement plane and room segmentation methods, as in \cite{s_graphs+}, with the addition of a floor segmentation module able to detect changes in the floor levels by measuring changes in the slope of the trajectory of keyframes. In practice, floor segmentation will be highly relevant to perform loop closure only within the same floor. This is a reasonable strategy, as local sensor readings are commonly limited to a single floor and connecting staircases. Additionally, it removes a high rate of false positives, caused by aliasing between floor levels.

Our back-end of the proposed approach acts on a four-layered optimizable factor graph, as in \cite{s_graphs+}, but adds several optimization strategies that leverage the hierarchy in the graph. Specifically, the optimization is performed at three levels. A local optimization acts over a window of $n$ keyframes every time new nodes or edges are added to the factor graph. A floor-level global optimization runs every time a loop closure candidate is found between two keyframes. Unlike typical global optimization threads in SLAM, that operate on all keyframes, our floor-level optimization focuses on a subset of the graph containing only the current floor level. It optimizes the keyframes and all the connected layers in the current floor level, and previously connected floor levels are only optimized when the current floor is revisited. A room-level local optimization marginalizes redundant keyframes and its connections observing the same room. This hierarchical optimizations minimizes computational overhead while maintaining the map and pose accuracy across connected floor levels. The global state $\mathbf{s}$ that we estimate is as follows
%
\begin{multline}
\mathbf{s} = [
\leftidx{^M}{\boldsymbol{\xi}}_{{1}}, \ \hdots, \ \leftidx{^M}{\boldsymbol{\xi}}_{F}, \\ 
\leftidx{^M}{\mathbf{x}}_{{R_1}_{\boldsymbol{\xi}_1}}, \ \hdots, \ \leftidx{^M}{\mathbf{x}}_{{R_T}_{\boldsymbol{\xi}_F}}, \ \leftidx{^M}{\boldsymbol{\pi}}_{{1}_{\boldsymbol{\xi}_1}}, \ \hdots, \ \leftidx{^M}{\boldsymbol{\pi}}_{{P}_{\boldsymbol{\xi}_F}}, \\
\leftidx{^M}{\boldsymbol{\rho}}_{{1}_{\boldsymbol{\xi}_1}}, \ \hdots, \ \leftidx{^M}{\boldsymbol{\rho}}_{{S}_{\boldsymbol{\xi}_F}}, \ \leftidx{^M}{\boldsymbol{\kappa}}_{{1}_{\boldsymbol{\xi}_1}}, \ \hdots, \  \leftidx{^M}{\boldsymbol{\kappa}}_{{K}_{\boldsymbol{\xi}_F}}, \ \\  \ \leftidx{^M}{\mathbf{x}}_{O}]^\top
\end{multline}
%
where $\leftidx{^M}{\boldsymbol{\xi}}_{f} \in \mathbb{R}^3, f \in \{1, \hdots \, F \}$ are the $F$ floors levels. All the nodes in the state at a given time contain the semantic of the floor-level it belongs to. $\leftidx{^M}{\mathbf{x}}_{{R_t}_{\boldsymbol{\xi}_f}} \in SE(3), \ t \in \{1, \hdots, T\}$ are the robot poses belonging to a given floor level. $\leftidx{^M}{\boldsymbol{\pi}}_{{i}_{\boldsymbol{\xi}_f}} \in \mathbb{R}^3, \ i \in \{1, \hdots, P\}$ are the plane parameters of the $P$ wall planes belonging to a given floor level. $\leftidx{^M}{\boldsymbol{\rho}}_{{j}_{\boldsymbol{\xi}_f}} \in \mathbb{R}^3, \ j \in \{1, \hdots, S\}$ contains the parameters of the $S$ four-wall rooms and $\leftidx{^M}{\boldsymbol{\kappa}}_{k} \in \mathbb{R}^3, \ k \in \{1, \hdots, K\}$ the parameters of the $K$ two-wall rooms, all belonging to a given floor-level ${\boldsymbol{\xi}_f}$. Finally, $\leftidx{^M}{\mathbf{x}}_{O}$ is the estimated drift between the map frame and the odometry frame. 

\begin{figure}[t]
  \centering
  \includegraphics[width=0.65\columnwidth]{images/stairway_detection.png}
  \caption{\textbf{Stairway Detection}, with the keyframes corresponding to stairways as blue dots, and those corresponding to different floor levels as red and green dots.}
  \label{fig:stairway_detection}
\end{figure}

\section{Front-End}

\subsection{Floor Segmentation}

\textbf{Floor Center.} The floor segmentation module extracts the center of a current floor level from the largest wall planes currently extracted in a scene. Each time the pipeline is run, it creates a default floor node with the center placed at the origin of frame $M$. It then utilizes the information from all walls at the current floor level $\boldsymbol{\xi}_f$ to create a sub-category of wall planes $\leftidx{^M}{\boldsymbol{\Pi}_{i_{\boldsymbol{\xi}_f}}}$ where $i=\{1, \hdots, T \}$. The walls sub-categories consist of $x$ vertical planar surfaces and $y$ vertical planar surfaces based on their normal orientation. After receiving a plane set, it computes the widths ${w}_x$ between all $x$-direction planes and similarly ${w}_y$ for $y$-direction planes as in \cite{s_graphs+}. 

The wall plane set with the largest $w_x$ and $w_y$ is the chosen candidate for the current floor level. These planar pairs in both $x$ and $y$ direction undergo an additional dot product check between their corresponding normal orientations, $|\mathbf{n}_{x_{a_1}} \cdot \mathbf{n}_{x_{b_1}}| < t_n$ and $|\mathbf{n}_{y_{a_1}} \cdot \mathbf{n}_{y_{b_1}}| < t_n$, to remove wall planes originating outside the building structure. The floor segmentation then computes the floor center node using the obtained wall plane candidates as explained in \cite{s_graphs+}.  

\textbf{Stairway Detection.} 
The stairway detection module is responsible for identifying transitions between floor levels and associating corresponding keyframes with the detected stairs. We follow a process similar to \cite{lexis} but extend it to our use case for detecting multiple floors and associating keyframes belonging to stairways and different floor-levels (see Fig.~\ref{fig:stairway_detection}). 

Our process relies on the analysis of sequential keyframes captured during navigation, particularly focusing on their vertical displacement. 
It begins by maintaining a queue of keyframes that are sequentially analyzed to compute the slope of their vertical trajectory. The slope is calculated using linear regression on the height values extracted from the poses of the keyframes. This slope quantifies the rate of vertical displacement and serves as the primary indicator for stairway traversal with its corresponding sign distinguishing upward or downward movements. A slope above threshold $\tau_s$ is identified as the start of the stairway, after which all the keyframes are added into stairway keyframe sequence. When the gradient drops below $\tau_s$ we mark end of that stairway, with all the keyframes $k_s$ within the sequence being part of the stairway sequence. A new floor node is then created with its height set using the height of the final keyframe in the sequence and all subsequent keyframes assigned to the new floor-level. 

% Keyframes associated with these transitions are temporarily stored and refined through a filtering process to remove outliers or irrelevant data points. 
% This filtering is achieved by retaining only the most relevant keyframes based on their trajectory consistency. When the slope stabilizes near zero, indicating a return to a level surface, the stair traversal is deemed complete. At this point, the system identifies the height of the new floor level using the final keyframe in the sequence and assigns it as a new floor node in the map.

\subsection{Floor-based Loop Closure}
One of the challenges for SLAM in large environments with multiple floors is the similarity in the geometry and visual appearance of rooms and environments in same areas of the different floors. This aliasing cause conventional geometry- or semantic-based loop closure algorithms to fail, leading to incorrect associations between keyframes from different floors. Such errors can severely degrade the accuracy of the estimated maps.

To address this issue, we leverage the inherent hierarchical structure of our graph in combination with the floor segmentation module to incorporate floor-level information across all layers. Specifically, given a complete set of keyframes $\boldsymbol{K}_i$ we create a subset of keyframes $\boldsymbol{K}_f$, $\forall f \in \boldsymbol{\xi}_f$ belonging to the current floor-level $ \boldsymbol{\xi}_f$, and perform a scan matching-based loop-closure algorithm \cite{s_graphs+} for this subset, ensuring that keyframes from different floors are excluded from matching. 

To avoid false positives in transitional scenarios, such as climbing up/down of the stairs, our floor-based loop closure is temporarily disabled in such areas, as LiDAR measurements might capture regions that may not distinctly belong to any single floor. Floor-based loop closure is resumed once there is a full transition to a new floor-level. 

\subsection{Room Keyframe Segmentation} \label{sec:room_keyframe_segmentation}
In order to perform room-level local optimization (explained in Section.~\ref{sec:back_end}) we need to identify keyframes which are bounded within a four-wall room. We exploit the hierarchy within a room and its corresponding four-walls to identify the keyframes set $\boldsymbol{K}_{\rho_{i}}$ belonging to a room $\rho_i$. To compute if a keyframe lies within a room, we first compute the vector $\boldsymbol{v}_d = \boldsymbol{p} - \boldsymbol{q}_i$, where $\boldsymbol{p}$ is the translation component of the keyframe and $\boldsymbol{q}_i$ is the point on one of the planes belonging to the room. We then evaluate the dot product of $\delta = \boldsymbol{n}_i \cdot \boldsymbol{v}_d$, where $\boldsymbol{n}_i$ is the normal orientation of the plane. Keyframe positions with $\delta$ positive are considered to be bounded within the walls of the room. 

\section{Back-End} \label{sec:back_end}

The global state of the robot defined in \ref{subsec:system_architecture} is optimized in three different stages exploiting the hierarchy of the graph. We mainly detail the optimization strategies, where relies the novelty of the paper. Details on the formulation of the cost function can be found in \cite{s_graphs+}.

\begin{figure}[b]
  \centering
  \includegraphics[width=.8\columnwidth]{images/local_optimization.png}
  \caption{\textbf{Local Optimization.} Orange- and blue-colored keyframes, along with their connected layers, are included in the local optimization, with blue keyframes being fixed. Red keyframes are not incorporated in the optimization, as they are outside the optimization window.}
  \label{fig:local_optimization}
\end{figure}


\subsection{Local Optimization}
As a robot navigates through a scene we generate specific keyframe nodes and its observations thus incorporating new nodes and its edges within the graph. Every time the graph increases in size we perform a local optimization over a subset of a graph called local graph choosing a window of last $\boldsymbol{K}_n$ keyframes. In order to incorporate other hierarchical connected layers from the graph in the local optimization, we first check the connections of $\boldsymbol{K}_n$ with the walls layer, incorporating all the wall $\boldsymbol{\pi}_p$ nodes and their edges with keyframes $\boldsymbol{K}_n$. Given all the wall $\boldsymbol{\pi}_p$ within the window, if the entire set or a subset of it have connections to two-wall or four-wall rooms we incorporate these rooms $\boldsymbol{\rho}_s$ and $\boldsymbol{\kappa}_k$ within the local optimization. Furthermore, we incorporate into the local graph the corresponding the floor node $\boldsymbol{\xi}_f$ the robot is currently navigating. In order to maintain consistency of the graph during optimization, we follow the strategy proposed in \cite{orb_slam3} to fix keyframes that are outside the current local optimization window but observe any of the walls $\boldsymbol{\pi}_p$ inside the local window. The state $\boldsymbol{s}_l$ is be optimized as: 
%
\begin{equation}
    \boldsymbol{\hat{s}}_l = \argmin_{\boldsymbol{s}_l} (c_{\boldsymbol{k}_n}, c_{\boldsymbol{k_g}}, c_{\boldsymbol{\pi}_p}, c_{\boldsymbol{\rho_s}}, c_{\boldsymbol{\xi}_f})
\end{equation}

where $c_{\boldsymbol{k}_n}, c_{\boldsymbol{\pi}_p}, c_{\boldsymbol{\rho_s}}$ are the cost functions for the $n, p, s$  unfixed keyframes, walls and rooms, ${\boldsymbol{k}_g}$ is the cost of the $g$ fixed keyframes and $ c_{\boldsymbol{\xi}_f}$ is the cost of the $f^{th}$ floor node currently being explored. 
Fig.~\ref{fig:local_optimization} shows the local optimization strategy, with orange color keyframes along with the connected layers being optimized while the fixed keyframes highlighted in blue. 

\begin{figure}[t]
  \centering
  \includegraphics[width=0.75\columnwidth]{images/floor_global_optimization.png}
  \caption{\textbf{Floor-level Global Optimization.} Floor-1 keyframes with orange colored boxes along with the connected wall, room and floor nodes are included in the floor-level global optimization on detection of a loop closure candidate on floor-1. Keyframes and its connected components from floor-0 are ignored during this optimization.}
  \label{fig:floor_global_optimization}
\end{figure}


\subsection{Floor-level Global Optimization}

%We perform an improved strategy for global optimization which  takes into account the floor levels visited by a robot to optimize for only the nodes and edges in the hierarchical graph which can be effected by global optimization. 
Floor-level global optimization is performed in mainly two situations, specifically, when a loop closure is found between the keyframes of a floor-level, and when a duplicate wall is found when a room is re-detected as explained in \cite{s_graphs+}. Our main motivation for this strategy is as follows.  When a new floor-level is explored, and a loop closure is found between two keyframes of such floor-level, this loop closure does not affect the keyframes or other connected layers of the preceding floor-levels. The same reasoning applies when duplicate walls are found for a detected room. 

When executing floor-level global optimization for floor-level $f$, we create a subset of the graph containing floor node $\boldsymbol{\xi}_f$, keyframe nodes $\boldsymbol{K}_{{n}_{\xi_f}}$ and its edges with the neighboring keyframes, then incorporate all wall nodes $\boldsymbol{\pi}_{p_{\xi_f}}$ with their edges to keyframe nodes $\boldsymbol{K}_{{n}_{\xi_f}}$. Floor-level global optimization incorporates all previously identified loop closure (relative pose) edges between the keyframes of that floor-level. 
We also incorporate four-wall $\boldsymbol{\rho}_{n_{\xi_f}}$ and two-wall room nodes $\boldsymbol{\kappa}_{n_{\xi_f}}$ with their edges to the underlying walls, finally also incorporating the edges between these rooms and the current floor node. To maintain consistency of the graph during optimization, we fix a keyframe which can either be the initial keyframe when the robot starts navigating, or it can be the last keyframe from the previous visited floor-level, this assures that the current floor-level map estimates do not diverge away from the previous floor-level map (See Fig.~\ref{fig:floor_global_optimization}). The state $\boldsymbol{s}_g$ using floor-level global optimization is optimized as:
%
\begin{multline}
\boldsymbol{\hat{s}}_g = \argmin_{\boldsymbol{s}_g} (c_{\boldsymbol{K}_{{n}_{\xi_f}}}, c_{\boldsymbol{K}_{{g}_{\xi_f}}}, c_{\boldsymbol{K}_{{l}_{\xi_f}}}, \\ 
c_{\boldsymbol{\pi}_{p_{\xi_f}}}, c_{\boldsymbol{\rho}_{s_{\xi_f}}}, c_{\boldsymbol{\kappa}_{k_{\xi_f}}}, c_{\boldsymbol{\xi}_f})
\end{multline}
%
where $c_{\boldsymbol{K}_{{n}_{\xi_f}}}, c_{\boldsymbol{K}_{{g}_{\xi_f}}}, c_{\boldsymbol{K}_{{lc}_{\xi_f}}}$ are the cost functions for the $n$ unfixed, $g$ fixed, $l$ loop closure keyframes. $c_{\boldsymbol{\pi}_{p_{\xi_f}}}, c_{\boldsymbol{\rho}_{s_{\xi_f}}}, c_{\boldsymbol{\kappa}_{k_{\xi_f}}}$ are the costs of $p$, $s$ and $k$ walls, two-wall and four-wall rooms respectively. $c_{\boldsymbol{\xi}_f}$ is the cost of the current floor node being explored. 

As a relevant effect of floor-level global optimization, if a robot \textit{revisits} a given floor-level and finds a suitable loop closure candidate, the floor-level global optimization incorporates all the nodes and edges not only from the current floor-level but also from all the previous visited ones. This ensures the correction of all the accumulated errors from the revisited floor-level, as well as the previously visited ones.   

\subsection{Room-level Local Optimization}

Room-level local optimization is performed every time a four-wall room (bounded by four walls) is detected by the room segmentation module and room keyframes belonging to that particular room are identified (Section.~\ref{sec:room_keyframe_segmentation}). The room-level local optimization creates a graph subset containing the room node $\boldsymbol{\rho}_i$ along with the walls nodes $\boldsymbol{\pi}_{p_{\rho_i}}$ lying within the room along with the connected keyframes $\boldsymbol{K}_{n_{\rho_i}}$. Additionally as in local and floor-level global optimization to maintain the consistency of the graph we incorporate fixed set of keyframes $\boldsymbol{K}_{g_{\rho_i}}$ which are outside the room but observe the walls of the room. The state $\boldsymbol{s}_r$ during room-level local optimization is given as:

\begin{equation}
    \boldsymbol{\hat{s}}_r = \argmin_{\boldsymbol{s}_r} (c_{\boldsymbol{k}_{n_{\rho_i}}}, c_{\boldsymbol{k}_{g_{\rho_i}}}, c_{\boldsymbol{\pi}_{p_{\rho_i}}}, c_{\boldsymbol{\rho_i}})
\end{equation}

Where $c_{\boldsymbol{k}_{n_{\rho_i}}}, c_{\boldsymbol{k}_{g_{\rho_i}}}$ is the cost for $n$ unfixed and $g$ fixed keyframes and $c_{\boldsymbol{\pi}_{p_{\rho_i}}}$ and $c_{\boldsymbol{\rho_i}}$ are the costs related to $p$ walls and the $i$-th room. 

The main strategy of the room-level local optimization is to marginalize redundant keyframe nodes and its edges that observe the same room structure. Thus, after a room-level local optimization is performed, all the keyframe nodes and it corresponding edges except the first keyframe node/edges are marginalized out from the main global graph. 
%The marginalized information from the room-level local optimization step is included to generate a compressed global graph excluding the marginalized keyframes and their edges. 
Marginalization leads to a generation of a disconnected graph between keyframes. These disconnections are between the marginalized keyframes and their non-marginalized neighbors. To obtain a connected global graph, we incrementally check for all the edges $e_{i,n}$, between the $i$-th marginalized keyframe $\boldsymbol{K}_i$ with $n$-th non-marginalized neighbor $\boldsymbol{K}_n$, and connect the closest non-marginalized neighbors $\boldsymbol{K}_{i-1}$ and $\boldsymbol{K}_n$ with a new edge $e_f$ (See Fig.~\ref{fig:room_local_optimization}). The information matrix of the new edge is the summation of the information matrices of the edges removed in $e_{i,n}$:
%
\begin{equation}
    \Omega_{e_f} = \sum_{(\boldsymbol{K}_i, \boldsymbol{K}_n) \in \mathcal{E}_m} \Omega_{e_{i,n}}
\end{equation}
%
where:
\begin{itemize}
    \item $\Omega_{e_{i,n}}$ is the information matrix of the original edge $e_{i,n}$.
    \item $\Omega_{e_f}$ is the resulting information matrix for the new edge $e_f = (K_{i-1}, K_n)$.
\end{itemize}

The new edge $e_f$ is added back to the global graph edges:
\begin{equation}
    \mathcal{E}_{global} \leftarrow \mathcal{E}_{global} \cup \{ (K_{i-1}, K_n, \Omega_{e_f}) \}
\end{equation}

The room-level local optimization is always performed after the execution of at least one of the local or floor-level global optimization.


\begin{figure}[t]
  \centering
  \includegraphics[width=0.85\columnwidth]{images/room_local_optimization.png}
  \caption{\textbf{Room-level Local Optimization.} Overview of the compressed graph after room-level local optimization is performed. Orange box shows the nodes used for performing room-level local optimization and box with dotted red lines refer to the keyframes marginalized after the optimization and green line refers to the new edge added between the first keyframe of the room with the first keyframe outside the room.}
  \label{fig:room_local_optimization}
\end{figure}

