\section{Introduction}

SLAM, that has typically focused on geometric aspects, should also enable a robot with a high-level understanding of its environment \cite{cadena2016past}. The combination of 3D scene graphs with traditional SLAM graphs have recently sparked great interest given its compact and intuitive approach to representing an environment as a hierarchy of layers \cite{hydra}, \cite{curb_sg}. In this direction, Situational Graphs (\textit{S-Graphs}) \cite{s_graphs+} model the environment as a four-layered optimizable graph composed of of \textit{Keyframes}, \textit{Walls}, \textit{Rooms} and \textit{Floors}. As the robot explores large environments, however, the size of the graph grows, along with the its computational footprint. Essentially, while \textit{S-Graphs} and 3D scene graphs are motivated by the need of higher-level representations, none of them have yet tailored their optimization to the specifics of these hierarchical representations. Leveraging them will result in a lower computational footprint, enabling real-time graph optimization for larger maps, which will result finally in enhanced accuracy.

\begin{figure}[t]
  \centering
  \includegraphics[width=1\columnwidth]{images/front_image.png}
  \caption{Three-story building map generated using \textit{S-Graphs 2.0}. The different floor levels estimated by our method are plotted in green, red, and orange colors. The estimated robot trajectory is plotted in black.}
  \label{fig:front_image}
\end{figure}

In this paper we present Situational Graphs 2.0 (\textit{S-Graphs 2.0}), that builds over the work of \cite{s_graphs+} and contributes with two novel approaches that improves its robustness and computational footprint. Our first novelty consists on proposing a floor detection module capable of segmenting different floor levels and the stairways that connect them. This floor-level information is incorporated in all the rooms, walls and keyframes layers of the graph ensuring robust localization and loop closures, even across similarly looking areas on different floors. Our second novelty is the proposal of an optimization strategy that leverages the hierarchical structure of the graph to reduce its computational complexity, while maintaining pose and map accuracy. Specifically, as a robot explores an environment at a given floor-level, it estimates a hierarchical graph by local optimization over a subset of the last $n$ keyframes and its connected layers. For loop closures at a given floor-level, we run a floor-level global optimization, optimizing only the subgraph of that floor-level. Additionally, on detection of a room candidate bounded by four walls, we exploit the \textit{room-wall} hierarchy to perform room-level local optimization to marginalize keyframes and its connections removing redundant keyframes observing the same room. To summarize our main contributions are:


\begin{itemize}
    \item \textit{S-Graphs 2.0}, a hierarchical SLAM leveraging semantic relations for efficient management and optimization of robot poses and map elements.
    \item Floor-level loop closure for robust localization in similar-looking environments at different levels. 
    \item Floor- and room-level hierarchical optimization strategies, marginalizing out unnecessary entities while maintaining the map accuracy.
\end{itemize}





\begin{figure*}[t]
  \centering
  \includegraphics[width=1.8\columnwidth]{images/system_architecture.png}
  \caption{\textbf{System Architecture.} The inputs to our method are the 3D LiDAR data along with the odometry measurements. Its contains different modules in the front-end modules for generating the four-layered hierarchical graph and organizing it floor-level. A back-end module which exploits the hierarchy in the graph to apply different optimization strategies.}
  \label{fig:system_architecture}
\end{figure*}
