\section{Experimental Results}

\subsection{Methodology}

We validate our algorithm on different real indoor environments from university buildings to construction sites, validating it with different state-of-the-art SLAM algorithms including the baseline \textit{S-Graphs+} \cite{s_graphs+}. Our datasets are recorded using either using Velodyne VLP-16 or Ouster OS-1 64 3D LiDAR, with the odometry generated either using robot encoders or FAST-LIO2 \cite{fast_lio2}. We perform the experiments in real in-house datasets dividing them into single-floor or multi-floor datasets.   

\textbf{Single-Floor Dataset.} 
The single-floor dataset consists of small-scale indoor environments with the robot navigating around one single floor. The main motivation for these experiments is to fairly validate the accuracy and cost reduction of our S-Graphs 2.0 against baselines, as most baselines accumulate significant drift or incur in false loop closures for multi-floor scenes. The first two experiments, \textit{C1F1} and \textit{C1F2}, are performed on two different floors of a construction site. Additionally, \textit{C2F0}, \textit{C2F1}, and \textit{C2F2} consist of three different floors of an ongoing construction site combining four individual houses. \textit{C3F1}, and \textit{C3F2} are two different floors two combined houses. To validate the accuracy of the given approaches we use the RMSE of the generated 3D map against the ground truth 3D map. Furthermore, we also report the computational time of our approach against the most accurate baseline \textit{S-Graphs+} \cite{s_graphs+}. 

\textbf{Multi-Floor Dataset.}
The multi-floor dataset is aimed at validating the accuracy of our proposed approach with baselines in presence larger trajectories covering multiple floors. It consists of five datasets ranging from \textit{BE1} and \textit{BE2} which are university buildings where the robot navigates the three and two floors respectively. \textit{LC1} consists of a university building with long corridors and robot traversing two floors with both the floors resembling each other. \textit{CS1} and \textit{CS2}are buildings in construction with robot traversing all the three floors. Due to absence of ground truth, to validate the accuracy of each approach we use the Mean Map Entropy (MME) presented in \cite{efficient_continous_SLAM} to measure the sharpness of the generated map. Additionally, to evaluate the accuracy of the algorithms to properly segment multiple floors, we perform an evaluation of Intersection over Union (IoU) of histogram of occurrences of $z$ values per floor-level from the generated map versus the percentage ground truth time spent per floor-level.  
\begin{table}[t]
\centering
\renewcommand{\arraystretch}{1.1} 
\setlength{\tabcolsep}{4pt}       
\scriptsize
\caption{Point cloud RMSE [cm] for our single-floor dataset. Best results are boldfaced, second best underlined.}
\begin{tabular}{l|ccccccc|c}
\toprule
\multirow{2}{*}{\textbf{Method}} & \multicolumn{7}{c|}{\textbf{Point Cloud RMSE [cm] $\downarrow$}} & \multirow{2}{*}{\textbf{Avg}} \\
\cmidrule(lr){2-8}
 & \textit{C1F1} & \textit{C1F2} & \textit{C2F0} & \textit{C2F1} & \textit{C2F2} & \textit{C3F1} & \textit{C3F2} \\
\midrule
HDL-SLAM \cite{hdl_graph_slam} & 33.5 & 19.8 & 18.5 & 21.1 & 19.5 & 22.9 & 19.4 & 22.1 \\
ALOAM \cite{loam}             & 52.6 & 33.6 & 34.1 & 45.1 & 29.9 & 36.5 & 43.4 & 39.3 \\
SCA-LOAM \cite{scan_context}   & 45.6 & 26.1 & 22.5 & 25.4 & 20.2 & 26.9 & 19.2 & 26.6 \\
BALM \cite{BALM}              & 34.9 & 22.5 & 18.2 & 19.7 & 99.0 & 26.9 & 22.6 & 34.8 \\
S-Graphs+ \cite{s_graphs+}     & \underline{32.9} & \textbf{18.9} & \textbf{16.9} & \textbf{18.9} & \textbf{17.6} & 22.3 & \textbf{18.7} & \underline{20.9} \\
\midrule
\textit{Ours}                & \textbf{31.3} & \underline{19.0} & \underline{17.0} & \underline{19.3} & \underline{18.0} & \textbf{22.0} & \underline{19.2} & \textbf{20.8} \\
\bottomrule
\end{tabular}
\label{tab:rmse_real_data}
\end{table}

\begin{table}[t]
\centering
\renewcommand{\arraystretch}{1.1} 
\setlength{\tabcolsep}{4pt}     
\scriptsize
\caption{Point cloud Mean Map Entropy (MME) for our multi floor dataset. Best results are boldfaced.}
\begin{tabular}{l|ccccc|c}
\toprule
\multirow{2}{*}{\textbf{Method}} & \multicolumn{5}{c|}{\textbf{Point Cloud MME $\downarrow$}} & \multirow{2}{*}{\textbf{Avg}} \\
\cmidrule(lr){2-6}
 & \textit{BE1} & \textit{BE2} & \textit{LC1} & \textit{CS1} & \textit{CS2} & \\
\midrule
HDL-SLAM \cite{hdl_graph_slam} & -1.23 & -1.38 & -1.31 & -0.97 & -1.14 & -1.21 \\ 
ALOAM \cite{loam}             & -1.37 & -1.23 & -1.32 & -0.73 & -1.07 & -1.14 \\ 
SCA-LOAM \cite{scan_context}   & -0.85 & -1.04 & -0.96 & -0.66 & -1.04 & -0.91 \\ 
BALM \cite{BALM}              & -1.35 & -1.27 & 1.62  & -0.77 & \textbf{-1.28} & -1.26 \\ 
S-Graphs+ \cite{s_graphs+}     & -1.23 & -1.28 & -1.37 & -1.06 & -1.11 & -1.21 \\ 
\midrule
\textit{Ours}                & \textbf{-1.58} & \textbf{-1.41} & \textbf{-1.67} & \textbf{-1.24} & \textbf{-1.28} & \textbf{-1.44} \\
\bottomrule
\end{tabular}
\label{tab:mme_real_data}
\end{table}


\begin{table}[]
\centering
\renewcommand{\arraystretch}{1.2} 
\setlength{\tabcolsep}{4pt}     
\scriptsize
\caption{Intersection over Union (IoU) for our multi-floor dataset. Best results are boldfaced.}
\label{tab:iou_multi_floor}
\begin{tabular}{l|ccccc|c}
\toprule
\multirow{2}{*}{\textbf{Method}} & \multicolumn{5}{c|}{\textbf{IoU} $\uparrow$} & \multirow{2}{*}{\textbf{Avg}} \\
\cmidrule(lr){2-6}
 & \textit{BE1} & \textit{BE2} & \textit{LC1} & \textit{CS1} & \textit{CS2} & \\
\midrule
HDL-SLAM \cite{hdl_graph_slam} & 0.76 & 0.90 & 0.55 & 0.37 & 0.35 & 0.59 \\ 
ALOAM \cite{loam}             & 0.50 & 0.78 & 0.69 & 0.69 & 0.31 & 0.59 \\ 
SCA-LOAM \cite{scan_context}   & 0.56 & 0.81 & 0.78 & 0.51 & 0.31 & 0.59 \\ 
BALM \cite{BALM}              & 0.67 & 0.71 & 0.77 & 0.79 & 0.80 & 0.75 \\ 
S-Graphs+ \cite{s_graphs+}     & 0.83 & 0.72 & 0.46 & 0.33 & 0.30 & 0.53 \\ 
\midrule
\textit{Ours}  & \textbf{0.90} & \textbf{0.91} & \textbf{0.93} & \textbf{0.90} & \textbf{0.91} & \textbf{0.91} \\
\bottomrule
\end{tabular}
\end{table}


\begin{figure}[t]
  \centering
  \includegraphics[width=0.85\columnwidth]{images/iou.png}
  \caption{\textbf{Floor Comparison.} \textit{CS2} dataset comparing the distribution of time spent per floor (ground truth) in percentage with the distribution of map-derived z-values converted to percentages, organized per floor.}
  \label{fig:iou}
\end{figure}


\subsection{Results and Discussions}

\textbf{Single-Floor Dataset.}
The results of the single-floor experiments, in Table.~\ref{tab:rmse_real_data}, show that our \textit{S-Graphs 2.0} is able to provide final map accuracy similar to the second best baseline \textit{S-Graphs+} with an average accuracy improvement of $0.48\%$. However, the benefits of the hierarchical optimization of our \textit{S-Graphs 2.0} clearly stand out in Table.~\ref{tab:computation_time}, where it has an average cost reduction of $79.87\%$ to the closest baseline. The benefits of the hierarchical optimization become even more clear for longer sequences. See, for example, how in \textit{C2F2}, \textit{C3F2} there is slight decrease in the accuracy with respect to \textit{S-Graphs+} ($2.27\%$ and $2.67\%$ respectively),but a large reduction the computation time ($85.56\%$ and $82.66\%$ respectively). Summing up, in these single-floor datasets, we show that the optimization strategy in \textit{S-Graphs 2.0} maintains the estimation accuracy while significantly reducing the computational cost. 


\begin{table}[]
\centering
\renewcommand{\arraystretch}{0.8} % reduce row height
\setlength{\tabcolsep}{3pt}       % adjust column spacing
\footnotesize
\caption{Computation time in milliseconds (ms) for our method compared against the baseline \textit{S-Graphs+}. Sequence lengths [s] are indicated for each dataset.}
\label{tab:computation_time}
\begin{tabular}{l|c|c c}
\toprule
& & \textbf{Computation Time} \\
\midrule
\textbf{Dataset} & \textbf{Seq. Length [s]} & S-Graphs+ [ms] & \textit{Ours} [ms] \\
\midrule
{\textit{C1F1}} & 487   & 74.0  & 36.0 \\
{\textit{C1F2}} & 657   & 106   & 38.0 \\
{\textit{C2F0}} & 238   & 87.3  & 4.00 \\
{\textit{C2F1}} & 672   & 169   & 23.0 \\
{\textit{C2F2}} & 1044  & 263   & 37.0 \\
{\textit{C3F1}} & 558   & 125   & 8.00 \\
{\textit{C3F2}} & 999   & 173   & 30.0 \\
\midrule
{\textit{BE1}}  & 1378  & 1106  & 88.0 \\
{\textit{BE2}}  & 1032  & 479   & 101  \\
{\textit{LC1}}  & 490   & 180   & 21.0 \\
{\textit{CS1}}  & 3000  & 1085  & 10.0 \\
{\textit{CS2}}  & 690   & 126   & 13.0 \\
\midrule
\textbf{Avg} & --  & 331.1 & \textbf{34.08} \\
\bottomrule
\end{tabular}
\end{table}

\begin{figure}[t]
\centering
\begin{subfigure}[t]{.45\textwidth}
\centering
\includegraphics[width=1\textwidth]{images/s_graphs_multi_map.png}
\caption{{\textit{S-Graphs 2.0}}}
\label{fig:3d_map_s_graphs}
\end{subfigure}
\begin{subfigure}[t]{0.45\textwidth}
\centering
\includegraphics[width=1\textwidth]{images/s_graphs_map.png}
\caption{\textit{S-Graphs+}}
\label{fig:3d_map_hdl_slam}
\end{subfigure}
\caption{\textbf{Qualitative Results} comparing \textit{S-Graphs 2.0} and \textit{S-Graphs+} in the multi-floor dataset \textit{LC1}. Note how the two floor levels collapse for \textit{S-Graphs+}, while our \textit{S-Graphs 2.0} shows correct results.}
\label{fig:3d_map_l1c1}
%\vspace{-2mm}, while our 
\end{figure}


\textbf{Multi-Floor Dataset.}
Table.~\ref{tab:mme_real_data} shows the Mean Map Entropy (MME) for the multi-floor dataset. Our method outperforms all the baselines by an average of $25.67\%$ when comparing the MME. Given the large area covered in these multi-floor experiments, all baselines fail at giving reasonable estimates, being unable to maintain the separation between the floor levels as well as accumulating significant drift. This can be assessed in  Fig.~\ref{fig:3d_map_l1c1}, that shows the maps estimated by \textit{S-Graphs 2.0} and \textit{S-Graphs+} for \textit{LC1}. As this environment is highly aliased between floors, \textit{S-Graphs+}'s loop closure gives a high number of false positives between levels. Our approach, given the floor segmentation, stairway detection, and floor-based loop closure is able to maintain the accurate map while segmenting the keyframes and its connected layers based on the given floor-level. Fig.~\ref{fig:3d_map_CS2} shows qualitative results for dataset \textit{CS2} demonstrating that our method is capable of segmenting the three floor-levels while maintaining a good map accuracy compared to \textit{S-Graphs+} (which again performs inaccurate loop closures between floor levels) and BALM. Table.~\ref{tab:iou_multi_floor} and Fig.~\ref{fig:iou} also validate the proper floor segmentation by comparing the IoU percentage of $z$ (height) values per floor-level from the maps generated by the algorithms versus the percentage ground truth time spend per floor-level. Observe that the IoU of our \textit{S-Graphs 2.0} correlates perfectly with the ground truth time, \textit{BALM} presents a weaker correlation and the rest of baselines offer very poor results.  

Furthermore, as can be seen from Table.~\ref{tab:computation_time} the average reduction of computation time in multi-floor dataset of our approach compared to \textit{S-Graphs+} is an average of $92.2\%$ given the fact that \textit{S-Graphs+} does not perform efficient computation with increased length of the sequence as compared to our approach which efficiently handles and optimizes the keyframes and its connected semantic-relations based on the local, floor-local and room-level local optimizations while performing accurate floor-based loop closures. Additional qualitative results: \url{https://www.youtube.com/watch?v=vOH8X7XnQms}. 

\begin{figure}[t]
\centering
\begin{subfigure}[t]{.2\textwidth}
\centering
\includegraphics[width=1\textwidth]{images/cs2_s_graphs_multi.png}
\caption{{\textit{S-Graphs 2.0}}}
\label{fig:3d_map_s_graphs}
\end{subfigure}
\begin{subfigure}[t]{0.2\textwidth}
\centering
\includegraphics[width=1\textwidth]{images/cs2_s_graphs.png}
\caption{\textit{S-Graphs+}}
\label{fig:3d_map_hdl_slam}
\end{subfigure}
\begin{subfigure}[t]{0.2\textwidth}
\centering
\includegraphics[width=1\textwidth]{images/cs2_balm.png}
\caption{\textit{BALM}}
\label{fig:3d_map_hdl_slam}
\end{subfigure}
\caption{\textbf{Qualitative Results} comparing \textit{S-Graphs 2.0}, \textit{S-Graphs+} and \textit{BALM} in the multi-floor dataset \textit{CS2}. Note the collapse of floor levels for \textit{S-Graphs+}, deviations for \textit{BALM} and accurate estimation results for our \textit{S-Graphs 2.0}.}
\label{fig:3d_map_CS2}
%\vspace{-2mm}
\end{figure}


