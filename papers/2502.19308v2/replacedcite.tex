\section{Background and Related Work}
\label{sec:related}

This section briefly describes the POMDP framework used for modeling our problems, and an overview of existing crop simulators and crop growth models. 
\paragraph{Partially Observable Markov Decision Process} We formulate our agromanagement problems using the framework of partially observable Markov decision process (POMDP)____.
POMDPs are well-suited for open-field agriculture problems, since many crop and soil-related features that are essential for defining the system's full state cannot be directly observed____. Formally, a POMDP is a tuple $M=\langle \mathcal{S}, \mathcal{A}, \mathcal{P}, \mathcal{R}, \Omega, \mathcal{O} \rangle$ where $\mathcal{S}$ is a set of states, $\mathcal{A}$ is a set of actions, $\mathcal{P}:\mathcal{S}\times \mathcal{A}\times\mathcal{S}\to[0,1]$ is the transition kernel, and $\mathcal{R}:\mathcal{S}\times\mathcal{A}\to\mathbb{R}$ is the reward function. $\Omega$ is the set of possible observations and $\mathcal{O}:\mathcal{S}\times\mathcal{A}\times\Omega\to[0,1]$ is the probability of obtaining observation $o$ when taking action $a$ in state $s$. A reward discount factor $\gamma$ determines the importance of immediate versus future rewards. The RL agent computes a policy $\pi:\Omega\times \mathcal{A}\to[0,1]$ that maximizes the expected sum of discounted rewards, $\mathbb{E}_{\rho^\pi}\left[\sum_{t=0}^T\gamma^t\mathcal{R}(s_t,a_t)\right]$, where $\rho^\pi$ is the distribution of states and actions induced by the policy $\pi$ and $T$ is the time horizon. 

\paragraph{RL for Crop Management}
Building on RL's success in robotics, autonomous driving, and healthcare, there is growing interest in applying RL to optimize crop yield____. While RL has proven effective in controlled greenhouse environments____, its application in open-field agriculture remains limited due to reduced sensing capabilities and long growing seasons. ____ proposed an imitation learning approach to learn expert actions under partial observability, but it has not been tested in the real-world. To bridge this gap, several crop simulators have been developed. CropGym simulates winter wheat in a nitrogen-limited soil via a Gym wrapper around a CGM____. Gym-DSSAT focuses on maize growth optimization through fertilization and irrigation decisions____. CyclesGym, built around the Cycles CGM____, focuses on learning crop rotation strategies for annual crops but is limited to soybeans and maize, lacking support for perennial crop modeling____. Table~\ref{tab:crop_benchmarks} summarizes the capabilities of different crop simulators.

Existing crop simulators support RL training for fertilization and irrigation but \emph{lack support for perennial crops}, a key research area____. Additionally, customization is infeasible without expert knowledge of the underlying CGMs, since most CGMs are run through separate executables. In contrast, WOFOSTGym offers easy domain customization for RL researchers while providing high-fidelity parameters for 23 annual and two perennial crops, high-fidelity model parameters for grape phenology for 32 cultivars, and access to diverse soil types and weather patterns.

\begin{table}[t]
\begin{tabular}{l cccccccc }
\hline
Name & \makecell{Perennial Crop\\Support} & \makecell{Multiple Crops\\and Farms}& \makecell{Easily\\Customizable} & \makecell{Models Crop\\Sub-processes}  \\
\hline
CyclesGym & \textcolor{red}{\ding{55}} & \textcolor{red}{\ding{55}} & \textcolor{green}{\ding{51}}& \textcolor{green}{\ding{51}}\\
\hline
CropGym & \textcolor{red}{\ding{55}} & \textcolor{red}{\ding{55}} & \textcolor{red}{\ding{55}}& \textcolor{green}{\ding{51}} \\ 
\hline
gym-DSSAT & \textcolor{red}{\ding{55}} &\textcolor{red}{\ding{55}} & \textcolor{red}{\ding{55}}& \textcolor{green}{\ding{51}} \\
\hline
FarmGym & \textcolor{red}{\ding{55}} & \textcolor{red}{\ding{55}} & \textcolor{green}{\ding{51}} & \textcolor{red}{\ding{55}} \\
\hline
WOFOSTGym (Ours) & \textcolor{green}{\ding{51}} & \textcolor{green}{\ding{51}} & \textcolor{green}{\ding{51}} & \textcolor{green}{\ding{51}} \\
\hline
\end{tabular}
\caption{Comparison of available crop simulators based on four important desiderata for use with RL. A simulator is easily customizable if it does not require agriculture domain expertise to run different experiments. Modeling crop sub-processes (phenology, roots, stems, leaves, etc.) as it generally leads to a higher fidelity model.}
\label{tab:crop_benchmarks}
\end{table}

\paragraph{Crop Growth Models}
Crop Growth Models (CGMs) simulate the growth of crops in varying environments subject to different agromanagement decisions____.
Examples of widely-used CGMs include WOFOST____, DSSAT____, APSIM____, EPIC____, CropSyst____, Cycles____ and AquaCrop____. None of the available CGMs support perennial crops. The relevant features of these CGMs are highlighted in the Appendix. 

Our simulator is built on WOFOST, a CGM that models annual crop growth subject to nutrient (nitrogen, phosphorus, and potassium) and water-limited conditions____. We choose WOFOST CGM since it can model the growth of perennial crops with a high fidelity____. It also accounts for varying CO2 concentrations, making it valuable for climate-impacted agricultural research____. Additionally, its modular design facilitates modifications to crop process models____, and its Python implementation enables seamless integration with OpenAI Gym____.