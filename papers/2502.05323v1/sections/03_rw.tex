\section{Related Work}

\subsection{Intermediation in LMICs}


In LMICs, where access to technology or digital literacy is limited, intermediated interactions emerge as a common practice to enable technology access and use for a vast number of people. According to Sambasivan et al. ~\cite{10.1145/1753326.1753718}, intermediation is accomplished when a digitally skilled user enables information seeking and use for a ``beneficiary-user” for whom technology is inaccessible due to non-literacy, lack of technology-operation skills, or financial constraints. Parikh and Ghosh ~\cite{1626204} articulate an understanding of intermediated tasks that identifies how technology availability, existing power relationships between users, and technical literacy can impact the ways in which interactions manifest. Prior work emphasizes the critical role of human relations or ``human infrastructure” in intermediated interactions and their potential for being ``more robust and pervasive than technology networks” ~\cite{10.1145/2369220.2369258} because they can overcome access and use constraints and can influence information penetration ~\cite{10.1145/2909609.2909664, 10.1145/1753326.1753610}. The human agent facilitating intermediation affords the flexibility to accommodate unexpected needs and behaviors ~\cite{10.1145/2737856.2738023}. However, human factors of intermediaries, such as their background and motivations, can influence intermediation ~\cite{10.1145/3449118, 10.1145/3313831.3376465, 10.1145/3411764.3445410}.

Intermediaries can support new tasks in addition to continuing their primary tasks. In Tanzania, mobile money agents, whose main purpose was to assist community members with mobile money services\lrem{, also supported} \ladd{began to support} feature phone users in using a \lrem{new }newly introduced mobile app ~\cite{10.1145/3613904.3642099}. In India, local mobile shops assisted in the dissemination of health education videos in addition to continuing their usual business ~\cite{10.1145/2909609.2909655}. Regardless of the tasks they support, intermediaries often contribute more than their technical literacy - they also facilitate information flow. In Digital Green ~\cite{4937388} and Projecting Health~\cite{10.1145/2737856.2738023} in India, mediators facilitated video-based education on domain-specific topics and discussions that furthered the audience’s understanding of the video content. In Bangladesh, small businesses, such as pharmacies and grocery shops, bridged the information gap between extremely impoverished people and low-cost healthcare services, connecting clients to affordable healthcare ~\cite{10.1145/3449118}. In these examples and beyond, intermediaries have been effective not only in supporting technology access and use, but also in disseminating information and encouraging people to engage with information ~\cite{10.1145/2369220.2369253}. 


\subsection{Interventions for Person-Centered Contraceptive Education and Decision-Making Support} 

There is a growing body of work in contraceptive education and decision-support that focuses on improving preference-sensitive decision-making, meaning that method choice can depend on a variety of factors such as perception of method safety, effectiveness, and side effects ~\cite{madden2015role, marshall2016young}. Research on preference-sensitive shared decision-making with a provider during contraceptive counseling demonstrates that women and service providers differ in their preferences for and prioritization of contraceptive characteristics ~\cite{weisberg2013women}. Such work has informed the development of person-centered contraceptive decision-support tools such as My Birth Control, a web-based tool in the U.S. used before counseling to improve women’s experiences of contraceptive counseling and support them in choosing contraceptive methods that fit their needs and preferences ~\cite{dehlendorf2019cluster}. The app includes educational models and surveys to indicate preferences for method characteristics. My Birth Control increased women’s knowledge of methods, reduced decision conflict, encouraged providers to acknowledge clients’ preferences and values, and clients appeared to be more confident in their method preferences ~\cite{holt2020patient, dehlendorf2019mixed}. Our app takes inspiration from My Birth Control’s person-centered focus. 

\ladd{Person-centeredness is also evident in HCI work that explores opportunities for technology to support information seeking of sexual and reproductive health in light of the challenges and needs faced by specific populations. Dewan et al. ~\cite{10.1145/3613904.3641934} offer guidance on technology development for teenagers seeking information on reproductive health given teenagers’ information seeking practices and struggles with finding appropriate information sources. Patel et al. ~\cite{10.1145/3679318.3685380} propose technology designs that facilitate safe and appropriate sexual and reproductive care for LGBTQ+ people with uteruses, considering their experiences with in-person care.}

\lrem{However, person-centered contraceptive decision-support tools}\ladd{However, aforementioned work is situated in high-income countries. Technologies that support person-centered contraceptive decision-support} have yet to be a focus for contraceptive interventions in LMICs. There are, however, interventions such as text messaging, phone calls, or voice messaging that are designed to facilitate contraceptive education and behavior change. While these include interventions tailored to a specific population, they are intended to engage users in health-related communications, increase contraceptive knowledge, or promote contraceptive uptake and continuation, which differs from supporting person-centered decision-making ~\cite{harrington2019mhealth, mccarthy2018development, 10.1145/2702123.2702124}. A plausible explanation for the lack of person-centered contraceptive decision-making tools in LMICs may be the paucity of data on contraceptive preferences and priorities of women in these countries. \lrem{In sub-Saharan Africa, where our work takes place, t}\ladd{T}here is even less data on what adolescents consider important to their contraceptive decisions, though there are studies that demonstrate adolescents’ concerns about future fertility and side effects ~\cite{velonjara2018motherhood, ochako2015barriers, gueye2015belief}. \ladd{While there are examples of HCI-driven interventions aimed to provide adolescence related sexual and reproductive health guidance in LMICs ~\cite{10.1145/3411764.3445694, wang2022artificial}, these interventions assume that adolescents have access to personal mobile devices that can be used privately, which is not common for adolescents in many LMICs ~\cite{madonsela2023development}.}\lrem{ While there are interventions focused on contraceptive information provision}\lrem{\mbox{ ~\cite{feroz2021using}}}{\lrem{, a} \ladd{Moreover, a}pproaches to support decision-making in community-based settings, such as pharmacies where many adolescents access contraception ~\cite{gonsalves2023pharmacies, gonsalves2020mixed, radovich2018meets}, are limited ~\cite{dev2019acceptability}. Our work \ladd{aims to improve the quality and person-centeredness of contraceptive care available to AGYW by tailoring our intervention to AGYW’s needs, values, and preferences.}\lrem{ aims to reach AGYW in a setting where they access contraceptive services to improve contraceptive care.}