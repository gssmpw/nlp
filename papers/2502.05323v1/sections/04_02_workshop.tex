\subsection{Co-design Workshops}
We conducted co-design workshops to understand how to integrate the intervention into the pharmacy setting and to gather potential app \lrem{users}\ladd{stakeholders, namely AGYW and pharmacy staff,} with varied preferences to share feedback on the app and ideate new designs or content. 

\subsubsection{Study Design}
\lrem{The workshops took place in Kisumu county, Kenya. Luo is the primary ethnic group of Kisumu county where DhoLuo, Kiswahili, and English are spoken. Public health clinics offer most contraceptive methods for free or for a small fee. Pharmacies usually offer condoms, emergency contraceptives, depot medroxyprogesterone acetate (injectables, also known as ``depo”), and combined oral contraceptive pills. Some pharmacies also place contraceptive implants on site, but most refer patients to nearby clinics for implants or intrauterine devices (IUD/IUCD).} 

We conducted four workshops - three with AGYW aged 15-24 years with pharmacy-based contraceptive care experience and one with pharmacy staff - to allow for iteration in app prototype development between workshops. Workshops were conducted between April - May 2024. Workshops were led by four facilitators, all who spoke English and Kiswahili and two who also spoke DhoLuo. Facilitators were also involved in the creation and revision of the study design. Workshop activities were mainly led by two facilitators with human-centered design experience (third and fourth authors). The first author was also present for the first three workshops. 

Each workshop began with a distribution of informed consent forms, followed by an explanation of consent guidelines by study staff and a pre-workshop questionnaire administered on a tablet by study staff. After the pre-workshop questionnaire, participants re-grouped to review the background and objectives of the study, and confidentiality and consent guidelines. We then proceeded to conduct activities where participants engaged in discussion and produced artifacts (e.g., post-it notes) and facilitators took field notes. Workshops lasted between 9AM and 5PM, including \lrem{a }\ladd{one 90-minute} lunch break and \ladd{two 30-minute} tea breaks \ladd{in between workshop activities}. Numerous small revisions to the app were made in response to feedback in between workshops and after completion of all workshops. 

\subsubsection{Participant Recruitment}
AGYW participants were recruited using community-based strategies to optimize the diversity of perspectives and experiences in the three AGYW workshops by age, education, contraceptive experience and experience with mobile apps. Purposive sampling ~\cite{tongco2007purposive} was used to recruit AGYW from peer mobilizers in the community and in-person in the pharmacy settings. Snowball sampling ~\cite{parker2019snowball} was also used, by which study staff asked interested participants to recommend peers for inclusion. Female study staff who spoke DhoLuo, Kiswahili, and English provided an introduction to the study and potential participants who met the inclusion criteria were assessed for their availability for workshop dates. AGYW under 18 years of age and not emancipated minors were encouraged to speak to a trusted adult about participating in the study. Eligible AGYW met the following criteria: 1) cisgender female; 2) age 15-24 years; 3) has accessed pharmacy-based contraception in the last three months; 4) not currently pregnant; 5) speaks DhoLuo, Kiswahili, and/or English; 6) feels comfortable reading DhoLuo, Kiswahili, and/or English; and 7) willing to sign informed consent. \ladd{This work targets AGYW aged 15-24 years because we determined that this age group is representative of a population that will most likely benefit from our app. We also note that most AGYW under the age of 15 are not sexually active, as only 8\% of Kenyan AGYW aged 15-24 years experienced first sexual intercourse before the age of 15 ~\cite{kenyadhs}. We also wanted to minimize the risks to a vulnerable population by working with slightly older AGYW.}

Pharmacy staff participants were recruited from pharmacies in Kisumu and periurban areas to balance the gender and age of pharmacy staff in the pharmacy staff workshop. Eligible pharmacists met the following criteria: 1) at least 18 years of age; 2) currently working in a pharmacy providing condoms, emergency contraception, combined oral contraceptive pills, and depot medroxyprogesterone acetate (hormonal contraceptive injection); 3) currently working in a pharmacy that has a counseling room/private space; 4) speaks Kiswahili and/or English; 5) feels comfortable reading DhoLuo, Kiswahili, and/or English; and 6) willing to sign informed consent. 

We recruited 33 AGYW and 10 pharmacy staff (6 female). \ladd{We conducted four workshops:}\lrem{We had} one workshop of 12 AGYW aged 15-17 years\footnote{\lrem{There was one 19-year-old in the workshop with a younger age group but}\ladd{All except one participant in the younger age group workshop was aged 15-17 years and so} we present the \lrem{two }\ladd{AGYW} workshops as having \ladd{two} distinct age groups.}\ladd{,} \lrem{and} two workshops with 10 and 11 AGYW aged 18-24 years\ladd{, and one workshop of 10 pharmacy staff}. The sample size was determined through the review of relevant literature in the health domain using similar methods ~\cite{hunter2021designing, wilkinson2022developing, harrington2021spoiled} and guidelines in qualitative research ~\cite{sandelowski1995sample}. AGYW ranged in age from 15-24 years \lrem{(mean=19.24, SD=2.75)}\ladd{and the median age was 19 (IQR 17-21).}\lrem{ and p} \ladd{P}harmacy staff ranged in age from 32-48 years \lrem{(mean=37.10, SD=6.87)}\ladd{and the median age was 33.5 (IQR 32-40)}. \ladd{A summary of participant characteristics for the three AGYW workshops and one pharmacy staff workshop are in Tables \ref{tab:AGYW-participants} and \ref{tab:pharm-participants}, respectively.} AGYW participants and \lrem{pharmacist}\ladd{pharmacy staff} participants were compensated with 1000 KSH and 2000 KSH (approximately 8 USD and 15 USD), respectively, for their participation. \ladd{The compensation was determined based on the judgment of the Kenyan ethics review board of what was appropriate and not coercive.}

\begin{table}[h!]
\caption{Participant characteristics of AGYW workshops.}
\Description{Table with characteristics and workshop ID for 3 AGYW workshops. Characteristics are in one column with median age (IQR), education level, and mobile phone use. Corresponding values are in the other columns where each column represents one workshop.}
\label{tab:AGYW-participants}
% \normalsize
\resizebox{\columnwidth}{!} {
\begin{tabular}{|p{4.4cm}||p{1.6cm}|p{1.6cm}|p{1.6cm}|}
\hline
\multicolumn{1}{|c||}{} &
\multicolumn{3}{c|}{\textbf{Workshop ID}} \\
\multicolumn{1}{|c||}{\textbf{Characteristic}} &
\multicolumn{1}{c}{\textbf{AGYW-1}} &
\multicolumn{1}{c}{\textbf{AGYW-2}} &
\multicolumn{1}{c|}{\textbf{AGYW-3}} \\
\hline
Total Participants, n & 10 & 12 & 11 \\
\hline
Median Age (IQR) & 20.5 (19.25-21.75) & 17 (15.75-17) & 21 (19-23) \\
\hline
Education Level, n &  &  & \\ 
\hspace{3mm}Primary school, not complete & 0 & 1 & 0 \\ 
\hspace{3mm}Primary school, complete & 1 & 3 & 1 \\
\hspace{3mm}Secondary school, not complete & 1 & 7 & 1\\
\hspace{3mm}Secondary school, complete & 3 & 1 & 7\\
\hspace{3mm}Post-secondary school & 5 & 0 & 2 \\ 
\hline
Mobile Phone Use, n &  &  & \\ 
\hspace{3mm}Has access to a mobile phone & 10 & 5 & 11 \\ 
\hspace{3mm}Owns a personal mobile phone & 10 & 4 & 11 \\
\hspace{3mm}Has access to a smartphone & 9 & 2 & 6 \\
\hline
\end{tabular}
}
\end{table}

\begin{table}[h!]
\caption{Participant characteristics of pharmacy staff workshop.}
\Description{Table with characteristics and workshop ID for pharmacy staff workshop. Characteristics are in one column with median age (IQR), education level, and years of professional experience. Corresponding values are in the other columns where each column represents one workshop.}
\label{tab:pharm-participants}
% \normalsize
\resizebox{\columnwidth}{!} {
\begin{tabular}{|p{7cm}||p{4cm}|}
\hline
\multicolumn{1}{|c||}{} &
\multicolumn{1}{c|}{\textbf{Workshop ID}} \\
\multicolumn{1}{|c||}{\textbf{Characteristic}} & \multicolumn{1}{c|}{\textbf{Pharmacy-1}} \\
\hline
Total Participants, n& 10 (6 female) \\
\hline
Median Age (IQR) & 33.5 (32-40) \\
\hline
%Highest Education Level &  \\ 
%\hspace{3mm}Bachelor's Degree & 1/10 \\ 
%\hspace{3mm}Diploma & 9/10 \\
Licensure, n & \\
\hspace{3mm} Pharmacy owner & 1\\
\hspace{3mm} Pharmacy technician & 8\\
\hspace{3mm} Other & 1\\
\hline
Years of Working in the Pharmacy, n &   \\ 
\hspace{3mm}5 or more years & 9 \\ 
\hspace{3mm}1-2 years & 1 \\
\hline
Median Score for Confidence in Advising AGYW (IQR) &   \\ 
\hspace{3mm}On family planning options & 5 (4-5) \\ 
\hspace{3mm}On how to use family planning methods & 5 (4-5) \\
\hspace{3mm}On family planning side effects & 5 (4.25-5) \\
\hline
\end{tabular}
}
\end{table}


\subsubsection{Data Collection}
Data was collected through pre-workshop questionnaires, artifacts (e.g., post-it notes), \ladd{written} field notes \ladd{in English, and audio-recordings. The primary data sources were artifacts and written notes that were produced for each workshop and all its activities, as the workshops were centered around small or large group activities that involved creating visual representations of ideas and concepts.}\lrem{, and audio-recordings of} \ladd{Large group interviews were audio-recorded as a secondary source of data to retain individual participant quotations and perspectives.} \lrem{ facilitated app engagement, app feedback, and group interviews.}\ladd{When participants were engaged in small-group activities or speaking one-on-one with a facilitator, it was impractical to rely on audio-recordings, so written note-taking was used.} Activities were conducted in the order they are presented in this section. \ladd{The flow of activities is described in Figure \ref{fig:flow}.}



\begin{figure*}[hbt!]
  \centering
\includegraphics[width=1.0\linewidth]{figures/flow.png}
\caption{Flow of workshop activities.}
\Description{Flow of workshop activities from left to right: pre-workshop questionnaire (1 hour), journey mapping (1-2 hours), free app engagement (20 mins), usability testing (1 hour), facilitated app engagement (1 hour), app feedback (1 hour), group interview (1-2 hours).}
\label{fig:flow}
\end{figure*}


\paragraph{Pre-workshop questionnaire}
Each participant individually completed a pre-workshop questionnaire with study staff. For AGYW, we collected 1) demographic information; 2) mobile phone and app use; 3) relationships and sexual history; and 4) reproductive history and contraceptive experience and desires. In this paper, we present data on aspects of AGYW’s demographic \lrem{information}\ladd{characteristics including age and education levels}, mobile phone \lrem{and app use}\ladd{use}, and contraceptive experiences. Other data will be presented in a future paper. For pharmacy staff, we collected 1) demographic information; and 2) confidence in advising AGYW on contraception. \ladd{For confidence in advising AGYW on contraception, we used a 5-point Likert scale, with 1 being not at all confident and 5 being extremely confident.} The pre-workshop questionnaire is included in Appendix \ref{question}. 

\paragraph{Journey mapping}
Facilitators led participants to create a journey map to visualize the entire flow of a pharmacy visit experience. We focused on ``pain points” - moments that create discomfort, embarrassment, or inconvenience - and moments where there is the potential for delight. Journey mapping lasted between one to two hours. 

For the AGYW workshop\ladd{s}, we applied different methods of journey mapping. In some workshops, participants used post-it notes to write their actions, thoughts, and goals that they have before, during, and after pharmacy visits (Figure \ref{fig:workshop} a). In other workshops, participants conducted an interactive role-play depicting a pharmacy visit, in which one participant acted as the pharmacist, another acted as an AGYW, and observing participants commented on their experiences during the role play (Figure \ref{fig:workshop} b). The purpose was to understand counseling from AGYW’s perspective.

In the pharmacy staff workshop, one participant acted as a pharmacist and one member of the study team acted as an AGYW to role-play a contraceptive counseling scenario (Figure \ref{fig:workshop} c). The purpose was to understand how a pharmacy staff would conduct counseling. After role-play, participants discussed how their experiences differed or aligned with what was presented.

\paragraph{Free app engagement}
Facilitators introduced the \ladd{Mara Divas} app to the participants, explaining its objective and the workshop’s goal to gather app feedback. Each participant was given a tablet and headphones. Participants were given unstructured time to engage with the app to gain familiarity. Workshop facilitators answered questions and documented real-time observations and feedback. Free app engagement \ladd{was conducted in both AGYW workshops and the pharmacy staff workshop. This activity} lasted approximately 20 minutes.

\paragraph{Usability testing}
In AGYW workshops, participants individually conducted a series of tasks on the app. The purpose was to test how easy the app was used by AGYW to inform changes for future design iterations. Each participant was given three tasks to complete (e.g., watch a video in DhoLuo of a peer talking about how to keep family planning methods private) and a facilitator observed and took notes as the participant attempted the tasks. We prepared 12 tasks total. \ladd{Usability testing was only conducted in the AGYW workshop}\lrem{We did not conduct usability testing in the pharmacy staff workshop} because pharmacy staff were not target users \ladd{of the app}. Usability testing lasted approximately one hour.

\paragraph{Facilitated app engagement}
As a group, participants discussed specific aspects or sections of the app. Participants were asked to discuss whether and how their opinions on contraception changed after engaging with the app, whether the content was helpful for addressing AGYW concerns, and how the app could influence their decision-making. In the pharmacy staff workshop, participants \ladd{also} discussed whether and how the app’s content differed from their training or experience. All participants also discussed the language toggle tool, audio, and video and compared their experiences and preferences with reading, listening, and watching. Discussions were audio-recorded. \ladd{Facilitated app engagement was conducted in both AGYW workshops and the pharmacy staff workshop.} This activity lasted approximately one hour.

\paragraph{App Feedback}
In a group, facilitators elicited structured positive feedback, criticisms, questions, and ideas regarding the app. We used the “I Like, I Wish, What If” method ~\cite{Dam_Siang_2023} because we thought this would be a helpful way to engage in participatory ideation for participants who do not have much experience with giving constructive critique. %Facilitators probed feedback on the app’s content, language toggle tool, audio, and video. 
Participants and/or facilitators recorded the feedback on post-it notes (Figure \ref{fig:workshop} d). Discussions were audio-recorded. \ladd{App feedback was conducted in both AGYW workshops and the pharmacy staff workshop.} This activity lasted approximately one hour.


\paragraph{Group interview}
Facilitators conducted a semi-structured group interview to discuss how participants’ experiences in the pharmacy would change after exploring the app. Participants discussed their concerns with app usage in the pharmacy setting, the pharmacist’s role in supporting AGYW with and after using the app, how to combine counseling with app engagement, the app’s usefulness to the pharmacy staff, and how contraceptive access can be made easier for AGYW. Discussions were audio-recorded. \ladd{The group interview was conducted in both AGYW workshops and the pharmacy staff workshop.} This activity lasted approximately one to two hours. 

\begin{figure*}[hbt!]
  \centering
\includegraphics[width=1.0\linewidth]{figures/workshop.jpeg}
\caption{Workshop activities. Enlarged images are in Appendix \ref{big_images}. (a) AGYW journey mapping where AGYW and facilitators wrote actions, thoughts, goals, and emotions before a pharmacy visit on post-it notes. (b) AGYW journey mapping with role-play, led by the facilitator. (c) Pharmacy staff journey mapping with role-play. (d) App feedback documented by AGYW on post-it notes using the ``I Like, I Wish, What If” method.}
\Description{Four images of workshop activities, described from left to right. (a) Poster paper with "BEFORE PHARMACY" at the top. Below, a post-it note labeled "ACTION/THOUGHT" and next to it multiple post-it notes with actions and thoughts AGYW have before visiting the pharmacy setting. Below is a post-it note labeled "GOAL" and next to it post-it notes of goals that AGYW have for visiting the pharmacy setting. Below are three post-it notes placed vertically, each with a different facial expression drawn on it, where one is happy, one is neutral, and one is sad. Post-it notes are placed next to it and connected with a string to indicate the emotional path. On each post-it note is a sticker with the same three facial expressions. (b) A girl wearing a sign labeled "AGYW" with a drawing of a girl is facing five girls (workshop participants) and a woman (workshop facilitator). (c) A sign on the wall that says "AMUA PHARMACY". A man wearing a sign labeled "Pharmacist Amua Pharmacy" with a drawing of a man is talking to a woman with a sign labeled "AGYW" and a drawing of a girl. A paper labeled ``AMUA pharmacy” is behind them, to illustrate that the characters are in a pharmacy. (d) Poster paper with "I LIKE" at the top with post-it notes of comments on what participants liked about the app, "I WISH" with post-it notes with comments on  what participants wished the app had, and "IDEAS" with post-it notes with comments on ideas participants had with improving the app.}
\label{fig:workshop}
\end{figure*}



%Figure 2. Workshop activities. (a) AGYW journey mapping where AGYW and facilitators wrote actions, thoughts, goals, and emotions before a pharmacy visit on post-it notes. (b) AGYW journey mapping with AGYW role-playing a pharmacy visit, led by the facilitator. (c) Pharmacy staff journey mapping with pharmacy staff and study staff role-playing a counseling scenario. (d) AGYW providing app feedback on post-it notes using the ``I Like, I Wish, What If” method.

\subsection{Data Analysis}


Data sources included detailed written notes \ladd{in English} recorded by a designated staff member during the workshops, selective transcriptions of workshop audio-recordings, and workshop artifacts (e.g., arranged post-it notes from journey maps and app feedback, photos). \ladd{The primary data sources were artifacts and written notes that were produced for all workshop activities and the secondary data source was audio-recordings from large group workshop activities.} %\ladd{The primary data sources were artifacts and written notes that were produced for each workshop, as the workshops were centered around small or large group activities that involved creating visual representations of ideas and concepts. Large group interview workshop activities were audio-recorded as a secondary source of data, in order to retain individual participant quotations and perspectives. When participants were engaged in small-group activities or speaking one-on-one with a facilitator, it was impractical to rely on audio-recordings, so written note-taking was used.}

Multimedia data from the workshops were analyzed using two processes in tandem: a collaborative analysis by the larger design team (including the first, second, third, and fourth authors), and an in-depth thematic analysis by the first author. Team-based collaborative analysis took place in real time immediately after each workshop and at weekly design team meetings \ladd{that} continued after workshop completion. The collaborative team was made up of Kenyan study staff and researchers from Kenyan and U.S. institutions, with expertise in human-centered design, computer science\ladd{/HCI}, and clinical reproductive health. In the meetings immediately after and between workshops, the team discussed, compared, and contrasted observations within and among workshops. The main goal of these discussions was to identify app design updates to be made for the next workshop, using key insights generated from workshop activities such as free app engagement, usability testing, facilitated app engagement, and app feedback. After data collection was complete, the collaborative design team conducted analytic virtual meetings to consolidate data sources and begin to interpret the data holistically. The team recollected and discussed workshop interactions, observations, and feedback from participants that was particularly insightful, emerged frequently, or was surprising, and cataloged the iterative changes to the app. The Kenyan team members contributed additional expertise in the local context and languages, which enriched and contextualized the data.

Concurrent with the collaborative process, the first author qualitatively analyzed the \lrem{data}\ladd{artifacts and the written notes} with a specific focus on AGYW interactions with the tablet-based app and perceptions of the role of the pharmacist with the app. Using an inductive approach, the first author independently reviewed the \lrem{data}\ladd{the artifacts and the written notes} from the journey mapping, with a focus on key actions and emotions that AGYW and pharmacists expressed, and created an initial matrix of concepts highlighting AGYW concerns, desires, and decision-making processes when accessing contraception in the pharmacy setting. These concepts contextualized participant responses generated from app-specific workshop activities\lrem{Transcripts from the workshops, alongside text from w} (i.e., free and facilitated app engagement, app feedback, group interview), which were coded using an inductive approach guided by the intermediation model. Using thematic analysis ~\cite{Braun_Clarke_2022}, the codes were consolidated into an initial set of themes and iteratively refined. Themes were then discussed between the first and last author to build consensus and develop cross-cutting insights across the workshops. \ladd{After identifying themes, the first author reviewed the audible audio-recordings to select representative participant quotations that speak to the themes and concepts and transcribed and translated them in English with support from the Kenyan study staff.} 




\subsubsection{Ethical Considerations}
Although minors were involved in this study, parental permission for study involvement was waived with approval from the \lrem{IRBs}\ladd{ethics review boards} in Kenya and the U.S. This decision was made through reflection on the reality that parental involvement in sensitive sexual and reproductive health research among adolescents may be harmful for adolescent research participants. To prepare to address conflicts arising from parents who have concerns about their child’s participation in the study, we encouraged minors to consult with a trusted adult prior to giving their assent. Participants were given a contact number so that parents may call to learn more or to arrange to talk with the study team. 
