\section{Introduction}

Information and communication technologies for development (ICTD) research ``involves a consideration of human and societal relations with the technological world and specifically considers the potential for positive socioeconomic change through this engagement” ~\cite{burrell2009constitutes}. ICTD research is typically set in low- and middle-income countries (LMICs) that face barriers - economic, access, literacy, and infrastructure - that impact how people interact with technologies and how technologies are developed, deployed, sustained, and received. In these settings, intermediated interactions emerge as a practice to overcome barriers to enable technology use and information access. Intermediated interaction is a multi-user cooperative effort in which digitally adept individuals (i.e., intermediaries) support individuals who are typically non-literate, lack technology-operation skills, and/or have financial constraints to use technology and access information ~\cite{10.1145/1753326.1753718, 1626204}.

We extend this model of intermediation to our work in a mobile health contraceptive project for adolescent girls and young women (AGYW) in Kenya, with experts in human-centered design, computer science/HCI, and clinical reproductive health in Kenya and the U.S. AGYW in sub-Saharan Africa face a disproportionately high risk of unintended pregnancy and poor health outcomes, and face unique barriers to contraceptive access due to their gender and age ~\cite{sully2020adding}. These challenges necessitate AGYW-centered contraceptive decision-support resources that reach this underserved and stigmatized population. Our work builds on the understanding that pharmacies are a key source of contraceptives for AGYW in Kenya ~\cite{gonsalves2023pharmacies, gonsalves2020mixed, radovich2018meets} and formative work that explores AGYW’s contraceptive preferences and needs. We developed a tablet-based application \ladd{(app) - the Mara Divas app -} that provides contraceptive education and decision-support for AGYW in the pharmacy setting in Kenya, in a way that aligns with their needs and preferences. Our app aims to not only provide recommendations for contraceptive methods that suit AGYW’s needs and preferences, but to also demystify information that is of importance to them. We created this mobile app to enable an assessment of its impact. Our motivation is to empower vulnerable AGYW to make informed contraceptive choices and contribute to improved reproductive health outcomes. 

We draw on the model of intermediation for two reasons. First, we want to understand how the substantial body of work in ICTD and human computer interaction for development (HCI4D) regarding intermediation could inform a successful intervention for AGYW in Kenya. Second, we want to apply lessons from this unique setting to inform a broader understanding of LMIC-based interventions. We address the following research questions:
\begin{itemize}
\item \textbf{RQ1:} How does the intermediation model allow us to understand how a pharmacy-based app for contraceptive education and decision-support for AGYW can be integrated into the pharmacy setting in Kenya?
\item \textbf{RQ2:} How does this experience inform approaches to the development of interventions for public intermediated scenarios?
\end{itemize}

We conducted workshops in Kisumu county, Kenya with AGYW aged 15-24 years and pharmacy staff to iteratively refine the app prototype and to understand how to integrate the intervention into the pharmacy setting. \ladd{Our findings highlight key moments in AGYW-pharmacist interactions in which pharmacists’ personal beliefs, interests, and priorities misalign with those of AGYW. We also describe differences in language, digital, and health literacy levels among AGYW subpopulations, and the critical roles of the pharmacist.} We contribute the following:

\begin{itemize}
\item \lrem{Descriptions of key moments in AGYW-pharmacist interactions, differences between subpopulations of AGYW, the app’s potential impact on AGYW-pharmacist interactions, and app feedback from participants that inspired changes.}\ladd{An articulation of the significance and role of intermediation by the pharmacist to facilitate app usage and more broadly, contraceptive education and decision-making for a wide range of AGYW.}
\item An argument that our intervention could have a differential impact on subpopulations of AGYW\ladd{, emphasizing the importance of extending impact to a hard-to-reach population}. 
\item \lrem{A reflection on implications for technological interventions in LMIC-based healthcare settings.}\ladd{Considerations for developing technologies for a complex ecology of LMIC-based healthcare, including commercial retail settings.} 
\end{itemize}

