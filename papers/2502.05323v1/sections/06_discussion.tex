\section{Discussion}

Our main findings are as follows: 
\begin{enumerate}
\item Pharmacists’ interests, goals, and priorities can differ from those of AGYW, contributing to AGYW’s anxiety and avoidance of approaching the pharmacy setting for contraceptive services.
\item Older AGYW aged 18-24 years and younger AGYW aged 15-17 years present different language, digital, and health literacy levels, where younger AGYW were less educated and hence less comfortable with reading, less experienced with technology, and less informed about contraception and health overall. %Younger AGYW were less comfortable reading than older AGYW, prompting them to utilize the app’s multimedia. Older AGYW were comfortable with technology use while younger AGYW required assistance. Older AGYW had foundational contraceptive knowledge and the app was able to aid in decision-making processes, while younger AGYW used the app to build foundational knowledge.  
\item Pharmacists and AGYW validated the role of the pharmacist. Pharmacists will possibly take on multiple roles after the app is introduced in the pharmacy setting to accommodate different AGYW’s needs.
\end{enumerate}

Overall, the app received positive feedback from workshop participants. Videos, especially those featuring peers, were popular among AGYW, confirming the strengths of video-based education as established in prior work in HCI4D ~\cite{10.1145/2737856.2738023, 4937388, 10.1145/3544548.3581458, 10.1145/3491102.3501950}. Co-design allowed us to amplify AGYW’s experiences with contraception and pharmacies and incorporate their voices into iterations of the app ~\cite{10.1145/3544548.3581458, 10.1145/3613904.3642532, 10.1145/3283458.3283476}.

In this section, we focus on three components: 1) potential role of intermediation in the pharmacy setting; 2) potential differential impact of LMIC-based interventions on subgroups; and 3) challenges and opportunities for technological interventions in LMIC healthcare settings.

\subsection{Intermediation in the Pharmacy Setting}
We first clarify why intermediation is an appropriate lens of analysis. The physical and human infrastructure of the pharmacy setting enables an environment where technology can be supported ~\cite{10.1145/1753326.1753718, 1626204}. The pharmacy setting affords a private room for AGYW to learn about contraception, provides a tablet that eliminates requirements of smartphone ownership, and has a pharmacist who can facilitate contraceptive education and decision-making. Similar to other ICTD/HCI4D interventions ~\cite{4937388, 10.1145/2737856.2738023}, a central motivation of this intervention is to provide complementary expertise to that of the pharmacist. In some ways, the app may have more information than what the pharmacist can provide to AGYW, given known limitations in counseling time and pharmacy staff knowledge ~\cite{gonsalves2020pharmacists, gonsalves2019regulating}. Potential intermediation in the pharmacy setting resembles those observed in LMICs where the intermediary takes on tasks in addition to their primary task ~\cite{10.1145/3613904.3642099, 10.1145/3449118, 10.1145/2909609.2909655}. With the introduction of the app, pharmacists are expected to not only help AGYW with achieving the primary goal of their pharmacy visit – to obtain contraceptive services – but also app usage. We anticipated AGYW's lack of interest in the app since it is not their main purpose for visiting the pharmacy and addressed this concern by adding interactive multimedia and creating an appealing user interface.

Next, we consider how intermediated interactions can contribute towards a successful intervention in the pharmacy setting, adding to literature that recognizes the importance of human relationships for a technological endeavor ~\cite{10.1145/2369220.2369258, 1626204, 10.1145/3613904.3642099}. 

\subsubsection{The differential role of the pharmacist}
One of our biggest learnings was the differences between older AGYW and younger AGYW. It was not feasible for us to create two apps and we wanted to make a single, robust app that fits both subpopulations. Because of the limitations of a single app, the human agent - the pharmacist - is critical for providing the flexibility to accommodate a broad range of AGYW’s needs and behaviors ~\cite{10.1145/2369220.2369258, 10.1145/3173574.3174213}. For older AGYW who had basic contraceptive knowledge, the primary benefit of the app was the reassurance it provided in their decision-making process. The app was not strictly viewed as an information provision tool, but also as a decision aid tool for evaluating contraceptive options. \ladd{Some older AGYW were more interested in obtaining in-depth information on specific contraceptive methods of interest, suggesting that they already had an idea of which contraceptive method was right for them and that they had topics in mind that they were wanting to learn about. The app provides more general information than what these older AGYW want. As such, the pharmacist will likely be relied upon to discuss topics of interest.} \textbf{For older AGYW, the pharmacist could be a ``last-mile connector” ~\cite{10.1145/2369220.2369258} who provides guidance on making informed choices or facilitates the last step in obtaining a contraceptive method (RQ1)}. 
 
On the other hand, for younger AGYW who were less technically experienced and \ladd{less} informed about contraception, pharmacists may need to play a more active role throughout the pharmacy visit. To support successful technology use, pharmacists may need to practice ``proximate enabling” ~\cite{10.1145/1753326.1753718} by helping AGYW with using the app. For younger AGYW who had nascent contraceptive knowledge, the app helped build foundational knowledge that is necessary for reaching the point of making informed choices. While the app exposed AGYW to standard information, it did not answer all of their questions, as \lrem{became}\ladd{was} evident \ladd{in the workshops} when younger AGYW asked questions \ladd{about side effects and length of method effectiveness}\lrem{ in the workshops} after using the app. \ladd{Pharmacy staff also expected that AGYW might ask questions about difficult health concepts, such as hormones, that the app does not explain.} To explain \lrem{difficult health concepts, such as hormones,}\ladd{health information} in understandable language, pharmacists may need to practice ``proximate translation,” which is the simplification of information ~\cite{10.1145/1753326.1753718}. \textbf{The pharmacist plays a critical role in bridging younger AGYW’s knowledge gap to build a foundation for making informed choices (RQ1)}. 

Pharmacists’ ability to execute various roles, however, could depend on their domain knowledge. Pharmacists may find that communicating with younger AGYW about contraceptive information is easier than providing nuanced guidance on method choice for older AGYW. However, the ease of communication with younger AGYW could depend on pharmacists’ comprehensive knowledge of contraception and even sexual and reproductive health holistically, as younger AGYW are likely to raise broader health questions. In this sense, pharmacists might struggle less in communicating with older AGYW who want a quick conversation to have their method choice validated by the pharmacist. However, older AGYW may want more method-specific information that was not addressed with the app, which would require pharmacists to have in-depth knowledge of particular contraceptive methods.

The following sections build on the strengths of intermediation, as described above, to discuss important considerations for technological interventions in intermediated scenarios in LMICs.
 
\subsection{Extending Impact to a Hard-to-Reach Population}

Especially in LMICs where technology use is growing, education levels vary, and access to resources are limited, it is important to consider who is being reached with interventions but also those who fall just outside their reach. We draw on a conversation in ICTD to illustrate how intervention decisions can include or exclude certain populations. Despite the strong presence of feature phone owners in LMICs ~\cite{Silver_Johnson_2018, Kenya_2023, Tanzania_2015, Uganda_2022}, smartphone users became a primary interest for ICTD projects as they became technologically eas\lrem{ier}\ladd{y}-to-reach due to growth in smartphone penetration in LMICs ~\cite{Delaporte_Bahia_2022} and the perception that smartphones are innovative, convenient, and desirable. However, some ICTD researchers advocate for the continued support of feature phone users ~\cite{10.1145/3613904.3642099, 10.1145/3555648, doi:10.1177/2050157918776684}. %Recognizing how important it is to represent or target diff segments with diff tech and diff aspects of the tech. 

We extend this reflection to our work to understand how intervention decisions can differently impact older AGYW who are technologically eas\lrem{ier}\ladd{y}-to-reach and younger AGYW who are hard\lrem{er}-to-reach. Older AGYW’s access to and experiences with technology in addition to their contraceptive knowledge likely contributed to them thinking that the app would be irrelevant to AGYW who were non-literate or lacked basic contraceptive knowledge. They also requested more app features and wanted the app publicly available so that smartphone users, like themselves, can use it before \lrem{their }visiting the pharmacy setting. While changes made according to these suggestions could encourage wider use of the app by eas\lrem{ier}\ladd{y}-to-reach\ladd{, older} AGYW who can access and use technology, \ladd{as explained earlier, the app needs to be studied and refined before it is ready to be made widely available. AGYW who have access to smartphones might share their devices with family members; an app installed on a shared device risks exposing AGYW, which could harm their safety. Moreover,} \textbf{over focusing on a technologically eas\lrem{ier}\ladd{y}-to-reach population could hinder efforts to reach a hard\lrem{er}-to-reach population with less resources (RQ2)}. \ladd{If the app were to be designed exclusively for older AGYW, younger AGYW, who are arguably the subpopulation who has a relatively larger knowledge gap, may not have other opportunities to learn about contraception. As such, we} \lrem{We }prioritize building a single, robust app that \lrem{can }extend\ladd{s} impact to AGYW who are hard\lrem{er}-to-reach because of their low levels of education, technology experience, and maturity around sexual and reproductive health. Not only does the pharmacy setting provide an environment where technology is accessible, but the intermediation of the pharmacist could extend the benefits of the app to a wide range of users ~\cite{10.1145/1753326.1753718}. For hard\lrem{er}-to-reach AGYW, the benefits of technology may only be realized with intermediation in the pharmacy setting ~\cite{10.1145/2369220.2369258}. 

 %Different technologies have potential to appeal to and involve a hard\lrem{er}-to-reach population, as we saw in younger AGYW’s positive responses to the app’s audio and video.


\subsection{Implications for Complex Healthcare Settings in LMICs}
There have been numerous efforts in HCI to understand how technologies fit into complicated and sensitive domains, including within health ~\cite{10.1145/3613904.3642245, 10.1145/3637323, 10.1145/3411764.3445410, 10.1145/3313831.3376465}. We contribute to these efforts by reflecting on how \lrem{our}\ladd{the Mara Divas} app could integrate into a social system with human relations, beliefs, and practices and a medical system with a standard for understanding contraception and improving health outcomes. In doing so, we discuss implications for technological interventions in LMIC-based healthcare settings with hopes to help HCI researchers in the health domain \lrem{identify affordances that can make }\ladd{clarify how} technology \ladd{can be} beneficial and impactful. \ladd{These implications and considerations are based on our understanding of the social and cultural contexts of this work and experiences with designing, developing, and getting feedback on an initial app prototype.}

Healthcare providers in LMICs might lack adequate training or motivation ~\cite{scott2016non, bitton2017primary}, possibly because healthcare settings are overburdened. Our intervention considered this concern by creating an app that in itself does not depend on pharmacy staff’s knowledge but rather aims to augment it. However, the app could complicate information penetration. In one way, the app could promote views that are difficult for the pharmacist to endorse. In another way, the app could inform AGYW about difficult topics that the pharmacist could not represent. \ladd{For example, in the pharmacy staff workshop role play, the pharmacist asked the AGYW if religion allowed her to use contraception, implying the influence of religion on a pharmacist’s stance on contraceptive service provision. It is plausible that a pharmacist might not encourage AGYW to use contraception. The app, on the other hand, provides evidence-based information that could help AGYW in meeting their goals to prevent pregnancy. However, the potential tension between pharmacists’ views and the app’s information risks exposing AGYW to contradicting views.} \lrem{To minimize the risk of exposing AGYW to contradicting views, the messaging that is promoted in the intervention and that represented by providers should be aligned. }\textbf{Standardization of health messaging is crucial because messaging could promote accurate and consistent information dissemination and penetration and impact behavior change and health outcomes ~\cite{agha2010intentions, mwaikambo2011works} (RQ2)}. However, achieving standardization in LMICs is challenging because of issues with inadequate funding, overburdened healthcare workers, and healthcare worker complacency ~\cite{adovor2021medical}. In commercial settings, such as pharmacies, profit-maximizing motives ~\cite{wulandari2021prevalence, miller2016performance} and the lack of monitoring and enforcement of rules ~\cite{wulandari2021prevalence} could pose additional barriers for ensuring that service providers adhere to standardized messaging. Moreover, the controversy surrounding sensitive topics including contraception complicates the process of standardization and acceptance of a chosen standard.

Client-provider relations can impact clients’ willingness to seek healthcare services and reception towards technological applications in healthcare ~\cite{10.1145/3613904.3642245}. Our population of interest felt discouraged from approaching pharmacies because of misaligned incentives, goals, and priorities with the pharmacist. \ladd{AGYW were anxious about interactions with the pharmacists from before their visit, causing them to rehearse the interactions, wear cover-ups to hide their identities, or avoid the pharmacy setting entirely.} \lrem{It is possible that s}\ladd{S}trained AGYW-pharmacist relationships could \ladd{consequently} dissuade AGYW from approaching the pharmacies to use the app. This \lrem{case}\ladd{concern} demonstrates the limitations of technology alone to make an impact. \textbf{While relationship-building was not intended to be a focus of our technological intervention, it may be necessitated as part of broader motivation to improve AGYW-centered contraceptive resources and contribute to improved health outcomes (RQ2)}. To realize the potential significance of intermediated interactions in the pharmacy setting, pharmacists will need to improve their relationships with AGYW, a critical foundation of intermediated interactions ~\cite{10.1145/2369220.2369258}, by building their competencies and attitudes to promote an AGYW-friendly environment ~\cite{chandra2014contraception}. 

The physical infrastructure of commercial healthcare settings, which tend to lack waiting areas and private counseling areas ~\cite{de2018adolescent}, raises unique considerations for interventions in these settings. \ladd{In deciding to place the app in the pharmacy setting, we also considered the privacy risks of using the app at the pharmacy counter where it is highly visible and open. As such, we}\lrem{ We} intend for the \ladd{Mara Divas} app to be used only in pharmacies with private counseling rooms to provide a safe space to learn and discuss a sensitive topic. The provision of a private room was also a motivation for targeting tablets for our app. \lrem{Requiring AGYW to use the app at the pharmacy counter where it could be visible to people would impose a higher privacy risk for AGYW than using the app in a private back room.}\textbf{The physical infrastructure of healthcare settings in LMICs, which is often restricted and overcrowded, is an important aspect of intervention design that, if not considered carefully, could harbor risks (RQ2)}.%These complexities and variety of concerns that could arise in healthcare settings in LMICs emphasize a strict consideration of settings, their challenges, and implications for interventions. 

\subsection{Limitations and Future Work}


We acknowledge that sampling in the workshops could bias the design of the app, which may not benefit all AGYW. We plan to address this by further exploring more AGYWs’ experiences with the app and in the pharmacy setting in the feasibility study, which took place between July - September 2024, where approximately 100 AGYW engaged with the app and provided feedback \ladd{and pharmacy staff provided their perspectives on the app intervention}. The goal of the feasibility study was to evaluate the feasibility, acceptability, and appropriateness of the app in the pharmacy setting from the perspectives of AGYW and pharmacy staff at study sites. Findings from the feasibility study will be presented in a future paper.

This study was carried out in a single county in Kenya and its findings are not generalizable to the whole of Kenya. However, we believe that the findings from this study and the feasibility study will be useful for informing the next phase of research, which is to refine the app with consideration for uses in rural areas and other regions of Kenya. Our team plans a hybrid effectiveness implementation trial to concurrently examine both clinical effectiveness and implementation outcomes in the intermediated pharmacy setting. This research will include ongoing refinement of the app and strategies for integrating it into the pharmacy setting, as well as a cluster-randomized trial to examine clinical outcomes. 
 
\ladd{At this stage of the research process, we think it is appropriate to optimize situating the app into the pharmacy setting where AGYW already seek contraceptive services. After the value and utility of the app in the pharmacy setting is studied and validated, there may be opportunities to extend the benefits of this app beyond the pharmacy setting. This process, though, will entail its own strategy and design changes to the app to ensure the app’s success as part of a wider contraception awareness agenda. Peer mobilizers, who were leveraged to recruit AGYW participants for the co-design workshops, could play an important role in helping peers prepare for pharmacy visits. For example, peer mobilizers with smartphones could assist AGYW without smartphone access in using the app before visiting the pharmacy setting. This process would entail training peer mobilizers to facilitate app usage, including smartphone usage. Future work could also involve the expansion of the population of interest for male peers by presenting male contraceptive methods in the app and expanding app usage to young men. Enabling wider availability of the app so that it can be used outside of the pharmacy setting could provide opportunities for AGYW and their male peers to use the app together and collaboratively learn about contraception.}
