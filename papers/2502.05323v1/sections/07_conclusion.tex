\section{Conclusion}

AGYW in Kenya and more broadly in sub-Saharan Africa lack contraceptive decision-support resources that are tailored to this population. To contribute to efforts to empower an underserved population and improve health outcomes, we developed a person-centered, tablet-based app \ladd{- the Mara Divas app -} that provides contraceptive education and decision-support for AGYW in the pharmacy setting in Kenya. Through co-design workshops, we gathered app feedback and explored how our intervention could integrate into the pharmacy setting. Our findings suggest that our intervention will have a differential impact on AGYW depending on their age, education, technology experience, and contraceptive knowledge. We highlight the significant role that intermediation could play in the context of the app, with the pharmacist taking on various roles to accommodate different AGYWs’ needs.

More broadly, this work presents important considerations to make for technological interventions in intermediated healthcare scenarios in LMICs. We reflect on who is being reached and who is unable to be reached with these interventions and advocate for extending impact to a hard\lrem{er}-to-reach population that is characterized by their low levels of education, health literacy, and access to and familiarity with technology. We also recognize that effectively and meaningfully designing technology within healthcare settings in LMICs warrants a reflection on the complex system it is being designed for, including the medical system, human relations, and physical infrastructure. We hope that our work contributes towards a holistic understanding of the challenges and opportunities in LMIC-based interventions.

