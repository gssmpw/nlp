\section{Background}

In LMICs, nearly 50\% of adolescent pregnancies are unintended, most of which occur among adolescents who want to avoid a pregnancy but are not using a modern contraceptive method ~\cite{sully2020adding}. In East Africa, AGYW aged 15-24 years encounter multiple barriers to accessing sexual and reproductive health services, such as stigma around female sexual activity and social pressure to have a child ~\cite{sully2020adding}. Despite a need to provide contraceptive services that meet AGYW’s needs, conventional health facilities struggle to reach AGYW \ladd{because their services are expensive and time-consuming, and AGYW are concerned about healthcare providers maintaining confidentiality} ~\cite{corroon2016key}.

Commercial drug sellers, including pharmacies, are key contraceptive access points for AGYW in sub-Saharan Africa ~\cite{gonsalves2023pharmacies, gonsalves2020mixed, radovich2018meets}. Pharmacies are preferred by many Kenyan AGYW because of their convenience, privacy, speed of service, and perception of less judgemental pharmacists ~\cite{gonsalves2023pharmacies, gonsalves2020mixed}. However, pharmacy-based contraceptive services have downsides, including sporadic and inaccurate counseling ~\cite{gonsalves2020pharmacists} due to lack of pharmacy provider training ~\cite{gonsalves2019regulating} and limited availability of contraceptive methods ~\cite{gonsalves2023pharmacies}. As such, contraceptive counseling in the pharmacy setting struggles to align with AGYW’s values and preferences. The impetus of \lrem{the}\ladd{our} app is to provide person-centered contraceptive decision-support tailored to AGYW seeking contraceptive services in the pharmacy setting.
