\section{Methods}

This study builds on formative work aimed to provide decision-making support for AGYW seeking contraception in pharmacies in Kenya. Our team previously explored factors that influence contraceptive preferences and decision-making for AGYW in Kisumu, Kenya, assessed the relative importance of contraceptive method and service delivery characteristics, and examined the quality of contraceptive care in pharmacies in western Kenya. Rigorous qualitative and qualitative research informed our intervention design and provided an understanding of contraceptive care for AGYW in Kenya seeking contraceptive services in the pharmacy setting.

We designed and developed \lrem{an app}\ladd{the Mara Divas app} to be placed in the pharmacy setting, a key access point for contraceptive services for Kenyan AGYW, to provide contraceptive education and decision-support tailored to AGYW's needs and preferences. We conducted co-design workshops with AGYW and pharmacy staff in Kisumu county, Kenya to understand how to integrate the app into the pharmacy setting and to gather app feedback. Feedback from the workshops inspired revisions to the app, which were made in between workshops and then after. \ladd{This research was approved by the Kenya Medical Research Institute Scientific Ethics Review Unit and the University of Washington Human Subjects Division. We obtained a research permit from the Kenya National Commission for Science, Technology, and Innovation.}\lrem{This study was approved by the IRB in Kenya and at our university after undergoing rigorous review.} 

\subsection{Setting}
\ladd{This work takes place in Kisumu county which is located in western Kenya. Its largest population center is the city of Kisumu, which is the third largest city in Kenya. In 2022 in Kisumu county, 9.2\% of AGYW aged 15-19 years had ever experienced a live birth and 11.1\% had ever experienced pregnancy ~\cite{DHS2023}. Luo is the primary ethnic group of Kisumu county where DhoLuo, Kiswahili (the national language), and English are spoken. Public health clinics offer most contraceptive methods for free or for a small fee. Pharmacies usually offer condoms, emergency contraceptives, depot medroxyprogesterone acetate (injectables, also known as ``depo”), and combined oral contraceptive pills. Some pharmacies also place contraceptive implants on site, but most refer patients to nearby clinics for implants or intrauterine devices (IUD/IUCD).}
