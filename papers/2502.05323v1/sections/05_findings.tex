\section{Findings}

We first present a summary of participant demographics. \lrem{Among 21 AGYW aged 18-24 years, seven were in post-secondary school, 10 had secondary school as their highest level of education, two entered secondary school but have not/did not complete, and two had primary school as their highest level of education. Among 12 AGYW aged 15-17 years, eight entered secondary school but have not/did not complete and four had primary school as their highest level of education. }In this paper, we refer to ``older AGYW” as those aged 18-24 years \ladd{(i.e., workshops AGYW-1 and AGYW-3) }and ``younger AGYW” as those aged 15-17 years \ladd{(i.e., workshop AGYW-2). Table \ref{tab:AGYW-participants} details participant characteristics for the three AGYW workshops. Younger AGYW, as expected, had overall lower levels of education than older AGYW. While all older AGYW owned a personal mobile phone, less than half of younger AGYW owned a personal mobile phone. Smartphone access was more common among older AGYW. This information regarding AGYW’s ownership of and access to a smartphone validates our decision to place it in the pharmacy setting, which we discuss in detail in section \ref{changes-not-made}}. Among all 33 AGYW, the most common contraceptive method that was used was condoms (n=25), followed by emergency contraceptive (n=13) and implants (n=13). 
%The median score for AGYW’s confidence that the methods they use or last used is right for them on a 5-point Likert scale was 4. 
\lrem{Among 22 AGYW aged 18-24 years, all owned or shared a mobile device with family and 16 used a smartphone. Among AGYW who owned or shared a mobile device, the median score for the level of comfort using mobile apps on a 5-point Likert scale was 4. On the other hand, among 12 AGYW aged 15-17 years, only five owned or shared a mobile device and only two used a smartphone.} 

\ladd{As indicated in Table \ref{tab:pharm-participants}, the 10 pharmacy staff’s confidence levels in advising AGYW were high. We measured the pharmacy staff’s confidence levels on advising AGYW on family planning options, on how to use family planning options, and on family planning side effects on a 5-point Likert scale, with 1 being not at all confident and 5 being extremely confident. The median scores were 5 (IQR 4-5), 5 (IQR 4-5), and 5 (IQR 4.25-5), respectively.} \lrem{Among 10 pharmacy staff, nine have been working in the pharmacy setting for five or more years and one has been working for one to two years. The median scores for confidence in advising AGYW on contraceptive options, advising AGYW on how to use contraceptive methods, and advising AGYW on contraceptive side effects were all 5. }

\lrem{In this section}\ladd{Next}, we first provide a summary of app feedback and how they impacted intervention decisions. Then, we elaborate on findings from group discussions and thematic analysis that expanded our understanding of how the \ladd{Mara Divas} app could integrate into the pharmacy setting. \ladd{In our findings, we refer to the workshop IDs except when they are obvious in the context of the findings (i.e., findings from younger AGYW are from workshop AGYW-2, findings from pharmacy staff are from workshop Pharmacy-1.)}

\subsection{App Feedback}

\subsubsection{Positive validation on app design}

Overall, we received positive app feedback. Pharmacists said that the app answered AGYW’s commonly asked questions in a \textit{``systematic”} way that helped \lrem{users step}\ladd{with stepping} through questions. AGYW also said that the app covered topics of importance and interest to them. AGYW and pharmacists appreciated the videos featuring providers, saying they added credibility to the app’s information. AGYW also praised the peer videos, noting how they felt encouraged when hearing their peers talk confidently about their contraceptive experiences. \ladd{AGYW also appreciated the provision of headphones, saying that they promoted privacy so that no one in the pharmacy could know what the AGYW is listening to on the app (AGYW-3).}

\subsubsection{Changes made}

Aside from significantly changing the app’s color scheme to bright purple and pink based on feedback from AGYW, most changes were to do with content or information presentation. One notable change is in the framing of the app. Pharmacists and AGYW said that ``family planning” was not relatable for AGYW who were not preparing to raise a family, but rather were prioritizing pregnancy prevention\lrem{, as summarized by a pharmacy staff}: \textit{``What is family planning, by the way? You are planning a family. This is a 16-year-old person who does not have a family. What are they planning? `I want to prevent pregnancy\lrem{.}’”} \ladd{(Pharmacy-1).} Prior work also reflects AGYW’s perceived dissonance between ``family planning” and ``pregnancy prevention” ~\cite{harrington2021spoiled}. To make the app’s framing more appropriate for AGYW’s needs and desires, we replaced the phrase ``family planning” with ``pregnancy prevention” throughout the app. 

We made signifiers on the app more findable and recognizable (e.g., larger buttons, text that prompts actions). Younger AGYW had difficulty understanding the icons that represented contraceptive methods and did not find contraceptive method labels (e.g., “IUCD (coil)”) helpful, likely because of their low language literacy. We complemented the icons with real-life images of methods so that AGYW of varying literacy levels can comprehend app content.

We also made changes to information presentation techniques. AGYW had difficulty finding information regarding how to prevent HIV and STIs and important things to know about the emergency contraceptive. Although this information was linked from other screens, many AGYW were unable to navigate through multiple screens to find this information. We added a button on the main screen (Figure \ref{fig:app} a), where all main topics were outlined, that linked directly to the information. This change was also made in light of pharmacists’ emphasis on the importance of educating AGYW about preventing HIV and STIs.

\subsubsection{Changes not made}
\label{changes-not-made}

In all workshops, participants suggested making the \ladd{Mara Divas} app available on the app store so that more people can access it. However, our intention is to restrict the app’s availability to the pharmacy setting. The main reason was AGYW’s lack of smartphone access, \lrem
{which was confirmed with}\ladd{as evident in} younger AGYW \ladd{participant demographics} \lrem{in our workshops }\ladd{(Table \ref{tab:AGYW-participants})}. We also considered the lack of privacy when using this app on personal or shared mobile devices\ladd{. Some AGYW have access to a mobile device but share it with family members. If the app were made available on the app store and installed on shared mobile devices, there is a risk that other users would discover that the AGYW is sexually active, which could cause harm to them. We were motivated to make our app low-barrier by placing it in}\lrem{ and the benefits of using the app in} a pharmacy’s private counseling room where AGYW would not be seen. \ladd{Another reason for restricting this app to the pharmacy setting is that at this research stage, the app is too premature to make it publicly available. The app and its effectiveness needs to be studied and refined to make it usable, understandable, and beneficial for AGYW before making it widely available.}

Participants from all of the workshops expressed that the app should include information on male contraceptives and/or aim to educate male peers. Aside from the lack of time to incorporate these ideas into the final version of the app, we did not add information targeting male youth because such feedback fell outside of the scope of the motivation of this app, namely focusing on supporting AGYW (not young boys or men) in contraceptive education and decision-making. 

Older AGYW also suggested adding all languages spoken in Kenya to the app. Aside from the lack of time to translate all content in all languages, we restricted the app to three languages because most AGYW in Kisumu county would only be comfortable with DhoLuo, Kiswahili, and/or English.

\subsection{Key moments in counseling} 

We shed light on pharmacist behaviors and attitudes and the realities of AGYW-pharmacist interactions that give context to challenges in pharmacy-based counseling. 

\subsubsection{Pharmacists’ personal beliefs affect AGYW-pharmacist interactions} 

We highlight key moments of the role play in the pharmacy staff workshop \ladd{(Figure \ref{fig:workshop} c)} to present examples of misalignment between pharmacists’ personal beliefs, interests, and priorities and those of AGYW. The entire role play lasted approximately 20 minutes.

\begin{enumerate}
\item Pharmacist asks AGYW whether her religion allows her to use contraceptives.
\item Pharmacist asks AGYW about her boyfriend, insists that she brings her boyfriend to her next visit, and asks for her boyfriend’s contact information. AGYW says she does not want her boyfriend involved, but the pharmacist insists.
\item Pharmacist is concerned about the AGYW’s potential exposure to HIV and STIs and risk of dropping out of school. AGYW says she is not concerned about HIV and STIs, only about getting pregnant.
\end{enumerate}

\ladd{In each of these key moments, there is misalignment between what the AGYW wants and what the pharmacist insists. The pharmacist’s priorities for the AGYW are also apparent, such as staying in school and minimizing exposure to HIV and STIs. The incongruence between the priorities of the AGYW and those of the pharmacists is one of the reasons why AGYW are anxious about approaching the pharmacy setting for contraceptive services.} \lrem{These moments were comparable to the ones raised by AGYW as those that make them anxious about approaching the pharmacists. These include, }\ladd{AGYW expressed that they have had negative experiences with pharmacists, such as }being scolded by the pharmacist for not knowing anything about contraceptives, being told that they are too young for sex, and being asked who their sexual partners are. \ladd{To avoid confrontation and uncomfortable interactions, AGYW rehearsed how to talk to the pharmacist so that they can avoid crying during counseling (AGYW-2), thought of ways to convince the pharmacist to maintain confidentiality (AGYW-2), and considered going to a pharmacist where the AGYW is not known to the pharmacist (AGYW-3).}

We highlight two topics that revealed disagreements among pharmacy staff. First, pharmacists expressed different sentiments with involving young men in contraceptive education. With respect to a screen on the app showing a female body (Figure \ref{fig:app} b), one pharmacy staff said, \textit{``If we put a picture of a man in the display here, we are trying to put this as a collective responsibility…So the moment you start omitting man here is the moment you walk this journey alone. But if you include the man here you will get a lot of support here and the impact will be great,”} indicating the benefits of educating a wider audience about contraception. However, some pharmacists said that the app should adhere to its original intent to focus on AGYWs' needs and preferences. 

Second, the young age of AGYW presented a challenge for some pharmacists, but not all, to provide contraceptive services. Pharmacists were concerned about being seen as a business that gave AGYW \textit{``a ticket to promiscuous-ity”} and being known as the \textit{``clinic that offers these services that makes our girls behave this way.”} However, not all pharmacists thought the client’s age swayed their decision to provide contraceptive services: \textit{``At the end of the day you end up giving what you will be giving. You end up exposing her to these questions, the methods. Whether she's 15 or 24, whether she's below 18… So is [age] really important?”}

Not all pharmacists shared the same beliefs, indicating the possibility of inconsistent practices. The variability of pharmacists’ attitudes seemed to contribute to AGYW’s anxiety and uncertainty about whether they will encounter a judgemental or friendly pharmacist. 

\subsubsection{Challenges in counseling}

In response to the 20 minute role play, observing pharmacy staff participants said that most AGYW were unwilling to stay for that long. \lrem{According to pharmacy staff, m}\ladd{M}any AGYW present themselves as \textit{``all-knowing,”} because they have been using a method for a while, know what they want from the pharmacist, and do not desire counseling \ladd{(Pharmacy-1)}. There are also AGYW who send others to request P-2 \ladd{(the brand name of an emergency contraceptive pill, Postinor-2)} on their behalf, resulting in challenges with properly counseling AGYW: \textit{``Some will even send boda boda \ladd{[motorcycle or bicycle taxi]} guys, they will send their younger siblings…the end users will not come, they will send someone else. So even to counsel them, it is not easy\lrem{.}”} \ladd{(Pharmacy-1).} Other times, while they are able to approach the pharmacists, \lrem{they}\ladd{AGYW} pretend to obtain contraceptives for someone else or hide their identities by wearing a hijab\ladd{, a head covering conventionally worn by Muslim women,} to cover their face \ladd{(AGYW-2)}. These deliberate choices that AGYW make for obtaining contraceptives present challenges for the pharmacy staff to counsel AGYW. \ladd{However, AGYW avoided interactions with pharmacists because they worried about the pharmacist’s demeanor towards AGYW, being seen by someone they know at the pharmacy, and the pharmacist not maintaining confidentiality. These concerns stemmed from AGYW’s prior negative interactions with pharmacists, such as pharmacists interrogating AGYW about why they were here, pharmacists saying loudly what the AGYW’s request is in a way that was audible to other clients, and pharmacists telling AGYW that they were too young for sex.}

\subsection{Differences between Older AGYW and Younger AGYW}

We explain differences between older AGYW and younger AGYW with respect to language, digital, and health literacy levels. We elaborate on differences in their perceptions of and interactions with the app. 
\subsubsection{Language literacy}

\ladd{While younger AGYW’s lower education levels were not unexpected, they influenced younger AGYW’s comfort with engaging in text-based content of the app.}\lrem{Younger AGYW’s education levels were a main reason for why they were less comfortable with reading compared to older AGYW, whose majority completed secondary school.} \ladd{Younger AGYW were overall less comfortable than older AGYW with respect to comprehending English. During workshop activities, younger AGYW used Kiswahili or a mixture of Kiswahili and English, while older AGYW were more comfortable conversing in English. Younger AGYW engaged with the app’s English content less often and instead relied on the Kiswahili content more. For both older and younger AGYW, DhoLuo was the least popular language option. We learned that DhoLuo is more commonly spoken by adults in Kisumu county, rather than the younger population.} Some younger AGYW preferred to only listen to audio and watch videos, saying that the audio helped them quickly understand the content, and noted their benefits for a non-literate individual. One participant said that the audio reminded her of listening to a teacher, which encouraged her to pay attention to the content more keenly than reading it (AGYW-2). A few younger AGYW used the audio feature without instruction or explanation from facilitators.

\subsubsection{Digital literacy}

Older AGYW used the tablets very comfortably, likely because of their prior experiences with smartphones. We did not notice any older AGYW requiring assistance with tablet or app usage. On the other hand, many younger AGYW were not familiar with touchscreen devices and were tapping parts of the screen that were irrelevant to the functionality of the app. As a result, facilitators assisted younger AGYW with app usage, such as with launching the app (i.e., tapping the app icon to open the app) and finding the volume adjustment buttons on the side of the tablet. This was expected given their lack of experience with and ownership of smartphones. One younger AGYW told us that while she found the audio feature helpful, her encounter with it was unintentional: \textit{``At first, I was just choosing randomly and I found it. Now I know how to find it”} (AGYW-2). This indicates that younger AGYW took an experimental approach to understanding how to use the app and tablet \ladd{as opposed to their interactions with the app and tablet being informed by prior experiences with technology}. 

\subsubsection{Health literacy}

Both older and younger AGYW expressed that the app debunked rumors and misinformation, indicating that both subpopulations were misinformed to some degree. However, older AGYW suggested that they were more familiar with the app’s information relative to younger AGYW. Older AGYW had prior experiences searching for contraceptive information and were mostly familiar with the contraceptive methods on the app. The app was helpful for confirming or augmenting prior knowledge with additional information, such as the administration of contraceptive methods. Some older AGYW were interested in \ladd{obtaining} in-depth information on specific contraceptive methods, such as injectables, as opposed to general contraceptive information that was on the app. Moreover, older AGYW expressed that the app provided them with the \textit{``reassurance [that] all will be well”} because it laid out all potential concerns \ladd{(AGYW-1) and the app’s content resonated with what the AGYW had experienced before (AGYW-3)}. For example, one participant said that the app’s information minimized their fear of side effects of injectables and gave them a better sense of the right method for them. Videos also contributed to feelings of reassurance \ladd{(AGYW-1)}. Provider-based videos provided additional information that boosted their confidence in decision-making and peer videos helped AGYW feel less isolated. 

On the other hand, younger AGYW expressed that much of the app’s information was novel to them. For example, one younger AGYW told us that they learned that only condoms could prevent HIV and STIs. Younger AGYW’s behaviors also suggested the novelty of the app’s information. They were writing notes on the app’s content, discussing among themselves about the content, and asking follow-up questions to the workshop facilitators about the side effects and the length of method effectiveness. Younger AGYW also discussed how their journeys in the pharmacy setting would be impacted by the app. They said that before the pharmacy visit, they would have limited knowledge on available methods, their use, benefits, and side effects. They would also be uncertain of how to engage with the pharmacist and express their needs, leading to high pressure. However, the app could allow AGYW to gain knowledge on contraception and feel more confident in expressing their needs to the pharmacist. These observations and responses imply that younger AGYW have less contraceptive knowledge to begin with as compared to older AGYW, but the app could allow them to gain new information and confidence in approaching the pharmacist.

\subsection{The Roles of the Pharmacist}
Pharmacists and AGYW agreed that pharmacists play a critical role in facilitating contraceptive education and decision-making. We elaborate on the different roles that the pharmacists might take on with the introduction of the app.

\subsubsection{Follow best counseling practices}

Pharmacists understood that AGYW avoided counseling due to fear of long questioning. One pharmacist said that \textit{``counseling and questioning are different,”} implying that counseling should not be an interrogation. Overall, pharmacists agreed on the importance of creating a safe space to allow AGYW to openly share their needs and concerns. Creating a safe space requires the pharmacist to place their personal beliefs aside: \textit{``whatever you are saying, don't judge them. Always let them have their way so that they can make the decision for their own. Don't decide for them”} \ladd{(Pharmacy-1)}. AGYW expressed that they want to be treated with respect and trust that pharmacists would maintain confidentiality. \ladd{They recommended providing pharmacy staff training so that they can better talk with AGYW (AGYW-3), suggesting that better counseling training could make AGYW feel more comfortable approaching the pharmacy setting.} Younger AGYW believed in the pharmacists’ professionalism to manage the app and were not concerned about the app being used in the pharmacy setting. 

\subsubsection{Accommodate different AGYW needs}

Older AGYW thought that counseling could occur at different moments of the pharmacy visit depending on the AGYW’s background. One older AGYW said that direct in-person counseling was more valuable than using the app for non-literate AGYW and another said that the app was more valuable for those who were more aware of contraception and had basic knowledge: \textit{``The ones who will use this app, they are aware. So for the ones who do not understand it need more consultation”} \ladd{(AGYW-1)}. However, younger AGYW did not question that they were a population of interest for the app. \ladd{While y}\lrem{Y}ounger AGYW \lrem{did }express\ladd{ed} that they might encounter issues with app usage, \lrem{but}\ladd{they} did not recognize this as a\lrem{n} \ladd{significant} issue because they thought that the pharmacist could \lrem{assist}\ladd{facilitate app use}. Pharmacists were willing to help AGYW, especially those who were not tech savvy, to use the app. Overall, these sentiments indicate AGYW’s perception that the pharmacist could support AGYW of varying language, digital, and health literacy levels. 

\subsubsection{Answer AGYW’s questions}

\ladd{In the workshops, younger AGYW asked follow-up questions about the side effects and the length of method effectiveness, suggesting that the app helped them structure questions.} Younger AGYW wanted the pharmacist to confirm the information they learned from the app and provide more explanations if needed\ladd{, suggesting that they trust the pharmacist’s expertise and guidance (AGYW-2)}. \lrem{Their behaviors suggested that they would need the pharmacist to answer the questions they structured after using the app.}Pharmacists also recognized that although the app provided answers to important questions about contraception, AGYW may not fully understand the concepts that were being addressed in the app, so \textit{``the service provider will be in a position to explain further…the terms [that AGYW do not understand]...`What are hormones and what do they do in our body, why will there be changes, and if I stop using them will there be other changes?'”} \ladd{(Pharmacy-1).} Pharmacists \lrem{were keen to \textit{``give [AGYW] room to ask questions,”}}\ladd{wanted to give room for AGYW to ask their questions at their own pace,} but recognized that they needed deep knowledge of contraception and have to learn to communicate with AGYW: \textit{``[pharmacists] need to be trained on the services and especially the methods, the side effects, the do’s and the don'ts\ladd{}\lrem{.}”} \ladd{(Pharmacy-1), echoing a similar sentiment that AGYW expressed on wanting the pharmacists to be trained on how to converse with AGYW (AGYW-3).} 

\subsubsection{Provide guidance on method choice}

The desire for the pharmacist to supportively \textit{``affirm your choice”} \ladd{(AGYW-1)} and provide guidance on appropriate methods was more strongly expressed by older AGYW than younger AGYW. Younger AGYW were more focused on thinking about how the pharmacist could support information provision as opposed to guidance on method choice. In discussing the benefits of making the app public, older AGYW wanted the pharmacist to facilitate the process of finalizing on a method choice, as summarized by this participant: \textit{``I feel like when you go to the hospital or pharmacy it's all about seeing the doctor…and tell him or her what's really happening or what I want. So I think the app should just be on the Play Store so that you can just have the information at hand. You can go and tell the doctor you want this and this\lrem{.}”} \ladd{(AGYW-1).} This AGYW wanted to spend more time discussing preferred method choices with the pharmacist than reviewing general contraceptive information with them. Older AGYW suggested that, if the app were available for use outside of the pharmacy setting, they would know what method they want by using the app beforehand and use the counseling time to request in-depth information about their preferred method or immediately request a transaction: \textit{``You can go for consultation but at least you know what you are going for. So maybe say you're going for depo, you need more explanation\lrem{.}”} \ladd{(AGYW-1).}  Older AGYW explained that they would be able to approach the pharmacist with less anxiety because the app would answer most of their questions prior to the pharmacy visit. 



