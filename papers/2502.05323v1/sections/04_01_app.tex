\subsection{App Design and Development}
Considering that AGYW have low levels of knowledge regarding side effects of contraception and hold strong beliefs about contraceptive harms ~\cite{harrington2021spoiled, ochako2015barriers, sedlander2018they}, we did not develop an app that only recommended the ``right” contraceptive method. We were motivated to provide information on various contraceptive options regarding topics that are important to AGYW, including bleeding patterns, privacy of contraceptive use, and influence on future fertility ~\cite{harrington2023adaptation, harrington2021spoiled} and demystify side effects. We were inspired by the person-centeredness of My Birth Control, a U.S.-based decision support tool that assists women with contraceptive method selection ~\cite{dehlendorf2019cluster, holt2020patient, dehlendorf2019mixed}. \ladd{We named our app the Mara Divas app, in which ``Mara” means ``my own” in DhoLuo, the local language of Kisumu county, Kenya.}

The \ladd{Mara Divas} app is \lrem{intended}\ladd{designed} to be used in a private counseling room in the pharmacy setting. \ladd{The room will provide a tablet with the app installed and a pair of headphones. The intended users of the app are AGYW seeking contraceptive services in the pharmacy setting. As such, when we use the term ``user” henceforth, we refer to the AGYW using the app.} When the AGYW approaches the pharmacist, the pharmacist would take them to a private counseling room to use the app for 5-20 minutes. Lastly, the pharmacist would communicate directly with the AGYW about their choice and provide either the desired method or a referral. 


\begin{figure*}[hbt!]
  \centering
\includegraphics[width=1.0\linewidth]{figures/app.png}
\caption{Mara Divas app configuration. Enlarged images of each screen are in Appendix \ref{big_images}. (a) Main screen with questions and topics that are of most interest and importance to AGYW. (b) Short explanations of each contraceptive method and where it is administered on a female body. (c) Information presented with text, audio, and videos (provider and peer). (d) Survey where users can indicate preferences for method characteristics. (e) Recommendations for contraceptive methods based on survey results.}
\Description{Five app screens, described from left to right. (a) Main screen with nine questions and topics that are of most interest and importance to AGYW. Top of the screen has three language toggle buttons labeled ``Kiswahili,” ``Dholuo,” and ``English” where ``English” is grayed out to indicate its selection. These buttons are presented in all of the screens explained after. Top of the screen also includes a ``Your Favorites” button with a thumbs-up sign preceding the text, which is also in screens (b) and (e). The screen is filled with buttons, labeled in order from top to button, ``What are my options?” ``What will happen to my period?” ``How long does the method work?” `1`What is my chance of getting pregnant?” ``Can I keep it private?” ``What if I’m ready to have a baby?” and ``Take our quiz and find your method!” Each button is accompanied by a representative image of that topic and an audio button. Two buttons labeled ``3 things you need to know about the E-pill (P\ladd{-}2)” and ``Preventing HIV and STIs” is also exhibited. (b) Top of the page is titled ``What are my options?” A female body in the center of the screen with contraceptive method icons around it. On the left side of the body are icons and accompanying labels of pills (daily pills), emergency pill (e-pill, P\ladd{-}2), IUCD (coil), and on the right side are injection (depo), implant, condom, and female condom. Below each label is a button ``Favorite it!” with a thumbs-up sign preceding the text. The injection (depo) is selected, indicated by a different color. A pop-up at the bottom of the screen shows a description of the injection (depo) with an audio button and a real image of the injection. (c) Top of the page is titled ``What will happen to my period?” Below the title are seven contraceptive methods with icons and labels, as described in screen (b). Implant is selected, indicated by a different color. Middle of the screen displays a text description of the method and an audio button. Below this is two videos - one of a provider, one of an AGYW peer - placed side by side. Below the videos is a button labeled ``WHY?” with a magnifying glass icon preceding the text (d) Top of the page is titled ``Quiz.” Multiple choice quiz/survey questions displayed are ``How do you feel about changes to your periods?” ``If you had to guess, when do you think you might want a pregnancy?” ``How long do you want your method to work?” and ``How important is it to you to keep your method private from your parents or partner?” Selections for each question are indicated by a different color. A scroll bar indicates continuation of the quiz. (e) Top of the page is titled ``Recommendations” and below it, ``Here are some recommendations that might be right for you!” Recommendations displayed are titled ``Based on how you feel about light, irregular periods, these methods might be a good choice for you:” and ``Based on how you feel about the possibility of your periods stopping, these methods might be a good choice for you:” with icons and labels of the depo and implant. Below each method label is a button labeled ``Learn More” and ``Favorite it!” with a thumbs-up sign preceding the text. Below the buttons are short relevant explanations for other recommendations. The bottom of the screen has a button labeled ``END SESSION.” A scroll bar indicates continuation of the quiz.}
\label{fig:app}
\end{figure*}

We built an Android app using the Flutter \ladd{3.19.5} platform\ladd{~\cite{flutter} that supports Android 5.0 and all subsequent versions. Our app targets} \lrem{targeting }Samsung Galaxy \ladd{Tab A8} tablets \ladd{with a 10.5-inch display}. We chose tablets because the pharmacy setting provides a large enough space for private table usage and the contents on a tablet would be easier to interact with than on a phone. We chose Android devices because of their low cost and AGYW’s familiarity with Android. Flutter was selected as an app development framework because it is a mature mobile app platform and provides a professional app appearance. 



%Fig 1: App configuration. (a) Main screen with questions and topics that are of most interest and importance to AGYW. (b) Screen presenting short explanations of each contraceptive method and where they are placed on a female body. (c) Screen with text, audio, and videos (provider and peer). (d) Survey to indicate preferences for method characteristics. (e) Recommendations for contraceptive methods based on survey results.

The content of the app was carefully crafted by a clinical reproductive health expert who has extensive research experience in Kenya. The app’s main screen presents sections with topics that are of most interest to AGYW (Figure \ref{fig:app} a), as learned through formative work. Each section leads the user to a screen with topic-specific information for seven contraceptive methods (Figure \ref{fig:app} b, c): male condom, female condom, daily contraceptive pills, implant, injection (depo), IUCD, and emergency contraceptive (e-pill). For each method, we also made a screen that briefly summarized information about that method. Users can also ``favorite” methods that they like and want to return to for review. 

Once all sections are visited, users complete a survey to indicate preferences for method characteristics (Figure \ref{fig:app} d). The survey is composed of five multiple choice questions. Although five questions are limited in understanding the users’ contraceptive preferences, we restricted the number of questions because AGYW dislike lengthy questioning ~\cite{gonsalves2020pharmacists}. Based on survey responses, the app recommends contraceptive methods by the user’s method characteristic preference (Figure \ref{fig:app} e). %For example, if the user indicated that it was extremely important to keep their method private, the recommendation would state, ``Based on how important it is to you to keep your method private, the following methods might be a good choice for you:” with methods that align with the user’s preference around privacy. 

We prepared the \ladd{Mara Divas} app in DhoLuo, Kiswahili (the national language), and English, which are languages spoken in Kisumu county, Kenya, where the workshops took place and the app is planned to be piloted. A language toggle tool is available on all screens so that users can engage with content in their preferred language. To reach AGYW with limited language literacy, we included audio and brief video components (Figure \ref{fig:app} c). Text is accompanied by a spoken version of the text which was recorded by study staff, indicated by an audio button next to the text. We included two types of videos: videos where providers share their expertise on contraceptive topics, and videos where AGYW share their experiences with using contraceptives. These videos were narrated by service providers and AGYW actors in Kisumu. We incorporated four provider videos and three peer videos. Videos were between one to three minutes long.
